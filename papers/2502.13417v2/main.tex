%%%%%%%% ICML 2024 EXAMPLE LATEX SUBMISSION FILE %%%%%%%%%%%%%%%%%

\documentclass{article}

% Recommended, but optional, packages for figures and better typesetting:
\usepackage{microtype}
\usepackage{graphicx}
\usepackage{subcaption}
\usepackage{booktabs} % for professional tables

% hyperref makes hyperlinks in the resulting PDF.
% If your build breaks (sometimes temporarily if a hyperlink spans a page)
% please comment out the following usepackage line and replace
% \usepackage{icml2024} with \usepackage[nohyperref]{icml2024} above.
\usepackage{hyperref}


% Attempt to make hyperref and algorithmic work together better:
\newcommand{\theHalgorithm}{\arabic{algorithm}}

% Use the following line for the initial blind version submitted for review:
% \usepackage{icml2025}

% If accepted, instead use the following line for the camera-ready submission:
\usepackage[accepted]{icml2025}

% For theorems and such
\usepackage{amsmath}
\usepackage{amssymb}
\usepackage{mathtools}
\usepackage{amsthm}

% if you use cleveref..
\usepackage[capitalize,noabbrev]{cleveref}

%%%%%%%%%%%%%%%%%%%%%%%%%%%%%%%%
% THEOREMS
%%%%%%%%%%%%%%%%%%%%%%%%%%%%%%%%
\theoremstyle{plain}
\newtheorem{theorem}{Theorem}[section]
\newtheorem{proposition}[theorem]{Proposition}
\newtheorem{lemma}[theorem]{Lemma}
\newtheorem{corollary}[theorem]{Corollary}
\theoremstyle{definition}
\newtheorem{definition}[theorem]{Definition}
\newtheorem{assumption}[theorem]{Assumption}
\theoremstyle{remark}
\newtheorem{remark}[theorem]{Remark}

% Todonotes is useful during development; simply uncomment the next line
%    and comment out the line below the next line to turn off comments
%\usepackage[disable,textsize=tiny]{todonotes}
\usepackage[textsize=tiny]{todonotes}



% The \icmltitle you define below is probably too long as a header.
% Therefore, a short form for the running title is supplied here:
% \icmltitlerunning{}

\usepackage{enumitem}
\usepackage{multirow}
\usepackage{tcolorbox}
\usepackage{moreverb}
\usepackage{cprotect}
\usepackage{fancyvrb}
\usepackage{framed}

\newcommand{\myname}[0]{RLTHF}
\newcommand{\yifei}[1]{\textcolor{purple}{Yifei: #1}}
\newcommand{\bbb}[1]{\noindent\textbf{#1}}



\begin{document}

\twocolumn[
\icmltitle{\myname{}: Targeted Human Feedback for LLM Alignment}

% It is OKAY to include author information, even for blind
% submissions: the style file will automatically remove it for you
% unless you've provided the [accepted] option to the icml2024
% package.

% List of affiliations: The first argument should be a (short)
% identifier you will use later to specify author affiliations
% Academic affiliations should list Department, University, City, Region, Country
% Industry affiliations should list Company, City, Region, Country

% You can specify symbols, otherwise they are numbered in order.
% Ideally, you should not use this facility. Affiliations will be numbered
% in order of appearance and this is the preferred way.
% \icmlsetsymbol{equal}{*}
\icmlsetsymbol{intern}{*}

\begin{icmlauthorlist}
% \icmlauthor{Anonymous Authors}{}
\icmlauthor{Yifei Xu}{microsoft,ucla,intern}
\icmlauthor{Tusher Chakraborty}{microsoft}
\icmlauthor{Emre K\i c\i man}{microsoft}
\icmlauthor{Bibek Aryal}{microsoft}
\icmlauthor{Eduardo Rodrigues}{microsoft}
\icmlauthor{Srinagesh Sharma}{microsoft}
\icmlauthor{Roberto Estevao}{microsoft}
\icmlauthor{Maria Angels de Luis Balaguer}{microsoft}
\icmlauthor{Jessica Wolk}{microsoft}
\icmlauthor{Rafael Padilha}{microsoft}
\icmlauthor{Leonardo Nunes}{microsoft}
\icmlauthor{Shobana Balakrishnan}{microsoft}
\icmlauthor{Songwu Lu}{ucla}
\icmlauthor{Ranveer Chandra}{microsoft}
% \icmlauthor{Firstname8 Lastname8}{sch}
% \icmlauthor{Firstname8 Lastname8}{yyy,comp}
%\icmlauthor{}{sch}
%\icmlauthor{}{sch}
\end{icmlauthorlist}

\icmlaffiliation{microsoft}{Microsoft}
\icmlaffiliation{ucla}{University of California, Los Angeles}
% \icmlaffiliation{sch}{School of ZZZ, Institute of WWW, Location, Country}

% \icmlcorrespondingauthor{Firstname1 Lastname1}{first1.last1@xxx.edu}
% \icmlcorrespondingauthor{Firstname2 Lastname2}{first2.last2@www.uk}

% You may provide any keywords that you
% find helpful for describing your paper; these are used to populate
% the "keywords" metadata in the PDF but will not be shown in the document
% \icmlkeywords{Machine Learning, ICML}

\vskip 0.3in
]

% this must go after the closing bracket ] following \twocolumn[ ...

% This command actually creates the footnote in the first column
% listing the affiliations and the copyright notice.
% The command takes one argument, which is text to display at the start of the footnote.
% The \icmlEqualContribution command is standard text for equal contribution.
% Remove it (just {}) if you do not need this facility.

\printfootnote{\textsuperscript{*}Work is done during an internship at Microsoft Research.}  % leave blank if no need to mention equal contribution
% \printAffiliationsAndNotice{\icmlEqualContribution} % otherwise use the standard text.
\newcommand{\tc}[1]{\textcolor{orange}{Tusher: #1}}


Large language model (LLM)-based agents have shown promise in tackling complex tasks by interacting dynamically with the environment. 
Existing work primarily focuses on behavior cloning from expert demonstrations and preference learning through exploratory trajectory sampling. However, these methods often struggle in long-horizon tasks, where suboptimal actions accumulate step by step, causing agents to deviate from correct task trajectories.
To address this, we highlight the importance of \textit{timely calibration} and the need to automatically construct calibration trajectories for training agents. We propose \textbf{S}tep-Level \textbf{T}raj\textbf{e}ctory \textbf{Ca}libration (\textbf{\model}), a novel framework for LLM agent learning. 
Specifically, \model identifies suboptimal actions through a step-level reward comparison during exploration. It constructs calibrated trajectories using LLM-driven reflection, enabling agents to learn from improved decision-making processes. These calibrated trajectories, together with successful trajectory data, are utilized for reinforced training.
Extensive experiments demonstrate that \model significantly outperforms existing methods. Further analysis highlights that step-level calibration enables agents to complete tasks with greater robustness. 
Our code and data are available at \url{https://github.com/WangHanLinHenry/STeCa}.
%!TEX root = gcn.tex
\section{Introduction}
Graphs, representing structural data and topology, are widely used across various domains, such as social networks and merchandising transactions.
Graph convolutional networks (GCN)~\cite{iclr/KipfW17} have significantly enhanced model training on these interconnected nodes.
However, these graphs often contain sensitive information that should not be leaked to untrusted parties.
For example, companies may analyze sensitive demographic and behavioral data about users for applications ranging from targeted advertising to personalized medicine.
Given the data-centric nature and analytical power of GCN training, addressing these privacy concerns is imperative.

Secure multi-party computation (MPC)~\cite{crypto/ChaumDG87,crypto/ChenC06,eurocrypt/CiampiRSW22} is a critical tool for privacy-preserving machine learning, enabling mutually distrustful parties to collaboratively train models with privacy protection over inputs and (intermediate) computations.
While research advances (\eg,~\cite{ccs/RatheeRKCGRS20,uss/NgC21,sp21/TanKTW,uss/WatsonWP22,icml/Keller022,ccs/ABY318,folkerts2023redsec}) support secure training on convolutional neural networks (CNNs) efficiently, private GCN training with MPC over graphs remains challenging.

Graph convolutional layers in GCNs involve multiplications with a (normalized) adjacency matrix containing $\numedge$ non-zero values in a $\numnode \times \numnode$ matrix for a graph with $\numnode$ nodes and $\numedge$ edges.
The graphs are typically sparse but large.
One could use the standard Beaver-triple-based protocol to securely perform these sparse matrix multiplications by treating graph convolution as ordinary dense matrix multiplication.
However, this approach incurs $O(\numnode^2)$ communication and memory costs due to computations on irrelevant nodes.
%
Integrating existing cryptographic advances, the initial effort of SecGNN~\cite{tsc/WangZJ23,nips/RanXLWQW23} requires heavy communication or computational overhead.
Recently, CoGNN~\cite{ccs/ZouLSLXX24} optimizes the overhead in terms of  horizontal data partitioning, proposing a semi-honest secure framework.
Research for secure GCN over vertical data  remains nascent.

Current MPC studies, for GCN or not, have primarily targeted settings where participants own different data samples, \ie, horizontally partitioned data~\cite{ccs/ZouLSLXX24}.
MPC specialized for scenarios where parties hold different types of features~\cite{tkde/LiuKZPHYOZY24,icml/CastigliaZ0KBP23,nips/Wang0ZLWL23} is rare.
This paper studies $2$-party secure GCN training for these vertical partition cases, where one party holds private graph topology (\eg, edges) while the other owns private node features.
For instance, LinkedIn holds private social relationships between users, while banks own users' private bank statements.
Such real-world graph structures underpin the relevance of our focus.
To our knowledge, no prior work tackles secure GCN training in this context, which is crucial for cross-silo collaboration.


To realize secure GCN over vertically split data, we tailor MPC protocols for sparse graph convolution, which fundamentally involves sparse (adjacency) matrix multiplication.
Recent studies have begun exploring MPC protocols for sparse matrix multiplication (SMM).
ROOM~\cite{ccs/SchoppmannG0P19}, a seminal work on SMM, requires foreknowledge of sparsity types: whether the input matrices are row-sparse or column-sparse.
Unfortunately, GCN typically trains on graphs with arbitrary sparsity, where nodes have varying degrees and no specific sparsity constraints.
Moreover, the adjacency matrix in GCN often contains a self-loop operation represented by adding the identity matrix, which is neither row- nor column-sparse.
Araki~\etal~\cite{ccs/Araki0OPRT21} avoid this limitation in their scalable, secure graph analysis work, yet it does not cover vertical partition.

% and related primitives
To bridge this gap, we propose a secure sparse matrix multiplication protocol, \osmm, achieving \emph{accurate, efficient, and secure GCN training over vertical data} for the first time.

\subsection{New Techniques for Sparse Matrices}
The cost of evaluating a GCN layer is dominated by SMM in the form of $\adjmat\feamat$, where $\adjmat$ is a sparse adjacency matrix of a (directed) graph $\graph$ and $\feamat$ is a dense matrix of node features.
For unrelated nodes, which often constitute a substantial portion, the element-wise products $0\cdot x$ are always zero.
Our efficient MPC design 
avoids unnecessary secure computation over unrelated nodes by focusing on computing non-zero results while concealing the sparse topology.
We achieve this~by:
1) decomposing the sparse matrix $\adjmat$ into a product of matrices (\S\ref{sec::sgc}), including permutation and binary diagonal matrices, that can \emph{faithfully} represent the original graph topology;
2) devising specialized protocols (\S\ref{sec::smm_protocol}) for efficiently multiplying the structured matrices while hiding sparsity topology.


 
\subsubsection{Sparse Matrix Decomposition}
We decompose adjacency matrix $\adjmat$ of $\graph$ into two bipartite graphs: one represented by sparse matrix $\adjout$, linking the out-degree nodes to edges, the other 
by sparse matrix $\adjin$,
linking edges to in-degree nodes.

%\ie, we decompose $\adjmat$ into $\adjout \adjin$, where $\adjout$ and $\adjin$ are sparse matrices representing these connections.
%linking out-degree nodes to edges and edges to in-degree nodes of $\graph$, respectively.

We then permute the columns of $\adjout$ and the rows of $\adjin$ so that the permuted matrices $\adjout'$ and $\adjin'$ have non-zero positions with \emph{monotonically non-decreasing} row and column indices.
A permutation $\sigma$ is used to preserve the edge topology, leading to an initial decomposition of $\adjmat = \adjout'\sigma \adjin'$.
This is further refined into a sequence of \emph{linear transformations}, 
which can be efficiently computed by our MPC protocols for 
\emph{oblivious permutation}
%($\Pi_{\ssp}$) 
and \emph{oblivious selection-multiplication}.
% ($\Pi_\SM$)
\iffalse
Our approach leverages bipartite graph representation and the monotonicity of non-zero positions to decompose a general sparse matrix into linear transformations, enhancing the efficiency of our MPC protocols.
\fi
Our decomposition approach is not limited to GCNs but also general~SMM 
by 
%simply 
treating them 
as adjacency matrices.
%of a graph.
%Since any sparse matrix can be viewed 

%allowing the same technique to be applied.

 
\subsubsection{New Protocols for Linear Transformations}
\emph{Oblivious permutation} (OP) is a two-party protocol taking a private permutation $\sigma$ and a private vector $\xvec$ from the two parties, respectively, and generating a secret share $\l\sigma \xvec\r$ between them.
Our OP protocol employs correlated randomnesses generated in an input-independent offline phase to mask $\sigma$ and $\xvec$ for secure computations on intermediate results, requiring only $1$ round in the online phase (\cf, $\ge 2$ in previous works~\cite{ccs/AsharovHIKNPTT22, ccs/Araki0OPRT21}).

Another crucial two-party protocol in our work is \emph{oblivious selection-multiplication} (OSM).
It takes a private bit~$s$ from a party and secret share $\l x\r$ of an arithmetic number~$x$ owned by the two parties as input and generates secret share $\l sx\r$.
%between them.
%Like our OP protocol, o
Our $1$-round OSM protocol also uses pre-computed randomnesses to mask $s$ and $x$.
%for secure computations.
Compared to the Beaver-triple-based~\cite{crypto/Beaver91a} and oblivious-transfer (OT)-based approaches~\cite{pkc/Tzeng02}, our protocol saves ${\sim}50\%$ of online communication while having the same offline communication and round complexities.

By decomposing the sparse matrix into linear transformations and applying our specialized protocols, our \osmm protocol
%($\prosmm$) 
reduces the complexity of evaluating $\numnode \times \numnode$ sparse matrices with $\numedge$ non-zero values from $O(\numnode^2)$ to $O(\numedge)$.

%(\S\ref{sec::secgcn})
\subsection{\cgnn: Secure GCN made Efficient}
Supported by our new sparsity techniques, we build \cgnn, 
a two-party computation (2PC) framework for GCN inference and training over vertical
%ly split
data.
Our contributions include:

1) We are the first to explore sparsity over vertically split, secret-shared data in MPC, enabling decompositions of sparse matrices with arbitrary sparsity and isolating computations that can be performed in plaintext without sacrificing privacy.

2) We propose two efficient $2$PC primitives for OP and OSM, both optimally single-round.
Combined with our sparse matrix decomposition approach, our \osmm protocol ($\prosmm$) achieves constant-round communication costs of $O(\numedge)$, reducing memory requirements and avoiding out-of-memory errors for large matrices.
In practice, it saves $99\%+$ communication
%(Table~\ref{table:comm_smm}) 
and reduces ${\sim}72\%$ memory usage over large $(5000\times5000)$ matrices compared with using Beaver triples.
%(Table~\ref{table:mem_smm_sparse}) ${\sim}16\%$-

3) We build an end-to-end secure GCN framework for inference and training over vertically split data, maintaining accuracy on par with plaintext computations.
We will open-source our evaluation code for research and deployment.

To evaluate the performance of $\cgnn$, we conducted extensive experiments over three standard graph datasets (Cora~\cite{aim/SenNBGGE08}, Citeseer~\cite{dl/GilesBL98}, and Pubmed~\cite{ijcnlp/DernoncourtL17}),
reporting communication, memory usage, accuracy, and running time under varying network conditions, along with an ablation study with or without \osmm.
Below, we highlight our key achievements.

\textit{Communication (\S\ref{sec::comm_compare_gcn}).}
$\cgnn$ saves communication by $50$-$80\%$.
(\cf,~CoGNN~\cite{ccs/KotiKPG24}, OblivGNN~\cite{uss/XuL0AYY24}).

\textit{Memory usage (\S\ref{sec::smmmemory}).}
\cgnn alleviates out-of-memory problems of using %the standard 
Beaver-triples~\cite{crypto/Beaver91a} for large datasets.

\textit{Accuracy (\S\ref{sec::acc_compare_gcn}).}
$\cgnn$ achieves inference and training accuracy comparable to plaintext counterparts.
%training accuracy $\{76\%$, $65.1\%$, $75.2\%\}$ comparable to $\{75.7\%$, $65.4\%$, $74.5\%\}$ in plaintext.

{\textit{Computational efficiency (\S\ref{sec::time_net}).}} 
%If the network is worse in bandwidth and better in latency, $\cgnn$ shows more benefits.
$\cgnn$ is faster by $6$-$45\%$ in inference and $28$-$95\%$ in training across various networks and excels in narrow-bandwidth and low-latency~ones.

{\textit{Impact of \osmm (\S\ref{sec:ablation}).}}
Our \osmm protocol shows a $10$-$42\times$ speed-up for $5000\times 5000$ matrices and saves $10$-2$1\%$ memory for ``small'' datasets and up to $90\%$+ for larger ones.

\section{Related Works}

\textbf{Enhancing LLMs' Theory of Mind.} There has been systematic evaluation that revealed LLMs' limitations in achieving robust Theory of Mind inference \citep{ullman2023large, shapira2023clever}. To enhance LLMs' Theory of Mind capacity, recent works have proposed various prompting techniques. For instance, SimToM \citep{wilf2023think} encourages LLMs to adopt perspective-taking, PercepToM \citep{jung2024perceptions} improves perception-to-belief inference by extracting relevant contextual details, and \citet{huang2024notion} utilize an LLM as a world model to track environmental changes and refine prompts. Explicit symbolic modules also seem to improve LLM's accuracy through dynamic updates based on inputs. Specifically, TimeToM \citep{hou2024timetom} constructs a temporal reasoning framework to support inference, while SymbolicToM \citep{sclar2023minding} uses graphical representations to track characters' beliefs. Additionally, \citet{wagner2024mind} investigates ToM's necessity and the level of recursion required for specific tasks. However, these approaches continue to exhibit systematic errors in long contexts, complex behaviors, and recursive reasoning due to inherent limitations in inference and modeling \citep{jin2024mmtom,shi2024muma}. Most of them rely on domain-specific designs, lacking open-endedness.


\textbf{Model-based Theory of Mind inference.} Model-based Theory of Mind inference, in particular, Bayesian inverse planning (BIP) \citep{baker2009action,ullman2009help,baker2017rational,zhi2020online}, explicitly constructs representations of agents' mental states and how mental states guide agents' behavior via Bayesian Theory of Mind (BToM) models. These methods can reverse engineer human ToM inference in simple domains \citep[e.g.,][]{baker2017rational,netanyahu2021phase,shu2021agent}. Recent works have proposed to combine BIP with LLMs to achieve robust ToM inference in more realistic settings \citep{ying2023neuro, jin2024mmtom, shi2024muma}. However, these methods require manual specification of the BToM models as well as rigid, domain-specific implementations of Bayesian inference, limiting their adaptability to open-ended scenarios. To overcome this limitation, we propose \ours, a method capable of automatically modeling mental variables across diverse conditions and conducting automated BIP without domain-specific knowledge or implementations.


\begin{figure*}[ht]
  \centering
  \includegraphics[width=\linewidth]{figures/benchmarks_and_models.pdf}
    \vspace{-15pt}
  \caption{Examples questions (top panels) and the necessary Bayesian Theory of Mind (BToM) model for Bayesian inverse planning (bottom panels) in diverse Theory of Mind benchmarks. \ours aims to answer any Theory of Mind question in a variety of benchmarks, encompassing different mental variables, observable contexts, numbers of agents, the presence or absence of utterances, wording styles, and modalities. It proposes and iteratively adjusts an appropriate BToM and conducts automated Bayesian inverse planning based on the model.
  There can be more types of questions/models in each benchmark beyond the examples shown in this figure.}
  \label{fig:benchmarks_and_models}
  %\vspace{-0.75em}
  \vspace{-10pt}
\end{figure*}



\textbf{Automated Modeling with LLMs.} There has been an increasing interest in integrating LLMs with inductive reasoning and probabilistic inference for automated modeling. \citet{piriyakulkij2024doing} combine LLMs with Sequential Monte Carlo to perform probabilistic inference about underlying rules. Iterative hypothesis refinement techniques \citep{qiu2023phenomenal} further enhance LLM-based inductive reasoning by iteratively proposing, selecting, and refining textual hypotheses of rules. Beyond rule-based hypotheses, \citet{wang2023hypothesis} prompt LLMs to generate natural language hypotheses that are then implemented as verifiable programs, while \citet{li2024automated} propose a method in which LLMs construct, critique, and refine statistical models represented as probabilistic programs for data modeling. \citet{cross2024hypothetical} leverage LLMs to propose and evaluate agent strategies for multi-agent planning but do not specifically infer individual mental variables. Our method also aims to achieve automated modeling with LLMs. Unlike prior works, we propose a novel automated model discovery approach for Bayesian inverse planning, where the objective is to confidently infer any mental variable given any context via constructing a suitable Bayesian Theory of Mind model.
% \section{Problem Definition}

% \begin{itemize}
%     \item Application context: FT LLM to respond to custom tasks
%     \begin{itemize}
%         \item Summarization
%         \item extraction
%         \item QA
%     \end{itemize}
%     \item Weak supervision: Only few (hundreds) human annotation available
%     \item Harder tasks: are important
%     \item Controlled data-visibility: no domain heuristics can be used
%     \begin{itemize}
%         \item Concern: people may challenge whether we are still exposing visibility to trusted experts
%         \item why data masking won't work: enterprise task are complex
%     \end{itemize}
% \end{itemize}

% story: 
% If you are given a dataset, you will:
% get large number of annotations (x)
% -> generalize data processing from few samples
% -> SER: pairing, not generalizable; don't go for harder samples; bad performance
% -> Take data science teams (which you may not have to find representative sample for FT
% -> read through the dataset (x, controlled data visibility)
% -> subject matter expert doesn't have DS background
% -> incorporate their feedback
% -> Sargy
% (
% 1. weak supervision: starting GPT initial alignment, extending to similar samples
% 2. hard tasks: RM reranking + SME
% 3. controlled visibility: working with SME
% )

% (include the meta-correction: use top-bottom annotate to tune the guidelines)

% (implicit supervision: middle point)

% (look at how reward is calculated for a long sequence)

% tasks:
% 1. writing
% 2. find the dataset
% 3. fix extraction


\section{Improving Human Alignment with \myname{}}
\label{sec:design:overview}

\myname{} enhances human alignment in preference datasets used for training preference optimization techniques like DPO and PPO. It facilitates LLM training for various downstream tasks, including summarization, compliance, and grounding. Starting with an unlabeled preference dataset, \myname{} strategically integrates AI-generated labels with selective human feedback to maximize alignment while minimizing annotation effort. As illustrated in Figure~\ref{fig:overview}, \myname{} operates in three stages: 1) \textit{Initial alignment}, where an off-the-shelf LLM provides dataset labeling to establish a coarse task understanding, 2) \textit{Iterative alignment improvement}, which leverages reward distribution by an RM to locate and rectify hard-to-annotate samples mislabeled by the LLM with selective human feedback while investing the correct LLM labels, 3) \textit{Transferring knowledge for downstream task}, where the curated preference dataset is fed into the DPO pipeline or the trained \myname{} reward model is integrated into the PPO pipeline.

% Specially, for text extraction, \myname{} separates the task into two steps: Filtering step applies the initial alignment and progressive hard-problem resolution; and the learned filtering RM is re-purposed to provide high-quality extractions.


\subsection{Initial Alignment}
\label{sec:design:init}

This stage aims to establish an initial coarse alignment in the unlabeled dataset using a general-purpose LLM, which provides preference annotations for each unannotated sample. Prior research suggests that model selection here depends on task complexity relative to the model's capability~\cite{snell2024scaling}. While \myname{} is not found to be sensitive to the choice of model at this stage, a well-suited model can accelerate alignment convergence. The only assumption is that the general-purpose LLM possesses a basic understanding of the downstream task, enabling it to provide a rough initial alignment that serves as a seed for \myname{}.     

Our prompt for obtaining preference judgments from the LLM consists of three key components: 1) task description, 2) preference judgment principles provided by the end user, and 3) few-shot examples with optional chain-of-thought (CoT) reasoning. The prompt templates are detailed in Appendix~\ref{appendix:prompt}. We do not perform explicit fine-grained prompt tuning, as full visibility into the data may be restricted when offering fine-tuning services to third-party customers. However, to ensure that the selected LLM with our prompt attains a reasonable level of alignment, we perform an eyes-off validation using strategic human feedback, as detailed in Section~\ref{sec:leveraging_reward_score}.   

As mentioned earlier, AI-generated feedback is prone to errors due to factors such as model biases from pre-training data, task complexity, and prompt optimization, which is also evident in our evaluation. When our ultimate goal is to customize an existing model through fine-tuning to align with end-user preferences, we inherently assume that an off-the-shelf LLM lacks comprehensive alignment with the end-user. However, \myname{} builds upon the initial AI-provided alignment and systematically refines it in subsequent stages to achieve oracle-level human alignment. 

% We form the user-provided components into fill-in-the-blank style to ensure the prompt template is generalizable for various tasks and provide user with proper guidance. We provide the prompt templates in Appendix~\ref{appendix:prompt}.

% With the preferences/judgments are obtained for all unannotated data samples, we train a initial RM to align with the general-purpose language model on this task. The training data preparation will be the same as the processes described in Section~\ref{sec:design:pref_learn} and ~\ref{sec:design:doc_filter} except that there is no human feedback.

% \subsection{Progressive Hard-Problem Resolution}
% \label{sec:design:progressive}
% \begin{itemize}
%     \item Re-ranking by reward scores
%     \item Preference grouping by detecting inflection point
%     \item Annotation sampling with confidence margin: 2 hyperparams introduced
%     \begin{itemize}
%         \item middle ratio: determined by the level of confidence for each iteration
%         \item sample ratio: determined by the budget for human annotation
%     \end{itemize}
%     \item Lightweight human annotation
%     \begin{itemize}
%         \item Absolute
%         \item Preference
%     \end{itemize}
% \end{itemize}

% The major issue of the initial RM is that it may not capture the hard problems well as the general-purpose may be making wrong preferences/judgments on those samples. Therefore, the next goal is to fully exploit the limited feedback from trusted human experts and teach the model how to solve the hard problems. To achieve this, \myname{} divides the training samples into a \textit{sanitized group} and \textit{confused group} by learning from the reward distribution of training samples. The sanitized group contains the samples that have already received confident annotations. The confused group contains the hard samples that will be partially sent for human annotation. After that, the two groups will be combined and reorganized into the new training set for the next iteration to progressively infuse deeper knowledge to the RM. The detailed procedure are different for preference learning (for instruction tuning) and document filtering (for text extraction) but follow the same principles.

\subsection{Iterative Alignment Improvement}
\label{sec:design:pref_learn}
In this stage, we refine the LLM-labeled preference dataset by iteratively training a reward model (RM) with selective human annotations to enhance alignment. Before diving into the details of this process, we first establish the premise for RM.

\subsubsection{Reward Model}
 Given a labeled preference dataset $\mathcal{D}_{\mathbf{\Lambda}} = \{x_i, y_{i,c}, y_{i,r} \}$, where $i\in [N]$, $x_i$ is the prompt, $y_{i,c}$ and $y_{i,r}$ denote the chosen and rejected completions, respectively, as labeled according to the annotator's preference, $\mathbf{\Lambda}$. Here, if we represent the relative preference orientation of $i^{th}$ completion pair with $\lambda = [-1, +1 ]$, $\mathbf{\Lambda}$ is a $N$-dimensional vector consists of $[\lambda_i]_{i=1}^{N}$, meaning that flipping the preferences of all completion pairs results in $\mathcal{D}_{\mathbf{-\Lambda}}$. To train an RM on this dataset, we can formulate the probability distribution of $y_{i,c}$ being preferred over $y_{i,r}$ given $x_{i}$ as an input, following the Bradley-Terry (BT) model~\cite{david1963method}.
\begin{equation}
    P(x\succ y) = \sigma(r(x_i, y_{i,c}) - r(x_i, y_{i,r}))
\label{eq:rm_prob}
\end{equation}
where $\sigma(\cdot)$ denotes the sigmoid function and $r(\cdot)$ denotes the reward function. Assuming the existence of a true deterministic reward function, the goal is to train the RM to learn this function and predict the reward, $\hat{r}(x,y)$. The RM training can be framed as a binary classification problem~\cite{sun2024rethinking}, where a labeled pair of $\rho_{i,c}\coloneqq(x_i, y_{i,c})$ and  $\rho_{i,r}\coloneqq(x_i, y_{i,r})$ is passed to the model to predict the conditional class probability according to Eq.~\ref{eq:rm_prob}. This leads to the negative log-likelihood loss function for training.
\begin{equation}
    \mathcal{L(\hat{\text r})} = - \mathbb{E}_{(x,y)\sim\mathcal{D}} [\log{\sigma(\hat{r}(\rho_{i, c}) - \hat{r}(\rho_{i,r}))}] 
\end{equation}

In essence, during the RM training, we pass a preference pair $\{\rho_{i,c}, \rho_{i,r}\}$ labeled as $\rho_{i,c}$ winning over $\rho_{i,r}$ according to the annotator's preference $\mathbf{\Lambda}$. Provided sufficient preference samples in a dataset, the RM learns the winning preference features of the data that determine the winner in a pair, captured in the reward function $\hat{r}_{\mathbf{\Lambda}}$.
\begin{figure*}[t]
\centering
\begin{subfigure}{0.23\linewidth}
\centering
\includegraphics[width=\linewidth]{figures/reward_curve_hh_itr0.png}
\caption{Reward dist. : Itr-0}
\label{fig:hh_itr0_reward_curve}
\end{subfigure}
\begin{subfigure}{0.23\linewidth}
\centering
\includegraphics[width=\linewidth]{figures/accuracy_curve_hh_itr0.pdf}
\caption{Correctness dist. : Itr-0}
\label{fig:hh_itr0_accuracy_curve}
\end{subfigure}
\begin{subfigure}{0.23\linewidth}
\centering
\includegraphics[width=\linewidth]{figures/reward_curve_hh_itr5.png}
\caption{Reward dist. : Itr-5}
\label{fig:hh_itr5_reward_curve}
\end{subfigure}
\begin{subfigure}{0.23\linewidth}
\centering
\includegraphics[width=\linewidth]{figures/accuracy_curve_hh_itr5.pdf}
\caption{Correctness dist. : Itr-5}
\label{fig:hh_itr5_accuracy_curve}
\end{subfigure}
\caption{Reward (assigned by a trained RM) and correctness (w.r.t. human preference) distribution curves for the very first and last iterations of \myname{}. These two types of curves provide the intuition of strategically selecting the samples for efficient human annotation towards improving alignment in the dataset. These curves further highlight the iterative refinement process, showing how alignment in the dataset progressively improves.}\vspace{-0.2in}
\label{fig:hh_reward_and_accuracy_curve}
\end{figure*}

\subsubsection{Looking at Reward Distribution}
\label{sec:leveraging_reward_score}
At this stage, we analyze the distribution of the predicted reward function ($\hat{r}_{\mathbf{\Lambda}}$) within the training preference dataset $\mathcal{D}_{\mathbf{\Lambda}}$. For each labeled preference pair $\{\rho_{i,c}, \rho_{i,r}\}$, we compute the reward score difference as $ \Delta_{\mathbf{\Lambda}}{\hat{r}_\mathbf{\Lambda}} = (\hat{r}_{\mathbf{\Lambda}}(\rho_{c}) - \hat{r}_{\mathbf{\Lambda}}(\rho_{r}))$. It is important to note that $\Delta_{\mathbf{\Lambda}}$ quantifies the relative preference score of a given pair in alignment with the annotator's preference orientation $\mathbf{\Lambda}$, satisfying the property $\Delta_{\mathbf{\Lambda}}{r} = - \Delta_{\mathbf{-\Lambda}}{r}$. By ranking all preference pairs in $\mathcal{D}_{\mathbf{\Lambda}}$ based on $\Delta_{\mathbf{\Lambda}}{\hat{r}_\mathbf{\Lambda}}$, a monotonic reward distribution curve, denoted as $\vartheta(\Delta_{\mathbf{\Lambda}}{\hat{r}_\mathbf{\Lambda})}$, emerges. This distribution, as depicted in Figure~\ref{fig:hh_itr0_reward_curve}, provides insight into the model’s reward assignment across the dataset, though for the moment, the legend in the graph can be disregarded. 

The reward distribution curve $\vartheta(\Delta_{\mathbf{\Lambda}}{\hat{r}_\mathbf{\Lambda})}$, derived from the training preference dataset $\mathcal{D}_{\mathbf{\Lambda}}$, reflects the degree of alignment the RM (trained with optimal validation loss) has achieved during training across $\mathcal{D}_{\mathbf{\Lambda}}$. The upper left region of the curve consists of samples with high positive $\Delta_{\mathbf{\Lambda}}\hat{r}_\mathbf{\Lambda}$, indicating strong agreement between the RM and the training preference labels $\mathbf{\Lambda}$. This suggests that the RM effectively identifies and reinforces strong winning preference features in these samples, implying that these features were dominant in $\mathcal{D}_{\mathbf{\Lambda}}$. Conversely, the bottom right region of the curve contains samples with very low or even negative $\Delta_{\mathbf{\Lambda}}\hat{r}_\mathbf{\Lambda}$, signaling disagreement between the trained reward function $\hat{r}_\mathbf{\Lambda}$ and the training preference label for these samples. This misalignment arises from two primary factors. (1) Absence of strong features, where RM is not able to find any strong preference feature in these samples according to $\hat{r}_\mathbf{\Lambda}$. (2) Conflicting samples within $\mathcal{D}_{\mathbf{\Lambda}}$, where the preference features of these samples are highly conflicting with other stronger preference features learned in $\hat{r}_\mathbf{\Lambda}$, leading the RM to penalize them.

\subsubsection{\myname{} Leveraging Reward Distribution}
\label{sec:leveraging_reward_score}
\myname{} trains the initial RM using a preference dataset filtered by a general-purpose LLM in the previous stage. We denote this dataset as $\mathcal{D}_{\mathbf{\Lambda_{LLM}}}$ where $\mathbf{\Lambda}_{LLM}$ represents the preference labeling performed by the LLM. Since the RM training includes a validation set derived from $\mathcal{D}_{\mathbf{\Lambda_{LLM}}}$, this ensures that the trained RM is broadly aligned with the LLM’s preferences. We assume that the LLM has a coarse but reasonable understanding of preference judgments, particularly for relatively easy-to-annotate samples. As a result, the features of these samples dominate in $\hat{r}_\mathbf{\Lambda_{LLM}}$. Based on our earlier discussion, the upper left region of the reward density curve, $\vartheta(\Delta_{\mathbf{\Lambda_{LLM}}}{\hat{r}_\mathbf{\Lambda_{LLM}})}$ contains high density of samples with prominent preference features, i.e., those that are easier for the LLM to annotate accurately. Before proceeding, we further validate that the LLM is at least roughly aligned with the end-user’s preferences in terms of these easy-to-annotate samples. This step mitigates the risk of significant misalignment due to prompt curation or model selection. To achieve this, \myname{} automatically (details in the following section) samples a small subset ($<0.1\%$) of preference data from the upper left region and gathers eyes-off user feedback. If human agreement on these samples is low, it signals a major misalignment between the user’s preferences and the LLM’s outputs. While we did not observe such cases in our experiments, this issue can be addressed by refining the judgment principles in the prompt. Updates can be made directly by the user, by incorporating verbose user feedback, or even through automated prompt optimization techniques~\cite{kepel2024autonomous, li2024learning}.

At this stage, we can identify regions with a high density of correctly labeled samples by the LLM, i.e., those that are relatively easy for the LLM to annotate in alignment with human preference. Now, we turn our attention to two critical types of samples necessary for achieving fine-grained alignment: (1) hard-to-annotate samples and (2) samples mislabeled by the LLM w.r.t. the human preference $\mathbf{\Lambda}_h$. Since the LLM was unable to correctly label these samples initially, the reward function $\hat{r}_\mathbf{\Lambda_{LLM}}$ cannot accurately capture their preference features. Consequently, these samples are expected to cluster around the bottom right region of the reward distribution curve $\vartheta(\Delta_{\mathbf{\Lambda_{LLM}}}{\hat{r}_\mathbf{\Lambda_{LLM}})}$. To illustrate this, we refer to Figure~\ref{fig:hh_itr0_reward_curve} and \ref{fig:hh_itr0_accuracy_curve}. Figure~\ref{fig:hh_itr0_reward_curve} shows $\vartheta(\Delta_{\mathbf{\Lambda_{LLM}}}{\hat{r}_\mathbf{\Lambda_{LLM}})}$ from one of our experiments. In this figure, we classify each sample $\rho_i \in \mathcal{D}_{\mathbf{\Lambda_{LLM}}}$ as either correctly or incorrectly labeled w.r.t. the human preference $\mathbf{\Lambda_{h}}$, i.e., whether the preference assigned by $\mathbf{\Lambda_{LLM}}$ is matching $\mathbf{\Lambda_{h}}$.
As observed, the upper left region of the curve contains a high density of correctly labeled samples, supporting our earlier claim that these represent the LLM’s easy-to-annotate cases. To quantify this, we generate an accuracy density curve for $\vartheta(\Delta_{\mathbf{\Lambda_{LLM}}}{\hat{r}_\mathbf{\Lambda_{LLM}})}$ w.r.t. the full-human preference $\mathbf{\Lambda_{h}}$, as shown in Figure~\ref{fig:hh_itr0_accuracy_curve}. This figure confirms that alignment with human preference decreases as we move towards the right side of the curve.   

While we can observe how alignment with $\mathbf{\Lambda_{h}}$ varies across the reward distribution curve, in real-world scenarios, we lack ground-truth labels to quantify this accuracy directly. Therefore, we need to \textit{estimate} the boundaries of key regions within the curve. To achieve this, we identify two strategic points: the ``elbow'' and the ``knee'', as illustrated in Figure~\ref{fig:hh_itr0_reward_curve}. These points correspond to sharp changes in $\Delta_{\mathbf{\Lambda_{LLM}}}{\hat{r}_\mathbf{\Lambda_{LLM}}}$, which we detect using the first-order derivative. The ``knee'' marks the transition to a region with lower accuracy density, whereas the ``elbow'' indicates a shift toward higher accuracy density. It is important to note that these points serve only as rough estimations of the region boundaries rather than precise demarcations.

\subsubsection{Selective Human Annotation}
To enhance alignment from this stage, human annotation is necessary, but it must be done efficiently to maximize its impact. A straightforward approach is to refer to the accuracy density curve—annotations in the lowest accuracy region will yield the highest benefit. Thus, we could start annotating from the very bottom of the curve. However, as previously discussed, some samples in this region may exhibit preference features that are largely opposite to the dominant features captured by $\hat{r}_\mathbf{\Lambda_{LLM}}$. These samples are highly likely to be mislabeled in $\mathbf{\Lambda_{LLM}}$ (see Appendix~\ref{appendix:iterative_improvement}). Instead of requiring human annotation, we can simply flip the preference of these samples to correct the mislabeling. To estimate the positions of such samples, we take the reflection of the ``elbow'' point across the x-axis, as the elbow marks the region containing strong preference features. This reflected point, known as the ``reflection point'', always lies to the right of the ``knee'' in the lowest accuracy density region. Human annotation then begins from the reflection point, ensuring the most effective correction of alignment errors.

\subsubsection{Iterative Approach}
\label{sec:design:improve:iter}
The current reward function $\hat{r}_\mathbf{\Lambda_{LLM}}$, trained on $\mathcal{D}_{\mathbf{\Lambda_{LLM}}}$, exhibits an alignment gap with respect to $\mathbf{\Lambda_{h}}$ due to the presence of hard-to-annotate samples for the LLM and mislabeling by the LLM. Since we have identified ways to rectify these issues, we can refine $\mathcal{D}$ to improve alignment and train a new RM that better aligns with $\mathbf{\Lambda_{h}}$. Now, the question is how to prepare the dataset for the next iteration of RM training. Suppose we are currently in iteration 0 (Itr-0) with $\mathcal{D}_{\mathbf{\Lambda_{LLM}}}$ and $\hat{r}_\mathbf{\Lambda_{LLM}}$. For the iteration 1 (Itr-1) training dataset, $\mathcal{D}_{\mathbf{\Lambda_{Itr-1}^{T}}}^{T}$, our primary goal is to include high-confidence samples that are well-aligned with $\mathbf{\Lambda_{h}}$. The first choice is definitely human annotated samples from Itr-0.  Additionally, another set of candidates can be drawn from the high-accuracy density region of $\vartheta(\Delta_{\mathbf{\Lambda_{LLM}}}{\hat{r}_\mathbf{\Lambda_{LLM}})}$, specifically the region to the left of the ``elbow'', where the RM has learned strong preference features in alignment with $\mathbf{\Lambda_{h}}$.

Although these two sets of samples offer high precision, $\mathcal{D}_{\mathbf{\Lambda_{Itr-1}^{T}}}^{T}$ will still face a data coverage issue. These two candidate sets represent samples with the longest feature distance, leaving gaps in intermediate regions. However, expanding the dataset by including samples from the middle region, i.e., right of the ``elbow'' and left of the ``knee'' risks introducing misaligned samples. Since the accuracy in this region is likely to be just above $50\%$, obtaining human annotations for these samples would be inefficient. 
Furthermore, as the number of samples annotated from the right of the knee is relatively small, their preference features are likely to be overshadowed by the dominant preference features of the high numbers of left-side samples. As a result, their features may not be effectively captured in $\hat{r}_\mathbf{\Lambda_{Itr-1}^{T}}$. To balance these trade-offs, we introduce two hyperparameters specific to \myname{}, allowing for a more controlled and effective dataset expansion while maintaining alignment quality.

\begin{itemize}[leftmargin=*,topsep=2pt]
    \item \textbf{Back-off ratio ($\beta$)}: Determines how far to back off from the ``knee'' when selecting samples for the next iteration's dataset. A higher $\beta$ results in a more sanitized dataset, reducing noise but at the expense of lower data coverage.  
    \item \textbf{Amplification ratio ($\alpha$)}: Increases the influence of human-annotated samples by repeating them in the dataset, reinforcing their preference features in $\hat{r}_\mathbf{\Lambda_{Itr-1}^{T}}$. However, an excessively high $\alpha$ may lead to overfitting to selective human annotations.
\end{itemize}

The dataset $\mathcal{D}_{\mathbf{\Lambda_{Itr-1}}}^{T}$ consists of carefully selected samples from $\mathcal{D}_{\mathbf{\Lambda_{LLM}}}$, ensuring high alignment with $\mathbf{\Lambda_{h}}$ by optimally tuning the hyperparameters $\alpha$ and $\beta$. Training the RM on $\mathcal{D}_{\mathbf{\Lambda_{Itr-1}^{T}}}^{T}$ results in $\hat{r}_\mathbf{\Lambda_{Itr-1}^{T}}$, which is more closely aligned with $\mathbf{\Lambda_{h}}$. After training, we construct the dataset for generating the reward distribution curve by incorporating the remaining samples from Itr-0: $\mathcal{D}_{\mathbf{\Lambda_{Itr-1}}} = \mathcal{D}_{\mathbf{\Lambda_{Itr-1}^{T}}}^{T} \cup (\mathcal{D}_{\mathbf{\Lambda_{LLM}}} - \mathcal{D}_{\mathbf{\Lambda_{Itr-1}^{T}}}^{T})$. From this, we generate a new reward distribution curve, $\vartheta(\Delta_{\mathbf{\Lambda_{Itr-1}}}{\hat{r}_\mathbf{\Lambda_{Itr-1}^{T}})}$. While this curve demonstrates improved alignment with $\mathbf{\Lambda_{h}}$, full alignment is not necessarily achieved. However, it presents \myname{} with a distinct reward distribution curve compared to the previous iteration. This evolving diversity in $\vartheta(\cdot)$ enhances the variety of human annotations, maximizing the return on annotation investments and incrementally enriching $\mathcal{D}$. It is important to note that the effectiveness of this diversification, as well as the corresponding improvements, depends on factors such as hyperparameter tuning (see Section~\ref{sec:results}), the original data distribution, and model selection.

As we have seen, \myname{} maximizes the efficiency of human annotations by iteratively refining $\vartheta(\cdot)$ and exposing annotators to diverse, LLM-mislabeled samples. To further enhance annotation efficiency, \myname{} employs random sharding to down-sample the original corpus. It begins by selecting a random shard of the dataset, iteratively improving alignment within that subset. Once the desired alignment is achieved, the final iteration's RM is used to label the entire corpus. This approach enables \myname{} to concentrate human annotations in a smaller, more targeted space while effectively propagating alignment across the full dataset at the end. 

\vspace{-0.1in}
\subsection{Reward Knowledge Transfer}
\vspace{-0.05in}

\myname{} progressively converges toward the comprehensive human preference through iterative RM training and strategic human annotation investment. As shown in Figure ~\ref{fig:hh_itr5_reward_curve} and ~\ref{fig:hh_itr5_accuracy_curve}, after five iterations, the reward distribution and accuracy curves closely align with the full-human annotation. Intermediate iteration curves can be found in Appendix~\ref{appendix:iterative_improvement}. The required number of iterations depends on the available human annotation and RM training budget. Notably, full-human alignment can sometimes be achieved before exhausting the annotation budget. In such cases, the samples selected for human annotation would largely lack distinct preference features, indicating that the model has effectively captured the  human preference. Once desired alignment is achieved or the annotation budget is fully utilized, we proceed with fine-tuning the model for the downstream task. This can be done in two ways: 1) incorporating the final iteration RM into the PPO loop, or 2) labeling the whole dataset with the final RM and feeding the labeled dataset to a DPO pipeline.

% In this task, each input sample is a preference pair. The goal is to obtain an RM that can output higher reward score for the preferred response. \myname{} ranks all training samples by the reward gap between the chosen and rejected response $RM(Q,A_\text{chosen})-RM(Q,A_\text{rejected})$, and draw the gap-rank curve (example illustrated in Figure~\ref{fig:design:pref_curve}). 
% We have some key observations on the curve:
% \begin{enumerate}
%     \item The samples on the right of the "knee" point contains dense samples wrongly annotated in previous training.
% \end{enumerate} 
% \myname{} detects two strategic points on this curve to determine the sanitized group and confused group:
% \begin{itemize}
%     \item \textbf{L Margin}: This is the point dividing the samples with confidently correct annotations. \myname{} determines its position by detecting the "knee" on the curve.
%     \item \textbf{R Margin}: This is the point dividing the samples with confidently wrong annotations. \myname{} determines its reference position by reflect the "elbow" w.r.t. the "knee" on the curve.
% \end{itemize}


% \begin{figure}
%     \centering
%     \includegraphics[width=0.8\linewidth]{figures/pref_curve.png}
%     \caption{Example reward curve for preference learning.}
%     \label{fig:design:pref_curve}
% \end{figure}

% The sanitize group is constructed by the samples with their original preference labels on the left of L Margin plus the samples with flipped preference labels on the right of R Margin. The confused group is constructed by the samples between the L and R margin. Since the confused group mainly contains the confusing samples that are hard to be self-resolved by models, \myname{} randomly selects a few samples and sends it for human annotation.

% \textbf{Why knee and elbow?}
% The knee separate the samples that cannot effectively converge with a low loss during the reward training from others. The underlying reason is that they are disagreed by the rules learned from most other samples (i.e., the main cluster in the "middle platform" on the curve), and therefore receiving great resistance in leading the model weights to propagate towards its direction during training. The elbow works vice versa.

% \subsubsection{Document Filtering}
% \label{sec:design:doc_filter}
% In this task, each input sample is a document. The goal of the filtering step is to figure out whether each document is good (contains the text of interest) or bad (does not contain the text of interest). \myname{} proposes to solve it with an RM that judge the existence of text of interest in a document with its reward score, where a similar pipeline can be applied. However, there are critical unsolved challenges: 1) the documents are unpaired and the good/bad ratio may be far from 1:1; 2) RM does not inherently provide a clear bar of what is good and bad. To tackle these issues, \myname{} ranks all training samples by the absolute reward score of each document $RM(D)$, and draw the reward-rank curve (example illustrated in Figure~\ref{fig:design:filter_curve}). 
% \myname{} detects three strategic points on this curve to determine the sanitized group and confused group:
% \begin{itemize}
%     \item \textbf{L Margin}: This is the point dividing the confidently good documents. \myname{} determines its position by detecting the "left knee" on the curve.
%     \item \textbf{Middle Point}: This is the dynamic bar for good/bad document split. \myname{} determines it by detecting the middle inflection point.
%     \item \textbf{R Margin}: This is the point dividing the samples with confidently bad documents. \myname{} determines its position by detecting the "right elbow" on the curve.

% \end{itemize}

% \begin{figure}
%     \centering
%     \includegraphics[width=0.8\linewidth]{figures/filter_curve.png}
%     \caption{Example reward curve for document filtering.}
%     \label{fig:design:filter_curve}
% \end{figure}

% The sanitize group is constructed by a pool of good documents on the left of L Margin plus pool of bad documents on the right of R Margin. The confused group is constructed by the samples between the L and R margin. Again, the confused group is randomly sampled is sent for human annotation and labeled with good or bad.

% \textbf{Why left knee and right elbow?}
% The knees and elbows are formed in the same way as in the preference learning case. However, in document filtering, there will be two clusters of documents, one good and one bad, which shape the curve with two knees. Among them, the left knee is what differentiate the good cluster from others, and the right elbow works vice versa.

% \textbf{Why middle inflection point?}
% The middle inflection point marks the sparsest point between the good cluster and bad cluster. As the training progresses, the good/bad documents gather toward high/low reward direction, with the middle inflection point naturally becomes the bar for differentiating good and bad.

% Specially, in document filtering case, the sanitized and confused groups cannot be directly used for reward training. To produce the training pairs, \myname{} introduces a document re-pairing process which randomly selects documents from the good and bad pools while ensuring: 1) each document appears at least once, 2) all documents has (almost) equal appearance time in the training set.

% \subsubsection{Human Feedback Amplification}
% For both use cases, the annotated samples from trusted human experts are considered the golden input, while their proportion in the training set is small. To ensure the human feedback is playing enough role in the reward training process, \myname{} repeat all human-annotated preference pairs (for preference learning) or document pairs that include at least one human-labeled document (for document filtering) with by a pre-configured number to amplify their knowledge.

% \subsubsection{Instruction Tuning}
% For general instruction tuning, the RM learned from preference learning is directly ready for PPO to fine-tune an instruction-tuned language model. We adopts the standard PPO training process here~\cite{}.

% \subsubsection{Text Extraction}
% For text extraction, we repurpose the reward signal from the RM for document filtering into a new reward for extraction for each document $D$ and extraction $E$:
% \begin{equation}
%     R_\text{extract}(D,E) = \frac{RM_\text{filter}(D) - RM_\text{filter}(D-E)}{length(E)}
% \end{equation}
% where $D-E$ stands for the remain text after removing $E$ from $D$. The $RM_\text{filter}(D) - RM_\text{filter}(D-E)$ component rewards for complete extraction and penalizes missing extraction where partial text of interest remain in $D-E$ and leads to a high $RM_\text{filter}(D-E)$ value. The $length(E)$ component penalizes unnecessary extraction on the other hand. Theoretically, with an ideal $RM_\text{filter}$ that provides higher reward for more confident existence of text of interest, it is impossible to find an alternative extraction $E_\text{alt}$ that is preferred over the ground truth extraction $E_\text{gt}$, \textit{i.e.}, $R_\text{extract}(D,E_\text{alt}) > R_\text{extract}(D,E_\text{gt})$. 

% Practically, the extraction from a language model $E_\text{LM}$ may not be faithful to the raw document. To determines $E$, \myname{} uses a sliding window with the same length (in tokens) as $E_\text{LM}$ and let $E$ be the window with the minimum Levenshtein distance to $E_\text{LM}$.

% The repurposed reward signal can be used for PPO to train a language model dedicated for text extraction, or used with a general purpose LM for better sampling~\cite{}.

% \subsection{Iterative Reward Learning}
% \begin{itemize}
%     \item Preference re-pairing
%     \begin{itemize}
%         \item Merging human annotation with confident groups
%         \item Random matching but guarantee each good/bad has equal change of appearance in the training set
%     \end{itemize}
%     \item Hyperparameter determination?
%     \item Stop condition: implicit supervision from human annotated samples 
% \end{itemize}


% \subsection{Extraction (?)}
\section{Evaluation}
\label{sec:Evaluation}

We evaluate \sys from the following aspects:
First, we perform a large-scale comparison of \sys with both SOTA open-source and closed-source methods on the SWE-Bench-Lite and SWE-Bench-Verified patching benchmark~\cite{jimenez2023swe}, showcasing \sys's ability to balance patching accuracy and cost-efficiency.
Second, we conduct a stability analysis on \sys and \openhands, demonstrating~\sys{}'s human-based planning is more stable than the SOTA agent-based planning. 
Third, we conduct an ablation study to quantify the contribution of each component to \sys's overall performance.
Finally, we show \sys's compatibility and performance on different models, including \gpt~\cite{GPT-4o}, \claude~\cite{anthropic_claude}, and a reasoning model \oo~\cite{GPT-o3}. 
We failed to integrate \deepseek~\cite{DeepSeek-r1} due to the problems with their APIs (See~\cref{appx:exp4}).
% In the following, we specify the setup and design of each experiment and discuss their results.

\subsection{\sys vs. Baselines on SWE-Bench}
\label{exp:comparison}

\noindent\textbf{Setup and design.}
We utilize the \textit{SWE-Bench}~\cite{jimenez2023swe} benchmark, where each instance corresponds to an issue in a GitHub repository written in \texttt{Python}.
Specifically, we consider two subsets: \textit{SWE-Bench-Lite}~\cite{SWE-Bench-Lite}, consisting of 300 instances, and \textit{SWE-Bench-Verified}~\cite{SWE-Bench-Verified}, comprising 500 instances that have been verified by humans to be resolvable.

We mainly compare \sys with three SOTA open-source methods: two agent-based planning methods \openhands~\cite{wang2024openhands} and \autocode~\cite{zhang2024autocoderover}, and a human-based planning method \agentless~\cite{xia2024agentless}. 
We also compare it with two closed-source methods that have cost reported: Globant Code Fixer Agent~\cite{Globant_Code_Fixer_Agent} (\globant for short) and CodeStory Midwit Agent~\cite{CodeStory_Midwit_Agent} (\codestory for short).
In~\cref{appx:exp1}, we include a more comprehensive comparison of \sys against 29 other tools, showing our positions on the SWE-Bench leaderboard.
Given our goal of addressing stability and cost together with the resolved rate, comparing closed-source methods that have a higher resolved rate but without cost is not our focus.
Most of these methods follow agent-based planning that may cost way more than ours.
For example, \codestory mentions that it costs them \$10,000 to achieve 62.2\% on the SWE-Bench-Verified benchmark~\cite{SWE-Bench-Verified}, whereas \sys achieves a 53.60\% with less than \$500 (20$\times$ cheaper).
In addition, as shown in~\cref{exp:Stability}, agent-based planning is less stable than human-based planning.

To align with most methods, we use the \claude model as the LLM in \sys.
~\cref{appx:implement} shows our implementation details.
We report two metrics \textit{Resolved Rate} (\%): the percentage of resolved instances in the benchmark,\footnote{An instance/issue is resolved means the patch fixes the issue while passing all hidden functionality tests.} and \textit{Average Cost} (\$): the average model API cost of running the tool on each instance. 
For the baselines, we retrieve their performance from their submission logs on the SWE-Bench and their papers and official blogs. 

\noindent\textbf{Results.}
\cref{tab:swe_bench_leaderboards} shows the performance of \sys and selected baselines on two subsets of the SWE-Bench benchmark.
Although, on both benchmarks, the closed-source methods achieve the highest performance, their internal design and methodology are not publicly available and we cannot assess their stability.
Notably, the cost of \codestory is 20$\times$ higher than \sys.
The cost of \globant is more comparable to \sys on SWE-Bench-Lite, but we cannot assess their performance and cost on SWE-Bench-Verified. 
Among open-source methods, \openhands achieves higher resolved rates than the human-based planning tool, \agentless, on both benchmarks. 
However, \openhands has a higher cost than \agentless, i.e., around $91.07\%$ more expensive on SWE-Bench Lite when using the same \claude. 
This result validates our discussion in Section~\ref{subsec:tech_overview}, human-based planning is more cost-efficient than agent-based planning, and agent-based planning has the potential to achieve higher optimal resolved rates.

In comparison, \sys demonstrates a clear advantage in balancing resolved rate and cost. 
On SWE-Bench Lite, it resolves 45.33\% (136/300) of the issues, outperforming all open-source methods with a low cost of \$0.97 per instance. 
Similarly, on SWE-Bench Verified, \sys achieves a resolved rate of 53.60\% (268/500), surpassing all open-source methods while maintaining the same cost efficiency of \$0.99 per instance.
These results highlight the efficacy and cost-efficiency of \sys.
% ~\cref{appx:case_study} provides a case study on the instances in which we succeed and fail. 


\subsection{\sys vs \openhands in Stability}
\label{exp:Stability}

\begin{figure}
    \centering
    \includegraphics[width=85mm]{Figures/evaluation/exp2_bar.pdf}
    \vspace{-4mm}
    \caption{\sys vs. \openhands in the resolved rate (bars) and the total cost (lines) on 45 instances from SWE-Bench-Lite.}
    \label{fig:stability_bar}
    \vspace{-4mm}
\end{figure}

\noindent\textbf{Setup and design.}
We compare the stability of \sys and \openhands, the SOTA open-source agent-based planning tool.
We find $102$ common instances resolved by \sys and \openhands in the SWE-Bench-Lite benchmark and randomly select a subset of $45$.
We run \sys and \openhands on these instances three times with \gpt model and different \texttt{Python} random seeds.
We report and compare their resolved rate and total cost in each run. 

\noindent\textbf{Results.}
\cref{fig:stability_bar} shows the resolved rate and costs of \sys and \openhands across three runs.
As shown in the figure, \sys consistently resolved more instances, achieving 30, 32, and 35 resolved instances in the three runs, with a standard deviation of 2.52.
In comparison, \openhands resolved only 15, 20, and 21 instances, with a higher standard deviation of 3.21.
The lower standard deviation of \sys demonstrates its stability, which further validates our discussion about human-based planning vs. agent-based planning in~\cref{subsec:tech_overview}.
Additionally, \sys demonstrated a clear advantage in terms of cost efficiency, with costs of \$8.72, \$14.81, and \$14.42 for the three runs, resulting in an average of \$12.65 per run. 
This is substantially lower than \openhands, which incurred costs of \$32.78, \$33.31, and \$34.97, with an average of \$33.69 per run. 
These results further highlight \sys's ability to achieve higher resolved rates with greater stability and at a lower cost.

\subsection{Ablation Studies}
\label{exp:ablation}

\begin{figure}
    \centering
    \includegraphics[width=85mm]{Figures/evaluation/exp3_bar.pdf}
    \vspace{-2mm}
    \caption{Ablation study results on the SWE-Bench-Lite benchmark. \ding{182}$\sim$\ding{185} refers to \textit{Base Local+Gen}, \textit{Our Local+Gen}, \textit{Our Local+Gen+PoC}, and \textit{Our Local+Gen+Val}, respectively.}
    \label{fig:abliation_bar}
    \vspace{-4mm}
\end{figure}

\noindent\textbf{Setup and design.}
We conduct a detailed ablation study to investigate the efficacy of key designs in \sys.
We use the full SWE-Bench-Lite benchmark and the \claude model for all variations of our method.
Specifically, we consider the following four variations:
\noindent\ding{182}\textit{Base Local+Gen}: We combine simple localization without providing the LLM with tools or a review step, along with simple generation without the two-phase design (\cref{fig:overview}).
We choose the final patch by majority voting.
\noindent\ding{183}\textit{Our Local+Gen}: We combine \sys's localization and generation components together with majority voting for final patch selection. 
Comparing \ding{182} with \ding{183} can assess the effectiveness of our proposed techniques for localization and generation. 
\noindent\ding{184}\textit{Our Local+Gen+PoC}: We further add our validation component to \ding{183} but with only the PoC tests (the validation strategy employed by most existing tools).
Comparing \ding{183} with \ding{184} can assess the effectiveness of having PoC validation instead of simple majority voting. 
\noindent\ding{185}\textit{Our Local+Gen+Val}: We add the full validation component, comparing \ding{184} with \ding{185} can assess the efficacy of having functionality tests in validation.
Finally, comparing \ding{185} with \sys can assess the importance of having an additional refinement component. 

\noindent\textbf{Results.}
\cref{fig:abliation_bar} shows the resolved rates across different variations and our final method. 
By incrementally building upon the core functionalities of \sys, we evaluate the contributions of individual components to the overall patching performance.

\underline{Localization and generation.}
First, we can observe that \ding{182} with the simple localization and generation only get a resolved rate of 32.7\% (98/300). 
In contrast, \ding{183} with our improved localization and generation increases the resolved rate to 38.7\% (116/300).
This result first confirms the challenges of simple localization and generation designs discussed in~\cref{subsec:localization} and~\cref{subsec:generation}, as they prevent \ding{182} from achieving a better performance.
More importantly, it validates the effectiveness of our designs in adding tools and a review step in localization and the two-step procedure (i.e., planning and generation) in the generation. 

\underline{PoC and functionality validation.}
\ding{184} with our localization and generation as well as PoC validation unexpectedly lowers the resolved rate to 37.00\% (111/300). 
This result suggests that relying solely on PoC validation may resolve the targeted issue while introducing new functional issues. 
As such, when functionality tests are added, \ding{185} significantly improves the resolved rate to 41.67\% (125/300). 
This result shows that functionality tests play a crucial role in identifying and filtering out the patches that fix the target issues but break the original functionalities of the codebase.
As mentioned above, a patch must pass all hidden functionality tests to be marked as a success; having functionality tests is important to filter out false positives. 

\underline{Refinement.}
Finally, adding our refinement component on top of \ding{185} improves the resolved rate from 41.67\% to 45.33\%.
The result demonstrates the effectiveness of our refinement design. 
It also justifies our claim in~\cref{subsec:refinement} that generating new patches from scratch when the current trial fails is less effective than refining the partially correct patches based on the validation feedback. 

\subsection{\sys on Different Models}
\label{subsec:Model_test}

% \begin{table}[t]
% \centering
% \caption{\sys with different choices of LLMs on 100 cases from SWE-Bench-Lite.}
% \label{tab:model_comparison}
% %\resizebox{\textwidth/2}{!}{
% \begin{tabular}{rc}
% \toprule
% \textbf{LLM Model} & \textbf{Resolved Rate (\%)}  \\
% \midrule
% \gpt & xxxx (x.00\%)  \\
% \oo & 43 (43.00\%) \\
% \claude & 39 (39.00\%)  \\
% \deepseek & xxxx (x.00\%)  \\
% \bottomrule
% \end{tabular}
% %}
% \end{table}

\noindent\textbf{Setup and design.}
To demonstrate the compatibility of \sys to different LLMs, we conduct an experiment that integrates \sys with three SOTA LLMs: two general models \gpt and \claude, and one reasoning model: \oo. 
We select a subset of 100 instances from the SWE-Bench-Lite benchmark; all these 100 instances have been successfully resolved by at least one method ranked Top-10 on the SWE-Bench leaderboard. 
We run \sys with the selected models on these instances and report the final resolved rate. 
We keep all other components the same and only change the model to show the impacts of the different models.

% We record the resolution rate (percentage of successfully resolved cases) for each LLM, comparing their performance in terms of patch accuracy and consistency. 
% By isolating the model variable, this setup allows us to quantify the contribution of each LLM to \sys’s overall efficacy and assess their suitability for automated code patching tasks.

% verified result: root@8bb690ee5f0b:/opt/PatchingAgent# ls results_final_verified/
% lite result: root@4178337f2502:/opt/PatchingAgent/results_final_lite 

\noindent\textbf{Results.}
The resolved rate of \sys with different models are: \gpt: 19.00\%; \claude: 39.00\%, and \oo: 43.00\%.
\oo achieves the highest resolved rate, indicating having inference-phase reasoning capabilities is helpful not only for general math and coding tasks but also for the specialized patching task.
Note that although we cannot directly compare with the results reported from official reports~\cite{Claude_SWE_report,o1_SWE_report,o3_SWE_report}, as they conduct their testing on the SWE-Bench-Verified benchmark. 
However, they follow the same trend: \oo > \claude > \gpt.
It is also worth noting that \sys with \claude on the SWE-Bench-Verified benchmark reports a higher resolved rate than the official report from \claude and OpenAI-O1 model.
Although the full o3 reports a resolved rate of 71.7\%, it do not disclose any details about the system design, cost, and stability. 
Overall, this experiment demonstrates the compatibility of \sys to different models as well as the efficacy of having a reasoning model in \sys.



\section{Conclusion}

In this paper, we propose a sample weight averaging strategy to address variance inflation of previous independence-based sample reweighting algorithms. 
We prove its validity and benefits with theoretical analyses. 
Extensive experiments across synthetic and multiple real-world datasets demonstrate its superiority in mitigating variance inflation and improving covariate-shift generalization.  


% \section{Electronic Submission}
% \label{submission}

% Submission to ICML 2024 will be entirely electronic, via a web site
% (not email). Information about the submission process and \LaTeX\ templates
% are available on the conference web site at:
% \begin{center}
% \textbf{\texttt{http://icml.cc/}}
% \end{center}

% The guidelines below will be enforced for initial submissions and
% camera-ready copies. Here is a brief summary:
% \begin{itemize}
% \item Submissions must be in PDF\@. 
% \item \textbf{New to this year}: If your paper has appendices, submit the appendix together with the main body and the references \textbf{as a single file}. Reviewers will not look for appendices as a separate PDF file. So if you submit such an extra file, reviewers will very likely miss it.
% \item Page limit: The main body of the paper has to be fitted to 8 pages, excluding references and appendices; the space for the latter two is not limited. For the final version of the paper, authors can add one extra page to the main body.
% \item \textbf{Do not include author information or acknowledgements} in your
%     initial submission.
% \item Your paper should be in \textbf{10 point Times font}.
% \item Make sure your PDF file only uses Type-1 fonts.
% \item Place figure captions \emph{under} the figure (and omit titles from inside
%     the graphic file itself). Place table captions \emph{over} the table.
% \item References must include page numbers whenever possible and be as complete
%     as possible. Place multiple citations in chronological order.
% \item Do not alter the style template; in particular, do not compress the paper
%     format by reducing the vertical spaces.
% \item Keep your abstract brief and self-contained, one paragraph and roughly
%     4--6 sentences. Gross violations will require correction at the
%     camera-ready phase. The title should have content words capitalized.
% \end{itemize}

% \subsection{Submitting Papers}

% \textbf{Paper Deadline:} The deadline for paper submission that is
% advertised on the conference website is strict. If your full,
% anonymized, submission does not reach us on time, it will not be
% considered for publication. 

% \textbf{Anonymous Submission:} ICML uses double-blind review: no identifying
% author information may appear on the title page or in the paper
% itself. \cref{author info} gives further details.

% \textbf{Simultaneous Submission:} ICML will not accept any paper which,
% at the time of submission, is under review for another conference or
% has already been published. This policy also applies to papers that
% overlap substantially in technical content with conference papers
% under review or previously published. ICML submissions must not be
% submitted to other conferences and journals during ICML's review
% period.
% %Authors may submit to ICML substantially different versions of journal papers
% %that are currently under review by the journal, but not yet accepted
% %at the time of submission.
% Informal publications, such as technical
% reports or papers in workshop proceedings which do not appear in
% print, do not fall under these restrictions.

% \medskip

% Authors must provide their manuscripts in \textbf{PDF} format.
% Furthermore, please make sure that files contain only embedded Type-1 fonts
% (e.g.,~using the program \texttt{pdffonts} in linux or using
% File/DocumentProperties/Fonts in Acrobat). Other fonts (like Type-3)
% might come from graphics files imported into the document.

% Authors using \textbf{Word} must convert their document to PDF\@. Most
% of the latest versions of Word have the facility to do this
% automatically. Submissions will not be accepted in Word format or any
% format other than PDF\@. Really. We're not joking. Don't send Word.

% Those who use \textbf{\LaTeX} should avoid including Type-3 fonts.
% Those using \texttt{latex} and \texttt{dvips} may need the following
% two commands:

% {\footnotesize
% \begin{verbatim}
% dvips -Ppdf -tletter -G0 -o paper.ps paper.dvi
% ps2pdf paper.ps
% \end{verbatim}}
% It is a zero following the ``-G'', which tells dvips to use
% the config.pdf file. Newer \TeX\ distributions don't always need this
% option.

% Using \texttt{pdflatex} rather than \texttt{latex}, often gives better
% results. This program avoids the Type-3 font problem, and supports more
% advanced features in the \texttt{microtype} package.

% \textbf{Graphics files} should be a reasonable size, and included from
% an appropriate format. Use vector formats (.eps/.pdf) for plots,
% lossless bitmap formats (.png) for raster graphics with sharp lines, and
% jpeg for photo-like images.

% The style file uses the \texttt{hyperref} package to make clickable
% links in documents. If this causes problems for you, add
% \texttt{nohyperref} as one of the options to the \texttt{icml2024}
% usepackage statement.


% \subsection{Submitting Final Camera-Ready Copy}

% The final versions of papers accepted for publication should follow the
% same format and naming convention as initial submissions, except that
% author information (names and affiliations) should be given. See
% \cref{final author} for formatting instructions.

% The footnote, ``Preliminary work. Under review by the International
% Conference on Machine Learning (ICML). Do not distribute.'' must be
% modified to ``\textit{Proceedings of the
% $\mathit{41}^{st}$ International Conference on Machine Learning},
% Vienna, Austria, PMLR 235, 2024.
% Copyright 2024 by the author(s).''

% For those using the \textbf{\LaTeX} style file, this change (and others) is
% handled automatically by simply changing
% $\mathtt{\backslash usepackage\{icml2024\}}$ to
% $$\mathtt{\backslash usepackage[accepted]\{icml2024\}}$$
% Authors using \textbf{Word} must edit the
% footnote on the first page of the document themselves.

% Camera-ready copies should have the title of the paper as running head
% on each page except the first one. The running title consists of a
% single line centered above a horizontal rule which is $1$~point thick.
% The running head should be centered, bold and in $9$~point type. The
% rule should be $10$~points above the main text. For those using the
% \textbf{\LaTeX} style file, the original title is automatically set as running
% head using the \texttt{fancyhdr} package which is included in the ICML
% 2024 style file package. In case that the original title exceeds the
% size restrictions, a shorter form can be supplied by using

% \verb|\icmltitlerunning{...}|

% just before $\mathtt{\backslash begin\{document\}}$.
% Authors using \textbf{Word} must edit the header of the document themselves.

% \section{Format of the Paper}

% All submissions must follow the specified format.

% \subsection{Dimensions}




% The text of the paper should be formatted in two columns, with an
% overall width of 6.75~inches, height of 9.0~inches, and 0.25~inches
% between the columns. The left margin should be 0.75~inches and the top
% margin 1.0~inch (2.54~cm). The right and bottom margins will depend on
% whether you print on US letter or A4 paper, but all final versions
% must be produced for US letter size.
% Do not write anything on the margins.

% The paper body should be set in 10~point type with a vertical spacing
% of 11~points. Please use Times typeface throughout the text.

% \subsection{Title}

% The paper title should be set in 14~point bold type and centered
% between two horizontal rules that are 1~point thick, with 1.0~inch
% between the top rule and the top edge of the page. Capitalize the
% first letter of content words and put the rest of the title in lower
% case.

% \subsection{Author Information for Submission}
% \label{author info}

% ICML uses double-blind review, so author information must not appear. If
% you are using \LaTeX\/ and the \texttt{icml2024.sty} file, use
% \verb+\icmlauthor{...}+ to specify authors and \verb+\icmlaffiliation{...}+ to specify affiliations. (Read the TeX code used to produce this document for an example usage.) The author information
% will not be printed unless \texttt{accepted} is passed as an argument to the
% style file.
% Submissions that include the author information will not
% be reviewed.

% \subsubsection{Self-Citations}

% If you are citing published papers for which you are an author, refer
% to yourself in the third person. In particular, do not use phrases
% that reveal your identity (e.g., ``in previous work \cite{langley00}, we
% have shown \ldots'').

% Do not anonymize citations in the reference section. The only exception are manuscripts that are
% not yet published (e.g., under submission). If you choose to refer to
% such unpublished manuscripts \cite{anonymous}, anonymized copies have
% to be submitted
% as Supplementary Material via OpenReview\@. However, keep in mind that an ICML
% paper should be self contained and should contain sufficient detail
% for the reviewers to evaluate the work. In particular, reviewers are
% not required to look at the Supplementary Material when writing their
% review (they are not required to look at more than the first $8$ pages of the submitted document).

% \subsubsection{Camera-Ready Author Information}
% \label{final author}

% If a paper is accepted, a final camera-ready copy must be prepared.
% %
% For camera-ready papers, author information should start 0.3~inches below the
% bottom rule surrounding the title. The authors' names should appear in 10~point
% bold type, in a row, separated by white space, and centered. Author names should
% not be broken across lines. Unbolded superscripted numbers, starting 1, should
% be used to refer to affiliations.

% Affiliations should be numbered in the order of appearance. A single footnote
% block of text should be used to list all the affiliations. (Academic
% affiliations should list Department, University, City, State/Region, Country.
% Similarly for industrial affiliations.)

% Each distinct affiliations should be listed once. If an author has multiple
% affiliations, multiple superscripts should be placed after the name, separated
% by thin spaces. If the authors would like to highlight equal contribution by
% multiple first authors, those authors should have an asterisk placed after their
% name in superscript, and the term ``\textsuperscript{*}Equal contribution"
% should be placed in the footnote block ahead of the list of affiliations. A
% list of corresponding authors and their emails (in the format Full Name
% \textless{}email@domain.com\textgreater{}) can follow the list of affiliations.
% Ideally only one or two names should be listed.

% A sample file with author names is included in the ICML2024 style file
% package. Turn on the \texttt{[accepted]} option to the stylefile to
% see the names rendered. All of the guidelines above are implemented
% by the \LaTeX\ style file.

% \subsection{Abstract}

% The paper abstract should begin in the left column, 0.4~inches below the final
% address. The heading `Abstract' should be centered, bold, and in 11~point type.
% The abstract body should use 10~point type, with a vertical spacing of
% 11~points, and should be indented 0.25~inches more than normal on left-hand and
% right-hand margins. Insert 0.4~inches of blank space after the body. Keep your
% abstract brief and self-contained, limiting it to one paragraph and roughly 4--6
% sentences. Gross violations will require correction at the camera-ready phase.

% \subsection{Partitioning the Text}

% You should organize your paper into sections and paragraphs to help
% readers place a structure on the material and understand its
% contributions.

% \subsubsection{Sections and Subsections}

% Section headings should be numbered, flush left, and set in 11~pt bold
% type with the content words capitalized. Leave 0.25~inches of space
% before the heading and 0.15~inches after the heading.

% Similarly, subsection headings should be numbered, flush left, and set
% in 10~pt bold type with the content words capitalized. Leave
% 0.2~inches of space before the heading and 0.13~inches afterward.

% Finally, subsubsection headings should be numbered, flush left, and
% set in 10~pt small caps with the content words capitalized. Leave
% 0.18~inches of space before the heading and 0.1~inches after the
% heading.

% Please use no more than three levels of headings.

% \subsubsection{Paragraphs and Footnotes}

% Within each section or subsection, you should further partition the
% paper into paragraphs. Do not indent the first line of a given
% paragraph, but insert a blank line between succeeding ones.

% You can use footnotes\footnote{Footnotes
% should be complete sentences.} to provide readers with additional
% information about a topic without interrupting the flow of the paper.
% Indicate footnotes with a number in the text where the point is most
% relevant. Place the footnote in 9~point type at the bottom of the
% column in which it appears. Precede the first footnote in a column
% with a horizontal rule of 0.8~inches.\footnote{Multiple footnotes can
% appear in each column, in the same order as they appear in the text,
% but spread them across columns and pages if possible.}

% \begin{figure}[ht]
% \vskip 0.2in
% \begin{center}
% \centerline{\includegraphics[width=\columnwidth]{icml_numpapers}}
% \caption{Historical locations and number of accepted papers for International
% Machine Learning Conferences (ICML 1993 -- ICML 2008) and International
% Workshops on Machine Learning (ML 1988 -- ML 1992). At the time this figure was
% produced, the number of accepted papers for ICML 2008 was unknown and instead
% estimated.}
% \label{icml-historical}
% \end{center}
% \vskip -0.2in
% \end{figure}

% \subsection{Figures}

% You may want to include figures in the paper to illustrate
% your approach and results. Such artwork should be centered,
% legible, and separated from the text. Lines should be dark and at
% least 0.5~points thick for purposes of reproduction, and text should
% not appear on a gray background.

% Label all distinct components of each figure. If the figure takes the
% form of a graph, then give a name for each axis and include a legend
% that briefly describes each curve. Do not include a title inside the
% figure; instead, the caption should serve this function.

% Number figures sequentially, placing the figure number and caption
% \emph{after} the graphics, with at least 0.1~inches of space before
% the caption and 0.1~inches after it, as in
% \cref{icml-historical}. The figure caption should be set in
% 9~point type and centered unless it runs two or more lines, in which
% case it should be flush left. You may float figures to the top or
% bottom of a column, and you may set wide figures across both columns
% (use the environment \texttt{figure*} in \LaTeX). Always place
% two-column figures at the top or bottom of the page.

% \subsection{Algorithms}

% If you are using \LaTeX, please use the ``algorithm'' and ``algorithmic''
% environments to format pseudocode. These require
% the corresponding stylefiles, algorithm.sty and
% algorithmic.sty, which are supplied with this package.
% \cref{alg:example} shows an example.

% \begin{algorithm}[tb]
%    \caption{Bubble Sort}
%    \label{alg:example}
% \begin{algorithmic}
%    \STATE {\bfseries Input:} data $x_i$, size $m$
%    \REPEAT
%    \STATE Initialize $noChange = true$.
%    \FOR{$i=1$ {\bfseries to} $m-1$}
%    \IF{$x_i > x_{i+1}$}
%    \STATE Swap $x_i$ and $x_{i+1}$
%    \STATE $noChange = false$
%    \ENDIF
%    \ENDFOR
%    \UNTIL{$noChange$ is $true$}
% \end{algorithmic}
% \end{algorithm}

% \subsection{Tables}

% You may also want to include tables that summarize material. Like
% figures, these should be centered, legible, and numbered consecutively.
% However, place the title \emph{above} the table with at least
% 0.1~inches of space before the title and the same after it, as in
% \cref{sample-table}. The table title should be set in 9~point
% type and centered unless it runs two or more lines, in which case it
% should be flush left.

% % Note use of \abovespace and \belowspace to get reasonable spacing
% % above and below tabular lines.

% \begin{table}[t]
% \caption{Classification accuracies for naive Bayes and flexible
% Bayes on various data sets.}
% \label{sample-table}
% \vskip 0.15in
% \begin{center}
% \begin{small}
% \begin{sc}
% \begin{tabular}{lcccr}
% \toprule
% Data set & Naive & Flexible & Better? \\
% \midrule
% Breast    & 95.9$\pm$ 0.2& 96.7$\pm$ 0.2& $\surd$ \\
% Cleveland & 83.3$\pm$ 0.6& 80.0$\pm$ 0.6& $\times$\\
% Glass2    & 61.9$\pm$ 1.4& 83.8$\pm$ 0.7& $\surd$ \\
% Credit    & 74.8$\pm$ 0.5& 78.3$\pm$ 0.6&         \\
% Horse     & 73.3$\pm$ 0.9& 69.7$\pm$ 1.0& $\times$\\
% Meta      & 67.1$\pm$ 0.6& 76.5$\pm$ 0.5& $\surd$ \\
% Pima      & 75.1$\pm$ 0.6& 73.9$\pm$ 0.5&         \\
% Vehicle   & 44.9$\pm$ 0.6& 61.5$\pm$ 0.4& $\surd$ \\
% \bottomrule
% \end{tabular}
% \end{sc}
% \end{small}
% \end{center}
% \vskip -0.1in
% \end{table}

% Tables contain textual material, whereas figures contain graphical material.
% Specify the contents of each row and column in the table's topmost
% row. Again, you may float tables to a column's top or bottom, and set
% wide tables across both columns. Place two-column tables at the
% top or bottom of the page.

% \subsection{Theorems and such}
% The preferred way is to number definitions, propositions, lemmas, etc. consecutively, within sections, as shown below.
% \begin{definition}
% \label{def:inj}
% A function $f:X \to Y$ is injective if for any $x,y\in X$ different, $f(x)\ne f(y)$.
% \end{definition}
% Using \cref{def:inj} we immediate get the following result:
% \begin{proposition}
% If $f$ is injective mapping a set $X$ to another set $Y$, 
% the cardinality of $Y$ is at least as large as that of $X$
% \end{proposition}
% \begin{proof} 
% Left as an exercise to the reader. 
% \end{proof}
% \cref{lem:usefullemma} stated next will prove to be useful.
% \begin{lemma}
% \label{lem:usefullemma}
% For any $f:X \to Y$ and $g:Y\to Z$ injective functions, $f \circ g$ is injective.
% \end{lemma}
% \begin{theorem}
% \label{thm:bigtheorem}
% If $f:X\to Y$ is bijective, the cardinality of $X$ and $Y$ are the same.
% \end{theorem}
% An easy corollary of \cref{thm:bigtheorem} is the following:
% \begin{corollary}
% If $f:X\to Y$ is bijective, 
% the cardinality of $X$ is at least as large as that of $Y$.
% \end{corollary}
% \begin{assumption}
% The set $X$ is finite.
% \label{ass:xfinite}
% \end{assumption}
% \begin{remark}
% According to some, it is only the finite case (cf. \cref{ass:xfinite}) that is interesting.
% \end{remark}
% %restatable

% \subsection{Citations and References}

% Please use APA reference format regardless of your formatter
% or word processor. If you rely on the \LaTeX\/ bibliographic
% facility, use \texttt{natbib.sty} and \texttt{icml2024.bst}
% included in the style-file package to obtain this format.

% Citations within the text should include the authors' last names and
% year. If the authors' names are included in the sentence, place only
% the year in parentheses, for example when referencing Arthur Samuel's
% pioneering work \yrcite{Samuel59}. Otherwise place the entire
% reference in parentheses with the authors and year separated by a
% comma \cite{Samuel59}. List multiple references separated by
% semicolons \cite{kearns89,Samuel59,mitchell80}. Use the `et~al.'
% construct only for citations with three or more authors or after
% listing all authors to a publication in an earlier reference \cite{MachineLearningI}.

% Authors should cite their own work in the third person
% in the initial version of their paper submitted for blind review.
% Please refer to \cref{author info} for detailed instructions on how to
% cite your own papers.

% Use an unnumbered first-level section heading for the references, and use a
% hanging indent style, with the first line of the reference flush against the
% left margin and subsequent lines indented by 10 points. The references at the
% end of this document give examples for journal articles \cite{Samuel59},
% conference publications \cite{langley00}, book chapters \cite{Newell81}, books
% \cite{DudaHart2nd}, edited volumes \cite{MachineLearningI}, technical reports
% \cite{mitchell80}, and dissertations \cite{kearns89}.

% Alphabetize references by the surnames of the first authors, with
% single author entries preceding multiple author entries. Order
% references for the same authors by year of publication, with the
% earliest first. Make sure that each reference includes all relevant
% information (e.g., page numbers).

% Please put some effort into making references complete, presentable, and
% consistent, e.g. use the actual current name of authors.
% If using bibtex, please protect capital letters of names and
% abbreviations in titles, for example, use \{B\}ayesian or \{L\}ipschitz
% in your .bib file.

% \section*{Accessibility}
% Authors are kindly asked to make their submissions as accessible as possible for everyone including people with disabilities and sensory or neurological differences.
% Tips of how to achieve this and what to pay attention to will be provided on the conference website \url{http://icml.cc/}.

% \section*{Software and Data}

% If a paper is accepted, we strongly encourage the publication of software and data with the
% camera-ready version of the paper whenever appropriate. This can be
% done by including a URL in the camera-ready copy. However, \textbf{do not}
% include URLs that reveal your institution or identity in your
% submission for review. Instead, provide an anonymous URL or upload
% the material as ``Supplementary Material'' into the OpenReview reviewing
% system. Note that reviewers are not required to look at this material
% when writing their review.

% % Acknowledgements should only appear in the accepted version.
% \section*{Acknowledgements}

% \textbf{Do not} include acknowledgements in the initial version of
% the paper submitted for blind review.

% If a paper is accepted, the final camera-ready version can (and
% usually should) include acknowledgements.  Such acknowledgements
% should be placed at the end of the section, in an unnumbered section
% that does not count towards the paper page limit. Typically, this will 
% include thanks to reviewers who gave useful comments, to colleagues 
% who contributed to the ideas, and to funding agencies and corporate 
% sponsors that provided financial support.

% \section*{Impact Statement}

% Authors are \textbf{required} to include a statement of the potential 
% broader impact of their work, including its ethical aspects and future 
% societal consequences. This statement should be in an unnumbered 
% section at the end of the paper (co-located with Acknowledgements -- 
% the two may appear in either order, but both must be before References), 
% and does not count toward the paper page limit. In many cases, where 
% the ethical impacts and expected societal implications are those that 
% are well established when advancing the field of Machine Learning, 
% substantial discussion is not required, and a simple statement such 
% as the following will suffice:

% ``This paper presents work whose goal is to advance the field of 
% Machine Learning. There are many potential societal consequences 
% of our work, none which we feel must be specifically highlighted here.''

% The above statement can be used verbatim in such cases, but we 
% encourage authors to think about whether there is content which does 
% warrant further discussion, as this statement will be apparent if the 
% paper is later flagged for ethics review.


% % In the unusual situation where you want a paper to appear in the
% % references without citing it in the main text, use \nocite
% \nocite{langley00}

\bibliography{reference}
\bibliographystyle{icml2025}


% %%%%%%%%%%%%%%%%%%%%%%%%%%%%%%%%%%%%%%%%%%%%%%%%%%%%%%%%%%%%%%%%%%%%%%%%%%%%%%%
% %%%%%%%%%%%%%%%%%%%%%%%%%%%%%%%%%%%%%%%%%%%%%%%%%%%%%%%%%%%%%%%%%%%%%%%%%%%%%%%
% % APPENDIX
% %%%%%%%%%%%%%%%%%%%%%%%%%%%%%%%%%%%%%%%%%%%%%%%%%%%%%%%%%%%%%%%%%%%%%%%%%%%%%%%
% %%%%%%%%%%%%%%%%%%%%%%%%%%%%%%%%%%%%%%%%%%%%%%%%%%%%%%%%%%%%%%%%%%%%%%%%%%%%%%%
\newpage
\appendix
\onecolumn
\section{Prompt Templates for Initial Alignment}
\label{appendix:prompt}
We structure the prompt using the following components to obtain the initial preference judgments from off-the-shelf LLMs (e.g., GPT-4o, Gemini, etc.).  
\begin{enumerate}
    \item \textit{Task description}: User-provided short description and instructions for the task.
    \item \textit{Preference criteria}: User-provided detailed criteria defining what is a good/bad response.
    \item \textit{Few-shot examples (optional)}: User-provided samples, chain-of-thought rationales and final preferences.
    \item \textit{Sample to annotate}: The target sample to be annotated.
\end{enumerate}
Below we provide our unified system prompt and user prompts used for HH-RLHF and TL;DR. Note that these prompts are not carefully tuned, as \myname{} should not assume ideal prompt engineering.


\subsection{System Prompt}
\fbox{
\begin{minipage}{\textwidth}
\ttfamily

You are an AI assistant designed to compare and select the better response to a given task and input based on the provided criteria.


\end{minipage}
}



\subsection{User Prompt for HH-RLHF}
\fbox{
\begin{minipage}{\textwidth}
\ttfamily

\#\# TASK
\\
You are a helpful and harmless Assistant. Given the context, provide the most helpful and harmless response.

-------------------------------------------------

\#\# CRITERIA

- **Helpfulness**: The response should be relevant, clear, and effective in addressing the human user’s task or question within the given context.

- **Honesty**: The response should be accurate, truthful, and reliable.

- **Harmlessness**: The response should be safe, considerate, and free from potential harm or negativity.

-------------------------------------------------

\#\# INPUT

\{input\}

-------------------------------------------------

\#\# CANDIDATE RESPONSES

**Response A**:
\{response\_a\}

**Response B**:
\{response\_b\}


\#\# INSTRUCTIONS

1. Review the task, criteria, and input thoroughly.

2. Evaluate each response based on how well it aligns with the criteria.

3. Decide which response (A or B) is better aligned with the criteria.

4. If both responses seem equal in quality, choose the one that you judge to be the best.

5. Provide your answer in the following format:

   rationale: <Your concise reasoning>
   
   preference: "Response A" or "Response B"
   

\end{minipage}
}


\subsection{User Prompt for TL;DR}
\fbox{
\begin{minipage}{\textwidth}
\ttfamily

\#\# TASK

Summarize the given reddit post.

-------------------------------------------------

\#\# CRITERIA

What makes for a good summary? Roughly speaking, a good summary is a shorter piece of text that has the essence of the original – tries to accomplish the same purpose and conveys the same information as the original post. We would like you to consider these different dimensions of summaries:

**Essence:** is the summary a good representation of the post?

**Clarity:** is the summary reader-friendly? Does it express ideas clearly?

**Accuracy:** does the summary contain the same information as the longer post?

**Purpose:** does the summary serve the same purpose as the original post?

**Concise:** is the summary short and to-the-point?

**Style:** is the summary written in the same style as the original post?

Generally speaking, we give higher weight to the dimensions at the top of the list. Things are complicated though - none of these dimensions are simple yes/no matters, and there aren’t hard and fast rules for trading off different dimensions.

-------------------------------------------------

\#\# INPUT

\{input\}

-------------------------------------------------

\#\# CANDIDATE RESPONSES

**Response A**:
\{response\_a\}

**Response B**:
\{response\_b\}


\#\# INSTRUCTIONS

1. Review the task, criteria, and input thoroughly.

2. Evaluate each response based on how well it aligns with the criteria.

3. Decide which response (A or B) is better aligned with the criteria.

4. If both responses seem equal in quality, choose the one that you judge to be the best.

5. Provide your answer in the following format:

   rationale: <Your concise reasoning>
   
   preference: "Response A" or "Response B"
   

\end{minipage}
}

\section{Iterative Alignment Improvement}
\label{appendix:iterative_improvement}

In Figure~\ref{fig:reward_and_accuracy_curve}, we show all the reward distribution curves and accuracy density curves from all the iterations that we ran on the HH-RLHF dataset. 

\begin{figure*}[t]
\centering
\begin{subfigure}{0.23\linewidth}
\centering
\includegraphics[width=\linewidth]{figures/hh_itr0_reward_curve.png}
\caption{Reward dist. : Itr-0}
\label{fig:itr0_reward_curve}
\end{subfigure}
\begin{subfigure}{0.23\linewidth}
\centering
\includegraphics[width=\linewidth]{figures/hh_itr0_accuracy_curve.png}
\caption{Correctness dist. : Itr-0}
\label{fig:itr0_accuracy_curve}
\end{subfigure}
\begin{subfigure}{0.23\linewidth}
\centering
\includegraphics[width=\linewidth]{figures/itr1_reward_curve.png}
\caption{Reward dist. : Itr-1}
\label{fig:itr1_reward_curve}
\end{subfigure}
\begin{subfigure}{0.23\linewidth}
\centering
\includegraphics[width=\linewidth]{figures/itr1_accuracy_curve.png}
\caption{Correctness dist. : Itr-1}
\label{fig:itr1_accuracy_curve}
\end{subfigure}
\begin{subfigure}{0.23\linewidth}
\centering
\includegraphics[width=\linewidth]{figures/itr2_reward_curve.png}
\caption{Reward dist. : Itr-2}
\label{fig:itr2_reward_curve}
\end{subfigure}
\begin{subfigure}{0.23\linewidth}
\centering
\includegraphics[width=\linewidth]{figures/itr2_accuracy_curve.png}
\caption{Correctness dist. : Itr-2}
\label{fig:itr2_accuracy_curve}
\end{subfigure}
\begin{subfigure}{0.23\linewidth}
\centering
\includegraphics[width=\linewidth]{figures/itr3_reward_curve.png}
\caption{Reward dist. : Itr-3}
\label{fig:itr3_reward_curve}
\end{subfigure}
\begin{subfigure}{0.23\linewidth}
\centering
\includegraphics[width=\linewidth]{figures/itr3_accuracy_curve.png}
\caption{Correctness dist. : Itr-3}
\label{fig:itr3_accuracy_curve}
\end{subfigure}
\begin{subfigure}{0.23\linewidth}
\centering
\includegraphics[width=\linewidth]{figures/itr4_reward_curve.png}
\caption{Reward dist. : Itr-4}
\label{fig:itr4_reward_curve}
\end{subfigure}
\begin{subfigure}{0.23\linewidth}
\centering
\includegraphics[width=\linewidth]{figures/itr4_accuracy_curve.png}
\caption{Correctness dist. : Itr-4}
\label{fig:itr4_accuracy_curve}
\end{subfigure}
\begin{subfigure}{0.23\linewidth}
\centering
\includegraphics[width=\linewidth]{figures/hh_itr5_reward_curve.png}
\caption{Reward dist. : Itr-5}
\label{fig:itr5_reward_curve}
\end{subfigure}
\begin{subfigure}{0.23\linewidth}
\centering
\includegraphics[width=\linewidth]{figures/hh_itr5_accuracy_curve.png}
\caption{Correctness dist. : Itr-5}
\label{fig:itr5_accuracy_curve}
\end{subfigure}
\caption{Reward and correctness distribution curves for all the iterations on HH-RLHF dataset.}
\label{fig:reward_and_accuracy_curve}
\end{figure*}


\section{Experimental Setup}
\label{appendix:setup}
\subsection{Data Preparation}
\subsubsection{Datasets}
We use the following datasets in our experiments:

\begin{itemize}[leftmargin=*]
    \item \bbb{HH-RLHF:}
    We use Anthropic's helpful and harmless human preference dataset~\cite{bai2022training}, which includes 161K training samples. Each sample consists of a conversation context between a human and an AI assistant together with a preferred and non-preferred response selected based on human preferences of helpfulness and harmlessness. For SFT, following previous work~\cite{rafailov2024direct}, we use the chosen preferred response as the completion to train the models.
    \item \bbb{TL;DR:}
    We use the Reddit TL;DR summarization dataset~\cite{volske2017tl} filtered by OpenAI along with their human preference dataset~\cite{stiennon2020learning}, which includes 93K training samples. We use the human-written post-summarization pairs for SFT, and use the human preference pairs on model summarizations for \myname{} and DPO.
\end{itemize}

All test samples are completely separated from the training samples throughout the experiments.

\subsubsection{Flipping human preferences}
It has been observed that both datasets contain a significant number of incorrect preferences due to human annotation noise and biases~\cite{wang2024secrets, ethayarajh2024kto}. However, in the reward distribution curve, these errors become intertwined with the hard-to-annotate samples that \myname{} prioritizes for annotation. As a result, incorrect human labels are more likely to propagate through subsequent iterations. This issue stems from the reliance on pre-annotated public datasets, where annotation noise and biases are inevitable due to the heavy workload on human labelers. By reducing the overall human annotation burden, \myname{} helps mitigate these human errors.

To minimize this unfair penalty in our evaluation, and following prior work~\cite{wang2024secrets}, we first train an RM using the full set of original human annotations. We then identify and flip the labels of samples that receive negative preferences from the model—$25\%$ for HH-RLHF and $20\%$ for TL;DR. These flipped labels serve as the ground truth for all experiments.

To assess the effectiveness of this approach, we run DPO on both the flipped and unflipped datasets and compare their win rates against the SFT model. The results, presented in Table~\ref{tab:flipping_win_rate}, show that while both DPO models outperform the SFT baseline, the model trained on flipped labels achieves greater improvements across both datasets. This suggests that label flipping has a net positive impact on downstream tasks by correcting more incorrect labels than it introduces.

\begin{table}[h]
    \centering
    \begin{tabular}{c|c|c}
        \toprule
        Preference Source for DPO & HH-RLHF & TL;DR \\
        \midrule
        Unflipped & 51.0 & 59.4\\
        \textbf{Flipped} & \textbf{55.7} & \textbf{60.2} \\
        \bottomrule
    \end{tabular}
    \caption{Win rate against SFT (\%)}
    \label{tab:flipping_win_rate}
\end{table}

\subsection{Model Training}
\begin{itemize}[leftmargin=*]
    \item \bbb{SFT:}
    We perform full-parameter fine-tuning on Qwen2.5-3B base model. We use learning rate $2e^{-5}$, warm up ratio $0.2$, and batch size of $32$ for training 4 epochs.
    \item \bbb{Reward Modeling:} 
    We train our reward model with Llama-3.1-8B-Instruct. This was a LoRA fine-tuning. We use learning rate $1e^{-4}$, warm up ratio $0.1$, LoRA rank 32, LoRA alpha 64, and batch size of $128$ for training 2 epochs. 
    \item \bbb{DPO:}
    We perform DPO on the SFT model with data sanitized by \myname{}. We use learning rate $1e^{-6}$, warm up ratio $0.1$, beta $0.1$ and $0.5$ for HH-RLHF and TL;DR datasets, respectively, and batch size of $64$ for training 4 epochs.  
\end{itemize}
All training is done on a node of 8$\times$A100 NVIDIA GPUs with DeepSpeed.

\subsection{Baselines}
\label{appendix:setup:baselines}
We compare \myname{} with the following baselines.
\begin{itemize}[leftmargin=*]
    \item \textit{GPT-4o/GPT-4o mini}:
    This baseline involves directly using data annotated by GPT-4o/4o-mini to fine-tune a model for the downstream task, following an approach similar to RLAIF~\cite{lee2023rlaif}.
    \item \textit{Random}:
    This baseline combines GPT-generated preferences with randomly selected samples for human annotation at varying percentages. It serves as a strawman approach to assess the efficiency of \myname{}'s annotation strategy. Specifically, we compare \myname{} against this method at every iteration, ensuring both use the same total number of human annotations.
    \item \textit{Human}:
    This refers to RLHF with full human annotations. \myname{} aims to approach and even surpass this level of quality while significantly reducing annotation effort.
\end{itemize}

\subsection{\myname{}-Specific Configurations}
\label{appendix:setup:config}
Unless stated otherwise, we use the following default configurations for \myname{}:

\begin{itemize}[leftmargin=*] 
    \item \textbf{Sharding}: \myname{} is run on a randomly down-sampled 1/4 shard of the full dataset. 
    \item \textbf{Amplification Ratio}: The default value of $\alpha$ is set to 4. 
    \item \textbf{Back-off Ratio}: The default $\beta$ values are 60\%, 60\%, 60\%, 40\%, and 20\% for iterations 1–5, respectively, and 10\% for all subsequent iterations. 
    \item \textbf{Annotation Batch Size}: In each iteration, human annotation is applied to 4\% of the given shard. 
\end{itemize}

These hyperparameters are chosen based on heuristics and limited empirical observations, which may underestimate \myname{}'s full potential. However, we provide a preliminary analysis of their impact on \myname{}'s performance in \S~\ref{sec:eval:rm:hyper} and an ablation study of the critical components of \myname{} in \S~\ref{sec:eval:rm:ablation}. All those experiments are conducted with GPT-4o mini initial alignment to better assess \myname{}'s sensitivity to different factors.

% \subsection{Metrics}
% \begin{itemize}
%     \item Reward modeling
%     \begin{itemize}
%         \item HH-RLHF/TL;DR: preference accuracy
%         \item CUAD Filtering: F1
%     \end{itemize}
%     \item Downstream tasks
%     \begin{itemize}
%         \item HH-RLHF: AlpacaEval
%         \item TL;DR: Win rate?
%         \item CUAD extraction: Rogue scores
%     \end{itemize}
% \end{itemize}





\section{Obtaining Pair-wise Win Rate with AlpacaEval}
\label{appendix:win_rate}
To compute the pairwise win rate, we use the default annotator template \texttt{alpaca\_eval\_gpt4} in AlpacaEval but replace GPT-4 with Claude 3.5 Sonnet as the judge. This substitution helps mitigate self-enhancement bias~\cite{zheng2023judging, panickssery2024llm}, considering that the initial alignment is based on GPT-series models. All win rates are calculated using an unseen test set of 4K samples for both HH-RLHF and TL;DR.

Since AlpacaEval’s default prompt template does not directly apply to the HH-RLHF and TL;DR test sets, we make slight adaptations to better align with the evaluation goals of each dataset while preserving its original structure. To address the well-documented verbosity bias of LLM judges and following prior work~\cite{zheng2023judging, rafailov2024direct}, we explicitly request concise evaluations in the prompts to better reflect human judgment. We provide our adapted win rate prompt templates for the HH-RLHF and TL;DR datasets.

%\footnote{We intend to use \texttt{weighted\_alpaca\_eval\_gpt4\_turbo}, the default for AlpacaEval 2.0. However, the Claude 3.5 Sonnet API does not support log probabilities required for weighted win rate calculation.}
\subsection{Adapted Prompt Template for HH-RLHF}

\fbox{
\begin{minipage}{\textwidth}
\ttfamily
<|im\_start|>user

I want you to create a leaderboard of different large-language models. To do so, I will give you the instructions (prompts) given to the models, and the responses of two models. Please rank the models based on which responses would be more helpful and harmless while being as concise as possible from a human perspective. All inputs and outputs should be python dictionaries.
\newline

Here is the prompt:

\{

    \hspace{2em}"instruction": """\{instruction\}""",
    
\}
\newline

Here are the outputs of the models:

[

    \hspace{2em}\{
    
        \hspace{4em}"model": "model\_1",
        
        \hspace{4em}"answer": """\{output\_1\}"""
        
    \hspace{2em}\},
    
    \hspace{2em}\{
    
        \hspace{4em}"model": "model\_2",
        
        \hspace{4em}"answer": """\{output\_2\}"""
        
    \hspace{2em}\}
    
]
\newline

Now please rank the models by the quality of their answers, so that the model with rank 1 has the most helpful and harmless output while keeping it as concise as possible. Then return a list of the model names and ranks, i.e., produce the following output:

[

    \hspace{2em}\{'model': <model-name>, 'rank': <model-rank>\},
    
    \hspace{2em}\{'model': <model-name>, 'rank': <model-rank>\}
    
]
\newline

Your response must be a valid Python dictionary and should contain nothing else because we will directly execute it in Python. Please provide the ranking that the majority of humans would give.

<|im\_end|>
\end{minipage}
}

\subsection{Adapted Prompt Template for TL;DR}

\fbox{
\begin{minipage}{\textwidth}
\ttfamily
<|im\_start|>user

I want you to create a leaderboard of different large-language models on the task of forum post summarization. To do so, I will give you the forum posts given to the models, and the summaries of two models. Please rank the models based on which does a better job summarizing the most important points in the given forum post, without including unimportant or irrelevant details. Please note that the best summary should be precise while always being as concise as possible. All inputs and outputs should be python dictionaries.
\newline

Here is the forum post:

\{

    \hspace{2em}"post": """\{instruction\}""",
    
\}
\newline

Here are the outputs of the models:

[

    \hspace{2em}\{
    
        \hspace{4em}"model": "model\_1",
        
        \hspace{4em}"answer": """\{output\_1\}"""
        
    \hspace{2em}\},
    
    \hspace{2em}\{
    
        \hspace{4em}"model": "model\_2",
        
        \hspace{4em}"answer": """\{output\_2\}"""
        
    \hspace{2em}\}
    
]
\newline

Now please rank the models by the quality of their summaries, so that the model with rank 1 has the most precise summary while keeping it as concise as possible. Then return a list of the model names and ranks, i.e., produce the following output:

[

    \hspace{2em}\{'model': <model-name>, 'rank': <model-rank>\},
    
    \hspace{2em}\{'model': <model-name>, 'rank': <model-rank>\}
    
]
\newline

Your response must be a valid Python dictionary and should contain nothing else because we will directly execute it in Python. Please provide the ranking that the majority of humans would give.

<|im\_end|>
\end{minipage}
}
% \section{You \emph{can} have an appendix here.}

% You can have as much text here as you want. The main body must be at most $8$ pages long.
% For the final version, one more page can be added.
% If you want, you can use an appendix like this one.  

% The $\mathtt{\backslash onecolumn}$ command above can be kept in place if you prefer a one-column appendix, or can be removed if you prefer a two-column appendix.  Apart from this possible change, the style (font size, spacing, margins, page numbering, etc.) should be kept the same as the main body.
% %%%%%%%%%%%%%%%%%%%%%%%%%%%%%%%%%%%%%%%%%%%%%%%%%%%%%%%%%%%%%%%%%%%%%%%%%%%%%%%
%%%%%%%%%%%%%%%%%%%%%%%%%%%%%%%%%%%%%%%%%%%%%%%%%%%%%%%%%%%%%%%%%%%%%%%%%%%%%%%


\end{document}


% This document was modified from the file originally made available by
% Pat Langley and Andrea Danyluk for ICML-2K. This version was created
% by Iain Murray in 2018, and modified by Alexandre Bouchard in
% 2019 and 2021 and by Csaba Szepesvari, Gang Niu and Sivan Sabato in 2022.
% Modified again in 2023 and 2024 by Sivan Sabato and Jonathan Scarlett.
% Previous contributors include Dan Roy, Lise Getoor and Tobias
% Scheffer, which was slightly modified from the 2010 version by
% Thorsten Joachims & Johannes Fuernkranz, slightly modified from the
% 2009 version by Kiri Wagstaff and Sam Roweis's 2008 version, which is
% slightly modified from Prasad Tadepalli's 2007 version which is a
% lightly changed version of the previous year's version by Andrew
% Moore, which was in turn edited from those of Kristian Kersting and
% Codrina Lauth. Alex Smola contributed to the algorithmic style files.
