\section{Experiments}
\subsection{Experimental Settings}
\paragraph{Datasets and metrics.} Following previous research \citep{DBLP:conf/acl/WangLSXDLCWS24,DBLP:conf/iclr/LightmanKBEBLLS24}, we evaluate value-based process verifiers based on their verification ability through best-of-$N$ sampling. Specifically, given a problem $p$, we sample $N$ candidate solutions from a generator. Then, the candidates are re-ranked according to the score generated by the given verifier. The candidate with the highest score will be treated as the best candidate and selected as the final solution. Finally, we compare the consistency of the final solution and ground-truth answer to determine whether the solution is correct. The statistical success rate will be reported. Following \citet{DBLP:journals/corr/abs-2408-03314}, we also include the beam-search metric to evaluate the performance of value-based process verifiers at test time. Specifically, the number of beams $N$ and beam size $M$ will be set. Given a problem $p$, the verifier is required to score every step during the generation process. For each iteration, $N$ beams will generate $M$ next-step candidates individually, the verifier will select the top $N$ candidates from the $N*M$ beams as the beams for the next iteration, and then continue the iteration until finished. Like Best-of-N, we compare the consistency of the final solution and ground-truth answer. The statistical success rate will be reported.

We conduct our experiments on the challenging MATH dataset\citet{DBLP:conf/nips/HendrycksBKABTS21} for verification. We use the test split following \citet{DBLP:conf/iclr/LightmanKBEBLLS24} as our test set, which consists of 500 randomly selected questions from MATH. As described by \citet{DBLP:conf/acl/WangLSXDLCWS24}, the subset evaluation produces similar results to the full-set evaluation. We uses MetaMATH\citep{DBLP:conf/iclr/YuJSYLZKLWL24} as the fine-tuning dataset as used in \citet{DBLP:conf/acl/WangLSXDLCWS24}.

\paragraph{Baselines and implementation details.}
The generator in our experiments is based on LLemma-7b. We train it on MetaMATH for 3 epochs to get the generator. Based on the train split of MATH dataset, we construct the training dataset of the process verifier. To clarify, We use the generator to sample 15 solutions per problem. Following previous works\citep{DBLP:conf/iclr/LightmanKBEBLLS24, DBLP:conf/acl/WangLSXDLCWS24}, we split each solution into steps by the pre-defined rule-based strategies (e.g. newline as the delimiters). For each step, we combine it with its previous steps to form an incomplete solution, then sample 8 rollouts to perform Monte Carlo estimation and annotate the state value. In general, we sample $15*8=120$ samples for each problem and the training dataset has 180k samples in total.

We use the Qwen2.5-Math-7B-Instruct\citep{yang2024qwen2} and deepseek-math-7b-instruct\citep{DBLP:journals/corr/abs-2402-03300} as the value-based process verifier base model. For the scalar-regression methods, similar to \citet{DBLP:conf/acl/WangLSXDLCWS24}, we train the verifier in the language modeling way, adding special tokens to the model's vocabulary and use the probability of generating the positive token as the predicted state value. For the expectation-based methods, we add a linear layer then use the softmax function to map the categorical output into a probability distribution. We compare the following methods:
\begin{itemize}
    \item Scalar Regression (outcome). Scalar-regression model that is trained on the final state value.
    \item Scalar Regression. Scalar-regression model that is trained on every intermediate state value, including the final state value.
    \item Expectation Regression (MSE). Expectation-based model that trained with mean-square error objective function on every intermediate state value, including the final state value.
    \item Expectation Classification (HL). Expectation-based model that trained with Histogram Loss objective function on a pre-defined distribution on every intermediate state value, including the final state value.
\end{itemize}

For the Expectation Classification method, we use the truncated Gaussian distribution where $\mu$ is the sampled state value, $\sigma^2$ is $\frac{\mu(1-\mu)}{k}$, as the Gaussian distribution can be used as an approximation of the binomial distribution, which can result in a more precise posterior distribution and thus reduce distribution mismatch. A further analysis of distribution selection can be found in section\ref{analysis_sampling}.

\subsection{Results}
\section{Appendix: Experimental results}
In this section, we present the results of the comprehensive experimental evaluation, which contains various combinations of query graphs, problem formulations, and classical and quantum optimizers. For each method proposed in this work, we evaluated the technique against five common query graph types: clique, star, chain, cycle, and tree. Each query graph is labeled using the format 'Graph name - number of nodes'. For each graph type and graph size, we randomly selected 20 query graph instances with cardinalities and selectivities. The cardinalities are randomly sampled from the range 10 to 50 and selectivity from interval $(0, 1]$. The costs are summed over 20 runs, describing a realistic cumulative error, and scaled with respect to the cost returned from the dynamic programming algorithm \textit{with} cross-products, which is the optimal left-deep plan. The anonymized code for this experimental evaluation can be found on GitHub \cite{anonymous2024qjoin}. Since we have used 20 query graph instances for five different graph types of sizes 3 to 60 and solved them with four different quantum and classical solvers, the total number of evaluated query graphs is in the thousands.

%After fixing one of the three proposed methods and a query graph instance, we have constructed the corresponding HUBO optimization problem. Then, we have submitted the problem for each selected solver. Depending on the solver's requirements, we might need to translate the HUBO problem into the equivalent QUBO problem. The available solvers include two quantum computing (D-wave quantum annealer, D-wave Hybrid quantum annealer) and two classical approaches (exact poly solver, Gurobi optimizer).

We have decided to focus on the quality of solutions instead of optimization time. Although time is crucial in real-life cases, integrating quantum computational systems with classical systems still brings an unavoidable overhead. Quantum computers work at the time scale of nano and milliseconds, but the classical pre-and post-processing makes the total computation time relatively long in practice. Concretely, these pre-and post-processing phases are demonstrated by such steps as encoding problems in HUBO/QUBO format, submitting them to a quantum computer over a possibly slow connection, and even waiting in line for the quantum computer to become available from the other users. After executing the workload, we need to translate the obtained results back into a format that allows us to interpret them in the light of the original problem.

\textbf{Summary of proposed methods.} We have proposed three algorithms to solve the join order selection problem with a higher-order unconstrained binary optimization model. Table \ref{table:methods} introduces names for these methods, which are used in this section.

\begin{table}[!ht]
\centering
\resizebox{0.5\columnwidth}{!}{
\begin{tabular}{|c|c|c|c|}
\hline
\textbf{Method name} & \textbf{Cost function} & \textbf{Validity constraint} \\ \hline
precise 1 & precise cost function & cost function dependent \\ \hline
precise 2 & precise cost function & cost function independent \\ \hline
heuristic & heuristic cost function & cost function dependent \\ \hline
\end{tabular}
}
\caption{Summary of proposed algorithms}
\label{table:methods}
\end{table}

\subsection{Evaluating \textrm{Precise 1} formulation}

First, we evaluate \textrm{Precise 1} formulation, which combines precise cost function and cost-dependent validity constraints. Fig.~\ref{fig:poly_solver_precise_1} shows the results of optimizing join order selection using the D-Wave's exact poly solver. Following the bounds given by Theorem \ref{thm:dp_bound}, the HUBO model consistently generates a plan that matches the quality of the plan produced by the dynamic programming algorithm without the cross products. We also see that the returned plans are only at most $0.7\%$ bigger than the optimal plan from dynamic programming with the cross products. We have excluded some results where the HUBO model produced the exact optimal plan: Clique-3, Cycle-3, Star-4, and Star-5.

\begin{figure*}[tbh]
    \centering
    \includegraphics[width=\textwidth]{results/precise_1_exact_poly_solver.png}
    \caption{Precise 1 results using the D-Wave's exact poly solver}
    \label{fig:poly_solver_precise_1}
    \Description[Other results]{}
\end{figure*}

% Graph Graph with 3 nodes and 3 edges is not included in the chart.
% costvalues: {'HUBO model': 1.0, 'Graph-DP': 1.0, 'Graph-Greedy': 1.0}
% graph type: clique
% Graph Graph with 3 nodes and 3 edges is not included in the chart.
% costvalues: {'HUBO model': 1.0, 'Graph-DP': 1.0, 'Graph-Greedy': 1.0}
% graph type: cycle
% Graph Graph with 4 nodes and 3 edges is not included in the chart.
% costvalues: {'HUBO model': 1.0000000000000002, 'Graph-DP': 1.0, 'Graph-Greedy': 1.0000000000000002}
% graph type: star
% Graph Graph with 5 nodes and 4 edges is not included in the chart.
% costvalues: {'HUBO model': 1.0000040588404762, 'Graph-DP': 1.000004058840476, 'Graph-Greedy': 1.0000040588404762}
% graph type: star
% Graph Graph with 5 nodes and 4 edges is not included in the chart.
% costvalues: {'HUBO model': 1.0, 'Graph-DP': 1.0, 'Graph-Greedy': 1.0}
% graph type: random

Second, we solved the same HUBO formulations using a classical Gurobi solver after the HUBO problem was translated into the equivalent QUBO problem. The results are presented in Fig.~\ref{fig:gurobi_precise_1}. The HUBO to QUBO translation does not decrease the algorithm's quality, and Gurobi finds the correct plans. The results stay very close to the optimal join tree, always being as good as a dynamic programming algorithm without cross products.

\begin{figure*}[tbh]
    \centering
    \includegraphics[width=\textwidth]{results/precise_1_gurobi.png}
    \caption{Precise 1 results using Gurobi solver}
    \label{fig:gurobi_precise_1}
    \Description[Other results]{}
\end{figure*}

% Graph Graph with 3 nodes and 3 edges is not included in the chart.
% costvalues: {'HUBO model': 1.0, 'Graph-DP': 1.0, 'Graph-Greedy': 1.0}
% graph type: clique
% Graph Graph with 3 nodes and 3 edges is not included in the chart.
% costvalues: {'HUBO model': 1.0, 'Graph-DP': 1.0, 'Graph-Greedy': 1.0}
% graph type: cycle
% Graph Graph with 4 nodes and 3 edges is not included in the chart.
% costvalues: {'HUBO model': 1.0000000000000002, 'Graph-DP': 1.0, 'Graph-Greedy': 1.0000000000000002}
% graph type: star
% Graph Graph with 5 nodes and 4 edges is not included in the chart.
% costvalues: {'HUBO model': 1.0000040588404762, 'Graph-DP': 1.000004058840476, 'Graph-Greedy': 1.0000040588404762}
% graph type: star
% Graph Graph with 4 nodes and 5 edges is not included in the chart.
% costvalues: {'HUBO model': 1.0, 'Graph-DP': 1.0, 'Graph-Greedy': 1.0}
% graph type: random

Third, we solved the same problems using D-wave's Leap Hybrid solver, a quantum-classical optimization platform in the cloud. In this case, the results are consistently as good as those from the dynamic program algorithm without the cross products, with some exceptions due to the heuristic nature of the quantum computer: Cycle-6, Chain-7, and Tree-6.

\begin{figure*}[tbh]
    \centering
    \includegraphics[width=\textwidth]{results/precise_1_dwave_LeapHybridSampler.png}
    \caption{Precise 1 results using D-Wave's Leap Hybrid solver}
    \label{fig:leap_precise_1}
    \Description[Other results]{}
\end{figure*}

% Graph Graph with 3 nodes and 3 edges is not included in the chart.
% costvalues: {'HUBO model': 1.0, 'Graph-DP': 1.0, 'Graph-Greedy': 1.0}
% graph type: clique
% Graph Graph with 3 nodes and 3 edges is not included in the chart.
% costvalues: {'HUBO model': 1.0, 'Graph-DP': 1.0, 'Graph-Greedy': 1.0}
% graph type: cycle
% Graph Graph with 4 nodes and 3 edges is not included in the chart.
% costvalues: {'HUBO model': 1.0000000000000002, 'Graph-DP': 1.0, 'Graph-Greedy': 1.0000000000000002}
% graph type: star
% Graph Graph with 5 nodes and 4 edges is not included in the chart.
% costvalues: {'HUBO model': 1.0000177876393395, 'Graph-DP': 1.000004058840476, 'Graph-Greedy': 1.0000040588404762}
% graph type: star

Finally, Fig.~\ref{fig:dwave_precise_1} shows the results from D-Wave's quantum annealer, which does not utilize hybrid features to increase solution quality. This resulted in performance that did not match the performance of the previous solvers, and this performance decrease was already identified in \cite{Schonberger_Scherzinger_Mauerer}. While quality was not as good as the previous solutions, the valid plans were still usable, mainly only a few percent from the global optimal. %Noise on the machine decreased the quality, and we could not embed many previous problems in the hardware. 

\begin{figure*}[tbh]
    \centering
    \includegraphics[width=\textwidth]{results/precise_1_dwave_DWaveSampler.png}
    \caption{Precise 1 results using D-Wave's standard solver}
    \label{fig:dwave_precise_1}
    \Description[Other results]{Everything is included in these results}
\end{figure*}
\subsection{Evaluating \textrm{Precise 2} formulation}

The key idea behind the Precise 2 formulation is to tackle larger join order optimization cases because the validity constraints are more efficient regarding the number of higher-order terms. We include the exact poly solver results to demonstrate that this formulation encodes precisely the correct plans. For the other solvers, we only show results that optimized larger queries compared to the previous Presice 1 method.

First, the results from the exact poly solver in Fig.~\ref{fig:precise_2_exact_poly_solver} demonstrate that this algorithm follows the bounds of Theorem \ref{thm:dp_bound}. In practice, the returned plans are again very close to the optimal plans. We can also see that compared to the Precise 1 method, the different sets of validity constraints work equally well.

\begin{figure*}[tbh]
    \centering
    \includegraphics[width=\textwidth]{results/precise_2_exact_poly_solver.png}
    \caption{Precise 2 results using the D-Wave's exact poly solver}
    \label{fig:precise_2_exact_poly_solver}
    \Description[Precise 2 results using the D-Wave's exact poly solver]{}
\end{figure*}

% Graph Graph with 3 nodes and 3 edges is not included in the chart.
% costvalues: {'Q-Join: Presice 2': 1.0, 'Dynamic programming w/o cross-products': 1.0, 'Greedy w/o cross-products': 1.0}
% graph type: clique
% Graph Graph with 3 nodes and 3 edges is not included in the chart.
% costvalues: {'Q-Join: Presice 2': 1.0, 'Dynamic programming w/o cross-products': 1.0, 'Greedy w/o cross-products': 1.0}
% graph type: cycle
% Graph Graph with 4 nodes and 3 edges is not included in the chart.
% costvalues: {'Q-Join: Presice 2': 1.0000000000000002, 'Dynamic programming w/o cross-products': 1.0, 'Greedy w/o cross-products': 1.0000000000000002}
% graph type: star
% Graph Graph with 5 nodes and 4 edges is not included in the chart.
% costvalues: {'Q-Join: Presice 2': 1.0000040588404762, 'Dynamic programming w/o cross-products': 1.000004058840476, 'Greedy w/o cross-products': 1.0000040588404762}
% graph type: star

Second, to evaluate the Gurobi solver, we scaled up the problem sizes remarkably from the Precise 1 method, although the experiments were performed on a standard laptop. The results are presented in Fig.~\ref{fig:precise_2_gurobi_1} and Fig.~\ref{fig:precise_2_gurobi_2}. We can see that finding the point that minimizes both cost and validity constraints becomes harder when the problem sizes increase. %The results remain close to optimal, with cost summed over multiple runs. However, even one suboptimal round can significantly impact the total cost and accuracy, so the results demonstrate solid performance considering the demanding evaluation setup.

\begin{figure*}[tbh]
    \centering
    \includegraphics[width=\textwidth]{results/precise_2_gurobi_1.png}
    \caption{Precise 2 results using Gurobi solver}
    \label{fig:precise_2_gurobi_1}
    \Description[Precise 2 results using Gurobi solver]{}
\end{figure*}

% Graph Graph with 3 nodes and 3 edges is not included in the chart.
% costvalues: {'Q-Join: Presice 2': 1.0, 'Dynamic programming w/o cross-products': 1.0, 'Greedy w/o cross-products': 1.0}
% graph type: clique
% Graph Graph with 3 nodes and 3 edges is not included in the chart.
% costvalues: {'Q-Join: Presice 2': 1.0, 'Dynamic programming w/o cross-products': 1.0, 'Greedy w/o cross-products': 1.0}
% graph type: clique
% Graph Graph with 3 nodes and 3 edges is not included in the chart.
% costvalues: {'Q-Join: Presice 2': 1.0, 'Dynamic programming w/o cross-products': 1.0, 'Greedy w/o cross-products': 1.0}
% graph type: cycle
% Graph Graph with 3 nodes and 3 edges is not included in the chart.
% costvalues: {'Q-Join: Presice 2': 1.0, 'Dynamic programming w/o cross-products': 1.0, 'Greedy w/o cross-products': 1.0}
% graph type: cycle
% Graph Graph with 4 nodes and 3 edges is not included in the chart.
% costvalues: {'Q-Join: Presice 2': 1.0000000000000002, 'Dynamic programming w/o cross-products': 1.0, 'Greedy w/o cross-products': 1.0000000000000002}
% graph type: star
% Graph Graph with 5 nodes and 4 edges is not included in the chart.
% costvalues: {'Q-Join: Presice 2': 1.0000040588404762, 'Dynamic programming w/o cross-products': 1.000004058840476, 'Greedy w/o cross-products': 1.0000040588404762}
% graph type: star
% Graph Graph with 6 nodes and 5 edges is not included in the chart.
% costvalues: {'Q-Join: Presice 2': 1.0000021564508312, 'Dynamic programming w/o cross-products': 1.0000011116483463, 'Greedy w/o cross-products': 1.0000011116483465}
% graph type: star
% Graph Graph with 4 nodes and 3 edges is not included in the chart.
% costvalues: {'Q-Join: Presice 2': 1.0000000000000002, 'Dynamic programming w/o cross-products': 1.0, 'Greedy w/o cross-products': 1.0000000000000002}
% graph type: star
% Graph Graph with 5 nodes and 4 edges is not included in the chart.
% costvalues: {'Q-Join: Presice 2': 1.0000040588404762, 'Dynamic programming w/o cross-products': 1.000004058840476, 'Greedy w/o cross-products': 1.0000040588404762}
% graph type: star

\begin{figure*}[tbh]
    \centering
    \includegraphics[width=\textwidth]{results/precise_2_gurobi_2.png}
    \caption{Precise 2 results using Gurobi solver}
    \label{fig:precise_2_gurobi_2}
    \Description[Precise 2 results using Gurobi solver]{}
\end{figure*}

Slightly unexpectedly, the Leap Hybrid solver did not perform as well as we expected, as shown in Fig.~\ref{fig:precise_2_dwave_LeapHybridSampler}. The solver does not have tunable hyperparameters, which we would be able to adjust to obtain better results. On the other hand, we used the developer access to the solver, which is limited to only one minute of quantum computing access per month. Finally, we did not include the results from the D-wave quantum solver due to space limitations since the solver did not scale to these cases. %It provided new results only for the Star-6 query graph with $7.07\%$ larger plans over 20 query graphs than the optimal ones.

\begin{figure*}[tbh]
    \centering
    \includegraphics[width=\textwidth]{results/precise_2_dwave_LeapHybridSampler.png}
    \caption{Precise 2 results using D-Wave's Leap Hybrid solver}
    \label{fig:precise_2_dwave_LeapHybridSampler}
    \Description[Precise 2 results using D-Wave's Leap Hybrid solver]{}
\end{figure*}

% Graph Graph with 3 nodes and 3 edges is not included in the chart.
% costvalues: {'Q-Join: Presice 2': 1.0, 'Dynamic programming w/o cross-products': 1.0, 'Greedy w/o cross-products': 1.0}
% graph type: clique
% Graph Graph with 3 nodes and 3 edges is not included in the chart.
% costvalues: {'Q-Join: Presice 2': 1.0, 'Dynamic programming w/o cross-products': 1.0, 'Greedy w/o cross-products': 1.0}
% graph type: clique
% Graph Graph with 3 nodes and 3 edges is not included in the chart.
% costvalues: {'Q-Join: Presice 2': 1.0, 'Dynamic programming w/o cross-products': 1.0, 'Greedy w/o cross-products': 1.0}
% graph type: cycle
% Graph Graph with 3 nodes and 3 edges is not included in the chart.
% costvalues: {'Q-Join: Presice 2': 1.0, 'Dynamic programming w/o cross-products': 1.0, 'Greedy w/o cross-products': 1.0}
% graph type: cycle
% Graph Graph with 4 nodes and 3 edges is not included in the chart.
% costvalues: {'Q-Join: Presice 2': 1.0000000000000002, 'Dynamic programming w/o cross-products': 1.0, 'Greedy w/o cross-products': 1.0000000000000002}
% graph type: star
% Graph Graph with 5 nodes and 4 edges is not included in the chart.
% costvalues: {'Q-Join: Presice 2': 1.0000141364642996, 'Dynamic programming w/o cross-products': 1.000004058840476, 'Greedy w/o cross-products': 1.0000040588404762}
% graph type: star
% Graph Graph with 6 nodes and 5 edges is not included in the chart.
% costvalues: {'Q-Join: Presice 2': 1.0000247692741169, 'Dynamic programming w/o cross-products': 1.0000011116483463, 'Greedy w/o cross-products': 1.0000011116483465}
% graph type: star
% Graph Graph with 4 nodes and 3 edges is not included in the chart.
% costvalues: {'Q-Join: Presice 2': 1.0000000000000002, 'Dynamic programming w/o cross-products': 1.0, 'Greedy w/o cross-products': 1.0000000000000002}
% graph type: star
% Graph Graph with 5 nodes and 4 edges is not included in the chart.
% costvalues: {'Q-Join: Presice 2': 1.0000177876393395, 'Dynamic programming w/o cross-products': 1.000004058840476, 'Greedy w/o cross-products': 1.0000040588404762}
% graph type: star
\subsection{Evaluating heuristic formulation}

The key motivation behind the heuristic formulation is to tackle even larger query graphs. Our main goal is to demonstrate that this algorithm reaches acceptable results with superior scalability compared to the previous Precise 1 and 2 formulations. The results also indicate that Theorem \ref{thm:greedy_bound} is respected in practice. The optimal results are computed with dynamic programming without cross products. Due to space limitations, we only included the results from the Gurobi solver, which we consider the most demonstrative, and we had unlimited access to it since it runs locally. 

The results are presented so that we have computed and scaled the difference between each pair of methods. A value that differs from 0 indicates that the two methods gave different join trees with different costs. Since one of the methods is near-optimal (DP without cross products), it is clear which method produced the suboptimal result. This way, we can compare all three methods at the same time. In all cases, we can see that the heuristic algorithm respects Theorem \ref{thm:greedy_bound} very well in practice, so the difference between quantum and greedy is always $0$.

Fig.~\ref{fig:clique_accuracies} shows the results of applying the heuristic method to clique query graphs. Although these results are good, the scalability in this hard case is modest. On the other hand, we are unaware of any quantum computing research that would have outperformed this scalability in the case of clique graphs. For example, the most scalable method \cite{10.14778/3632093.3632112} excluded clique graphs from their results.

\begin{figure}
    \centering
    \includegraphics[width=\linewidth]{results/clique_accuracies.png}
    \caption{Heuristic results for clique query graphs using Gurobi solver}
    \label{fig:clique_accuracies}
\end{figure}

The results for the tree (Fig.~\ref{fig:tree_accuracies}), chain (Fig.~\ref{fig:chain_accuracies}), cycle (Fig.~\ref{fig:cycle_accuracies}), and star graphs demonstrate the best scalability. We computed the results up to 60 tables to demonstrate advantageous scalability over the most scalable method in the previous research \cite{10.14778/3632093.3632112} where they considered queries up to 50 relations.

\begin{figure}
    \centering
    \includegraphics[width=\linewidth]{results/tree_accuracies.png}
    \caption{Heuristic results for tree query graphs using Gurobi solver}
    \label{fig:tree_accuracies}
\end{figure}

\begin{figure}
    \centering
    \includegraphics[width=\linewidth]{results/chain_accuracies.png}
    \caption{Heuristic results for chain query graphs using Gurobi solver}
    \label{fig:chain_accuracies}
\end{figure}
    
\begin{figure}
    \centering
    \includegraphics[width=\linewidth]{results/cycle_accuracies.png}
    \caption{Heuristic results for cycle query graphs using Gurobi solver}
    \label{fig:cycle_accuracies}
\end{figure}

We exclude the results for the star query graphs because, in this case, all three methods performed identically across up to 60 graphs and over 20 iterations, with no difference observed (a relative scaled difference of 0). These results may be because star graphs typically do not benefit from cross products \cite{10.14778/3632093.3632112}, and our method, which excludes them, performs better with such types of queries.


For the BoN experiments, we report the Best-of-N results and $N\in\{8,16,32,64,128\}$. For the beam search experiments, we report the result of both the number of beams and beam size are the same value, $M=N\in\{4,8\}$. The results are shown in Table \ref{tab:experiment-results}. Under the categorical distribution definition, the expectation-based models that directly optimized the categorical distribution by its expectation, or incorporating the pre-defined Gaussian distribution then directly optimized the shape of the categorical distribution using Histogram loss, perform consistently better than the scalar regression baseline, showing that the structural prior injection method can effectively improve the performance of process verifier trained via different optimization objective functions. The results also indicate that under reasonable structural prior, the expectation-based method can be an elegant replacement for the scalar-regression verifier at little-to-no cost.

We also see a superior performance of the outcome-supervised verifier compared with its process-supervised version on the Best-of-N task, when using Qwen2.5-Math-7B-Instruct as the base model of the verifier. One possible reason is that the value annotations are accurate for the final states. For the intermediate states, the labels contain error and noise due to Monte Carlo estimation and thus result in inferior performance. We provide more information about the Monte Carlo error analysis in Appendix \ref{app:noise}. However, though the outcome-supervised verifier has satisfied performance for the Best-of-N task, it doesn't perform well in Beam Search, which shows the importance of process supervision and the broad advantages of our method.