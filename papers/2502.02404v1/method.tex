%\section{Methodology [Sjoerd]}
%\subsection{Literature selection} 

\label{select-lit-section}
We performed the literature collection in this survey study as a three-step process. First, we employed the online literature search tool Scopus \cite{elsevierScopus} to gather relevant literature from several publishers. Second, we performed a more in-depth search for articles by specific publishers. Third, a selection of conferences and journals were manually checked for relevant papers in proceedings and articles, respectively, to include publications that were possibly omitted in previous steps.  Finally, all duplicates were filtered out and we manually reviewed the resulting publications for relevance.

\paragraph{Step 1: Scopus-wide search}
%To collect relevant literature for this review, 
We used Scopus to search for publications from over 7,000 publishers. We used the following search criteria: a) the word ``FPGA'' appears in the title, abstract, and/or keywords, b) at least one of the authors has a Dutch affiliation, and c) the publication year is 2019 or later. These translate into the following Scopus query:  
%\begin{center} \begin{verbatim} 
\texttt{TITLE-ABS-KEY (FPGA) AND AFFILCOUNTRY (Netherlands) AND PUBYEAR > 2019}.
%\end{verbatim}
%\end{center}
The search was performed using this query on the October 30, 2023, which delivered a list of 186 papers. As of the publication date, this query is expected to yield an increasing number of papers, as more papers aligning with the search criteria are published. %At the date of publication, this query will deliver more results as more papers are published that fit the query. 
To put things in context, we also conducted a world-wide search for papers on FPGAs published in the last 5 years. The search resulted in a list of over 27,600 titles from 84 countries with at least 10 publications, indicating the Netherlands contributes roughly 1\% to the world-wide research on FPGAs, and ranks on position 22 in terms of output volume. In Europe, Dutch FPGA research ranks 10th in terms of output volume. 
 % that are considered in this survey. 

%. Scopus allows searching for peer-reviewed literature from over 7000 publishers. Scopus also enables filtering literature for specific elements such as keywords, words in abstract, and author affiliations.
%In order to search for all relevant literature in Scopus, an advanced search query is constructed. The query should filter literature based on the following three elements.
%\begin{itemize}

%    \item The word FPGA is present in the title, abstract or keywords
%    \item Any author is affiliated to a Dutch organization
%    \item The publication date is later than 2019
%\end{itemize}
%These filters are implemented using the following advanced search query.
%\begin{verbatim}
%TITLE-ABS-KEY (FPGA) AND AFFILCOUNTRY (Netherlands) AND PUBYEAR > 2019
%\end{verbatim}
%The query on Scopus resulted in a total of 186 results, which are considered in this survey.

\paragraph{Step 2: Select publishers}
In addition to the Scopus-based search, we conducted an in-depth search for relevant papers in proceedings and articles published by ACM and IEEE. Publications (co-)authored by Dutch institutes, published in 2019 or later, containing both the terms ``FPGA'' and ``HPC'' %anywhere in the text 
were queried for.  
Compared to the initial Scopus-based search, this new search is more specific in terms of keywords (by using one additional keyword), but more inclusive because it considers the entire text for keyword matches. %, instead of only the abstract, title, and list of keywords. 
We used advanced search tools provided by IEEE and ACM \cite{acm_advanced_search} \cite{ieee_advanced_search}, and obtained a list of 65 articles (including possible duplicates). %Performing this step in the discovery process yielded 65 articles, including possible duplicates in the previous step.

\paragraph{Step 3: Select conferences and journals}
%The third step of our literature selection process 
We conducted a final search within topic-relevant conferences and journals;  %As a final step to the discovery phase, a selection of journals and conferences that we deem highly relevant to the subjects covered in this survey 
we restricted the scope of this manual search to conference proceedings and journal issues published in the last 5 years. The following conferences and journals were considered in this step: 
%reviewed by assessing the abstracts of the relevant publications. Issues from 2019 and onward by the following journals and conferences were included in this step:
\begin{itemize}
    \item International Symposium on Field-Programmable Gate Arrays (FPGA)
    \item International Symposium on Field-Programmable Custom Computing Machines (FCCM)
    \item International Conference on Field Programmable Logic and Applications (FPL)
    \item International Conference on Field-Programmable Technology (FPT)
    \item ACM Transactions on Reconfigurable Technology and Systems (TRETS)
\end{itemize}
%Selecting the articles (co-)authored by Dutch institutes in these issues yielded 21 relevant articles, including possible duplicates in the previous steps.
This search delivered a list of 21 publications from Dutch-affiliated authors (including possible duplicates).

\paragraph{Final literature selection}

%Upon completion of the 3-step literature collection process, 
We verified all collected publications for relevance and excluded duplicates. This process yielded a total of 212 relevant publications, which we further classified into research themes. A significant number of these papers (120, over 55\%) focus on FPGA applications. Thus, we selected six applications for further review, based on the number of relevant papers and the significance of FPGAs in the application; the selection  resulted in a shortlist of 49 application papers. These applications are discussed in section \ref{sec:applications}. Other applications which are not reviewed in depth include Network Processing~\cite{Kundel2021OpenBNG:Hardware}, Cryptography~\cite{Massolino2020ASIKE}, Control Systems~\cite{Moonen2021Simulink-BasedSystems} and Weather Prediction~\cite{Singh2020NERO:Modeling}.  % based on their abstract. 
Section~\ref{themes-section} introduces the research themes by which all reviewed papers were categorized. 

%Having gathered a satisfactory number of publications relevant to the scope of this survey, the gathered articles were manually reviewed by reading only the abstracts of each article. This was done to confirm the relevance of the article to the topics of this survey and to gain insight into common themes covered in the gathered literature. After the manual review of the abstracts a total of x (\textit{currently 234}) unique articles are found to have sufficient relevance, and are included in this review. In section \ref{themes-section} the common themes by which the articles are grouped are expanded upon.

