This section discusses deployment challenges and ongoing research efforts in Dutch institutions toward achieving robust FPGA implementations. The robustness of FPGAs and the designs they support is a multi-dimensional challenge, affecting several stages of an FPGA-based systems' production chain. Here, we delve into two critical aspects: reliability (Sec.~\ref{reliabilitysec}) and security (Sec.~\ref{securitysec}). %of FPGA devices and their corresponding designs, 
We discuss the intricacies, and present collaborative efforts required to address these challenges.

\subsection{Reliability}
\label{reliabilitysec}
This section delves into the reliability of FPGA devices %and their designs 
in high-radiation environments, such as space and particle colliders---a topic that is at the forefront of research by ESTEC in the Netherlands.
%
% On the resilience of FPGA (memory) to radiation and on error correction methods, which are often used to counteract these issues.
%
%\subsubsection{Background}
%  How they can help improve research through faster architectural testing, what are the challenges of FPGAs in space
FPGAs, particularly those based on SRAM, are vulnerable to charged or high-energy particles. The Configuration RAM (CRAM) of SRAM-based FPGAs is prone to Single Event Upsets (SEUs), which can alter bits within the configuration memory. These alterations can potentially change the logic functions and connections within the device, leading to Single Event Functional Interruptions (SEFIs). To mitigate these risks, protective measures such as Triple Modular Redundancy (TMR) or Error Correcting Code (ECC) scrubbers are often implemented. However, these solutions can complicate the design and add significant overhead. %Innovations like PyXEL's~\cite{DeSio2023PyXEL:FPGAs} bitstream analysis tool have been developed to enhance FPGA robustness by automating reliability analysis and facilitating mitigation solutions. 
As an alternative, Flash-based FPGAs offer a robust option since their configuration memory is inherently immune to SEUs, although they face constraints in terms of computing resources and less mature toolchains. %The Payload-XL project~\cite{Viel2023Payload-XL:FPGA} exemplifies practical deployment, validating the BRAVE FPGA's performance in orbit and demonstrating the system's real-world reliability under space conditions.

%In the context of Technology Readiness Levels (TRL), 


% Why are FPGAs vulnerable to adverse conditions?
% where can we find these adverse conditions? Space/particle accelerators/etc
% Difference between RTL and device
% introduce TRL

\subsubsection*{\bf{Research topics}}

FPGAs serve as an ideal test platform for advancing the maturity of RTL designs, especially in the context of Technology Readiness Levels (TRL) in space applications. %Initially, 
FPGAs can be 
used to simulate early-stage prototypes, allowing for iterative refinement. In subsequent phases, they can be used to %FPGAs 
validate system components under simulated space-like conditions, ensuring RTL designs work as intended in the actual hardware configuration. In the final stages, FPGAs integrated with RTL designs are tested in operational environments, including actual space missions, demonstrating the system’s performance in real-world conditions. This approach %not only 
provides a flexible and cost-effective method for testing and validating technology, and %but also 
helps managing %the 
risks associated with deploying new technologies in space. Dutch research on FPGA devices focuses primarily on technology validation to assess and enhance their resilience to adverse conditions such as radiation, alongside the development of novel reliability techniques to mitigate vulnerabilities like high susceptibility to SEUs. Additionally, FPGAs are extensively employed for %studied as versatile platforms for 
testing 
low TRL % Technology 
%Readiness Level (TRL)
innovations and for realizing robust operational systems in various applications. The following list elaborates on these core research areas.

%%%%%%%%%%%%%%%%%%%%%%%%%%%%%%%%% Pure draft:

% \begin{itemize}
    % \item Technology validation of FPGA devices:
    \paragraph{Technology validation of FPGA devices}
    This topic assesses the resilience of FPGA devices under the effect of charged and high-energy particles. The Payload-XL project~\cite{Viel2023Payload-XL:FPGA} exemplifies practical deployment, validating the BRAVE FPGA's performance in orbit and demonstrating the system's real-world reliability under space conditions. %~\cite{Viel2023Payload-XL:FPGA}. 
    Studies that %like those 
    use ultra-high-energy heavy ions~\cite{Vlagkoulis2021SingleIons,Du2019UltrahighFPGA} characterize and quantify how various FPGA technologies respond to environmental challenges, such as in space. \citet{Leon2021DevelopmentBenchmarks} employ %Specific areas of interest include using 
    diverse benchmarks to evaluate the performance of new radiation hardened devices. %~\cite{Leon2021DevelopmentBenchmarks}.
    
    % \item Advancements in FPGA Device Reliability
    \paragraph{Advancements in FPGA device reliability}
    Research here develops innovative techniques aimed at enhancing FPGA reliability. %~\cite{DeSio2023PyXEL:FPGAs}. 
    \citet{DeSio2023PyXEL:FPGAs} describe %The %Innovations like 
   % 
   PyXEL's %~\cite{DeSio2023PyXEL:FPGAs}
   bitstream analysis tool for %have been developed to 
   enhancing FPGA robustness by automating reliability analysis and facilitating mitigation solutions.  \citet{Vlagkoulis2022ConfigurationTechnique} present 
   %
   %This includes 
   configuration memory scrubbing methods that integrate mixed 2-D coding techniques, while %~\cite{Vlagkoulis2022ConfigurationTechnique},
\citet{Mousavi2023MTTRSensitivity}
   discuss scrubbing methods %or 
   that strive to reduce the mean time to repair, %~\cite{Mousavi2023MTTRSensitivity}, 
   thereby significantly improving error correction capabilities.
    
    % \item Utilization of FPGAs as Research and Development Platforms: 
    \paragraph{%Utilization of 
    FPGAs as research and development platforms}
    Various studies explore the role of FPGAs as both experimental test beds and final platforms for payload systems~\cite{Viel2023Payload-XL:FPGA,Forlin2023AnSounds,Bohmer2023NeutronFPGAs}. 
    \citet{Gambardella2022AcceleratedTraining} conducted accelerated radiation tests on quantized neural networks, while \citet{Anders2023AProcessors} and \citet{Hoozemans2018IncreasingProcessor} underline the capabilities of FPGAs to support the development of new processor architectures, such as RISC-V processors, and new computing paradigms, such as polymorphic VLIW processors, respectively. 
  
    
   % ~\cite{Gambardella2022AcceleratedTraining} underline FPGAs' capability to support both developmental testing and operational deployment of accelerators and soft-core processors~\cite{Anders2023AProcessors,Hoozemans2018IncreasingProcessor}.
   
% \end{itemize}

\subsubsection*{\bf{Future directions}}

The future of reliability work on FPGAs in the Netherlands is increasingly shaped by the demand from the ESTEC ESA center and the expanding New Space economy for advanced space applications. Dutch companies, such as Technolution, with its FreNox RISC-V soft-core, play a pivotal role in building this high-reliability ecosystem. As new FPGA providers in Europe, like NanoExplore, emerge, the need for technology-agnostic designs that enable the seamless integration of external IPs becomes more critical. The primary focus moving forward will be on expanding this ecosystem by adapting and validating existing common IPs for high-reliability applications, while maintaining ease of integration across various platforms.


\subsection{Hardware security}
\label{securitysec}
% This section explores the complex landscape of FPGA security, focusing on protection strategies in cloud computing and evolving hardware environments. It addresses concerns from cloud service providers and clients about intellectual property (IP) protection and hardware design integrity.

This section explores the security challenges and innovative solutions in FPGA technology.
FPGAs present unique security risks due to their reconfigurability and wide deployment range, from cloud data centers to edge devices. Their flexibility and complex ecosystem makes them susceptible to various threats, including hardware Trojans, fault injection, and side-channel attacks. 

%\subsubsection{Background}

% Security in FPGAs is a multi dimensional problem, you have to consider the security of the design, as well as the security of the device. These dimensions influence each other and play a role in what type of security concerns come to mind, what types of attacks and countermeasures. 

% basically explain the figure, claiming that most/all of the security works will fit into one or more of the trust relationships. Security is part of trust when it is guaranteed as a service.

% Security, especially on the FPGA domain, is based on the guarantee and enforcement of trust and isolation. This trust is propagated through the supply chain of the FPGA market as presented in~\cite{Zhang2014ASystems}. Trust is required between supplier and consumer (e.g., Foundries and FPGA vendors), and propagated to the consumers down the line e.g., a system developer trusts in the FPGA vendor to guarantee its FPGA (or EDA tool) does not have hardware Trojans~\cite{Zeitouni2021TrustedFPGAs,Labafniya2020OnPrevention,Nikiema2023TowardsDevices,Palumbo2022IsAnswer}. The direct relationships of trust are shown in Figure~\ref{fig:FPGA-trust}. 

% However, at a certain point in the chain, the relationship between service provider and consumer are no longer transactional and become operational. In Figure~\ref{fig:FPGA-trust} this is shown by the dotted line. Such is the case with the end user and the system developer, or between the cloud service provider and the end user. In these cases, the interaction between the parties is no longer one-way, and trust is no longer the only issue. In these cases the critical aspects become isolation in the cloud environment~\cite{Zeitouni2021TrustedFPGAs}, protection from malicious users~\cite{Koylu2022ExploitingAttacks,Garaffa2021RevealingNeuron}, or providing countermeasures to attacks~\cite{Socha2020Side-channelHardware}, among many others.

Security, particularly in the realm of FPGAs, hinges on establishing and maintaining trust and isolation. This trust permeates the entire supply chain of the FPGA market, as detailed by \citet{Zhang2014ASystems}. Trust must exist between suppliers and consumers—for instance, between foundries and FPGA vendors—and is subsequently extended to downstream users. For example, system developers rely on FPGA vendors to ensure that their hardware and EDA tools are free from hardware Trojans, as discussed in various studies~\cite{Zeitouni2021TrustedFPGAs,Labafniya2020OnPrevention,Nikiema2023TowardsDevices,Palumbo2022IsAnswer}. The direct relationships of trust within this framework are illustrated in Figure~\ref{fig:FPGA-trust}. We observe that, at certain stages within the supply chain, the dynamics between service providers and consumers go beyond mere transactional interactions and become operational. This shift is represented by the dotted line in the figure. %Figure~\ref{fig:FPGA-trust}. 
Typical examples include the relationship between end users and system developers, or between cloud service providers and end users. In these scenarios, interactions between the involved parties are more reciprocal, and the focus expands beyond trust. Critical issues then include ensuring isolation within cloud environments, protecting against malicious users, and implementing countermeasures to various attacks, as highlighted in recent research~\cite{Zeitouni2021TrustedFPGAs,Koylu2022ExploitingAttacks,Garaffa2021RevealingNeuron,Socha2020Side-channelHardware}.

\begin{figure}
    \centering
    \includegraphics[width=0.7\linewidth]{figures/FPGA_trust_chain.png}
    \caption{Trust chain demonstrated in the FPGA-based system market. The ``interaction barrier'' marks the point where the interactions become complex with trust being provided and required for the transactions. Adapted from~\cite{Zhang2014ASystems}.}
    \label{fig:FPGA-trust}
\end{figure}

\subsubsection*{\bf{Research topics}}

% FPGAs are increasingly used in security-sensitive applications due to their flexibility and reconfigurability. However, securing FPGA configurations in cloud environments poses unique challenges. On one hand, clients need to protect their IP by encrypting configuration bitstreams. However, service providers must ensure that these configurations are not malicious before deployment. A trusted FPGA shell, as demonstrated on a Xilinx VCU118 board, represents a practical solution that balances IP protection with cloud provider requirements by conducting verifiable checks on hardware designs with minimal resource overhead.

% Innovative approaches like evolvable hardware (EH) are also being explored for dynamic configuration changes to combat hardware trojans, capable of autonomously detecting and correcting internal circuit errors. Techniques such as Genetic Programming (GP) and Cartesian Genetic Programming (CGP) enhance FPGA security by adapting circuit configurations in response to environmental inputs, thereby preventing trojan activation.

% In the Netherlands, research in security crosses multiple technical domains and utilizes many techniques in order to provide operational level guarantees. Some of this research focuses on verifying the implicit trust from elements at the left of the dotted line in Figure~\ref{fig:FPGA-trust}. These mainly focus on the search for and disruption of hardware trojans~\cite{Zeitouni2021TrustedFPGAs,Labafniya2020OnPrevention,Nikiema2023TowardsDevices,Palumbo2022IsAnswer}. 
% In the case of hardware trojans, although the FPGA vendor might not be directly responsible for the attack, the relationship of trust is inherited from the upstream members (e.g. foundry).
% % This is represented in Figure~\ref{fig:FPGA-trust} by the d), e), f), g) and i) relations. 

% % represented by relations g), h) and i). 
% Most of the research in the Netherlands is focused in the right-hand side of Figure~\ref{fig:FPGA-trust}. 
% This focuses mainly on the relations between cloud services, system developers, and end users. These include the creation of hardware primitives such as PUFs~\cite{Jin2022ProgrammableApplications,Jin2020ErasableDesign} and Roots-of-Trust~\cite{Nikiema2023TowardsDevices}. The evaluation of Fault Injection and protection mechanisms~\cite{Miteloudi2022ROCKY:Data,Nikiema2023TowardsDevices,Koylu2022InstructionAttacks,Koylu2022ExploitingAttacks}. The detection and protection of Side-Channel Attacks~\cite{Lahr2020SideImplementation,Miteloudi2021EvaluatingLeakage, Socha2020Side-channelHardware, Garaffa2021RevealingNeuron} and $\mu$Architectural attacks~\cite{Arikan2022ProcessorSketches,Nikiema2023TowardsDevices}.


In the Netherlands, research on security spans multiple technical domains, employing a variety of techniques to ensure operational level guarantees.
%{\bf BRUNO, can you add one sentence here to introduce/explain the separation into vendor-side and user-side?}

\paragraph{Vendor-side:} Part of this research is devoted to assessing the implicit trust indicated by elements to the left-hand side of the dotted line in Figure~\ref{fig:FPGA-trust}. These efforts predominantly aim to detect and neutralize hardware trojans, as discussed in various studies~\cite{Zeitouni2021TrustedFPGAs,Labafniya2020OnPrevention,Nikiema2023TowardsDevices,Palumbo2022IsAnswer}. In cases involving hardware trojans, while FPGA vendors might not be directly accountable for the intrusion, they inherit a trust relationship from upstream entities such as foundries.

\paragraph{User-side:}
On the other hand, much of the research focus in the Netherlands is directed toward the interactions depicted on the right-hand side of the dotted line in Figure~\ref{fig:FPGA-trust}. This includes examining the connections among cloud services, system developers, and end users. Key focal areas include the development of hardware primitives like Programmable Unclonable Functions (PUFs)~\cite{Jin2022ProgrammableApplications,Jin2020ErasableDesign} and Roots-of-Trust ~\cite{Nikiema2023TowardsDevices}, as well as the evaluation of fault injection and protection mechanisms ~\cite{Miteloudi2022ROCKY:Data,Nikiema2023TowardsDevices,Koylu2022InstructionAttacks,Koylu2022ExploitingAttacks}. Additionally, significant attention is given to the detection and mitigation of side-channel attacks ~\cite{Lahr2020SideImplementation,Miteloudi2021EvaluatingLeakage, Socha2020Side-channelHardware, Garaffa2021RevealingNeuron} and microarchitectural attacks~\cite{Arikan2022ProcessorSketches,Nikiema2023TowardsDevices}.

\subsubsection*{\bf{Future directions}}

Current research in the Netherlands extensively utilizes FPGA systems across all stages of end development. However, there is relatively less focus on enhancing the resilience and security of FPGA production, likely due to the absence of commercial players developing these systems within the country. As a result, future research is expected to increasingly emphasize the integration of FPGAs into system-level solutions, addressing critical issues in security from an end-user perspective. The inherent flexibility of FPGA platforms also presents significant opportunities for innovative system design. They can serve as an initial gateway into a broader design ecosystem, where the development of intellectual properties (IPs) is validated for functional metrics such as side-channel leakage and protection mechanisms. This approach ensures that new designs meet stringent standards before deployment, fostering a robust and secure technological landscape.