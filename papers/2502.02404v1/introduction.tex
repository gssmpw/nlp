
% Introduce the field of FPGAs in general. What wil be reviewed: FPGA developments in academia and industry in the Netherlands. Specify which questions are to be answered in this paper. Introduce how this paper is divided up into (sub)sections. 

% SURF provided a list of questions, part of which we intend to answer in this survey paper. The questions posed are as follows:
% \begin{enumerate}
%     \item Can you please provide more information about FPGA-based technology in the context of HPC?
%     \item What are the primary applications of this technology? 
%     \item Who are the primary stakeholders in the Netherlands and Europe? 
%     \item In what areas should FPGA-based architectures be deployed, such as data centres, edge computing, network infrastructure, or other places?
%     \item How can programming models be made more accessible?
%     \item How can we collaborate with companies like AMD and Intel?
%     \item Regarding scientific computing applications, what are the strategic advantages of using FPGA-based technologies? Is it more research-based, production-based, or prototyping-based?
%     \item In the next 5-8 years, would it be beneficial for SURF or other organisations to host FPGA-based technology?
%     \item Who needs to be involved and informed in the Netherlands about this initiative, and who will benefit from it?
% \end{enumerate}



%\section{Background [NIKOS,...?]}
% Introduce what problems FPGAs are supposed to solve, why do we need fast dedicated hardware? Problems such as big data computations, high bandwidth data streaming and communication, efficient computing.

% After introducing the problems, introduce the (potential) solution: FPGAs. For those with little prior knowledge about FPGAs: what are FPGAs, how do they work in general and why are the applicable and suitable to these problems.

%\subsection{Societal need for accelerated computing [...?]}
% Outline some major challenges faced in society that require FPGAs as a solution. Relate to known applications such as radio astronomy, machine learning, etc. Briefly describe why these applications require a lot of computation.


%FPGA technology is highly relevant today due to 

The current surge in computationally intensive workloads, such as artificial intelligence (AI) %, machine learning (ML), 
and high-performance computing (HPC), creates an unprecedented need for ever-increasing computational capacity. Most modern-day applications and technologies are data-driven and must process large volumes of data at high speeds, or operate under stringent time constraints. Emerging technologies, such as distributed sensor networks in radio astronomy \cite{big-data-radio-astronomy} and next generation sequencing in genetics \cite{genetic-big-data}, enable the accumulation of vast amounts of data, further intensifying the need for more % to be processed, requiring additional 
computational power. %At the same time, 
Keeping up with the ever-increasing demand for computational capacity requires excessive amounts of energy, at a time when there is an urgent need to reduce unsustainable energy consumption. %that need customized hardware solutions. 
%In the present day there is an increasing demand for computing power. 
AI applications, powered by large language and generative models, are rapidly increasing in scale and complexity, further increasing their energy demand \cite{energy-llm}.
Given the increasing amounts of data to be processed and the high computational requirements of modern-day applications, the energy needed to sustain these applications can often not be supplied exclusively by renewable sources of energy \cite{green-data-centers, enegry-efficiency-cloud-dc}. This  %thereby %. In order to reduce unsustainable energy usage required to run computationally expensive applications, there is an
raises the need for energy-efficient hardware that is able to facilitate the next generation of data-driven applications.

%\subsection{Why use FPGAs? [...?]}
% Introduce why FPGAs might offer a solution to these problems. Outlines what is needed for these type of challenges in terms of hardware, e.g. parallel processing, fast access to memory, distributed computing, etc. Then expand on why FPGAs are suitable for tackling these challenges, also compared to other hardware such as GPUs, CPUs (and ASICs?). Expand on the differences between these types of hardware, to make clear why FPGAs are specifically interesting in some instances. 
To enable the deployment of modern applications that need to process large amounts of data faster, energy efficiently, and/or in real time, hardware acceleration is necessary. Various hardware technologies can be used for this purpose, such as GPUs, FPGAs, and ASICs. % applied to enable accelerated processing. 
A GPU (Graphics Processing Unit) is a massively parallel architecture that comprises numerous small processing cores~\cite{gpuComputing}, making it well-suited for computationally intensive workloads such as graphics rendering and parallel computing, particularly for vector-processing operations. GPUs are widely deployed for accelerating AI/ML model training. %However, a GPU is not the most energy-efficient solution for all applications%for many applications, it is not the most energy efficient solution 
%~\cite{hw-efficiency-compare-2, hw-efficiency-compare}.
%
%in fields such as machine learning and graphics processing this is the most prevalent form of hardware acceleration. 
%The GPU employs the parallelization of a large number of cores, with modern GPUs running thousands of cores, where each core can perform operations in parallel to other cores. Through abundant parallelization, the GPU can achieve a high throughput of basic operations \cite{gpuComputing}. The GPU can be an effective hardware acceleration platform, and is nowadays accessible to implement through programming models such as CUDA, however, for many applications it is not the most energy efficient solution \cite{hw-efficiency-compare-2} \cite{hw-efficiency-compare}. 
An ASIC (Application-Specific Integrated Circuit) %, on the other hand, %
%(ASICs), as their name suggests, 
offers the highest performance and energy efficiency for an application by implementing in hardware only the logic that is actually needed by the target application. Designing and fabricating an ASIC, however, requires an extensive design process and is extremely costly \cite{asic-challenges}. These limitations can be alleviated by using an FPGA (Field Programmable Gate Array). FPGAs are frequently used instead of GPUs as well, because of their I/O capabilities, e.g., optical links or direct network connections.

FPGAs have revolutionized digital design and prototyping due to their versatility and adaptability to varying computational requirements. Unlike ASICs, FPGAs are reprogrammable and can be reconfigured after manufacturing multiple times, allowing for cost-effectively realizing highly efficient custom computer architectures. FPGA technology relies on an array of configurable logic blocks that are interconnected through programmable routing channels; the configuration of these elements results in the implementation of specialized digital circuits that are tailored to the specific requirements of a particular domain, application, or even workload. Eliminating unnecessary microarchitecture-level components and optimizing the design for a specific domain or application leads to considerably higher performance and energy efficiency than general-purpose processors without the prohibitively high costs for manufacturing an ASIC.

%can offer hardware acceleration by implementing logic in hardware that is specific to an application. Customized ASICs can offer the most performance when it comes to hardware acceleration, but designing and fabricating an ASIC requires an extensive design process and is extremely costly \cite{asic-challenges}. For these reasons, ASICs cannot be used to offer flexible hardware acceleration for various applications. This limitation is alleviated when using a Field Programmable Gate Array (FPGA). The FPGA is a widely applicable device, since, unlike on the ASIC, the logic on an FPGA can be reconfigured on-the-fly to emulate specific hardware architectures, albeit at the cost of performance when compared to an ASIC. By optimizing the hardware architecture for specific applications, the FPGA can be both more energy efficient, and gain greater acceleration than a GPU, without the overhead that designing an ASIC requires. The challenge that comes with using FPGAs for hardware acceleration is that designing custom hardware architectures introduces increased complexity, compared to solutions designed solely in software. Hardware architecture design for specific applications can be a time-costly and specialized process compared to designing a solution in software that can run on a processor such as a GPU or CPU. For this reason, tools are being researched and designed which increase the accessibility of implementing FPGAs. By means of these tools, a larger number of engineers are able to deploy FPGAs to accelerate their application, resulting in optimized, energy efficient hardware architectures, which can be realized on FPGA hardware.

% why the netherlands? because the NL has greate presence in technological development. sustain this, find some eu project where NL has a presence.

%\subsection{Dutch research towards FPGAs [Sjoerd, ...?]}

\paragraph{Context} The unique ability of modern FPGA technology to balance performance and power efficiency through customization makes it a pivotal technology in addressing the diverse and evolving challenges of today's highly heterogeneous computing landscape. Research in the Netherlands plays a major role in developing new innovative FPGA technologies. The country has a strong reputation for research excellence, particularly in areas such as technology, engineering, agriculture, environmental science, and healthcare. Dutch universities and research institutions are actively involved in collaborative international projects, fostering a culture of knowledge exchange and cooperation, while Dutch companies contribute to global initiatives and advancements in fields like sustainable energy, water management, and digital technology. Besides universities and industry, public institutes such as the European Space Research and Technology Center (ESTEC), the National Institute for Nuclear and High energy physics (Nikhef), and the Netherlands Institute for Radio Astronomy (ASTRON) also conduct FPGA-related research. Figure~\ref{fig:org-publications} provides an overview of Dutch organizations that conduct research in this area, and the number of recent publications per organization. 
The figure shows that %the public organizations are the main contributing publishers, 
research by universities and ESTEC constitutes the majority of the relevant published work. Two private companies, IMEC NL and KPN, have also contributed with at least two publications, indicating that private organizations also play a role in the scientific development of FPGA technology. Overall, FPGA technology is an active field of research in the Dutch scientific community, with relevance in academia and industry. 

\begin{figure}[t]
    \centering
    \includegraphics[width=\textwidth]{figures/total_no_papers_2.pdf}
    \caption{The number of FPGA related publications published by Dutch organizations within the past 5 years. Only organizations with at least two relevant publications are shown.}
    \label{fig:org-publications}
\end{figure}

Advancing FPGA technology aligns with European goals, and strengthens the position of the Netherlands in a future where efficient large-scale computing will be increasingly important. %This survey further explores the themes and subjects that current FPGA related research in the Netherlands is focused on. Through this survey we aim to map out the fields of Dutch research where FPGA technology is developed and applied, in both academia and industry, enabling collaborations and highlighting the importance of FPGA researching with a perspective on future applications.
In the past decade, the Netherlands has made significant scientific contributions to a large number of European research projects, accelerating the development of various technologies. Within the European Horizon 2020 funding program, the Netherlands has been %among the most 
successfully %countries in 
participating %fulfilling 
in scientific European projects %grants 
when normalized to the number of scientific person-year effort in the country. Within Horizon 2020, the Netherlands has made a comparatively high contribution to the program pillar focusing on Excellent Science \cite{rathenauNederlandHorizon}. Current major strategic goals outlined by the European Commission are the development of autonomous technologies, such as artificial intelligence, and the development of technologies to combat climate change \cite{rathenauEuropeseWetenschap}. Both of these goals align with developing energy efficient hardware acceleration. The Netherlands also houses the European Space Research and Technology Center (ESTEC), which is the main center for research and development of the European Space Agency (ESA). Space applications are another specific field where FPGA technology has a prominent role. The flexibility and performance that FPGAs offer aligns with the needs of space applications. Recently, space-grade FPGA technology has been developed as a European research effort, strengthening the application of FPGAs in space projects \cite{EuEsaSpaceGradeFpgProject}.  Overall, Dutch contributions to research and technology are important on both European and global scale in various scientific and technological domains.

\paragraph{Survey focus} Harnessing the full potential of advanced FPGA-based systems demands more than just comprehending the inherent capabilities and limitations of FPGA technology. It necessitates a deep understanding of how these attributes align with and serve the diverse computational requirements across various domains and applications. To this end, we survey the present landscape of FPGA innovation research in the Netherlands.  %, a country that plays a significant role in research and technology, contributing valuable innovations and insights that are important for both Europe and the world. 
%The goal of this survey is to %provide insights into the current landscape of FPGA innovation research and 
%serve as the basis for informing and possibly guiding future national-level investments. 
By delving into the ongoing projects, advancements, and breakthroughs in the field, this survey aims to provide valuable insights that go beyond a mere snapshot of the present landscape. More importantly, it aspires to serve as a foundational resource, possibly guiding future national-level investments in FPGA technology. By understanding the strengths, challenges, and emerging trends in FPGA research, policymakers, researchers, and industry stakeholders can make informed decisions about allocating resources and shaping strategies that will contribute to the continued growth and impact of FPGA technology. This survey endeavors to be a pivotal tool in fostering a forward-looking approach, ensuring that the nation remains at the forefront of FPGA innovation and leverages its potential for future technological advancements. Considering the relevance of FPGA technology in the European and Dutch context, there is ample reason for the Netherlands to invest in FPGA technology.
% Give a rough outline of the role of Dutch institutes and companies in FPGA development. How this research and development can be supported by making HPC accelerated with FPGAs accessible to the institutes and companies.

%FPGAs have been a common theme of research at various universities and institutes in the Netherlands, and have widely been used in applications designed by companies in the Netherlands. Besides universities, public institutes such as ESTEC, the National Institute for Nuclear and High energy physics (Nikhef) and the Netherlands Institute for Radio Astronomy (ASTRON) have publish FPGA related research. Figure \ref{fig:org-publications} gives an overview of Dutch organizations with at least two publications relevant to FPGA innovation, that have been gathered in this survey. The figure shows the number of relevant publications that each organization has contributed.
%\begin{figure}[!htb]
%    \centering
%    \includegraphics[width=\textwidth]{figures/most-cited-dutch-institutes.png}
%    \caption{Number of publications {\bf MORE INFO} relevant to this review per organization. Showing all organizations with at least two relevant publications.}
%   \label{fig:org-publications}
%\end{figure}
%Figure \ref{fig:org-publications} shows that the public organizations are the main contributing publishers, research by universities and ESTEC constitutes the majority of the relevant published work. Two private companies, namely IMEC and KPN, have also contributed more than one published work, indicating that private organizations also play a role in the scientific development of FPGA technology. Overall, FPGA technology is an active field of research in the Dutch scientific community, with relevance in academia and industry. Advancing FPGA technology aligns with European goals, and strengthens the position of the Netherlands in a future where efficient large scale computing will be increasingly relevant. This survey further explores the themes and subjects that current FPGA related research in the Netherlands is focused on. Through this survey we aim to map out the fields of Dutch research where FPGA technology is developed and applied, in both academia and industry, enabling collaborations and highlighting the importance of FPGA researching with a perspective on future applications.