


%\zaidnote{Please mention the methodology that we followed to get to these conclusion: like the survey and the discussions. Steven: we interviewed several industry partners in the Netherlands, discussed the initial survey results with them and included their feedback here.}


To understand the alignment between FPGA research in Dutch academia and industry, a questionnaire was initially distributed to Dutch companies working in FPGA technology. Nine companies responded, representing a broad range of FPGA applications, from custom hardware architecture development to programming models, tools, and application-specific designs. Thereafter, we conducted interviews with several industry representatives in the Netherlands, reviewed the initial survey findings with them, and integrated their feedback into this analysis. The survey posed nine questions, addressing the following subjects:
\begin{enumerate}
    \item The themes and applications where companies apply FPGAs (question 1 and 2)
    \item The reasons for companies to choose FPGA technology (question 3)
    \item To what extend the companies create custom FPGA designs (question 4)
    \item The satisfaction of companies with respect to the available FPGA technology on the market (question 5 and 6)
    \item The most beneficial advances in FPGA technology to companies (question 7)
    \item The involvement of companies in academic research (question 8 and 9)
\end{enumerate}
The companies involved and consented to their names being published are ProDrive Technologies, HDL Works, Sioux Technologies, Technolution, QBayLogic, Demcon and Core-Vision\footnote{\url{https://prodrive-technologies.com}, \url{https://hdlworks.com}, \url{https://sioux.eu}, \url{https://technolution.com}, \url{https://qbaylogic.com}, \url{https://demcon.com}, \url{https://www.core-vision.nl}}. Other companies choose to remain anonymous. 

\begin{table}[ht]
\caption{Most cited applications where FPGAs are used, and reasons for preferring a design involving FPGAs}
\label{tab:survey_results}
\resizebox{\columnwidth}{!}{%
\begin{tabular}{lc|lc}
\textbf{Applications for FPGA design} & \textbf{Companies} & \textbf{What makes FPGAs the preferred technology} & \textbf{Companies} \\ \hline
Control systems                       & 7                       & High degree of parallelism                         & 4                       \\
Near memory processing                & 3                       & High bandwidth and throughput                      & 3                       \\
Reliability and resilient design      & 3                       & Flexibility of development                         & 3                       \\
Machine learning                      & 2                       & Custom hardware interfacing                        & 2                       \\
Distributed computing                 & 2                       & Low-latency                                        & 2                       \\
Coarse grained architecture design    & 2                       & Affordable compared to ASIC design             & 2                      
\end{tabular}%
}
\end{table}

The most frequently cited applications for FPGAs are control systems, near-memory processing and resilient designs. Table \ref{tab:survey_results} shows an overview of the most cited applications (each company could submit multiple applications). Most companies also reported that they use existing FPGA technology and software to develop custom IP designs for their products, while a fraction (three companies) also highlighted their focus on custom software development to create FPGA solutions.

Companies cite a variety of reasons for choosing FPGAs over GPUs or CPUs in their solutions. The most frequently mentioned motivations are shown in table \ref{tab:survey_results} (there can be more than one motivation per company). These motivations align with the typical advantages of FPGAs, including high parallelism, high throughput, and substantial processing bandwidth. Additionally, companies appreciate the flexibility in hardware design that FPGAs offer, allowing for adjustments throughout a product's lifetime without major redesigns. Several companies also highlight the deterministic behavior and precise timing of FPGAs as key factors in their decision-making. Furthermore, the cost-effectiveness of FPGAs for implementing custom hardware designs—especially for low-volume products, compared to developing a custom ASIC—plays a significant role in their preference. Other advantages mentioned include flexible hardware interfacing, low latency, low power consumption, and sufficient I/O bandwidth.



Most companies express general satisfaction with the current state of accessible FPGA technology, with only two reporting dissatisfaction. However, they identify several developments that could enhance their FPGA designs, such as on-chip memory for smaller FPGA devices, improved quality of IP cores, better software for managing HDL dependencies and packages, and high-level languages that retain the details of hardware design. Many companies suggest that FPGA technology would be more attractive if high-level design efforts shifted focus from making FPGAs easier for software developers to enhancing the capability of hardware designers to work at a higher level of abstraction. Overall, tooling is regarded as a critical area for improvement. Frequent changes in vendor tooling often lead companies to develop custom solutions. Additionally, vendor-specific tools create varying workflows, making it challenging to work with devices from different manufacturers. While open-source tools exist, they are generally not considered suitable for professional use.





In conclusion, companies employ FPGA technology when their solutions require the specific advantages that FPGAs provide, such as low latency, high processing bandwidth, deterministic behavior, and flexible interfacing. The accessibility and adoption of FPGAs in the industry would significantly improve with the development of vendor-agnostic, open-source toolchains that are suitable for professional use. Moreover, creating a high-level programming language tailored for hardware designers, instead of focusing on accessibility for software developers, would greatly enhance FPGA design in industrial applications.