


This paper presented an overview of the current research landscape, applications, and future potential of FPGA technology in the Netherlands. It highlights how academia and industry in the Netherlands play an important role in developing new and innovative FPGA-based technologies and solutions to address various important societal challenges ranging from healthcare to power efficiency. We gathered a total of %234
212 FPGA-related papers published in the past 5 year, classifying % and classifies the current research on done FPGAs 
them into five major themes: a) FPGA architecture, with 11 published papers, b) data center infrastructure \& HPC, with 40 papers, c) programming models \& tools, with 15 papers, d) robustness of FPGAs, with 26 papers, and e) applications, with 120 papers. Applications with a significant number of relevant papers and where FPGAs play an important role were reviewed in depth, leaving 49 application papers over 6 subjects. Most contributions to this research effort stem from 16 organizations, including major Dutch universities, research institutes, and % as well as 
industry. 
%In addition, the paper 
This survey indicates that FPGAs have emerged as powerful accelerators for a wide range of applications. Specific application domains that engender significant FPGA research interest include: machine learning, astronomy, particle physics experiments, quantum computing, space applications,
and bioinformatics. 

The paper emphasizes that while FPGA technology holds tremendous potential, there are ongoing challenges, particularly in terms of development tools, performance predictability, and hardware security. Future research will need to focus on improving these aspects, especially with regards to user-friendly programming models and more robust performance estimation frameworks. The continued investment in FPGA technology will be vital for maintaining the Netherlands' competitive edge and addressing the growing demand for energy-efficient, high-performance computing solutions. Furthermore, fostering open-source tools, as well as deeper industry-academia collaboration, will be essential for sustaining long-term growth and innovation in this rapidly evolving field.