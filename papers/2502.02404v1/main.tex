%%
%% This is file `sample-manuscript.tex',
%% generated with the docstrip utility.
%%
%% The original source files were:
%%
%% samples.dtx  (with options: `manuscript')
%% 
%% IMPORTANT NOTICE:
%% 
%% For the copyright see the source file.
%% 
%% Any modified versions of this file must be renamed
%% with new filenames distinct from sample-manuscript.tex.
%% 
%% For distribution of the original source see the terms
%% for copying and modification in the file samples.dtx.
%% 
%% This generated file may be distributed as long as the
%% original source files, as listed above, are part of the
%% same distribution. (The sources need not necessarily be
%% in the same archive or directory.)
%%
%% Commands for TeXCount
%TC:macro \cite [option:text,text]
%TC:macro \citep [option:text,text]
%TC:macro \citet [option:text,text]
%TC:envir table 0 1
%TC:envir table* 0 1
%TC:envir tabular [ignore] word
%TC:envir displaymath 0 word
%TC:envir math 0 word
%TC:envir comment 0 0
%%
%%
%% The first command in your LaTeX source must be the \documentclass command.
\documentclass[acmsmall,screen,review=false,authorversion,nonacm]{acmart}

\colorlet{RED}{red}
\newcommand{\zaidnote}[1]{ {\textcolor{red} { ***Zaid: #1 }}}


\title[FPGA Innovation Research in the Netherlands]{FPGA Innovation Research in the Netherlands: \\Present Landscape and Future Outlook}

%%
%% \BibTeX command to typeset BibTeX logo in the docs
\AtBeginDocument{%
  \providecommand\BibTeX{{%
    \normalfont B\kern-0.5em{\scshape i\kern-0.25em b}\kern-0.8em\TeX}}}

%% Rights management information.  This information is sent to you
%% when you complete the rights form.  These commands have SAMPLE
%% values in them; it is your responsibility as an author to replace
%% the commands and values with those provided to you when you
%% complete the rights form.
\setcopyright{none} %{acmlicensed}
\makeatletter
\let\@authorsaddresses\@empty
\makeatother
% \copyrightyear{2018}
% \acmYear{2018}
% \acmDOI{XXXXXXX.XXXXXXX}

%% These commands are for a PROCEEDINGS abstract or paper.
% \acmConference[Conference acronym 'XX]{Make sure to enter the correct
%   conference title from your rights confirmation emai}{June 03--05,
%   2018}{Woodstock, NY}
% \acmISBN{978-1-4503-XXXX-X/18/06}


%%
%% Submission ID.
%% Use this when submitting an article to a sponsored event. You'll
%% receive a unique submission ID from the organizers
%% of the event, and this ID should be used as the parameter to this command.
%%\acmSubmissionID{123-A56-BU3}

%%
%% For managing citations, it is recommended to use bibliography
%% files in BibTeX format.
%%
%% You can then either use BibTeX with the ACM-Reference-Format style,
%% or BibLaTeX with the acmnumeric or acmauthoryear sytles, that include
%% support for advanced citation of software artefact from the
%% biblatex-software package, also separately available on CTAN.
%%
%% Look at the sample-*-biblatex.tex files for templates showcasing
%% the biblatex styles.
%%

%%
%% The majority of ACM publications use numbered citations and
%% references.  The command \citestyle{authoryear} switches to the
%% "author year" style.
%%
%% If you are preparing content for an event
%% sponsored by ACM SIGGRAPH, you must use the "author year" style of
%% citations and references.
%% Uncommenting
%% the next command will enable that style.
%%\citestyle{acmauthoryear}
%%
%% end of the preamble, start of the body of the document source.
\begin{document}

%%
%% The "title" command has an optional parameter,
%% allowing the author to define a "short title" to be used in page headers.
%or

%%
%% The "author" command and its associated commands are used to define
%% the authors and their affiliations.
%% Of note is the shared affiliation of the first two authors, and the
%% "authornote" and "authornotemark" commands
%% used to denote shared contribution to the research.
\author{Nikolaos Alachiotis}
\email{n.alachiotis@utwente.nl}
\orcid{0000-0001-8162-379}
\affiliation{%
  \institution{University of Twente}
  \streetaddress{}
  \city{}
  \state{}
  \country{The Netherlands}
  \postcode{}
}

\author{Sjoerd van den Belt}
\email{s.p.vandenbelt@utwente.nl}
\orcid{0009-0005-8812-9929}
\affiliation{%
  \institution{University of Twente}
  \streetaddress{}
  \city{}
  \state{}
  \country{the Netherlands}
  \postcode{}
}

\author{Steven van der Vlugt}
\email{vlugt@astron.nl}
\orcid{0000-0001-6834-4860}
\affiliation{%
  \institution{Netherlands Institute for Radio Astronomy (ASTRON)}
  \streetaddress{Oude Hoogeveensedijk 4}
  \city{Dwingeloo}
  \state{Drenthe}
  \country{the Netherlands}
  \postcode{7990 AA}
}

\author{Reinier van der Walle}
\email{walle@astron.nl}
\orcid{0009-0005-8097-6170}
\affiliation{%
  \institution{Netherlands Institute for Radio Astronomy (ASTRON)}
  \streetaddress{Oude Hoogeveensedijk 4}
  \city{Dwingeloo}
  \state{Drenthe}
  \country{the Netherlands}
  \postcode{7990 AA}
}

\author{Mohsen Safari}
\email{mohsen.safari@surf.nl}
\orcid{0000-0003-0839-3251}
\affiliation{%
  \institution{SURF}
  \streetaddress{}
  \city{}
  \state{}
  \country{the Netherlands}
  \postcode{}
}

\author{Bruno Endres Forlin}
\email{b.endresforlin@utwente.nl}
\orcid{0000-0003-4822-1841}
\affiliation{%
  \institution{University of Twente}
  \streetaddress{}
  \city{Enschede}
  \state{}
  \country{the Netherlands}
  \postcode{}
}

\author{Tiziano De Matteis}
\email{t.de.matteis@vu.nl}
\orcid{0000-0002-9158-6849}
\affiliation{%
  \institution{Vrije Universiteit Amsterdam}
  \streetaddress{}
  \city{Amsterdam}
  \state{}
  \country{The Netherlands}
  \postcode{}
}

\author{Zaid Al-Ars}
\email{z.al-ars@tudelft.nl}
\orcid{0000-0001-7670-8572}
\affiliation{%
  \institution{Delft University of Technology}
  \streetaddress{}
  \city{Delft}
  \state{}
  \country{the Netherlands}
  \postcode{}
}

\author{Roel Jordans}
  \email{r.jordans@tue.nl}
  \affiliation{
  \institution{Eindhoven University of Technology}
  \streetaddress{Den Dolech 2}
  \city{Eindhoven}
  \country{the Netherlands}
}

\additionalaffiliation{
  \institution{Radboud Radiolab, Institute for Mathemathics, Astrophysics and Particle Physics (IMAPP), Radboud University}
  \streetaddress{Heyendaalseweg 135}
  \city{Nijmegen}
  \country{the Netherlands}
}

\author{Ant\'onio J. Sousa de Almeida}
\email{ajsousal@gmail.com}
\orcid{0000-0002-7024-2262}
\affiliation{%
  \institution{University of Twente}
  \streetaddress{}
  \city{}
  \state{}
  \country{the Netherlands}
  \postcode{}
}

\author{Federico Corradi}
\email{f.corradi@tue.nl}
\orcid{0000-0002-5868-8077}
\affiliation{%
  \institution{Eindhoven University of Technology}
  \streetaddress{}
  \city{}
  \state{}
  \country{the Netherlands}
  \postcode{}
}

\author{Christiaan Baaij}
\email{christiaan@qbaylogic.com}
\affiliation{%
  \institution{QBayLogic B.V.}
  \streetaddress{}
  \city{}
  \state{}
  \country{the Netherlands}
  \postcode{}
}

\author{Ana-Lucia Varbanescu}
\email{a.l.varbanescu@utwente.nl}
\orcid{0000-0002-4932-1900}
\affiliation{%
  \institution{University of Twente}
  \streetaddress{}
  \city{}
  \state{}
  \country{The Netherlands}
  \postcode{}
}

%\author{\textcolor{red}{\textbf{?????!!!!!!???ADD YOUR NAME ????!!!!? ???!!}}}
%\email{}
%\affiliation{%
%  \institution{}
%  \streetaddress{}
%  \city{}
%  \state{}
%  \country{the Netherlands}
%  \postcode{}
%}

%%
%% By default, the full list of authors will be used in the page
%% headers. Often, this list is too long, and will overlap
%% other information printed in the page headers. This command allows
%% the author to define a more concise list
%% of authors' names for this purpose.
\renewcommand{\shortauthors}{Alachiotis et al.}

%%
%% The abstract is a short summary of the work to be presented in the
%% article.
\begin{abstract}

FPGAs have transformed digital design by enabling versatile and customizable solutions that balance performance and power efficiency, yielding them essential for today’s diverse computing challenges.
Research in the Netherlands, both in academia and industry, plays a major role in developing new innovative FPGA solutions. This survey presents the current landscape of FPGA
innovation research in the Netherlands by delving into ongoing projects, advancements, and
breakthroughs in the field. Focusing on recent research outcome (within the past 5 years), we have identified five key research areas: a) FPGA architecture, b) FPGA robustness, c) data center infrastructure and high-performance computing, d) programming models and tools, and e) applications. 
This survey provides in-depth insights beyond a mere snapshot of the current innovation research landscape by highlighting future research directions within each key area; these insights can serve as a foundational resource to inform potential national-level investments in FPGA technology.

\end{abstract}

%%
%% The code below is generated by the tool at http://dl.acm.org/ccs.cfm.
%% Please copy and paste the code instead of the example below.
%%

% \begin{CCSXML}
% <ccs2012>
%    <concept>
%        <concept_id>10002944.10011122.10002945</concept_id>
%        <concept_desc>General and reference~Surveys and overviews</concept_desc>
%        <concept_significance>500</concept_significance>
%        </concept>
%    <concept>
%        <concept_id>10010520.10010521.10010542.10010543</concept_id>
%        <concept_desc>Computer systems organization~Reconfigurable computing</concept_desc>
%        <concept_significance>500</concept_significance>
%        </concept>
%  </ccs2012>
% \end{CCSXML}

% \ccsdesc[500]{General and reference~Surveys and overviews}
% \ccsdesc[500]{Computer systems organization~Reconfigurable computing}





%%
%% Keywords. The author(s) should pick words that accurately describe
%% the work being presented. Separate the keywords with commas.
% \keywords{Field Programmable Gate Array, Architecture, Data center, High Performance Computing, Robustness, Applications}

% \received{18 November 2024}
% \received[revised]{12 March 2009}
% \received[accepted]{5 June 2009}

%%
%% This command processes the author and affiliation and title
%% information and builds the first part of the formatted document.
\maketitle

% remove when finished
%\newpage
%\tableofcontents
%\newpage




\section{Introduction}
\documentclass[../main.tex]{subfiles}
\graphicspath{{../images/}}
\makeatletter
\def\input@path{{../images/}}
\makeatother
\begin{document}
\section{Introduction}
\begin{figure}
\centering
\begin{tikzpicture}
\node[inner sep=0pt] (ws) at (0, 0) {
\includegraphics[height=.4\textwidth, trim={10cm 0 10cm 0},clip]{world_space.png}};
\node[inner sep=0pt] (cs) at (6,0) {\includegraphics[height=.4\textwidth, trim={10cm 1cm 10cm 4cm},clip]{conf_space.png}};
\end{tikzpicture}
\vspace{-5pt}
\label{fig:pbrm_intro}
\caption{\textbf{Left}: Shows world space obstacles as grey spheres. Robots start and goal configuration is colored red and green, respectively. Configurations along the computed path are colored transparent blue. \textbf{Right:} Mapped world space scenario to configuration space. Obstacle region is the grey mesh. Red spheres are collision-free regions computed by the neural SCDF. The optimized shortest path in the convex corridor is the blue curve.}
\vspace{-25pt}
\end{figure}
Motion planning is the problem of finding a collision-free trajectory that connects a given start and goal configuration. The planning takes place in the configuration space of the robot. For single body robots, like mobile robots or drones, the configuration space and the world space are usually the same. This simplifies the planning, since explicit obstacle representations are available which enables geometrical tools like separating hyperplanes, smallest distance to obstacles etc., to be used when designing motion planning algorithms. For multi-body robots like manipulators, the situation is completely different. The world space obstacles are usually mapped to non-convex regions, and to make the problem even harder, the mapping is usually not known. Forming explicit representations of the obstacle region in the configuration space is usually too expensive or intractable. Despite all of this, sampling based planners are used with great success, which mainly is due to their use of implicit representations of the obstacle region. The basic idea is to construct a graph in the configuration space that covers and connects the collision-free region. From this graph, a path can be extracted that connects a given start and goal configuration. The approach is computationally expensive, since the graph is constructed with the smallest geometrical building block available, points, which represents a collision-check. Furthermore, the extracted paths from the graph are non-smooth and jagged due to the stochastic nature of the approach. This adds an additional post-processing step to the process, where the paths are shortcutted and smoothened, before the path can be used for tracking. Clearly a lot of time is invested to form this graph and produce smooth paths. Thus, if the obstacles start to move, then all of this work is done in no use, since all points that make up this graph need to be re-verified, which is simply too time consuming to be done in real time.
\\\\
In this work, we want to address the existing drawbacks of the sampling based planners. Our main contribution is an improved motion planner where each vertex in the graph covers a collision-free region in the form of a sphere instead of a point and where the edges are formed with neighboring intersecting spheres. This representation has the advantage of instead of returning piecewise linear paths, returning a sequence of overlapping spheres, i.e. a convex corridor, that connects a given start and goal configuration, illustrated in Figure \ref{fig:pbrm_intro}. This convex corridor allows us to use convex optimization to produce smooth trajectories, instead of computationally expensive post-processing methods. The representation further allows us to estimate the coverage of the collision-free space, which gives us awareness and feedback in the offline roadmap construction phase. Finally, our representation is simple to adapt to moving obstacles, simply requery for the new radii and recheck for intersections. 
\\\\
The spherical collision-free regions are formed using a signed distance function (SDF), which is a function that returns the smallest distance from an arbitrary point to the boundary of an obstacle. As the name implies, the distance is signed, thus if the point is inside the obstacle it is negative otherwise positive. If the distance is positive, a sphere with radius equal to the distance is guaranteed to cover a collision-free region. Using an SDF in motion planning is not new, but what is novel about our approach is that we express the distance in the configuration space instead of the world space and by doing so allows us to form these convex collision-free regions. We refer to the resulting SDF as a signed configuration distance function (SCDF). Computing an SCDF analytically is non-trivial, our approach is therefore to parameterize the SCDF with a deep neural network and learn the mapping by supervised learning. Our resulting neural SCDF can compute distances for different parameter values of obstacle shapes and we also show how multiple distances can be combined, thus making our approach flexible.
\section{Related work}
Motion planning algorithms can roughly be divided into three families, grid-based, sampling based and optimization based methods. Grid-based methods (GBM) discretize the planning space from which a graph is then compiled. A standard search method is A$^\star$ \citep{a_star}, which is classified as an \textit{informed} search method, since it employs a heuristic function to speed up the search. A$^\star$ guarantees to return an optimal path at the level of discretization used. GBMs usually discretize the planning space by a regular lattice and this limits the GBMs to problems with low dimensionality due to the curse of dimensionality. Thus, GBMs are usually limited to single-body robots where the degrees of freedom (DOF) are low. To overcome the inherent scaling problem with the GBMs, stochastic methods are usually used for multi-body robots. These methods are termed as sampling-based methods (SBM) and core members within this family are the rapidly-exploring random trees (RRT) \citep{rrt} and the probabilistic roadmap (PRM) \citep{prm}. RRT grows a tree from the start configuration and explores the collision-free region in a rapid way until it is able to connect to the goal region. RRT is usually improved by bi-directional planning \citep{rrt_connect}, i.e. an additional tree is grown from the goal configuration and the trees are tested for connection after any tree has been expanded. RRT is a single-query method, thus it searches for a path from scratch each time it is queried. Contrary to this, PRM is a multi-query method, which solves for multiple queries without starting from scratch. PRM does this by creating a roadmap (graph) that covers the collision-free space as an offline step. The graph is then used to solve for multiple queries. PRMs are used in cases where the environment does not change since the extra offline step is too computationally costly and needs to be re-done if the environment is changed. In our work, we address this inherent issue by using a different roadmap representation. Our vertices in the graph cover a collision-free region in the form of spheres and we form the edges by checking for intersecting spheres. If something in the environment changes, we recompute the spheres radii and recheck the intersections, without relying on collision detection. We use a trained neural network to compute the sphere radius, therefore querying for the radius can be done fast, hence our representation enables the PRM for dynamic environments.
\\\\
In the recent decades, optimization based methods (OBM) \citep{chomp, schulman, itomp, stomp} have been introduced as an alternative to SBM for multi-body robots. Like the SBM, the OBMs scale well to higher dimensional problems and produce smoother motion. It is common to use a SDF in the optimization since it is a smooth function, thus enabling gradient-based methods. However, the standard way of expressing the SDF is in world space. The distance therefore needs to be mapped to the configuration space by the forward kinematics. This mapping makes the optimization problem a non-linear program (NLP), which is computationally expensive to solve. Recently, a different approach has been proposed. In \cite{mp_gcs} motion planning is formulated as a convex optimization problem by using the graph of convex sets framework \citep{gcs}. The underlying idea is to decompose the collision-free space into intersecting convex sets from which a convex optimization problem is formulated. In cases where an explicit representation of the obstacles in the configuration space exists, like for single-body robots, creating collision-free convex regions can be done fast \citep{iris}. For multi-body robots, this is non-trivial. Existing work does this successfully \citep{iris_nlp, iris_c} by an optimization based approach, but the methods are still too time consuming to be used in the presence of moving obstacles. Our approach is instead to use deep learning to learn an SDF expressed in the configuration space. With this, we can query for shortest distances to the collision boundary, which allows us to expand spherical regions which are collision-free. Our approach is fast and therefore enables our suggested roadmap planner to be used in dynamic environments.
\\\\
Recent research has focused on learning collision detection \citep{fk_kernel_distance, diffco, graphdistnet} by predicting the signed distance between the robot links and the surrounding obstacles in the world space. The learned SDF is used in trajectory optimization but since the distance is expressed in the world space, the problem becomes an NLP and therefore takes a long time to solve. We take a novel approach and suggest to instead express the signed distance in the configuration space. This allows us to improve the PRM at the same time as it enables convex optimization for trajectory optimization, which runs faster and is more reliable than NLP solvers. In \cite{cspf} a learned signed distance function in the configuration space is proposed similar to our approach. However, their approach is restricted to point cloud representations, while we propose to represent the obstacles as parameterized geometric shapes, e.g. spheres. Furthermore, we also show how to use our learned SCDF to improve an existing roadmap planner.
\section{Problem formulation}
A robot is located in the world space, $\W \subset \R^3 $. The unique location of the robot is given by its configuration $\q \in \C$, where $\C$ is the configuration space. The set of points covered by the robots bodies at a certain configuration is expressed as $\B(\q) \subset \W$. The robot is surrounded by $\NrObst$ obstacles $\O = \bigcup_{i=1}^{\NrObst} \O_i$, where  $\O_i \subset \W$. The representation of the obstacle in the configuration space is the set $\C\O_i = \{\q \in \C \: |\: \B(\q) \cap \O_i \neq \emptyset \}$. The obstacle space is formed as $\Co = \bigcup_{i=1}^{\NrObst} \C \O_i$. The complement is referred to as the free space, $\Cf = \C \setminus \Co$. The path planning problem is a tuple, ($\Cf$, $\qStart$, $\qGoal$), where we want to connect a query pair, consisting of a start, $\qStart$, and goal configuration, $\qGoal$, with a geometric path, $\q(s): [0, 1] \mapsto \Cf$, such that $\q(0)=\qStart$ and $\q(1)=\qGoal$, or report correctly when such a path does not exist.
\end{document}


\section{Survey Method}
% \begin{figure}
%     \centering
%     \includegraphics[width=0.5\linewidth]{Move_teaser.pdf}
%     \caption{Comparison of different dynamic compute approaches. length of arrow indicates residual transformation per token while width indicates velocity of transformation.}
%     \label{fig:enter-label}
% \end{figure}

\section{Method}
\label{sec:method}
Residual connections play a crucial role in shaping token representations, yet their dynamics remain underexplored in the context of efficient decoding. In this work, we delve deeper into transformer residual dynamics and investigate how modulating residual transformation velocity can improve inference efficiency in token-level processing, optimizing both dense and sparse MoE transformers.


\subsection{Residual Dynamics and Motivation for Multi-rate Residuals} \label{sec:motivation}

To analyze how hidden representations evolve across different layers of a transformer architecture, it's crucial to consider the effect of residual connections. Each transformer decoder layer typically has residual connections across attention and MLP submodules. As the residual stream $h_i$ traverses from interval $E_j$ to $E_{j+1}$, it undergoes a residual transformation given by:  
% \begin{equation}
% \label{eq:slow_residual_transformation}
% H_{E_{j+1}} = H_{E_j} \prod_{i=E_j}^{E_{j+1}} \left( I + \mathcal{A}_i \right) \left( I + \mathcal{M}_i \right) \quad \text{where} \quad \mathcal{A}_i = f(c_i, h_{i}), \mathcal{M}_i = g(h_i)
% \end{equation}

\begin{equation} \label{eq:slow_residual_transformation}
h_{E_{j+1}} = h_{E_j} + \sum_{i=E_j}^{E_{j+1}-1} \left( \mathcal{A}_i(h_i) + \mathcal{M}_i(h_i + \mathcal{A}_i(h_i)) \right) \quad \text{where} \quad \mathcal{A}_i = f(c_i, h_{i}), \mathcal{M}_i = g(h_i). 
\end{equation}

Here, \( \mathcal{A}_i \) denotes the non-linear transformation introduced by the multi-head attention mechanism at layer \( i \), while \( \mathcal{M}_i \) corresponds to the non-linear transformation of the MLP block at the same layer. These transformations depend on the input residual stream \( h_i \) and, in the case of \( \mathcal{A}_i \), the previous contextual representation \( c_i \).\footnote{Normalization layers are typically applied in practice but are omitted here for simplicity of the argument.}


% For easy tokens, the magnitude and direction of this delta transformation become progressively smaller with each successive layer as shown in \cref{fig:delta_transformation}. Consequently, it is feasible to predict these tokens after only a few residual connections, whereas harder tokens necessitate more extensive processing through additional layers.

\begin{figure}[ht]
    \centering
    \begin{subfigure}{0.48\textwidth}
        \centering
        \includegraphics[width=\textwidth]{sections/figures/residual_change.pdf}
        \caption{}
        \label{fig:residual_change}
    \end{subfigure}%
    \hfill
    \begin{subfigure}{0.48\textwidth}
        \centering
        \includegraphics[width=\textwidth]{sections/figures/alignment_wrt_dedicated_model.pdf}
        \caption{}
    \label{fig:alignment_wrt_dedicated_model}
    \end{subfigure}
    \caption{(a) As residual streams propagate through the model, the directional shifts in the residuals become progressively smaller. (b) A dedicated model with $k$ layers achieves a faster rate of change in residual streams and higher alignment than base model leveraging early exit mechanisms at layer $k$.}
    \label{fig}
\end{figure}


To examine whether residual transformations can be accelerated across layers, we conducted experiments using a diverse set of prompts on a pre-trained Phi3 model~\cite{phi3_report}. As illustrated in \cref{fig:residual_change}, we measured the directional shift in residual states as \( 1 - \mathcal{C}(h_{i-1}, h_i) \), where \(\mathcal{C}\) denotes normalized cosine similarity. This shift is notably higher in the initial layers, gradually decreasing in subsequent layers. This behavior allows traditional early exit approaches to effectively accelerate decoding by enabling earlier exits for simpler tokens. However, these approaches typically rely on a distance-based approximation, where the full residual transformation of the model is approximated by the residual transformations of the initial layers. To gain deeper insights into the distance versus velocity aspects of residual transformation, we conducted a comparative study. Specifically, we trained an early exit head at layer $k$ of the Phi3 model, which consists of 32 layers, restricting the distance traveled by each token. To accelerate the residual transformation relative to number of layers, we trained a smaller model consisting of only $k$ layers, while keeping all other hyperparameters consistent. We then compared the next-token prediction accuracy of the early exit head of the base model with that of the smaller model. To ensure an equal number of trainable parameters, we inserted low-rank adapters into the smaller model and trained only these adapters, whereas, in the distance-based approach, we trained solely the early exit head. In addition, to accelerate the residual transformation in smaller model, we distilled the residual streams from the larger model by incorporating a distillation loss ~\cite{sanh2019distilbert} between the residual state at layer \(i\) of the smaller model and the residual state at layer \(4 \times i\) of the larger model. As shown in ~\cref{fig:alignment_wrt_dedicated_model} the smaller model demonstrates a significantly faster rate of change in residual streams, leading to higher next token prediction accuracy after $k$ layers compared to the base model that employs traditional early exit mechanisms after $k$ layers \cite{schuster2022confident, chen2023eellm, varshney-etal-2024-investigating}. This experimental setup, which modifies only the rate of change in residual streams while keeping other factors constant, suggests that dense transformers, trained with a fixed number of layers, may inherently possess a slow residual transformation bias.

This observation raises an intriguing question: if the rate of change in residual streams could be accelerated relative to the number of layers, is it possible to facilitate earlier alignment for a greater proportion of tokens? Earlier alignment would be beneficial to not only facilitate dynamic computation but also for generating speculative tokens efficiently with high acceptance rates in speculative decoding setups ~\cite{leviathan2023fast, chen2023accelerating}. 

%thereby enhancing the efficiency of early exiting? 
 % This bias likely constrains the effectiveness of early exiting, particularly for easier tokens. By addressing this limitation through accelerated residual transformations, we hypothesize that it is possible to substantially improve the efficiency and accuracy of early exit strategies in transformer models.

\subsection{Multi-Rate Residual Transformation} \label{m2r2_method}

To address the slow residual transformation bias described in ~\cref{sec:motivation}, we introduce \textit{accelerated residual streams} that operate at rate $R$ relative to original slow residual stream. We pair slow residual stream, $h$ with an accelerated residual stream, $p$, which has an intrinsic bias towards earlier alignment. Relative to ~\cref{eq:slow_residual_transformation}, accelerated residual transformation from interval $E_j$ to $E_{j+1}$ can be represented as: 

% \begin{equation}
% \label{eq:fast_residual_transformation}
% P_{E_{j+1}} = P_{E_j} \prod_{i=E_j}^{E_{j+1}} \left( I + \hat{\mathcal{A}_i} \right) \left( I + \hat{\mathcal{M}_i} \right) \quad \text{where} \quad \hat{\mathcal{A}_i} = \hat{f}(c_i, P_{i}), \hat{\mathcal{M}_i} = \hat{g}(P_{i})
% \end{equation}


\begin{equation} \label{eq:fast_residual_transformation}
p_{E_{j+1}} = p_{E_j} + \sum_{i=E_j}^{E_{j+1}-1} \left( \hat{\mathcal{A}_i}(p_i) + \hat{\mathcal{M}_i}(p_i + \hat{\mathcal{A}_i}(p_i)) \right) \quad \text{where} \quad \hat{\mathcal{A}_i} = \hat{f}(c_i, p_{i}), \hat{\mathcal{M}_i} = \hat{g}(h_i), 
\end{equation}



where $\hat{\mathcal{A}_i}$ and $\hat{\mathcal{M}_i}$ denote non-linear transformation added by layer $i$ to previous accelerated residual $p_{i}$. Similar to $\mathcal{A}_i$, non-linear transformation $\hat{\mathcal{A}_i}$ attends to same context $c_i$ but uses a different transformation $\hat{f}$ for accelerating $p_{E_j}$ relative to $h_{E_j}$. 

We integrate accelerated residual transformation directly into the base network using parallel accelerator adapters such that rank of accelerator adapters $R_p << d$ where $d$ denotes base model hidden dimension. This setup allows the slow residual stream $h_{E_j}$ to pass through the base model layers while the accelerated residual stream $p_{E_j}$ utilizes these parallel adapters as shown in ~\cref{fig:m2r2_main}. Both slow and accelerated residuals are processed in same forward pass via attention masking and incur negligible additional inference latency in memory bound decoding setups, while in compute bound decoding setups where FLOPs optimization is essential, accelerated residual stream utilizes a fraction of attention heads that of slow residual (see ~\cref{sec:flops_optimization}). Additionally, to maximize the utility of accelerated residual transformations without introducing dedicated KV caches, we propose a shared caching mechanism between the slow and accelerated streams which minimally impact alignment benefits of our approach while offering substantial memory savings (see ~\cref{fig:koala_alignment}). Specifically, the attention operation on the slow residuals \( \text{MHA}(h_t, h_{\leq t}, h_{\leq t}) \) is redefined for accelerated residuals as 
\[
\hat{\mathcal{A}} = MHA(p_t, h_{<t} \oplus p_t, h_{<t} \oplus p_t),
\]
where the accelerated residual at time-step $t$, \( p_t \) attends to the slow residual’s KV cache, facilitating the reuse of contextual information across both residual streams without incurring additional caching costs. Here, \(MHA(q, k, v) \) represents multi-head attention between query \( q \), key \( k \), and value \( v \).

\begin{figure}
    \centering
    \includegraphics[width=0.8\linewidth]{sections//figures/m2r2_main2.pdf}
    \caption{Multi-rate Residuals Framework: Slow residual stream of base model is accompanied by a faster stream that operates at a $2-(J+1)\times$ rate relative to the slow stream, undergoing transformations via accelerator adapters as detailed in \cref{m2r2_method}, where J denotes number of early exit intervals. Colors within the slow and fast residual streams indicate similarity, with matching colors representing the most closely aligned residual states. At the beginning of the forward pass and at each exit point, the accelerated residual state is initialized from the corresponding slow residual state to avoid gradient conflict during training (see ~\cref{sec:grad_conflict}). Early exiting decisions are informed by the Accelerated Residual Latent Attention (ARLA) mechanism, described in \cref{method_arla}, which evaluates residual dynamics across consecutive exit gates.}
    \label{fig:m2r2_main}
\end{figure}

% Furthermore. to maximize the benefits of fast residual transformations without using dedicated KV caches, we propose sharing the fast network’s cache with the slow network. Formally speaking, We modify attention operation on slow residuals $MHA(H_t, H_{<=t}, H_{<=t})$ as $MHA(P_{t}, H_{<t} \oplus P_t, H_{<t}  \oplus P_t)$ such that accelerated residuals attend to previous slow context KV cache, where $MHA(q,k,v)$ denotes multi head attention between query, $q$, key $k$ and value $v$.


\subsection{Enhanced Early Residual Alignment}
Early residual alignment is instrumental in optimizing early exiting, speculative decoding, and Mixture-of-Experts (MoE) inference mechanisms. In this section, we provide a detailed analysis of how accelerated residuals enhance these inference setups.

% By aligning the residual states of intermediate layers with the final output representations, the model can maintain high prediction accuracy even when computations are truncated at earlier layers. This enables more reliable early exiting, reducing the overall computational cost while preserving performance. Additionally, in speculative decoding, early residual alignment allows the model to make confident predictions using faster, partial computations, thereby accelerating inference without sacrificing output quality.


\subsubsection{Early Exiting} \label{method_early_exiting}

A prevalent strategy for enabling early exiting at an intermediate layer $E_{j}$ involves approximating the residual transformation between $E_{j}$ and the final layer $N-1$ using a linear, context independent mapping, $\mathcal{T}$, such that $H_{N-1} \approx \mathcal{T}(H_{E_{j}})$. This approximation has been extensively employed in conventional approaches ~\cite{schuster2022confident, chen2023eellm, varshney-etal-2024-investigating}, providing a computationally efficient means to project the output of deeper layers from intermediate states. Specifically, residual state of layer $N-1$ with this approximation can be expressed as:


% \begin{equation}
% \label{eq: vanila_ea_assumption}
% \Phi(H_{E_{j}}) \sim H_{E_{j}} \prod_{i=E_{j}}^{N}\left( I + \mathcal{A}_i \right) \left( I + \mathcal{M}_i \right) \quad \text{where} \quad \Phi \perp C
% \end{equation}

\begin{equation} \label{eq:early_exiting}
h_{E_j} + \sum_{i=E_j}^{N-1} \left( \mathcal{A}_i(h_i) + \mathcal{M}_i(h_i + \mathcal{A}_i(h_i)) \right) \sim \mathcal{T}(h_{E_{j}})  \quad \text{where} \quad \mathcal{T} \perp c. 
\end{equation}


Here, $\mathcal{A}_i$ and $\mathcal{M}_i$ represent the residual contributions of the multi-head attention and MLP layers, respectively, while $\mathcal{T}$ remains independent of $c$, the preceding context.

This approach is inherently limited by two major factors: first, the assumption of linearity between $h_{E_{j}}$ and $h_{N-1}$ may not hold uniformly for all tokens, particularly when $E_j \ll N$. Second, the linear transformation $\mathcal{T}$ disregards the influence of the context $c$ and fails to account for the latent representations of previous contextual states. In contrast, M2R2 accelerated residual states mitigate both of these challenges by approximating the slow residual transformation of all layers via a faster residual transformation of fewer layers as:
% \begin{equation}
% H_{E_j} \prod_{i=E_j}^{N}\left( I + \mathcal{A}_i \right) \left( I + \mathcal{M}_i \right) \sim P_{E_j} \prod_{i=E_j}^{E_j+1}\left( I + \hat{\mathcal{A}_i} \right) \left( I + \hat{\mathcal{M}_i} \right)
% \end{equation}


\begin{equation} \label{eq:m2r2_approximating_ea}
h_{E_j} + \sum_{i=E_j}^{N-1} \left( \mathcal{A}_i(h_i) + \mathcal{M}_i(h_i + \mathcal{A}_i(h_i)) \right) \sim p_{E_j} + \sum_{i=E_j}^{E_{j+1}-1} \left( \hat{\mathcal{A}_i}(p_i) + \hat{\mathcal{M}_i}(p_i + \hat{\mathcal{A}_i}(p_i)) \right), 
\end{equation}

% \begin{equation} \label{eq:fast_residual_transformation}
% p_{E_{j+1}} = p_{E_j} + \sum_{i=E_j}^{E_{j+1}-1} \left( \hat{\mathcal{A}_i}(p_i) + \hat{\mathcal{M}_i}(p_i + \hat{\mathcal{A}_i}(p_i)) \right) \quad \text{where} \quad \hat{\mathcal{A}_i} = \hat{f}(c_i, p_{i}), \hat{\mathcal{M}_i} = \hat{g}(h_i) 
% \end{equation}






where $p_{E_j}$ is initialized from the slow residual state $h_{E_j}$ at each early exit interval $E_j$ using an identity transformation (see ~\cref{fig:m2r2_main}). As shown in ~\cref{fig:m2r2_residual_sim}, accelerated residuals offer a smoother, more consistent shift in residual direction across layers, in contrast to the abrupt changes typically seen at early exit points in standard early exit methods. Moreover, the normalized cosine similarity between accelerated states at early exit intervals and final residual states is substantially higher compared to traditional early exit techniques, highlighting improved alignment with final layer representations. Traditional adaptive compute methods are constrained by two principal factors: the number of tokens eligible for early exit at intermediate layers and the precision of early exit decision. If residual streams fail to saturate early, the majority of tokens remain ineligible for exit, thereby diminishing potential speedups. Additionally, imprecise delineations between tokens suitable for early exit can lead to underthinking (premature exits that adversely affect accuracy) or overthinking (unnecessary processing that compromises efficiency) ~\cite{zhou2020self, dai2020dynamic}. Enhanced early alignment using ~\cref{eq:m2r2_approximating_ea} helps to address  first issue. To address the second issue we introduce Accelerated Residual Latent Attention, which dynamically assesses the saturation of the residual stream, allowing for a more precise differentiation between tokens that can exit early and those requiring further processing.

% This results in uniform change in residual direction    
% % We keep $\mathcal{A} = \hat{\mathcal{A}}$, while $\hat{\mathcal{M}}$ is accelerated by a factor of $2 - (N_{E}+1)X$ relative to the slower residual transformation $\mathcal{M}$, where $N_E$ represents number of early exiting intervals.
% Figure~\cref{fig:rate_change_comparison} illustrates the comparative rate of change between these transformation streams.



% fig:rate_change_comparison
% - grid plot x axis -> layer id (0, 8) , y axis -> layer id -> dark color cell for max similarity , lighter for lower 
% 
-------------------------------------------------------
Let's consider residual stream $h_i$ traverses through interval $E_j$ to $E_{j+1}$ and undergoes residual transformation given by 
\begin{equation}
h_{E_{j+1}} = h_{E_j} \prod_{i=E_j}^{E_{j+1}} \left( 1 + \delta_i \right)    
\end{equation}

where $\delta_i$ denotes non-linear transformation added by layer $i$. Each non-linear transformation of layer $i$ is a function of previous contextual representation, $c_i$ and input residual stream $h_i-1$ as
$\delta_i = f(c_i, h_{i-1})$ 

One way to exit early at exit $E_j+1$ is to assume that residual transformation from $E_j+1$ to final layer $N-1$ can be approximated by a linear function $\phi$ as $h_{N-1} \sim \Phi(h_{E_j+1})$ and most conventional approaches such as \todo{cite EA papers} use this approach. In other words, 

\begin{equation}
\Phi(h_{E_j+1} \sim h_{E_j+1} \prod_{i=E_j+1}^{N} \left( 1 + \delta_i \right)   
\end{equation}

This approach suffers from two primary issues, linearity assumption from $h_E_j+1$ to $H_N-1$ if often incorrect, particularly when $E_j << N$. More importantly, linear transformation $\Phi$ doesn't consider effect of context $C_i$. M2R2  effectively addresses these issues as accelerated residual stream at interval $E_j+1$ can be represented as 

\begin{equation}
r_{E_{j+1}} = r_{E_j} \prod_{i=E_j}^{E_{j+1}} \left( 1 + \gamma_i \right)    
\end{equation}

where $\gamma_i$ denotes non-linear transformation added by layer $i$ to previous accelerated residual $r_i-1$. Similar to $\delta_i$, non-linear transformation $\gamma_i$ considers context $C_i$ as 
$\gamma_i = g(c_i, r_{i-1})$. So in summary, slow residual transformation is approximated by accelerated residual as: 

\begin{equation}
h_{E_j} \prod_{i=E_j}^{N} \left( 1 + \delta_i \right) \sim h_{E_j} \prod_{i=E_j}^{E_j+1} \left( 1 + \gamma_i \right)
\end{equation}

It's worth noting that accelerated residual $r_i$ and slow residual $h_i$ are processed concurrently at layer $i$ by constructing proper attention mask such as attention of slow residual is represented as 

$MHA(H_it, H_{i<=t}, H_{i<=t}$ while attention of fast residual is computed as 

$MHA(r_it, H_{i<=t}, H_{i<=t}$ where $MHA(q,k,v$ denotes multi head attention between query, $q$, key $k$ and value $v$.


------------------------------------------------------------------

Vertical latent attention on accelerated residual is computed as 
$MHA(S_mt, S(Ej<=i<=m)t, S(Ej<=i<=m)t)$ where $Smt$ denotes query/key/value projection in latent domain at layer $m$ at time $t$. 
------------------------------------------------------------------

Gradient conflict Avoidance: 

Let's consider $w_j$ is a trainable parameter that belongs to a layer between $E_j$ and $E_j+1$. Consider early exit loss at gate $E_j+1$, $L_j+1$, gradient propagation of $w_j$ at another trainable parameter $w_j-n$ can be gives as 

$\sum_{k=E_j-n}^{E_j} \beta_k \frac{\partial L_{E_k}}{\partial w_k}$

where $\beta_j$ denotes backward transformation coefficient for weight $w_j$ to reach gate $E_j$. 
 
On the other hand, gradient propagation in proposed approach can be represented as 

\[
\frac{\partial L_{E_j}}{\partial w_j} = 
\begin{cases} 
\beta_j \frac{\partial L_{E_j}}{\partial w_j} & \text{if } E_j \leq w_j \leq E_{j+1} \\
0 & \text{otherwise}
\end{cases}
\]







% \begin{figure}[ht]
%     \centering
%     \includegraphics[width=0.8\textwidth, height=5cm]{rate_change_comparison.png}
%     \caption{Rate of change comparison between fast and slow residual streams.}
%     \label{fig:rate_change_comparison}
% \end{figure}

%vary k and and plot EA accuracy for larger and smaller models. 

% \begin{figure}[ht]
%     \centering
%     \includegraphics[width=0.5\textwidth,height=5cm]{sections/figures/alignment_comparison_dialogsum.pdf}
%     \caption{Alignment of exited tokens for different early exit layers using traditional early exiting heads, dedicated faster networks, and faster residuals.}
%     \label{fig:small_model_early_exiting}
% \end{figure}


\textbf{Accelerated Residual Latent Attention} \label{method_arla}

In the context of residual streams, we observe that the decision to exit at a given layer can be more effectively informed by analyzing the dynamics of residual stream transformations, instead of solely relying on a classification head applied at the early exit interval $E_j$. To capture the subtle dynamics of residual acceleration, we propose a \textit{Accelerated Residual Latent Attention} (ARLA) mechanism. This approach involves making the exit decision at gate $E_j$ by attending to the residuals spanning from gate $E_{j-1}$ to $E_j$, rather than considering only the residual at gate $E_j$. To minimize the computational overhead associated with exit decision-making, the attention mechanism operates within the latent domain as depicted in ~\cref{fig:arla_arch}. Formally, for each interval $[E_j, E_{j+1}]$, the accelerated residuals are projected into Query ($Q^s_{E_j}, \ldots, Q^s_{E_{j+1}}$), Key ($K^s_{E_j}, \ldots, K^s_{E_{j+1}}$), and Value ($V^s_{E_j}, \ldots, V^s_{E_{j+1}}$) vectors, with latent dimension $d^s$ for $Q^s$, $K^s$, and $V^s$ being significantly smaller than hidden dimension of $p$.\footnote{We use $d^s = 64$ for experiments described in ~\cref{sec:experiments}.} Notably, when the router is allowed to make exit decisions at gate $E_j$ based on residual change dynamics, we observe that the attention is not confined to the residual state at $E_j$ but is distributed across residual states from $E_{j-1}$ to $E_j$, %as illustrated in Figure~\ref{fig:vertical_latent_attention_dynamics}. 
This broader focus on residual dynamics significantly reduces decision ambiguity in early exits, as demonstrated in Figure~\ref{fig:roc_arla}, which contrasts routers based on the last hidden state, and the proposed ARLA router.

%show R -> S transformation. 
%show parameter and flop overhead as compared to adapter on last hidden state.

% \begin{figure}[ht]
%     \centering
%     \includegraphics[width=0.5\textwidth,height=5cm]{sections/figures/roc_arla.pdf}
%     \caption{ROC curves of early exit decision strategies: confidence-based methods (CALM/LITE), routers based on the accelerated hidden state, and latent attention routers.}
%     \label{fig:decision_making_comparison}
% \end{figure}

% \begin{figure}[ht]
%     \centering
%     \includegraphics[width=0.5\textwidth,height=5cm]{vertical_latent_attention.png}
%     \caption{Vertical latent attention mechanism for optimizing early exit decisions by considering residuals from gate \(M\) through \(M-1\).}
%     \label{fig:vertical_latent_attention}
% \end{figure}

\begin{figure}[ht]
    \centering
    \begin{subfigure}{0.52\textwidth}
        \centering
        \includegraphics[width=\textwidth, height = 4cm]{sections/figures/arla_arch.pdf}
        \caption{Accelerated Residual Latent Attention (ARLA): Accelerated residuals between early exit gates are projected into latent domain and attention over residual states within the interval is computed to capture residual dynamics and exit decision is made based on residual saturation.}
        \label{fig:arla_arch}
    \end{subfigure}%
    \hfill
    \begin{subfigure}{0.45\textwidth}
        \centering
        \includegraphics[width=\textwidth, height = 4.5cm]{sections/figures/vla_roc.pdf}
        \caption{ROC classification curves of early exit decision strategies using a linear router used on last residual state ~\cite{schuster2022confident, varshney-etal-2024-investigating, chen2023eellm}  and using ARLA approach that considers residual dynamics. }
        \label{fig:roc_arla}
    \end{subfigure}
    \caption{Effectiveness of ARLA in capturing residual dynamics for early exiting decisions.}


\end{figure}



% \begin{figure}[ht]
%     \centering
%     \includegraphics[width=1\textwidth,height=5cm]{sections/figures/arla.pdf}
%     \caption{fig that plots 32 rows 2 cols heatmap showing attention at each gate}
%     \label{fig:vertical_latent_attention_dynamics}
% \end{figure}

\subsubsection{Self Speculative Decoding} \label{method_self_speculative_decoding}

An alternative means to exploit the early alignment properties of our approach is through the use of accelerated residual states for speculative token sampling to accelerate autoregressive decoding. Speculative decoding aims to speed up memory-bound transformer inference by employing a lightweight draft model to predict candidate tokens, while verifying speculated tokens in parallel and advancing token generation by more than one token per full model invocation \cite{leviathan2023fast, chen2023accelerating, xia2023speculative, miao2023specinfer}. Despite its effectiveness in accelerating large language models (LLMs), speculative decoding introduces substantial complexity in both deployment and training. A separate draft model must be specifically trained and aligned with the target model for each application, which increases the training load and operational complexity ~\cite{chen2023accelerating}. Additionally, this approach is resource-inefficient, as it requires both the draft and target models to be simultaneously maintained in memory during inference \cite{leviathan2023fast, chen2023accelerating}. 

One strategy to address this inefficiency is to leverage the initial layers of the target model itself to generate speculative candidates, as depicted in ~\cite{Tang2024}. While this method reduces the autoregressive overhead associated with speculation, it suffers from suboptimal acceptance rates. This occurs because the linear transformation employed for translating hidden states from layer $k$ to the final layer $N$ is typically a poor approximation, as discussed in ~\cref{sec:motivation} and ~\cref{method_early_exiting}. Our approach resolves this limitation by utilizing accelerated residuals, which demonstrate higher fidelity to their slower counterparts. By utilizing accelerated residuals operating at a rate of $N/k$, where $k$ denotes the number of layers used for candidate speculation, we are able to efficiently generate speculative tokens for decoding.\footnote{We typically set $k = 4$ to balance the trade-off between autoregressive drafting overhead and acceptance rate, as discussed in~\cref{sec:experiments}.}
 This technique not only obviates the need for multiple models during inference but also improves the overall efficiency and effectiveness of speculative decoding.

\begin{figure}
    \centering    \includegraphics[width=1\linewidth]{sections/figures/m2r2_aot_loading.pdf}
    \caption{Ahead-of-Time Expert Loading: M2R2 accelerated residual stream predicts experts required for future layers, reducing reliance on on-demand lazy loading. Speculative pre-loading is efficiently overlapped with computation of multi-head attention (MHA) and MLP transformations. Only incorrectly speculated experts are loaded lazily, resulting in faster inference steps and improved computational efficiency. Here, H indicates LBM Host while D indicates HBM Device.}
    \label{fig:moe_expert_aot_loading}
\end{figure}


\subsubsection{Ahead of Time Expert Loading:} \label{method_aot_expert_loading}

Recent advancements in sparse Mixture-of-Experts (MoE) architectures ~\cite{shazeer2017outrageously, fedus2022switch, artetxe2019massively, lepikhin2020gshard, zoph2022designing} have introduced a paradigm shift in token generation by dynamically activating only a subset of experts per input, achieving superior efficiency in comparison to dense models, particularly under memory-bound constraints of autoregressive decoding \cite{fedus2022switch, zoph2022designing}. This sparse activation approach enables MoE-based language models to generate tokens more swiftly, leveraging the efficiency of selective expert usage and avoiding the overhead of full dense layer invocation. In dense transformer models, pre-loading layers is a common strategy to enhance throughput, as computations of current layer can be overlapped with pre-loading of next layer parameters ~\cite{narayanan2021efficient, shoeybi2020megatron}. However, MoE models face a unique challenge: expert selection occurs dynamically based on previous layer’s output, making it infeasible to preload next layer’s experts in parallel. This limitation results in inherent latency, as expert loading becomes a sequential, on-demand process ~\cite{lepikhin2020gshard, fedus2022switch}.

To address this inefficiency, our method introduces a mechanism with \textit{accelerated residuals}, which not only captures key characteristics of base slower residual states but also exhibit high cosine similarity with their final counterparts (as illustrated in \cref{fig:m2r2_residual_sim}). By employing accelerated residual streams, we can effectively predict the necessary experts for future layers well in advance of their actual invocation. Specifically, using a $2\times$ accelerated residual, the experts needed for layers $2i+2$ and $2i+3$ can be identified while still computing in layer $i$, thus overcoming the bottleneck of sequential, on-demand expert selection and mitigating latency in the decoding pipeline, as shown in \cref{fig:moe_expert_aot_loading}. Note that, we use fixed set of accelerator adapters for transforming accelerated residuals (as discussed in ~\cref{m2r2_method}) while slow residual is transformed via expert routing mechanism. 

Furthermore, our approach integrates a Least Recently Used (LRU) caching strategy, which enhances memory efficiency by replacing the least recently used experts with speculated experts that are anticipated to be needed in upcoming layers. This hybrid approach of preemptive expert loading with LRU caching yields substantial improvements over traditional on-demand loading or standalone caching strategies. By minimizing cache misses and efficiently managing memory, this approach addresses both compute and memory bottlenecks, leading to faster, more resource-efficient token generation in MoE architectures. A comprehensive evaluation of this strategy, in relation to state-of-the-art methods, is provided in \cref{experiments_aot}, and the compute and memory traces on an A100 GPU are detailed in \cref{fig:moe_aot_cuda_trace}.



% Recent advancements in sparse Mixture-of-Experts (MoE) architectures have introduced the concept of utilizing distinct computational paths for different tokens \cite{shazeer2017outrageously}. This approach, wherein only a subset of experts are activated per input, enables MoE-based language models to generate tokens more swiftly compared to their dense counterparts due to memory-bound nature of auto-regressive decoding. In dense models, pre-loading layers in advance is a common strategy to enhance computational efficiency. However, this technique is not applicable to MoE models, where expert selection occurs dynamically based on the outputs of previous layers, preventing parallel pre-fetching of experts.

% Our proposed method addresses this inefficiency. Accelerated residuals, which are highly similar to their slower counterparts (see \cref{fig:similarity}), can reliably predict the necessary experts ahead of time. For instance, by utilizing $2X$ accelerated residual stream, we can predict the experts needed for the layer $2i+1$ and $2i+3$ while carrying out computation in layer $i$. This enables us to commence expert loading significantly earlier, as illustrated in \cref{expert_loading}, effectively mitigating the delays observed with the naive on-demand expert loading. Additionally, our method benefits from incorporating a Least Recently Used (LRU) strategy, where speculated experts replace those that are least recently utilized, resulting in improved performance compared to using either strategy alone. For a comprehensive evaluation, refer to \cref{moe_trace}, which provides a CUDA compute and memory trace of our approach executed on <>.



% A naive solution involves using the residual state of the previous layer along with the gating function of the next layer to predict which experts need to be loaded, and initiating the expert loading process in parallel with the attention computation of the next layer. Yet, as shown in \cref{fig:MOE_attn_vs_loading_time}, the attention computation for medium to long contexts is considerably faster than the expert loading time, making this approach inefficient.




\subsection{Training} \label{method_training}
% This approach is feasible due to the absence of gradient conflicts, as discussed in \cref{sec:grad_conflict}.

To accelerate residual streams, we employ parallel accelerator adapters as described in \cref{m2r2_method}.  For the early exiting use-case outlined in \cref{method_early_exiting}, we define the training objective for these adapters using the following loss function, which combines cross-entropy loss at each exit $E_j$ with distillation loss at each layer $i$. Loss weights coefficients $\alpha_0$ and $\alpha_1$ are employed to balance contribution of corresponding losses.

\begin{align} \label{eq:mr_loss}
L_{\text{m2r2}} = \underbrace{-\alpha_0 \sum_{j=1}^{J} \sum_{t=1}^{T} \log p_{\theta} \left( \hat{y}_t^{E_j} \mid y_{<t}, x \right)}_{\text{cross-entropy loss}} 
+ \underbrace{\alpha_1\sum_{i=1}^{E_{J-1}} \sum_{t=1}^{T} \| \mathbf{p}_{t}^{i} - \mathbf{h}_{t}^{((i - E_{j(i)}) \cdot R_i) + E_{j(i)})} \|^2}_{\text{distillation loss}}.
\end{align}

where $\hat{y}_t^{E_j}$ denotes the predictions from the accelerated residual stream at layer $E_j$ and time step $t$, $y_t$ represents the corresponding ground truth tokens, and $x$ indicates previous context tokens. The distillation loss at each layer $i$ is computed by comparing accelerated residuals at layer $i$ with slow residuals at layer $(i - E_{j(i)}) \cdot R_i + E_{j(i)}$, where $R_i$ denotes the rate of accelerated residuals at layer $i$ while $E_{j(i)}$ represents the most recent gate layer index such that $E_{j(i)} <= i$. \( J \) represents the total number of early exit gates, N denotes number of hidden layers and $E_j$ denotes layer index corresponding to gate index $j$ and \( T \) denotes the sequence length. 

In dynamic compute settings, after training of accelerator adapters, we optimize the query, key, and value parameters governing the ARLA routers (see ~\cref{method_arla}) across all exits in parallel on binary cross entropy loss between predicted decision and ground truth exiting decision. The ground truth labels for the router are determined based on whether the application of the final logit head on $\hat{y}_t^{E_j}$ yields the correct next-token prediction. 


% The objective for this optimization is defined by the following loss function:


%TODO are equations required ? 
% \begin{equation} \label{eq:arla_loss_combined}\small
%     L_{\text{arla}} = -\frac{1}{N} \sum_{t=1}^{T} \left( \sum_{j=1}^{E_n} \left[ O_t^{E_j} \log(\hat{O}_t^{E_j}) + (1 - O_t^{E_j}) \log(1 - \hat{O}_t^{E_j}) \right] \right), \quad \text{where} \quad 
%     O_t^{E_j} = \begin{cases} 
%     1, & \text{if } L(\hat{y}_t^{E_j}) = y_t^{E_j} \\
%     0, & \text{otherwise}
%     \end{cases}
% \end{equation}

% where $\hat{O}_t^{E_j}$ represents the binary predicted logits produced by the vertical latent attention router, as described in \cref{sec:arla}, at gate $E_j$ and time step $t$, and $O_t^{E_j}$ denotes the corresponding ground truth labels. The ground truth labels for the router are determined based on whether the application of the logit head on $\hat{y}_t^{E_j}$ yields the correct next-token prediction. The parameters controlling vertical latent attention are trained concurrently to ensure consistency and efficient use of computational resources.

For self-speculative decoding, as described in \cref{method_self_speculative_decoding}, the training objective remains the same as \cref{eq:mr_loss}, but with the number of intervals set to $J = 1$ and the rate of residual transformation set to $R_n = N/k$, where the first $k$ layers generate speculative candidate tokens. In the context of Ahead-of-Time Expert Loading for Mixture-of-Experts (MoE) models (see \cref{method_aot_expert_loading}), setting the rate of residual transformation to $R_n = 2$ typically offers a good trade-off between the accuracy of expert speculation and AoT pre-loading of experts. 

% Thus, we set $J = 1$ and $E_1 = 16$.


~\subsection{FLOPs Optimization} \label{sec:flops_optimization}

Naively implemented, M2R2 incurs higher FLOP overhead compared to traditional speculative decoding and early exiting approaches such as ~\cite{medusa, schuster2022confident, Tang2024}. However, modern accelerators demonstrate compute bandwidth that exceeds memory access bandwidth by an order of magnitude or more~\cite{databricksLLMInference2023, jouppi2021ten}, meaning increased FLOPs do not necessarily translate to increased decoding latency. Nevertheless, to ensure fair comparison and efficiency in compute bound scenarios, we introduce targeted optimizations.

~\textbf{Attention FLOPs Optimization} For medium-to-long context lengths, attention computation dominates FLOPs in the self-attention layer, surpassing the contribution from MLP layers. Specifically, matrix multiplications involving queries, cached keys, and cached values scale with $l_{kv} * l_{q}$ where $l_{kv}$ denotes previous context length and $l_q$ denotes current query length. Since M2R2 pairs accelerated residuals with slow residuals, a naive implementation results in twice the FLOPs consumption compared to a standard attention layer. To address this, we limit the attention of accelerated residual stream to selectively attend to the top-k most relevant tokens, identified by the slow residual stream based on top attention coefficients\footnote{We set to k = 64 and attend to top 64 tokens as identified by the slow residual stream.}. This is possible since slow and accelerated residual streams are processed in same forward pass and accelerated streams have access to attention coefficients of slow stream. Note that, the faster residual stream still retains the flexibility to assign distinct attention coefficients to these tokens. Furthermore, we design the faster residual stream to employ only 8 attention heads, compared to the 32 heads used in the slow residual stream of the Phi-3 model, reducing query, key, value, and output projection FLOPs by a factor of 1/4. ~\cref{fig:m2r2_num_heads_ablation} indicates effect of using a slicker stream on alignment. As depicted, using $\hat{n}_h = 8$ offers a good trade-off between alignment and FLOPs overhead. 

~\textbf{MLP FLOPs Optimization} The accelerator adapters operating on the accelerated residual stream are intentionally designed with lower rank than their counterparts in the base model. This reduces FLOP overhead by a factor proportional to $hiddenSize / rank$. Additionally, since the faster residual stream uses only 8 attention heads (compared to 32 in the slow residual stream of Phi-3), the subsequent MLP layers process a smaller set of activations, further reducing FLOPs by another factor of 1/4.

These optimizations significantly reduce the FLOP overhead per speculative draft generation, as illustrated in ~\cref{fig:flops_optmization}. Notably, while traditional early-exiting speculative approaches such as DEED require propagating the full slow residual state through the initial layers, incurring substantial computational costs, M2R2 achieves efficient token generation via slimmer, low-rank faster residual streams. In contrast, Medusa introduces considerable FLOP overhead due to per-head computations scaling with $d^2+dv$\footnote{Here $d$ denotes hidden state dimension while $v$ denotes vocab size.}, whereas M2R2 employs low-rank layers for both MLP and language modeling heads, maintaining computational efficiency. All experiments involving the M2R2 approach, as detailed in ~\cref{sec:experiments}, are conducted using these FLOPs optimizations.









% \[
% O_t^{E_j} = 
% \begin{cases} 
% 1, & \text{if } L(\hat{y}_t^{E_j}) = y_t^{E_j} \\
% 0, & \text{otherwise}
% \end{cases}
% \]




%add distillation
% We train accelerator adapters described in \cref{m2r2_method} to accelerate residual streams on next token prediction all in parallel since there are no gradient conflict issues as described in \cref{sec:grad_conflict}.

% \begin{align} \label{eq:mr_loss}
% L_{mr} =  & -\sum_{j = 1}^{E_n} (\sum_{t=1}^{T}\log p_{\theta} (\hat{y}_t^{E_j} | \hat{y}_{<t}, x)) \nonumber
% \end{align}

% where $\hat{y_t^{E_j}}$ denotes predicted logits obtained from accelerated residual stream at gate $E_j$ and time-step $t$ while $y_t^{E_j}$ denotes corresponding truth tokens. 

% Upon training of adapters responsible for accelerating residual streams, we train query, key, value parameters responsible for vertical latent attention of all gates in parallel as

% \begin{equation} \label{eq:arla_loss}
%     L_{arla} = -\frac{1}{N} (\sum_{t=1}^{T}(1\sum_{j=1}^{E_n} \left[ O_t^{E_j} \log(\hat{O}_t^{E_j}) + (1 - o_t^{E_j}) \log(1 - \hat{o_t}_{E_j}) \right]))
% \end{equation}

% where $\hat{O_t^{E_j}}$ denotes binary predicted logits obtained from vertical latent attention router described in \cref{sec:arla} at gate $E_j$ and timestep $t$ while $O_t^{E_j}$ denotes corresponding truth label. Truth labels for router are obtained by computing whether logit head application on $\hat{y}_t^j$ results in true next token prediction. Formally speaking, 

% $O_t^{E_j} = 1 if L(\hat{y_t^{E_j}}) == y_t^{E_j} , 0 otherwise$. 

% Parameters responsible for vertical latent attention are also trained in parallel as well. 

%todo: training slow and fast residuals together and distillation can be two training mdoes. 
%Distillation can be an ablation. 




% Although transformer decoding is memory bound on most mainstream accelerators, there could be scenarios where flop savings are crucial. For instance, on on-device settings power consumption is directly correlated with flops per decoding step and reducing flops does help with overall energy consumption. Vanilla early exiting methods help with flop reduction but suffer from mismatch between training and inference due to early exited tokens. If token at decoding step $t$, $T_t$ exited at layer $E_i$, while token $T_{t+k}$ exits at layer $E_j$ such that $E_i < E_j$, hidden state $H_{t+k}l$ does not have corresponding hidden state $H_tl$ to attend to where $E_i < l <= E_j$. One solution that's often used in literature is to rely on last hidden state available, $H_t{E_j}$, however it tends to be sub-optimal and does affect generation quality \cite{ref}.  To alleviate this mismatch while reducing flops, we train router such that attention mask between token $T_{t+k}$ and token $T_{<t+k}$ is given by: 

% \begin{equation}
%     a_{T_{{t+k}{T_{<t+k}}} = 1 if  E_{T_{<t+k}} >= E{T_{t+k}}
%     else 0
% \end{equation}

% This attention mask enables router to account for exited tokens and get trained accordingly. Since attention mechanism during decoding remains exactly same as that during training, impact on generation quality tends to be minimal as noted in \cref{fig:gen_auality_with_and_without_recompute_attention_show_flops}.  Although MoD does not suffer from training and inference mismatch, we observe that it suffers from discountinuity between pre-training and super-vised fine-tuning resulting in sub-optimal perplexity. On the other hand, our method doesn't not require pre-training , doesn't suffer from discountinuity, and achieves much better perplexity in super-vised fine-tuning and instruction tuning setups as shown in \cref{fig:Mod_vs_m2r2_loss_curves}.






% Our techniques are directly applicable in such scenarios.    




%expert loading with cuda streams in experiments

\section{Research Themes}
\label{themes-section}
Based on the literature found through the process outlined in Section \ref{select-lit-section}, articles with similar subjects or covering similar themes of research are grouped together. The themes are selected in such a way that most publications can be exclusively divided into one of the themes, i.e., the themes should not have significant overlap. Furthermore the themes should effectively separate domain specific research and more generally applicable research. %Finally we have a specific interest in the developments within the domain of data center and HPC research. 
Considering these requirements the following list of themes is selected:
% We start off by separating the literature into two distinct categories: (1) articles which implement an FPGA for a specific application, which covers a wide variety of research purposes, and (2) articles on the advancement of FPGA hardware and tooling software itself. The second of the two is further divided into more specific categories, resulting in the following list of themes:

\begin{enumerate}
    \item FPGA architecture
    \item Robustness of FPGAs
    \item Data center infrastructure \& HPC
    \item Programming models \& tools
    \item Applications
\end{enumerate}

Figure~\ref{fig:theme-distribution} illustrates how these themes cover both general and domain-specific development, and shows whether a theme is hardware or software focused. The theme ``Applications'' focuses on research applying FPGA technology to domain specific problems. This can be in the form of hardware architectures for domain-specific applications, as well as software tools enabling FPGA technology in a specific domain of research. The themes ``Programming models and tools'' and ``FPGA architecture'' focus on research and development of solutions that are generally applicable in a wide range of domains, while the ``Robustness of FPGAs'' and ``Data center infrastructure \& HPC'' themes feature both hardware- and software-focused research. Moreover, these themes focus on a narrower selection of FPGA applications and can, therefore, be considered  domain-specific. 

% add some explanation of figure
\begin{figure}[!htbp]
    \centering
    \includegraphics[width=0.5\textwidth]{figures/theme_distribution_diagram.pdf}
    \caption{The themes that are selected can be differentiated based on their domain-specificity, ranging from very domain-focused to general purpose, and based on whether the main focus is on hardware or software. }
    \label{fig:theme-distribution}
\end{figure}
Based on common subjects in each theme, the themes are further organized into %more specific 
subcategories. Table \ref{tab:overview-themes-most-cited} shows the subcategories and %by which each theme is subdivided, as well as 
the prevalence of each  subject based on the number of published articles. % covered in it. 
Finally, the most influential articles, based on the highest number of citations within each category (Google Scholar), is shown. This selection excludes survey publications.

\begin{table}[!ht]
\centering
\caption{Overview of highly cited papers per category, with number of published articles per theme and category in parentheses.}
\label{tab:overview-themes-most-cited}
{\small
\begin{tabular}{lll}
\textbf{Theme and category} & \textbf{Highly cited publications} & \textbf{Dutch affiliation} \\ \hline
\textbf{\textbf{FPGA architecture (11)}} &  &  \\ \cline{1-1}
Near-memory processing (4) & \citet{Singh2021FPGA-BasedApplications} & Eindhoven University \\
Coarse-grained reconfigurable architecture (4) & \citet{Wijtvliet2019Blocks:Efficiency} & Eindhoven University \\
Network-on-Chip (3) & \citet{RibotGonzalez2020HopliteRT:FPGA} & Eindhoven University \\ \hline
\multicolumn{2}{l}{\textbf{\textbf{Data center infrastructure \& HPC (40)}}}  &  \\ \cline{1-1}
Big data processing and analytics (22) & \citet{Peltenburg2019Fletcher:Arrow} & Delft University of Tech. \\
Distributed computing (5) & \citet{Bielski2018DReDBox:Datacenter} & Sintecs B.V. \\
Optical hardware communication (9) & \citet{Yan2018HiFOST:Switches} & Eindhoven University \\
High performance computing (4) & \citet{Katevenis2018NextDevelopment} & MonetDB Solutions \\ \hline
\multicolumn{2}{l}{\textbf{\textbf{Programming models \& tools (15)}} }  &  \\ \cline{1-1}
Programming models and frameworks (8) & \citet{Peltenburg2020Tydi:Streams} & Delft University of Tech.  \\
Performance prediction (7) & \citet{Yasudo2018PerformancePlatforms} & University of Amsterdam \\ \hline
\textbf{\textbf{Robustness of FPGAs (26)}} &  &  \\ \cline{1-1}
Reliability (12) & \citet{Du2019UltrahighFPGA} & ESTEC \\
Hardware security (14) & \citet{Labafniya2020OnPrevention} & Delft University of Tech. \\ \hline
\textbf{\textbf{Applications (49)}} &  &  \\ \cline{1-1}
Machine learning (12) & \citet{Rocha2020BinaryWrist-PPG} & IMEC NL \\
Astronomy (11) & \citet{Ashton2020ATelescopes} & University of Amsterdam \\
Particle physics experiments (7) & \citet{FernandezPrieto2020PhaseExperiment} & Nikhef \\
Quantum computing (5) & \citet{Philips-nat-2022} & Delft University of Tech. \\
Space (9) & \citet{Barrios2020SHyLoCMissions} & ESTEC \\
Bioinformatics (5) & \citet{Malakonakis2020ExploringRAxML} & University of Twente \\ \hline
\end{tabular}
}
\end{table}


Figure~\ref{fig:org-publish-per-theme} illustrates an overview of publications per theme for each organization with more than one publication. It is clear that most major contributors to FPGA research publish mostly application-specific research. Out of the major contributors, Delft University of Technology focuses more on the ``Data center \& infrastructure'' domain, while Eindhoven University of Technology is a larger contributor to the ``FPGA architecture'' theme. A brief description of each theme is provided below. 


\begin{figure}[!htb]
    \centering
    \includegraphics[width=\textwidth]{figures/per_chapter_no_papers_2.pdf}
    \caption{Number of publications per theme for each organization with more than one relevant publication.}
    \label{fig:org-publish-per-theme}
\end{figure}


\begin{itemize}
    \item {\bf FPGA architecture}: This research theme covers literature related to the design of novel digital hardware architectures. Efficient architectures, fast on-chip memory access, coarse grained hardware design, and partially reconfigurable hardware are covered in this theme.
\item {\bf Data center infrastructure \& HPC}: This theme includes literature on FPGAs used in high-performance computing environments. This covers papers on the rapid processing of big data, and research towards distributed computing infrastructures deploying FPGAs. Furthermore, research focusing on employing FPGAs for processing communication between computing nodes using optical links is also covered here.
\item {\bf Programming models \& tools}:
This theme covers literature related to tools and models used to program FPGAs, ranging from research on high-level synthesis (HLS) tools to tools that enable accessible hardware acceleration of conventional software. This theme also features research efforts on tools for accurate performance prediction of synthesized 
FPGA solutions.
\item {\bf Robustness of FPGAs}:
This theme covers literature regarding the reliability and resilience of FPGAs to specific environments. Specifically, resilience to radiation in environments where this is prevalent is a common subject. Furthermore, this theme expands on the security of FPGAs with regards to cyberattacks.
\item {\bf Applications}: The literature on specific applications using FPGAs is more extensive than that of the other themes. This is expected since FPGAs can be applied in various fields, whereas the advancement of FPGA architectures and development tools is generally a more narrow area of research. 
%Within this theme, 
Machine learning has been the main focus in recent years.
\end{itemize}

%\paragraph{Hardware and architecture}
%This research theme covers literature related to the design of novel digital hardware architecture. Efficient architectures, fast on-chip memory access, coarse grained hardware design and partially reconfigurable hardware are subjects that are covered within this theme.

%\paragraph{Robustness of FPGAs}
%In the theme of robustness the literature regarding the reliability and resilience of FPGAs to specific environments is covered. Specifically the resilience to radiation in environments where this is prevalent is a common subject. Furthermore, this theme expands on the security of FPGAs with regards to cyberattacks.

%\paragraph{Data center \& infrastructure}
%This theme encapsulates the literature on FPGAs used in high-performance computing environments. This includes literature on the rapid processing of big data and research towards distributed computing infrastructures implementing FPGAs. Furthermore, ample research is being done towards using FPGAs for processing the communication between computing nodes using optical links. This is also covered under this theme.

%\paragraph{Programming models \& tools}
%This research theme covers literature related to the tools and models used to program FPGAs. This ranges from research towards HLS tools, to tools which allow accessible hardware acceleration of conventional software. This theme also features research towards tools for efficient and accurate performance prediction of synthesized FPGA solutions.

%\paragraph{Applications of FPGAs}
%The literature on specific applications using FPGAs is more numerous than other themes. This is not unexpected since FPGAs can be applied in numerous fields, whilst the advancement of FPGAs and FPGA associated tools itself is a more narrow area of research. Within this theme, machine learning is the most researched subject in the recent past.

\section{FPGA Architecture}
\label{sec:archi}
\begin{figure*}[t]
\vskip 0.2in
\begin{center}
\centerline{
\includegraphics[width=\textwidth, height=9cm]{figures/architecture_img.pdf}}   
\vspace{-3mm}
\caption{\textbf{Overview of our method at the blending stage. }
% condition
Two input images or concepts are encoded into embeddings, mapped to a shared text space via the Linear Prior Converter from unCLIP~\citep{ramesh2022hierarchical}. These embeddings condition the U-Net: one for downsampling, the other for upsampling.
% module
During the blending stage, a blending latent $L_b$ initialized with Gaussian noise is processed in the Feedback Interpolation Module, conditioned on image embeddings. Noise $\epsilon$ is added to the embeddings to generate initial auxiliary latents, which are interpolated into $L^{(t)}_{b}$ with an increasing weight $p$. The  $L^{(t)}_{a}$ is combined with interpolated latent $L'^{(t)}_{b}$ by proportion $p$. All updated $L'^{(t)}_{a}$ are refined in the auxiliary inference to retain original features using the text prompt for corresponding categories, and $L'^{(t)}_{b}$ is denoised via the blending inference.
% refinement
Finally, the refined $ L_b $ is passed into the VAE decoder to generate the final blending image. 
}
\label{architecture}
\end{center}
% \vskip -0.4in
\vspace{-8mm}
\end{figure*}


\section{Data center infrastructure \& HPC}
\label{sec:HPC}
%This subsection contains all papers regarding FPGA use in dealing with big data, high performance computing and distributed computing.

%%%% This is a long text for intro, but we can polish it and provide a summary as it. Please feel free to revise the text and add/remove sentences:
\iffalse

In this section, we overview the research and activity related to FPGAs in HPC and data centers. In a broader view, even though GPUs are still the dominant accelerators, and AI-specific hardware is growing fast, but FPGAs also have recently gained attraction.

There are several reasons that FPGAs are emerging in HPC and data centers. First, in some cases, they are more cost and energy efficient compared to CPUs and GPUs, which is one of the most important factors in HPC centers nowadays.
Second, FPGAs are capable of direct I/O connection. For example they can directly attach to the network without host intervene through dedicated network stack implemented on FPGAs. 
Third, it enables spatial programming paradigm, e.g., data flow implementation, to reduce data movement (from memory to compute units) compared to the traditional control-based procedural programming~\cite{Licht2022PythonDesign}. 
Fourth, FPGAs as re-configurable hardware provide more flexibility to application developers to have HW-SW co-design and implement domain specific applications with their own constraint metrics. Although, there has been recently enormous effort to mitigate the drawback of programmability of these devices to software developers and end users, but in most cases it is still a challenge to extract good performance out of it and to yield an optimized implementation of an algorithm.

Major data centers such as Microsoft, Alibaba, Amazon, Baidu, Huawei, etc. benefit from FPGAs in their infrastructures~\cite{firestone2018azure,PutnamAServices,caulfield2016cloud,ernst2020competing,xilinx_alibaba}. Some of them only use FPGAs for their internal developed applications. For example in Microsoft Catapult project, FPGAs are used in Microsoft Bing search service by placing a re-configurable logic layer (i.e., FPGAs) between network switches and servers~\cite{PutnamAServices,caulfield2016cloud}. 
On the other hand, some of data centers expose FPGAs as a service to application developers, e.g., AWS. There is an overview in~\cite{Bobda2022TheCloud} of existing academic and commercial efforts of using FPGA acceleration in data centers. They discussed different aspects from architectures, scalability, abstractions to middleware, applications, security and vulnerability of these devices. 

On HPC side, we can also see that FPGAs are emerging as a different type of accelerators. For instance, Fugaku extends its supercomputer center with a scalable FPGA-cluster system~\cite{Sano2023ESSPER:Fugaku}. 
Under AMD university program~\cite{amd_hacc}, some research institutes such as Paderborn University, ETH Zurich, University of California, Los Angeles (UCLA), University of Illinois at Urbana Champagne (UIUC) and National University of Singapore (NUS) deploy Heterogeneous Accelerated Compute Clusters (HACCs).  All these clusters support adaptive computing by incorporating FPGAs in their compute nodes to accelerate scientific applications.   
On a different line of research, Intel and Vmware in collaboration with the University of Toronto, University of Texas at Austin, Carnegie Mellon University initiate Crossroads 3D-FPGA Academic Research Center~\cite{crossroads_fpga}. Their ambition is to define a role for FPGAs as a central function in future data center servers. All these activities indicate the importance role of FPGAs in the future of HPC ecosystem and data centers.

Along with the rapid adaptation of FPGA technology in HPC and data centers, whether it becomes a dominant accelerator is still an open question. One important factor will be the economic advantage, whether they provide more performance with less energy and hence cost for a variety of applications?


%%%%%%%%%%%%%%%%%%%%%%%%% Rephrasing above information with a story line:

In this section, we overview the research and activity related to FPGAs in HPC and data centers. The entrance of FPGAs in HPC and data centers bring a fundamental question: in which position FPGAs can play a significant role in the workflow of HPC and data centers?
%Where in the system design and in which role FPGAs can be beneficial in HPC and data centers workflow?
One historical position of FPGAs in this ecosystem is in the network and communication. This is due to the capability of these devices to have direct I/O connection (without host intervene) to attach to network components (e.g., switches and routers) through dedicated network stack implemented on FPGAs. For example in Microsoft Catapult project, FPGAs are used in Microsoft Bing search service by placing a re-configurable logic layer (i.e., FPGAs) between network switches and servers~\cite{PutnamAServices,caulfield2016cloud}. 

Another straight forward answer to the role of FPGAs is to be as a new type of accelerators sitting along with CPUs and GPUs in HPC and data centers compute nodes. This is due to re-configurability and flexibility of these devices which enables HW-SW co-design and implementation of domain specific applications with their own constraint metrics. Moreover, FPGAs as accelerators enable spatial programming paradigm, e.g., data flow implementation, to reduce data movement (from memory to compute units) compared to the traditional control-based procedural programming~\cite{Licht2022PythonDesign}. 
As an instance of FPGAs as accelerators, we can see that Fugaku extends its supercomputer center with a scalable FPGA-cluster system~\cite{Sano2023ESSPER:Fugaku}. 
Another example is AMD university program~\cite{amd_hacc} where some research institutes such as Paderborn University, ETH Zurich, University of California, Los Angeles (UCLA), University of Illinois at Urbana Champagne (UIUC) and National University of Singapore (NUS) deploy Heterogeneous Accelerated Compute Clusters (HACCs).  All these clusters support adaptive computing by incorporating FPGAs in their compute nodes to accelerate scientific applications.   
With all the efforts have been done to mitigate the drawback of programmability of these devices to software developers and end users, but in most cases it is still a challenge to extract good performance out of it and to yield an optimized implementation of an algorithm.

Intel and Vmware in collaboration with the University of Toronto, University of Texas at Austin, Carnegie Mellon University initiate Crossroads 3D-FPGA Academic Research Center to re-think and find a permanent solution for this question~\cite{crossroads_fpga}. Their ambition is to define a fix role for FPGAs as a central function in future data center servers. In their perspective, FPGAs will be at the heart of the servers as data movement and transformation engine between network, traditional compute units, accelerators and storage.

All these activities indicate the importance, but still ambiguous role of FPGAs in the future of HPC ecosystem and data centers. There is an overview in~\cite{Bobda2022TheCloud} of existing academic and commercial efforts of using FPGAs in data centers. Although this is still an open question, and different ad-hoc solutions have been proposed, but we can say with certainty that one important factor will be the economic advantage; i.e., whether they provide more performance with less energy and hence cost for a variety of applications?

In the rest of this section, we overview the landscape of the FPGA research within the Netherlands in four categories: big data processing and analytics, distributed computing, optical hardware communication and high performance computing.
\fi

%%%%%%%%%%%%%%%%%%%%%%%%%%%%%%%%%%%%%%%%%%A bit Shorter general background
In this section, we overview research and activities related to FPGAs in HPC and data centers. The use of FPGAs in HPC and data centers raises a fundamental question: in which position FPGAs can play a significant role in the workflow of HPC and data centers?
%Where in the system design and in which role FPGAs can be beneficial in HPC and data centers workflow?
One historical position of FPGAs in this ecosystem is in the network and communication. 
This is due to the direct I/O connection capabilities of these devices, allowing them to attach to network components (e.g., switches and routers) through a dedicated network stack directly implemented on FPGAs.
In the Microsoft Catapult project~\cite{caulfield2016cloud, PutnamAServices}, for instance, FPGAs are used in the Microsoft Bing search service as %by placing 
a re-configurable logic layer %(i.e., FPGAs) 
between network switches and servers. %~\cite{PutnamAServices,caulfield2016cloud}. 

Another straightforward answer to this question is to deploy %place 
FPGAs as dedicated %a new type of 
accelerators/co-processors. % along with CPUs and GPUs in HPC and data centers. 
Due to their re-configurability and flexibility, % of these devices, 
FPGAs enable hardware-software co-design and implementation of domain specific applications. Moreover, FPGAs as accelerators facilitate %the % enable 
spatial programming, % paradigm, 
e.g., dataflow implementations, to reduce data movement (from memory to compute units) compared to the traditional, control-based procedural programming~\cite{Licht2022PythonDesign}. 
As an instance of FPGAs as accelerators, Fugaku extends its supercomputer center with a scalable FPGA-cluster system~\cite{Sano2023ESSPER:Fugaku}. 
Another example is through the AMD university program~\cite{amd_hacc}, where some research institutes %all 
around the world deploy Heterogeneous Accelerated Compute Clusters (HACCs).  %All 
These clusters support adaptive computing by incorporating FPGAs in their compute nodes %in order 
to accelerate scientific applications.
Despite substantial efforts to improve the programmability of these devices for software developers and end users, achieving %extracting good 
high performance through %and achieving
an optimized implementation of an algorithm remains a significant challenge in most cases.
%Although there has been enormous effort to mitigate the programmability of these devices to software developers and end users, but in most cases it is still a challenge to extract good performance and to yield an optimized implementation of an algorithm.
Intel and Vmware, in collaboration with %some 
research institutes and universities, established the  Crossroads 3D-FPGA Academic Research Center~\cite{crossroads_fpga} to re-consider and find a permanent solution for this question. Their ambition is to define a fixed role for FPGAs as a central function in future data center servers. From %In 
their perspective, FPGAs will serve as %be at 
the core %heart 
of %the 
servers, acting as data movement and %data 
transformation engines between the network, traditional compute units, accelerators, and storage.


The aforementioned %All these 
activities indicate the important, yet %but still 
ambiguous role of FPGAs in the future of HPC ecosystems and data centers. ~\citet {Bobda2022TheCloud} provide %There is 
an overview %in 
of existing academic and commercial efforts of employing %using FPGAs 
in data centers. Among the commercial efforts, we observe that major data centers such as Microsoft, Alibaba, Amazon, Baidu, and Huawei %, etc. 
benefit from FPGAs in their infrastructures~\cite{firestone2018azure,PutnamAServices,caulfield2016cloud,ernst2020competing,xilinx_alibaba}. 
Although this is still an open question, and various %different 
ad-hoc solutions have been proposed, %but 
%can say with certainty that 
one important factor will be the economic advantage; %Specifically, 
it will depend on whether these solutions can deliver more performance with less energy consumption and lower costs across a range of applications.
%i.e., whether they provide more performance with less energy and hence cost for a variety of applications? 
The rest of this section presents an %, we 
overview of the FPGA research landscape %of the FPGA research with
in the Netherlands, organized into four categories: big data processing and analytics (\ref{sec:big-data-processing-analytics}), distributed computing (\ref{distcomp}), optical hardware communication (\ref{opthwcom}) and high performance computing (\ref{sec:high-performance-computing}).





%%%%%%%%%%%%%%%%%%%%%%%%%%%%%%%%%%%%%%%%%%%%%%%%%%%%%%%%%%%%%%%%%%%%%%%%
% Other strong points of FPGAs:
% - Interfacing in general (e.g. Optical communication)
% - Network attached accelerators https://www.researchgate.net/publication/373405337_FPGA-Based_Network-Attached_Accelerators_-_An_Environmental_Life_Cycle_Perspective
% - SmartNICs https://www.intel.com/content/www/us/en/products/details/fpga/platforms/smartnic.html

\subsection{Big data processing and analytics} \label{sec:big-data-processing-analytics}
% This section discusses FPGAs as data center accelerators for big data processing and analytics. A relatively large amount of papers have been published on big data processing and analytics by Dutch organizations and in collaboration with Dutch organizations. 

% \paragraph{Background}
Several studies in Dutch academia 
 have assessed the domain of big data processing and analytics~\cite{Hoozemans2021FPGAOpportunities, Peltenburg2021GeneratingArrow, Rellermeyer2019TheProcessing, Fang2020In-memorySurvey}, identifying opportunities for FPGA accelerators %in this domain 
 and describing the challenges faced in the wide adoption of FPGA technology. \citet{Peltenburg2021GeneratingArrow} identify %the following %The 
 %main challenges: %identified
 %\cite{Peltenburg2021GeneratingArrow} include : 
 %\textbf{
 the programmability %} 
 of the accelerators, %\textbf{
 the portability %} 
 of the implementation, %\textbf{
 the interface design %} 
 to the data, and %\textbf{
 the infrastructure %} 
 for data movement to/from the accelerator and across % and between 
 %\textit{
 kernels %} 
 running on %in 
 the accelerator, as the main challenges. Solutions %are proposed 
 leveraging various existing technologies have been proposed, e.g., Apache Spark~\cite{ApacheSpark}, Apache Arrow Flight~\cite{ArrowFlight}, the IBM POWER architecture~\cite{7924241}, and OpenCAPI~\cite{OpenCAPI}, while applications 
%Application 
of FPGA accelerators in this domain %, in the Dutch research community, are found in the domains of 
involve database search~\cite{Fang2020In-memorySurvey}, real-time data analysis~\cite{Chrysos2019DataNode}, graph-based processing~\cite{Iosup2023GraphContinuum, Prodan2022TowardsEurope}, high-frequency trading~\cite{Chen2021FPGAAlgorithm}, DNA analysis~\cite{Voicu2019SparkJNI:Spark}, and machine learning~\cite{Rellermeyer2019TheProcessing}.

%: I/O challenges, data format specifications, (de-)compression, applications in graph processing, data base searches etc. 

%On fast retrieval of big data from memory and the communication of big data between hardware nodes through FPGAs. Also some applications of big data processing using FPGAs.

\subsubsection*{\bf{Research topics}}
Several challenges in using FPGAs effectively as accelerators for big data processing and analytics have been addressed by the Dutch research community.
% \begin{itemize}
%     \item \textbf{Interface design and Infrastructure}. Many data structures used in data bases are not well matched with the architecture of an FPGA, thus making processing on an FPGA inefficient. Apache Arrow Flight organizes data movement in a coherent and transparent way across various systems and applications. Fletcher \cite{Peltenburg2021GeneratingArrow}, \cite{Ahmad2022BenchmarkingMicroservices} extends Apache Arrow Flight towards FPGA and defines inter kernel infrastructure between processing kernels implemented in FPGA. Complementary work provides (on-line) data conversion from the often used Parquet \cite{Peltenburg2020BattlingFPGA} and JSON \cite{Peltenburg2021TensAccelerators} formats to Arrow. 
    
%     \item \textbf{Frameworks and tooling}. Addressing \textbf{Programmability}, \textbf{Portability}, \textbf{Interface design} and \textbf{Infrastructure} challenges. Several frameworks are developed to ease programming of FPGA accelerators for big data processing and analytics. \textit{Fletcher} integrates FPGA accelerators with tools and frameworks that use Apache Arrow in their back-ends \cite{Peltenburg2019Fletcher:Arrow}.
%     The open stream-oriented specification Tydi-spec \cite{Peltenburg2020Tydi:Streams} and language Tydi-lang \cite{Tian2022TydilangAL} are developed to specify and implement complex, dynamically sized data structures onto hardware streams.
%     \textit{SparkJNI} enables heterogeneous CPU - FPGA systems based on the Apache Spark unified engine for large-scale data analytics \cite{Voicu2019SparkJNI:Spark}.
%     The work by Abrahamse et al. \cite{Abrahamse2022Memory-DisaggregatedApplications} extends the ThymesisFlow \cite{9252003} memory disaggregration system with a framework leveraging IBM POWER9 and FPGA accelerators.
    
%     \item \textbf{Compression and Decompression}. Addressing the \textbf{Infrastructure} challenge. Both data storage size as well as data movement bandwidth from storage to data processor impose significant challenges in the efficient deployment of accelerators. Data compression is used to reduce both the data storage size and bandwidth requirements. However, compression and decompression of data requires a significant amount of resources. Efforts have been made to (de-)compress data on FPGA either to process the data directly on FPGA or to send them to another component in a system for further processing or storage. Implementations are made based on the Snappy \cite{Fang2019AModel}, LZ77 \cite{Fang2020AnLogic} and Zstd \cite{Chen2021FPGAAlgorithm} (de-)compression algorithms. Also, an energy-efficient co-processor, supporting a range of decompression algorithms, was designed and tested on FPGA \cite{Hoozemans2021EnergyASIP}.
% \end{itemize}
\paragraph{Interface design and infrastructure} Many data structures used in databases do %are 
not map well to the architecture of an FPGA, for example the alignment of data format or the method of data retrieval, %matched with the architecture of an FPGA, 
thus making processing on an FPGA inefficient. Apache Arrow Flight~\cite{ArrowFlight} organizes data movement in a coherent and transparent way across various systems and applications. Fletcher~ \cite{Peltenburg2021GeneratingArrow, Ahmad2022BenchmarkingMicroservices} extends Apache Arrow Flight with FPGA support %for %towards 
%FPGAs 
and defines inter-kernel infrastructure between 
processing kernels implemented in FPGA. Complementary work provides (on-line) data conversion from the widely %often 
used Parquet \cite{Peltenburg2020BattlingFPGA} and JSON \cite{Peltenburg2021TensAccelerators} formats to Arrow. 

\paragraph{Frameworks and tooling} %Addressing %\textbf{Programmability}, \textbf{Portability}, \textbf{Interface design} and \textbf{Infrastructure} challenges. 
Several frameworks have been %are 
developed to ease the programming of FPGA accelerators for big data processing and analytics. %\textit{
Fletcher %} 
integrates FPGA accelerators with tools and frameworks that use Apache Arrow as their back-end~\cite{Peltenburg2019Fletcher:Arrow}.
The open stream-oriented specification Tydi-spec \cite{Peltenburg2020Tydi:Streams} and language Tydi-lang \cite{Tian2022TydilangAL} are developed to specify and implement complex, dynamically sized data structures onto hardware streams.
%\textit{
SparkJNI %} 
enables heterogeneous CPU-FPGA systems based on the Apache Spark unified engine for large-scale data analytics \cite{Voicu2019SparkJNI:Spark}, while 
%The work by Abrahamse et al. 
\citet{Abrahamse2022Memory-DisaggregatedApplications} extend the ThymesisFlow \cite{9252003} memory disaggregration system with a framework leveraging IBM POWER9 and FPGA accelerators.

\paragraph{Compression and decompression} %Addressing the \textbf{Infrastructure} challenge. 

Both the data storage size and the bandwidth required to move data from/to storage %to the processor 
present significant challenges in efficiently deploying accelerators. Data compression is used to mitigate these challenges by reducing both storage size and bandwidth requirements. However, the processes of 
compression and decompression require considerable resources, and efforts have been undertaken to (de-)compress data on FPGAs, either to enable direct data processing on the FPGA itself, or to facilitate data transfers to another system component for further processing or storage. Solutions %Implementations are made 
based on various (de-)compression algorithms have been presented, such as Snappy~\cite{Fang2019AModel}, LZ77~\cite{Fang2020AnLogic} and Zstd~\cite{Chen2021FPGAAlgorithm}. % (de-)compression algorithms. 
%In addition, 
\citet{Hoozemans2021EnergyASIP} present an energy-efficient, FPGA-based co-processor that supports several %a range of 
decompression algorithms. %, was designed and tested on FPGA .

\subsubsection*{\bf{Future directions}}

Research to develop frameworks that enable the efficient use of FPGA accelerators for big data processing and analytics is ongoing. By adopting high-level workflows tailored to these tasks, FPGA accelerators are becoming increasingly applicable within general data center infrastructures and applications. 
%Many work is ongoing on frameworks enabling efficient use of FPGA accelerators for big data processing and analytics. With a higher level workflow suited to big data processing and analytics, FPGA accelerators are more likely to be applied in general data centre infrastructure and applications. 
We see the work referred to in section \ref{sec:big-data-processing-analytics} being continued, %especially the work from the Accelerated Big Data Systems group at TU Delft with IBM and AMD,
as well as being extended with other partners in the industry %, e.g.,  %such as Voltron Data
%for example in 
\cite{10.1145/3624062.3624541, 10305451, Reukers2023AnIR, groet2024leveraging}.
Furthermore, one can not overstep the current rise of ML and AI, which, when applied to big data processing and analytics \cite{Rellermeyer2019TheProcessing} can benefit from FPGA acceleration \cite{10.1145/3613963}. The above listed technologies being developed in the Netherlands can enable the use of FPGAs as accelerators for ML and AI in big data analytics. 

%%%%%%%%%%%%%%%%%%%%%%%%%%%%%%%%%%%%%%%%%%%%%%%%%%%%%%%%%%%%%%%%
\subsection{Distributed computing}
\label{distcomp}
% This section discusses the topic of distributed and decentral computing to accelerate computations through the use of (multiple) FPGAs.

% \subsubsection{Background}
Distributed computing involves the deployment of multiple computing nodes in parallel to increase performance and solve large computational problems. %is a common approach where either we hit the resource limit within a compute node due to large problem size, or we aim at increasing the performance by using different compute resources in parallel. 
%Distributed computing is a natural approach to mitigate the challenge of resources limitation within a physical compute node. 
%When we encounter a situation where the size of the problem is much larger to keep in one compute node, distributing different computation elements to different compute nodes is an intuitive solution. Moreover, we can also distribute the computation workflow on different compute nodes in order to increase performance. 
%In contrast for CPUs and GPUs which use DRAM-based cache-hierarchy memory structure, FPGAs use various memory technologies such as LUTs, BRAM, URAM, HBM, etc. which have lower latency and higher bandwidth compared to DRAM. 
%Thus, FPGAs appear to be more efficient in memory-bound computation tasks. 
While %The field of 
research on distributed computing involving CPU and GPU nodes is well established, %for CPU, and also GPU nodes is not a new area. However, as 
the emergence of FPGAs %are emerging 
as a new type of computational resources and accelerators within data center infrastructures introduces a new and challenging area of research. %, the research in this direction is new and challenging. 
The Dutch academia has mainly focused on applications that use distributed multi-FPGA systems for %
%The applications that in Dutch academia are investigated using distributed multi-FPGAs are 
large-scale graph processing~\cite{Sahebi2023DistributedFPGAs} and deep neural networks (DNNs)~\cite{Alonso2021Elastic-DF:Partitioning}.  

\subsubsection*{\bf{Research topics}}
%There are 
Several research topics have been %that are 
investigated by Dutch researchers. 

% \begin{itemize}
%     \item \textbf{Communication overhead}. Reducing communication is a key factor in distributed computing, and in particular in multi-FPGA systems. By reducing communication overhead, computation time and latency reduces and efficiency increases. In order to reach this goal, researchers propose interconnection frameworks to establish flexible, reliable, efficient and custom communication protocols in multi-FPGA systems~\cite{salazar2020plasticnet,Salazar-Garcia2021PlasticNet+:Transceivers,Salazar-Garcia2022AApplications}. In addition to reducing latency, these proposed frameworks are designed to work with different topology schemes, and different FPGA technologies.      
    
%     \item \textbf{Partitioning and performance scaling}. 
%     In order to increase performance in multi-FPGAs systems, researchers propose an open-source distributed resource partitioning and allocator tool on FPGAs for data flow architectures targeting DNN inference which works in conjunction with FINN compiler. They demonstrate their methodology enables super-linear scaling of throughput, by benefiting from model parallelism and direct FPGA-FPGA communication~\cite{Alonso2021Elastic-DF:Partitioning}.
%     Another different research proposes a (multi-FPGA) framework to process large-scale graph processing. The framework uses an offline partitioning mechanism, and it uses Hadoop to map the graph into the underlying hardware. They show that graph partitioning using FPGA architecture results in better performance on large graphs included millions of vertices and billions of edges. Their results indicate a significant speed-up compared to the state-of-the art CPU, GPU and FPGA solutions~\cite{Sahebi2023DistributedFPGAs}.  


    
% \end{itemize}
\paragraph{Communication overhead} Reducing communication is a key factor in distributed computing, and in particular in multi-FPGA systems. By reducing communication overhead, computation time and latency reduces and efficiency increases. To reach this goal, researchers propose interconnection frameworks to establish flexible, reliable, efficient and custom communication protocols in multi-FPGA systems~\cite{salazar2020plasticnet,Salazar-Garcia2021PlasticNet+:Transceivers,Salazar-Garcia2022AApplications}. In addition to reducing latency, these %proposed 
frameworks are designed to work with various %different 
topology schemes and different FPGA technologies.

\paragraph{Partitioning and performance scaling} %In order 
To increase performance in multi-FPGA systems, \citet{Alonso2021Elastic-DF:Partitioning} %researchers have 
propose an open-source, distributed resource partitioning and allocator tool on FPGAs for data flow architectures targeting DNN inference; it %which 
works in conjunction with the FINN compiler~\cite{umuroglu2017finn}. The authors show that %demonstrate 
their methodology enables super-linear scaling of throughput by benefiting from model parallelism and direct FPGA-FPGA communication. %~\cite{Alonso2021Elastic-DF:Partitioning}.
\citet{Sahebi2023DistributedFPGAs} %Another different research 
propose a (multi-FPGA) framework for %to process 
large-scale graph processing. The framework uses an offline partitioning mechanism, and relies on %it uses 
Hadoop to map the graph into the underlying hardware. The authors show that graph partitioning using an FPGA architecture results in better performance on large graphs that include millions of vertices and billions of edges. Their results indicate %a 
significant speed-ups over %compared to the 
state-of-the art CPU, GPU, and FPGA solutions.


\subsubsection*{\bf{Future directions}}

There are several %different 
challenges in distributed computing using multi-FPGA systems, thereby %which 
necessitating %requiring 
further research in this direction. For instance, overcoming communication barriers and designing protocols for FPGA-FPGA communication is an ongoing research domain. Moreover, at the application level, developing (standard) MPI-like collective communication libraries for multi-FPGA systems would be beneficial. %desired. 
Also, %We also need 
more case studies are needed in order to investigate and design efficient partitioning and workload distribution schemes %and %assigning different 
%distribute computational tasks 
for FPGA resources. Therefore, to bring ease-of-use and automation for distributed computing on FPGAs, developing libraries and tools  is crucial. 


\iffalse
\subsubsection{PlasticNet: A low latency flexible network architecture for interconnected multi-FPGA systems; PlasticNet+: Extending multi-FPGA interconnect architecture via Gigabit transceivers}
The paper focuses on addressing the challenges of multi-FPGA system communication. They propose an extension over PlasticNet framework via flexible, efficient, reliable, custom protocol. PlasticNet framework is a FPGA interconnect architecture of processing units for both within a board or among neighboring FPGA boards. Their extended proposal improves PlasticNet over area, channel overhead and latency. They evaluate their approach using a ring-based topology of Zynq ZC706 FPGA boards. They report the best-case latency of 300 ns which is half the latency of an Ethernet 10G link. Another advantage of the proposed approach is the adaptability to a wider range of interconnect topologies.

\subsubsection{Elastic-DF: Scaling Performance of DNN Inference in FPGA Clouds through Automatic Partitioning}
Power dissemination and multi-tenancy are two important factors for data center operators. FPGAs gains attraction to data centers, in particular for Deep Neural Networks (DNN) inference applications, as they are well aligned with the two factors. However, it is challenging as DNN inference on FPGA requires the right choice of architecture and a set of implementation tools. Therefore, the authors proposed an open-source distributed resource partitioning and allocator tool on FPGAs for data flow architectures targeting DNN inference which works in conjunction with DFA compiler FINN. They experiment their approach on FPGA cluster at ETH Zurich. It includes four FPGA-equipped nodes; a mix of Alveo U250 and U280 (10 cards in total). The FPGAs are connected using 100 Gb/s interfaces. They use different FPGA accelerator deployment software: XRT, PYNQ, Jupyter Lab. Dask, and InAccel Coral. They evaluated several DNN accelerator implementations of FINN-generated DF accelerators for quantized MN and RN-50 classifiers. They demonstrate their methodology enables super-linear scaling of throughput, by benefiting from model parallelism and direct FPGA-FPGA communication over 100 G bps Ethernet connection. Concretely, they show 44\% latency decrease on U280 for ResNet-50, and 78\% throughput increase on U200 and U280 for MobileNetV1.

\subsubsection{A custom interconnection multi-FPGA framework for distributed processing applications}
One of the main challenges in FPGA clusters is to reduce communication overhead between network elements to reduce computation time and maximize efficiency of processing elements on FPGAs. Therefore, the authors of the paper propose a multi-FPGA interconnection framework targeting distributed applications. Thus, they build multi-FPGA systems included 5 Zynq ZC706 FPGAs over their custom network. They assume an application can be accelerated by decomposing its computation and distributed into different processing elements on a multi-FPGA machine. The authors show the effectiveness of their framework using matrix multiplication algorithm. With the aggregated bandwidth of 25 Gbps per FPGA, their framework shown the latency of 200 ns, an efficiency of 97.25\% and throughput of 21.4 GFLOPS. Another advantage of their approach is portability of the proposed network interconnect to newer generation of FPGAs.

\subsubsection{Distributed large-scale graph processing on FPGAs}
Large-scale graph processing is challenging and causes performance degradation due to irregular structure and memory access on both CPUs and GPUs. The authors propose a FPGA engine, as part of a framework to overlap and hide data transfers in order to maximize utilization of FPGA accelerator. The framework uses an offline partitioning mechanism, and it uses Hadoop to map the graph into the underlying hardware. They show that graph partitioning using FPGA architecture results in better performance on large graphs included millions of vertices and billions of edges. They benefit from the partitioning scheme in GridGraph library, but on FPGA instead. They use Xilinx Vivado HLS toolchain for their implementation on Alveo U250 card. Their optimized implementation of the PageRank for a single FPGA outperforms state-of-the art CPU, GPU and FPGA solutions: a speed up to GridGraph by 2x, Cugraph by 4.4x and VITIS LIB by 26x. Even when the size of graphs limit the performance of a FPGA, their approach shows a speed up about 12x using multi-FPGAs. 
\fi
%%%%%%%%%%%%%%%%%%%%%%%%%%%%%%%%%%%%%%%%%%%%%%%%%%%%%%%%%%%%%%%%%%%%%%%5
\subsection{Optical hardware communication}
\label{opthwcom}
% This section examines the use of FPGA-based systems in optical hardware communication, a field that is gaining traction for its potential to revolutionize data center network (DCN) infrastructures.

% \subsubsection{Background}
Optical hardware communication is at the forefront of addressing the critical challenges faced by contemporary data center network (DCN) infrastructures, such as bandwidth limitations, latency issues, and scalability concerns. Optical communication is a viable alternative to conventional electrical data pathways, offering significant improvements in terms of efficiency and performance. The integration of FPGAs into optical communication systems has been a key development, providing the necessary flexibility and speed for dynamic network reconfiguration and management.

\subsubsection*{\bf{Research topics}}
The exploration of optical hardware communication utilizing FPGAs encompasses a variety of innovative research topics covered by Dutch organizations.
\paragraph{Optical wireless datacenter networks}
%Implementing semiconductor optical amplifier (SOA)-based wavelength selectors and arrayed waveguide grating routers (AWGRs) controlled by fast FPGA-based switch schedulers. 
\citet{Zhang2022Low-LatencyRouter} have developed an optical wireless (OW)-DCN architecture that promises enhanced flexibility and scalability for DCNs, supporting high-speed optical packet-switching transmissions. FPGA-based switch schedulers are used for control of the implementation based on semiconductor optical amplifier (SOA)-based wavelength selectors and arrayed waveguide grating routers (AWGRs).

\paragraph{Disaggregated optical networks}
The DACON project~\cite{Guo2022DACON:Invited} introduces a Disaggregated, Application-Centric Optical Network that utilizes hybrid optical switches and FPGA-based controllers, resulting in improved application performance and reduced latency.

\paragraph{Low-latency edge networks}
The Electro-Optical Communication group at TU/e has proposed an edge data center network architecture that employs photonics and FPGA-based supervisory channels to achieve microsecond-time control and deterministic latency \cite{Santana2023SOA-BasedApplications}.

\paragraph{Nanosecond optical switching}
A novel optical switching and control system has been designed to address the bandwidth bottlenecks of electrical switching, featuring a distributed network architecture with optical label channels and the Optical Flow Control (OFC) protocol \cite{Xue2022NanosecondNetworks}.

\paragraph{Hybrid datacenter architectures} The HiFOST DCN architecture~\cite{Yan2018HiFOST:Switches} integrates flow-controlled fast optical switches with modified top-of-the-rack switches, offering substantial improvements in latency and cost efficiency.

\paragraph{Beyond 5G networks} \citet{Santana2022TransparentApplications} present a new Edge Cloud Network design %has been put forward, 
that uses %utilizing 
FPGA-based controllers for rapid reconfiguration of optical networks, catering to the low-latency requirements of 5G applications and beyond.  %\cite{Santana2022TransparentApplications}.

% \begin{itemize}
%     \item \textbf{Optical Wireless Data-Center Networks}. Implementing semiconductor optical amplifier (SOA)-based wavelength selectors and arrayed waveguide grating routers (AWGRs) controlled by fast FPGA-based switch schedulers. Researchers have developed an optical wireless (OW)-DCN architecture that promises enhanced flexibility and scalability for DCNs, supporting high-speed optical packet-switching transmissions.\cite{Zhang2022Low-LatencyRouter}

%     \item \textbf{Disaggregated Optical Networks}. The DACON project introduces a Disaggregated, Application-Centric Optical Network that utilizes hybrid optical switches and FPGA-based controllers, resulting in improved application performance and reduced latency.\cite{Guo2022DACON:Invited}

%     \item \textbf{Low-Latency Edge Networks}. The Electro-Optical Communumuroglu2017finnumuroglu2017finnication group at TU/e has proposed an edge data center network architecture that employs photonics and FPGA-based supervisory channels to achieve microsecond-time control and deterministic latency.\cite{Santana2023SOA-BasedApplications}

%     \item \textbf{Nanosecond Optical Switching}. A novel optical switching and control system has been designed to address the bandwidth bottlenecks of electrical switching, featuring a distributed network architecture with optical label channels and the Optical Flow Control (OFC) protocol.\cite{Xue2022NanosecondNetworks}

%     \item \textbf{Hybrid Data Center Architectures}. The HiFOST DCN architecture integrates flow-controlled fast optical switches with modified top-of-the-rack switches, offering substantial improvements in latency and cost efficiency.\cite{Yan2018HiFOST:Switches}

%     \item \textbf{Beyond 5G Networks}. A new Edge Cloud Network design has been put forward, utilizing FPGA-based controllers for rapid reconfiguration of optical networks, catering to the low-latency requirements of 5G and beyond applications.\cite{Santana2022TransparentApplications}

% \end{itemize}

\subsubsection*{\bf{Future directions}}
% The field of optical hardware communication is poised for significant advancements, with ongoing research directed towards:
% \begin{itemize}
%     \item \textbf{Visible Light Communications (VLC)}. Efforts to mitigate LED nonlinearity have led to the development of a Legendre-polynomials-based post-compensator optimized for FPGA implementation, enhancing the bit rate efficiency of high-speed VLC systems.\cite{Niu2021LEDCommunications}
%     \item \textbf{Real-Time LED Modeling}. The introduction of a real-time FPGA-based implementation of a nonlinear LED model and post-compensator marks a substantial contribution to VLC technology, enabling high data rates over bandwidth-limited LEDs.\cite{Deng2022Physics-BasedImplementation}
%     \item \textbf{Concurrency-Aware Mapping in HPC}. A concurrency-aware mapping technique has been developed to reduce optical packet collisions in Architecture-On-Demand  (AoD) network infrastructures, improving buffer utilization and execution time degradation in HPC systems.\cite{Meyer2018OpticalPerformance}
% \end{itemize}

Through various ongoing developments, the field of optical hardware communication is poised for significant advancements. Efforts to mitigate LED nonlinearity have led to the development of a Legendre-polynomials-based post-compensator optimized for FPGA implementation, enhancing the bit rate efficiency of high-speed Visible Light Communications (VLC) systems \cite{Niu2021LEDCommunications}. The introduction of a real-time FPGA-based implementation of a nonlinear LED model and post-compensator marks a substantial contribution to VLC technology, enabling high data rates over bandwidth-limited LEDs \cite{Deng2022Physics-BasedImplementation}. 
A concurrency-aware mapping technique has been developed to reduce optical packet collisions in Architecture-On-Demand  (AoD) network infrastructures, improving buffer utilization and execution time degradation in HPC systems \cite{Meyer2018OpticalPerformance}.


%\cite{mendely.bib key}

%%%%%%%%%%%%%%%%%%%%%%%%%%%%%%%%%%%%%%%%%%%%%%%%%%%%%%%%%%%%%%%%%
\subsection{High performance computing}\label{sec:high-performance-computing}
% In this section, we discuss the research of utilizing FPGAs in HPC ecosystem by Dutch researchers.

%\subsubsection{Background}
Benefiting from FPGAs in %for 
HPC applications is an active research area. % nowadays. 
Even though GPUs remain %are still 
the most prevalent %dominant 
accelerator technology in HPC, and AI-specific hardware is being  increasingly adopted, %growing fast, but 
FPGAs are also %have recently gained attraction, and even we can observe (experimental) FPGA deployments on 
increasingly employed in HPC centers. 

\subsubsection*{\bf{Research topics}}
Dutch institutes have been involved in European projects, e.g., %such as 
ExaNeSt~\cite{Katevenis2018NextDevelopment} and MANGO~\cite{Flich2018ExploringApproach}, to %in 
design %ing of 
large-scale heterogeneous compute systems. We can observe 
the important role of FPGAs in these projects, facilitating network communication or accelerating execution. % as either considering in the network or as accelerators: 

% \begin{itemize}
%     \item \textbf{Architecture and system design}.
\paragraph{Architecture and system design}
    The %At 
    ExaNeSt European project~\cite{Katevenis2018NextDevelopment} deploys FPGAs %are proposed 
    as accelerators in % for 
    a European 
    exascale supercomputer based on low-cost, low-power %many 
    ARM cores. They also employ an FPGA-based testbed for a low-latency, high bandwidth unified Remote Direct Memory Access (RDMA) interconnect, and present %hey design 
    a custom FPGA-based switch to support inner-cabinet communications.
   The MANGO project~\cite{Flich2018ExploringApproach} aims at addressing the PPP (Performance, Power, and Predictability) space in HPC %: Performance, Power and Predictability 
   by exploring %and investigating 
   customizabe and deeply heterogeneous accelerators. Their hardware concept consists of General-purpose compute Nodes (GNs) with %, having 
   commercial accelerators such as Xeon Phi and NVIDIA GPUs, along with Heterogeneous Nodes (HNs). HNs are clusters of many-core chips coupled with customized heterogeneous computing resources, including high-capacity clusters of FPGAs.
    
    % \item \textbf{Programming languages, tools and applications}. 
\paragraph{Programming languages, tools, and applications}
    Within the ExaNeSt project, \citet{Katevenis2018NextDevelopment} %, they 
    design a novel microarchitecture as Top-of-Rack switches. In one of their experiment, they port the OpenCL kernels of the molecular dynamics simulator LAMMPS~\cite{plimpton1995fast} to FPGAs using HLS tools. 
    %They report that the use of an FPGA improves %the 
    %performance compared to using ARM cores, more than a factor of two.
    They report that running the kernel on an FPGA requires 0.56 seconds while the 4 ARM cores requires 1.3 seconds. That is an improvement of more than a factor 2 in speed up.
    Within the MANGO project, ~\citet{Flich2018ExploringApproach} target three applications with significant QoS aspects: 1) online video transcoding, 2) rendering for medical imaging, and 3) error correcting codes in communication. The MANGO project relies on LLVM~\cite{lattner2004llvm} and their programming model is an extension of %the 
    existing languages and libraries (e.g., OpenCL~\cite{opencl}) for HPC by integrating the expression of new architectural features as well as QoS concerns and parameters. This is achieved %They do it 
    by augmenting the runtime library API with new functions, pragmas and keywords to the existing HPC languages (e.g., clang C/C++ frontend). 
    
    % \item \textbf{Performance Models}. 
\paragraph{Performance models}
    Combining the advantages of reconfigurability, dataflow computation, and heterogeneity results in %yields 
    Reconfigurable Dataflow Platforms (RDPs) as a promising building block in %the 
    next-generation, large-scale high-performance machines. RDPs rely on %include 
    Reconfigurable Dataflow Accelerators (RDAs) to realize multiple streaming pipelines, each % which each 
    comprising many parallel operations. Due to the % such 
    heterogeneous hierarchy, 
    performance prediction of RDPs is very challenging, in particular to detect bottlenecks within reasonable time and accuracy. %Therefore, 
\citet{Yasudo2021AnalyticalPlatforms,Yasudo2018PerformancePlatforms} %Dutch researchers have been involved in a project to 
propose a performance estimate framework for reconfigurable dataflow applications %(i.e., RDPs), 
    named Performance Estimation for Reconfigurable Kernels and Systems (PERKS). %~\cite{Yasudo2021AnalyticalPlatforms,Yasudo2018PerformancePlatforms}. 
    It %PERKS 
    automatically extracts specific parameters from the application, the hardware, and the platform to calibrate the model. They use eight applications for their evaluation: AdPredictor (an online machine learning algorithm), N-body simulation, Monte Carlo simulation, sequence alignment, Asian option pricing, Jacobi solver, and Regression/regularisation solver. Their results show that PERKS achieves %performs an 
    accuracy of 91\% on these applications.
% \end{itemize}

\subsubsection*{\bf{Future directions}}
\iffalse
Determining the role of FPGAs in HPC necessitates %demands 
more research and %it 
raises several %many 
questions, such as 
1) Do we have to find a permanent position for FPGAs in HPC ecosystems to maximize their impact? % the most? 
If so, what would that position be, %Where would be that position 
from both architectural system design and application workflow perspectives?,
2) How can we bridge %fill out 
the gap between software developers and FPGA programming models and tools? Should we focus on HLS approaches or compiler-specific tools, or a combination of both?,
%Through HLS approach or compiler specific tools?
3) What types of HPC applications can benefit from FPGAs?, and last but not least 4) Are FPGAs cost-efficient in terms of energy and performance to warrant a permanent position in future HPC centers? 
%Whether FPGAs would be cost efficient in terms of energy and performance to dictate a permanent position in HPC centers? 
Further research and case studies are required to obtain %gain 
more insights in order to answer %address 
these questions. This future direction and the corresponding outcomes will indicate how important FPGAs will be in the future of HPC and data centers. 
\fi

Determining the role of FPGAs in HPC necessitates more research from both data center architecture design and FPGA programming model.
From a data center design perspective, the positioning of FPGAs in the architecture of HPC centers needs more investigation. This also depends on the targeted application workflow and how FPGA can impact the most.    
From a user perspective, the programmability of these devices is an important factor. Therefore, the gap between software developers and FPGA programming models and tools should be reduced further to use FPGA as mainstream HPC devices.

\iffalse
\subsubsection{Next generation of Exascale-class systems: ExaNeSt project and the status of its interconnect and storage development}
ExaNeSt European project merges industry and academia in the area of system cooling, storage, network and interconnect, and HPC applications. They aim to develop system-level interconnect and distributed non-volatile memory storage for a European exascale supercomputer based on low cost and power many ARM cores and computing accelerators implemented in programmable components (FPGAs). In this paper they explain the project in terms of hardware architecture and software stack development. The breakdown of the components in this project is as follows: 1) A low-latency, high bandwidth unified RDMA interconnect. They use FPGA-based testbed for this purpose. 2) Providing low-latency inter-process communication as needed on HPC workflows. 3) A novel distributed storage architecture. 4) A set of exascale scientific applications such as MonetDB and LAMMPS 5) Packaging and advanced cooling system.
The ExaNeSt interconnect has three components: 1) Network interface which bridges the processes that run on ARM cores with the communication layer of the network. 2) Intra-rack network IP based on APEnet which provides switching and routing features and manages communications over links. 3) A novel micoarchitecture as Top-of-Rack switches. They design a custom FPGA-based switch to support inner-cabinet communications.
The unit of the system is the Xilinx Zynq UltraScale+ FPGA integrating four 64 bit ARMv8 Cortex-A53 hard-cores running 1.5 GHz. 
In one of their experiment, they port the OpenCL kernels of LAMMPS to FPGA using HLS tools. They report the use of FPGA improves the performance significantly compared to using ARM cores, more than a factor of 2. 

\subsubsection{Exploring manycore architectures for next-generation HPC systems through the MANGO approach}
The paper explains the main approach and architectural solution, application scenario and software stack in MONGO project. The MANGO project aims at addressing the PPP space in HPC: Performance, Power and Predictability by exploring and investigating customizabe and deeply heterogeneous accelerators. They target three applications with significant QoS aspects: 1) Online video transcoding 2) Rendering for medical imaging 3) Error correcting code in communication. Their hardware concept consists of General-purpose compute Nodes (GNs), having commercial accelerators such as Xeon Phi and NVIDIA GPUs, along with Heterogeneous Nodes (HNs). HNs are clusters of manycore chips coupled with customized heterogeneous computing resources. Their deployment platform consists of 16 GNn with standard processors, e.g., Intel Xeon E5 and Kepler GPUs, and 64 HNs consists of ASIC ARM cores and high-capacity cluster of FPGAs. GNs and HNs are connected via infiniband. Their programming model is an extension of the existing languages and libraries for HPC by providing new architectural features and QoS concerns and parameters. They do it by augmenting the runtime library API with new functions, pragmas and keywords to the language.

\subsubsection{Analytical Performance Estimation for Large-Scale Reconfigurable Dataflow Platforms; Performance Estimation for Exascale Reconfigurable Dataflow Platforms}
Combining the advantages of reconfigurability, dataflow computation and heterogeneity yields Reconfigurable Dataflow Platforms (RDPs) as promising building block in the next generation of large-scale high-performance machines. RDPs include Reconfigurable Dataflow Accelerators (RDAs). Various hardware, computation, storage and communication elements are costumized for a specific hardware design to implement an algorithm. As such performance prediction of RDPs becomes very challenging, in particular to detect bottlenecks within reasonable time and accuracy. The authors of the paper propose a performance estimate framework for reconfigurable dataflow applications (i.e., RDPs), named Performance Estimation for Reconfigurable Kernels and Systems (PERKS). PERKS automatically extract specific parameters from the application, hardware and platform to calibrate the model. PERKS allows to predict performance of multi-accelerator systems using analytical model along with machine and application parameters. Their experimental setup is RDPs from Maxeler, known as Maxeler DFEs. A node of 12-core Intel Xeon CPU connects to 4 DFEs (via PCI Express). Each DFE has a Xilinx v6-SXT475 FPGA and 48 GB of DRAM. They use 8 applications for their evaluation: AdPredictor (an online machine learning algorithm), N-body simulation, Monte Carlo simulation, sequence alignment, Asian option pricing, Jacobi solver, and Regression/regularisation solver. Their results show that PERKS performs an accuracy of 91\% on current reconfigurable workloads.
\fi

%1. One paragraph summary (of each paper):
%        What is the contribution
%            system design/architecture (FPGA position), e.g. PCIe device, network attached accelerator etc.
%            Application/case study/technology development
%            In case of accelerator, what is the workflow. In case of infrastructure 
%            Product/research? How does this relate to research topics?
%               If it's a product, which research groups did contribute to it
%               open source?, actively maintained? 
%           Is the research used somewhere?
%        Advantages/Disadvantages
%        Forward look, where do we see this going in the future?

% Paragraph:
% Publication title
% short answers to above questions

 

%2. Merge/combine all ones into one (sub)section: Deadline March 31



\section{Programming Models and Tools}% (Tiziano, Steven, Christiaan) (CONT)}
\label{sec:programming}
% This section discusses models, tools, and techniques that facilitate porting of existing software to hardware accelerators. 

% \textcolor{blue}{Notes for Nikos/Sjoerd:
% \begin{itemize}
%     \item We don't have a lot of paper per subtopic, and inside each subtopic the papers are also quite different (in terms of goals). So we propose reducing the topic to just two: programming models/frameworks and performance prediction (or tools)
%     \item All the papers may have additional details embedded as latex comments
%     \item one of the select papers ( "Modeling FPGA-Based Systems via Few-Shot Learning") is actually a poster with a short abstract. We believe this is a subset of another paper ("LEAPER...") and therefore we suggest removing it.
%     \item Regarding the "Future direction": the current picture of the Dutch research in programming models and tools is quite scattered and with no clear (at least to us) direction. Out of the 14 papers, only 4 are led by Dutch institutions, and it is a bit hard to say on what Dutch research groups are leaders. At the moment, the corresponding subsections contain high-level overviews.
%     \item a comment that applies to both subtopics is that there are very few open-source tools, or even just publicly available artifacts, so we can make the call for action in this regard (probably it applies to other sections)
%     \item we noticed that there are some shared references with Sect. \ref{sec:big-data-processing-analytics} (Big data processing -- e.g., tydi) and Sect. \ref{sec:high-performance-computing} (High-Performance computing -- e.g., PERKS), so maybe it makes sense to link them together
%     \item you can remove these notes afterward
% \end{itemize}
% }


This section discusses approaches that boost developer productivity; Section~\ref{prog_models_frameworks} reviews programming models and frameworks that raise the abstraction level of describing hardware, while Section~\ref{perf_pred} presents 
techniques that predict performance of synthesized programs.

\subsection{Programming models and frameworks}
\label{prog_models_frameworks}
% This section presents developments of programming models and frameworks, including HLS-based, aimed at reducing development time for FPGA-based designs.

% \subsubsection{Background}
FPGA development is traditionally characterized by a steep learning curve, especially for non-experts. For this reason, High-Level Synthesis (HLS) tools and, more generally, high-level programming models and frameworks have been proposed to increase productivity by raising the abstraction level. 
HLS tools became commercial products in the early 2010s. Since then, they have been used in various application domains, including, but not limited to, deep learning, multimedia, graph processing, and genome sequencing ~\cite{Cong-2022}. HLS tools can reduce the average development time (up to two-thirds compared to RTL~\cite{Lahti-2019}). However, they still require considerable expertise to optimize the FPGA designs and achieve a Quality-of-Result that is on par with the one obtained through hardware description languages. For this reason, higher-level programming models and frameworks are now being proposed. They allow developers %user 
to describe hardware %write their 
%programs 
in a more convenient formalism (e.g., Clash~\cite{clash-2010,clash-website}, HeteroCL~\cite{heterocl-2019}, PyLog~\cite{pylog-2021}, and DaCe~\cite{dace-2021} which currently support Haskell, Python or a Python-embedded DSL), and %they %take care of 
automatically, or via user-provided hints, generate optimized %lower it into 
HLS/HDL descriptions. %and optimize the final design. 

%\subsubsection{Current research in the Netherlands}

\subsubsection*{\bf{Research topics}} Several papers have been published by Dutch organizations focusing on reducing development time for hardware design using HLS programming models and tools.

\paragraph{Abeto framework%: a Solution for Heterogeneous IP Management
}
% Reviewer: Tiziano
% Contribution:
%   - this is not a programming framework, rather a tool for IP management and workflow automation
%   - it should be general enough to be used for various application mains
%   - it is not open-source, probably still maintained
%   2. Advantages/Disadvantages, how to position this work related to state of the art: 
%       - in the paper the authors mention several other IP management applications, but they say these are usually "rigid" in the expected IP core format and require considerable effort

%Sanchez et al. propose Abeto 
\citet{Sanchez2022AbetoManagement} propose Abeto, a software tool for IP management and workflow automation. Historically, there has been no established standard for packing, documenting, and distributing IP core designs.  This prevents their re-usability, as each IP core has its unique learning curve and challenges for using them in an EDA toolchain. Abeto allows the user to operate in a unified manner with heterogeneous IP cores, and conveniently configure and launch the different stages of the IP workflow. To add an IP core, Abeto requires some auxiliary information to be provided: a database definition (containing information about the directory structure of the IP core) and a command dictionary (which includes the list of supported IP commands and how they must be executed). The tool has been validated against a subset of the ESA portfolio of IP cores\footnote{\url{https://www.esa.int/Enabling_Support/Space_Engineering_Technology/Microelectronics/ESA_HDL_IP_Cores_Portfolio_Overview}}, which constitutes a heterogeneous group of IP cores, demonstrating the tool's versatility.


\paragraph{%A Complete Open Source
Design flow for Gowin FPGAs}
% Reviewer: Christiaan
% Contributions
% - The contribution describes a framework to configure Gowin FPGAs using open-source tools
% - It is general purpose insofar that any application can make use of the new flow
% - It is an open source technology development, and contributions have been merged upstream
% - The work is analytical in that is describes the method for documenting the bitstream format.
%De Vos et. al.~
\citet{vos2020gowin} describe a method %and results 
to create an open-source design flow for the Gowin LittleBee family of FPGAs.
The design flow is based on well-known open-source tools such as Yosys and nextpnr, as well as the newly developed bitstream generator.
Architectural details of the FPGA family were documented using input fuzzing and comparing results from the existing closed-source vendor tool flow.
While the created open-source flow is capable of synthesizing a full RISC-V core, many aspects, such as DSPs, RAMs, and PLLs are currently unsupported. The authors report that documenting the bitstream format for all of these features is %will be 
the subject of future work.


%\paragraph{AEx: Automated High-Level Synthesis of Compiler Programmable Co-Processors -- \cite{Hirvonen2023AEx:Co-Processors}} 

\paragraph{AEx framework} 


% Reviewer: Tiziano
% Contributions:
% - a framework for overlays/FPGA design
% - seems general purpose but this is not clear
% - it is the result of a European project (https://fitoptivis.eu/) but there is no mention that this is opensource
\citet{Hirvonen2023AEx:Co-Processors} propose %in this paper 
AEx, a framework for automated High-Level synthesis of compiler programmable co-processors. AEx can be used to produce Application-Specific Instruction-Set (ASIP) architectures. ASIP processors have been proposed as a way to produce FPGA overlays starting from a software-programmable template. The program being executed can be easily changed, reducing design time and costs. The template being used by AEx is Transport Triggered Architecture (TTA).
% In case, a short explanation of TTAs
%TTAs are architectures in which programs have more low-level control of data transfers between processor functional units. This is in contrast with  directly control the internal transport buses of a processor. 
AEx includes heuristics for design space exploration and pruning, aimed at finding the best architecture able to satisfy real-time execution time and clock frequency constraints. The user can then choose the results that better fit their need (e.g., minimum resource utilization). Evaluation %The results 
shows how the tool is able to produce results in a reasonable amount of time, achieving %with 
performance close to that of the fixed-function implementations generated by HLS vendor tools such as AMD/Xilinx Vitis.


%\paragraph{Exploration of Synthesis Methods from Simulink Models to FPGA for Aerospace Applications}

\paragraph{Synthesis from Simulink Models to FPGA for Aerospace Applications}

% Reviewer: Tiziano
% Contributions:
% - a survey/analysis of methods to generate FPGA design from Matlab Simulink model (for space applications)
% - they considered a specific use case (Simulink)
% - nature of the work: comparative. It is not an actual framework, but rather a collection of suggestions
% - How to position this work wrt background? Not the first paper I believe to discuss the advantage of FPGA in aerospace. Maybe one of the few to discuss how to leverage HLS in a convenient way for aerospace engineers

Reconfigurable hardware is becoming an attractive solution for aerospace applications, thanks to its power efficiency and capabilities of in-flight configuration. Algorithms are usually expressed in model-based programming frameworks, e.g., Matlab Simulink, but turning them into low-level hardware description languages can be cumbersome. \citet{Curzel2023ExplorationApplications} analyze solutions to automatically synthesize Simulink models. Matlab already provides an automated method (HDL coder) to translate part of Simulink models into Verilog/VHDL, but this still requires a certain level of expertise. Therefore, the authors propose to apply HLS on the code generated by Matlab's Embedded Coder tool, further automatizing the design process. Experiments with three %different 
benchmarks show that this solution is more efficient than relying on HDL coder, and it does not require specific hardware expertise. % to the user.
% Future work: how to automatize HW/SW partitioning, quality of generated C code.

\paragraph{HLS optimizations for  post-quantum cryptography on Lattice FPGAs}
%\paragraph{Optimizing Lattice-based Post-Quantum Cryptography Codes for High-Level Synthesis}

% Reviewer: Tiziano:
% Contributions:
% - analysis and improvement of HLS based implementation of post-quantum crypto algorithms
% - they considered a specific use case
% - nature of the work: implementation. They claim they would open-sourced it, but I can not find it
% - How to position this paper: this paper does not introduce anything new, but applies known optimization to a specific application domain
%In this paper, Guerrieri et al.~
\citet{Guerrieri2022OptimizingSynthesis} discuss the process of porting Post-Quantum Cryptographic algorithms to an FPGA using HLS. While it can be reasonably straightforward %easy 
to port an existing CPU implementation to an FPGA, %the 
performance can be low and resource utilization %of resources 
is not optimal. The authors discuss how, applying well-known HLS-specific optimization techniques, the code can be rewritten to leverage the capabilities of HLS tools and produce more efficient designs, reducing the computation latency of up to two orders of magnitudes in specific cases.


%\paragraph{CONT Optimizing Industrial Applications for Heterogeneous HPC Systems: The OPTIMA Project Intermediate stage}
\paragraph{Optimizations for Heterogeneous HPC Systems (OPTIMA)}
% Reviewer: Christiaan
% Contributions
% - Documentation of the two HPC systems, and the results of porting certain applications
% - It's a case study for specific algorithms on specific HPC systems
% - The applications that are ported seem to be open-source and actively maintained
% - Positioning the paper: while the paper reports results, it does not demonstrate how these results compare to SOTA
%Theodoropoulos et al.~
\citet{Theodoropoulos2023optima} demonstrate the results of porting and optimizing industrial applications to two new heterogeneous HPC systems within the OPTIMA project. The results highlight the performance increase of using the available FPGA-based accelerators versus a pure software implementation running on the CPUs of the HPC system.


%\paragraph{The VINEYARD Framework for Heterogeneous Cloud Applications: The BrainFrame Case}

\paragraph{A Framework for Heterogeneous Cloud Applications (VINEYARD)}
% Reviewer: Christiaan
% Contributions:
% - A framework for deploying different parts of an application across multiple accelerators in a cloud environement
% - Though they consider multiple use-cases, they only work out a specific use case: spiking neural networks
% - Supposedly there is an open-source marketplace, http://www.accel-store.com/, but it returns error 503
% Positioning the paper: while the paper reports results, it does not demonstrate how these results compare to SOTA
%In this paper, Sidiropoulos et al.~
\citet{Sidiropoulos2018vineyard} describe a framework to accelerate different parts of an application across different accelerators, like GPUs and FPGAs. They demonstrate the utility of this framework by creating a platform for computational neuroscience, called BrainFrame. The BrainFrame platform allows one to simulate spiking neural networks, and depending on the number of neurons and their interconnectivity, certain combinations of accelerators achieved the shortest simulation times.

%\paragraph{Tydi: an open specification for complex data structures over hardware streams}
\paragraph{Mapping data structures to hardware streams (Tydi)}
%Pelterberg et.al.~
\citet{Peltenburg2020Tydi:Streams} describe a specification for mapping complex, dynamically sized data structures onto a fixed number of hardware streams.
%SV: see also data centre big data analytics section for more about Tidy

\subsubsection*{\bf{Future directions}}
Traditional FPGA programming has been done using Hardware Description Languages, which have a steep learning curve that does not favor adopting reconfigurable devices in the scientific and industry community. To address this issue, there is a collective effort to increase the level of abstraction for FPGA designs without compromising performance. Achieving this goal requires a multidisciplinary approach involving programming languages, compilers, and optimization techniques. HLS tools and high-level approaches are being used in various application domains.
Although the current direction in this field in the Netherlands is unclear, our analysis has pinpointed specific domains of interest within local research communities, such as aerospace and accelerated big data processing, that could benefit from more accessible programming methods for FPGA devices.

    % \item in the Netherlands:
    % \begin{itemize}
    % \item no winner application domain, even though aerospace is often considered (due to ESA)
    % \item TUD with Tydi (and the related lab research) seems to be investing a lot in architectures/hardware for accelerating big data processing
    % %SV: work is continued and implemented in Voltron Data ?
    % \end{itemize}
%\end{itemize}


\subsection{Performance prediction}
\label{perf_pred}
% This section describes tool to improve and speed up the performance prediction (e.g., area and efficiency) of FPGA designs.

% \subsubsection{Background}
FPGA design and development processes are time-consuming activities due to, among others, the very fine granularity reconfigurability of FPGA designs, which translates into a large design space and long synthesis time. For this reason, it is crucial to enable quick performance prediction of synthesized programs to improve early-stage design analysis and exploration, and performance debugging.
We can distinguish between two main types of performance prediction models: analytical and ML-based. Analytical models (such as HLscope+~\cite{hlscope-2017} and COMBA~\cite{comba-2020}) analyze the source code and use mathematical modeling to estimate performance and resource utilization. They are able to produce quick estimates at the cost of reduced accuracy. ML-Based models~\cite{oneal-2018, ustun-2020, Sun-2021}, on the other hand, %instead 
aim at improving prediction accuracy by considering device-specific features, but typically %often 
require long and expensive training procedures. 


%\subsubsection{Current research in the Netherlands}

\subsubsection*{\bf{Research topics}} Several papers have been published by Dutch organizations focusing on performance prediction of synthesized codes.

%\paragraph{LEAPER: Fast and Accurate FPGA-based System Performance Prediction via Transfer learning}

\paragraph{System Performance Prediction via Transfer learning (LEAPER)}

% Reviewr: Christiaan
% Contributions:
%  - The contribution describes a framework.
%  - It is general purpose insofar that any application can make use of the new flow, it does however only support C/C++ based entry.
%   - Technology development
%   - The work is analytical
%Singh et.al.~
\citet{Singha2022leaper} describe a method for predicting system performance and resource usage of FPGA accelerators using transfer learning.
They trained a performance predictor model for an edge/embedded FPGA, and used transfer learning so that the model can also be used for cloud/high-end FPGAs.
The method allows for design space exploration of mapping C/C++ programs to cloud/high-end FPGAs using HLS. The authors showed that it is 10x faster than the state of the art, achieving 85\% accuracy.

\paragraph{Modeling FPGA-Based Systems via Few-Shot Learning}
% Reviewer Tiziano
% This is a poster, and very few details are provided in the abstract
Machine learning based models are being proposed to provide fast and accurate performance predictions of FPGA-based designs. However, training these models is expensive, due to the time-consuming FPGA design cycle. %In this poster, 
%Singh et al.~
\citet{Singh2021ModelingLearning} propose a transfer-learning-based approach for FPGA-based systems, to adapt an existing ML model, trained for a specific device, to a new, unknown environment, reducing the training costs.


%\paragraph{CGRA-EAM—Rapid Energy and Area Estimation for Coarse-grained Reconfigurable Architectures}
\paragraph{%CGRA-EAM—Rapid 
Energy and Area Estimation for Coarse-grained Reconfigurable Architectures}
% Reviewer: Christiaan
% Contributions:
% - Analytical model for power and area of CGRAs, and the method to create said model
% - While the method to create the model is only applied to one CGRA architecture, the paper claims it should work for many different CGRAs as long as the RTL is available/can be generated.
% - It does not seem the code artifacts or models can be easily downloaded anywhere
% - Positioning the paper: This work is analytical vs a machine-learning approach, focus on CGRA (instead of FPGA or ASIC). 
Design space exploration is often required to achieve good Pareto points when creating reconfigurable architectures. %Wijtvliet et. al.~
\citet{Wijtvliet2021cgra} introduce the CGRA-EAM  model for energy and area estimation for CGRAs. It %which 
achieves a 15.5\% error for energy and 2.1\% error for area estimation for the Blocks~\cite{Wijtvliet2019Blocks:Efficiency} CGRA. %The novelty of the work lies on the focus on CGRAs and the that it works over multiple different application running on an CGRA.  
The novelty of this work lies in its focus on CGRAs and its ability to handle multiple different applications running on a CGRA.

\paragraph{Analytical Performance Estimation for Large-Scale Reconfigurable Dataflow Platforms}
% Reviewer: Christiaan & Steven

% Contributions:
% - framework for analytical model + statistics 
% - automatic approach for predicting performance on reconfigurable dataflow platforms
% - The novelty of the work lies in the fact that it is applicable to performance estimation for large-scale workloads on heterogeneous systems
% Keio University, Japan; Imperial College London, United Kingdom; University of Amsterdam, The Netherlands; Maxeler Technologies Ltd., United Kingdom
%The work from Yasudo et al., introduced in 
\citet{Yasudo2018PerformancePlatforms, Yasudo2021AnalyticalPlatforms} %and \citet{Yasudo2021perf} 
introduced and further expanded %in ~\cite{Yasudo2021perf}, proposes 
\emph{PERKS}, a performance estimation framework for reconfigurable dataflow platforms. The authors propose %In the work it is proposed 
that reconfigurable accelerators, such as FPGAs, will play an important role in future exascale computing platforms and that such a framework is essential in the efficient deployment of applications on heterogenous platforms with reconfigurable accelerators. The PERKS framework uses parameters from the target platform and the application to build an analytical model to predict the performance of multi-accelerator systems. Experimental results with different reconfigurable dataflow applications are presented, showing that the framework can predict the performance of current workloads with high accuracy.

%propose the PERKS performance estimation framework for reconfigurable dataflow programs in order to assess the feasibility of large-scale heterogeneous systems. Experimental show an above 91\% accuracy for execution time of five different applications for two reconfigurable dataflow processing platforms. It is an analytical model that adops profiling and statistical methods for calibration and improving accuracy. The novelty of the work lies in the fact that it is applicable to performance estimation for large-scale workloads on heterogeneous systems.


\paragraph{Memory and Communication Profiling for Accelerator-Based Platforms}
% Reviewer: Steven
% Contributions:
% - MCPROF profiling tool, open-source (last activity 7 years ago?)
% - Improvements compared to existing work include: faster, provinding better insights
% - Case study of image processing applications for FPGA and GPU, allowing to get insigt if application fits well to the architecture
% TU Delt + QuTech
% They descrive future work on "relating the datacommunication information generated by MCPROF with performance estimates generated by the profiling tools provided by Xilinx and Nvidia" and "utilization of the currently generated information by MCPROF to automatically generate SDSoC and OpenACC [53] pragmas for FPGA- and GPUbased accelerators, respectively".
% Looking at the publications from the main author of this work, it might have ended up in Quantum programming tooling? https://scholar.google.nl/citations?user=N_tCHwkAAAAJ&hl=nl
%The work by Ashraf et al. 
\citet{Ashraf2018MemoryPlatforms} present \emph{MCPROF}, an open-source memory-access and data-communication profiler. The tool provides a detailed profile of memory-access behavior for heterogeneous systems (CPU, GPU, and FPGA) for C/C++ applications, as well as data-communication-aware mapping of applications on these architectures. Comparison with the state of the art show that the proposed profiler has an order of magnitude, on average, lower overhead than state-of-the-art data-communication profilers over a wide range of benchmarks. A case study  with several image processing applications for heterogeneous multi-core platforms containing an FPGA and a GPU as accelerators was conducted. The authors also demonstrate that the tool can provide insights into whether %or not 
a specific %certain 
accelerator (GPU or FPGA) is a good fit 
for the application.

%\paragraph{Performance Estimation for Exascale Reconfigurable Dataflow Platforms}
% combined with: Analytical Performance Estimation for Large-Scale Reconfigurable Dataflow Platforms
% Reviewer: Steven
% Contributions:
% - 
% Keio University, Japan; Imperial College London, United Kingdom; University of Amsterdam, The Netherlands; Maxeler Technologies Ltd., United Kingdom
%
% CB: "Analytical Performance Estimation for Large-Scale Reconfigurable Dataflow Platforms" is _also_ introduces PERKS, is newer, and at the end of the introduction section it says:
% "This article extends the brief description of PERKS from Yasudo et al. [44] as follows. Section 2
% on related work is new. Section 3 covers a simpler model than the one described in the work of
% Yasudo et al. [44]. Section 4, which describes the automated method for performance estimation, is
% new. Section 5 includes further material on our experimental setup and on verifying the proposed
% model. Section 6 contains additional large-scale workload scenarios to illustrate our approach.
% Finally, Section 7 highlights the main take-home messages of this work."
% 
% CB: So perhaps these two works could be combined into one paragraph?

%This work by Yasudo et al. \cite{Yasudo2018PerformancePlatformsb} proposes \emph{PERKS}, a  performance estimation framework for reconfigurable dataflow platforms. In the work it is proposed that reconfigurable accelerators (such as FPGA) will play an important role in future exascale computing platforms and that such a framework is essential in the efficient deployment of applications on heterogenous platforms with reconfigurable accelerators. The PERKS framework uses parameters from the target platform and the application to build an analytical model to predict the performance of multi-accelerator systems. Experimental results with different reconfigurable dataflow applications are presented, showing that the framework can predict the performance of current workloads at high accuracy. 

\paragraph{Delay Prediction for ASIC HLS}%: Comparing Graph-Based and Nongraph-Based Learning Models}
% Reviewer: Christiaan
% Contributions
% - Thorough analysis of both graph and nongraph-based learning models for delay estimation
% - It's a general method tested against multiple applications including data and control oriented designs
% - It does not seem the code artifacts or models can be easily downloaded
% - Positioning the paper: existing work focuses mostly on FPGA, whilst the focus of this work is more on ASIC.
The delay estimates of HLS tools can often deviate significantly from results obtained from logic synthesis. %De et. al.~
\citet{De2023hls} propose %using 
a hybrid model by incorporating graph-based learning models, which can infer structural features from a design, into traditional non-graph-based learning models for delay estimates. The hybrid model improves delay prediction by 93\% in comparison to the delay prediction reported %given 
by a %the 
commercial HLS tool. 

\subsubsection*{\bf{Future directions}}
% \begin{itemize}
%     \item performance prediction for FPGA and, more in general, reconfigurable dataflow devices, is a challenging task. Traditionally, this was addressed by using analytical approaches, but more recently machine learning based approaches are becoming mainstream, and we can expect them to become a popular option
%     \item with CGRA-like architecture being commercialized, the interest in exploring these devices is increasing
%     \item we advocate the need for open-source tools (or at least reproducibility): performance prediction is a device-specific task, but very few of the papers reviewed release the code used. If we want to increase the chance of collaborations, and reproducibility, this has to change.
% \end{itemize}

Performance prediction for FPGAs and, more generally, reconfigurable dataflow devices is challenging. Traditionally, this was addressed with %by using 
analytical approaches, but more recently, machine-learning-based approaches are becoming mainstream, and we can expect them to become a popular option given their successes in recent research. With CGRA-like architectures being commercialized, the interest in exploring these devices is increasing, and we %should 
expect future work to focus more on the predictability of the 
performance of such devices. Performance prediction is a device-specific task, yet only a %fraction %very 
few 
of the reviewed papers favor the reproducibility of the presented results, or publicly release the % share the 
code used to generate these results. 
We should shift toward more reproducible and transparent research to foster %increase the chance of 
collaborations and facilitate general progress. Thus, we advocate for the need for open-source practices %also 
in performance prediction as well.
% Describe the future direction that this research, in the Netherlands, is likely to head into. This section should describe what ongoing research seems most promising and what we might expect in future research on this topic. This paragraph can also expand on potential applications that these developments can be applied to, giving insight into why this is important.
% - it is quite disjoint, so in case we can mention what's the trend in the world
% - it is also fine to say something about who is leading (e.g. work X was led by this dutch institution) or who contributed (in the last 5 years)


\section{Robustness of FPGAs}
\label{sec:robustness}
\begin{table}
    \vspace{-0.5cm}
    \centering
    \caption{Watermark Verification AUC under each image perturbation. Cr. \& Dr. refers to random crop and random drop.}
    \label{table:robustness}
    % \vspace{-.25cm}
    \resizebox{\columnwidth}{!}{%  
    \begin{tabular}{@{}cccccccc@{}}
    \toprule
    \textbf{Methods} & \textbf{JPEG} & \textbf{Cr. \& Dr.} & \textbf{Resize} & \textbf{GauBlur} & \textbf{MedFilter} & \textbf{Brightness} & \textbf{Avg} \\ \midrule
    Tree-Ring & 0.987 & 0.993 & 0.992 & 0.985 & 0.988 & 0.991 & 0.990 \\
    DiffuseTrace & 0.962 & 0.993 & 0.985 & 0.966 & 0.969 & 0.922 & 0.968 \\
    Gaussian Shading & 0.999 & 1.000 & 1.000 & 1.000 & 1.000 & 0.999 & 0.999 \\
    $\text{G-S}_{ChaCha20}$ & 0.999 & 1.000 & 1.000 & 1.000 & 1.000 & 0.999 & 0.999 \\ \midrule
    \tool(T-R) & 0.952 & 0.955 & 0.983 & 0.951 & 0.969 & 0.946 & 0.957 \\
    \tool(D-T) & 0.939 & 0.974 & 0.977 & 0.939 & 0.950 & 0.965 & 0.959 \\
    \tool(G-S) & 0.965 & 0.982 & 0.988 & 0.973 & 0.978 & 0.937 & 0.972 \\ \bottomrule
    \end{tabular}
    }
% \vspace{-.25cm}
    \vspace{-0.5cm}
\end{table}


%\newpage
\section{Applications}
\label{sec:applications}
\section{Applications.}

\begin{figure*}
\includegraphics[width=\linewidth]{fig_supp_dof_v2.pdf}
\caption{Depth of field render using 100\% shading rate (a) and using a combination of 100\% and 12.5\% shading rate (b,c). In (b,c) triangles falling partly inside the focal range are shaded into a 100\% texture atlas; triangles falling outside of the focal range are shaded into a second atlas at 12.5\% shading rate. Using \cite{Neff2022MSA} with the same settings (b) introduces artifacts such as the seams on the awning and the coarse shading on the balcony and menu due to undersampling (highlighted in red). Using FastAtlas (c), the rendered outputs are nearly identical.} 
\label{fig:depthOfField}
\end{figure*}


\begin{figure*}
\includegraphics[width=\linewidth]{fig_supp_fov_v2.pdf}
\caption{Foveated rendering using a 100\% texture atlas (a) and using a combination of 100\% and 25\% shading rate atlases (b,c). In (b,c) triangles falling inside the foveated region are rendered into a high resolution texture atlas; triangles falling outside of the shaded region are rendered into a low-resolution texture atlas and then blurred. (b) Using \cite{Neff2022MSA} with the same setting, the area outside the foveated region has undersampling artifacts on areas such as the wall boundary and shadows (inset). Using FastAtlas (c), we are able to obtain high-quality foveated rendering suitable for VR and AR displays with eye tracking.}
\label{fig:foveated_rendering}
\end{figure*}


\label{sec:applications}
A key advantage of texture-space shading is that shading rate can be decoupled from screen space resolution. This decoupling means that one can generate multiple shading atlases for the same scene and use them to shade different content at different shading resolutions. In particular, one can shade a portion of a scene using a high-resolution atlas, and another portion using a low resolution one. In these scenarios, only a portion of the high-resolution atlas is shaded, significantly reducing computational costs.
Applications that can benefit from this technique are ones where only a portion of a scene requires high quality shading, whereas the rest of the 
scene can be rendered with low resolution or blurry shading. We evaluate the applicability of our method to such scenarios by applying it to two representative examples: depth of field and foveated rendering. 
 
\paragraph*{Depth-of-Field.} 
We modify the depth-of-field method of Bukowski et al. \shortcite{Bukowski2013DepthOfField} to utilize two texture atlases. For the first atlas, we set the target scale to 100\% and use an $8K \times 8K$ atlas that provided enough space to fit all charts at this scale. Our second atlas is allocated to be  $1K \times 1K$ (1/64th the number of original texels). We generate a single packing by computing charts and atlases for the entire visible scene for the $8k$ atlas. 
The packing for the low resolution atlas is generated by simply scaling the coordinates of all charts by factor 0.125 along both axes. When packing the 8K atlas we allocate a gutter size sufficient to ensure that charts continue to not touch if the $8k$ packing is scaled down to fit the $1k$ atlas. We then use this packing to shade both atlases. We shade the entire low-resolution atlas, and only shade a triangle in the $8k$ atlas if it is contained or partly contained within the focal field. We then read from the high-resolution atlas when rasterizing areas in the focal field or its transition region, and use the low resolution one otherwise. The net result is that any shading values outside of the focal field are shaded at a reduced one sixty-fourth shading rate. As  Fig.~\ref{fig:depthOfField}c demonstrates, using our atlases achieves compelling depth-of-field effects while requiring one sixty-fourth the amount of shading for textures outside of the focal field.
We used the exact same setup for \cite{Neff2022MSA} using 8K and 1K atlases. See accompanying video and Fig.~\ref{fig:depthOfField} for a side by side comparison.  

\paragraph*{Foveated Rendering.} We use the same technique for foveated rendering. We generate an $8K$ atlas packing, and then scale it down to create an identical $2K$ atlas packing. We then shade a low-resolution $2K$ atlas for the entire scene, and a matching high-resolution $8K$ atlas with the same packing only for those triangles contained inside the fovea region. The low resolution view is then blurred using a two-pass Gaussian blur and composited with a high-resolution rendering of the foveal region to produce the final image (Fig. \ref{fig:foveated_rendering}c). By providing a fast, robust texture-space shading method, our seamless texture atlases make foveated rendering practical for modern virtual reality systems with eye tracking. We used the exact same setup for \cite{Neff2022MSA} (Fig.~\ref{fig:foveated_rendering}b).

The quality of renders generated using a mix of high and low resolution atlases is clearly contingent on the quality of the shading generated using the low-resolution atlases. Patney et al.~\shortcite{patney2016towards} note that static TSS methods do not provide sufficient quality low resolution results to enable this approach for foveated rendering.
This observation is aligned with our measurements (Tab.~\ref{tab:supp_flip_fixed_atlas}, Tab.~\ref{tab:supp_flip_fixed_sr}) and figures which show that these methods fail to produce adequate quality shading at low resolutions. Shading using the method of \cite{Neff2022MSA} at these resolutions produces undesirable undersampling and highly noticeable visible seams at meshlet boundaries (Figs. \ref{fig:depthOfField}b and \ref{fig:foveated_rendering}b.)  FastAtlas generates atlases of sufficient quality at low resolutions for the needs of these applications.


\section{Research and Development in Industry}
\label{sec:industry}
\begin{table}[t]
\centering
\setlength{\abovecaptionskip}{0.05cm}
\setlength{\belowcaptionskip}{0.2cm}
\caption{Performance on industrial dataset}
\setlength{\tabcolsep}{2mm}{
\resizebox{0.75\textwidth}{!}{
\begin{tabular}{c|c|c|c|c|c}
\toprule
    \textbf{Metric} & \textbf{PRAUC}$\uparrow$ & \textbf{PCOC}$\uparrow$ & \textbf{AUC}$\uparrow$ & \textbf{LogLoss}$\downarrow$ & \textbf{Params}$\downarrow$ \\ \midrule
    \textbf{baseline} & 0.91913 & 0.96561 & 0.84141 & 0.40721 & 3142.3MB \\
    \textbf{MEC} & \textbf{0.91918} & \textbf{0.96673} & \textbf{0.84143} & \textbf{0.40683} & \textbf{7.014MB} \\
\bottomrule
\end{tabular}
}}
\label{tab:industry}
% \vspace{-10pt}
\end{table}


%\section{Discussion}
%This work identifies signal collapse as a critical bottleneck in one-shot neural network pruning. Performance loss in pruned networks is due to \textbf{signal collapse} in addition to the removal of critical parameters. We propose \textbf{REFLOW} (\textbf{Re}storing \textbf{F}low of \textbf{Low}-variance signals), a simple yet effective method that mitigates signal collapse without computationally expensive weight updates. By focusing on signal preservation, REFLOW highlights the importance of mitigating signal collapse in sparse networks and enables magnitude pruning to match or surpass state-of-the-art one-shot pruning methods such as CHITA, CBS, and WF.

REFLOW consistently achieves state-of-the-art accuracy across diverse architectures, restoring ResNeXt-101 from under 4.1\% to 78.9\% top-1 accuracy at 80\% sparsity on ImageNet. Its lightweight design makes it a practical solution for both research and deployment, delivering high-quality sparse models without the overhead of traditional approaches. These findings challenge the traditional emphasis on weight selection strategies and underscore the critical role of signal propagation for achieving high-quality sparse networks in the context of one-shot pruning.




%\newpage
\section{Conclusions}
This paper presents an overview of the current research landscape, applications, and future potential of FPGA technology in the Netherlands. It highlights how academia and industry in the Netherlands play an important role in developing new and innovative FPGA-based technologies and solutions to address various important societal challenges ranging from healthcare to power efficiency. 

\subsection{Summary and main findings}
We selected a total of %234
212 relevant FPGA-related papers published in the past 5 years. Most contributions to the national FPGA research effort stem from 16 organizations, including major Dutch universities, research institutes, and % as well as 
industry. Many of these publications are collaborations of national and international partners, mostly European. We note that the survey covers about 1\% of the relevant world-wide FPGA publications (from the same period), with the Netherlands ranking 22nd in the world and 10th in Europe in relevant FPGA-research output. 

We classified the selected papers into five major themes: a) FPGA architecture, with 11 published papers, b) data center infrastructure \& HPC, with 40 papers, c) programming models \& tools, with 15 papers, d) robustness of FPGAs, with 26 papers, and e) applications, with 120 papers. We paid specific attention and reviewed in depth popular FPGA applications, i.e, those applications with a significant number of relevant publications where FPGAs play an important role. We found 49 application papers over 6 subjects; these publications indicate that FPGAs have emerged as powerful accelerators for a wide range of applications, such as machine learning, astronomy, particle physics experiments, quantum computing, space applications, and bioinformatics. 

Our survey revealed ample future directions across all themes. In terms of architecture (see Section ~\ref{sec:archi}), specialized FPGA architectures (e.g., for in-memory computing) as well as embedding FPGA technology as part of (complex) computing systems beyond the computation (e.g., for on-chip and off-chip communication or for memory systems) are  promising paths towards reducing the memory gap of traditional systems. For datacenters and HPC infrastructure (see Section~\ref{sec:HPC}), we identify a promising research direction in FPGA-based acceleration in general, and tooling to enable its seamless integration into large-scale systems in particular; a second research direction should focus on supporting novel communication protocols and technologies, to further improve data movement in supercomputers, datacenters, and the computing continuum. For programming models and tools (see Section~\ref{sec:programming}), the main research developments target high-level tooling for easier development and deployment, accessible for more educated users than HDL experts, and tooling support for performance analysis, modeling, and prediction, which will help solution feasibility analysis and fair comparison against competing technologies. For robustness of FPGAs (see Section~\ref{sec:robustness}), reliability and security are key aspects where more research is needed in architecture and tooling for moving from proof-of-concept to complex (eco)systems, where FPGAs can guarantee such essential non-functional requirements in the most adverse conditions (e.g., for deployment in space). Finally, for applications, future research should focus on the adoption of FPGAs in more application domains where energy-efficient acceleration is essential; we suggest further research towards more efficient developments in AI (for both training acceleration and latency-sensitive deployment), as well as for various computing continuum layers, fast/specialized communication, and HPC kernels. We further expect more developments towards successful acceleration in simulation and digital twin applications, where FPGAs could play an interesting role in a simulation-emulation continuum.     

\subsection{Limitations}
This work identifies signal collapse as a critical bottleneck in one-shot neural network pruning. Performance loss in pruned networks is due to \textbf{signal collapse} in addition to the removal of critical parameters. We propose \textbf{REFLOW} (\textbf{Re}storing \textbf{F}low of \textbf{Low}-variance signals), a simple yet effective method that mitigates signal collapse without computationally expensive weight updates. By focusing on signal preservation, REFLOW highlights the importance of mitigating signal collapse in sparse networks and enables magnitude pruning to match or surpass state-of-the-art one-shot pruning methods such as CHITA, CBS, and WF.

REFLOW consistently achieves state-of-the-art accuracy across diverse architectures, restoring ResNeXt-101 from under 4.1\% to 78.9\% top-1 accuracy at 80\% sparsity on ImageNet. Its lightweight design makes it a practical solution for both research and deployment, delivering high-quality sparse models without the overhead of traditional approaches. These findings challenge the traditional emphasis on weight selection strategies and underscore the critical role of signal propagation for achieving high-quality sparse networks in the context of one-shot pruning.




\subsection{Future Work}
   
Our detailed analysis of the last half-decade of Dutch FPGA research does already emphasize that, while the technology holds tremendous potential, there are ongoing challenges, particularly in terms of development tools, performance predictability, and hardware security. Future research will need to focus on improving these aspects, especially with regards to user-friendly programming models and more robust performance estimation frameworks. The continued investment in FPGA technology will be vital for maintaining the Netherlands' competitive edge and addressing the growing demand for energy-efficient, high-performance computing solutions. Furthermore, fostering open-source tools, as well as deeper industry-academia collaboration, will be essential for sustaining long-term growth and innovation in this rapidly evolving field.

To further enlarge the scope of this analysis, we welcome a broader community-driven research, to survey the developments and needs of FPGA research and innovation at European or even world-wide level. Such analysis will reveal additional collaboration opportunities, cross-theme and cross-domain, across multiple layers of modern computing systems. While we recognize the scale and scope of such an effort are significantly larger, we argue that it is an effective way to map the current landscape of FPGA research, and further focus on relevant research challenges and opportunities.  
%\section{Conclusion Remarks}
This work proposes a RBG graph model for disease spreading via hubs. We study the joint effect of the agent density, hub density, and connection function. The existence of a critical hub density depends only on the boundedness of the support of the connection function, which relates to curbing the traveling distance of individuals. When it comes to dispersion, both the degree distribution and the percolation threshold suggest that increasing dispersion helps spread the disease. The percolation properties of RBG graphs relate to unipartite graphs with modified connection functions. 
An interesting question in this direction is if and when the properties of the RBG graphs can be well represented by unipartite graphs with some modified connection functions. Our conjecture is that for independent connections between different pairs of agents, such representation is unlikely due to the oblivion of the local dependence (present in the RBG models). 
 Another direction is to consider hybrid models where agents may get infected either through common hubs or direct interactions between agents. The former infection mechanism is more centralized than the latter. 


%\newpage
\bibliographystyle{ACM-Reference-Format-num}
\bibliography{references,mendeley}


\end{document}
