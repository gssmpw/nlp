%%
%% This is file `sample-manuscript.tex',
%% generated with the docstrip utility.
%%
%% The original source files were:
%%
%% samples.dtx  (with options: `manuscript')
%% 
%% IMPORTANT NOTICE:
%% 
%% For the copyright see the source file.
%% 
%% Any modified versions of this file must be renamed
%% with new filenames distinct from sample-manuscript.tex.
%% 
%% For distribution of the original source see the terms
%% for copying and modification in the file samples.dtx.
%% 
%% This generated file may be distributed as long as the
%% original source files, as listed above, are part of the
%% same distribution. (The sources need not necessarily be
%% in the same archive or directory.)
%%
%% Commands for TeXCount
%TC:macro \cite [option:text,text]
%TC:macro \citep [option:text,text]
%TC:macro \citet [option:text,text]
%TC:envir table 0 1
%TC:envir table* 0 1
%TC:envir tabular [ignore] word
%TC:envir displaymath 0 word
%TC:envir math 0 word
%TC:envir comment 0 0
%%
%%
%% The first command in your LaTeX source must be the \documentclass command.
\documentclass[acmsmall,screen,review=false,authorversion,nonacm]{acmart}

\colorlet{RED}{red}
\newcommand{\zaidnote}[1]{ {\textcolor{red} { ***Zaid: #1 }}}


\title[FPGA Innovation Research in the Netherlands]{FPGA Innovation Research in the Netherlands: \\Present Landscape and Future Outlook}

%%
%% \BibTeX command to typeset BibTeX logo in the docs
\AtBeginDocument{%
  \providecommand\BibTeX{{%
    \normalfont B\kern-0.5em{\scshape i\kern-0.25em b}\kern-0.8em\TeX}}}

%% Rights management information.  This information is sent to you
%% when you complete the rights form.  These commands have SAMPLE
%% values in them; it is your responsibility as an author to replace
%% the commands and values with those provided to you when you
%% complete the rights form.
\setcopyright{none} %{acmlicensed}
\makeatletter
\let\@authorsaddresses\@empty
\makeatother
% \copyrightyear{2018}
% \acmYear{2018}
% \acmDOI{XXXXXXX.XXXXXXX}

%% These commands are for a PROCEEDINGS abstract or paper.
% \acmConference[Conference acronym 'XX]{Make sure to enter the correct
%   conference title from your rights confirmation emai}{June 03--05,
%   2018}{Woodstock, NY}
% \acmISBN{978-1-4503-XXXX-X/18/06}


%%
%% Submission ID.
%% Use this when submitting an article to a sponsored event. You'll
%% receive a unique submission ID from the organizers
%% of the event, and this ID should be used as the parameter to this command.
%%\acmSubmissionID{123-A56-BU3}

%%
%% For managing citations, it is recommended to use bibliography
%% files in BibTeX format.
%%
%% You can then either use BibTeX with the ACM-Reference-Format style,
%% or BibLaTeX with the acmnumeric or acmauthoryear sytles, that include
%% support for advanced citation of software artefact from the
%% biblatex-software package, also separately available on CTAN.
%%
%% Look at the sample-*-biblatex.tex files for templates showcasing
%% the biblatex styles.
%%

%%
%% The majority of ACM publications use numbered citations and
%% references.  The command \citestyle{authoryear} switches to the
%% "author year" style.
%%
%% If you are preparing content for an event
%% sponsored by ACM SIGGRAPH, you must use the "author year" style of
%% citations and references.
%% Uncommenting
%% the next command will enable that style.
%%\citestyle{acmauthoryear}
%%
%% end of the preamble, start of the body of the document source.
\begin{document}

%%
%% The "title" command has an optional parameter,
%% allowing the author to define a "short title" to be used in page headers.
%or

%%
%% The "author" command and its associated commands are used to define
%% the authors and their affiliations.
%% Of note is the shared affiliation of the first two authors, and the
%% "authornote" and "authornotemark" commands
%% used to denote shared contribution to the research.
\author{Nikolaos Alachiotis}
\email{n.alachiotis@utwente.nl}
\orcid{0000-0001-8162-379}
\affiliation{%
  \institution{University of Twente}
  \streetaddress{}
  \city{}
  \state{}
  \country{The Netherlands}
  \postcode{}
}

\author{Sjoerd van den Belt}
\email{s.p.vandenbelt@utwente.nl}
\orcid{0009-0005-8812-9929}
\affiliation{%
  \institution{University of Twente}
  \streetaddress{}
  \city{}
  \state{}
  \country{the Netherlands}
  \postcode{}
}

\author{Steven van der Vlugt}
\email{vlugt@astron.nl}
\orcid{0000-0001-6834-4860}
\affiliation{%
  \institution{Netherlands Institute for Radio Astronomy (ASTRON)}
  \streetaddress{Oude Hoogeveensedijk 4}
  \city{Dwingeloo}
  \state{Drenthe}
  \country{the Netherlands}
  \postcode{7990 AA}
}

\author{Reinier van der Walle}
\email{walle@astron.nl}
\orcid{0009-0005-8097-6170}
\affiliation{%
  \institution{Netherlands Institute for Radio Astronomy (ASTRON)}
  \streetaddress{Oude Hoogeveensedijk 4}
  \city{Dwingeloo}
  \state{Drenthe}
  \country{the Netherlands}
  \postcode{7990 AA}
}

\author{Mohsen Safari}
\email{mohsen.safari@surf.nl}
\orcid{0000-0003-0839-3251}
\affiliation{%
  \institution{SURF}
  \streetaddress{}
  \city{}
  \state{}
  \country{the Netherlands}
  \postcode{}
}

\author{Bruno Endres Forlin}
\email{b.endresforlin@utwente.nl}
\orcid{0000-0003-4822-1841}
\affiliation{%
  \institution{University of Twente}
  \streetaddress{}
  \city{Enschede}
  \state{}
  \country{the Netherlands}
  \postcode{}
}

\author{Tiziano De Matteis}
\email{t.de.matteis@vu.nl}
\orcid{0000-0002-9158-6849}
\affiliation{%
  \institution{Vrije Universiteit Amsterdam}
  \streetaddress{}
  \city{Amsterdam}
  \state{}
  \country{The Netherlands}
  \postcode{}
}

\author{Zaid Al-Ars}
\email{z.al-ars@tudelft.nl}
\orcid{0000-0001-7670-8572}
\affiliation{%
  \institution{Delft University of Technology}
  \streetaddress{}
  \city{Delft}
  \state{}
  \country{the Netherlands}
  \postcode{}
}

\author{Roel Jordans}
  \email{r.jordans@tue.nl}
  \affiliation{
  \institution{Eindhoven University of Technology}
  \streetaddress{Den Dolech 2}
  \city{Eindhoven}
  \country{the Netherlands}
}

\additionalaffiliation{
  \institution{Radboud Radiolab, Institute for Mathemathics, Astrophysics and Particle Physics (IMAPP), Radboud University}
  \streetaddress{Heyendaalseweg 135}
  \city{Nijmegen}
  \country{the Netherlands}
}

\author{Ant\'onio J. Sousa de Almeida}
\email{ajsousal@gmail.com}
\orcid{0000-0002-7024-2262}
\affiliation{%
  \institution{University of Twente}
  \streetaddress{}
  \city{}
  \state{}
  \country{the Netherlands}
  \postcode{}
}

\author{Federico Corradi}
\email{f.corradi@tue.nl}
\orcid{0000-0002-5868-8077}
\affiliation{%
  \institution{Eindhoven University of Technology}
  \streetaddress{}
  \city{}
  \state{}
  \country{the Netherlands}
  \postcode{}
}

\author{Christiaan Baaij}
\email{christiaan@qbaylogic.com}
\affiliation{%
  \institution{QBayLogic B.V.}
  \streetaddress{}
  \city{}
  \state{}
  \country{the Netherlands}
  \postcode{}
}

\author{Ana-Lucia Varbanescu}
\email{a.l.varbanescu@utwente.nl}
\orcid{0000-0002-4932-1900}
\affiliation{%
  \institution{University of Twente}
  \streetaddress{}
  \city{}
  \state{}
  \country{The Netherlands}
  \postcode{}
}

%\author{\textcolor{red}{\textbf{?????!!!!!!???ADD YOUR NAME ????!!!!? ???!!}}}
%\email{}
%\affiliation{%
%  \institution{}
%  \streetaddress{}
%  \city{}
%  \state{}
%  \country{the Netherlands}
%  \postcode{}
%}

%%
%% By default, the full list of authors will be used in the page
%% headers. Often, this list is too long, and will overlap
%% other information printed in the page headers. This command allows
%% the author to define a more concise list
%% of authors' names for this purpose.
\renewcommand{\shortauthors}{Alachiotis et al.}

%%
%% The abstract is a short summary of the work to be presented in the
%% article.
\begin{abstract}

FPGAs have transformed digital design by enabling versatile and customizable solutions that balance performance and power efficiency, yielding them essential for today’s diverse computing challenges.
Research in the Netherlands, both in academia and industry, plays a major role in developing new innovative FPGA solutions. This survey presents the current landscape of FPGA
innovation research in the Netherlands by delving into ongoing projects, advancements, and
breakthroughs in the field. Focusing on recent research outcome (within the past 5 years), we have identified five key research areas: a) FPGA architecture, b) FPGA robustness, c) data center infrastructure and high-performance computing, d) programming models and tools, and e) applications. 
This survey provides in-depth insights beyond a mere snapshot of the current innovation research landscape by highlighting future research directions within each key area; these insights can serve as a foundational resource to inform potential national-level investments in FPGA technology.

\end{abstract}

%%
%% The code below is generated by the tool at http://dl.acm.org/ccs.cfm.
%% Please copy and paste the code instead of the example below.
%%

% \begin{CCSXML}
% <ccs2012>
%    <concept>
%        <concept_id>10002944.10011122.10002945</concept_id>
%        <concept_desc>General and reference~Surveys and overviews</concept_desc>
%        <concept_significance>500</concept_significance>
%        </concept>
%    <concept>
%        <concept_id>10010520.10010521.10010542.10010543</concept_id>
%        <concept_desc>Computer systems organization~Reconfigurable computing</concept_desc>
%        <concept_significance>500</concept_significance>
%        </concept>
%  </ccs2012>
% \end{CCSXML}

% \ccsdesc[500]{General and reference~Surveys and overviews}
% \ccsdesc[500]{Computer systems organization~Reconfigurable computing}





%%
%% Keywords. The author(s) should pick words that accurately describe
%% the work being presented. Separate the keywords with commas.
% \keywords{Field Programmable Gate Array, Architecture, Data center, High Performance Computing, Robustness, Applications}

% \received{18 November 2024}
% \received[revised]{12 March 2009}
% \received[accepted]{5 June 2009}

%%
%% This command processes the author and affiliation and title
%% information and builds the first part of the formatted document.
\maketitle

% remove when finished
%\newpage
%\tableofcontents
%\newpage




\section{Introduction}
\section{Introduction}
\label{sec:introduction}
The business processes of organizations are experiencing ever-increasing complexity due to the large amount of data, high number of users, and high-tech devices involved \cite{martin2021pmopportunitieschallenges, beerepoot2023biggestbpmproblems}. This complexity may cause business processes to deviate from normal control flow due to unforeseen and disruptive anomalies \cite{adams2023proceddsriftdetection}. These control-flow anomalies manifest as unknown, skipped, and wrongly-ordered activities in the traces of event logs monitored from the execution of business processes \cite{ko2023adsystematicreview}. For the sake of clarity, let us consider an illustrative example of such anomalies. Figure \ref{FP_ANOMALIES} shows a so-called event log footprint, which captures the control flow relations of four activities of a hypothetical event log. In particular, this footprint captures the control-flow relations between activities \texttt{a}, \texttt{b}, \texttt{c} and \texttt{d}. These are the causal ($\rightarrow$) relation, concurrent ($\parallel$) relation, and other ($\#$) relations such as exclusivity or non-local dependency \cite{aalst2022pmhandbook}. In addition, on the right are six traces, of which five exhibit skipped, wrongly-ordered and unknown control-flow anomalies. For example, $\langle$\texttt{a b d}$\rangle$ has a skipped activity, which is \texttt{c}. Because of this skipped activity, the control-flow relation \texttt{b}$\,\#\,$\texttt{d} is violated, since \texttt{d} directly follows \texttt{b} in the anomalous trace.
\begin{figure}[!t]
\centering
\includegraphics[width=0.9\columnwidth]{images/FP_ANOMALIES.png}
\caption{An example event log footprint with six traces, of which five exhibit control-flow anomalies.}
\label{FP_ANOMALIES}
\end{figure}

\subsection{Control-flow anomaly detection}
Control-flow anomaly detection techniques aim to characterize the normal control flow from event logs and verify whether these deviations occur in new event logs \cite{ko2023adsystematicreview}. To develop control-flow anomaly detection techniques, \revision{process mining} has seen widespread adoption owing to process discovery and \revision{conformance checking}. On the one hand, process discovery is a set of algorithms that encode control-flow relations as a set of model elements and constraints according to a given modeling formalism \cite{aalst2022pmhandbook}; hereafter, we refer to the Petri net, a widespread modeling formalism. On the other hand, \revision{conformance checking} is an explainable set of algorithms that allows linking any deviations with the reference Petri net and providing the fitness measure, namely a measure of how much the Petri net fits the new event log \cite{aalst2022pmhandbook}. Many control-flow anomaly detection techniques based on \revision{conformance checking} (hereafter, \revision{conformance checking}-based techniques) use the fitness measure to determine whether an event log is anomalous \cite{bezerra2009pmad, bezerra2013adlogspais, myers2018icsadpm, pecchia2020applicationfailuresanalysispm}. 

The scientific literature also includes many \revision{conformance checking}-independent techniques for control-flow anomaly detection that combine specific types of trace encodings with machine/deep learning \cite{ko2023adsystematicreview, tavares2023pmtraceencoding}. Whereas these techniques are very effective, their explainability is challenging due to both the type of trace encoding employed and the machine/deep learning model used \cite{rawal2022trustworthyaiadvances,li2023explainablead}. Hence, in the following, we focus on the shortcomings of \revision{conformance checking}-based techniques to investigate whether it is possible to support the development of competitive control-flow anomaly detection techniques while maintaining the explainable nature of \revision{conformance checking}.
\begin{figure}[!t]
\centering
\includegraphics[width=\columnwidth]{images/HIGH_LEVEL_VIEW.png}
\caption{A high-level view of the proposed framework for combining \revision{process mining}-based feature extraction with dimensionality reduction for control-flow anomaly detection.}
\label{HIGH_LEVEL_VIEW}
\end{figure}

\subsection{Shortcomings of \revision{conformance checking}-based techniques}
Unfortunately, the detection effectiveness of \revision{conformance checking}-based techniques is affected by noisy data and low-quality Petri nets, which may be due to human errors in the modeling process or representational bias of process discovery algorithms \cite{bezerra2013adlogspais, pecchia2020applicationfailuresanalysispm, aalst2016pm}. Specifically, on the one hand, noisy data may introduce infrequent and deceptive control-flow relations that may result in inconsistent fitness measures, whereas, on the other hand, checking event logs against a low-quality Petri net could lead to an unreliable distribution of fitness measures. Nonetheless, such Petri nets can still be used as references to obtain insightful information for \revision{process mining}-based feature extraction, supporting the development of competitive and explainable \revision{conformance checking}-based techniques for control-flow anomaly detection despite the problems above. For example, a few works outline that token-based \revision{conformance checking} can be used for \revision{process mining}-based feature extraction to build tabular data and develop effective \revision{conformance checking}-based techniques for control-flow anomaly detection \cite{singh2022lapmsh, debenedictis2023dtadiiot}. However, to the best of our knowledge, the scientific literature lacks a structured proposal for \revision{process mining}-based feature extraction using the state-of-the-art \revision{conformance checking} variant, namely alignment-based \revision{conformance checking}.

\subsection{Contributions}
We propose a novel \revision{process mining}-based feature extraction approach with alignment-based \revision{conformance checking}. This variant aligns the deviating control flow with a reference Petri net; the resulting alignment can be inspected to extract additional statistics such as the number of times a given activity caused mismatches \cite{aalst2022pmhandbook}. We integrate this approach into a flexible and explainable framework for developing techniques for control-flow anomaly detection. The framework combines \revision{process mining}-based feature extraction and dimensionality reduction to handle high-dimensional feature sets, achieve detection effectiveness, and support explainability. Notably, in addition to our proposed \revision{process mining}-based feature extraction approach, the framework allows employing other approaches, enabling a fair comparison of multiple \revision{conformance checking}-based and \revision{conformance checking}-independent techniques for control-flow anomaly detection. Figure \ref{HIGH_LEVEL_VIEW} shows a high-level view of the framework. Business processes are monitored, and event logs obtained from the database of information systems. Subsequently, \revision{process mining}-based feature extraction is applied to these event logs and tabular data input to dimensionality reduction to identify control-flow anomalies. We apply several \revision{conformance checking}-based and \revision{conformance checking}-independent framework techniques to publicly available datasets, simulated data of a case study from railways, and real-world data of a case study from healthcare. We show that the framework techniques implementing our approach outperform the baseline \revision{conformance checking}-based techniques while maintaining the explainable nature of \revision{conformance checking}.

In summary, the contributions of this paper are as follows.
\begin{itemize}
    \item{
        A novel \revision{process mining}-based feature extraction approach to support the development of competitive and explainable \revision{conformance checking}-based techniques for control-flow anomaly detection.
    }
    \item{
        A flexible and explainable framework for developing techniques for control-flow anomaly detection using \revision{process mining}-based feature extraction and dimensionality reduction.
    }
    \item{
        Application to synthetic and real-world datasets of several \revision{conformance checking}-based and \revision{conformance checking}-independent framework techniques, evaluating their detection effectiveness and explainability.
    }
\end{itemize}

The rest of the paper is organized as follows.
\begin{itemize}
    \item Section \ref{sec:related_work} reviews the existing techniques for control-flow anomaly detection, categorizing them into \revision{conformance checking}-based and \revision{conformance checking}-independent techniques.
    \item Section \ref{sec:abccfe} provides the preliminaries of \revision{process mining} to establish the notation used throughout the paper, and delves into the details of the proposed \revision{process mining}-based feature extraction approach with alignment-based \revision{conformance checking}.
    \item Section \ref{sec:framework} describes the framework for developing \revision{conformance checking}-based and \revision{conformance checking}-independent techniques for control-flow anomaly detection that combine \revision{process mining}-based feature extraction and dimensionality reduction.
    \item Section \ref{sec:evaluation} presents the experiments conducted with multiple framework and baseline techniques using data from publicly available datasets and case studies.
    \item Section \ref{sec:conclusions} draws the conclusions and presents future work.
\end{itemize}

\section{Survey Method}
\section{Method}\label{sec:method}
\begin{figure}
    \centering
    \includegraphics[width=0.85\textwidth]{imgs/heatmap_acc.pdf}
    \caption{\textbf{Visualization of the proposed periodic Bayesian flow with mean parameter $\mu$ and accumulated accuracy parameter $c$ which corresponds to the entropy/uncertainty}. For $x = 0.3, \beta(1) = 1000$ and $\alpha_i$ defined in \cref{appd:bfn_cir}, this figure plots three colored stochastic parameter trajectories for receiver mean parameter $m$ and accumulated accuracy parameter $c$, superimposed on a log-scale heatmap of the Bayesian flow distribution $p_F(m|x,\senderacc)$ and $p_F(c|x,\senderacc)$. Note the \emph{non-monotonicity} and \emph{non-additive} property of $c$ which could inform the network the entropy of the mean parameter $m$ as a condition and the \emph{periodicity} of $m$. %\jj{Shrink the figures to save space}\hanlin{Do we need to make this figure one-column?}
    }
    \label{fig:vmbf_vis}
    \vskip -0.1in
\end{figure}
% \begin{wrapfigure}{r}{0.5\textwidth}
%     \centering
%     \includegraphics[width=0.49\textwidth]{imgs/heatmap_acc.pdf}
%     \caption{\textbf{Visualization of hyper-torus Bayesian flow based on von Mises Distribution}. For $x = 0.3, \beta(1) = 1000$ and $\alpha_i$ defined in \cref{appd:bfn_cir}, this figure plots three colored stochastic parameter trajectories for receiver mean parameter $m$ and accumulated accuracy parameter $c$, superimposed on a log-scale heatmap of the Bayesian flow distribution $p_F(m|x,\senderacc)$ and $p_F(c|x,\senderacc)$. Note the \emph{non-monotonicity} and \emph{non-additive} property of $c$. \jj{Shrink the figures to save space}}
%     \label{fig:vmbf_vis}
%     \vspace{-30pt}
% \end{wrapfigure}


In this section, we explain the detailed design of CrysBFN tackling theoretical and practical challenges. First, we describe how to derive our new formulation of Bayesian Flow Networks over hyper-torus $\mathbb{T}^{D}$ from scratch. Next, we illustrate the two key differences between \modelname and the original form of BFN: $1)$ a meticulously designed novel base distribution with different Bayesian update rules; and $2)$ different properties over the accuracy scheduling resulted from the periodicity and the new Bayesian update rules. Then, we present in detail the overall framework of \modelname over each manifold of the crystal space (\textit{i.e.} fractional coordinates, lattice vectors, atom types) respecting \textit{periodic E(3) invariance}. 

% In this section, we first demonstrate how to build Bayesian flow on hyper-torus $\mathbb{T}^{D}$ by overcoming theoretical and practical problems to provide a low-noise parameter-space approach to fractional atom coordinate generation. Next, we present how \modelname models each manifold of crystal space respecting \textit{periodic E(3) invariance}. 

\subsection{Periodic Bayesian Flow on Hyper-torus \texorpdfstring{$\mathbb{T}^{D}$}{}} 
For generative modeling of fractional coordinates in crystal, we first construct a periodic Bayesian flow on \texorpdfstring{$\mathbb{T}^{D}$}{} by designing every component of the totally new Bayesian update process which we demonstrate to be distinct from the original Bayesian flow (please see \cref{fig:non_add}). 
 %:) 
 
 The fractional atom coordinate system \citep{jiao2023crystal} inherently distributes over a hyper-torus support $\mathbb{T}^{3\times N}$. Hence, the normal distribution support on $\R$ used in the original \citep{bfn} is not suitable for this scenario. 
% The key problem of generative modeling for crystal is the periodicity of Cartesian atom coordinates $\vX$ requiring:
% \begin{equation}\label{eq:periodcity}
% p(\vA,\vL,\vX)=p(\vA,\vL,\vX+\vec{LK}),\text{where}~\vec{K}=\vec{k}\vec{1}_{1\times N},\forall\vec{k}\in\mathbb{Z}^{3\times1}
% \end{equation}
% However, there does not exist such a distribution supporting on $\R$ to model such property because the integration of such distribution over $\R$ will not be finite and equal to 1. Therefore, the normal distribution used in \citet{bfn} can not meet this condition.

To tackle this problem, the circular distribution~\citep{mardia2009directional} over the finite interval $[-\pi,\pi)$ is a natural choice as the base distribution for deriving the BFN on $\mathbb{T}^D$. 
% one natural choice is to 
% we would like to consider the circular distribution over the finite interval as the base 
% we find that circular distributions \citep{mardia2009directional} defined on a finite interval with lengths of $2\pi$ can be used as the instantiation of input distribution for the BFN on $\mathbb{T}^D$.
Specifically, circular distributions enjoy desirable periodic properties: $1)$ the integration over any interval length of $2\pi$ equals 1; $2)$ the probability distribution function is periodic with period $2\pi$.  Sharing the same intrinsic with fractional coordinates, such periodic property of circular distribution makes it suitable for the instantiation of BFN's input distribution, in parameterizing the belief towards ground truth $\x$ on $\mathbb{T}^D$. 
% \yuxuan{this is very complicated from my perspective.} \hanlin{But this property is exactly beautiful and perfectly fit into the BFN.}

\textbf{von Mises Distribution and its Bayesian Update} We choose von Mises distribution \citep{mardia2009directional} from various circular distributions as the form of input distribution, based on the appealing conjugacy property required in the derivation of the BFN framework.
% to leverage the Bayesian conjugacy property of von Mises distribution which is required by the BFN framework. 
That is, the posterior of a von Mises distribution parameterized likelihood is still in the family of von Mises distributions. The probability density function of von Mises distribution with mean direction parameter $m$ and concentration parameter $c$ (describing the entropy/uncertainty of $m$) is defined as: 
\begin{equation}
f(x|m,c)=vM(x|m,c)=\frac{\exp(c\cos(x-m))}{2\pi I_0(c)}
\end{equation}
where $I_0(c)$ is zeroth order modified Bessel function of the first kind as the normalizing constant. Given the last univariate belief parameterized by von Mises distribution with parameter $\theta_{i-1}=\{m_{i-1},\ c_{i-1}\}$ and the sample $y$ from sender distribution with unknown data sample $x$ and known accuracy $\alpha$ describing the entropy/uncertainty of $y$,  Bayesian update for the receiver is deducted as:
\begin{equation}
 h(\{m_{i-1},c_{i-1}\},y,\alpha)=\{m_i,c_i \}, \text{where}
\end{equation}
\begin{equation}\label{eq:h_m}
m_i=\text{atan2}(\alpha\sin y+c_{i-1}\sin m_{i-1}, {\alpha\cos y+c_{i-1}\cos m_{i-1}})
\end{equation}
\begin{equation}\label{eq:h_c}
c_i =\sqrt{\alpha^2+c_{i-1}^2+2\alpha c_{i-1}\cos(y-m_{i-1})}
\end{equation}
The proof of the above equations can be found in \cref{apdx:bayesian_update_function}. The atan2 function refers to  2-argument arctangent. Independently conducting  Bayesian update for each dimension, we can obtain the Bayesian update distribution by marginalizing $\y$:
\begin{equation}
p_U(\vtheta'|\vtheta,\bold{x};\alpha)=\mathbb{E}_{p_S(\bold{y}|\bold{x};\alpha)}\delta(\vtheta'-h(\vtheta,\bold{y},\alpha))=\mathbb{E}_{vM(\bold{y}|\bold{x},\alpha)}\delta(\vtheta'-h(\vtheta,\bold{y},\alpha))
\end{equation} 
\begin{figure}
    \centering
    \vskip -0.15in
    \includegraphics[width=0.95\linewidth]{imgs/non_add.pdf}
    \caption{An intuitive illustration of non-additive accuracy Bayesian update on the torus. The lengths of arrows represent the uncertainty/entropy of the belief (\emph{e.g.}~$1/\sigma^2$ for Gaussian and $c$ for von Mises). The directions of the arrows represent the believed location (\emph{e.g.}~ $\mu$ for Gaussian and $m$ for von Mises).}
    \label{fig:non_add}
    \vskip -0.15in
\end{figure}
\textbf{Non-additive Accuracy} 
The additive accuracy is a nice property held with the Gaussian-formed sender distribution of the original BFN expressed as:
\begin{align}
\label{eq:standard_id}
    \update(\parsn{}'' \mid \parsn{}, \x; \alpha_a+\alpha_b) = \E_{\update(\parsn{}' \mid \parsn{}, \x; \alpha_a)} \update(\parsn{}'' \mid \parsn{}', \x; \alpha_b)
\end{align}
Such property is mainly derived based on the standard identity of Gaussian variable:
\begin{equation}
X \sim \mathcal{N}\left(\mu_X, \sigma_X^2\right), Y \sim \mathcal{N}\left(\mu_Y, \sigma_Y^2\right) \Longrightarrow X+Y \sim \mathcal{N}\left(\mu_X+\mu_Y, \sigma_X^2+\sigma_Y^2\right)
\end{equation}
The additive accuracy property makes it feasible to derive the Bayesian flow distribution $
p_F(\boldsymbol{\theta} \mid \mathbf{x} ; i)=p_U\left(\boldsymbol{\theta} \mid \boldsymbol{\theta}_0, \mathbf{x}, \sum_{k=1}^{i} \alpha_i \right)
$ for the simulation-free training of \cref{eq:loss_n}.
It should be noted that the standard identity in \cref{eq:standard_id} does not hold in the von Mises distribution. Hence there exists an important difference between the original Bayesian flow defined on Euclidean space and the Bayesian flow of circular data on $\mathbb{T}^D$ based on von Mises distribution. With prior $\btheta = \{\bold{0},\bold{0}\}$, we could formally represent the non-additive accuracy issue as:
% The additive accuracy property implies the fact that the "confidence" for the data sample after observing a series of the noisy samples with accuracy ${\alpha_1, \cdots, \alpha_i}$ could be  as the accuracy sum  which could be  
% Here we 
% Here we emphasize the specific property of BFN based on von Mises distribution.
% Note that 
% \begin{equation}
% \update(\parsn'' \mid \parsn, \x; \alpha_a+\alpha_b) \ne \E_{\update(\parsn' \mid \parsn, \x; \alpha_a)} \update(\parsn'' \mid \parsn', \x; \alpha_b)
% \end{equation}
% \oyyw{please check whether the below equation is better}
% \yuxuan{I fill somehow confusing on what is the update distribution with $\alpha$. }
% \begin{equation}
% \update(\parsn{}'' \mid \parsn{}, \x; \alpha_a+\alpha_b) \ne \E_{\update(\parsn{}' \mid \parsn{}, \x; \alpha_a)} \update(\parsn{}'' \mid \parsn{}', \x; \alpha_b)
% \end{equation}
% We give an intuitive visualization of such difference in \cref{fig:non_add}. The untenability of this property can materialize by considering the following case: with prior $\btheta = \{\bold{0},\bold{0}\}$, check the two-step Bayesian update distribution with $\alpha_a,\alpha_b$ and one-step Bayesian update with $\alpha=\alpha_a+\alpha_b$:
\begin{align}
\label{eq:nonadd}
     &\update(c'' \mid \parsn, \x; \alpha_a+\alpha_b)  = \delta(c-\alpha_a-\alpha_b)
     \ne  \mathbb{E}_{p_U(\parsn' \mid \parsn, \x; \alpha_a)}\update(c'' \mid \parsn', \x; \alpha_b) \nonumber \\&= \mathbb{E}_{vM(\bold{y}_b|\bold{x},\alpha_a)}\mathbb{E}_{vM(\bold{y}_a|\bold{x},\alpha_b)}\delta(c-||[\alpha_a \cos\y_a+\alpha_b\cos \y_b,\alpha_a \sin\y_a+\alpha_b\sin \y_b]^T||_2)
\end{align}
A more intuitive visualization could be found in \cref{fig:non_add}. This fundamental difference between periodic Bayesian flow and that of \citet{bfn} presents both theoretical and practical challenges, which we will explain and address in the following contents.

% This makes constructing Bayesian flow based on von Mises distribution intrinsically different from previous Bayesian flows (\citet{bfn}).

% Thus, we must reformulate the framework of Bayesian flow networks  accordingly. % and do necessary reformulations of BFN. 

% \yuxuan{overall I feel this part is complicated by using the language of update distribution. I would like to suggest simply use bayesian update, to provide intuitive explantion.}\hanlin{See the illustration in \cref{fig:non_add}}

% That introduces a cascade of problems, and we investigate the following issues: $(1)$ Accuracies between sender and receiver are not synchronized and need to be differentiated. $(2)$ There is no tractable Bayesian flow distribution for a one-step sample conditioned on a given time step $i$, and naively simulating the Bayesian flow results in computational overhead. $(3)$ It is difficult to control the entropy of the Bayesian flow. $(4)$ Accuracy is no longer a function of $t$ and becomes a distribution conditioned on $t$, which can be different across dimensions.
%\jj{Edited till here}

\textbf{Entropy Conditioning} As a common practice in generative models~\citep{ddpm,flowmatching,bfn}, timestep $t$ is widely used to distinguish among generation states by feeding the timestep information into the networks. However, this paper shows that for periodic Bayesian flow, the accumulated accuracy $\vc_i$ is more effective than time-based conditioning by informing the network about the entropy and certainty of the states $\parsnt{i}$. This stems from the intrinsic non-additive accuracy which makes the receiver's accumulated accuracy $c$ not bijective function of $t$, but a distribution conditioned on accumulated accuracies $\vc_i$ instead. Therefore, the entropy parameter $\vc$ is taken logarithm and fed into the network to describe the entropy of the input corrupted structure. We verify this consideration in \cref{sec:exp_ablation}. 
% \yuxuan{implement variant. traditionally, the timestep is widely used to distinguish the different states by putting the timestep embedding into the networks. citation of FM, diffusion, BFN. However, we find that conditioned on time in periodic flow could not provide extra benefits. To further boost the performance, we introduce a simple yet effective modification term entropy conditional. This is based on that the accumulated accuracy which represents the current uncertainty or entropy could be a better indicator to distinguish different states. + Describe how you do this. }



\textbf{Reformulations of BFN}. Recall the original update function with Gaussian sender distribution, after receiving noisy samples $\y_1,\y_2,\dots,\y_i$ with accuracies $\senderacc$, the accumulated accuracies of the receiver side could be analytically obtained by the additive property and it is consistent with the sender side.
% Since observing sample $\y$ with $\alpha_i$ can not result in exact accuracy increment $\alpha_i$ for receiver, the accuracies between sender and receiver are not synchronized which need to be differentiated. 
However, as previously mentioned, this does not apply to periodic Bayesian flow, and some of the notations in original BFN~\citep{bfn} need to be adjusted accordingly. We maintain the notations of sender side's one-step accuracy $\alpha$ and added accuracy $\beta$, and alter the notation of receiver's accuracy parameter as $c$, which is needed to be simulated by cascade of Bayesian updates. We emphasize that the receiver's accumulated accuracy $c$ is no longer a function of $t$ (differently from the Gaussian case), and it becomes a distribution conditioned on received accuracies $\senderacc$ from the sender. Therefore, we represent the Bayesian flow distribution of von Mises distribution as $p_F(\btheta|\x;\alpha_1,\alpha_2,\dots,\alpha_i)$. And the original simulation-free training with Bayesian flow distribution is no longer applicable in this scenario.
% Different from previous BFNs where the accumulated accuracy $\rho$ is not explicitly modeled, the accumulated accuracy parameter $c$ (visualized in \cref{fig:vmbf_vis}) needs to be explicitly modeled by feeding it to the network to avoid information loss.
% the randomaccuracy parameter $c$ (visualized in \cref{fig:vmbf_vis}) implies that there exists information in $c$ from the sender just like $m$, meaning that $c$ also should be fed into the network to avoid information loss. 
% We ablate this consideration in  \cref{sec:exp_ablation}. 

\textbf{Fast Sampling from Equivalent Bayesian Flow Distribution} Based on the above reformulations, the Bayesian flow distribution of von Mises distribution is reframed as: 
\begin{equation}\label{eq:flow_frac}
p_F(\btheta_i|\x;\alpha_1,\alpha_2,\dots,\alpha_i)=\E_{\update(\parsnt{1} \mid \parsnt{0}, \x ; \alphat{1})}\dots\E_{\update(\parsn_{i-1} \mid \parsnt{i-2}, \x; \alphat{i-1})} \update(\parsnt{i} | \parsnt{i-1},\x;\alphat{i} )
\end{equation}
Naively sampling from \cref{eq:flow_frac} requires slow auto-regressive iterated simulation, making training unaffordable. Noticing the mathematical properties of \cref{eq:h_m,eq:h_c}, we  transform \cref{eq:flow_frac} to the equivalent form:
\begin{equation}\label{eq:cirflow_equiv}
p_F(\vec{m}_i|\x;\alpha_1,\alpha_2,\dots,\alpha_i)=\E_{vM(\y_1|\x,\alpha_1)\dots vM(\y_i|\x,\alpha_i)} \delta(\vec{m}_i-\text{atan2}(\sum_{j=1}^i \alpha_j \cos \y_j,\sum_{j=1}^i \alpha_j \sin \y_j))
\end{equation}
\begin{equation}\label{eq:cirflow_equiv2}
p_F(\vec{c}_i|\x;\alpha_1,\alpha_2,\dots,\alpha_i)=\E_{vM(\y_1|\x,\alpha_1)\dots vM(\y_i|\x,\alpha_i)}  \delta(\vec{c}_i-||[\sum_{j=1}^i \alpha_j \cos \y_j,\sum_{j=1}^i \alpha_j \sin \y_j]^T||_2)
\end{equation}
which bypasses the computation of intermediate variables and allows pure tensor operations, with negligible computational overhead.
\begin{restatable}{proposition}{cirflowequiv}
The probability density function of Bayesian flow distribution defined by \cref{eq:cirflow_equiv,eq:cirflow_equiv2} is equivalent to the original definition in \cref{eq:flow_frac}. 
\end{restatable}
\textbf{Numerical Determination of Linear Entropy Sender Accuracy Schedule} ~Original BFN designs the accuracy schedule $\beta(t)$ to make the entropy of input distribution linearly decrease. As for crystal generation task, to ensure information coherence between modalities, we choose a sender accuracy schedule $\senderacc$ that makes the receiver's belief entropy $H(t_i)=H(p_I(\cdot|\vtheta_i))=H(p_I(\cdot|\vc_i))$ linearly decrease \emph{w.r.t.} time $t_i$, given the initial and final accuracy parameter $c(0)$ and $c(1)$. Due to the intractability of \cref{eq:vm_entropy}, we first use numerical binary search in $[0,c(1)]$ to determine the receiver's $c(t_i)$ for $i=1,\dots, n$ by solving the equation $H(c(t_i))=(1-t_i)H(c(0))+tH(c(1))$. Next, with $c(t_i)$, we conduct numerical binary search for each $\alpha_i$ in $[0,c(1)]$ by solving the equations $\E_{y\sim vM(x,\alpha_i)}[\sqrt{\alpha_i^2+c_{i-1}^2+2\alpha_i c_{i-1}\cos(y-m_{i-1})}]=c(t_i)$ from $i=1$ to $i=n$ for arbitrarily selected $x\in[-\pi,\pi)$.

After tackling all those issues, we have now arrived at a new BFN architecture for effectively modeling crystals. Such BFN can also be adapted to other type of data located in hyper-torus $\mathbb{T}^{D}$.

\subsection{Equivariant Bayesian Flow for Crystal}
With the above Bayesian flow designed for generative modeling of fractional coordinate $\vF$, we are able to build equivariant Bayesian flow for each modality of crystal. In this section, we first give an overview of the general training and sampling algorithm of \modelname (visualized in \cref{fig:framework}). Then, we describe the details of the Bayesian flow of every modality. The training and sampling algorithm can be found in \cref{alg:train} and \cref{alg:sampling}.

\textbf{Overview} Operating in the parameter space $\bthetaM=\{\bthetaA,\bthetaL,\bthetaF\}$, \modelname generates high-fidelity crystals through a joint BFN sampling process on the parameter of  atom type $\bthetaA$, lattice parameter $\vec{\theta}^L=\{\bmuL,\brhoL\}$, and the parameter of fractional coordinate matrix $\bthetaF=\{\bmF,\bcF\}$. We index the $n$-steps of the generation process in a discrete manner $i$, and denote the corresponding continuous notation $t_i=i/n$ from prior parameter $\thetaM_0$ to a considerably low variance parameter $\thetaM_n$ (\emph{i.e.} large $\vrho^L,\bmF$, and centered $\bthetaA$).

At training time, \modelname samples time $i\sim U\{1,n\}$ and $\bthetaM_{i-1}$ from the Bayesian flow distribution of each modality, serving as the input to the network. The network $\net$ outputs $\net(\parsnt{i-1}^\mathcal{M},t_{i-1})=\net(\parsnt{i-1}^A,\parsnt{i-1}^F,\parsnt{i-1}^L,t_{i-1})$ and conducts gradient descents on loss function \cref{eq:loss_n} for each modality. After proper training, the sender distribution $p_S$ can be approximated by the receiver distribution $p_R$. 

At inference time, from predefined $\thetaM_0$, we conduct transitions from $\thetaM_{i-1}$ to $\thetaM_{i}$ by: $(1)$ sampling $\y_i\sim p_R(\bold{y}|\thetaM_{i-1};t_i,\alpha_i)$ according to network prediction $\predM{i-1}$; and $(2)$ performing Bayesian update $h(\thetaM_{i-1},\y^\calM_{i-1},\alpha_i)$ for each dimension. 

% Alternatively, we complete this transition using the flow-back technique by sampling 
% $\thetaM_{i}$ from Bayesian flow distribution $\flow(\btheta^M_{i}|\predM{i-1};t_{i-1})$. 

% The training objective of $\net$ is to minimize the KL divergence between sender distribution and receiver distribution for every modality as defined in \cref{eq:loss_n} which is equivalent to optimizing the negative variational lower bound $\calL^{VLB}$ as discussed in \cref{sec:preliminaries}. 

%In the following part, we will present the Bayesian flow of each modality in detail.

\textbf{Bayesian Flow of Fractional Coordinate $\vF$}~The distribution of the prior parameter $\bthetaF_0$ is defined as:
\begin{equation}\label{eq:prior_frac}
    p(\bthetaF_0) \defeq \{vM(\vm_0^F|\vec{0}_{3\times N},\vec{0}_{3\times N}),\delta(\vc_0^F-\vec{0}_{3\times N})\} = \{U(\vec{0},\vec{1}),\delta(\vc_0^F-\vec{0}_{3\times N})\}
\end{equation}
Note that this prior distribution of $\vm_0^F$ is uniform over $[\vec{0},\vec{1})$, ensuring the periodic translation invariance property in \cref{De:pi}. The training objective is minimizing the KL divergence between sender and receiver distribution (deduction can be found in \cref{appd:cir_loss}): 
%\oyyw{replace $\vF$ with $\x$?} \hanlin{notations follow Preliminary?}
\begin{align}\label{loss_frac}
\calL_F = n \E_{i \sim \ui{n}, \flow(\parsn{}^F \mid \vF ; \senderacc)} \alpha_i\frac{I_1(\alpha_i)}{I_0(\alpha_i)}(1-\cos(\vF-\predF{i-1}))
\end{align}
where $I_0(x)$ and $I_1(x)$ are the zeroth and the first order of modified Bessel functions. The transition from $\bthetaF_{i-1}$ to $\bthetaF_{i}$ is the Bayesian update distribution based on network prediction:
\begin{equation}\label{eq:transi_frac}
    p(\btheta^F_{i}|\parsnt{i-1}^\calM)=\mathbb{E}_{vM(\bold{y}|\predF{i-1},\alpha_i)}\delta(\btheta^F_{i}-h(\btheta^F_{i-1},\bold{y},\alpha_i))
\end{equation}
\begin{restatable}{proposition}{fracinv}
With $\net_{F}$ as a periodic translation equivariant function namely $\net_F(\parsnt{}^A,w(\parsnt{}^F+\vt),\parsnt{}^L,t)=w(\net_F(\parsnt{}^A,\parsnt{}^F,\parsnt{}^L,t)+\vt), \forall\vt\in\R^3$, the marginal distribution of $p(\vF_n)$ defined by \cref{eq:prior_frac,eq:transi_frac} is periodic translation invariant. 
\end{restatable}
\textbf{Bayesian Flow of Lattice Parameter \texorpdfstring{$\boldsymbol{L}$}{}}   
Noting the lattice parameter $\bm{L}$ located in Euclidean space, we set prior as the parameter of a isotropic multivariate normal distribution $\btheta^L_0\defeq\{\vmu_0^L,\vrho_0^L\}=\{\bm{0}_{3\times3},\bm{1}_{3\times3}\}$
% \begin{equation}\label{eq:lattice_prior}
% \btheta^L_0\defeq\{\vmu_0^L,\vrho_0^L\}=\{\bm{0}_{3\times3},\bm{1}_{3\times3}\}
% \end{equation}
such that the prior distribution of the Markov process on $\vmu^L$ is the Dirac distribution $\delta(\vec{\mu_0}-\vec{0})$ and $\delta(\vec{\rho_0}-\vec{1})$, 
% \begin{equation}
%     p_I^L(\boldsymbol{L}|\btheta_0^L)=\mathcal{N}(\bm{L}|\bm{0},\bm{I})
% \end{equation}
which ensures O(3)-invariance of prior distribution of $\vL$. By Eq. 77 from \citet{bfn}, the Bayesian flow distribution of the lattice parameter $\bm{L}$ is: 
\begin{align}% =p_U(\bmuL|\btheta_0^L,\bm{L},\beta(t))
p_F^L(\bmuL|\bm{L};t) &=\mathcal{N}(\bmuL|\gamma(t)\bm{L},\gamma(t)(1-\gamma(t))\bm{I}) 
\end{align}
where $\gamma(t) = 1 - \sigma_1^{2t}$ and $\sigma_1$ is the predefined hyper-parameter controlling the variance of input distribution at $t=1$ under linear entropy accuracy schedule. The variance parameter $\vrho$ does not need to be modeled and fed to the network, since it is deterministic given the accuracy schedule. After sampling $\bmuL_i$ from $p_F^L$, the training objective is defined as minimizing KL divergence between sender and receiver distribution (based on Eq. 96 in \citet{bfn}):
\begin{align}
\mathcal{L}_{L} = \frac{n}{2}\left(1-\sigma_1^{2/n}\right)\E_{i \sim \ui{n}}\E_{\flow(\bmuL_{i-1} |\vL ; t_{i-1})}  \frac{\left\|\vL -\predL{i-1}\right\|^2}{\sigma_1^{2i/n}},\label{eq:lattice_loss}
\end{align}
where the prediction term $\predL{i-1}$ is the lattice parameter part of network output. After training, the generation process is defined as the Bayesian update distribution given network prediction:
\begin{equation}\label{eq:lattice_sampling}
    p(\bmuL_{i}|\parsnt{i-1}^\calM)=\update^L(\bmuL_{i}|\predL{i-1},\bmuL_{i-1};t_{i-1})
\end{equation}
    

% The final prediction of the lattice parameter is given by $\bmuL_n = \predL{n-1}$.
% \begin{equation}\label{eq:final_lattice}
%     \bmuL_n = \predL{n-1}
% \end{equation}

\begin{restatable}{proposition}{latticeinv}\label{prop:latticeinv}
With $\net_{L}$ as  O(3)-equivariant function namely $\net_L(\parsnt{}^A,\parsnt{}^F,\vQ\parsnt{}^L,t)=\vQ\net_L(\parsnt{}^A,\parsnt{}^F,\parsnt{}^L,t),\forall\vQ^T\vQ=\vI$, the marginal distribution of $p(\bmuL_n)$ defined by \cref{eq:lattice_sampling} is O(3)-invariant. 
\end{restatable}


\textbf{Bayesian Flow of Atom Types \texorpdfstring{$\boldsymbol{A}$}{}} 
Given that atom types are discrete random variables located in a simplex $\calS^K$, the prior parameter of $\boldsymbol{A}$ is the discrete uniform distribution over the vocabulary $\parsnt{0}^A \defeq \frac{1}{K}\vec{1}_{1\times N}$. 
% \begin{align}\label{eq:disc_input_prior}
% \parsnt{0}^A \defeq \frac{1}{K}\vec{1}_{1\times N}
% \end{align}
% \begin{align}
%     (\oh{j}{K})_k \defeq \delta_{j k}, \text{where }\oh{j}{K}\in \R^{K},\oh{\vA}{KD} \defeq \left(\oh{a_1}{K},\dots,\oh{a_N}{K}\right) \in \R^{K\times N}
% \end{align}
With the notation of the projection from the class index $j$ to the length $K$ one-hot vector $ (\oh{j}{K})_k \defeq \delta_{j k}, \text{where }\oh{j}{K}\in \R^{K},\oh{\vA}{KD} \defeq \left(\oh{a_1}{K},\dots,\oh{a_N}{K}\right) \in \R^{K\times N}$, the Bayesian flow distribution of atom types $\vA$ is derived in \citet{bfn}:
\begin{align}
\flow^{A}(\parsn^A \mid \vA; t) &= \E_{\N{\y \mid \beta^A(t)\left(K \oh{\vA}{K\times N} - \vec{1}_{K\times N}\right)}{\beta^A(t) K \vec{I}_{K\times N \times N}}} \delta\left(\parsn^A - \frac{e^{\y}\parsnt{0}^A}{\sum_{k=1}^K e^{\y_k}(\parsnt{0})_{k}^A}\right).
\end{align}
where $\beta^A(t)$ is the predefined accuracy schedule for atom types. Sampling $\btheta_i^A$ from $p_F^A$ as the training signal, the training objective is the $n$-step discrete-time loss for discrete variable \citep{bfn}: 
% \oyyw{can we simplify the next equation? Such as remove $K \times N, K \times N \times N$}
% \begin{align}
% &\calL_A = n\E_{i \sim U\{1,n\},\flow^A(\parsn^A \mid \vA ; t_{i-1}),\N{\y \mid \alphat{i}\left(K \oh{\vA}{KD} - \vec{1}_{K\times N}\right)}{\alphat{i} K \vec{I}_{K\times N \times N}}} \ln \N{\y \mid \alphat{i}\left(K \oh{\vA}{K\times N} - \vec{1}_{K\times N}\right)}{\alphat{i} K \vec{I}_{K\times N \times N}}\nonumber\\
% &\qquad\qquad\qquad-\sum_{d=1}^N \ln \left(\sum_{k=1}^K \out^{(d)}(k \mid \parsn^A; t_{i-1}) \N{\ydd{d} \mid \alphat{i}\left(K\oh{k}{K}- \vec{1}_{K\times N}\right)}{\alphat{i} K \vec{I}_{K\times N \times N}}\right)\label{discdisc_t_loss_exp}
% \end{align}
\begin{align}
&\calL_A = n\E_{i \sim U\{1,n\},\flow^A(\parsn^A \mid \vA ; t_{i-1}),\N{\y \mid \alphat{i}\left(K \oh{\vA}{KD} - \vec{1}\right)}{\alphat{i} K \vec{I}}} \ln \N{\y \mid \alphat{i}\left(K \oh{\vA}{K\times N} - \vec{1}\right)}{\alphat{i} K \vec{I}}\nonumber\\
&\qquad\qquad\qquad-\sum_{d=1}^N \ln \left(\sum_{k=1}^K \out^{(d)}(k \mid \parsn^A; t_{i-1}) \N{\ydd{d} \mid \alphat{i}\left(K\oh{k}{K}- \vec{1}\right)}{\alphat{i} K \vec{I}}\right)\label{discdisc_t_loss_exp}
\end{align}
where $\vec{I}\in \R^{K\times N \times N}$ and $\vec{1}\in\R^{K\times D}$. When sampling, the transition from $\bthetaA_{i-1}$ to $\bthetaA_{i}$ is derived as:
\begin{equation}
    p(\btheta^A_{i}|\parsnt{i-1}^\calM)=\update^A(\btheta^A_{i}|\btheta^A_{i-1},\predA{i-1};t_{i-1})
\end{equation}

The detailed training and sampling algorithm could be found in \cref{alg:train} and \cref{alg:sampling}.





\section{Research Themes}
\label{themes-section}
Based on the literature found through the process outlined in Section \ref{select-lit-section}, articles with similar subjects or covering similar themes of research are grouped together. The themes are selected in such a way that most publications can be exclusively divided into one of the themes, i.e., the themes should not have significant overlap. Furthermore the themes should effectively separate domain specific research and more generally applicable research. %Finally we have a specific interest in the developments within the domain of data center and HPC research. 
Considering these requirements the following list of themes is selected:
% We start off by separating the literature into two distinct categories: (1) articles which implement an FPGA for a specific application, which covers a wide variety of research purposes, and (2) articles on the advancement of FPGA hardware and tooling software itself. The second of the two is further divided into more specific categories, resulting in the following list of themes:

\begin{enumerate}
    \item FPGA architecture
    \item Robustness of FPGAs
    \item Data center infrastructure \& HPC
    \item Programming models \& tools
    \item Applications
\end{enumerate}

Figure~\ref{fig:theme-distribution} illustrates how these themes cover both general and domain-specific development, and shows whether a theme is hardware or software focused. The theme ``Applications'' focuses on research applying FPGA technology to domain specific problems. This can be in the form of hardware architectures for domain-specific applications, as well as software tools enabling FPGA technology in a specific domain of research. The themes ``Programming models and tools'' and ``FPGA architecture'' focus on research and development of solutions that are generally applicable in a wide range of domains, while the ``Robustness of FPGAs'' and ``Data center infrastructure \& HPC'' themes feature both hardware- and software-focused research. Moreover, these themes focus on a narrower selection of FPGA applications and can, therefore, be considered  domain-specific. 

% add some explanation of figure
\begin{figure}[!htbp]
    \centering
    \includegraphics[width=0.5\textwidth]{figures/theme_distribution_diagram.pdf}
    \caption{The themes that are selected can be differentiated based on their domain-specificity, ranging from very domain-focused to general purpose, and based on whether the main focus is on hardware or software. }
    \label{fig:theme-distribution}
\end{figure}
Based on common subjects in each theme, the themes are further organized into %more specific 
subcategories. Table \ref{tab:overview-themes-most-cited} shows the subcategories and %by which each theme is subdivided, as well as 
the prevalence of each  subject based on the number of published articles. % covered in it. 
Finally, the most influential articles, based on the highest number of citations within each category (Google Scholar), is shown. This selection excludes survey publications.

\begin{table}[!ht]
\centering
\caption{Overview of highly cited papers per category, with number of published articles per theme and category in parentheses.}
\label{tab:overview-themes-most-cited}
{\small
\begin{tabular}{lll}
\textbf{Theme and category} & \textbf{Highly cited publications} & \textbf{Dutch affiliation} \\ \hline
\textbf{\textbf{FPGA architecture (11)}} &  &  \\ \cline{1-1}
Near-memory processing (4) & \citet{Singh2021FPGA-BasedApplications} & Eindhoven University \\
Coarse-grained reconfigurable architecture (4) & \citet{Wijtvliet2019Blocks:Efficiency} & Eindhoven University \\
Network-on-Chip (3) & \citet{RibotGonzalez2020HopliteRT:FPGA} & Eindhoven University \\ \hline
\multicolumn{2}{l}{\textbf{\textbf{Data center infrastructure \& HPC (40)}}}  &  \\ \cline{1-1}
Big data processing and analytics (22) & \citet{Peltenburg2019Fletcher:Arrow} & Delft University of Tech. \\
Distributed computing (5) & \citet{Bielski2018DReDBox:Datacenter} & Sintecs B.V. \\
Optical hardware communication (9) & \citet{Yan2018HiFOST:Switches} & Eindhoven University \\
High performance computing (4) & \citet{Katevenis2018NextDevelopment} & MonetDB Solutions \\ \hline
\multicolumn{2}{l}{\textbf{\textbf{Programming models \& tools (15)}} }  &  \\ \cline{1-1}
Programming models and frameworks (8) & \citet{Peltenburg2020Tydi:Streams} & Delft University of Tech.  \\
Performance prediction (7) & \citet{Yasudo2018PerformancePlatforms} & University of Amsterdam \\ \hline
\textbf{\textbf{Robustness of FPGAs (26)}} &  &  \\ \cline{1-1}
Reliability (12) & \citet{Du2019UltrahighFPGA} & ESTEC \\
Hardware security (14) & \citet{Labafniya2020OnPrevention} & Delft University of Tech. \\ \hline
\textbf{\textbf{Applications (49)}} &  &  \\ \cline{1-1}
Machine learning (12) & \citet{Rocha2020BinaryWrist-PPG} & IMEC NL \\
Astronomy (11) & \citet{Ashton2020ATelescopes} & University of Amsterdam \\
Particle physics experiments (7) & \citet{FernandezPrieto2020PhaseExperiment} & Nikhef \\
Quantum computing (5) & \citet{Philips-nat-2022} & Delft University of Tech. \\
Space (9) & \citet{Barrios2020SHyLoCMissions} & ESTEC \\
Bioinformatics (5) & \citet{Malakonakis2020ExploringRAxML} & University of Twente \\ \hline
\end{tabular}
}
\end{table}


Figure~\ref{fig:org-publish-per-theme} illustrates an overview of publications per theme for each organization with more than one publication. It is clear that most major contributors to FPGA research publish mostly application-specific research. Out of the major contributors, Delft University of Technology focuses more on the ``Data center \& infrastructure'' domain, while Eindhoven University of Technology is a larger contributor to the ``FPGA architecture'' theme. A brief description of each theme is provided below. 


\begin{figure}[!htb]
    \centering
    \includegraphics[width=\textwidth]{figures/per_chapter_no_papers_2.pdf}
    \caption{Number of publications per theme for each organization with more than one relevant publication.}
    \label{fig:org-publish-per-theme}
\end{figure}


\begin{itemize}
    \item {\bf FPGA architecture}: This research theme covers literature related to the design of novel digital hardware architectures. Efficient architectures, fast on-chip memory access, coarse grained hardware design, and partially reconfigurable hardware are covered in this theme.
\item {\bf Data center infrastructure \& HPC}: This theme includes literature on FPGAs used in high-performance computing environments. This covers papers on the rapid processing of big data, and research towards distributed computing infrastructures deploying FPGAs. Furthermore, research focusing on employing FPGAs for processing communication between computing nodes using optical links is also covered here.
\item {\bf Programming models \& tools}:
This theme covers literature related to tools and models used to program FPGAs, ranging from research on high-level synthesis (HLS) tools to tools that enable accessible hardware acceleration of conventional software. This theme also features research efforts on tools for accurate performance prediction of synthesized 
FPGA solutions.
\item {\bf Robustness of FPGAs}:
This theme covers literature regarding the reliability and resilience of FPGAs to specific environments. Specifically, resilience to radiation in environments where this is prevalent is a common subject. Furthermore, this theme expands on the security of FPGAs with regards to cyberattacks.
\item {\bf Applications}: The literature on specific applications using FPGAs is more extensive than that of the other themes. This is expected since FPGAs can be applied in various fields, whereas the advancement of FPGA architectures and development tools is generally a more narrow area of research. 
%Within this theme, 
Machine learning has been the main focus in recent years.
\end{itemize}

%\paragraph{Hardware and architecture}
%This research theme covers literature related to the design of novel digital hardware architecture. Efficient architectures, fast on-chip memory access, coarse grained hardware design and partially reconfigurable hardware are subjects that are covered within this theme.

%\paragraph{Robustness of FPGAs}
%In the theme of robustness the literature regarding the reliability and resilience of FPGAs to specific environments is covered. Specifically the resilience to radiation in environments where this is prevalent is a common subject. Furthermore, this theme expands on the security of FPGAs with regards to cyberattacks.

%\paragraph{Data center \& infrastructure}
%This theme encapsulates the literature on FPGAs used in high-performance computing environments. This includes literature on the rapid processing of big data and research towards distributed computing infrastructures implementing FPGAs. Furthermore, ample research is being done towards using FPGAs for processing the communication between computing nodes using optical links. This is also covered under this theme.

%\paragraph{Programming models \& tools}
%This research theme covers literature related to the tools and models used to program FPGAs. This ranges from research towards HLS tools, to tools which allow accessible hardware acceleration of conventional software. This theme also features research towards tools for efficient and accurate performance prediction of synthesized FPGA solutions.

%\paragraph{Applications of FPGAs}
%The literature on specific applications using FPGAs is more numerous than other themes. This is not unexpected since FPGAs can be applied in numerous fields, whilst the advancement of FPGAs and FPGA associated tools itself is a more narrow area of research. Within this theme, machine learning is the most researched subject in the recent past.

\section{FPGA Architecture}
\label{sec:archi}
%\vspace{-3mm}
\subsection{Multi-destination active message format}
\label{section:message_format}
\vspace{-0.7cm}
\begin{figure}[h!]
	\scriptsize
        \centering
    % \hspace{-1cm}
    \includegraphics[width=1\columnwidth]{diagrams/message_format.pdf}
    \vspace{-0.4cm}
	\caption{Message format} 
	\label{fig:message_format}
	\vspace{-.3cm}
\end{figure}
\textit{Nexus Machine} extends the fundamental Active Message primitives to accommodate a multi-destination based routing mechanism. 
Fig.~\ref{fig:message_format} illustrates the message format: the first 12 bits specify intermediate destinations (\textit{R1}, \textit{R2}, \textit{R3}), based on our workload analysis. 
The next 4 bits contain the Program Counter (PC) for the next instruction (\textit{N\_PC}), followed by 4 bits for the \textit{Opcode}. 
A single bit (\textit{Res\_c}) indicates if the message carries a result. 
The subsequent 2 bits (\textit{Op1\_c} and \textit{Op2\_c}) identify whether \textit{Op1} and \textit{Op2} are addresses or values. 
Depending on \textit{Res\_c}, the \textit{Result} field contains the final result or its address, while the next 16 bits hold data for Operand1 (\textit{Op1}) and Operand2 (\textit{Op2}).

When a message arrives at a router, the first destination (\textit{R1}) is processed by the \textit{Route Computation} logic and then allocated to the appropriate output port. After reaching \textit{R1}, the message is handled by the \textit{Input Network Interface}, and the remaining destinations are cyclically rotated, making \textit{R2} the first and \textit{R3} the second. 

In the \textit{Nexus Machine}, a message is equivalent to a packet or flit (all messages are a single-flit packet).
\begin{comment}
\begin{figure*}[h!]
	\scriptsize
	\centering
	\includegraphics[width=\textwidth]{diagrams/architecture.pdf}
	\caption{\textit{Nexus Machine} microarchitecture. \textit{Nexus Machine} is a fabric of homogenous PEs interconnected by a mesh network for communicating Active messages, enhancing fabric utilization by executing messages en-route.} 
	\label{fig:detail_arch}
	%\vspace{-.5cm}
\end{figure*}
\end{comment}
%\vspace{-3mm}
\subsection{Nexus Machine Micro-architecture}
\begin{figure*}[h!]
	\scriptsize
	\centering
	\includegraphics[width=0.9\textwidth]{diagrams/architecture.pdf}
    \vspace{-.15cm}
	\caption{\textit{Nexus Machine} microarchitecture. A fabric of homogenous PEs interconnected by a mesh network for communicating Active Messages which carry instructions that can be launched en-route at any PE, enhancing fabric utilization and runtime.} 
    \vspace{-0.3cm}
 %\color{red}{\bf Peh: I suggest replacing (d) with one of the Compute Unit, cos it's a major component of Nexus and yet do not feature in any figure. We need to highlight to reviewers that our compute unit consists of ALU :)}} 
	\label{fig:detail_arch}
	%\vspace{-.5cm}
\end{figure*}
%\subsubsection{Top Level}
As presented in Fig.~\ref{fig:detail_arch}(a), the \textit{Nexus Machine}'s fabric comprises homogeneous processing elements (PEs) interconnected with a mesh network, with a global termination detector. Each PE is linked to four neighboring PEs in North, East, South, and West directions.
The off-chip memory is connected to the four PEs located along the left edge.


\subsubsection{Processing Elements (PEs).}
%As presented in figure~\ref{fig:detail_arch}(b), each PE combines a compute unit, a dynamic router for network connectivity with congestion control, a decode unit with local data memory, an Input Network Interface which contains an instruction memory for handling incoming AMs and an AM Network Interface unit for spawning new AMs. \\
As presented in Fig.~\ref{fig:detail_arch}(b), each PE combines a compute unit, a dynamic router for network connectivity with congestion control, a decode unit, and two Network Interface logic.
Specifically, \textit{Input Network Interface} unit is responsible for efficiently handling incoming AMs from the NoC, while the AM Network Interface unit initiates the injection of new messages into the NoC.

\textbf{Input Network Interface.}
%The Input Network Interface logic triggers the loading of subsequent instruction on AM arrival.
%The arrival of a Decode AM triggers loading of the data element 
The \textit{Input Network Interface} unit manages \textit{incoming AMs} to a PE.
Depending on the message, \\%it performs either of these two operations.\\
%(a) It either updates the instruction contained in the message based on the next Program Counter (N\_PC) value provided within the message body.\\
(a) If it pertains to an ALU operation, it is directed to the \textit{Compute Unit} for execution.\\
(b) Alternatively, in case of a memory operation, the message is forwarded to the \textit{Decode} unit. 
This unit initiates a load or store operation, utilizing the operand address information (\textit{Op1} or \textit{Op2}) contained in the message.\\
Once these operations are completed, the resulting \textit{output dynamic AM} is dispatched to the \textit{AM Network Interface} for injecting into the network.
%The message enters the network via the local input port, which feeds the \textit{Compute} unit.

\textbf{Compute Unit.}
The \textit{compute unit} within a PE can perform 16-bit arithmetic operations, logic operations, multiplication, and division on its ALU.

An incoming AM at the \textit{Input Network Interface} dispatches two operands, \textit{Op1} and \textit{Op2} along with the \textit{Opcode} field in the message to the compute unit.
After computation, it generates an output that is combined with the original AM, replacing the \textit{Op1} field in the message.
Finally, this modified AM is forwarded to the \textit{AM Network Interface} for injecting into the network.

%{\bf Peh: There needs to be detailed information on how an AM launches computation! This is the thesis of AM! For instance, what's the format of the AM, when it's received, which field is used to configure the ALU? How is the PC set? What happens in the beginning of execution? read config memory? are there registers? what happens if data operand is not present -- can that happen? stall? Lots of details needed here.}

\textbf{Decode Unit.}
The \textit{Decode Unit}, as shown in Fig.~\ref{fig:detail_arch}(e), can be flexibly configured to operate in dereference and streaming modes.
In \textbf{dereference mode}, the operand address field (\textit{Op1} or \textit{Op2}) in the message triggers the loading of a single element. This gets embedded into the output \textit{dynamic AM}.
Conversely, in \textbf{streaming mode}, the message initiates the loading of multiple elements from memory, generating multiple output AMs.
In this mode, the operand address is considered the base address, along with a count to access and load the elements from memory sequentially.
These two modes suffice for our benchmarks; however, our architecture allows for integration of additional modes if needed.

\textbf{Active Message (AM) Network Interface.}
The \textit{AM Network Interface logic} is responsible for injecting AMs into the network.
%The AM Network Interface logic consists of an AM Queue and a configuration memory.
%The AM Queue is a 1KB FIFO, initialized with 44-bit precompiled entries.
%The configuration memory is 16-bit wide, containing 8 configurations.
This module comprises two primary components: an \textit{AM Queue} and a \textit{configuration memory}. 
The \textit{AM Queue} is a 16KB FIFO initialized with 70-bit precompiled entries. 
The \textit{configuration memory}, 10-bit wide, accommodates 8 distinct configurations.

%Depending on the availability of the output dynamic AM from the \textit{Input Network Interface}, it either
It either performs these two operations, as shown in Fig.~\ref{fig:detail_arch}(b):
(1) If the output \textit{dynamic AM} is available from \textit{Input Network Interface}, the subsequent configuration is loaded from memory based on the \textit{N\_PC} field of the AM (see Fig.~\ref{fig:message_format}). 
This configuration is combined with the output \textit{dynamic AM} and forwarded into the injection port of the router.\\
(2) Alternatively, a \textit{static AM} is injected into the network to keep it occupied. 
This \textit{static AM} is the concatenation of the next precompiled entry from the \textit{AM Queue} with the first configuration loaded from memory.
The generation rate of \textit{static AMs} is determined by the backpressure signal at the router's injection port.

The highlighted blue fields in the message format (see Fig.~\ref{fig:message_format}) depict data from the configuration memory used to construct the subsequent dynamic AM, with fields \textit{Res\_c}, \textit{Op1\_c}, and \textit{Op2\_c} stored to prevent redundancy.
%The AM Network Interface logic consists of a 1KB AM Queue, a FIFO containing 44-bit pre-compiled entries.
%, alongside a 13-bit wide configuration register. 

%As shown in Figure~\ref{fig:detail_arch}(e), the output dynamic AMs from \textit{Input Network Interface} trigger loading the next subsequent configuration from the memory with the N\_PC field of the AM.
%These are further concatenated with the output dynamic AMs and pushed into the injection port of the router.
%The injection rate is managed by the backpressure signal at the injection port of the router.

%To keep the network occupied, static AMs are containing the first precompiled entry from AM queue
%As shown in Figure~\ref{fig:detail_arch}(e), it concatenates an AM Queue entry with the first configuration loaded from the memory to generate a static AM, which is subsequently pushed intothe injection port of the router. 

\subsubsection{Dynamic and Congestion Aware Routing.}
\textit{Nexus Machine} supports turn model routing~\cite{noc_peh}, with each router containing five input and five output ports.
%Specifically, these input ports correspond to AM, local, north, east, south and west, whereas output ports correspond to local and four directions.
Specifically, these input ports are designated for messages coming from \textit{AM Network Interface} unit, as well as north, east, south, and west directions, whereas output ports are designated for messages going to \textit{Input Network Interface} unit and four directions.
%The AM input port receives recently generated messages from the \textit{AM Network Interface unit}, while the local port handles messages coming from the \textit{Input Network Interface}. 
Each input port has a buffer comprising three registers to manage in-flight messages, accompanied by congestion control logic. \textit{Nexus Machine}'s design choice of employing only three registers is motivated by the goal of minimizing overall power consumption.

As presented in Fig.~\ref{fig:detail_arch}(c), each router contains a Route Computation Unit, Separable Allocator, and a Crossbar.

\textbf{Route Computation} logic considers the destination of messages from all the input ports. It compares it with the positional ID of the PE, and calculates the output port to be requested. This is sent as an input to the allocator.

%{\bf Peh: Separable alllocation is a well-known previously proposed technique... so there's no need to elaborate... just cite a NoCs textbook}
A toy example of \textbf{Separable Allocation} process is presented in Fig.~\ref{fig:detail_arch}(d)~\cite{noc_peh}.
\iffalse
The request matrix's rows correspond to input ports, and columns correspond to output ports.
The process consists of two stages of 6:1 and 5:1 fixed priority arbiters. The first stage prunes the matrix to ensure that each output port (or resource) receives requests from at most one input port (or requestor). Subsequently, the backpressure signal is applied to each output port, enabling congestion control, as explained below. The second stage further prunes the matrix to guarantee one grant per input port.
The allocator executes within a single cycle, marking granted requests as issued immediately to prevent them from bidding again.
\fi

\textbf{On/Off Congestion control} involves the transmission of a signal to the upstream router when the count of available buffers falls below a threshold, ensuring all in-flight messages will have buffers on arrival. Each of the five ports transmits an OFF signal when their corresponding available buffer space is reduced to 1, i.e., $T_{OFF} = 1$, and conversely, an ON signal when their buffer space reaches 2, i.e., $T_{ON} = 2$.

The output of the allocator is sent to a 6x5 \textbf{Crossbar}.
%, which forms many-to-many connections among internal and external datapaths.\\ Peh: A crossbar by definition forms many-to-many connections between its input and output ports, so no need to explain. 

\subsubsection{Off-chip Memory Datapath.}
Each off-chip memory port connects to a row of the PE array via an AXI bus, delivering a combined bandwidth of 1.28GBps. During data loading, data transfers from off-chip memory to the \textit{AM queues} and \textit{data memory} in each PE. 
The \textit{AM queues} are actively consumed during execution, effectively hiding data loading latency by performing it concurrently with the execution. 
However, data loading into \textit{data memories} occurs after tile execution is complete.

\subsubsection{Bit-vector Scanners.}
The first sparse operand is encoded in \textit{static AMs}. For subsequent sparse operands, bit-vector scanner hardware assists in efficient iteration, providing coordinates within compressed vectors as described in \cite{capstan}. \textit{Nexus Machine} integrates a modified version of this with its AXI bus controller to obtain these coordinates. It can vectorize 16 non-zeros within 128 elements, allowing it to handle matrices with densities exceeding 12\%.
\vspace{-0.2cm}
\subsection{Deadlock avoidance}
%\textcolor{blue}{\bf Peh: Write up a short blurb on how Nexus machine addresses various deadlock scenarios and explain design choice: (1) Within network: Flow control deadlocks addressed by bubble buffer [cite bubble flow control] instead of VCs so as to minimize buffering; Routing deadlocks addressed by turn model, so as to provide high throughput without complex adaptive routing hardware; (2) AM also introduces potential network-PE deadlocks -- addressed by compiler preventing such cyclic dependencies + runtime timeouts} 

Given the dynamic nature, \textit{Nexus Machine} can potentially encounter deadlock without careful design. We address various deadlock scenarios with these specific design choices: 
(1) To mitigate flow control deadlocks within the network, we adopt the bubble NoC~\cite{bubble_flow} approach over Virtual Channels (VCs), with the aim to minimize buffering. 
(2) Routing deadlocks are mitigated by using the turn model~\cite{noc_peh}, that ensures high throughput without the need for complex adaptive routing hardware.
(3) AMs can potentially create deadlocks between the network and PEs. 
These are effectively mitigated by the compiler through strategic data placement and runtime timeouts.
%\textit{Nexus Machine} currently uses a simple heuristic for data placement strategy within the compiler.
%\textcolor{red}{\bf Peh: Is the simple heuristic related to deadlocks? Cos the above sentence seems to contradict the earlier sentence. Elaborate on the heuristic and timeouts??}
Future research will explore more optimized data placement strategies.

\section{Data center infrastructure \& HPC}
\label{sec:HPC}
%This subsection contains all papers regarding FPGA use in dealing with big data, high performance computing and distributed computing.

%%%% This is a long text for intro, but we can polish it and provide a summary as it. Please feel free to revise the text and add/remove sentences:
\iffalse

In this section, we overview the research and activity related to FPGAs in HPC and data centers. In a broader view, even though GPUs are still the dominant accelerators, and AI-specific hardware is growing fast, but FPGAs also have recently gained attraction.

There are several reasons that FPGAs are emerging in HPC and data centers. First, in some cases, they are more cost and energy efficient compared to CPUs and GPUs, which is one of the most important factors in HPC centers nowadays.
Second, FPGAs are capable of direct I/O connection. For example they can directly attach to the network without host intervene through dedicated network stack implemented on FPGAs. 
Third, it enables spatial programming paradigm, e.g., data flow implementation, to reduce data movement (from memory to compute units) compared to the traditional control-based procedural programming~\cite{Licht2022PythonDesign}. 
Fourth, FPGAs as re-configurable hardware provide more flexibility to application developers to have HW-SW co-design and implement domain specific applications with their own constraint metrics. Although, there has been recently enormous effort to mitigate the drawback of programmability of these devices to software developers and end users, but in most cases it is still a challenge to extract good performance out of it and to yield an optimized implementation of an algorithm.

Major data centers such as Microsoft, Alibaba, Amazon, Baidu, Huawei, etc. benefit from FPGAs in their infrastructures~\cite{firestone2018azure,PutnamAServices,caulfield2016cloud,ernst2020competing,xilinx_alibaba}. Some of them only use FPGAs for their internal developed applications. For example in Microsoft Catapult project, FPGAs are used in Microsoft Bing search service by placing a re-configurable logic layer (i.e., FPGAs) between network switches and servers~\cite{PutnamAServices,caulfield2016cloud}. 
On the other hand, some of data centers expose FPGAs as a service to application developers, e.g., AWS. There is an overview in~\cite{Bobda2022TheCloud} of existing academic and commercial efforts of using FPGA acceleration in data centers. They discussed different aspects from architectures, scalability, abstractions to middleware, applications, security and vulnerability of these devices. 

On HPC side, we can also see that FPGAs are emerging as a different type of accelerators. For instance, Fugaku extends its supercomputer center with a scalable FPGA-cluster system~\cite{Sano2023ESSPER:Fugaku}. 
Under AMD university program~\cite{amd_hacc}, some research institutes such as Paderborn University, ETH Zurich, University of California, Los Angeles (UCLA), University of Illinois at Urbana Champagne (UIUC) and National University of Singapore (NUS) deploy Heterogeneous Accelerated Compute Clusters (HACCs).  All these clusters support adaptive computing by incorporating FPGAs in their compute nodes to accelerate scientific applications.   
On a different line of research, Intel and Vmware in collaboration with the University of Toronto, University of Texas at Austin, Carnegie Mellon University initiate Crossroads 3D-FPGA Academic Research Center~\cite{crossroads_fpga}. Their ambition is to define a role for FPGAs as a central function in future data center servers. All these activities indicate the importance role of FPGAs in the future of HPC ecosystem and data centers.

Along with the rapid adaptation of FPGA technology in HPC and data centers, whether it becomes a dominant accelerator is still an open question. One important factor will be the economic advantage, whether they provide more performance with less energy and hence cost for a variety of applications?


%%%%%%%%%%%%%%%%%%%%%%%%% Rephrasing above information with a story line:

In this section, we overview the research and activity related to FPGAs in HPC and data centers. The entrance of FPGAs in HPC and data centers bring a fundamental question: in which position FPGAs can play a significant role in the workflow of HPC and data centers?
%Where in the system design and in which role FPGAs can be beneficial in HPC and data centers workflow?
One historical position of FPGAs in this ecosystem is in the network and communication. This is due to the capability of these devices to have direct I/O connection (without host intervene) to attach to network components (e.g., switches and routers) through dedicated network stack implemented on FPGAs. For example in Microsoft Catapult project, FPGAs are used in Microsoft Bing search service by placing a re-configurable logic layer (i.e., FPGAs) between network switches and servers~\cite{PutnamAServices,caulfield2016cloud}. 

Another straight forward answer to the role of FPGAs is to be as a new type of accelerators sitting along with CPUs and GPUs in HPC and data centers compute nodes. This is due to re-configurability and flexibility of these devices which enables HW-SW co-design and implementation of domain specific applications with their own constraint metrics. Moreover, FPGAs as accelerators enable spatial programming paradigm, e.g., data flow implementation, to reduce data movement (from memory to compute units) compared to the traditional control-based procedural programming~\cite{Licht2022PythonDesign}. 
As an instance of FPGAs as accelerators, we can see that Fugaku extends its supercomputer center with a scalable FPGA-cluster system~\cite{Sano2023ESSPER:Fugaku}. 
Another example is AMD university program~\cite{amd_hacc} where some research institutes such as Paderborn University, ETH Zurich, University of California, Los Angeles (UCLA), University of Illinois at Urbana Champagne (UIUC) and National University of Singapore (NUS) deploy Heterogeneous Accelerated Compute Clusters (HACCs).  All these clusters support adaptive computing by incorporating FPGAs in their compute nodes to accelerate scientific applications.   
With all the efforts have been done to mitigate the drawback of programmability of these devices to software developers and end users, but in most cases it is still a challenge to extract good performance out of it and to yield an optimized implementation of an algorithm.

Intel and Vmware in collaboration with the University of Toronto, University of Texas at Austin, Carnegie Mellon University initiate Crossroads 3D-FPGA Academic Research Center to re-think and find a permanent solution for this question~\cite{crossroads_fpga}. Their ambition is to define a fix role for FPGAs as a central function in future data center servers. In their perspective, FPGAs will be at the heart of the servers as data movement and transformation engine between network, traditional compute units, accelerators and storage.

All these activities indicate the importance, but still ambiguous role of FPGAs in the future of HPC ecosystem and data centers. There is an overview in~\cite{Bobda2022TheCloud} of existing academic and commercial efforts of using FPGAs in data centers. Although this is still an open question, and different ad-hoc solutions have been proposed, but we can say with certainty that one important factor will be the economic advantage; i.e., whether they provide more performance with less energy and hence cost for a variety of applications?

In the rest of this section, we overview the landscape of the FPGA research within the Netherlands in four categories: big data processing and analytics, distributed computing, optical hardware communication and high performance computing.
\fi

%%%%%%%%%%%%%%%%%%%%%%%%%%%%%%%%%%%%%%%%%%A bit Shorter general background
In this section, we overview research and activities related to FPGAs in HPC and data centers. The use of FPGAs in HPC and data centers raises a fundamental question: in which position FPGAs can play a significant role in the workflow of HPC and data centers?
%Where in the system design and in which role FPGAs can be beneficial in HPC and data centers workflow?
One historical position of FPGAs in this ecosystem is in the network and communication. 
This is due to the direct I/O connection capabilities of these devices, allowing them to attach to network components (e.g., switches and routers) through a dedicated network stack directly implemented on FPGAs.
In the Microsoft Catapult project~\cite{caulfield2016cloud, PutnamAServices}, for instance, FPGAs are used in the Microsoft Bing search service as %by placing 
a re-configurable logic layer %(i.e., FPGAs) 
between network switches and servers. %~\cite{PutnamAServices,caulfield2016cloud}. 

Another straightforward answer to this question is to deploy %place 
FPGAs as dedicated %a new type of 
accelerators/co-processors. % along with CPUs and GPUs in HPC and data centers. 
Due to their re-configurability and flexibility, % of these devices, 
FPGAs enable hardware-software co-design and implementation of domain specific applications. Moreover, FPGAs as accelerators facilitate %the % enable 
spatial programming, % paradigm, 
e.g., dataflow implementations, to reduce data movement (from memory to compute units) compared to the traditional, control-based procedural programming~\cite{Licht2022PythonDesign}. 
As an instance of FPGAs as accelerators, Fugaku extends its supercomputer center with a scalable FPGA-cluster system~\cite{Sano2023ESSPER:Fugaku}. 
Another example is through the AMD university program~\cite{amd_hacc}, where some research institutes %all 
around the world deploy Heterogeneous Accelerated Compute Clusters (HACCs).  %All 
These clusters support adaptive computing by incorporating FPGAs in their compute nodes %in order 
to accelerate scientific applications.
Despite substantial efforts to improve the programmability of these devices for software developers and end users, achieving %extracting good 
high performance through %and achieving
an optimized implementation of an algorithm remains a significant challenge in most cases.
%Although there has been enormous effort to mitigate the programmability of these devices to software developers and end users, but in most cases it is still a challenge to extract good performance and to yield an optimized implementation of an algorithm.
Intel and Vmware, in collaboration with %some 
research institutes and universities, established the  Crossroads 3D-FPGA Academic Research Center~\cite{crossroads_fpga} to re-consider and find a permanent solution for this question. Their ambition is to define a fixed role for FPGAs as a central function in future data center servers. From %In 
their perspective, FPGAs will serve as %be at 
the core %heart 
of %the 
servers, acting as data movement and %data 
transformation engines between the network, traditional compute units, accelerators, and storage.


The aforementioned %All these 
activities indicate the important, yet %but still 
ambiguous role of FPGAs in the future of HPC ecosystems and data centers. ~\citet {Bobda2022TheCloud} provide %There is 
an overview %in 
of existing academic and commercial efforts of employing %using FPGAs 
in data centers. Among the commercial efforts, we observe that major data centers such as Microsoft, Alibaba, Amazon, Baidu, and Huawei %, etc. 
benefit from FPGAs in their infrastructures~\cite{firestone2018azure,PutnamAServices,caulfield2016cloud,ernst2020competing,xilinx_alibaba}. 
Although this is still an open question, and various %different 
ad-hoc solutions have been proposed, %but 
%can say with certainty that 
one important factor will be the economic advantage; %Specifically, 
it will depend on whether these solutions can deliver more performance with less energy consumption and lower costs across a range of applications.
%i.e., whether they provide more performance with less energy and hence cost for a variety of applications? 
The rest of this section presents an %, we 
overview of the FPGA research landscape %of the FPGA research with
in the Netherlands, organized into four categories: big data processing and analytics (\ref{sec:big-data-processing-analytics}), distributed computing (\ref{distcomp}), optical hardware communication (\ref{opthwcom}) and high performance computing (\ref{sec:high-performance-computing}).





%%%%%%%%%%%%%%%%%%%%%%%%%%%%%%%%%%%%%%%%%%%%%%%%%%%%%%%%%%%%%%%%%%%%%%%%
% Other strong points of FPGAs:
% - Interfacing in general (e.g. Optical communication)
% - Network attached accelerators https://www.researchgate.net/publication/373405337_FPGA-Based_Network-Attached_Accelerators_-_An_Environmental_Life_Cycle_Perspective
% - SmartNICs https://www.intel.com/content/www/us/en/products/details/fpga/platforms/smartnic.html

\subsection{Big data processing and analytics} \label{sec:big-data-processing-analytics}
% This section discusses FPGAs as data center accelerators for big data processing and analytics. A relatively large amount of papers have been published on big data processing and analytics by Dutch organizations and in collaboration with Dutch organizations. 

% \paragraph{Background}
Several studies in Dutch academia 
 have assessed the domain of big data processing and analytics~\cite{Hoozemans2021FPGAOpportunities, Peltenburg2021GeneratingArrow, Rellermeyer2019TheProcessing, Fang2020In-memorySurvey}, identifying opportunities for FPGA accelerators %in this domain 
 and describing the challenges faced in the wide adoption of FPGA technology. \citet{Peltenburg2021GeneratingArrow} identify %the following %The 
 %main challenges: %identified
 %\cite{Peltenburg2021GeneratingArrow} include : 
 %\textbf{
 the programmability %} 
 of the accelerators, %\textbf{
 the portability %} 
 of the implementation, %\textbf{
 the interface design %} 
 to the data, and %\textbf{
 the infrastructure %} 
 for data movement to/from the accelerator and across % and between 
 %\textit{
 kernels %} 
 running on %in 
 the accelerator, as the main challenges. Solutions %are proposed 
 leveraging various existing technologies have been proposed, e.g., Apache Spark~\cite{ApacheSpark}, Apache Arrow Flight~\cite{ArrowFlight}, the IBM POWER architecture~\cite{7924241}, and OpenCAPI~\cite{OpenCAPI}, while applications 
%Application 
of FPGA accelerators in this domain %, in the Dutch research community, are found in the domains of 
involve database search~\cite{Fang2020In-memorySurvey}, real-time data analysis~\cite{Chrysos2019DataNode}, graph-based processing~\cite{Iosup2023GraphContinuum, Prodan2022TowardsEurope}, high-frequency trading~\cite{Chen2021FPGAAlgorithm}, DNA analysis~\cite{Voicu2019SparkJNI:Spark}, and machine learning~\cite{Rellermeyer2019TheProcessing}.

%: I/O challenges, data format specifications, (de-)compression, applications in graph processing, data base searches etc. 

%On fast retrieval of big data from memory and the communication of big data between hardware nodes through FPGAs. Also some applications of big data processing using FPGAs.

\subsubsection*{\bf{Research topics}}
Several challenges in using FPGAs effectively as accelerators for big data processing and analytics have been addressed by the Dutch research community.
% \begin{itemize}
%     \item \textbf{Interface design and Infrastructure}. Many data structures used in data bases are not well matched with the architecture of an FPGA, thus making processing on an FPGA inefficient. Apache Arrow Flight organizes data movement in a coherent and transparent way across various systems and applications. Fletcher \cite{Peltenburg2021GeneratingArrow}, \cite{Ahmad2022BenchmarkingMicroservices} extends Apache Arrow Flight towards FPGA and defines inter kernel infrastructure between processing kernels implemented in FPGA. Complementary work provides (on-line) data conversion from the often used Parquet \cite{Peltenburg2020BattlingFPGA} and JSON \cite{Peltenburg2021TensAccelerators} formats to Arrow. 
    
%     \item \textbf{Frameworks and tooling}. Addressing \textbf{Programmability}, \textbf{Portability}, \textbf{Interface design} and \textbf{Infrastructure} challenges. Several frameworks are developed to ease programming of FPGA accelerators for big data processing and analytics. \textit{Fletcher} integrates FPGA accelerators with tools and frameworks that use Apache Arrow in their back-ends \cite{Peltenburg2019Fletcher:Arrow}.
%     The open stream-oriented specification Tydi-spec \cite{Peltenburg2020Tydi:Streams} and language Tydi-lang \cite{Tian2022TydilangAL} are developed to specify and implement complex, dynamically sized data structures onto hardware streams.
%     \textit{SparkJNI} enables heterogeneous CPU - FPGA systems based on the Apache Spark unified engine for large-scale data analytics \cite{Voicu2019SparkJNI:Spark}.
%     The work by Abrahamse et al. \cite{Abrahamse2022Memory-DisaggregatedApplications} extends the ThymesisFlow \cite{9252003} memory disaggregration system with a framework leveraging IBM POWER9 and FPGA accelerators.
    
%     \item \textbf{Compression and Decompression}. Addressing the \textbf{Infrastructure} challenge. Both data storage size as well as data movement bandwidth from storage to data processor impose significant challenges in the efficient deployment of accelerators. Data compression is used to reduce both the data storage size and bandwidth requirements. However, compression and decompression of data requires a significant amount of resources. Efforts have been made to (de-)compress data on FPGA either to process the data directly on FPGA or to send them to another component in a system for further processing or storage. Implementations are made based on the Snappy \cite{Fang2019AModel}, LZ77 \cite{Fang2020AnLogic} and Zstd \cite{Chen2021FPGAAlgorithm} (de-)compression algorithms. Also, an energy-efficient co-processor, supporting a range of decompression algorithms, was designed and tested on FPGA \cite{Hoozemans2021EnergyASIP}.
% \end{itemize}
\paragraph{Interface design and infrastructure} Many data structures used in databases do %are 
not map well to the architecture of an FPGA, for example the alignment of data format or the method of data retrieval, %matched with the architecture of an FPGA, 
thus making processing on an FPGA inefficient. Apache Arrow Flight~\cite{ArrowFlight} organizes data movement in a coherent and transparent way across various systems and applications. Fletcher~ \cite{Peltenburg2021GeneratingArrow, Ahmad2022BenchmarkingMicroservices} extends Apache Arrow Flight with FPGA support %for %towards 
%FPGAs 
and defines inter-kernel infrastructure between 
processing kernels implemented in FPGA. Complementary work provides (on-line) data conversion from the widely %often 
used Parquet \cite{Peltenburg2020BattlingFPGA} and JSON \cite{Peltenburg2021TensAccelerators} formats to Arrow. 

\paragraph{Frameworks and tooling} %Addressing %\textbf{Programmability}, \textbf{Portability}, \textbf{Interface design} and \textbf{Infrastructure} challenges. 
Several frameworks have been %are 
developed to ease the programming of FPGA accelerators for big data processing and analytics. %\textit{
Fletcher %} 
integrates FPGA accelerators with tools and frameworks that use Apache Arrow as their back-end~\cite{Peltenburg2019Fletcher:Arrow}.
The open stream-oriented specification Tydi-spec \cite{Peltenburg2020Tydi:Streams} and language Tydi-lang \cite{Tian2022TydilangAL} are developed to specify and implement complex, dynamically sized data structures onto hardware streams.
%\textit{
SparkJNI %} 
enables heterogeneous CPU-FPGA systems based on the Apache Spark unified engine for large-scale data analytics \cite{Voicu2019SparkJNI:Spark}, while 
%The work by Abrahamse et al. 
\citet{Abrahamse2022Memory-DisaggregatedApplications} extend the ThymesisFlow \cite{9252003} memory disaggregration system with a framework leveraging IBM POWER9 and FPGA accelerators.

\paragraph{Compression and decompression} %Addressing the \textbf{Infrastructure} challenge. 

Both the data storage size and the bandwidth required to move data from/to storage %to the processor 
present significant challenges in efficiently deploying accelerators. Data compression is used to mitigate these challenges by reducing both storage size and bandwidth requirements. However, the processes of 
compression and decompression require considerable resources, and efforts have been undertaken to (de-)compress data on FPGAs, either to enable direct data processing on the FPGA itself, or to facilitate data transfers to another system component for further processing or storage. Solutions %Implementations are made 
based on various (de-)compression algorithms have been presented, such as Snappy~\cite{Fang2019AModel}, LZ77~\cite{Fang2020AnLogic} and Zstd~\cite{Chen2021FPGAAlgorithm}. % (de-)compression algorithms. 
%In addition, 
\citet{Hoozemans2021EnergyASIP} present an energy-efficient, FPGA-based co-processor that supports several %a range of 
decompression algorithms. %, was designed and tested on FPGA .

\subsubsection*{\bf{Future directions}}

Research to develop frameworks that enable the efficient use of FPGA accelerators for big data processing and analytics is ongoing. By adopting high-level workflows tailored to these tasks, FPGA accelerators are becoming increasingly applicable within general data center infrastructures and applications. 
%Many work is ongoing on frameworks enabling efficient use of FPGA accelerators for big data processing and analytics. With a higher level workflow suited to big data processing and analytics, FPGA accelerators are more likely to be applied in general data centre infrastructure and applications. 
We see the work referred to in section \ref{sec:big-data-processing-analytics} being continued, %especially the work from the Accelerated Big Data Systems group at TU Delft with IBM and AMD,
as well as being extended with other partners in the industry %, e.g.,  %such as Voltron Data
%for example in 
\cite{10.1145/3624062.3624541, 10305451, Reukers2023AnIR, groet2024leveraging}.
Furthermore, one can not overstep the current rise of ML and AI, which, when applied to big data processing and analytics \cite{Rellermeyer2019TheProcessing} can benefit from FPGA acceleration \cite{10.1145/3613963}. The above listed technologies being developed in the Netherlands can enable the use of FPGAs as accelerators for ML and AI in big data analytics. 

%%%%%%%%%%%%%%%%%%%%%%%%%%%%%%%%%%%%%%%%%%%%%%%%%%%%%%%%%%%%%%%%
\subsection{Distributed computing}
\label{distcomp}
% This section discusses the topic of distributed and decentral computing to accelerate computations through the use of (multiple) FPGAs.

% \subsubsection{Background}
Distributed computing involves the deployment of multiple computing nodes in parallel to increase performance and solve large computational problems. %is a common approach where either we hit the resource limit within a compute node due to large problem size, or we aim at increasing the performance by using different compute resources in parallel. 
%Distributed computing is a natural approach to mitigate the challenge of resources limitation within a physical compute node. 
%When we encounter a situation where the size of the problem is much larger to keep in one compute node, distributing different computation elements to different compute nodes is an intuitive solution. Moreover, we can also distribute the computation workflow on different compute nodes in order to increase performance. 
%In contrast for CPUs and GPUs which use DRAM-based cache-hierarchy memory structure, FPGAs use various memory technologies such as LUTs, BRAM, URAM, HBM, etc. which have lower latency and higher bandwidth compared to DRAM. 
%Thus, FPGAs appear to be more efficient in memory-bound computation tasks. 
While %The field of 
research on distributed computing involving CPU and GPU nodes is well established, %for CPU, and also GPU nodes is not a new area. However, as 
the emergence of FPGAs %are emerging 
as a new type of computational resources and accelerators within data center infrastructures introduces a new and challenging area of research. %, the research in this direction is new and challenging. 
The Dutch academia has mainly focused on applications that use distributed multi-FPGA systems for %
%The applications that in Dutch academia are investigated using distributed multi-FPGAs are 
large-scale graph processing~\cite{Sahebi2023DistributedFPGAs} and deep neural networks (DNNs)~\cite{Alonso2021Elastic-DF:Partitioning}.  

\subsubsection*{\bf{Research topics}}
%There are 
Several research topics have been %that are 
investigated by Dutch researchers. 

% \begin{itemize}
%     \item \textbf{Communication overhead}. Reducing communication is a key factor in distributed computing, and in particular in multi-FPGA systems. By reducing communication overhead, computation time and latency reduces and efficiency increases. In order to reach this goal, researchers propose interconnection frameworks to establish flexible, reliable, efficient and custom communication protocols in multi-FPGA systems~\cite{salazar2020plasticnet,Salazar-Garcia2021PlasticNet+:Transceivers,Salazar-Garcia2022AApplications}. In addition to reducing latency, these proposed frameworks are designed to work with different topology schemes, and different FPGA technologies.      
    
%     \item \textbf{Partitioning and performance scaling}. 
%     In order to increase performance in multi-FPGAs systems, researchers propose an open-source distributed resource partitioning and allocator tool on FPGAs for data flow architectures targeting DNN inference which works in conjunction with FINN compiler. They demonstrate their methodology enables super-linear scaling of throughput, by benefiting from model parallelism and direct FPGA-FPGA communication~\cite{Alonso2021Elastic-DF:Partitioning}.
%     Another different research proposes a (multi-FPGA) framework to process large-scale graph processing. The framework uses an offline partitioning mechanism, and it uses Hadoop to map the graph into the underlying hardware. They show that graph partitioning using FPGA architecture results in better performance on large graphs included millions of vertices and billions of edges. Their results indicate a significant speed-up compared to the state-of-the art CPU, GPU and FPGA solutions~\cite{Sahebi2023DistributedFPGAs}.  


    
% \end{itemize}
\paragraph{Communication overhead} Reducing communication is a key factor in distributed computing, and in particular in multi-FPGA systems. By reducing communication overhead, computation time and latency reduces and efficiency increases. To reach this goal, researchers propose interconnection frameworks to establish flexible, reliable, efficient and custom communication protocols in multi-FPGA systems~\cite{salazar2020plasticnet,Salazar-Garcia2021PlasticNet+:Transceivers,Salazar-Garcia2022AApplications}. In addition to reducing latency, these %proposed 
frameworks are designed to work with various %different 
topology schemes and different FPGA technologies.

\paragraph{Partitioning and performance scaling} %In order 
To increase performance in multi-FPGA systems, \citet{Alonso2021Elastic-DF:Partitioning} %researchers have 
propose an open-source, distributed resource partitioning and allocator tool on FPGAs for data flow architectures targeting DNN inference; it %which 
works in conjunction with the FINN compiler~\cite{umuroglu2017finn}. The authors show that %demonstrate 
their methodology enables super-linear scaling of throughput by benefiting from model parallelism and direct FPGA-FPGA communication. %~\cite{Alonso2021Elastic-DF:Partitioning}.
\citet{Sahebi2023DistributedFPGAs} %Another different research 
propose a (multi-FPGA) framework for %to process 
large-scale graph processing. The framework uses an offline partitioning mechanism, and relies on %it uses 
Hadoop to map the graph into the underlying hardware. The authors show that graph partitioning using an FPGA architecture results in better performance on large graphs that include millions of vertices and billions of edges. Their results indicate %a 
significant speed-ups over %compared to the 
state-of-the art CPU, GPU, and FPGA solutions.


\subsubsection*{\bf{Future directions}}

There are several %different 
challenges in distributed computing using multi-FPGA systems, thereby %which 
necessitating %requiring 
further research in this direction. For instance, overcoming communication barriers and designing protocols for FPGA-FPGA communication is an ongoing research domain. Moreover, at the application level, developing (standard) MPI-like collective communication libraries for multi-FPGA systems would be beneficial. %desired. 
Also, %We also need 
more case studies are needed in order to investigate and design efficient partitioning and workload distribution schemes %and %assigning different 
%distribute computational tasks 
for FPGA resources. Therefore, to bring ease-of-use and automation for distributed computing on FPGAs, developing libraries and tools  is crucial. 


\iffalse
\subsubsection{PlasticNet: A low latency flexible network architecture for interconnected multi-FPGA systems; PlasticNet+: Extending multi-FPGA interconnect architecture via Gigabit transceivers}
The paper focuses on addressing the challenges of multi-FPGA system communication. They propose an extension over PlasticNet framework via flexible, efficient, reliable, custom protocol. PlasticNet framework is a FPGA interconnect architecture of processing units for both within a board or among neighboring FPGA boards. Their extended proposal improves PlasticNet over area, channel overhead and latency. They evaluate their approach using a ring-based topology of Zynq ZC706 FPGA boards. They report the best-case latency of 300 ns which is half the latency of an Ethernet 10G link. Another advantage of the proposed approach is the adaptability to a wider range of interconnect topologies.

\subsubsection{Elastic-DF: Scaling Performance of DNN Inference in FPGA Clouds through Automatic Partitioning}
Power dissemination and multi-tenancy are two important factors for data center operators. FPGAs gains attraction to data centers, in particular for Deep Neural Networks (DNN) inference applications, as they are well aligned with the two factors. However, it is challenging as DNN inference on FPGA requires the right choice of architecture and a set of implementation tools. Therefore, the authors proposed an open-source distributed resource partitioning and allocator tool on FPGAs for data flow architectures targeting DNN inference which works in conjunction with DFA compiler FINN. They experiment their approach on FPGA cluster at ETH Zurich. It includes four FPGA-equipped nodes; a mix of Alveo U250 and U280 (10 cards in total). The FPGAs are connected using 100 Gb/s interfaces. They use different FPGA accelerator deployment software: XRT, PYNQ, Jupyter Lab. Dask, and InAccel Coral. They evaluated several DNN accelerator implementations of FINN-generated DF accelerators for quantized MN and RN-50 classifiers. They demonstrate their methodology enables super-linear scaling of throughput, by benefiting from model parallelism and direct FPGA-FPGA communication over 100 G bps Ethernet connection. Concretely, they show 44\% latency decrease on U280 for ResNet-50, and 78\% throughput increase on U200 and U280 for MobileNetV1.

\subsubsection{A custom interconnection multi-FPGA framework for distributed processing applications}
One of the main challenges in FPGA clusters is to reduce communication overhead between network elements to reduce computation time and maximize efficiency of processing elements on FPGAs. Therefore, the authors of the paper propose a multi-FPGA interconnection framework targeting distributed applications. Thus, they build multi-FPGA systems included 5 Zynq ZC706 FPGAs over their custom network. They assume an application can be accelerated by decomposing its computation and distributed into different processing elements on a multi-FPGA machine. The authors show the effectiveness of their framework using matrix multiplication algorithm. With the aggregated bandwidth of 25 Gbps per FPGA, their framework shown the latency of 200 ns, an efficiency of 97.25\% and throughput of 21.4 GFLOPS. Another advantage of their approach is portability of the proposed network interconnect to newer generation of FPGAs.

\subsubsection{Distributed large-scale graph processing on FPGAs}
Large-scale graph processing is challenging and causes performance degradation due to irregular structure and memory access on both CPUs and GPUs. The authors propose a FPGA engine, as part of a framework to overlap and hide data transfers in order to maximize utilization of FPGA accelerator. The framework uses an offline partitioning mechanism, and it uses Hadoop to map the graph into the underlying hardware. They show that graph partitioning using FPGA architecture results in better performance on large graphs included millions of vertices and billions of edges. They benefit from the partitioning scheme in GridGraph library, but on FPGA instead. They use Xilinx Vivado HLS toolchain for their implementation on Alveo U250 card. Their optimized implementation of the PageRank for a single FPGA outperforms state-of-the art CPU, GPU and FPGA solutions: a speed up to GridGraph by 2x, Cugraph by 4.4x and VITIS LIB by 26x. Even when the size of graphs limit the performance of a FPGA, their approach shows a speed up about 12x using multi-FPGAs. 
\fi
%%%%%%%%%%%%%%%%%%%%%%%%%%%%%%%%%%%%%%%%%%%%%%%%%%%%%%%%%%%%%%%%%%%%%%%5
\subsection{Optical hardware communication}
\label{opthwcom}
% This section examines the use of FPGA-based systems in optical hardware communication, a field that is gaining traction for its potential to revolutionize data center network (DCN) infrastructures.

% \subsubsection{Background}
Optical hardware communication is at the forefront of addressing the critical challenges faced by contemporary data center network (DCN) infrastructures, such as bandwidth limitations, latency issues, and scalability concerns. Optical communication is a viable alternative to conventional electrical data pathways, offering significant improvements in terms of efficiency and performance. The integration of FPGAs into optical communication systems has been a key development, providing the necessary flexibility and speed for dynamic network reconfiguration and management.

\subsubsection*{\bf{Research topics}}
The exploration of optical hardware communication utilizing FPGAs encompasses a variety of innovative research topics covered by Dutch organizations.
\paragraph{Optical wireless datacenter networks}
%Implementing semiconductor optical amplifier (SOA)-based wavelength selectors and arrayed waveguide grating routers (AWGRs) controlled by fast FPGA-based switch schedulers. 
\citet{Zhang2022Low-LatencyRouter} have developed an optical wireless (OW)-DCN architecture that promises enhanced flexibility and scalability for DCNs, supporting high-speed optical packet-switching transmissions. FPGA-based switch schedulers are used for control of the implementation based on semiconductor optical amplifier (SOA)-based wavelength selectors and arrayed waveguide grating routers (AWGRs).

\paragraph{Disaggregated optical networks}
The DACON project~\cite{Guo2022DACON:Invited} introduces a Disaggregated, Application-Centric Optical Network that utilizes hybrid optical switches and FPGA-based controllers, resulting in improved application performance and reduced latency.

\paragraph{Low-latency edge networks}
The Electro-Optical Communication group at TU/e has proposed an edge data center network architecture that employs photonics and FPGA-based supervisory channels to achieve microsecond-time control and deterministic latency \cite{Santana2023SOA-BasedApplications}.

\paragraph{Nanosecond optical switching}
A novel optical switching and control system has been designed to address the bandwidth bottlenecks of electrical switching, featuring a distributed network architecture with optical label channels and the Optical Flow Control (OFC) protocol \cite{Xue2022NanosecondNetworks}.

\paragraph{Hybrid datacenter architectures} The HiFOST DCN architecture~\cite{Yan2018HiFOST:Switches} integrates flow-controlled fast optical switches with modified top-of-the-rack switches, offering substantial improvements in latency and cost efficiency.

\paragraph{Beyond 5G networks} \citet{Santana2022TransparentApplications} present a new Edge Cloud Network design %has been put forward, 
that uses %utilizing 
FPGA-based controllers for rapid reconfiguration of optical networks, catering to the low-latency requirements of 5G applications and beyond.  %\cite{Santana2022TransparentApplications}.

% \begin{itemize}
%     \item \textbf{Optical Wireless Data-Center Networks}. Implementing semiconductor optical amplifier (SOA)-based wavelength selectors and arrayed waveguide grating routers (AWGRs) controlled by fast FPGA-based switch schedulers. Researchers have developed an optical wireless (OW)-DCN architecture that promises enhanced flexibility and scalability for DCNs, supporting high-speed optical packet-switching transmissions.\cite{Zhang2022Low-LatencyRouter}

%     \item \textbf{Disaggregated Optical Networks}. The DACON project introduces a Disaggregated, Application-Centric Optical Network that utilizes hybrid optical switches and FPGA-based controllers, resulting in improved application performance and reduced latency.\cite{Guo2022DACON:Invited}

%     \item \textbf{Low-Latency Edge Networks}. The Electro-Optical Communumuroglu2017finnumuroglu2017finnication group at TU/e has proposed an edge data center network architecture that employs photonics and FPGA-based supervisory channels to achieve microsecond-time control and deterministic latency.\cite{Santana2023SOA-BasedApplications}

%     \item \textbf{Nanosecond Optical Switching}. A novel optical switching and control system has been designed to address the bandwidth bottlenecks of electrical switching, featuring a distributed network architecture with optical label channels and the Optical Flow Control (OFC) protocol.\cite{Xue2022NanosecondNetworks}

%     \item \textbf{Hybrid Data Center Architectures}. The HiFOST DCN architecture integrates flow-controlled fast optical switches with modified top-of-the-rack switches, offering substantial improvements in latency and cost efficiency.\cite{Yan2018HiFOST:Switches}

%     \item \textbf{Beyond 5G Networks}. A new Edge Cloud Network design has been put forward, utilizing FPGA-based controllers for rapid reconfiguration of optical networks, catering to the low-latency requirements of 5G and beyond applications.\cite{Santana2022TransparentApplications}

% \end{itemize}

\subsubsection*{\bf{Future directions}}
% The field of optical hardware communication is poised for significant advancements, with ongoing research directed towards:
% \begin{itemize}
%     \item \textbf{Visible Light Communications (VLC)}. Efforts to mitigate LED nonlinearity have led to the development of a Legendre-polynomials-based post-compensator optimized for FPGA implementation, enhancing the bit rate efficiency of high-speed VLC systems.\cite{Niu2021LEDCommunications}
%     \item \textbf{Real-Time LED Modeling}. The introduction of a real-time FPGA-based implementation of a nonlinear LED model and post-compensator marks a substantial contribution to VLC technology, enabling high data rates over bandwidth-limited LEDs.\cite{Deng2022Physics-BasedImplementation}
%     \item \textbf{Concurrency-Aware Mapping in HPC}. A concurrency-aware mapping technique has been developed to reduce optical packet collisions in Architecture-On-Demand  (AoD) network infrastructures, improving buffer utilization and execution time degradation in HPC systems.\cite{Meyer2018OpticalPerformance}
% \end{itemize}

Through various ongoing developments, the field of optical hardware communication is poised for significant advancements. Efforts to mitigate LED nonlinearity have led to the development of a Legendre-polynomials-based post-compensator optimized for FPGA implementation, enhancing the bit rate efficiency of high-speed Visible Light Communications (VLC) systems \cite{Niu2021LEDCommunications}. The introduction of a real-time FPGA-based implementation of a nonlinear LED model and post-compensator marks a substantial contribution to VLC technology, enabling high data rates over bandwidth-limited LEDs \cite{Deng2022Physics-BasedImplementation}. 
A concurrency-aware mapping technique has been developed to reduce optical packet collisions in Architecture-On-Demand  (AoD) network infrastructures, improving buffer utilization and execution time degradation in HPC systems \cite{Meyer2018OpticalPerformance}.


%\cite{mendely.bib key}

%%%%%%%%%%%%%%%%%%%%%%%%%%%%%%%%%%%%%%%%%%%%%%%%%%%%%%%%%%%%%%%%%
\subsection{High performance computing}\label{sec:high-performance-computing}
% In this section, we discuss the research of utilizing FPGAs in HPC ecosystem by Dutch researchers.

%\subsubsection{Background}
Benefiting from FPGAs in %for 
HPC applications is an active research area. % nowadays. 
Even though GPUs remain %are still 
the most prevalent %dominant 
accelerator technology in HPC, and AI-specific hardware is being  increasingly adopted, %growing fast, but 
FPGAs are also %have recently gained attraction, and even we can observe (experimental) FPGA deployments on 
increasingly employed in HPC centers. 

\subsubsection*{\bf{Research topics}}
Dutch institutes have been involved in European projects, e.g., %such as 
ExaNeSt~\cite{Katevenis2018NextDevelopment} and MANGO~\cite{Flich2018ExploringApproach}, to %in 
design %ing of 
large-scale heterogeneous compute systems. We can observe 
the important role of FPGAs in these projects, facilitating network communication or accelerating execution. % as either considering in the network or as accelerators: 

% \begin{itemize}
%     \item \textbf{Architecture and system design}.
\paragraph{Architecture and system design}
    The %At 
    ExaNeSt European project~\cite{Katevenis2018NextDevelopment} deploys FPGAs %are proposed 
    as accelerators in % for 
    a European 
    exascale supercomputer based on low-cost, low-power %many 
    ARM cores. They also employ an FPGA-based testbed for a low-latency, high bandwidth unified Remote Direct Memory Access (RDMA) interconnect, and present %hey design 
    a custom FPGA-based switch to support inner-cabinet communications.
   The MANGO project~\cite{Flich2018ExploringApproach} aims at addressing the PPP (Performance, Power, and Predictability) space in HPC %: Performance, Power and Predictability 
   by exploring %and investigating 
   customizabe and deeply heterogeneous accelerators. Their hardware concept consists of General-purpose compute Nodes (GNs) with %, having 
   commercial accelerators such as Xeon Phi and NVIDIA GPUs, along with Heterogeneous Nodes (HNs). HNs are clusters of many-core chips coupled with customized heterogeneous computing resources, including high-capacity clusters of FPGAs.
    
    % \item \textbf{Programming languages, tools and applications}. 
\paragraph{Programming languages, tools, and applications}
    Within the ExaNeSt project, \citet{Katevenis2018NextDevelopment} %, they 
    design a novel microarchitecture as Top-of-Rack switches. In one of their experiment, they port the OpenCL kernels of the molecular dynamics simulator LAMMPS~\cite{plimpton1995fast} to FPGAs using HLS tools. 
    %They report that the use of an FPGA improves %the 
    %performance compared to using ARM cores, more than a factor of two.
    They report that running the kernel on an FPGA requires 0.56 seconds while the 4 ARM cores requires 1.3 seconds. That is an improvement of more than a factor 2 in speed up.
    Within the MANGO project, ~\citet{Flich2018ExploringApproach} target three applications with significant QoS aspects: 1) online video transcoding, 2) rendering for medical imaging, and 3) error correcting codes in communication. The MANGO project relies on LLVM~\cite{lattner2004llvm} and their programming model is an extension of %the 
    existing languages and libraries (e.g., OpenCL~\cite{opencl}) for HPC by integrating the expression of new architectural features as well as QoS concerns and parameters. This is achieved %They do it 
    by augmenting the runtime library API with new functions, pragmas and keywords to the existing HPC languages (e.g., clang C/C++ frontend). 
    
    % \item \textbf{Performance Models}. 
\paragraph{Performance models}
    Combining the advantages of reconfigurability, dataflow computation, and heterogeneity results in %yields 
    Reconfigurable Dataflow Platforms (RDPs) as a promising building block in %the 
    next-generation, large-scale high-performance machines. RDPs rely on %include 
    Reconfigurable Dataflow Accelerators (RDAs) to realize multiple streaming pipelines, each % which each 
    comprising many parallel operations. Due to the % such 
    heterogeneous hierarchy, 
    performance prediction of RDPs is very challenging, in particular to detect bottlenecks within reasonable time and accuracy. %Therefore, 
\citet{Yasudo2021AnalyticalPlatforms,Yasudo2018PerformancePlatforms} %Dutch researchers have been involved in a project to 
propose a performance estimate framework for reconfigurable dataflow applications %(i.e., RDPs), 
    named Performance Estimation for Reconfigurable Kernels and Systems (PERKS). %~\cite{Yasudo2021AnalyticalPlatforms,Yasudo2018PerformancePlatforms}. 
    It %PERKS 
    automatically extracts specific parameters from the application, the hardware, and the platform to calibrate the model. They use eight applications for their evaluation: AdPredictor (an online machine learning algorithm), N-body simulation, Monte Carlo simulation, sequence alignment, Asian option pricing, Jacobi solver, and Regression/regularisation solver. Their results show that PERKS achieves %performs an 
    accuracy of 91\% on these applications.
% \end{itemize}

\subsubsection*{\bf{Future directions}}
\iffalse
Determining the role of FPGAs in HPC necessitates %demands 
more research and %it 
raises several %many 
questions, such as 
1) Do we have to find a permanent position for FPGAs in HPC ecosystems to maximize their impact? % the most? 
If so, what would that position be, %Where would be that position 
from both architectural system design and application workflow perspectives?,
2) How can we bridge %fill out 
the gap between software developers and FPGA programming models and tools? Should we focus on HLS approaches or compiler-specific tools, or a combination of both?,
%Through HLS approach or compiler specific tools?
3) What types of HPC applications can benefit from FPGAs?, and last but not least 4) Are FPGAs cost-efficient in terms of energy and performance to warrant a permanent position in future HPC centers? 
%Whether FPGAs would be cost efficient in terms of energy and performance to dictate a permanent position in HPC centers? 
Further research and case studies are required to obtain %gain 
more insights in order to answer %address 
these questions. This future direction and the corresponding outcomes will indicate how important FPGAs will be in the future of HPC and data centers. 
\fi

Determining the role of FPGAs in HPC necessitates more research from both data center architecture design and FPGA programming model.
From a data center design perspective, the positioning of FPGAs in the architecture of HPC centers needs more investigation. This also depends on the targeted application workflow and how FPGA can impact the most.    
From a user perspective, the programmability of these devices is an important factor. Therefore, the gap between software developers and FPGA programming models and tools should be reduced further to use FPGA as mainstream HPC devices.

\iffalse
\subsubsection{Next generation of Exascale-class systems: ExaNeSt project and the status of its interconnect and storage development}
ExaNeSt European project merges industry and academia in the area of system cooling, storage, network and interconnect, and HPC applications. They aim to develop system-level interconnect and distributed non-volatile memory storage for a European exascale supercomputer based on low cost and power many ARM cores and computing accelerators implemented in programmable components (FPGAs). In this paper they explain the project in terms of hardware architecture and software stack development. The breakdown of the components in this project is as follows: 1) A low-latency, high bandwidth unified RDMA interconnect. They use FPGA-based testbed for this purpose. 2) Providing low-latency inter-process communication as needed on HPC workflows. 3) A novel distributed storage architecture. 4) A set of exascale scientific applications such as MonetDB and LAMMPS 5) Packaging and advanced cooling system.
The ExaNeSt interconnect has three components: 1) Network interface which bridges the processes that run on ARM cores with the communication layer of the network. 2) Intra-rack network IP based on APEnet which provides switching and routing features and manages communications over links. 3) A novel micoarchitecture as Top-of-Rack switches. They design a custom FPGA-based switch to support inner-cabinet communications.
The unit of the system is the Xilinx Zynq UltraScale+ FPGA integrating four 64 bit ARMv8 Cortex-A53 hard-cores running 1.5 GHz. 
In one of their experiment, they port the OpenCL kernels of LAMMPS to FPGA using HLS tools. They report the use of FPGA improves the performance significantly compared to using ARM cores, more than a factor of 2. 

\subsubsection{Exploring manycore architectures for next-generation HPC systems through the MANGO approach}
The paper explains the main approach and architectural solution, application scenario and software stack in MONGO project. The MANGO project aims at addressing the PPP space in HPC: Performance, Power and Predictability by exploring and investigating customizabe and deeply heterogeneous accelerators. They target three applications with significant QoS aspects: 1) Online video transcoding 2) Rendering for medical imaging 3) Error correcting code in communication. Their hardware concept consists of General-purpose compute Nodes (GNs), having commercial accelerators such as Xeon Phi and NVIDIA GPUs, along with Heterogeneous Nodes (HNs). HNs are clusters of manycore chips coupled with customized heterogeneous computing resources. Their deployment platform consists of 16 GNn with standard processors, e.g., Intel Xeon E5 and Kepler GPUs, and 64 HNs consists of ASIC ARM cores and high-capacity cluster of FPGAs. GNs and HNs are connected via infiniband. Their programming model is an extension of the existing languages and libraries for HPC by providing new architectural features and QoS concerns and parameters. They do it by augmenting the runtime library API with new functions, pragmas and keywords to the language.

\subsubsection{Analytical Performance Estimation for Large-Scale Reconfigurable Dataflow Platforms; Performance Estimation for Exascale Reconfigurable Dataflow Platforms}
Combining the advantages of reconfigurability, dataflow computation and heterogeneity yields Reconfigurable Dataflow Platforms (RDPs) as promising building block in the next generation of large-scale high-performance machines. RDPs include Reconfigurable Dataflow Accelerators (RDAs). Various hardware, computation, storage and communication elements are costumized for a specific hardware design to implement an algorithm. As such performance prediction of RDPs becomes very challenging, in particular to detect bottlenecks within reasonable time and accuracy. The authors of the paper propose a performance estimate framework for reconfigurable dataflow applications (i.e., RDPs), named Performance Estimation for Reconfigurable Kernels and Systems (PERKS). PERKS automatically extract specific parameters from the application, hardware and platform to calibrate the model. PERKS allows to predict performance of multi-accelerator systems using analytical model along with machine and application parameters. Their experimental setup is RDPs from Maxeler, known as Maxeler DFEs. A node of 12-core Intel Xeon CPU connects to 4 DFEs (via PCI Express). Each DFE has a Xilinx v6-SXT475 FPGA and 48 GB of DRAM. They use 8 applications for their evaluation: AdPredictor (an online machine learning algorithm), N-body simulation, Monte Carlo simulation, sequence alignment, Asian option pricing, Jacobi solver, and Regression/regularisation solver. Their results show that PERKS performs an accuracy of 91\% on current reconfigurable workloads.
\fi

%1. One paragraph summary (of each paper):
%        What is the contribution
%            system design/architecture (FPGA position), e.g. PCIe device, network attached accelerator etc.
%            Application/case study/technology development
%            In case of accelerator, what is the workflow. In case of infrastructure 
%            Product/research? How does this relate to research topics?
%               If it's a product, which research groups did contribute to it
%               open source?, actively maintained? 
%           Is the research used somewhere?
%        Advantages/Disadvantages
%        Forward look, where do we see this going in the future?

% Paragraph:
% Publication title
% short answers to above questions

 

%2. Merge/combine all ones into one (sub)section: Deadline March 31



\section{Programming Models and Tools}% (Tiziano, Steven, Christiaan) (CONT)}
\label{sec:programming}
% This section discusses models, tools, and techniques that facilitate porting of existing software to hardware accelerators. 

% \textcolor{blue}{Notes for Nikos/Sjoerd:
% \begin{itemize}
%     \item We don't have a lot of paper per subtopic, and inside each subtopic the papers are also quite different (in terms of goals). So we propose reducing the topic to just two: programming models/frameworks and performance prediction (or tools)
%     \item All the papers may have additional details embedded as latex comments
%     \item one of the select papers ( "Modeling FPGA-Based Systems via Few-Shot Learning") is actually a poster with a short abstract. We believe this is a subset of another paper ("LEAPER...") and therefore we suggest removing it.
%     \item Regarding the "Future direction": the current picture of the Dutch research in programming models and tools is quite scattered and with no clear (at least to us) direction. Out of the 14 papers, only 4 are led by Dutch institutions, and it is a bit hard to say on what Dutch research groups are leaders. At the moment, the corresponding subsections contain high-level overviews.
%     \item a comment that applies to both subtopics is that there are very few open-source tools, or even just publicly available artifacts, so we can make the call for action in this regard (probably it applies to other sections)
%     \item we noticed that there are some shared references with Sect. \ref{sec:big-data-processing-analytics} (Big data processing -- e.g., tydi) and Sect. \ref{sec:high-performance-computing} (High-Performance computing -- e.g., PERKS), so maybe it makes sense to link them together
%     \item you can remove these notes afterward
% \end{itemize}
% }


This section discusses approaches that boost developer productivity; Section~\ref{prog_models_frameworks} reviews programming models and frameworks that raise the abstraction level of describing hardware, while Section~\ref{perf_pred} presents 
techniques that predict performance of synthesized programs.

\subsection{Programming models and frameworks}
\label{prog_models_frameworks}
% This section presents developments of programming models and frameworks, including HLS-based, aimed at reducing development time for FPGA-based designs.

% \subsubsection{Background}
FPGA development is traditionally characterized by a steep learning curve, especially for non-experts. For this reason, High-Level Synthesis (HLS) tools and, more generally, high-level programming models and frameworks have been proposed to increase productivity by raising the abstraction level. 
HLS tools became commercial products in the early 2010s. Since then, they have been used in various application domains, including, but not limited to, deep learning, multimedia, graph processing, and genome sequencing ~\cite{Cong-2022}. HLS tools can reduce the average development time (up to two-thirds compared to RTL~\cite{Lahti-2019}). However, they still require considerable expertise to optimize the FPGA designs and achieve a Quality-of-Result that is on par with the one obtained through hardware description languages. For this reason, higher-level programming models and frameworks are now being proposed. They allow developers %user 
to describe hardware %write their 
%programs 
in a more convenient formalism (e.g., Clash~\cite{clash-2010,clash-website}, HeteroCL~\cite{heterocl-2019}, PyLog~\cite{pylog-2021}, and DaCe~\cite{dace-2021} which currently support Haskell, Python or a Python-embedded DSL), and %they %take care of 
automatically, or via user-provided hints, generate optimized %lower it into 
HLS/HDL descriptions. %and optimize the final design. 

%\subsubsection{Current research in the Netherlands}

\subsubsection*{\bf{Research topics}} Several papers have been published by Dutch organizations focusing on reducing development time for hardware design using HLS programming models and tools.

\paragraph{Abeto framework%: a Solution for Heterogeneous IP Management
}
% Reviewer: Tiziano
% Contribution:
%   - this is not a programming framework, rather a tool for IP management and workflow automation
%   - it should be general enough to be used for various application mains
%   - it is not open-source, probably still maintained
%   2. Advantages/Disadvantages, how to position this work related to state of the art: 
%       - in the paper the authors mention several other IP management applications, but they say these are usually "rigid" in the expected IP core format and require considerable effort

%Sanchez et al. propose Abeto 
\citet{Sanchez2022AbetoManagement} propose Abeto, a software tool for IP management and workflow automation. Historically, there has been no established standard for packing, documenting, and distributing IP core designs.  This prevents their re-usability, as each IP core has its unique learning curve and challenges for using them in an EDA toolchain. Abeto allows the user to operate in a unified manner with heterogeneous IP cores, and conveniently configure and launch the different stages of the IP workflow. To add an IP core, Abeto requires some auxiliary information to be provided: a database definition (containing information about the directory structure of the IP core) and a command dictionary (which includes the list of supported IP commands and how they must be executed). The tool has been validated against a subset of the ESA portfolio of IP cores\footnote{\url{https://www.esa.int/Enabling_Support/Space_Engineering_Technology/Microelectronics/ESA_HDL_IP_Cores_Portfolio_Overview}}, which constitutes a heterogeneous group of IP cores, demonstrating the tool's versatility.


\paragraph{%A Complete Open Source
Design flow for Gowin FPGAs}
% Reviewer: Christiaan
% Contributions
% - The contribution describes a framework to configure Gowin FPGAs using open-source tools
% - It is general purpose insofar that any application can make use of the new flow
% - It is an open source technology development, and contributions have been merged upstream
% - The work is analytical in that is describes the method for documenting the bitstream format.
%De Vos et. al.~
\citet{vos2020gowin} describe a method %and results 
to create an open-source design flow for the Gowin LittleBee family of FPGAs.
The design flow is based on well-known open-source tools such as Yosys and nextpnr, as well as the newly developed bitstream generator.
Architectural details of the FPGA family were documented using input fuzzing and comparing results from the existing closed-source vendor tool flow.
While the created open-source flow is capable of synthesizing a full RISC-V core, many aspects, such as DSPs, RAMs, and PLLs are currently unsupported. The authors report that documenting the bitstream format for all of these features is %will be 
the subject of future work.


%\paragraph{AEx: Automated High-Level Synthesis of Compiler Programmable Co-Processors -- \cite{Hirvonen2023AEx:Co-Processors}} 

\paragraph{AEx framework} 


% Reviewer: Tiziano
% Contributions:
% - a framework for overlays/FPGA design
% - seems general purpose but this is not clear
% - it is the result of a European project (https://fitoptivis.eu/) but there is no mention that this is opensource
\citet{Hirvonen2023AEx:Co-Processors} propose %in this paper 
AEx, a framework for automated High-Level synthesis of compiler programmable co-processors. AEx can be used to produce Application-Specific Instruction-Set (ASIP) architectures. ASIP processors have been proposed as a way to produce FPGA overlays starting from a software-programmable template. The program being executed can be easily changed, reducing design time and costs. The template being used by AEx is Transport Triggered Architecture (TTA).
% In case, a short explanation of TTAs
%TTAs are architectures in which programs have more low-level control of data transfers between processor functional units. This is in contrast with  directly control the internal transport buses of a processor. 
AEx includes heuristics for design space exploration and pruning, aimed at finding the best architecture able to satisfy real-time execution time and clock frequency constraints. The user can then choose the results that better fit their need (e.g., minimum resource utilization). Evaluation %The results 
shows how the tool is able to produce results in a reasonable amount of time, achieving %with 
performance close to that of the fixed-function implementations generated by HLS vendor tools such as AMD/Xilinx Vitis.


%\paragraph{Exploration of Synthesis Methods from Simulink Models to FPGA for Aerospace Applications}

\paragraph{Synthesis from Simulink Models to FPGA for Aerospace Applications}

% Reviewer: Tiziano
% Contributions:
% - a survey/analysis of methods to generate FPGA design from Matlab Simulink model (for space applications)
% - they considered a specific use case (Simulink)
% - nature of the work: comparative. It is not an actual framework, but rather a collection of suggestions
% - How to position this work wrt background? Not the first paper I believe to discuss the advantage of FPGA in aerospace. Maybe one of the few to discuss how to leverage HLS in a convenient way for aerospace engineers

Reconfigurable hardware is becoming an attractive solution for aerospace applications, thanks to its power efficiency and capabilities of in-flight configuration. Algorithms are usually expressed in model-based programming frameworks, e.g., Matlab Simulink, but turning them into low-level hardware description languages can be cumbersome. \citet{Curzel2023ExplorationApplications} analyze solutions to automatically synthesize Simulink models. Matlab already provides an automated method (HDL coder) to translate part of Simulink models into Verilog/VHDL, but this still requires a certain level of expertise. Therefore, the authors propose to apply HLS on the code generated by Matlab's Embedded Coder tool, further automatizing the design process. Experiments with three %different 
benchmarks show that this solution is more efficient than relying on HDL coder, and it does not require specific hardware expertise. % to the user.
% Future work: how to automatize HW/SW partitioning, quality of generated C code.

\paragraph{HLS optimizations for  post-quantum cryptography on Lattice FPGAs}
%\paragraph{Optimizing Lattice-based Post-Quantum Cryptography Codes for High-Level Synthesis}

% Reviewer: Tiziano:
% Contributions:
% - analysis and improvement of HLS based implementation of post-quantum crypto algorithms
% - they considered a specific use case
% - nature of the work: implementation. They claim they would open-sourced it, but I can not find it
% - How to position this paper: this paper does not introduce anything new, but applies known optimization to a specific application domain
%In this paper, Guerrieri et al.~
\citet{Guerrieri2022OptimizingSynthesis} discuss the process of porting Post-Quantum Cryptographic algorithms to an FPGA using HLS. While it can be reasonably straightforward %easy 
to port an existing CPU implementation to an FPGA, %the 
performance can be low and resource utilization %of resources 
is not optimal. The authors discuss how, applying well-known HLS-specific optimization techniques, the code can be rewritten to leverage the capabilities of HLS tools and produce more efficient designs, reducing the computation latency of up to two orders of magnitudes in specific cases.


%\paragraph{CONT Optimizing Industrial Applications for Heterogeneous HPC Systems: The OPTIMA Project Intermediate stage}
\paragraph{Optimizations for Heterogeneous HPC Systems (OPTIMA)}
% Reviewer: Christiaan
% Contributions
% - Documentation of the two HPC systems, and the results of porting certain applications
% - It's a case study for specific algorithms on specific HPC systems
% - The applications that are ported seem to be open-source and actively maintained
% - Positioning the paper: while the paper reports results, it does not demonstrate how these results compare to SOTA
%Theodoropoulos et al.~
\citet{Theodoropoulos2023optima} demonstrate the results of porting and optimizing industrial applications to two new heterogeneous HPC systems within the OPTIMA project. The results highlight the performance increase of using the available FPGA-based accelerators versus a pure software implementation running on the CPUs of the HPC system.


%\paragraph{The VINEYARD Framework for Heterogeneous Cloud Applications: The BrainFrame Case}

\paragraph{A Framework for Heterogeneous Cloud Applications (VINEYARD)}
% Reviewer: Christiaan
% Contributions:
% - A framework for deploying different parts of an application across multiple accelerators in a cloud environement
% - Though they consider multiple use-cases, they only work out a specific use case: spiking neural networks
% - Supposedly there is an open-source marketplace, http://www.accel-store.com/, but it returns error 503
% Positioning the paper: while the paper reports results, it does not demonstrate how these results compare to SOTA
%In this paper, Sidiropoulos et al.~
\citet{Sidiropoulos2018vineyard} describe a framework to accelerate different parts of an application across different accelerators, like GPUs and FPGAs. They demonstrate the utility of this framework by creating a platform for computational neuroscience, called BrainFrame. The BrainFrame platform allows one to simulate spiking neural networks, and depending on the number of neurons and their interconnectivity, certain combinations of accelerators achieved the shortest simulation times.

%\paragraph{Tydi: an open specification for complex data structures over hardware streams}
\paragraph{Mapping data structures to hardware streams (Tydi)}
%Pelterberg et.al.~
\citet{Peltenburg2020Tydi:Streams} describe a specification for mapping complex, dynamically sized data structures onto a fixed number of hardware streams.
%SV: see also data centre big data analytics section for more about Tidy

\subsubsection*{\bf{Future directions}}
Traditional FPGA programming has been done using Hardware Description Languages, which have a steep learning curve that does not favor adopting reconfigurable devices in the scientific and industry community. To address this issue, there is a collective effort to increase the level of abstraction for FPGA designs without compromising performance. Achieving this goal requires a multidisciplinary approach involving programming languages, compilers, and optimization techniques. HLS tools and high-level approaches are being used in various application domains.
Although the current direction in this field in the Netherlands is unclear, our analysis has pinpointed specific domains of interest within local research communities, such as aerospace and accelerated big data processing, that could benefit from more accessible programming methods for FPGA devices.

    % \item in the Netherlands:
    % \begin{itemize}
    % \item no winner application domain, even though aerospace is often considered (due to ESA)
    % \item TUD with Tydi (and the related lab research) seems to be investing a lot in architectures/hardware for accelerating big data processing
    % %SV: work is continued and implemented in Voltron Data ?
    % \end{itemize}
%\end{itemize}


\subsection{Performance prediction}
\label{perf_pred}
% This section describes tool to improve and speed up the performance prediction (e.g., area and efficiency) of FPGA designs.

% \subsubsection{Background}
FPGA design and development processes are time-consuming activities due to, among others, the very fine granularity reconfigurability of FPGA designs, which translates into a large design space and long synthesis time. For this reason, it is crucial to enable quick performance prediction of synthesized programs to improve early-stage design analysis and exploration, and performance debugging.
We can distinguish between two main types of performance prediction models: analytical and ML-based. Analytical models (such as HLscope+~\cite{hlscope-2017} and COMBA~\cite{comba-2020}) analyze the source code and use mathematical modeling to estimate performance and resource utilization. They are able to produce quick estimates at the cost of reduced accuracy. ML-Based models~\cite{oneal-2018, ustun-2020, Sun-2021}, on the other hand, %instead 
aim at improving prediction accuracy by considering device-specific features, but typically %often 
require long and expensive training procedures. 


%\subsubsection{Current research in the Netherlands}

\subsubsection*{\bf{Research topics}} Several papers have been published by Dutch organizations focusing on performance prediction of synthesized codes.

%\paragraph{LEAPER: Fast and Accurate FPGA-based System Performance Prediction via Transfer learning}

\paragraph{System Performance Prediction via Transfer learning (LEAPER)}

% Reviewr: Christiaan
% Contributions:
%  - The contribution describes a framework.
%  - It is general purpose insofar that any application can make use of the new flow, it does however only support C/C++ based entry.
%   - Technology development
%   - The work is analytical
%Singh et.al.~
\citet{Singha2022leaper} describe a method for predicting system performance and resource usage of FPGA accelerators using transfer learning.
They trained a performance predictor model for an edge/embedded FPGA, and used transfer learning so that the model can also be used for cloud/high-end FPGAs.
The method allows for design space exploration of mapping C/C++ programs to cloud/high-end FPGAs using HLS. The authors showed that it is 10x faster than the state of the art, achieving 85\% accuracy.

\paragraph{Modeling FPGA-Based Systems via Few-Shot Learning}
% Reviewer Tiziano
% This is a poster, and very few details are provided in the abstract
Machine learning based models are being proposed to provide fast and accurate performance predictions of FPGA-based designs. However, training these models is expensive, due to the time-consuming FPGA design cycle. %In this poster, 
%Singh et al.~
\citet{Singh2021ModelingLearning} propose a transfer-learning-based approach for FPGA-based systems, to adapt an existing ML model, trained for a specific device, to a new, unknown environment, reducing the training costs.


%\paragraph{CGRA-EAM—Rapid Energy and Area Estimation for Coarse-grained Reconfigurable Architectures}
\paragraph{%CGRA-EAM—Rapid 
Energy and Area Estimation for Coarse-grained Reconfigurable Architectures}
% Reviewer: Christiaan
% Contributions:
% - Analytical model for power and area of CGRAs, and the method to create said model
% - While the method to create the model is only applied to one CGRA architecture, the paper claims it should work for many different CGRAs as long as the RTL is available/can be generated.
% - It does not seem the code artifacts or models can be easily downloaded anywhere
% - Positioning the paper: This work is analytical vs a machine-learning approach, focus on CGRA (instead of FPGA or ASIC). 
Design space exploration is often required to achieve good Pareto points when creating reconfigurable architectures. %Wijtvliet et. al.~
\citet{Wijtvliet2021cgra} introduce the CGRA-EAM  model for energy and area estimation for CGRAs. It %which 
achieves a 15.5\% error for energy and 2.1\% error for area estimation for the Blocks~\cite{Wijtvliet2019Blocks:Efficiency} CGRA. %The novelty of the work lies on the focus on CGRAs and the that it works over multiple different application running on an CGRA.  
The novelty of this work lies in its focus on CGRAs and its ability to handle multiple different applications running on a CGRA.

\paragraph{Analytical Performance Estimation for Large-Scale Reconfigurable Dataflow Platforms}
% Reviewer: Christiaan & Steven

% Contributions:
% - framework for analytical model + statistics 
% - automatic approach for predicting performance on reconfigurable dataflow platforms
% - The novelty of the work lies in the fact that it is applicable to performance estimation for large-scale workloads on heterogeneous systems
% Keio University, Japan; Imperial College London, United Kingdom; University of Amsterdam, The Netherlands; Maxeler Technologies Ltd., United Kingdom
%The work from Yasudo et al., introduced in 
\citet{Yasudo2018PerformancePlatforms, Yasudo2021AnalyticalPlatforms} %and \citet{Yasudo2021perf} 
introduced and further expanded %in ~\cite{Yasudo2021perf}, proposes 
\emph{PERKS}, a performance estimation framework for reconfigurable dataflow platforms. The authors propose %In the work it is proposed 
that reconfigurable accelerators, such as FPGAs, will play an important role in future exascale computing platforms and that such a framework is essential in the efficient deployment of applications on heterogenous platforms with reconfigurable accelerators. The PERKS framework uses parameters from the target platform and the application to build an analytical model to predict the performance of multi-accelerator systems. Experimental results with different reconfigurable dataflow applications are presented, showing that the framework can predict the performance of current workloads with high accuracy.

%propose the PERKS performance estimation framework for reconfigurable dataflow programs in order to assess the feasibility of large-scale heterogeneous systems. Experimental show an above 91\% accuracy for execution time of five different applications for two reconfigurable dataflow processing platforms. It is an analytical model that adops profiling and statistical methods for calibration and improving accuracy. The novelty of the work lies in the fact that it is applicable to performance estimation for large-scale workloads on heterogeneous systems.


\paragraph{Memory and Communication Profiling for Accelerator-Based Platforms}
% Reviewer: Steven
% Contributions:
% - MCPROF profiling tool, open-source (last activity 7 years ago?)
% - Improvements compared to existing work include: faster, provinding better insights
% - Case study of image processing applications for FPGA and GPU, allowing to get insigt if application fits well to the architecture
% TU Delt + QuTech
% They descrive future work on "relating the datacommunication information generated by MCPROF with performance estimates generated by the profiling tools provided by Xilinx and Nvidia" and "utilization of the currently generated information by MCPROF to automatically generate SDSoC and OpenACC [53] pragmas for FPGA- and GPUbased accelerators, respectively".
% Looking at the publications from the main author of this work, it might have ended up in Quantum programming tooling? https://scholar.google.nl/citations?user=N_tCHwkAAAAJ&hl=nl
%The work by Ashraf et al. 
\citet{Ashraf2018MemoryPlatforms} present \emph{MCPROF}, an open-source memory-access and data-communication profiler. The tool provides a detailed profile of memory-access behavior for heterogeneous systems (CPU, GPU, and FPGA) for C/C++ applications, as well as data-communication-aware mapping of applications on these architectures. Comparison with the state of the art show that the proposed profiler has an order of magnitude, on average, lower overhead than state-of-the-art data-communication profilers over a wide range of benchmarks. A case study  with several image processing applications for heterogeneous multi-core platforms containing an FPGA and a GPU as accelerators was conducted. The authors also demonstrate that the tool can provide insights into whether %or not 
a specific %certain 
accelerator (GPU or FPGA) is a good fit 
for the application.

%\paragraph{Performance Estimation for Exascale Reconfigurable Dataflow Platforms}
% combined with: Analytical Performance Estimation for Large-Scale Reconfigurable Dataflow Platforms
% Reviewer: Steven
% Contributions:
% - 
% Keio University, Japan; Imperial College London, United Kingdom; University of Amsterdam, The Netherlands; Maxeler Technologies Ltd., United Kingdom
%
% CB: "Analytical Performance Estimation for Large-Scale Reconfigurable Dataflow Platforms" is _also_ introduces PERKS, is newer, and at the end of the introduction section it says:
% "This article extends the brief description of PERKS from Yasudo et al. [44] as follows. Section 2
% on related work is new. Section 3 covers a simpler model than the one described in the work of
% Yasudo et al. [44]. Section 4, which describes the automated method for performance estimation, is
% new. Section 5 includes further material on our experimental setup and on verifying the proposed
% model. Section 6 contains additional large-scale workload scenarios to illustrate our approach.
% Finally, Section 7 highlights the main take-home messages of this work."
% 
% CB: So perhaps these two works could be combined into one paragraph?

%This work by Yasudo et al. \cite{Yasudo2018PerformancePlatformsb} proposes \emph{PERKS}, a  performance estimation framework for reconfigurable dataflow platforms. In the work it is proposed that reconfigurable accelerators (such as FPGA) will play an important role in future exascale computing platforms and that such a framework is essential in the efficient deployment of applications on heterogenous platforms with reconfigurable accelerators. The PERKS framework uses parameters from the target platform and the application to build an analytical model to predict the performance of multi-accelerator systems. Experimental results with different reconfigurable dataflow applications are presented, showing that the framework can predict the performance of current workloads at high accuracy. 

\paragraph{Delay Prediction for ASIC HLS}%: Comparing Graph-Based and Nongraph-Based Learning Models}
% Reviewer: Christiaan
% Contributions
% - Thorough analysis of both graph and nongraph-based learning models for delay estimation
% - It's a general method tested against multiple applications including data and control oriented designs
% - It does not seem the code artifacts or models can be easily downloaded
% - Positioning the paper: existing work focuses mostly on FPGA, whilst the focus of this work is more on ASIC.
The delay estimates of HLS tools can often deviate significantly from results obtained from logic synthesis. %De et. al.~
\citet{De2023hls} propose %using 
a hybrid model by incorporating graph-based learning models, which can infer structural features from a design, into traditional non-graph-based learning models for delay estimates. The hybrid model improves delay prediction by 93\% in comparison to the delay prediction reported %given 
by a %the 
commercial HLS tool. 

\subsubsection*{\bf{Future directions}}
% \begin{itemize}
%     \item performance prediction for FPGA and, more in general, reconfigurable dataflow devices, is a challenging task. Traditionally, this was addressed by using analytical approaches, but more recently machine learning based approaches are becoming mainstream, and we can expect them to become a popular option
%     \item with CGRA-like architecture being commercialized, the interest in exploring these devices is increasing
%     \item we advocate the need for open-source tools (or at least reproducibility): performance prediction is a device-specific task, but very few of the papers reviewed release the code used. If we want to increase the chance of collaborations, and reproducibility, this has to change.
% \end{itemize}

Performance prediction for FPGAs and, more generally, reconfigurable dataflow devices is challenging. Traditionally, this was addressed with %by using 
analytical approaches, but more recently, machine-learning-based approaches are becoming mainstream, and we can expect them to become a popular option given their successes in recent research. With CGRA-like architectures being commercialized, the interest in exploring these devices is increasing, and we %should 
expect future work to focus more on the predictability of the 
performance of such devices. Performance prediction is a device-specific task, yet only a %fraction %very 
few 
of the reviewed papers favor the reproducibility of the presented results, or publicly release the % share the 
code used to generate these results. 
We should shift toward more reproducible and transparent research to foster %increase the chance of 
collaborations and facilitate general progress. Thus, we advocate for the need for open-source practices %also 
in performance prediction as well.
% Describe the future direction that this research, in the Netherlands, is likely to head into. This section should describe what ongoing research seems most promising and what we might expect in future research on this topic. This paragraph can also expand on potential applications that these developments can be applied to, giving insight into why this is important.
% - it is quite disjoint, so in case we can mention what's the trend in the world
% - it is also fine to say something about who is leading (e.g. work X was led by this dutch institution) or who contributed (in the last 5 years)


\section{Robustness of FPGAs}
\label{sec:robustness}

\section{Loss Robustness}
\label{sec:robustness}

% We extend the concept of label noise to the autoregressive language modeling domain, focusing on asymmetric or class-conditional noise. Specifically, at each step $t$, the label $\xbm_t$ in the training data of the black-box model is flipped to  $\tilde \xbm_t \in V$ with probability $p^*(\tilde \xbm_t|\xbm_t)$, while the feature vectors or preceding tokens $(\xbm_{t-1:1})$ remain unchanged. Consequently, the black-box model observes samples from a noisy distribution given by  $p^*(\tilde \xbm_t, \xbm_{t-1:1}) = \sum_{\xbm_t}p^*(\tilde \xbm_t | \xbm_t)p^*(\xbm_t|\xbm_{t-1:1})p^*(\xbm_{t-1:1})$.

% Denote by $T_t  \in [0, 1]^{|V|\times |V|}$, the noise transition matrix at step $t$ specifying the probability of one label being flipped to another, so that $\forall i, j \;\; T_{t_{ij}}=p^*(\tilde \xbm_t = \ebm^j | \xbm_t = \ebm^i)$. The matrix is row-stochastic and not necessarily symmetric across the classes. 

% To address asymmetric label noise, we modify the loss $\bm{\ell}$ to ensure robustness. Initially, assuming the noise transition matrix $T_t$ is known, we apply a loss correction inspired by prior work~\citep{patrini2017making, sukhbaatar2015training}. Subsequently, we relax this assumption and estimate $T_t$ directly, forming the foundation of our plugin model approach.

We extend label noise modeling to the autoregressive language setting, focusing on asymmetric or class-conditional noise. At each step $t$, the label $\xbm_t$ in the black-box model’s training data is flipped to $\tilde \xbm_t \in V$ with probability $p^*(\tilde \xbm_t|\xbm_t)$, while preceding tokens $(\xbm_{t-1:1})$ remain unchanged. As a result, the black-box model observes samples from a noisy distribution: $p^*(\tilde \xbm_t, \xbm_{t-1:1}) = \sum_{\xbm_t} p^*(\tilde \xbm_t | \xbm_t) p^*(\xbm_t|\xbm_{t-1:1}) p^*(\xbm_{t-1:1}).$

We define the noise transition matrix $T_t \in [0,1]^{|V|\times |V|}$ at step $t$, where each entry $T_{t_{ij}} = p^*(\tilde \xbm_t = \ebm^j | \xbm_t = \ebm^i)$ represents the probability of label flipping. This matrix is row-stochastic but not necessarily symmetric.

To handle asymmetric label noise, we modify the loss $\bm{\ell}$ for robustness. Initially, assuming a known $T_t$, we apply a loss correction inspired by~\citep{patrini2017making, sukhbaatar2015training}. We then relax this assumption by estimating $T_t$ directly, forming the basis of our \textit{Plugin} model approach.

We observe that a language model trained with no loss correction would result in a predictor for noisy labels $b(\tilde \xbm_t | \xbm_{t-1:1})$. We can make explicit the dependence on $T_t$. For example, with cross-entropy we have:

\begin{align*}
&\ell(\ebm^i, b(\tilde \xbm_t | \xbm_{t-1:1})) = -\log b(\tilde\xbm_t = \ebm^i | \xbm_{t-1:1}) \\
&= -\log \sum_{j=1}^{|V|} p^*(\tilde\xbm_t = \ebm^i | \xbm_t = \ebm^j) b(\xbm_t = \ebm^j | \xbm_{t-1:1}) \\ 
&= -\log \sum_{j=1}^{|V|} T_{t_{ji}} {b}(\xbm_t = \ebm^j | \xbm_{t-1:1}), \numberthis
\label{eq:fc}
\end{align*}
or in matrix form
\begin{equation}
    \label{eq:fc-mat}
    \bm{\ell}(b(\tilde \xbm_t|\xbm_{t-1:1})) = -\log T_t^\top b(\xbm_t|\xbm_{t-1:1}).
\end{equation}

% This loss compares the noisy label $\tilde \xbm_t$ to the noisy predictions averaged using the transition matrix $T_t$ at step $t$. Note that the cross-entropy loss is commonly employed for next-token prediction tasks. Cross-entropy is a \emph{proper composite loss} with the softmax function as its \emph{inverse link function}~\citep{patrini2017making}. Consequently, from Theorem 2 of~\citep{patrini2017making}, the minimizer of the \emph{forwardly-corrected} loss in Equation~\eqref{eq:fc-mat} for noisy data corresponds to the minimizer of the actual loss for clean data. Formally, this can be expressed as:

% \begin{align*}
%     \label{eq:loss-minimizers}
%     & \argmin_{w} E^*_{\tilde \xbm_t,\xbm_{t-1:1}}\Big[\bm{\ell}(\xbm_t, T_t^T b(\xbm_t|\xbm_{t-1:1})) \Big] \\ &= 
%     \argmin_{w} E^*_{\xbm_t,\xbm_{t-1:1}}\Big[\bm{\ell}(\xbm_t, b(\xbm_t|\xbm_{t-1:1})) \Big],
% \end{align*}
% where $w$ are the weights of the language model, and their dependence is implicitly embedded in the definition of the softmax output $b$ from the black-box language model. This result indicates that if the transition matrix $T_t$ were known, we could transform the softmax output $b(\bm{x}_t \mid \bm{x}_{t-1:1})$ using $T_t^T$, use the transformed predictions as the final outputs, and re-train the black-box model accordingly with the corrected loss. However, the transition matrix $T_t$ is not known a priori, and we do not have access to the training data. Thus, estimating $T_t$ from clean data becomes a crucial step in our approach.

This loss compares the noisy label $\tilde \xbm_t$ to the noisy predictions averaged via the transition matrix $T_t$ at step $t$. Cross-entropy loss, commonly used for next-token prediction, is a \emph{proper composite loss} with the softmax function as its \emph{inverse link function}~\citep{patrini2017making}. Consequently, from Theorem 2 of~\citet{patrini2017making}, the minimizer of the \emph{forwardly-corrected} loss in Equation~\eqref{eq:fc-mat} on noisy data aligns with the minimizer of the true loss on clean data, i.e., 
\begin{align*}
    \label{eq:loss-minimizers}
    & \argmin_{w} E^*_{\tilde \xbm_t,\xbm_{t-1:1}}\Big[\bm{\ell}(\xbm_t, T_t^\top b(\xbm_t|\xbm_{t-1:1})) \Big] \\ &= 
    \argmin_{w} E^*_{\xbm_t,\xbm_{t-1:1}}\Big[\bm{\ell}(\xbm_t, b(\xbm_t|\xbm_{t-1:1})) \Big],
\end{align*}
where $w$ are the language model’s weights, implicitly embedded in the softmax output $b$ from the black-box model. This result suggests that if $T_t$ were known, we could transform the softmax output $b(\xbm_t \mid \xbm_{t-1:1})$ using $T_t^T$, use the transformed predictions as final outputs, and retrain the model accordingly. However, since $T_t$ is unknown and training data is inaccessible, estimating $T_t$ from clean data is essential to our approach.


\subsection{Estimation of Transition Matrix}
\label{ssec:estimatingT}

% In our problem setup, we assume access to a small amount of clean language data for the task. Under the assumption that the black-box model is expressive enough to model $p^*(\tilde{\bm{x}}_t \mid \bm{x}_{t-1:1})$ (Assumption (2) in Theorem 3 of~\citep{patrini2017making}), the transition matrix $T_t$ can be estimated using this clean data. Considering the supervised classification problem at step $t$, let $\mathcal{X}_t^i$ denote all samples in the clean data where $\bm{x}_t = \bm{e}^i$ and the preceding tokens are $(\bm{x}_{t-1}, \dots, \bm{x}_1)$. A naive estimate of the transition matrix can be computed as follows:

We assume access to a small amount of target language data for the task. Given that the black-box model is expressive enough to approximate $p^*(\tilde{\xbm}_t \mid \xbm_{t-1:1})$ (Assumption (2) in Theorem 3 of~\citet{patrini2017making}), the transition matrix $T_t$ can be estimated from this target data. Considering the supervised classification setting at step $t$, let $\mathcal{X}_t^i$ represent all target data samples where $\xbm_t = \ebm^i$ and the preceding tokens are $(\xbm_{t-1:1})$. A naive estimate of the transition matrix is: $\hat T_{t_{ij}}=b(\tilde \xbm_t = \ebm^j|\xbm_t=\ebm^i)=\frac{1}{|\mathcal{X}_t^i|}\sum_{x\in\mathcal{X}_t^i}b(\tilde \xbm_t = \ebm^j|\xbm_{t-1:1})$.


While this setup works for a single step $t$, there are two key challenges in extending it across all steps in the token prediction task:

\begin{enumerate}[leftmargin=0.4cm]
    \item \textbf{Limited sample availability:} The number of samples where $\bm{x}_t = \bm{e}^i$ and the preceding tokens $(\bm{x}_{t-1}, \dots, \bm{x}_1)$ match exactly is limited in the clean data, especially with large vocabulary sizes (e.g., $|V| = O(100K)$ for LLaMA~\citep{dubey2024llama}). This necessitates modeling the transition matrix as a function of features derived from $\bm{x}_{t-1:1}$, akin to text-based autoregressive models.
    \item \textbf{Large parameter space:} With a vocabulary size of $|V| = O(100K)$, the transition matrix $T_t$ at step $t$ contains approximately 10 billion parameters. This scale may exceed the size of the closed-source LLM and cannot be effectively learned from limited target data. Therefore, structural restrictions must be imposed on $T_t$ to reduce its complexity.
\end{enumerate}

To address these challenges, we impose the restriction that the transition matrix $T_t$ is diagonal. While various constraints could be applied to simplify the problem, assuming $T_t$ is diagonal offers two key advantages. First, it allows the transition matrix—effectively a vector in this case—to be modeled using standard autoregressive language models, such as a \emph{GPT-2 model with $k$ transformer blocks}, a \emph{LLaMA model with $d$-dimensional embeddings}, or a fine-tuned \emph{GPT-2-small} model. These architectures can be adjusted based on the size of the target data. Second, a diagonal transition matrix corresponds to a symmetric or class-independent label noise setup, where $\xbm_t = \ebm^i$ flips to any other class with equal probability in the training data. This assumption, while simplifying, remains realistic within the framework of label noise models.

By enforcing this diagonal structure, we ensure efficient estimation of the transition matrix while maintaining practical applicability within our framework. In the next section, we outline our approach for adapting closed-source language models to target data.














%\newpage
\section{Applications}
\label{sec:applications}
%Placeholder for general introduction of the Accelerators and Applications theme sections. Themes of applications include machine-learning, bioinformatics, space applications, radio astronomy and weather simulations. Some of these references will overlap with other sections, e.g. when contributions are made on applying effective distributed computing for the purpose of weather forecasting.

FPGAs have emerged as powerful accelerators for a wide range of applications. In this section, we discuss FPGA-based solutions in machine learning (Section~\ref{sec:ml}), astronomy (Section~\ref{sec:astr}), particle physics experiments (Section~\ref{sec:phys}), quantum computing (Section~\ref{sec:quant}), space applications (Section~\ref{sec:space}), and bioinformatics (Section~\ref{sec:bio}).

\subsection{Machine learning}
\label{sec:ml}
% Three main parts, adapting an existing ML approach to hardware, designing hardware to accelerate an existing ML approach, (co-)design hardware for exotic ML approach.
% Main categories of evaluation are throughput, power, hardware area / resources, accuracy.
% \begin{itemize}
%     \item Accelerating existing ML models with new hardware design
%         \begin{itemize}
%             \item CNN acceleration (5)
%             \item TPU (1)
%             \item Benchmarking FPGAs (1)
%     \item Co-design existing ML models to hardware accelerate
%         \begin{itemize}
%             \item Pruning
%             \item Quantization / fixed point
%             \item Weight sharing
%             \item NAS adaptive to hardware
%         \end{itemize}
%     \item Design new hardware for exotic ML model
%         \begin{itemize}
%             \item Spiking / neuromorphic (7)
%             \item Bayesian (1)
%             \item Oscillating (2)
%         \end{itemize}
%     \end{itemize}
%     \item Hardware for 
% \end{itemize}
% \subsubsection{Background}
In the field of machine learning, and in particular deep learning, hardware acceleration plays a vital role. GPUs are the predominant method for hardware acceleration due to their high parallelism, but FPGA research is showing promising results. FPGAs enable inference at greater speed and better power efficiency when compared to GPUs \cite{hw-efficiency-compare} by designing model-specific accelerated pipelines \cite{ml-energy-efficient-cnn}. Through the co-design of machine learning models and machine learning hardware on FPGAs, models are accelerated without compromising on performance metrics and utilizing limited FPGA resources. In addition, the flexibility of the FPGA's architecture enables the realization of unconventional deep learning technology, such as Spiking Neural Networks (SNNs). 
%These networks can operate on a fraction of the power required by conventional networks on CPU or GPU.

%\subsubsection{Research topics}
\paragraph{Hardware acceleration} Ample research on hardware acceleration focuses on accelerating existing neural network architectures. One common class of architectures is convolutional neural networks (CNNs), which learn image filters in order to identify abstract image features. CNNs are often deployed in embedded applications which require real-time image processing and low energy consumption, making FPGAs a suitable candidate for CNN acceleration. \citet{ml-energy-efficient-cnn} propose an implementation of the LeNet architecture using Vitis HLS, pipelining the CNN layers, and outperforms other FPGA based implementations at a processing time of $70 \mu s$. One downside to this approach is the inflexibility of designing a specific model architecture in HLS which can be resolved by using partial reconfiguration \cite{ml-cnn-acclr-part-reconf}. To increase CNN throughput, further parallelization can be exploited, and in combination with the use of the high bandwidth OpenCAPI interface, can achieve a latency of less than $10 \mu s$ on the LeNet-5 model, streaming data from an HDMI interface \cite{ml-FPQNet}. In each of these implementations, fully pipelined CNNs are possible due to the limited number of parameters in small CNNs. As larger pipelined networks are deployed on FPGAs, parallelization puts strain on the available resources, and in particular the amount of on-chip-memory becomes a bottleneck. A proposed solution to this is using Frequency Compensated Memory Packing \cite{ml-mem-efficient-df-inf}.

In addition to CNN acceleration, general neural network acceleration has been developed by means of a programmable Tensor Processing Unit (TPU) as an overlay on an FPGA accelerator \cite{ml-agile-tuned-tpu}. Deep learning acceleration using FPGAs is also relevant to space technology research. Since the reprogrammability of FPGAs make them a suitable contender for deployment on space missions, FPGA implementations of existing deep learning models are being benchmarked for space applications \cite{ml-myriad-2-space-cnn} \cite{ml-mem-efficient-df-inf}.

\paragraph{Spiking neural networks} Spiking Neural Networks (SNNs) are computational models formed using spiking neuronal units that operate in parallel and mimic the basic operational principles of biological systems. These features endow SNNs with potentially richer dynamics than traditional artificial neural network models based on the McCulloch-Pitts point neurons or simple ReLU activation functions that do not incorporate timing information. Thus, SNNs excel in handling temporal information streams and are well-suited for innovative non-von-Neumann computer architectures, which differ from traditional sequential processing systems. SNNs are particularly well-suited for implementation in FPGAs due to their massive parallelism and requirement for significant on-chip memories with high-memory bandwidth for storing neuron states and synaptic weights. Additionally, SNNs use sparse binary communication, which is beneficial for low-latency operations because both computing and memory updates are triggered by events. FPGAs' inherent flexibility allows for reprogramming and customization, which enable reprogrammable SNNs in FPGAs, resulting in flexible, efficient, and low-latency systems~\cite{Corradi2021Gyro:Analytics,Irmak2021ADesigns,SankaranAnInference}. \citet{corradi2024accelerated} demonstrated the application of a Spiking Convolutional Neural Network (SCNN) to population genomics. The SCNN architecture achieved comparable classification accuracy to state-of-the-art CNNs while processing only about 59.9\% of the input data, reaching 97.6\% of CNN accuracy for classifying selective-sweep and recombination-hotspot genomic regions. This was enabled by % success is attributed to 
the SCNN's capability to temporize genetic information, allowing it to produce classification outputs without processing the entire genomic input sequence. Additionally, when implemented on FPGA hardware, the SCNN model exhibited over three times higher throughput and more than 100 times greater energy efficiency than a GPU implementation, markedly enhancing the processing of large-scale population genomics datasets.


\paragraph{Model/hardware co-design} Previous examples demonstrate that existing deep neural network models can be accelerated using FPGAs. Typically, research in this area focuses on designing an optimal hardware solution for an existing model. A more effective approach, however, is to co-design the model and the hardware accelerator simultaneously. However, simultaneous co-design of DNN models and accelerators is challenging. DNN designers often need more specialized knowledge to consider hardware constraints, while hardware designers may need help to maintain the quality and accuracy of DNN models. Furthermore, efficiently exploring the extensive co-design space is a significant challenge. This co-design methodology leads to better performance, leveraging FPGAs' flexibility and rapid prototyping capabilities. For example, \citet{Rocha2020BinaryWrist-PPG}, by co-designing the bCorNET framework, which combines binary CNNs and LSTMs, they were able to create an efficient hardware accelerator that processes HR estimation from PPG signals in real-time. The pipelined architecture and quantization strategies employed allowed for significant reductions in memory footprint and computational complexity, enabling real-time processing with low latency.

In SNNs, encoding information in spike streams is a crucial co-design aspect. SNNs primarily use two encoding strategies: rate-coding and time-to-first-spike (TTFS) coding. Rate coding is common in SNN models, encoding information based on the instantaneous frequency of spike streams. Higher spike frequencies result in higher precision but at the cost of increased energy consumption due to frequent spiking. While rate coding offers accuracy, it reduces sparsity. In FPGA implementations, rate coding is often used for its robustness, simplicity, ease of training through the conversion of analog neural networks to spiking neural networks, and practicality in multi-sensor data fusion, where it helps represent real values from various sensors (radars, cameras) even in the presence of jitter or imperfect synchronization~\cite{Corradi2021Gyro:Analytics}.
Conversely, TTFS coding has been demonstrated in SNNs implemented on FPGAs to enhance sparsity and has the potential of reducing energy consumption by encoding information in spike timing. For instance, Pes et al.~\cite{Pes2024ActiveNetworks} introduced a novel SNN model with active dendrites to address catastrophic forgetting in sequential learning tasks. Active dendrites enable the SNN to dynamically select different sub-networks for different tasks, improving continual learning and mitigating catastrophic forgetting. This model was implemented on a Xilinx Zynq-7020 SoC FPGA, demonstrating practical viability with a high accuracy of 80\% and an average inference time of 37.3 ms, indicating significant potential for real-world deployment in edge devices.

%To overcome this challenges, Cong et al in \textcolor{red}{\textcolor{red}{~\cite{FPGA/DNN Co-Design: An Efficient Design Methodology for IoT
%Intelligence on the Edge}}} introduced a co-design methodology for FPGAs and DNNs that integrates both bottom-up and top-down approaches, in which a bottom-up search for DNN models that prioritize high accuracy is paired with a top-down design of FPGA accelerators tailored to the specific characteristics of DNNs.
%Other methods leverage an automatic toolchain comprising  auto-DNN engine for hardware-aware DNN model optimization and an auto-HLS engine to generate FPGA-suitable synthesizable code, or hardware-aware Neural Architecture Search (NAS). When co-design is applied, it typicaly produces DNN models and FPGA accelerators that outperform state-of-the-art FPGA designs in various metrics, including accuracy, speed, power consumption, and energy efficiency \textcolor{red}{~\cite{When Neural Architecture Search Meets Hardware Implementation: from Hardware Awareness to Co-Design}.

%\textcolor{blue}{Co-design is critical in developing FPGA-based systems, merging hardware and software engineering from the initial design stages. This integrated method is essential for optimizing system performance, functionality, and cost-effectiveness. Co-design leverages the adaptable nature of FPGAs, tailoring the computing workload to meet specific hardware needs and adjusting the hardware to suit software demands. This synergy results in improved system performance and greater energy efficiency.}
%\textcolor{blue}{Many co-design examples exists in literature that demonstrate how clever distributed memory layouts can results in increased performances~\cite{}. }

%\paragraph{Novel hardware architecture} 

%\textcolor{blue}{Modern co-design methodologies allow the generation of hardware architectures and applications for advanced Reconfigurable Acceleration Devices (RAD) that go beyond traditional FPGA capabilities. These devices integrate FPGA fabric with other components like general-purpose processors, specialized accelerators, and high-performance networks-on-chip (NoCs) within a system-in-package framework. This integration enables complex data center applications to be handled more efficiently than conventional FPGAs. In particular, Boutrous et al in \cite{Architecture and Application Co-Design for Beyond-FPGA Reconfigurable Acceleration Devices} introduce RAD-Sim, an architecture simulator, to aid in the design space exploration of RADs. This allows for the study of interactions between different system components. Notably, they demonstrated mapping deep learning FPGA overlays to different RAD configurations, demonstrating how RAD-Sim can guide the adaptation of applications to exploit the novel features of RADs effectively.}


\subsection{Astronomy}
\label{sec:astr}
%\subsubsection{Introduction}

Astronomy is the study of everything in the universe beyond our Earth's atmosphere. Observations are done at different modalities and wavelengths, such as detection of a range of different particles (e.g., Cherenkov detector based systems such as KM3NeT \cite{KM3NeT:2009xxi}), gravitational waves, optical observations, gamma and x-ray observations and radio (e.g., WSRT \cite{van_Cappellen_2022}, LOFAR \cite{van_Haarlem_2013}, SKA \cite{book-SKA}). Observations can be done from space or from earth; in this section, we limit the scope to ground-based astronomy. A common denominator for instruments required for observation of the different modalities and different wave lengths is that the systems need to be very sensitive in order to observe very faint signals from outside the Earth's atmosphere. Instruments are typically large and/or distributed over a large area %in order 
to achieve %reach 
good sensitivity and resolution. Different modalities and wavelengths require distinct types of sensors, cameras, or antennas to convert observed phenomena into electrical signals. Each system is tailored to its specific modality and wavelength, necessitating specialized components to accurately capture and translate the data. %At different modalities and different wave lengths, the systems each require different kinds of sensors, camera's or antenna's that convert the observed phenomenon to an electrical signal. 
%The electrical signal is at some point in the signal chain converted to the digital domain and processed in various stages into an end product used by scientists. 
At a certain stage in the signal chain, the electrical signal is converted into the digital domain, where it undergoes multiple processing stages. This processed signal ultimately results in an end product that can be utilized by scientists for analysis and research purposes.
Systems can roughly be split into two parts, a front-end and a back-end. The front-end requires interfacing with and processing of data from the sensor; electronics commonly deployed in the front-end are constrained in space (size), temperature, power, cost, RFI, environmental conditions and serviceability. The back-end processes data produced by the front-end(s) either in an online or offline fashion, which is usually %typically be 
done with server infrastructure in a data center. % environment, either on site or centralized. 
In the back-end, the main challenges are the high data bandwidth and large data size coming from the front-ends. Although the environment is more flexible, systems are still constrained in space, power, and cost.

%\subsubsection{Background}

FPGAs have been used in astronomy instrumentation for quite some time, as they 
%FPGAs have since long found applications in astronomy instrumentation. 
%Typically FPGA's 
are %very 
efficient in interfacing with Analog to Digital Converters (ADCs), and well suited to the conditions faced in instrumentation front-ends (e.g. NCLE \cite{karapakula2024ncle}). Moreover, FPGA are also used further down the processing stages for various signal processing operations, both in the front-ends (e.g., Uniboard2 in LOFAR \cite{doi:10.1142/S225117171950003X}) as well as in the back-ends of systems (e.g., MeerKAT \cite{2022JATIS...8a1006V} and SKA \cite{SKA-CBF}). GPUs represent a good alternative in back-end processing (e.g., LOFAR's system COBALT \cite{Broekema_2018}) as well. The work by Veenboer et al. \cite{10.1007/978-3-030-29400-7_36} describes a trade-off between using a GPU and an FPGA accelerator in the implementation of an image processing operation in a radio telescope back-end.

%\subsubsection{Research topics}
%Dutch academia has contributed to several astronomy instruments:
%Often large international consortia, not immidiately clear what the role of the Dutch partners was. But also some work which is mainly done by Dutch institutes
\paragraph{Hardware Development for the Radio Neutrino Observatory in Greenland (RNO-G)}
The RNO-G \cite{Smith2022HardwareRNO-G} is a radio detection array for neutrinos. It consists of 35 autonomous stations deployed over a $5 \times 6$ km grid near the NSF Summit Station
in Greenland. Each station includes an FPGA-based phased trigger. The station has to operate in a 25 W power envelope. The implementation on FPGA seems to be favorable due to environmental conditions and operation constraints.

\paragraph{Implementation of a Correlator onto a Hardware Beam-Former to Calculate Beam-Weights}
The Apertif Phased Array Feed (PAF) \cite{van_Cappellen_2022} is a radio telescope front-end used in the WSRT system in the Netherlands. FPGAs are used for antenna read out as well as signal processing close to the antenna. Schoonderbeek et al. \cite{Schoonderbeek2020ImplementationBeam-Weights} describe the transformation and implementation of a beamformer algorithm on FPGA in order to build a more efficient system.

\paragraph{Near Memory Acceleration and Reduced-Precision Acceleration for High Resolution Radio Astronomy Imaging}
\citet{Corda2020NearImaging} describe the implementation of a 2D FFT on FPGA, leveraging Near-Memory Computing. The 2D FFT is applied to an image processing implementation on FPGA in the back-end of a radio telescope and compared with implementations on CPU and GPU. \citet{Corda2022Reduced-PrecisionHardware} explore %the concept of 
reduced-precision computation on an FPGA %is explored 
for the same image processing application. %They propose an implementation on an FPGA accelerator and compare with an implementation on CPU and GPU.

\paragraph{The MUSCAT Readout Electronics Backend: Design and Pre-deployment Performance}
The MUSCAT is a large single dish radio telescope with 1458 receives in the focal plane. The system uses FPGA based electronics to read out and pre-process the data from the receivers \cite{Rowe2023ThePerformance}. %\emph{Electronics Backend} in this case relates to the electronics close to the antenna, referred to as front-end in our description here.

% Small contribution by NL through SRON

\paragraph{Cherenkov Telescope Arrays}
Three different contributions have been made to three different Cherenkov based Telescope Arrays.
%\paragraph{A NECTAr-based upgrade for the Cherenkov cameras of the H.E.S.S. 12-meter telescopes}
Ashton et al.~\cite{Ashton2020ATelescopes} describe a system for the High Energy Stereoscopic System (H.E.S.S.) where a custom board with ARM CPU and an FPGA is used to read out and pre-process a custom designed NECTAr digitizer chip in the front-end of the system. After pre-processing, the data is distributed to a back-end over Ethernet.
%Anton Pannekoek Institute for Astronomy
%\paragraph{A White Rabbit-Synchronized Accurate Time-Stamping Solution for the Small-Sized Cameras of the Cherenkov Telescope Array}
Sánchez-Garrido et al.~\cite{Sanchez-Garrido2021AArray} present the design of a Zynq FPGA SoC based platform for White Rabbit time synchronization in the ZEN-CTA telescope array front-ends. Data captured and pre-processed at the front-ends is distributed over Ethernet to the back-end including the time stamp.
%\paragraph{Architecture and performance of the KM3NeT front-end firmware}
Aiello et al.~\cite{Aiello2021ArchitectureFirmware} outline the architecture and performance of the KM3Net front-end firmware. The KM3NeT telescope consists of two deep-sea three-dimensional sensor grids being deployed in the Mediterranean Sea. A central logic board with FPGA in the front-end serves as a Time to Digital Converter to record events and time at the sensors; the data is transmitted and further processed in a back-end on shore.
%S. Aiello et al., “KM3NeT front-end and readout electronics system:
%hardware, firmware, and software,” J. Astronomical Telescopes Inst.
%Syst., vol. 5, no. 4, pp. 1–15, 2019.

%\subsubsection{Future direction}

\vspace{0.4cm}
%In the works included in this survey, 
FPGA are mainly used for front-end sensor interfacing and pre-processing. \citet{Corda2020NearImaging, Corda2022Reduced-PrecisionHardware} underline that FPGAs are also still relevant in the back-end, providing improved performance over CPU and on-par performance with GPU accelerators. FPGA are expected to remain the dominant choice for platforms in astronomy instrumentation front-ends due to the strong interfacing capabilities and the adaptability and suitability to the constraints imposed by instrumentation front-ends. In the back-end, FPGAs provide a viable solution to application acceleration, but will have to compete with other accelerator architectures, e.g., GPUs~\cite{10.1007/978-3-030-29400-7_36}. 
An emerging new technology are the Artificial Intelligence Engines in the AMD Versal Adaptive SoC. The work from \citet{Versal-ACAP} evaluated the AI Engines for a signal processing application in radio astronomy. The flexibility and programmability of the AI Engines, combined with the interfacing capabilities of the FPGA can lead to a powerful platform for telescope front-ends.

\subsection{Particle physics experiments}
\label{sec:phys}
The Large Hadron Collider (LHC) features various particle accelerators to facilitate particle physics experiments. Experiments performed using particle accelerators can produce massive amounts of data that needs to be propagated and preprocessed at high speeds before the reduced relevant data is recorded for offline storage. FPGAs are widely employed throughout systems LHC particle accelerators, such as ATLAS and LHCb, for their high-bandwidth capabilities, and the flexibility that reconfigurable hardware offers without requiring hardware alterations to the system. Recently both the ATLAS and LHCb particle accelerators have been commissioned for upgrades. The Dutch Institute for Subatomic Physics (Nikhef) is one of the collaborating institutes working on the LHC accelerators.

%\subsubsection{Research topics}
%\paragraph{TODO - revise text into research topics}

LHCb is a particle accelerator that specializes in experiments that study the bottom quark. Major upgrades to the LHCb that enable handling a higher collision rate require new front-end and back-end electronics. To facilitate the back-end of the upgrade, the LHCb implements the custom PCIe40 board, which features an Intel Arria 10 FPGA. Four PCIe40 boards are dedicated for controlling part of the LHCb system, and 52 PCIe40 boards are used to read out each of the detector’s slices, producing an aggregated data rate of 2.85 Tb/s \cite{FernandezPrieto2020PhaseExperiment}.

ATLAS is one of the general particle accelerators of the LHC. ATLAS uses two trigger stages in order to record only the particle interactions of interest. In an upgrade to the ATLAS accelerator, ASIC-based calorimeter trigger preprocessor boards are replaced by FPGA-based hardware. Using FPGAs for this purpose allows implementing enhanced signal processing methods \cite{Aad2020PerformanceTrigger}. After the two trigger stages, FPGAs are deployed to process the triggers for tracking particles \cite{Aad2021TheSystem}.

Ongoing upgrades to the LHC particle accelerators, referred to as High Luminosity LHC (HL-LHC), will facilitate higher energy collisions. HL-LHC will produce increased background rates. To reduce false triggers due to background, the New Small Wheel checks for coinciding hits. Each trigger processor features Virtex-7, Kintex Ultrascale and Zynq FPGAs \cite{Iakovidis2023TheElectronics}. Interaction to and from front-end hardware is done through Front-End Link eXchange (FELiX) boards. As part of the HL-LHC upgrades, each FELiX board must facilitate a maximum throughput of 200 Gbps. To enable this, Remote Direct Memory Access (RDMA) over Converged Ethernet (RoCE) as part of the FELiX FPGA system is proposed \cite{Vasile2022FPGALHC, Vasile2023IntegrationLHC}. The performance of the FELiX upgrade in combination with an upgraded Software ReadOut Driver (SW ROD) satisfies the data transfer requirements of the upgraded ATLAS system \cite{Gottardo2020FEliXSystem}.




\subsection{Quantum computing}
\label{sec:quant}
Quantum computing promises to help solving many global challenges of our time such as quantum chemistry problems to design new medicines, the prediction of material properties for efficient energy storage, and the handling of big data needed for complex climate physics~\cite{Gibney-nat-2014}. The most promising quantum algorithms demand systems comprising thousands to millions of quantum bits~\cite{Meter-2013}, the quantum counterpart of a classical bit. A quantum processor comprising up to 50 qubits has been realized using solid-state superconducting qubits~\cite{Arute-nat-2019}, but its operation requires a combination of cryogenic temperatures below ~100 mK and hundreds of coaxial lines for qubit control and readout. %Furthermore, 
While in systems with a few qubits, this can be controlled using off-the-shelf electronic equipment, such approach becomes infeasible when scaling qubit systems toward thousands or millions of qubits that are required for a practical quantum computer. 

%\subsubsection{Research topics}

A means to tackle the foreseeable bottleneck in scaling the operation of qubit systems is to integrate FPGA technology in the control and readout of solid-state qubits. FPGAs have been used to generate highly-stable waveforms suitable for the control of quantum bits with latency significantly lower than software alternatives~\cite{Ireland-2020}. In systems of semiconductor spin qubits, FPGAs have provided in-hardware syncing of quantum dot control voltages with the signal acquisition and buffering and thus enabled the observation of real-time charge-tunneling events~\cite{Hartman-2023}. FPGAs have also been used to configure and synchronize a cryo-controller with an arbitrary waveform generator required to generate complex pulse shapes and perform quantum operations~\cite{Xue-nat-2021}. Such setup has enabled the demonstration of universal control of a quantum processor hosting six semiconductor spin qubits~\cite{Philips-nat-2022}. FPGAs have proven to be essential for implementing quantum error correction algorithms, which are critical for mitigating the effects of dephasing and decoherence in solid-state qubits. %FPGAs have also been shown to essential for the implementation of quantum error correction algorithms needed to mitigate the effects of dephasing and decoherence in solid-state qubits. 
In qubit systems based on superconducting quantum circuits, the first efficient demonstration of quantum error correction was made possible by a FPGA-controlled data acquisition system which provided dynamic real-time feedback on the evolution of the quantum system~\cite{Ofek-nat-2016}. It has been further predicted that FPGA can enable highly-efficient quantum error correction based on neural-network decoders~\cite{Overwater-2022}.

%\subsubsection{Future directions}

FPGA technology has proven invaluable in the development of the emerging research field of quantum computing.
However, the complexity of programming FPGA circuits hinders their implementation in quantum computing systems. Commercial efforts have been done toward providing graphical tools for designing FPGA programs, namely the Quantum Researchers Toolkit by Keysight Technologies and the FPGA-based multi-instrument platform Moku-Pro developed by Liquid Instruments. These tools are essential for implementing customized algorithms without the need for dedicated expertise in hardware description languages. Future research is also needed in integrating FPGAs in cryogenic platforms required to operate qubit systems. Such capability has already been demonstrated; commercial FPGAs can operate at temperatures below 4 K and be integrated in a cryogenic platform for qubit control~\cite{Homulle-2017}. These efforts provide evidence that FPGA technology is of great interest for enabling a scalable and practically applicable quantum computer. 


\subsection{Space}
\label{sec:space}
The flexibility of FPGA technology makes it a suitable platform for many applications on-board space missions. The European Space Research and Technology Centre (ESTEC), as part of the European Space Agency (ESA) actively explores FPGA technology for space applications, and has an extensive portfolio of FPGA Intellectual Property (IP) Cores~\cite{esa_ip}.

%\subsubsection{Research topics}

FPGAs can flexibly route its input and output ports, and can be configured to support many different communication protocols. This makes FPGAs good contenders as devices that communicate with the various hardware platforms and sensors on a space mission. FPGAs and have been implemented as interface devices in novel on-board machine learning and digital signal processing  implementations~\cite{Leon2021ImprovingSoC, Leon2021FPGABenchmarks, karapakula2024ncle}. 

An on-board task for which FPGAs are used is hyperspectral imaging. This type of on-board imaging produces vast amounts of data. To reduce transmission bandwidth requirements when transmitting the sensory data to earth, real-time on-board compression handling high data rates is required. FPGAs are well-suited for such tasks, and research has been done on using space-grade radiation-hardened FPGAs \cite{Barrios2020SHyLoCMissions} as well as commercial off-the-shelf (COTS) FPGAs \cite{Rodriguez2019ScalableCompression} for on-board hyperspectral image compression. COTS devices are generally cheaper than space-grade devices, but the higher susceptibility of these devices to radiation-induced effects makes them challenging to employ.

Communication between on-board systems often requires high data-rates and is susceptible to radiation induced effects. To deal with the unique constraints of space applications, dedicated communication protocols such as SpaceWire, and its successor, SpaceFibre have been developed. These protocols are available as FPGA IP implementations, and testing environments of SpaceFibre have been developed \cite{MystkowskaSimulationSpaceFibre, AnSection}. SpaceWire can interface with the common AXI4 protocol using a dedicated bridge \cite{RubattuASystems}, enabling its integration with SpaceWire interfaces. Direct Memory Access (DMA) allows peripherals to transfer data to and from an FPGA without going through a CPU. The application of DMA in space is being investigated, however its application as of now is limited since DMA is susceptibility to radiation-induced effects \cite{Portaluri2022Radiation-inducedDevices}.



\subsection{Bioinformatics}
\label{sec:bio}
FPGA technology has been extensively explored for accelerating Bioinformatics kernels. Bioinformatics is an interdisciplinary scientific field that combines biology, computer science, mathematics, and statistics to analyze and interpret biological data. The field primarily focuses on the development and application of methods, algorithms, and tools to handle, process, and analyze large sets of biological data, such as DNA sequences, protein structures, and gene expression patterns.Continuous advances in DNA sequencing technologies~\cite{hu2021next} have led to the rapid accumulation of biological data, creating an urgent need for high-performance computational solutions capable of efficiently managing increasingly larger datasets.

\citet{Shahroodi2022KrakenOnMem:Profiling} describe a hardware/software co-designed framework to accelerate and improve energy consumption of taxonomic profiling. In metagenomics, the main goal is to understand the role of each organism in our environment in order to
improve our quality of life, and taxonomic profiling involves the identification and categorization of the various types of organisms present in a biological sample by analyzing DNA or protein sequences from the sample to determine which species or taxa are represented. The study focuses on boosting performance of table lookup, which is the primary bottleneck in taxonomic profilers, by proposing a processing-in-memory hardware accelerator. Using large-scale simulations, the authors report an average of 63.1\% faster execution and orders of magnitude higher energy efficiency than the  widely used metagenomic analysis tool Kraken2~\cite{wood2019improved} executed on a 128-core server with AMD EPYC 7742 processors  operating at 2.25 GHz. An FPGA was used for prototyping and emulation purposes.

\citet{Corts2022AcceleratedFPGAs} employ FPGAs to accelerate the detection of traces of positive natural selection in genomes. The authors designed a hardware accelerator for the $\omega$ statistic~\cite{kim2004linkage}, which is extensively used in population genetics as an indicator of positive selection. In comparison with a single CPU core,
the FPGA accelerator can deliver up to $57.1\times$ faster
computation of the $\omega$ statistic, using the OmegaPlus~\cite{alachiotis2012omegaplus} software implementation as reference.


%\citet{Ahmad2022Communication-EfficientFlight}



\citet{Malakonakis2020ExploringRAxML} use FPGAs to accelerate the widely used phylogenetics software tool RAxML~\cite{stamatakis2014raxml}. The study implements the Phylogenetic Likelihood Function (PLF), which is used for evaluating phylogenetic trees, on a Xilinx ZCU102 development board and a cloud-based Amazon AWS EC2 F1 instance. The first system (ZCU102) can deploy two accelerator instances, operating at 250MHz, and delivers up to $7.7\times$ faster executions than sequential software execution on a AWS EC2 F1 instance. %Xeon processors. 
The AWS-based accelerated system is $5.2\times$ faster than the same software implementation. %In comparison with previous work by Alachiotis et al.~\cite{alachiotis2009exploring[7_12]}, the implementation on the Xilinx development board is about 2.35x faster. %, but the older technology should certainly be taken into consideration. 



\citet{Alachiotis2021AcceleratingCloud} also target the PLF implementation in RAxML, and propose an optimization method for data movement in PCI-attached accelerators using tree-search algorithms. They developed a software cache controller that leverages data dependencies between consecutive PLF calls to cache data on the accelerator card. In combination with double buffering over PCIe, this approach led to nearly $4\times$ improvement in the performance of an FPGA-based PLF accelerator. Executing the complete RAxML algorithm on an AWS EC2 F1 instance, the authors observed up to $9.2\times$ faster processing of protein data than a $2.7$ GHz Xeon processor in the same cloud environment.

With genomic datasets continuing to expand, bioinformatics analyses are likely to increasingly rely on cloud computing in the future. This shift will be supported by new programming models and frameworks designed to address the data-movement challenges posed by cloud-based hardware accelerators. These accelerators, such as FPGAs and GPUs, need data transfers from the host processor, which can significantly impact execution times and negate gains from computation improvements. Fortunately, similar data-movement concerns exist for both FPGAs and GPUs, and ongoing engineering efforts are likely to converge on common solutions~\cite{Corts2023AGenetics}. This will help bring optimized, hardware-accelerate processing techniques into more widespread use among computational biologists and bioinformaticians in the future.






\section{Research and Development in Industry}
\label{sec:industry}
\begin{table}[t]
\centering
\setlength{\abovecaptionskip}{0.05cm}
\setlength{\belowcaptionskip}{0.2cm}
\caption{Performance on industrial dataset}
\setlength{\tabcolsep}{2mm}{
\resizebox{0.75\textwidth}{!}{
\begin{tabular}{c|c|c|c|c|c}
\toprule
    \textbf{Metric} & \textbf{PRAUC}$\uparrow$ & \textbf{PCOC}$\uparrow$ & \textbf{AUC}$\uparrow$ & \textbf{LogLoss}$\downarrow$ & \textbf{Params}$\downarrow$ \\ \midrule
    \textbf{baseline} & 0.91913 & 0.96561 & 0.84141 & 0.40721 & 3142.3MB \\
    \textbf{MEC} & \textbf{0.91918} & \textbf{0.96673} & \textbf{0.84143} & \textbf{0.40683} & \textbf{7.014MB} \\
\bottomrule
\end{tabular}
}}
\label{tab:industry}
% \vspace{-10pt}
\end{table}


%\section{Discussion}
%\section{Discussion of Assumptions}\label{sec:discussion}
In this paper, we have made several assumptions for the sake of clarity and simplicity. In this section, we discuss the rationale behind these assumptions, the extent to which these assumptions hold in practice, and the consequences for our protocol when these assumptions hold.

\subsection{Assumptions on the Demand}

There are two simplifying assumptions we make about the demand. First, we assume the demand at any time is relatively small compared to the channel capacities. Second, we take the demand to be constant over time. We elaborate upon both these points below.

\paragraph{Small demands} The assumption that demands are small relative to channel capacities is made precise in \eqref{eq:large_capacity_assumption}. This assumption simplifies two major aspects of our protocol. First, it largely removes congestion from consideration. In \eqref{eq:primal_problem}, there is no constraint ensuring that total flow in both directions stays below capacity--this is always met. Consequently, there is no Lagrange multiplier for congestion and no congestion pricing; only imbalance penalties apply. In contrast, protocols in \cite{sivaraman2020high, varma2021throughput, wang2024fence} include congestion fees due to explicit congestion constraints. Second, the bound \eqref{eq:large_capacity_assumption} ensures that as long as channels remain balanced, the network can always meet demand, no matter how the demand is routed. Since channels can rebalance when necessary, they never drop transactions. This allows prices and flows to adjust as per the equations in \eqref{eq:algorithm}, which makes it easier to prove the protocol's convergence guarantees. This also preserves the key property that a channel's price remains proportional to net money flow through it.

In practice, payment channel networks are used most often for micro-payments, for which on-chain transactions are prohibitively expensive; large transactions typically take place directly on the blockchain. For example, according to \cite{river2023lightning}, the average channel capacity is roughly $0.1$ BTC ($5,000$ BTC distributed over $50,000$ channels), while the average transaction amount is less than $0.0004$ BTC ($44.7k$ satoshis). Thus, the small demand assumption is not too unrealistic. Additionally, the occasional large transaction can be treated as a sequence of smaller transactions by breaking it into packets and executing each packet serially (as done by \cite{sivaraman2020high}).
Lastly, a good path discovery process that favors large capacity channels over small capacity ones can help ensure that the bound in \eqref{eq:large_capacity_assumption} holds.

\paragraph{Constant demands} 
In this work, we assume that any transacting pair of nodes have a steady transaction demand between them (see Section \ref{sec:transaction_requests}). Making this assumption is necessary to obtain the kind of guarantees that we have presented in this paper. Unless the demand is steady, it is unreasonable to expect that the flows converge to a steady value. Weaker assumptions on the demand lead to weaker guarantees. For example, with the more general setting of stochastic, but i.i.d. demand between any two nodes, \cite{varma2021throughput} shows that the channel queue lengths are bounded in expectation. If the demand can be arbitrary, then it is very hard to get any meaningful performance guarantees; \cite{wang2024fence} shows that even for a single bidirectional channel, the competitive ratio is infinite. Indeed, because a PCN is a decentralized system and decisions must be made based on local information alone, it is difficult for the network to find the optimal detailed balance flow at every time step with a time-varying demand.  With a steady demand, the network can discover the optimal flows in a reasonably short time, as our work shows.

We view the constant demand assumption as an approximation for a more general demand process that could be piece-wise constant, stochastic, or both (see simulations in Figure \ref{fig:five_nodes_variable_demand}).
We believe it should be possible to merge ideas from our work and \cite{varma2021throughput} to provide guarantees in a setting with random demands with arbitrary means. We leave this for future work. In addition, our work suggests that a reasonable method of handling stochastic demands is to queue the transaction requests \textit{at the source node} itself. This queuing action should be viewed in conjunction with flow-control. Indeed, a temporarily high unidirectional demand would raise prices for the sender, incentivizing the sender to stop sending the transactions. If the sender queues the transactions, they can send them later when prices drop. This form of queuing does not require any overhaul of the basic PCN infrastructure and is therefore simpler to implement than per-channel queues as suggested by \cite{sivaraman2020high} and \cite{varma2021throughput}.

\subsection{The Incentive of Channels}
The actions of the channels as prescribed by the DEBT control protocol can be summarized as follows. Channels adjust their prices in proportion to the net flow through them. They rebalance themselves whenever necessary and execute any transaction request that has been made of them. We discuss both these aspects below.

\paragraph{On Prices}
In this work, the exclusive role of channel prices is to ensure that the flows through each channel remains balanced. In practice, it would be important to include other components in a channel's price/fee as well: a congestion price  and an incentive price. The congestion price, as suggested by \cite{varma2021throughput}, would depend on the total flow of transactions through the channel, and would incentivize nodes to balance the load over different paths. The incentive price, which is commonly used in practice \cite{river2023lightning}, is necessary to provide channels with an incentive to serve as an intermediary for different channels. In practice, we expect both these components to be smaller than the imbalance price. Consequently, we expect the behavior of our protocol to be similar to our theoretical results even with these additional prices.

A key aspect of our protocol is that channel fees are allowed to be negative. Although the original Lightning network whitepaper \cite{poon2016bitcoin} suggests that negative channel prices may be a good solution to promote rebalancing, the idea of negative prices in not very popular in the literature. To our knowledge, the only prior work with this feature is \cite{varma2021throughput}. Indeed, in papers such as \cite{van2021merchant} and \cite{wang2024fence}, the price function is explicitly modified such that the channel price is never negative. The results of our paper show the benefits of negative prices. For one, in steady state, equal flows in both directions ensure that a channel doesn't loose any money (the other price components mentioned above ensure that the channel will only gain money). More importantly, negative prices are important to ensure that the protocol selectively stifles acyclic flows while allowing circulations to flow. Indeed, in the example of Section \ref{sec:flow_control_example}, the flows between nodes $A$ and $C$ are left on only because the large positive price over one channel is canceled by the corresponding negative price over the other channel, leading to a net zero price.

Lastly, observe that in the DEBT control protocol, the price charged by a channel does not depend on its capacity. This is a natural consequence of the price being the Lagrange multiplier for the net-zero flow constraint, which also does not depend on the channel capacity. In contrast, in many other works, the imbalance price is normalized by the channel capacity \cite{ren2018optimal, lin2020funds, wang2024fence}; this is shown to work well in practice. The rationale for such a price structure is explained well in \cite{wang2024fence}, where this fee is derived with the aim of always maintaining some balance (liquidity) at each end of every channel. This is a reasonable aim if a channel is to never rebalance itself; the experiments of the aforementioned papers are conducted in such a regime. In this work, however, we allow the channels to rebalance themselves a few times in order to settle on a detailed balance flow. This is because our focus is on the long-term steady state performance of the protocol. This difference in perspective also shows up in how the price depends on the channel imbalance. \cite{lin2020funds} and \cite{wang2024fence} advocate for strictly convex prices whereas this work and \cite{varma2021throughput} propose linear prices.

\paragraph{On Rebalancing} 
Recall that the DEBT control protocol ensures that the flows in the network converge to a detailed balance flow, which can be sustained perpetually without any rebalancing. However, during the transient phase (before convergence), channels may have to perform on-chain rebalancing a few times. Since rebalancing is an expensive operation, it is worthwhile discussing methods by which channels can reduce the extent of rebalancing. One option for the channels to reduce the extent of rebalancing is to increase their capacity; however, this comes at the cost of locking in more capital. Each channel can decide for itself the optimum amount of capital to lock in. Another option, which we discuss in Section \ref{sec:five_node}, is for channels to increase the rate $\gamma$ at which they adjust prices. 

Ultimately, whether or not it is beneficial for a channel to rebalance depends on the time-horizon under consideration. Our protocol is based on the assumption that the demand remains steady for a long period of time. If this is indeed the case, it would be worthwhile for a channel to rebalance itself as it can make up this cost through the incentive fees gained from the flow of transactions through it in steady state. If a channel chooses not to rebalance itself, however, there is a risk of being trapped in a deadlock, which is suboptimal for not only the nodes but also the channel.

\section{Conclusion}
This work presents DEBT control: a protocol for payment channel networks that uses source routing and flow control based on channel prices. The protocol is derived by posing a network utility maximization problem and analyzing its dual minimization. It is shown that under steady demands, the protocol guides the network to an optimal, sustainable point. Simulations show its robustness to demand variations. The work demonstrates that simple protocols with strong theoretical guarantees are possible for PCNs and we hope it inspires further theoretical research in this direction.

%\newpage
\section{Conclusions}
This paper presents an overview of the current research landscape, applications, and future potential of FPGA technology in the Netherlands. It highlights how academia and industry in the Netherlands play an important role in developing new and innovative FPGA-based technologies and solutions to address various important societal challenges ranging from healthcare to power efficiency. 

\subsection{Summary and main findings}
We selected a total of %234
212 relevant FPGA-related papers published in the past 5 years. Most contributions to the national FPGA research effort stem from 16 organizations, including major Dutch universities, research institutes, and % as well as 
industry. Many of these publications are collaborations of national and international partners, mostly European. We note that the survey covers about 1\% of the relevant world-wide FPGA publications (from the same period), with the Netherlands ranking 22nd in the world and 10th in Europe in relevant FPGA-research output. 

We classified the selected papers into five major themes: a) FPGA architecture, with 11 published papers, b) data center infrastructure \& HPC, with 40 papers, c) programming models \& tools, with 15 papers, d) robustness of FPGAs, with 26 papers, and e) applications, with 120 papers. We paid specific attention and reviewed in depth popular FPGA applications, i.e, those applications with a significant number of relevant publications where FPGAs play an important role. We found 49 application papers over 6 subjects; these publications indicate that FPGAs have emerged as powerful accelerators for a wide range of applications, such as machine learning, astronomy, particle physics experiments, quantum computing, space applications, and bioinformatics. 

Our survey revealed ample future directions across all themes. In terms of architecture (see Section ~\ref{sec:archi}), specialized FPGA architectures (e.g., for in-memory computing) as well as embedding FPGA technology as part of (complex) computing systems beyond the computation (e.g., for on-chip and off-chip communication or for memory systems) are  promising paths towards reducing the memory gap of traditional systems. For datacenters and HPC infrastructure (see Section~\ref{sec:HPC}), we identify a promising research direction in FPGA-based acceleration in general, and tooling to enable its seamless integration into large-scale systems in particular; a second research direction should focus on supporting novel communication protocols and technologies, to further improve data movement in supercomputers, datacenters, and the computing continuum. For programming models and tools (see Section~\ref{sec:programming}), the main research developments target high-level tooling for easier development and deployment, accessible for more educated users than HDL experts, and tooling support for performance analysis, modeling, and prediction, which will help solution feasibility analysis and fair comparison against competing technologies. For robustness of FPGAs (see Section~\ref{sec:robustness}), reliability and security are key aspects where more research is needed in architecture and tooling for moving from proof-of-concept to complex (eco)systems, where FPGAs can guarantee such essential non-functional requirements in the most adverse conditions (e.g., for deployment in space). Finally, for applications, future research should focus on the adoption of FPGAs in more application domains where energy-efficient acceleration is essential; we suggest further research towards more efficient developments in AI (for both training acceleration and latency-sensitive deployment), as well as for various computing continuum layers, fast/specialized communication, and HPC kernels. We further expect more developments towards successful acceleration in simulation and digital twin applications, where FPGAs could play an interesting role in a simulation-emulation continuum.     

\subsection{Limitations}
\section{Discussion of Assumptions}\label{sec:discussion}
In this paper, we have made several assumptions for the sake of clarity and simplicity. In this section, we discuss the rationale behind these assumptions, the extent to which these assumptions hold in practice, and the consequences for our protocol when these assumptions hold.

\subsection{Assumptions on the Demand}

There are two simplifying assumptions we make about the demand. First, we assume the demand at any time is relatively small compared to the channel capacities. Second, we take the demand to be constant over time. We elaborate upon both these points below.

\paragraph{Small demands} The assumption that demands are small relative to channel capacities is made precise in \eqref{eq:large_capacity_assumption}. This assumption simplifies two major aspects of our protocol. First, it largely removes congestion from consideration. In \eqref{eq:primal_problem}, there is no constraint ensuring that total flow in both directions stays below capacity--this is always met. Consequently, there is no Lagrange multiplier for congestion and no congestion pricing; only imbalance penalties apply. In contrast, protocols in \cite{sivaraman2020high, varma2021throughput, wang2024fence} include congestion fees due to explicit congestion constraints. Second, the bound \eqref{eq:large_capacity_assumption} ensures that as long as channels remain balanced, the network can always meet demand, no matter how the demand is routed. Since channels can rebalance when necessary, they never drop transactions. This allows prices and flows to adjust as per the equations in \eqref{eq:algorithm}, which makes it easier to prove the protocol's convergence guarantees. This also preserves the key property that a channel's price remains proportional to net money flow through it.

In practice, payment channel networks are used most often for micro-payments, for which on-chain transactions are prohibitively expensive; large transactions typically take place directly on the blockchain. For example, according to \cite{river2023lightning}, the average channel capacity is roughly $0.1$ BTC ($5,000$ BTC distributed over $50,000$ channels), while the average transaction amount is less than $0.0004$ BTC ($44.7k$ satoshis). Thus, the small demand assumption is not too unrealistic. Additionally, the occasional large transaction can be treated as a sequence of smaller transactions by breaking it into packets and executing each packet serially (as done by \cite{sivaraman2020high}).
Lastly, a good path discovery process that favors large capacity channels over small capacity ones can help ensure that the bound in \eqref{eq:large_capacity_assumption} holds.

\paragraph{Constant demands} 
In this work, we assume that any transacting pair of nodes have a steady transaction demand between them (see Section \ref{sec:transaction_requests}). Making this assumption is necessary to obtain the kind of guarantees that we have presented in this paper. Unless the demand is steady, it is unreasonable to expect that the flows converge to a steady value. Weaker assumptions on the demand lead to weaker guarantees. For example, with the more general setting of stochastic, but i.i.d. demand between any two nodes, \cite{varma2021throughput} shows that the channel queue lengths are bounded in expectation. If the demand can be arbitrary, then it is very hard to get any meaningful performance guarantees; \cite{wang2024fence} shows that even for a single bidirectional channel, the competitive ratio is infinite. Indeed, because a PCN is a decentralized system and decisions must be made based on local information alone, it is difficult for the network to find the optimal detailed balance flow at every time step with a time-varying demand.  With a steady demand, the network can discover the optimal flows in a reasonably short time, as our work shows.

We view the constant demand assumption as an approximation for a more general demand process that could be piece-wise constant, stochastic, or both (see simulations in Figure \ref{fig:five_nodes_variable_demand}).
We believe it should be possible to merge ideas from our work and \cite{varma2021throughput} to provide guarantees in a setting with random demands with arbitrary means. We leave this for future work. In addition, our work suggests that a reasonable method of handling stochastic demands is to queue the transaction requests \textit{at the source node} itself. This queuing action should be viewed in conjunction with flow-control. Indeed, a temporarily high unidirectional demand would raise prices for the sender, incentivizing the sender to stop sending the transactions. If the sender queues the transactions, they can send them later when prices drop. This form of queuing does not require any overhaul of the basic PCN infrastructure and is therefore simpler to implement than per-channel queues as suggested by \cite{sivaraman2020high} and \cite{varma2021throughput}.

\subsection{The Incentive of Channels}
The actions of the channels as prescribed by the DEBT control protocol can be summarized as follows. Channels adjust their prices in proportion to the net flow through them. They rebalance themselves whenever necessary and execute any transaction request that has been made of them. We discuss both these aspects below.

\paragraph{On Prices}
In this work, the exclusive role of channel prices is to ensure that the flows through each channel remains balanced. In practice, it would be important to include other components in a channel's price/fee as well: a congestion price  and an incentive price. The congestion price, as suggested by \cite{varma2021throughput}, would depend on the total flow of transactions through the channel, and would incentivize nodes to balance the load over different paths. The incentive price, which is commonly used in practice \cite{river2023lightning}, is necessary to provide channels with an incentive to serve as an intermediary for different channels. In practice, we expect both these components to be smaller than the imbalance price. Consequently, we expect the behavior of our protocol to be similar to our theoretical results even with these additional prices.

A key aspect of our protocol is that channel fees are allowed to be negative. Although the original Lightning network whitepaper \cite{poon2016bitcoin} suggests that negative channel prices may be a good solution to promote rebalancing, the idea of negative prices in not very popular in the literature. To our knowledge, the only prior work with this feature is \cite{varma2021throughput}. Indeed, in papers such as \cite{van2021merchant} and \cite{wang2024fence}, the price function is explicitly modified such that the channel price is never negative. The results of our paper show the benefits of negative prices. For one, in steady state, equal flows in both directions ensure that a channel doesn't loose any money (the other price components mentioned above ensure that the channel will only gain money). More importantly, negative prices are important to ensure that the protocol selectively stifles acyclic flows while allowing circulations to flow. Indeed, in the example of Section \ref{sec:flow_control_example}, the flows between nodes $A$ and $C$ are left on only because the large positive price over one channel is canceled by the corresponding negative price over the other channel, leading to a net zero price.

Lastly, observe that in the DEBT control protocol, the price charged by a channel does not depend on its capacity. This is a natural consequence of the price being the Lagrange multiplier for the net-zero flow constraint, which also does not depend on the channel capacity. In contrast, in many other works, the imbalance price is normalized by the channel capacity \cite{ren2018optimal, lin2020funds, wang2024fence}; this is shown to work well in practice. The rationale for such a price structure is explained well in \cite{wang2024fence}, where this fee is derived with the aim of always maintaining some balance (liquidity) at each end of every channel. This is a reasonable aim if a channel is to never rebalance itself; the experiments of the aforementioned papers are conducted in such a regime. In this work, however, we allow the channels to rebalance themselves a few times in order to settle on a detailed balance flow. This is because our focus is on the long-term steady state performance of the protocol. This difference in perspective also shows up in how the price depends on the channel imbalance. \cite{lin2020funds} and \cite{wang2024fence} advocate for strictly convex prices whereas this work and \cite{varma2021throughput} propose linear prices.

\paragraph{On Rebalancing} 
Recall that the DEBT control protocol ensures that the flows in the network converge to a detailed balance flow, which can be sustained perpetually without any rebalancing. However, during the transient phase (before convergence), channels may have to perform on-chain rebalancing a few times. Since rebalancing is an expensive operation, it is worthwhile discussing methods by which channels can reduce the extent of rebalancing. One option for the channels to reduce the extent of rebalancing is to increase their capacity; however, this comes at the cost of locking in more capital. Each channel can decide for itself the optimum amount of capital to lock in. Another option, which we discuss in Section \ref{sec:five_node}, is for channels to increase the rate $\gamma$ at which they adjust prices. 

Ultimately, whether or not it is beneficial for a channel to rebalance depends on the time-horizon under consideration. Our protocol is based on the assumption that the demand remains steady for a long period of time. If this is indeed the case, it would be worthwhile for a channel to rebalance itself as it can make up this cost through the incentive fees gained from the flow of transactions through it in steady state. If a channel chooses not to rebalance itself, however, there is a risk of being trapped in a deadlock, which is suboptimal for not only the nodes but also the channel.

\section{Conclusion}
This work presents DEBT control: a protocol for payment channel networks that uses source routing and flow control based on channel prices. The protocol is derived by posing a network utility maximization problem and analyzing its dual minimization. It is shown that under steady demands, the protocol guides the network to an optimal, sustainable point. Simulations show its robustness to demand variations. The work demonstrates that simple protocols with strong theoretical guarantees are possible for PCNs and we hope it inspires further theoretical research in this direction.

\subsection{Future Work}
   
Our detailed analysis of the last half-decade of Dutch FPGA research does already emphasize that, while the technology holds tremendous potential, there are ongoing challenges, particularly in terms of development tools, performance predictability, and hardware security. Future research will need to focus on improving these aspects, especially with regards to user-friendly programming models and more robust performance estimation frameworks. The continued investment in FPGA technology will be vital for maintaining the Netherlands' competitive edge and addressing the growing demand for energy-efficient, high-performance computing solutions. Furthermore, fostering open-source tools, as well as deeper industry-academia collaboration, will be essential for sustaining long-term growth and innovation in this rapidly evolving field.

To further enlarge the scope of this analysis, we welcome a broader community-driven research, to survey the developments and needs of FPGA research and innovation at European or even world-wide level. Such analysis will reveal additional collaboration opportunities, cross-theme and cross-domain, across multiple layers of modern computing systems. While we recognize the scale and scope of such an effort are significantly larger, we argue that it is an effective way to map the current landscape of FPGA research, and further focus on relevant research challenges and opportunities.  
%\vspace{-0.2cm}
\section{Impact: Why Free Scientific Knowledge?}
\vspace{-0.1cm}

Historically, making knowledge widely available has driven transformative progress. Gutenberg’s printing press broke medieval monopolies on information, increasing literacy and contributing to the Renaissance and Scientific Revolution. In today's world, open source projects such as GNU/Linux and Wikipedia show that freely accessible and modifiable knowledge fosters innovation while ensuring creators are credited through copyleft licenses. These examples highlight a key idea: \textit{access to essential knowledge supports overall advancement.} 

This aligns with the arguments made by Prabhakaran et al. \cite{humanrightsbasedapproachresponsible}, who specifically highlight the \textbf{ human right to participate in scientific advancement} as enshrined in the Universal Declaration of Human Rights. They emphasize that this right underscores the importance of \textit{ equal access to the benefits of scientific progress for all}, a principle directly supported by our proposal for Knowledge Units. The UN Special Rapporteur on Cultural Rights further reinforces this, advocating for the expansion of copyright exceptions to broaden access to scientific knowledge as a crucial component of the right to science and culture \cite{scienceright}. 

However, current intellectual property regimes often create ``patently unfair" barriers to this knowledge, preventing innovation and access, especially in areas critical to human rights, as Hale compellingly argues \cite{patentlyunfair}. Finding a solution requires carefully balancing the imperative of open access with the legitimate rights of authors. As Austin and Ginsburg remind us, authors' rights are also human rights, necessitating robust protection \cite{authorhumanrights}. Shareable knowledge entities like Knowledge Units offer a potential mechanism to achieve this delicate balance in the scientific domain, enabling wider dissemination of research findings while respecting authors' fundamental rights.

\vspace{-0.2cm}
\subsection{Impact Across Sectors}

\textbf{Researchers:} Collaboration across different fields becomes easier when knowledge is shared openly. For instance, combining machine learning with biology or applying quantum principles to cryptography can lead to important breakthroughs. Removing copyright restrictions allows researchers to freely use data and methods, speeding up discoveries while respecting original contributions.

\textbf{Practitioners:} Professionals, especially in healthcare, benefit from immediate access to the latest research. Quick access to newer insights on the effectiveness of drugs, and alternative treatments speeds up adoption and awareness, potentially saving lives. Additionally, open knowledge helps developing countries gain access to health innovations.

\textbf{Education:} Education becomes more accessible when teachers use the latest research to create up-to-date curricula without prohibitive costs. Students can access high-quality research materials and use LM assistance to better understand complex topics, enhancing their learning experience and making high-quality education more accessible.

\textbf{Public Trust:} When information is transparent and accessible, the public can better understand and trust decision-making processes. Open access to government policies and industry practices allows people to review and verify information, helping to reduce misinformation. This transparency encourages critical thinking and builds trust in scientific and governmental institutions.

Overall, making scientific knowledge accessible supports global fairness. By viewing knowledge as a common resource rather than a product to be sold, we can speed up innovation, encourage critical thinking, and empower communities to address important challenges.

\vspace{-0.2cm}
\section{Open Problems}
\vspace{-0.1cm}

Moving forward, we identify key research directions to further exploit the potential of converting original texts into shareable knowledge entities such as demonstrated by the conversion into Knowledge Units in this work:


\textbf{1. Enhancing Factual Accuracy and Reliability:}  Refining KUs through cross-referencing with source texts and incorporating community-driven correction mechanisms, similar to Wikipedia, can minimize hallucinations and ensure the long-term accuracy of knowledge-based datasets at scale.

\textbf{2. Developing Applications for Education and Research:}  Using KU-based conversion for datasets to be employed in practical tools, such as search interfaces and learning platforms, can ensure rapid dissemination of any new knowledge into shareable downstream resources, significantly improving the accessibility, spread, and impact of KUs.

\textbf{3. Establishing Standards for Knowledge Interoperability and Reuse:}  Future research should focus on defining standardized formats for entities like KU and knowledge graph layouts \citep{lenat1990cyc}. These standards are essential to unlock seamless interoperability, facilitate reuse across diverse platforms, and foster a vibrant ecosystem of open scientific knowledge. 

\textbf{4. Interconnecting Shareable Knowledge for Scientific Workflow Assistance and Automation:} There might be further potential in constructing a semantic web that interconnects publicly shared knowledge, together with mechanisms that continually update and validate all shareable knowledge units. This can be starting point for a platform that uses all collected knowledge to assist scientific workflows, for instance by feeding such a semantic web into recently developed reasoning models equipped with retrieval augmented generation. Such assistance could assemble knowledge across multiple scientific papers, guiding scientists more efficiently through vast research landscapes. Given further progress in model capabilities, validation, self-repair and evolving new knowledge from already existing vast collection in the semantic web can lead to automation of scientific discovery, assuming that knowledge data in the semantic web can be freely shared.

We open-source our code and encourage collaboration to improve extraction pipelines, enhance Knowledge Unit capabilities, and expand coverage to additional fields.

\vspace{-0.2cm}
\section{Conclusion}
\vspace{-0.1cm}

In this paper, we highlight the potential of systematically separating factual scientific knowledge from protected artistic or stylistic expression. By representing scientific insights as structured facts and relationships, prototypes like Knowledge Units (KUs) offer a pathway to broaden access to scientific knowledge without infringing copyright, aligning with legal principles like German \S 24(1) UrhG and U.S. fair use standards. Extensive testing across a range of domains and models shows evidence that Knowledge Units (KUs) can feasibly retain core information. These findings offer a promising way forward for openly disseminating scientific information while respecting copyright constraints.

\section*{Author Contributions}

Christoph conceived the project and led organization. Christoph and Gollam led all the experiments. Nick and Huu led the legal aspects. Tawsif led the data collection. Ameya and Andreas led the manuscript writing. Ludwig, Sören, Robert, Jenia and Matthias provided feedback. advice and scientific supervision throughout the project. 

\section*{Acknowledgements}

The authors would like to thank (in alphabetical order): Sebastian Dziadzio, Kristof Meding, Tea Mustać, Shantanu Prabhat for insightful feedback and suggestions. Special thanks to Andrej Radonjic for help in scaling up data collection. GR and SA acknowledge financial support by the German Research Foundation (DFG) for the NFDI4DataScience Initiative (project number 460234259). AP and MB acknowledge financial support by the Federal Ministry of Education and Research (BMBF), FKZ: 011524085B and Open Philanthropy Foundation funded by the Good Ventures Foundation. AH acknowledges financial support by the Federal Ministry of Education and Research (BMBF), FKZ: 01IS24079A and the Carl Zeiss Foundation through the project "Certification and Foundations of Safe ML Systems" as well as the support from the International Max Planck Research School for Intelligent Systems (IMPRS-IS). JJ acknowledges funding by the Federal Ministry of Education and Research of Germany (BMBF) under grant no. 01IS22094B (WestAI - AI Service Center West), under grant no. 01IS24085C (OPENHAFM) and under the grant DE002571 (MINERVA), as well as co-funding by EU from EuroHPC Joint Undertaking programm under grant no. 101182737 (MINERVA) and from Digital Europe Programme under grant no. 101195233 (openEuroLLM) 


%\newpage
\bibliographystyle{ACM-Reference-Format-num}
\bibliography{references,mendeley}


\end{document}
