\label{themes-section}
Based on the literature found through the process outlined in Section \ref{select-lit-section}, articles with similar subjects or covering similar themes of research are grouped together. The themes are selected in such a way that most publications can be exclusively divided into one of the themes, i.e., the themes should not have significant overlap. Furthermore the themes should effectively separate domain specific research and more generally applicable research. %Finally we have a specific interest in the developments within the domain of data center and HPC research. 
Considering these requirements the following list of themes is selected:
% We start off by separating the literature into two distinct categories: (1) articles which implement an FPGA for a specific application, which covers a wide variety of research purposes, and (2) articles on the advancement of FPGA hardware and tooling software itself. The second of the two is further divided into more specific categories, resulting in the following list of themes:

\begin{enumerate}
    \item FPGA architecture
    \item Robustness of FPGAs
    \item Data center infrastructure \& HPC
    \item Programming models \& tools
    \item Applications
\end{enumerate}

Figure~\ref{fig:theme-distribution} illustrates how these themes cover both general and domain-specific development, and shows whether a theme is hardware or software focused. The theme ``Applications'' focuses on research applying FPGA technology to domain specific problems. This can be in the form of hardware architectures for domain-specific applications, as well as software tools enabling FPGA technology in a specific domain of research. The themes ``Programming models and tools'' and ``FPGA architecture'' focus on research and development of solutions that are generally applicable in a wide range of domains, while the ``Robustness of FPGAs'' and ``Data center infrastructure \& HPC'' themes feature both hardware- and software-focused research. Moreover, these themes focus on a narrower selection of FPGA applications and can, therefore, be considered  domain-specific. 

% add some explanation of figure
\begin{figure}[!htbp]
    \centering
    \includegraphics[width=0.5\textwidth]{figures/theme_distribution_diagram.pdf}
    \caption{The themes that are selected can be differentiated based on their domain-specificity, ranging from very domain-focused to general purpose, and based on whether the main focus is on hardware or software. }
    \label{fig:theme-distribution}
\end{figure}
Based on common subjects in each theme, the themes are further organized into %more specific 
subcategories. Table \ref{tab:overview-themes-most-cited} shows the subcategories and %by which each theme is subdivided, as well as 
the prevalence of each  subject based on the number of published articles. % covered in it. 
Finally, the most influential articles, based on the highest number of citations within each category (Google Scholar), is shown. This selection excludes survey publications.

\begin{table}[!ht]
\centering
\caption{Overview of highly cited papers per category, with number of published articles per theme and category in parentheses.}
\label{tab:overview-themes-most-cited}
{\small
\begin{tabular}{lll}
\textbf{Theme and category} & \textbf{Highly cited publications} & \textbf{Dutch affiliation} \\ \hline
\textbf{\textbf{FPGA architecture (11)}} &  &  \\ \cline{1-1}
Near-memory processing (4) & \citet{Singh2021FPGA-BasedApplications} & Eindhoven University \\
Coarse-grained reconfigurable architecture (4) & \citet{Wijtvliet2019Blocks:Efficiency} & Eindhoven University \\
Network-on-Chip (3) & \citet{RibotGonzalez2020HopliteRT:FPGA} & Eindhoven University \\ \hline
\multicolumn{2}{l}{\textbf{\textbf{Data center infrastructure \& HPC (40)}}}  &  \\ \cline{1-1}
Big data processing and analytics (22) & \citet{Peltenburg2019Fletcher:Arrow} & Delft University of Tech. \\
Distributed computing (5) & \citet{Bielski2018DReDBox:Datacenter} & Sintecs B.V. \\
Optical hardware communication (9) & \citet{Yan2018HiFOST:Switches} & Eindhoven University \\
High performance computing (4) & \citet{Katevenis2018NextDevelopment} & MonetDB Solutions \\ \hline
\multicolumn{2}{l}{\textbf{\textbf{Programming models \& tools (15)}} }  &  \\ \cline{1-1}
Programming models and frameworks (8) & \citet{Peltenburg2020Tydi:Streams} & Delft University of Tech.  \\
Performance prediction (7) & \citet{Yasudo2018PerformancePlatforms} & University of Amsterdam \\ \hline
\textbf{\textbf{Robustness of FPGAs (26)}} &  &  \\ \cline{1-1}
Reliability (12) & \citet{Du2019UltrahighFPGA} & ESTEC \\
Hardware security (14) & \citet{Labafniya2020OnPrevention} & Delft University of Tech. \\ \hline
\textbf{\textbf{Applications (49)}} &  &  \\ \cline{1-1}
Machine learning (12) & \citet{Rocha2020BinaryWrist-PPG} & IMEC NL \\
Astronomy (11) & \citet{Ashton2020ATelescopes} & University of Amsterdam \\
Particle physics experiments (7) & \citet{FernandezPrieto2020PhaseExperiment} & Nikhef \\
Quantum computing (5) & \citet{Philips-nat-2022} & Delft University of Tech. \\
Space (9) & \citet{Barrios2020SHyLoCMissions} & ESTEC \\
Bioinformatics (5) & \citet{Malakonakis2020ExploringRAxML} & University of Twente \\ \hline
\end{tabular}
}
\end{table}


Figure~\ref{fig:org-publish-per-theme} illustrates an overview of publications per theme for each organization with more than one publication. It is clear that most major contributors to FPGA research publish mostly application-specific research. Out of the major contributors, Delft University of Technology focuses more on the ``Data center \& infrastructure'' domain, while Eindhoven University of Technology is a larger contributor to the ``FPGA architecture'' theme. A brief description of each theme is provided below. 


\begin{figure}[!htb]
    \centering
    \includegraphics[width=\textwidth]{figures/per_chapter_no_papers_2.pdf}
    \caption{Number of publications per theme for each organization with more than one relevant publication.}
    \label{fig:org-publish-per-theme}
\end{figure}


\begin{itemize}
    \item {\bf FPGA architecture}: This research theme covers literature related to the design of novel digital hardware architectures. Efficient architectures, fast on-chip memory access, coarse grained hardware design, and partially reconfigurable hardware are covered in this theme.
\item {\bf Data center infrastructure \& HPC}: This theme includes literature on FPGAs used in high-performance computing environments. This covers papers on the rapid processing of big data, and research towards distributed computing infrastructures deploying FPGAs. Furthermore, research focusing on employing FPGAs for processing communication between computing nodes using optical links is also covered here.
\item {\bf Programming models \& tools}:
This theme covers literature related to tools and models used to program FPGAs, ranging from research on high-level synthesis (HLS) tools to tools that enable accessible hardware acceleration of conventional software. This theme also features research efforts on tools for accurate performance prediction of synthesized 
FPGA solutions.
\item {\bf Robustness of FPGAs}:
This theme covers literature regarding the reliability and resilience of FPGAs to specific environments. Specifically, resilience to radiation in environments where this is prevalent is a common subject. Furthermore, this theme expands on the security of FPGAs with regards to cyberattacks.
\item {\bf Applications}: The literature on specific applications using FPGAs is more extensive than that of the other themes. This is expected since FPGAs can be applied in various fields, whereas the advancement of FPGA architectures and development tools is generally a more narrow area of research. 
%Within this theme, 
Machine learning has been the main focus in recent years.
\end{itemize}

%\paragraph{Hardware and architecture}
%This research theme covers literature related to the design of novel digital hardware architecture. Efficient architectures, fast on-chip memory access, coarse grained hardware design and partially reconfigurable hardware are subjects that are covered within this theme.

%\paragraph{Robustness of FPGAs}
%In the theme of robustness the literature regarding the reliability and resilience of FPGAs to specific environments is covered. Specifically the resilience to radiation in environments where this is prevalent is a common subject. Furthermore, this theme expands on the security of FPGAs with regards to cyberattacks.

%\paragraph{Data center \& infrastructure}
%This theme encapsulates the literature on FPGAs used in high-performance computing environments. This includes literature on the rapid processing of big data and research towards distributed computing infrastructures implementing FPGAs. Furthermore, ample research is being done towards using FPGAs for processing the communication between computing nodes using optical links. This is also covered under this theme.

%\paragraph{Programming models \& tools}
%This research theme covers literature related to the tools and models used to program FPGAs. This ranges from research towards HLS tools, to tools which allow accessible hardware acceleration of conventional software. This theme also features research towards tools for efficient and accurate performance prediction of synthesized FPGA solutions.

%\paragraph{Applications of FPGAs}
%The literature on specific applications using FPGAs is more numerous than other themes. This is not unexpected since FPGAs can be applied in numerous fields, whilst the advancement of FPGAs and FPGA associated tools itself is a more narrow area of research. Within this theme, machine learning is the most researched subject in the recent past.