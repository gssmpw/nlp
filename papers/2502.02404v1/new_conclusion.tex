This paper presents an overview of the current research landscape, applications, and future potential of FPGA technology in the Netherlands. It highlights how academia and industry in the Netherlands play an important role in developing new and innovative FPGA-based technologies and solutions to address various important societal challenges ranging from healthcare to power efficiency. 

\subsection{Summary and main findings}
We selected a total of %234
212 relevant FPGA-related papers published in the past 5 years. Most contributions to the national FPGA research effort stem from 16 organizations, including major Dutch universities, research institutes, and % as well as 
industry. Many of these publications are collaborations of national and international partners, mostly European. We note that the survey covers about 1\% of the relevant world-wide FPGA publications (from the same period), with the Netherlands ranking 22nd in the world and 10th in Europe in relevant FPGA-research output. 

We classified the selected papers into five major themes: a) FPGA architecture, with 11 published papers, b) data center infrastructure \& HPC, with 40 papers, c) programming models \& tools, with 15 papers, d) robustness of FPGAs, with 26 papers, and e) applications, with 120 papers. We paid specific attention and reviewed in depth popular FPGA applications, i.e, those applications with a significant number of relevant publications where FPGAs play an important role. We found 49 application papers over 6 subjects; these publications indicate that FPGAs have emerged as powerful accelerators for a wide range of applications, such as machine learning, astronomy, particle physics experiments, quantum computing, space applications, and bioinformatics. 

Our survey revealed ample future directions across all themes. In terms of architecture (see Section ~\ref{sec:archi}), specialized FPGA architectures (e.g., for in-memory computing) as well as embedding FPGA technology as part of (complex) computing systems beyond the computation (e.g., for on-chip and off-chip communication or for memory systems) are  promising paths towards reducing the memory gap of traditional systems. For datacenters and HPC infrastructure (see Section~\ref{sec:HPC}), we identify a promising research direction in FPGA-based acceleration in general, and tooling to enable its seamless integration into large-scale systems in particular; a second research direction should focus on supporting novel communication protocols and technologies, to further improve data movement in supercomputers, datacenters, and the computing continuum. For programming models and tools (see Section~\ref{sec:programming}), the main research developments target high-level tooling for easier development and deployment, accessible for more educated users than HDL experts, and tooling support for performance analysis, modeling, and prediction, which will help solution feasibility analysis and fair comparison against competing technologies. For robustness of FPGAs (see Section~\ref{sec:robustness}), reliability and security are key aspects where more research is needed in architecture and tooling for moving from proof-of-concept to complex (eco)systems, where FPGAs can guarantee such essential non-functional requirements in the most adverse conditions (e.g., for deployment in space). Finally, for applications, future research should focus on the adoption of FPGAs in more application domains where energy-efficient acceleration is essential; we suggest further research towards more efficient developments in AI (for both training acceleration and latency-sensitive deployment), as well as for various computing continuum layers, fast/specialized communication, and HPC kernels. We further expect more developments towards successful acceleration in simulation and digital twin applications, where FPGAs could play an interesting role in a simulation-emulation continuum.     

\subsection{Limitations}
This work identifies signal collapse as a critical bottleneck in one-shot neural network pruning. Performance loss in pruned networks is due to \textbf{signal collapse} in addition to the removal of critical parameters. We propose \textbf{REFLOW} (\textbf{Re}storing \textbf{F}low of \textbf{Low}-variance signals), a simple yet effective method that mitigates signal collapse without computationally expensive weight updates. By focusing on signal preservation, REFLOW highlights the importance of mitigating signal collapse in sparse networks and enables magnitude pruning to match or surpass state-of-the-art one-shot pruning methods such as CHITA, CBS, and WF.

REFLOW consistently achieves state-of-the-art accuracy across diverse architectures, restoring ResNeXt-101 from under 4.1\% to 78.9\% top-1 accuracy at 80\% sparsity on ImageNet. Its lightweight design makes it a practical solution for both research and deployment, delivering high-quality sparse models without the overhead of traditional approaches. These findings challenge the traditional emphasis on weight selection strategies and underscore the critical role of signal propagation for achieving high-quality sparse networks in the context of one-shot pruning.




\subsection{Future Work}
   
Our detailed analysis of the last half-decade of Dutch FPGA research does already emphasize that, while the technology holds tremendous potential, there are ongoing challenges, particularly in terms of development tools, performance predictability, and hardware security. Future research will need to focus on improving these aspects, especially with regards to user-friendly programming models and more robust performance estimation frameworks. The continued investment in FPGA technology will be vital for maintaining the Netherlands' competitive edge and addressing the growing demand for energy-efficient, high-performance computing solutions. Furthermore, fostering open-source tools, as well as deeper industry-academia collaboration, will be essential for sustaining long-term growth and innovation in this rapidly evolving field.

To further enlarge the scope of this analysis, we welcome a broader community-driven research, to survey the developments and needs of FPGA research and innovation at European or even world-wide level. Such analysis will reveal additional collaboration opportunities, cross-theme and cross-domain, across multiple layers of modern computing systems. While we recognize the scale and scope of such an effort are significantly larger, we argue that it is an effective way to map the current landscape of FPGA research, and further focus on relevant research challenges and opportunities.  