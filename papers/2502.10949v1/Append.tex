\section*{Appendix: Proof of Theorems from Section~\ref{sec_a221}}


\noindent\underline{\bf Proof of Lemma~\ref{lem_1}:}\\
Let $z(t)=y(t+T)=y(\eta)$, where $\eta=t+T$, for $t\in\mbb R$. Then
\begin{align}\label{eq_64}
  &
  \frac{dz}{dt} = \frac{dy}{d\eta} = f(y(\eta),\eta) = f(z(t),t+T) = f(z,t),
\end{align}
where we have used equations~\eqref{eq_1} and~\eqref{eq_a9}.
If $y(t_0)=y(t_0+T))$, then
\begin{align}\label{eq_65}
  z(t_0) = y(t_0+T) = y(t_0)=y_0.
\end{align}
Comparing the initial value problem consisting of~\eqref{eq_64}--\eqref{eq_65} and the problem~\eqref{eq_1},
and in light of the uniqueness under Assumption~\ref{ass_1},
we conclude that $z(t)=y(t)$ for all $t\in\mbb R$.
It follows that $y(t)$ is a periodic function with a period $T$.


\vspace{8pt}
\noindent\underline{\bf Proof of Theorem~\ref{thm_a1}:}\\
Let $\Psi(y_0,t_0,\xi)=\psi(y_0,t_0+T,\xi)=\psi(y_0,\eta,\xi)$,
where $\eta=t_0+T$, for
$(y_0,t_0,\xi)\in\mbb R^n\times\mbb R\times[0,h_{\max}]$.
Then
\begin{subequations}\label{eq_66}
  \begin{align}
    &
  \frac{\partial\Psi}{\partial\xi}
  = \left.\frac{\partial\psi}{\partial\xi}\right|_{(y_0,\eta,\xi)}
  = f(\psi(y_0,\eta,\xi),\eta+\xi)
  = f(\Psi(y_0,t_0,\xi),t_0+\xi+T) = f(\Psi,t_0+\xi), \\
  &
  \Psi(y_0,t_0,0) = \psi(y_0,t_0+T,0) = y_0,
\end{align}
\end{subequations}
where we have used equation~\eqref{eq_9} and the periodicity of $f(y,t)$.
Comparing the problems~\eqref{eq_66} and~\eqref{eq_9},
and in light of the uniqueness under Assumption~\ref{ass_1},
we conclude that $\Psi(y_0,t_0,\xi)=\psi(y_0,t_0,\xi)$
for all $(y_0,t_0,\xi)\in\mbb R^n\times\mbb R\times[0,h_{\max}]$.
It follows that $\psi(y_0,t_0,\xi)$ is periodic with respect to
$t_0$ with a period $T$
on $(y_0,t_0,\xi)\in\mbb R^n\times\mbb R\times[0,h_{\max}]$.

\vspace{8pt}
\noindent\underline{\bf Proof of Theorem~\ref{thm_a2}:}\\
Let $\Psi(y_0,t_0,\xi)=\psi(y_0+L_i\mbs e_i,t_0,\xi)=\psi(z_0,t_0,\xi)$,
where $z_0=y_0+L_i\mbs e_i\in\mbb R^n$, for
all $(y_0,t_0,\xi)\in\mbb R^n\times\mbb R\times[0,h_{\max}]$.
Then
\begin{subequations}\label{eq_67}
  \begin{align}
    &
    \frac{\partial\Psi}{\partial\xi}
    = \left.\frac{\partial\psi}{\partial\xi}\right|_{(z_0,t_0,\xi)}
    = f(\psi(z_0,t_0,\xi),t_0+\xi)
    = f(\Psi(y_0,t_0,\xi),t_0+\xi), \\
    &
    \Psi(y_0,t_0,0) = \psi(y_0+L_i\mbs e_i,t_0,0) = y_0+L_i\mbs e_i,
  \end{align}
\end{subequations}
where we have used equation~\eqref{eq_9}.

Let $\Phi(y_0,t_0,\xi) = \psi(y_0,t_0,\xi) + L_i\mbs e_i$,
for all $(y_0,t_0,\xi)\in\mbb R^n\times\mbb R\times[0,h_{\max}]$.
Then
\begin{subequations}\label{eq_68}
  \begin{align}
    &
    \frac{\partial\Phi}{\partial\xi}
    = \left.\frac{\partial\psi}{\partial\xi}\right|_{(y_0,t_0,\xi)}
    = f(\psi(y_0,t_0,\xi),t_0+\xi)
    = f(\psi(y_0,t_0,\xi)+L_i\mbs e_i,t_0+\xi) \notag \\
    &\qquad\qquad\qquad\qquad
    = f(\Phi(y_0,t_0,\xi),t_0+\xi), \\
    &
    \Phi(y_0,t_0,0) = \psi(y_0,t_0,0)+L_i\mbs e_i = y_0+L_i\mbs e_i,
  \end{align}
\end{subequations}
where we have used equation~\eqref{eq_9} and
the periodicity condition~\eqref{eq_a10} for $f(y,t)$.

Comparing the problem~\eqref{eq_67} with the problem~\eqref{eq_68},
and in light of the uniqueness under Assumption~\ref{ass_1},
we conclude that $\Psi(y_0,t_0,\xi)=\Phi(y_0,t_0,\xi)$
for all $(y_0,t_0,\xi)\in\mbb R^n\times\mbb R\times[0,h_{\max}]$.
This leads to equation~\eqref{eq_a11}.
