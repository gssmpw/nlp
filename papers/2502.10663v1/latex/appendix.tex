\section{Appendix}

\subsection{Scientific Artifacts}
\label{sec:copyright}
We list the name, source, and license for each of the scientific artifact we use below.

\begin{table}[h]
    \centering
    \resizebox{\columnwidth}{!}{
    \begin{tabular}{l l l}
        \hline
        \textbf{Name} & \textbf{Source} & \textbf{License} \\
        \hline
        iNaturalist & \citet{van2018inaturalist} & iNaturalist terms of service \\
        Birds & \citet{WahCUB_200_2011} & Open to non-commercial research \\
        UnRel & \citet{peyre2017weakly} & UnRel code and data license \\
        SPICE & \citet{hong2018inferring} & GNU Affero General license v3.0 \\
        CLIPScore & \citet{hessel2021clipscore} & MIT license \\
        ROUGE & \citet{lin2004rouge} & Apache License v2.0 \\
        BLEU & \citet{papineni2002bleu} & Apache License v2.0 \\
        \hline
    \end{tabular}
    }
    \caption{Scientific artifact sources and licenses.}
    \label{tab:dataset_licenses}
\end{table}


\subsection{MTurk Details}
\label{sec:mturk}
The workers are from diverse English-speaking backgrounds, and all of them are MTurk Master qualifiers. They were paid \$0.05 for each example. based on the average completion time for each task, the estimated wage rate is \$9/hour, which is higher than the US minimum wage (\$7.25/hour). All data we use are granted for research purposes. Figure~\ref{fig:MTurk} shows the interface for our MTurk evaluation.

\begin{figure}[h!]
    \centering
    \includegraphics[width=\columnwidth]{figures/mturk.pdf}
    \caption{An example question that is sent to workers.}
    \label{fig:MTurk}
\end{figure}