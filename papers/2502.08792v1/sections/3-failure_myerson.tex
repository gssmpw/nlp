\section{Bayesian Update and Applying Myerson}\label{sec:failure_myerson}


In our setting, the seller obtains the signals $s_i$ prior to selecting the mechanism. After obtaining $s_i$, the seller's posterior belief about $v_i$ is given by:
\begin{equation}\label{eq:density-f} f_{\gamma_i,s_i}^i(v) = \gamma_i \cdot f_i(v) + (1-\gamma_i)\cdot \delta_{s_i}(v),\end{equation}
where $\delta_{s_i}(\cdot)$ is the Dirac delta function that places a unit of mass at $s_i$ and zero mass everywhere else. Equivalently, 
\begin{equation}\label{eq:cumulative-F}
F_{\gamma_i,s_i}^i(v) = \begin{cases}
\gamma_i \cdot f_i(v) \quad \text{for $v<s_i$},\\
\gamma_i \cdot f_i(v) + (1-\gamma_i) \quad \text{for $v \geq s_i$}.
\end{cases}
\end{equation}

The question we aim to address can thus be rephrased as what is the revenue-maximizing auction when the valuation of buyer $i$ is drawn according to $F_{\gamma_i,s_i}^i$. 

\vspace{.1in}
\noindent \textbf{On the distribution of hallucinations.} We will assume throughout the paper that the value $v_i$ and any potential hallucination $w_i$ are drawn from the same distribution. However, if we were to assume that the value were drawn from density $g_i$ and the hallucination from density $f_i$, where these distributions are absolutely continuous with respect to each other, we could obtain a similar formula via Bayesian updating. Let $Z_i$ represent whether a hallucination occurred. The posterior density would then be given by:
\begin{eqnarray*}
    f_{\gamma_i,s_i}^i(v) &=&  P(Z_i \mid s_i) \cdot f_{\gamma_i,s_i}^i(v \mid Z_i) +  P(\hbox{not } Z_i \mid s_i) \cdot f_{\gamma_i,s_i}^i(v \mid \hbox{not }Z_i)\\ 
    &=& P(Z_i \mid s_i)\cdot f_i(v) + P(\hbox{not }Z_i \mid s_i)\cdot \delta_{s_i}(v)\\ 
    &=& \frac{\gamma_i \cdot f_i(s_i)}{\gamma_i \cdot f_i(s_i) + (1-\gamma_i) \cdot g_i(s_i)}f_i(v) + \frac{(1-\gamma_i) \cdot g(s_i)}{\gamma_i \cdot f_i(s_i) + (1-\gamma_i) \cdot g_i(s_i)}\delta_{s_i}(v)\\
    &=& \widetilde {\gamma}^i_{s_i} \cdot f(v) + (1-\widetilde {\gamma}^i_{s_i}) \cdot \delta_{s_i}(v), 
\end{eqnarray*}
where
$\widetilde \gamma_{s_i} = \left(1+\frac{1-\gamma_i}{\gamma_i}\frac{g_i(s_i)}{f_i(s_i)}\right)^{-1}$. That is, our results from the rest of the paper would apply if we replace $\gamma_i$ with $\widetilde \gamma_{s_i}$.

\subsection{Applying Myerson}

\citet{myerson1981optimal} tells us that in a private values setting, the revenue-maximizing auction is given by calculating the virtual value of each agent (which might require ironing) and then allocating the item to the agent with the highest non-negative virtual value, or discarding the item if all of the virtual values are negative. Since virtual values are computed separately for each buyer, we will suppress the buyer index $i$ from the notation whenever possible to lighten the notational burden. 

For a given density $f$ and cumulative distribution $F$, the pre-ironing virtual value function is $\varphi_F(v) = v - (1-F(v))/f(v)$. For the density and cumulative distributions given by Eqs. \eqref{eq:density-f}
 and \eqref{eq:cumulative-F}, we have:
 \[ \varphi_{F_{\gamma,s}}(v) =  
 \begin{cases}
     v - \frac{1/\gamma - F(v)}{f(v)}, &\hbox{ for } v < s,\\
     v - \frac{1 - F(v)}{f(v)}, &\hbox{ for } v > s.\\
 \end{cases}\]
 We note that the virtual value function is not well-defined at $s$, but we will ignore this issue for now since that is a single point. The function $\varphi_{F_{\gamma,s}}$ does not need to be ironed after $s$ since $\varphi_{F_{\gamma,s}}(v) = \varphi_{F} (v)$ for $v > s$ and we have assumed $F$ is regular. Ironing could be necessary before $s$ depending on the choice of $F$. 

 Let's apply this to single-buyer, uniform over $[0,1]$ case. For this particular $F$, we obtain:
  \begin{equation}\label{eq:misleading-F} \varphi_{F_{\gamma,s}}(v) =  
 \begin{cases}
     2v - 1/\gamma, &\hbox{ for } v < s,\\
     2v - 1, &\hbox{ for } v > s.\\
 \end{cases}\end{equation}
 For this particular distribution, ironing is not necessary before $s$ since $2v - 1/\gamma$ is an increasing function of $v$. 
 Consider the special case $s=1/2-\epsilon$ and $\gamma = \epsilon$, for a small $\epsilon$. Eq. \eqref{eq:misleading-F} crosses zero at $v=1/2$, implying that the optimal price is $1/2$. However, this cannot be the correct optimal price. The revenue generated by this price is bounded above by $\epsilon$ since it requires $s$ to be a hallucination as a necessary condition for a sale to occur. Meanwhile, using the signal $1/2-\epsilon$ as the price would generate at least $(1-\epsilon)\cdot(1/2-\epsilon)$ in revenue. 

 It turns out that ignoring what occurred at $s$, where the density $f_{\gamma,s}$ is not well-defined, and applying Myerson's technique naively was a mistake. To obtain a correct optimal auction, we will need to use a more sophisticated characterization of optimal auctions that applies for distributions that do not admit densities. 