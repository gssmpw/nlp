\section{The Model}\label{sec:model}

A seller has one indivisible good to sell, and there are $n$ potential buyers. Each buyer $i$ has a private value $v_i$, which is drawn from a cumulative distribution $F_i$. The distributions $F_i$ are assumed to satisfy all of the assumptions as in \citet{myerson1981optimal}: they admit densities $f_i$, which are strictly positive everywhere within a support $[a_i,b_i]$. We will also assume the value distributions are regular.  

\begin{assumption}[Regularity]\label{ass:regular} For every $i$, the virtual value function $v-(1-F_i(v))/f_i(v)$ is assumed to be non-decreasing over the support of buyer $i$'s valuation. 
\end{assumption}

The seller has access to a value prediction technology, which generates a signal $s_i$ for each $i$. The signal is a hallucination with probability $\gamma_i \in (0,1)$. If the signal is a hallucination, then $s_i = w_i$, where $w_i$ is a random variable also drawn from distribution $F_i$ that is independent of buyer $i$'s value $v_i$ (we discuss the case where $w_i$ drawn from a different distribution than $v_i$ in  \Cref{sec:failure_myerson}). If the signal is not a hallucination, then the signal is assumed to be accurate: $s_i = v_i$. The seller is assumed to know the values $\gamma_i$, but not whether a given realization is a hallucination or not. We assume that the realizations of hallucinations, $v_i$ and $w_i$ are independent across buyers. 

We will use $\bm{\gamma}$ and $\bm{s}$ to represent the vectors of hallucination probabilities and signals, respectively. Given a signal, the seller can perform a Bayesian update to obtain what we call the posterior distribution of a buyer's value. We will denote by $\bm{F}_{\bm{\gamma},\bm{s}}$ the posterior distribution of the buyers' values and by $\bm{F}_{\bm{\gamma},\bm{s},-i}$ the posterior distribution of the buyers' values excluding the $i^{th}$ buyer.
 
The question we aim to address in this paper is what is the seller's revenue-maximizing mechanism in the presence of this value prediction technology. A mechanism is defined by a pair $(\bm{x},\bm{p})$, where $\bm{x}$ (resp. $\bm{p}$) is an allocation (resp. payment) function which takes as input the vector of reported types $\bm{\theta}$ and the vector of observed signals $\bm{s}$ and outputs the vector of probability of allocation (resp. of payment) for each buyer. We assume that all agents have quasi-linear utilities. For a given vector of signals $\bm{s}$, we will explore the following problem:
\begin{subequations}
\label{eq:optimal_mechanism}
\begin{alignat}{2}
&\!\sup_{(\bm{x},\bm{p})} &\;& \mathbb{E}_{\bm{\theta} \sim \bm{F_{\gamma,s}}} \left[  \sum_{i=1}^n  p_i(\bm{\theta},\bm{s}) \right]    \\
&\text{s.t.} &      &  \mathbb{E}_{\bm{\theta_{-i}} \sim \bm{F}_{\bm{\gamma},\bm{s},-i}} \left[ \theta_i \cdot x_i(\theta_i,\bm{\theta}_{-i},\bm{s}) - p_i(\theta_i,\bm{\theta}_{-i},\bm{s}) \right] \nonumber \\ 
 &  &  & \qquad \geq \mathbb{E}_{\bm{\theta_{-i}} \sim \bm{F}_{\bm{\gamma},\bm{s},-i}} \left[ \theta_i \cdot x_i(\theta'_i,\bm{\theta}_{-i},\bm{s}) - p_i(\theta'_i,\bm{\theta}_{-i},\bm{s}) \right] \quad \text{for every $i, \theta_i, \theta'_i$,} \label{eq:IC} \\
 &  &  &\mathbb{E}_{\bm{\theta_{-i}} \sim \bm{F}_{\bm{\gamma},\bm{s},-i}} \left[ \theta_i \cdot x_i(\theta_i,\bm{\theta}_{-i},\bm{s}) - p_i(\theta_i,\bm{\theta}_{-i},\bm{s}) \right] \geq 0 \quad \text{for every $i, \theta_i$,} \label{eq:IR}\\
& & & \sum_{i=1}^n x_i(\bm{\theta}) \leq 1 \quad \text{for every $\bm{\theta}$.}
\end{alignat}
\end{subequations}

 \noindent \textbf{Signal-revealing direct mechanisms.} Problem \eqref{eq:optimal_mechanism} specifies the problem of finding the optimal signal-revealing direct mechanism. A direct mechanism is one where the seller chooses an incentive-compatible allocation and payment scheme, and asks the buyers to reveal their types. In standard mechanism design, restricting to direct mechanisms is without loss of optimality \citep{myerson1981optimal}. We define a signal-revealing mechanism to be one where the seller shares the signals alongside the allocation and payment rules. Exploring non-signal-revealing mechanisms is a potentially difficult problem, as the choice of allocation and payment rule will reveal the signals unless the seller explicitly pools signals (i.e., chooses a mechanism that is at least partially non-responsive to signals). We leave the question of whether restricting attention to signal-revealing mechanisms is without loss of optimality open for future work. Note that by assuming the mechanism is signal-revealing we made the formulation relatively straightforward: both the objective and the IC and IR constraints use the posterior distributions given signals rather than the priors.\\

\noindent \textbf{On the correctness of non-hallucinatory signals.}
A natural question regarding this model is why we assume that, when a signal is not a hallucination, it equals the buyer's private value. In reality, errors from deep neural network models are likely a combination of hallucinations and classical Gaussian noise. We analyze pure hallucination in this paper in order to achieve a clean characterization. If we added a Gaussian noise on top of the hallucination, the answer would not be as crisp as the near-decomposition obtained in \Cref{thm:main}. This strict separation between hallucinations and Gaussian noise also allows for a sharp comparison of their respective implications (\Cref{fig:single_buyer}).
 
 
