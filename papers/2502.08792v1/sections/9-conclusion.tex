\section{Conclusion}


In this paper, we studied how Bayesian mechanism design can be adapted to address the challenges posed by hallucination-prone predictions generated by modern machine learning models. By introducing a novel Bayesian framework, we modeled these imperfect signals and rigorously characterized the structure of optimal mechanisms, extending classical results like those of \citet{myerson1981optimal} to settings where posterior distributions lack continuous densities. Our findings provide new insights into how sellers can navigate uncertainty and optimize revenue in environments shaped by unreliable predictions.

Our framework has three main implications.  First, it bridges the gap between traditional auction theory and contemporary machine learning applications, offering a pathway to integrate uncertain predictive signals into practical mechanism design. Second, our comparative analysis with an alternative model, the value-with-noise model, underscores the sensitivity of optimal mechanisms to the underlying assumptions about signal generation, thereby encouraging careful model selection in real-world implementations. Finally, in contrast with the now classical formulation in the algorithm with prediction literature which assumes that advice are either correct or adversarially chosen, our Bayesian framework captures the fact that when the prediction of a machine learning model is wrong, it is in fact ``randomly'' wrong: we believe that exploring this paradigm for other problem classes could design algorithms which are not tailored towards worst-case analyses. 

Despite these contributions, several exciting questions remain. A critical open question lies in analyzing non-direct mechanisms, where signals are not directly disclosed to buyers and strategic interactions become significantly more complex. Understanding the revenue implications (if any) and computational challenges in such settings would greatly add to the value of our framework. Additionally, our results assume that the hallucination probability is known to the seller; relaxing this assumption to consider uncertainty in hallucination probabilities could further align the model with real-world applications. 