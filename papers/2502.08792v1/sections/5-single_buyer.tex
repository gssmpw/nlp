\section{The Single Buyer Case}\label{sec:single-buyer}



In this section, we first leverage \Cref{sec:main} to study the structure of the optimal mechanism for a single buyer. We then, compare the mechanism obtained in our model of hallucination-prone signals with another model which corresponds to the classical model of Gaussian noise.

\subsection{Optimal Mechanism for One Buyer}

An important implication of \Cref{thm:main} is the following characterization of the optimal mechanism for a single buyer. In this setting, the optimal mechanism is a posted price. 

\begin{proposition}
\label{cor:optimal_price}
Assume $n = 1$ and $F$ is regular on $[a,b]$ with continuous density. Then, for any $s \in [a,b]$ and any $\gamma \in [0,1]$, there exist two thresholds $L_{\gamma}$ and $U_{\gamma}$ such that the optimal price satisfies:
\[ p^* = 
\begin{cases}
    p^{\hbox{ignore}} &\hbox{ if } s < L_{\gamma},\\
    s &\hbox{ if } L_{\gamma} \leq s < U_{\gamma},\\
    p^{\hbox{cap}}&\hbox{ if } s \geq U_{\gamma},
\end{cases}\]
where $p^{\hbox{ignore}}$ and $p^{\hbox{cap}}$ satisfy:
\[p^{\hbox{ignore}} - \frac{1 - F(p^{\hbox{ignore}})}{f(p^{\hbox{ignore}})} = 0 \quad \hbox{ and } \quad p^{\hbox{cap}} - \frac{1/\gamma - F(p^{\hbox{cap}})}{f(p^{\hbox{cap}})} = 0.\]
\end{proposition}


\Cref{cor:optimal_price} shows that, when using hallucination-prone signals, there are three different regimes defining the optimal price. When the signal is low (i.e., lower than $L_\gamma$) the optimal price corresponds to the monopoly price under the prior distribution. In that case the seller bets on the signal being a hallucination and completely disregards it. The intuition is that even if the signal is actually equal to the true value the best achievable revenue would be equal to the signal which is low in that regime. When the signal is in the intermediate region, the seller completely trusts the signal and prices at the value of the signal. Finally, if the signal is too high, pricing at the signal is too risky as the signal may be a hallucination. In that case, the seller posts a capped price. We provide a visual representation of  the virtual values under these three different regimes in \Cref{fig:single_buyer}.


\begin{figure}[h!]
    \centering
\subfigure[Ignore]{
    \begin{tikzpicture}[scale = 0.65]
    \begin{axis}[
        width=10cm,
        height=8cm,
        xmin=0,xmax=1,
        ymin=-1.5,ymax=1.5,
        table/col sep=comma,
        xlabel={$v$},
        ylabel={virtual value},
        grid=both,
        legend pos=north west
    ]
    

    \addplot [black,  thick,unbounded coords=jump] table[x=x,y={s=0.1}] {Data/virtual_value_gamma=075_uniform.csv};
    \addlegendentry{Low signal (s=0.1)}
    \addplot[only marks, black, mark=*,forget plot] coordinates {(0.102, -0.18356179)};
    \draw[black, dashed] (axis cs:0.5, -1.5) -- (axis cs:0.5, 0);
    
    

    \filldraw[black] (axis cs:0.5,-1.5) circle (2pt) node[anchor=south west]{\footnotesize $p^*=0.5$};


    \end{axis}
    \end{tikzpicture}
    }
    \subfigure[Follow]{
    \begin{tikzpicture}[scale = 0.65]
    \begin{axis}[
        width=10cm,
        height=8cm,
        xmin=0,xmax=1,
        ymin=-1.5,ymax=1.5,
        table/col sep=comma,
        xlabel={$v$},
        ylabel={virtual value},
        grid=both,
        legend pos=north west
    ]
    

    \addplot [black,  thick,unbounded coords=jump] table[x=x,y={s=0.4}] {Data/virtual_value_gamma=075_uniform.csv};
    \addlegendentry{Medium signal (s=0.4)}
    \addplot[only marks, black, mark=*,forget plot] coordinates {(0.41, 0.25170764)};
    \draw[black, dashed] (axis cs:0.4, -1.5) -- (axis cs:0.4, 0.25170764);
    

    \filldraw[black] (axis cs:0.4,-1.5) circle (2pt) node[anchor=south east]{\footnotesize $p^*=s$};
    

    \end{axis}
    \end{tikzpicture}
    }
    \subfigure[Cap]{
    \begin{tikzpicture}[scale=0.65]
    \begin{axis}[
        width=10cm,
        height=8cm,
        xmin=0,xmax=1,
        ymin=-1.5,ymax=1.5,
        table/col sep=comma,
        xlabel={$v$},
        ylabel={virtual value},
        grid=both,
        legend pos=north west
    ]
    

    \addplot [black,  thick,unbounded coords=jump] table[x=x,y={s=0.8}] {Data/virtual_value_gamma=075_uniform.csv};
    \addlegendentry{High signal (s=0.8)}
    \addplot[only marks, black, mark=*,forget plot] coordinates {(0.8, 0.76930105)};
    \draw[black, dashed] (axis cs:0.67, -1.5) -- (axis cs:0.67, 0);



    \filldraw[black] (axis cs:0.67,-1.5) circle (2pt) node[anchor=south west]{\footnotesize $p^*=0.66$};
    
    \end{axis}
    \end{tikzpicture}
    }
    \caption{\textbf{Illustration of the three different regimes in the single-buyer case.} The figure represents the correct virtual value functions under the three different regimes described in \Cref{cor:optimal_price}, when $F$ is the uniform distribution and $\gamma = 0.75$. The place where the virtual value crosses zero is the optimal price.}
    \label{fig:single_buyer}
    \end{figure}


\subsection{Comparison to the Value-with-noise Model} \label{sec:noise}
We next contrast the optimal prices under our hallucination model with the ones that emerge from a more classical model where the signal corresponds to the true value plus some Gaussian noise.
 In this alternative model, we assume that the signal $s$ observed by the decision-maker satisfies $s = v + \varepsilon$, where $v$ is the private value of the buyer and $\varepsilon$ is a random variable independently sampled from a zero-mean Gaussian distribution with variance $\sigma^2$.

In some sense, the key difference between the value-with-noise model and the hallucination-prone one is that the error is relatively local in the former, whereas it is more global for the latter. For instance, when the variance $\sigma^2$ is small, the signal obtained will likely be close to the true value, whereas a small hallucination probability $\gamma$ still implies that when the signal is wrong it can be arbitrarily far from the true value and is completely uncorrelated to it. We compare in \Cref{fig:comparison_hall_noise} the optimal price for these two models.
\begin{figure}[h!]
    \centering
    \begin{tikzpicture}[]
    \begin{axis}[
        %width=10cm,
        %height=8cm,
        xmin=0,xmax=1,
        ymin=0.2,ymax=0.8,
        table/col sep=comma,
        xlabel={$s$},
        ylabel={Optimal price},
        grid=both,
        legend pos=north west,
        legend style ={font ={\footnotesize}}
    ]


    \addplot [blue,  thick,unbounded coords=jump] table[x=s,y={best_price_signal_noise}] {Data/comparison_noise_vs_hallucination.csv};
    \addlegendentry{Value-with-noise}

    \addplot [red,  thick,unbounded coords=jump] table[x=s,y={best_price_hallucination}] {Data/comparison_noise_vs_hallucination.csv};
    \addlegendentry{Hallucination}
    \end{axis}
    \end{tikzpicture}
    \caption{\textbf{Optimal price as a function of the signal.} We compare the optimal price for the value-with-noise and the hallucination-prone models, assuming a uniform prior in both cases. The value $\gamma$ is set to $0.75$ and $\sigma^2$ is chosen the match the variance of the hallucination model when $s=0.5$.} 
    \label{fig:comparison_hall_noise}
\end{figure}

We  observe in \Cref{fig:comparison_hall_noise} that the structure of the optimal mechanism starkly differs depending on the underlying model assumed for the signal generation. Under the value-with-noise model, the optimal price inflates the signal when it is too low (when $s \leq 0.4$ in our example) and deflates the signal when it is too high (above $0.4$ in this case), which is very different from the 3-regime optimal approach under hallucinations. This highlights that the optimal mechanism structure heavily depends on the assumption made on the learning algorithm used to generate the signals. 



