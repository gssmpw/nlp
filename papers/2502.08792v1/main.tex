\documentclass[11pt]{article}
\usepackage[utf8]{inputenc}
\usepackage[T1]{fontenc}
\usepackage[english]{babel}
\usepackage[]{algorithm2e}
\usepackage{mwe}
\usepackage{amsmath,mathtools}
\usepackage[dvipsnames]{xcolor}

%\usepackage[inline]{showlabels}

\usepackage{amsthm,bm}
\usepackage{lineno}
%\linenumbers

\usepackage{tikz,pgfplots}
\pgfplotsset{compat=1.5}
\usepackage{subfigure}
\usetikzlibrary{intersections}
\usetikzlibrary{patterns}
\usepackage{multirow}


\usepackage[shortlabels]{enumitem}
\usepackage[colorlinks=true, linkcolor=blue, citecolor=blue, hypertexnames=false]{hyperref}  
\usepackage{nameref}
\usepackage{cleveref}
%\crefname{subsection}{subsection}{subsections}

\usepackage{color}              % Need the color package
% \usepackage[suppress]{color-edits}
\usepackage{color-edits}
\addauthor{Omar}{magenta}
\addauthor{Humberto}{blue}
\addauthor{Ilan}{red}



\usepackage[right=1.0in, top=1in, bottom=1.0in, left=1.0in]{geometry}

\usepackage{lmodern}


\usepackage{graphicx}
\usepackage{lscape}
\usepackage[final]{pdfpages}
\usepackage{amsthm}
\usepackage[authoryear,round]{natbib}
\usepackage{amsfonts}
\usepackage{mathtools}
\usepackage{bbm}
\usepackage{dsfont}

\usepackage{minitoc}
\renewcommand \thepart{}
\renewcommand \partname{}


\newcommand{\revision}[1]{{\color{red}{#1}}}

\usepackage{bbm}
\newtheorem{lemma}{Lemma}
\newtheorem{proposition}{Proposition}
\newtheorem{corollary}{Corollary}
\newtheorem{theorem}{Theorem}
\newtheorem*{theorem*}{Theorem}
\newtheorem{assumption}{Assumption}
\newtheorem{property}{Property}
\newtheorem{definition}{Definition}
\newtheorem{conjecture}{Conjecture}
\newtheorem{remark}{Remark}
\newtheorem{example}{Example}

\usepackage{authblk}

\usepackage{setspace}
\setstretch{1.5}


\newcommand{\aF}[1][x]{\alpha^*_F(#1)}
\newcommand{\bF}[1][x]{\beta^*_F(#1)}


\newcommand{\agF}[1][x]{\alpha^*_{\gamma F}(#1)}
\newcommand{\bgF}[1][x]{\beta^*_{\gamma F}(#1)}

\newcommand{\aFs}[1][x]{\tilde{\alpha}(#1)}
\newcommand{\bFs}[1][x]{\tilde{\beta}(#1)}

\date{February 10, 2025}


\title{Auction Design using Value Prediction with Hallucinations}

\begin{document}

\author[1]{Ilan Lobel}
\author[2]{Humberto Moreira}
\author[3]{Omar Mouchtaki}
\affil[1]{NYU Stern School of Business, \texttt{ilobel@stern.nyu.edu}}
\affil[2]{FGV/EPGE Escola Brasileira de Economia e Finança, \texttt{humberto.moreira@fgv.br}}
\affil[3]{NYU Stern School of Business, \texttt{om2166@stern.nyu.edu}}

\maketitle

\begin{abstract}
We investigate a Bayesian mechanism design problem where a seller seeks to maximize revenue by selling an indivisible good to one of \(n\) buyers, incorporating potentially unreliable predictions (signals) of buyers' private values derived from a machine learning model. We propose a framework where these signals are sometimes reflective of buyers' true valuations but other times are hallucinations, which are  uncorrelated with the buyers' true valuations. 
Our main contribution is a characterization of the optimal auction under this framework. 
Our characterization establishes a near-decomposition of how to treat types above and below the signal. 
For the one buyer case, the seller's optimal strategy is to post one of three fairly intuitive prices depending on the signal,  which we call the ``ignore'', ``follow'' and ``cap'' actions. 
\end{abstract}

\doparttoc % Tell to minitoc to generate a toc for the parts
\faketableofcontents % Run a fake tableofcontents command for the partocs


\section{Introduction}

Despite the remarkable capabilities of large language models (LLMs)~\cite{DBLP:conf/emnlp/QinZ0CYY23,DBLP:journals/corr/abs-2307-09288}, they often inevitably exhibit hallucinations due to incorrect or outdated knowledge embedded in their parameters~\cite{DBLP:journals/corr/abs-2309-01219, DBLP:journals/corr/abs-2302-12813, DBLP:journals/csur/JiLFYSXIBMF23}.
Given the significant time and expense required to retrain LLMs, there has been growing interest in \emph{model editing} (a.k.a., \emph{knowledge editing})~\cite{DBLP:conf/iclr/SinitsinPPPB20, DBLP:journals/corr/abs-2012-00363, DBLP:conf/acl/DaiDHSCW22, DBLP:conf/icml/MitchellLBMF22, DBLP:conf/nips/MengBAB22, DBLP:conf/iclr/MengSABB23, DBLP:conf/emnlp/YaoWT0LDC023, DBLP:conf/emnlp/ZhongWMPC23, DBLP:conf/icml/MaL0G24, DBLP:journals/corr/abs-2401-04700}, 
which aims to update the knowledge of LLMs cost-effectively.
Some existing methods of model editing achieve this by modifying model parameters, which can be generally divided into two categories~\cite{DBLP:journals/corr/abs-2308-07269, DBLP:conf/emnlp/YaoWT0LDC023}.
Specifically, one type is based on \emph{Meta-Learning}~\cite{DBLP:conf/emnlp/CaoAT21, DBLP:conf/acl/DaiDHSCW22}, while the other is based on \emph{Locate-then-Edit}~\cite{DBLP:conf/acl/DaiDHSCW22, DBLP:conf/nips/MengBAB22, DBLP:conf/iclr/MengSABB23}. This paper primarily focuses on the latter.

\begin{figure}[t]
  \centering
  \includegraphics[width=0.48\textwidth]{figures/demonstration.pdf}
  \vspace{-4mm}
  \caption{(a) Comparison of regular model editing and EAC. EAC compresses the editing information into the dimensions where the editing anchors are located. Here, we utilize the gradients generated during training and the magnitude of the updated knowledge vector to identify anchors. (b) Comparison of general downstream task performance before editing, after regular editing, and after constrained editing by EAC.}
  \vspace{-3mm}
  \label{demo}
\end{figure}

\emph{Sequential} model editing~\cite{DBLP:conf/emnlp/YaoWT0LDC023} can expedite the continual learning of LLMs where a series of consecutive edits are conducted.
This is very important in real-world scenarios because new knowledge continually appears, requiring the model to retain previous knowledge while conducting new edits. 
Some studies have experimentally revealed that in sequential editing, existing methods lead to a decrease in the general abilities of the model across downstream tasks~\cite{DBLP:journals/corr/abs-2401-04700, DBLP:conf/acl/GuptaRA24, DBLP:conf/acl/Yang0MLYC24, DBLP:conf/acl/HuC00024}. 
Besides, \citet{ma2024perturbation} have performed a theoretical analysis to elucidate the bottleneck of the general abilities during sequential editing.
However, previous work has not introduced an effective method that maintains editing performance while preserving general abilities in sequential editing.
This impacts model scalability and presents major challenges for continuous learning in LLMs.

In this paper, a statistical analysis is first conducted to help understand how the model is affected during sequential editing using two popular editing methods, including ROME~\cite{DBLP:conf/nips/MengBAB22} and MEMIT~\cite{DBLP:conf/iclr/MengSABB23}.
Matrix norms, particularly the L1 norm, have been shown to be effective indicators of matrix properties such as sparsity, stability, and conditioning, as evidenced by several theoretical works~\cite{kahan2013tutorial}. In our analysis of matrix norms, we observe significant deviations in the parameter matrix after sequential editing.
Besides, the semantic differences between the facts before and after editing are also visualized, and we find that the differences become larger as the deviation of the parameter matrix after editing increases.
Therefore, we assume that each edit during sequential editing not only updates the editing fact as expected but also unintentionally introduces non-trivial noise that can cause the edited model to deviate from its original semantics space.
Furthermore, the accumulation of non-trivial noise can amplify the negative impact on the general abilities of LLMs.

Inspired by these findings, a framework termed \textbf{E}diting \textbf{A}nchor \textbf{C}ompression (EAC) is proposed to constrain the deviation of the parameter matrix during sequential editing by reducing the norm of the update matrix at each step. 
As shown in Figure~\ref{demo}, EAC first selects a subset of dimension with a high product of gradient and magnitude values, namely editing anchors, that are considered crucial for encoding the new relation through a weighted gradient saliency map.
Retraining is then performed on the dimensions where these important editing anchors are located, effectively compressing the editing information.
By compressing information only in certain dimensions and leaving other dimensions unmodified, the deviation of the parameter matrix after editing is constrained. 
To further regulate changes in the L1 norm of the edited matrix to constrain the deviation, we incorporate a scored elastic net ~\cite{zou2005regularization} into the retraining process, optimizing the previously selected editing anchors.

To validate the effectiveness of the proposed EAC, experiments of applying EAC to \textbf{two popular editing methods} including ROME and MEMIT are conducted.
In addition, \textbf{three LLMs of varying sizes} including GPT2-XL~\cite{radford2019language}, LLaMA-3 (8B)~\cite{llama3} and LLaMA-2 (13B)~\cite{DBLP:journals/corr/abs-2307-09288} and \textbf{four representative tasks} including 
natural language inference~\cite{DBLP:conf/mlcw/DaganGM05}, 
summarization~\cite{gliwa-etal-2019-samsum},
open-domain question-answering~\cite{DBLP:journals/tacl/KwiatkowskiPRCP19},  
and sentiment analysis~\cite{DBLP:conf/emnlp/SocherPWCMNP13} are selected to extensively demonstrate the impact of model editing on the general abilities of LLMs. 
Experimental results demonstrate that in sequential editing, EAC can effectively preserve over 70\% of the general abilities of the model across downstream tasks and better retain the edited knowledge.

In summary, our contributions to this paper are three-fold:
(1) This paper statistically elucidates how deviations in the parameter matrix after editing are responsible for the decreased general abilities of the model across downstream tasks after sequential editing.
(2) A framework termed EAC is proposed, which ultimately aims to constrain the deviation of the parameter matrix after editing by compressing the editing information into editing anchors. 
(3) It is discovered that on models like GPT2-XL and LLaMA-3 (8B), EAC significantly preserves over 70\% of the general abilities across downstream tasks and retains the edited knowledge better.
\section{Model, Notation, and Preliminaries}
\label{sec:model}
The goal of constructing an incident database is to determine whether some system that individuals interact with---for example, an (algorithmic) loan decision system, or a medical treatment---results in disproportionate harm to some meaningful subgroups. 
For the incident database associated with a particular system, we will use $\Badevent \in \{0,1\}$ as an indicator variable that denotes the undesirable event corresponding to that system. For example, in loan decisions, this could correspond to the event that a highly-qualified individual was denied a loan; in the medical setting, this may be an adverse physical side effect due to the treatment. 

\paragraph{Subgroup definitions.}
Individuals are characterized with feature vectors $X \in \X$, and we index individuals as 
$X_i$ (``features of individual $i$'') or 
$X_t$ (``features of the individual who reports at time $t$'').
Every individual $X_i$ ``belongs to'' at least one group $G$, and we will denote the event that $X_i$ belongs to $G$ as $\{X_i \in G\}$; we will use $\Groups$ to denote the set of all possible groups. 
This set of possible groups $\Groups$ can be defined arbitrarily as long as all groups can be determined as a function of covariates $\X$. We allow for groups to be overlapping---that is, we allow each individual $X_i$ to be in multiple groups so that $|\{G' \in \Groups: X_i \in G'\}| \geq 1$. 
\iftoggle{icml}{}{For example, it is possible to set $\Groups := 2^{\X}$ as in \citet{hebert2018multicalibration}. }

\paragraph{Reference population.}
The system for which the database is constructed naturally has a corresponding reference population of eligible individuals. For example, this could be everyone who has applied for a loan, or everyone who has been prescribed a certain medication. Thus, given a set of groups $\mathcal G$, we assume that it is possible to compute the composition of the reference population. 
\begin{assumption}[Reference population]
\label{assn:ref}
   For every $G \in \Groups$, the quantity $\Basegroup := \Pr[X \in G]$ is known. Throughout this work, we refer to the set $\{\Basegroup\}_{G \in \Groups}$ as \emph{base preponderances}.
\end{assumption}

\paragraph{Probabilistic model of reporting.}
As the database administrator, the high-level goal is to determine whether there exists some subgroup $G \in \Groups$ where $\Pr[\Badevent\mid X \in G]$ is abnormally high. 
Crucially, the database does not have access to information about every individual who interacts with the system; instead, individuals \textit{may} report to the database if they believe that they experienced bad event $\Badevent$. 
We thus let $R_i$ be a random variable representing whether individual $i$ decides to report (with $R_i = 0$ indicating no report). 

Each report $X_t$ is received sequentially, and assumed to be sampled i.i.d. from some underlying reporting distribution.\iftoggle{icml}{}{\footnote{Note that the events $\{X_t \in G\}$ and $\{X_t \in G'\}$ are correlated for any $G, G' \in \Groups$, i.e. the independence does not hold across groups. The key point in our case is independence across time.}}
Given a group $G$, we denote its corresponding mean among reports $\Pr[X_t \in G \mid R_t = 1]$ as  $\mu_G$.
We will sometimes refer to $\{\mu_G\}_{G \in \Groups}$ as (reporting) preponderances, as they represent the proportion of \textit{reports} that each $G$ comprises. A central claim of this paper is that comparing $\mu_G$ to $\Basegroup$---i.e., the extent to which group $G$ is (over)represented within the reporting database---can be a useful signal for $\Pr[Y \mid G]$ in a wide class of applications.\footnote
{Because we allow groups to overlap, we cannot enforce $\sum_G \Basegroup = 1$ or $\sum_G\mu_G = 1$.} \iftoggle{icml}{}{The i.i.d. model of course simplifies the analysis and exposition, but itself is not intrinsic to modeling the incident reporting problem as a sequential hypothesis test. As we discuss in Appendix \ref{subsec:practical}, the explicit i.i.d. assumption can be relaxed; more generally, any probabilistic model for sequential testing can be adapted to incident reporting. 
}
\section{Bayesian Update and Applying Myerson}\label{sec:failure_myerson}


In our setting, the seller obtains the signals $s_i$ prior to selecting the mechanism. After obtaining $s_i$, the seller's posterior belief about $v_i$ is given by:
\begin{equation}\label{eq:density-f} f_{\gamma_i,s_i}^i(v) = \gamma_i \cdot f_i(v) + (1-\gamma_i)\cdot \delta_{s_i}(v),\end{equation}
where $\delta_{s_i}(\cdot)$ is the Dirac delta function that places a unit of mass at $s_i$ and zero mass everywhere else. Equivalently, 
\begin{equation}\label{eq:cumulative-F}
F_{\gamma_i,s_i}^i(v) = \begin{cases}
\gamma_i \cdot f_i(v) \quad \text{for $v<s_i$},\\
\gamma_i \cdot f_i(v) + (1-\gamma_i) \quad \text{for $v \geq s_i$}.
\end{cases}
\end{equation}

The question we aim to address can thus be rephrased as what is the revenue-maximizing auction when the valuation of buyer $i$ is drawn according to $F_{\gamma_i,s_i}^i$. 

\vspace{.1in}
\noindent \textbf{On the distribution of hallucinations.} We will assume throughout the paper that the value $v_i$ and any potential hallucination $w_i$ are drawn from the same distribution. However, if we were to assume that the value were drawn from density $g_i$ and the hallucination from density $f_i$, where these distributions are absolutely continuous with respect to each other, we could obtain a similar formula via Bayesian updating. Let $Z_i$ represent whether a hallucination occurred. The posterior density would then be given by:
\begin{eqnarray*}
    f_{\gamma_i,s_i}^i(v) &=&  P(Z_i \mid s_i) \cdot f_{\gamma_i,s_i}^i(v \mid Z_i) +  P(\hbox{not } Z_i \mid s_i) \cdot f_{\gamma_i,s_i}^i(v \mid \hbox{not }Z_i)\\ 
    &=& P(Z_i \mid s_i)\cdot f_i(v) + P(\hbox{not }Z_i \mid s_i)\cdot \delta_{s_i}(v)\\ 
    &=& \frac{\gamma_i \cdot f_i(s_i)}{\gamma_i \cdot f_i(s_i) + (1-\gamma_i) \cdot g_i(s_i)}f_i(v) + \frac{(1-\gamma_i) \cdot g(s_i)}{\gamma_i \cdot f_i(s_i) + (1-\gamma_i) \cdot g_i(s_i)}\delta_{s_i}(v)\\
    &=& \widetilde {\gamma}^i_{s_i} \cdot f(v) + (1-\widetilde {\gamma}^i_{s_i}) \cdot \delta_{s_i}(v), 
\end{eqnarray*}
where
$\widetilde \gamma_{s_i} = \left(1+\frac{1-\gamma_i}{\gamma_i}\frac{g_i(s_i)}{f_i(s_i)}\right)^{-1}$. That is, our results from the rest of the paper would apply if we replace $\gamma_i$ with $\widetilde \gamma_{s_i}$.

\subsection{Applying Myerson}

\citet{myerson1981optimal} tells us that in a private values setting, the revenue-maximizing auction is given by calculating the virtual value of each agent (which might require ironing) and then allocating the item to the agent with the highest non-negative virtual value, or discarding the item if all of the virtual values are negative. Since virtual values are computed separately for each buyer, we will suppress the buyer index $i$ from the notation whenever possible to lighten the notational burden. 

For a given density $f$ and cumulative distribution $F$, the pre-ironing virtual value function is $\varphi_F(v) = v - (1-F(v))/f(v)$. For the density and cumulative distributions given by Eqs. \eqref{eq:density-f}
 and \eqref{eq:cumulative-F}, we have:
 \[ \varphi_{F_{\gamma,s}}(v) =  
 \begin{cases}
     v - \frac{1/\gamma - F(v)}{f(v)}, &\hbox{ for } v < s,\\
     v - \frac{1 - F(v)}{f(v)}, &\hbox{ for } v > s.\\
 \end{cases}\]
 We note that the virtual value function is not well-defined at $s$, but we will ignore this issue for now since that is a single point. The function $\varphi_{F_{\gamma,s}}$ does not need to be ironed after $s$ since $\varphi_{F_{\gamma,s}}(v) = \varphi_{F} (v)$ for $v > s$ and we have assumed $F$ is regular. Ironing could be necessary before $s$ depending on the choice of $F$. 

 Let's apply this to single-buyer, uniform over $[0,1]$ case. For this particular $F$, we obtain:
  \begin{equation}\label{eq:misleading-F} \varphi_{F_{\gamma,s}}(v) =  
 \begin{cases}
     2v - 1/\gamma, &\hbox{ for } v < s,\\
     2v - 1, &\hbox{ for } v > s.\\
 \end{cases}\end{equation}
 For this particular distribution, ironing is not necessary before $s$ since $2v - 1/\gamma$ is an increasing function of $v$. 
 Consider the special case $s=1/2-\epsilon$ and $\gamma = \epsilon$, for a small $\epsilon$. Eq. \eqref{eq:misleading-F} crosses zero at $v=1/2$, implying that the optimal price is $1/2$. However, this cannot be the correct optimal price. The revenue generated by this price is bounded above by $\epsilon$ since it requires $s$ to be a hallucination as a necessary condition for a sale to occur. Meanwhile, using the signal $1/2-\epsilon$ as the price would generate at least $(1-\epsilon)\cdot(1/2-\epsilon)$ in revenue. 

 It turns out that ignoring what occurred at $s$, where the density $f_{\gamma,s}$ is not well-defined, and applying Myerson's technique naively was a mistake. To obtain a correct optimal auction, we will need to use a more sophisticated characterization of optimal auctions that applies for distributions that do not admit densities. 
\section{Characterization of the Optimal Auction}\label{sec:optimal_auction}

In this section, we first introduce a slight generalization of Myerson's ironing operation, which we will need to state our results. We then present our main theorem, and demonstrate what it implies for some simple distributions. We also show that our main theorem fails if we remove the regularity assumption. 

\subsection{Truncated Myerson Ironing}\label{sec:ironing}

Consider a distribution $F$ supported on $[a,b]$ and which admits a positive density on its support.  In that case $F$ is strictly increasing on $[a,b]$ and therefore it admits an inverse function $F^{-1}$ strictly increasing on $[0,1]$. When the virtual value function of $F$ defined for every $x \in [a,b]$ as $\varphi_{F}(x)$  
is not monotonic non-decreasing, \citet{myerson1981optimal} proposes a general procedure called ironing to characterize the optimal auction. In what follows we introduce our slight generalization of Myerson's ironing operator. The only difference between the operator we introduce below and the one presented in \citet{myerson1981optimal} is that we also allow for the operation to be performed only in an interval of the quantile space rather than over the entire quantile space. Hence, we call this operation the truncated Myerson ironing. If we restrict $x$ to be equal to 1 in what follows, we would be mimic the definition of the original Myerson ironing operator.



For every quantile $q \in [0,1]$, let
\begin{equation}
\label{eq:J}
    J(q) = \int_0^q \varphi_{F}(F^{-1}(r)) dr.
\end{equation}
Furthermore, for every $x \in [0,1]$, let $G_x:[0,x] \to \mathbb{R}$ be the convex hull of the restriction of the function $J$ on $[0,x]$, formally defined for every $q \in [0,x]$ as,
\begin{equation*}
    G_x(q) = \min_{ \substack{(\lambda,r_1,r_2) \in [0,1]\times[0,x]^2\\ \text{s.t. } \lambda \cdot r_1 + (1-\lambda) \cdot r_2 = q} } \lambda \cdot J(r_1) + (1-\lambda) \cdot J(r_2) 
\end{equation*}
By definition, $G_x$ is convex on $[0,x]$. Therefore, it is continuously differentiable on $[0,x]$ except at countably many points. For every $q \in [0,x]$, we define the function $g$ as,
\begin{equation*}
    g_x(q) = \begin{cases}
        G'_x(q) \quad \text{if $G$ is differentiable at $q$}\\
        \lim_{\tilde{q} \downarrow q} G'_x(\tilde{q}) \quad \text{otherwise.}
    \end{cases}
\end{equation*}
The convexity of $G_x$ implies that $g_x$ is monotone non-decreasing. For any $t \in [a,b]$ we define the truncated ironed virtual of $F$ on $[a,t]$ as the mapping,
\begin{equation*}
    \mathrm{IRON}_{[a,t]}[F] : \begin{cases}
        [a,t] \to \mathbb{R}\\
        v \mapsto g_{F^{-1}(t)}(F(v)).
    \end{cases}
\end{equation*}


We note that $\mathrm{IRON}_{[a,b]}[F]$ corresponds to the classical notion of ironing introduced in \citet{myerson1981optimal}. We emphasize that when $t < b$, the mapping $\mathrm{IRON}_{[a,t]}[F]$ is in general different from the restriction of $\mathrm{IRON}_{[a,b]}[F]$ on $[a,t]$ (see \Cref{fig:ironing_operator}). 

 \begin{figure}[h!]
    \centering
    \subfigure[Convexification in quantile space]{
    \begin{tikzpicture}[scale=.65]
    \begin{axis}[
        width=10cm,
        height=10cm,
        xmin=-0.,xmax=1.0,
        ymin=-0.12,ymax=0.01,
        scaled y ticks={base 10:2},
        table/col sep=comma,
        xlabel={$q$},
        ylabel={$H(q)$},
        grid=both,
        legend pos=south west
    ]


    \addplot [blue, dashed, line width=.7mm] table[x=F,y=H] {Data/ironing_example_mix_truncated_normals.csv};
    \addlegendentry{Before ironing}
    


    \addplot [red, very thick] table[x=F,y=psi] {Data/ironing_example_mix_truncated_normals.csv};
    \addlegendentry{$\mathrm{IRON}_{[0,2]}$}
    

    \addplot [teal, very thick] table[x=F,y=psi_cut05] {Data/ironing_example_mix_truncated_normals.csv};
    \addlegendentry{$\mathrm{IRON}_{[0,0.5]}$}
    


    \addplot [black, very thick] table[x=F,y=psi_cut02] {Data/ironing_example_mix_truncated_normals.csv};
    \addlegendentry{$\mathrm{IRON}_{[0,0.2]}$}
    


    \end{axis}
    \end{tikzpicture}
    }
    \subfigure[Virtual value]{
    \begin{tikzpicture}[scale=.65]
    \begin{axis}[
        width=10cm,
        height=10cm,
        xmin=0,xmax=2.0,
        ymin=-2.5,ymax=2,
        table/col sep=comma,
        xlabel={$v$},
        ylabel={virtual value},
        grid=both,
        legend pos=south east
    ]
    
    \addplot [blue,  dashed, line width=.7mm] table[x=x,y={virtual_value_preiron}] {Data/ironing_example_mix_truncated_normals.csv};
    \addlegendentry{Before ironing}

    \addplot [red, very thick] table[x=x,y=virtual_value] {Data/ironing_example_mix_truncated_normals.csv};
    \addlegendentry{$\mathrm{IRON}_{[0,2]}$}

    \addplot [teal, very thick] table[x=x,y=virtual_value_cut05] {Data/ironing_example_mix_truncated_normals.csv};
    \addlegendentry{$\mathrm{IRON}_{[0,0.5]}$}

    \addplot [black, very thick] table[x=x,y=virtual_value_cut02] {Data/ironing_example_mix_truncated_normals.csv};
    \addlegendentry{$\mathrm{IRON}_{[0,0.2]}$}

    \end{axis}
    \end{tikzpicture}
    }
    \caption{ 
    The figure illustrates the truncated ironing procedure. The distribution $F$ used is a mixture of two truncated normals on $[0,2]$ with parameters $(0.1,0.04)$ and $(1.9,1.8)$ and respective weights $0.8$ and $0.2$.  (a) The figure shows the initial $J$ function (in blue) and the convex envelopes of this function on different intervals: $F^{-1}(0.2)$, $F^{-1}(0.5)$ and $F^{-1}(2)$.  (b) The figure shows the induced virtual value function before ironing and by ironing on three subintervals: $0.2$, $0.5$ and $2$.} 
    \label{fig:ironing_operator}
    \end{figure}


\subsection{Main Result}\label{sec:main}

If the distribution $F$ does not admit a density that is positive everywhere in the support, the classical Myerson ironing procedure is not applicable since it relies on the existence of the inverse $F^{-1}$. In this case, there exists a more general virtual value characterization developed by \citet{monteiro2010optimal} that is still applicable. That characterization is difficult to work with because it involves generalized convex hulls, rather than the standard convexification used by Myerson. We defer the presentation and discussion of how to use this complex machinery until Section \ref{sec:technical_work}. We are now ready to state the main result of the paper, which states that if the value distributions are regular, then an ironing procedure that has the same complexity as Myerson does apply. 


\begin{theorem}\label{thm:main}
Let $F_i$ be distributions satisfying Assumption \ref{ass:regular}. Then, there exists a direct mechanism that is revenue-maximizing. In this mechanism, given reported values $\hat{v}_i$, the seller allocates the good to the buyer with the highest non-negative value of $\bar{\varphi}^i_{\gamma_i, s_i}(\hat{v}_i)$, where the function $\bar{\varphi}^i_{\gamma_i, s_i}(\hat{v}_i)$ is defined as:
\begin{equation}
\label{eq:ironed-vv} 
\bar{\varphi}^i_{\gamma_i, s_i}(v) = 
\begin{cases}
    \mathrm{IRON}_{[0, s_i]}[\gamma_i F_i](v), & \text{if } a \leq v < s_i, \\
    \varphi_{F_i}(T_i), & \text{if } s_i \leq v < T_i, \\
    \varphi_{F_i}(v), & \text{if } T_i \leq v \leq b.
\end{cases}
\end{equation}
for every $v \in [a_i, b_i]$. Furthermore, the winning bidder pays  the minimum amount they would need to bid to still win.
The constants $(T_i)_{i \in \{1, \ldots, n\}}$ are defined in \Cref{prop:from_F_to_feasible_Fs}, and the operator $\mathrm{IRON}$ is as specified in Section \ref{sec:ironing}.
\end{theorem}
We present the key technical arguments required to proof \Cref{thm:main} in \Cref{sec:technical_work}.


\Cref{thm:main} above states that $\bar\varphi^i_{\gamma_i,s_i}$ is the correct notion of ironed virtual value function given posterior beliefs $F_{\gamma_i,s_i}^i$. Before the signal $s_i$, the correct pre-ironing virtual value is given by $\mathrm{IRON}_{[0, s_i]}[\gamma_i F_i]$, which might require ironing, but where ironing can be done using Myerson's classical approach but with the domain truncated to $[0,s_i]$. Immediately after the signal, we need to iron out a segment $[s_i,T_i]$ of the virtual value to account for the mass at $s_i$. After $T_i$, the original virtual value function $\varphi_{F_i}$ applies. 

The theorem can be interpreted as a near-decomposition result. Ironing the section strictly before the signal yields $\mathrm{IRON}_{[0, s_i]}[\gamma_i F_i]$ while ironing the virtual value from $s_i$ (inclusive) onward yields the second and third clauses of Eq. \eqref{eq:ironed-vv}.
We call this a near-decomposition, not a full decomposition, because $T_i$ creates a link between the two sides, as the value of $T_i$ depends on the distribution before the signal.

The key assumption that enables this near-decomposition is the regularity of $F_i$. The next example shows that if $F_i$ is irregular, then Theorem \ref{thm:main} may fail.

\begin{example}
    Consider the distribution $F$ putting a $0.8$ weight on a truncated normal on $[0.5,0.52]$ with mean $0.51$ and std $0.05$, and a $0.2$ weight on the Uniform over $[0,1]$. We note that this distribution is not regular. In \Cref{fig:counter_example}, we compare the value of $\mathrm{IRON}_{[0,s]}[\gamma F]$ and the actual generalized ironed virtual value of  $F_{\gamma,s}$  computed using the method described in \Cref{sec:technical_work}, for $s = 0.53$ and $\gamma = 0.9$.
\begin{figure}[h!]
    \centering
    \begin{tikzpicture}[scale = 0.65]
    \begin{axis}[
        width=10cm,
        height=10cm,
        xmin=0.5,xmax=0.55,
        ymin=0.4,ymax=0.55,
        table/col sep=comma,
        xlabel={$v$},
        ylabel={virtual value},
        grid=both,
        legend pos=north west
    ]
    
    \addplot [black,  line width = 0.7 mm,unbounded coords=jump] table[x=x,y={virtual_value}] {Data/counter_example.csv};
    \addlegendentry{Ironed virtual value}

    \addplot [red,  line width = 0.7 mm,unbounded coords=jump] table[x=x,y={virtual_value_pre_s},restrict expr to domain={\thisrow{x}}{0:0.531}] {Data/counter_example.csv};
    \addlegendentry{$\mathrm{IRON}_{[0,s]}(\gamma F)$}
    
    \draw[blue, dashed, thick] (axis cs:0.53, 0.4) -- (axis cs:0.53, 0.5);
    \filldraw[blue] (axis cs:0.53,0.4) circle (2pt) node[anchor=south west]{\footnotesize $s=0.53$};

    \end{axis}
    \end{tikzpicture}
    \caption{\textbf{Numerical counter-example to the near-decomposition property without regularity.}}
    \label{fig:counter_example}
\end{figure}

\Cref{thm:main} claims that the generalized ironed virtual value of  $F_{\gamma,s}$ should be equal to $\mathrm{IRON}_{[0,s]}[\gamma F]$ for every $v < s$. However, \Cref{fig:counter_example} demonstrates that this statement does not hold in our example. This figure shows that when $F$ is not regular, the ironing procedure cannot independently be executed on the intervals $[0,s]$ and $[s,1]$ as described in \Cref{thm:main}. Intuitively, when $F$ is not regular, $s$ may lie in a region that already required ironing under the prior distribution $F$. Consequently, when considering the posterior distribution $F_{\gamma,s}$ the values before and after $s$  be taken into account to properly compute the ironed virtual value around $s$. 
\end{example}


It is useful to see what Theorem \ref{thm:main} implies for some simple distributions. If $F$ is a uniform [0,1] distribution, then the virtual value is given by:
  \begin{equation*} \bar \varphi_{F_{\gamma,s}}(v) =   \begin{cases}
     2v - 1/\gamma, &\hbox{ for } v < s,\\
     2T - 1, &\hbox{ for } s \leq v < T,\\
     2v - 1, &\hbox{ for } v \leq  T.\\
 \end{cases}\end{equation*}
 If $F$ is an exponential distribution, then ironing might be required to the left of the signal. Note that the exponential distribution is not only a regular distribution, but satisfies the even stronger condition of monotone hazard rate. Despite this, the pre-signal distribution still sometimes requires ironing (see \Cref{fig:illustration_theorem1}).

    \begin{figure}[h]
    \centering
    \subfigure[Exponential prior $(\lambda = 1), \gamma = 0.95$]{
    \begin{tikzpicture}[scale = 0.65]
    \begin{axis}[
        width=10cm,
        height=10cm,
        xmin=0,xmax=6.5,
        ymin=-4,ymax=6,
        table/col sep=comma,
        xlabel={$v$},
        ylabel={virtual value},
        grid=both,
        legend pos=north west
    ]

    \addplot [blue, dashed,  thick,unbounded coords=jump] table[x=x,y={preiron_s=5}] {Data/virtual_value_gamma=095_exponential.csv};
    \addlegendentry{Unironed (s=5)}
    
    \addplot [red,  thick,unbounded coords=jump] table[x=x,y={s=5}] {Data/virtual_value_gamma=095_exponential.csv};
    \addlegendentry{Ironed (s=5)}
    \addplot[only marks, red, mark=*,forget plot] coordinates {(5, 4.952221754226520)};
    \end{axis}
    \end{tikzpicture}
    }
    \subfigure[Uniform prior, $\gamma = 0.75$]{
    \begin{tikzpicture}[scale = 0.65]
    \begin{axis}[
        width=10cm,
        height=10cm,
        xmin=0,xmax=1,
        ymin=-1.5,ymax=1.5,
        table/col sep=comma,
        xlabel={$v$},
        ylabel={virtual value},
        grid=both,
        legend pos=north west
    ]
    


    \addplot[domain=0:0.4,samples=50,thick,dashed,blue] {2*x - 1/0.75};  % For x < s
    \addplot[domain=0.4:1,samples=50,thick,dashed,blue,forget plot] {2*x - 1};        % For x > s
    \addlegendentry{Unironed (s=0.4)}

    \addplot [red,  thick,unbounded coords=jump] table[x=x,y={s=0.4}] {Data/virtual_value_gamma=075_uniform.csv};
    \addlegendentry{Ironed (s=0.4)}
    \addplot[only marks, red, mark=*,forget plot] coordinates {(0.41, 0.25170764)};
   

    \end{axis}
    \end{tikzpicture}
    }
    \caption{\textbf{Ironed virtual value for different priors.} In each plot the unironed virtual value corresponds to the naive evaluation $\varphi_{F_{\gamma,s}}$, wherever it is well defined (i.e., everywhere but at $s$). The ironed virtual value corresponds to the virtual value characterized in \Cref{thm:main}.}
     \label{fig:illustration_theorem1}
    \end{figure}
\section{The Single Buyer Case}\label{sec:single-buyer}



In this section, we first leverage \Cref{sec:main} to study the structure of the optimal mechanism for a single buyer. We then, compare the mechanism obtained in our model of hallucination-prone signals with another model which corresponds to the classical model of Gaussian noise.

\subsection{Optimal Mechanism for One Buyer}

An important implication of \Cref{thm:main} is the following characterization of the optimal mechanism for a single buyer. In this setting, the optimal mechanism is a posted price. 

\begin{proposition}
\label{cor:optimal_price}
Assume $n = 1$ and $F$ is regular on $[a,b]$ with continuous density. Then, for any $s \in [a,b]$ and any $\gamma \in [0,1]$, there exist two thresholds $L_{\gamma}$ and $U_{\gamma}$ such that the optimal price satisfies:
\[ p^* = 
\begin{cases}
    p^{\hbox{ignore}} &\hbox{ if } s < L_{\gamma},\\
    s &\hbox{ if } L_{\gamma} \leq s < U_{\gamma},\\
    p^{\hbox{cap}}&\hbox{ if } s \geq U_{\gamma},
\end{cases}\]
where $p^{\hbox{ignore}}$ and $p^{\hbox{cap}}$ satisfy:
\[p^{\hbox{ignore}} - \frac{1 - F(p^{\hbox{ignore}})}{f(p^{\hbox{ignore}})} = 0 \quad \hbox{ and } \quad p^{\hbox{cap}} - \frac{1/\gamma - F(p^{\hbox{cap}})}{f(p^{\hbox{cap}})} = 0.\]
\end{proposition}


\Cref{cor:optimal_price} shows that, when using hallucination-prone signals, there are three different regimes defining the optimal price. When the signal is low (i.e., lower than $L_\gamma$) the optimal price corresponds to the monopoly price under the prior distribution. In that case the seller bets on the signal being a hallucination and completely disregards it. The intuition is that even if the signal is actually equal to the true value the best achievable revenue would be equal to the signal which is low in that regime. When the signal is in the intermediate region, the seller completely trusts the signal and prices at the value of the signal. Finally, if the signal is too high, pricing at the signal is too risky as the signal may be a hallucination. In that case, the seller posts a capped price. We provide a visual representation of  the virtual values under these three different regimes in \Cref{fig:single_buyer}.


\begin{figure}[h!]
    \centering
\subfigure[Ignore]{
    \begin{tikzpicture}[scale = 0.65]
    \begin{axis}[
        width=10cm,
        height=8cm,
        xmin=0,xmax=1,
        ymin=-1.5,ymax=1.5,
        table/col sep=comma,
        xlabel={$v$},
        ylabel={virtual value},
        grid=both,
        legend pos=north west
    ]
    

    \addplot [black,  thick,unbounded coords=jump] table[x=x,y={s=0.1}] {Data/virtual_value_gamma=075_uniform.csv};
    \addlegendentry{Low signal (s=0.1)}
    \addplot[only marks, black, mark=*,forget plot] coordinates {(0.102, -0.18356179)};
    \draw[black, dashed] (axis cs:0.5, -1.5) -- (axis cs:0.5, 0);
    
    

    \filldraw[black] (axis cs:0.5,-1.5) circle (2pt) node[anchor=south west]{\footnotesize $p^*=0.5$};


    \end{axis}
    \end{tikzpicture}
    }
    \subfigure[Follow]{
    \begin{tikzpicture}[scale = 0.65]
    \begin{axis}[
        width=10cm,
        height=8cm,
        xmin=0,xmax=1,
        ymin=-1.5,ymax=1.5,
        table/col sep=comma,
        xlabel={$v$},
        ylabel={virtual value},
        grid=both,
        legend pos=north west
    ]
    

    \addplot [black,  thick,unbounded coords=jump] table[x=x,y={s=0.4}] {Data/virtual_value_gamma=075_uniform.csv};
    \addlegendentry{Medium signal (s=0.4)}
    \addplot[only marks, black, mark=*,forget plot] coordinates {(0.41, 0.25170764)};
    \draw[black, dashed] (axis cs:0.4, -1.5) -- (axis cs:0.4, 0.25170764);
    

    \filldraw[black] (axis cs:0.4,-1.5) circle (2pt) node[anchor=south east]{\footnotesize $p^*=s$};
    

    \end{axis}
    \end{tikzpicture}
    }
    \subfigure[Cap]{
    \begin{tikzpicture}[scale=0.65]
    \begin{axis}[
        width=10cm,
        height=8cm,
        xmin=0,xmax=1,
        ymin=-1.5,ymax=1.5,
        table/col sep=comma,
        xlabel={$v$},
        ylabel={virtual value},
        grid=both,
        legend pos=north west
    ]
    

    \addplot [black,  thick,unbounded coords=jump] table[x=x,y={s=0.8}] {Data/virtual_value_gamma=075_uniform.csv};
    \addlegendentry{High signal (s=0.8)}
    \addplot[only marks, black, mark=*,forget plot] coordinates {(0.8, 0.76930105)};
    \draw[black, dashed] (axis cs:0.67, -1.5) -- (axis cs:0.67, 0);



    \filldraw[black] (axis cs:0.67,-1.5) circle (2pt) node[anchor=south west]{\footnotesize $p^*=0.66$};
    
    \end{axis}
    \end{tikzpicture}
    }
    \caption{\textbf{Illustration of the three different regimes in the single-buyer case.} The figure represents the correct virtual value functions under the three different regimes described in \Cref{cor:optimal_price}, when $F$ is the uniform distribution and $\gamma = 0.75$. The place where the virtual value crosses zero is the optimal price.}
    \label{fig:single_buyer}
    \end{figure}


\subsection{Comparison to the Value-with-noise Model} \label{sec:noise}
We next contrast the optimal prices under our hallucination model with the ones that emerge from a more classical model where the signal corresponds to the true value plus some Gaussian noise.
 In this alternative model, we assume that the signal $s$ observed by the decision-maker satisfies $s = v + \varepsilon$, where $v$ is the private value of the buyer and $\varepsilon$ is a random variable independently sampled from a zero-mean Gaussian distribution with variance $\sigma^2$.

In some sense, the key difference between the value-with-noise model and the hallucination-prone one is that the error is relatively local in the former, whereas it is more global for the latter. For instance, when the variance $\sigma^2$ is small, the signal obtained will likely be close to the true value, whereas a small hallucination probability $\gamma$ still implies that when the signal is wrong it can be arbitrarily far from the true value and is completely uncorrelated to it. We compare in \Cref{fig:comparison_hall_noise} the optimal price for these two models.
\begin{figure}[h!]
    \centering
    \begin{tikzpicture}[]
    \begin{axis}[
        %width=10cm,
        %height=8cm,
        xmin=0,xmax=1,
        ymin=0.2,ymax=0.8,
        table/col sep=comma,
        xlabel={$s$},
        ylabel={Optimal price},
        grid=both,
        legend pos=north west,
        legend style ={font ={\footnotesize}}
    ]


    \addplot [blue,  thick,unbounded coords=jump] table[x=s,y={best_price_signal_noise}] {Data/comparison_noise_vs_hallucination.csv};
    \addlegendentry{Value-with-noise}

    \addplot [red,  thick,unbounded coords=jump] table[x=s,y={best_price_hallucination}] {Data/comparison_noise_vs_hallucination.csv};
    \addlegendentry{Hallucination}
    \end{axis}
    \end{tikzpicture}
    \caption{\textbf{Optimal price as a function of the signal.} We compare the optimal price for the value-with-noise and the hallucination-prone models, assuming a uniform prior in both cases. The value $\gamma$ is set to $0.75$ and $\sigma^2$ is chosen the match the variance of the hallucination model when $s=0.5$.} 
    \label{fig:comparison_hall_noise}
\end{figure}

We  observe in \Cref{fig:comparison_hall_noise} that the structure of the optimal mechanism starkly differs depending on the underlying model assumed for the signal generation. Under the value-with-noise model, the optimal price inflates the signal when it is too low (when $s \leq 0.4$ in our example) and deflates the signal when it is too high (above $0.4$ in this case), which is very different from the 3-regime optimal approach under hallucinations. This highlights that the optimal mechanism structure heavily depends on the assumption made on the learning algorithm used to generate the signals. 




%\section{The Hallucination Probability}\label{sec:gamma}


%\section{A Heuristic Solution}\label{sec:heuristic}






\section{Key Technical Arguments}
\label{sec:technical_work}
In this section we present the key technical arguments needed to prove \Cref{thm:main}. We first describe the family of semi-infinite dimensional problems developed in \citet{monteiro2010optimal} to characterize the ironed virtual value for arbitrary distributions. We then solve this family of problems to obtain our closed-form solution.

\subsection{Ironing for Arbitrary Distributions}
\label{sec:gen_ironing}
Let $F$ be a regular distribution which admits a positive density $f$ on its support. 
For any $\gamma \in (0,1)$ and any $s$ in the support of $F$, recall the definition of the post-signal distribution $F_{\gamma,s}$ defined in Eq.~\eqref{eq:cumulative-F}.
We note that the post-signal distribution does not admit a density at $v = s$. In this setting, the virtual value function used to iron in the Myerson sense (see Section \ref{sec:ironing}), and which is defined for every distribution $F$ with positive density on its support
is not well-defined. In what follows, we present the formalism developed in \citet{monteiro2010optimal} to characterize the optimal auction for general distributions. This formalism generalizes Myerson's characterization.

For every distribution $F$ (which does not need to have a density), we define for every $x \in [a,b]$ the function
\begin{equation*}
H_{F}(x) = \int_{a}^x t  dF(t) - \int_a^x (1-F(t))dt.
\end{equation*}
Fix $t \in [a,b]$. For every $x \in [a,t]$, we define the generalized convex hull of $H_F$ as,
\begin{subequations}
\label{eq:gen_virtual_value}
\begin{alignat}{2}
\Psi_{F}^t(x) = \; &\!\sup_{\alpha,\beta \in \mathbb{R}} &\;& \alpha + \beta \cdot F(x) \\
&\text{s.t.} &      &  \alpha + \beta \cdot F(y) \leq H_{F}(y) \quad \forall y \in [a,t]. 
\end{alignat}
\end{subequations}
Let $\partial \Psi_{F}^t(x)$ be the generalized sub-differential of $\Psi_{F}^t$ at $x$ defined as the set of $\beta \in \mathbb{R}$ such that
\begin{equation}
\label{eq:subgradient}
\Psi_{F}^t(z) \geq \Psi^t_{F}(x) + \beta \cdot (F(z) - F(x)) \quad \text{for every $z \in [a,t]$}.  
\end{equation}
Equivalently (see Section 2 of \citet{monteiro2010optimal}), one has that
\begin{equation}
    \label{eq:subgrad_are_solutions}
    \partial \Psi_{F}^t(z) = \{ \beta \in \mathbb{R} \text{ s.t. there exists $\alpha \in \mathbb{R}$ such that $(\alpha,\beta)$ is optimal for \eqref{eq:gen_virtual_value}} \}.
\end{equation}
Furthermore, let $\ell^t_{F}(x) = \inf \partial \Psi^t_{F}(x)$ and $s^t_{F}(x) = \sup \partial \Psi^t_{F}(x)$\footnote{Note that we will drop dependencies in $t$ when $t = b$, as $\Psi_F^b$ corresponds to the generalized convex hull of $H_F$ on the whole domain $[a,b]$.
}.

\begin{figure}[h]
    \centering
    \begin{tikzpicture}[every text node part/.style={align=center},transform shape,]
    \begin{axis}[
        width=10cm,
        height=8cm,
        xmin=-0.,xmax=2.0,
        ymin=-0.3,ymax=0.01,
        scaled y ticks={base 10:2},
        table/col sep=comma,
        xlabel={$v$},
        ylabel={Negative Revenue},
        grid=both,
        legend pos=south east
    ]

    \addplot [blue, dashed, line width=.7mm] table[x=x,y=H] {Data/ironing_example_mix_truncated_normals.csv};
    \addlegendentry{Before ironing}


    \addplot [black, line width=.5mm] table[x=x,y=psi] {Data/ironing_example_mix_truncated_normals.csv} ;
    \addlegendentry{Ironed curve}


 \addplot [violet, line width=.3mm] table[x=x,y expr={-1.22+1.2*\thisrow{F}}] {Data/ironing_example_mix_truncated_normals.csv} node[pos = 0.7,below right] {\footnotesize $y = -1+1.2  F(v)$};



    \addplot [red, line width=.3mm] table[x=x,y expr={-0.03-0.2*\thisrow{F}}] {Data/ironing_example_mix_truncated_normals.csv} node[pos = 0.7,below left] {\footnotesize $y = -0.03-0.2  F(v)$};


   

    \addplot [teal, line width=.3mm] table[x=x,y expr={-0.044-0.08*\thisrow{F}}] {Data/ironing_example_mix_truncated_normals.csv} node[pos = 0.3,below] { \footnotesize $y = -0.044-0.08  F(v)$};

    \end{axis}
    \end{tikzpicture}
    \caption{The figure illustrates the ironing procedure defined by \citet{monteiro2010optimal}. The distribution $F$ used is mixture of two truncated normals on $[0,2]$ with parameters $(0.1,0.04)$ and $(1.9,1.8)$ and respective weights $0.8$ and $0.2$. Instead of the standard convexification in quantile space, \cite{monteiro2010optimal} perform a generalized convexification in the value space where affine functions of $F$ are used to iron the revenue curve.}
    \label{fig:monteiro}
    \end{figure}


\citet{monteiro2010optimal} show that the mapping $\ell_F$ generalizes the notion of ironed virtual value functions for distributions which do not necessarily have a positive density. Figure \ref{fig:monteiro} shows an example of this kind of ironing works via generalized convexification in value space. In particular, they show that when $F$ admits a positive density on its support, $\ell_F$ is equal to the usual Myerson ironing operator $\mathrm{IRON}_{[a,b]}[F].$ Our next result extends this result to the truncated ironing operator.

\begin{proposition}
    \label{prop:Myerson_and_Monteiro}
    Let $F$ be a distribution with positive density on $[a,b]$. Then, for every $t \in [a,b]$, $\ell_F^t = \mathrm{IRON}_{[a,t]}[F]$. 
\end{proposition}



\if false
     \begin{figure}
    \centering
    \begin{tikzpicture}[every text node part/.style={align=center},transform shape,]
    \begin{axis}[
        width=10cm,
        height=8cm,
        xmin=-0.,xmax=1.0,
        ymin=-0.7,ymax=0.01,
        table/col sep=comma,
        xlabel={$v$},
        ylabel={Negative Revenue},
        grid=both,
        legend pos=south east,
        legend style ={font ={\tiny}}
    ]

    \addplot [blue, dashed, line width=.7mm,unbounded coords=jump] table[x=x,y=H] {Data/iron_uniform_with_hal_s=04_gamma=075.csv};
    \addlegendentry{Before ironing}

    \addplot [red, line width=.6mm, unbounded coords=jump]  table[x=x, y=psi, restrict expr to domain={\thisrow{x}}{0.625:1}]{Data/iron_uniform_with_hal_s=04_gamma=075.csv};
    \addlegendentry{Ironed}
    


    \addplot [red, line width=.6mm, unbounded coords=jump]  table[x=x, y=psi, restrict expr to domain={\thisrow{x}}{0.4:1}]{Data/iron_uniform_with_hal_s=04_gamma=075.csv};



    \addplot [violet, line width=.3mm,unbounded coords=jump] table[x=x,y expr={-0.36+0.25*\thisrow{F}}] {Data/iron_uniform_with_hal_s=04_gamma=075.csv};
    
    
    \filldraw[black] (axis cs:0.625,-0.7) circle (2pt) node[anchor=south east]{$T$};
    
    \end{axis}
    \end{tikzpicture}
    \end{figure}


        \begin{figure}[h]
    \centering
    \begin{tikzpicture}[every text node part/.style={align=center},transform shape,]
    \begin{axis}[
        width=10cm,
        height=8cm,
        xmin=-0.,xmax=7.0,
        ymin=-0.7,ymax=0.01,
        table/col sep=comma,
        xlabel={$v$},
        ylabel={Negative Revenue},
        grid=both,
        legend pos=south east,
        legend style ={font ={\tiny}}
    ]

    \addplot [blue, dashed, line width=.7mm,unbounded coords=jump] table[x=x,y=H] {Data/iron_exp_hal_s=5_gamma=095.csv};
    \addlegendentry{Before ironing}


    

     \addplot [red, line width=.6mm, unbounded coords=jump]  table[x=x, y=psi]{Data/iron_exp_hal_s=5_gamma=095.csv};%\addlegendentry{Ironed}}
     \addlegendentry{Ironed}




    \addplot [teal, line width=.3mm,unbounded coords=jump] table[x=x,y expr={-0.87+0.62*\thisrow{F}}] {Data/iron_exp_hal_s=5_gamma=095.csv};



    
    \end{axis}
    \end{tikzpicture}
    \caption{Revenue curve before and after ironing for the exponential distribution with $\lambda = 1$, signal $s=5$ and $\gamma = 0.5$. The figure also shows the affine function of $F$ that is used to ``convexify'' the value segment just before the signal $s$.}
    \label{fig:convexified-revenue}
    \end{figure}
%\end{frame}
\fi


We note that while \citet{monteiro2010optimal} provide a structural result about the general ironed virtual value function, one still needs to solve in general infinitely many semi-infinite optimization problems to be able to implement the optimal auction. In what follows, we characterize we solve Problem \eqref{eq:gen_virtual_value} for our model.

\subsection{Outline of the proof of \Cref{thm:main}}
\label{sec:outline}
Fix a regular distribution $F$ with positive continuous density $f$ on its support $[a,b]$.
The generalized convex hull of $H_{F_{\gamma,s}}$ is defined as,
\begin{subequations}
\begin{alignat}{2}
\Psi_{F_{\gamma,s}}(x) = \; &\!\sup_{\alpha,\beta \in \mathbb{R}} &\;& \alpha + \beta \cdot F_{\gamma,s}(x) \nonumber \\
&\text{s.t.} &      &  \alpha + \beta \cdot F_{\gamma,s}(y) \leq H_{F_{\gamma,s}}(y) \quad \forall y \in [a,b]. \nonumber
\end{alignat}
\end{subequations}

By expressing $F_{\gamma,s}$ and $H_{F_{\gamma,s}}$ as a function of $F$, $H_F$, $\gamma$ and $s$ (see \Cref{lem:F_and_H}), we obtain the following equivalent expression for $\Psi_{F_{\gamma,s}}$. For every $x$ we have that,
\begin{subequations}\label{eq:F_gamma_after_s}
\begin{alignat}{2}
\Psi_{F_{\gamma,s}}(x) = \; &\!\sup_{\alpha,\beta \in \mathbb{R}} &\;& \alpha  + \beta \cdot \gamma \cdot F(x) + \mathbbm{1} \{ x \geq s \} \cdot \beta \cdot (1-\gamma) \\
&\text{s.t.} &      &  \alpha + \beta \cdot \gamma \cdot F(y) \leq \gamma \cdot H_{F}(y) - (1-\gamma) \cdot y \quad \forall y < s. \label{eq:constraint_pre_s} \\ 
&  &      &  \alpha + \beta \cdot (1-\gamma) + \beta \cdot \gamma \cdot F(y) \leq \gamma \cdot H_{F}(y) \quad \forall y \geq s. \label{eq:constraint_post_s}
\end{alignat}
\end{subequations}
%To prove our result, we first characterize the solutions of Problem \eqref{eq:F_gamma_after_s} by constructing a threshold $T$ such that for every $x \geq T$, solutions of  Problem  \eqref{eq:gen_virtual_value} can be related to the ones of  Problem \eqref{eq:F_gamma_after_s}.

To prove \Cref{thm:main}, we aim to relate $\ell_{F_{\gamma,s}}$ to $\ell_{\gamma F}^s$ on the interval $[a,s)$ and $\ell_{F_{\gamma,s}}$ to $\ell_{F}$ on the interval $[s,b]$. Then, by applying \Cref{prop:Myerson_and_Monteiro}, we obtain the desired expression.\\

\noindent \textbf{Key proof technique.} To establish this result, we first prove that the generalized virtual value functions we consider are well-behaved on every interval which does not include $s$. We prove more generally the following result on the regularity of the generalized virtual value function.
\begin{lemma}
    \label{lem:continuity}
    Let $I$ be an interval included in $[a,b]$. Assume that $G$ admits a density $g$ that is positive and continuous on $I$. Then, $\ell_{G}$ is continuous on $I$.
\end{lemma}
Given a distribution $G$, recall that $\ell_{G}$ is the lowest generalized sub-gradient of the function $\Psi_G$ which is itself the generalized convex hull of the function $H_{G}$. Therefore, \Cref{lem:continuity} extends the statement that ``the convex hull of a differentiable function of one variable is continuously differentiable'' to our generalized notions of convexity and differentials. 

% From the expression of ${F_{\gamma,s}}$ derived in \Cref{lem:F_and_H}, and by the assumption that $F$ admits a positive and continuous density on $[a,b]$, we have that the distribution ${F_{\gamma,s}}$ admits a positive and continuous density on any interval which does not admit $s$. Hence, \Cref{lem:continuity} implies that the generalized virtual value $\ell_{F_{\gamma,s}}$


In turn, the key argument to prove that two distributions of interest $F$ and $G$ have the same virtual value function on some interval consists in first establishing the continuity of $\ell_F$ and $\ell_G$ by using \Cref{lem:continuity}. We then prove that $\ell_F$ is a generalized sub-gradient of $\Psi_G$ on the whole interval and conclude applying the following lemma.
\begin{lemma}
\label{lem:inclusion_to_eq}
Let $F$ and $G$ be two distributions on $[a,b]$, and let $I$ be an interval included in $[a,b]$. If $\ell_F(x) \in \partial \Psi_{G}(x)$ for all $x \in I$, and if $\ell_F$ and $\ell_G$ are continuous on $I$, then $\ell_F = \ell_G$ on $I$.
\end{lemma}
We next show how we relate the generalized virtual value functions of the distributions of interest on the intervals $[a,s)$ and $[s,b]$.\\

\noindent \textbf{Analysis on the interval $[s,b]$.}
We first prove that for some $T$ (defined in \Cref{prop:from_F_to_feasible_Fs}) we have that $\ell_F = \ell_{F_{\gamma,s}}$ over the interval $[T,b]$. As discussed previously, we establish this result by leveraging \Cref{lem:inclusion_to_eq}. Hence, it is sufficient to prove that $\ell_{F}(x) \in \partial \Psi_{F_{\gamma,s}}$ for every $x \in [T,b]$. We note that the definition of the generalized differential presented in \eqref{eq:subgrad_are_solutions} implies that $\ell_{F}(x) \in \partial \Psi_{F_{\gamma,s}}$ if and only if there exists an optimal solution for Problem \eqref{eq:F_gamma_after_s} where $\beta = \ell_F(x)$. In what follows, we construct such a solution.


Let $x \in [s,b]$ and remark that \eqref{eq:subgrad_are_solutions} implies that there there exists $(\aF,\bF)$ such that $\bF = \ell_{F}(x)$ which is optimal for Problem \eqref{eq:gen_virtual_value}. We define our related candidate solution for Problem \eqref{eq:F_gamma_after_s} as,
\begin{equation}
\label{eq:candidate}
(\aFs,\bFs) = (\gamma \cdot \aF - (1-\gamma) \cdot \bF, \bF).
\end{equation}
A critical aspect of the construction in \eqref{eq:candidate} is that $\bFs = \bF = \ell_{F}(x)$. Therefore, proving optimality of $(\aFs,\bFs)$ for Problem \eqref{eq:F_gamma_after_s} implies that $\ell_F(x) \in \partial \Psi_{F_{\gamma,s}}$.

A straightforward algebraic manipulation allows us to show that for every $x \in [s,b]$ the couple $(\aFs,\bFs)$ satisfies the constraint \eqref{eq:constraint_post_s} for every $y \geq s$.
However, the constraints \eqref{eq:constraint_pre_s} are not necessarily satisfied for all $x \in [s,b]$. We define the threshold $T$ such that $(\aFs,\bFs)$ satisfies the constraint \eqref{eq:constraint_pre_s} for all $y < s$.  
To that end, we define the following auxiliary mapping. For every $y \leq s$, let $\mu_y$ be defined as,
\begin{equation*}
\mu_y(x) = \aFs + \gamma \cdot \bFs \cdot F(y) - \gamma \cdot H_{F}(y) + (1-\gamma) \cdot y \quad \text{for every $x \in [s,b]$.}
\end{equation*}
This definition, implies that $(\aFs,\bFs)$ satisfies the constraint \eqref{eq:constraint_pre_s} at a given $y$ if and only if, $\mu_{y}(x) \leq 0$.  Consequently, the feasibility of $(\aFs,\bFs)$ for Problem \eqref{eq:F_gamma_after_s} reduces to the analysis of the sign of $\mu_y$. Our next result provides structural properties about $\mu_y$.
\begin{lemma}\label{lem:prop_mu}
\,
\begin{enumerate}
\item[(i)] $\mu_y$ is non-increasing for every $y \leq s$.
\item[(ii)] If $y > y'$, then for every $x \in [s,b]$, $\mu_{y}(x) > \mu_{y'}(x)$.
\item[(iii)] $\mu_s(s) > 0$ and $\mu_s(b) \leq 0$.
\end{enumerate}
\end{lemma}
\Cref{lem:prop_mu} implies, by property $(ii)$, that for every $(\aFs,\bFs)$ the most stringent constraint \eqref{eq:constraint_pre_s}  is for $y =s$. Furthermore, property $(i)$ implies that if $(\aFs,\bFs)$ satisfies \eqref{eq:constraint_pre_s} for a given $y$ and a given $x$ then for all $x' \geq x$, $(\aFs[x'],\bFs[x'])$ also satisfies \eqref{eq:constraint_pre_s} at $y$. By using these results, we construct a threshold $T$ such that the $(\aFs,\bFs)$ is feasible for all $x \geq T$. More generally, we prove the optimality of $(\aFs,\bFs)$ for Problem \eqref{eq:F_gamma_after_s} and establish the following result.
\begin{lemma}
\label{prop:from_F_to_feasible_Fs}
There exists $T \in (s,b]$ such that $\mu_s(T) = 0$. Furthermore, for every $x \in [T,b]$, we have that $\ell_{F}(x) \in \partial \Psi_{F_{\gamma,s}}(x)$.
\end{lemma}
Combining \Cref{lem:continuity}, \Cref{lem:inclusion_to_eq} and \Cref{prop:from_F_to_feasible_Fs} we conclude that $\ell_F = \ell_{F_{\gamma,s}}$ on $[T,b]$. We complete the proof on the interval $[s,b]$ by showing that $\ell_{F_{\gamma,s}}$ is constant on $[s,T]$.\\

\noindent \textbf{Analysis on the interval $[a,s)$.}
On this interval, we show that $\ell_F=\ell_{\gamma F}^s$, where $\ell_{\gamma F}^s$ is defined as the smallest generalized sub-gradient of the function, defined for every $x \in [a,s)$ as
\begin{subequations}\label{eq:main-gammaF}
\begin{alignat}{2}
\Psi^s_{\gamma F}(x) = \; &\!\sup_{\alpha,\beta \in \mathbb{R}} &\;& \alpha +  \beta \cdot \gamma \cdot F(x) \\
&\text{s.t.} &      &  \alpha + \beta \cdot \gamma \cdot F(y) \leq \gamma \cdot H_{F}(y) - (1-\gamma) \cdot y \quad \forall y \in [a,s]. %\label{eq:constraint_gF}
\end{alignat}
\end{subequations}
We note that for every $x \in [a,s)$, Problem \eqref{eq:main-gammaF} is a relaxation of Problem \eqref{eq:F_gamma_after_s} in which we removed the constraint \eqref{eq:constraint_post_s}. The main argument consists in proving that the relaxation is tight in the sense that the value of both problems is the same. 
In particular, we establish that for every $x \in [a,s)$, either $\ell_{F_{\gamma,s}}(x) = \ell_{F_{\gamma,s}}(s)$ or $\ell_{F_{\gamma,s}}(x) \in \partial \Psi_{\gamma F}^s(x)$. By \Cref{lem:inclusion_to_eq} we then conclude that $\ell_{F_{\gamma,s}}(x) \in \{\ell_{F_{\gamma,s}}(s), \ell_{\gamma F}^s(x)\}.$
Using a continuity argument, we conclude that $\ell_{F_{\gamma,s}}$ must equal $\ell_{\gamma F}^s$ on the whole interval $[a,s)$.





The complete proof of \Cref{thm:main} is presented in \Cref{sec:apx_main_proof}.

    
\section{Conclusion}
This paper presents {\tool}, a novel interactive text-to-SQL annotation system that enables users to create high-quality, schema-specific text-to-SQL datasets.
{\tool} integrates multiple functionalities, including schema customization, database synthesis, query alignment, dataset analysis, and additional features such as confidence scoring.
A user study with 12 participants demonstrates that by combining these features, {\tool} significantly reduces annotation time while improving the quality of the annotated data.
{\tool} effectively bridges the gap resulting from insufficient training and evaluation datasets for new or unexplored schemas.

%\newpage

\bibliographystyle{agsm}
\bibliography{ref}

\newpage

\appendix

\renewcommand{\theequation}{\thesection-\arabic{equation}}
\renewcommand{\theproposition}{\thesection-\arabic{proposition}}
\renewcommand{\thelemma}{\thesection-\arabic{lemma}}
\renewcommand{\thetheorem}{\thesection-\arabic{theorem}}
\renewcommand{\thedefinition}{\thesection-\arabic{definition}}
\pagenumbering{arabic}
\renewcommand{\thepage}{App-\arabic{page}}

\setcounter{equation}{0}
\setcounter{proposition}{0}
\setcounter{definition}{0}
\setcounter{lemma}{0}
\setcounter{theorem}{0}

\subsection{Lloyd-Max Algorithm}
\label{subsec:Lloyd-Max}
For a given quantization bitwidth $B$ and an operand $\bm{X}$, the Lloyd-Max algorithm finds $2^B$ quantization levels $\{\hat{x}_i\}_{i=1}^{2^B}$ such that quantizing $\bm{X}$ by rounding each scalar in $\bm{X}$ to the nearest quantization level minimizes the quantization MSE. 

The algorithm starts with an initial guess of quantization levels and then iteratively computes quantization thresholds $\{\tau_i\}_{i=1}^{2^B-1}$ and updates quantization levels $\{\hat{x}_i\}_{i=1}^{2^B}$. Specifically, at iteration $n$, thresholds are set to the midpoints of the previous iteration's levels:
\begin{align*}
    \tau_i^{(n)}=\frac{\hat{x}_i^{(n-1)}+\hat{x}_{i+1}^{(n-1)}}2 \text{ for } i=1\ldots 2^B-1
\end{align*}
Subsequently, the quantization levels are re-computed as conditional means of the data regions defined by the new thresholds:
\begin{align*}
    \hat{x}_i^{(n)}=\mathbb{E}\left[ \bm{X} \big| \bm{X}\in [\tau_{i-1}^{(n)},\tau_i^{(n)}] \right] \text{ for } i=1\ldots 2^B
\end{align*}
where to satisfy boundary conditions we have $\tau_0=-\infty$ and $\tau_{2^B}=\infty$. The algorithm iterates the above steps until convergence.

Figure \ref{fig:lm_quant} compares the quantization levels of a $7$-bit floating point (E3M3) quantizer (left) to a $7$-bit Lloyd-Max quantizer (right) when quantizing a layer of weights from the GPT3-126M model at a per-tensor granularity. As shown, the Lloyd-Max quantizer achieves substantially lower quantization MSE. Further, Table \ref{tab:FP7_vs_LM7} shows the superior perplexity achieved by Lloyd-Max quantizers for bitwidths of $7$, $6$ and $5$. The difference between the quantizers is clear at 5 bits, where per-tensor FP quantization incurs a drastic and unacceptable increase in perplexity, while Lloyd-Max quantization incurs a much smaller increase. Nevertheless, we note that even the optimal Lloyd-Max quantizer incurs a notable ($\sim 1.5$) increase in perplexity due to the coarse granularity of quantization. 

\begin{figure}[h]
  \centering
  \includegraphics[width=0.7\linewidth]{sections/figures/LM7_FP7.pdf}
  \caption{\small Quantization levels and the corresponding quantization MSE of Floating Point (left) vs Lloyd-Max (right) Quantizers for a layer of weights in the GPT3-126M model.}
  \label{fig:lm_quant}
\end{figure}

\begin{table}[h]\scriptsize
\begin{center}
\caption{\label{tab:FP7_vs_LM7} \small Comparing perplexity (lower is better) achieved by floating point quantizers and Lloyd-Max quantizers on a GPT3-126M model for the Wikitext-103 dataset.}
\begin{tabular}{c|cc|c}
\hline
 \multirow{2}{*}{\textbf{Bitwidth}} & \multicolumn{2}{|c|}{\textbf{Floating-Point Quantizer}} & \textbf{Lloyd-Max Quantizer} \\
 & Best Format & Wikitext-103 Perplexity & Wikitext-103 Perplexity \\
\hline
7 & E3M3 & 18.32 & 18.27 \\
6 & E3M2 & 19.07 & 18.51 \\
5 & E4M0 & 43.89 & 19.71 \\
\hline
\end{tabular}
\end{center}
\end{table}

\subsection{Proof of Local Optimality of LO-BCQ}
\label{subsec:lobcq_opt_proof}
For a given block $\bm{b}_j$, the quantization MSE during LO-BCQ can be empirically evaluated as $\frac{1}{L_b}\lVert \bm{b}_j- \bm{\hat{b}}_j\rVert^2_2$ where $\bm{\hat{b}}_j$ is computed from equation (\ref{eq:clustered_quantization_definition}) as $C_{f(\bm{b}_j)}(\bm{b}_j)$. Further, for a given block cluster $\mathcal{B}_i$, we compute the quantization MSE as $\frac{1}{|\mathcal{B}_{i}|}\sum_{\bm{b} \in \mathcal{B}_{i}} \frac{1}{L_b}\lVert \bm{b}- C_i^{(n)}(\bm{b})\rVert^2_2$. Therefore, at the end of iteration $n$, we evaluate the overall quantization MSE $J^{(n)}$ for a given operand $\bm{X}$ composed of $N_c$ block clusters as:
\begin{align*}
    \label{eq:mse_iter_n}
    J^{(n)} = \frac{1}{N_c} \sum_{i=1}^{N_c} \frac{1}{|\mathcal{B}_{i}^{(n)}|}\sum_{\bm{v} \in \mathcal{B}_{i}^{(n)}} \frac{1}{L_b}\lVert \bm{b}- B_i^{(n)}(\bm{b})\rVert^2_2
\end{align*}

At the end of iteration $n$, the codebooks are updated from $\mathcal{C}^{(n-1)}$ to $\mathcal{C}^{(n)}$. However, the mapping of a given vector $\bm{b}_j$ to quantizers $\mathcal{C}^{(n)}$ remains as  $f^{(n)}(\bm{b}_j)$. At the next iteration, during the vector clustering step, $f^{(n+1)}(\bm{b}_j)$ finds new mapping of $\bm{b}_j$ to updated codebooks $\mathcal{C}^{(n)}$ such that the quantization MSE over the candidate codebooks is minimized. Therefore, we obtain the following result for $\bm{b}_j$:
\begin{align*}
\frac{1}{L_b}\lVert \bm{b}_j - C_{f^{(n+1)}(\bm{b}_j)}^{(n)}(\bm{b}_j)\rVert^2_2 \le \frac{1}{L_b}\lVert \bm{b}_j - C_{f^{(n)}(\bm{b}_j)}^{(n)}(\bm{b}_j)\rVert^2_2
\end{align*}

That is, quantizing $\bm{b}_j$ at the end of the block clustering step of iteration $n+1$ results in lower quantization MSE compared to quantizing at the end of iteration $n$. Since this is true for all $\bm{b} \in \bm{X}$, we assert the following:
\begin{equation}
\begin{split}
\label{eq:mse_ineq_1}
    \tilde{J}^{(n+1)} &= \frac{1}{N_c} \sum_{i=1}^{N_c} \frac{1}{|\mathcal{B}_{i}^{(n+1)}|}\sum_{\bm{b} \in \mathcal{B}_{i}^{(n+1)}} \frac{1}{L_b}\lVert \bm{b} - C_i^{(n)}(b)\rVert^2_2 \le J^{(n)}
\end{split}
\end{equation}
where $\tilde{J}^{(n+1)}$ is the the quantization MSE after the vector clustering step at iteration $n+1$.

Next, during the codebook update step (\ref{eq:quantizers_update}) at iteration $n+1$, the per-cluster codebooks $\mathcal{C}^{(n)}$ are updated to $\mathcal{C}^{(n+1)}$ by invoking the Lloyd-Max algorithm \citep{Lloyd}. We know that for any given value distribution, the Lloyd-Max algorithm minimizes the quantization MSE. Therefore, for a given vector cluster $\mathcal{B}_i$ we obtain the following result:

\begin{equation}
    \frac{1}{|\mathcal{B}_{i}^{(n+1)}|}\sum_{\bm{b} \in \mathcal{B}_{i}^{(n+1)}} \frac{1}{L_b}\lVert \bm{b}- C_i^{(n+1)}(\bm{b})\rVert^2_2 \le \frac{1}{|\mathcal{B}_{i}^{(n+1)}|}\sum_{\bm{b} \in \mathcal{B}_{i}^{(n+1)}} \frac{1}{L_b}\lVert \bm{b}- C_i^{(n)}(\bm{b})\rVert^2_2
\end{equation}

The above equation states that quantizing the given block cluster $\mathcal{B}_i$ after updating the associated codebook from $C_i^{(n)}$ to $C_i^{(n+1)}$ results in lower quantization MSE. Since this is true for all the block clusters, we derive the following result: 
\begin{equation}
\begin{split}
\label{eq:mse_ineq_2}
     J^{(n+1)} &= \frac{1}{N_c} \sum_{i=1}^{N_c} \frac{1}{|\mathcal{B}_{i}^{(n+1)}|}\sum_{\bm{b} \in \mathcal{B}_{i}^{(n+1)}} \frac{1}{L_b}\lVert \bm{b}- C_i^{(n+1)}(\bm{b})\rVert^2_2  \le \tilde{J}^{(n+1)}   
\end{split}
\end{equation}

Following (\ref{eq:mse_ineq_1}) and (\ref{eq:mse_ineq_2}), we find that the quantization MSE is non-increasing for each iteration, that is, $J^{(1)} \ge J^{(2)} \ge J^{(3)} \ge \ldots \ge J^{(M)}$ where $M$ is the maximum number of iterations. 
%Therefore, we can say that if the algorithm converges, then it must be that it has converged to a local minimum. 
\hfill $\blacksquare$


\begin{figure}
    \begin{center}
    \includegraphics[width=0.5\textwidth]{sections//figures/mse_vs_iter.pdf}
    \end{center}
    \caption{\small NMSE vs iterations during LO-BCQ compared to other block quantization proposals}
    \label{fig:nmse_vs_iter}
\end{figure}

Figure \ref{fig:nmse_vs_iter} shows the empirical convergence of LO-BCQ across several block lengths and number of codebooks. Also, the MSE achieved by LO-BCQ is compared to baselines such as MXFP and VSQ. As shown, LO-BCQ converges to a lower MSE than the baselines. Further, we achieve better convergence for larger number of codebooks ($N_c$) and for a smaller block length ($L_b$), both of which increase the bitwidth of BCQ (see Eq \ref{eq:bitwidth_bcq}).


\subsection{Additional Accuracy Results}
%Table \ref{tab:lobcq_config} lists the various LOBCQ configurations and their corresponding bitwidths.
\begin{table}
\setlength{\tabcolsep}{4.75pt}
\begin{center}
\caption{\label{tab:lobcq_config} Various LO-BCQ configurations and their bitwidths.}
\begin{tabular}{|c||c|c|c|c||c|c||c|} 
\hline
 & \multicolumn{4}{|c||}{$L_b=8$} & \multicolumn{2}{|c||}{$L_b=4$} & $L_b=2$ \\
 \hline
 \backslashbox{$L_A$\kern-1em}{\kern-1em$N_c$} & 2 & 4 & 8 & 16 & 2 & 4 & 2 \\
 \hline
 64 & 4.25 & 4.375 & 4.5 & 4.625 & 4.375 & 4.625 & 4.625\\
 \hline
 32 & 4.375 & 4.5 & 4.625& 4.75 & 4.5 & 4.75 & 4.75 \\
 \hline
 16 & 4.625 & 4.75& 4.875 & 5 & 4.75 & 5 & 5 \\
 \hline
\end{tabular}
\end{center}
\end{table}

%\subsection{Perplexity achieved by various LO-BCQ configurations on Wikitext-103 dataset}

\begin{table} \centering
\begin{tabular}{|c||c|c|c|c||c|c||c|} 
\hline
 $L_b \rightarrow$& \multicolumn{4}{c||}{8} & \multicolumn{2}{c||}{4} & 2\\
 \hline
 \backslashbox{$L_A$\kern-1em}{\kern-1em$N_c$} & 2 & 4 & 8 & 16 & 2 & 4 & 2  \\
 %$N_c \rightarrow$ & 2 & 4 & 8 & 16 & 2 & 4 & 2 \\
 \hline
 \hline
 \multicolumn{8}{c}{GPT3-1.3B (FP32 PPL = 9.98)} \\ 
 \hline
 \hline
 64 & 10.40 & 10.23 & 10.17 & 10.15 &  10.28 & 10.18 & 10.19 \\
 \hline
 32 & 10.25 & 10.20 & 10.15 & 10.12 &  10.23 & 10.17 & 10.17 \\
 \hline
 16 & 10.22 & 10.16 & 10.10 & 10.09 &  10.21 & 10.14 & 10.16 \\
 \hline
  \hline
 \multicolumn{8}{c}{GPT3-8B (FP32 PPL = 7.38)} \\ 
 \hline
 \hline
 64 & 7.61 & 7.52 & 7.48 &  7.47 &  7.55 &  7.49 & 7.50 \\
 \hline
 32 & 7.52 & 7.50 & 7.46 &  7.45 &  7.52 &  7.48 & 7.48  \\
 \hline
 16 & 7.51 & 7.48 & 7.44 &  7.44 &  7.51 &  7.49 & 7.47  \\
 \hline
\end{tabular}
\caption{\label{tab:ppl_gpt3_abalation} Wikitext-103 perplexity across GPT3-1.3B and 8B models.}
\end{table}

\begin{table} \centering
\begin{tabular}{|c||c|c|c|c||} 
\hline
 $L_b \rightarrow$& \multicolumn{4}{c||}{8}\\
 \hline
 \backslashbox{$L_A$\kern-1em}{\kern-1em$N_c$} & 2 & 4 & 8 & 16 \\
 %$N_c \rightarrow$ & 2 & 4 & 8 & 16 & 2 & 4 & 2 \\
 \hline
 \hline
 \multicolumn{5}{|c|}{Llama2-7B (FP32 PPL = 5.06)} \\ 
 \hline
 \hline
 64 & 5.31 & 5.26 & 5.19 & 5.18  \\
 \hline
 32 & 5.23 & 5.25 & 5.18 & 5.15  \\
 \hline
 16 & 5.23 & 5.19 & 5.16 & 5.14  \\
 \hline
 \multicolumn{5}{|c|}{Nemotron4-15B (FP32 PPL = 5.87)} \\ 
 \hline
 \hline
 64  & 6.3 & 6.20 & 6.13 & 6.08  \\
 \hline
 32  & 6.24 & 6.12 & 6.07 & 6.03  \\
 \hline
 16  & 6.12 & 6.14 & 6.04 & 6.02  \\
 \hline
 \multicolumn{5}{|c|}{Nemotron4-340B (FP32 PPL = 3.48)} \\ 
 \hline
 \hline
 64 & 3.67 & 3.62 & 3.60 & 3.59 \\
 \hline
 32 & 3.63 & 3.61 & 3.59 & 3.56 \\
 \hline
 16 & 3.61 & 3.58 & 3.57 & 3.55 \\
 \hline
\end{tabular}
\caption{\label{tab:ppl_llama7B_nemo15B} Wikitext-103 perplexity compared to FP32 baseline in Llama2-7B and Nemotron4-15B, 340B models}
\end{table}

%\subsection{Perplexity achieved by various LO-BCQ configurations on MMLU dataset}


\begin{table} \centering
\begin{tabular}{|c||c|c|c|c||c|c|c|c|} 
\hline
 $L_b \rightarrow$& \multicolumn{4}{c||}{8} & \multicolumn{4}{c||}{8}\\
 \hline
 \backslashbox{$L_A$\kern-1em}{\kern-1em$N_c$} & 2 & 4 & 8 & 16 & 2 & 4 & 8 & 16  \\
 %$N_c \rightarrow$ & 2 & 4 & 8 & 16 & 2 & 4 & 2 \\
 \hline
 \hline
 \multicolumn{5}{|c|}{Llama2-7B (FP32 Accuracy = 45.8\%)} & \multicolumn{4}{|c|}{Llama2-70B (FP32 Accuracy = 69.12\%)} \\ 
 \hline
 \hline
 64 & 43.9 & 43.4 & 43.9 & 44.9 & 68.07 & 68.27 & 68.17 & 68.75 \\
 \hline
 32 & 44.5 & 43.8 & 44.9 & 44.5 & 68.37 & 68.51 & 68.35 & 68.27  \\
 \hline
 16 & 43.9 & 42.7 & 44.9 & 45 & 68.12 & 68.77 & 68.31 & 68.59  \\
 \hline
 \hline
 \multicolumn{5}{|c|}{GPT3-22B (FP32 Accuracy = 38.75\%)} & \multicolumn{4}{|c|}{Nemotron4-15B (FP32 Accuracy = 64.3\%)} \\ 
 \hline
 \hline
 64 & 36.71 & 38.85 & 38.13 & 38.92 & 63.17 & 62.36 & 63.72 & 64.09 \\
 \hline
 32 & 37.95 & 38.69 & 39.45 & 38.34 & 64.05 & 62.30 & 63.8 & 64.33  \\
 \hline
 16 & 38.88 & 38.80 & 38.31 & 38.92 & 63.22 & 63.51 & 63.93 & 64.43  \\
 \hline
\end{tabular}
\caption{\label{tab:mmlu_abalation} Accuracy on MMLU dataset across GPT3-22B, Llama2-7B, 70B and Nemotron4-15B models.}
\end{table}


%\subsection{Perplexity achieved by various LO-BCQ configurations on LM evaluation harness}

\begin{table} \centering
\begin{tabular}{|c||c|c|c|c||c|c|c|c|} 
\hline
 $L_b \rightarrow$& \multicolumn{4}{c||}{8} & \multicolumn{4}{c||}{8}\\
 \hline
 \backslashbox{$L_A$\kern-1em}{\kern-1em$N_c$} & 2 & 4 & 8 & 16 & 2 & 4 & 8 & 16  \\
 %$N_c \rightarrow$ & 2 & 4 & 8 & 16 & 2 & 4 & 2 \\
 \hline
 \hline
 \multicolumn{5}{|c|}{Race (FP32 Accuracy = 37.51\%)} & \multicolumn{4}{|c|}{Boolq (FP32 Accuracy = 64.62\%)} \\ 
 \hline
 \hline
 64 & 36.94 & 37.13 & 36.27 & 37.13 & 63.73 & 62.26 & 63.49 & 63.36 \\
 \hline
 32 & 37.03 & 36.36 & 36.08 & 37.03 & 62.54 & 63.51 & 63.49 & 63.55  \\
 \hline
 16 & 37.03 & 37.03 & 36.46 & 37.03 & 61.1 & 63.79 & 63.58 & 63.33  \\
 \hline
 \hline
 \multicolumn{5}{|c|}{Winogrande (FP32 Accuracy = 58.01\%)} & \multicolumn{4}{|c|}{Piqa (FP32 Accuracy = 74.21\%)} \\ 
 \hline
 \hline
 64 & 58.17 & 57.22 & 57.85 & 58.33 & 73.01 & 73.07 & 73.07 & 72.80 \\
 \hline
 32 & 59.12 & 58.09 & 57.85 & 58.41 & 73.01 & 73.94 & 72.74 & 73.18  \\
 \hline
 16 & 57.93 & 58.88 & 57.93 & 58.56 & 73.94 & 72.80 & 73.01 & 73.94  \\
 \hline
\end{tabular}
\caption{\label{tab:mmlu_abalation} Accuracy on LM evaluation harness tasks on GPT3-1.3B model.}
\end{table}

\begin{table} \centering
\begin{tabular}{|c||c|c|c|c||c|c|c|c|} 
\hline
 $L_b \rightarrow$& \multicolumn{4}{c||}{8} & \multicolumn{4}{c||}{8}\\
 \hline
 \backslashbox{$L_A$\kern-1em}{\kern-1em$N_c$} & 2 & 4 & 8 & 16 & 2 & 4 & 8 & 16  \\
 %$N_c \rightarrow$ & 2 & 4 & 8 & 16 & 2 & 4 & 2 \\
 \hline
 \hline
 \multicolumn{5}{|c|}{Race (FP32 Accuracy = 41.34\%)} & \multicolumn{4}{|c|}{Boolq (FP32 Accuracy = 68.32\%)} \\ 
 \hline
 \hline
 64 & 40.48 & 40.10 & 39.43 & 39.90 & 69.20 & 68.41 & 69.45 & 68.56 \\
 \hline
 32 & 39.52 & 39.52 & 40.77 & 39.62 & 68.32 & 67.43 & 68.17 & 69.30  \\
 \hline
 16 & 39.81 & 39.71 & 39.90 & 40.38 & 68.10 & 66.33 & 69.51 & 69.42  \\
 \hline
 \hline
 \multicolumn{5}{|c|}{Winogrande (FP32 Accuracy = 67.88\%)} & \multicolumn{4}{|c|}{Piqa (FP32 Accuracy = 78.78\%)} \\ 
 \hline
 \hline
 64 & 66.85 & 66.61 & 67.72 & 67.88 & 77.31 & 77.42 & 77.75 & 77.64 \\
 \hline
 32 & 67.25 & 67.72 & 67.72 & 67.00 & 77.31 & 77.04 & 77.80 & 77.37  \\
 \hline
 16 & 68.11 & 68.90 & 67.88 & 67.48 & 77.37 & 78.13 & 78.13 & 77.69  \\
 \hline
\end{tabular}
\caption{\label{tab:mmlu_abalation} Accuracy on LM evaluation harness tasks on GPT3-8B model.}
\end{table}

\begin{table} \centering
\begin{tabular}{|c||c|c|c|c||c|c|c|c|} 
\hline
 $L_b \rightarrow$& \multicolumn{4}{c||}{8} & \multicolumn{4}{c||}{8}\\
 \hline
 \backslashbox{$L_A$\kern-1em}{\kern-1em$N_c$} & 2 & 4 & 8 & 16 & 2 & 4 & 8 & 16  \\
 %$N_c \rightarrow$ & 2 & 4 & 8 & 16 & 2 & 4 & 2 \\
 \hline
 \hline
 \multicolumn{5}{|c|}{Race (FP32 Accuracy = 40.67\%)} & \multicolumn{4}{|c|}{Boolq (FP32 Accuracy = 76.54\%)} \\ 
 \hline
 \hline
 64 & 40.48 & 40.10 & 39.43 & 39.90 & 75.41 & 75.11 & 77.09 & 75.66 \\
 \hline
 32 & 39.52 & 39.52 & 40.77 & 39.62 & 76.02 & 76.02 & 75.96 & 75.35  \\
 \hline
 16 & 39.81 & 39.71 & 39.90 & 40.38 & 75.05 & 73.82 & 75.72 & 76.09  \\
 \hline
 \hline
 \multicolumn{5}{|c|}{Winogrande (FP32 Accuracy = 70.64\%)} & \multicolumn{4}{|c|}{Piqa (FP32 Accuracy = 79.16\%)} \\ 
 \hline
 \hline
 64 & 69.14 & 70.17 & 70.17 & 70.56 & 78.24 & 79.00 & 78.62 & 78.73 \\
 \hline
 32 & 70.96 & 69.69 & 71.27 & 69.30 & 78.56 & 79.49 & 79.16 & 78.89  \\
 \hline
 16 & 71.03 & 69.53 & 69.69 & 70.40 & 78.13 & 79.16 & 79.00 & 79.00  \\
 \hline
\end{tabular}
\caption{\label{tab:mmlu_abalation} Accuracy on LM evaluation harness tasks on GPT3-22B model.}
\end{table}

\begin{table} \centering
\begin{tabular}{|c||c|c|c|c||c|c|c|c|} 
\hline
 $L_b \rightarrow$& \multicolumn{4}{c||}{8} & \multicolumn{4}{c||}{8}\\
 \hline
 \backslashbox{$L_A$\kern-1em}{\kern-1em$N_c$} & 2 & 4 & 8 & 16 & 2 & 4 & 8 & 16  \\
 %$N_c \rightarrow$ & 2 & 4 & 8 & 16 & 2 & 4 & 2 \\
 \hline
 \hline
 \multicolumn{5}{|c|}{Race (FP32 Accuracy = 44.4\%)} & \multicolumn{4}{|c|}{Boolq (FP32 Accuracy = 79.29\%)} \\ 
 \hline
 \hline
 64 & 42.49 & 42.51 & 42.58 & 43.45 & 77.58 & 77.37 & 77.43 & 78.1 \\
 \hline
 32 & 43.35 & 42.49 & 43.64 & 43.73 & 77.86 & 75.32 & 77.28 & 77.86  \\
 \hline
 16 & 44.21 & 44.21 & 43.64 & 42.97 & 78.65 & 77 & 76.94 & 77.98  \\
 \hline
 \hline
 \multicolumn{5}{|c|}{Winogrande (FP32 Accuracy = 69.38\%)} & \multicolumn{4}{|c|}{Piqa (FP32 Accuracy = 78.07\%)} \\ 
 \hline
 \hline
 64 & 68.9 & 68.43 & 69.77 & 68.19 & 77.09 & 76.82 & 77.09 & 77.86 \\
 \hline
 32 & 69.38 & 68.51 & 68.82 & 68.90 & 78.07 & 76.71 & 78.07 & 77.86  \\
 \hline
 16 & 69.53 & 67.09 & 69.38 & 68.90 & 77.37 & 77.8 & 77.91 & 77.69  \\
 \hline
\end{tabular}
\caption{\label{tab:mmlu_abalation} Accuracy on LM evaluation harness tasks on Llama2-7B model.}
\end{table}

\begin{table} \centering
\begin{tabular}{|c||c|c|c|c||c|c|c|c|} 
\hline
 $L_b \rightarrow$& \multicolumn{4}{c||}{8} & \multicolumn{4}{c||}{8}\\
 \hline
 \backslashbox{$L_A$\kern-1em}{\kern-1em$N_c$} & 2 & 4 & 8 & 16 & 2 & 4 & 8 & 16  \\
 %$N_c \rightarrow$ & 2 & 4 & 8 & 16 & 2 & 4 & 2 \\
 \hline
 \hline
 \multicolumn{5}{|c|}{Race (FP32 Accuracy = 48.8\%)} & \multicolumn{4}{|c|}{Boolq (FP32 Accuracy = 85.23\%)} \\ 
 \hline
 \hline
 64 & 49.00 & 49.00 & 49.28 & 48.71 & 82.82 & 84.28 & 84.03 & 84.25 \\
 \hline
 32 & 49.57 & 48.52 & 48.33 & 49.28 & 83.85 & 84.46 & 84.31 & 84.93  \\
 \hline
 16 & 49.85 & 49.09 & 49.28 & 48.99 & 85.11 & 84.46 & 84.61 & 83.94  \\
 \hline
 \hline
 \multicolumn{5}{|c|}{Winogrande (FP32 Accuracy = 79.95\%)} & \multicolumn{4}{|c|}{Piqa (FP32 Accuracy = 81.56\%)} \\ 
 \hline
 \hline
 64 & 78.77 & 78.45 & 78.37 & 79.16 & 81.45 & 80.69 & 81.45 & 81.5 \\
 \hline
 32 & 78.45 & 79.01 & 78.69 & 80.66 & 81.56 & 80.58 & 81.18 & 81.34  \\
 \hline
 16 & 79.95 & 79.56 & 79.79 & 79.72 & 81.28 & 81.66 & 81.28 & 80.96  \\
 \hline
\end{tabular}
\caption{\label{tab:mmlu_abalation} Accuracy on LM evaluation harness tasks on Llama2-70B model.}
\end{table}

%\section{MSE Studies}
%\textcolor{red}{TODO}


\subsection{Number Formats and Quantization Method}
\label{subsec:numFormats_quantMethod}
\subsubsection{Integer Format}
An $n$-bit signed integer (INT) is typically represented with a 2s-complement format \citep{yao2022zeroquant,xiao2023smoothquant,dai2021vsq}, where the most significant bit denotes the sign.

\subsubsection{Floating Point Format}
An $n$-bit signed floating point (FP) number $x$ comprises of a 1-bit sign ($x_{\mathrm{sign}}$), $B_m$-bit mantissa ($x_{\mathrm{mant}}$) and $B_e$-bit exponent ($x_{\mathrm{exp}}$) such that $B_m+B_e=n-1$. The associated constant exponent bias ($E_{\mathrm{bias}}$) is computed as $(2^{{B_e}-1}-1)$. We denote this format as $E_{B_e}M_{B_m}$.  

\subsubsection{Quantization Scheme}
\label{subsec:quant_method}
A quantization scheme dictates how a given unquantized tensor is converted to its quantized representation. We consider FP formats for the purpose of illustration. Given an unquantized tensor $\bm{X}$ and an FP format $E_{B_e}M_{B_m}$, we first, we compute the quantization scale factor $s_X$ that maps the maximum absolute value of $\bm{X}$ to the maximum quantization level of the $E_{B_e}M_{B_m}$ format as follows:
\begin{align}
\label{eq:sf}
    s_X = \frac{\mathrm{max}(|\bm{X}|)}{\mathrm{max}(E_{B_e}M_{B_m})}
\end{align}
In the above equation, $|\cdot|$ denotes the absolute value function.

Next, we scale $\bm{X}$ by $s_X$ and quantize it to $\hat{\bm{X}}$ by rounding it to the nearest quantization level of $E_{B_e}M_{B_m}$ as:

\begin{align}
\label{eq:tensor_quant}
    \hat{\bm{X}} = \text{round-to-nearest}\left(\frac{\bm{X}}{s_X}, E_{B_e}M_{B_m}\right)
\end{align}

We perform dynamic max-scaled quantization \citep{wu2020integer}, where the scale factor $s$ for activations is dynamically computed during runtime.

\subsection{Vector Scaled Quantization}
\begin{wrapfigure}{r}{0.35\linewidth}
  \centering
  \includegraphics[width=\linewidth]{sections/figures/vsquant.jpg}
  \caption{\small Vectorwise decomposition for per-vector scaled quantization (VSQ \citep{dai2021vsq}).}
  \label{fig:vsquant}
\end{wrapfigure}
During VSQ \citep{dai2021vsq}, the operand tensors are decomposed into 1D vectors in a hardware friendly manner as shown in Figure \ref{fig:vsquant}. Since the decomposed tensors are used as operands in matrix multiplications during inference, it is beneficial to perform this decomposition along the reduction dimension of the multiplication. The vectorwise quantization is performed similar to tensorwise quantization described in Equations \ref{eq:sf} and \ref{eq:tensor_quant}, where a scale factor $s_v$ is required for each vector $\bm{v}$ that maps the maximum absolute value of that vector to the maximum quantization level. While smaller vector lengths can lead to larger accuracy gains, the associated memory and computational overheads due to the per-vector scale factors increases. To alleviate these overheads, VSQ \citep{dai2021vsq} proposed a second level quantization of the per-vector scale factors to unsigned integers, while MX \citep{rouhani2023shared} quantizes them to integer powers of 2 (denoted as $2^{INT}$).

\subsubsection{MX Format}
The MX format proposed in \citep{rouhani2023microscaling} introduces the concept of sub-block shifting. For every two scalar elements of $b$-bits each, there is a shared exponent bit. The value of this exponent bit is determined through an empirical analysis that targets minimizing quantization MSE. We note that the FP format $E_{1}M_{b}$ is strictly better than MX from an accuracy perspective since it allocates a dedicated exponent bit to each scalar as opposed to sharing it across two scalars. Therefore, we conservatively bound the accuracy of a $b+2$-bit signed MX format with that of a $E_{1}M_{b}$ format in our comparisons. For instance, we use E1M2 format as a proxy for MX4.

\begin{figure}
    \centering
    \includegraphics[width=1\linewidth]{sections//figures/BlockFormats.pdf}
    \caption{\small Comparing LO-BCQ to MX format.}
    \label{fig:block_formats}
\end{figure}

Figure \ref{fig:block_formats} compares our $4$-bit LO-BCQ block format to MX \citep{rouhani2023microscaling}. As shown, both LO-BCQ and MX decompose a given operand tensor into block arrays and each block array into blocks. Similar to MX, we find that per-block quantization ($L_b < L_A$) leads to better accuracy due to increased flexibility. While MX achieves this through per-block $1$-bit micro-scales, we associate a dedicated codebook to each block through a per-block codebook selector. Further, MX quantizes the per-block array scale-factor to E8M0 format without per-tensor scaling. In contrast during LO-BCQ, we find that per-tensor scaling combined with quantization of per-block array scale-factor to E4M3 format results in superior inference accuracy across models. 




\end{document}