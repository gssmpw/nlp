\section{Dataset Generation}
\label{sec:dataset}
\revise{
To train the proposed GNN, we constructed a dataset of building structures and a subset of these structures were subjected to fire simulations using FEA. The dataset generation process is illustrated in \figref{fig:dataset_generation_procedure}. Initially, a total of 33,000 building structures with geometrical details, material properties, and gravity loads were created. Due to randomness in generating these structures, a filter is applied to remove unreasonable data after gravity load simulation, which included 15,377 structures. A trade-off between computational feasibility and model performance is made among the remaining 17,623 structures. As further labeling structures with MIDR requires resource-intensive fire simulations via OpenSeesRT, a large proportion of 16,050 structures is selected as unlabeled dataset. On the other hand, each of the other 1,573 structures was further subjected to 30 different fire simulations, forming the labeled dataset containing $1,573\times 30 = 47,190$ fire cases.} This section details the step-by-step process for generating the dataset, including geometry creation, material property assignment, and simulations due to gravity loads and fire scenarios. 
% To train the proposed neural network, we constructed a dataset comprising building structure data and a subset of fire scenario data. The dataset generation process is illustrated in \figref{fig:dataset_generation_procedure}. 
% A total of 33,000 building structures with geometric details, material properties, and gravity loads were initially created. Out of these, 3,000 structures were selected as labeled data, and the remaining 30,000 were designated as unlabeled data. Further, about half of them filtered out due to instability under gravity loads only. 
\begin{figure*}[h!]
    \centering
    \includegraphics[width=0.8\linewidth]{figures/dataset_filter_procedure.pdf}
    \caption{Workflow for dataset generation (geometry, material property, gravity loads, and fire scenarios).}
    \label{fig:dataset_generation_procedure}
\end{figure*}

\subsection{Geometry Generation}
\label{subsec:geometry_generation}
The geometry of the building structures forms the foundation of the dataset. Regular 
\revise{3D structures} resembling multi-story parking structures or shopping malls were generated, with parameters such as building floor dimensions and story heights selected randomly. Each building structure is composed of multiple rooms, which serve as the basic unit in this study. A room herein is a cuboid space defined by specific length, width, and height. Within a structure, rooms of the same dimensions are uniformly arranged along the length, width, and height, corresponding to the $x$-, $y$-, and $z$-axes, respectively. Structures vary in room size and number of rooms along each axis. Specifically, the room length, width, and height are independently sampled from a uniform distribution within the interval $[2, 5]$ meters along the three directions of the structure. Similarly, the room number along each axis is uniformly sampled independently as an integer within the interval $[2, 7]$, i.e., the maximum number of stories of the buildings simulated in this study is 7.

To introduce variability and simulate real-world scenarios, approximately $8\%$ of structural elements (beams or columns) are randomly removed after initial geometry creation. 
\revise{Such removal is not fire-induced damage, but reflects functional diversity often observed in real buildings, such as open spaces designed for activities in shopping malls, e.g., ice skating rinks. Examples of the generated geometries are illustrated in \figref{fig:example_generated_geometry}, showcasing the diversity and realism of the dataset. This element removal does not affect the definition of room's geometry in the structure and nor does it affect the number of considered fire scenarios.} 

\revise{A range of coefficient of variation values ($3.3\%$ to $17.5\%$) was derived from prior studies that investigated the statistics of geometrical and material properties of structural components of buildings (e.g., \cite{mirza1979variations, lee2004probabilistic}). These studies provide empirical data on the natural variability in parameters such as Young's modulus, yield strength, and dimensions of structural elements due to manufacturing tolerances and material inconsistencies. By selecting $8\%$ for the removal of structural elements in our database, we aimed to maintain a level of variability that is representative of real-world uncertainties while ensuring computational feasibility. This choice ensures that the database captures realistic deviations without introducing extreme cases that may not be commonly encountered in practice.}

\begin{figure*}[h!]
    \centering
    \includegraphics[width=\linewidth]{figures/example_generated_geometry.pdf}
    \caption{Examples of generated structural geometry of different sizes (all dimensions in meters).}
    \label{fig:example_generated_geometry} 
\end{figure*}

{\blockRevise

In this study, we opted for a deterministic square, dimension of $0.1$ m, solid cross-sectional steel elements due to their simplicity in modeling and analysis. Square sections exhibit uniform geometrical properties in all directions, simplifying the computation of structural responses and avoiding complications associated with more complex shapes, such as wide-flange sections, facilitating the computational efficiency and scalability to generate a large dataset. This choice also helps to mitigate issues related to stress concentrations and facilitates a more straightforward representation of structural behavior under thermal loads. 

\textit{Remark:} The selected cross-section provides a comparable flexural rigidity to a $W 130 \times 130 \times 28.1$ wide-flange section (metric units), albeit with significantly higher axial rigidity. This cross-section is acceptable for gravity-load-designed frames under service loading conditions where the models assume fully rigid, moment-resisting beam-column connections for the evaluation of the IDR under thermal loading. This assumption is reasonable in this computational study where the primary interest is to understand the global deformation response of frames under fire conditions. The selection of uniform square cross-sections for both beams and columns, rather than adherence to standard capacity design principles, was made here primarily for computational efficiency and to reduce design parameters in the database generation process. This choice allows for simplified and scalable approach to analyze the fire-induced response of generic steel frames without the need for large section variations, where this study mainly focuses on the fire vulnerability assessment using ML-based predictions. However, if additional loading conditions, e.g., seismic or wind loads, were to be considered, larger sections, strong-column/weak-beam principle, and ductile detailing would be required in the generated buildings for realistic structural behavior under combined loading conditions. Future studies may also consider investigating the influence of variable cross-sectional dimensions and semi-rigid connections on the structural performance under fire conditions. 
} % blockRevise

\subsection{Material Properties}
Steel is chosen as the material for the structures. To reflect real-world variations, we randomly assign one of five slightly different steel material types to each structural element. \revise{
The ranges of material properties are provided in \tabref{tab:material_property_ranges} and the properties are sampled from uniform distributions of the corresponding ranges. These variations simulate differences arising from manufacturing batches or regional material properties. That these properties are at ambient temperature and change when the temperature rises due to a fire. The selection of materials with varying properties is aimed at increasing the diversity of the data. Our goal is to represent as wide a range of data as possible with a limited amount of building structure data, thereby enhancing the generalization ability of the GNN. Our assumed material property ranges are expected to be wider than the real-world conditions based on findings in \cite{mirza1979variations, lee2004probabilistic}. Therefore, we are essentially tackling a more challenging and general task. If we can solve this problem, we are confident that our method will perform equally well or even better in real-world scenarios.
}
\begin{table}[h!]
    \centering
    \caption{Material properties ranges for considered steel structures.}
    \begin{tabular}{lc}
        \toprule
        Property & Range \\
        \midrule
        Young's modulus & [168, 252] GPa \\
        Yield strength & [220, 330] MPa \\
        Strain-hardening ratio & [0.8, 1.2] \% \\
        \bottomrule
    \end{tabular}
    \label{tab:material_property_ranges}
\end{table}

\subsection{Gravity Loads}
Gravity loads are applied to columns and beams based on their \revise{influence (tributary) areas as typically conducted in structural analysis. The considered ``service'' load conditions include the column self-weight and the additional loads directly supported on the beams from their self-weight and weights of the reinforced concrete slabs, people as live load, and building content. An edge beam typically carries approximately half the gravity load supported by a parallel interior beam}. The ranges of gravity loads are listed in \tabref{tab:gravity_load_ranges}. \revise{The loads are sampled from uniform distributions of the corresponding ranges.} Structures that failed to meet an MIDR threshold of $1\%$ under gravity loads were deemed unacceptable designs and filtered out, as such configurations of randomly chosen geometry, material, and gravity load combinations were considered unrealistic from a regulatory and practicality points of view.
\begin{table}[h!]
    \centering
    \caption{Gravity load ranges for considered beams and columns.}
    \begin{tabular}{lc}
        \toprule
        Element & Range (kN/m)  \\
        \midrule
        Column & [0.5, 1.0]  \\
        Edge beam & [1.5, 4.5]  \\
        Interior beam & [3.0, 7.5]  \\
        \bottomrule
    \end{tabular}
    \label{tab:gravity_load_ranges}
\end{table} 

\subsection{Rule-based Thermal Load Generation}
\label{subsec:thermal_load_generation}
To evaluate a building's structural response during a fire event, we employed a simplified rule-based approach for thermal load generation. 
% Previous studies \cite{nan_structuralfire_2023} have demonstrated that steel structures rapidly equilibrate with surrounding gases temperatures due to efficient heat exchange. Consequently, gas temperatures can be directly used as inputs for FEA tools, e.g., OpenSees, simplifying the process of modeling thermal loads. 
% Accurately simulating temperature fields in fire scenarios poses significant challenges. Advanced thermodynamic simulations, such as those performed using Fire Dynamics Simulator (FDS) \cite{mcgrattan_fire_2000}, provide precise temperature predictions. However, these methods are hindered by high computational costs, prolonging execution times, and limited scalability, making them impractical for generating large datasets. Additionally, real-world fire loads often display substantial spatial variability across different rooms \cite{dundar_fire_2023}, resulting in scenario-specific temperature fields with limited generalizability. For example, studies on bridge fires \cite{he_study_2024} have demonstrated that environmental factors, such as wind speeds, can significantly influence temperature distributions. Furthermore, even within identical scenarios, variations in fire modeling methodologies can produce distinctly different temperature fields \cite{zhang_temperature_2020, du_new_2012}. These challenges emphasize the need for efficient and adaptable methods to generate fire temperature data.
% To address these issues, we adopted a rule-based approach to model temperature variations. 
According to \cite{spearpoint_fire_2008}, a typical fire development follows a predictable pattern. During the {\em{growth stage}}, the temperature rises slowly and approximately linearly after ignition. This is followed by the {\em{flashover stage}}, where temperatures increase rapidly to peak values. After reaching the peak, the temperature either stabilizes or continues to rise slowly until the {\em{decay stage}} begins. Inspired by this fire development pattern, we describe the temperature evolution in time, $t$, prior to the decay stage in two distinct stages:
\begin{enumerate}
    \item {\bf{Initial linear increase stage}}: For $t \in [0, t_1)$, temperature increases gradually and linearly as the fire spreads through the building. This stage represents the time before the fire directly affects a structural element.  
    \item {\bf{ISO 834 fire curve stage}}: For $t \in [t_1, t_{\thre}]$, temperature rises rapidly following the ISO 834 curve \cite{ISO834}, modeling the direct impact of the fire on the structural element. 
\end{enumerate}
The slope of the linear temperature increase, $c$, and the transition time, $t_1$, are influenced by the spatial relationship between the fire source and the structural element. For the second stage of temperature evolution, we utilize the ISO 834 curve, a widely accepted standard for fire resistance testing. This standardized fire curve describes the temperature rise over time, enabling rapid and consistent thermal fields across various scenarios. The duration of fire simulation in this study is set to $t_{\thre}=60$ minutes. This value represents the upper limit for the temperature evolution of each structural element, providing a consistent basis for analyzing the structural response to fire.

Let $(x, y, z)$ represents the midpoint of a structural element and $(x_{\subfire}, y_{\subfire}, z_{\subfire})$ the fire source point. \revise{Integer parameters $h$ and $h_{\subfire}$ correspond to the respective floor levels of the element and the fire source}. The temperature evolution for each element is expressed as follows:
\begin{enumerate}
    \item Linear increase stage ($0 < t < t_1$):
    \begin{equation}
    T(t) = c \cdot t,
    \end{equation}
    where $c$, the rate of temperature increase ($^\circ\mathrm{C}/\mathrm{min}$), depends on the height difference between the element, $h$, and the fire source, $h_{\subfire}$:
    \begin{equation}
        c = 
        \begin{cases} 
        5\left/\left(h - h_{\subfire} + 1\right)\right., & h \geq h_{\subfire}, \\
        2\left/\left(h_{\subfire} - h\right)\right., & h < h_{\subfire}.
        \end{cases}
    \end{equation}
     \item ISO 834 stage ($t \geq t_1$):
\begin{equation}
    T(t) = c \cdot t_1 + 345 \log_{10} \left(8 \left(t - t_1\right) + 1\right).
\end{equation}
\end{enumerate}

The transition (arrival) time $t_1$, marking the end of the linear stage, depends on the spatial distance between the fire source and the element. We define the following two Euclidean distances $L_p$ in the $xy$ plane and $L_s$ in the $xyz$ space:
\begin{eqnarray}
L_p & \triangleq & \sqrt{(x - x_{\subfire})^2 + (y - y_{\subfire})^2}, \\
\label{eq:Lp}
L_s & \triangleq & \sqrt{(x - x_{\subfire})^2 + (y - y_{\subfire})^2 + (z - z_{\subfire})^2}.
\label{eq:Ls}
\end{eqnarray}
Accordingly, the transition time, $t_1$, is expressed as follows:
\begin{equation}
    t_1 = 
    \begin{cases}
    \beta_{1} \cdot \left(1 - \exp\left\{- L_s\left/\alpha_{1}\right.\right\}\right), & h > h_{\subfire}, \\
    \beta_{2} \cdot \left(1 - \exp\left\{- L_p\left/\alpha_{2}\right.\right\}\right), & h = h_{\subfire}, \\
    \beta_{3} \cdot \left(1 - \exp\left\{- L_s\left/\alpha_{3}\right.\right\}\right), & h < h_{\subfire} .
    \end{cases}
    \label{eq:t1}
\end{equation}
The parameters $\beta_i$ and $\alpha_i$ for determining $t_1$ are summarized in Table~\ref{tab:fire_spread_parameters}. In this study, we take $r_{\mathrm{up}}=0.95$ and $r_{\mathrm{down}}=0.97$.
\begin{table}[ht]
    \centering
    \caption{Fire spread parameters for $t_1$ calculations.}
    \begin{tabular}{lcc}
        \toprule
        Case  & $\beta_i$ & $\alpha_i$  \\
        \midrule
        $i=1$, Upward spread & $16 \left.\left(1-r_{\mathrm{up}}^{\left|h-h_{\subfire}\right|}\right)\right/\left(1-r_{\mathrm{up}}\right)$ & $10$  \\
        $i=2$, Horizontal spread & $18$ & $18$  \\
        $i=3$, Downward spread & $30 \left.\left(1-r_{\mathrm{down}}^{\left|h-h_{\subfire}\right|}\right)\right/\left(1-r_{\mathrm{down}}\right)$ & $5$  \\
        \bottomrule
    \end{tabular}
    \label{tab:fire_spread_parameters}
\end{table}

\figref{fig:t1_curve} illustrates the $t_1$ curves for various fire scenarios: (1) fire originating on the lower floor, $h-h_{\subfire}=1$ with rapid upward spread, (2) fire on the same floor, $h=h_{\subfire}$ with the fastest spread, and (3) fire on the upper floor, $h_{\subfire}-h=1$ with slow downward spread. The exponential decay in $t_1$ reflects the accelerating fire propagation speed as the distance increases. \figref{fig:t1_curve} also indicates that the employed simplified model is consistent with the Markov chain-based dynamic model given by \cite{cheng_dynamic_2011}, where the rooms at the same floor of the fire point start flashover slightly before the corresponding upper floors. Additionally, $\beta_{1}$ and $\beta_{3}$ are the summation of a geometric sequence, where story level $h$ is the index. The common ratios $r_{\mathrm{up}}<1$ in $\beta_{1}$ and $r_{\mathrm{down}}<1$ in $\beta_{3}$ indicate that the fire speeds up to spread through the next story, which is consistent with the real-world fire spread mechanism given in \cite{hokugo_mechanism_2000}. The temperature profile within the range $t \in [0, t_{\thre}]$ is subsequently used as the thermal load in OpenSeesRT simulations to compute displacements at each structural node at time $t_{\thre}$.
\begin{figure}[h!]
    \centering
    \includegraphics[width=0.8\linewidth]{figures/m204_t1_curve.pdf}
    \caption{Three examples for the $t_1$ curve.}
    \label{fig:t1_curve}
\end{figure}

\revise{
\textit{Remark:} The effects of structural elements, such as concrete floor slabs and partitions, are not explicitly modeled in our approach. Instead, their influence is implicitly captured through the careful selection of the parameters $ \alpha, \beta, r_\mathrm{up} $, and $ r_\mathrm{down} $. This parameterization provides a unified framework for generating temperature fields. Indeed, fire propagation is governed by a multitude of factors and remains an open research question. For instance, if the fire resistance of a floor slab is enhanced by fire protective coating, the corresponding model can account for this by decreasing $\alpha_1$ \& $\alpha_3$, increasing $\beta_1$ \& $\beta_3$, and adopting larger values for $r_\mathrm{up}$ \& $r_\mathrm{down}$, which collectively slow down the vertical spread of fire. Conversely, scenarios involving higher amounts of combustible materials would warrant the opposite adjustments. This flexible and integrated approach avoids the need to design separate models for different fire propagation scenarios while still capturing the essential effects.
}

\revise{
In conclusion, our rule-based approach is a computationally efficient method for approximating fire temperature fields, enabling large-scale dataset generation to train predictive models. By combining ISO 834 fire curves with spatial considerations and embedding structural effects through parameter calibration, the method achieves a balanced trade-off between accuracy and scalability, making it a practical solution for thermal load modeling in fire scenarios. After generating the temperature of each beam or column according to the middle point, the temperature is applied as uniform thermal load to the elements of the structure in question using OpenSeesRT. 
}

% In conclusion, this rule-based approach is a computationally efficient method to approximate fire temperature fields, enabling large-scale dataset generation to train predictive models. By combining ISO 834 fire curves with spatial considerations, the method balances accuracy and scalability, making it a practical solution for thermal load modeling in fire scenarios.

% \subsection{Interstory Drift Ratio}
\subsection{OpenSeesRT Simulation}
\label{subsec:opensees_simulation}

The thermal and mechanical responses of 3D frame structures under combined fire and gravity loads are simulated using OpenSeesRT \cite{perez2024openseesrt}. \revise{In the simulation, the IDR of each node at $t_{\thre}$ is computed using the computed nodal displacements. Each structural model features six degrees of freedom per node (3 translational  and 3 rotational), with linear geometrical transformations (\texttt{geomTransf: Linear}) defining how the element local coordinate systems are mapped to the global coordinate system and assuming small displacements and rotations. Although OpenSeesRT allows a variety of options for modeling finite deformations, in the present simulations and mainly for simplicity, we did not consider large deformations. All bottom nodes (nodes on the ground) are fully constrained in all six degrees of freedom, while degrees of freedom os all other nodes are free.} Material behavior is temperature-dependent and modeled with \texttt{Steel01Thermal}, while fiber-based sections (\texttt{FiberThermal}) capture nonlinear interactions between thermal and mechanical responses at the cross-section level. \revise{Structural elements are represented as displacement-based Euler-Bernoulli beam-columns (\texttt{dispBeamColumnThermal}). This element  formulation accounts for thermal strains (temperature gradients) in the section, which is discretized into fibers. Numerical integration is used along the length of each element using three integration (Gauss) points, one at each end and the third in the middle of the element.}

{\revise{Thermal expansion of steel members plays a crucial role in IDR development. In reality, reinforced concrete floor slabs heat at a different rate than steel members due to their higher thermal mass and lower thermal conductivity. This differential heating can lead to restrained thermal expansion, introducing axial compression in beams and affecting the overall structural response. In this study, explicit {\em{composite action}} between steel members and concrete slabs is not modeled. Instead, our approach focuses on isolating the response of the steel structural frame, which is often the critical load-bearing component in fire scenarios. This assumption aligns with prior studies \cite{Possidente_2024} demonstrating that steel structures reach thermal equilibrium with surrounding gases quickly, allowing the use of uniform thermal loading in fire analysis. Future work could enhance this framework by incorporating slab-beam interaction effects, through a refined FEA for an extended dataset where constraints imposed by floor slabs are explicitly considered.}

The analysis begins with the application of gravity loads, followed by incremental thermal loads simulating the fire exposure. A static nonlinear solver using  \texttt{ExpressNewton} algorithm ensures convergence, while the \texttt{NormDispIncr} test maintains accuracy. An incremental \texttt{LoadControl} scheme with small step sizes is employed to guarantee numerical stability, using 10\% for gravity loads and 1\% for thermal loads. 

\revise{
In the thermal load analysis, uniform thermal load is applied to each beam or column, i.e., the temperature of each element is set to be that at the middle point, according to \secref{subsec:thermal_load_generation}. The \texttt{Steel01Thermal} material allows the properties (e.g., Young's modulus and yield strength) to be adjusted at increasing temperatures according to \cite{EN1993} using its Table 3.1: Reduction factors for the stress-strain relationship of carbon steel at elevated temperatures. For example, if the Young’s modulus at ambient temperature is $E_0$, then as the temperature ($T$) increases, the modulus changes as $E(T) = \eta (T) \times E_0$. \cite{EN1993} directly provides the values of $\eta(T) \in \left[0,1\right] $ at every $100 ^\circ\mathrm{C}$ interval and recommends using linear interpolation to obtain $\eta(T)$ for intermediate values of $T$.
} OpenSeesRT documentation \cite{OpenSeesThermalExamples} provides several examples of thermal analyses.

This modeling framework accommodates variations in material properties, cross-sectional geometries, and temperature profiles, providing robust simulations of structural behavior under fire conditions. The primary settings and configurations for the OpenSeesRT simulations are summarized in \tabref{tab:ops_detail}.
\begin{table}[h!]
    \centering
        \caption{Key settings of OpenSeesRT simulations.}
    \begin{tabular}{l|>{\raggedright\arraybackslash}p{0.6\linewidth}} %
    \toprule
    Modeling Aspect     & Details \\
    \midrule
    Geometry            & 3D models; 6 degrees of freedom per node \\
    Transformation      & geomTransf: Linear \\ 
    Material            & Steel01Thermal \\
    Section             & FiberThermal; Cross-section: $0.1$ m $\times$ $0.1$ m \\ 
    Element type        & {dispBeamColumnThermal} \\ 
    Loading             & Gravity loads: {beamUniform}; Thermal loads: {beamThermal} \\
    Integration scheme  & Incremental {LoadControl}; Step size: $10\%$ (gravity analysis), $1\%$ (thermal analysis) \\
    Nonlinear solver    & {ExpressNewton} algorithm; {UmfPack} solver; Convergence test: {NormDispIncr} tolerance: $10^{-8}$; Maximum \# iterations per step: $1000$. \\ 
    \bottomrule
    \end{tabular}
    \label{tab:ops_detail}
\end{table}

For each structure in the labeled dataset, 30 fire points are selected using a dual-granularity approach, \revise{i.e., two-stage sampling strategy,} to ensure they are well-distributed. Specifically, rooms are sequentially selected, with one fire point randomly chosen within each selected room. If a building is large and contains more than 30 rooms, we randomly select 30 rooms without replacement, i.e., ensuring that no more than one fire point is located in the same room. Conversely, if the building is small and has fewer than 30 rooms, all rooms are initially selected, with one fire point randomly assigned to each room. Additionally, rooms are then selected with replacement until a total of 30 fire points are assigned. \revise{The room-level sampling prioritizes selecting distinct rooms to avoid spatial clustering of fire points, while the point-level sampling ensures intra-room variability. This approach aligns with stratified sampling principles commonly used for efficient spatial representation, where multi-stage sampling strategies optimize coverage and variability, e.g., \cite{arunachalam_generalized_2023}, and enables a more comprehensive characterizing of how the structures respond under fire conditions.}
% This selection method prevents fire points from clustering too closely while maintaining an element of randomness. By distributing fire points in this manner, the 30 fire scenarios are effectively utilized, enabling a more comprehensive characterizing of how the structures respond under fire conditions.

\subsection{Summary of the Dataset Generation}
As discussed in this section and related to  \figref{fig:dataset_generation_procedure}, three key steps were considered in the development of the dataset: 
\begin{enumerate}
    \item {\bf{Filtering process}}: Structures with MIDR exceeding $1\%$ under gravity loads were excluded,  resulting in $1,573$ labeled structures retained for fire simulation and $16,050$ unlabeled structures for training the MFSP predictor.
    \item {\bf{Fire simulations}}: For each retained labeled structure, 30 fire scenarios were simulated using OpenSeesRT, yielding $47,190$ fire cases.
    \item {\bf{Data distribution check}}: MIDR distributions for labeled and unlabeled data under gravity loads were highly similar, because both datasets were generated using the same method. Under fire conditions, the MIDR distribution shifted, reflecting significant structural deformation with values reaching a maximum of about 6\%, an average of 1.70\%, and a standard deviation of 1.12\%. This step ensured a diverse and comprehensive dataset for the proposed predictive framework.
\end{enumerate}
The statistical distribution histograms for MIDR (after applying the $1\%$ filtering threshold \revise{for gravity load responses}) under different loading conditions are plotted in \figref{fig:histogram_mdr}. Figures \ref{fig:histogram_mdr}(a) and \ref{fig:histogram_mdr}(b) show the MIDR distributions of the labeled and unlabeled data, respectively, under gravity loads only. \figref{fig:histogram_mdr}(c) shows the MIDR distribution of the labeled data under the combined effects of gravity and fire loads. Fire load causes the structures to significantly deform, leading to a noticeably \revise{right-skewed} MIDR distribution.

\begin{figure*}[h!]
    \centering
    \includegraphics[width=\linewidth]{figures/histogram_mdr.pdf}
    \caption{Histograms of MIDR for labeled and unlabeled structures with gravity loads and fire cases.}
    \label{fig:histogram_mdr}
\end{figure*}

\revise{
This dataset provides the basis for training and testing the performance of the GNN-based framework. Although we employed a simplified rule-based thermal load generation method compared with conventional CFD-based simulations, the temperature field, the changes of the material properties, and the response of the structures, are all still highly nonlinear and complex. Therefore, it is still a challenging task for the NN to predict the MIDRs based on this dataset.
}
\vspace{-0.3cm}

\section{Experiments}

In this section, we present and analyze the experiments on four public datasets, aiming to 

\subsection{Settings}

\paragraph{Training Datasets.}

To evaluate the distillation method proposed in this paper, we conduct experiments on four commonly used and publicly available datasets with variable statistics in ~\cref{tab:data_stats}.
\begin{enumerate}[label=\arabic{enumi}), leftmargin=0.5cm]  %
    \item \emph{Amazon}\footnote{\url{http://snap.stanford.edu/data/amazon/productGraph/categoryFiles/}} includes Amazon product reviews and metadata infomation. For our empirical study, we mainly focus on the categories of "Magazine\_Subscriptions".
    \item \emph{MovieLens}\footnote{\url{https://grouplens.org/datasets/movielens/}} is maintained by GroupLens and it contains movie ratings from the MovieLens recommendation service. We use the "ml-100k" and "ml-1m" versions for our experiments.
    \item \emph{Epinions}\footnote{\url{https://cseweb.ucsd.edu/~jmcauley/datasets.html\#social_data}} was collected by~\cite{zhao2014leveraging} from Epinions.com, a popular online consumer review website. It describes consumer reviews and also contains trust relationships amongst users and spans more than a decade, from January 2001 to November 2013.
\end{enumerate}

\paragraph{Evaluation Metrics.}

We adopt the leave-one-out strategy for evaluation, following prior research~\cite{devlin2018bert, apgl4sr, yin2024dataset}. For each sequence, the most recent interaction is used for testing, the second for validation, and the rest for training. To expedite evaluation, as in previous studies~\cite{SASRec}, we randomly sample 100 negative items to rank with the ground-truth item, which closely approximates full-ranking results while significantly reducing the computational cost. We assess performance using HR, NDCG, and MRR. HR@k checks if the target item appears within the top-k recommendations, NDCG@k considers the item's rank, and MRR@k computes the average reciprocal rank of the first relevant item, with k $\in \{5, 10, 20\}$.

\paragraph{Implementation Details.}

We implement TD3 using PyTorch and develop recommendation models based on the library of Recbole~\cite{recbole}. Throughout the distillation process, we use SASRec~\cite{SASRec} serves as the learner across all datasets, utilizing attention heads $\in \{1, 2\}$, layers $\in \{1, 2\}$, hidden size $\in \{64, 128\}$, inner size $\in \{64, 128, 256\}$, and attention dropout probability
of 0.5 and hidden dropout probability of 0.2. For magazine and epinions dataset, we set $d1,d2\in\{8,16\}$, while for ml-100k and ml-1m dataset, we set $d1,d2\in\{16,32,64\}$. To evaluate the cross-architecture generalization of the proposed TD3, we employ GRU4Rec~\cite{GRU4Rec}, BERT4Rec~\cite{bert4rec}, and NARM~\cite{narm} for performance assessment. For all distilled datasets, we apply the Adam optimizer~\cite{diederik2014adam} for both the \emph{inner-loop} and \emph{outer-loop} optimization. 

In the \emph{outer-loop}, the synthetic sequence summary optimizer is configured with a learning rate $\alpha \in \{0.01, 0.03\}$ and a weight decay of 0.0001, using a cosine scheduler to adjust the learning rate throughout the process. In the \emph{inner-loop}, the learner optimizer employs a learning rate $\eta \in \{0.003, 0.005, 0.01\}$, with a weight decay of 0.00005. The inner steps are set to 200, using a random truncated window of 40 for backpropagation through time in RaT-BPTT, implemented via the Higher~\cite{higher} package across all datasets.


\section{Baseline} \label{sec:splitgraph}

The baseline method for batch-$k$DP solves each query using flow-augmenting path-based methods, which rely on the concept of \textit{split-graphs}~\cite{baseline_moreverbose, baseline1step2, baselineOnlySplitP1}. 
% For each query, paths are iteratively found in a split-graph, which is updated after each iteration.
% A split-graph is constructed by two transformations of the original graph:
% (1) reversing result-set paths, simulating flow-augmentation, and 
% (2) splitting vertices within these paths, giving rise to the name ``split-graph."

\textbf{Definition: Split-Graph~\cite{baselineOnlySplitP1}} 
Given a graph \( G = (V, E) \) and a set \( P \) of disjoint paths from \( s \) to \( t \), the split-graph \( \iG_{G,P} = (\iV_{G,P}, \iE_{G,P}) \) is constructed as follows:
(1) Initializing \( \iV_{G,P} = V \) and \( \iE_{G,P} = E \).
(2) For each edge in \( E(P) \), reversing the corresponding edge in \( \iE_{G,P} \).
(3) Splitting vertices \(v \in V(P) \setminus \{s, t\}\) into \(v^{in}\) and \(v^{out}\), and connecting them accordingly.
(4) Replacing edges in \(\iE_{G,P}\) with updated vertex connections, preserving incoming and outgoing edges.

% \textbf{Example}: 
% Fig.~\ref{fig:eg_split} shows the split-graph construction for the graph \( G \) in Fig.~\ref{fig:g} with $P= \{p_1=\{a, e, d, h\}\}$. Changes are shown in red.


% \vspace{-10pt}
\begin{figure}[h!]
\newcommand{\mylinewidth}{\linewidth}
\centering
    \begin{subfigure}[t]{0.35\mylinewidth}
        \centering
        % \resizebox{\mylinewidth}{!}
        {\includegraphics[width=\linewidth]{pic/eg/g}}
        \caption{Disjoint paths for $(a, h)$.}
        \label{fig:g}
    \end{subfigure}
    \begin{subfigure}[t]{0.6\mylinewidth}
        \centering
        % \resizebox{\mylinewidth}{!}
        {\includegraphics[width=\linewidth]{pic/eg/steps_red_new.pdf}}
        \caption{Split-graph with $P= \{p_1=\{$a$, $e$, $d$, $h$\}\}$.}
        \label{fig:eg_split}
    \end{subfigure}
    \caption{Examples of disjoint paths and split-graph.}
    % \label{fig:fg_share_intuition}
\end{figure} 
% \vspace{-5pt}

% 删除 begin
Given a graph \( G \) and vertices \( s \) and \( t \), the algorithm proceeds as follows:
% (1) Initialize \( P = \emptyset \) and \( \iG_{G,P} = G \).
% (2) Find the first path \( p_1 \) using a path-finding algorithm (e.g., BFS) in \( \iG_{G,P} \) and update \( \iG_{G,P} \).
% (3) Find the second path \( p_2 \), update found paths following an approach similar to augmenting paths in the maximum flow problem~\cite{baseline_moreverbose}, then update \( \iG_{G,P} \). More paths are found in a similar manner.
(1) Initialize $P = \emptyset$ and $\iG_{G, P} = G$.
(2) Find the first path $p_1$ in $\iG_{G, P}$ using any path-finding algorithm (e.g., BFS), forming $P_1 = \{p_1\}$, and update $\iG_{G, P}$ to $\iG_{G, P_1}$.
(3) Search for $p_2$ in $\iG_{G, P_1}$, yielding $P_2 = \{p_1, p_2\}$, and adjust $P_2$ following an approach similar to augmenting flows~\cite{baseline_moreverbose}.
Then update $\iG_{G, P_1}$ to $\iG_{G, P_2}$.
(4) Search for $p_3$ in $\iG_{G, P_2}$. More paths are found in a similar manner.
% 删除 end

\subsection{Overall Performance}
\vspace{1mm}

% add significance
\begin{table*}[ht]
  
  \centering
  \resizebox{\linewidth}{!}{
  % \renewcommand{\arraystretch}{1.5} % 增加行高
  % \footnotesize % 调整字体大小为小号
  % \scriptsize
  %\setlength{\tabcolsep}{5pt} % 调整列与列之间的间距
  % \resizebox{\linewidth}{!}{
  \begin{tabular}{l|c|cc|cccccc|cc} 
    
    \hline
    % \cline{3-7}
    % & \multicolumn{4}{c|}{\textbf{Original Query}} & \textbf{Classical QE} & \multicolumn{6}{c}{\textbf{LLM-based QE}}\\
    % \multirow{2}{*}{\textbf{Model($\rightarrow$) }} & \multirow{2}{*}{\textbf{BM25}} & \multicolumn{5}{c}{\textbf{Finetuned Dense Retrievers}}\\
    % & & DPR & ANCE & Contriever & BGE & QE-LLaMA\\
    % \cline{3-7}
    \multirow{2}{*}{\textbf{Task }} & \multirow{2}{*}{\textbf{BM25}} & \multicolumn{8}{c|}{\textbf{Unsupervised Dense Retrievers}}& \multicolumn{2}{c}{\textbf{Supervised Dense Retrievers}}\\ \cline{3-12}
     &   & \textbf{coCondenser} & \textbf{Contriever}\rlap{$\text{}^{\dagger}$} & \textbf{PRF}\rlap{$\text{}^{\diamond}$} & \textbf{Q2Q} & \textbf{Q2E} & \textbf{Q2C} & \textbf{Q2D}\rlap{$\text{}^{\mathsection}$} & \textbf{LLM-QE} & \textbf{Contriever}\rlap{$\text{}^{\mathparagraph}$}  & \textbf{LLM-QE}\\
     \hline
    
    MS MARCO           & 22.8          & 16.2         & 20.55\rlap{$\text{}^{\diamond}$}         & 16.66  & 22.07          & 21.38         & 22.10         & 23.00\rlap{$\text{}^{\dagger \diamond}$}        & 25.20\rlap{$\text{}^{\dagger \diamond \mathsection}$}          & \uline{34.33}  & \textbf{34.70} \\
    Trec-COVID         & \uline{65.6}  & 40.4         & 27.45         & 27.71  & 38.76          & 48.64         & 58.81         & 57.25\rlap{$\text{}^{\dagger \diamond}$}        & 59.66\rlap{$\text{}^{\dagger \diamond}$}          & 34.16  & \textbf{68.62}\rlap{$\text{}^{\mathparagraph}$}  \\
    NFCorpus           & 32.5          & 28.9         & 31.73\rlap{$\text{}^{\diamond}$}         & 27.49  & 31.53          & 32.90         & 32.80         & 33.20\rlap{$\text{}^{\dagger \diamond}$}        & \textbf{33.61}\rlap{$\text{}^{\dagger \diamond}$} & 32.71   & \uline{33.47} \\
    NQ                 & 32.9          & 17.8         & 25.37\rlap{$\text{}^{\diamond}$}         & 20.98  & 34.80          & 29.05         & 36.82         & 38.91\rlap{$\text{}^{\dagger \diamond}$}        & \uline{43.26}\rlap{$\text{}^{\dagger \diamond \mathsection}$}  & 34.02  & \textbf{51.47}\rlap{$\text{}^{\mathparagraph}$} \\
    HotpotQA           & 60.3          & 34.0         & 48.07\rlap{$\text{}^{\diamond}$}         & 40.43  & 56.15          & 46.15         & 59.82         & 61.84\rlap{$\text{}^{\dagger \diamond}$}        & \uline{65.82}\rlap{$\text{}^{\dagger \diamond \mathsection }$}  & 58.78   & \textbf{67.44}\rlap{$\text{}^{\mathparagraph}$} \\
    FiQA               & 23.6          & 25.1         & 24.50\rlap{$\text{}^{\diamond}$}         & 19.65  & 26.69          & 25.20         & 27.23         & 27.38\rlap{$\text{}^{\dagger \diamond}$}        & \uline{30.12}\rlap{$\text{}^{\dagger \diamond \mathsection}$}  & 28.04  & \textbf{33.48}\rlap{$\text{}^{\mathparagraph}$} \\
    ArguAna            & 31.5          & 44.4         & 37.90         & 38.19  & 42.89          & 43.24         & 41.83         & 42.90\rlap{$\text{}^{\dagger \diamond}$}        & 43.06\rlap{$\text{}^{\dagger \diamond}$}          & \uline{52.70} & \textbf{52.92} \\
    Touche-2020        & \textbf{36.7} & 11.7         & 16.68\rlap{$\text{}^{\diamond}$}         & 14.26  & 12.93          & 18.01         & 23.12         & 26.33\rlap{$\text{}^{\dagger \diamond}$}        & 24.34\rlap{$\text{}^{\dagger \diamond}$}          & 10.46  & \uline{26.61}\rlap{$\text{}^{\mathparagraph}$}  \\
    CQADupStack        & 29.9          & 30.9         & 28.43\rlap{$\text{}^{\diamond \mathsection}$}         & 23.18  & 25.21          & 26.74         & 21.90         & 24.69\rlap{$\text{}^{\diamond}$}        & 27.84\rlap{$\text{}^{\diamond \mathsection}$}          & \uline{31.60}
    & \textbf{33.35}\rlap{$\text{}^{\mathparagraph}$} \\
    Quora              & 78.9          & 82.1         & \uline{83.50}\rlap{$\text{}^{\diamond \mathsection}$} & 81.43  & 81.65          & 82.28         & 80.80         & 81.53        & 82.54\rlap{$\text{}^{\diamond \mathsection}$}          & \textbf{85.53}  & 81.96          \\
    DBPedia            & 31.3          & 21.5         & 29.16\rlap{$\text{}^{\diamond}$}         & 23.43  & 32.18          & 29.13         & 34.27         & 36.10\rlap{$\text{}^{\dagger \diamond}$}        & \uline{38.20}\rlap{$\text{}^{\dagger \diamond \mathsection}$}  & \textbf{38.22}  & 37.77 \\
    Scidocs            & 15.8          & 13.6         & 14.91\rlap{$\text{}^{\diamond}$}         & 13.51  & 15.32          & 15.12         & 15.17         & 15.52\rlap{$\text{}^{\dagger \diamond}$}        & \uline{16.63}\rlap{$\text{}^{\dagger \diamond \mathsection}$}  & 15.67  & \textbf{17.27}\rlap{$\text{}^{\mathparagraph}$} \\
    FEVER              & 75.3          & 61.5         & 68.20\rlap{$\text{}^{\diamond}$}         & 58.95  & 70.07          & 66.93         & 75.36         & 78.62\rlap{$\text{}^{\dagger \diamond}$}        & \uline{82.80}\rlap{$\text{}^{\dagger \diamond \mathsection}$}  & 82.49  & \textbf{85.03}\rlap{$\text{}^{\mathparagraph}$} \\
    Climate-FEVER      & 21.4          & 16.9         & 15.50\rlap{$\text{}^{\diamond}$}         & 13.52  & 15.40          & 15.02         & 22.28         & 19.43\rlap{$\text{}^{\dagger \diamond}$}        & 21.16\rlap{$\text{}^{\dagger \diamond \mathsection}$}          & \uline{23.04}   & \textbf{23.08} \\
    Scifact            & 66.5          & 56.1         & 64.92\rlap{$\text{}^{\diamond}$}         & 60.56  & 67.05          & 66.73         & 66.35         & 66.52\rlap{$\text{}^{\diamond}$}        & \uline{67.74}\rlap{$\text{}^{\dagger \diamond}$}  & \textbf{68.64}  & 66.28          \\
    \hline
    Avg. BEIR14        & 43.0          & 34.6         & 36.88{$\text{}^{\diamond}$}         & 33.09  & 39.33          & 38.94         & 42.61         & 43.59\rlap{$\text{}^{\dagger \diamond}$}        & \uline{45.48}\rlap{$\text{}^{\dagger \diamond \mathsection}$}  & 42.59  & \textbf{48.48}\rlap{$\text{}^{\mathparagraph}$} \\
    Avg. All           & 41.7          & 33.4         & 35.79{$\text{}^{\diamond}$}         & 32.00  & 38.18          & 37.77         & 41.24         & 42.21\rlap{$\text{}^{\dagger \diamond}$}        & \uline{44.13}\rlap{$\text{}^{\dagger \diamond \mathsection}$}  & 42.04  & \textbf{47.56}\rlap{$\text{}^{\mathparagraph}$} \\
    % \hline
    Best on            & 1             & 0            & 0             & 0      & 0              & 0             & 0             & 0            & 1              & 3  
    & \textbf{10}             \\
    \hline
     
  \end{tabular}}
  \caption{Overall Performance of LLM-QE. We follow previous work~\cite{izacard2021unsupervised} and report the average performance across 14 BEIR tasks (BEIR14) and all tasks (All). \textbf{Bold} and \uline{underlined} scores indicate the best and second-best results. $\dagger$, $\diamond$, and $\mathsection$ denote significant improvements over Contriver, PRF, and Q2D in the unsupervised setting, while $\mathparagraph$ indicates a significant improvement over Contriver in the supervised setting.}
  \label{tab:overall}
\end{table*}

% \end{document}




% \begin{table*}[ht]
%   \label{tab:overall}
%   \centering
%   % \renewcommand{\arraystretch}{1.5} % 增加行高
%   % \footnotesize % 调整字体大小为小号
%   \scriptsize
%   % \setlength{\tabcolsep}{4pt} % 调整列与列之间的间距
%   % \resizebox{\linewidth}{!}{
%   \begin{tabular}{l|cccc|c|cccc|c|c} 
    
%     \hline
%     % \cline{3-7}
%     & \multicolumn{4}{c|}{\textbf{Original Query}} & \textbf{Classical QE} & \multicolumn{6}{c}{\textbf{LLM-based QE}}\\
%     % \multirow{2}{*}{\textbf{Model($\rightarrow$) }} & \multirow{2}{*}{\textbf{BM25}} & \multicolumn{5}{c}{\textbf{Finetuned Dense Retrievers}}\\
%     % & & DPR & ANCE & Contriever & BGE & QE-LLaMA\\
%     % \cline{3-7}
%     & {} & \multicolumn{9}{c}{\textbf{Unsupervised Dense Retrievers}}& \textbf{Fintuned}\\
%     {\textbf{Task }} & \textbf{BM25}  & \textbf{coCondenser} & \textbf{Contriever} & \textbf{Anchor-DR}  & \textbf{PRF} & \textbf{Q2Q} & \textbf{Q2E} & \textbf{Q2C} & \textbf{Q2D} & \textbf{LLM-QE} & \textbf{LLM-QE*}\\
%      \hline
    
%     MS MARCO           & 22.8          & 16.2         & 20.55         & \uline{26.15}  & 16.66  & 22.07          & 21.38         & 22.10         & 23.00        & & \textbf{33.08} \\
%     Trec-COVID         & 65.6          & 40.4         & 27.45         & \textbf{72.16} & 27.71  & 38.76          & 48.64         & 58.81         & 57.25        & & \uline{71.07}  \\
%     NFCorpus           & 32.5          & 28.9         & 31.73         & 30.70          & 27.49  & 31.53          & 32.90         & 32.80         & \uline{33.20}& & \textbf{33.83} \\
%     NQ                 & 32.9          & 17.8         & 25.37         & 28.83          & 20.98  & 34.80          & 29.05         & 36.82         & \uline{38.91}& & \textbf{49.38} \\
%     HotpotQA           & 60.3          & 34.0         & 48.07         & 53.81          & 40.43  & 56.15          & 46.15         & 59.82         & \uline{61.84}& & \textbf{64.34} \\
%     FiQA               & 23.6          & 25.1         & 24.50         & 23.79          & 19.65  & 26.69          & 25.20         & 27.23         & \uline{27.38}& & \textbf{34.51} \\
%     ArguAna            & 31.5          & \uline{44.4} & 37.90         & 28.39          & 38.19  & 42.89          & 43.24         & 41.83         & 42.90        & & \textbf{51.82} \\
%     Touche-2020        & \textbf{36.7} & 11.7         & 16.68         & 21.85          & 14.26  & 12.93          & 18.01         & 23.12         & 26.33        & & \uline{28.57}  \\
%     CQADupStack        & 29.9          & \uline{30.9} & 28.43         & 28.82          & 23.18  & 25.21          & 26.74         & 21.90         & 24.69        & & \textbf{32.90} \\
%     Quora              & 78.9          & 82.1         & \uline{83.50} & \textbf{85.57} & 81.43  & 81.65          & 82.28         & 80.80         & 81.53        & & 83.19          \\
%     DBPedia            & 31.3          & 21.5         & 29.16         & 34.61          & 23.43  & 32.18          & 29.13         & 34.27         & \uline{36.10}& & \textbf{37.79} \\
%     Scidocs            & \uline{15.8}  & 13.6         & 14.91         & 13.42          & 13.51  & 15.32          & 15.12         & 15.17         & 15.52        & & \textbf{17.49} \\
%     FEVER              & 75.3          & 61.5         & 68.20         & 72.15          & 58.95  & 70.07          & 66.93         & 75.36         & \uline{78.62}& & \textbf{85.49} \\
%     Climate-FEVER      & 21.4          & 16.9         & 15.50         & 18.87          & 13.52  & 15.40          & 15.02         & \uline{22.28} & 19.43        & & \textbf{23.08} \\
%     Scifact            & 66.5          & 56.1         & 64.92         & 58.84          & 60.56  & \textbf{67.05} & \uline{66.73} & 66.35         & 66.52        & & 66.63          \\
%     \hline
%     Avg. BEIR14        & 43.0          & 34.6         & 36.88         & 40.84          & 33.09  & 39.33          & 38.94         & 42.61         & \uline{43.59}& & \textbf{48.58} \\
%     Avg. All           & 41.7          & 33.4         & 35.79         & 39.86          & 32.00  & 38.18          & 37.77         & 41.24         & \uline{42.21}& & \textbf{47.54} \\
%     Best on            & 1             & 0            & 0             & 2              & 0      & 1              & 0             & 0             & 0            & & 11             \\
%     \hline
     
%   \end{tabular}
%   \caption{Overall Performance on MS MARCO and BEIR under nDCG@10. We follow previous work~\cite{izacard2021unsupervised} and report the average performance on 14 BEIR tasks (BEIR14) and MSMARCO (All). The results of ANCE, Contriever and bge-Large are evaluated using their released checkpoints. The results of other baselines are copied from their original papers. Bold and underlined scores indicate the best and second best results, respectively. ${\dagger}$, ${\ddagger}$ and ${\mathsection}$ indicate statistically significant improvements over $\text{ANCE}^{\dagger}$, $\text{Contriever}^{\ddagger}$ and $\text{bge-Large}^{\mathsection}$, respectively. }
% \end{table*}


We evaluated TD3's performance across various synthetic sequence summary sizes and diverse datasets. In \cref{tab:overall_results}, we compare models trained on full original datasets with those using various-sized synthetic sequence summaries. Additionally, \cref{tab:baselines} and \cref{fig:baseline_results} compare TD3's performance with Farzi and heuristic sampling methods: \emph{random sampling}, which selects sequences uniformly, and \emph{longest sampling}, which selects sequences in descending order of length. Our findings are as follows: 1) TD3 achieves comparable training performance, even with substantial data compression. This shows that small-batch synthetic summaries distilled from the original dataset effectively capture essential information, preserving data integrity for model training. 2) In datasets such as Magazine and Epinions, models trained on TD3-distilled summaries outperform those trained on the original datasets. This highlights the value of high-quality, smaller datasets over larger, noisier ones, underscoring the importance of data quality in model training. 3) As illustrated in \cref{tab:baselines} and \cref{fig:baseline_results}, TD3 is more sample-efficient than Farzi and heuristic methods, showing superior data utilization. 

These results illustrate the transformative potential of data distillation in improving sequential recommendation systems. This approach represents a shift towards a data-centric paradigm in recommender systems, where prioritizing data quantity and quality and strategic compression can create more robust and efficient algorithms, reducing computational costs and storage demands. This evolution paves the way for the next generation of recommendation algorithms, focusing on maximizing value from the minimal data.


\begin{table}[t]
    \centering
    \caption{Number of real multiplications and sum rate for different methods (``Ericsson'' site, $N_{\sf{T}} = 64$, $N_{\sf{U}} = 4$, SNR = 27dB).}
    \resizebox{\columnwidth}{!}{
    \begin{tabular}{ccc}
        \toprule
        \textbf{Methods} & \textbf{\# of Multiplications} & \textbf{Sum-Rate (b/s/Hz)}\\
        \midrule
        WMMSE \cite{shi2011iteratively} ($I=12.5$)    & 36.1\,M & 37.29
        \\ 
        MAML-CNN (zero-shot) & 3.77\,M & 37.01 \\
        \textbf{PaPP} (zero-shot) & \textbf{1.05\,M} & \textbf{38.10}\\
        Zero Forcing \cite{nayebi2017precoding} & 8.4\,K & 32.14\\
        \bottomrule
    \end{tabular} }
    \label{tab:complexity}
\end{table}
Table \ref{tab:complexity} shows an example of the computational complexity of different precoding methods, measured by the number of real multiplications required for processing when the deployment site is ``Ericsson''. This analysis offers insights into the trade-offs between computational demands and sum-rate performance.

The \gls{WMMSE} algorithm is renowned for achieving good weighted sum rate in mMIMO systems. However, this performance comes at the cost of high computational complexity. Specifically, with stopping criteria of $10^{-3}$ and an average of 12.5 iterations for this setup, the WMMSE method requires approximately 36 million multiplications. This complexity can limit real-time applications or systems constrained by computational resources. The total number of real multiplications for the WMMSE method is given by
\[
 4I \bigg(\frac{2}{3} N_{\sf{T}}^3N_{\sf{U}} + N_{\sf{T}}^2N_{\sf{U}} + 2N_{\sf{T}}(2N_{\sf{U}}^2+N_{\sf{U}})+ N_{\sf{U}}^2+\frac{14}{3}N_{\sf{U}}\bigg) \, ,
\]
where $I$ is the total number of iterations. 

As another baseline, we consider the method proposed in \cite{lyu2023downlink}, which combines a multilayer perceptron (MLP) model with the WMMSE algorithm. However, we replace the original MLP model with a CNN model similar to the approach in \cite{hojatian2021unsupervised}.
This modification was made to better capture spatial features in the \gls{CSI}, enabling the model to handle the increased complexity introduced by a larger number of antennas and users. By leveraging the representational power of CNNs, MAML-CNN achieves more competitive results in our experimental settings compared to the original MAML-MLP. Since it combines a \gls{DNN} with an additional matrix inversion step, it exhibits significantly higher complexity than \gls{ZF} but remains less demanding than WMMSE. 
The number of real multiplications required by MAML-CNN is given by
\begin{align*}
 & \
C_{\text{out}}N_{\sf{T}}N_{\sf{U}}C_{\text{in}}k^2 + C_{\text{out}}N_{\sf{T}}N_{\sf{U}}(3N_{\sf{U}}+1) \\
  & + 8\bigg(\frac{4}{3} N_{\sf{T}}^3 + N_{\sf{T}}^2 (3N_{\sf{U}}+2) + N_{\sf{T}}(2N_{\sf{U}}+3)\bigg) \, ,
\end{align*}
where $C_{\text{in}}$ and  $C_{\text{out}}$ are the input and output channels of the CNN layer, and $k$  is the kernel size. 

The proposed PaPP method leverages a DNN to provide a zero-shot precoding solution. While it exhibits greater computational complexity compared to traditional methods like ZF due to its convolutional and fully connected layers, this approach offers significant performance benefits, including improved interference mitigation and adaptability to unseen sites. The total number of real multiplications required for the PaPP method can be expressed as
\begin{align*}
 & \ C_{\text{out}}N_{\sf{T}}N_{\sf{U}}C_{\text{in}}k^2 + C_{\text{out}}N_{\sf{T}}N_{\sf{U}}D_{\text{FC1}} \\
& +D_{\text{FC1}}D_{\text{FC2}} + D_{\text{FC2}}D_{\text{FC3}} + D_{\text{FC3}}D_{\text{FC4r}} + D_{\text{FC3}}D_{\text{FC4i}}\, ,
\end{align*}
where $D_{\text{FCi}}$ ($i=1,2,3,4$) represent the sizes of the \glspl{FCL}. 

The \gls{ZF} \cite{nayebi2017precoding} precoding method offers significantly lower computational complexity, requiring approximately 8.4 thousand multiplications. This efficiency stems from its reliance on simpler linear algebra operations, specifically the inversion of smaller matrices when $N_{\sf{T}}>N_{\sf{U}}$. Despite its computational efficiency, ZF is susceptible to performance degradation in high-interference scenarios or under adverse channel conditions. The total number of real multiplications required by ZF is
\[
8 N_{\sf{U}}^2 N_{\sf{T}} + \frac{8}{3} N_{\sf{U}}^3 \, .
\]\vspace{-3mm}

\subsection{Time and Memory Analysis}

\vspace{1mm}

\subsubsection{Theoretical Memory Complexity}

As discussed in Section \ref{sec:related_work}, Farzi \cite{sachdeva2023farzi} decomposes synthetic data $\syn \in \mathbb{R}^{\mu \times \zeta \times |\vocab|}$ to a latent summary $\mathbf{D} \in \mathbb{R}^{\mu \times \zeta \times d}$ and a decoder $\mathbf{M} \in \mathbb{R}^{d \times |\vocab|}$, where $d \ll |\vocab|$. To highlight the advantages of employing Tucker decomposition for three-dimensional data, we perform a comparative analysis of our proposed approach with that of Farzi. In particular, we investigate the computational footprint associated with the bi-level optimization framework by evaluating memory usage during a single outer-loop step, providing insights into the efficiency gains achieved through our method as follows:
\begin{align*}
    \operatorname{Farzi~:} & 
    ~ \mathcal{O}\big( |\Phi| + |\mathcal{B}^{\real}| \cdot \zeta \cdot d_3 + |\mathcal{B}^{\syn}| \cdot \zeta \cdot |\vocab| + \mu \cdot \zeta \cdot d + d \cdot |\vocab| \big) \\
    \operatorname{TD3~:} & 
    ~ \mathcal{O}\big( |\Phi| + |\mathcal{B}^{\real}| \cdot \zeta \cdot d_3 + |\mathcal{B}^{\syn}| \cdot \zeta \cdot |\vocab| + (\mu + \zeta) \cdot d_1 + d_1^2 \cdot d_3 \big) 
\end{align*}
where $|\mathcal{B}^{\real}|$ and $|\mathcal{B}^{\syn}|$ denote the batch size of real and synthetic data, respectively, while $d_3$ represents the item embeddings' hidden dimension. Additionally, $d$, $d_1$, and $d_3$ are of the same order of magnitude and much smaller than $|\vocab|$. When the item set is very large, or the distilled sequence summary requires larger values of $\mu$ and $\zeta$, the inequality $(\mu + \zeta) \cdot d_1 + d_1^2 \cdot d_3 \ll \mu \cdot \zeta \cdot d + d \cdot |\vocab|$ holds, the method proposed in our work will offer a spatial advantage. 

\subsubsection{Empirical Computational Complexity}

The data distillation process consumes considerable wall-clock time and GPU memory, so we conducted a detailed quantitative analysis of these requirements for both distillation and model training on the original and distilled datasets, as shown in \cref{tab:time_memory}. Wall-clock time is reported in single A100 (80GB) GPU hours. While distillation generally takes longer and uses more memory than training on the original data, its cost is often amortizable in real-world scenarios where multiple models must be trained on the same dataset. 
The amortization, based on the ratios in the table, varies by dataset and distillation scale. Notably, for large datasets, where training time is typically lengthy, training on significantly reduced distilled data can shorten this process by several orders of magnitude. This trade-off substantially decreases future training time, making distillation a one-time cost that yields long-term benefits for various downstream tasks, such as hyperparameter tuning and architecture exploration. Hence, the distillation process and its amortization will be well justified.

\begin{figure}[htbp]
    \centering
    \begin{subfigure}{0.48\textwidth}
        \centering
        \includegraphics[width=\linewidth]{figure/magazine.pdf}
    \end{subfigure}
    \hfill
    \vspace{1mm}
    \begin{subfigure}{0.48\textwidth}
        \centering
        \includegraphics[width=\linewidth]{figure/ml100k.pdf}
    \end{subfigure}
    \vspace{0.8mm}
    \caption{Illustration of the performance comparison of the TD3 against: \emph{Farzi}, \emph{random sampling} and \emph{longest sampling}.}
    \label{fig:baseline_results}
\end{figure}
\vspace{-1cm}


\subsection{Cross-Architecture Generalization}

\vspace{1mm}

Since the synthetic sequence summary is carefully tailored for optimizing a specific learner model, we assess its generalizability across various unseen architectures, as shown in \cref{tab:cross-arch}. We first distill the Epinions dataset using SASRec~\cite{SASRec}, resulting in a condensed synthetic summary of size $[50 \times 20]$. This summary is then used to train several alternative architectures, including GRU4Rec~\cite{GRU4Rec}, which models sequential behavior using Gated Recurrent Units (GRU); NARM~\cite{li2017neural}, which enhances GRU4Rec with an attention mechanism to emphasize user intent; and BERT4Rec~\cite{bert4rec}, which uses bidirectional self-attention to learn sequence representations. The models trained on the synthetic summary demonstrate strong generalization across diverse architectures, maintaining high predictive performance. In some cases, they even outperform models trained on the original dataset, highlighting the effectiveness of our proposed TD3 method in enabling cross-architecture transferability.

\def\arraystretch{1.2}
\setlength{\tabcolsep}{0.5em} %

\begin{table}[t!] \centering
    \caption{
    Evaluation of generalization performance on unseen architectures using a synthetic summary of size $[50 \times 20]$ distilled from the Epinions dataset via SASRec. 
    }
    \vspace{-8pt}
    \label{tab:cross-arch}
    \begin{center}
    \resizebox{0.98\linewidth}{!}{
        \begin{tabular}{ c | c c c c }
            \toprule
            \multirow{4}{*}{\textbf{Metric}} & \multicolumn{4}{c}{\textbf{Architecture}} \\
            & \multicolumn{4}{c}{{Synthetic Data / Original Data}} \\
            & SASRec & NARM & GRU4Rec & BERT4Rec \\
            \midrule
            HR@10   & \underline{19.75}~/~19.61 & \underline{19.60}~/~19.43 & 19.11~/~20.25 & \underline{18.98}~/~17.14 \\
            HR@20   & \underline{31.30}~/~30.90 & 30.55~/~31.10 & 29.49~/~31.49 & \underline{29.08}~/~27.62 \\
            NDCG@10 & \underline{10.61}~/~10.45 & \underline{10.67}~/~10.25 & 10.55~/~10.93 & \underline{10.31}~/~9.29 \\
            NDCG@20 & \underline{13.51}~/~13.28 & \underline{13.43}~/~13.18 & 13.15~/~13.75 & \underline{12.83}~/~11.91 \\
            MRR@10  & \underline{7.85}~/~7.69   & \underline{7.98}~/~7.48   & 7.96~/~8.13   & \underline{7.70}~/~6.93 \\
            MRR@20  & \underline{8.64}~/~8.46   & \underline{8.73}~/~8.28   & 8.66~/~8.90   & \underline{8.38}~/~7.63 \\
            \bottomrule
        \end{tabular}
    }
    \end{center}
\end{table}






\subsection{Ablation Studies}

To analyze our method's components, we conducted ablation studies on \emph{ML-100k}. Results for \emph{Feature Space Alignment} (FSA) and \emph{Augmented Learner Training} (ALT) are in \cref{tab:ablation_results}. FSA and ALT complement each other. ALT significantly boosts TD3's performance by enhancing contextual understanding and reducing dependence on specific sequence patterns, improving sequence information capture and data updates in the \emph{outer-loop}. FSA alone also enhances performance across metrics by strengthening the objective function, aiding convergence to a similar solution in the feature space, and maintaining performance on original and synthetic data. Utilizing both in the \emph{inner-loop} and \emph{outer-loop} maximizes distillation benefits.

% To analyze our method's components, we performed ablation studies on \emph{ML-100k}. Results for \emph{Feature Space Alignment} (FSA) and \emph{Augmented Learner Training} (ALT) are shown in \cref{tab:ablation_results}. FSA and ALT complement each other, with ALT significantly enhancing TD3's performance by improving contextual understanding and sequence information capture. FSA alone boosts performance by strengthening the objective function and aiding convergence. Using both in the \emph{inner-loop} and \emph{outer-loop} maximizes distillation benefits.

\begin{table*}
  [t]
  \centering
  \resizebox{\textwidth}{!}{%
  \begin{tabular}{cccccccccccc}
    \toprule \multicolumn{2}{c}{Components}                                                             & \multicolumn{5}{c}{Re-executability Rate (\%)} & \multicolumn{5}{c}{Readability (\#)} \\
    \cmidrule(lr){1-2} \cmidrule(lr){3-7} \cmidrule(lr){8-12}        \hspace{8pt}\labelemoji\hspace{8pt}                                                                & \hspace{8pt}\toolemoji\hspace{8pt}                                      & O0                                 & O1             & O2             & O3             & AVG            & O0             & O1             & O2             & O3             & AVG            \\
    \hline
    \rowcolor[rgb]{0.93,0.93,0.93}\multicolumn{12}{c}{\textbf{Initialize with LLM4Decompile-End-6.7B~\citep{llm4decompile}}}   \\
    \xmark                                                                                              & \xmark                                    & 69.51                              & 46.95          & 50.61          & 46.34          & 53.35          & 3.98 & 3.41 & 3.44 & 3.38 & 3.55 \\
    \cmark                                                                                              & \xmark                                    & 75.61                              & 50.61          & 50.00          & 50.00          & 56.55          & 4.01 & 3.44 & 3.39 & \textbf{3.49} & 3.58 \\
    \xmark                                                                                              & \cmark                                    & 83.54                     & \textbf{56.10}          & 51.22          & 50.61 & 60.37 & 4.05 & 3.51 & 3.51 & 3.42 & 3.62 \\
    \cmark                                                                                              & \cmark                                    & \textbf{85.37}                            & \textbf{56.10}                     & \textbf{51.83} & \textbf{52.43}          & \textbf{61.43} & \textbf{4.13} & \textbf{3.60} & \textbf{3.54} & \textbf{3.49} & \textbf{3.69} \\

    \rowcolor[rgb]{0.93,0.93,0.93}\multicolumn{12}{c}{\textbf{Initialize with Deepseek-Coder-6.7B-base~\citep{deepseekcoder}}} \\
    \xmark                                                                                              & \xmark                                    & 59.15                              & 35.98          & 39.02          & 37.80          & 42.99          & 3.71 & 3.05 & 3.16 & 3.05 & 3.24 \\
    \cmark                                                                                              & \xmark                                    & 66.46                              & 41.46          & 38.41          & 36.59          & 45.73          & 3.76 & 3.17 & \textbf{3.21} & 3.08 & 3.31 \\
    \xmark                                                                                              & \cmark                                    & 70.73                              & 39.63          & 39.02          & 40.24          & 47.41          & 3.90 & 3.17 & 3.08 & 3.11 & 3.31 \\
    \cmark                                                                                              & \cmark                                    & \textbf{79.88}                     & \textbf{45.73} & \textbf{43.90} & \textbf{42.68} & \textbf{53.05} & \textbf{3.96} & \textbf{3.21} & 3.18 & \textbf{3.19} & \textbf{3.38} \\
    \bottomrule
  \end{tabular}%
  }
  \caption{The ablation study of different methods across four optimization levels
  (O0, O1, O2, O3), as well as their average scores (AVG). The results in bold represent the optimal performance. The ~\labelemoji~ and ~\toolemoji~ means Relabedling and Function Call. \textbf{Bold} denotes the best performance.}
  \label{tab:ablation}
\end{table*}
