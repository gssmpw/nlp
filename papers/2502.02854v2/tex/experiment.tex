\section{Dataset}
\label{sec:dataset}

\subsection{Data Collection}

To analyze political discussions on Discord, we followed the methodology in \cite{singh2024Cross-Platform}, collecting messages from politically-oriented public servers in compliance with Discord's platform policies.

Using Discord's Discovery feature, we employed a web scraper to extract server invitation links, names, and descriptions, focusing on public servers accessible without participation. Invitation links were used to access data via the Discord API. To ensure relevance, we filtered servers using keywords related to the 2024 U.S. elections (e.g., Trump, Kamala, MAGA), as outlined in \cite{balasubramanian2024publicdatasettrackingsocial}. This resulted in 302 server links, further narrowed to 81 English-speaking, politics-focused servers based on their names and descriptions.

Public messages were retrieved from these servers using the Discord API, collecting metadata such as \textit{content}, \textit{user ID}, \textit{username}, \textit{timestamp}, \textit{bot flag}, \textit{mentions}, and \textit{interactions}. Through this process, we gathered \textbf{33,373,229 messages} from \textbf{82,109 users} across \textbf{81 servers}, including \textbf{1,912,750 messages} from \textbf{633 bots}. Data collection occurred between November 13th and 15th, covering messages sent from January 1st to November 12th, just after the 2024 U.S. election.

\subsection{Characterizing the Political Spectrum}
\label{sec:timeline}

A key aspect of our research is distinguishing between Republican- and Democratic-aligned Discord servers. To categorize their political alignment, we relied on server names and self-descriptions, which often include rules, community guidelines, and references to key ideologies or figures. Each server's name and description were manually reviewed based on predefined, objective criteria, focusing on explicit political themes or mentions of prominent figures. This process allowed us to classify servers into three categories, ensuring a systematic and unbiased alignment determination.

\begin{itemize}
    \item \textbf{Republican-aligned}: Servers referencing Republican and right-wing and ideologies, movements, or figures (e.g., MAGA, Conservative, Traditional, Trump).  
    \item \textbf{Democratic-aligned}: Servers mentioning Democratic and left-wing ideologies, movements, or figures (e.g., Progressive, Liberal, Socialist, Biden, Kamala).  
    \item \textbf{Unaligned}: Servers with no defined spectrum and ideologies or opened to general political debate from all orientations.
\end{itemize}

To ensure the reliability and consistency of our classification, three independent reviewers assessed the classification following the specified set of criteria. The inter-rater agreement of their classifications was evaluated using Fleiss' Kappa \cite{fleiss1971measuring}, with a resulting Kappa value of \( 0.8191 \), indicating an almost perfect agreement among the reviewers. Disagreements were resolved by adopting the majority classification, as there were no instances where a server received different classifications from all three reviewers. This process guaranteed the consistency and accuracy of the final categorization.

Through this process, we identified \textbf{7 Republican-aligned servers}, \textbf{9 Democratic-aligned servers}, and \textbf{65 unaligned servers}.

Table \ref{tab:statistics} shows the statistics of the collected data. Notably, while Democratic- and Republican-aligned servers had a comparable number of user messages, users in the latter servers were significantly more active, posting more than double the number of messages per user compared to their Democratic counterparts. 
This suggests that, in our sample, Democratic-aligned servers attract more users, but these users were less engaged in text-based discussions. Additionally, around 10\% of the messages across all server categories were posted by bots. 

\subsection{Temporal Data} 

Throughout this paper, we refer to the election candidates using the names adopted by their respective campaigns: \textit{Kamala}, \textit{Biden}, and \textit{Trump}. To examine how the content of text messages evolves based on the political alignment of servers, we divided the 2024 election year into three periods: \textbf{Biden vs Trump} (January 1 to July 21), \textbf{Kamala vs Trump} (July 21 to September 20), and the \textbf{Voting Period} (after September 20). These periods reflect key phases of the election: the early campaign dominated by Biden and Trump, the shift in dynamics with Kamala Harris replacing Joe Biden as the Democratic candidate, and the final voting stage focused on electoral outcomes and their implications. This segmentation enables an analysis of how discourse responds to pivotal electoral moments.

Figure \ref{fig:line-plot} illustrates the distribution of messages over time, highlighting trends in total messages volume and mentions of each candidate. Prior to Biden's withdrawal on July 21, mentions of Biden and Trump were relatively balanced. However, following Kamala's entry into the race, mentions of Trump surged significantly, a trend further amplified by an assassination attempt on him, solidifying his dominance in the discourse. The only instance where Trump’s mentions were exceeded occurred during the first debate, as concerns about Biden’s age and cognitive abilities temporarily shifted the focus. In the final stages of the election, mentions of all three candidates rose, with Trump’s mentions peaking as he emerged as the victor.
\vspace{-0.3cm}

\section{Experiments}

In this section, we present and analyze the experiments on four public datasets, aiming to 

\subsection{Settings}

\paragraph{Training Datasets.}

To evaluate the distillation method proposed in this paper, we conduct experiments on four commonly used and publicly available datasets with variable statistics in ~\cref{tab:data_stats}.
\begin{enumerate}[label=\arabic{enumi}), leftmargin=0.5cm]  %
    \item \emph{Amazon}\footnote{\url{http://snap.stanford.edu/data/amazon/productGraph/categoryFiles/}} includes Amazon product reviews and metadata infomation. For our empirical study, we mainly focus on the categories of "Magazine\_Subscriptions".
    \item \emph{MovieLens}\footnote{\url{https://grouplens.org/datasets/movielens/}} is maintained by GroupLens and it contains movie ratings from the MovieLens recommendation service. We use the "ml-100k" and "ml-1m" versions for our experiments.
    \item \emph{Epinions}\footnote{\url{https://cseweb.ucsd.edu/~jmcauley/datasets.html\#social_data}} was collected by~\cite{zhao2014leveraging} from Epinions.com, a popular online consumer review website. It describes consumer reviews and also contains trust relationships amongst users and spans more than a decade, from January 2001 to November 2013.
\end{enumerate}

\paragraph{Evaluation Metrics.}

We adopt the leave-one-out strategy for evaluation, following prior research~\cite{devlin2018bert, apgl4sr, yin2024dataset}. For each sequence, the most recent interaction is used for testing, the second for validation, and the rest for training. To expedite evaluation, as in previous studies~\cite{SASRec}, we randomly sample 100 negative items to rank with the ground-truth item, which closely approximates full-ranking results while significantly reducing the computational cost. We assess performance using HR, NDCG, and MRR. HR@k checks if the target item appears within the top-k recommendations, NDCG@k considers the item's rank, and MRR@k computes the average reciprocal rank of the first relevant item, with k $\in \{5, 10, 20\}$.

\paragraph{Implementation Details.}

We implement TD3 using PyTorch and develop recommendation models based on the library of Recbole~\cite{recbole}. Throughout the distillation process, we use SASRec~\cite{SASRec} serves as the learner across all datasets, utilizing attention heads $\in \{1, 2\}$, layers $\in \{1, 2\}$, hidden size $\in \{64, 128\}$, inner size $\in \{64, 128, 256\}$, and attention dropout probability
of 0.5 and hidden dropout probability of 0.2. For magazine and epinions dataset, we set $d1,d2\in\{8,16\}$, while for ml-100k and ml-1m dataset, we set $d1,d2\in\{16,32,64\}$. To evaluate the cross-architecture generalization of the proposed TD3, we employ GRU4Rec~\cite{GRU4Rec}, BERT4Rec~\cite{bert4rec}, and NARM~\cite{narm} for performance assessment. For all distilled datasets, we apply the Adam optimizer~\cite{diederik2014adam} for both the \emph{inner-loop} and \emph{outer-loop} optimization. 

In the \emph{outer-loop}, the synthetic sequence summary optimizer is configured with a learning rate $\alpha \in \{0.01, 0.03\}$ and a weight decay of 0.0001, using a cosine scheduler to adjust the learning rate throughout the process. In the \emph{inner-loop}, the learner optimizer employs a learning rate $\eta \in \{0.003, 0.005, 0.01\}$, with a weight decay of 0.00005. The inner steps are set to 200, using a random truncated window of 40 for backpropagation through time in RaT-BPTT, implemented via the Higher~\cite{higher} package across all datasets.


\newcommand{\boformat}[1]{$\mathbf{#1}$}
\newcommand{\blformat}[1]{\textcolor{blue}{$\mathbf{#1}$}}

\def\arraystretch{1.1}
\setlength{\tabcolsep}{0.5em} %

\begin{table}[t!] \centering
    \caption{Comparison of TD3 with existing dataset distillation techniques and heuristic sampling based methods.
    }
    \vspace{-8pt}
    \label{tab:baselines}
    \begin{center}
    \resizebox{0.48\textwidth}{!}{
    \begin{tabular}{cc|cc|cc|c}
        \toprule
        \multirow{2.8}{*}{Dataset} & \multirow{2.8}{*}{Metric} & \multicolumn{2}{c|}{Sampling} & \multicolumn{2}{c|}{Distillation} & \multirow{2.8}{*}{Full-Data} \\[3pt]
        
        \cmidrule{3-6}
        & & \multicolumn{1}{c}{Random.} & \multicolumn{1}{c|}{Longest.} & \multicolumn{1}{c}{Farzi} & \multicolumn{1}{c|}{\sampler} \\[2pt]
        
        \midrule
        \multirow{4}{*}{\STAB{Magazine\\\texttt{[30$\times$20]}}} 
        & HR@10 $\uparrow$     & 15.44 \std{$\pm$2.68} & 19.62 \std{$\pm$0.31} &      41.46 \std{$\pm$1.27} & \textbf{\ul{47.87} \std{$\pm$1.54}} & \textit{45.40} \\
        & HR@20 $\uparrow$     & 27.50 \std{$\pm$1.03} & 33.66 \std{$\pm$2.73} & \ul{58.70} \std{$\pm$0.13} & \textbf{\ul{61.17} \std{$\pm$1.83}} & \textit{56.98} \\
        & NDCG@10 $\uparrow$   &  7.75 \std{$\pm$1.16} &  9.58 \std{$\pm$0.58} &      24.27 \std{$\pm$0.20} & \textbf{\ul{27.90} \std{$\pm$0.88}} & \textit{25.63} \\
        & NDCG@20 $\uparrow$   & 10.75 \std{$\pm$0.77} & 13.10 \std{$\pm$0.30} & \ul{28.65} \std{$\pm$0.13} & \textbf{\ul{31.25} \std{$\pm$0.98}} & \textit{28.57} \\
        \midrule
        \multirow{4}{*}{\STAB{Epinions\\\texttt{[15$\times$30]}}} 
        & HR@10 $\uparrow$     & 10.67 \std{$\pm$0.90} & 10.24 \std{$\pm$0.21} &      18.99 \std{$\pm$0.30} & \textbf{\ul{19.86} \std{$\pm$0.10}} & \textit{19.13} \\
        & HR@20 $\uparrow$     & 20.25 \std{$\pm$1.25} & 20.01 \std{$\pm$0.41} &      29.82 \std{$\pm$0.39} & \textbf{\ul{31.06} \std{$\pm$0.19}} & \textit{30.09} \\
        & NDCG@10 $\uparrow$   &  4.93 \std{$\pm$0.36} &  4.79 \std{$\pm$0.06} & \ul{10.38} \std{$\pm$0.17} & \textbf{\ul{10.67} \std{$\pm$0.05}} & \textit{10.25} \\
        & NDCG@20 $\uparrow$   &  7.31 \std{$\pm$0.45} &  7.22 \std{$\pm$0.11} & \ul{13.09} \std{$\pm$0.07} & \textbf{\ul{13.49} \std{$\pm$0.08}} & \textit{13.00} \\
        \midrule
        \multirow{4}{*}{\STAB{ML-100k\\\texttt{[30$\times$50]}}} 
        & HR@10 $\uparrow$     & 10.85 \std{$\pm$2.30} & 13.36 \std{$\pm$0.45} & 62.92 \std{$\pm$1.39} & \textbf{66.14 \std{$\pm$1.16}} & \textit{68.93} \\
        & HR@20 $\uparrow$     & 21.49 \std{$\pm$3.98} & 25.59 \std{$\pm$0.64} & 77.84 \std{$\pm$0.30} & \textbf{81.37 \std{$\pm$0.43}} & \textit{83.78} \\
        & NDCG@10 $\uparrow$   &  4.81 \std{$\pm$1.10} &  5.97 \std{$\pm$0.32} & 34.92 \std{$\pm$0.71} & \textbf{38.56 \std{$\pm$0.86}} & \textit{40.97} \\
        & NDCG@20 $\uparrow$   &  7.48 \std{$\pm$1.28} &  9.02 \std{$\pm$0.36} & 38.72 \std{$\pm$0.40} & \textbf{42.44 \std{$\pm$0.72}} & \textit{44.76} \\
        \midrule
        \multirow{4}{*}{\STAB{ML-1M\\\texttt{[200$\times$50]}}} 
        & HR@10 $\uparrow$     & 15.88 \std{$\pm$0.22} & 16.60 \std{$\pm$0.51} & 38.01 \std{$\pm$0.98} & \textbf{70.52 \std{$\pm$0.36}} & \textit{79.32} \\
        & HR@20 $\uparrow$     & 28.09 \std{$\pm$0.31} & 30.93 \std{$\pm$0.45} & 56.10 \std{$\pm$1.24} & \textbf{82.52 \std{$\pm$0.25}} & \textit{87.60} \\
        & NDCG@10 $\uparrow$   &  7.40 \std{$\pm$0.11} &  7.77 \std{$\pm$0.16} & 19.85 \std{$\pm$0.49} & \textbf{45.93 \std{$\pm$0.37}} & \textit{58.82} \\
        & NDCG@20 $\uparrow$   & 10.47 \std{$\pm$0.10} & 11.36 \std{$\pm$0.14} & 24.41 \std{$\pm$0.55} & \textbf{48.97 \std{$\pm$0.30}} & \textit{60.93} \\
        \bottomrule
    \end{tabular}}
    \end{center}
\end{table}


\subsection{Overall Performance}
\vspace{1mm}

\begin{table*}[t]
\centering
\small
\begin{tabular}{l|c|ccccc|cc} 
\hline
\multirow{3}{*}{\textbf{Method}} & \multirow{3}{*}{\textbf{Parameters}} & \multicolumn{5}{c|}{\textbf{Question Answering}} & \multicolumn{2}{c}{\textbf{Fact Verification}} \\ \cline{3-9}
 & &  \textbf{TABMWP} & \textbf{WTQ} & \textbf{HiTab} & \textbf{TAT-QA} & \textbf{FeTaQA} & \textbf{TabFact} & \textbf{InfoTabs} \\ 
 & & (Acc.) & (Acc.) & (Acc.) & (Acc.) & (BLEU) & (Acc.) & (Acc.) \\ 
\hline
\multicolumn{9}{l}{{\cellcolor[rgb]{0.957,0.957,0.957}}\textit{LLM (Text)}} \\
Llama2 & 7B & 22.82 & 16.39 & 10.72 & 13.73 & 10.93 & 9.20 & 38.92 \\
TableLlama & 7B & 10.10 & 24.97 & 46.57 & 19.04 & 38.38 & 79.37 & 46.57 \\
Llama3-Instruct & 8B & 42.01 & 21.24 & 6.97 & 13.08 & 12.66 & 73.89 & 54.00 \\
\multicolumn{9}{l}{{\cellcolor[rgb]{0.957,0.957,0.957}}\textit{MLLM (Image)}} \\
MiniGPT-4 & 7B & 0.22 & 0.90 & 0.20 & 0.13 & 0.39 & 0 & 0.10 \\
Qwen-VL & 7B & 3.30 & 0.09 & 0.06 & 0.13 & 0.45 & 1.12 & 0.65 \\
InternLM-XComposer& 7B & 0.06 & 0.05 & 0.12 & 0.26 & 2.62 & 1.19 & 1.11 \\
mPLUG-Owl & 7B & 1.76 & 0.62 & 0.25 & 0.13 & 7.42 & 7.46 & 5.53 \\
mPLUG-Owl2 & 7B & 6.83 & 0.67 & 0.13 & 0.39 & 11.91 & 8.21 & 26.19 \\
LLaVA v1.5 & 7B & 6.05 & 1.24 & 2.03 & 2.97 & 8.24 & 18.9 & 28.31 \\
Vary-toy & 1.8B & 4.42 & 7.96 & 3.42 & 8.81 & 2.44 & 6.33 & 6.98 \\
Monkey & 7B & 13.26 & 19.07 & 6.41 & 12.31 & 3.41 & 22.56 & 22.11 \\
Table-LLaVA  & 7B & 57.78 & 18.43 & 10.09 & 12.82 & 25.60 & 59.85 & 65.26 \\
Table-LLaVA & 13B & 59.77 & 20.41 & 10.85 & 15.67 & 28.03 & 65.00 & 66.91 \\
MiniCPM-V-2.6 & 8B & 83.68  & 47.97 & 56.53 & 51.55 & 32.68 & 78.48 & 73.03 \\

\multicolumn{9}{l}{{\cellcolor[rgb]{0.957,0.957,0.957}}\textit{MLLM (Image \& Text)}} \\
Table-LLaVA & 13B & 84.58 & 39.89 & 46.00 & 29.27 & \textbf{33.50} & 69.93 & 74.88\\
MiniCPM-V-2.6 & 8B & \uline{86.06} & 52.30 & 58.56 & 52.46 & 32.96 & 79.31 & 73.18 \\
w/ Vanilla SFT & 8B & 76.69 & \uline{55.54} & \uline{62.88} & \uline{58.91} & 16.92 & \textbf{82.54} & \textbf{76.22} \\
w/ \method{}  & 8B & \textbf{87.50} & \textbf{55.77} & \textbf{63.00} & \textbf{60.75} & \uline{33.18} & \uline{82.27} & \uline{75.74} \\
\hline
\end{tabular}
\caption{Overall Performance on TQA and TFV Tasks. The \textbf{best} results are marked in bold, while the \uline{second-best} results are underlined. We establish baselines using LLM (Text) and MLLM (Image) by feeding unimodal table representations to language models. Next, we use image-based and text-based table representations as inputs to train various MLLM (Image \& Text) models, demonstrating the effectiveness of our \method{}.} \label{tab:overall}
\end{table*}

We evaluated TD3's performance across various synthetic sequence summary sizes and diverse datasets. In \cref{tab:overall_results}, we compare models trained on full original datasets with those using various-sized synthetic sequence summaries. Additionally, \cref{tab:baselines} and \cref{fig:baseline_results} compare TD3's performance with Farzi and heuristic sampling methods: \emph{random sampling}, which selects sequences uniformly, and \emph{longest sampling}, which selects sequences in descending order of length. Our findings are as follows: 1) TD3 achieves comparable training performance, even with substantial data compression. This shows that small-batch synthetic summaries distilled from the original dataset effectively capture essential information, preserving data integrity for model training. 2) In datasets such as Magazine and Epinions, models trained on TD3-distilled summaries outperform those trained on the original datasets. This highlights the value of high-quality, smaller datasets over larger, noisier ones, underscoring the importance of data quality in model training. 3) As illustrated in \cref{tab:baselines} and \cref{fig:baseline_results}, TD3 is more sample-efficient than Farzi and heuristic methods, showing superior data utilization. 

These results illustrate the transformative potential of data distillation in improving sequential recommendation systems. This approach represents a shift towards a data-centric paradigm in recommender systems, where prioritizing data quantity and quality and strategic compression can create more robust and efficient algorithms, reducing computational costs and storage demands. This evolution paves the way for the next generation of recommendation algorithms, focusing on maximizing value from the minimal data.


\section{Computational complexity}\label{sec:compl}

In this section, we show that both of our problems are \np-hard.
We saw in the previous section that if
we set $\alpha  = \infty$, then \problemcdcsm reduces to \problemdts, which can be solved exactly in
polynomial time. However, \problemcdcsm is \np-hard when $\alpha = 0$.
%However, we show next that the complexity of fair densestsubgraph $\problemcdcsm$ which allows $0 < \alpha \leq 1$ is $\np$-hard.


\begin{proposition} 
\label{prop:np}
\problemcdcsm is \np-hard.
\end{proposition} 
\begin{proof}
We prove the hardness from $k$-\prbclique, a problem where, given a graph $H$, we are asked if there is a clique of size at least $k$.

Assume that we are given a graph $H = (V, E)$ with $n$ nodes, $n \geq k$. We set $\alpha = 0$.
The graph snapshot $G_1$ consists of the
graph $H$ and an additional set of $k$ singleton vertices $U$. $G_2$ consists of a $k$-clique connecting the vertices in $U$. 

We claim that there is a subset $S$ yielding $\dens{S, \calG} =  (k - 1)/2$ if and only if there is an $k$-clique in $H$.

Assume that  there is a subset $S$ yielding $\dens{S, \calG} =  (k - 1)/2$.
Since the value of objective is $(k - 1)/2$, we have $\dens{S, G_1} = \dens{S, G_2} = (k - 1)/4$. 
Let $S = W \cup T$ where $W \subseteq V$ and denotes the subset of vertices from $H$ and $T \subseteq U$ is the subset of vertices from  $U$ in $S$. 

Assume that $\abs{W} < \abs{T}$. Since $\abs{T} \leq k$, $\abs{W} < k$.  The density induced on $G_1$ is bounded by $\dens{S, G_1} \leq \frac{{\abs{W} \choose 2}}{\abs{T} + \abs{W}} < \frac{{\abs{W} \choose 2}}{2\abs{W}} < \frac{k - 1}{4}$, which is a contradiction.
Assume that  $\abs{W} > \abs{T}$. Then the density induced on $G_2$ is bounded by $\dens{S, G_2} = \frac{{\abs{T} \choose 2}}{\abs{T} + \abs{W}} < \frac{{\abs{T} \choose 2}}{2\abs{T}} = \frac{\abs{T} - 1}{4} \leq \frac{k - 1}{4}$, again a contradiction. Therefore, $\abs{W} =  \abs{T}$. 

Consequently, $\dens{S, G_2} = (\abs{T} - 1)/4$, implying that $\abs{T} = k$. Finally, $\frac{k - 1}{4} = \dens{S, G_1} = \frac{\abs{E(S)}}{2k}$ implies that $\abs{E} = {k \choose 2}$, that is, $W$ is a $k$-clique in $H$.

%Let  $d_1$ and $d_2$ are the densities induced on $G_1$ and $G_2$ respectively by $S$.
%Based on our assumption, $\dens{S, \calG} =  (k - 1)/2$ yields and 
%the only way to obtain a density value  of $f(S, G_2) = (k - 1)/4$  which satisfies marginal density constraint is that  $S = W \cup U$ and thus there should be an $k$-clique in $H$.
%Therefore, if the density is $  (k - 1)/4$ there is an $k$-clique in $H$.


On the other hand, assume there is a clique $C$ of size $k$ in $H$.
%Let clique $C$ is formed by the set of vertices in $W$.
% Let $d_1$ and $d_2$ are the optimal densities induced on $G_1$ and $G_2$.
% Since the density constraint should be satisfied $d_1 \geq (d_1 + d_2)/2$ and $d_2 \geq (d_1 + d_2)/2$ which implies $d_1 \geq d_2$ and $d_2 \geq d_1$. Therefore, $d_1 = d_2$.  
% Let $k_1$ and $k_2$ number of non-singleton nodes contribute for $d_1$ and $d_2$ respectively.
% Since $d_1 = d_2$,  $\frac{ k_1(k_1 - 1)}{ 2(k_1 + k_2)} = \frac{k_2 (k_2 - 1)}{ 2(k_1 + k_2)}$ which implies $k_1 =  k_2$. Then  the sum of the densities is given by $\frac{ (k_1 - 1)}{ 2} $ where the optimal is  when $k_1  = k =  k_2$.
Set $S = C \cup U$. 
Immediately, $\dens{S, \calG} =   (k - 1)/2$ proving the claim.
\qed
\end{proof}

\iffalse
\begin{proposition}
\label{prop:inapproximability}
\problemcdcsm does not have any polynomial time approximation algorithm with an approximation ratio better than $n^{1-\epsilon}$ for any constant $\epsilon > 0$, unless $\np = \zpp$.
\end{proposition}
\fi

A similar proof will show that \problemcdcsdiff is \np-hard, and inapproximable.

\begin{proposition} 
\label{prop:np2}
\problemcdcsdiff is \np-hard.
Unless $\poly = \np$, there is no polynomial-time algorithm with multiplicative approximation guarantee for \problemcdcsdiff. 
\end{proposition} 

\begin{proof}
We use the same reduction from $k$-\prbclique as in the proof of Proposition~\ref{prop:np}.
We also set $\sigma = (k - 1)/2$. If there is a clique $C$ in $H$, then selecting $S = C \cup U$ yields $\diff{S, \calG} = 0$.
On the other hand, if $\diff{S, \calG} = 0$, then $\dens{S, G_1} = \dens{S, G_2} = (k - 1)/4$, and the argument in the proof of Proposition~\ref{prop:np} shows that there must be a $k$-clique in $H$.
In summary, the difference $\diff{S, \calG} = 0$ for a solution $S$ if and only if there is a $k$-clique in $H$.

This also immediately implies that there is no polynomial-time algorithm with multiplicative approximation guarantee since this algorithm can be then used to test whether there is a set $S$ with $\diff{S, \calG} = 0$.
\qed
\end{proof}\vspace{-3mm}

\subsection{Time and Memory Analysis}

\vspace{1mm}

\subsubsection{Theoretical Memory Complexity}

As discussed in Section \ref{sec:related_work}, Farzi \cite{sachdeva2023farzi} decomposes synthetic data $\syn \in \mathbb{R}^{\mu \times \zeta \times |\vocab|}$ to a latent summary $\mathbf{D} \in \mathbb{R}^{\mu \times \zeta \times d}$ and a decoder $\mathbf{M} \in \mathbb{R}^{d \times |\vocab|}$, where $d \ll |\vocab|$. To highlight the advantages of employing Tucker decomposition for three-dimensional data, we perform a comparative analysis of our proposed approach with that of Farzi. In particular, we investigate the computational footprint associated with the bi-level optimization framework by evaluating memory usage during a single outer-loop step, providing insights into the efficiency gains achieved through our method as follows:
\begin{align*}
    \operatorname{Farzi~:} & 
    ~ \mathcal{O}\big( |\Phi| + |\mathcal{B}^{\real}| \cdot \zeta \cdot d_3 + |\mathcal{B}^{\syn}| \cdot \zeta \cdot |\vocab| + \mu \cdot \zeta \cdot d + d \cdot |\vocab| \big) \\
    \operatorname{TD3~:} & 
    ~ \mathcal{O}\big( |\Phi| + |\mathcal{B}^{\real}| \cdot \zeta \cdot d_3 + |\mathcal{B}^{\syn}| \cdot \zeta \cdot |\vocab| + (\mu + \zeta) \cdot d_1 + d_1^2 \cdot d_3 \big) 
\end{align*}
where $|\mathcal{B}^{\real}|$ and $|\mathcal{B}^{\syn}|$ denote the batch size of real and synthetic data, respectively, while $d_3$ represents the item embeddings' hidden dimension. Additionally, $d$, $d_1$, and $d_3$ are of the same order of magnitude and much smaller than $|\vocab|$. When the item set is very large, or the distilled sequence summary requires larger values of $\mu$ and $\zeta$, the inequality $(\mu + \zeta) \cdot d_1 + d_1^2 \cdot d_3 \ll \mu \cdot \zeta \cdot d + d \cdot |\vocab|$ holds, the method proposed in our work will offer a spatial advantage. 

\subsubsection{Empirical Computational Complexity}

The data distillation process consumes considerable wall-clock time and GPU memory, so we conducted a detailed quantitative analysis of these requirements for both distillation and model training on the original and distilled datasets, as shown in \cref{tab:time_memory}. Wall-clock time is reported in single A100 (80GB) GPU hours. While distillation generally takes longer and uses more memory than training on the original data, its cost is often amortizable in real-world scenarios where multiple models must be trained on the same dataset. 
The amortization, based on the ratios in the table, varies by dataset and distillation scale. Notably, for large datasets, where training time is typically lengthy, training on significantly reduced distilled data can shorten this process by several orders of magnitude. This trade-off substantially decreases future training time, making distillation a one-time cost that yields long-term benefits for various downstream tasks, such as hyperparameter tuning and architecture exploration. Hence, the distillation process and its amortization will be well justified.

\begin{figure}[htbp]
    \centering
    \begin{subfigure}{0.48\textwidth}
        \centering
        \includegraphics[width=\linewidth]{figure/magazine.pdf}
    \end{subfigure}
    \hfill
    \vspace{1mm}
    \begin{subfigure}{0.48\textwidth}
        \centering
        \includegraphics[width=\linewidth]{figure/ml100k.pdf}
    \end{subfigure}
    \vspace{0.8mm}
    \caption{Illustration of the performance comparison of the TD3 against: \emph{Farzi}, \emph{random sampling} and \emph{longest sampling}.}
    \label{fig:baseline_results}
\end{figure}
\vspace{-1cm}


\subsection{Cross-Architecture Generalization}

\vspace{1mm}

Since the synthetic sequence summary is carefully tailored for optimizing a specific learner model, we assess its generalizability across various unseen architectures, as shown in \cref{tab:cross-arch}. We first distill the Epinions dataset using SASRec~\cite{SASRec}, resulting in a condensed synthetic summary of size $[50 \times 20]$. This summary is then used to train several alternative architectures, including GRU4Rec~\cite{GRU4Rec}, which models sequential behavior using Gated Recurrent Units (GRU); NARM~\cite{li2017neural}, which enhances GRU4Rec with an attention mechanism to emphasize user intent; and BERT4Rec~\cite{bert4rec}, which uses bidirectional self-attention to learn sequence representations. The models trained on the synthetic summary demonstrate strong generalization across diverse architectures, maintaining high predictive performance. In some cases, they even outperform models trained on the original dataset, highlighting the effectiveness of our proposed TD3 method in enabling cross-architecture transferability.

\def\arraystretch{1.2}
\setlength{\tabcolsep}{0.5em} %

\begin{table}[t!] \centering
    \caption{
    Evaluation of generalization performance on unseen architectures using a synthetic summary of size $[50 \times 20]$ distilled from the Epinions dataset via SASRec. 
    }
    \vspace{-8pt}
    \label{tab:cross-arch}
    \begin{center}
    \resizebox{0.98\linewidth}{!}{
        \begin{tabular}{ c | c c c c }
            \toprule
            \multirow{4}{*}{\textbf{Metric}} & \multicolumn{4}{c}{\textbf{Architecture}} \\
            & \multicolumn{4}{c}{{Synthetic Data / Original Data}} \\
            & SASRec & NARM & GRU4Rec & BERT4Rec \\
            \midrule
            HR@10   & \underline{19.75}~/~19.61 & \underline{19.60}~/~19.43 & 19.11~/~20.25 & \underline{18.98}~/~17.14 \\
            HR@20   & \underline{31.30}~/~30.90 & 30.55~/~31.10 & 29.49~/~31.49 & \underline{29.08}~/~27.62 \\
            NDCG@10 & \underline{10.61}~/~10.45 & \underline{10.67}~/~10.25 & 10.55~/~10.93 & \underline{10.31}~/~9.29 \\
            NDCG@20 & \underline{13.51}~/~13.28 & \underline{13.43}~/~13.18 & 13.15~/~13.75 & \underline{12.83}~/~11.91 \\
            MRR@10  & \underline{7.85}~/~7.69   & \underline{7.98}~/~7.48   & 7.96~/~8.13   & \underline{7.70}~/~6.93 \\
            MRR@20  & \underline{8.64}~/~8.46   & \underline{8.73}~/~8.28   & 8.66~/~8.90   & \underline{8.38}~/~7.63 \\
            \bottomrule
        \end{tabular}
    }
    \end{center}
\end{table}






\subsection{Ablation Studies}

To analyze our method's components, we conducted ablation studies on \emph{ML-100k}. Results for \emph{Feature Space Alignment} (FSA) and \emph{Augmented Learner Training} (ALT) are in \cref{tab:ablation_results}. FSA and ALT complement each other. ALT significantly boosts TD3's performance by enhancing contextual understanding and reducing dependence on specific sequence patterns, improving sequence information capture and data updates in the \emph{outer-loop}. FSA alone also enhances performance across metrics by strengthening the objective function, aiding convergence to a similar solution in the feature space, and maintaining performance on original and synthetic data. Utilizing both in the \emph{inner-loop} and \emph{outer-loop} maximizes distillation benefits.

% To analyze our method's components, we performed ablation studies on \emph{ML-100k}. Results for \emph{Feature Space Alignment} (FSA) and \emph{Augmented Learner Training} (ALT) are shown in \cref{tab:ablation_results}. FSA and ALT complement each other, with ALT significantly enhancing TD3's performance by improving contextual understanding and sequence information capture. FSA alone boosts performance by strengthening the objective function and aiding convergence. Using both in the \emph{inner-loop} and \emph{outer-loop} maximizes distillation benefits.

% \begin{table}[!t]
% \centering
% \scalebox{0.68}{
%     \begin{tabular}{ll cccc}
%       \toprule
%       & \multicolumn{4}{c}{\textbf{Intellipro Dataset}}\\
%       & \multicolumn{2}{c}{Rank Resume} & \multicolumn{2}{c}{Rank Job} \\
%       \cmidrule(lr){2-3} \cmidrule(lr){4-5} 
%       \textbf{Method}
%       &  Recall@100 & nDCG@100 & Recall@10 & nDCG@10 \\
%       \midrule
%       \confitold{}
%       & 71.28 &34.79 &76.50 &52.57 
%       \\
%       \cmidrule{2-5}
%       \confitsimple{}
%     & 82.53 &48.17
%        & 85.58 &64.91
     
%        \\
%        +\RunnerUpMiningShort{}
%     &85.43 &50.99 &91.38 &71.34 
%       \\
%       +\HyReShort
%         &- & -
%        &-&-\\
       
%       \bottomrule

%     \end{tabular}
%   }
% \caption{Ablation studies using Jina-v2-base as the encoder. ``\confitsimple{}'' refers using a simplified encoder architecture. \framework{} trains \confitsimple{} with \RunnerUpMiningShort{} and \HyReShort{}.}
% \label{tbl:ablation}
% \end{table}
\begin{table*}[!t]
\centering
\scalebox{0.75}{
    \begin{tabular}{l cccc cccc}
      \toprule
      & \multicolumn{4}{c}{\textbf{Recruiting Dataset}}
      & \multicolumn{4}{c}{\textbf{AliYun Dataset}}\\
      & \multicolumn{2}{c}{Rank Resume} & \multicolumn{2}{c}{Rank Job} 
      & \multicolumn{2}{c}{Rank Resume} & \multicolumn{2}{c}{Rank Job}\\
      \cmidrule(lr){2-3} \cmidrule(lr){4-5} 
      \cmidrule(lr){6-7} \cmidrule(lr){8-9} 
      \textbf{Method}
      & Recall@100 & nDCG@100 & Recall@10 & nDCG@10
      & Recall@100 & nDCG@100 & Recall@10 & nDCG@10\\
      \midrule
      \confitold{}
      & 71.28 & 34.79 & 76.50 & 52.57 
      & 87.81 & 65.06 & 72.39 & 56.12
      \\
      \cmidrule{2-9}
      \confitsimple{}
      & 82.53 & 48.17 & 85.58 & 64.91
      & 94.90&78.40 & 78.70& 65.45
       \\
      +\HyReShort{}
       &85.28 & 49.50
       &90.25 & 70.22
       & 96.62&81.99 & \textbf{81.16}& 67.63
       \\
      +\RunnerUpMiningShort{}
       % & 85.14& 49.82
       % &90.75&72.51
       & \textbf{86.13}&\textbf{51.90} & \textbf{94.25}&\textbf{73.32}
       & \textbf{97.07}&\textbf{83.11} & 80.49& \textbf{68.02}
       \\
   %     +\RunnerUpMiningShort{}
   %    & 85.43 & 50.99 & 91.38 & 71.34 
   %    & 96.24 & 82.95 & 80.12 & 66.96
   %    \\
   %    +\HyReShort{} old
   %     &85.28 & 49.50
   %     &90.25 & 70.22
   %     & 96.62&81.99 & 81.16& 67.63
   %     \\
   % +\HyReShort{} 
   %     % & 85.14& 49.82
   %     % &90.75&72.51
   %     & 86.83&51.77 &92.00 &72.04
   %     & 97.07&83.11 & 80.49& 68.02
   %     \\
      \bottomrule

    \end{tabular}
  }
\caption{\framework{} ablation studies. ``\confitsimple{}'' refers using a simplified encoder architecture. \framework{} trains \confitsimple{} with \RunnerUpMiningShort{} and \HyReShort{}. We use Jina-v2-base as the encoder due to its better performance.
}
\label{tbl:ablation}
\end{table*}
