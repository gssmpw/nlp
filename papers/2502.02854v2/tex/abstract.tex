\begin{abstract}




In the era of data-centric AI, the focus of recommender systems has shifted from model-centric innovations to data-centric approaches. The success of modern AI models is built on large-scale datasets, but this also results in significant training costs. Dataset distillation has emerged as a key solution, condensing large datasets to accelerate model training while preserving model performance. However, condensing discrete and sequentially correlated user-item interactions, particularly with extensive item sets, presents considerable challenges. This paper introduces \textbf{TD3}, a novel \textbf{T}ucker \textbf{D}ecomposition based \textbf{D}ataset \textbf{D}istillation method within a meta-learning framework, designed for sequential recommendation. TD3 distills a fully expressive \emph{synthetic sequence summary} from original data. To efficiently reduce computational complexity and extract refined latent patterns, Tucker decomposition decouples the summary into four factors: \emph{synthetic user latent factor}, \emph{temporal dynamics latent factor}, \emph{shared item latent factor}, and a \emph{relation core} that models their interconnections. Additionally, a surrogate objective in bi-level optimization is proposed to align feature spaces extracted from models trained on both original data and synthetic sequence summary beyond the na\"ive performance matching approach. In the \emph{inner-loop}, an augmentation technique allows the learner to closely fit the synthetic summary, ensuring an accurate update of it in the \emph{outer-loop}. To accelerate the optimization process and address long dependencies, RaT-BPTT is employed for bi-level optimization. Experiments and analyses on multiple public datasets have confirmed the superiority and cross-architecture generalizability of the proposed designs. Codes are released at \textcolor{blue}{\url{https://github.com/USTC-StarTeam/TD3}}.

\begin{figure}[t!] \centering
    \centering
    \vspace{-4pt}
\end{figure}

\end{abstract}
