\section{Introduction}

Pan-Tilt-Zoom (PTZ) cameras are widely used in video broadcasting~\cite{Liu:2023:LAC,homayounfar2017sports} and security surveillance~\cite{Wu:2013:KPTZ, Eldrandaly:2019:Coverage}. For applications, like panorama creation, localization or digital twins ~\cite{lalonde2007system,rameau2013self,wu2012keeping,Song:2006:Traffic}, accurately estimating their orientation and intrinsic parameters is highly desirable.
However, the camera parameters provided by the hardware contain multiple errors~\cite{Wu:2013:KPTZ}, which will impact the effectiveness of their applications. Therefore, it is necessary to propose an efficient and accurate calibration method for PTZ cameras.

Two types of PTZ camera calibration tasks are popular. One category focuses on sports field registration, which is usually equivalent to the homography estimation problem, achieved by local feature matching~\cite{homayounfar2017sports,Gupta:2011:ULE,Nie2021ARA,Sha:2020:EtE} or learning-based methods~\cite{jiang2020optimizing,SFLNet}.
While these methods are effective in specific applications, their assumption of planarity and known fixed structure restricts the use in complex 3D scenes. Another category addresses the more general task of PTZ camera calibration. To achieve efficient and precise estimation results, these methods often incorporate numerous assumptions to simplify the problem. For example, the camera has a known initial position~\cite{PTZ-SLAM}, rotates only~\cite{selfcalib1997}, lacks distortion~\cite{chen2018two}, or assumes simple illumination and environmental changes~\cite{Sinha:2006:Mosaic}. 

\begin{figure}[!t]
    \centering
    \includegraphics[width=\linewidth]{figures/teaser_v1.pdf}
    \caption{\textbf{Overview of PTZ-Calib}. Given a set of camera images(a), our method can automatically calibrate them in the local coordinate system(b) and further align them geographically using global 3D references(c).}
    \label{fig:teaser}
\end{figure}

To address these challenges, our proposed method robustly estimates camera parameters, such as camera pose, focal length, and distortion coefficients, for any viewpoint via a two-stage process. In the offline stage, we select uncalibrated reference frames and track 2D feature points across them using feature matching algorithm. We then apply our novel PTZ-IBA (PTZ Incremental Bundle Adjustment) algorithm for automatic calibration in local coordinate system. To further optimize camera parameters, we incorporate global geographic reference 3D data, aligning them with the geographic coordinate system.
In the online stage, we calibrate camera parameters for new viewpoints using feature matching and optimization techniques.
This method effectively balances accuracy and computational efficiency.

Extensive evaluations validate the efficacy of our method. In sports field registration, our algorithm outperforms leading open-sourced state-of-the-art methods and achieves the best results. We also build a set of 3D synthetic scenes to evaluate the calibration accuracy. Our method consistently delivers superior performance both quantitatively and qualitatively. Moreover, we demonstrate the application of our algorithm in real-world settings, illustrating the broad applicability in complex scenarios. We summarize our contributions as follows. 

\begin{itemize}
    \item We propose a robust two-stage PTZ camera calibration method \emph{PTZ-Calib} to provide efficient and accurate camera parameters for arbitrary viewpoints.
    \item We introduce the PTZ-IBA algorithm, which automatically calibrates camera views within a local coordinate system. Besides, we can further align parameters to real-world global coordinates using extra 3D information.
    \item Extensive evaluations demonstrate that our method outperforms current SOTA methods across diverse real and synthetic datasets, as well as various applications.
\end{itemize}
