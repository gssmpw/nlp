\section{Related Work}\label{sec:related}

\textbf{Sports Field Registration}
involves estimating correspondences (such as lines and circles) between a sports field and images captured by a fixed, rotatable camera, traditionally treated as a homography estimation problem. Early methods identified correspondences and obtained homography using Direct Linear Transform (DLT) or optimization-based method~\cite{homayounfar2017sports, Gupta:2011:ULE,Nie2021ARA}. Recent methods have shifted towards semantic segmentation to learn the sports field's representation, directly predicting or regressing an initial homograph matrix~\cite{jiang2020optimizing}, or searching for the optimal homography in a database of synthetic images with predefined matrices or known camera parameters~\cite{Sha:2020:EtE,Chen2019SportsCC}. End-to-end approaches~\cite{SFLNet} also obtained promising results.
Recently,~\cite{Theiner:2023:TVCalib,Marc:2024:NBJW} consider the task as a camera calibration problem, leveraging the segment correspondences to predict the camera pose and focal length. 
Despite achieving notable successes, the limitation lies in their inability to be readily adapted to alternative application scenarios due to the scarcity of training data and complex 3D scenes.

\textbf{PTZ Camera Calibration }
aims to estimate full camera parameters, including intrinsic and extrinsic parameters.
Earlier approaches often rely on various assumptions to simplify the problem. \cite{Wu:2013:KPTZ} develops a comprehensive model that includes a dynamic correction process to maintain accurate calibration over time. However, this model assumes a consistent relationship between zoom scale and focal length.
Other methods assume that cameras start with known initial camera poses~\cite{PTZ-SLAM}, rotate only~\cite{selfcalib1997} or have no lens distortions~\cite{chen2018two}, which is impractical in real-world applications. \cite{Liu:2023:LAC} presents a novel linear auto-calibration method for bullet-type PTZ cameras only and cannot directly apply to other cameras without modifications. 

In comparison, our method relies on minimal assumptions while fully calibrating the camera parameters. Moreover, with additional 2D-3D annotations, we can further improve the camera parameters and obtain absolute poses in the geographic coordinate system. By employing a two-stage design, our method effectively balances computational efficiency with accuracy, making it highly practical for various applications.