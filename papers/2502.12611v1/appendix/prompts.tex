% \clearpage
\section{Prompt}
\label{sec:prompt-content}
In this section, we present two types of prompts used in our study. The \textbf{System Prompt} is generated based on each learner’s attributes, including sex, academic background, English proficiency level, and country of origin, to construct a more targeted background persona. The \textbf{Task Prompt} provides the specific writing instructions, topic, length requirements, format, and spell-check reminders for the learners.


\subsection{System Prompt}
\label{sec:system_prompt}

We construct the \textbf{System Prompt} by concatenating four short statements reflecting the following dimensions in order: ``Sex,'' ``Acad.~Genre,'' ``CEFR,'' and ``Country.'' Each dimension--value pair maps to a predefined prompt string. For instance, if ``Sex'' = Male, then the prompt part is ``You are a male author.'' The final prompt strings for the other attributes are similarly selected from the respective lists. 

\vspace{1em}
\noindent
\begin{showcase}[title=Example]


|\classbg{Sex = Male}|
--> "You are a male author."

|\classbg{Acad. Genre = Science \& Technology}|
--> "Your academic background is in Science \& Technology."

|\classbg{CEFR = A2\_0}| 
--> "Your English proficiency level is CEFR A2."

|\classbg{Country = CHN}| 
--> "You are from China, an EFL environment."


\end{showcase}


When concatenated, these form the complete system prompt:

\begin{prompt}[title={Prompt \thetcbcounter: Male, Sci\_Tech, A2\_0, CHN}]
You are a male author. 
Your academic background is in Science \& Technology. 
Your English proficiency level is CEFR A2. 
You are from China, an EFL environment (English as a Foreign Language). 
\end{prompt}

\noindent
The JSON representation below details all possible dimension--value pairs and their corresponding prompt strings, which we combine to produce the final System Prompt.



\begin{showcase}[title=Prompt: Sex]


|\classbg{Sex = Male}|
You are a male author.

|\classbg{Sex = Female}|
You are a female author.

\end{showcase}


\begin{showcase}[title=Prompt: Acad. Genre]


|\classbg{Acad. Genre = Humanities}|
Your academic background is in the Humanities.

|\classbg{Acad. Genre = Social Sciences}|
Your academic background is in the Social Sciences.

|\classbg{Acad. Genre = Science \& Technology}|
Your academic background is in Science & Technology.

|\classbg{Acad. Genre = Life Science}|
Your academic background is in Life Science.

\end{showcase}



\begin{showcase}[title=Prompt: CEFR]


|\classbg{CEFR = A2\_0}|
Your English proficiency level is CEFR A2.

|\classbg{CEFR = B1\_1}|
Your English proficiency level is CEFR B1 (lower).

|\classbg{CEFR = B1\_2}|
Your English proficiency level is CEFR B1 (upper).

|\classbg{CEFR = B2\_0}|
Your English proficiency level is CEFR B2+.

|\classbg{CEFR = XX\_0}|
You are a native English speaker.

\end{showcase}


\begin{showcase}[title=Prompt: Country]


|\classbg{Country = CHN}|
You are from China, an EFL environment (English as a Foreign Language).

|\classbg{Country = THA}|
You are from Thailand, an EFL environment (English as a Foreign Language).

|\classbg{Country = JPN}|
You are from Japan, an EFL environment (English as a Foreign Language).

|\classbg{Country = KOR}|
You are from Korea, an EFL environment (English as a Foreign Language).

|\classbg{Country = IDN}|
You are from Indonesia, an EFL environment (English as a Foreign Language).

|\classbg{Country = PHL}|
You are from the Philippines, an ESL environment (English as a Second Language).

|\classbg{Country = PAK}|
You are from Pakistan, an ESL environment (English as a Second Language).

|\classbg{Country = SIN}|
You are from Singapore, an ESL environment (English as a Second Language).

|\classbg{Region = TWN}|
You are from Taiwan, an EFL environment (English as a Foreign Language).

|\classbg{Country = ENS\_USA}|
You are from the United States, a native English-speaking (NS) environment.

|\classbg{Country = HKG}|
You are from Hong Kong, an ESL environment (English as a Second Language).

|\classbg{Country = ENS\_GBR}|
You are from the United Kingdom, a native English-speaking (NS) environment.

|\classbg{Country = ENS\_CAN}|
You are from Canada, a native English-speaking (NS) environment.

|\classbg{Country = ENS\_AUS}|
You are from Australia, a native English-speaking (NS) environment.

|\classbg{Country = ENS\_NZL}|
You are from New Zealand, a native English-speaking (NS) environment.

\end{showcase}

\subsection{Prompt Illustration}
\label{sec:appendix_prompt}

For clarity, the template for constructing the \textit{System Prompt} is:
\begin{prompt}[title={Prompt Template}]
\{\texttt{sex\_prompt}\} 
\{\texttt{acad\_genre\_prompt}\} 
\{\texttt{cefr\_prompt}\} 
\{\texttt{country\_prompt}\}
\end{prompt}


Each placeholder (e.g., \{\texttt{sex\_prompt}\}) is replaced by the corresponding string from the JSON snippets in Section~\ref{sec:system_prompt}. Below is a concrete illustration using one combination of attributes:

\begin{prompt}[title={Sample Prompt: F, Humanities, B1\_1, SIN}]
You are a female author. 
Your academic background is in the Humanities. 
Your English proficiency level is CEFR B1 (lower). 
You are from Singapore, an ESL environment (English as a Second Language). 
\end{prompt}

Thus, the System Prompt succinctly captures user-specific information to guide the subsequent text generation process.


\subsection{Task Prompt}
\label{appendix:task_prompt}
We use two topics for the writing tasks, each with the following requirements. Learners must use a word processor, keep the word count between 200 and 300 words, and run spell check before finalizing.

\begin{prompt}[title={Prompt \thetcbcounter: PTJ task prompt}]
Do you agree or disagree with the following statements? Use reasons and specific details to support your opinion. \\
(Topic) It is important for college students to have a part-time job. \\ \\
Instructions \\
1. Clarify your opinions and show the reasons and some examples. \\
2. The length of your single essay should be from 200 to 300 WORDS (not letters). Too short or too long essays cannot be accepted. \\
3. You must run spell check before completing your writing.\\
\end{prompt}    


\begin{prompt}[title={Prompt \thetcbcounter: SMK task prompt}]
Do you agree or disagree with the following statements? Use reasons and specific details to support your opinion. \\
(Topic) Smoking should be completely banned at all the restaurants in the country.\\ \\
Instructions \\
1. Clarify your opinions and show the reasons and some examples.\\
2. The length of your single essay should be from 200 to 300 WORDS (not letters). Too short or too long essays cannot be accepted. \\
3. You must run spell check before completing your writing. \\
\end{prompt}    