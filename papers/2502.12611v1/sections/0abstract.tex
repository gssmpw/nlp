\begin{abstract}
The rise of Large Language Models (LLMs) necessitates accurate AI-generated text detection. However, current approaches largely overlook the influence of author characteristics. We investigate how sociolinguistic attributes—gender, CEFR proficiency, academic field, and language environment—impact state-of-the-art AI text detectors. Using the ICNALE corpus of human-authored texts and parallel AI-generated texts from diverse LLMs, we conduct a rigorous evaluation employing multi-factor ANOVA and weighted least squares (WLS). Our results reveal significant biases: CEFR proficiency and language environment consistently affected detector accuracy, while gender and academic field showed detector-dependent effects. These findings highlight the crucial need for socially aware AI text detection to avoid unfairly penalizing specific demographic groups. We offer novel empirical evidence, a robust statistical framework, and actionable insights for developing more equitable and reliable detection systems in real-world, out-of-domain contexts. This work paves the way for future research on bias mitigation, inclusive evaluation benchmarks, and socially responsible LLM detectors.
\end{abstract}