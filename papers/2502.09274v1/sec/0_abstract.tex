\begin{abstract}
3D scene understanding is a critical yet challenging task in autonomous driving, primarily due to the irregularity and sparsity of LiDAR data, as well as the computational demands of processing large-scale point clouds. Recent methods leverage the range-view representation to improve processing efficiency. To mitigate the performance drop caused by information loss inherent to the "many-to-one" problem, where multiple nearby 3D points are mapped to the same 2D grids and only the closest is retained, prior works tend to choose a higher azimuth resolution for range-view projection. However, this can bring the drawback of reducing the proportion of pixels that carry information and heavier computation within the network. We argue that it is not the optimal solution and show that, in contrast, decreasing the resolution is more advantageous in both efficiency and accuracy. In this work, we present a comprehensive re-design of the workflow for range-view-based LiDAR semantic segmentation. Our approach addresses data representation, augmentation, and post-processing methods for improvements. Through extensive experiments on two public datasets, we demonstrate that our pipeline significantly enhances the performance of various network architectures over their baselines, paving the way for more effective LiDAR-based perception in autonomous systems. The code will be released based on the acceptance.
\end{abstract}
