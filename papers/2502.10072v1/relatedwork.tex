\section{Literature Review}
Regulating traffic flow without hindering it, and managing the logistics chain without disrupting its efficiency, are two perennial challenges that successive governments have endeavoured to address. These issues become particularly significant when discussing legislation aimed at controlling the load on commercial vehicles. Two key socioeconomic questions arise from such regulatory efforts:

\vspace{0.5\baselineskip}

1. Impact on Traffic Flow Due to Load Estimation: A primary concern is whether the process of load estimation will lead to significant traffic delays. Requiring truck operators to divert from established routes to obtain clearance at designated locations could result in substantial disruptions. This scenario is especially critical for perishable goods, where time is of the essence. The economic implications of delays further exacerbate this issue, as the rising costs of fuel and the need for timely deliveries are paramount for logistics firms. Moreover, imposing additional steps for load verification could lead to inefficiencies that ripple through the supply chain, ultimately affecting consumers. Therefore, it is essential that any load verification processes are integrated into the existing traffic flow to avoid unnecessary disruptions, particularly as logistical costs continue to escalate. 

2. Potential Backlash from the Trucking Industry: 
Increased regulation may provoke backlash from the trucking sector, which is predominantly composed of blue-collar workers who often feel disenfranchised by governmental oversight. This concern is underscored by the military adage attributed to General Omar Bradley: “Amateurs talk strategy; professionals talk logistics” [11]. The United States' global dominance is, in part, a function of its exceptional logistical capabilities, which enable military force projection worldwide within 24 hours. Similarly, the U.S. industrial sector benefits from a robust logistics framework, reliant on the contributions of truckers who facilitate the movement of goods. India's trucking industry, by comparison, is comprised of a rather obsolete, outdated vehicles and little-to-no consolidation. While nationwide delivery services like Delhivery or DTDC operate with astounding efficiency, it is exponentially more difficult to transport tonnes of fruit than a single package. Transportation in India is generally run by localized businesses, who seek to establish a monopoly under their area of operations. New regulatory burdens could jeopardise this equilibrium and lead to a potential labour shortage if truckers perceive these regulations as excessively punitive. 

\vspace{0.5\baselineskip}

Furthermore, many truck drivers already face significant financial pressures, including rising operational costs and the burden of taxes. The addition of stringent regulations may exacerbate their grievances, leading to heightened resistance against governmental oversight. Should profit margins dwindle due to these regulatory measures, there exists a tangible risk that many truckers may exit the industry, thus undermining the critical supply chain that supports economic activity.

Currently, most governments impose standardised limits on the maximum permissible weight for commercial vehicles and specific weight restrictions per axle. However, these regulations often fail to account for essential factors such as road conditions, topography, and incline variations. The enforcement of these regulations poses its own challenges, as trucks are not routinely subjected to inspections, which allows many logistics companies to overload their vehicles. The potential revenue generated from overloading frequently outweighs the penalties imposed for non-compliance, resulting in a regulatory environment that struggles to maintain efficacy.

One potential solution to this challenge is the implementation of weighbridges, strategically located at critical checkpoints such as border crossings, bridges, and toll booths to measure a vehicle’s total weight in real time. This system offers a non-intrusive means of ensuring compliance without necessitating detours for trucks. 

\subsection{An overview of weighbridges}

Bwire and Nairobi [12] conducted an extensive analysis of the implementation and regulation of weighbridges in Kenya, highlighting how the Traffic Act of Kenya Cap 403 (2018) defines permissible gross vehicle weights (GVW) by axle configurations. For example, the 2 Configuration (two axles with single wheels) permits up to 18,000 kg, while the 2A Configuration (two axles with front single and rear double wheels) also allows up to 18,000 kg. Larger configurations, like the 6A and 7 Configurations for articulated trucks, can support up to 56,000 kg for vehicles with seven axles. 

Fixed weighbridges are permanent structures installed along roads, offering high accuracy and durability, making them ideal for high-traffic areas where enforcing load regulations is critical. In contrast, portable weighbridges are mobile units that can be installed temporarily for unscheduled checks. While they offer flexibility, they are generally less durable and less accurate, often deployed in remote areas where frequent location changes are necessary.

Due to their high accuracy, fixed weighbridges are the most commonly used in Kenya. They can measure both gross vehicle weights (GVW) and axle loads, enabling trucks to redistribute loads when overloaded. Smaller single axle weighbridges, which weigh one axle at a time, are easier to relocate but slower, as operators must sum individual axle readings to calculate the total weight. As noted by Victor [13], this process can be time-consuming. On the other hand, axle unit weighbridges weigh multiple axles in one operation, providing faster and more accurate results. These are commonly used in high-traffic areas, with platform sizes ranging from 3.2m x 3m to 3.2m x 4m [12],[13].

Multi-deck weighbridges, which use multiple platforms to weigh multi-axle vehicles simultaneously, offer the highest levels of accuracy and efficiency, making them the preferred choice for monitoring trucks on major highways. Mobile weighbridges consist of movable pads placed on the road surface, where axle weights are calculated by summing the wheel loads. These weighbridges are typically used for random, unscheduled checks, but they require levelling mats and frequent calibration to ensure accuracy.

Weighbridges operate in two main modes: static and dynamic. In the static method, vehicles must stop on the weighbridge platform, offering higher accuracy, which is essential for legal enforcement. However, it is time-consuming, leading to potential delays in high-traffic areas. In this mode, the total mass transmitted from all axles to the wheels is measured over a 15-second time span [14]. The dynamic method, or Weigh-in-Motion (WIM), allows vehicles to be weighed while moving, reducing traffic congestion. However, while WIM systems help maintain traffic flow, they are less accurate than static methods, making them less suitable for legal proceedings like establishing liability in case of accidents [15].

Accuracy is critical for prosecuting truck operators for overloading. In Kenya, permissible errors depend on the weighbridge's capacity, ranging from 20 kg for an 80-tonne capacity weighbridge to 80 kg for a 400-tonne capacity unit upon re-verification. For first-time verification, stricter tolerances apply, such as 10 kg for an 80-tonne weighbridge and 40 kg for a 400-tonne unit. For comparison, New Zealand's tolerances range from ±40 kg for loads between 10 and 40 tonnes, while the US National Institute of Standards Handbook 44 specifies an acceptance tolerance of 0.1\%, equating to $\pm$40 kg for a 40-tonne load [16].

The type of weighbridge selected also depends on traffic volumes and road classification. Multi-deck weighbridges are used on high-traffic roads, while single-axle or portable weighbridges may suffice for roads with lower traffic. Weighbridge installations are capital-intensive projects, with costs influenced by platform size, load capacity, and supporting infrastructure. In 2008, single-axle weighbridges cost between 0.4 and 1 million USD, while multi-deck weighbridges ranged from 6.0 to 8.0 million USD. Adjusted for inflation, these costs would be approximately \$584,198 to \$1,460,495 for single-axle weighbridges and \$8,762,971 to \$11,683,961 for multi-deck weighbridges, inclusive of procurement, installation, and maintenance [17].

Despite the high costs, implementation has proven to be fruitful. For instance, in 1995, an Indian government official visiting Chicago observed how technology was integrated to streamline dumping ground procedures. Upon returning to Calcutta, he faced opposition from local mafias but successfully installed a static weighbridge system at a cost of 80 lakhs (then), replacing a trip-based payment system with one based on the tonnage of waste dumped. This reduced the number of trips to 500 and significantly improved the efficiency of truck loading [18]. In more recent times, Mkhize and De Beer [19] conducted a statistical analysis of vehicle loads using a novel Stress-in-Motion (SIM) mechanism at a traffic signal near Heidelberg, Germany. Their study demonstrated strong repeatability and reproducibility, with the SIM system underestimating the actual Gross Vehicle Mass (GVM) by only 6\%. This slight underestimation highlights the system's potential for real-time vehicle load monitoring, complementing existing weighbridge methods and offering a promising alternative for dynamic load assessment in traffic-heavy environments.

\subsection{Alternative approaches}

Jeuken [20] recognised that weighbridges were permanent, capital-intensive investments often unnecessary for simple applications, such as weighing livestock or agricultural produce. To address this issue, he developed a `free-hanging cattle cage' as a low-cost, easily portable alternative to traditional weighbridges. In another project, he aimed to replicate the functionality of a weighbridge within the vehicle by utilising cargo compartments, or bulk tanks, as load receptors in various configurations. He integrated the cargo compartment rigidly with the chassis of a lorry through multiple strain gauge load cells, effectively making the weighing system an integral part of the vehicle. The load cells were strategically positioned between the subframe and the cargo compartment, designed to be robust enough to withstand operational stresses. However, the specific dimensions of these load cells resulted in limited measurement resolution, meaning that accurate weighing was primarily feasible when the vehicle was on a level surface.

Additionally, the project explored a specialised application of this weighing system within semi-trailers. In this configuration, load cells were installed between the tank and the running gear on one side, and between the semi-trailer coupling tray and the truck's subframe on the other. This design enhanced the overall measurement capability while accommodating the dynamic characteristics of trailer operation.

In another project, Longo et al. [21] proposed a simplified physics-based model for a compact angular head, known as RHEvo, which was primarily used for hemming in production lines. This tool consisted of mechanical components, including springs, rollers, skates, and bearings. After developing the model, the authors employed a neural network to estimate the current state of these internal components. Their physical analysis showed that aging impacted the elastic coefficient of the springs due to fatigue degradation. They also calculated the remaining useful life (RUL) of the internal springs using a stochastic model. This approach was applicable to various devices that utilised springs, including automobile suspension systems, weapon recoil mechanisms, and engine shut-off valves.

Tinga and Loendersloot [22] also discussed methods and tools that enhanced predictive maintenance. The authors highlighted that vibration-based machinery health monitoring techniques were effective for detecting damage, diagnosing system health, and predicting the remaining life of machinery. They introduced a decision support tool that guided users in selecting the most suitable predictive maintenance strategies and monitoring techniques.

Building on this, Yang et al. [23] applied neural network algorithms trained on engine data to predict vehicle overloading. Their model aimed to detect engine performance patterns correlated with overload conditions, providing a proactive approach to fleet management.

Similarly, Praveena et al. [24] developed a real-time load monitoring system designed to ensure trucks comply with weight regulations. This system used load cells installed between the chassis and trailer to continuously measure weight. When the permissible weight limit was exceeded, the system automatically disconnected the battery, preventing the engine from starting. To address uneven load distribution on inclines, a gyroscopic sensor was integrated, allowing the system to disable this restriction when the vehicle was on a slope.

Complementing this work, Chen and Chen [25] designed a module that utilised data from a Tire Pressure Monitoring System (TPMS) to detect overloading. By correlating changes in tyre pressure with vehicle weight, their system provided an additional layer of monitoring, ensuring compliance with weight regulations and helping prevent overloading.

Arena et al. [26] conducted a comprehensive survey on predictive maintenance mechanisms, focusing on how these systems could identify overloading patterns in trucks and commercial vehicles. By establishing a baseline and monitoring the frequency of maintenance recommendations, one may be able to infer potential overloading occurrences. This approach offers insights into operational stresses and fleet management inefficiencies. However, the analysis rests on the assumption of ceteris paribus—that all external factors remain constant. It assumes that drivers are adhering to optimal driving practices, no accidents are occurring, road conditions are normal, and vehicle loads are evenly distributed. While this provides a controlled environment for analysis, real-world variables such as driver behaviour, uneven load distribution, mechanical wear, and changing road conditions introduce significant variability, affecting the accuracy of the findings. To address these issues, further research is required to develop adaptive models that consider these dynamic factors in predictive maintenance strategies.