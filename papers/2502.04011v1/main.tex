\documentclass[conference]{IEEEtran}
\IEEEoverridecommandlockouts
% The preceding line is only needed to identify funding in the first footnote. If that is unneeded, please comment it out.
\usepackage{cite}
\usepackage{amsmath,amssymb,amsfonts}
\usepackage{algorithmic}
\usepackage{graphicx}
\usepackage{textcomp}
\usepackage{xcolor}
\usepackage[tikz]{bclogo}
\usepackage[framemethod=tikz]{mdframed}
\usepackage{lipsum}

\usepackage[many]{tcolorbox}
\usepackage{multirow}
\usepackage{array}
\usepackage{arydshln}
% \linespread{0.96}
 \usepackage[subtle,tracking=normal]{savetrees}
\definecolor{bgblue}{RGB}{245,243,253}
\definecolor{ttblue}{RGB}{91,194,224}

\mdfdefinestyle{mystyle}{%
  rightline=true,
  innerleftmargin=10,
  innerrightmargin=10,
  outerlinewidth=0pt,
  topline=false,
  rightline=true,
  bottomline=false,
  skipabove=\topsep,
  skipbelow=\topsep
}

\newtcolorbox{myboxi}[1][]{
  breakable,
  title=#1,
  colback=white,
  colbacktitle=white,
  coltitle=black,
  fonttitle=\bfseries,
  bottomrule=0pt,
  toprule=0pt,
  leftrule=3pt,
  rightrule=3pt,
  titlerule=0pt,
  arc=0pt,
  outer arc=0pt,
  colframe=black,
}

\newtcolorbox{myboxii}[1][]{
  breakable,
  freelance,
  title=#1,
  colback=white,
  colbacktitle=white,
  coltitle=black,
  fonttitle=\bfseries,
  bottomrule=0pt,
  boxrule=0pt,
  colframe=white,
  overlay unbroken and first={
  \draw[black,line width=3pt]
    ([xshift=5pt]frame.north west) -- 
    (frame.north west) -- 
    (frame.south west);
  \draw[black,line width=3pt]
    ([xshift=-5pt]frame.north east) -- 
    (frame.north east) -- 
    (frame.south east);
  },
  overlay unbroken app={
  \draw[black,line width=3pt,line cap=rect]
    (frame.south west) -- 
    ([xshift=5pt]frame.south west);
  \draw[black,line width=3pt,line cap=rect]
    (frame.south east) -- 
    ([xshift=-5pt]frame.south east);
  },
  overlay middle and last={
  \draw[black,line width=3pt]
    (frame.north west) -- 
    (frame.south west);
  \draw[black,line width=3pt]
    (frame.north east) -- 
    (frame.south east);
  },
  overlay last app={
  \draw[black,line width=3pt,line cap=rect]
    (frame.south west) --
    ([xshift=5pt]frame.south west);
  \draw[black,line width=3pt,line cap=rect]
    (frame.south east) --
    ([xshift=-5pt]frame.south east);
  },
}
\usepackage{xcolor}
\usepackage{framed}
\colorlet{shadecolor}{lightgray}

\def\BibTeX{{\rm B\kern-.05em{\sc i\kern-.025em b}\kern-.08em
    T\kern-.1667em\lower.7ex\hbox{E}\kern-.125emX}}
\begin{document}

\title{Debiasing Architectural Decision-Making: An Experiment With Students and Practitioners
\thanks{ Part of this work was supported by Funetec-PB – Ct.: Phoebus 01/24 and Snet 04/22.}
}


\author{\IEEEauthorblockN{1\textsuperscript{st} Klara Borowa}
\IEEEauthorblockA{\textit{Warsaw University of Technology,}\\ 
\textit{Institute of Control and Computation Engineering}\\
Warsaw, Poland \\
klara.borowa@pw.edu.pl}
\and
\IEEEauthorblockN{2\textsuperscript{nd} Rodrigo Rebouças de Almeida}
\IEEEauthorblockA{\textit{Federal University of Paraíba} \\\textit{Department of Exact Sciences} \\
Rio Tinto/PB, Brazil \\
rodrigor@dcx.ufpb.br}
\and
\IEEEauthorblockN{3\textsuperscript{rd} Marion Wiese}
\IEEEauthorblockA{\textit{Universität Hamburg} \\
\textit{Department of Informatics}\\
Hamburg, Germany \\
marion.wiese@uni-hamburg.de}
}

\maketitle

\begin{abstract}
%Context
Cognitive biases are predictable, systematic errors in human reasoning. They influence decision-making in various areas, including architectural decision-making, where architects face many choices. \color{black} For example, anchoring can cause architects to unconsciously prefer the first architectural solution that they came up with, without considering any solution alternatives. \color{black} Prior research suggests that training individuals in debiasing techniques during a practical workshop can help reduce the impact of biases.
%Goal
The goal of this study was to design and evaluate a debiasing workshop with individuals at various stages of their professional careers.
%Method
To test the workshop's effectiveness, we performed an experiment with 16 students and 20 practitioners, split into control and workshop group pairs. We recorded and analyzed their think-aloud discussions about improving the architectures of systems they collaborated on.
%Result
The workshop improved the participants' argumentation when discussing architectural decisions and increased the use of debiasing techniques taught during the workshop. This led to the successful reduction of the researched biases' occurrences. In particular, anchoring and optimism bias occurrences decreased significantly.
We also found that practitioners were more susceptible to cognitive biases than students, so the workshop had a more substantial impact on practitioners.
We assume that the practitioners' attachment to their systems may be the cause of their susceptibility to biases.
\color{black} Finally, we identified factors that may reduce the effectiveness of the debiasing workshop. On that basis, we prepared a set of teaching suggestions for educators. \color{black} 
%Conclusion
Overall, we recommend using this workshop to educate both students and experienced practitioners about the typical harmful influences of cognitive bias on architectural decisions and how to avoid them.
\end{abstract}

\begin{IEEEkeywords}
Software Architecture, Architectural Decisions, Architectural Decision-Making, Cognitive bias, Debiasing
\end{IEEEkeywords}




\section{Introduction}
\label{sec:Introduction}
        
        %biases
        \textbf{Cognitive biases} are systematic errors caused by the heuristics the human mind uses to reduce the complexity of various tasks \cite{Tversky1974}, including decision-making.
        Cognitive biases distort human decision-making in various domains: from clinicians making erroneous medical diagnoses \cite{norman2017causes} to developers copying code without reading it \cite{Chattopadhyay2020}.
        %biases in SE
        Stacy and Macmillan \cite{Stacy1995} first observed the possible impact of cognitive biases in the realm of software engineering. 
        \color{black} In particular, how \textit{representativeness,} \textit{availability bias} and \textit{confirmation bias} could influence software developers' daily activities. 
        \color{black}
        Since then, research on cognitive biases has significantly expanded to include at least 36 biases and their impact on various software engineering knowledge areas~\cite{Mohanani2018}.
        %biases in SA
        The influence of cognitive biases is particularly impactful in architectural decision-making (ADM)~\cite{Tang2011}~\cite{VanVliet2016}\color{black}, \color{black} since software architecture can be considered as a set of design decisions \cite{jansen2005software}.
        %rationalizing ADM
        However, few studies have focused on behavioral factors in ADM, with even fewer containing any empirical validation of decision-making techniques \cite{Razavian2019}.
        Notable examples of studies that have focused on improving the process of ADM include those that assess the usefulness of reflection \color{black}-- \color{black}Razavian et al. \cite{razavian2016two} (experiment on students), and Tang et al. \cite{Tang2018} (experiment on students and professionals).

        %Debiasing
        \textbf{Debiasing} is a process that improves one's judgment by using techniques decreasing the impact of a particular cognitive bias~\cite{fischhoff1982debiasing}.
        A workshop can effectively mitigate some influences of cognitive biases on software practitioners'~\cite{Shepperd2018}.
        %Debiasing in SA
        %rr del: However, in the context of architectural decision-making, as far as we know, the only empirically proved debiasing training, proposed by Borowa et al. \cite{Borowa2022}, was an experiment on groups of students with barely any real-life experience in software development. 
        %%rr add:
        As far as we know, the only empirically tested debiasing training for ADM was conducted on a group of students with limited experience in software development~\cite{Borowa2022}. This study had students perform theoretical tasks, and \color{black} employed a pretest-posttest design. \color{black} %compared the students' performance before and after a debiasing workshop.
        %Rationale
        Given the doubts expressed by the software engineering community about the validity of the findings in experiments with student participants~\cite{galster2023empirical}~\cite{ baltes2022sampling}, as well as numerous threats to validity associated with pretest-posttest experiments~\cite{Knapp2016}, further evaluation of debiasing through workshops is warranted.
        % Additionally, one of the three debiasing techniques proposed by Borowa et al. \cite{Borowa2022}, i.e., having one person monitoring the discussion to avoid confirmation bias, proved to have a poor debiasing effect. This means that a different debiasing technique could be explored instead.
        %Goal 
        Therefore, the \textbf{goal} of this study is to conduct a debiasing intervention with a control group to ascertain its effectiveness and investigate its impact on individuals at various stages of their careers, i.e., students and practitioners. 
        % Our intervention is similarly designed to the one by Borowa et al. \cite{Borowa2022}, with \textbf{the significant change} of changing one of three proposed debiasing techniques.
         
        % RQs
        \noindent
            As such, we strive to answer the following research question:
            \begin{itemize}
                \item [] \textbf{RQ: How does the proposed debiasing architectural decision-making workshop influence students and practitioners?}
            \end{itemize}

        \noindent In particular, we explored whether the following \color{black} effects would appear:
        \textbf{(1) Would the workshop decrease the number of cognitive bias occurrences? (2) Would the workshop increase the participants' use of debiasing techniques?}


                %\item \textbf{Would the debiasing techniques decrease the amount of cognitive biases?}
            
\color{black}
        
        %Method
        To answer this research question, we performed 18 controlled experiments with 16 students and 20 software practitioners, divided into control and workshop groups. %Participants were recruited from three countries ([hidden for double-blind review]) over nine companies.
        %rr del: Since the results of our experiments were mixed and only partially positive, we analyzed the data to understand the sources of the results.
        %Contribution

        \noindent
            The study's main \textbf{contribution} is the design and empirical evaluation of the workshop, which resulted in the following:
            \begin{enumerate}
                \item \color{black}The workshop reduced the occurrence of biases for both students and practitioners. It decreased \color{black} the occurrence of all three researched biases (\textit{anchoring}, \textit{confirmation bias}, \textit{and optimism bias}). The decrease in anchoring and optimism bias was statistically significant. %However, the reduction of confirmation bias was less effective.
                \color{black}
                \item The workshop increased the use of all three proposed debiasing techniques. This change was statistically significant in the case of the ``listing multiple solutions'' and ``discussing drawbacks'' techniques. 
                %This was not the case for the ``discussing risks'' technique.
                
                \item The workshop significantly improved the amount of non-biased arguments for and against architectural solutions while reducing biased arguments.
                \color{black}
                % However, the effects of \textit{anchoring} and \textit{confirmation bias} were not significantly decreased.
                % \item We observed that experienced practitioners react differently to such debiasing workshop than students.
                %rr del: \item We established that when practitioners over-produce arguments in favor of choosing a particular solution, this leads to numerous biased arguments. As such, we propose that future debiasing efforts focus more on the argument's quality than quantity.
                %rr update:
                % \item We found that when professionals over-produce arguments in favor of choosing a specific solution, biased arguments increase. Therefore, we suggest that future debiasing efforts should prioritize the quality of arguments over the quantity.
                %%rr del: \item We found that increasing the number of counterarguments opposing a particular solution has the opposite effect - the number of non-biased counterarguments increases.
                %rr update:
                %\item We observed that when the number of counterarguments opposing a particular solution increases, the number of non-biased counterarguments also increases.
                %rr del: \item We identified that additional factors, such as group decision-making and individual high confidence levels, can impact practitioners' biased thinking.
                %%rr update:
                %\item We found that group decision-making and high confidence levels can influence biased thinking.
                \item The students discussing their university projects were less impacted by cognitive biases than practitioners in the case of real-life architectures.
                \item The positive effects of the workshop on practitioners were more substantial, 
                \color{black}
                i.e., the decrease in bias occurrences was greater for practitioners. This change was possible because of the higher initial bias level of practitioners, which left more room for improvement.
                \item A set of teaching suggestions based on the factors that impacted the few unsuccessful participants.
                \color{black}
            \end{enumerate}
            
            
        % study outline 
            %rr update:
            Section \ref{sec:related_work} provides information on previous research relevant to this study. Section \ref{sec:Method} describes the research method. The results of the experiment are presented in Section \ref{sec:Results} and discussed in Section \ref{sec:Discussion}. Section \ref{sec:ThreatsToValidity} explores the threats to validity.
            Finally, we summarize this study in Section \ref{sec:Conclusion}.
    
    \section{Related work}
    \label{sec:related_work}
    
    \subsection{Cognitive biases}
    The seminal work of Tversky and Kahneman from 1974~\cite{Tversky1974} introduced the concept of cognitive biases, describing three of them: \textit{representatives}, \textit{availability}, and \textit{anchoring}. In their work, the researchers found that human beings rely heavily on heuristics during the decision-making process while often being blind to logical, statistical facts.  

    These findings later evolved into the dual process theory, describing the human mind as divided into Systems 1 and 2. System 1 performs fast and intuitive decisions that heavily rely on heuristics. Inversely, System 2 performs slow decisions that are logical and rule-based. By using energy mainly for important decisions (System 2), this natural phenomenon allows the human body to save precious energy when making simplified decisions (System 1).
    %This natural phenomenon allows individuals to save precious energy by using the energy for important decisions (System 2), while simple decisions can be made by System 1.
    However, humans are prone to using the energy-saving System 1 for decision-making, even in cases that require rule-based thinking. In these cases, cognitive biases might occur \cite{kahneman2011think}. 

    As a counterpoint to biased reasoning, \textbf{rational reasoning} can be described based on the research by William James~\cite{james1890} as having two components: (1) perception of a specific piece of \textbf{factual information}, and (2) a \textbf{logical consequence} of this information \cite{james1890}. 

    \noindent
    %Researchers have found numerous ways in which cognitive biases impact various areas of software engineering. 
    % mapping studies
    
    %In their systematic mapping study, Mohanani et al. \cite{Mohanani2018} found over 37 biases that have been researched in the software engineering context. The four most commonly researched biases are \textit{anchoring and adjustment, availability bias, confirmation bias}, and \textit{optimism bias}. Most studies on cognitive biases are primarily empirical studies (72\%), predominantly experiments. 
    
    % The following three are found to be particluarly relevant for architectural decision-making~\cite{}:
    % \begin{itemize}
    %     \item 
    % \end{itemize}
    

    
    %It impacts the individuals who expect more successful outcomes for themselves than their peers \cite{weinstein1989optimistic}.
    
    %\subsection{Cognitive biases impact on Software Engineering}
    
    %Two SWEBOK knowledge areas that researchers mainly focused on so far are Management and Design.
    
    %Chakraborty et al. \cite{Chakraborty2018} found ten categories of biases that influence software developers during code implementation: (1) preconceptions (related to developers' beliefs), (2) ownership (related to preferring code created by the specific developer), (3) Fixation (related to fixating on particular ideas/solutions), (4) Resort o default (related to keeping the status-quo), (5) Optimism (related to over-optimistic assumptions), (6) Convenience (preferring short-term benefits over long-term), (7) Subconscious action (prioritizing most visible issues), (8) Blisfull ignorance (related to habitualy ignoring inconvenient information), (9) Superficial selection (related to choosing solutions that 'look good' at first glance and (10) memory bias (preference for newly obtained information). An example of bias impact from the fixation group is how one experiment participant anchored on the idea that a function's definition had to be changed when, in fact, the software only had to be rebooted.

%single instances
    %In testing, \textit{confirmation bias }impacts developers who are prone to writing unit tests proving that the software works correctly instead of finding defects \cite{Calikli2013a}.

    % In project risk management, optimism bias makes individuals focus more on the pros than the cons of choosing a risk management plan \cite{Shalev2014}.
    

    %In software effort estimation, developers are strongly influenced by \textit{anchoring} and thus often fixate on the first numerical estimate that appears during a conversation \cite{Løhre2016}.
    

%rr update:
    %\subsection{Cognitive biases and Architectural Decision-Making}
    Viewing software architecture as a set of design decisions is a well-established concept~\cite{jansen2005software}. It has resulted in a substantial amount of research regarding issues related to ADM, such as documenting architectural decisions~~\cite{vanHeesch2012documentation}, models of architectural decisions~\cite{zimmermann2009managing}, decision-making best practices \cite{Tang2021}, and human aspects of ADM \cite{Tang2017}, particularly cognitive biases \cite{VanVliet2016}.

    The impact of cognitive biases on ADM can have severe consequences, such as designing sub-par solutions \cite{Manjunath2018a} or incurring dangerous architectural technical debt \cite{Borowa2021b}.

    Based on previous research on cognitive biases in ADM \cite{VanVliet2016,Zalewski2017,Borowa2021b}, as well as the most often researched biases in software engineering \cite{Mohanani2018}, this study focuses on the following three cognitive biases: %\textit{anchoring, confirmation bias, and optimism bias.}

    \textbf{\textit{Anchoring bias}} is the decision-maker's preference for initial information/ideas/so\-lu\-tions (which then become an `anchor') \cite{Tversky1974}. 
    \color{black}
    In ADM, for example, architects may anchor on the first solution idea that comes to their minds, and they may refuse to change or adjust this solution, even when it becomes necessary~\cite{Tang2011}.
    \color{black}
    % Anchoring may, for example, influence developers by fixating them on the first numerical estimate that appears during a conversation, which may cause their time estimates to be inaccurate \cite{Løhre2016}.
    %An example impact of \textit{anchoring} can be observed in cooperative group decision-making, where decisions are made in a fast implicit negotiation among group members that usually happens in a heuristic fashion \cite{deWilde2018anchoring}.

    \textbf{\textit{Confirmation bias}} is the tendency to purposefully search and interpret information to verify one's beliefs \cite{nickerson1998confirmation}. 
\color{black}
    For example, due to confirmation bias, an architect may have the belief that microservices are the best architectural style and insist on designing a microservice-based system, even when it is not appropriate~\cite{Zalewski2017}.
    \color{black}
    % An example of confirmation bias would be developers' proneness to writing unit tests proving that the software works correctly instead of finding defects \cite{Calikli2013a}.
    %It makes humans less likely to change their views, e.g., political views, despite counter-evidence \cite{kaplan2016neural}.
    
    \textbf{\textit{Optimism bias}} is the overestimation of the probability of positive future outcomes~\cite{sharot2011optimism}. 
    \color{black}
    This may affect architects by leading them to underestimate the maximum amount of requests that would be sent to a particular component~\cite{Zalewski2017}.
    \color{black}
    % An example of optimism bias is its impact on project risk management, where individuals ignore potential risks while they focus on the pros of their preferred choices \cite{Shalev2014}.


% \begin{table*}
% \caption{Participants}
% \label{tab:participants}
% \footnotesize
% \begin{tabular}{|l|p{0.6cm}|l|p{0.9cm}|p{1.1cm}|p{1.9cm}|p{2.5cm}|p{1.6cm}|p{0.8cm}|}%|p{1.2cm}|}
% \hline
% %No.&Pair No.& Group & Age (years) & Experience working in IT (years)& Role & Company Domain &  Company size (no. of employees) & Pilot or Main & Decisions discussed\\
% No.&Pair & Group & Age & IT  & Role & Company  &  Company & Pilot\\%  & Decisions \\
% &No.&  & (years) &  exp. &  & domain &  size &   % & discussed\\
% & &  &  &  & (years)&  &  &  (employees) &   \\%&  \\
% \hline
% 1& P1 &Workshop &28  & 7  &	Developer				& Digital payment					& 21-100 &Yes \\% & 4\\ %Software Developer
% %\hline
% 2& P1 &Control &28  & 2  & 	Developer				& Digital payment					& 21-100&Yes\\% & 5\\ %Software Developer
% %\hline
% 3 & P2 &Workshop & 40 & 18 &  Analyst & Digital payment & 21 - 100&Yes \\% & 4\\ %Systems Analyst
% %\hline 
% 4& P2 &Control & 37 & 5 &    Product Manager & Digital payment & 21 - 100&Yes \\% & 2\\ % Product Project Manager
% \hline 
% 5&P3 &Workshop & 52	& 20 &	Developer		& Media						& over 8000&No \\% & 6 \\ %Senior Software Developer
% %\hline
% 6&P3&Control& 59	& 38 &	Developer		& Media						& over 8000&No \\% & 11\\ %Senior Software Developer
% %\hline
% 7&P4&Workshop& 42	& 20 &	Developer		& Marketing Services	& 101-500&No \\% & 9\\ %Senior Software Developer
% %\hline
% 8&P4&Control& 40	& 20 &	 Developer		& Marketing Services	& 101-500&No \\% & 19\\ %Senior Software Developer
% %\hline
% 9&P5&Workshop& 42	& 20 &	CTO								& Finance					& 101-500&No \\% &11\\
% %\hline
% 10&P5&Control& 52	& 30 &	Architect						& Finance					& 101-500&No \\% &8\\
% %\hline
% 11&P6&Workshop& 39	& 7	 &	Architect 	& Signal processing 		& 500-5000&No \\% 11\\ %Software Engineer/Architect
% %\hline
% 12&P6&Control& 59	& 38 &	Systems Engineer				& Signal processing		& 500-5000&No \\% &12\\
% %\hline
% 13&P7&Workshop& 46	& 20 &	Architect				& Retail					& over 5000&No \\% &9\\ %Solution Architect
% %\hline
% 14&P7&Control &41	& 19 &	Architect				& Retail					& over 5000&No \\% &6\\ %Solution Architect
% %\hline

% 15&P8&Workshop& 29	& 5	 &	Developer 	& Education		& 500-5000&No \\% &5\\ %Software Developer
% %\hline 
% 16&P8&Control& 30	& 4 &	Developer				& Education		& 500-5000&No \\% &12\\ %Software Developer
% %\hline
% 17&P9&?& 37	& 5 &	Product Manager				& Entertainment					& 21-1000&No \\% &16\\ %Product Project Manager
% %\hline
% 18&P9&? &40	& 18 &	Analyst				& Entertainment					& 21-1000&No \\% &18\\ %Systems Analyst	
% \hline
% \end{tabular}
% \end{table*}





% \begin{table}
% \caption{Participants \\ \tiny{W-workshop group / C-Control group, \\ 
% Age - Age in years, \\
% IT exp. - commercial experience in years, \\
% DOM - Company domain, \\
% SIZE - Company size: XS = below 20, S=21-100, M=101-500, L=501-5000, XL= over 5000
% }}
% \label{tab:participants}
% \footnotesize
% %\begin{tabular}{|c|p{0.2cm}|p{0.25cm}|l|p{0.9cm}|p{1cm}|p{1.5cm}|p{1.5cm}|p{1.6cm}|p{0.7cm}|}%|p{1.2cm}|}
% \begin{tabular}{|c|p{0.2cm}|p{0.25cm}|p{0.4cm}|p{0.6cm}|p{0.6cm}|p{1.2cm}|p{1.2cm}|p{0.4cm}|}%p{0.2cm}|}%|p{1.2cm}|}
% \hline
% %No.&Pair No.& Group & Age (years) & Experience working in IT (years)& Role & Company Domain &  Company size (no. of employees) & Pilot or Main & Decisions discussed\\
% &No&Pair & C/W & Age   & IT   & Role & DOM & SIZE \\%  & Decisions \\
% &   &  &      & & exp.&       &   &       \\% & discussed\\
% %&   &No.  &      & (years)& exp.&       & domain  &  size     \\% & discussed\\
% %&   &     &      &        & (years)&     &         &  (staff)    \\%&  \\
% \hline

% \multirow{4}{*}{\rotatebox[origin=c]{90}{Pilot}}
% &1& P1 &W &28  & 7  &	Developer				& Digital payment					& S  \\
% % & 4\\ %Software Developer
% %\hdashline
% %\hline
% &2& P1 &C &28  & 2  & 	Developer				& Digital payment					& S\\% & 5\\ %Software Developer
% %\hline
% &3 & P2 &W & 40 & 18 &  Analyst & Digital payment & S \\% & 4\\ %Systems Analyst
% %\hline 
% &4& P2 &C & 37 & 5 &    PM & Digital payment & S \\% & 2\\ % Product Project Manager
% \hline 
% \multirow{14}{*}{\rotatebox[origin=c]{90}{Main - practitioners}}
% &5&P3 &W & 52	& 20 &	Developer		& Media						& XL \\% & 6 \\ %Senior Software Developer
% %\hline
% &6&P3&C& 59	& 38 &	Developer		& Media						& XL \\% & 11\\ %Senior Software Developer
% %\hline
% &7&P4&W& 42	& 20 &	Developer		& Marketing& M \\% & 9\\ %Senior Software Developer
% %\hline
% &8&P4&C& 40	& 20 &	 Developer		& Marketing& M \\% & 19\\ %Senior Software Developer
% %\hline
% &9&P5&W& 42	& 20 &	CTO								& Finance					& M \\% &11\\
% %\hline
% &10&P5&C& 52	& 30 &	Architect						& Finance					& M \\% &8\\
% %\hline
% &11&P6&W& 39	& 7	 &	Architect 	& Audio 		& L \\% 11\\ %Software Engineer/Architect %Signal processing
% %\hline
% &12&P6&C& 59	& 38 &	Sys. Eng.				& Audio	& L \\% &12\\
% %\hline
% &13&P7&W& 46	& 20 &	Architect				& Retail					& XL \\% &9\\ %Solution Architect
% %\hline
% &14&P7&C &41	& 19 &	Architect				& Retail					& XL \\% &6\\ %Solution Architect
% %\hline

% &15&P8&W& 29	& 5	 &	Developer 	& Education		& L \\% &5\\ %Software Developer
% %\hline 
% &16&P8&C& 30	& 4 &	Developer				& Education		& L \\% &12\\ %Software Developer
% %\hline
% &17&P9&W& 39	& 20 &	PM			& Government					& S \\% &16\\ %Product Project Manager
% %\hline
% &18&P9&C &42	& 19 &	Coordinator				& Government					& S \\%Systems Development Coordinator
% &19&P10&W& 23	& 2 &	Developer				& Education					& S \\% &16\\ %Product Project Manager
% %\hline
% &20&P10&C &23	& 2 &	Developer				& Education					& S \\% &18\\ %Systems Analyst	
% \hline
% \multirow{14}{*}{\rotatebox[origin=c]{90}{Main - students}}
% &	21	&	S1	&	W	&	24	&	0	&	-	&	-	&	-	\\
% &	22	&	S1	&	C	&	24	&	2	&	Developer	&	Software	&	XL	\\ %SW Engineer/Developer
% &	23	&	S2	&	W	&	23	&	0.6	&	Developer	&	Electronics	&	XL	\\ %Python developer
% &	24	&	S2	&	C	&	24	&	3	&	Developer	&	Utilities	&	L	\\ %Software Engineer
% &	25	&	S3	&	W	&	23	&	2	&	Developer	&	Insurance&	S	\\ %Fullstack 
% &	26	&	S3	&	C	&	23	&	1.5	&	Developer	&	ERP	&	M	\\ %Junior 
% &	27	&	S4	&	W	&	21	&	0.4	&	Developer	&	\mbox{Web design}	&	S	\\ %Frontend Developer
% &	28	&	S4	&	C	&	22	&	1.5	&	Developer	&	Security	&	XL\\ %Cybersecurity T-SQL Developer
% &	29	&	S5	&	W	&	20	&	0	&	-	&	-	&	-	\\
% &	30	&	S5	&	C	&	21	&	0	&	-	&	-	&	-	\\
% &	31	&	S6	&	W	&	23	&	1	&	Tester	&	\mbox{Web dev.}&	S	\\ %Data Procesing website	
% &	32	&	S6	&	C	&	24	&	1.5	&	Developer	&	ERP	&	M	\\
% &	33	&	S7	&	W	&	26	&	1.5	&	Pre-sales	&	Sales	&	XL	\\
% &	34	&	S7	&	C	&	23	&	0.8	&	BMS Eng.&	Energetics 	&	L	\\
% &	35	&	S9	&	W	&	27	&	3	&	ML Eng.	&	Security	&	L	\\
% &	36	&	S8	&	C	&	26	&	1	&	DevOps	&	Earth Observation	&	M	\\
% \hline
% \end{tabular}
% \vspace*{-1.5\baselineskip}
% \end{table}


\begin{table}
\caption{Participants \\ \tiny{W-workshop group / C-Control group, \\ 
Age - Age in years, \\
IT exp. - commercial experience in years, \\
DOM - Company domain, \\
SIZE - Company size: S = below 100, M=101-500, L=501-5000, XL= over 5000
}}
\label{tab:participants}
\center
\footnotesize
\scalebox{0.8}{
%\begin{tabular}{|c|p{0.2cm}|p{0.25cm}|l|p{0.9cm}|p{1cm}|p{1.5cm}|p{1.5cm}|p{1.6cm}|p{0.7cm}|}%|p{1.2cm}|}
\begin{tabular}{|c|c|c|c|c|c|c|c|}%p{0.2cm}|}%|p{1.2cm}|}
\hline
%No.&Pair No.& Group & Age (years) & Experience working in IT (years)& Role & Company Domain &  Company size (no. of employees) & Pilot or Main & Decisions discussed\\
&No&Pair & C/W   & IT   & Role & DOM & SIZE \\%  & Decisions \\
&   &  &      & exp.&       &   &       \\% & discussed\\
%&   &No.  &      & (years)& exp.&       & domain  &  size     \\% & discussed\\
%&   &     &      &        & (years)&     &         &  (staff)    \\%&  \\
\hline

\multirow{4}{*}{\rotatebox[origin=c]{90}{Pilot}}
&1& P1 &W   & 6-10  &	Developer				& Digital					& S  \\
& &   &  &   &    			& payment					&   \\% & 4\\ %Software % & 4\\ %Software Developer
%\hdashline
%\hline
&2& P1 &C   & 1-2  & 	Developer				& Digital					& S\\% 
& &   &  &   &    		& payment					&   \\% & 4\\ %Software & 5\\ %Software Developer
%\hline
&3 & P2 &W  & 11-20 &  Analyst & Digital & S \\% & 4\\ %Systems Analyst
& &   &  &   &    	& payment					&   \\% & 4\\ %Software 
%\hline 
&4& P2 &C  & 3-5 &    Product & Digital & S \\% & 2\\ % Product Project Manager
&&&&  &	Manager			& payment 				&  \\% &16\\ 
\hline 
\multirow{14}{*}{\rotatebox[origin=c]{90}{Main - practitioners}}
&5&P3 &W & 11-20 &	Developer		& Media						& XL \\% & 6 \\ %Senior Software Developer
%\hline
&6&P3&C	& \textgreater 20 &	Developer		& Media						& XL \\% & 11\\ %Senior Software Developer
%\hline
&7&P4&W	& 11-20 &	Developer		& Marketing& M \\% & 9\\ %Senior Software Developer
%\hline
&8&P4&C	& 11-20 &	 Developer		& Marketing& M \\% & 19\\ %Senior Software Developer
%\hline
&9&P5&W	& 11-20 &	CTO								& Finance					& M \\% &11\\
%\hline
&10&P5&C& \textgreater 20 &	Architect						& Finance					& M \\% &8\\
%\hline
&11&P6&W	& 6-10	 &	Architect 	& Audio 		& L \\% 11\\ %Software Engineer/Architect %Signal processing
%\hline
&12&P6&C& \textgreater 20 &	Sys. Eng.				& Audio	& L \\% &12\\
%\hline
&13&P7&W	& 11-20 &	Architect				& Retail					& XL \\% &9\\ %Solution Architect
%\hline
&14&P7&C & 11-20 &	Architect				& Retail					& XL \\% &6\\ %Solution Architect
%\hline

&15&P8&W&  3-5	 &	Developer 	& Education		& L \\% &5\\ %Software Developer
%\hline 
&16&P8&C& 3-5 &	Developer				& Education		& L \\% &12\\ %Software Developer
%\hline
&17&P9&W& 11-20 &	Product			& Government					& S \\% &16\\ 
&&&&  &	Manager			& 				&  \\% &16\\ 
%Product Project Manager
%\hline
&18&P9&C& 11-20 &	Coordinator				& Government					& S \\%Systems Development Coordinator
&19&P10&W& 1-2 &	Developer				& Education					& L \\% &16\\ %Product Project Manager
%\hline
&20&P10&C& 1-2 &	Developer				& Education					& L \\% &18\\ %Systems Analyst	
\hline
\multirow{14}{*}{\rotatebox[origin=c]{90}{Main - students}}
&	21	&	S1	&	W	&	-	&	-	&	-	&	-	\\
&	22	&	S1	&	C	&	1-2	&	Developer	&	Mobile	&	XL	\\ %SW Engineer/Developer
&	23	&	S2	&	W	&	\textless 1	&	Developer	&	Electronics	&	XL	\\ %Python developer
&	24	&	S2	&	C		&	3-5	&	Developer	&	Utilities	&	L	\\ %Software Engineer
&	25	&	S3	&	W	&	1-2		&	Developer	&	Insurance&	S	\\ %Fullstack 
&	26	&	S3	&	C	&	1-2		&	Developer	&	ERP	&	M	\\ %Junior 
&	27	&	S4	&	W	&	\textless 1		&	Developer	&	Ecommerce	&	S	\\ %Frontend Developer
&	28	&	S4	&	C	&	1-2		&	Developer	&	Security	&	XL\\ %Cybersecurity T-SQL Developer
&	29	&	S5	&	W	&	-	&	-	&	-	&	-	\\
&	30	&	S5	&	C	&	-	&	-	&	-	&	-	\\
&	31	&	S6	&	W	&	1-2	&	Tester	&	\mbox{Web dev.}&	S	\\ %Data Procesing website	
&	32	&	S6	&	C	&	1-2		&	Developer	&	ERP	&	M	\\
&	33	&	S7	&	W	&	1-2		&	Pre-sales	&	Sales	&	XL	\\
&	34	&	S7	&	C	&	\textless 1		&	Developer&	Energetics 	&	L	\\
&	35	&	S9	&	W	&	3-5	&	Developer	&	Security	&	L	\\
&	36	&	S8	&	C	&	1-2	&	DevOps	&	Earth	&	M	\\
&		&		&		&		&		&	Observation	&	\\
\hline
\end{tabular}
}
\vspace*{-1.5\baselineskip}
\end{table}


    \subsection{Debiasing} 
%rr add
Debiasing refers to the process of identifying and mitigating cognitive biases that may affect decision-making.
    Possible types of debiasing treatments can be described using the levels proposed by Fischhoff \cite{fischhoff1982debiasing}. A- and B-level treatments consist of informing practitioners about cognitive biases (A) and how they may impact practitioners (B). Levels C and D require more organizational resources: a C-level treatment requires giving personalized feedback to each debiased person, and the D-level requires extensive long-term training.

    %Debiasing has been a topic of interest in many research domains, e.g., economics \cite{jugnandan2023towards}~\cite{pavicevic2021role}. I
    In the field of software engineering, two notable C-level debiasing treatments were reportedly successful. Firstly,  rationalizing development time estimates by Shepperd et al.~\cite{Shepperd2018}, where researchers purposefully anchored software developers on pessimistic estimates to counter over-optimistic predictions.     
    Secondly, in ADM, in their short paper, Borowa et al. \cite{Borowa2022} reported achieving a debiasing effect on a group of student participants. However, their experiment had several weaknesses that our study avoids:
    \begin{enumerate}
        \item The experiment employed the pretest-posttest design, which is known for numerous threats to validity~\cite{Knapp2016}. For example, the debiasing effect may have not resulted from the workshop but from the student participants' learning while performing the pretest task.
        \item The participants were all students\color{black}, making it unclear whether the results would be the same for \color{black} experienced practitioners.
        \item The participants performed different tasks before and after the workshop, so comparing these may not have been reliable. 
        \color{black}
        For example, if designing one architecture was more difficult than the other, participants' bias occurrence count could be increased by this added difficulty. 
        \color{black}
        \item The participants performed purely theoretical tasks by designing architectures for domains that they may not have had any prior knowledge about. This may have caused mistakes due to their lack of domain knowledge.
        % \item The experiment was three hours long with no breaks. As such, the participants may have been impacted by fatigue during the posttest task.
        \color{black}
        \item The debiasing effect was achieved by improving the amount of non-biased arguments, \textbf{but, despite the authors' efforts, the bias occurrence was not reduced}. We measured the bias occurrences using a similar method to Borowa et al.~\cite{Borowa2022} and did achieve a bias reduction. 
        \color{black}
    \end{enumerate}
    
    
    % \textbf{the debiased students were not impacted by biases less often, but instead used more non-biased arguments in their reasoning}. Additionally, the students performed \textbf{purely theoretical tasks} and their performance was compared through them performing \textbf{two different tasks before and after a workshop}.
    % However, neither this nor any other recent study ascertained whether the effect on experienced practitioners, and real-life architectures, would be similar. Through our study, we have strived to empirically verify whether a debiasing workshop is effective for experienced practitioners, working on existing architectures. 
    
    %This verification is crucial because of the ongoing debate in the software engineering research community about whether students are a reliable source of data \cite{galster2023empirical, baltes2022sampling}.

        
    \section{Method}
    \label{sec:Method}
    We ran an experiment built on top of an existing debiasing workshop structure~\cite{Borowa2022}.
    However, we improved the workshop and experiment, as explained in the following subsections.
    
    The study was divided into two phases: (1) a pilot phase with two experiments, i.e., two pairs of practitioners from one company, and (2) the main experiment, i.e., with eight pairs of students and eight pairs of practitioners from seven different companies.
    While we present the pilot's results in this paper, the study plan underwent major changes between these phases, which led us to exclude the pilot's data from quantitative calculations (i.e., code count averages and p-values).
    The differences between the pilot and the main study are also explained in the respective subsections.

\begin{figure*}[h]
\centering
\includegraphics[width=0.8\textwidth]{Method2.drawio.pdf}
\caption{Study steps}
\label{fig:experiment_steps}
\end{figure*}
  
\subsection{Sample}
\label{sec:sample}
We recruited two types of participants: students and practitioners. 
\color{black}
We define students as individuals studying to obtain a computer science-related degree who had performed major ADM tasks in an industrial setting. Practitioners, on the other hand, are individuals who have performed major ADM tasks in an industrial setting.
\color{black}
Student pairs discussed projects from their coursework during this study, while practitioners discussed real-life industrial projects. However, it is worth noting that most student participants were also young practitioners with some industrial experience. 
%This diverse sample of participants at various stages of their professional careers was a key factor in examining the workshop's impact on different individuals.

The first author recruited student participants from Warsaw University of Technology through convenience sampling, by advertising the search for participants through their institution's channels. 
To qualify for participation, a pair of students had to have developed a software system from scratch as part of a team during their coursework, i.e., they had to have taken part in requirements gathering, architectural design, implementation, and testing of a system with at least three major components. Sixteen students applied and met these criteria.
\color{black}
Students could obtain extra credit in a software architecture course by participating, though only two students used this opportunity. %, and the rest did not receive any rewards.
\color{black}
%With the exception of 3 undergraduate students, all students had some commercial experience in software development. 


We recruited practitioner participants through convenience sampling from our network in three countries (Brazil, Germany, and Poland) \color{black} and offered no rewards to practitioner participants. \color{black} %Brazil, Germany, and Poland.
This was the most appropriate method of acquiring participants for two reasons:
\begin{itemize}
    \item The experiment required participants to discuss existing systems used by their companies. As such, participants needed to trust the researchers to handle sensitive data appropriately, which allowed us to face real-world designs instead of theoretical examples.
    \item %The experiment was designed to show if practitioners participating in the workshop would make less biased decisions. To increase the validity of our research in this regard, w
    We needed to find pairs of practitioners who had worked together on developing the same system and who had similar roles. Explaining the nuances to the participants required personal discussion with one of the researchers.
\end{itemize}

The participants were informed beforehand that the experiment was related to improving ADM and about the overall steps of the experiment. However, they were not informed \color{black} about cognitive biases, had no access to the workshop materials, and did not know the Step 4 task of the experiment. No pair of participants had the chance to contact another pair to disclose such information. 
\color{black}
% We finished the data gathering for the main experiment after seven participant pairs participated.  In the domain of usability testing, where think-aloud protocol studies are widely used, a group of five participants is considered big enough to perform a study \cite{turner2006determining}. However, we planned to use the non-parametric Wilcoxon Signed Rank Test \cite{wilcoxon1992individual} to calculate p-values of code counts during the analysis. Therefore, we opted for a slightly larger amount of participants since this method is only suited to yield results for at least seven sample pairs. 
Table \ref{tab:participants} presents basic information about the participants. %, such as experience, role, and domain. 


\subsection{The experiment}
\label{sec:experiment}
\color{black}
We conducted an experiment based on an existing debiasing workshop structure~\cite{Borowa2022} with several modifications.
\color{black}
% In this study, we ran an experiment built on top of an existing debiasing workshop structure~\cite{Borowa2022}. % which was effective on student participants in countering the optimism, anchoring, and confirmation biases.
% However, we made a set of changes. %differences on four specific points.
\color{black}
First, we replaced a debiasing practice that was having one person monitor the discussion for statements influenced by \textit{confirmation-bias}, since the previous workshop's authors admit that their participants were rarely able to use this technique at all~\cite{Borowa2022}. 
\color{black}
%However, \textbf{we changed one of the debiasing practices taught during the workshop}, which was the least effective, i.e., having one person monitor a discussion for \textit{confirmation-bias}-influenced statements. 
Instead, our participants were taught to list multiple solution options for their architectural decisions.

\color{black}
Second, 
       we performed an experiment with a matched pair design~\cite{myers2006experimental} instead of the pretest-posttest design. 
       As such, we compared the results for the same architectural task done by matched pairs: (1) where one had attended the workshop (workshop group), and (2) the other had not (control group). 
%This was not the case in~\cite{Borowa2022}, where students performed different tasks before and after the workshop.
By \color{black} contrast, in the experiment of~\cite{Borowa2022}, researchers compared different tasks done before and after the workshop by the same students. 
\color{black}
The matched pair design, combined with recruiting pairs of individuals who had worked on a project together, allowed us to explore \textbf{whether the discussed software systems could have benefited from having an individual participate in the debiasing workshop,} while enhancing internal validity by randomizing each pair's split into workshop/control groups.


% students performed tasks before and after the workshop.
% Third, our participants were all experienced practitioners, as described in Section \ref{sec:sample}.
Third, the architectural designs used in the experiment came from students' coursework projects or real-life cases from practitioners' companies, not hypothetical scenarios.
\color{black}

The experiment itself consisted of (see Figure~\ref{fig:experiment_steps}):
\begin{enumerate}
    \item \textit{Step 1: Discussing existing architecture (up to 30 min.):} Both participants took part in a meeting with a researcher. First, they were asked to fill out a questionnaire to gather demographic data. % (age, experience, role in the company, etc.). 
    Then, they were asked to explain the architecture of a system they had both worked on previously. In the case of student participants, they had to discuss a university project where they performed the whole development process, from gathering requirements through architectural design, implementation, and testing. Practitioner participants discussed real-life industrial projects.    
    Participants drew a simple ``boxes and arrows'' diagram of the architecture and gave the researcher the following information: (1) the overall idea and context of the system, (2) the system's main components, (3) relationships between components, and (4) the technologies used. This way, we ensured that both participants knew the architecture at the same level.
    
    \item \textit{Step 2: Debiasing workshop (around 60 min.):} During this step, the workshop group participant took part in the debiasing workshop, as described in Section \ref{sec:workshop}. The control group participant had a break.
    
    \item \textit{Step 3: Break:} % The work of Borowa et al. \cite{Borowa2022}, as well as other 
    %
    Research on cognitive biases~\cite{kahneman2011think} notes that rational logic-based thinking is physically exhausting. To avoid tiring the workshop participants more than the control participants, there was always a break before the last step of the experiment. The length varied but, was always long enough for at least one meal. 
    
    \item \textit{Step 4: Architecture improvement (up to 30 min.):} \color{black} Participants were asked to identify issues with the Step~1 system's architecture. They were asked to propose possible improvements to resolve these issues. \color{black}  The participants were requested to note the changes on the boxes-and-arrows diagram from Step~1. % where appropriate. % while discussing. 
    
    \textbf{In the pilot run} with two pairs, we allowed participants to discuss together. %these issues and issues' solutions together. %similarly as in the study conducted by Borowa et al. \cite{Borowa2022}. 
    %rr del: However, we noticed that this approach made one of the participants afraid to voice their opinions, probably because the other participant was their superior. % at their company. 
    %rr update:
    However, in one pilot experiment, we noticed that this approach caused a participant to omit their opinions. This was caused by the fact that the other participant, their superior, dominated the discussion. 
    %Therefore, we changed our approach for the main experiment.
    
    %\textbf{In the main experiment}, was performed in the form of a ``think-aloud protocol'' %data gathering 
    %session \cite{Ericsson1993}. 
    \textbf{In the main experiment}, we changed our data gathering approach to a ``think-aloud protocol'' session \cite{Ericsson1993}. 
    This method was suggested by Razavian et al. \cite{Razavian2019} for studying behavioral factors in ADM. Accordingly, the participants proposed their improvements separately     \color{black} 
    (i.e. one after another, in a separate room, with no option to hear each other) in the subsequent sessions, to avoid interferences between participants.
    The participants \color{black} were asked to voice their thoughts, i.e., ``think aloud'', during the whole session. We could not use evaluation methods such as questionnaires or having the participant write down their argumentation since cognitive biases occur during an individual's thought process, and their occurrence may not be visible when the participants have additional time to curate their answers. 
    
    %Before starting this step, participants were tasked with a short ``think aloud'' warm-up task - they had to think aloud while folding a paper plane or building a simple construction from Lego bricks.
\end{enumerate}

The experiment's Step 4 was recorded, with the audio recordings transcribed for further analysis. All the experiment sessions were performed in the native language of the participants by a native-speaking researcher, who subsequently translated the transcript to English to make it accessible to all the authors. We allowed participants to choose between on-site and online participation. Four experiments were performed during a personal meeting (two pilot and two practitioner main), and the rest were held online (all main). 
\color{black}
All the experiments for each pair were performed separately, so no step was performed with many pairs present in the same room at the same time. 
\color{black}
The experiment plan is available as part of the additional material~\cite{additional_material}.

% \begin{figure}[h]
%   \centering
%   \includegraphics[width=\linewidth]{sample-franklin}
%   \caption{1907 Franklin Model D roadster. Photograph by Harris \&
%     Ewing, Inc. [Public domain], via Wikimedia
%     Commons. (\url{https://goo.gl/VLCRBB}).}
%   \Description{A woman and a girl in white dresses sit in an open car.}
% \end{figure}


\begin{table*}
\caption{Coding scheme: adapted from \cite{Borowa2022}}
\label{tab:codes}
\scalebox{0.8}{
\begin{tabular}{|p{.05\textwidth}|p{.15\textwidth}|p{.9\textwidth}|}
\hline
Code & Code meaning & Description\\
\hline
\textbf{Arg}  & \textbf{Argument} & A statement in \textbf{support} of a possible solution alternative. \color{black}E.g.~``MS SQL would be a better choice than Oracle since it is the cheaper solution.''\color{black} \\%, even when no particular rationale is given for the decision.\\
\textbf{Carg} & \textbf{Counterargument} & A statement in \textbf{opposition} of a possible solution alternative. \color{black}E.g.~``This solution requires turning off the access to users during maintenance''.\color{black}\\
\textbf{Anch}	& \textbf{Anchoring}	& A statement suggesting that the participant is impacted by \textbf{anchoring}. \color{black}E.g.~``Django would be the best choice'' when no additional contextual information suggests why Django was chosen or whether alternatives were considered.\color{black}\\
\textbf{Conf}	& \textbf{Confirmation bias}	& A statement suggesting that the participant is impacted by \textbf{confirmation bias}. \color{black}E.g.~``We already decided on a centralized system, so let's leave it like this,'' when issues with this choice were being discussed. \color{black}\\
\textbf{Opt}	& \textbf{Optimism bias}	& A statement suggesting that the participant is impacted by \textbf{optimism bias}. \color{black}E.g.~``I am sure we can easily connect to some external REST API to handle the payment,'' without researching or having prior knowledge about payment systems. \color{black}\\
\textbf{Ddraw}	& \textbf{Decision's drawback}  & Use of the \textbf{anti-anchoring technique}, i.e., a statement where the participant discusses a drawback of the solution alternative. \color{black}E.g.~``Splitting this service into microservices would require additional work on orchestration.''\color{black}\\ % (Drawbacks, unlike risks, always have an impact on the solution, e.g. a particular component's license is expensive.) \\
\textbf{Dmulti}	& \textbf{Decision with multiple alternatives} & Use of the \textbf{anti-confirmation bias and anti-anchoring technique}, i.e., a statement where the participant mentions more than one solution alternative. \color{black}E.g.~``We could consider various free RDBMS: like MySQL, PostgreSQL, Maria DB or even SQLite.''\color{black}\\
\textbf{Drisk}	& \textbf{Decision's risk} & Use of the \textbf{anti-optimism bias technique}, i.e. a statement where the participant discusses a risk associated with a solution alternative. \color{black}E.g.~``This SSH Server is a possible bottleneck in the system and may stop responding due to too many requests.''\color{black}\\ %(Risks, unlike drawbacks, can have an impact on the solution. However, the impact of a risk is determined by its probability of occurrence, which means there is a possibility that the risk may not occur at all. For instance, an online web application could become exceedingly popular and struggle to handle customer traffic.)\\
%\textbf{Dec[No]} & \textbf{Decision} & \textbf{Code} used to mark statements related to the same architectural decision, in case the coder identifies more than one. \\
\hline
\end{tabular}
 }
\end{table*}








\subsection{Debiasing Workshop}
\label{sec:workshop}


%The debiasing workshop was performed similarly to the one designed by Borowa et al. \cite{Borowa2022}, with two main differences: (1) We changed one debiasing technique which proved to be the least effective in the work of Borowa et al. \cite{Borowa2022}; (2) We provided examples of the use of the debiasing techniques during workshop;
Our workshop was an intervention that included not only informing about cognitive biases and the specific effects that they may have on the participants but also trained the participants through a teaching experience with personalized feedback. %(in our case - workshop) 
Therefore, it is a C-level debiasing intervention on Fischoff’s debiasing scale \cite{fischhoff1982debiasing}. %Such an approach proved effective in debiasing software developers to increase the accuracy of effort estimates \cite{Shepperd2018}.

Overall, the workshop comprised (1) a short lecture about cognitive biases and the work of Daniel Kahneman and Amos Tversky in general~\cite{Tversky1974}~\cite{kahneman2011think}, (2) an explanation of how cognitive biases affect ADM in particular, (3)
a design session, during which the participant was supposed to design a solution for a fictional architecture task and make use of the debiasing techniques. 
We taught the following debiasing techniques to the participants:
\begin{enumerate}
\color{black}
\item  \textit{Generating multiple solution options:} This is a new debiasing technique that aims to counter \textit{anchoring} and \textit{confirmation biases}.
 The main impact of both these biases is limiting the scope of considered solution alternatives~\cite{Zalewski2017}, i.e., anchoring makes individuals fixated on one solution idea~\cite{Tang2011} and confirmation bias makes them unlikely to change their initial decision based on evidence~\cite{borowa2021knowledge}. As such, it seems prudent to explicitly list solution alternatives before architects believe that one single solution is the best option before considering any others.

Listing multiple solutions is also a basic reasoning technique suggested by many ADM researchers \cite{Tang2021}~\cite{soliman2021exploring}~\cite{cervantes2016designing} for improved reasoning.
\color{black}

% This bias combination may lead architects to overlook issues with a solution (\textit{confirmation bias}) after deciding on an initial one (\textit{anchoring}).

% % Suggested by GrammarlyGO:
% This technique aims to counter anchoring and confirmation bias. Architects may overlook issues with a solution after deciding on an initial one.

%This technique replaces the one taught by Borowa et al. \cite{Borowa2022} that was the least effective, i.e. having a designated team member monitoring the discussion to avoid the omission of ``unwanted'' solution alternatives.
\item \textit{Listing at least one drawback of the solution alternative:} This technique is targeted towards countering the effects of \textit{anchoring}, since \textit{anchoring} often makes practitioners over-focused on one solution's advantages \cite{Borowa2022}. Unlike risks, drawbacks always impact the solution, e.g., a component's license is expensive. 
\item \textit{Listing at least one risk associated with the solution alternative:} Since individuals are prone to over-optimistic predictions, explicitly discussing risks is targeted towards countering \textit{optimism bias} \cite{Borowa2022}.
Risks, unlike drawbacks, have a probability associated with their occurrence. This means there is a possibility that the risk may not occur at all.
\end{enumerate}

%

The researcher leading the workshop demonstrated the debiasing techniques using examples and actively assisted the participants in the design session when they were struggling with the techniques. %This was done to provide personalized feedback required of a level C debasing intervention \cite{fischhoff1982debiasing}.

%After the workshop, the participants were asked to fill out a short questionnaire containing 10 test questions \cite{additional_material}, which were supposed to assess the knowledge gained during the workshop.








\subsection{Data Analysis}
\label{sec:data_analysis}



For the data analysis procedure, we used the hypothesis coding technique \cite{Saldana2013}, which involved creating a set of codes before the coding process. 
%Since we wanted to prove the debiasing workshops' effectiveness as Borowa et al. \cite{Borowa2022}, we used the same coding scheme. 
The codes that we used are listed in Table \ref{tab:codes}
\color{black}
and were prepared based on the same coding scheme used by Borowa et al.~\cite{Borowa2022}. 
\color{black} 
A coding guide with examples from this study's participants is available online~\cite{additional_material}.

Recordings from the sessions were transcribed and translated to English by the researcher organizing the workshop, then coded by another author and, for the main experiment, the coding was reviewed by an additional author.

% In the case of the pilot phase sessions, the coder was not informed which transcript belonged to the treatment or control groups. In the case of the main experiment, as suggested by Ericsson and Simon\cite{Ericsson1993}, we employed \textbf{context-free coding}.
% This method aims to decrease the possibility of the researchers' bias affecting the study's outcome. We expected that a coder may unconsciously find more bias-related codes in the control group participants and since the code counts are our most important measurements, avoiding this bias was a priority.

% Our context-free coding process was as follows:
% \begin{enumerate}
%     \item The workshop organizer divided the transcripts into segments (one for each architectural decision discussed). This was a natural split since participants usually discussed one possible change to the architecture after another.
%     \item For each pair of participants, the control and workshop groups' segments were randomly ordered. 
%     \item Another researcher coded the segments - without knowledge of whether the particular segment belonged to the treatment or control group participant. Although they did know which segment belonged to which architecture.
%     \item An additional researcher reviewed the coding and discussed changes with the original coder until reaching a consensus on all codes.
%     \item The workshop organizer revealed the participants' information to create a summary of each workshop's coding.
% \end{enumerate}

% Having obtained the code counts containing measured values for each participant, we summarized the following: (1) The sum of all codes for a particular measurement in the control and workshop participants, (2) checked whether the measurements decreased for cognitive biases and bias-impacted statements in each pair, (3) checked if measurements for debiasing techniques and non-biased statements increased. 
% %Context-free coding is not the standard in all current think-aloud protocol research, because contextual information may be lost due to it \cite{CHINGYANG200395}.


% Alternative order and correction suggestion

In the pilot phase, the coder was unaware of which \color{black}participant was the control/workshop one. % belonged to the workshop or control groups. 
The main experiment, as suggested by Ericsson and Simon~\cite{Ericsson1993}, \color{black} employed \textbf{context-free coding} as follows:
\begin{enumerate}
    \item The workshop organizer divided the transcripts into segments (one for each architectural decision). This was a natural split since the participants usually discussed one possible change to the architecture after another.
    \item For each pair of participants, the control and workshop groups' segments were randomly ordered. 
    \item Another researcher coded the segments, without knowing whether a particular segment belonged to the workshop or control group participant. %Although they did know which segment belonged to which architecture.
    \item An additional researcher reviewed the coding and discussed change proposals with the original coder until a consensus was reached on all codes.
    \item The workshop organizer revealed the participants' information to create a summary of each experiment's coding.
\end{enumerate}

      \begin{figure*}[!htb]
            \centering
            \includegraphics[width=0.85\textwidth]{results_overview_table.png}
            \caption{Average amount of codes for control and workshop group for (1) All participants, (2) students, (3) practitioners.}
            \label{fig:codes_avg}
            %\vspace{-5mm}
        \end{figure*}
        
\begin{table*}[!htb]
\centering
\caption{(Non-)Biased Arguments and Counterarguments \tiny{(B -- Biased, NB -- Non-biased)}}
\label{tab:arguments}
\scalebox{0.9}{
\begin{tabular}{{|c|c|p{0.7cm}|p{0.7cm}|p{0.7cm}|p{0.7cm}|p{0.7cm}|p{0.7cm}|p{0.7cm}|p{0.7cm}|p{1.7cm}|p{1.7cm}|p{1.7cm}|}}
\hline
& & \multicolumn{4}{c|}{Control}   & \multicolumn{4}{c|}{Workshop}  & \textbf{Control} & \textbf{Workshop} & \textbf{Difference}\\
& & \multicolumn{2}{c|}{Arguments}& \multicolumn{2}{c|}{Counterarg.} &\multicolumn{2}{c|}{Arguments}& \multicolumn{2}{c|}{Counterarg.} &\textbf{Biased} &\textbf{Biased } &\textbf{Biased }\\
 &Pair& B  &NB  &B &NB &B &NB &B&NB &\textbf{statements [\%]} &\textbf{statements [\%]} &\textbf{statements [pp]}\\
\hline

\multirow{2}{*}{\rotatebox[origin=c]{90}{Pilot}}
	&	P1	&	5	&	2	&	1	&	2	&	4	&	0	&	1	&	6	&	60\%	&	38\%	&	-22	\\
	&	P2	&	8	&	1	&	0	&	1	&	1	&	3	&	0	&	1	&	80\%	&	17\%	&	-63	\\
\hline
\multirow{7}{*}{\rotatebox[origin=c]{90}{Main: practitioner}}
	&	P3	&	12	&	10	&	4	&	1	&	8	&	8	&	2	&	9	&	59\%	&	36\%	&	-24	\\
	&	P4	&	3	&	6	&	1	&	1	&	11	&	5	&	9	&	4	&	36\%	&	67\%	&	30	\\
	&	P5	&	14	&	22	&	1	&	4	&	3	&	20	&	0	&	1	&	37\%	&	11\%	&	-26	\\
	&	P6	&	9	&	10	&	2	&	3	&	5	&	13	&	3	&	7	&	46\%	&	26\%	&	-20	\\
	&	P7	&	24	&	17	&	6	&	4	&	10	&	54	&	3	&	13	&	59\%	&	15\%	&	-43	\\
	&	P8	&	3	&	1	&	0	&	1	&	1	&	10	&	0	&	1	&	60\%	&	8\%	&	-52	\\
	&	P9	&	7	&	9	&	2	&	1	&	3	&	11	&	2	&	4	&	47\%	&	24\%	&	-24	\\
&	P10	&	5	&	3	&	1	&	1	&	2	&	6	&	1	&	1	&	60	\%&	30	\%	&	-30	\\
\hline
\multirow{7}{*}{\rotatebox[origin=c]{90}{Main: student}}
&	S1	&	1	&	3	&	1	&	1	&	1	&	4	&	0	&	8	&	33	\%&	8	\%	&	-25	\\
&	S2	&	1	&	3	&	1	&	1	&	1	&	8	&	0	&	1	&	33	\%&	10	\%	&	-23	\\
&	S3	&	1	&	2	&	2	&	1	&	1	&	15	&	2	&	11	&	50	\%&	10	\%	&	-40	\\
&	S4	&	2	&	1	&	2	&	0	&	1	&	5	&	1	&	3	&	80	\%&	20	\%	&	-60	\\
&	S5	&	3	&	1	&	3	&	0	&	3	&	3	&	3	&	5	&	86	\%&	43	\%	&	-43	\\
&	S6	&	2	&	1	&	0	&	5	&	4	&	4	&	1	&	2	&	25	\%&	45	\%	&	20	\\
&	S7	&	5	&	7	&	2	&	5	&	0	&	11	&	0	&	4	&	37	\%&	0	\%	&	-37	\\
&	S8	&	3	&	2	&	1	&	2	&	0	&	6	&	1	&	7	&	50	\%&	7	\%	&	-43	\\





\hline
%Sum & 72 & 119 & 13 & 8 & 47 & 101 & 18 & 29 &  &  & \\
%\hline
\end{tabular}
}

\vspace*{-1.5\baselineskip}
\end{table*}

\begin{table*}[!htb]
\centering
\caption{Occurrences of Cognitive Biases \tiny{(ANCH -- Anchoring, OPT -- Optimism bias, CONF -- Confirmation bias)}}
\label{tab:biases}
\scalebox{0.9}{
%\begin{tabular}{|c|c|c|c|c|c|c|c|c|c|p{2cm}|}
 \begin{tabular}{|c|c|r|r|r|r|r|r|r|r|r|} %p{2cm}|}
\hline
& & \multicolumn{3}{c|}{Control}   & \multicolumn{3}{c|}{Workshop}  & \textbf{Control} & \textbf{Workshop} & \textbf{Difference}\\
 & Pair &ANCH &CONF  &OPT &ANCH &CONF &OPT &\textbf{Bias sum} &\textbf{Bias sum} &\textbf{Bias sum}\\
\hline
\multirow{2}{*}{\rotatebox[origin=c]{90}{Pilot}}
	&	P1	&	6	&	3	&	1	&	1	&	1	&	1	&	10	&	3	&	-7	\\
	&	P2	&	7	&	3	&	1	&	1	&	0	&	0	&	11	&	1	&	-10	\\
\hline
\multirow{7}{*}{\rotatebox[origin=c]{90}{Main: practitioner}}
	&	P3	&	9	&	7	&	2	&	7	&	4	&	0	&	18	&	11	&	-7	\\
	&	P4	&	3	&	1	&	2	&	12	&	11	&	0	&	6	&	23	&	17	\\
	&	P5	&	10	&	3	&	8	&	1	&	2	&	3	&	21	&	6	&	-15	\\
	&	P6	&	9	&	3	&	0	&	5	&	3	&	1	&	12	&	9	&	-3	\\
	&	P7	&	6	&	21	&	23	&	6	&	6	&	4	&	50	&	16	&	-34	\\
	&	P8	&	1	&	0	&	3	&	0	&	0	&	2	&	4	&	2	&	-2	\\
	&	P9	&	7	&	2	&	0	&	2	&	4	&	2	&	9	&	8	&	-1	\\
 &	P10	&	6	&	0	&	0	&	1	&	0	&	2	&	6	&	3	&	-3	\\
 \hline
\multirow{7}{*}{\rotatebox[origin=c]{90}{Main: student}}
&	S1	&	0	&	1	&	1	&	0	&	1	&	1	&	2	&	2	&	0	\\
&	S2	&	0	&	1	&	1	&	1	&	0	&	0	&	2	&	1	&	-1	\\
&	S3	&	3	&	2	&	3	&	1	&	1	&	1	&	8	&	3	&	-5	\\
&	S4	&	2	&	1	&	1	&	1	&	1	&	1	&	4	&	3	&	-1	\\
&	S5	&	4	&	3	&	1	&	4	&	3	&	1	&	8	&	8	&	0	\\
&	S6	&	2	&	1	&	1	&	2	&	3	&	0	&	4	&	5	&	1	\\
&	S7	&	4	&	4	&	0	&	0	&	0	&	0	&	8	&	0	&	-8	\\
&	S8	&	0	&	2	&	4	&	1	&	0	&	0	&	6	&	1	&	-5	\\


\hline
%Sum & 53 & 31 & 36 & 28 & 23 & 8 & 129 & 59 & \\
%\bottomrule
\end{tabular}
}
\vspace*{-1.5\baselineskip}
\end{table*}

\color{black}We anticipated that a coder might unconsciously try to find codes proving the research hypotheses.
The context-free coding method aims to decrease the possibility of this occurring. \color{black}

%Having obtained the code counts containing measured values for each participant, we summarized the following: 
We evaluated our coding as follows: (1)~We calculated the average of all codes for a particular measurement for each group of participants. (2)~We checked whether the measurements for cognitive biases and bias-impacted statements decreased. (3)~We checked whether the measurements of debiasing techniques and non-biased statements increased. 

In the work of Borowa et al. \cite{Borowa2022}, the authors assumed that if most arguments used while discussing one decision were non-biased, then the decision was non-biased. 
We have not followed this assumption since one argument (biased or non-biased) can be decisive in decision-making. %rs to choose a solution.% despite other arguments. 


In order to determine whether the \color{black} debiasing intervention \color{black} was successful, we defined a set of research hypotheses (H\textsubscript{R}):
\begin{itemize}
    \item In the case of \textbf{values that the workshop strived to decrease} (biased arguments/counterarguments, cognitive biases), the research hypothesis (H\textsubscript{R}) was that the measurements with the workshop (workshop group) were smaller than without it (control group). Our null hypothesis (H\textsubscript{0}) was that the control group measurements were smaller than or equal to the workshop group measurements. %For example, the control group showed less biased arguments than the workshop group.
    \item The opposite was the case with \textbf{values that the workshop tried to increase} (not biased arguments/counter\-ar\-gu\-ments, use of debiasing techniques). \color{black} In this case, \color{black} our research hypothesis (H\textsubscript{R}) was that the measurements with the workshop (workshop group) were bigger than without it (control group). In these cases, the null hypothesis (H\textsubscript{0}) was that the control group measurements were bigger (or equal to) the workshop group measurements. % which would mean, e.g., that the control group showed more use of debiasing techniques than the workshop group. 
\end{itemize}



To examine which results are statistically significant and can be considered as accepted with a high confidence level, we performed the non-parametric Wilcoxon Signed Rank test \cite{wilcoxon1992individual} for each of the measurements and the percentage of biased statements (i.e. arguments and counterarguments). This test is used to measure whether a statistical difference can be observed between small paired (dependent) data samples. We consider the control group and workshop group participant pairs to be dependent since they both shared an understanding of the particular architecture and performed the task of discussing improvements for the same systems.
We accept H\textsubscript{R} as statistically significant if the p-value is smaller than 0.05, i.e., there is a less than 5\% chance that H\textsubscript{R} is false.
%In the case of p-values smaller than 0.1 but greater than 0.05, we cannot accept H\textsubscript{R} as statistically significant. % but we consider the possibility that this may be due to our limited sample. This means there is a considerable probability that further research might still show the respective research hypotheses can be accepted.

%Our approach leads to code counts that are better comparable since, depending on the particularly considered architecture, the number of possible arguments and counterarguments can differ strongly. This effect was visible in the results: for example, pair P8 participants used only 5 (control group)/12(workshop group) arguments and counterarguments, while pair P5 voiced 41 (control group)/ 24 (workshop group) arguments and counterarguments (see Table \ref{tab:arguments}).

%Context-free coding is not the standard in all current think-aloud protocol research, because contextual information may be lost due to it \cite{CHINGYANG200395}.




\section{Results}
    \label{sec:Results}




In this section, we present the results of our analysis. Due to the change in data-gathering methods, the results from the pilot study were not included in any calculations (p-values, overall code sum/average for the control/workshop group), but raw values are presented in this section as well.

    
%In this section, we present our study's results. 
% Similarly to Borowa et al.~\cite{Borowa2022}, we counted the codes for arguments/counterarguments, and biased/unbiased statements, and used debiasing techniques. %However, we did not summarize "Decisions" in the same way. 
% However, in the work of Borowa et al. \cite{Borowa2022}, the authors assumed that if most arguments used while discussing one decision were non-biased, then the decision was non-biased. 
% In our work, we do not follow this assumption since one argument (both biased and non-biased) can be decisive in leading decision-makers to choose a solution despite other arguments. 
% Additionally, such measurement simply results from counting biased/non-biased arguments/counterarguments, which makes it unnecessary.

% In the case of \textbf{values that the workshop strived to decrease} (biased arguments/counterarguments, cognitive biases), the research hypothesis (H\textsubscript{R}) was that the measurements with the workshop (workshop group) were smaller than without it (control group). 
% Our null hypothesis (H\textsubscript{0}) was that the control group measurements were smaller than or equal to the treatment group measurements. 
%For example, the control group showed less biased arguments than the workshop group.

% The opposite was the case with \textbf{values that the workshop tried to increase} (not biased arguments/counter\-ar\-gu\-ments, use of debiasing techniques).

%where our research hypothesis (H\textsubscript{R}) was that the measurements with the workshop (workshop group) were bigger than without it (control group). 
%In these cases, the null hypothesis (H\textsubscript{0}) was that the control group measurements were bigger (or equal to) the treatment group measurements, which would mean, e.g., that the control group showed more use of debiasing techniques than the workshop group. 
% Figure \ref{fig:codes_avg} showcases the average code counts for each measurement. %Additionally, in Table \ref{tab:decisions}, we included how many decisions each participant chose to discuss.



% To examine which results are statistically significant, we performed the non-parametric Wilcoxon Signed Rank test \cite{wilcoxon1992individual} for each of the measurements presented in this section. This parametric test is used to measure if a statistical difference can be observed between small paired (dependent) data samples. We consider the control group and workshop group participants to be dependent since they both shared an understanding of the particular architecture and performed the task of discussing improvements for the same systems. 

%This is contrary to Borowa et al.~\cite{Borowa2022}, which compared the same groups of students before and after a workshop, whose before and after tasks were different. 
%Our approach leads to results that are more comparable, since the number of decisions, and as such arguments and counterarguments to be considered (see Table \ref{tab:arguments}) strongly differs depending on the particularly considered architecture.

% Our approach leads to code counts that are better comparable since, depending on the particularly considered architecture, the number of possible arguments and counterarguments can differ strongly. This effect was visible in the results: for example, pair P8 participants used only 5 (control group)/12(workshop group) arguments and counterarguments, while pair P5 voiced 41 (control group)/ 24 (workshop group) arguments and counterarguments (see Table \ref{tab:arguments}).

Figure \ref{fig:codes_avg} showcases the average code counts for each measurement totaled across all participants, \color{black}i.e. for each code, we summed the number of occurrences in the transcript for each participant and calculated the averages for participant groups. \color{black}When considering only the code averages, all measured values changed in accordance with the research hypothesis H\textsubscript{R}. That is: (1) the amount of arguments and counterarguments increased, (2) the amount of biased arguments and counterarguments decreased, (3) the amount of non-biased arguments and counterarguments increased, (4) the amount of bias occurrences decreased, (5) the amount of debiasing techniques use increased.

\subsection{\color{black}Statistical Significance\color{black}}

        \begin{table}[!htb]
\caption{p-values for each measurement \\ \tiny{W-workshop group / C-Control group, \\
AVG - Average of code sums, \\
H\textsubscript{R} - Research hypothesis\\
}}
\label{tab:pvalues}
\center
\scalebox{0.9}{
\begin{tabular}{{|p{2.5cm}|p{1cm}|p{1cm}|p{1cm}|p{0.8cm}|}}
\hline
Measurement	&	C-AVG & W-AVG & p-value	& (H\textsubscript{R})
 \\
\hline
% Dec	&	0.1355	&	Before <= After	&	After > Before	\\
Arg								& 12.13 		& 14.06      	&	0.1619 				&	W \textgreater C	\\
\textbf{Carg	}			            & 3.25 		& 7.28      	& \textbf{0.0022	}	&	W \textgreater C	\\
\hline           			        	           	
\textbf{Anch	}						& 4.13 		& 2.75       	&	\textbf{0.0414	}&	W \textless C	\\
Conf							& 3.25 		& 2.24       	&	0.2062				&	W \textless C	\\
\textbf{Opt	}				            & 3.13 		& 1.13        	&	\textbf{0.0404	}&	W \textless C	\\
\textbf{Biases sum	}					& 10.50 		& 6.31      	& \textbf{0.0108	}			&	W \textless C	\\
\hline          			        	           	
\textbf{Biased Args	}					& 5.94 		& 3.38       	&	\textbf{0.0205}			&	W \textless C	\\
Biased Cargs					&  1.81		& 1.71       	&	0.1482					&	W \textless C	\\
\textbf{Non-biased Args	}				& 6.13 		& 11.44 		& \textbf{0.0016	}		&	W \textgreater C	\\
\textbf{Non-biased Cargs}		& 1.94 		& 5.06      	&	\textbf{0.0057}	&	W \textgreater C	\\
\textbf{\% Biased statements}			& 49,90\% 	& 23,13\% 	&		\textbf{0.0008}			&	W \textless C	\\
\hline                  	            
\textbf{Ddraw	}						& 2.75 		& 5.00      	&	\textbf{0.0123}	&	W \textgreater C	\\
\textbf{Dmulti	}				& 1.75 		& 3.81      	& \textbf{0.0018}		&	W \textgreater C	\\
Drisk							& 2.13 		& 3.25      	&	0.0692				&	W \textgreater C	\\
\textbf{Techniques use sum	}			& 10.57 		& 14.57      	&	\textbf{0.0038	}			&	W \textgreater C	\\
% \% of not biased args/cargs	&	0.1179	&	Control <= Workshop	&	Workshop > Control	\\
% \% not biased args	&	0.3438	&	Control <= Workshop	&	Workshop > Control	\\
% \% not biased cargs	&	0.05742	&	Control <= Workshop	&	Workshop > Control	\\

\hline

\end{tabular}
}
\vspace*{-0.5\baselineskip}
\end{table}

The p-values calculated using the Wilcoxon Signed Rank test \cite{wilcoxon1992individual} are presented in Table~\ref{tab:pvalues}. We calculated the p-values for all measurements that we coded and the percentage of biased statements. Most observed measurement changes were statistically significant, with the exception of (1) increased argument amount, (2) decreased amount of biased counterarguments, (3) decreased amount of confirmation bias occurrences, and (4) increased amount of the ``listing risks'' technique uses.

%TECHNIQUES CUT TABLE
% \begin{center}
% \begin{table*}
% \centering
% \caption{Use of Debiasing techniques\\ \tiny{(DRAW -- Stating drawback, MULTI -- Listing solution alternatives , RISK -- Stating risks) }}
% \label{tab:techniques}
% \begin{tabular}{|c|c|c|c|c|c|c|c|c|c|c|c|}
% \hline
% & & \multicolumn{3}{c|}{Control}   & \multicolumn{3}{c|}{Workshop}  & \textbf{Control} & \textbf{Workshop} & \textbf{Difference}\\
% & Pair &DRAW &MULTI &RISK &DRAW &MULTI &RISK &\textbf{Techniques} &\textbf{Techniques} &\textbf{Techniques} \\
% \hline
% \multirow{2}{*}{\rotatebox[origin=c]{90}{Pilot}}
% 	&	P1	&	2	&	0	&	2	&	6	&	1	&	2	&	4	&	9	&	5	\\
% 	&	P2	&	1	&	0	&	0	&	1	&	0	&	1	&	1	&	2	&	1	\\
% \hline
% \multirow{7}{*}{\rotatebox[origin=c]{90}{Main: practitioner}}
% 	&	P3	&	4	&	0	&	2	&	5	&	3	&	2	&	6	&	10	&	4	\\
% 	&	P4	&	2	&	0	&	2	&	2	&	5	&	0	&	4	&	7	&	3	\\
% 	&	P5	&	8	&	5	&	11	&	5	&	5	&	7	&	24	&	17	&	-7	\\
% 	&	P6	&	0	&	1	&	0	&	5	&	3	&	2	&	1	&	10	&	9	\\
% 	&	P7	&	9	&	9	&	12	&	13	&	9	&	20	&	30	&	42	&	12	\\
% 	&	P8	&	2	&	0	&	1	&	5	&	1	&	2	&	3	&	8	&	5	\\
% 	&	P9	&	4	&	0	&	2	&	4	&	3	&	1	&	6	&	8	&	2	\\
%  &	P10	&	2	&	2	&	0	&	7	&	6	&	1	&	4	&	14	&	10	\\
%  \hline
%  \multirow{7}{*}{\rotatebox[origin=c]{90}{Main: student}}
% &	S1	&	0	&	0	&	0	&	5	&	2	&	1	&	0	&	8	&	8	\\
% &	S2	&	3	&	1	&	2	&	3	&	4	&	1	&	6	&	8	&	2	\\
% &	S3	&	0	&	1	&	0	&	11	&	8	&	5	&	1	&	24	&	23	\\
% &	S4	&	0	&	3	&	0	&	1	&	2	&	2	&	3	&	5	&	2	\\
% &	S5	&	1	&	2	&	0	&	1	&	2	&	0	&	3	&	3	&	0	\\
% &	S6	&	5	&	0	&	1	&	2	&	0	&	1	&	6	&	3	&	-3	\\
% &	S7	&	0	&	3	&	1	&	5	&	5	&	3	&	4	&	13	&	9	\\
% &	S8	&	4	&	1	&	0	&	6	&	3	&	4	&	5	&	13	&	8	\\


% \hline
% %Sum & 30 & 15 & 30 & 34 & 26 & 28 & 75 & 85 & 10 \\
% %\bottomrule
% \end{tabular}
% \vspace*{-0.5\baselineskip}
% \end{table*}

% \end{center}


% \begin{center}
    
%DECISIONS CUT TABLE
% \begin{table*}
% \centering
% \caption{Decisions discussed}
% \label{tab:decisions}
% \begin{tabular}{|l|l|l|l|l|l|l|l|l|l|l|l|l|l|l|l|l|l|l|l|}
% \hline
% & \multicolumn{10}{c|}{Pair number} & \multicolumn{8}{c|}{Pair number}\\
% & P1 &P2 &P3 &P4 &P5 &P6 &P7 &P8 &P9 & P10 &S1 & S2 & S3 & S4 & S5 & S6& S7& S8\\
% \hline
% Control &	4	&	4	&	6	&	9	&	11	&	11	&	9	&	5	&	16	&	10	&	4	&	4	&	9	&	5	&	9	&	6	&	10	&	7	\\
% Workshop	&5	&	2	&	11	&	19	&	8	&	12	&	6	&	12	&	18	&	10	&	11	&	7	&	13	&	7	&	8	&	9	&	8	&	7	\\
% \hline
% Difference	&1	&	-2	&	5	&	10	&	-3	&	1	&	-3	&	7	&	2	&	0	&	7	&	3	&	4	&	2	&	-1	&	3	&	-2	&	0	\\
% \hline
% %Sum & 30 & 15 & 30 & 34 & 26 & 28 & 75 & 85 & 10 \\
% %\bottomrule
% \end{tabular}
% \vspace*{-1.5\baselineskip}
% \end{table*}

% \end{center}

\subsection{Arguments and Counterarguments} 
\label{sec:Arguments}

      % \begin{figure}%[H]
      %       \centering
      %       \includegraphics[width=0.45\textwidth]{img/Arg_Carg.png}
      %       \caption{(Non-)Biased Arguments \textit{(Args) }and Counterarguments \textit{(Cargs)}}
      %       \label{fig:Arg_Carg}
      %       %\vspace{-5mm}
      %   \end{figure}


\color{black}We coded as arguments the statements that supported choosing a particular architectural solution, e.g. Participant 6 said ``(...) what is special about this architecture is that everything is dynamically configurable.''
Counterarguments are statements that give reasons against choosing a solution, e.g. Participant 2 notes price ``(...) AWS itself is another third party service that is paid as well.'' \color{black}
Table \ref{tab:arguments} presents the numbers of arguments and counterarguments for each pair of participants. %scared guy and lazy cto 
When combined with Figure \ref{fig:codes_avg}, showing the average of each type of statement in the control and workshop groups, it allows us to make the following observations:
\begin{itemize}
    \item The number of arguments and counterarguments used was higher for workshop participants.
    \item Workshop participants used fewer biased arguments and counterarguments overall.
    \item Practitioners used more arguments than students in both workshop and control groups. 
    \item The number of practitioners' biased counterarguments slightly increased after the workshop.
    
\end{itemize}

Additionally, based on the data shown in Table~\ref{tab:pvalues}, we can distinguish a set of statistically significant results related to arguments and counterarguments. \textbf{The following differences between participant pairs were statistically significant: (1) increased counterargument amount, (2) increased non-biased statements amount for both arguments and counterarguments, (3) decreased amount of biased arguments.} However, the increase in arguments amount in favor of a solution was not significant, nor was the decrease in biased counterargument amount.  
% Additionally, decreasing the biased counterargument amount was achieved on student participants but not on practitioners.
Additionally, when looking at the individual values per pair, two participants, one student (pair S6), and one practitioner (pair P4), used more biased statements, despite the workshop's intervention.

% Unlike Borowa et al. \cite{Borowa2022}, we did not find a statistically significant decrease in the percentage of biased statements or the a statistically significant increase in the number of non-biased arguments. However, this may be due to our limited sample. Since only in one case did the workshop participant exhibit a higher percentage of biased statements than the control group participant (pair P4) such visible effect did occur despite not being statistically significant.
% However, \textbf{the increase of non-biased counterarguments was significant} (p-value = 0.0449).
% An interesting point is that \textbf{the number of biased arguments also decreased notably}. While this decrease is not statistically significant to the 5\% significance level, this may be due to our limited sample (p-value = 0.0739).




\subsection{\color{black}Cognitive Biases\color{black}}
\label{sec:CognitiveBiases}

\color{black}We coded biased statements using three codes. We identified anchoring in arguments favoring solutions without factual data or solution alternatives, e.g., Participant 14 said the following, without providing additional rationale: ``I would move all this work on data, manipulation of this data, certainly, to the server.'' 
Second, we used the confirmation bias code when participants avoided discussing improvements despite identifying problems; Participant 7 remarked: ``Even if we were to talk about splitting up the middleware or converting it to microservices or other structures, I wouldn't do any of that because that's what [colleague] and I have defined as the actual system.''
Finally, optimism bias was coded for overly optimistic predictions, e.g., Participant 36 proposed parallel processing, stating ``Surely this would also speed up the whole process,'' without considering communication overheads. 
\color{black}

Table \ref{tab:biases} presents the number of biased statements from each participant pair. Figure \ref{fig:codes_avg} summarizes of average bias occurrences in each of the control and workshop groups.
% While Figure \ref{fig:codes_avg} shows the overall sum of bias occurrences in each of the control and workshop groups, Table \ref{tab:biases} presents the number of bias occurrences for each participant pair.
Overall, the workshop group participants were less impacted by each of the researched biases. \textbf{The results were statistically significant (see Table~\ref{tab:pvalues}) in the cases of decreasing anchoring and optimism bias occurrences.}  %However,

However, two differences between practitioners and students are noticeable: \textbf{(1) the average bias occurrence amount (both control and workshop) was higher in the case of practitioners}, and \textbf{(2) decreased bias occurrence was higher in practitioners than students}. Overall, the occurrence of biases of three student pairs (S1, S2, S5) did not decrease, which was the case for only one practitioner pair (P4).

%While these overall decreases can not be considered statistically significant to the 5\% level, they are noticeable in almost every participant pair for every researched bias. The one sole exception to this is pair P4.

% In Borowa et al.'s research \cite{Borowa2022}, the participants of the workshop did not exhibit decreased susceptibility to cognitive biases at all. This means that such an effect was achieved for the first time in this study.


%In the case of our study the decrease may not have been statistically significant, but it is noticeable in almost each participant pair for each researched bias. 
%Such results may suggest that our not finding statistical significance may simply result from a relatively low sample.


% \begin{figure}[h]
% \centering
% \includegraphics[width=8cm]{img/Biases.png}
% %\caption{Comparison of cognitive bias occurrences in the workshop and control groups}
% \caption{Occurrences of Cognitive Bias}

% \label{fig:biases}
% \end{figure}






\subsection{\color{black}Debiasing Techniques\color{black}}
\label{sec:DebiasinTechniques}

\color{black}\color{black}
% Table \ref{tab:techniques} shows how many debiasing techniques were used by each participant, while 

Figure \ref{fig:codes_avg} shows the usage of each technique in the workshop and control groups.
Overall, \textbf{the workshop participants used each of the three techniques more frequently than the control group participants}. This was statistically significant (see Table~\ref{tab:pvalues}) in the cases of ``discussing drawbacks'' and ``listing multiple solutions'', but not for the ``discussing risks'' technique.
Two participants' use of debiasing techniques did not improve after the workshop, one practitioner (pair P5) and one student (pair S6).
\color{black}The code count for each participant's use of debiasing techniques is part of the additional material~\cite{additional_material}.\color{black}


% this was not the case only for pair P5. 
% Additionally, the \textbf{``listing multiple options'' technique was used significantly more (p-value = 0.029)} by workshop participants. 
%However, unlike Borowa et al. \cite{Borowa2022}, we did not find a significant increase in the overall debiasing technique's usage after the workshop. 



\subsection{Decisions}

\color{black}\color{black}
% Table \ref{tab:decisions} showcases how many architectural decisions were discussed by each participant. 

We did not attempt to influence the decision amount through this experiment, so no p-values were calculated. \color{black}However, during the experiment, we noticed that the amount of discussed decisions varied, so we decided to explore whether decision count could possibly be linked to cognitive bias susceptibility. 
In most cases (11 out of 18), the workshop participants discussed more architectural decisions than their control group collaborators. Pair P4 was an outlier in decision count, with the workshop participant discussing 19 decisions against 9 discussed by the control participant, which was the biggest difference of all the pairs. This pair is the only practitioner pair where the workshop participant's bias occurrences and percentage of biased statements were higher than their control group counterparts. %This happened despite the workshop participant's higher use of debiasing techniques. 
Detailed decision count data is presented in the additional material~\cite{additional_material}.
\color{black}



% \begin{table}[!ht]
% \caption{Decisions}
% \label{tab:decisions}
% \begin{tabular}{{|c|c|c|p{2cm}|}}
% \toprule
% Pair& Control  &Treatment  &Decision count increase \\
% \midrule
% P1 &4    &5   &\textbf{1} \\
% P2 &4    &2  &\textbf{-2} \\
% P3 &6   &11  &\textbf{5} \\
% P4 &9    &19   &\textbf{10} \\
% P5 &11   &8  &\textbf{-3} \\
% P6 &11    &12  &\textbf{1} \\
% P7 &9   &6  &\textbf{-3} \\
% \bottomrule
% \end{tabular}
% \end{table}


      % \begin{figure}%[H]
      %       \centering
      %       \includegraphics[width=0.45\textwidth]{img/Techniques.png}
      %       \caption{Use of Debiasing Techniques}
      %       \label{fig:techniques}
      %       %\vspace{-5mm}
      %   \end{figure}



% \subsection{After workshop test}

% \begin{table} [!ht]
% \caption{After workshop test results}
% \label{tab:after_workshop_quiz}
% \begin{tabular}{|c|c|c|}
% \toprule
% Participant No.& Pair & Score [points]\\
% \midrule
% 1& P1 & 4 \\
% 3& P2 & 2 \\
% 5 & P3 & 5\\
% 7& P4 &5\\
% 9& P5 &7\\
% 11& P6 &6\\
% 13& P7 &5\\
% \bottomrule
% \end{tabular}
% \end{table}
% Table \ref{tab:after_workshop_quiz} contains the result, on a 0-10 scale, from a quiz given to the treatment group participants at the end of the workshop. We found no notable relationship between test scores and the debiasing workshop's outcome.


% \subsection{Correlations with debiasing techniques}
% \label{sec:Correlations}
% If our experiment was successful, three subsequent effects could occur:
% \begin{enumerate}
%     \item The workshop might lead to the usage of debiasing techniques.
%     \item The debiasing techniques might reduce the impact of biases.
%     \item The workshop may directly (through learning about biases) reduce the impact of biases. This may occur even if the debiasing techniques do not noticeably impact the biases.
% \end{enumerate}
% To separate these effects from one another, we analyzed the effect of debiasing techniques' use on the occurrence of biases.
% For this, we calculated the Pearson Correlation Coefficient (R) between the following measurements: 
% \begin{itemize}
%     \item Use of debiasing techniques versus biased / non-biased arguments/counterarguments
%     \item Use of debiasing techniques versus cognitive biases
% \end{itemize}
% The effect size is interpreted as medium if it is higher than at 0.3 and as strong if higher than 0.6.

% Table \ref{tab:correlations_techniques_args} showcases the correlations that we found between debiasing technique use and argument types. 

% %% This is from my previous paper. I'm not sure if you can even measure the correlation without having the significance first.
% %We used the $\phi$-coefficient of Pearson's $\tilde{\chi}^2$-correlation \cite{Cohen1988,Pearson1900} for dichotomous variables and assumed a significant correlation from a significance level of 0.05, i.e., 5\%.
% %The effect size is interpreted as medium if it is higher than at 0.3 and as strong if higher than 0.6.

% \begin{table*} % [!ht]
% \caption{Correlations between \textit{debiasing technique use} and \textit{argument types}\\ \small{ARG -- arguments, CARG -- counterarguments,   B -- Biased, N -- Non-biased}}
% \label{tab:correlations_techniques_args}
% \begin{tabular}{|p{1.3cm}|p{2cm}|p{1.5cm}|p{1.5cm}|p{1.5cm}|p{1.5cm}|p{1.5cm}|p{1.5cm}|}
% \toprule
% & & & & \multicolumn{2}{c|}{Biased}   & \multicolumn{2}{c|}{Non-biased}  \\
% Participant No.	& Techniques use	&	ARG	&	CARG	&	ARG	&	CARG	&	ARG	&	CARG \\
% \midrule
% 1	&	4	&	7	&	3	&	5	&	1	&	2	&	2	\\
% 2	&	9	&	5	&	7	&	4	&	1	&	0	&	6	\\
% 3	&	1	&	13	&	1	&	8	&	0	&	1	&	1	\\
% 4	&	2	&	4	&	1	&	1	&	0	&	3	&	1	\\
% 5	&	6	&	22	&	5	&	12	&	4	&	10	&	1	\\
% 6	&	10	&	16	&	11	&	8	&	2	&	8	&	9	\\
% 7	&	4	&	9	&	2	&	3	&	1	&	6	&	1	\\
% 8	&	7	&	16	&	13	&	11	&	9	&	5	&	4	\\
% 9	&	32	&	84	&	1	&	14	&	1	&	70	&	0	\\
% 10	&	17	&	27	&	1	&	5	&	0	&	20	&	1	\\
% 11	&	1	&	19	&	5	&	9	&	2	&	10	&	3	\\
% 12	&	10	&	18	&	9	&	5	&	3	&	13	&	7	\\
% 13	&	27	&	41	&	6	&	21	&	4	&	20	&	0	\\
% 14	&	33	&	89	&	4	&	13	&	3	&	52	&	1	\\

% \midrule
% 	\multicolumn{2}{|l|}{Pearson Correlation Coefficient (R)}	&	0.9107	&	-0.0847	&	0.6551	&	0.0977	&	0.8796	&	-0.2792	\\
%  \midrule
% 	\multicolumn{2}{|l|}{Correlation type}	&	Strong positive	&	Weak negative	&	Strong positive	&	Weak positive	&	Strong positive	&	Weak negative	\\

% \bottomrule
% \end{tabular}
% \end{table*}


% We found strong positive correlations (R > 0.5) between the overall number of debiasing techniques and the number of pro-solution arguments - both biased and not biased. This was not the case for counterarguments.

% \begin{table*} [!ht]
% \caption{Correlations between \underline{debiasing technique use} and \underline{biased statements}}
% \label{tab:correlations_techniques_bias}
% \begin{tabular}{|p{1.3cm}|p{2cm}|c|c|c|c|}
% \toprule
% Participant No.	&	Techniques sum	&	Anchoring	&	Confirmation	&	Optimism	&	Bias sum	\\

% \midrule
% 1	&	4	&	6	&	3	&	1	&	10	\\
% 2	&	9	&	1	&	1	&	1	&	3	\\
% 3	&	1	&	7	&	3	&	1	&	11	\\
% 4	&	2	&	1	&	0	&	0	&	1	\\
% 5	&	6	&	9	&	7	&	2	&	18	\\
% 6	&	10	&	7	&	4	&	0	&	11	\\
% 7	&	4	&	3	&	1	&	2	&	6	\\
% 8	&	7	&	12	&	11	&	0	&	23	\\
% 9	&	32	&	12	&	5	&	7	&	24	\\
% 10	&	17	&	1	&	2	&	3	&	6	\\
% 11	&	1	&	9	&	3	&	0	&	12	\\
% 12	&	10	&	5	&	3	&	1	&	9	\\
% 13	&	27	&	5	&	20	&	23	&	48	\\
% 14	&	33	&	1	&	2	&	3	&	6	\\


% \midrule
% 	\multicolumn{2}{|l|}{Pearson Correlation Coefficient (R)}	&	-0.0519	&	0.3452	&	0.6006	&	0.4317	\\

%  \midrule
% 	\multicolumn{2}{|l|}{Correlation type}&	Weak negative	&	Medium positive	&	Strong positive	&	Medium positive	\\

% \bottomrule
% \end{tabular}
% \end{table*}

% Table \ref{tab:correlations_techniques_bias} presents the correlations between the number of debiasing techniques used and various types of biased statements. We found a strong positive correlation with optimism bias-impacted statements. Additionally, there is a weak positive correlation with confirmation bias-impacted statements and the overall amount of biased statements.
% There was no correlation between the number of debiasing techniques used and the amount of anchoring-impacted statements.

% The implications of these correlations are discussed in Section \ref{sec:Discussion}.


% \begin{table*} [!ht]
% \caption{Correlations testing hypotheses H1, H2, H3 and H4}
% \label{tab:correlations}
% \begin{tabular}{|l|l|l|l|l|}%|p{1.5cm}|p{2cm}|p{2cm}|p{2cm}|}
% \toprule
% Hypothesis & Value 1& Value 2 & Pearson Correlation Coefficient (R) & Strength\\
% \midrule
% H1 & Listing multiple solutions& Optimism bias & 0.5772 & Strong Positive\\
% H2 & Listing multiple solutions& Non-biased counterarguments & -0.1493 & Weak Negative\\
% H3 & No. of Counterarg. & Non-biased Counterarg. & 0.7582 & Strong Positive \\
% H4 & No. of Arguments& Biased Arguments &0.6378 & Strong Positive \\
% \bottomrule
% \end{tabular}
% \end{table*}


\section{Discussion}
\label{sec:Discussion}
In this section, we discuss our results in the context of (1) comparison with previous work on debiasing ADM, \color{black} (2) differences between student and practitioner participants, (3) a set of teaching suggestions for educators, (4) implications for researchers.\color{black}

% Generally, the workshops' results were as predicted, and all measured values changed in accordance with our research hypotheses. Notably, statistically significant results were obtained in the case of:
% \begin{itemize}
%     \item Increased use of the ``discussing drawbacks'' and ``listing multiple solutions'' techniques.
%     \item Decreasing the occurrences of anchoring and optimism bias.
%     \item Increased use of non-biased arguments and counterarguments.
%     \item Decreased amount of biased pro-arguments.
%     \item Decreased percentage of biased statements.
% \end{itemize}

\subsection{\color{black}Comparison with Previous Work} 
Compared to the only study \color{black}reporting an effective debiasing ADM intervention by Borowa et al.~\cite{Borowa2022}, \color{black}performed with solely student participants, \textbf{our results show more positive outcomes}. In their study, the amount of non-biased arguments and counterarguments and the use of debiasing techniques were improved. Our study identified similar outcomes and additionally showed:  \textbf{(1) a decrease in bias occurrences (optimism bias and anchoring decreased significantly) and (2) a significant decrease of biased arguments in support of a solution}. This may be the result of two factors: (1) improved experiment and workshop design and (2) the practitioner participants' stronger response to the debiasing workshop \color{black}(i.e. bigger decrease in bias occurrences) \color{black} in comparison to student participants.

% \begin{shaded*}\noindent\textbf{Finding 1: The workshop is effective, but its impact is stronger on practitioners}
% Compared to students, experienced practitioners have fewer problems in specifying non-biased arguments in support of decisions. As such, when teaching practitioners, more focus should be on discussing fact-based counterarguments. 
% \end{shaded*}

\subsection{\color{black}Comparision between Practitioners' and Students'} 
\label{sec:discussion:pract_vs_students}
We discovered that there are notable differences in how students and practitioners reason about architectural decisions and respond to the debiasing workshop. 


First, \textbf{practitioners were much more skilled overall in using arguments in support of a solution}, in both control and workshop groups, \color{black} i.e. on average, they used almost twice as many arguments in favor of solutions than students. \color{black} We suspect this may be due to the practitioners' having the preexisting skill of arguing in favor of their chosen solutions since they most likely had to do this repeatedly in their workplace. In contrast, students may find the formulation of any arguments, both for and against architectural solutions, new due to their limited experience. % with different architectural solutions.%since university courses do not usually allow students much creative freedom with architectural decisions. 

Second, \textbf{the number of biased counterarguments slightly increased in the case of practitioners}, which was not the case with students. \color{black} This may be because practitioners had more previous negative experiences (e.g. failed projects), so they had a higher tendency to prematurely dismiss solutions that they had a personal negative experience with.

Thirdly, \textbf{practitioners from both control and workshop groups were overall more impacted by biases than students}\color{black}, i.e., the average bias occurrence measurement for all biases was higher in the case of practitioners. \color{black}As a result, practitioners' decrease in bias occurrence was more extreme than in the case of students. Having industrial experience may actually make practitioners more prone to the impact of cognitive biases. 
This higher bias susceptibility may be due to practitioners' higher attachment to the solutions that they worked on. In the case of a finished student project, the student can freely critique the solution since there is no danger of losing customers or damaging their reputation. Conversely, practitioners spend years polishing their systems, which they have to sell to customers. 
% On the other hand, students may find formulating both pro and con arguments challenging, as they often have less experience and lack seniority.
%This high attachment to their solutions is additionally reflected in practitioners' extremely high use of pro-arguments to support their solutions and the overall non-significant decrease of confirmation bias occurrences, i.e., the one bias that makes an individual dismiss information that is not compatible with their beliefs. 
This strong attachment to their solutions is further evidenced by practitioners’ frequent use of arguments to support their choices and the non-significant decrease in the occurrences of confirmation bias—i.e., the tendency to dismiss information that contradicts their beliefs.

\color{black} Similarly, in the software testing domain, Calikli et al.~\cite{calikli2010empirical} found that active software testers were more prone to confirmation bias than ex-testers or trained students. They attributed this to habits formed by routine tasks and suggested breaking routines by periodically performing varying tasks. A similar approach could also benefit software architects, such as participating in non-architectural activities. \color{black}
%but this requires further research.

% \subsection{\color{black}Improvement Areas} 
% Despite our workshop's overall success, it seems that it did not fully achieve three goals in a statistically significant manner:  (1) decreasing the confirmation bias occurrences, (2) raising the ``discussing risks'' technique's use, and (3) decreasing the number of biased counterarguments. Future debiasing efforts could focus on these shortfalls. This could include further debiasing techniques to counteract confirmation bias, and providing additional guidance to effectively discuss risks associated with architectural decisions.


% tOur debiasing workshop \textbf{succeeded in decreasing the occurrence of all three biases} (\textit{anchoring}, \textit{confirmation bias}, \textit{optimism bias}) for the workshop participants. In two cases,  this decrease was statistically significant (\textit{anchoring} and \textit{confirmation bias}). This had never been achieved previously. 

% The amounts of uses of \textbf{each debiasing technique that we taught increased} for workshop participants as well, with a statistically significant increase of the ``discussing drawbacks'' and ``listing multiple solutions'' techniques.

% Additionally, we observed \textbf{statistically significant changes} regarding an increased amount of non-biased arguments and counterarguments, decreased amount of biased pro-arguments, and decreased percentage of biased 

% While most measured values did change as we hypothesized, in most cases, the results were not statistically significant. However, this does not necessarily mean this study's results are not valuable. It does mean however, that not statistically significant results have weaker evidence supporting their validity and should be considered more carefully. Proper calculation of statistical significance in the case of such small samples is not easily achievable, and we used the p-values in this study mainly to distinguish results with the highest validity.

% We found it intriguing that, in comparison to the study on students~\cite{Borowa2022}, the frequency of non-biased arguments did not increase in a statistically significant manner. This implies that, for practitioners, the increase in non-biased counterarguments was more severe than in non-biased arguments. We suspect this may be due to the practitioners' having the preexisting skill of arguing in favor of their chosen solutions since they most likely had to do this repeatedly in their workplace. On the other hand, students may find the formulation of both pro and con arguments new since university courses do not usually allow students much creative freedom with architectural decisions. 

% We found it interesting that contrary to the study on students \cite{Borowa2022}, the frequency of non-biased arguments did not increase. We believe that this may be due to the practitioners' already preexisting skill of arguing in favor of their solutions, since they most likely had to do so often in their workplace. On the other hand,  students may find the formulation of both pro and con arguments to be new, since university courses usually do not allow students much creative freedom. What practitioners did learn from the workshop is the more rarely used skill of using counterarguments. They focused on this so much, that both biased and non-biased counterargument measurements were higher in the workshop group. Still, this increase was significantly higher only for non-biased counterarguments.

% \noindent This led us to derive four hypotheses:
% \begin{itemize}
%     \item H1: Listing multiple solutions may decrease over-optimistic statements.
%     \item H2: Listing multiple solutions may decrease the amount of non-biased counterarguments. 
%     \item H3: Increasing the number of counterarguments may cause a decrease in biased counterarguments.
%     \item H4: Increasing the number of arguments may cause an increase in biased arguments.
% \end{itemize}
% To verify these hypotheses, we calculated the Pearson Correlation Coefficients (R) to find if there were correlations supporting these hypotheses. We consider an R > 0.5 value to mean a strong correlation, an R between 0.3 and 0.5 to be a medium correlation, and smaller than 0.3 to be a weak correlation.
% The results of these calculations are presented in Table~\ref{tab:correlations}.

% \subsection{What increases rationality?}
% The debiasing techniques of listing drawbacks and counterarguments' did not increase significantly through our workshop. 
% Additionally, while the technique of listing multiple solutions was applied significantly more in the treatment group, this technique did not impact the debiasing effect directly. There is no negative correlation with \textit{optimism bias} and no positive one with non-biased counterarguments.

% This means that the positive impact of the workshop was not directly a result of the debiasing techniques. Instead, it is more likely that the overall educational value of the workshop made treatment group participants aware that over-optimism should be avoided and that discussing counterarguments can provide a debiasing effect.

% \subsection{Is more always better?}
% Through our debiasing efforts, we encouraged participants to argue their choices. 

% This seems to be the correct course of action in the case of counterarguments: since the bigger amount of counterarguments strongly correlates with non-biased counterarguments.
% However, the opposite is true in the case of arguments in support of a solution. More arguments strongly correlate with more biased arguments. 

% This may be explained by the practitioners' subconscious desire to prove that the solution they prefer is correct. When the practitioners wanted to support their choices, at some point rational arguments ran out, and biased arguments became more prevalent. This could be a major impact of \textit{confirmation bias} that the workshop did not counter.


% These counter-intuitive results might however be explained by a different chain of events. Simply, there is a strong correlation between the number of debiasing techniques used and overall arguments \& counterarguments, including the ones that are biased. Trained to produce arguments, the practitioners do so. 
% % I do not know what you want to say here, but some alternatives from GrammarlyGO:
% %The practitioners are trained to produce arguments, and they do so effectively.
% %The practitioners, who have been trained to create persuasive arguments, do just that.
% %The practitioners are trained to produce arguments, so they do it.

% That would mean that \textbf{the practitioners' success of avoiding over-optimistic statements, was not directly tied to the debiasing techniques, but more to the overall information learned from the workshop}.

% This does not necessarily mean that we should avoid debiasing techniques to avoid biased statements. However, it seems that the over-production of low-quality arguments is a side-effect of over-focusing on debiasing techniques. The students that took part in Borowa et al.'s \cite{Borowa2022} workshop, on average used 8 debiasing techniques, while our practitioner participants had an average of 11.6. 


% Our debiasing intervention did increase non-biased counterarguments, yet to counter \textit{anchoring} and \textit{confirmation bias}, there is still a pressing need to stop the overflow of low-quality biased pro arguments. 
% To achieve this, we propose a new decision-making technique: \textbf{When discussing the pros and cons of a solution, limit the number of pro arguments to a small amount of the most significant arguments}.



%\textbf{Lessons learned} \newline
% \subsection{\color{black}Unsuccessful Participants}
% Some of the participants did not perform as expected. 

% Firstly, practitioner participant No. 3 (pair P2 from the pilot) discussed only two design decisions, which was the smallest number of decisions discussed by all of the participants. In this case, we suspect that this resulted from a socio-cultural factor associated with group decision-making \cite{muccini2018group} - \textbf{the junior participant felt internal pressure due to the presence of the senior participant (control group).}

% Pair P4 was the only practitioner pair where the workshop participant's bias occurrences and percentage of biased statements were higher than his control group counterparts. This happened despite the workshop participant's higher use of debiasing techniques. The only value that seems to distinguish this pair is that the workshop participant discussed many more decisions than the control group participant. This difference, of 19 decisions discussed over 9, was the highest of all pairs. As such, it seems that \textbf{discussing too many decisions in a short time span may increase the likelihood of cognitive bias occurrence and should be avoided.} We suspect that this could be explained by a heightened cognitive load~\cite{gonccales2021measuring} experienced by the participant who attempted to discuss such a large amount of design decisions swiftly. Psychology research on cognitive load does show that it has an impact on \textit{anchored} time estimates~\cite{blankenship2008elaboration} and as such, we suspect that it may have also impacted our participant. Especially since anchoring was the main bias impacting this participant. %However, we found no relevant research on cognitive load in software architecture that would link . % that could support this hypothesis. As such, further research would be required to examine it.

% Student pair S6 was the only student pair where the workshop group participant did not perform better in almost any regard: (1) they had more bias occurrences, (2) they had more biased statements, and (3) they used fewer debiasing techniques. This student's case was quite distinct because, during the workshop (before which we did not disclose that the study is about cognitive biases), they informed the researcher conducting the workshop that they read Kahneman's book~\cite{kahneman2011think}. As such, this participant had a high confidence in their knowledge of cognitive biases.\textbf{ We believe that this exact confidence may have made our debiasing efforts less successful in their case.}
% Similarly, the workshop participant from pair P5 was the only practitioner who used a smaller amount of debiasing techniques than his control group colleague. \textbf{This may have been caused by their high confidence as well, which could be associated with being a CTO in their company.}

% As such, we presume that the debiasing workshop's effect can be negatively impacted by: (1) socio-cultural factors associated with group decision-making, (2) discussing too many decisions in a short time span, (3) high confidence level, e.g. due to prior knowledge about biases or a high position in an organization.

\subsection{\color{black}Recommendations for Educators}
\color{black}
Overall, the workshop was a successful debiasing intervention and we recommend its use by educators. However, we additionally investigated the participants who did not perform as expected, to explore what educators might take into account to improve this debiasing intervention. 

\textbf{Teaching suggestion 1: Debiasing might only be effective when team members of all seniority levels participate in a debiasing intervention.}
As shown by Participant 3 (pair P2 from the pilot), who discussed only two design decisions, group discussion can be heavily impacted by pressure from senior colleagues. As such, socio-cultural factors associated with group decision-making \cite{muccini2018group} make it crucial to perform the intervention on all individuals, not only a few team members, who may be unable to pass their knowledge about biases to their colleagues. 
%If only junior practitioners take part in the intervention, they are unlikely to be able to pass their knowledge about biases to seniors.

\textbf{Teaching suggestion 2: Avoid discussing too many decisions.}
Pair P4 was the only practitioner pair where the workshop participant's bias occurrences were higher than their control group colleague's. The only value that seems to distinguish this pair is that the workshop participant discussed many more decisions than the control group participant. This difference was the highest of all the pairs. As such, discussing too many decisions in a short time span may increase the likelihood of cognitive bias and should be avoided. This might result from a heightened cognitive load~\cite{gonccales2021measuring} experienced by the participant, who attempted to discuss a large number of design decisions swiftly. Psychological research shows that cognitive load impacts \textit{anchored} time estimates~\cite{blankenship2008elaboration}. Thus, we suspect that it may also have impacted our participant. Especially since anchoring was the main bias impacting this participant.

\textbf{Teaching suggestion 3: Address overconfidence to counter susceptibility to cognitive biases.}
Student pair S6 was the only student pair where the workshop group participant did not perform better in almost any regard (bias occurrences, biased statements, and debiasing techniques). This student's case was quite distinct because, during the workshop (before which we did not disclose that the study is about cognitive biases), they informed the researcher conducting the workshop that they had read Kahneman's famous book about cognitive biases~\cite{kahneman2011think}. As such, this participant had a high confidence in their knowledge of cognitive biases.
Similarly, the workshop participant from pair P5 was the only practitioner who used a smaller amount of debiasing techniques than his control group colleague. Notably, the workshop group participant was the company's CTO. During the experiment, the researcher noticed that the attitude of the workshop participants indicated that they considered the workshop's contents trivial. The participant's confidence level was most likely heightened due to their high position in the company, where they were previously the head architect for many years. Therefore, we suggest that educators performing debiasing interventions should warn unusually confident individuals about their possibly higher susceptibility to biases.

\textbf{Teaching suggestion 4: Limit low-quality counterarguments.} In the case of practitioners, a slight increase in the use of biased counterarguments by workshop participants occurred. This may be because the workshop focused on negative factors, such as decision-related risks and drawbacks. As such, some participants may have over-focused on creating numerous counterarguments with less care about their quality. Educators may ask participants to limit the number of counterarguments to a set number of crucial ones during the workshop.
% \begin{shaded*}\noindent\textbf{Teaching suggestion 1: Focus on high-quality counterarguments.}
% Compared to students, experienced practitioners have fewer problems in specifying non-biased arguments in support of decisions. As such, when teaching practitioners, more focus should be on discussing fact-based counterarguments. 
% \end{shaded*}


\subsection{\color{black}Implications for Researchers}
Our study provides researchers with a valuable starting point, showcasing how a successful ADM debiasing intervention can be performed. However, we strongly advise that researchers consider the following issues.
First, since confirmation bias was the only cognitive bias that did not decrease significantly, researchers should focus on finding methods of countering its impact. Since we suspect that routine, previous negative experiences and overconfidence may be the source of this problem, debiasing methods could consider routine interruption, discussing previous negative experiences, and warning overconfident individuals.
Second, since practitioners' bias levels were higher and attachment to an existing architectural solution seems to be a strong factor affecting biases, future studies should focus on debiasing practitioners on real-life industrial systems.
Finally, since the use of the ``discussing risks'' technique did not increase significantly, the workshop did not fully teach participants how to perform it. Researchers could attempt to change the workshop's contents related to teaching this technique or even consider replacing it with a different technique altogether.

% since the practitioners bias levels were overall much higher than students, we strongly suggest that future debiasing interventions should be tested using 
% Despite our workshop's overall success, it seems that it did not fully achieve three goals in a statistically significant manner:  (1) decreasing the confirmation bias occurrences, (2) raising the ``discussing risks'' technique's use, and (3) decreasing the number of biased counterarguments. Future debiasing efforts could focus on these shortfalls. This could include further debiasing techniques to counteract confirmation bias, and providing additional guidance to effectively discuss risks associated with architectural decisions. Since we suspect that
\color{black}

% produced notably \textbf{unexpected results.} 

% A side effect of the workshop seems to be a slight increase in the use of biased counterarguments by workshop participants, most likely because the workshop focused on negative factors such as decision-related risks and drawbacks. As such, workshop participants may have over-focused on creating numerous counterarguments with less care about their quality. However, the beneficial increase in non-biased counterarguments was significantly higher.
% \begin{shaded*}\noindent\textbf{Teaching suggestion 1: Focus on high-quality counterarguments.}
% Compared to students, experienced practitioners have fewer problems in specifying non-biased arguments in support of decisions. As such, when teaching practitioners, more focus should be on discussing fact-based counterarguments. 
% \end{shaded*}

% Additionally, Participant No 3 (Pair P2 from the pilot) discussed only two design decisions, which was the smallest number of decisions discussed by all the participants. In this case, we suspect that this resulted from \textbf{a socio-cultural factor} associated with group decision-making \cite{muccini2018group} - the junior participant felt internal pressure due to the presence of the senior participant (control group). %To avoid further impact from this factor on our study, we changed the Step 3 formula from group discussion to think-aloud protocol.
% \begin{shaded*}\noindent\textbf{Teaching suggestion 2: Debiasing might only be effective when team members of all seniority levels take part in it.}
% Practitioners must be made aware that this training's debiasing effect can be significantly decreased if thr participant's colleagues who are higher in the hierarchy dismiss their concerns. As such, teaching only younger developers while omitting senior team members or leaders may not be effective.
% \end{shaded*}

% Furthermore, the use of debiasing techniques also increased for almost all participant pairs, except one. In the case of pair P5, Participant No 9, who was the workshop group participant, was also the CTO of the company. During the experiment, the author conducting it noticed that the attitude of the participant suggested that they considered the contents of the workshop to be trivial. The participant's \textbf{confidence level was most likely heightened due to their high position in the company}, where they were previously the head architect for many years. It is likely that this made them more susceptible to cognitive biases and less likely to use the debiasing techniques.
% \begin{shaded*}\noindent\textbf{Teaching suggestion 3: Focus on high-level and experienced team members.}
% If the participant has a high position in the company and vast engineering expertise, they may have a high confidence level in the architecture of their system and thus be less likely to learn from this workshop. In the case of these participants, take care to specifically inform them of this danger. 
% \end{shaded*}

% Finally, participant No.~7 (Pair~4) was the only workshop group participant who gave more biased statements than the control group participant. The only metric that seemingly distinguishes participant No.~7 is that they choose to discuss \textbf{the highest number of architectural decisions (19)} from all the participants. This is in line with classic research on cognitive biases, which defines them as a natural heuristic used by the human mind to avoid overload with too many tasks \cite{kahneman2011think}. This means that when too many decisions are considered, the human mind will behave in an energy-saving manner that makes biases more likely to occur. 

% \begin{shaded*}\noindent\textbf{Teaching suggestion 4: Avoid discussing too many decisions.}
% Inform participants that considering too many architectural decisions in a short time span is likely to produce an increase in cognitive bias influences. Try to moderate the training in such a way that focuses on high-quality argumentation of a lower number of key decisions.
% \end{shaded*}
% %In the case of Participant No.7, they discussed 16 pro arguments while Participant No. 8 only discussed 8.
% %This could be explained by our observation that more supporting arguments strongly increase the amount of biased arguments. 


    \section{Threats to validity}
    \label{sec:ThreatsToValidity}
        Our threats to validity are described based on the guidelines %for experiments in software engineering 
        created by Wohlin et al.~\cite{Wohlin2000}:

             %rr updated
    	\textbf{Construct Validity --}
    	% CUT:Construct validity refers to the appropriateness of the chosen methods.
     % As we performed a controlled experiment, our study heavily relies on its construct validity.
     %While we carefully designed the experiment to maximize construct validity (see Section~\ref{sec:Method}), 
     \color{black} It is important to note that every human's thought process is nuanced and multifaceted, so the ``biased statements'' that we measured are a simplification of this process. As such, there exists the threat that our study over-simplified this complex phenomenon by, for example, not taking into account how several biases may impact decision-making simultaneously, and the relationships between them. \color{black}
     
     % one threat that could impact the results is how each researcher could lead the debiasing workshop differently. To mitigate the potential particularities and cultural factors that could affect the experiment, we discussed possible scenarios that could occur during the experiment before conducting it and then used the same workshop plan and slides.
     % Furthermore, during the pilot run, we observed that one pilot study participant being the other's superior may be an additional factor decreasing validity. Therefore, we changed the last step of the experiment from group discussion to separate think-aloud protocol sessions.
     
     % Additionally, we had the participants take a break before Step 4 of the study, to avoid overtiring workshop participants.
     % Finally, every control/workshop pair of participants discussed the same architecture during Step 4 of the study, which makes the results for both groups simpler to compare. This is a difference from previous research that performed the comparison of before and after the workshop, using two different architectural tasks~\cite{Borowa2022}.
     


         	\textbf{Internal Validity --}
        %CUT: Internal validity threats are related to possible wrong conclusions about causal relationships between treatment and outcome. 
        % We discussed some causalities of our findings in Section~\ref{sec:Discussion}. 
        % However, the real causalities could be different.
        % We discussed the results with three researchers from different cultural backgrounds. 
        % We could support our assumptions on the causalities with the correlations, we found.
        % We suggest doing future work (more experiments, long-term observation) to support our findings even more.
        % While the experimental environment was carefully controlled to ascertain internal validity, some threats have to be considered. 
        First, our sample may be impacted by selection bias since our participants were volunteers. It is possible that individuals less interested in architectural decision-making may be impacted by cognitive biases and the workshop differently.
        % Secondly, while we strived to recruit pairs of participants with similar levels of expertise, both participants in each pair could not have the same exact level of knowledge and the same previous experiences. However, we decided that this is preferable to the pretest-posttest experimental design, which poses a significant threat to validity due to the participants' learning and maturation during the pretest task.
        Second, student participants discussed systems from a university project, while practitioners described real-life systems. Therefore, when comparing their outcomes, it must be noted that some result differences may be rooted in the nature of university projects and not the participant's experience level. We take this into account when discussing our results (see Section~\ref{sec:discussion:pract_vs_students}). This was unavoidable since there was no different way of testing how students reason if they had no experience in designing real-world architectures.
        \color{black}
        Third, while our experiment, which employed a matched pair design, did include randomized splitting into the control/workshop group, internal validity could have been enhanced even further by a fully random assignment design~\cite{ACM_standards_experiment}. However, in that case, all the participants would have had to discuss the same task, making it impossible for them to use their knowledge about the system fully, thus limiting external validity.
        Finally, a key limitation of the experiment lies in the design of the control group. The control group participants were given no alternative intervention and simply took a break. This raises the possibility that the observed improvements may stem from any intervention rather than the debiasing workshop. Future studies should include a placebo~\cite{myers2006experimental} intervention for the control group participants to mitigate this threat. 
        \color{black}
 % We took care not to allow external factors to impact the participants: (1) We did not tell them about cognitive biases beforehand. (2) They did not know about Step 4 of the experiment beforehand, so they could not prepare for it. (3) During Step 4, participants did not get the chance to contact anyone or use the internet for help.  Additionally, we employed context-free coding to avoid the coder's unconscious desire to showcase that the workshop is effective. An additional author then reviewed this coding to avoid errors that could result from having one coder.
  

        
        \textbf{Conclusion Validity --}
        %// add info from  previous study, and correlations
    	%CUT: Conclusion validity relates to the reliability of the conclusions drawn from the results.
         We have the threat of a relatively low number of experiments executed, which is due to the fact that it was hard to find participants, especially practitioners, willing to invest over two hours in a scientific experiment. Finding participants was particularly challenging since our experiment required (1) pairs of participants who worked together on the same project and (2), in the case of practitioners, sensitive data about their real-life architecture.
         This low sample size places a risk on the statistical significance of the findings.
         To counteract this threat, we performed the non-parametric Wilcoxon Signed Rank Test, which makes it possible to check the statistical significance of small paired samples. 
         Still, a larger sample would provide more confidence in the findings.
         %Additionally, while our data gathering method differed in the pilot (discussion) and main (think-aloud protocol) experiments, we included the data from the pilot experimental session where the treatment group participant discussed the smallest amount of decisions (only two) from the pilot study in our calculations. We did this to avoid 'fishing' for the desired results, which Wohlin et al.~\cite{Wohlin2000} warn about.   


    \textbf{External Validity --}
    	%CUT: External validity threats refer to the ability to generalize the result.
    As common for a controlled experiment, the external validity is hampered by the strictly controlled setting. However, we tried to perform the experiment in the participants' preferred environment to lessen this effect. %: in their company's office or using their preferred videoconferencing software.
    Additionally, the student participants have less domain knowledge than practitioners, which could lead to poor design decisions. We attempted to lessen this effect by having students discuss a project that they designed and implemented themselves during their coursework, which means that they had to analyze the project's domain beforehand.
    % Furthermore, all our practitioner participants discussed real-life architectures from their workplaces.
    % Finally, we strived to obtain a diverse group of participants, in order to ascertain the workshop's effectiveness on various individuals and projects.
Finally, we could not research the long-term effect of the debiasing workshop. The workshop's beneficial impact may lessen over time. Future work could focus on studying the long-term effects of our workshop. %are not studied => no long-term prognosis (=> future work)

    
       % Wohlin et al.~\cite{Wohlin2000} differentiate three types of interactions with treatment: people, place, and time.//industry participants
       % In the case of \textbf{people}, the main threat to validity is the rather small number of participating practitioners. However, this was unavoidable due to the experiment requiring very sensitive data from the practitioners, which called for an extraordinary level of trust towards the researchers. However, we attempted to make this threat less impactful by gathering data from 6 different companies, in 3 different countries. 
       % ///->
       
       % //where they felt comfortable
       % The \textbf{place} kind of threats are related to the setting of the treatment. In our case, by design, the setting varied heavily: three different researchers organizing workshop sessions, sessions online and in-person, in three different countries. This way, even if the setting had an influence, it would be different for each pair of participants. The common point for all experiments was: the unified experiment plan and resources (available as additional material \cite{additionsl_material}. Additionally, to ensure that the architectures discussed were representative of real-life practice, during the experiment, participants discussed real-life systems from their current company.

       % //break //not enough time for experiment
       % As for \textbf{time} related threats to validity: to avoid their influence, we ensured that paired control and treatment group participants took part in the study either simultaneously (Step 1, pilot phase Step 3) or one after the other (main phase Step 3), to avoid the impact of additional external factors. Finally, all participants were coworkers, which means that they had similar kinds of influences on them. 
    
\section{Conclusion}
\label{sec:Conclusion}
We performed a debiasing workshop in order to explore its impact on students and practitioners, i.e. individuals at various career stages.
The design of our study was based on the structure by Borowa et al. \cite{Borowa2022}, with some adaptations (as explained in Section~\ref{sec:Method}). %Instead of studying a group `before' and `after' the treatment, 
We divided participants into separate workshop (treatment) and control groups as it is common in similar studies~\cite{razavian2016two,Tang2017}, striving for a higher level of validity than in a pretest-posttest experiment~\cite{Knapp2016}. We also employed the think-aloud protocol study method suggested by Razavian et al. \cite{Razavian2019} as appropriate for researching behavioral aspects of ADM. 
%These were all intentional decisions made to use well-established, reliable methods for our study.

The research question that we aimed to answer was:
\begin{itemize}
                \item [] \textbf{RQ: How does the proposed debiasing architectural decision-making workshop influence students and practitioners?}
    \end{itemize}
Our main contribution includes the design and evaluation of a debiasing workshop. Through this evaluation, we found:
\begin{enumerate}
    \item The use of the proposed debiasing techniques that were taught increased. In particular, the use of``discussing drawbacks'' and ``listing multiple solutions'' techniques' increased significantly.
     \item The occurrences of cognitive bias for all three researched biases (anchoring, confirmation bias, and optimism bias) decreased. This decrease was significant in the case of anchoring and optimism bias.
    \item Practitioners were commonly more influenced by all researched biases than students, and as such, the improvement achieved in their case was more prominent. 
    % \item Two effects were statistically significant: (1)~the rise of non-biased counterarguments and (2)~the increased frequency of using the listing multiple solutions technique.
    \item \color{black} A set of teaching suggestions based on factors that negatively impacted some of the study's participants.\color{black}
    %successfully debiasing ADM, such as (1)~socio-cultural factors, (2)~confidence level, and (3)~discussing too many architectural decisions in a short time span.
\end{enumerate}
% explored whether the following positive effects would appear: (1) Would the workshop decrease the amount of cognitive biases occurrences? (2) Would the workshop increase the participant's  use of debiasing techniques?
% \textbf{Both positive effects were observed as the workshop's result.} All researched biases influenced were decreased and the use of all techniques was higher for workshop participants. Additionally, two effects were statistically significant: (1) the rise of non-biased counterarguments and (1) the increased frequency of using the listing multiple solutions technique.

            %     The workshop decreased the occurrence of all three researched biases: \textit{anchoring}, \textit{confirmation bias}, and \textit{optimism bias}.
            %     The decrease in the amount of \textit{optimism bias} influences was statistically significant. This was not the case for \textit{anchoring} and\textit{ confirmation bias}.
            %     \item \textbf{Would the workshop increase the participant's  use of debiasing techniques?}
            %     The workshop did make participants list solution alternatives significantly more often. However, this was not the case for the other two debiasing techniques.
            %     \item \textbf{Would the debiasing techniques decrease the amount of cognitive biases?} 
            %     The debiasing techniques' use did not correlate with the decrease in the amount of cognitive bias influences.
            % \end{itemize}

% In order to achieve that, we improved and adapted the debiasing workshop of Borowa et al. \cite{Borowa2022} that had a positive debiasing effect on student experiment participants. 
% We found seven pairs of practitioners, separated them into a control and treatment (workshop) group, and had them discuss improvements for an existing real-world architectural design. 

% The answer to the RQ is complex, since some parts of the experiment were successful, while others were not. Our main findings include:
%     \begin{itemize}
%         \item The workshop was successful in the regard of decreasing the impact of optimism bias. This result is statistically significant. 
%         \item The workshop was not fully successful in decreasing the impact of anchoring and confirmation biases. In this case, the results were not statistically significant. 
%         \item Despite the workshop training, unlike students in previous work \cite{Borowa2022}, debiased practitioners use of all debiasing techniques did was not statistically significant in comparison to the control group.
%         \item The only debiasing technique which's use was significantly higher in the treatment group, was listing multiple solution options.
%         \item The workshop caused a significant increase in non-biased counterarguments for the treatment group participants.
%         \item Over-focusing on producing positive arguments as well as avoiding the use of arguments can be a source of cognitive biases.
%         Therefore, debiasing architectural decision-making should focus more on the quality of argumentation, and not only quantity.
%         \item Additional social or individual factors can be a source of cognitive biases and influence the experiment.
%     \end{itemize}


\textbf{Practitioners and educators} intending to perform such a debiasing workshop can use the teaching materials we prepared for this study either directly or using them as a basis for their own materials~\cite{additional_material}. \textbf{Researchers} may make use of this study as a basis for creating improved debiasing interventions and evaluate them by employing our experimental design. 

%Additionally, we encourage them to employ the suggestions presented in Section~\ref{sec:Discussion}.


% Additionally, for those intending to use the workshop we suggest the following:
% \begin{itemize}
%     \item 
%     \item 
%     \item 
%     \item 
% \end{itemize}

% Our study resulted in the following contributions:
% \begin{itemize}
%         \item We created a workshop that decreased the overall occurrence of cognitive biases in experienced practitioners.
%         \item The workshop, most likely as a result of educating about cognitive biases influence (not through the debiasing techniques), caused a statistically significant: (1) increase in non-biased counterarguments and (2) a decrease in \textit{optimism bias} impact for the treatment group participants.
%         \item Over-focusing on producing positive arguments can be a direct source of biased arguments.
%         Therefore, debiasing architectural decision-making should focus more on the quality of argumentation and not only quantity.
%         \item Contrary to pro arguments, focus on counterarguments does result in more non-biased counterarguments. %Contrary to that, focusing on producing pro arguments leads to more biased arguments.
%         \item Additional social or individual factors can be a source of cognitive biases and influence the experiment.
%     \end{itemize}

% \noindent
%Our work has its merits for both researchers and practitioners. 
% \textbf{Researchers} have obtained an effective debiasing workshop that decreases the occurrence of cognitive biases and improves the use of debiasing techniques.

% The workshop's effect is statistically significant in regards to an increased amount of non-biased counterarguments and improved tge a decrease in the impact of \textit{optimism bias}. We also propose a simple technique that could improve the workshop in terms of reducing biased arguments, which could additionally help counter \textit{anchoring} and \textit{confirmation bias}: limiting the number of pro arguments to a small amount of the most significant ones. 
% Additionally, researchers now should be aware that experienced practitioners require more sophisticated debiasing training for architectural decision-making than students. %rr del: Since many positive results that Borowa et al. \cite{Borowa2022} obtained through her debiasing workshop on students, e.g. significant decrease in the percentage of biased statements, did not occur in our experiment.
%rr update:
% Different from Borowa et al. \cite{Borowa2022}, performed with students, with practitioners the significant decrease in the percentage of biased statements, did not occur.

%\noindent
% \textbf{Practitioners} can use our workshop in order to improve the architectural decision-making in their companies and improve awareness about the potential problems caused by cognitive bias in the team's routine. The workshop material is openly available in our additional material~\cite{additional_material}
% \\
    
%\subsection{Future Work}
\textbf{For future work,} we propose to focus on (1) longitudinal studies in a real-world setting to ensure that the debiasing treatment has a long-term impact, (2) further improvements meant to reduce the impact of confirmation bias, \color{black}(3) an in-depth qualitative analysis of the biases in ADM.\color{black}


\section{Data availability}
The experiment plan, workshop plan, workshop slides, participant questionnaire, coding results, and a coding guide with examples from the participants are available online~\cite{additional_material}.










% \section{Introduction}
% This document is a model and instructions for \LaTeX.
% Please observe the conference page limits. 

% \section{Ease of Use}

% \subsection{Maintaining the Integrity of the Specifications}

% The IEEEtran class file is used to format your paper and style the text. All margins, 
% column widths, line spaces, and text fonts are prescribed; please do not 
% alter them. You may note peculiarities. For example, the head margin
% measures proportionately more than is customary. This measurement 
% and others are deliberate, using specifications that anticipate your paper 
% as one part of the entire proceedings, and not as an independent document. 
% Please do not revise any of the current designations.

% \section{Prepare Your Paper Before Styling}
% Before you begin to format your paper, first write and save the content as a 
% separate text file. Complete all content and organizational editing before 
% formatting. Please note sections \ref{AA}--\ref{SCM} below for more information on 
% proofreading, spelling and grammar.

% Keep your text and graphic files separate until after the text has been 
% formatted and styled. Do not number text heads---{\LaTeX} will do that 
% for you.

% \subsection{Abbreviations and Acronyms}\label{AA}
% Define abbreviations and acronyms the first time they are used in the text, 
% even after they have been defined in the abstract. Abbreviations such as 
% IEEE, SI, MKS, CGS, ac, dc, and rms do not have to be defined. Do not use 
% abbreviations in the title or heads unless they are unavoidable.

% \subsection{Units}
% \begin{itemize}
% \item Use either SI (MKS) or CGS as primary units. (SI units are encouraged.) English units may be used as secondary units (in parentheses). An exception would be the use of English units as identifiers in trade, such as ``3.5-inch disk drive''.
% \item Avoid combining SI and CGS units, such as current in amperes and magnetic field in oersteds. This often leads to confusion because equations do not balance dimensionally. If you must use mixed units, clearly state the units for each quantity that you use in an equation.
% \item Do not mix complete spellings and abbreviations of units: ``Wb/m\textsuperscript{2}'' or ``webers per square meter'', not ``webers/m\textsuperscript{2}''. Spell out units when they appear in text: ``. . . a few henries'', not ``. . . a few H''.
% \item Use a zero before decimal points: ``0.25'', not ``.25''. Use ``cm\textsuperscript{3}'', not ``cc''.)
% \end{itemize}

% \subsection{Equations}
% Number equations consecutively. To make your 
% equations more compact, you may use the solidus (~/~), the exp function, or 
% appropriate exponents. Italicize Roman symbols for quantities and variables, 
% but not Greek symbols. Use a long dash rather than a hyphen for a minus 
% sign. Punctuate equations with commas or periods when they are part of a 
% sentence, as in:
% \begin{equation}
% a+b=\gamma\label{eq}
% \end{equation}

% Be sure that the 
% symbols in your equation have been defined before or immediately following 
% the equation. Use ``\eqref{eq}'', not ``Eq.~\eqref{eq}'' or ``equation \eqref{eq}'', except at 
% the beginning of a sentence: ``Equation \eqref{eq} is . . .''

% \subsection{\LaTeX-Specific Advice}

% Please use ``soft'' (e.g., \verb|\eqref{Eq}|) cross references instead
% of ``hard'' references (e.g., \verb|(1)|). That will make it possible
% to combine sections, add equations, or change the order of figures or
% citations without having to go through the file line by line.

% Please don't use the \verb|{eqnarray}| equation environment. Use
% \verb|{align}| or \verb|{IEEEeqnarray}| instead. The \verb|{eqnarray}|
% environment leaves unsightly spaces around relation symbols.

% Please note that the \verb|{subequations}| environment in {\LaTeX}
% will increment the main equation counter even when there are no
% equation numbers displayed. If you forget that, you might write an
% article in which the equation numbers skip from (17) to (20), causing
% the copy editors to wonder if you've discovered a new method of
% counting.

% {\BibTeX} does not work by magic. It doesn't get the bibliographic
% data from thin air but from .bib files. If you use {\BibTeX} to produce a
% bibliography you must send the .bib files. 

% {\LaTeX} can't read your mind. If you assign the same label to a
% subsubsection and a table, you might find that Table I has been cross
% referenced as Table IV-B3. 

% {\LaTeX} does not have precognitive abilities. If you put a
% \verb|\label| command before the command that updates the counter it's
% supposed to be using, the label will pick up the last counter to be
% cross referenced instead. In particular, a \verb|\label| command
% should not go before the caption of a figure or a table.

% Do not use \verb|\nonumber| inside the \verb|{array}| environment. It
% will not stop equation numbers inside \verb|{array}| (there won't be
% any anyway) and it might stop a wanted equation number in the
% surrounding equation.

% \subsection{Some Common Mistakes}\label{SCM}
% \begin{itemize}
% \item The word ``data'' is plural, not singular.
% \item The subscript for the permeability of vacuum $\mu_{0}$, and other common scientific constants, is zero with subscript formatting, not a lowercase letter ``o''.
% \item In American English, commas, semicolons, periods, question and exclamation marks are located within quotation marks only when a complete thought or name is cited, such as a title or full quotation. When quotation marks are used, instead of a bold or italic typeface, to highlight a word or phrase, punctuation should appear outside of the quotation marks. A parenthetical phrase or statement at the end of a sentence is punctuated outside of the closing parenthesis (like this). (A parenthetical sentence is punctuated within the parentheses.)
% \item A graph within a graph is an ``inset'', not an ``insert''. The word alternatively is preferred to the word ``alternately'' (unless you really mean something that alternates).
% \item Do not use the word ``essentially'' to mean ``approximately'' or ``effectively''.
% \item In your paper title, if the words ``that uses'' can accurately replace the word ``using'', capitalize the ``u''; if not, keep using lower-cased.
% \item Be aware of the different meanings of the homophones ``affect'' and ``effect'', ``complement'' and ``compliment'', ``discreet'' and ``discrete'', ``principal'' and ``principle''.
% \item Do not confuse ``imply'' and ``infer''.
% \item The prefix ``non'' is not a word; it should be joined to the word it modifies, usually without a hyphen.
% \item There is no period after the ``et'' in the Latin abbreviation ``et al.''.
% \item The abbreviation ``i.e.'' means ``that is'', and the abbreviation ``e.g.'' means ``for example''.
% \end{itemize}
% An excellent style manual for science writers is \cite{b7}.

% \subsection{Authors and Affiliations}
% \textbf{The class file is designed for, but not limited to, six authors.} A 
% minimum of one author is required for all conference articles. Author names 
% should be listed starting from left to right and then moving down to the 
% next line. This is the author sequence that will be used in future citations 
% and by indexing services. Names should not be listed in columns nor group by 
% affiliation. Please keep your affiliations as succinct as possible (for 
% example, do not differentiate among departments of the same organization).

% \subsection{Identify the Headings}
% Headings, or heads, are organizational devices that guide the reader through 
% your paper. There are two types: component heads and text heads.

% Component heads identify the different components of your paper and are not 
% topically subordinate to each other. Examples include Acknowledgments and 
% References and, for these, the correct style to use is ``Heading 5''. Use 
% ``figure caption'' for your Figure captions, and ``table head'' for your 
% table title. Run-in heads, such as ``Abstract'', will require you to apply a 
% style (in this case, italic) in addition to the style provided by the drop 
% down menu to differentiate the head from the text.

% Text heads organize the topics on a relational, hierarchical basis. For 
% example, the paper title is the primary text head because all subsequent 
% material relates and elaborates on this one topic. If there are two or more 
% sub-topics, the next level head (uppercase Roman numerals) should be used 
% and, conversely, if there are not at least two sub-topics, then no subheads 
% should be introduced.

% \subsection{Figures and Tables}
% \paragraph{Positioning Figures and Tables} Place figures and tables at the top and 
% bottom of columns. Avoid placing them in the middle of columns. Large 
% figures and tables may span across both columns. Figure captions should be 
% below the figures; table heads should appear above the tables. Insert 
% figures and tables after they are cited in the text. Use the abbreviation 
% ``Fig.~\ref{fig}'', even at the beginning of a sentence.

% \begin{table}[htbp]
% \caption{Table Type Styles}
% \begin{center}
% \begin{tabular}{|c|c|c|c|}
% \hline
% \textbf{Table}&\multicolumn{3}{|c|}{\textbf{Table Column Head}} \\
% \cline{2-4} 
% \textbf{Head} & \textbf{\textit{Table column subhead}}& \textbf{\textit{Subhead}}& \textbf{\textit{Subhead}} \\
% \hline
% copy& More table copy$^{\mathrm{a}}$& &  \\
% \hline
% \multicolumn{4}{l}{$^{\mathrm{a}}$Sample of a Table footnote.}
% \end{tabular}
% \label{tab1}
% \end{center}
% \end{table}

% \begin{figure}[htbp]
% \centerline{\includegraphics{fig1.png}}
% \caption{Example of a figure caption.}
% \label{fig}
% \end{figure}

% Figure Labels: Use 8 point Times New Roman for Figure labels. Use words 
% rather than symbols or abbreviations when writing Figure axis labels to 
% avoid confusing the reader. As an example, write the quantity 
% ``Magnetization'', or ``Magnetization, M'', not just ``M''. If including 
% units in the label, present them within parentheses. Do not label axes only 
% with units. In the example, write ``Magnetization (A/m)'' or ``Magnetization 
% \{A[m(1)]\}'', not just ``A/m''. Do not label axes with a ratio of 
% quantities and units. For example, write ``Temperature (K)'', not 
% ``Temperature/K''.

% \section*{Acknowledgment}

% The preferred spelling of the word ``acknowledgment'' in America is without 
% an ``e'' after the ``g''. Avoid the stilted expression ``one of us (R. B. 
% G.) thanks $\ldots$''. Instead, try ``R. B. G. thanks$\ldots$''. Put sponsor 
% acknowledgments in the unnumbered footnote on the first page.

% \section*{References}

% Please number citations consecutively within brackets \cite{b1}. The 
% sentence punctuation follows the bracket \cite{b2}. Refer simply to the reference 
% number, as in \cite{b3}---do not use ``Ref. \cite{b3}'' or ``reference \cite{b3}'' except at 
% the beginning of a sentence: ``Reference \cite{b3} was the first $\ldots$''

% Number footnotes separately in superscripts. Place the actual footnote at 
% the bottom of the column in which it was cited. Do not put footnotes in the 
% abstract or reference list. Use letters for table footnotes.

% Unless there are six authors or more give all authors' names; do not use 
% ``et al.''. Papers that have not been published, even if they have been 
% submitted for publication, should be cited as ``unpublished'' \cite{b4}. Papers 
% that have been accepted for publication should be cited as ``in press'' \cite{b5}. 
% Capitalize only the first word in a paper title, except for proper nouns and 
% element symbols.

% For papers published in translation journals, please give the English 
% citation first, followed by the original foreign-language citation \cite{b6}.

% \begin{thebibliography}{00}
% \bibitem{b1} G. Eason, B. Noble, and I. N. Sneddon, ``On certain integrals of Lipschitz-Hankel type involving products of Bessel functions,'' Phil. Trans. Roy. Soc. London, vol. A247, pp. 529--551, April 1955.
% \bibitem{b2} J. Clerk Maxwell, A Treatise on Electricity and Magnetism, 3rd ed., vol. 2. Oxford: Clarendon, 1892, pp.68--73.
% \bibitem{b3} I. S. Jacobs and C. P. Bean, ``Fine particles, thin films and exchange anisotropy,'' in Magnetism, vol. III, G. T. Rado and H. Suhl, Eds. New York: Academic, 1963, pp. 271--350.
% \bibitem{b4} K. Elissa, ``Title of paper if known,'' unpublished.
% \bibitem{b5} R. Nicole, ``Title of paper with only first word capitalized,'' J. Name Stand. Abbrev., in press.
% \bibitem{b6} Y. Yorozu, M. Hirano, K. Oka, and Y. Tagawa, ``Electron spectroscopy studies on magneto-optical media and plastic substrate interface,'' IEEE Transl. J. Magn. Japan, vol. 2, pp. 740--741, August 1987 [Digests 9th Annual Conf. Magnetics Japan, p. 301, 1982].
% \bibitem{b7} M. Young, The Technical Writer's Handbook. Mill Valley, CA: University Science, 1989.
% \end{thebibliography}
% \vspace{12pt}
% \color{black}
% IEEE conference templates contain guidance text for composing and formatting conference papers. Please ensure that all template text is removed from your conference paper prior to submission to the conference. Failure to remove the template text from your paper may result in your paper not being published.


\bibliographystyle{IEEEtran}
\bibliography{bibliography}

\end{document}
