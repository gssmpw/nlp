\section{Related work}
\label{sec:related_work}
    
    \subsection{Cognitive biases}
    The seminal work of Tversky and Kahneman from 1974____ introduced the concept of cognitive biases, describing three of them: \textit{representatives}, \textit{availability}, and \textit{anchoring}. In their work, the researchers found that human beings rely heavily on heuristics during the decision-making process while often being blind to logical, statistical facts.  

    These findings later evolved into the dual process theory, describing the human mind as divided into Systems 1 and 2. System 1 performs fast and intuitive decisions that heavily rely on heuristics. Inversely, System 2 performs slow decisions that are logical and rule-based. By using energy mainly for important decisions (System 2), this natural phenomenon allows the human body to save precious energy when making simplified decisions (System 1).
    %This natural phenomenon allows individuals to save precious energy by using the energy for important decisions (System 2), while simple decisions can be made by System 1.
    However, humans are prone to using the energy-saving System 1 for decision-making, even in cases that require rule-based thinking. In these cases, cognitive biases might occur ____. 

    As a counterpoint to biased reasoning, \textbf{rational reasoning} can be described based on the research by William James____ as having two components: (1) perception of a specific piece of \textbf{factual information}, and (2) a \textbf{logical consequence} of this information ____. 

    \noindent
    %Researchers have found numerous ways in which cognitive biases impact various areas of software engineering. 
    % mapping studies
    
    %In their systematic mapping study, Mohanani et al. ____ found over 37 biases that have been researched in the software engineering context. The four most commonly researched biases are \textit{anchoring and adjustment, availability bias, confirmation bias}, and \textit{optimism bias}. Most studies on cognitive biases are primarily empirical studies (72\%), predominantly experiments. 
    
    % The following three are found to be particluarly relevant for architectural decision-making____:
    % \begin{itemize}
    %     \item 
    % \end{itemize}
    

    
    %It impacts the individuals who expect more successful outcomes for themselves than their peers ____.
    
    %\subsection{Cognitive biases impact on Software Engineering}
    
    %Two SWEBOK knowledge areas that researchers mainly focused on so far are Management and Design.
    
    %Chakraborty et al. ____ found ten categories of biases that influence software developers during code implementation: (1) preconceptions (related to developers' beliefs), (2) ownership (related to preferring code created by the specific developer), (3) Fixation (related to fixating on particular ideas/solutions), (4) Resort o default (related to keeping the status-quo), (5) Optimism (related to over-optimistic assumptions), (6) Convenience (preferring short-term benefits over long-term), (7) Subconscious action (prioritizing most visible issues), (8) Blisfull ignorance (related to habitualy ignoring inconvenient information), (9) Superficial selection (related to choosing solutions that 'look good' at first glance and (10) memory bias (preference for newly obtained information). An example of bias impact from the fixation group is how one experiment participant anchored on the idea that a function's definition had to be changed when, in fact, the software only had to be rebooted.

%single instances
    %In testing, \textit{confirmation bias }impacts developers who are prone to writing unit tests proving that the software works correctly instead of finding defects ____.

    % In project risk management, optimism bias makes individuals focus more on the pros than the cons of choosing a risk management plan ____.
    

    %In software effort estimation, developers are strongly influenced by \textit{anchoring} and thus often fixate on the first numerical estimate that appears during a conversation ____.
    

%rr update:
    %\subsection{Cognitive biases and Architectural Decision-Making}
    Viewing software architecture as a set of design decisions is a well-established concept____. It has resulted in a substantial amount of research regarding issues related to ADM, such as documenting architectural decisions~____, models of architectural decisions____, decision-making best practices ____, and human aspects of ADM ____, particularly cognitive biases ____.

    The impact of cognitive biases on ADM can have severe consequences, such as designing sub-par solutions ____ or incurring dangerous architectural technical debt ____.

    Based on previous research on cognitive biases in ADM ____, as well as the most often researched biases in software engineering ____, this study focuses on the following three cognitive biases: %\textit{anchoring, confirmation bias, and optimism bias.}

    \textbf{\textit{Anchoring bias}} is the decision-maker's preference for initial information/ideas/so\-lu\-tions (which then become an `anchor') ____. 
    \color{black}
    In ADM, for example, architects may anchor on the first solution idea that comes to their minds, and they may refuse to change or adjust this solution, even when it becomes necessary____.
    \color{black}
    % Anchoring may, for example, influence developers by fixating them on the first numerical estimate that appears during a conversation, which may cause their time estimates to be inaccurate ____.
    %An example impact of \textit{anchoring} can be observed in cooperative group decision-making, where decisions are made in a fast implicit negotiation among group members that usually happens in a heuristic fashion ____.

    \textbf{\textit{Confirmation bias}} is the tendency to purposefully search and interpret information to verify one's beliefs ____. 
\color{black}
    For example, due to confirmation bias, an architect may have the belief that microservices are the best architectural style and insist on designing a microservice-based system, even when it is not appropriate____.
    \color{black}
    % An example of confirmation bias would be developers' proneness to writing unit tests proving that the software works correctly instead of finding defects ____.
    %It makes humans less likely to change their views, e.g., political views, despite counter-evidence ____.
    
    \textbf{\textit{Optimism bias}} is the overestimation of the probability of positive future outcomes____. 
    \color{black}
    This may affect architects by leading them to underestimate the maximum amount of requests that would be sent to a particular component____.
    \color{black}
    % An example of optimism bias is its impact on project risk management, where individuals ignore potential risks while they focus on the pros of their preferred choices ____.


% \begin{table*}
% \caption{Participants}
% \label{tab:participants}
% \footnotesize
% \begin{tabular}{|l|p{0.6cm}|l|p{0.9cm}|p{1.1cm}|p{1.9cm}|p{2.5cm}|p{1.6cm}|p{0.8cm}|}%|p{1.2cm}|}
% \hline
% %No.&Pair No.& Group & Age (years) & Experience working in IT (years)& Role & Company Domain &  Company size (no. of employees) & Pilot or Main & Decisions discussed\\
% No.&Pair & Group & Age & IT  & Role & Company  &  Company & Pilot\\%  & Decisions \\
% &No.&  & (years) &  exp. &  & domain &  size &   % & discussed\\
% & &  &  &  & (years)&  &  &  (employees) &   \\%&  \\
% \hline
% 1& P1 &Workshop &28  & 7  &	Developer				& Digital payment					& 21-100 &Yes \\% & 4\\ %Software Developer
% %\hline
% 2& P1 &Control &28  & 2  & 	Developer				& Digital payment					& 21-100&Yes\\% & 5\\ %Software Developer
% %\hline
% 3 & P2 &Workshop & 40 & 18 &  Analyst & Digital payment & 21 - 100&Yes \\% & 4\\ %Systems Analyst
% %\hline 
% 4& P2 &Control & 37 & 5 &    Product Manager & Digital payment & 21 - 100&Yes \\% & 2\\ % Product Project Manager
% \hline 
% 5&P3 &Workshop & 52	& 20 &	Developer		& Media						& over 8000&No \\% & 6 \\ %Senior Software Developer
% %\hline
% 6&P3&Control& 59	& 38 &	Developer		& Media						& over 8000&No \\% & 11\\ %Senior Software Developer
% %\hline
% 7&P4&Workshop& 42	& 20 &	Developer		& Marketing Services	& 101-500&No \\% & 9\\ %Senior Software Developer
% %\hline
% 8&P4&Control& 40	& 20 &	 Developer		& Marketing Services	& 101-500&No \\% & 19\\ %Senior Software Developer
% %\hline
% 9&P5&Workshop& 42	& 20 &	CTO								& Finance					& 101-500&No \\% &11\\
% %\hline
% 10&P5&Control& 52	& 30 &	Architect						& Finance					& 101-500&No \\% &8\\
% %\hline
% 11&P6&Workshop& 39	& 7	 &	Architect 	& Signal processing 		& 500-5000&No \\% 11\\ %Software Engineer/Architect
% %\hline
% 12&P6&Control& 59	& 38 &	Systems Engineer				& Signal processing		& 500-5000&No \\% &12\\
% %\hline
% 13&P7&Workshop& 46	& 20 &	Architect				& Retail					& over 5000&No \\% &9\\ %Solution Architect
% %\hline
% 14&P7&Control &41	& 19 &	Architect				& Retail					& over 5000&No \\% &6\\ %Solution Architect
% %\hline

% 15&P8&Workshop& 29	& 5	 &	Developer 	& Education		& 500-5000&No \\% &5\\ %Software Developer
% %\hline 
% 16&P8&Control& 30	& 4 &	Developer				& Education		& 500-5000&No \\% &12\\ %Software Developer
% %\hline
% 17&P9&?& 37	& 5 &	Product Manager				& Entertainment					& 21-1000&No \\% &16\\ %Product Project Manager
% %\hline
% 18&P9&? &40	& 18 &	Analyst				& Entertainment					& 21-1000&No \\% &18\\ %Systems Analyst	
% \hline
% \end{tabular}
% \end{table*}





% \begin{table}
% \caption{Participants \\ \tiny{W-workshop group / C-Control group, \\ 
% Age - Age in years, \\
% IT exp. - commercial experience in years, \\
% DOM - Company domain, \\
% SIZE - Company size: XS = below 20, S=21-100, M=101-500, L=501-5000, XL= over 5000
% }}
% \label{tab:participants}
% \footnotesize
% %\begin{tabular}{|c|p{0.2cm}|p{0.25cm}|l|p{0.9cm}|p{1cm}|p{1.5cm}|p{1.5cm}|p{1.6cm}|p{0.7cm}|}%|p{1.2cm}|}
% \begin{tabular}{|c|p{0.2cm}|p{0.25cm}|p{0.4cm}|p{0.6cm}|p{0.6cm}|p{1.2cm}|p{1.2cm}|p{0.4cm}|}%p{0.2cm}|}%|p{1.2cm}|}
% \hline
% %No.&Pair No.& Group & Age (years) & Experience working in IT (years)& Role & Company Domain &  Company size (no. of employees) & Pilot or Main & Decisions discussed\\
% &No&Pair & C/W & Age   & IT   & Role & DOM & SIZE \\%  & Decisions \\
% &   &  &      & & exp.&       &   &       \\% & discussed\\
% %&   &No.  &      & (years)& exp.&       & domain  &  size     \\% & discussed\\
% %&   &     &      &        & (years)&     &         &  (staff)    \\%&  \\
% \hline

% \multirow{4}{*}{\rotatebox[origin=c]{90}{Pilot}}
% &1& P1 &W &28  & 7  &	Developer				& Digital payment					& S  \\
% % & 4\\ %Software Developer
% %\hdashline
% %\hline
% &2& P1 &C &28  & 2  & 	Developer				& Digital payment					& S\\% & 5\\ %Software Developer
% %\hline
% &3 & P2 &W & 40 & 18 &  Analyst & Digital payment & S \\% & 4\\ %Systems Analyst
% %\hline 
% &4& P2 &C & 37 & 5 &    PM & Digital payment & S \\% & 2\\ % Product Project Manager
% \hline 
% \multirow{14}{*}{\rotatebox[origin=c]{90}{Main - practitioners}}
% &5&P3 &W & 52	& 20 &	Developer		& Media						& XL \\% & 6 \\ %Senior Software Developer
% %\hline
% &6&P3&C& 59	& 38 &	Developer		& Media						& XL \\% & 11\\ %Senior Software Developer
% %\hline
% &7&P4&W& 42	& 20 &	Developer		& Marketing& M \\% & 9\\ %Senior Software Developer
% %\hline
% &8&P4&C& 40	& 20 &	 Developer		& Marketing& M \\% & 19\\ %Senior Software Developer
% %\hline
% &9&P5&W& 42	& 20 &	CTO								& Finance					& M \\% &11\\
% %\hline
% &10&P5&C& 52	& 30 &	Architect						& Finance					& M \\% &8\\
% %\hline
% &11&P6&W& 39	& 7	 &	Architect 	& Audio 		& L \\% 11\\ %Software Engineer/Architect %Signal processing
% %\hline
% &12&P6&C& 59	& 38 &	Sys. Eng.				& Audio	& L \\% &12\\
% %\hline
% &13&P7&W& 46	& 20 &	Architect				& Retail					& XL \\% &9\\ %Solution Architect
% %\hline
% &14&P7&C &41	& 19 &	Architect				& Retail					& XL \\% &6\\ %Solution Architect
% %\hline

% &15&P8&W& 29	& 5	 &	Developer 	& Education		& L \\% &5\\ %Software Developer
% %\hline 
% &16&P8&C& 30	& 4 &	Developer				& Education		& L \\% &12\\ %Software Developer
% %\hline
% &17&P9&W& 39	& 20 &	PM			& Government					& S \\% &16\\ %Product Project Manager
% %\hline
% &18&P9&C &42	& 19 &	Coordinator				& Government					& S \\%Systems Development Coordinator
% &19&P10&W& 23	& 2 &	Developer				& Education					& S \\% &16\\ %Product Project Manager
% %\hline
% &20&P10&C &23	& 2 &	Developer				& Education					& S \\% &18\\ %Systems Analyst	
% \hline
% \multirow{14}{*}{\rotatebox[origin=c]{90}{Main - students}}
% &	21	&	S1	&	W	&	24	&	0	&	-	&	-	&	-	\\
% &	22	&	S1	&	C	&	24	&	2	&	Developer	&	Software	&	XL	\\ %SW Engineer/Developer
% &	23	&	S2	&	W	&	23	&	0.6	&	Developer	&	Electronics	&	XL	\\ %Python developer
% &	24	&	S2	&	C	&	24	&	3	&	Developer	&	Utilities	&	L	\\ %Software Engineer
% &	25	&	S3	&	W	&	23	&	2	&	Developer	&	Insurance&	S	\\ %Fullstack 
% &	26	&	S3	&	C	&	23	&	1.5	&	Developer	&	ERP	&	M	\\ %Junior 
% &	27	&	S4	&	W	&	21	&	0.4	&	Developer	&	\mbox{Web design}	&	S	\\ %Frontend Developer
% &	28	&	S4	&	C	&	22	&	1.5	&	Developer	&	Security	&	XL\\ %Cybersecurity T-SQL Developer
% &	29	&	S5	&	W	&	20	&	0	&	-	&	-	&	-	\\
% &	30	&	S5	&	C	&	21	&	0	&	-	&	-	&	-	\\
% &	31	&	S6	&	W	&	23	&	1	&	Tester	&	\mbox{Web dev.}&	S	\\ %Data Procesing website	
% &	32	&	S6	&	C	&	24	&	1.5	&	Developer	&	ERP	&	M	\\
% &	33	&	S7	&	W	&	26	&	1.5	&	Pre-sales	&	Sales	&	XL	\\
% &	34	&	S7	&	C	&	23	&	0.8	&	BMS Eng.&	Energetics 	&	L	\\
% &	35	&	S9	&	W	&	27	&	3	&	ML Eng.	&	Security	&	L	\\
% &	36	&	S8	&	C	&	26	&	1	&	DevOps	&	Earth Observation	&	M	\\
% \hline
% \end{tabular}
% \vspace*{-1.5\baselineskip}
% \end{table}


\begin{table}
\caption{Participants \\ \tiny{W-workshop group / C-Control group, \\ 
Age - Age in years, \\
IT exp. - commercial experience in years, \\
DOM - Company domain, \\
SIZE - Company size: S = below 100, M=101-500, L=501-5000, XL= over 5000
}}
\label{tab:participants}
\center
\footnotesize
\scalebox{0.8}{
%\begin{tabular}{|c|p{0.2cm}|p{0.25cm}|l|p{0.9cm}|p{1cm}|p{1.5cm}|p{1.5cm}|p{1.6cm}|p{0.7cm}|}%|p{1.2cm}|}
\begin{tabular}{|c|c|c|c|c|c|c|c|}%p{0.2cm}|}%|p{1.2cm}|}
\hline
%No.&Pair No.& Group & Age (years) & Experience working in IT (years)& Role & Company Domain &  Company size (no. of employees) & Pilot or Main & Decisions discussed\\
&No&Pair & C/W   & IT   & Role & DOM & SIZE \\%  & Decisions \\
&   &  &      & exp.&       &   &       \\% & discussed\\
%&   &No.  &      & (years)& exp.&       & domain  &  size     \\% & discussed\\
%&   &     &      &        & (years)&     &         &  (staff)    \\%&  \\
\hline

\multirow{4}{*}{\rotatebox[origin=c]{90}{Pilot}}
&1& P1 &W   & 6-10  &	Developer				& Digital					& S  \\
& &   &  &   &    			& payment					&   \\% & 4\\ %Software % & 4\\ %Software Developer
%\hdashline
%\hline
&2& P1 &C   & 1-2  & 	Developer				& Digital					& S\\% 
& &   &  &   &    		& payment					&   \\% & 4\\ %Software & 5\\ %Software Developer
%\hline
&3 & P2 &W  & 11-20 &  Analyst & Digital & S \\% & 4\\ %Systems Analyst
& &   &  &   &    	& payment					&   \\% & 4\\ %Software 
%\hline 
&4& P2 &C  & 3-5 &    Product & Digital & S \\% & 2\\ % Product Project Manager
&&&&  &	Manager			& payment 				&  \\% &16\\ 
\hline 
\multirow{14}{*}{\rotatebox[origin=c]{90}{Main - practitioners}}
&5&P3 &W & 11-20 &	Developer		& Media						& XL \\% & 6 \\ %Senior Software Developer
%\hline
&6&P3&C	& \textgreater 20 &	Developer		& Media						& XL \\% & 11\\ %Senior Software Developer
%\hline
&7&P4&W	& 11-20 &	Developer		& Marketing& M \\% & 9\\ %Senior Software Developer
%\hline
&8&P4&C	& 11-20 &	 Developer		& Marketing& M \\% & 19\\ %Senior Software Developer
%\hline
&9&P5&W	& 11-20 &	CTO								& Finance					& M \\% &11\\
%\hline
&10&P5&C& \textgreater 20 &	Architect						& Finance					& M \\% &8\\
%\hline
&11&P6&W	& 6-10	 &	Architect 	& Audio 		& L \\% 11\\ %Software Engineer/Architect %Signal processing
%\hline
&12&P6&C& \textgreater 20 &	Sys. Eng.				& Audio	& L \\% &12\\
%\hline
&13&P7&W	& 11-20 &	Architect				& Retail					& XL \\% &9\\ %Solution Architect
%\hline
&14&P7&C & 11-20 &	Architect				& Retail					& XL \\% &6\\ %Solution Architect
%\hline

&15&P8&W&  3-5	 &	Developer 	& Education		& L \\% &5\\ %Software Developer
%\hline 
&16&P8&C& 3-5 &	Developer				& Education		& L \\% &12\\ %Software Developer
%\hline
&17&P9&W& 11-20 &	Product			& Government					& S \\% &16\\ 
&&&&  &	Manager			& 				&  \\% &16\\ 
%Product Project Manager
%\hline
&18&P9&C& 11-20 &	Coordinator				& Government					& S \\%Systems Development Coordinator
&19&P10&W& 1-2 &	Developer				& Education					& L \\% &16\\ %Product Project Manager
%\hline
&20&P10&C& 1-2 &	Developer				& Education					& L \\% &18\\ %Systems Analyst	
\hline
\multirow{14}{*}{\rotatebox[origin=c]{90}{Main - students}}
&	21	&	S1	&	W	&	-	&	-	&	-	&	-	\\
&	22	&	S1	&	C	&	1-2	&	Developer	&	Mobile	&	XL	\\ %SW Engineer/Developer
&	23	&	S2	&	W	&	\textless 1	&	Developer	&	Electronics	&	XL	\\ %Python developer
&	24	&	S2	&	C		&	3-5	&	Developer	&	Utilities	&	L	\\ %Software Engineer
&	25	&	S3	&	W	&	1-2		&	Developer	&	Insurance&	S	\\ %Fullstack 
&	26	&	S3	&	C	&	1-2		&	Developer	&	ERP	&	M	\\ %Junior 
&	27	&	S4	&	W	&	\textless 1		&	Developer	&	Ecommerce	&	S	\\ %Frontend Developer
&	28	&	S4	&	C	&	1-2		&	Developer	&	Security	&	XL\\ %Cybersecurity T-SQL Developer
&	29	&	S5	&	W	&	-	&	-	&	-	&	-	\\
&	30	&	S5	&	C	&	-	&	-	&	-	&	-	\\
&	31	&	S6	&	W	&	1-2	&	Tester	&	\mbox{Web dev.}&	S	\\ %Data Procesing website	
&	32	&	S6	&	C	&	1-2		&	Developer	&	ERP	&	M	\\
&	33	&	S7	&	W	&	1-2		&	Pre-sales	&	Sales	&	XL	\\
&	34	&	S7	&	C	&	\textless 1		&	Developer&	Energetics 	&	L	\\
&	35	&	S9	&	W	&	3-5	&	Developer	&	Security	&	L	\\
&	36	&	S8	&	C	&	1-2	&	DevOps	&	Earth	&	M	\\
&		&		&		&		&		&	Observation	&	\\
\hline
\end{tabular}
}
\vspace*{-1.5\baselineskip}
\end{table}


    \subsection{Debiasing} 
%rr add
Debiasing refers to the process of identifying and mitigating cognitive biases that may affect decision-making.
    Possible types of debiasing treatments can be described using the levels proposed by Fischhoff ____. A- and B-level treatments consist of informing practitioners about cognitive biases (A) and how they may impact practitioners (B). Levels C and D require more organizational resources: a C-level treatment requires giving personalized feedback to each debiased person, and the D-level requires extensive long-term training.

    %Debiasing has been a topic of interest in many research domains, e.g., economics ________. I
    In the field of software engineering, two notable C-level debiasing treatments were reportedly successful. Firstly,  rationalizing development time estimates by Shepperd et al.____, where researchers purposefully anchored software developers on pessimistic estimates to counter over-optimistic predictions.     
    Secondly, in ADM, in their short paper, Borowa et al. ____ reported achieving a debiasing effect on a group of student participants. However, their experiment had several weaknesses that our study avoids:
    \begin{enumerate}
        \item The experiment employed the pretest-posttest design, which is known for numerous threats to validity____. For example, the debiasing effect may have not resulted from the workshop but from the student participants' learning while performing the pretest task.
        \item The participants were all students\color{black}, making it unclear whether the results would be the same for \color{black} experienced practitioners.
        \item The participants performed different tasks before and after the workshop, so comparing these may not have been reliable. 
        \color{black}
        For example, if designing one architecture was more difficult than the other, participants' bias occurrence count could be increased by this added difficulty. 
        \color{black}
        \item The participants performed purely theoretical tasks by designing architectures for domains that they may not have had any prior knowledge about. This may have caused mistakes due to their lack of domain knowledge.
        % \item The experiment was three hours long with no breaks. As such, the participants may have been impacted by fatigue during the posttest task.
        \color{black}
        \item The debiasing effect was achieved by improving the amount of non-biased arguments, \textbf{but, despite the authors' efforts, the bias occurrence was not reduced}. We measured the bias occurrences using a similar method to Borowa et al.____ and did achieve a bias reduction. 
        \color{black}
    \end{enumerate}
    
    
    % \textbf{the debiased students were not impacted by biases less often, but instead used more non-biased arguments in their reasoning}. Additionally, the students performed \textbf{purely theoretical tasks} and their performance was compared through them performing \textbf{two different tasks before and after a workshop}.
    % However, neither this nor any other recent study ascertained whether the effect on experienced practitioners, and real-life architectures, would be similar. Through our study, we have strived to empirically verify whether a debiasing workshop is effective for experienced practitioners, working on existing architectures. 
    
    %This verification is crucial because of the ongoing debate in the software engineering research community about whether students are a reliable source of data ____.