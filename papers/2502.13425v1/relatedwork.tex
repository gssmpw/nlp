\section{Related work}
\label{sec:related}
Kd-Trees were introduced by Jon Bentley in 1975~\cite{Bentley1975MultidimensionalBS}. Further improvements were made by him, Friedman and Finkel~\cite{kd2_bentley} and many others. Bregman divergences were introduced by Lev Bregman in 1967~\cite{BREGMAN1967200}.

\ifapprox
    Many computational geometry techniques have been extended to operate with Bregman divergences instead of a metric. In the context of nearest neighbour search, Cayton blazed the trail by extending~\cite{CaytonBBTrees} ball-trees and implementing prototype software for the KL and IS divergences. He also proved theoretical results towards extending Kd-trees~\cite{caytonphd2009}, which we strengthen as well as provide algorithms and an efficient implementation. In turn, Nielsen, Piro, and Barlaud extended Vantage point trees~\cite{5202635, VPTree}. The same authors further improved Bregman ball-trees~\cite{Bregman_Ball_Trees, BBTreepp}. R-trees and VA-files were extended by Zhang and collaborators~\cite{RTreesVAFiles}, inspiring Song and collaborators~\cite{song2020brepartition}. Many approximate search methods have been adapted to the Bregman setting. We discuss related works and present experimental comparisons in Appendix~\ref{sec:approxDisc}.
\else
    Many computational geometry techniques have been extended to operate with Bregman divergences instead of a metric. In the context of nearest neighbour search, Cayton blazed the trail by extending~\cite{CaytonBBTrees} ball-trees and implementing prototype software for the KL and IS divergences. He also proved theoretical results towards extending Kd-trees~\cite{caytonphd2009}, which we strengthen as well as provide algorithms and an efficient implementation. In turn, Nielsen, Piro, and Barlaud extended Vantage point trees~\cite{5202635, VPTree}. The same authors further improved Bregman ball-trees~\cite{Bregman_Ball_Trees, BBTreeGithub}. R-trees and VA-files were extended by Zhang and collaborators~\cite{RTreesVAFiles}, inspiring Song and collaborators~\cite{song2020brepartition}. Ring-trees combined with a quad-tree decomposition have been proven to work sublinearly for finding approximate nearest neighbours by Abdullah, Moeller, and Venkatasubramanian~\cite{Bregman_ring_tree}. In 2013, Boytsov and Naidan developed their own Bregman VP-trees extension~\cite{BoytsovNaidan_VPTrees} for approximate nearest neighbours. Naidan later incorporated his VP-trees and Cayton's ball-trees into the Non-Metric Space Library (NMSLIB)\cite{nmslib}. This library also includes other approximate Bregman similarity searches including small world graphs~\cite{MalkovPonomarenkoLogvinovKrylov_SWG}. The hierarchical navigable small world graph has been a popular choice for similarity searches in vector databases~\cite{HNSW_Zilliz, HNSW_MariaDB, HNSW_MongoDB} and perform well in benchmarks for metrics~\cite{ANN_Benchmarks}. However, its implementation in NMSLIB is currently experimental for Bregman divergences~\cite{HNSW_Git}.
    Recently Abdelkader, Arya, da Fonseca and Mount proposed an approach to proximity search in non-metric settings, which includes Bregman divergences~\cite{abdelkader2019approximate}; as we understand it, this has not yet been implemented.
\fi

More broadly, Banerjee and collaborators extended $k$-means clustering~\cite{JMLR:v6:banerjee05b} to arbitrary Bregman divergences -- with the surprising twist that the existing algorithm works without changes. Coresets have also been extended to the Bregman setting by Ackermann and Bl{\"o}mer~\cite{Ackermann_Blomer_k_median}. Nielsen, Boissonnat, and Nock developed Bregman Voronoi diagrams and Delaunay triangulations~\cite{Bregman_Voronoi}. Edelsbrunner and Wagner extended topological data analysis methods to the Bregman case~\cite{EdWa16}. 

% This should be here. The software I mention below also includes the below things.
%On a slightly different note, we mention some developments related to geometries on the standard simplex coming from the computational geometry community. One example is the work of Nielsen and Shao, considering Hilbert geometries on the simplex~\cite{HSGball-2017}. Kyng, Phillips and Venkatasubramanian showed Johnson--Lindenstrauss-type results on the simplex~\cite{kyng2010johnson, kyng2011dimensionality}.

In the Euclidean case, robust software is available for all of these techniques.
One popular package in the Euclidean case is the ANN library by Mount and Arya~\cite{ANN_kd, ANN_Manual, ANN_boundary, ANN_optimalAlgorithm}. Our current implementation is inspired by this library. % which we will compare with the Bregman ball-tree implementation by Cayton~\cite{BBTreeGithub} and heuristic methods in NMSLIB~\cite{nmslib}.