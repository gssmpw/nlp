\section{Related Work}
% broad view of essentialness
In previous works, PCD has been developed to account for uncertainties in robot controllers and environment sensing to ensure the safety of robots operating in the real world, such as drones and automatic cars ____. However, PCD for high-DOF robots is particularly challenging as they often need to interact closely with objects in cluttered scenes, such as during pick-and-place tasks. Although these robots typically operate at slower speeds, the complexity of cluttered and unstructured environments amplifies the impact of sensing uncertainty. The spatial information of the environmental objects is retrieved by noisy sensors and usually from a few partial views ____. 

Early PCD methods, such as Monte Carlo-based approaches, compute the collision probability accurately but are computationally demanding, limiting their practical use in real-time applications ____. Some PCD methods that have closed-form solutions are fast to compute and are suitable for high-speed robots ____. Nevertheless, they only support using simple geometrics (\eg{ points, spheres, or ellipsoids}) to represent robots and environmental objects, which can give a conservative result if the objects cannot be accurately represented. Besides, they only consider the position estimation errors of objects. Although some PCD methods support convex complex geometric models (\eg{ mesh}), their performance either depends on the surface complexity of the model ____ or needs to be iteratively improved ____. Nevertheless, they only support the position estimation errors of objects. The learning-based method uses the point cloud of the environment and does not assume the probabilistic model per object, but the computation time is too expansive and needs to be trained for new robots ____. 

Despite these advancements, current PCD methods still face limitations in accurately modeling complex shapes and handling combined position and orientation uncertainties. This work addresses these gaps by introducing a robust PCD method that leverages superquadrics for improved shape approximation and integrates position and orientation uncertainties for enhanced robustness in real-world applications.