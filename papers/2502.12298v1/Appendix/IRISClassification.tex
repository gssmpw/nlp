\textbf{Experiment I: IRIS.}  %The IRIS dataset consists of 50 samples of three species of the iris flower.
%The features correspond to the length and width of the sepals and petals for each sample.
This dataset is relatively small; consequently, we only consider a shallow network with three fully connected layers and 2953 parameters. We set the history size and maximum iterations for the proposed approach and L-BFGS to 10. Fig.\ \ref{fig:IRIS}(a) shows the comparative performance of all the methods. Note that our proposed method (ARCs-LSR1) achieves the highest classification accuracy in the fewest number of epochs.

\begin{figure}[!ht]
    \centering
    %\subfloat[IRIS dataset]
{\adjincludegraphics[width=0.75\linewidth, trim={{.05\width}  {0.05\height} {.06\width} {0.15\height}},clip]{./Figures/iris_test_paper.png}}
    \caption{The classification accuracy results for \textbf{Experiment I: IRIS}. 
	%(a) Training loss of the network. The $y$-axis represents the negative log-likelihood loss and the $x$-axis represents the number of epochs. (b) 
	%The classification accuracy for each method, i.e., 
	The percentage of testing samples correctly predicted in the testing dataset for each method is presented. Note that the proposed method (ARCs-LSR1) achieves the highest classification accuracy within the fewest number of epochs.}
    \label{fig:enter-label}
\end{figure}