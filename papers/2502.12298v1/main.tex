
\documentclass[10pt]{article} % For LaTeX2e
\usepackage[preprint]{tmlr}
% If accepted, instead use the following line for the camera-ready submission:
%\usepackage[accepted]{tmlr}
% To de-anonymize and remove mentions to TMLR (for example for posting to preprint servers), instead use the following:
%\usepackage[preprint]{tmlr}

% Optional math commands from https://github.com/goodfeli/dlbook_notation.
%%%%% NEW MATH DEFINITIONS %%%%%

\usepackage{amsmath,amsfonts,bm}
\usepackage{derivative}
% Mark sections of captions for referring to divisions of figures
\newcommand{\figleft}{{\em (Left)}}
\newcommand{\figcenter}{{\em (Center)}}
\newcommand{\figright}{{\em (Right)}}
\newcommand{\figtop}{{\em (Top)}}
\newcommand{\figbottom}{{\em (Bottom)}}
\newcommand{\captiona}{{\em (a)}}
\newcommand{\captionb}{{\em (b)}}
\newcommand{\captionc}{{\em (c)}}
\newcommand{\captiond}{{\em (d)}}

% Highlight a newly defined term
\newcommand{\newterm}[1]{{\bf #1}}

% Derivative d 
\newcommand{\deriv}{{\mathrm{d}}}

% Figure reference, lower-case.
\def\figref#1{figure~\ref{#1}}
% Figure reference, capital. For start of sentence
\def\Figref#1{Figure~\ref{#1}}
\def\twofigref#1#2{figures \ref{#1} and \ref{#2}}
\def\quadfigref#1#2#3#4{figures \ref{#1}, \ref{#2}, \ref{#3} and \ref{#4}}
% Section reference, lower-case.
\def\secref#1{section~\ref{#1}}
% Section reference, capital.
\def\Secref#1{Section~\ref{#1}}
% Reference to two sections.
\def\twosecrefs#1#2{sections \ref{#1} and \ref{#2}}
% Reference to three sections.
\def\secrefs#1#2#3{sections \ref{#1}, \ref{#2} and \ref{#3}}
% Reference to an equation, lower-case.
\def\eqref#1{equation~\ref{#1}}
% Reference to an equation, upper case
\def\Eqref#1{Equation~\ref{#1}}
% A raw reference to an equation---avoid using if possible
\def\plaineqref#1{\ref{#1}}
% Reference to a chapter, lower-case.
\def\chapref#1{chapter~\ref{#1}}
% Reference to an equation, upper case.
\def\Chapref#1{Chapter~\ref{#1}}
% Reference to a range of chapters
\def\rangechapref#1#2{chapters\ref{#1}--\ref{#2}}
% Reference to an algorithm, lower-case.
\def\algref#1{algorithm~\ref{#1}}
% Reference to an algorithm, upper case.
\def\Algref#1{Algorithm~\ref{#1}}
\def\twoalgref#1#2{algorithms \ref{#1} and \ref{#2}}
\def\Twoalgref#1#2{Algorithms \ref{#1} and \ref{#2}}
% Reference to a part, lower case
\def\partref#1{part~\ref{#1}}
% Reference to a part, upper case
\def\Partref#1{Part~\ref{#1}}
\def\twopartref#1#2{parts \ref{#1} and \ref{#2}}

\def\ceil#1{\lceil #1 \rceil}
\def\floor#1{\lfloor #1 \rfloor}
\def\1{\bm{1}}
\newcommand{\train}{\mathcal{D}}
\newcommand{\valid}{\mathcal{D_{\mathrm{valid}}}}
\newcommand{\test}{\mathcal{D_{\mathrm{test}}}}

\def\eps{{\epsilon}}


% Random variables
\def\reta{{\textnormal{$\eta$}}}
\def\ra{{\textnormal{a}}}
\def\rb{{\textnormal{b}}}
\def\rc{{\textnormal{c}}}
\def\rd{{\textnormal{d}}}
\def\re{{\textnormal{e}}}
\def\rf{{\textnormal{f}}}
\def\rg{{\textnormal{g}}}
\def\rh{{\textnormal{h}}}
\def\ri{{\textnormal{i}}}
\def\rj{{\textnormal{j}}}
\def\rk{{\textnormal{k}}}
\def\rl{{\textnormal{l}}}
% rm is already a command, just don't name any random variables m
\def\rn{{\textnormal{n}}}
\def\ro{{\textnormal{o}}}
\def\rp{{\textnormal{p}}}
\def\rq{{\textnormal{q}}}
\def\rr{{\textnormal{r}}}
\def\rs{{\textnormal{s}}}
\def\rt{{\textnormal{t}}}
\def\ru{{\textnormal{u}}}
\def\rv{{\textnormal{v}}}
\def\rw{{\textnormal{w}}}
\def\rx{{\textnormal{x}}}
\def\ry{{\textnormal{y}}}
\def\rz{{\textnormal{z}}}

% Random vectors
\def\rvepsilon{{\mathbf{\epsilon}}}
\def\rvphi{{\mathbf{\phi}}}
\def\rvtheta{{\mathbf{\theta}}}
\def\rva{{\mathbf{a}}}
\def\rvb{{\mathbf{b}}}
\def\rvc{{\mathbf{c}}}
\def\rvd{{\mathbf{d}}}
\def\rve{{\mathbf{e}}}
\def\rvf{{\mathbf{f}}}
\def\rvg{{\mathbf{g}}}
\def\rvh{{\mathbf{h}}}
\def\rvu{{\mathbf{i}}}
\def\rvj{{\mathbf{j}}}
\def\rvk{{\mathbf{k}}}
\def\rvl{{\mathbf{l}}}
\def\rvm{{\mathbf{m}}}
\def\rvn{{\mathbf{n}}}
\def\rvo{{\mathbf{o}}}
\def\rvp{{\mathbf{p}}}
\def\rvq{{\mathbf{q}}}
\def\rvr{{\mathbf{r}}}
\def\rvs{{\mathbf{s}}}
\def\rvt{{\mathbf{t}}}
\def\rvu{{\mathbf{u}}}
\def\rvv{{\mathbf{v}}}
\def\rvw{{\mathbf{w}}}
\def\rvx{{\mathbf{x}}}
\def\rvy{{\mathbf{y}}}
\def\rvz{{\mathbf{z}}}

% Elements of random vectors
\def\erva{{\textnormal{a}}}
\def\ervb{{\textnormal{b}}}
\def\ervc{{\textnormal{c}}}
\def\ervd{{\textnormal{d}}}
\def\erve{{\textnormal{e}}}
\def\ervf{{\textnormal{f}}}
\def\ervg{{\textnormal{g}}}
\def\ervh{{\textnormal{h}}}
\def\ervi{{\textnormal{i}}}
\def\ervj{{\textnormal{j}}}
\def\ervk{{\textnormal{k}}}
\def\ervl{{\textnormal{l}}}
\def\ervm{{\textnormal{m}}}
\def\ervn{{\textnormal{n}}}
\def\ervo{{\textnormal{o}}}
\def\ervp{{\textnormal{p}}}
\def\ervq{{\textnormal{q}}}
\def\ervr{{\textnormal{r}}}
\def\ervs{{\textnormal{s}}}
\def\ervt{{\textnormal{t}}}
\def\ervu{{\textnormal{u}}}
\def\ervv{{\textnormal{v}}}
\def\ervw{{\textnormal{w}}}
\def\ervx{{\textnormal{x}}}
\def\ervy{{\textnormal{y}}}
\def\ervz{{\textnormal{z}}}

% Random matrices
\def\rmA{{\mathbf{A}}}
\def\rmB{{\mathbf{B}}}
\def\rmC{{\mathbf{C}}}
\def\rmD{{\mathbf{D}}}
\def\rmE{{\mathbf{E}}}
\def\rmF{{\mathbf{F}}}
\def\rmG{{\mathbf{G}}}
\def\rmH{{\mathbf{H}}}
\def\rmI{{\mathbf{I}}}
\def\rmJ{{\mathbf{J}}}
\def\rmK{{\mathbf{K}}}
\def\rmL{{\mathbf{L}}}
\def\rmM{{\mathbf{M}}}
\def\rmN{{\mathbf{N}}}
\def\rmO{{\mathbf{O}}}
\def\rmP{{\mathbf{P}}}
\def\rmQ{{\mathbf{Q}}}
\def\rmR{{\mathbf{R}}}
\def\rmS{{\mathbf{S}}}
\def\rmT{{\mathbf{T}}}
\def\rmU{{\mathbf{U}}}
\def\rmV{{\mathbf{V}}}
\def\rmW{{\mathbf{W}}}
\def\rmX{{\mathbf{X}}}
\def\rmY{{\mathbf{Y}}}
\def\rmZ{{\mathbf{Z}}}

% Elements of random matrices
\def\ermA{{\textnormal{A}}}
\def\ermB{{\textnormal{B}}}
\def\ermC{{\textnormal{C}}}
\def\ermD{{\textnormal{D}}}
\def\ermE{{\textnormal{E}}}
\def\ermF{{\textnormal{F}}}
\def\ermG{{\textnormal{G}}}
\def\ermH{{\textnormal{H}}}
\def\ermI{{\textnormal{I}}}
\def\ermJ{{\textnormal{J}}}
\def\ermK{{\textnormal{K}}}
\def\ermL{{\textnormal{L}}}
\def\ermM{{\textnormal{M}}}
\def\ermN{{\textnormal{N}}}
\def\ermO{{\textnormal{O}}}
\def\ermP{{\textnormal{P}}}
\def\ermQ{{\textnormal{Q}}}
\def\ermR{{\textnormal{R}}}
\def\ermS{{\textnormal{S}}}
\def\ermT{{\textnormal{T}}}
\def\ermU{{\textnormal{U}}}
\def\ermV{{\textnormal{V}}}
\def\ermW{{\textnormal{W}}}
\def\ermX{{\textnormal{X}}}
\def\ermY{{\textnormal{Y}}}
\def\ermZ{{\textnormal{Z}}}

% Vectors
\def\vzero{{\bm{0}}}
\def\vone{{\bm{1}}}
\def\vmu{{\bm{\mu}}}
\def\vtheta{{\bm{\theta}}}
\def\vphi{{\bm{\phi}}}
\def\va{{\bm{a}}}
\def\vb{{\bm{b}}}
\def\vc{{\bm{c}}}
\def\vd{{\bm{d}}}
\def\ve{{\bm{e}}}
\def\vf{{\bm{f}}}
\def\vg{{\bm{g}}}
\def\vh{{\bm{h}}}
\def\vi{{\bm{i}}}
\def\vj{{\bm{j}}}
\def\vk{{\bm{k}}}
\def\vl{{\bm{l}}}
\def\vm{{\bm{m}}}
\def\vn{{\bm{n}}}
\def\vo{{\bm{o}}}
\def\vp{{\bm{p}}}
\def\vq{{\bm{q}}}
\def\vr{{\bm{r}}}
\def\vs{{\bm{s}}}
\def\vt{{\bm{t}}}
\def\vu{{\bm{u}}}
\def\vv{{\bm{v}}}
\def\vw{{\bm{w}}}
\def\vx{{\bm{x}}}
\def\vy{{\bm{y}}}
\def\vz{{\bm{z}}}

% Elements of vectors
\def\evalpha{{\alpha}}
\def\evbeta{{\beta}}
\def\evepsilon{{\epsilon}}
\def\evlambda{{\lambda}}
\def\evomega{{\omega}}
\def\evmu{{\mu}}
\def\evpsi{{\psi}}
\def\evsigma{{\sigma}}
\def\evtheta{{\theta}}
\def\eva{{a}}
\def\evb{{b}}
\def\evc{{c}}
\def\evd{{d}}
\def\eve{{e}}
\def\evf{{f}}
\def\evg{{g}}
\def\evh{{h}}
\def\evi{{i}}
\def\evj{{j}}
\def\evk{{k}}
\def\evl{{l}}
\def\evm{{m}}
\def\evn{{n}}
\def\evo{{o}}
\def\evp{{p}}
\def\evq{{q}}
\def\evr{{r}}
\def\evs{{s}}
\def\evt{{t}}
\def\evu{{u}}
\def\evv{{v}}
\def\evw{{w}}
\def\evx{{x}}
\def\evy{{y}}
\def\evz{{z}}

% Matrix
\def\mA{{\bm{A}}}
\def\mB{{\bm{B}}}
\def\mC{{\bm{C}}}
\def\mD{{\bm{D}}}
\def\mE{{\bm{E}}}
\def\mF{{\bm{F}}}
\def\mG{{\bm{G}}}
\def\mH{{\bm{H}}}
\def\mI{{\bm{I}}}
\def\mJ{{\bm{J}}}
\def\mK{{\bm{K}}}
\def\mL{{\bm{L}}}
\def\mM{{\bm{M}}}
\def\mN{{\bm{N}}}
\def\mO{{\bm{O}}}
\def\mP{{\bm{P}}}
\def\mQ{{\bm{Q}}}
\def\mR{{\bm{R}}}
\def\mS{{\bm{S}}}
\def\mT{{\bm{T}}}
\def\mU{{\bm{U}}}
\def\mV{{\bm{V}}}
\def\mW{{\bm{W}}}
\def\mX{{\bm{X}}}
\def\mY{{\bm{Y}}}
\def\mZ{{\bm{Z}}}
\def\mBeta{{\bm{\beta}}}
\def\mPhi{{\bm{\Phi}}}
\def\mLambda{{\bm{\Lambda}}}
\def\mSigma{{\bm{\Sigma}}}

% Tensor
\DeclareMathAlphabet{\mathsfit}{\encodingdefault}{\sfdefault}{m}{sl}
\SetMathAlphabet{\mathsfit}{bold}{\encodingdefault}{\sfdefault}{bx}{n}
\newcommand{\tens}[1]{\bm{\mathsfit{#1}}}
\def\tA{{\tens{A}}}
\def\tB{{\tens{B}}}
\def\tC{{\tens{C}}}
\def\tD{{\tens{D}}}
\def\tE{{\tens{E}}}
\def\tF{{\tens{F}}}
\def\tG{{\tens{G}}}
\def\tH{{\tens{H}}}
\def\tI{{\tens{I}}}
\def\tJ{{\tens{J}}}
\def\tK{{\tens{K}}}
\def\tL{{\tens{L}}}
\def\tM{{\tens{M}}}
\def\tN{{\tens{N}}}
\def\tO{{\tens{O}}}
\def\tP{{\tens{P}}}
\def\tQ{{\tens{Q}}}
\def\tR{{\tens{R}}}
\def\tS{{\tens{S}}}
\def\tT{{\tens{T}}}
\def\tU{{\tens{U}}}
\def\tV{{\tens{V}}}
\def\tW{{\tens{W}}}
\def\tX{{\tens{X}}}
\def\tY{{\tens{Y}}}
\def\tZ{{\tens{Z}}}


% Graph
\def\gA{{\mathcal{A}}}
\def\gB{{\mathcal{B}}}
\def\gC{{\mathcal{C}}}
\def\gD{{\mathcal{D}}}
\def\gE{{\mathcal{E}}}
\def\gF{{\mathcal{F}}}
\def\gG{{\mathcal{G}}}
\def\gH{{\mathcal{H}}}
\def\gI{{\mathcal{I}}}
\def\gJ{{\mathcal{J}}}
\def\gK{{\mathcal{K}}}
\def\gL{{\mathcal{L}}}
\def\gM{{\mathcal{M}}}
\def\gN{{\mathcal{N}}}
\def\gO{{\mathcal{O}}}
\def\gP{{\mathcal{P}}}
\def\gQ{{\mathcal{Q}}}
\def\gR{{\mathcal{R}}}
\def\gS{{\mathcal{S}}}
\def\gT{{\mathcal{T}}}
\def\gU{{\mathcal{U}}}
\def\gV{{\mathcal{V}}}
\def\gW{{\mathcal{W}}}
\def\gX{{\mathcal{X}}}
\def\gY{{\mathcal{Y}}}
\def\gZ{{\mathcal{Z}}}

% Sets
\def\sA{{\mathbb{A}}}
\def\sB{{\mathbb{B}}}
\def\sC{{\mathbb{C}}}
\def\sD{{\mathbb{D}}}
% Don't use a set called E, because this would be the same as our symbol
% for expectation.
\def\sF{{\mathbb{F}}}
\def\sG{{\mathbb{G}}}
\def\sH{{\mathbb{H}}}
\def\sI{{\mathbb{I}}}
\def\sJ{{\mathbb{J}}}
\def\sK{{\mathbb{K}}}
\def\sL{{\mathbb{L}}}
\def\sM{{\mathbb{M}}}
\def\sN{{\mathbb{N}}}
\def\sO{{\mathbb{O}}}
\def\sP{{\mathbb{P}}}
\def\sQ{{\mathbb{Q}}}
\def\sR{{\mathbb{R}}}
\def\sS{{\mathbb{S}}}
\def\sT{{\mathbb{T}}}
\def\sU{{\mathbb{U}}}
\def\sV{{\mathbb{V}}}
\def\sW{{\mathbb{W}}}
\def\sX{{\mathbb{X}}}
\def\sY{{\mathbb{Y}}}
\def\sZ{{\mathbb{Z}}}

% Entries of a matrix
\def\emLambda{{\Lambda}}
\def\emA{{A}}
\def\emB{{B}}
\def\emC{{C}}
\def\emD{{D}}
\def\emE{{E}}
\def\emF{{F}}
\def\emG{{G}}
\def\emH{{H}}
\def\emI{{I}}
\def\emJ{{J}}
\def\emK{{K}}
\def\emL{{L}}
\def\emM{{M}}
\def\emN{{N}}
\def\emO{{O}}
\def\emP{{P}}
\def\emQ{{Q}}
\def\emR{{R}}
\def\emS{{S}}
\def\emT{{T}}
\def\emU{{U}}
\def\emV{{V}}
\def\emW{{W}}
\def\emX{{X}}
\def\emY{{Y}}
\def\emZ{{Z}}
\def\emSigma{{\Sigma}}

% entries of a tensor
% Same font as tensor, without \bm wrapper
\newcommand{\etens}[1]{\mathsfit{#1}}
\def\etLambda{{\etens{\Lambda}}}
\def\etA{{\etens{A}}}
\def\etB{{\etens{B}}}
\def\etC{{\etens{C}}}
\def\etD{{\etens{D}}}
\def\etE{{\etens{E}}}
\def\etF{{\etens{F}}}
\def\etG{{\etens{G}}}
\def\etH{{\etens{H}}}
\def\etI{{\etens{I}}}
\def\etJ{{\etens{J}}}
\def\etK{{\etens{K}}}
\def\etL{{\etens{L}}}
\def\etM{{\etens{M}}}
\def\etN{{\etens{N}}}
\def\etO{{\etens{O}}}
\def\etP{{\etens{P}}}
\def\etQ{{\etens{Q}}}
\def\etR{{\etens{R}}}
\def\etS{{\etens{S}}}
\def\etT{{\etens{T}}}
\def\etU{{\etens{U}}}
\def\etV{{\etens{V}}}
\def\etW{{\etens{W}}}
\def\etX{{\etens{X}}}
\def\etY{{\etens{Y}}}
\def\etZ{{\etens{Z}}}

% The true underlying data generating distribution
\newcommand{\pdata}{p_{\rm{data}}}
\newcommand{\ptarget}{p_{\rm{target}}}
\newcommand{\pprior}{p_{\rm{prior}}}
\newcommand{\pbase}{p_{\rm{base}}}
\newcommand{\pref}{p_{\rm{ref}}}

% The empirical distribution defined by the training set
\newcommand{\ptrain}{\hat{p}_{\rm{data}}}
\newcommand{\Ptrain}{\hat{P}_{\rm{data}}}
% The model distribution
\newcommand{\pmodel}{p_{\rm{model}}}
\newcommand{\Pmodel}{P_{\rm{model}}}
\newcommand{\ptildemodel}{\tilde{p}_{\rm{model}}}
% Stochastic autoencoder distributions
\newcommand{\pencode}{p_{\rm{encoder}}}
\newcommand{\pdecode}{p_{\rm{decoder}}}
\newcommand{\precons}{p_{\rm{reconstruct}}}

\newcommand{\laplace}{\mathrm{Laplace}} % Laplace distribution

\newcommand{\E}{\mathbb{E}}
\newcommand{\Ls}{\mathcal{L}}
\newcommand{\R}{\mathbb{R}}
\newcommand{\emp}{\tilde{p}}
\newcommand{\lr}{\alpha}
\newcommand{\reg}{\lambda}
\newcommand{\rect}{\mathrm{rectifier}}
\newcommand{\softmax}{\mathrm{softmax}}
\newcommand{\sigmoid}{\sigma}
\newcommand{\softplus}{\zeta}
\newcommand{\KL}{D_{\mathrm{KL}}}
\newcommand{\Var}{\mathrm{Var}}
\newcommand{\standarderror}{\mathrm{SE}}
\newcommand{\Cov}{\mathrm{Cov}}
% Wolfram Mathworld says $L^2$ is for function spaces and $\ell^2$ is for vectors
% But then they seem to use $L^2$ for vectors throughout the site, and so does
% wikipedia.
\newcommand{\normlzero}{L^0}
\newcommand{\normlone}{L^1}
\newcommand{\normltwo}{L^2}
\newcommand{\normlp}{L^p}
\newcommand{\normmax}{L^\infty}

\newcommand{\parents}{Pa} % See usage in notation.tex. Chosen to match Daphne's book.

\DeclareMathOperator*{\argmax}{arg\,max}
\DeclareMathOperator*{\argmin}{arg\,min}

\DeclareMathOperator{\sign}{sign}
\DeclareMathOperator{\Tr}{Tr}
\let\ab\allowbreak


\usepackage{amsmath}
\usepackage{amssymb}
\usepackage{algorithm}

\let\classAND\AND
\let\AND\relax

\usepackage{algorithmic}

\let\algoAND\AND
\let\AND\classAND


\AtBeginEnvironment{algorithmic}{\let\AND\algoAND}


\newtheorem{theorem}{Theorem}[section]
\newtheorem{lemma}[theorem]{Lemma}
\newtheorem{corollary}[theorem]{Corollary}
\newtheorem{proposition}[theorem]{Proposition}
\usepackage{adjustbox}
\usepackage{subfig}

% \theoremstyle{definition}
\newtheorem{definition}[theorem]{Definition}
\newtheorem{example}[theorem]{Example}
\usepackage{hyperref}
\usepackage{url}
\usepackage{cleveref}




\title{Symmetric Rank-One Quasi-Newton Methods for Deep Learning Using Cubic Regularization}

% Authors must not appear in the submitted version. They should be hidden
% as long as the tmlr package is used without the [accepted] or [preprint] options.
% Non-anonymous submissions will be rejected without review.

\author{%
  Aditya Ranganath \\
  Center for Applied Scientific Computing\\
  Lawrence Livermore National Laboratory\\
  7000 East Avenue, 
  Livermore, CA 94550 \\
  \texttt{ranganath2@llnl.gov} 
  \AND
  Mukesh Singhal \\
  Electrical Engineering and Computer Science \\
  University of California, Merced\\
  5200 N Lake Road\\
  Merced, CA 95343 \\
  \texttt{msinghal@ucmerced.edu}
  \AND
  Roummel Marcia \\
  Applied Mathematics \\
  University of California, Merced\\
  5200 N Lake Road \\
  Merced, CA 95343 \\
  \texttt{rmarcia@ucmerced.edu}
}

% The \author macro works with any number of authors. Use \AND 
% to separate the names and addresses of multiple authors.

\newcommand{\fix}{\marginpar{FIX}}
\newcommand{\new}{\marginpar{NEW}}

\def\month{MM}  % Insert correct month for camera-ready version
\def\year{YYYY} % Insert correct year for camera-ready version
\def\openreview{\url{https://openreview.net/forum?id=XXXX}} % Insert correct link to OpenReview for camera-ready version


\begin{document}


\maketitle



\begin{abstract}
Stochastic gradient descent and other first-order variants, such as Adam and AdaGrad, are commonly used in the field of deep learning due to their computational efficiency and low-storage memory requirements. However, these methods do not exploit curvature information. Consequently, iterates can converge to saddle points or poor local minima. On the other hand, Quasi-Newton methods compute Hessian approximations which exploit this information with a comparable computational budget. Quasi-Newton methods re-use previously computed iterates and gradients to compute a low-rank structured update. The most widely used quasi-Newton update is the L-BFGS, which guarantees a positive semi-definite Hessian approximation, making it suitable in a line search setting. However, the loss functions in DNNs are non-convex, where the Hessian is potentially non-positive definite. In this paper, we propose using a limited-memory symmetric rank-one quasi-Newton approach which allows for indefinite Hessian approximations, enabling directions of negative curvature to be exploited. Furthermore, we use a modified adaptive regularized cubics approach, which generates a sequence of cubic subproblems that have closed-form solutions with suitable  regularization choices. We investigate the performance of our proposed method on autoencoders and feed-forward neural network models and compare our approach to state-of-the-art first-order adaptive stochastic methods as well as other quasi-Newton methods.
\end{abstract}

\section{Introduction}
\label{sec:intro}
\section{Introduction}
Implicit Neural Representations (INRs), which fit the target function using only input coordinates, have recently gained significant attention.
%
By leveraging the powerful fitting capability of Multilayer Perceptrons (MLPs), INRs can implicitly represent the target function without requiring their analytical expressions. 
%
The versatility of MLPs allows INRs to be applied in various fields, including inverse graphics~\citep{mildenhall2021nerf, barron2023zip, martin2021nerf}, image super-resolution~\citep{chen2021learning, yuan2022sobolev, gao2023implicit}, 
image generation~\citep{skorokhodov2021adversarial}, and more~\citep{chen2021nerv, strumpler2022implicit, shue20233d}.
%
\begin{figure}
    \includegraphics[width=0.5\textwidth]{Image/Fig2.pdf}
    \caption{As illustrated at the circled blue regions and green regions, it can be observed that even with well-chosen standard deviation/scale, as experimented in \autoref{figure:combined}, the results are still unsatisfactory. However, using our proposed method, the noise is significantly alleviated while further enhancing the high-frequency details.}
    \label{fig:var}
    \vspace{-10pt}
\end{figure}

\begin{figure*}[!ht]
    \centering
    \begin{minipage}[b]{0.25\textwidth}
        \centering
        \includegraphics[width=1.\textwidth]{Image/fig_cropped.pdf} % 替换为你的小图文件
        \label{figure:small_image}
        \vspace{-20pt}
    \end{minipage}%
    \hfill
    \begin{minipage}[b]{0.75\textwidth}
        \centering
        \includegraphics[width=1.\textwidth]{Image/psnr_trends_rff_pe_simplified.pdf} % 替换为你的大图文件
        \vspace{-20pt}
        \label{figure:large_image}
        
    \end{minipage}
    \caption{We test the performance of MLPs with Random Fourier Features (RFF) and MLPs with Positional Encoding (PE) on a 1024-resolution image to better distinguish between high- and low-frequency regions, as demonstrated on the left-hand side of this figure. We find that the performance of MLPs+RFF degrades rapidly with increasing standard deviation compared with MLPs+PE. Since positional encoding is deterministic, scale=512 can be considered to have standard deviation around 121.}
    \label{figure:combined}
    \vspace{-10pt}
\end{figure*}
Varying the sampling standard deviation/scale may lead to degradation results, as shown in \autoref{figure:combined}.
%
However, MLPs face a significant challenge known as the spectral bias, where low-frequency signals are typically favored during training~\citep{rahaman2019spectral}. 
A common solution is to map coordinates into the frequency domain using Fourier features, such as Random Fourier Features and Positional Encoding, which can be understood as manually set high-frequency correspondence prior to accelerating the learning of high-frequency targets.~\citep{tancik2020fourier}. 
This embeddings widely applied to the INRs for novel view synthesis~\citep{mildenhall2021nerf,barron2021mip}, dynamic scene reconstruction~\citep{pumarola2021d}, object tracking~\citep{wang2023tracking}, and medical imaging~\citep{corona2022mednerf}.
% \begin{figure}[!h]
%     \centering
%     \includegraphics[width=1.\textwidth]{Image/psnr_trends_rff_pe_simplified.pdf}
%     \caption{This figure shows the change of PSNR on the whole, low-frequency region, and high-frequency region of the image fitting by using two Fourier Features Embedding with varying scale of variance: (Right) Positional Encoding (PE) (Left) Random Fourier Features (RFF). Both PE and RFF will degrade the low-frequency regions of the target image when variance increases.}
%     \vspace{-20pt} 
%     \label{figure:stats}
% \end{figure}


Although many INRs' downstream application scenarios use this encoding type, it has certain limitations when applied to specific tasks.
%
It depends heavily on two key hyperparameters: the sampling standard deviation/scale (available sampling range of frequencies) and the number of samples.
%
Even with a proper choice of sampling standard deviation/scale, the output remains unsatisfactory, as shown in \autoref{fig:var}: Noisy low-frequency regions and degraded high-frequency regions persist with well chosen sampling standard deviation/scale with the grid-searched standard deviation/scale, which may potentially affect the performance of the downstream applications resulting in noisy or coarse output.
%
However, limited research has contributed to explaining the reason and finding a proper frequency embeddings for input~\citep{landgraf2022pins, yuce2022structured}.

In this paper, we aim to offer a potential explanation for the high-frequency noise and propose an effective solution to the inherent drawbacks of Fourier feature embeddings for INRs.
%
Firstly, we hypothesize that the noisy output arises from the interaction between Fourier feature embeddings and multi-layer perceptrons (MLPs). We argue that these two elements can enhance each other's representation capabilities when combined. However, this combination also introduces the inherent properties of the Fourier series into the MLPs.
%
To support our hypothesis, we propose a simple theorem stating that the unsampled frequency components of the embeddings establish a lower bound on the expected performance. This underpins our hypothesis, as the primary fitting error in finitely sampled Fourier series originates from these unsampled frequencies.

Inspired by the analysis of noisy output and the properties of Fourier series expansion, we propose an approach to address this issue by enabling INRs to adaptively filter out unnecessary high-frequency components in low-frequency regions while enriching the input frequencies of the embeddings if possible.
%
To achieve this, we employ bias-free (additive term-free) MLPs. These MLPs function as adaptive linear filters due to their strictly linear and scale-invariant properties~\citep{mohan2019robust}, which preserves the input pattern through each activation layer and potentially enhances the expressive capability of the embeddings.
%
Moreover, by viewing the learning rate of the proposed filter and INRs as a dynamically balancing problem, we introduce a custom line-search algorithm to adjust the learning rate during training. This algorithm tackles an optimization problem to approximate a global minimum solution. Integrating these approaches leads to significant performance improvements in both low-frequency and high-frequency regions, as demonstrated in the comparison shown in \autoref{fig:var}.
%
Finally, to evaluate the performance of the proposed method, we test it on various INRs tasks and compare it with state-of-the-art models, including BACON~\citep{lindell2022bacon}, SIREN~\citep{sitzmann2020implicit}, GAUSS~\citep{ramasinghe2022beyond} and WIRE~\citep{saragadam2023wire}. 
The experimental results prove that our approach enables MLPs to capture finer details via Fourier Features while effectively reducing high-frequency noise without causing oversmoothness.
%
To summarize, the following are the main contributions of this work:
\begin{itemize}
    \item From the perspective of Fourier features embeddings and MLPs, we hypothesize that the representation capacity of their combination is also the combination of their strengths and limitations. A simple lemma offers partial validation of this hypothesis.

    
    \item  We propose a method that employs a bias-free MLP as an adaptive linear filter to suppress unnecessary high frequencies. Additionally, a custom line-search algorithm is introduced to dynamically optimize the learning rate, achieving a balance between the filter and INRs modules.

    \item To validate our approach, we conduct extensive experiments across a variety of tasks, including image regression, 3D shape regression, and inverse graphics. These experiments demonstrate the effectiveness of our method in significantly reducing noisy outputs while avoiding the common issue of excessive smoothing.
\end{itemize}


\noindent\section{Proposed approach
%Adaptive Regularization using Cubics with L-SR1 Updates
}
\label{sec:ProposedApproach}
In this section, we describe our proposed approach by first discussing the L-SR1 update.

\noindent \textbf{Limited-memory symmetric rank-one updates.} 
Unlike the BFGS update (\ref{eqn:LBFGS}), which is a rank-two update, the SR1 update is a rank-one update, which is  given by
\begin{equation}\label{eq:SR1}
	\mathbf{B}_{k+1} = \mathbf{B}_{k} + \frac{(\mathbf{y}_k - \mathbf{B}_k\mathbf{s}_k)
	(\mathbf{y}_k - \mathbf{B}_k\mathbf{s}_k)^{\top}}{\mathbf{s}_k^{\top}(\mathbf{y}_k - \mathbf{B}_k\mathbf{s}_k)}
\end{equation}
(see \citet{KhaBS93}).  As previously mentioned, $\mathbf{B}_{k+1}$ in (\ref{eq:SR1}) is not guaranteed to be definite.  However,
it can be shown that the SR1 matrices can converge to the true Hessian (see \citet{Conn1991} for details).
We note that the pair $(\mathbf{s}_k, \mathbf{y}_k)$ is accepted only when 
\begin{equation}\label{eq:acceptance1}
|\mathbf{s}_k^{\top}(\mathbf{y}_k - \mathbf{B}_k\mathbf{s}_k)| > \varepsilon \| \mathbf{s}_k \|_2 \| \mathbf{y}_k - \mathbf{B}_k \mathbf{s}_k \|_2,
\end{equation}
for some constant $\varepsilon > 0$ (see \citet{NoceWrig06}, Sec.\ 6.2, for details). The SR1 update can be defined recursively as
\begin{equation}\label{eq:SR1_B0}
	\mathbf{B}_{k+1} = \mathbf{B}_{0} + 
	\sum_{j = 0}^k \frac{(\mathbf{y}_j - \mathbf{B}_j\mathbf{s}_j)
	(\mathbf{y}_j - \mathbf{B}_j\mathbf{s}_j)^{\top}}{\mathbf{s}_j^{\top}(\mathbf{y}_j - \mathbf{B}_j\mathbf{s}_j)}.
\end{equation}
In limited-memory settings, only the last $m \ll n$ pairs of $(\mathbf{s}_j, \mathbf{y}_j)$ 
are stored and used.  
For ease of presentation, here we choose $k < m$.  We define
$$\mathbf{S}_{k} = [ \ \mathbf{s}_0 \ \  \mathbf{s}_1 \ \ \cdots  \ \ \mathbf{s}_{k-1} \ ] \quad \text{and} \quad 
\mathbf{Y}_{k} = [ \ \mathbf{y}_0 \ \ \mathbf{y}_1 \ \ \cdots \  \ \mathbf{y}_{k-1} \ ].
$$
Then 
$\mathbf{B}_{k}$ admits a compact representation of the form
\begin{equation}\label{eqn:compactSR1}
	\mathbf{B}_{k} \ = \ \mathbf{B}_0 + 
	\begin{bmatrix}
	\\
	\mathbf{\Psi}_{k}  \\
	\phantom{t}
	\end{bmatrix}
	\hspace{-.3cm}
	\begin{array}{c}
	\left  [ \  \mathbf{M}_{k}^{\phantom{h}}  \right ] \\
	\\
	\\
	\end{array}
	\hspace{-.3cm}
	\begin{array}{c}
	\left [  \ \quad \mathbf{\Psi}_{k}^{\top} \quad \ \right ] \\
	\\
	\\
	\end{array},
\end{equation}
where  $\mathbf{\Psi}_{k} = \mathbf{Y}_{k} -  \mathbf{B}_0 \mathbf{S}_{k}$ and 
%\begin{align}\label{eq:PsiM}
%	\nonumber
%	\mathbf{\Psi}_{k+1} &= \mathbf{Y}_{k+1}\!  -\! \mathbf{B}_0 \mathbf{S}_{k+1}  \ \\\nonumber \text{and}\\ 
$$
        \mathbf{M}_{k} = (\mathbf{D}_{k} \!+\! \mathbf{L}_{k} \!+\! \mathbf{L}_{k}^{\top} \!-\! \mathbf{S}_{k}^{\top}\!\mathbf{B}_0\mathbf{S}_{k})^{-1}\!,
$$
%\end{align}
where $\mathbf{L}_{k}$ is the strictly lower triangular part, $\mathbf{V}_{k}$ is the strictly
upper triangular part, and $\mathbf{D}_{k}$ is the diagonal part of 
$
	\mathbf{S}_{k}^{\top}\mathbf{Y}_{k} =   \mathbf{L}_{k} + \mathbf{D}_{k} + \mathbf{V}_{k}
$
(see \citet{ByrNS94} for further details).  


Because of the compact representation of $\mathbf{B}_{k}$, 
its partial eigendecomposition can be computed (see  \citet{ErwM15}).  
In particular, if we compute the QR decomposition of $\mathbf{\Psi}_{k} = \mathbf{QR}$
and the eigendecomposition $\mathbf{RMR}^\top= \mathbf{P} \hat{\mathbf{\Lambda}}_{k} \mathbf{P}^\top$,
then we can write 
$$
\mathbf{B}_{k} = \mathbf{B}_0 + \mathbf{U}_{\parallel} \hat{\mathbf{\Lambda}}_{k} 
\mathbf{U}_{\parallel}^{\top},
$$
where $\mathbf{U}_{\parallel}  = \mathbf{QP} \in \mathbb{R}^{n \times k}$ has
orthonormal columns and $\hat{\mathbf{\Lambda}}_{k} \in \mathbb{R}^{k \times k}$ 
is a  diagonal matrix.  
If $\mathbf{B}_0 =  \delta_k \mathbf{I}$ (see e.g., Lemma 2.4 in  \citet{Erway2020TrustregionAF}), 
where $0 < \delta_k < \delta_{\max}$ is some scalar and $\mathbf{I}$ is the identity matrix, 
then we obtain the eigendecomposition 
\begin{equation}
\mathbf{B}_{k} = \mathbf{U}_{k}\mathbf{\Lambda}_{k}\mathbf{U}_{k}^{\top}
=
\bigg [ \ 
\mathbf{U}_{\parallel}  \ \ \ \mathbf{U}_{\perp}
\bigg ]
\begin{bmatrix}
\hat{\mathbf{\Lambda}}_{k} + \delta_k \mathbf{I} & 0 \\
0 & \delta_k \mathbf{I} 
\end{bmatrix}
\begin{bmatrix}
\ \mathbf{U}_{\parallel}^{\top} \ 
\\[.2cm]
\mathbf{U}_{\perp}^{\top}
\end{bmatrix}
\end{equation}
where $\mathbf{U}_{k} = [  \ \mathbf{U}_{\parallel}  \ \ \mathbf{U}_{\perp} \ ]$ is an orthogonal 
matrix and
$\mathbf{U}_{\perp} \in \mathbb{R}^{n \times (n-k)}$ is a matrix
whose columns form an orthonormal basis orthogonal to the range space of $\mathbf{U}_{\parallel}$.
% and $\mathbf{U}_{k+1}^{\top} \mathbf{U}_{k+1}^{\phantom{\top}} = \mathbf{I}$.  
Here, 
\begin{equation}
	(\mathbf{\Lambda}_{k})_i =
	\begin{cases}  
	\delta_k + \hat{\lambda}_i & \text{ if $i \le k$} \\
	\delta_k & \text{ if $i > k$}
	\end{cases}.
\end{equation}
%$(\mathbf{\Lambda}_{k+1})_i = \delta_k + \hat{\lambda}_i$ for $i \le k+1$, where
%$\hat{\lambda}_i$ is the $i$th diagonal in $\hat{\mathbf{\Lambda}}_{k+1}$,
%and $(\mathbf{\Lambda}_{k+1})_i = \delta_k $ for $i > k+1$.  


%In particular, if $\mathbf{\Phi}_{k+1} = \mathbf{Q}_{k+1} \mathbf{R}_{k+1}$ is the
%QR decomposition of $\mathbf{\Phi}_{k+1}$ and $\mathbf{R_{k+1}M_{k+1}R_{k+1}^{\top} =
%\mathbf{P}_{k+1}\mathbf{\Lambda}_{k+1}\mathbf{P}_{k+1}^{\top}}$ is the eigendecomposition of the product 
%$\mathbf{R}_{k+1}\mathbf{M}_{k+1}\mathbf{R}_{k+1}^{\top}$, then
%$$
%	\mathbf{B}_{k+1} = 
%	\mathbf{B}_0 + \mathbf{Q}_{k+1}\mathbf{P}_{k+1}
%	\mathbf{\Lambda}_{k+1}\mathbf{P}_{k+1}^{\top}\mathbf{Q}_{k+1}^{\top}.
%$$

%\begin{figure}[t]
%	\centering
%		\begin{tabular}{cc}
%		\includegraphics[width=3.175cm, trim={2cm 1cm 2cm 0},clip]{Figures/cubic1.pdf} &
%		\includegraphics[width=3.175cm, trim={2cm 1cm 2cm 0},clip]{Figures/cubic2.pdf}  \\
%		(a) $\lambda > 0$ and $g > 0$ & (b) $\lambda > 0$ and $g < 0$ \\
%		\includegraphics[width=3.175cm, trim={2cm 1cm 2cm 0},clip]{Figures/cubic3.pdf} &
%		\includegraphics[width=3.175cm, trim={2cm 1cm 2cm 0},clip]{Figures/cubic4.pdf} \\
%		(c) $\lambda < 0$ and $g > 0$ & (d)  $\lambda < 0$ and $g < 0$
%		\end{tabular}
%%	\begin{tabular}{cccc}
%%		\includegraphics[width=3.175cm, trim={2cm 1cm 2cm 0},clip]{Figures/cubic1.pdf} &
%%		\includegraphics[width=3.175cm, trim={2cm 1cm 2cm 0},clip]{Figures/cubic2.pdf} &
%%		\includegraphics[width=3.175cm, trim={2cm 1cm 2cm 0},clip]{Figures/cubic3.pdf} &
%%		\includegraphics[width=3.175cm, trim={2cm 1cm 2cm 0},clip]{Figures/cubic4.pdf} 
%%		\\
%%		\footnotesize (a) $\lambda > 0$ and $\bar{g} > 0$ & 
%%		\footnotesize (b) $\lambda > 0$ and $\bar{g} < 0$ &
%%		\footnotesize (c) $\lambda < 0$ and $\bar{g} > 0$ &
%%		\footnotesize (d)  $\lambda < 0$ and $\bar{g} < 0$
%%	\end{tabular}
%	\caption{Illustration of the piece-wise cubic function $m(s)$. When $\lambda > 0$, 
%		$m(s)$ is a convex function and has a unique local minimum, which is also the global minimum 
%		((a) and (b)).
%		If $\lambda < 0$, then $m(s)$ has two local minima ((c) and (d)).
%		The scalar $\bar{g}$ corresponds to the slope of $m(s)$ at $s = 0$, i.e., $m'(0) = \bar{g}$.
%		If $\bar{g} > 0$, the minimum $s^*$ of $m(s)$
%		corresponds to the minimum of $m_-(s)$ (black circle in (a) and (c)), 	
%		and if $\bar{g} < 0$, then $s^*$ corresponds to the minimum of $m_+(s)$ (red circle in (b) and (d)).
%		\label{fig:cubic}}
%\end{figure}

\noindent \textbf{Adaptive regularization using cubics.} 
 Since the SR1 Hessian approximation can be indefinite, some safeguard must be implemented to ensure that the resulting search direction $\mathbf{s}_k$ is a descent direction.  One such safeguard is to use a ``regularization" term.
The Adaptive Regularization using Cubics (ARCs) method (see \citet{Griewank1981,NesP06,cartis2011adaptive}) can be viewed as an alternative to line-search and trust-region methods. At each iteration, an approximate global minimizer of a local (cubic) model,
\begin{equation}\label{eq:cr}
	\underset{\mathbf{s}\in \mathbb{R}^n}{\text{min}}  \ m_k(\mathbf{s}) 
	\equiv 
	%\underset{s\in \mathcal{R}^n}{\text{min}} 
	\mathbf{g}_k^{\top}\mathbf{s}
	+ \frac{1}{2} \mathbf{s}^{\top}\mathbf{B}_k \mathbf{s} + \frac{\mu_k}{3} (\Phi_k(\mathbf{s}))^3,
\end{equation}
is determined, where  $\mathbf{g}_k = \nabla f(\Theta_k)$, $\mu_k > 0$ is a regularization parameter, and
$\Phi_k$ is a function (norm) that regularizes $\mathbf{s}$.   Typically, the Euclidean norm is used.
In this work, we use an alternative ``shape-changing" norm that allows us to solve each subproblem 
(\ref{eq:cr}) exactly.  Proposed in \citet{Burdakov2017}, this shape-changing norm is
based on the partial eigendecomposition of $\mathbf{B}_{k}$.  Specifically, if 
$\mathbf{B}_{k} = \mathbf{U}_k \mathbf{\Lambda}_k \mathbf{U}_k^{\top}$ is the eigendecomposition
of $\mathbf{B}_k$, then we can define the norm 
$$
 \|\mathbf{s}\|_{\mathbf{U}_k}\overset{\text{def}}{=}\|\mathbf{U}_k^\top \mathbf{s}\|_3.
 $$
 It can be shown using H\"{o}lder's Inequality that 
 $$
 \frac{1}{\sqrt[\leftroot{1} 6]{n}}
 %n^{-1/6} 
 \| \mathbf{s} \|_2 
\le  \| \mathbf{s} \|_{\mathbf{U}_k}
\le  \| \mathbf{s} \|_2.
$$

As per the authors' literature review, this is the first time the adaptive regularized cubics has been used in conjunction with a shape changing norm in a deep learning setting. The main motivation of using this adaptive regularized cubics comes from better convergence properties when compared with a trust-region approach (see \citet{cartis2011adaptive}). Using the shape-changing norm allows us to solve the subproblem exactly.
 
 \noindent \textbf{Closed-form solution.} 
 Applying a change of basis with
$\bar{\mathbf{s}} = \mathbf{U}_k^{\top} \mathbf{s}$ and 
$\bar{\mathbf{g}}_k = \mathbf{U}_k^{\top}\mathbf{g}_k$, 
we can redefine the cubic subproblem as
%The ARC algorithm has claimed to achieve a 2-norm of the gradient $\norm{g} = \norm{\nabla f}$ below the desired accuracy $\epsilon$ in at most $\mathcal{O}(\epsilon^{-1.5})$ steps. Now we are ready to explain the regularization function $\Phi_k(s)$.
%\textbf{New basis:} If $\Phi(s)$ in (\ref{eq:cr}) is a two-norm operation on $s$, then our model function will, at most, have only two local minima, making it cumbersome to find the global minima. We propose a transformation of the parameter space $s$ to $\bar{s}$ such that $ \norm{s}_\mathbf{U}\overset{\text{def}}{=}\norm{\mathbf{U}^\top s}_3$ and redefine our objective function as
\begin{equation}\label{eq:modcr}
	\underset{\bar{\mathbf{s}} \in \mathbb{R}^n}{\text{min}}  \ \bar{{m}}_{k} (\bar{\mathbf{s}})
	= \bar{\mathbf{g}}_k^\top\bar{\mathbf{s}}
	+ \frac{1}{2}\bar{\mathbf{s}}^\top \mathbf{\Lambda}_k\bar{\mathbf{s}}
	+ \frac{\mu_k}{3}\|\bar{\mathbf{s}}\|_3^3.
\end{equation}
With this change of basis, we can easily find a closed-form solution of (\ref{eq:modcr}), which is generally not the case for other choices of norms.  
Note that $\bar{m}_k(\bar{\mathbf{s}})$ is a separable function,  
meaning we can write $\bar{m}_k(\bar{\mathbf{s}})$ as
$$
	\bar{m}_k(\bar{\mathbf{s}})
	=
	\sum_{i=1}^n
	\left \{
	(\bar{\mathbf{g}}_k)_i (\bar{\mathbf{s}})_i
	+
	\frac{1}{2}(\mathbf{\Lambda}_k)_i(\bar{\mathbf{s}})_i^2
	+
	\frac{\mu_k}{3} |(\bar{\mathbf{s}})_i |^3
	\right \}.
$$
Consequently, we can  solve (\ref{eq:modcr}) by solving one-dimensional problems 
of the form 
\begin{equation}\label{eq:modcr1}
	\underset{\bar{s} \in \mathbb{R}}{\text{min}}  \ \ \bar{m}(\bar{s})
	= \bar{g} \bar{s}   
	+ \frac{1}{2}\lambda \bar{s}^2
	+ \frac{\mu_k}{3}|\bar{s}|^3,
\end{equation}
where $\bar{g} \in \mathbb{R}$ corresponds to entries in $\bar{\mathbf{g}}_k$ and
$\lambda \in \mathbb{R}$ corresponds to diagonal entries in $\mathbf{\Lambda}_k$.  
To find the minimizer of (\ref{eq:modcr1}), we first write $\bar{m}(\bar{s})$ as follows:
\begin{equation*}
	\bar{m}(\bar{s}) = 
	\begin{cases}
		\bar{m}_+(s) = \bar{g} \bar{s}  
	+ \frac{1}{2}\lambda \bar{s}^2
	+ \frac{\mu_k}{3}\bar{s}^3 & \text{if $\bar{s} \ge 0$}, \\
		\bar{m}_-(\bar{s}) = \bar{g}\bar{s}  
	+ \frac{1}{2}\lambda \bar{s}^2
	- \frac{\mu_k}{3}\bar{s}^3 & \text{if $\bar{s} \le 0$}. 	
	\end{cases}
\end{equation*}
%The corresponding derivative is given by
%\begin{equation*}
%	m'(s) = 
%	\begin{cases}
%		m_+'(s) = g
%	+ \lambda s
%	+ \mu_ks^2 & \text{if $s \ge 0$},\\
%		m_-'(s) = g 
%	+ \lambda s
%	- \mu_ks^2 & \text{if $s \le 0$}.
%	\end{cases}
%\end{equation*}
The minimizer $\bar{s}^*$ of $\bar{m}(\bar{s})$ is obtained by setting $\bar{m}'(\bar{s})$ to zero and will depend on the sign of $\bar{g}$ because $\bar{g}$ is the slope of $\bar{m}(\bar{s})$ at $\bar{s} = 0$, i.e., $\bar{m}'(0) = \bar{g}$.  
In particular,
if $\bar{g} > 0$, then $\bar{s}^*$  is the minimizer of $\bar{m}_-(\bar{s})$,
%(see Figs.\ \ref{fig:cubic}(a) and (c)), 
namely
$\bar{s}^* = (-\lambda + \sqrt{\lambda^2 + 4\bar{g}\mu})/(-2\mu).$
If $\bar{g} < 0$, then $\bar{s}^*$ is the minimizer of $\bar{m}_+(\bar{s})$,
% (see Figs.\ \ref{fig:cubic}(b) and (d)), 
which is given by
$	\bar{s}^* = (-\lambda + \sqrt{\lambda^2 - 4\bar{g}\mu})/(2\mu).$
Note that these two expressions
for $\bar{s}^*$ are equivalent to the following formula:
$$
	\bar{s}^* = \frac{-2\bar{g}}{\lambda + \sqrt{\lambda^2 + 4|\bar{g}|\mu}},
$$
In the original space, $\mathbf{s}^* = \mathbf{U}_k \bar{\mathbf{s}}^*$ and 
$\mathbf{g}_k = \mathbf{U}_k \bar{\mathbf{g}}_k$.
Letting 
\begin{equation}\label{eq:Ck}
	\mathbf{C}_k = \text{diag} (\bar{c}_1, \dots, \bar{c}_n), \quad \text{where \ } \bar{c}_i =  \frac{2}{\lambda_i + \sqrt{\lambda_i^2 + 4|\bar{\mathbf{g}}_i|\mu}},
\end{equation}
then the solution $\mathbf{s}^*$ in the original space is  given by
\begin{equation}\label{eq:sstar}
	\mathbf{s}^* = \mathbf{U}_k \bar{\mathbf{s}}^* =  -\mathbf{U}_k  \mathbf{C}_k \mathbf{U}_k^{\top} \mathbf{g}_k.
\end{equation}
%For a more detailed description of the closed form solution, see  Appendix \ref{sec:Solution}. 




\noindent \textbf{Practical implementation.} While computing 
$\mathbf{U}_{\parallel} \in \mathbb{R}^{n \times k}$
in the matrix $\mathbf{U}_{k} = [  \ \mathbf{U}_{\parallel}  \ \ \mathbf{U}_{\perp} \ ]$
is feasible since 
$k \ll n$, computing $\mathbf{U}_{\perp}$ explicitly is not.  Thus, we must be able to compute 
$\mathbf{s}^*$ without needing $\mathbf{U}_{\perp}$.  
First, we define the following quantities
$$
\begin{array}{lllllll}
\bar{\mathbf{s}}_{\parallel} 
 = \mathbf{U}_{\parallel}^{\top} \mathbf{s} 
& \text{and}  
& \bar{\mathbf{s}}_{\perp} = \mathbf{U}_{\perp}^{\top} \mathbf{s},
\\[.2cm]
\bar{\mathbf{g}}_{\parallel} = \mathbf{U}_{\parallel}^{\top} \mathbf{g}_k
& \text{and} 
& \bar{\mathbf{g}}_{\perp} = \mathbf{U}_{\perp}^{\top} \mathbf{g}_k.
\end{array}
$$
Then the cubic subproblem (\ref{eq:modcr})  becomes
\begin{equation}
\underset{\bar{\mathbf{s}}\in \mathbb{R}^n
}{\text{minimize}}  \ \bar{{m}}_{k} (\bar{\mathbf{s}})
	\ = \ 
\underset{\bar{\mathbf{s}}_{\parallel} \in \mathbb{R}^k
}{\text{minimize}}  \ \bar{{m}}_{\parallel} (\bar{\mathbf{s}}_{\parallel}) + 
\underset{\bar{\mathbf{s}}_{\perp} \in \mathbb{R}^{n-k}
}{\text{minimize}}  \ \bar{{m}}_{\perp} (\bar{\mathbf{s}}_{\perp}),
\end{equation}
where
\begin{eqnarray} \label{eq:mparallel}
	\bar{m}_{\parallel}( \bar{\mathbf{s}}_{\parallel}) \!  \ \ &=& 
	\bar{\mathbf{g}}_{\parallel}^\top\bar{\mathbf{s}}_{\parallel}
	+ \frac{1}{2}\bar{\mathbf{s}}_{\parallel}^\top \hat{\mathbf{\Lambda}}_k\bar{\mathbf{s}}_{\parallel}
	+ \frac{\mu_k}{3}\|\bar{\mathbf{s}}_{\parallel} \|_3^3,
	\\
	\label{eq:mperp}
	\bar{m}_{\perp} ( \bar{\mathbf{s}}_{\perp}) &=& 
	\bar{\mathbf{g}}_{\perp}^\top\bar{\mathbf{s}}_{\perp}
	+ \frac{\delta_k}{2} \| \bar{\mathbf{s}}_{\perp} \|_2^2
	+ \frac{\mu_k}{3}\|\bar{\mathbf{s}}_{\perp}\|_3^3.
\end{eqnarray}
We minimize $\bar{m}_{\parallel}(\bar{s}_{\parallel})$ in (\ref{eq:mparallel}) similar to how we solved (\ref{eq:modcr1}).
In particular, if we let 
\begin{equation}\label{eq:Cparallel}
	\mathbf{C}_{\parallel} = \text{diag} (c_1, \dots, c_n), \ \text{where } c_i =  \frac{2}{\lambda_i + \sqrt{\lambda_i^2 + 4| (\bar{\mathbf{g}}_{\parallel})_i|\mu}},
\end{equation}
then the solution is given by 
\begin{equation}\label{eq:sparallelstar}
	\mathbf{s}_{\parallel}^* =
	-\mathbf{C}_{\parallel} \bar{\mathbf{g}}_{\parallel}.
\end{equation}



Minimizing $\bar{m}_{\perp}(\bar{s}_{\perp})$ in (\ref{eq:mperp}) is more challenging.
The only restriction on the matrix $\mathbf{U}_{\perp}$ is that its columns must form an orthonormal basis for the orthogonal complement of the range space of $\mathbf{U}_{\parallel}$.  
We are thus free to choose the columns of $\mathbf{U}_{\perp}$ as long as they satisfy this restriction.
In particular, we can choose the first column of $\mathbf{U}_{\perp}$ to be the normalized orthogonal projection of $\mathbf{g}_k$ onto the orthogonal complement of the range space of $\mathbf{U}_{\parallel}$, i.e.,
$$
	(\mathbf{U}_{\perp})_1 = ( \mathbf{I} - \mathbf{U}_{\parallel}\mathbf{U}_{\parallel}^{\top})\mathbf{g}_k
	/ \| ( \mathbf{I} - \mathbf{U}_{\parallel}\mathbf{U}_{\parallel}^{\top})\mathbf{g}_k \|_2.
$$
If $\mathbf{g}_k \in $ Range($\mathbf{U}_{\parallel}$), then 
$\bar{\mathbf{g}}_{\perp} = \mathbf{U}_{\perp}^{\top}\mathbf{g}_k = 0$  
% because g_k = U_parallel b for some b 
and the minimizer of (\ref{eq:mperp}) %$\bar{m}_{\perp}(\bar{\mathbf{s}}_{\perp})$ 
is $\bar{\mathbf{s}}_{\perp}^* = 0$ (since $\delta_k > 0$ and $\mu_k > 0$).
If $\mathbf{g}_k \notin $ Range($\mathbf{U}_{\parallel}$), then $(\mathbf{U}_{\perp})_1 \ne 0$ and 
we can choose vectors $ (\mathbf{U}_{\perp})_i \in \text{Range}(\mathbf{U}_{\parallel})^{\perp}$
such that $(\mathbf{U}_{\perp})_i^{\top} (\mathbf{U}_{\perp})_1 = 0$ for all $2 \le i \le n-k$.
Consequently,  $\mathbf{U}_{\perp}^{\top} (\mathbf{U}_{\perp})_1 = \kappa \mathbf{e}_1$,
where $\kappa$ is some constant and $\mathbf{e}_1$ is the first column of the identity matrix.  
Specifically,  
$$
	\kappa \mathbf{e}_1 
	=
	\mathbf{U}_{\perp}^{\top} (\mathbf{U}_{\perp})_1 
	= 
	\mathbf{U}_{\perp}^{\top}  \left ( \mathbf{U}_{\perp}  \mathbf{U}_{\perp}^{\top} \mathbf{g}_k \right )
	=
	\mathbf{U}_{\perp}^{\top} \mathbf{g}_k
	=
	\bar{\mathbf{g}}_{\perp},
$$
which implies $\kappa = \| \bar{\mathbf{g}}_{\perp} \|_2$.  Thus $ \bar{\mathbf{g}}_{\perp}$
has only one non-zero component (the first component) and therefore, the minimizer 
$\bar{\mathbf{s}}_{\perp}^*$ of 
$\bar{m}_{\perp} ( \bar{\mathbf{s}}_{\perp}) $ in (\ref{eq:mperp}) also has only one non-zero compoent (the first component as well).  In particular, 
\begin{align*}
	(\bar{\mathbf{s}}_{\perp}^*)_i
	&=
	\begin{cases}
		\displaystyle 
		-\alpha^* \| \bar{\mathbf{g}}_{\perp} \|_2
		& \text{if $i = 1$} 
		\\
		0 & \text{otherwise}
	\end{cases},
\end{align*}
where
\begin{equation}\label{eq:alphastar}
	\alpha=  \frac{2 }{ \delta_k 
		+ \sqrt{\delta_k^2 + 4 \mu \| \bar{\mathbf{g}}_{\perp} \|_2} }.
\end{equation}
Equivalently, $\bar{\mathbf{s}}_{\perp}^*=- \alpha^* \bar{\mathbf{g}}_{\perp}$.  
Note that the quantity $ \|  \bar{\mathbf{g}}_{\perp}\|_2$ can be computed without computing 
$ \bar{\mathbf{g}}_{\perp}$ from  the fact that $\| \mathbf{g} \|_2^2=
 \| \bar{\mathbf{g}}_{\parallel}\|_2^2 +  \| \bar{\mathbf{g}}_{\perp} \|_2^2$.  
 
 Combining the expressions for $\bar{s}_{\parallel}^*$ in (\ref{eq:sparallelstar}) and for 
 $\bar{\mathbf{s}}_{\perp}^*$, the solution in the original space is given by
 \begin{align*}
 	\mathbf{s}^* &=
	\mathbf{U}_{\parallel} \mathbf{s}_{\parallel}^* + 
	\mathbf{U}_{\perp}^{\phantom{^*}} \mathbf{s}_{\perp}^* \\
	&=
%	-\mathbf{U}_{\parallel}\mathbf{C}_{\parallel}\bar{\mathbf{g}}_{\parallel} - \alpha^* \bar{\mathbf{g}}_{\perp}
%= 
%-\mathbf{C}_{\parallel}\bar{\mathbf{g}}_{\parallel} 
- \mathbf{U}_{\parallel}\mathbf{C}_{\parallel}\mathbf{U}_{\parallel}^{\top} \mathbf{g} 
- \alpha^* (\mathbf{I}_n - \mathbf{U}_{\parallel}\mathbf{U}_{\parallel}^{\top}) \mathbf{g}\\
&= -\alpha^* \mathbf{g}  + \mathbf{U}_{\parallel}(\alpha^* \mathbf{I} - \mathbf{C}_{\parallel})\mathbf{U}_{\parallel}^{\top} \mathbf{g}.
 \end{align*}
 Note that computing $\mathbf{s}^*$ neither  involves forming $\mathbf{U}_{\perp}$ nor
 computing $\bar{\mathbf{g}}_{\perp}$ explicitly.
 
\bigskip




\noindent \textbf{Termination criteria.} 
With each cubic subproblem solved, the iterations are terminated when 
the change in iterates, $\mathbf{s}_k$, is sufficiently small, i.e., 
\begin{equation}\label{eq:acceptance2}
\| \mathbf{s}_k \|_2 < \tilde{\epsilon} \| \mathbf{y}_k - \mathbf{B}_k \mathbf{s}_k\|_2,
\end{equation}
for some $\tilde{\epsilon}$, 
or when the maximum number of allowable iterations is achieved.
The proposed Adaptive Regularization using Cubics with L-SR1 (ARCs-LSR1) algorithm is given in Algorithm \ref{alg:LSR1ARC}.






\begin{algorithm}[!h]
	\caption{Adaptive Regularization using Cubics with Limited-Memory SR1 (ARCs-LSR1) }
	\begin{algorithmic}[1]
		\STATE $\textbf{Given: }\Theta_0, \gamma_2 \geq \gamma_1, 1 > \eta_2 \geq \eta_1 > 0,\  \sigma_0 > 0, \tilde{\epsilon} > 0, k = 0,$	 and $k_{\text{max}} > 0$
%		\Require $S_k = \{s_0, \ldots, s_k\}$, $Y_k = \{y_0, \ldots, y_k\}$
		\WHILE {$k < k_{\text{max}} \ \text{and} \  \| \mathbf{s}_k \|_2 \ge \tilde{\epsilon} \| \mathbf{y}_k - \mathbf{B}_k \mathbf{s}_k\|_2$}
		\STATE {Obtain $\mathbf{S}_k = [ \ \mathbf{s}_0 \ \  \cdots \ \  \mathbf{s}_k \ ]$ and $\mathbf{Y}_k = [ \ \mathbf{y}_0 \ \  \cdots \ \ \mathbf{y	}_k \ ]$}
		\STATE {Solve the generalized eigenvalue problem $\mathbf{S}_k^{\top}\mathbf{Y}_k \mathbf{u} = \hat{\lambda}\mathbf{S}_k^{\top}\mathbf{S}_k \mathbf{u}$ 
		and let $\delta_k=\min\{ \hat{\lambda}_i\}$}
		\STATE {Compute $\mathbf{\Psi}_k = \mathbf{Y}_k - \delta_k \mathbf{S}_k$}
%		\If {Cholesky is available}
%		\State {$\Psi^{\top} \Psi = R^{\top}R$}
%		\State {$Q = \Psi R^{-1}$}
%		\Else  { Perform QR-decomposition of $\Psi$}
		\STATE {Perform QR decomposition of $\mathbf{\Psi}_k = \mathbf{Q}\mathbf{R}$}
%		\EndIf
		\STATE {Compute eigendecomposition 
		%\begin{equation}%\label{eqn:eigenvaluedecomposition}
		$	\mathbf{RMR}^\top= \mathbf{P\Lambda P}^\top$
		%\end{equation}
		
		}
		\STATE {Assign $\mathbf{U}_\parallel = \mathbf{QP}$ and $\mathbf{U}_{\parallel}^{\top} = \mathbf{P}^{\top} \mathbf{Q}^{\top}$}
		\STATE {%With $D = \text{diag}(\lambda_0,\ldots, \lambda_{m})$, 
		Define $\mathbf{C}_\parallel = \text{diag}(c_1,\ldots, c_k)$, where $c_i = \frac{2}{\lambda_i + \sqrt{\lambda_i^2 + 4\mu | (\bar{\mathbf{g}}_{\parallel})_{i}|}}$ 
		and  $\bar{\mathbf{g}}_\parallel = \mathbf{U}^\top_\parallel \mathbf{g}$}
		\STATE Compute {$\alpha^{*}$ in \eqref{eq:alphastar}} %= \frac{2}{\delta_k + \sqrt{\delta_k^2 + 4\mu\| \bar{\mathbf{g}}_{\perp}\|}}$} %where $\mathbf{g}_{\perp} = \mathbf{g} - \mathbf{U}_\parallel \bar{\mathbf{g}}_\parallel$
		\STATE {Compute step $\! \mathbf{s}^* = -\alpha^{*}\mathbf{g} + \mathbf{U}_{\parallel}(\alpha^{*}\mathbf{I} - \mathbf{C}_{\parallel})\mathbf{U}_{\parallel}^{\top}\mathbf{g}$}
		\STATE Compute $m_k(\mathbf{s}^*\!)$ \! and \!  $\rho_k \! \!=\!  (f(\Theta_k) 
		\!-\! f(\Theta_{k+1})\!)\!/m_k(\mathbf{s}^*\!)$% from (\ref{eq:ratio}) in Appendix A
		\STATE {Set 
		\begin{align*}
			\Theta_{k+1} &=
			\begin{cases}
				\Theta_k + \mathbf{s}^* \hspace{.85cm}  & \text{if }\rho_k\geq\eta_1\\
				\Theta_k, & \text{otherwise}		
			\end{cases}, \quad \text{and} 
			\\
%			 \left\{ 
%			\begin{array}{lr}
%				\Theta_k + s_k, & \text{if }\rho_k\geq\eta_1,\\
%				\Theta_k, & \text{otherwise}
%			\end{array}\right\}.
%		\end{align*}
%		\begin{align*}
			\mu_{k+1} &=
			\begin{cases}
				\tfrac{1}{2} \mu_k & \text{if }\rho_k > \eta_2,\\
				\tfrac{1}{2} \mu_k (1 + \gamma_1) & \text{if }\eta_1 \leq \rho_k \leq \eta_2,\\
				\tfrac{1}{2} \mu_k (\gamma_1 + \gamma_2) & \text{otherwise}			
			\end{cases}
%			\left\{\begin{array}{lr}
%				[0, \sigma_k] & \text{if }\rho_k > \eta_2,\\
%				\left[\sigma_k, \gamma_1\sigma_k\right] & \text{if }\eta_1 \leq \rho_k \leq \eta_2,\\
%				\left[\gamma_1\sigma_k, \gamma_2\sigma_k\right] & \text{otherwise}
%			\end{array}\right\}.
		\end{align*}
		}
		\STATE {$k \leftarrow k+1$}
			\ENDWHILE
	\end{algorithmic}\label{alg:LSR1ARC}
\end{algorithm}

%\begin{algorithm}[!h]
%	\caption{Adaptive Regularization using Cubics (ARC)}
%	\begin{algorithmic}
%		\State $\textbf{Given: }\Theta_0, \gamma_2 \geq \gamma_1, 1 > \eta_2 \geq \eta_1 > 0,\ \text{and}\ \sigma_0 > 0$
%		\While{$k \leq k_{\text{max}}$}
%		\State Compute a step $s_k$ using Algorithm \ref{alg:LSR1obs}
%		\State Compute $\rho_k$ using modified formula ratio of actual reduction to model reduction
%		\State Set 
%		\begin{equation*}
%			\Theta_{k+1} = \left\{ 
%			\begin{array}{lr}
%				\Theta_k + s_k, & \text{if }\rho_k\geq\eta_1,\\
%				\Theta_k, & \text{otherwise}
%			\end{array}\right\}.
%		\end{equation*}
%		\State Set 
%		\begin{equation*}
%			\sigma_{k+1} \in \left\{\begin{array}{lr}
%				[0, \sigma_k] & \text{if }\rho_k > \eta_2,\\
%				\left[\sigma_k, \gamma_1\sigma_k\right] & \text{if }\eta_1 \leq \rho_k \leq \eta_2,\\
%				\left[\gamma_1\sigma_k, \gamma_2\sigma_k\right] & \text{otherwise}
%			\end{array}\right\}.
%		\end{equation*}
%		\EndWhile
%	\end{algorithmic}\label{alg:adaptivereg}
%\end{algorithm}
%


%
%\subsection{Contributions}
%The main contributions of this paper are as follows:
%\begin{enumerate}[leftmargin=0.45cm]
%	\itemsep 0em
%	\item \textbf{L-SR1 quasi-Newton methods.} The most commonly used quasi-Newton approach is the L-BFGS method.  In this work, we use the L-SR1 update to better model potentially indefinite Hessians of the non-convex loss function. 
%	\item \textbf{Adaptive Regularization using Cubics (ARCs).} Given that the quasi-Newton approximation is allowed to be indefinite, we use an Adaptive Regularized using Cubics approach to 
%safeguard each search direction.
%	\item \textbf{Shape-changing regularizer.} 
%	We use a shape-changing norm to define the cubic regularization term, which allows us 
%	to compute the closed form solution to cubic subproblem (\ref{eq:cr}).  
%	\textbf{Computational complexity.} Let  $m$ be the number of previous iterates and gradients stored in memory. The proposed LSR1 ARC approach is comparable to L-BFGS in terms of storage and compute complexity (see Table \ref{tbl:storagecomplexity}).  
%	\begin{table*}[h]
%		\centering
%		\caption{Storage and compute complexity of the methods used in our experiments.}
%		\begin{tabular}{|c|c|c|}
%			\hline
%			\textbf{Algorithms} & \textbf{Storage complexity} & \textbf{Compute complexity}\\
%			\hline
%			SGD/Adaptive methods & $\mathcal{O}(n)$ & $\mathcal{O}(n)$ \\
%			L-BFGS & $\mathcal{O}(n + mn)$ &  $\mathcal{O}(mn)$\\
%			ARCs-LSR1 & $\mathcal{O}(n + mn)$ & $\mathcal{O}(m^3 + 2mn)$ \\ 
%			\hline
%		\end{tabular}\label{tbl:storagecomplexity}
%		\centering
%	\end{table*}
%\end{enumerate}
%%

\medskip

\noindent \textbf{Convergence.} 
Here, we prove convergence properties of the proposed method (ARCs-LSR1 in Algorithm \ref{alg:LSR1ARC}).
The following theoretical guarantees follow the ideas from \citet{Benson2018,cartis2011adaptive}.
First, we make the following mild assumptions:


\medskip

\noindent 
\textbf{A1.} The loss function $f(\Theta)$ is continuously differentiable, i.e., 
$f \in C^1(\mathbb{R}^n)$.

\noindent
\textbf{A2.} The loss function $f(\Theta)$ is bounded below.

\medskip

%
%It is reasonable to assume that the function $f$ in \eqref{eq:emp} is bounded below by some value $K$ and continuous.
%\begin{lemma}\label{con:lemma1}
%		$f \in C^1(\mathbb{R}^n)$
%\end{lemma}

\noindent 
Next, 
%under the assumption that the norm of the rank-1 matrix $(\mathbf{y}_j - \mathbf{B}_j\mathbf{s}_j)
%	(\mathbf{y}_j - \mathbf{B}_j\mathbf{s}_j)^{\top}$ 
%	in (\ref{eq:SR1_B0}) is bounded above
%	(see \cite{Benson2018}), 
%we obtain a upper bound on the norm of the Hessian approximation $\mathbf{B}_k$.
we prove that the matrix $\mathbf{B}_k$ in (\ref{eq:SR1_B0}) is bounded.  

\begin{lemma}\label{lemma:1}
%If $\| (\mathbf{y}_j - \mathbf{B}_j\mathbf{s}_j)
%(\mathbf{y}_j - \mathbf{B}_j\mathbf{s}_j)^{\top} \|_F \le K$ for some constant $K > 0$, then
The SR1 matrix $\mathbf{B}_{k+1}$  in (\ref{eq:SR1_B0}) satsifies
$$
	\text{$\|\mathbf{B}_{k+1}\|_F  \leq \kappa_B$  \ \ \text{for all $k \geq$ 1}}
$$
for some $\kappa_B$ $>$ 0.
\end{lemma}

\textit{Proof:} 
Using the limited-memory SR1 update with memory parameter $m$ in (\ref{eq:SR1_B0}), we have
$$
	\| \mathbf{B}_{k+1} \|_F \le \| \mathbf{B}_0 \|_F + 
	\hspace{-.4cm}
	\sum_{j = k-m+1}^k 
	\hspace{-.4cm} 
	\frac{\| (\mathbf{y}_j - \mathbf{B}_j\mathbf{s}_j) (\mathbf{y}_j - \mathbf{B}_j\mathbf{s}_j)^{\! \top} \! \|_F}
	{| \mathbf{s}_j^{\top} ( \mathbf{y}_j - \mathbf{B}_j\mathbf{s}_j) |}.
%	\le \delta_{\max} + \frac{m K}{\varepsilon} \equiv \kappa_B.
$$
Because $\mathbf{B}_0 = \delta_k \mathbf{I}$ with $\delta_k < \delta_{\max}$ for some $\delta_{\max} > 0$,
we have that $\| \mathbf{B}_0 \|_F = \sqrt{n} \delta_{\max}$.  
Using a property of the Frobenius norm,
namely, for real matrices $\mathbf{A}$, $\| \mathbf{A} \|_F^2 = \text{trace}(\mathbf{AA}^{\top})$, we have that
$\| (\mathbf{y}_j - \mathbf{B}_j\mathbf{s}_j) (\mathbf{y}_j - \mathbf{B}_j\mathbf{s}_j)^{\top} \|_F 
= \| \mathbf{y}_j - \mathbf{B}_j\mathbf{s}_j \|_2^2$.
Since the pair $(\mathbf{s}_j, \mathbf{y}_j)$ is accepted only when $|\mathbf{s}_j^{\top}(\mathbf{y}_j - \mathbf{B}_j\mathbf{s}_j)| > \varepsilon \| \mathbf{s}_j \|_2 \| \mathbf{y}_j - \mathbf{B}_j \mathbf{s}_j \|_2$, for some constant $\varepsilon > 0$, and since $\| \mathbf{s}_k \|_2 \ge \tilde{\epsilon} \| \mathbf{y}_k - \mathbf{B}_k \mathbf{s}_k\|_2$, we have
$$
	\| \mathbf{B}_{k+1} \|_F \le \sqrt{n} \delta_{\max} + \frac{m}{\varepsilon \tilde{\epsilon}} \equiv \kappa_B,
$$
which completes the proof.
$\square$

\medskip

\noindent 
Given the bound on $\| \mathbf{B}_{k+1} \|_F$, we obtain the following result, which is similar to Theorem 2.5 in \citet{cartis2011adaptive}.

\begin{theorem}\label{thm:liminf}
	Under Assumptions \textbf{A1}  and \textbf{A2}, if Lemma \ref{lemma:1} holds, then
	$$\underset{k \to \infty}{\text{lim inf}} \  \|\mathbf{g}_k\| = 0.$$
\end{theorem}

\noindent 
Finally, we consider the following assumption, which can be satisfied when the gradient, $\mathbf{g}(\Theta)$, is Lipschitz continuous on $\Theta$. 

\medskip

\noindent 
\textbf{A3.} If $\{ \Theta_{t_i} \}$ and $\{ \Theta_{l_i} \}$ are subsequences of $\{ \Theta_k \}$, then  $\| \mathbf{g}_{t_i} - \mathbf{g}_{l_i} \| \rightarrow 0$ whenever 
$\| \Theta_{t_i} - \Theta_{l_i} \| \rightarrow 0$ as $i \rightarrow \infty$.


\medskip

\noindent 
If we further make Assumption  \textbf{A3}, we have the following stronger result (which is based on Corollary 2.6 in \citet{cartis2011adaptive}):

\begin{corollary}\label{cor:ARCs}
Under Assumptions \textbf{A1},  \textbf{A2}, and \textbf{A3}, 
 if Lemma \ref{lemma:1} holds, then
	$$\underset{k \to \infty}{\text{lim}} \|\mathbf{g}_k\| = 0.$$
\end{corollary}

By Corollary 2.3, the proposed ARCs-LSR1 method converges to first-order critical points.   


\noindent \textbf{Stochastic implementation.} Because full gradient computation is very expensive to perform, we impement a stochastic version 
of the proposed ARCs-LSR1 method.  In particular, we use the batch gradient approximation
$$
	\tilde{\mathbf{g}}_k \equiv \frac{1}{| \mathcal{B}_k |} \sum_{i \in \mathcal{B}_k} \nabla f_i (\Theta_k).
$$
In defining the SR1 matrix, we use the quasi-Newton pairs $(\mathbf{s}_k, \tilde{\mathbf{y}}_k)$,
where $\tilde{\mathbf{y}}_k = \tilde{\mathbf{g}}_{k+1} - \tilde{\mathbf{g}}_k$ (see e.g., \citet{Erway2020TrustregionAF}).
We make the following additional assumption (similar to Assumption 4 in  \citet{Erway2020TrustregionAF}) to guarantee that the loss function $f(\Theta)$ decreases over time:

\medskip

\noindent
\textbf{A4.} The loss function $f(\Theta)$ is fully evaluated at every $J > 1$ iterations (for example, 
at iterates $\Theta_{J_0}, \Theta_{J_1}, \Theta_{J_2}, \dots,$ where $0 \le J_0 < J$ and
$J = J_1 - J_0 = J_2 - J_1 = \cdots $) and nowhere else in the algorithm.  The batch size $d$ is increased 
monotonically if $f(\Theta_{J_{\ell}}) > f(\Theta_{J_{\ell - 1}}) - \tau$ for some $\tau > 0$.

\medskip

\noindent 
With this added assumption, we can show that the stochastic version of the proposed ARCs-LSR1 method converges.

\begin{theorem}\label{thm:sARCs}
	The stochastic version of ARCs-LSR1 converges with  
	$$\underset{k \to \infty}{\text{lim}} \|\mathbf{g}_k\| = 0.$$
\end{theorem}

\textit{Proof:} Let $\widehat{\Theta}_i = \Theta_{J_i}$.  By Assumption 4, $f(\Theta)$ must 
decrease monotonically over the subsequence $\{ \widehat{\Theta}_i \}$ or $d \rightarrow |\mathcal{D}|$,
where $|\mathcal{D}|$ is the size of the dataset.    If the objective function is decreased 
$\iota_k$ times over the subsequence $ \{ \widehat{\Theta}_i\}_{i=0}^k$, then
%\begin{eqnarray*}
%	f(\widehat{\Theta}_k) &=& f(\hat{\Theta}_0) + \sum_{i=1}^{\iota_k}
%	\left \{
%		f(\widehat{\Theta}_i) - f(\widehat{\Theta}_{i-1})
%	\right \} \\
%	&\le& f(\widehat{\Theta}_0) - \iota_k \tau.
%\end{eqnarray*}
\begin{eqnarray*}
	f(\widehat{\Theta}_k) = f(\hat{\Theta}_0)  + \sum_{i=1}^{\iota_k}
	\left \{
		f(\widehat{\Theta}_i)  -  f(\widehat{\Theta}_{i-1})
	\right \} \le  f(\widehat{\Theta}_0) - \iota_k \tau.
\end{eqnarray*}
If $d \rightarrow |\mathcal{D}|$, then $\iota_k \rightarrow \infty$ as $k \rightarrow \infty$.
By Assumption \textbf{A2}, $f(\Theta)$ is bounded below, which implies $\iota_k$ is finite.  
Thus, $d \rightarrow |\mathcal{D}|$, and the algorithm reduces to the full ARCs-LSR1 method,
whose convergence is guaranteed by Corollary \ref{cor:ARCs}.  $\square$

\medskip

\noindent 
We note that the proof to Theorem \ref{thm:sARCs} follows very closely the proof of Theorem 2.2 in 
\citet{Erway2020TrustregionAF}.


%\ref{appnd:stochastic}.  

%\begin{equation*}
%	\mathbf{B}_{k+1} = \mathbf{B}_0 + \mathbf{Y}_k -\delta_k \mathbf{S}_k \mathbf{M}_k (\mathbf{Y}_k -\delta_k \mathbf{S}_k)^\top,
%\end{equation*}
%where $\mathbf{B}_0 = \delta_k I$. Using the triangle inequality, we have
%\begin{equation*}
%	\norm{\mathbf{B}_{k+1}} \leq \norm{\mathbf{B}_0}  + \norm{\mathbf{Y}_k -\delta_k \mathbf{S}_k \mathbf{M}_k (\mathbf{Y}_k -\delta_k \mathbf{S}_k)^\top}.
%\end{equation*}
%We rewrite the equation above as follows:
%\begin{equation*}
%	\norm{\mathbf{B}_{k+1}} \leq \norm{\mathbf{B}_0}  + \frac{m}{\epsilon} \max_{k - m + 1\leq i \leq k} \norm{\mathbf{y}_i - \mathbf{B}_i \mathbf{s}_i}_2^2
%\end{equation*}
%
%It is reasonable to assume that $\norm{\mathbf{y}_i - \mathbf{B}_i \mathbf{s}_i}_2^2 = (\mathbf{y}_i - \mathbf{B}_i \mathbf{s}_i)(\mathbf{y}_i - \mathbf{B}_i \mathbf{s}_i)^T$ is bounded above in norm (see \cite{Benson2018} Lemma 1). Thus, we prove $\mathbf{B}_k$ is bounded:
%
%If lemma \ref{con:lemma1} and lemma \ref{con:lemma2} hold, the following theorem is obtained.

%\begin{theorem}\label{thm:con}
%	$\underset{k \to \infty}{\text{lim inf}} \norm{g_k} = 0$ 
%\end{theorem}
%For proof, please refer \cite{Benson2018}, theorem 2.5.

%Additionally, we make the following assumption,
%\begin{assumption}\label{con}
%	$\norm{g_t - g_l} \to 0$ whenever $\norm{\Theta_t - \Theta_l} \to 0$, $i \to \infty$.
%\end{assumption}
%
%
%
%Since Lemma \ref{con:lemma1}, Lemma \ref{con:lemma2}, Theorem \ref{thm:con},\ref{con} hold and $f(\Theta_k)$is bounded below, we state the following corollary
%\begin{corollary}
%	$\underset{k \to \infty}{\text{lim}} \norm{g_k} = 0$.
%\end{corollary}

\noindent \textbf{Complexity analysis.} 
SGD methods and the related adaptive methods require $\mathcal{O}(n)$ memory storage to store
the gradient and $\mathcal{O}(n)$ computational complexity to update each iterate.
Such low memory and computational requirements make these methods easily implementable.  
Quasi-Newton methods store the previous $m$ gradients and use them to compute the update at each iteration.  Consequently,  L-BFGS methods require $\mathcal{O}(mn)$ memory storage to store
the gradients and $\mathcal{O}(mn)$ computational complexity to update each iterate 
(see \citet{Burdakov2017} for details).  Our proposed ARCs-LSR1 approach also uses 
$\mathcal{O}(mn)$ memory storage to store the gradients, but the computational 
complexity to update each iterate requires an additional eigendecomposition of the $m \times m$
matrix $\mathbf{RMR}^{\top}$, so that the overall computational complexity at each iteration is
$\mathcal{O}(m^3+ mn)$.  However, since $m \ll n$, this additional factorization does not significantly
increase the computational time.% (see Table \ref{tbl:storagecomplexity}).

%	\begin{table}[!h]
%		\centering
%		\caption{Storage and compute complexity of the methods used in our experiments.}
%		\begin{tabular}{|c|c|c|}
%			\hline
%			\textbf{Algorithms} & \textbf{Storage complexity} & \textbf{Compute complexity}\\
%			\hline
%			SGD/Adaptive methods & $\mathcal{O}(n)$ & $\mathcal{O}(n)$ \\
%			L-BFGS & $\mathcal{O}(n + mn)$ &  $\mathcal{O}(mn)$\\
%			ARCs-LSR1 & $\mathcal{O}(n + mn)$ & $\mathcal{O}(m^3 + 2mn)$ \\ 
%			\hline
%		\end{tabular}\label{tbl:storagecomplexity}
%		\centering
%	\end{table}

\section{Results}
\label{sec:Experiments}
\section{Experimental Analysis}
\label{sec:exp}
We now describe in detail our experimental analysis. The experimental section is organized as follows:
%\begin{enumerate}[noitemsep,topsep=0pt,parsep=0pt,partopsep=0pt,leftmargin=0.5cm]
%\item 

\noindent In {\bf 
Section~\ref{exp:setup}}, we introduce the datasets and methods to evaluate the previously defined accuracy measures.

%\item
\noindent In {\bf 
Section~\ref{exp:qual}}, we illustrate the limitations of existing measures with some selected qualitative examples.

%\item 
\noindent In {\bf 
Section~\ref{exp:quant}}, we continue by measuring quantitatively the benefits of our proposed measures in terms of {\it robustness} to lag, noise, and normal/abnormal ratio.

%\item 
\noindent In {\bf 
Section~\ref{exp:separability}}, we evaluate the {\it separability} degree of accurate and inaccurate methods, using the existing and our proposed approaches.

%\item
\noindent In {\bf 
Section~\ref{sec:entropy}}, we conduct a {\it consistency} evaluation, in which we analyze the variation of ranks that an AD method can have with an accuracy measures used.

%\item 
\noindent In {\bf 
Section~\ref{sec:exectime}}, we conduct an {\it execution time} evaluation, in which we analyze the impact of different parameters related to the accuracy measures and the time series characteristics. 
We focus especially on the comparison of the different VUS implementations.
%\end{enumerate}

\begin{table}[tb]
\caption{Summary characteristics (averaged per dataset) of the public datasets of TSB-UAD (S.: Size, Ano.: Anomalies, Ab.: Abnormal, Den.: Density)}
\label{table:charac}
%\vspace{-0.2cm}
\footnotesize
\begin{center}
\scalebox{0.82}{
\begin{tabular}{ |r|r|r|r|r|r|} 
 \hline
\textbf{\begin{tabular}[c]{@{}c@{}}Dataset \end{tabular}} & 
\textbf{\begin{tabular}[c]{@{}c@{}}S. \end{tabular}} & 
\textbf{\begin{tabular}[c]{c@{}} Len.\end{tabular}} & 
\textbf{\begin{tabular}[c]{c@{}} \# \\ Ano. \end{tabular}} &
\textbf{\begin{tabular}[c]{c@{}c@{}} \# \\ Ab. \\ Points\end{tabular}} &
\textbf{\begin{tabular}[c]{c@{}c@{}} Ab. \\ Den. \\ (\%)\end{tabular}} \\ \hline
Dodgers \cite{10.1145/1150402.1150428} & 1 & 50400   & 133.0     & 5612.0  &11.14 \\ \hline
SED \cite{doi:10.1177/1475921710395811}& 1 & 100000   & 75.0     & 3750.0  & 3.7\\ \hline
ECG \cite{goldberger_physiobank_2000}   & 52 & 230351  & 195.6     & 15634.0  &6.8 \\ \hline
IOPS \cite{IOPS}   & 58 & 102119  & 46.5     & 2312.3   &2.1 \\ \hline
KDD21 \cite{kdd} & 250 &77415   & 1      & 196.5   &0.56 \\ \hline
MGAB \cite{markus_thill_2020_3762385}   & 10 & 100000  & 10.0     & 200.0   &0.20 \\ \hline
NAB \cite{ahmad_unsupervised_2017}   & 58 & 6301   & 2.0      & 575.5   &8.8 \\ \hline
NASA-M. \cite{10.1145/3449726.3459411}   & 27 & 2730   & 1.33      & 286.3   &11.97 \\ \hline
NASA-S. \cite{10.1145/3449726.3459411}   & 54 & 8066   & 1.26      & 1032.4   &12.39 \\ \hline
SensorS. \cite{YAO20101059}   & 23 & 27038   & 11.2     & 6110.4   &22.5 \\ \hline
YAHOO \cite{yahoo}  & 367 & 1561   & 5.9      & 10.7   &0.70 \\ \hline 
\end{tabular}}
\end{center}
\end{table}











\subsection{Experimental Setup and Settings}
\label{exp:setup}
%\vspace{-0.1cm}

\begin{figure*}[tb]
  \centering
  \includegraphics[width=1\linewidth]{figures/quality.pdf}
  %\vspace{-0.7cm}
  \caption{Comparison of evaluation measures (proposed measures illustrated in subplots (b,c,d,e); all others summarized in subplots (f)) on two examples ((A)AE and OCSM applied on MBA(805) and (B) LOF and OCSVM applied on MBA(806)), illustrating the limitations of existing measures for scores with noise or containing a lag. }
  \label{fig:quality}
  %\vspace{-0.1cm}
\end{figure*}

We implemented the experimental scripts in Python 3.8 with the following main dependencies: sklearn 0.23.0, tensorflow 2.3.0, pandas 1.2.5, and networkx 2.6.3. In addition, we used implementations from our TSB-UAD benchmark suite.\footnote{\scriptsize \url{https://www.timeseries.org/TSB-UAD}} For reproducibility purposes, we make our datasets and code available.\footnote{\scriptsize \url{https://www.timeseries.org/VUS}}
\newline \textbf{Datasets: } For our evaluation purposes, we use the public datasets identified in our TSB-UAD benchmark. The latter corresponds to $10$ datasets proposed in the past decades in the literature containing $900$ time series with labeled anomalies. Specifically, each point in every time series is labeled as normal or abnormal. Table~\ref{table:charac} summarizes relevant characteristics of the datasets, including their size, length, and statistics about the anomalies. In more detail:

\begin{itemize}
    \item {\bf SED}~\cite{doi:10.1177/1475921710395811}, from the NASA Rotary Dynamics Laboratory, records disk revolutions measured over several runs (3K rpm speed).
	\item {\bf ECG}~\cite{goldberger_physiobank_2000} is a standard electrocardiogram dataset and the anomalies represent ventricular premature contractions. MBA(14046) is split to $47$ series.
	\item {\bf IOPS}~\cite{IOPS} is a dataset with performance indicators that reflect the scale, quality of web services, and health status of a machine.
	\item {\bf KDD21}~\cite{kdd} is a composite dataset released in a SIGKDD 2021 competition with 250 time series.
	\item {\bf MGAB}~\cite{markus_thill_2020_3762385} is composed of Mackey-Glass time series with non-trivial anomalies. Mackey-Glass data series exhibit chaotic behavior that is difficult for the human eye to distinguish.
	\item {\bf NAB}~\cite{ahmad_unsupervised_2017} is composed of labeled real-world and artificial time series including AWS server metrics, online advertisement clicking rates, real time traffic data, and a collection of Twitter mentions of large publicly-traded companies.
	\item {\bf NASA-SMAP} and {\bf NASA-MSL}~\cite{10.1145/3449726.3459411} are two real spacecraft telemetry data with anomalies from Soil Moisture Active Passive (SMAP) satellite and Curiosity Rover on Mars (MSL).
	\item {\bf SensorScope}~\cite{YAO20101059} is a collection of environmental data, such as temperature, humidity, and solar radiation, collected from a sensor measurement system.
	\item {\bf Yahoo}~\cite{yahoo} is a dataset consisting of real and synthetic time series based on the real production traffic to some of the Yahoo production systems.
\end{itemize}


\textbf{Anomaly Detection Methods: }  For the experimental evaluation, we consider the following baselines. 

\begin{itemize}
\item {\bf Isolation Forest (IForest)}~\cite{liu_isolation_2008} constructs binary trees based on random space splitting. The nodes (subsequences in our specific case) with shorter path lengths to the root (averaged over every random tree) are more likely to be anomalies. 
\item {\bf The Local Outlier Factor (LOF)}~\cite{breunig_lof_2000} computes the ratio of the neighbor density to the local density. 
\item {\bf Matrix Profile (MP)}~\cite{yeh_time_2018} detects as anomaly the subsequence with the most significant 1-NN distance. 
\item {\bf NormA}~\cite{boniol_unsupervised_2021} identifies the normal patterns based on clustering and calculates each point's distance to normal patterns weighted using statistical criteria. 
\item {\bf Principal Component Analysis (PCA)}~\cite{aggarwal_outlier_2017} projects data to a lower-dimensional hyperplane. Outliers are points with a large distance from this plane. 
\item {\bf Autoencoder (AE)} \cite{10.1145/2689746.2689747} projects data to a lower-dimensional space and reconstructs it. Outliers are expected to have larger reconstruction errors. 
\item {\bf LSTM-AD}~\cite{malhotra_long_2015} use an LSTM network that predicts future values from the current subsequence. The prediction error is used to identify anomalies.
\item {\bf Polynomial Approximation (POLY)} \cite{li_unifying_2007} fits a polynomial model that tries to predict the values of the data series from the previous subsequences. Outliers are detected with the prediction error. 
\item {\bf CNN} \cite{8581424} built, using a convolutional deep neural network, a correlation between current and previous subsequences, and outliers are detected by the deviation between the prediction and the actual value. 
\item {\bf One-class Support Vector Machines (OCSVM)} \cite{scholkopf_support_1999} is a support vector method that fits a training dataset and finds the normal data's boundary.
\end{itemize}

\subsection{Qualitative Analysis}
\label{exp:qual}



We first use two examples to demonstrate qualitatively the limitations of existing accuracy evaluation measures in the presence of lag and noise, and to motivate the need for a new approach. 
These two examples are depicted in Figure~\ref{fig:quality}. 
The first example, in Figure~\ref{fig:quality}(A), corresponds to OCSVM and AE on the MBA(805) dataset (named MBA\_ECG805\_data.out in the ECG dataset). 

We observe in Figure~\ref{fig:quality}(A)(a.1) and (a.2) that both scores identify most of the anomalies (highlighted in red). However, the OCSVM score points to more false positives (at the end of the time series) and only captures small sections of the anomalies. On the contrary, the AE score points to fewer false positives and captures all abnormal subsequences. Thus we can conclude that, visually, AE should obtain a better accuracy score than OCSVM. Nevertheless, we also observe that the AE score is lagged with the labels and contains more noise. The latter has a significant impact on the accuracy of evaluation measures. First, Figure~\ref{fig:quality}(A)(c) is showing that AUC-PR is better for OCSM (0.73) than for AE (0.57). This is contradictory with what is visually observed from Figure~\ref{fig:quality}(A)(a.1) and (a.2). However, when using our proposed measure R-AUC-PR, OCSVM obtains a lower score (0.83) than AE (0.89). This confirms that, in this example, a buffer region before the labels helps to capture the true value of an anomaly score. Overall, Figure~\ref{fig:quality}(A)(f) is showing in green and red the evolution of accuracy score for the 13 accuracy measures for AE and OCSVM, respectively. The latter shows that, in addition to Precision@k and Precision, our proposed approach captures the quality order between the two methods well.

We now present a second example, on a different time series, illustrated in Figure~\ref{fig:quality}(B). 
In this case, we demonstrate the anomaly score of OCSVM and LOF (depicted in Figure~\ref{fig:quality}(B)(a.1) and (a.2)) applied on the MBA(806) dataset (named MBA\_ECG806\_data.out in the ECG dataset). 
We observe that both methods produce the same level of noise. However, LOF points to fewer false positives and captures more sections of the abnormal subsequences than OCSVM. 
Nevertheless, the LOF score is slightly lagged with the labels such that the maximum values in the LOF score are slightly outside of the labeled sections. 
Thus, as illustrated in Figure~\ref{fig:quality}(B)(f), even though we can visually consider that LOF is performing better than OCSM, all usual measures (Precision, Recall, F, precision@k, and AUC-PR) are judging OCSM better than AE. On the contrary, measures that consider lag (Rprecision, Rrecall, RF) rank the methods correctly. 
However, due to threshold issues, these measures are very close for the two methods. Overall, only AUC-ROC and our proposed measures give a higher score for LOF than for OCSVM.

\subsection{Quantitative Analysis}
\label{exp:case}

\begin{figure}[t]
  \centering
  \includegraphics[width=1\linewidth]{figures/eval_case_study.pdf}
  %\vspace*{-0.7cm}
  \caption{\commentRed{
  Comparison of evaluation measures for synthetic data examples across various scenarios. S8 represents the oracle case, where predictions perfectly align with labeled anomalies. Problematic cases are highlighted in the red region.}}
  %\vspace*{-0.5cm}
  \label{fig:eval_case_study}
\end{figure}
\commentRed{
We present the evaluation results for different synthetic data scenarios, as shown in Figure~\ref{fig:eval_case_study}. These scenarios range from S1, where predictions occur before the ground truth anomaly, to S12, where predictions fall within the ground truth region. The red-shaded regions highlight problematic cases caused by a lack of adaptability to lags. For instance, in scenarios S1 and S2, a slight shift in the prediction leads to measures (e.g., AUC-PR, F score) that fail to account for lags, resulting in a zero score for S1 and a significant discrepancy between the results of S1 and S2. Thus, we observe that our proposed VUS effectively addresses these issues and provides robust evaluations results.}

%\subsection{Quantitative Analysis}
%\subsection{Sensitivity and Separability Analysis}
\subsection{Robustness Analysis}
\label{exp:quant}


\begin{figure}[tb]
  \centering
  \includegraphics[width=1\linewidth]{figures/lag_sensitivity_analysis.pdf}
  %\vspace*{-0.7cm}
  \caption{For each method, we compute the accuracy measures 10 times with random lag $\ell \in [-0.25*\ell,0.25*\ell]$ injected in the anomaly score. We center the accuracy average to 0.}
  %\vspace*{-0.5cm}
  \label{fig:lagsensitivity}
\end{figure}

We have illustrated with specific examples several of the limitations of current measures. 
We now evaluate quantitatively the robustness of the proposed measures when compared to the currently used measures. 
We first evaluate the robustness to noise, lag, and normal versus abnormal points ratio. We then measure their ability to separate accurate and inaccurate methods.
%\newline \textbf{Sensitivity Analysis: } 
We first analyze the robustness of different approaches quantitatively to different factors: (i) lag, (ii) noise, and (iii) normal/abnormal ratio. As already mentioned, these factors are realistic. For instance, lag can be either introduced by the anomaly detection methods (such as methods that produce a score per subsequences are only high at the beginning of abnormal subsequences) or by human labeling approximation. Furthermore, even though lag and noises are injected, an optimal evaluation metric should not vary significantly. Therefore, we aim to measure the variance of the evaluation measures when we vary the lag, noise, and normal/abnormal ratio. We proceed as follows:

\begin{enumerate}[noitemsep,topsep=0pt,parsep=0pt,partopsep=0pt,leftmargin=0.5cm]
\item For each anomaly detection method, we first compute the anomaly score on a given time series.
\item We then inject either lag $l$, noise $n$ or change the normal/abnormal ratio $r$. For 10 different values of $l \in [-0.25*\ell,0.25*\ell]$, $n \in [-0.05*(max(S_T)-min(S_T)),0.05*(max(S_T)-min(S_T))]$ and $r \in [0.01,0.2]$, we compute the 13 different measures.
\item For each evaluation measure, we compute the standard deviation of the ten different values. Figure~\ref{fig:lagsensitivity}(b) depicts the different lag values for six AD methods applied on a data series in the ECG dataset.
\item We compute the average standard deviation for the 13 different AD quality measures. For example, figure~\ref{fig:lagsensitivity}(a) depicts the average standard deviation for ten different lag values over the AD methods applied on the MBA(805) time series.
\item We compute the average standard deviation for the every time series in each dataset (as illustrated in Figure~\ref{fig:sensitivity_per_data}(b to j) for nine datasets of the benchmark.
\item We compute the average standard deviation for the every dataset (as illustrated in Figure~\ref{fig:sensitivity_per_data}(a.1) for lag, Figure~\ref{fig:sensitivity_per_data}(a.2) for noise and Figure~\ref{fig:sensitivity_per_data}(a.3) for normal/abnormal ratio).
\item We finally compute the Wilcoxon test~\cite{10.2307/3001968} and display the critical diagram over the average standard deviation for every time series (as illustrated in Figure~\ref{fig:sensitivity}(a.1) for lag, Figure~\ref{fig:sensitivity}(a.2) for noise and Figure~\ref{fig:sensitivity}(a.3) for normal/abnormal ratio).
\end{enumerate}

%height=8.5cm,

\begin{figure}[tb]
  \centering
  \includegraphics[width=\linewidth]{figures/sensitivity_per_data_long.pdf}
%  %\vspace*{-0.3cm}
  \caption{Robustness Analysis for nine datasets: we report, over the entire benchmark, the average standard deviation of the accuracy values of the measures, under varying (a.1) lag, (a.2) noise, and (a.3) normal/abnormal ratio. }
  \label{fig:sensitivity_per_data}
\end{figure}

\begin{figure*}[tb]
  \centering
  \includegraphics[width=\linewidth]{figures/sensitivity_analysis.pdf}
  %\vspace*{-0.7cm}
  \caption{Critical difference diagram computed using the signed-rank Wilkoxon test (with $\alpha=0.1$) for the robustness to (a.1) lag, (a.2) noise and (a.3) normal/abnormal ratio.}
  \label{fig:sensitivity}
\end{figure*}

The methods with the smallest standard deviation can be considered more robust to lag, noise, or normal/abnormal ratio from the above framework. 
First, as stated in the introduction, we observe that non-threshold-based measures (such as AUC-ROC and AUC-PR) are indeed robust to noise (see Figure~\ref{fig:sensitivity_per_data}(a.2)), but not to lag. Figure~\ref{fig:sensitivity}(a.1) demonstrates that our proposed measures VUS-ROC, VUS-PR, R-AUC-ROC, and R-AUC-PR are significantly more robust to lag. Similarly, Figure~\ref{fig:sensitivity}(a.2) confirms that our proposed measures are significantly more robust to noise. However, we observe that, among our proposed measures, only VUS-ROC and R-AUC-ROC are robust to the normal/abnormal ratio and not VUS-PR and R-AUC-PR. This is explained by the fact that Precision-based measures vary significantly when this ratio changes. This is confirmed by Figure~\ref{fig:sensitivity_per_data}(a.3), in which we observe that Precision and Rprecision have a high standard deviation. Overall, we observe that VUS-ROC is significantly more robust to lag, noise, and normal/abnormal ratio than other measures.




\subsection{Separability Analysis}
\label{exp:separability}

%\newline \textbf{Separability Analysis: } 
We now evaluate the separability capacities of the different evaluation metrics. 
\commentRed{The main objective is to measure the ability of accuracy measures to separate accurate methods from inaccurate ones. More precisely, an appropriate measure should return accuracy scores that are significantly higher for accurate anomaly scores than for inaccurate ones.}
We thus manually select accurate and inaccurate anomaly detection methods and verify if the accuracy evaluation scores are indeed higher for the accurate than for the inaccurate methods. Figure~\ref{fig:separability} depicts the latter separability analysis applied to the MBA(805) and the SED series. 
The accurate and inaccurate anomaly scores are plotted in green and red, respectively. 
We then consider 12 different pairs of accurate/inaccurate methods among the eight previously mentioned anomaly scores. 
We slightly modify each score 50 different times in which we inject lag and noises and compute the accuracy measures. 
Figure~\ref{fig:separability}(a.4) and Figure~\ref{fig:separability}(b.4) are divided into four different subplots corresponding to 4 pairs (selected among the twelve different pairs due to lack of space). 
Each subplot corresponds to two box plots per accuracy measure. 
The green and red box plots correspond to the 50 accuracy measures on the accurate and inaccurate methods. 
If the red and green box plots are well separated, we can conclude that the corresponding accuracy measures are separating the accurate and inaccurate methods well. 
We observe that some accuracy measures (such as VUS-ROC) are more separable than others (such as RF). We thus measure the separability of the two box-plots by computing the Z-test. 

\begin{figure*}[tb]
  \centering
  \includegraphics[width=1\linewidth]{figures/pairwise_comp_example_long.pdf}
  %\vspace*{-0.5cm}
  \caption{Separability analysis applied on 4 pairs of accurate (green) and inaccurate (red) methods on (a) the MBA(805) data series, and (b) the SED data series.}
  %\vspace*{-0.3cm}
  \label{fig:separability}
\end{figure*}

We now aggregate all the results and compute the average Z-test for all pairs of accurate/inaccurate datasets (examples are shown in Figures~\ref{fig:separability}(a.2) and (b.2) for accurate anomaly scores, and in Figures~\ref{fig:separability}(a.3) and (b.3) for inaccurate anomaly scores, for the MBA(805) and SED series, respectively). 
Next, we perform the same operation over three different data series: MBA (805), MBA(820), and SED. 
Then, we depict the average Z-test for these three datasets in Figure~\ref{fig:separability_agg}(a). 
Finally, we show the average Z-test for all datasets in Figure~\ref{fig:separability_agg}(b). 


We observe that our proposed VUS-based and Range-based measures are significantly more separable than other current accuracy measures (up to two times for AUC-ROC, the best measures of all current ones). Furthermore, when analyzed in detail in Figure~\ref{fig:separability} and Figure~\ref{fig:separability_agg}, we confirm that VUS-based and Range-based are more separable over all three datasets. 

\begin{figure}[tb]
  \centering
  \includegraphics[width=\linewidth]{figures/agregated_sep_analysis.pdf}
  %\vspace*{-0.5cm}
  \caption{Overall separability analysis (averaged z-test between the accuracy values distributions of accurate and inaccurate methods) applied on 36 pairs on 3 datasets.}
  \label{fig:separability_agg}
\end{figure}


\noindent \textbf{Global Analysis: } Overall, we observe that VUS-ROC is the most robust (cf. Figure~\ref{fig:sensitivity}) and separable (cf. Figure~\ref{fig:separability_agg}) measure. 
On the contrary, Precision and Rprecision are non-robust and non-separable. 
Among all previous accuracy measures, only AUC-ROC is robust and separable. 
Popular measures, such as, F, RF, AUC-ROC, and AUC-PR are robust but non-separable.

In order to visualize the global statistical analysis, we merge the robustness and the separability analysis into a single plot. Figure~\ref{fig:global} depicts one scatter point per accuracy measure. 
The x-axis represents the averaged standard deviation of lag and noise (averaged values from Figure~\ref{fig:sensitivity_per_data}(a.1) and (a.2)). The y-axis corresponds to the averaged Z-test (averaged value from Figure~\ref{fig:separability_agg}). 
Finally, the size of the points corresponds to the sensitivity to the normal/abnormal ratio (values from Figure~\ref{fig:sensitivity_per_data}(a.3)). 
Figure~\ref{fig:global} demonstrates that our proposed measures (located at the top left section of the plot) are both the most robust and the most separable. 
Among all previous accuracy measures, only AUC-ROC is on the top left section of the plot. 
Popular measures, such as, F, RF, AUC-ROC, AUC-PR are on the bottom left section of the plot. 
The latter underlines the fact that these measures are robust but non-separable.
Overall, Figure~\ref{fig:global} confirms the effectiveness and superiority of our proposed measures, especially of VUS-ROC and VUS-PR.


\begin{figure}[tb]
  \centering
  \includegraphics[width=\linewidth]{figures/final_result.pdf}
  \caption{Evaluation of all measures based on: (y-axis) their separability (avg. z-test), (x-axis) avg. standard deviation of the accuracy values when varying lag and noise, (circle size) avg. standard deviation of the accuracy values when varying the normal/abnormal ratio.}
  \label{fig:global}
\end{figure}




\subsection{Consistency Analysis}
\label{sec:entropy}

In this section, we analyze the accuracy of the anomaly detection methods provided by the 13 accuracy measures. The objective is to observe the changes in the global ranking of anomaly detection methods. For that purpose, we formulate the following assumptions. First, we assume that the data series in each benchmark dataset are similar (i.e., from the same domain and sharing some common characteristics). As a matter of fact, we can assume that an anomaly detection method should perform similarly on these data series of a given dataset. This is confirmed when observing that the best anomaly detection methods are not the same based on which dataset was analyzed. Thus the ranking of the anomaly detection methods should be different for different datasets, but similar for every data series in each dataset. 
Therefore, for a given method $A$ and a given dataset $D$ containing data series of the same type and domain, we assume that a good accuracy measure results in a consistent rank for the method $A$ across the dataset $D$. 
The consistency of a method's ranks over a dataset can be measured by computing the entropy of these ranks. 
For instance, a measure that returns a random score (and thus, a random rank for a method $A$) will result in a high entropy. 
On the contrary, a measure that always returns (approximately) the same ranks for a given method $A$ will result in a low entropy. 
Thus, for a given method $A$ and a given dataset $D$ containing data series of the same type and domain, we assume that a good accuracy measure results in a low entropy for the different ranks for method $A$ on dataset $D$.

\begin{figure*}[tb]
  \centering
  \includegraphics[width=\linewidth]{figures/entropy_long.pdf}
  %\vspace*{-0.5cm}
  \caption{Accuracy evaluation of the anomaly detection methods. (a) Overall average entropy per category of measures. Analysis of the (b) averaged rank and (c) averaged rank entropy for each method and each accuracy measure over the entire benchmark. Example of (b.1) average rank and (c.1) entropy on the YAHOO dataset, KDD21 dataset (b.2, c.2). }
  \label{fig:entropy}
\end{figure*}

We now compute the accuracy measures for the nine different methods (we compute the anomaly scores ten different times, and we use the average accuracy). 
Figures~\ref{fig:entropy}(b.1) and (b.2) report the average ranking of the anomaly detection methods obtained on the YAHOO and KDD21 datasets, respectively. 
The x-axis corresponds to the different accuracy measures. We first observe that the rankings are more separated using Range-AUC and VUS measures for these two datasets. Figure~\ref{fig:entropy}(b) depicts the average ranking over the entire benchmark. The latter confirms the previous observation that VUS measures provide more separated rankings than threshold-based and AUC-based measures. We also observe an interesting ranking evolution for the YAHOO dataset illustrated in Figure~\ref{fig:entropy}(b.1). We notice that both LOF and MatrixProfile (brown and pink curve) have a low rank (between 4 and 5) using threshold and AUC-based measures. However, we observe that their ranks increase significantly for range-based and VUS-based measures (between 2.5 and 3). As we noticed by looking at specific examples (see Figure~\ref{exp:qual}), LOF and MatrixProfile can suffer from a lag issue even though the anomalies are well-identified. Therefore, the range-based and VUS-based measures better evaluate these two methods' detection capability.


Overall, the ranking curves show that the ranks appear more chaotic for threshold-based than AUC-, Range-AUC-, and VUS-based measures. 
In order to quantify this observation, we compute the Shannon Entropy of the ranks of each anomaly detection method. 
In practice, we extract the ranks of methods across one dataset and compute Shannon's Entropy of the different ranks. 
Figures~\ref{fig:entropy}(c.1) and (c.2) depict the entropy of each of the nine methods for the YAHOO and KDD21 datasets, respectively. 
Figure~\ref{fig:entropy}(c) illustrates the averaged entropy for all datasets in the benchmark for each measure and method, while Figure~\ref{fig:entropy}(a) shows the averaged entropy for each category of measures.
We observe that both for the general case (Figure~\ref{fig:entropy}(a) and Figure~\ref{fig:entropy}(c)) and some specific cases (Figures~\ref{fig:entropy}(c.1) and (c.2)), the entropy is reducing when using AUC-, Range-AUC-, and VUS-based measures. 
We report the lowest entropy for VUS-based measures. 
Moreover, we notice a significant drop between threshold-based and AUC-based. 
This confirms that the ranks provided by AUC- and VUS-based measures are consistent for data series belonging to one specific dataset. 


Therefore, based on the assumption formulated at the beginning of the section, we can thus conclude that AUC, range-AUC, and VUS-based measures are providing more consistent rankings. Finally, as illustrated in Figure~\ref{fig:entropy}, we also observe that VUS-based measures result in the most ordered and similar rankings for data series from the same type and domain.










\subsection{Execution Time Analysis}
\label{sec:exectime}

In this section, we evaluate the execution time required to compute different evaluation measures. 
In Section~\ref{sec:synthetic_eval_time}, we first measure the influence of different time series characteristics and VUS parameters on the execution time. In Section~\ref{sec:TSB_eval_time}, we  measure the execution time of VUS (VUS-ROC and VUS-PR simultaneously), R-AUC (R-AUC-ROC and R-AUC-PR simultaneously), and AUC-based measures (AUC-ROC and AUC-PR simultaneously) on the TSB-UAD benchmark. \commentRed{As demonstrated in the previous section, threshold-based measures are not robust, have a low separability power, and are inconsistent. 
Such measures are not suitable for evaluating anomaly detection methods. Thus, in this section, we do not consider threshold-based measures.}


\subsubsection{Evaluation on Synthetic Time Series}\hfill\\
\label{sec:synthetic_eval_time}

We first analyze the impact that time series characteristics and parameters have on the computation time of VUS-based measures. 
to that effect, we generate synthetic time series and labels, where we vary the following parameters: (i) the number of anomalies {\bf$\alpha$} in the time series, (ii) the average \textbf{$\mu(\ell_a)$} and standard deviation $\sigma(\ell_a)$ of the anomalies lengths in the time series (all the anomalies can have different lengths), (iii) the length of the time series \textbf{$|T|$}, (iv) the maximum buffer length \textbf{$L$}, and (v) the number of thresholds \textbf{$N$}.


We also measure the influence on the execution time of the R-AUC- and AUC- related parameter, that is, the number of thresholds ($N$).
The default values and the range of variation of these parameters are listed in Table~\ref{tab:parameter_range_time}. 
For VUS-based measures, we evaluate the execution time of the initial VUS implementation, as well as the two optimized versions, VUS$_{opt}$ and VUS$_{opt}^{mem}$.

\begin{table}[tb]
    \centering
    \caption{Value ranges for the parameters: number of anomalies ($\alpha$), average and standard deviation anomaly length ($\mu(\ell_a)$,$\sigma(\ell_a)$), time series length ($|T|$), maximum buffer length ($L$), and number of thresholds ($N$).}
    \begin{tabular}{|c|c|c|c|c|c|c|} 
 \hline
 Param. & $\alpha$ & $\mu(\ell_a)$ & $\sigma(\ell_{a})$ & $|T|$ & $L$ & $N$ \\ [0.5ex] 
 \hline\hline
 \textbf{Default} & 10 & 10 & 0 & $10^5$ & 5 & 250\\ 
 \hline
 Min. & 0 & 0 & 0 & $10^3$ & 0 & 2 \\
 \hline
 Max. & $2*10^3$ & $10^3$ & $10$ & $10^5$ & $10^3$ & $10^3$ \\ [1ex] 
 \hline
\end{tabular}
    \label{tab:parameter_range_time}
\end{table}


Figure~\ref{fig:sythetic_exp_time} depicts the execution time (averaged over ten runs) for each parameter listed in Table~\ref{tab:parameter_range_time}. 
Overall, we observe that the execution time of AUC-based and R-AUC-based measures is significantly smaller than VUS-based measures.
In the following paragraph, we analyze the influence of each parameter and compare the experimental execution time evaluation to the theoretical complexity reported in Table~\ref{tab:complexity_summary}.

\vspace{0.2cm}
\noindent {\bf [Influence of $\alpha$]}:
In Figure~\ref{fig:sythetic_exp_time}(a), we observe that the VUS, VUS$_{opt}$, and VUS$_{opt}^{mem}$ execution times are linearly increasing with $\alpha$. 
The increase in execution time for VUS, VUS$_{opt}$, and VUS$_{opt}^{mem}$ is more pronounced when we vary $\alpha$, in contrast to $l_a$ (which nevertheless, has a similar effect on the overall complexity). 
We also observe that the VUS$_{opt}^{mem}$ execution time grows slower than $VUS_{opt}$ when $\alpha$ increases. 
This is explained by the use of 2-dimensional arrays for the storage of predictions, which use contiguous memory locations that allow for faster access, decreasing the dependency on $\alpha$.

\vspace{0.2cm}
\noindent {\bf [Influence of $\mu(\ell_a)$]}:
As shown in Figure~\ref{fig:sythetic_exp_time}(b), the execution time variation of VUS, VUS$_{opt}$, and VUS$_{opt}^{mem}$ caused by $\ell_a$ is rather insignificant. 
We also observe that the VUS$_{opt}$ and VUS$_{opt}^{mem}$ execution times are significantly lower when compared to VUS. 
This is explained by the smaller dependency of the complexity of these algorithms on the time series length $|T|$. 
Overall, the execution time for both VUS$_{opt}$ and VUS$_{opt}^{mem}$ is significantly lower than VUS, and follows a similar trend. 

\vspace{0.2cm}
\noindent {\bf [Influence of $\sigma(\ell_a)$]}: 
As depicted in Figure~\ref{fig:sythetic_exp_time}(d) and inferred from the theoretical complexities in Table~\ref{tab:complexity_summary}, none of the measures are affected by the standard deviation of the anomaly lengths.

\vspace{0.2cm}
\noindent {\bf [Influence of $|T|$]}:
For short time series (small values of $|T|$), we note that O($T_1$) becomes comparable to O($T_2$). 
Thus, the theoretical complexities approximate to $O(NL(T_1+T_2))$, $O(N*(T_1+T_2))+O(NLT_2)$ and $O(N(T_1+T_2))$ for VUS, VUS$_{opt}$, and VUS$_{opt}^{mem}$, respectively. 
Indeed, we observe in Figure~\ref{fig:sythetic_exp_time}(c) that the execution times of VUS, VUS$_{opt}$, and VUS$_{opt}^{mem}$ are similar for small values of $|T|$. However, for larger values of $|T|$, $O(T_1)$ is much higher compared to $O(T_2)$, thus resulting in an effective complexity of $O(NLT_1)$ for VUS, and $O(NT_1)$ for VUS$_{opt}$, and VUS$_{opt}^{mem}$. 
This translates to a significant improvement in execution time complexity for VUS$_{opt}$ and VUS$_{opt}^{mem}$ compared to VUS, which is confirmed by the results in Figure~\ref{fig:sythetic_exp_time}(c).

\vspace{0.2cm}
\noindent {\bf [Influence of $N$]}: 
Given the theoretical complexity depicted in Table~\ref{tab:complexity_summary}, it is evident that the number of thresholds affects all measures in a linear fashion.
Figure~\ref{fig:sythetic_exp_time}(e) demonstrates this point: the results of varying $N$ show a linear dependency for VUS, VUS$_{opt}$, and VUS$_{opt}^{mem}$ (i.e., a logarithmic trend with a log scale on the y axis). \commentRed{Moreover, we observe that the AUC and range-AUC execution time is almost constant regardless of the number of thresholds used. The latter is explained by the very efficient implementation of AUC measures. Therefore, the linear dependency on the number of thresholds is not visible in Figure~\ref{fig:sythetic_exp_time}(e).}

\vspace{0.2cm}
\noindent {\bf [Influence of $L$]}: Figure~\ref{fig:sythetic_exp_time}(f) depicts the influence of the maximum buffer length $L$ on the execution time of all measures. 
We observe that, as $L$ grows, the execution time of VUS$_{opt}$ and VUS$_{opt}^{mem}$ increases slower than VUS. 
We also observe that VUS$_{opt}^{mem}$ is more scalable with $L$ when compared to VUS$_{opt}$. 
This is consistent with the theoretical complexity (cf. Table~\ref{tab:complexity_summary}), which indicates that the dependence on $L$ decreases from $O(NL(T_1+T_2+\ell_a \alpha))$ for VUS to $O(NL(T_2+\ell_a \alpha)$ and $O(NL(\ell_a \alpha))$ for $VUS_{opt}$, and $VUS_{opt}^{mem}$.





\begin{figure*}[tb]
  \centering
  \includegraphics[width=\linewidth]{figures/synthetic_res.pdf}
  %\vspace*{-0.5cm}
  \caption{Execution time of VUS, R-AUC, AUC-based measures when we vary the parameters listed in Table~\ref{tab:parameter_range_time}. The solid lines correspond to the average execution time over 10 runs. The colored envelopes are to the standard deviation.}
  \label{fig:sythetic_exp_time}
\end{figure*}


\vspace{0.2cm}
In order to obtain a more accurate picture of the influence of each of the above parameters, we fit the execution time (as affected by the parameter values) using linear regression; we can then use the regression slope coefficient of each parameter to evaluate the influence of that parameter. 
In practice, we fit each parameter individually, and report the regression slope coefficient, as well as the coefficient of determination $R^2$.
Table~\ref{tab:parameter_linear_coeff} reports the coefficients mentioned above for each parameter associated with VUS, VUS$_{opt}$, and VUS$_{opt}^{mem}$.



\begin{table}[tb]
    \centering
    \caption{Linear regression slope coefficients ($C.$) for VUS execution times, for each parameter independently. }
    \begin{tabular}{|c|c|c|c|c|c|c|} 
 \hline
 Measure & Param. & $\alpha$ & $l_a$ & $|T|$ & $L$ & $N$\\ [0.5ex] 
 \hline\hline
 \multirow{2}{*}{$VUS$} & $C.$ & 21.9 & 0.02 & 2.13 & 212 & 6.24\\\cline{2-7}
 & {$R^2$} & 0.99 & 0.15 & 0.99 & 0.99 & 0.99 \\   
 \hline
  \multirow{2}{*}{$VUS_{opt}$} & $C.$ & 24.2  & 0.06 & 0.19 & 27.8 & 1.23\\\cline{2-7}
  & $R^2$& 0.99 & 0.86 & 0.99 & 0.99 & 0.99\\ 
 \hline
 \multirow{2}{*}{$VUS_{opt}^{mem}$} & $C.$ & 21.5 & 0.05 & 0.21 & 15.7 & 1.16\\\cline{2-7}
  & $R^2$ & 0.99 & 0.89 & 0.99 & 0.99 & 0.99\\[1ex] 
 \hline
\end{tabular}
    \label{tab:parameter_linear_coeff}
\end{table}

Table~\ref{tab:parameter_linear_coeff} shows that the linear regression between $\alpha$ and the execution time has a $R^2=0.99$. Thus, the dependence of execution time on $\alpha$ is linear. We also observe that VUS$_{opt}$ execution time is more dependent on $\alpha$ than VUS and VUS$_{opt}^{mem}$ execution time.
Moreover, the dependence of the execution time on the time series length ($|T|$) is higher for VUS than for VUS$_{opt}$ and VUS$_{opt}^{mem}$. 
More importantly, VUS$_{opt}$ and VUS$_{opt}^{mem}$ are significantly less dependent than VUS on the number of thresholds and the maximal buffer length. 







\subsubsection{Evaluation on TSB-UAD Time Series}\hfill\\
\label{sec:TSB_eval_time}

In this section, we verify the conclusions outlined in the previous section with real-world time series from the TSB-UAD benchmark. 
In this setting, the parameters $\alpha$, $\ell_a$, and $|T|$ are calculated from the series in the benchmark and cannot be changed. Moreover, $L$ and $N$ are parameters for the computation of VUS, regardless of the time series (synthetic or real). Thus, we do not consider these two parameters in this section.

\begin{figure*}[tb]
  \centering
  \includegraphics[width=\linewidth]{figures/TSB2.pdf}
  \caption{Execution time of VUS, R-AUC, AUC-based measures on the TSB-UAD benchmark, versus $\alpha$, $\ell_a$, and $|T|$.}
  \label{fig:TSB}
\end{figure*}

Figure~\ref{fig:TSB} depicts the execution time of AUC, R-AUC, and VUS-based measures versus $\alpha$, $\mu(\ell_a)$, and $|T|$.
We first confirm with Figure~\ref{fig:TSB}(a) the linear relationship between $\alpha$ and the execution time for VUS, VUS$_{opt}$ and VUS$_{opt}^{mem}$.
On further inspection, it is possible to see two separate lines for almost all the measures. 
These lines can be attributed to the time series length $|T|$. 
The convergence of VUS and $VUS_{opt}$ when $\alpha$ grows shows the stronger dependence that $VUS_{opt}$ execution time has on $\alpha$, as already observed with the synthetic data (cf. Section~\ref{sec:synthetic_eval_time}). 

In Figure~\ref{fig:TSB}(b), we observe that the variation of the execution time with $\ell_a$ is limited when compared to the two other parameters. We conclude that the variation of $\ell_a$ is not a key factor in determining the execution time of the measures.
Furthermore, as depicted in Figure~\ref{fig:TSB}(c), $VUS_{opt}$ and $VUS_{opt}^{mem}$ are more scalable than VUS when $|T|$ increases. 
We also confirm the linear dependence of execution time on the time series length for all the accuracy measures, which is consistent with the experiments on the synthetic data. 
The two abrupt jumps visible in Figure~\ref{fig:TSB}(c) are explained by significant increases of $\alpha$ in time series of the same length. 

\begin{table}[tb]
\centering
\caption{Linear regression slope coefficients ($C.$) for VUS execution time, for all time series parameters all-together.}
\begin{tabular}{|c|ccc|c|} 
 \hline
Measure & $\alpha$ & $|T|$ & $l_a$ & $R^2$ \\ [0.5ex] 
 \hline\hline
 \multirow{1}{*}{${VUS}$} & 7.87 & 13.5 & -0.08 & 0.99  \\ 
 %\cline{2-5} & $R^2$ & \multicolumn{3}{c|}{ 0.99}\\
 \hline
 \multirow{1}{*}{$VUS_{opt}$} & 10.2 & 1.70 & 0.09 & 0.96 \\
 %\cline{2-5} & $R^2$ & \multicolumn{3}{c|}{0.96}\\
\hline
 \multirow{1}{*}{$VUS_{opt}^{mem}$} & 9.27 & 1.60 & 0.11 & 0.96 \\
 %\cline{2-5} & $R^2$ & \multicolumn{3}{c|}{0.96} \\
 \hline
\end{tabular}
\label{tab:parameter_linear_coeff_TSB}
\end{table}



We now perform a linear regression between the execution time of VUS, VUS$_{opt}$ and VUS$_{opt}^{mem}$, and $\alpha$, $\ell_a$ and $|T|$.
We report in Table~\ref{tab:parameter_linear_coeff_TSB} the slope coefficient for each parameter, as well as the $R^2$.  
The latter shows that the VUS$_{opt}$ and VUS$_{opt}^{mem}$ execution times are impacted by $\alpha$ at a larger degree than $\alpha$ affects VUS. 
On the other hand, the VUS$_{opt}$ and VUS$_{opt}^{mem}$ execution times are impacted to a significantly smaller degree by the time series length when compared to VUS. 
We also confirm that the anomaly length does not impact the execution time of VUS, VUS$_{opt}$, or VUS$_{opt}^{mem}$.
Finally, our experiments show that our optimized implementations VUS$_{opt}$ and VUS$_{opt}^{mem}$ significantly speedup the execution of the VUS measures (i.e., they can be computed within the same order of magnitude as R-AUC), rendering them practical in the real world.











\subsection{Summary of Results}


Figure~\ref{fig:overalltable} depicts the ranking of the accuracy measures for the different tests performed in this paper. The robustness test is divided into three sub-categories (i.e., lag, noise, and Normal vs. abnormal ratio). We also show the overall average ranking of all accuracy measures (most right column of Figure~\ref{fig:overalltable}).
Overall, we see that VUS-ROC is always the best, and VUS-PR and Range-AUC-based measures are, on average, second, third, and fourth. We thus conclude that VUS-ROC is the overall winner of our experimental analysis.

\commentRed{In addition, our experimental evaluation shows that the optimized version of VUS accelerates the computation by a factor of two. Nevertheless, VUS execution time is still significantly slower than AUC-based approaches. However, it is important to mention that the efficiency of accuracy measures is an orthogonal problem with anomaly detection. In real-time applications, we do not have ground truth labels, and we do not use any of those measures to evaluate accuracy. Measuring accuracy is an offline step to help the community assess methods and improve wrong practices. Thus, execution time should not be the main criterion for selecting an evaluation measure.}

%
% \section{Results}
\label{sec:Results}
\section{Results}
\label{sec:Results}

In this section, we present various analysis results that demonstrate the adoption of code obfuscation in Google Play.

\subsection{Overall Obfuscation Trends} 
\label{sec:obstrend}

\subsubsection{Presence of obfuscation} Out of the 548,967 Google Play Store APKs analyzed, we identified 308,782 obfuscated apps, representing approximately 56.25\% of the total. In Figure~\ref{fig:obfuscated_percentage}, we show the year-wise percentage of obfuscated apps for 2016-2023. There is an overall obfuscation increase of 13\% between 2016 and 2023, and as can be seen, the percentage of obfuscated apps has been increasing in the last few years, barring 2019 and 2020. As explained in Section~\ref{subsec:dataset}, 2019 and 2020 contain apps that are more likely to be abandoned by developers, and as such, they may not use advanced development practices.

\begin{figure}[h!]
\centering
    \includegraphics[width=\linewidth]{Figures/Only_obfuscation_trendV2.pdf}
    \caption{Percentage of obfuscated apps by year} \vspace{-4mm}
    \label{fig:obfuscated_percentage}
\end{figure}


From 2016 to 2018, the obfuscation levels were relatively stable at around 50-55\%, while from 2021 to 2023, there was a marked rise, reaching approximately 66\% in 2023. This indicates a growing focus on app protection measures among developers, likely driven by heightened security and IP concerns and the availability of advanced obfuscation tools.


\subsubsection{Obfuscation tools} Among the obfuscated APKs, our tool detector identified that 40.92\% of the apps use Proguard, 36.64\% use Allatori, 1.01\% use DashO, and 21.43\% use other (i.e., unknown) tools. We show the yearly trends in Figure~\ref{fig:ofbuscated_tool}. Note that we omit results in 2019 and 2020 ({\bf cf.} Section~\ref{subsec:dataset}).

ProGuard and Allatori are the most consistently used obfuscation tools, with ProGuard showing a slight overall increase in popularity and Allatori demonstrating variability. This inclination could be attributed to ProGuard being the default obfuscator integrated into Android Studio, a widely used development environment for Android applications. Notably, ProGuard usage increased by 13\% from 2018 to 2021, likely due to the introduction of R8 in April 2019~\cite{release_note_android}, which further simplified ProGuard integration with Android apps.

\begin{figure}[h]
\centering
    \includegraphics[width=\linewidth]{Figures/Initial_Tool_Trend_2019_dropV2.pdf} 
    \caption{Yearly obfuscation tool usage}
    \label{fig:ofbuscated_tool}
\end{figure}


DashO consistently remains low in usage, likely due to its high cost. The use of other obfuscation tools decreased until 2018 but has shown a resurgence from 2021 to 2023. This suggests that developers might be using other or custom tools, or our detector might be predicting some apps obfuscated with Proguard or Allatori as `other.' To investigate, we manually checked a sample of apps from the `other' category and confirmed they are indeed obfuscated. However, we could not determine which obfuscation tools the developers used. We discuss this potential limitation further in Section~\ref{sec:limitations}.


\subsubsection{Obfuscation techniques} We show the year-wise breakdown of obfuscation technique usage in Figure~\ref{fig:obfuscated_tech}. Among the various obfuscation techniques, Identifier Renaming emerged as the most prevalent, with 99.62\% of obfuscated apps using it alone or in combination with other methods (Categories of Only IR, IR and CF, IR and SE, or All three). Furthermore, 81.04\% of obfuscated apps used Control Flow Modification, and 62.76\% used String Encryption. The pervasive use of Identifier Renaming (IR) can be attributed to the fact that all obfuscation tools support it ({\bf cf.} Table~\ref{tab:ob_tool_cap}). Similarly, lower adoption of Control Flow Modification and String Encryption can be attributed to Proguard not supporting it. 

\begin{figure}[h]
\centering
    \includegraphics[width=\linewidth]{Figures/Initial_Tech_Trend_2019_dropV2.pdf} 
    \caption{Yearly obfuscation technique usage}
    \label{fig:obfuscated_tech}
\end{figure}



Next, we investigate the adoption of obfuscation on Google Play Store from various perspectives. Same as earlier, due to the smaller dataset size and possible bias ({\bf cf.} Section~\ref{subsec:dataset}), we exclude the APKs from 2019 and 2020 from this analyses.


\subsection{App Genre}
\label{sec:app_genre}

First, we investigate whether the obfuscation practices vary according to the App genre. Initially, we analysed all the APKs together before separating them into two snapshots.


\begin{figure*}[h]
    \centering
    \includegraphics[width=\linewidth]{Figures/AppGenreObfuscationV3.pdf}
    \caption{Obfuscated app percentage by genre (overall)}
    \label{fig:app_genre_overall}
\end{figure*}

Figure~\ref{fig:app_genre_overall} shows the genre-wise obfuscated app percentage. We note that 19 genres have more than 60\% of the apps obfuscated, and almost all the genres have more than 40\% obfuscation percentage. \textit{Casino} genre has the highest obfuscation percentage rate at 80\%, and overall, game genres tend to be more obfuscated than the other genres. The higher obfuscation usage in casino apps is logical due to their nature. These apps often simulate or involve gambling activities and handle monetary transactions and sensitive data related to in-game purchases, making them attractive targets for reverse engineering and hacking. This necessitates robust security measures to prevent fraud and protect user data. 


\begin{figure}[h]
    \centering
    \includegraphics[width=\linewidth]{Figures/AppGenre2018_2023ComparisonV3.pdf}
    \caption{Percentage of obfuscated apps by genre (2018-2023)}
    \label{fig:app_genre_comparison}
\end{figure}



\subsubsection{Genre-wise obfuscation trends in the two snapshots} To investigate the adoption of obfuscation over time, we study the two snapshots of Google Play separately, i.e., APKs from 2016-2018 as one group and APKs from 2021-2023 as another. 

Figure~\ref{fig:app_genre_comparison} illustrates the change in obfuscation levels by app genre between 2016-2018 to 2021-2023. Notably, app categories such as Education, Weather, and Parenting, which had obfuscation levels below the 2018 average, have increased to above the 2023 average by 2023. One possible reason for this in Education and Parenting apps can be the increase in online education activities during and after COVID-19 and the developers identifying the need for app hardening.

There are some genres, such as Casino and Action, for which the percentage of obfuscated apps didn't change across the two snapshots (i.e., purple and orange circles are close together in Figure~\ref{fig:app_genre_comparison}). This is because these genres are highly obfuscated from the beginning. Finally, the four genres, including Simulation and Role Playing, have a lower percentage of obfuscated apps in the 2021-2023 dataset. Our manual analysis didn't result in a conclusion as to why.


\begin{figure}[!h]
    \centering
    \includegraphics[width=\linewidth]{Figures/AppGenreTechAllV5.pdf}
    \caption{Obfuscation technique usage by genre (overall)}
    \label{fig:app_genre_all_tech}
\end{figure}


\subsubsection{Obfuscation techniques in different app genres} In Figure~\ref{fig:app_genre_all_tech}, we show the prevalence of key obfuscation techniques among various genres. As expected, almost all obfuscated apps in all genres used  Identifier Renaming. Also, it can be noted that genres with more obfuscated app percentages tend to use all three obfuscation techniques. Notably, more than 85\% of \textit{Casino} genre apps employ multiple obfuscation techniques

\subsubsection{Obfuscation tool usage in different app genres} We also investigated whether specific obfuscation tools are favoured by developers in different genres. However, apart from the expected observation that  ProGuard and Allatori being the most used tools, we didn't find any other interesting patterns. Therefore, we haven't included those measurement results.

\subsection{App Developers}
Next, we investigate individual developer-wise code obfuscation practices. From the pool of analyzed APKs, we identified the number of apps associated with each developer. Subsequently, we sorted the developers according to the number of apps they had created and selected the top 100 developers with the highest number of APKs for the 2016-2018 and 2021-2023 datasets. For the 2018 snapshot, we had 8,349 apps among the top 100 developers, while for the 2023 snapshot, we had 11,338 apps among the top 100 developers.

We then proceeded to detect whether or not these developers obfuscate their apps and, if so, what kind of tools and techniques they use. We present our results in five levels; developer obfuscating over 80\% of their apps, 60\%--80\% of apps, 40\%--60\% of apps, less than 40\%, and no obfuscation.

Figure~\ref{fig:developer_trend_my_apps_all} compares the two datasets in terms of developer obfuscation adoption. It shows that more developers have moved to obfuscate more than 80\% of their apps in the 2021-2023 dataset (76\%) compared to the 2016-2018 dataset (48\%).

We also found that among developers who obfuscate more than 80\% of their apps, 73\% in 2018 and 93\% in 2023 used the same obfuscation tool. Additionally, these top developers employ Control Flow Modification (CF) and String Encryption (SE) above the average values discussed in Section~\ref{sec:obstrend}. Specifically, in 2018, top developers used CF in 81.3\% of cases and SE in 66.7\%, while in 2023, these figures increased to 88.2\% and 78.9\%. This results in two insights: 1) Most top developers obfuscate all their apps with advanced techniques, possibly due to concerns about IP and security, and 2) Developers stick to a single tool, possibly due to specialized knowledge or because they bought a commercial licence.

\begin{figure}[]
    \centering
    \includegraphics[width=\linewidth]{Figures/Developer_Analysed_Comparison.pdf}
    \caption{Obfuscation usage (Top-100 developers)}
    \label{fig:developer_trend_my_apps_all}
\end{figure}


Finally, we investigate the obfuscation practices of developers with only one app in Table~\ref{tab:my-table}. According to the table, from those developers, 45.5\% of them obfuscated their apps in the 2016-2018 dataset and 57.2\% obfuscated their apps in the 2021-2023 dataset, showing a clear increase. However, these percentages are approximately 10\% lower than the average obfuscation rate in both cohorts discussed in Section~\ref{sec:obstrend}. This indicates that single-app developers may be less aware or concerned about code protection.


\begin{table}[]
\caption{Developers with only one app}
\label{tab:my-table}
\resizebox{\columnwidth}{!}{%
\begin{tabular}{cccccc}
\hline
\textbf{Year} & \textbf{\begin{tabular}[c]{@{}c@{}}Non\\ Obfuscated\end{tabular}} & \multicolumn{4}{c}{\textbf{Obfuscated}} \\ \hline
\multirow{3}{*}{\textbf{\begin{tabular}[c]{@{}c@{}}2018 \\ Snapshot\end{tabular}}} & \multirow{3}{*}{\begin{tabular}[c]{@{}c@{}}26,581 \\ (54.5\%)\end{tabular}} & \multicolumn{4}{c}{\begin{tabular}[c]{@{}c@{}}22,214 (45.5\%)\end{tabular}} \\ \cline{3-6} 
 &  & \textbf{ProGuard} & \textbf{Allatori} & \textbf{DashO} & \textbf{Other} \\ \cline{3-6} 
 &  & 6,131 & 8,050 & 658 & 7,375 \\ \hline
\multirow{3}{*}{\textbf{\begin{tabular}[c]{@{}c@{}}2023 \\ Snapshot\end{tabular}}} & \multirow{3}{*}{\begin{tabular}[c]{@{}c@{}}19,510 \\ (42.8\%)\end{tabular}} & \multicolumn{4}{c}{\begin{tabular}[c]{@{}c@{}}26,084 (57.2\%)\end{tabular}} \\ \cline{3-6} 
 &  & \textbf{ProGuard} & \textbf{Allatori} & \textbf{DashO} & \textbf{Other} \\ \cline{3-6} 
 &  & 12,697 & 9,672 & 234 & 3,581 \\ \hline
\end{tabular}%
}
\end{table}

\subsection{Top-k Apps}

Next, we investigate the obfuscation practices of top apps in Google Play Store. First, we rank the apps using the same criterion used by our previous work~\cite{rajasegaran2019multi, karunanayake2020multi, seneviratne2015early}. That is, we sort the apps in descending order of number of downloads, average rating, and rating count, with the intuition that top apps have high download numbers and high ratings, even when reviewed by a large number of users. Then, we investigated the percentage of obfuscated apps and obfuscation tools and technique usage as summarized in Table~\ref{tab:top_k_apps_2018_2023}.

When considering the highly ranked applications (i.e., top-1,000), the obfuscation percentage is notably higher, at around 93\%, in both datasets, which is significantly higher than the average percentage of obfuscation we observed in Section~\ref{sec:obstrend}. Top-ranked apps, likely due to their higher visibility and potential revenue, invest more in obfuscation to safeguard their intellectual property and enhance security. 

The obfuscation percentage decreases when going from the top 1,000 apps to the top 30,000 apps. Nonetheless, the obfuscation percentage in both datasets remains around similar values until the top 30,000 (e.g., $\sim$74\% for top-30,000). This indicates that the major increase in obfuscation in the 2021-2023 dataset comes from apps beyond the top 30,000.

When observing the tools used, the usage of ProGuard increases as we move from top to lower-ranked apps in both datasets. This may be because ProGuard is free and the default in Android Studio, while commercial tools like Allatori and DashO are expensive. There is a notable increase in the use of Allatori among the top apps in the 2021-2023 dataset. Regarding obfuscation techniques, the top 1,000 apps utilize all three techniques more frequently than lower-ranked apps in both snapshots. This indicates that the top 1,000 apps are more heavily protected compared to lower-ranked ones.

\begin{table*}[]
\caption{Summary of analysis results for Top-k apps in 2018 and 2023}
\label{tab:top_k_apps_2018_2023}
\resizebox{\textwidth}{!}{%
\begin{tabular}{lccccccccc}
\hline
\multicolumn{1}{c}{\begin{tabular}[c]{@{}c@{}}Top k apps - \\ Year\end{tabular}} & \begin{tabular}[c]{@{}c@{}}Total \\ Apps\end{tabular} & \begin{tabular}[c]{@{}c@{}}Obfuscation\\ Percentage\end{tabular} & \begin{tabular}[c]{@{}c@{}}ProGuard\\ Percentage\end{tabular} & \begin{tabular}[c]{@{}c@{}}Allatori\\ Percentage\end{tabular} & \begin{tabular}[c]{@{}c@{}}DashO\\ Percentage\end{tabular} & \begin{tabular}[c]{@{}c@{}}Other\\ Percentage\end{tabular} & \begin{tabular}[c]{@{}c@{}}IR\\ Percentage\end{tabular} & \begin{tabular}[c]{@{}c@{}}CF\\ Percentage\end{tabular} & \begin{tabular}[c]{@{}c@{}}SE\\ Percentage\end{tabular} \\ \hline
1k (2018) & 1,000 & 93.40 & 29.98 & 28.48 & 0.64 & 40.90 & 99.90 & 88.76 & 65.42 \\
10k (2018) & 10,000 & 85.19 & 25.55 & 35.32 & 0.47 & 38.65 & 99.90 & 88.76 & 71.91 \\
20k (2018) & 20,000 & 78.42 & 26.31 & 36.76 & 0.57 & 36.36 & 99.87 & 87.37 & 71.49 \\
30k (2018) & 30,000 & 74.40 & 27.30 & 37.71 & 0.64 & 34.36 & 99.82 & 86.75 & 71.11 \\
30k+ (2018) & 314,568 & 53.36 & 36.72 & 34.70 & 1.33 & 27.24 & 99.34 & 83.54 & 63.11 \\ \hline
1k (2023) & 1,000 & 92.50 & 24.00 & 51.89 & 1.95 & 22.16 & 100.0 & 92.54 & 83.68 \\
10k (2023) & 10,000 & 81.88 & 26.03 & 56.20 & 1.03 & 16.74 & 99.89 & 89.40 & 82.01 \\
20k (2023) & 20,000 & 76.62 & 30.48 & 52.92 & 0.96 & 15.64 & 99.93 & 85.80 & 78.01 \\
30k (2023) & 30,000 & 73.72 & 33.87 & 50.34 & 0.89 & 14.90 & 99.95 & 83.31 & 75.34 \\
30k+ (2023) & 206,216 & 61.90 & 46.56 & 38.21 & 0.64 & 14.59 & 99.97 & 77.51 & 62.50 \\ \hline
\end{tabular}%
}
\end{table*}



\section{Conclusion}
\label{sec:Conclusion}
We present RiskHarvester, a risk-based tool to compute a security risk score based on the value of the asset and ease of attack on a database. We calculated the value of asset by identifying the sensitive data categories present in a database from the database keywords. We utilized data flow analysis, SQL, and Object Relational Mapper (ORM) parsing to identify the database keywords. To calculate the ease of attack, we utilized passive network analysis to retrieve the database host information. To evaluate RiskHarvester, we curated RiskBench, a benchmark of 1,791 database secret-asset pairs with sensitive data categories and host information manually retrieved from 188 GitHub repositories. RiskHarvester demonstrates precision of (95\%) and recall (90\%) in detecting database keywords for the value of asset and precision of (96\%) and recall (94\%) in detecting valid hosts for ease of attack. Finally, we conducted an online survey to understand whether developers prioritize secret removal based on security risk score. We found that 86\% of the developers prioritized the secrets for removal with descending security risk scores.

\section{Acknowledgments}
\label{sec:Ack}
%%%%%%%%%%%%%%%%%%%%%%%%%%%%%%%%%%%%%%%%%%%%%%%%%%%%%%%%%%%%%%%%%%%%%%%%
%%%%%%%%%%%%%%%%%%%%%%%%%%%% ACKNOWLEDGMENT %%%%%%%%%%%%%%%%%%%%%%%%%%%%
%%%%%%%%%%%%%%%%%%%%%%%%%%%%%%%%%%%%%%%%%%%%%%%%%%%%%%%%%%%%%%%%%%%%%%%%

\section*{Acknowledgment}

This research was supported by Coordena\c{c}\~{a}o de Aperfei\c{c}oamento de Pessoal de N\'{i}vel Superior - Brasil (CAPES) - Finance Code 001, FAPERGS 09/2023 PqG (N\textsuperscript{o} 24/2551-0001400-4), and CNPq Research Program (N\textsuperscript{o}306511/2021-5).

%%%%%%%%%%%%%%%%%%%%%%%%%%%%%%%%%%%%%%%%%%%%%%%%%%%%%%%%%%%%%%%%%%%%%%%%

% %%%%%%%%%%%%%%%%%%%%%%%%%%%%%%%%%%%%%%%%%%%%%%%%%%%%%%%%%%%%
\bibliography{refs}
\bibliographystyle{tmlr}


\end{document}
