
\documentclass[10pt]{article} % For LaTeX2e
\usepackage[preprint]{tmlr}
% If accepted, instead use the following line for the camera-ready submission:
%\usepackage[accepted]{tmlr}
% To de-anonymize and remove mentions to TMLR (for example for posting to preprint servers), instead use the following:
%\usepackage[preprint]{tmlr}

% Optional math commands from https://github.com/goodfeli/dlbook_notation.
%%%%% NEW MATH DEFINITIONS %%%%%

% \usepackage{amsmath,amsfonts,bm}
\usepackage{amsmath,amsfonts}

\usepackage{pifont}


\newcommand{\R}{\mathbb{R}}


\def\va{{\mathbf{a}}}
\def\vg{{\mathbf{g}}}

% Sets
\def\sR{\mathbb{R}}
\def\sC{\mathbb{C}}
\def\sZ{\mathbb{Z}}
\def\sN{\mathbb{N}}
\def\sQ{\mathbb{Q}}

\def\sS{\mathcal{S}}



% Vectors
\def\vzero{{\mathbf{0}}}
\def\vone{{\mathbf{1}}}
\def\vmu{{\mathbf{\mu}}}
\def\vtheta{{\mathbf{\theta}}}
\def\va{{\mathbf{a}}}
\def\vb{{\mathbf{b}}}
\def\vc{{\mathbf{c}}}
\def\vd{{\mathbf{d}}}
\def\ve{{\mathbf{e}}}
\def\vf{{\mathbf{f}}}
\def\vg{{\mathbf{g}}}
\def\vh{{\mathbf{h}}}
\def\vi{{\mathbf{i}}}
\def\vj{{\mathbf{j}}}
\def\vk{{\mathbf{k}}}
\def\vl{{\mathbf{l}}}
\def\vm{{\mathbf{m}}}
\def\vn{{\mathbf{n}}}
\def\vo{{\mathbf{o}}}
\def\vp{{\mathbf{p}}}
\def\vq{{\mathbf{q}}}
\def\vr{{\mathbf{r}}}
\def\vs{{\mathbf{s}}}
\def\vt{{\mathbf{t}}}
\def\vu{{\mathbf{u}}}
\def\vv{{\mathbf{v}}}
\def\vw{{\mathbf{w}}}
\def\vx{{\mathbf{x}}}
\def\vy{{\mathbf{y}}}
\def\vz{{\mathbf{z}}}
\def\vzeta{{\mathbf{\zeta}}}

% Matrix
\def\mA{{\mathbf{A}}}
\def\mB{{\mathbf{B}}}
\def\mC{{\mathbf{C}}}
\def\mD{{\mathbf{D}}}
\def\mE{{\mathbf{E}}}
\def\mF{{\mathbf{F}}}
\def\mG{{\mathbf{G}}}
\def\mH{{\mathbf{H}}}
\def\mI{{\mathbf{I}}}
\def\mJ{{\mathbf{J}}}
\def\mK{{\mathbf{K}}}
\def\mL{{\mathbf{L}}}
\def\mM{{\mathbf{M}}}
\def\mN{{\mathbf{N}}}
\def\mO{{\mathbf{O}}}
\def\mP{{\mathbf{P}}}
\def\mQ{{\mathbf{Q}}}
\def\mR{{\mathbf{R}}}
\def\mS{{\mathbf{S}}}
\def\mT{{\mathbf{T}}}
\def\mU{{\mathbf{U}}}
\def\mV{{\mathbf{V}}}
\def\mW{{\mathbf{W}}}
\def\mX{{\mathbf{X}}}
\def\mY{{\mathbf{Y}}}
\def\mZ{{\mathbf{Z}}}
\def\mBeta{{\mathbf{\beta}}}
\def\mPhi{{\mathbf{\Phi}}}
\def\mLambda{{\mathbf{\Lambda}}}
\def\mSigma{{\mathbf{\Sigma}}}


% Expectation
% \def\eE{\mathop{\mathbb{E}}\limits}
\def\eE{\mathbb{E}}

% Probability
\def\pP{\mathbb{P}}

% Tilde
\def\tf{\tilde{f}}
\def\tS{\tilde{S}}
\def\wtF{\widetilde{\mathcal{F}}}
\def\whR{\widehat{R}}
\def\tvx{\tilde{\mathbf{x}}}
\def\ty{\tilde{y}}


\def\defeq{\overset{\textup{def}}{=}}
% \def\defeq{\overset{.}{=}}
\def\defone{\overset{\text{\ding{172}}}{=}}
\def\deftwo{\overset{\text{\ding{173}}}{=}}
\def\leqone{\overset{\text{\ding{172}}}{\leq}}
\def\leqtwo{\overset{\text{\ding{173}}}{\leq}}
\def\leqthree{\overset{\text{\ding{174}}}{\leq}}
\def\leqfour{\overset{\text{\ding{175}}}{\leq}}
\def\eqone{\overset{\text{\ding{172}}}{=}}
\def\eqtwo{\overset{\text{\ding{173}}}{=}}
\def\eqthree{\overset{\text{\ding{174}}}{=}}
\def\eqfour{\overset{\text{\ding{175}}}{=}}
\def\geqfive{\overset{\text{\ding{176}}}{\geq}}

\usepackage{amsmath}
\usepackage{amssymb}
\usepackage{algorithm}

\let\classAND\AND
\let\AND\relax

\usepackage{algorithmic}

\let\algoAND\AND
\let\AND\classAND


\AtBeginEnvironment{algorithmic}{\let\AND\algoAND}


\newtheorem{theorem}{Theorem}[section]
\newtheorem{lemma}[theorem]{Lemma}
\newtheorem{corollary}[theorem]{Corollary}
\newtheorem{proposition}[theorem]{Proposition}
\usepackage{adjustbox}
\usepackage{subfig}

% \theoremstyle{definition}
\newtheorem{definition}[theorem]{Definition}
\newtheorem{example}[theorem]{Example}
\usepackage{hyperref}
\usepackage{url}
\usepackage{cleveref}




\title{Symmetric Rank-One Quasi-Newton Methods for Deep Learning Using Cubic Regularization}

% Authors must not appear in the submitted version. They should be hidden
% as long as the tmlr package is used without the [accepted] or [preprint] options.
% Non-anonymous submissions will be rejected without review.

\author{%
  Aditya Ranganath \\
  Center for Applied Scientific Computing\\
  Lawrence Livermore National Laboratory\\
  7000 East Avenue, 
  Livermore, CA 94550 \\
  \texttt{ranganath2@llnl.gov} 
  \AND
  Mukesh Singhal \\
  Electrical Engineering and Computer Science \\
  University of California, Merced\\
  5200 N Lake Road\\
  Merced, CA 95343 \\
  \texttt{msinghal@ucmerced.edu}
  \AND
  Roummel Marcia \\
  Applied Mathematics \\
  University of California, Merced\\
  5200 N Lake Road \\
  Merced, CA 95343 \\
  \texttt{rmarcia@ucmerced.edu}
}

% The \author macro works with any number of authors. Use \AND 
% to separate the names and addresses of multiple authors.

\newcommand{\fix}{\marginpar{FIX}}
\newcommand{\new}{\marginpar{NEW}}

\def\month{MM}  % Insert correct month for camera-ready version
\def\year{YYYY} % Insert correct year for camera-ready version
\def\openreview{\url{https://openreview.net/forum?id=XXXX}} % Insert correct link to OpenReview for camera-ready version


\begin{document}


\maketitle



\begin{abstract}
Stochastic gradient descent and other first-order variants, such as Adam and AdaGrad, are commonly used in the field of deep learning due to their computational efficiency and low-storage memory requirements. However, these methods do not exploit curvature information. Consequently, iterates can converge to saddle points or poor local minima. On the other hand, Quasi-Newton methods compute Hessian approximations which exploit this information with a comparable computational budget. Quasi-Newton methods re-use previously computed iterates and gradients to compute a low-rank structured update. The most widely used quasi-Newton update is the L-BFGS, which guarantees a positive semi-definite Hessian approximation, making it suitable in a line search setting. However, the loss functions in DNNs are non-convex, where the Hessian is potentially non-positive definite. In this paper, we propose using a limited-memory symmetric rank-one quasi-Newton approach which allows for indefinite Hessian approximations, enabling directions of negative curvature to be exploited. Furthermore, we use a modified adaptive regularized cubics approach, which generates a sequence of cubic subproblems that have closed-form solutions with suitable  regularization choices. We investigate the performance of our proposed method on autoencoders and feed-forward neural network models and compare our approach to state-of-the-art first-order adaptive stochastic methods as well as other quasi-Newton methods.
\end{abstract}

\section{Introduction}
\label{sec:intro}
\section{Introduction}


Sequential resource allocation is a fundamental problem in many domains, including healthcare, finance, and public policy \cite{considine2023optimizing,boehmer2024optimizing, yu2024fincon}. This task involves allocating limited resources over time while accounting for dynamic changes and competing demands. Deep reinforcement learning (RL) is an effective method to optimize decision-making for such challenges, offering efficient and scalable policies~\cite{yu2021reinforcement,talaat2022effective, xiong2023reinforcement,zhao2024towards}. However, deep RL policies generally provide action recommendations without human-readable reasoning and explanations. Such lack of interpretability poses a major challenge in critical domains where decisions must be transparent, justifiable, and in line with human decision-makers to ensure trust and compliance with ethical and regulatory standards.



For example, doctors may need to decide whether to prioritize intervention for Patient A or Patient B based on their current vital signs~\cite{boehmer2024optimizing}. An RL algorithm might suggest: \textit{ ``Intervene with Patient A "} with the implicit goal of maximizing the value function. However, the underlying reasoning may not be clear to the doctors, leaving them uncertain about the factors influencing the decision \cite{milani2024explainable}. For doctors, a more effective suggestion could be risk-based with specific information, e.g., \textit{``Patient A's vital signs are likely to deteriorate leading to higher potential risk compared to Patient B, so intervention with Patient A is prioritized"} \cite{gebrael2023enhancing, boatin2021wireless}.




\begin{figure*}[tbp]
    \centering
    \includegraphics[width=0.99\linewidth]{Figures/icml25_ProposedFramework.pdf}
    \caption{Overview of the \rbrl framework for joint sequential decision-making and explanation generation at time instance $t$. Starting with current state $\bs_t$,  a state-to-language descriptor generates \lang($\bs_t$), which is used to create the input prompt 
$\bp_t$. The LLM processes 
$\bp_t$
  to produce a thought 
$\pmb{\tau}_t$  and a set of candidate rules 
$\cR_t$ . An attention-based policy network selects a rule 
$\arule_t$ , which is used to derive an executable action $\aenv_t$ satisfying the budget constraint $B(\bs_t)$ for the environment 
  and a human-readable explanation $\pmb{\ell}_t^{expl}$, while also providing a rule reward $r_t^{\text{rule}}$ 
 . The environment transitions to the next state 
$\bs_{t+1}$ , returning an environment reward $r_t^{\text{env}}$ 
 . This process is repeated iteratively at subsequent time steps. 
}
    \label{fig:Proposed_framework}
\end{figure*}


Language agents \cite{sumers2024cognitive} leverage large language models (LLMs) for multi-step decision-making using reasoning techniques like chain of thought (CoT) \cite{wei2022chain} and ReAct \cite{yao2023react}. They enable natural language goal specification \cite{du2023guiding} and enhance human understanding \cite{hu2023language, srivastava2024policy}. However, LLMs struggle with complex sequential decision-making, such as resource allocation \cite{furuta2024exposing}, making RL a crucial tool for refining them into effective policy networks \cite{carta2023grounding, tantrue, wen2024reinforcing, zhai2024fine}. Yet, fine-tuning LLMs for policy learning is highly challenging due to the substantial computational costs and the complexity of token-level optimization \cite{rashid2024critical}, which remains an open challenge, particularly in sequential resource allocation.

Consequently, aiming to combine the strengths of both deep RL and language agents, we pose the following question:


\vspace{-0.1in}
\begin{tcolorbox}[colback=white!5!white,colframe=white!75!white]
\textit{%
Can we design a language agent framework that can simultaneously perform sequential resource allocation and provide human-readable explanations? }
\end{tcolorbox}
\vspace{-0.15in}






Motivated by existing work that employs predefined rules or concepts to explain RL policies \cite{Das2023State2Explanation} or guide RL exploration \cite{likmeta2020combining}, we explore the potential of using rules to prioritize individuals in resource allocation problems. In the context of language agents, rules are defined as ``structured statements" that capture prioritization among choices in a given state, aligning with the agent's goals \cite{srivastava2024policy}. 
Rules offer a flexible framework for encoding high-level decision criteria and priority logic, similar as the celebrated index policy for prioritizing arms in resource allocation problems \cite{whittle1988restless}, making them ideal for guiding resource allocation strategies while explaining the rationale behind decisions.%



Building on this, we propose a novel framework called Rule-Bottleneck Reinforcement Learning (\rbrl), which integrates the strengths of LLMs and RL to bridge the gap between decision-making and interoperability, by optimizing LLM-generated rules with RL. 
\rbrl provides a framework (as shown in Figure \ref{fig:Proposed_framework}) that simultaneously makes sequential resource allocation decisions and provides human-readable explanations. \rbrl leverages LLMs to generate candidate rules and employs RL to optimize policy selection, enabling the creation of effective decision policies while simultaneously providing human-understandable explanations. 

Our contributions are summarized as follows. \textit{First}, to avoid the computational cost and complexity of directly fine-tuning language agents, we leverage LLMs to generate a diverse set of rules, where each rule serves as a prioritization strategy for individuals in resource allocation. This approach enhances flexibility and interpretability in decision-making.
\textit{Second}, we extend the conventional state-action space by integrating the thoughts and rules generated by LLMs, creating a novel framework that enables reinforcement learning to operate on a richer, more interpretable decision structure.
\textit{Third}, we introduce an attention-based training framework that maps states to queries and rules to keys. The rule selection process is optimized by a policy network trained using the Soft Actor-Critic (SAC) algorithm \cite{haarnoja2018soft}, ensuring robust and efficient decision-making. In particular, the LLM also acts as a feedback mechanism, providing guidance during RL exploration to improve policy optimization and promote more effective learning. 
 



We evaluate our method in three environments from two real-world domains: \texttt{HeatAlerts}, where resources are allocated to mitigate extreme heat events; and \texttt{WearableDeviceAssignment}, for distributing monitoring devices to patients. 
Using cost-effective LLMs such as gpt-4o-mini \cite{openai2024gpt4omini} and Llama 3.1 8B \cite{meta2024llama3.1}, we first assess decision performance by comparing \rbrl with pure RL methods and language agent baselines. We then evaluate explanation quality through a human survey conducted under IRB approval. The results demonstrate \rbrl's effectiveness in both decision quality and interpretability.















\noindent\section{Proposed approach
%Adaptive Regularization using Cubics with L-SR1 Updates
}
\label{sec:ProposedApproach}
In this section, we describe our proposed approach by first discussing the L-SR1 update.

\noindent \textbf{Limited-memory symmetric rank-one updates.} 
Unlike the BFGS update (\ref{eqn:LBFGS}), which is a rank-two update, the SR1 update is a rank-one update, which is  given by
\begin{equation}\label{eq:SR1}
	\mathbf{B}_{k+1} = \mathbf{B}_{k} + \frac{(\mathbf{y}_k - \mathbf{B}_k\mathbf{s}_k)
	(\mathbf{y}_k - \mathbf{B}_k\mathbf{s}_k)^{\top}}{\mathbf{s}_k^{\top}(\mathbf{y}_k - \mathbf{B}_k\mathbf{s}_k)}
\end{equation}
(see \citet{KhaBS93}).  As previously mentioned, $\mathbf{B}_{k+1}$ in (\ref{eq:SR1}) is not guaranteed to be definite.  However,
it can be shown that the SR1 matrices can converge to the true Hessian (see \citet{Conn1991} for details).
We note that the pair $(\mathbf{s}_k, \mathbf{y}_k)$ is accepted only when 
\begin{equation}\label{eq:acceptance1}
|\mathbf{s}_k^{\top}(\mathbf{y}_k - \mathbf{B}_k\mathbf{s}_k)| > \varepsilon \| \mathbf{s}_k \|_2 \| \mathbf{y}_k - \mathbf{B}_k \mathbf{s}_k \|_2,
\end{equation}
for some constant $\varepsilon > 0$ (see \citet{NoceWrig06}, Sec.\ 6.2, for details). The SR1 update can be defined recursively as
\begin{equation}\label{eq:SR1_B0}
	\mathbf{B}_{k+1} = \mathbf{B}_{0} + 
	\sum_{j = 0}^k \frac{(\mathbf{y}_j - \mathbf{B}_j\mathbf{s}_j)
	(\mathbf{y}_j - \mathbf{B}_j\mathbf{s}_j)^{\top}}{\mathbf{s}_j^{\top}(\mathbf{y}_j - \mathbf{B}_j\mathbf{s}_j)}.
\end{equation}
In limited-memory settings, only the last $m \ll n$ pairs of $(\mathbf{s}_j, \mathbf{y}_j)$ 
are stored and used.  
For ease of presentation, here we choose $k < m$.  We define
$$\mathbf{S}_{k} = [ \ \mathbf{s}_0 \ \  \mathbf{s}_1 \ \ \cdots  \ \ \mathbf{s}_{k-1} \ ] \quad \text{and} \quad 
\mathbf{Y}_{k} = [ \ \mathbf{y}_0 \ \ \mathbf{y}_1 \ \ \cdots \  \ \mathbf{y}_{k-1} \ ].
$$
Then 
$\mathbf{B}_{k}$ admits a compact representation of the form
\begin{equation}\label{eqn:compactSR1}
	\mathbf{B}_{k} \ = \ \mathbf{B}_0 + 
	\begin{bmatrix}
	\\
	\mathbf{\Psi}_{k}  \\
	\phantom{t}
	\end{bmatrix}
	\hspace{-.3cm}
	\begin{array}{c}
	\left  [ \  \mathbf{M}_{k}^{\phantom{h}}  \right ] \\
	\\
	\\
	\end{array}
	\hspace{-.3cm}
	\begin{array}{c}
	\left [  \ \quad \mathbf{\Psi}_{k}^{\top} \quad \ \right ] \\
	\\
	\\
	\end{array},
\end{equation}
where  $\mathbf{\Psi}_{k} = \mathbf{Y}_{k} -  \mathbf{B}_0 \mathbf{S}_{k}$ and 
%\begin{align}\label{eq:PsiM}
%	\nonumber
%	\mathbf{\Psi}_{k+1} &= \mathbf{Y}_{k+1}\!  -\! \mathbf{B}_0 \mathbf{S}_{k+1}  \ \\\nonumber \text{and}\\ 
$$
        \mathbf{M}_{k} = (\mathbf{D}_{k} \!+\! \mathbf{L}_{k} \!+\! \mathbf{L}_{k}^{\top} \!-\! \mathbf{S}_{k}^{\top}\!\mathbf{B}_0\mathbf{S}_{k})^{-1}\!,
$$
%\end{align}
where $\mathbf{L}_{k}$ is the strictly lower triangular part, $\mathbf{V}_{k}$ is the strictly
upper triangular part, and $\mathbf{D}_{k}$ is the diagonal part of 
$
	\mathbf{S}_{k}^{\top}\mathbf{Y}_{k} =   \mathbf{L}_{k} + \mathbf{D}_{k} + \mathbf{V}_{k}
$
(see \citet{ByrNS94} for further details).  


Because of the compact representation of $\mathbf{B}_{k}$, 
its partial eigendecomposition can be computed (see  \citet{ErwM15}).  
In particular, if we compute the QR decomposition of $\mathbf{\Psi}_{k} = \mathbf{QR}$
and the eigendecomposition $\mathbf{RMR}^\top= \mathbf{P} \hat{\mathbf{\Lambda}}_{k} \mathbf{P}^\top$,
then we can write 
$$
\mathbf{B}_{k} = \mathbf{B}_0 + \mathbf{U}_{\parallel} \hat{\mathbf{\Lambda}}_{k} 
\mathbf{U}_{\parallel}^{\top},
$$
where $\mathbf{U}_{\parallel}  = \mathbf{QP} \in \mathbb{R}^{n \times k}$ has
orthonormal columns and $\hat{\mathbf{\Lambda}}_{k} \in \mathbb{R}^{k \times k}$ 
is a  diagonal matrix.  
If $\mathbf{B}_0 =  \delta_k \mathbf{I}$ (see e.g., Lemma 2.4 in  \citet{Erway2020TrustregionAF}), 
where $0 < \delta_k < \delta_{\max}$ is some scalar and $\mathbf{I}$ is the identity matrix, 
then we obtain the eigendecomposition 
\begin{equation}
\mathbf{B}_{k} = \mathbf{U}_{k}\mathbf{\Lambda}_{k}\mathbf{U}_{k}^{\top}
=
\bigg [ \ 
\mathbf{U}_{\parallel}  \ \ \ \mathbf{U}_{\perp}
\bigg ]
\begin{bmatrix}
\hat{\mathbf{\Lambda}}_{k} + \delta_k \mathbf{I} & 0 \\
0 & \delta_k \mathbf{I} 
\end{bmatrix}
\begin{bmatrix}
\ \mathbf{U}_{\parallel}^{\top} \ 
\\[.2cm]
\mathbf{U}_{\perp}^{\top}
\end{bmatrix}
\end{equation}
where $\mathbf{U}_{k} = [  \ \mathbf{U}_{\parallel}  \ \ \mathbf{U}_{\perp} \ ]$ is an orthogonal 
matrix and
$\mathbf{U}_{\perp} \in \mathbb{R}^{n \times (n-k)}$ is a matrix
whose columns form an orthonormal basis orthogonal to the range space of $\mathbf{U}_{\parallel}$.
% and $\mathbf{U}_{k+1}^{\top} \mathbf{U}_{k+1}^{\phantom{\top}} = \mathbf{I}$.  
Here, 
\begin{equation}
	(\mathbf{\Lambda}_{k})_i =
	\begin{cases}  
	\delta_k + \hat{\lambda}_i & \text{ if $i \le k$} \\
	\delta_k & \text{ if $i > k$}
	\end{cases}.
\end{equation}
%$(\mathbf{\Lambda}_{k+1})_i = \delta_k + \hat{\lambda}_i$ for $i \le k+1$, where
%$\hat{\lambda}_i$ is the $i$th diagonal in $\hat{\mathbf{\Lambda}}_{k+1}$,
%and $(\mathbf{\Lambda}_{k+1})_i = \delta_k $ for $i > k+1$.  


%In particular, if $\mathbf{\Phi}_{k+1} = \mathbf{Q}_{k+1} \mathbf{R}_{k+1}$ is the
%QR decomposition of $\mathbf{\Phi}_{k+1}$ and $\mathbf{R_{k+1}M_{k+1}R_{k+1}^{\top} =
%\mathbf{P}_{k+1}\mathbf{\Lambda}_{k+1}\mathbf{P}_{k+1}^{\top}}$ is the eigendecomposition of the product 
%$\mathbf{R}_{k+1}\mathbf{M}_{k+1}\mathbf{R}_{k+1}^{\top}$, then
%$$
%	\mathbf{B}_{k+1} = 
%	\mathbf{B}_0 + \mathbf{Q}_{k+1}\mathbf{P}_{k+1}
%	\mathbf{\Lambda}_{k+1}\mathbf{P}_{k+1}^{\top}\mathbf{Q}_{k+1}^{\top}.
%$$

%\begin{figure}[t]
%	\centering
%		\begin{tabular}{cc}
%		\includegraphics[width=3.175cm, trim={2cm 1cm 2cm 0},clip]{Figures/cubic1.pdf} &
%		\includegraphics[width=3.175cm, trim={2cm 1cm 2cm 0},clip]{Figures/cubic2.pdf}  \\
%		(a) $\lambda > 0$ and $g > 0$ & (b) $\lambda > 0$ and $g < 0$ \\
%		\includegraphics[width=3.175cm, trim={2cm 1cm 2cm 0},clip]{Figures/cubic3.pdf} &
%		\includegraphics[width=3.175cm, trim={2cm 1cm 2cm 0},clip]{Figures/cubic4.pdf} \\
%		(c) $\lambda < 0$ and $g > 0$ & (d)  $\lambda < 0$ and $g < 0$
%		\end{tabular}
%%	\begin{tabular}{cccc}
%%		\includegraphics[width=3.175cm, trim={2cm 1cm 2cm 0},clip]{Figures/cubic1.pdf} &
%%		\includegraphics[width=3.175cm, trim={2cm 1cm 2cm 0},clip]{Figures/cubic2.pdf} &
%%		\includegraphics[width=3.175cm, trim={2cm 1cm 2cm 0},clip]{Figures/cubic3.pdf} &
%%		\includegraphics[width=3.175cm, trim={2cm 1cm 2cm 0},clip]{Figures/cubic4.pdf} 
%%		\\
%%		\footnotesize (a) $\lambda > 0$ and $\bar{g} > 0$ & 
%%		\footnotesize (b) $\lambda > 0$ and $\bar{g} < 0$ &
%%		\footnotesize (c) $\lambda < 0$ and $\bar{g} > 0$ &
%%		\footnotesize (d)  $\lambda < 0$ and $\bar{g} < 0$
%%	\end{tabular}
%	\caption{Illustration of the piece-wise cubic function $m(s)$. When $\lambda > 0$, 
%		$m(s)$ is a convex function and has a unique local minimum, which is also the global minimum 
%		((a) and (b)).
%		If $\lambda < 0$, then $m(s)$ has two local minima ((c) and (d)).
%		The scalar $\bar{g}$ corresponds to the slope of $m(s)$ at $s = 0$, i.e., $m'(0) = \bar{g}$.
%		If $\bar{g} > 0$, the minimum $s^*$ of $m(s)$
%		corresponds to the minimum of $m_-(s)$ (black circle in (a) and (c)), 	
%		and if $\bar{g} < 0$, then $s^*$ corresponds to the minimum of $m_+(s)$ (red circle in (b) and (d)).
%		\label{fig:cubic}}
%\end{figure}

\noindent \textbf{Adaptive regularization using cubics.} 
 Since the SR1 Hessian approximation can be indefinite, some safeguard must be implemented to ensure that the resulting search direction $\mathbf{s}_k$ is a descent direction.  One such safeguard is to use a ``regularization" term.
The Adaptive Regularization using Cubics (ARCs) method (see \citet{Griewank1981,NesP06,cartis2011adaptive}) can be viewed as an alternative to line-search and trust-region methods. At each iteration, an approximate global minimizer of a local (cubic) model,
\begin{equation}\label{eq:cr}
	\underset{\mathbf{s}\in \mathbb{R}^n}{\text{min}}  \ m_k(\mathbf{s}) 
	\equiv 
	%\underset{s\in \mathcal{R}^n}{\text{min}} 
	\mathbf{g}_k^{\top}\mathbf{s}
	+ \frac{1}{2} \mathbf{s}^{\top}\mathbf{B}_k \mathbf{s} + \frac{\mu_k}{3} (\Phi_k(\mathbf{s}))^3,
\end{equation}
is determined, where  $\mathbf{g}_k = \nabla f(\Theta_k)$, $\mu_k > 0$ is a regularization parameter, and
$\Phi_k$ is a function (norm) that regularizes $\mathbf{s}$.   Typically, the Euclidean norm is used.
In this work, we use an alternative ``shape-changing" norm that allows us to solve each subproblem 
(\ref{eq:cr}) exactly.  Proposed in \citet{Burdakov2017}, this shape-changing norm is
based on the partial eigendecomposition of $\mathbf{B}_{k}$.  Specifically, if 
$\mathbf{B}_{k} = \mathbf{U}_k \mathbf{\Lambda}_k \mathbf{U}_k^{\top}$ is the eigendecomposition
of $\mathbf{B}_k$, then we can define the norm 
$$
 \|\mathbf{s}\|_{\mathbf{U}_k}\overset{\text{def}}{=}\|\mathbf{U}_k^\top \mathbf{s}\|_3.
 $$
 It can be shown using H\"{o}lder's Inequality that 
 $$
 \frac{1}{\sqrt[\leftroot{1} 6]{n}}
 %n^{-1/6} 
 \| \mathbf{s} \|_2 
\le  \| \mathbf{s} \|_{\mathbf{U}_k}
\le  \| \mathbf{s} \|_2.
$$

As per the authors' literature review, this is the first time the adaptive regularized cubics has been used in conjunction with a shape changing norm in a deep learning setting. The main motivation of using this adaptive regularized cubics comes from better convergence properties when compared with a trust-region approach (see \citet{cartis2011adaptive}). Using the shape-changing norm allows us to solve the subproblem exactly.
 
 \noindent \textbf{Closed-form solution.} 
 Applying a change of basis with
$\bar{\mathbf{s}} = \mathbf{U}_k^{\top} \mathbf{s}$ and 
$\bar{\mathbf{g}}_k = \mathbf{U}_k^{\top}\mathbf{g}_k$, 
we can redefine the cubic subproblem as
%The ARC algorithm has claimed to achieve a 2-norm of the gradient $\norm{g} = \norm{\nabla f}$ below the desired accuracy $\epsilon$ in at most $\mathcal{O}(\epsilon^{-1.5})$ steps. Now we are ready to explain the regularization function $\Phi_k(s)$.
%\textbf{New basis:} If $\Phi(s)$ in (\ref{eq:cr}) is a two-norm operation on $s$, then our model function will, at most, have only two local minima, making it cumbersome to find the global minima. We propose a transformation of the parameter space $s$ to $\bar{s}$ such that $ \norm{s}_\mathbf{U}\overset{\text{def}}{=}\norm{\mathbf{U}^\top s}_3$ and redefine our objective function as
\begin{equation}\label{eq:modcr}
	\underset{\bar{\mathbf{s}} \in \mathbb{R}^n}{\text{min}}  \ \bar{{m}}_{k} (\bar{\mathbf{s}})
	= \bar{\mathbf{g}}_k^\top\bar{\mathbf{s}}
	+ \frac{1}{2}\bar{\mathbf{s}}^\top \mathbf{\Lambda}_k\bar{\mathbf{s}}
	+ \frac{\mu_k}{3}\|\bar{\mathbf{s}}\|_3^3.
\end{equation}
With this change of basis, we can easily find a closed-form solution of (\ref{eq:modcr}), which is generally not the case for other choices of norms.  
Note that $\bar{m}_k(\bar{\mathbf{s}})$ is a separable function,  
meaning we can write $\bar{m}_k(\bar{\mathbf{s}})$ as
$$
	\bar{m}_k(\bar{\mathbf{s}})
	=
	\sum_{i=1}^n
	\left \{
	(\bar{\mathbf{g}}_k)_i (\bar{\mathbf{s}})_i
	+
	\frac{1}{2}(\mathbf{\Lambda}_k)_i(\bar{\mathbf{s}})_i^2
	+
	\frac{\mu_k}{3} |(\bar{\mathbf{s}})_i |^3
	\right \}.
$$
Consequently, we can  solve (\ref{eq:modcr}) by solving one-dimensional problems 
of the form 
\begin{equation}\label{eq:modcr1}
	\underset{\bar{s} \in \mathbb{R}}{\text{min}}  \ \ \bar{m}(\bar{s})
	= \bar{g} \bar{s}   
	+ \frac{1}{2}\lambda \bar{s}^2
	+ \frac{\mu_k}{3}|\bar{s}|^3,
\end{equation}
where $\bar{g} \in \mathbb{R}$ corresponds to entries in $\bar{\mathbf{g}}_k$ and
$\lambda \in \mathbb{R}$ corresponds to diagonal entries in $\mathbf{\Lambda}_k$.  
To find the minimizer of (\ref{eq:modcr1}), we first write $\bar{m}(\bar{s})$ as follows:
\begin{equation*}
	\bar{m}(\bar{s}) = 
	\begin{cases}
		\bar{m}_+(s) = \bar{g} \bar{s}  
	+ \frac{1}{2}\lambda \bar{s}^2
	+ \frac{\mu_k}{3}\bar{s}^3 & \text{if $\bar{s} \ge 0$}, \\
		\bar{m}_-(\bar{s}) = \bar{g}\bar{s}  
	+ \frac{1}{2}\lambda \bar{s}^2
	- \frac{\mu_k}{3}\bar{s}^3 & \text{if $\bar{s} \le 0$}. 	
	\end{cases}
\end{equation*}
%The corresponding derivative is given by
%\begin{equation*}
%	m'(s) = 
%	\begin{cases}
%		m_+'(s) = g
%	+ \lambda s
%	+ \mu_ks^2 & \text{if $s \ge 0$},\\
%		m_-'(s) = g 
%	+ \lambda s
%	- \mu_ks^2 & \text{if $s \le 0$}.
%	\end{cases}
%\end{equation*}
The minimizer $\bar{s}^*$ of $\bar{m}(\bar{s})$ is obtained by setting $\bar{m}'(\bar{s})$ to zero and will depend on the sign of $\bar{g}$ because $\bar{g}$ is the slope of $\bar{m}(\bar{s})$ at $\bar{s} = 0$, i.e., $\bar{m}'(0) = \bar{g}$.  
In particular,
if $\bar{g} > 0$, then $\bar{s}^*$  is the minimizer of $\bar{m}_-(\bar{s})$,
%(see Figs.\ \ref{fig:cubic}(a) and (c)), 
namely
$\bar{s}^* = (-\lambda + \sqrt{\lambda^2 + 4\bar{g}\mu})/(-2\mu).$
If $\bar{g} < 0$, then $\bar{s}^*$ is the minimizer of $\bar{m}_+(\bar{s})$,
% (see Figs.\ \ref{fig:cubic}(b) and (d)), 
which is given by
$	\bar{s}^* = (-\lambda + \sqrt{\lambda^2 - 4\bar{g}\mu})/(2\mu).$
Note that these two expressions
for $\bar{s}^*$ are equivalent to the following formula:
$$
	\bar{s}^* = \frac{-2\bar{g}}{\lambda + \sqrt{\lambda^2 + 4|\bar{g}|\mu}},
$$
In the original space, $\mathbf{s}^* = \mathbf{U}_k \bar{\mathbf{s}}^*$ and 
$\mathbf{g}_k = \mathbf{U}_k \bar{\mathbf{g}}_k$.
Letting 
\begin{equation}\label{eq:Ck}
	\mathbf{C}_k = \text{diag} (\bar{c}_1, \dots, \bar{c}_n), \quad \text{where \ } \bar{c}_i =  \frac{2}{\lambda_i + \sqrt{\lambda_i^2 + 4|\bar{\mathbf{g}}_i|\mu}},
\end{equation}
then the solution $\mathbf{s}^*$ in the original space is  given by
\begin{equation}\label{eq:sstar}
	\mathbf{s}^* = \mathbf{U}_k \bar{\mathbf{s}}^* =  -\mathbf{U}_k  \mathbf{C}_k \mathbf{U}_k^{\top} \mathbf{g}_k.
\end{equation}
%For a more detailed description of the closed form solution, see  Appendix \ref{sec:Solution}. 




\noindent \textbf{Practical implementation.} While computing 
$\mathbf{U}_{\parallel} \in \mathbb{R}^{n \times k}$
in the matrix $\mathbf{U}_{k} = [  \ \mathbf{U}_{\parallel}  \ \ \mathbf{U}_{\perp} \ ]$
is feasible since 
$k \ll n$, computing $\mathbf{U}_{\perp}$ explicitly is not.  Thus, we must be able to compute 
$\mathbf{s}^*$ without needing $\mathbf{U}_{\perp}$.  
First, we define the following quantities
$$
\begin{array}{lllllll}
\bar{\mathbf{s}}_{\parallel} 
 = \mathbf{U}_{\parallel}^{\top} \mathbf{s} 
& \text{and}  
& \bar{\mathbf{s}}_{\perp} = \mathbf{U}_{\perp}^{\top} \mathbf{s},
\\[.2cm]
\bar{\mathbf{g}}_{\parallel} = \mathbf{U}_{\parallel}^{\top} \mathbf{g}_k
& \text{and} 
& \bar{\mathbf{g}}_{\perp} = \mathbf{U}_{\perp}^{\top} \mathbf{g}_k.
\end{array}
$$
Then the cubic subproblem (\ref{eq:modcr})  becomes
\begin{equation}
\underset{\bar{\mathbf{s}}\in \mathbb{R}^n
}{\text{minimize}}  \ \bar{{m}}_{k} (\bar{\mathbf{s}})
	\ = \ 
\underset{\bar{\mathbf{s}}_{\parallel} \in \mathbb{R}^k
}{\text{minimize}}  \ \bar{{m}}_{\parallel} (\bar{\mathbf{s}}_{\parallel}) + 
\underset{\bar{\mathbf{s}}_{\perp} \in \mathbb{R}^{n-k}
}{\text{minimize}}  \ \bar{{m}}_{\perp} (\bar{\mathbf{s}}_{\perp}),
\end{equation}
where
\begin{eqnarray} \label{eq:mparallel}
	\bar{m}_{\parallel}( \bar{\mathbf{s}}_{\parallel}) \!  \ \ &=& 
	\bar{\mathbf{g}}_{\parallel}^\top\bar{\mathbf{s}}_{\parallel}
	+ \frac{1}{2}\bar{\mathbf{s}}_{\parallel}^\top \hat{\mathbf{\Lambda}}_k\bar{\mathbf{s}}_{\parallel}
	+ \frac{\mu_k}{3}\|\bar{\mathbf{s}}_{\parallel} \|_3^3,
	\\
	\label{eq:mperp}
	\bar{m}_{\perp} ( \bar{\mathbf{s}}_{\perp}) &=& 
	\bar{\mathbf{g}}_{\perp}^\top\bar{\mathbf{s}}_{\perp}
	+ \frac{\delta_k}{2} \| \bar{\mathbf{s}}_{\perp} \|_2^2
	+ \frac{\mu_k}{3}\|\bar{\mathbf{s}}_{\perp}\|_3^3.
\end{eqnarray}
We minimize $\bar{m}_{\parallel}(\bar{s}_{\parallel})$ in (\ref{eq:mparallel}) similar to how we solved (\ref{eq:modcr1}).
In particular, if we let 
\begin{equation}\label{eq:Cparallel}
	\mathbf{C}_{\parallel} = \text{diag} (c_1, \dots, c_n), \ \text{where } c_i =  \frac{2}{\lambda_i + \sqrt{\lambda_i^2 + 4| (\bar{\mathbf{g}}_{\parallel})_i|\mu}},
\end{equation}
then the solution is given by 
\begin{equation}\label{eq:sparallelstar}
	\mathbf{s}_{\parallel}^* =
	-\mathbf{C}_{\parallel} \bar{\mathbf{g}}_{\parallel}.
\end{equation}



Minimizing $\bar{m}_{\perp}(\bar{s}_{\perp})$ in (\ref{eq:mperp}) is more challenging.
The only restriction on the matrix $\mathbf{U}_{\perp}$ is that its columns must form an orthonormal basis for the orthogonal complement of the range space of $\mathbf{U}_{\parallel}$.  
We are thus free to choose the columns of $\mathbf{U}_{\perp}$ as long as they satisfy this restriction.
In particular, we can choose the first column of $\mathbf{U}_{\perp}$ to be the normalized orthogonal projection of $\mathbf{g}_k$ onto the orthogonal complement of the range space of $\mathbf{U}_{\parallel}$, i.e.,
$$
	(\mathbf{U}_{\perp})_1 = ( \mathbf{I} - \mathbf{U}_{\parallel}\mathbf{U}_{\parallel}^{\top})\mathbf{g}_k
	/ \| ( \mathbf{I} - \mathbf{U}_{\parallel}\mathbf{U}_{\parallel}^{\top})\mathbf{g}_k \|_2.
$$
If $\mathbf{g}_k \in $ Range($\mathbf{U}_{\parallel}$), then 
$\bar{\mathbf{g}}_{\perp} = \mathbf{U}_{\perp}^{\top}\mathbf{g}_k = 0$  
% because g_k = U_parallel b for some b 
and the minimizer of (\ref{eq:mperp}) %$\bar{m}_{\perp}(\bar{\mathbf{s}}_{\perp})$ 
is $\bar{\mathbf{s}}_{\perp}^* = 0$ (since $\delta_k > 0$ and $\mu_k > 0$).
If $\mathbf{g}_k \notin $ Range($\mathbf{U}_{\parallel}$), then $(\mathbf{U}_{\perp})_1 \ne 0$ and 
we can choose vectors $ (\mathbf{U}_{\perp})_i \in \text{Range}(\mathbf{U}_{\parallel})^{\perp}$
such that $(\mathbf{U}_{\perp})_i^{\top} (\mathbf{U}_{\perp})_1 = 0$ for all $2 \le i \le n-k$.
Consequently,  $\mathbf{U}_{\perp}^{\top} (\mathbf{U}_{\perp})_1 = \kappa \mathbf{e}_1$,
where $\kappa$ is some constant and $\mathbf{e}_1$ is the first column of the identity matrix.  
Specifically,  
$$
	\kappa \mathbf{e}_1 
	=
	\mathbf{U}_{\perp}^{\top} (\mathbf{U}_{\perp})_1 
	= 
	\mathbf{U}_{\perp}^{\top}  \left ( \mathbf{U}_{\perp}  \mathbf{U}_{\perp}^{\top} \mathbf{g}_k \right )
	=
	\mathbf{U}_{\perp}^{\top} \mathbf{g}_k
	=
	\bar{\mathbf{g}}_{\perp},
$$
which implies $\kappa = \| \bar{\mathbf{g}}_{\perp} \|_2$.  Thus $ \bar{\mathbf{g}}_{\perp}$
has only one non-zero component (the first component) and therefore, the minimizer 
$\bar{\mathbf{s}}_{\perp}^*$ of 
$\bar{m}_{\perp} ( \bar{\mathbf{s}}_{\perp}) $ in (\ref{eq:mperp}) also has only one non-zero compoent (the first component as well).  In particular, 
\begin{align*}
	(\bar{\mathbf{s}}_{\perp}^*)_i
	&=
	\begin{cases}
		\displaystyle 
		-\alpha^* \| \bar{\mathbf{g}}_{\perp} \|_2
		& \text{if $i = 1$} 
		\\
		0 & \text{otherwise}
	\end{cases},
\end{align*}
where
\begin{equation}\label{eq:alphastar}
	\alpha=  \frac{2 }{ \delta_k 
		+ \sqrt{\delta_k^2 + 4 \mu \| \bar{\mathbf{g}}_{\perp} \|_2} }.
\end{equation}
Equivalently, $\bar{\mathbf{s}}_{\perp}^*=- \alpha^* \bar{\mathbf{g}}_{\perp}$.  
Note that the quantity $ \|  \bar{\mathbf{g}}_{\perp}\|_2$ can be computed without computing 
$ \bar{\mathbf{g}}_{\perp}$ from  the fact that $\| \mathbf{g} \|_2^2=
 \| \bar{\mathbf{g}}_{\parallel}\|_2^2 +  \| \bar{\mathbf{g}}_{\perp} \|_2^2$.  
 
 Combining the expressions for $\bar{s}_{\parallel}^*$ in (\ref{eq:sparallelstar}) and for 
 $\bar{\mathbf{s}}_{\perp}^*$, the solution in the original space is given by
 \begin{align*}
 	\mathbf{s}^* &=
	\mathbf{U}_{\parallel} \mathbf{s}_{\parallel}^* + 
	\mathbf{U}_{\perp}^{\phantom{^*}} \mathbf{s}_{\perp}^* \\
	&=
%	-\mathbf{U}_{\parallel}\mathbf{C}_{\parallel}\bar{\mathbf{g}}_{\parallel} - \alpha^* \bar{\mathbf{g}}_{\perp}
%= 
%-\mathbf{C}_{\parallel}\bar{\mathbf{g}}_{\parallel} 
- \mathbf{U}_{\parallel}\mathbf{C}_{\parallel}\mathbf{U}_{\parallel}^{\top} \mathbf{g} 
- \alpha^* (\mathbf{I}_n - \mathbf{U}_{\parallel}\mathbf{U}_{\parallel}^{\top}) \mathbf{g}\\
&= -\alpha^* \mathbf{g}  + \mathbf{U}_{\parallel}(\alpha^* \mathbf{I} - \mathbf{C}_{\parallel})\mathbf{U}_{\parallel}^{\top} \mathbf{g}.
 \end{align*}
 Note that computing $\mathbf{s}^*$ neither  involves forming $\mathbf{U}_{\perp}$ nor
 computing $\bar{\mathbf{g}}_{\perp}$ explicitly.
 
\bigskip




\noindent \textbf{Termination criteria.} 
With each cubic subproblem solved, the iterations are terminated when 
the change in iterates, $\mathbf{s}_k$, is sufficiently small, i.e., 
\begin{equation}\label{eq:acceptance2}
\| \mathbf{s}_k \|_2 < \tilde{\epsilon} \| \mathbf{y}_k - \mathbf{B}_k \mathbf{s}_k\|_2,
\end{equation}
for some $\tilde{\epsilon}$, 
or when the maximum number of allowable iterations is achieved.
The proposed Adaptive Regularization using Cubics with L-SR1 (ARCs-LSR1) algorithm is given in Algorithm \ref{alg:LSR1ARC}.






\begin{algorithm}[!h]
	\caption{Adaptive Regularization using Cubics with Limited-Memory SR1 (ARCs-LSR1) }
	\begin{algorithmic}[1]
		\STATE $\textbf{Given: }\Theta_0, \gamma_2 \geq \gamma_1, 1 > \eta_2 \geq \eta_1 > 0,\  \sigma_0 > 0, \tilde{\epsilon} > 0, k = 0,$	 and $k_{\text{max}} > 0$
%		\Require $S_k = \{s_0, \ldots, s_k\}$, $Y_k = \{y_0, \ldots, y_k\}$
		\WHILE {$k < k_{\text{max}} \ \text{and} \  \| \mathbf{s}_k \|_2 \ge \tilde{\epsilon} \| \mathbf{y}_k - \mathbf{B}_k \mathbf{s}_k\|_2$}
		\STATE {Obtain $\mathbf{S}_k = [ \ \mathbf{s}_0 \ \  \cdots \ \  \mathbf{s}_k \ ]$ and $\mathbf{Y}_k = [ \ \mathbf{y}_0 \ \  \cdots \ \ \mathbf{y	}_k \ ]$}
		\STATE {Solve the generalized eigenvalue problem $\mathbf{S}_k^{\top}\mathbf{Y}_k \mathbf{u} = \hat{\lambda}\mathbf{S}_k^{\top}\mathbf{S}_k \mathbf{u}$ 
		and let $\delta_k=\min\{ \hat{\lambda}_i\}$}
		\STATE {Compute $\mathbf{\Psi}_k = \mathbf{Y}_k - \delta_k \mathbf{S}_k$}
%		\If {Cholesky is available}
%		\State {$\Psi^{\top} \Psi = R^{\top}R$}
%		\State {$Q = \Psi R^{-1}$}
%		\Else  { Perform QR-decomposition of $\Psi$}
		\STATE {Perform QR decomposition of $\mathbf{\Psi}_k = \mathbf{Q}\mathbf{R}$}
%		\EndIf
		\STATE {Compute eigendecomposition 
		%\begin{equation}%\label{eqn:eigenvaluedecomposition}
		$	\mathbf{RMR}^\top= \mathbf{P\Lambda P}^\top$
		%\end{equation}
		
		}
		\STATE {Assign $\mathbf{U}_\parallel = \mathbf{QP}$ and $\mathbf{U}_{\parallel}^{\top} = \mathbf{P}^{\top} \mathbf{Q}^{\top}$}
		\STATE {%With $D = \text{diag}(\lambda_0,\ldots, \lambda_{m})$, 
		Define $\mathbf{C}_\parallel = \text{diag}(c_1,\ldots, c_k)$, where $c_i = \frac{2}{\lambda_i + \sqrt{\lambda_i^2 + 4\mu | (\bar{\mathbf{g}}_{\parallel})_{i}|}}$ 
		and  $\bar{\mathbf{g}}_\parallel = \mathbf{U}^\top_\parallel \mathbf{g}$}
		\STATE Compute {$\alpha^{*}$ in \eqref{eq:alphastar}} %= \frac{2}{\delta_k + \sqrt{\delta_k^2 + 4\mu\| \bar{\mathbf{g}}_{\perp}\|}}$} %where $\mathbf{g}_{\perp} = \mathbf{g} - \mathbf{U}_\parallel \bar{\mathbf{g}}_\parallel$
		\STATE {Compute step $\! \mathbf{s}^* = -\alpha^{*}\mathbf{g} + \mathbf{U}_{\parallel}(\alpha^{*}\mathbf{I} - \mathbf{C}_{\parallel})\mathbf{U}_{\parallel}^{\top}\mathbf{g}$}
		\STATE Compute $m_k(\mathbf{s}^*\!)$ \! and \!  $\rho_k \! \!=\!  (f(\Theta_k) 
		\!-\! f(\Theta_{k+1})\!)\!/m_k(\mathbf{s}^*\!)$% from (\ref{eq:ratio}) in Appendix A
		\STATE {Set 
		\begin{align*}
			\Theta_{k+1} &=
			\begin{cases}
				\Theta_k + \mathbf{s}^* \hspace{.85cm}  & \text{if }\rho_k\geq\eta_1\\
				\Theta_k, & \text{otherwise}		
			\end{cases}, \quad \text{and} 
			\\
%			 \left\{ 
%			\begin{array}{lr}
%				\Theta_k + s_k, & \text{if }\rho_k\geq\eta_1,\\
%				\Theta_k, & \text{otherwise}
%			\end{array}\right\}.
%		\end{align*}
%		\begin{align*}
			\mu_{k+1} &=
			\begin{cases}
				\tfrac{1}{2} \mu_k & \text{if }\rho_k > \eta_2,\\
				\tfrac{1}{2} \mu_k (1 + \gamma_1) & \text{if }\eta_1 \leq \rho_k \leq \eta_2,\\
				\tfrac{1}{2} \mu_k (\gamma_1 + \gamma_2) & \text{otherwise}			
			\end{cases}
%			\left\{\begin{array}{lr}
%				[0, \sigma_k] & \text{if }\rho_k > \eta_2,\\
%				\left[\sigma_k, \gamma_1\sigma_k\right] & \text{if }\eta_1 \leq \rho_k \leq \eta_2,\\
%				\left[\gamma_1\sigma_k, \gamma_2\sigma_k\right] & \text{otherwise}
%			\end{array}\right\}.
		\end{align*}
		}
		\STATE {$k \leftarrow k+1$}
			\ENDWHILE
	\end{algorithmic}\label{alg:LSR1ARC}
\end{algorithm}

%\begin{algorithm}[!h]
%	\caption{Adaptive Regularization using Cubics (ARC)}
%	\begin{algorithmic}
%		\State $\textbf{Given: }\Theta_0, \gamma_2 \geq \gamma_1, 1 > \eta_2 \geq \eta_1 > 0,\ \text{and}\ \sigma_0 > 0$
%		\While{$k \leq k_{\text{max}}$}
%		\State Compute a step $s_k$ using Algorithm \ref{alg:LSR1obs}
%		\State Compute $\rho_k$ using modified formula ratio of actual reduction to model reduction
%		\State Set 
%		\begin{equation*}
%			\Theta_{k+1} = \left\{ 
%			\begin{array}{lr}
%				\Theta_k + s_k, & \text{if }\rho_k\geq\eta_1,\\
%				\Theta_k, & \text{otherwise}
%			\end{array}\right\}.
%		\end{equation*}
%		\State Set 
%		\begin{equation*}
%			\sigma_{k+1} \in \left\{\begin{array}{lr}
%				[0, \sigma_k] & \text{if }\rho_k > \eta_2,\\
%				\left[\sigma_k, \gamma_1\sigma_k\right] & \text{if }\eta_1 \leq \rho_k \leq \eta_2,\\
%				\left[\gamma_1\sigma_k, \gamma_2\sigma_k\right] & \text{otherwise}
%			\end{array}\right\}.
%		\end{equation*}
%		\EndWhile
%	\end{algorithmic}\label{alg:adaptivereg}
%\end{algorithm}
%


%
%\subsection{Contributions}
%The main contributions of this paper are as follows:
%\begin{enumerate}[leftmargin=0.45cm]
%	\itemsep 0em
%	\item \textbf{L-SR1 quasi-Newton methods.} The most commonly used quasi-Newton approach is the L-BFGS method.  In this work, we use the L-SR1 update to better model potentially indefinite Hessians of the non-convex loss function. 
%	\item \textbf{Adaptive Regularization using Cubics (ARCs).} Given that the quasi-Newton approximation is allowed to be indefinite, we use an Adaptive Regularized using Cubics approach to 
%safeguard each search direction.
%	\item \textbf{Shape-changing regularizer.} 
%	We use a shape-changing norm to define the cubic regularization term, which allows us 
%	to compute the closed form solution to cubic subproblem (\ref{eq:cr}).  
%	\textbf{Computational complexity.} Let  $m$ be the number of previous iterates and gradients stored in memory. The proposed LSR1 ARC approach is comparable to L-BFGS in terms of storage and compute complexity (see Table \ref{tbl:storagecomplexity}).  
%	\begin{table*}[h]
%		\centering
%		\caption{Storage and compute complexity of the methods used in our experiments.}
%		\begin{tabular}{|c|c|c|}
%			\hline
%			\textbf{Algorithms} & \textbf{Storage complexity} & \textbf{Compute complexity}\\
%			\hline
%			SGD/Adaptive methods & $\mathcal{O}(n)$ & $\mathcal{O}(n)$ \\
%			L-BFGS & $\mathcal{O}(n + mn)$ &  $\mathcal{O}(mn)$\\
%			ARCs-LSR1 & $\mathcal{O}(n + mn)$ & $\mathcal{O}(m^3 + 2mn)$ \\ 
%			\hline
%		\end{tabular}\label{tbl:storagecomplexity}
%		\centering
%	\end{table*}
%\end{enumerate}
%%

\medskip

\noindent \textbf{Convergence.} 
Here, we prove convergence properties of the proposed method (ARCs-LSR1 in Algorithm \ref{alg:LSR1ARC}).
The following theoretical guarantees follow the ideas from \citet{Benson2018,cartis2011adaptive}.
First, we make the following mild assumptions:


\medskip

\noindent 
\textbf{A1.} The loss function $f(\Theta)$ is continuously differentiable, i.e., 
$f \in C^1(\mathbb{R}^n)$.

\noindent
\textbf{A2.} The loss function $f(\Theta)$ is bounded below.

\medskip

%
%It is reasonable to assume that the function $f$ in \eqref{eq:emp} is bounded below by some value $K$ and continuous.
%\begin{lemma}\label{con:lemma1}
%		$f \in C^1(\mathbb{R}^n)$
%\end{lemma}

\noindent 
Next, 
%under the assumption that the norm of the rank-1 matrix $(\mathbf{y}_j - \mathbf{B}_j\mathbf{s}_j)
%	(\mathbf{y}_j - \mathbf{B}_j\mathbf{s}_j)^{\top}$ 
%	in (\ref{eq:SR1_B0}) is bounded above
%	(see \cite{Benson2018}), 
%we obtain a upper bound on the norm of the Hessian approximation $\mathbf{B}_k$.
we prove that the matrix $\mathbf{B}_k$ in (\ref{eq:SR1_B0}) is bounded.  

\begin{lemma}\label{lemma:1}
%If $\| (\mathbf{y}_j - \mathbf{B}_j\mathbf{s}_j)
%(\mathbf{y}_j - \mathbf{B}_j\mathbf{s}_j)^{\top} \|_F \le K$ for some constant $K > 0$, then
The SR1 matrix $\mathbf{B}_{k+1}$  in (\ref{eq:SR1_B0}) satsifies
$$
	\text{$\|\mathbf{B}_{k+1}\|_F  \leq \kappa_B$  \ \ \text{for all $k \geq$ 1}}
$$
for some $\kappa_B$ $>$ 0.
\end{lemma}

\textit{Proof:} 
Using the limited-memory SR1 update with memory parameter $m$ in (\ref{eq:SR1_B0}), we have
$$
	\| \mathbf{B}_{k+1} \|_F \le \| \mathbf{B}_0 \|_F + 
	\hspace{-.4cm}
	\sum_{j = k-m+1}^k 
	\hspace{-.4cm} 
	\frac{\| (\mathbf{y}_j - \mathbf{B}_j\mathbf{s}_j) (\mathbf{y}_j - \mathbf{B}_j\mathbf{s}_j)^{\! \top} \! \|_F}
	{| \mathbf{s}_j^{\top} ( \mathbf{y}_j - \mathbf{B}_j\mathbf{s}_j) |}.
%	\le \delta_{\max} + \frac{m K}{\varepsilon} \equiv \kappa_B.
$$
Because $\mathbf{B}_0 = \delta_k \mathbf{I}$ with $\delta_k < \delta_{\max}$ for some $\delta_{\max} > 0$,
we have that $\| \mathbf{B}_0 \|_F = \sqrt{n} \delta_{\max}$.  
Using a property of the Frobenius norm,
namely, for real matrices $\mathbf{A}$, $\| \mathbf{A} \|_F^2 = \text{trace}(\mathbf{AA}^{\top})$, we have that
$\| (\mathbf{y}_j - \mathbf{B}_j\mathbf{s}_j) (\mathbf{y}_j - \mathbf{B}_j\mathbf{s}_j)^{\top} \|_F 
= \| \mathbf{y}_j - \mathbf{B}_j\mathbf{s}_j \|_2^2$.
Since the pair $(\mathbf{s}_j, \mathbf{y}_j)$ is accepted only when $|\mathbf{s}_j^{\top}(\mathbf{y}_j - \mathbf{B}_j\mathbf{s}_j)| > \varepsilon \| \mathbf{s}_j \|_2 \| \mathbf{y}_j - \mathbf{B}_j \mathbf{s}_j \|_2$, for some constant $\varepsilon > 0$, and since $\| \mathbf{s}_k \|_2 \ge \tilde{\epsilon} \| \mathbf{y}_k - \mathbf{B}_k \mathbf{s}_k\|_2$, we have
$$
	\| \mathbf{B}_{k+1} \|_F \le \sqrt{n} \delta_{\max} + \frac{m}{\varepsilon \tilde{\epsilon}} \equiv \kappa_B,
$$
which completes the proof.
$\square$

\medskip

\noindent 
Given the bound on $\| \mathbf{B}_{k+1} \|_F$, we obtain the following result, which is similar to Theorem 2.5 in \citet{cartis2011adaptive}.

\begin{theorem}\label{thm:liminf}
	Under Assumptions \textbf{A1}  and \textbf{A2}, if Lemma \ref{lemma:1} holds, then
	$$\underset{k \to \infty}{\text{lim inf}} \  \|\mathbf{g}_k\| = 0.$$
\end{theorem}

\noindent 
Finally, we consider the following assumption, which can be satisfied when the gradient, $\mathbf{g}(\Theta)$, is Lipschitz continuous on $\Theta$. 

\medskip

\noindent 
\textbf{A3.} If $\{ \Theta_{t_i} \}$ and $\{ \Theta_{l_i} \}$ are subsequences of $\{ \Theta_k \}$, then  $\| \mathbf{g}_{t_i} - \mathbf{g}_{l_i} \| \rightarrow 0$ whenever 
$\| \Theta_{t_i} - \Theta_{l_i} \| \rightarrow 0$ as $i \rightarrow \infty$.


\medskip

\noindent 
If we further make Assumption  \textbf{A3}, we have the following stronger result (which is based on Corollary 2.6 in \citet{cartis2011adaptive}):

\begin{corollary}\label{cor:ARCs}
Under Assumptions \textbf{A1},  \textbf{A2}, and \textbf{A3}, 
 if Lemma \ref{lemma:1} holds, then
	$$\underset{k \to \infty}{\text{lim}} \|\mathbf{g}_k\| = 0.$$
\end{corollary}

By Corollary 2.3, the proposed ARCs-LSR1 method converges to first-order critical points.   


\noindent \textbf{Stochastic implementation.} Because full gradient computation is very expensive to perform, we impement a stochastic version 
of the proposed ARCs-LSR1 method.  In particular, we use the batch gradient approximation
$$
	\tilde{\mathbf{g}}_k \equiv \frac{1}{| \mathcal{B}_k |} \sum_{i \in \mathcal{B}_k} \nabla f_i (\Theta_k).
$$
In defining the SR1 matrix, we use the quasi-Newton pairs $(\mathbf{s}_k, \tilde{\mathbf{y}}_k)$,
where $\tilde{\mathbf{y}}_k = \tilde{\mathbf{g}}_{k+1} - \tilde{\mathbf{g}}_k$ (see e.g., \citet{Erway2020TrustregionAF}).
We make the following additional assumption (similar to Assumption 4 in  \citet{Erway2020TrustregionAF}) to guarantee that the loss function $f(\Theta)$ decreases over time:

\medskip

\noindent
\textbf{A4.} The loss function $f(\Theta)$ is fully evaluated at every $J > 1$ iterations (for example, 
at iterates $\Theta_{J_0}, \Theta_{J_1}, \Theta_{J_2}, \dots,$ where $0 \le J_0 < J$ and
$J = J_1 - J_0 = J_2 - J_1 = \cdots $) and nowhere else in the algorithm.  The batch size $d$ is increased 
monotonically if $f(\Theta_{J_{\ell}}) > f(\Theta_{J_{\ell - 1}}) - \tau$ for some $\tau > 0$.

\medskip

\noindent 
With this added assumption, we can show that the stochastic version of the proposed ARCs-LSR1 method converges.

\begin{theorem}\label{thm:sARCs}
	The stochastic version of ARCs-LSR1 converges with  
	$$\underset{k \to \infty}{\text{lim}} \|\mathbf{g}_k\| = 0.$$
\end{theorem}

\textit{Proof:} Let $\widehat{\Theta}_i = \Theta_{J_i}$.  By Assumption 4, $f(\Theta)$ must 
decrease monotonically over the subsequence $\{ \widehat{\Theta}_i \}$ or $d \rightarrow |\mathcal{D}|$,
where $|\mathcal{D}|$ is the size of the dataset.    If the objective function is decreased 
$\iota_k$ times over the subsequence $ \{ \widehat{\Theta}_i\}_{i=0}^k$, then
%\begin{eqnarray*}
%	f(\widehat{\Theta}_k) &=& f(\hat{\Theta}_0) + \sum_{i=1}^{\iota_k}
%	\left \{
%		f(\widehat{\Theta}_i) - f(\widehat{\Theta}_{i-1})
%	\right \} \\
%	&\le& f(\widehat{\Theta}_0) - \iota_k \tau.
%\end{eqnarray*}
\begin{eqnarray*}
	f(\widehat{\Theta}_k) = f(\hat{\Theta}_0)  + \sum_{i=1}^{\iota_k}
	\left \{
		f(\widehat{\Theta}_i)  -  f(\widehat{\Theta}_{i-1})
	\right \} \le  f(\widehat{\Theta}_0) - \iota_k \tau.
\end{eqnarray*}
If $d \rightarrow |\mathcal{D}|$, then $\iota_k \rightarrow \infty$ as $k \rightarrow \infty$.
By Assumption \textbf{A2}, $f(\Theta)$ is bounded below, which implies $\iota_k$ is finite.  
Thus, $d \rightarrow |\mathcal{D}|$, and the algorithm reduces to the full ARCs-LSR1 method,
whose convergence is guaranteed by Corollary \ref{cor:ARCs}.  $\square$

\medskip

\noindent 
We note that the proof to Theorem \ref{thm:sARCs} follows very closely the proof of Theorem 2.2 in 
\citet{Erway2020TrustregionAF}.


%\ref{appnd:stochastic}.  

%\begin{equation*}
%	\mathbf{B}_{k+1} = \mathbf{B}_0 + \mathbf{Y}_k -\delta_k \mathbf{S}_k \mathbf{M}_k (\mathbf{Y}_k -\delta_k \mathbf{S}_k)^\top,
%\end{equation*}
%where $\mathbf{B}_0 = \delta_k I$. Using the triangle inequality, we have
%\begin{equation*}
%	\norm{\mathbf{B}_{k+1}} \leq \norm{\mathbf{B}_0}  + \norm{\mathbf{Y}_k -\delta_k \mathbf{S}_k \mathbf{M}_k (\mathbf{Y}_k -\delta_k \mathbf{S}_k)^\top}.
%\end{equation*}
%We rewrite the equation above as follows:
%\begin{equation*}
%	\norm{\mathbf{B}_{k+1}} \leq \norm{\mathbf{B}_0}  + \frac{m}{\epsilon} \max_{k - m + 1\leq i \leq k} \norm{\mathbf{y}_i - \mathbf{B}_i \mathbf{s}_i}_2^2
%\end{equation*}
%
%It is reasonable to assume that $\norm{\mathbf{y}_i - \mathbf{B}_i \mathbf{s}_i}_2^2 = (\mathbf{y}_i - \mathbf{B}_i \mathbf{s}_i)(\mathbf{y}_i - \mathbf{B}_i \mathbf{s}_i)^T$ is bounded above in norm (see \cite{Benson2018} Lemma 1). Thus, we prove $\mathbf{B}_k$ is bounded:
%
%If lemma \ref{con:lemma1} and lemma \ref{con:lemma2} hold, the following theorem is obtained.

%\begin{theorem}\label{thm:con}
%	$\underset{k \to \infty}{\text{lim inf}} \norm{g_k} = 0$ 
%\end{theorem}
%For proof, please refer \cite{Benson2018}, theorem 2.5.

%Additionally, we make the following assumption,
%\begin{assumption}\label{con}
%	$\norm{g_t - g_l} \to 0$ whenever $\norm{\Theta_t - \Theta_l} \to 0$, $i \to \infty$.
%\end{assumption}
%
%
%
%Since Lemma \ref{con:lemma1}, Lemma \ref{con:lemma2}, Theorem \ref{thm:con},\ref{con} hold and $f(\Theta_k)$is bounded below, we state the following corollary
%\begin{corollary}
%	$\underset{k \to \infty}{\text{lim}} \norm{g_k} = 0$.
%\end{corollary}

\noindent \textbf{Complexity analysis.} 
SGD methods and the related adaptive methods require $\mathcal{O}(n)$ memory storage to store
the gradient and $\mathcal{O}(n)$ computational complexity to update each iterate.
Such low memory and computational requirements make these methods easily implementable.  
Quasi-Newton methods store the previous $m$ gradients and use them to compute the update at each iteration.  Consequently,  L-BFGS methods require $\mathcal{O}(mn)$ memory storage to store
the gradients and $\mathcal{O}(mn)$ computational complexity to update each iterate 
(see \citet{Burdakov2017} for details).  Our proposed ARCs-LSR1 approach also uses 
$\mathcal{O}(mn)$ memory storage to store the gradients, but the computational 
complexity to update each iterate requires an additional eigendecomposition of the $m \times m$
matrix $\mathbf{RMR}^{\top}$, so that the overall computational complexity at each iteration is
$\mathcal{O}(m^3+ mn)$.  However, since $m \ll n$, this additional factorization does not significantly
increase the computational time.% (see Table \ref{tbl:storagecomplexity}).

%	\begin{table}[!h]
%		\centering
%		\caption{Storage and compute complexity of the methods used in our experiments.}
%		\begin{tabular}{|c|c|c|}
%			\hline
%			\textbf{Algorithms} & \textbf{Storage complexity} & \textbf{Compute complexity}\\
%			\hline
%			SGD/Adaptive methods & $\mathcal{O}(n)$ & $\mathcal{O}(n)$ \\
%			L-BFGS & $\mathcal{O}(n + mn)$ &  $\mathcal{O}(mn)$\\
%			ARCs-LSR1 & $\mathcal{O}(n + mn)$ & $\mathcal{O}(m^3 + 2mn)$ \\ 
%			\hline
%		\end{tabular}\label{tbl:storagecomplexity}
%		\centering
%	\end{table}

\section{Results}
\label{sec:Experiments}
In this section, we empirically compare the proposed algorithm on both sequence windows and time windows with existing methods.
\paragraph{Datasets} For the sequence-based model, we used two synthetic datasets and two cross-language datasets. The statistics of the datasets are provided in Table \ref{table:statistics}:

\begin{table}[t]
    \centering
    \caption{The statistics of the datasets. The datasets satisfy $1 \leq \|\vx\|\|\vy\| \leq R $.}
    \label{table:statistics}
    \begin{tabular}{|c|c|c|c|c|c|}
    \hline
        Dataset & $n$ & $m_x$ & $m_y$ & $N$ & $R$ \\ \hline
        SYNTHETIC(1) & 100,000 & 1,000 & 2,000 & 50,000 & 65 \\ \hline
        SYNTHETIC(2) & 100,000 & 1,000 & 2,000 & 50,000 & 724 \\ \hline
        APR & 23,235 & 28,017 & 42,833 & 10,000 & 773 \\ \hline
        PAN11 & 88,977 & 5,121 & 9,959 & 10,000 & 5,548 \\ \hline
        EURO & 475,834 & 7,247 & 8,768 & 100,000 & 107,840 \\ \hline
    \end{tabular}
\end{table}

\begin{itemize}
    \item Synthetic: The elements of the two synthetic datasets are initially uniformly sampled from the range (0,1), then multiplied by a coefficient to adjust the maximum column squared norm $R$. The X matrix has 1,000 rows, and the Y matrix has 2,000 rows, each with 100,000 columns. The window size is set to 50,000.
    \item APR: The Amazon Product Reviews (APR) dataset is a publicly available collection containing product reviews and related information from the Amazon website. This dataset consists of millions of sentences in both English and French. We structured it into a review matrix where the X matrix has 28,017 rows, and the Y matrix has 42,833 rows, with both matrices sharing 23,235 columns. The window size is 10,000.
    \item PAN11: PANPC-11 (PAN11) is a dataset designed for text analysis, particularly for tasks such as plagiarism detection, author identification, and near-duplicate detection. The dataset includes texts in English and French. The X and Y matrices contain 5,121 and 9,959 rows, respectively, with both matrices having 88,977 columns. The window size is 10,000.
\end{itemize}
We evaluate the time-based model on another real-world dataset:
\begin{itemize}
    \item EURO: The Europarl (EURO) dataset is a widely used multilingual parallel corpus, comprising the proceedings of the European Parliament. We selected a subset of its English and French text portions. The X and Y matrices contain 7,247 and 8,768 rows, respectively, and both matrices share 475,834 columns. Timestamps are generated using the $Poisson$ $Arrival$ $Process$ with a rate parameter of $\lambda=2$. The window size is set to 100,000, with approximately 30,000 columns of data on average in each window.
\end{itemize}

\paragraph{Setup} For the sequence-based model, we compare the proposed hDS-COD and  aDS-COD with EH-COD~\cite{yao2024approximate} and DI-COD~\cite{yao2024approximate}. We do not consider the Sampling algorithm as a baseline, as its performance is inferior to that of EH-COD and DI-CID, as demonstrated in \cite{yao2024approximate}. %The hDS-COD is adjusted by the parameter $\ell$ and the maximum number of levels $L = \log{R}$, where $R$ is the prior estimate of the maximum squared column norm of the dataset. DI-COD similarly requires a prior estimate of $R$ to limit the maximum number of levels $L = \log{(R/\varepsilon})$. In contrast, aDS-COD and EH-COD do not require an estimate of $R$; their error-space balance is controlled by the parameter $\ell = \frac{1}{\varepsilon}$. 
For the time-based model, we compare the proposed hDS-COD and  aDS-COD with EH-COD and the Sampling algorithm since DI-COD cannot be applied to time-based sliding window model. To achieve the same error bound, the maximum number of levels for hDS-COD is set to $L = \log{(\varepsilon NR)}$, and the initial threshold for aDS-COD is set to $1$.

Our experiments aim to illustrate the trade-offs between space and approximation errors. The x-axis represents two metrics for space: final sketch size and total space cost. The final sketch size refers to the number of columns in the result sketches $\mA$ and $\mB$ generated by the algorithm, representing a compression ratio. The total space cost refers to the maximum space required during the algorithm's execution, measured by the number of columns.We evaluate the approximation performance of all algorithms based on correlation errors $\operatorname{corr-err}(\mathbf{X}_W \mathbf{Y}_W^\top, \mathbf{A} \mathbf{B}^\top)$, which is reflected on the y-axis. Every 1,000 iterations, all algorithms query the window and record the average and maximum errors across all sampled windows.

The experiments for all algorithms were conducted using MATLAB (R2023a), with all algorithms running on a Windows server equipped with 32GB of memory and a single processor of Intel i9-13900K.

\paragraph{Performance} Figure \ref{fig:error vs l} and Figure \ref{fig:error vs space} illustrate the space efficiency comparison of the algorithms on sequence-based datasets. Panels (a-d) show the average errors across all sampled windows, while panels (e-h) display the maximum errors.

Figure \ref{fig:error vs l} evaluates the compression effect of the final sketch. The hDS-COD, aDS-COD, and EH-COD show similar compression performances. But the DS series is more stable, particularly on the synthetic datasets, where they significantly outperform EH-COD and DI-COD. The performance of hDS-COD and aDS-COD is nearly the same, indicating that the adaptive threshold trick in aDS-COD does not have a noticeable negative impact on it, maintaining the same error as hDS-COD.

Figure \ref{fig:error vs space} measures the total space cost of the algorithms. hDS-COD and aDS-COD show a significant advantage over existing methods, as they can achieve the  $\varepsilon$-approximation error with much less space. For the same space cost, the correlation errors of hDS-COD and aDS-COD are much smaller than those of EH-COD and DI-COD. Also, aDS-COD has better space efficiency than hDS-COD because aDS only uses a single-level structure while hDS requires $\log R+1$ levels. We find that hDS-COD requires more space on  SYNTHETIC(2) dataset compared to SYNTHETIC(1) dataset. This phenomenon occurs because SYNTHETIC(2) dataset has a larger $R$, which confirms the dependence on $R$ as stated in Theorem~\ref{thm:hds}. 

Figure \ref{fig:time-based} compares the performance of algorithms on time-based windows. Panels (a) and (b) present the error against the final sketch size, which show that our aDS-COD and hDS-COD algorithms enjoy similar performance as EH-COD and significantly outperform the sampling algorithm. On the other hand, as shown in panels (c) and (d), our methods outperform baselines in terms of total space cost.

%
% \section{Results}
\label{sec:Results}

% \begin{figure*}[htpb!]
% \label{}
% \centering

%     {{\label{ROCIowaCedar} \includegraphics[width=\textwidth/3]{figures/IowaCedar_roc.png}}}%
%     \qquad
%     {{\label{ROCIowaDesMoines} \includegraphics[width=\textwidth/3]{figures/IowaDesMoines_roc.png} }%
%   \captionsetup{justification=centering}
%   \caption{\Acf{ROC} curves for \acf{RW} Iowa (CR) and  \acf{RW} Iowa (DM) dataset. Dummy model here represents a model whose output is solely a ``no Flood'' for all pixels.}
%   \label{fig:RW_ROC_Curves}%
% \end{figure*}



\section{Results and Discussions}
\label{sec:Results}

In this section, we aim to answer three main questions. First, we want to validate our hypothesis that \ac{SYN} data is a viable proxy for \ac{RW} data when training ML models for downscaling. Secondly, we seek to assess how much more skillful ML-based downscaling is compared to classical, non-data-driven techniques, such as our baseline methods, \textit{i.e.}, thresholded bicubic and Lanczos interpolation. Finally, we would like to appraise the extent to which data-driven models like ours are transferable (in terms of usefulness) to other regions without major performance degradations.  
To assess the quality of the models, we conduct a multiple comparison test --namely the Holm-Bonferroni procedure \cite{HolmBonferroni1979} -- that is designed to control the \ac{FWER}. We notice that, with a \ac{FWER} of $10^{-3}$, all the differences in model performance are significant. The only exception to this trend was observed in \ac{RW}-GH for whom the pairwise differences between \ac{RCAN} and \ac{ESRT}, Lanczos and Bicubic were not significant with the aforementioned \ac{FWER}. 

%Finally, we aim to find out the factors influencing the transferability of our models from one region to another.

\subsection{Potential of using SYN Data for RW downscaling}

In order to evaluate the utility of synthetic data for training, we compare performances of our candidate models on both \ac{SYN} and \ac{RW} Iowa data whose results are presented in Table \ref{tab:IowaResults}. We notice that 
\textbf{(i)} For the Iowa datasets, there is a drop in performance of all the models when going from \ac{SYN} to \ac{RW} datasets, 
\textbf{(ii)} for the \ac{RW}-IA (CR) as well as \ac{RW}-IA (DM) datasets, both bicubic and Lanczos interpolation have accuracies and MCC up to 70.89\% and 0.42 respectively while the deep learning models have accuracies and MCC up to 73.34\% and 0.46 respectively, 
\textbf{(iii)} There is a roughly 6\% accuracy improvement for the \ac{SYN} data for the deep learning models compared to the bicubic and lanczos models and this improvement drops to about 3\% for \ac{RW} data,  
\textbf{(iv)} the performance of all the models remain consistent across both \ac{RW}-IA datasets and \textbf{(v)} in \figref{fig:RW_ROC_Curves}, we observe that there is a high degree of overlap among the \ac{ROC} curves for the data-driven models.

From (i) and (iv) we can conclude that \ac{SYN} data is more intricate than \ac{RW} data. This implies that the benefits yielded by training with \ac{SYN} dataset, while significant, is not as prominent in the \ac{RW} Iowa datasets. 
% This may be due to sensor noise prevalent in the \ac{RW} Landsat-8 data that can be harder to reproduce in the synthetically generated examples. 
(i), (iii) and (v) implies that while \ac{SYN} data is not an exact replacement for \ac{RW} data, it provides a rather significant edge, which is all the more important when there is insufficient \ac{RW} for training. From (ii) we can conclude that the three proposed data driven models outperform classical super-resolution techniques such as bicubic and lanczos, conclusion supported by the \ac{ROC} curves in Figure \ref{fig:RW_ROC_Curves} for whom the data-driven models, in general, lie above the non-data-driven alternatives. Observation (iv) shows that  for the climatically similar \ac{RW}-Iowa(CR) and \ac{RW}-Iowa(DM) regions, training on \ac{SYN} Iowa data does indeed provide an edge. 

% have similar climate. 

\begin{figure*}[t!]
    \centering
    \begin{subfigure}[t]{0.5\textwidth}
        \centering
        \includegraphics[width=\textwidth/2]{figures/IowaCedar_roc.png}
        \caption{}
    \end{subfigure}%
    ~ 
    \begin{subfigure}[t]{0.5\textwidth}
        \centering
        \includegraphics[width=\textwidth/2]{figures/IowaDesMoines_roc.png}
        \caption{}
    \end{subfigure}
    \vspace*{0.5cm}
    \caption{    \label{fig:RW_ROC_Curves} \Acf{ROC} curves for (a) RW-IA (CR) and (b) RW-IA (DM) dataset. Na\"ive model here represents a model whose output is solely a ``no Flood'' for all pixels. Star here represents the pixel-wise classifier with a threshold of 0.5.}
\end{figure*}


\subsection{Effectiveness of data-driven approaches}

In order to evaluate the effectiveness of ML models in the downscaling task, we compare performances of our candidate models to Lanczos and bicubic interpolation methods by looking at figures of some sample predictions from Iowa (Figure \ref{fig:RWIowaDesMoines}), performance comparison in the region of Iowa in Table \ref{tab:IowaResults} and the ROC curves in Figure \ref{fig:RW_ROC_Curves} for \ac{RW} data. We notice that 
\textbf{(vi)} For RW-IA (DM) samples, the deep learning models maintain a higher degree of spatial continuity in the predicted \ac{FIM}, 
\textbf{(vii)} We observe that  bicubic and Lanczos interpolation produces over-smoothed \ac{FIM} reconstructions, while the plain \ac{RDN}, \ac{RCAN} and \ac{ESRT} models are more detail-inclusive. Similar conclusions can be drawn upon inspecting the \ac{ROC} curves in Figure \ref{fig:RW_ROC_Curves} and 
\textbf{(viii)} For RW-IA (CR), the ML models show a performance improvement of 3.06\% when comparing the best ML model and non-data-driven method and, while for RW-IA (DM) there is a performance improvement of 2.45\%.


Figures \ref{fig:EUSamples} and \ref{fig:RWIowaDesMoines} show the spatial disparity among the models whose details are often obscured in aggregated metrics such as accuracy. (vi) This implies that these data-driven models are better are recognizing an underlying stream network geometry than the classical methods. However, when it comes to narrow river streams, all the models struggle capturing the nuances of the \ac{FIM} resultant from localized high elevation features such as small islands within rivers or man-made structures. (vii) shows a clear advantage of our data-driven approaches over the non-data-driven alternatives. (viii) indicates the benefits of the data-driven models when evaluated over Iowa. 



\subsection{Applicability of our models to external regions}

To evaluate how transferable our models are, we draw conclusions from figures of the sample predictions from Western Europe (Figure \ref{fig:EUSamples}) and Ghana (Figure \ref{fig:GhanaSamples}) as well as the performance comparison in Table \ref{tab:ExternalResults}. We notice that 
\textbf{(ix)} for Ghana all of the models fail to adequately inundate the pixels over separated areas on account of several disconnected regions of inundation in the chosen area,
\textbf{(x)} the ML models outperform non-data driven methods for RW-EU, 
\textbf{(xi)} for the RW-EU dataset, there is an improvement of 4.89\% when comparing the accuracy of the best data- and non-data-driven methods, 
\textbf{(xii)} For RW-RR and RW-GH, there is marginal improvement (up to 0.77\% in accuracy) of the ML methods over the non-data driven methods and 
\textbf{(xiii)} For RW-EU, we notice that the ML models produce more connected streams over the non-data-driven models. 

(x) and (xi) implies that the models are transferable when considering hydroclimaticalogically similar regions since Iowa and the Meuse river in Europe lie within mid temperate zones. Similar to the observation (vi) for RW-IA (DM), (xiii) implies that the benefits of the ML model in identifying underlying network streams is also transferable to hydroclimatologically similar regions. In contrast, (xii) and (ix) both imply that the trained ML models struggle to generalize to RW-RR \& RW-GH. We speculate that this may be due to the significant differences in geography and climate when compared to Iowa. 

% More specifically, we notice that Ghana has a lot of disconnected regions when compared to Iowa and Western Europe, possibly indicating a geomorphological dissimilarity. Additionally, in the case of Red River and Ghana, we also speculate that they include drivers to flood inundation that are different from Iowa and Western Europe, which lie within mild temperate zones. Ghana on the other hand has a tropical (dry and hot) climate.

Our study directly implies that good quality synthetic data can be useful surrogates for downscaling low-resolution \acp{WFM} to high-resolution \acp{FIM} in regions, where such data are hard to come by, even when downscaling by a factor of 10. We noticed that such models were readily transferable to climatically similar regions as the region of training. However, Such derived ML models did not feature significantly different transferability when evaluated over hydroclimatologically dissimilar regions, which we attribute to different flood inundation characteristics, primarily at finer scales. A possible avenue to circumvent such issues is to explore additional training approaches that fall under the general area of domain adaptation. Nevertheless, data-driven models are still advantageous (and, hence, preferable) over non-data-driven alternatives in transfer scenarios like the one we considered here. 


%%%%%%%%%%%%%%%%%%%%%%%%%%%%%%% unused text %%%%%%%%%%%%%%%%%%%%%%%%%%%%%%%%%%%%%%%



% \tabref{tab:AccuracyResults} depicts test accuracies obtained by our models on both \ac{SYN} and \ac{RW} data. For Iowan floods, a comparison of \ac{SYN} and \ac{RW} results shows \textbf{(i)} bicubic and Lanczos interpolations remarkably gaining about $3\%$ in accuracy, as well as \textbf{(ii)} \ac{RDN} and \ac{RCAN} remaining relatively stable, while \textbf{(iii)} topography-aware models loosing $2.7\%$ in performance. From (i) one can conclude that \ac{SYN} data are morphologically slightly more intricate than \ac{RW} data. Also, (i) and (ii) likely imply that \ac{SYN} data, excluding topography, can serve as satisfactory surrogates of \ac{RW} data. However, as implied by (iii), our topography-dependent models seems to be particularly sensitive to distributional shifts of their combined inputs (\acp{WFM} and topographic features). More specifically, the topography-informed models' performance edge, while still statistically significant, is extremely marginal, even when compared to our non-data-driven approaches. Next, when comparing results between the cases of Iowan and Ghanaian \ac{RW} data, one observes that \textbf{(iv)} the accuracy of bicubic and Lanczos interpolations drops by almost $5\%$ due to over-smoothing. This may imply that Ghanaian \acp{FIM} bare a more complex morphology, when compared to Iowan \acp{FIM}. Also, \textbf{(v)} our topography-agnostic, data-driven models' performance degrades more gracefully (by about $2\%$), while \textbf{(vi)} our topography-aware models perform, virtually, as bad as our non-data-driven approaches. Hence, the differences in the data populations of the two regions we considered are significant enough to render our topography-dependent models noncompetitive. 





\section{Conclusion}
\label{sec:Conclusion}
Software development is increasingly conceived as a collaboration activity between developers and AIs. Indeed, IDEs already implement features to enable interactive development, with AI suggesting implementations that are reused by developers.

Although multiple studies show this interaction can be successful, there is still limited understanding of how the models must be configured and used in the context of code generation tasks. This study addresses this gap, systematically investigating the impact of several key parameters, including the repeated submission of a prompt to accommodate for the non-deterministic nature of the models.

Our study reveals several key findings about the usage of ChatGPT. In particular, we discovered how creativity, although up to a limited extent, is useful to increase the range of methods whose code can be generated correctly. A major role is played by parameter top-p, which is commonly underrated, and instead has a major impact on the correctness of the results, with lower values producing better results. Finally, prompts should be submitted multiple times, with $5$ repetitions combined with a temperature of $1.2$ resulting in an effective configuration in our experiments.  

Future work concerns two main research directions. One is about replicating this experiment with other AI assistants, to validate our findings in multiple contexts. The second research direction concerns finding strategies to deal with the need to submit the same prompt multiple times to obtain a useful result, and thus developing approaches able to select or merge multiple responses automatically. 

\section{Acknowledgments}
\label{sec:Ack}
R.\ Marcia's research is partially supported by NSF Grants IIS 1741490 and DMS 1840265.  A.\ Ranganath's work was performed under the auspices of the U.\ S.\ Department of Energy by Lawrence Livermore National Laboratory under Contract DE-AC52-07NA27344. 
% %%%%%%%%%%%%%%%%%%%%%%%%%%%%%%%%%%%%%%%%%%%%%%%%%%%%%%%%%%%%
\bibliography{refs}
\bibliographystyle{tmlr}


\end{document}
