


Here, we demonstrate how the cubic subproblem
\begin{equation}\label{eqn:modcr0}\tag{8}
	\underset{\bar{\mathbf{s}} \in \mathbb{R}^n}{\text{min}} \  \ \bar{{m}}_{k} (\bar{\mathbf{s}})
	= \bar{\mathbf{g}}_k^\top\bar{\mathbf{s}}
	+ \frac{1}{2}\bar{\mathbf{s}}^\top \mathbf{\Lambda}_k\bar{\mathbf{s}}
	+ \frac{\mu_k}{3}\norm{\bar{\mathbf{s}}}_3^3
\end{equation}
is solved analytically.  Note that $\bar{m}_k(\bar{\mathbf{s}})$ is a separable function,  
meaning we can write $\bar{m}_k(\bar{\mathbf{s}})$ as
$$
	\bar{m}_k(\bar{\mathbf{s}})
	=
	\sum_{i=1}^n
	\left \{
	(\bar{\mathbf{g}}_k)_i (\bar{\mathbf{s}})_i
	+
	\frac{1}{2}(\mathbf{\Lambda}_k)_i(\bar{\mathbf{s}})_i^2
	+
	\frac{\mu_k}{3} |(\bar{\mathbf{s}})_i |_3^3
	\right \}.
$$
Consequently, we can  solve (\ref{eqn:modcr}) by solving one-dimensional problems 
of the form 
\begin{equation}\label{eqn:modcr1}
	\underset{s \in \mathbb{R}}{\text{min}}  \ \ m(s)
	= gs  
	+ \frac{1}{2}\lambda s^2
	+ \frac{\mu_k}{3}|s|^3,
\end{equation}
where $g \in \mathbb{R}$ corresponds to entries in $\bar{\mathbf{g}}_k$ and
$\lambda \in \mathbb{R}$ corresponds to diagonal entries in $\mathbf{\Lambda}_k$.  

To find the minimizer of (\ref{eqn:modcr1}), we first write $m(s)$ as follows:
\begin{equation*}
	m(s) = 
	\begin{cases}
		m_+(s) = gs  
	+ \frac{1}{2}\lambda s^2
	+ \frac{\mu_k}{3}s^3 & \text{if $s \ge 0$}, \\
		m_-(s) = gs  
	+ \frac{1}{2}\lambda s^2
	- \frac{\mu_k}{3}s^3 & \text{if $s \le 0$}. 	
	\end{cases}
\end{equation*}
The corresponding derivative is given by
\begin{equation*}
	m'(s) = 
	\begin{cases}
		m_+'(s) = g
	+ \lambda s
	+ \mu_ks^2 & \text{if $s \ge 0$},\\
		m_-'(s) = g 
	+ \lambda s
	- \mu_ks^2 & \text{if $s \le 0$}.
	\end{cases}
\end{equation*}
The minimizer $s^*$ of $m(s)$ is obtained by setting $m'(s)$ to zero and will depend on the sign of $g$ because $g$ is the slope of $m(s)$ at $s = 0$, i.e., $m'(0) = g$.  
In particular,
if $g > 0$, then $s^*$  is the minimizer of $m_-(s)$ (see Figs.\ \ref{fig:cubic}(a) and (c)), namely
\begin{equation*}
	s^* = \frac{-\lambda + \sqrt{\lambda^2 + 4gu}}{-2\mu}.
\end{equation*}
If $g < 0$, then $s^*$ is the minimizer of $m_+(s)$ (see Figs.\ \ref{fig:cubic}(b) and (d)), which is given by
\begin{equation*}
	s^* = \frac{-\lambda + \sqrt{\lambda^2 - 4gu}}{2\mu}.
\end{equation*}
Note that these two expressions
for $s^*$ are equivalent to the following formula:
$$
	s^* = \frac{-2g}{\lambda + \sqrt{\lambda^2 + 4|g|\mu}}.
$$






\begin{figure}[ht!]
	\begin{tabular}{cc}
	\includegraphics[width=3.75cm, trim={2cm 1cm 2cm 0},clip]{Figures/cubic1.pdf} &
	\includegraphics[width=3.75cm, trim={2cm 1cm 2cm 0},clip]{Figures/cubic2.pdf}  \\
	(a) $\lambda > 0$ and $g > 0$ & (b) $\lambda > 0$ and $g < 0$ \\
	\includegraphics[width=3.75cm, trim={2cm 1cm 2cm 0},clip]{Figures/cubic3.pdf} &
	\includegraphics[width=3.75cm, trim={2cm 1cm 2cm 0},clip]{Figures/cubic4.pdf} \\
	(c) $\lambda < 0$ and $g > 0$ & (d)  $\lambda < 0$ and $g < 0$
	\end{tabular}
	\caption{Illustration of the piece-wise cubic function $m(s)$. When $\lambda > 0$, 
	$m(s)$ is a convex function and has a unique local minimum, which is also the global minimum 
	((a) and (b)).
	If $\lambda < 0$, then $m(s)$ has two local minima ((c) and (d)).
	The scalar $g$ corresponds to the slope of $m(s)$ at $s = 0$, i.e., $m'(0) = g$.
	If $g > 0$, the minimum $s^*$ of $m(s)$
	corresponds to the minimum of $m_-(s)$ (black circle in (a) and (c)), 	
	and if $g < 0$, then $s^*$ corresponds to the minimum of $m_+(s)$ (red circle in (b) and (d)).
	\label{fig:cubic}}
\end{figure}







