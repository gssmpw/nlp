\documentclass{article}
\usepackage{log_2024}						% for camera-ready version
% \usepackage[review]{log_2024}				% for anonymous submission to proceedings track
% \usepackage[review,eatrack]{log_2024}		% for anonymous submission to extended abstract track
% \usepackage[preprint]{log_2024}			% for preprint version
% \usepackage[eatrack]{log_2024}				% for accepted extended abstracts

%%%%% NEW MATH DEFINITIONS %%%%%

\usepackage{amsmath,amsfonts,bm}
\usepackage{derivative}
% Mark sections of captions for referring to divisions of figures
\newcommand{\figleft}{{\em (Left)}}
\newcommand{\figcenter}{{\em (Center)}}
\newcommand{\figright}{{\em (Right)}}
\newcommand{\figtop}{{\em (Top)}}
\newcommand{\figbottom}{{\em (Bottom)}}
\newcommand{\captiona}{{\em (a)}}
\newcommand{\captionb}{{\em (b)}}
\newcommand{\captionc}{{\em (c)}}
\newcommand{\captiond}{{\em (d)}}

% Highlight a newly defined term
\newcommand{\newterm}[1]{{\bf #1}}

% Derivative d 
\newcommand{\deriv}{{\mathrm{d}}}

% Figure reference, lower-case.
\def\figref#1{figure~\ref{#1}}
% Figure reference, capital. For start of sentence
\def\Figref#1{Figure~\ref{#1}}
\def\twofigref#1#2{figures \ref{#1} and \ref{#2}}
\def\quadfigref#1#2#3#4{figures \ref{#1}, \ref{#2}, \ref{#3} and \ref{#4}}
% Section reference, lower-case.
\def\secref#1{section~\ref{#1}}
% Section reference, capital.
\def\Secref#1{Section~\ref{#1}}
% Reference to two sections.
\def\twosecrefs#1#2{sections \ref{#1} and \ref{#2}}
% Reference to three sections.
\def\secrefs#1#2#3{sections \ref{#1}, \ref{#2} and \ref{#3}}
% Reference to an equation, lower-case.
\def\eqref#1{equation~\ref{#1}}
% Reference to an equation, upper case
\def\Eqref#1{Equation~\ref{#1}}
% A raw reference to an equation---avoid using if possible
\def\plaineqref#1{\ref{#1}}
% Reference to a chapter, lower-case.
\def\chapref#1{chapter~\ref{#1}}
% Reference to an equation, upper case.
\def\Chapref#1{Chapter~\ref{#1}}
% Reference to a range of chapters
\def\rangechapref#1#2{chapters\ref{#1}--\ref{#2}}
% Reference to an algorithm, lower-case.
\def\algref#1{algorithm~\ref{#1}}
% Reference to an algorithm, upper case.
\def\Algref#1{Algorithm~\ref{#1}}
\def\twoalgref#1#2{algorithms \ref{#1} and \ref{#2}}
\def\Twoalgref#1#2{Algorithms \ref{#1} and \ref{#2}}
% Reference to a part, lower case
\def\partref#1{part~\ref{#1}}
% Reference to a part, upper case
\def\Partref#1{Part~\ref{#1}}
\def\twopartref#1#2{parts \ref{#1} and \ref{#2}}

\def\ceil#1{\lceil #1 \rceil}
\def\floor#1{\lfloor #1 \rfloor}
\def\1{\bm{1}}
\newcommand{\train}{\mathcal{D}}
\newcommand{\valid}{\mathcal{D_{\mathrm{valid}}}}
\newcommand{\test}{\mathcal{D_{\mathrm{test}}}}

\def\eps{{\epsilon}}


% Random variables
\def\reta{{\textnormal{$\eta$}}}
\def\ra{{\textnormal{a}}}
\def\rb{{\textnormal{b}}}
\def\rc{{\textnormal{c}}}
\def\rd{{\textnormal{d}}}
\def\re{{\textnormal{e}}}
\def\rf{{\textnormal{f}}}
\def\rg{{\textnormal{g}}}
\def\rh{{\textnormal{h}}}
\def\ri{{\textnormal{i}}}
\def\rj{{\textnormal{j}}}
\def\rk{{\textnormal{k}}}
\def\rl{{\textnormal{l}}}
% rm is already a command, just don't name any random variables m
\def\rn{{\textnormal{n}}}
\def\ro{{\textnormal{o}}}
\def\rp{{\textnormal{p}}}
\def\rq{{\textnormal{q}}}
\def\rr{{\textnormal{r}}}
\def\rs{{\textnormal{s}}}
\def\rt{{\textnormal{t}}}
\def\ru{{\textnormal{u}}}
\def\rv{{\textnormal{v}}}
\def\rw{{\textnormal{w}}}
\def\rx{{\textnormal{x}}}
\def\ry{{\textnormal{y}}}
\def\rz{{\textnormal{z}}}

% Random vectors
\def\rvepsilon{{\mathbf{\epsilon}}}
\def\rvphi{{\mathbf{\phi}}}
\def\rvtheta{{\mathbf{\theta}}}
\def\rva{{\mathbf{a}}}
\def\rvb{{\mathbf{b}}}
\def\rvc{{\mathbf{c}}}
\def\rvd{{\mathbf{d}}}
\def\rve{{\mathbf{e}}}
\def\rvf{{\mathbf{f}}}
\def\rvg{{\mathbf{g}}}
\def\rvh{{\mathbf{h}}}
\def\rvu{{\mathbf{i}}}
\def\rvj{{\mathbf{j}}}
\def\rvk{{\mathbf{k}}}
\def\rvl{{\mathbf{l}}}
\def\rvm{{\mathbf{m}}}
\def\rvn{{\mathbf{n}}}
\def\rvo{{\mathbf{o}}}
\def\rvp{{\mathbf{p}}}
\def\rvq{{\mathbf{q}}}
\def\rvr{{\mathbf{r}}}
\def\rvs{{\mathbf{s}}}
\def\rvt{{\mathbf{t}}}
\def\rvu{{\mathbf{u}}}
\def\rvv{{\mathbf{v}}}
\def\rvw{{\mathbf{w}}}
\def\rvx{{\mathbf{x}}}
\def\rvy{{\mathbf{y}}}
\def\rvz{{\mathbf{z}}}

% Elements of random vectors
\def\erva{{\textnormal{a}}}
\def\ervb{{\textnormal{b}}}
\def\ervc{{\textnormal{c}}}
\def\ervd{{\textnormal{d}}}
\def\erve{{\textnormal{e}}}
\def\ervf{{\textnormal{f}}}
\def\ervg{{\textnormal{g}}}
\def\ervh{{\textnormal{h}}}
\def\ervi{{\textnormal{i}}}
\def\ervj{{\textnormal{j}}}
\def\ervk{{\textnormal{k}}}
\def\ervl{{\textnormal{l}}}
\def\ervm{{\textnormal{m}}}
\def\ervn{{\textnormal{n}}}
\def\ervo{{\textnormal{o}}}
\def\ervp{{\textnormal{p}}}
\def\ervq{{\textnormal{q}}}
\def\ervr{{\textnormal{r}}}
\def\ervs{{\textnormal{s}}}
\def\ervt{{\textnormal{t}}}
\def\ervu{{\textnormal{u}}}
\def\ervv{{\textnormal{v}}}
\def\ervw{{\textnormal{w}}}
\def\ervx{{\textnormal{x}}}
\def\ervy{{\textnormal{y}}}
\def\ervz{{\textnormal{z}}}

% Random matrices
\def\rmA{{\mathbf{A}}}
\def\rmB{{\mathbf{B}}}
\def\rmC{{\mathbf{C}}}
\def\rmD{{\mathbf{D}}}
\def\rmE{{\mathbf{E}}}
\def\rmF{{\mathbf{F}}}
\def\rmG{{\mathbf{G}}}
\def\rmH{{\mathbf{H}}}
\def\rmI{{\mathbf{I}}}
\def\rmJ{{\mathbf{J}}}
\def\rmK{{\mathbf{K}}}
\def\rmL{{\mathbf{L}}}
\def\rmM{{\mathbf{M}}}
\def\rmN{{\mathbf{N}}}
\def\rmO{{\mathbf{O}}}
\def\rmP{{\mathbf{P}}}
\def\rmQ{{\mathbf{Q}}}
\def\rmR{{\mathbf{R}}}
\def\rmS{{\mathbf{S}}}
\def\rmT{{\mathbf{T}}}
\def\rmU{{\mathbf{U}}}
\def\rmV{{\mathbf{V}}}
\def\rmW{{\mathbf{W}}}
\def\rmX{{\mathbf{X}}}
\def\rmY{{\mathbf{Y}}}
\def\rmZ{{\mathbf{Z}}}

% Elements of random matrices
\def\ermA{{\textnormal{A}}}
\def\ermB{{\textnormal{B}}}
\def\ermC{{\textnormal{C}}}
\def\ermD{{\textnormal{D}}}
\def\ermE{{\textnormal{E}}}
\def\ermF{{\textnormal{F}}}
\def\ermG{{\textnormal{G}}}
\def\ermH{{\textnormal{H}}}
\def\ermI{{\textnormal{I}}}
\def\ermJ{{\textnormal{J}}}
\def\ermK{{\textnormal{K}}}
\def\ermL{{\textnormal{L}}}
\def\ermM{{\textnormal{M}}}
\def\ermN{{\textnormal{N}}}
\def\ermO{{\textnormal{O}}}
\def\ermP{{\textnormal{P}}}
\def\ermQ{{\textnormal{Q}}}
\def\ermR{{\textnormal{R}}}
\def\ermS{{\textnormal{S}}}
\def\ermT{{\textnormal{T}}}
\def\ermU{{\textnormal{U}}}
\def\ermV{{\textnormal{V}}}
\def\ermW{{\textnormal{W}}}
\def\ermX{{\textnormal{X}}}
\def\ermY{{\textnormal{Y}}}
\def\ermZ{{\textnormal{Z}}}

% Vectors
\def\vzero{{\bm{0}}}
\def\vone{{\bm{1}}}
\def\vmu{{\bm{\mu}}}
\def\vtheta{{\bm{\theta}}}
\def\vphi{{\bm{\phi}}}
\def\va{{\bm{a}}}
\def\vb{{\bm{b}}}
\def\vc{{\bm{c}}}
\def\vd{{\bm{d}}}
\def\ve{{\bm{e}}}
\def\vf{{\bm{f}}}
\def\vg{{\bm{g}}}
\def\vh{{\bm{h}}}
\def\vi{{\bm{i}}}
\def\vj{{\bm{j}}}
\def\vk{{\bm{k}}}
\def\vl{{\bm{l}}}
\def\vm{{\bm{m}}}
\def\vn{{\bm{n}}}
\def\vo{{\bm{o}}}
\def\vp{{\bm{p}}}
\def\vq{{\bm{q}}}
\def\vr{{\bm{r}}}
\def\vs{{\bm{s}}}
\def\vt{{\bm{t}}}
\def\vu{{\bm{u}}}
\def\vv{{\bm{v}}}
\def\vw{{\bm{w}}}
\def\vx{{\bm{x}}}
\def\vy{{\bm{y}}}
\def\vz{{\bm{z}}}

% Elements of vectors
\def\evalpha{{\alpha}}
\def\evbeta{{\beta}}
\def\evepsilon{{\epsilon}}
\def\evlambda{{\lambda}}
\def\evomega{{\omega}}
\def\evmu{{\mu}}
\def\evpsi{{\psi}}
\def\evsigma{{\sigma}}
\def\evtheta{{\theta}}
\def\eva{{a}}
\def\evb{{b}}
\def\evc{{c}}
\def\evd{{d}}
\def\eve{{e}}
\def\evf{{f}}
\def\evg{{g}}
\def\evh{{h}}
\def\evi{{i}}
\def\evj{{j}}
\def\evk{{k}}
\def\evl{{l}}
\def\evm{{m}}
\def\evn{{n}}
\def\evo{{o}}
\def\evp{{p}}
\def\evq{{q}}
\def\evr{{r}}
\def\evs{{s}}
\def\evt{{t}}
\def\evu{{u}}
\def\evv{{v}}
\def\evw{{w}}
\def\evx{{x}}
\def\evy{{y}}
\def\evz{{z}}

% Matrix
\def\mA{{\bm{A}}}
\def\mB{{\bm{B}}}
\def\mC{{\bm{C}}}
\def\mD{{\bm{D}}}
\def\mE{{\bm{E}}}
\def\mF{{\bm{F}}}
\def\mG{{\bm{G}}}
\def\mH{{\bm{H}}}
\def\mI{{\bm{I}}}
\def\mJ{{\bm{J}}}
\def\mK{{\bm{K}}}
\def\mL{{\bm{L}}}
\def\mM{{\bm{M}}}
\def\mN{{\bm{N}}}
\def\mO{{\bm{O}}}
\def\mP{{\bm{P}}}
\def\mQ{{\bm{Q}}}
\def\mR{{\bm{R}}}
\def\mS{{\bm{S}}}
\def\mT{{\bm{T}}}
\def\mU{{\bm{U}}}
\def\mV{{\bm{V}}}
\def\mW{{\bm{W}}}
\def\mX{{\bm{X}}}
\def\mY{{\bm{Y}}}
\def\mZ{{\bm{Z}}}
\def\mBeta{{\bm{\beta}}}
\def\mPhi{{\bm{\Phi}}}
\def\mLambda{{\bm{\Lambda}}}
\def\mSigma{{\bm{\Sigma}}}

% Tensor
\DeclareMathAlphabet{\mathsfit}{\encodingdefault}{\sfdefault}{m}{sl}
\SetMathAlphabet{\mathsfit}{bold}{\encodingdefault}{\sfdefault}{bx}{n}
\newcommand{\tens}[1]{\bm{\mathsfit{#1}}}
\def\tA{{\tens{A}}}
\def\tB{{\tens{B}}}
\def\tC{{\tens{C}}}
\def\tD{{\tens{D}}}
\def\tE{{\tens{E}}}
\def\tF{{\tens{F}}}
\def\tG{{\tens{G}}}
\def\tH{{\tens{H}}}
\def\tI{{\tens{I}}}
\def\tJ{{\tens{J}}}
\def\tK{{\tens{K}}}
\def\tL{{\tens{L}}}
\def\tM{{\tens{M}}}
\def\tN{{\tens{N}}}
\def\tO{{\tens{O}}}
\def\tP{{\tens{P}}}
\def\tQ{{\tens{Q}}}
\def\tR{{\tens{R}}}
\def\tS{{\tens{S}}}
\def\tT{{\tens{T}}}
\def\tU{{\tens{U}}}
\def\tV{{\tens{V}}}
\def\tW{{\tens{W}}}
\def\tX{{\tens{X}}}
\def\tY{{\tens{Y}}}
\def\tZ{{\tens{Z}}}


% Graph
\def\gA{{\mathcal{A}}}
\def\gB{{\mathcal{B}}}
\def\gC{{\mathcal{C}}}
\def\gD{{\mathcal{D}}}
\def\gE{{\mathcal{E}}}
\def\gF{{\mathcal{F}}}
\def\gG{{\mathcal{G}}}
\def\gH{{\mathcal{H}}}
\def\gI{{\mathcal{I}}}
\def\gJ{{\mathcal{J}}}
\def\gK{{\mathcal{K}}}
\def\gL{{\mathcal{L}}}
\def\gM{{\mathcal{M}}}
\def\gN{{\mathcal{N}}}
\def\gO{{\mathcal{O}}}
\def\gP{{\mathcal{P}}}
\def\gQ{{\mathcal{Q}}}
\def\gR{{\mathcal{R}}}
\def\gS{{\mathcal{S}}}
\def\gT{{\mathcal{T}}}
\def\gU{{\mathcal{U}}}
\def\gV{{\mathcal{V}}}
\def\gW{{\mathcal{W}}}
\def\gX{{\mathcal{X}}}
\def\gY{{\mathcal{Y}}}
\def\gZ{{\mathcal{Z}}}

% Sets
\def\sA{{\mathbb{A}}}
\def\sB{{\mathbb{B}}}
\def\sC{{\mathbb{C}}}
\def\sD{{\mathbb{D}}}
% Don't use a set called E, because this would be the same as our symbol
% for expectation.
\def\sF{{\mathbb{F}}}
\def\sG{{\mathbb{G}}}
\def\sH{{\mathbb{H}}}
\def\sI{{\mathbb{I}}}
\def\sJ{{\mathbb{J}}}
\def\sK{{\mathbb{K}}}
\def\sL{{\mathbb{L}}}
\def\sM{{\mathbb{M}}}
\def\sN{{\mathbb{N}}}
\def\sO{{\mathbb{O}}}
\def\sP{{\mathbb{P}}}
\def\sQ{{\mathbb{Q}}}
\def\sR{{\mathbb{R}}}
\def\sS{{\mathbb{S}}}
\def\sT{{\mathbb{T}}}
\def\sU{{\mathbb{U}}}
\def\sV{{\mathbb{V}}}
\def\sW{{\mathbb{W}}}
\def\sX{{\mathbb{X}}}
\def\sY{{\mathbb{Y}}}
\def\sZ{{\mathbb{Z}}}

% Entries of a matrix
\def\emLambda{{\Lambda}}
\def\emA{{A}}
\def\emB{{B}}
\def\emC{{C}}
\def\emD{{D}}
\def\emE{{E}}
\def\emF{{F}}
\def\emG{{G}}
\def\emH{{H}}
\def\emI{{I}}
\def\emJ{{J}}
\def\emK{{K}}
\def\emL{{L}}
\def\emM{{M}}
\def\emN{{N}}
\def\emO{{O}}
\def\emP{{P}}
\def\emQ{{Q}}
\def\emR{{R}}
\def\emS{{S}}
\def\emT{{T}}
\def\emU{{U}}
\def\emV{{V}}
\def\emW{{W}}
\def\emX{{X}}
\def\emY{{Y}}
\def\emZ{{Z}}
\def\emSigma{{\Sigma}}

% entries of a tensor
% Same font as tensor, without \bm wrapper
\newcommand{\etens}[1]{\mathsfit{#1}}
\def\etLambda{{\etens{\Lambda}}}
\def\etA{{\etens{A}}}
\def\etB{{\etens{B}}}
\def\etC{{\etens{C}}}
\def\etD{{\etens{D}}}
\def\etE{{\etens{E}}}
\def\etF{{\etens{F}}}
\def\etG{{\etens{G}}}
\def\etH{{\etens{H}}}
\def\etI{{\etens{I}}}
\def\etJ{{\etens{J}}}
\def\etK{{\etens{K}}}
\def\etL{{\etens{L}}}
\def\etM{{\etens{M}}}
\def\etN{{\etens{N}}}
\def\etO{{\etens{O}}}
\def\etP{{\etens{P}}}
\def\etQ{{\etens{Q}}}
\def\etR{{\etens{R}}}
\def\etS{{\etens{S}}}
\def\etT{{\etens{T}}}
\def\etU{{\etens{U}}}
\def\etV{{\etens{V}}}
\def\etW{{\etens{W}}}
\def\etX{{\etens{X}}}
\def\etY{{\etens{Y}}}
\def\etZ{{\etens{Z}}}

% The true underlying data generating distribution
\newcommand{\pdata}{p_{\rm{data}}}
\newcommand{\ptarget}{p_{\rm{target}}}
\newcommand{\pprior}{p_{\rm{prior}}}
\newcommand{\pbase}{p_{\rm{base}}}
\newcommand{\pref}{p_{\rm{ref}}}

% The empirical distribution defined by the training set
\newcommand{\ptrain}{\hat{p}_{\rm{data}}}
\newcommand{\Ptrain}{\hat{P}_{\rm{data}}}
% The model distribution
\newcommand{\pmodel}{p_{\rm{model}}}
\newcommand{\Pmodel}{P_{\rm{model}}}
\newcommand{\ptildemodel}{\tilde{p}_{\rm{model}}}
% Stochastic autoencoder distributions
\newcommand{\pencode}{p_{\rm{encoder}}}
\newcommand{\pdecode}{p_{\rm{decoder}}}
\newcommand{\precons}{p_{\rm{reconstruct}}}

\newcommand{\laplace}{\mathrm{Laplace}} % Laplace distribution

\newcommand{\E}{\mathbb{E}}
\newcommand{\Ls}{\mathcal{L}}
\newcommand{\R}{\mathbb{R}}
\newcommand{\emp}{\tilde{p}}
\newcommand{\lr}{\alpha}
\newcommand{\reg}{\lambda}
\newcommand{\rect}{\mathrm{rectifier}}
\newcommand{\softmax}{\mathrm{softmax}}
\newcommand{\sigmoid}{\sigma}
\newcommand{\softplus}{\zeta}
\newcommand{\KL}{D_{\mathrm{KL}}}
\newcommand{\Var}{\mathrm{Var}}
\newcommand{\standarderror}{\mathrm{SE}}
\newcommand{\Cov}{\mathrm{Cov}}
% Wolfram Mathworld says $L^2$ is for function spaces and $\ell^2$ is for vectors
% But then they seem to use $L^2$ for vectors throughout the site, and so does
% wikipedia.
\newcommand{\normlzero}{L^0}
\newcommand{\normlone}{L^1}
\newcommand{\normltwo}{L^2}
\newcommand{\normlp}{L^p}
\newcommand{\normmax}{L^\infty}

\newcommand{\parents}{Pa} % See usage in notation.tex. Chosen to match Daphne's book.

\DeclareMathOperator*{\argmax}{arg\,max}
\DeclareMathOperator*{\argmin}{arg\,min}

\DeclareMathOperator{\sign}{sign}
\DeclareMathOperator{\Tr}{Tr}
\let\ab\allowbreak


\usepackage{booktabs}						% professional-quality tables
\usepackage{multirow}						% tabular cells spanning multiple rows
\usepackage{amsfonts}						% blackboard math symbols
\usepackage{graphicx}						% figures
\usepackage{duckuments}						% sample images

% If you want to use natbib:
\usepackage[numbers,compress,sort]{natbib}	% for numerical citations
% \usepackage[sort,round]{natbib}			% for textual citations

% If you want to use bibLaTeX, uncomment below:
% \usepackage[
%      backend=biber,
%      style=numeric-comp,
%      backref=true,
%      natbib=true]{biblatex}
% \addbibresource{reference.bib}

\usepackage[utf8]{inputenc} % allow utf-8 input
\usepackage[T1]{fontenc}    % use 8-bit T1 fonts
\usepackage[backref=page]{hyperref}       % hyperlinks
\usepackage{url}            % simple URL typesetting
\usepackage{multirow}
\usepackage{booktabs}       % professional-quality tables
\usepackage{amsfonts}       % blackboard math symbols
\usepackage{nicefrac}       % compact symbols for 1/2, etc.
\usepackage{microtype}      % microtypography
\usepackage{xcolor}         % colors
\usepackage{xspace}
\usepackage{adjustbox}
%\usepackage
\usepackage{xcolor}
% \hypersetup{
%     colorlinks,
%     linkcolor={red!50!black},
%     citecolor={blue!50!black},
%     urlcolor={blue!80!black}
% }
\usepackage{wrapfig}


\usepackage{bbm}
\usepackage{CJKutf8}

%%%%%% NEW MATH DEFINITIONS %%%%%

\usepackage{amsmath,amsfonts,bm}
\usepackage{derivative}
% Mark sections of captions for referring to divisions of figures
\newcommand{\figleft}{{\em (Left)}}
\newcommand{\figcenter}{{\em (Center)}}
\newcommand{\figright}{{\em (Right)}}
\newcommand{\figtop}{{\em (Top)}}
\newcommand{\figbottom}{{\em (Bottom)}}
\newcommand{\captiona}{{\em (a)}}
\newcommand{\captionb}{{\em (b)}}
\newcommand{\captionc}{{\em (c)}}
\newcommand{\captiond}{{\em (d)}}

% Highlight a newly defined term
\newcommand{\newterm}[1]{{\bf #1}}

% Derivative d 
\newcommand{\deriv}{{\mathrm{d}}}

% Figure reference, lower-case.
\def\figref#1{figure~\ref{#1}}
% Figure reference, capital. For start of sentence
\def\Figref#1{Figure~\ref{#1}}
\def\twofigref#1#2{figures \ref{#1} and \ref{#2}}
\def\quadfigref#1#2#3#4{figures \ref{#1}, \ref{#2}, \ref{#3} and \ref{#4}}
% Section reference, lower-case.
\def\secref#1{section~\ref{#1}}
% Section reference, capital.
\def\Secref#1{Section~\ref{#1}}
% Reference to two sections.
\def\twosecrefs#1#2{sections \ref{#1} and \ref{#2}}
% Reference to three sections.
\def\secrefs#1#2#3{sections \ref{#1}, \ref{#2} and \ref{#3}}
% Reference to an equation, lower-case.
\def\eqref#1{equation~\ref{#1}}
% Reference to an equation, upper case
\def\Eqref#1{Equation~\ref{#1}}
% A raw reference to an equation---avoid using if possible
\def\plaineqref#1{\ref{#1}}
% Reference to a chapter, lower-case.
\def\chapref#1{chapter~\ref{#1}}
% Reference to an equation, upper case.
\def\Chapref#1{Chapter~\ref{#1}}
% Reference to a range of chapters
\def\rangechapref#1#2{chapters\ref{#1}--\ref{#2}}
% Reference to an algorithm, lower-case.
\def\algref#1{algorithm~\ref{#1}}
% Reference to an algorithm, upper case.
\def\Algref#1{Algorithm~\ref{#1}}
\def\twoalgref#1#2{algorithms \ref{#1} and \ref{#2}}
\def\Twoalgref#1#2{Algorithms \ref{#1} and \ref{#2}}
% Reference to a part, lower case
\def\partref#1{part~\ref{#1}}
% Reference to a part, upper case
\def\Partref#1{Part~\ref{#1}}
\def\twopartref#1#2{parts \ref{#1} and \ref{#2}}

\def\ceil#1{\lceil #1 \rceil}
\def\floor#1{\lfloor #1 \rfloor}
\def\1{\bm{1}}
\newcommand{\train}{\mathcal{D}}
\newcommand{\valid}{\mathcal{D_{\mathrm{valid}}}}
\newcommand{\test}{\mathcal{D_{\mathrm{test}}}}

\def\eps{{\epsilon}}


% Random variables
\def\reta{{\textnormal{$\eta$}}}
\def\ra{{\textnormal{a}}}
\def\rb{{\textnormal{b}}}
\def\rc{{\textnormal{c}}}
\def\rd{{\textnormal{d}}}
\def\re{{\textnormal{e}}}
\def\rf{{\textnormal{f}}}
\def\rg{{\textnormal{g}}}
\def\rh{{\textnormal{h}}}
\def\ri{{\textnormal{i}}}
\def\rj{{\textnormal{j}}}
\def\rk{{\textnormal{k}}}
\def\rl{{\textnormal{l}}}
% rm is already a command, just don't name any random variables m
\def\rn{{\textnormal{n}}}
\def\ro{{\textnormal{o}}}
\def\rp{{\textnormal{p}}}
\def\rq{{\textnormal{q}}}
\def\rr{{\textnormal{r}}}
\def\rs{{\textnormal{s}}}
\def\rt{{\textnormal{t}}}
\def\ru{{\textnormal{u}}}
\def\rv{{\textnormal{v}}}
\def\rw{{\textnormal{w}}}
\def\rx{{\textnormal{x}}}
\def\ry{{\textnormal{y}}}
\def\rz{{\textnormal{z}}}

% Random vectors
\def\rvepsilon{{\mathbf{\epsilon}}}
\def\rvphi{{\mathbf{\phi}}}
\def\rvtheta{{\mathbf{\theta}}}
\def\rva{{\mathbf{a}}}
\def\rvb{{\mathbf{b}}}
\def\rvc{{\mathbf{c}}}
\def\rvd{{\mathbf{d}}}
\def\rve{{\mathbf{e}}}
\def\rvf{{\mathbf{f}}}
\def\rvg{{\mathbf{g}}}
\def\rvh{{\mathbf{h}}}
\def\rvu{{\mathbf{i}}}
\def\rvj{{\mathbf{j}}}
\def\rvk{{\mathbf{k}}}
\def\rvl{{\mathbf{l}}}
\def\rvm{{\mathbf{m}}}
\def\rvn{{\mathbf{n}}}
\def\rvo{{\mathbf{o}}}
\def\rvp{{\mathbf{p}}}
\def\rvq{{\mathbf{q}}}
\def\rvr{{\mathbf{r}}}
\def\rvs{{\mathbf{s}}}
\def\rvt{{\mathbf{t}}}
\def\rvu{{\mathbf{u}}}
\def\rvv{{\mathbf{v}}}
\def\rvw{{\mathbf{w}}}
\def\rvx{{\mathbf{x}}}
\def\rvy{{\mathbf{y}}}
\def\rvz{{\mathbf{z}}}

% Elements of random vectors
\def\erva{{\textnormal{a}}}
\def\ervb{{\textnormal{b}}}
\def\ervc{{\textnormal{c}}}
\def\ervd{{\textnormal{d}}}
\def\erve{{\textnormal{e}}}
\def\ervf{{\textnormal{f}}}
\def\ervg{{\textnormal{g}}}
\def\ervh{{\textnormal{h}}}
\def\ervi{{\textnormal{i}}}
\def\ervj{{\textnormal{j}}}
\def\ervk{{\textnormal{k}}}
\def\ervl{{\textnormal{l}}}
\def\ervm{{\textnormal{m}}}
\def\ervn{{\textnormal{n}}}
\def\ervo{{\textnormal{o}}}
\def\ervp{{\textnormal{p}}}
\def\ervq{{\textnormal{q}}}
\def\ervr{{\textnormal{r}}}
\def\ervs{{\textnormal{s}}}
\def\ervt{{\textnormal{t}}}
\def\ervu{{\textnormal{u}}}
\def\ervv{{\textnormal{v}}}
\def\ervw{{\textnormal{w}}}
\def\ervx{{\textnormal{x}}}
\def\ervy{{\textnormal{y}}}
\def\ervz{{\textnormal{z}}}

% Random matrices
\def\rmA{{\mathbf{A}}}
\def\rmB{{\mathbf{B}}}
\def\rmC{{\mathbf{C}}}
\def\rmD{{\mathbf{D}}}
\def\rmE{{\mathbf{E}}}
\def\rmF{{\mathbf{F}}}
\def\rmG{{\mathbf{G}}}
\def\rmH{{\mathbf{H}}}
\def\rmI{{\mathbf{I}}}
\def\rmJ{{\mathbf{J}}}
\def\rmK{{\mathbf{K}}}
\def\rmL{{\mathbf{L}}}
\def\rmM{{\mathbf{M}}}
\def\rmN{{\mathbf{N}}}
\def\rmO{{\mathbf{O}}}
\def\rmP{{\mathbf{P}}}
\def\rmQ{{\mathbf{Q}}}
\def\rmR{{\mathbf{R}}}
\def\rmS{{\mathbf{S}}}
\def\rmT{{\mathbf{T}}}
\def\rmU{{\mathbf{U}}}
\def\rmV{{\mathbf{V}}}
\def\rmW{{\mathbf{W}}}
\def\rmX{{\mathbf{X}}}
\def\rmY{{\mathbf{Y}}}
\def\rmZ{{\mathbf{Z}}}

% Elements of random matrices
\def\ermA{{\textnormal{A}}}
\def\ermB{{\textnormal{B}}}
\def\ermC{{\textnormal{C}}}
\def\ermD{{\textnormal{D}}}
\def\ermE{{\textnormal{E}}}
\def\ermF{{\textnormal{F}}}
\def\ermG{{\textnormal{G}}}
\def\ermH{{\textnormal{H}}}
\def\ermI{{\textnormal{I}}}
\def\ermJ{{\textnormal{J}}}
\def\ermK{{\textnormal{K}}}
\def\ermL{{\textnormal{L}}}
\def\ermM{{\textnormal{M}}}
\def\ermN{{\textnormal{N}}}
\def\ermO{{\textnormal{O}}}
\def\ermP{{\textnormal{P}}}
\def\ermQ{{\textnormal{Q}}}
\def\ermR{{\textnormal{R}}}
\def\ermS{{\textnormal{S}}}
\def\ermT{{\textnormal{T}}}
\def\ermU{{\textnormal{U}}}
\def\ermV{{\textnormal{V}}}
\def\ermW{{\textnormal{W}}}
\def\ermX{{\textnormal{X}}}
\def\ermY{{\textnormal{Y}}}
\def\ermZ{{\textnormal{Z}}}

% Vectors
\def\vzero{{\bm{0}}}
\def\vone{{\bm{1}}}
\def\vmu{{\bm{\mu}}}
\def\vtheta{{\bm{\theta}}}
\def\vphi{{\bm{\phi}}}
\def\va{{\bm{a}}}
\def\vb{{\bm{b}}}
\def\vc{{\bm{c}}}
\def\vd{{\bm{d}}}
\def\ve{{\bm{e}}}
\def\vf{{\bm{f}}}
\def\vg{{\bm{g}}}
\def\vh{{\bm{h}}}
\def\vi{{\bm{i}}}
\def\vj{{\bm{j}}}
\def\vk{{\bm{k}}}
\def\vl{{\bm{l}}}
\def\vm{{\bm{m}}}
\def\vn{{\bm{n}}}
\def\vo{{\bm{o}}}
\def\vp{{\bm{p}}}
\def\vq{{\bm{q}}}
\def\vr{{\bm{r}}}
\def\vs{{\bm{s}}}
\def\vt{{\bm{t}}}
\def\vu{{\bm{u}}}
\def\vv{{\bm{v}}}
\def\vw{{\bm{w}}}
\def\vx{{\bm{x}}}
\def\vy{{\bm{y}}}
\def\vz{{\bm{z}}}

% Elements of vectors
\def\evalpha{{\alpha}}
\def\evbeta{{\beta}}
\def\evepsilon{{\epsilon}}
\def\evlambda{{\lambda}}
\def\evomega{{\omega}}
\def\evmu{{\mu}}
\def\evpsi{{\psi}}
\def\evsigma{{\sigma}}
\def\evtheta{{\theta}}
\def\eva{{a}}
\def\evb{{b}}
\def\evc{{c}}
\def\evd{{d}}
\def\eve{{e}}
\def\evf{{f}}
\def\evg{{g}}
\def\evh{{h}}
\def\evi{{i}}
\def\evj{{j}}
\def\evk{{k}}
\def\evl{{l}}
\def\evm{{m}}
\def\evn{{n}}
\def\evo{{o}}
\def\evp{{p}}
\def\evq{{q}}
\def\evr{{r}}
\def\evs{{s}}
\def\evt{{t}}
\def\evu{{u}}
\def\evv{{v}}
\def\evw{{w}}
\def\evx{{x}}
\def\evy{{y}}
\def\evz{{z}}

% Matrix
\def\mA{{\bm{A}}}
\def\mB{{\bm{B}}}
\def\mC{{\bm{C}}}
\def\mD{{\bm{D}}}
\def\mE{{\bm{E}}}
\def\mF{{\bm{F}}}
\def\mG{{\bm{G}}}
\def\mH{{\bm{H}}}
\def\mI{{\bm{I}}}
\def\mJ{{\bm{J}}}
\def\mK{{\bm{K}}}
\def\mL{{\bm{L}}}
\def\mM{{\bm{M}}}
\def\mN{{\bm{N}}}
\def\mO{{\bm{O}}}
\def\mP{{\bm{P}}}
\def\mQ{{\bm{Q}}}
\def\mR{{\bm{R}}}
\def\mS{{\bm{S}}}
\def\mT{{\bm{T}}}
\def\mU{{\bm{U}}}
\def\mV{{\bm{V}}}
\def\mW{{\bm{W}}}
\def\mX{{\bm{X}}}
\def\mY{{\bm{Y}}}
\def\mZ{{\bm{Z}}}
\def\mBeta{{\bm{\beta}}}
\def\mPhi{{\bm{\Phi}}}
\def\mLambda{{\bm{\Lambda}}}
\def\mSigma{{\bm{\Sigma}}}

% Tensor
\DeclareMathAlphabet{\mathsfit}{\encodingdefault}{\sfdefault}{m}{sl}
\SetMathAlphabet{\mathsfit}{bold}{\encodingdefault}{\sfdefault}{bx}{n}
\newcommand{\tens}[1]{\bm{\mathsfit{#1}}}
\def\tA{{\tens{A}}}
\def\tB{{\tens{B}}}
\def\tC{{\tens{C}}}
\def\tD{{\tens{D}}}
\def\tE{{\tens{E}}}
\def\tF{{\tens{F}}}
\def\tG{{\tens{G}}}
\def\tH{{\tens{H}}}
\def\tI{{\tens{I}}}
\def\tJ{{\tens{J}}}
\def\tK{{\tens{K}}}
\def\tL{{\tens{L}}}
\def\tM{{\tens{M}}}
\def\tN{{\tens{N}}}
\def\tO{{\tens{O}}}
\def\tP{{\tens{P}}}
\def\tQ{{\tens{Q}}}
\def\tR{{\tens{R}}}
\def\tS{{\tens{S}}}
\def\tT{{\tens{T}}}
\def\tU{{\tens{U}}}
\def\tV{{\tens{V}}}
\def\tW{{\tens{W}}}
\def\tX{{\tens{X}}}
\def\tY{{\tens{Y}}}
\def\tZ{{\tens{Z}}}


% Graph
\def\gA{{\mathcal{A}}}
\def\gB{{\mathcal{B}}}
\def\gC{{\mathcal{C}}}
\def\gD{{\mathcal{D}}}
\def\gE{{\mathcal{E}}}
\def\gF{{\mathcal{F}}}
\def\gG{{\mathcal{G}}}
\def\gH{{\mathcal{H}}}
\def\gI{{\mathcal{I}}}
\def\gJ{{\mathcal{J}}}
\def\gK{{\mathcal{K}}}
\def\gL{{\mathcal{L}}}
\def\gM{{\mathcal{M}}}
\def\gN{{\mathcal{N}}}
\def\gO{{\mathcal{O}}}
\def\gP{{\mathcal{P}}}
\def\gQ{{\mathcal{Q}}}
\def\gR{{\mathcal{R}}}
\def\gS{{\mathcal{S}}}
\def\gT{{\mathcal{T}}}
\def\gU{{\mathcal{U}}}
\def\gV{{\mathcal{V}}}
\def\gW{{\mathcal{W}}}
\def\gX{{\mathcal{X}}}
\def\gY{{\mathcal{Y}}}
\def\gZ{{\mathcal{Z}}}

% Sets
\def\sA{{\mathbb{A}}}
\def\sB{{\mathbb{B}}}
\def\sC{{\mathbb{C}}}
\def\sD{{\mathbb{D}}}
% Don't use a set called E, because this would be the same as our symbol
% for expectation.
\def\sF{{\mathbb{F}}}
\def\sG{{\mathbb{G}}}
\def\sH{{\mathbb{H}}}
\def\sI{{\mathbb{I}}}
\def\sJ{{\mathbb{J}}}
\def\sK{{\mathbb{K}}}
\def\sL{{\mathbb{L}}}
\def\sM{{\mathbb{M}}}
\def\sN{{\mathbb{N}}}
\def\sO{{\mathbb{O}}}
\def\sP{{\mathbb{P}}}
\def\sQ{{\mathbb{Q}}}
\def\sR{{\mathbb{R}}}
\def\sS{{\mathbb{S}}}
\def\sT{{\mathbb{T}}}
\def\sU{{\mathbb{U}}}
\def\sV{{\mathbb{V}}}
\def\sW{{\mathbb{W}}}
\def\sX{{\mathbb{X}}}
\def\sY{{\mathbb{Y}}}
\def\sZ{{\mathbb{Z}}}

% Entries of a matrix
\def\emLambda{{\Lambda}}
\def\emA{{A}}
\def\emB{{B}}
\def\emC{{C}}
\def\emD{{D}}
\def\emE{{E}}
\def\emF{{F}}
\def\emG{{G}}
\def\emH{{H}}
\def\emI{{I}}
\def\emJ{{J}}
\def\emK{{K}}
\def\emL{{L}}
\def\emM{{M}}
\def\emN{{N}}
\def\emO{{O}}
\def\emP{{P}}
\def\emQ{{Q}}
\def\emR{{R}}
\def\emS{{S}}
\def\emT{{T}}
\def\emU{{U}}
\def\emV{{V}}
\def\emW{{W}}
\def\emX{{X}}
\def\emY{{Y}}
\def\emZ{{Z}}
\def\emSigma{{\Sigma}}

% entries of a tensor
% Same font as tensor, without \bm wrapper
\newcommand{\etens}[1]{\mathsfit{#1}}
\def\etLambda{{\etens{\Lambda}}}
\def\etA{{\etens{A}}}
\def\etB{{\etens{B}}}
\def\etC{{\etens{C}}}
\def\etD{{\etens{D}}}
\def\etE{{\etens{E}}}
\def\etF{{\etens{F}}}
\def\etG{{\etens{G}}}
\def\etH{{\etens{H}}}
\def\etI{{\etens{I}}}
\def\etJ{{\etens{J}}}
\def\etK{{\etens{K}}}
\def\etL{{\etens{L}}}
\def\etM{{\etens{M}}}
\def\etN{{\etens{N}}}
\def\etO{{\etens{O}}}
\def\etP{{\etens{P}}}
\def\etQ{{\etens{Q}}}
\def\etR{{\etens{R}}}
\def\etS{{\etens{S}}}
\def\etT{{\etens{T}}}
\def\etU{{\etens{U}}}
\def\etV{{\etens{V}}}
\def\etW{{\etens{W}}}
\def\etX{{\etens{X}}}
\def\etY{{\etens{Y}}}
\def\etZ{{\etens{Z}}}

% The true underlying data generating distribution
\newcommand{\pdata}{p_{\rm{data}}}
\newcommand{\ptarget}{p_{\rm{target}}}
\newcommand{\pprior}{p_{\rm{prior}}}
\newcommand{\pbase}{p_{\rm{base}}}
\newcommand{\pref}{p_{\rm{ref}}}

% The empirical distribution defined by the training set
\newcommand{\ptrain}{\hat{p}_{\rm{data}}}
\newcommand{\Ptrain}{\hat{P}_{\rm{data}}}
% The model distribution
\newcommand{\pmodel}{p_{\rm{model}}}
\newcommand{\Pmodel}{P_{\rm{model}}}
\newcommand{\ptildemodel}{\tilde{p}_{\rm{model}}}
% Stochastic autoencoder distributions
\newcommand{\pencode}{p_{\rm{encoder}}}
\newcommand{\pdecode}{p_{\rm{decoder}}}
\newcommand{\precons}{p_{\rm{reconstruct}}}

\newcommand{\laplace}{\mathrm{Laplace}} % Laplace distribution

\newcommand{\E}{\mathbb{E}}
\newcommand{\Ls}{\mathcal{L}}
\newcommand{\R}{\mathbb{R}}
\newcommand{\emp}{\tilde{p}}
\newcommand{\lr}{\alpha}
\newcommand{\reg}{\lambda}
\newcommand{\rect}{\mathrm{rectifier}}
\newcommand{\softmax}{\mathrm{softmax}}
\newcommand{\sigmoid}{\sigma}
\newcommand{\softplus}{\zeta}
\newcommand{\KL}{D_{\mathrm{KL}}}
\newcommand{\Var}{\mathrm{Var}}
\newcommand{\standarderror}{\mathrm{SE}}
\newcommand{\Cov}{\mathrm{Cov}}
% Wolfram Mathworld says $L^2$ is for function spaces and $\ell^2$ is for vectors
% But then they seem to use $L^2$ for vectors throughout the site, and so does
% wikipedia.
\newcommand{\normlzero}{L^0}
\newcommand{\normlone}{L^1}
\newcommand{\normltwo}{L^2}
\newcommand{\normlp}{L^p}
\newcommand{\normmax}{L^\infty}

\newcommand{\parents}{Pa} % See usage in notation.tex. Chosen to match Daphne's book.

\DeclareMathOperator*{\argmax}{arg\,max}
\DeclareMathOperator*{\argmin}{arg\,min}

\DeclareMathOperator{\sign}{sign}
\DeclareMathOperator{\Tr}{Tr}
\let\ab\allowbreak

\usepackage{cleveref}
%\usepackage{graphicx}
\usepackage{subcaption}

\usepackage{algpseudocode}
\usepackage{algorithm}

\newcommand{\ourmethod}{TRIX\xspace}


\title{TRIX: A More Expressive \revision{Model for Zero-shot\\ 
Domain Transfer in Knowledge Graphs}}

\author[Y. Zhang et al.]{%
Yucheng Zhang\\
Purdue University\\
\email{zhan4332@purdue.edu}\And
Beatrice Bevilacqua\\
Purdue University\\
\email{bbevilac@purdue.edu}\AND
Mikhail Galkin\\
Intel AI Lab\\
\email{mikhail.galkin@intel.com}\And
Bruno Ribeiro\\
Purdue University\\
\email{ribeirob@purdue.edu}
}

\begin{document}

\maketitle

\begin{abstract}
\revision{Fully inductive knowledge graph models can be trained on multiple domains and subsequently perform zero-shot knowledge graph completion (KGC) in new unseen domains.
This is an important capability towards the goal of having foundation models for knowledge graphs.}
  In this work, we introduce a more expressive and capable \revision{fully inductive model}, dubbed \ourmethod, which not only yields strictly more expressive triplet embeddings (head entity, relation, tail entity)  compared to state-of-the-art methods, but also introduces a new capability: directly handling both entity and relation prediction tasks in inductive settings. 
  Empirically, we show that \ourmethod outperforms the state-of-the-art fully inductive models in zero-shot entity and relation predictions in new domains, and outperforms large-context LLMs in out-of-domain predictions.
  % Empirically, we show that \ourmethod\ surpasses state-of-the-art \revision{fully inductive models}, with an average improvement of up to 7.4\% in hits@10 accuracy in zero-shot entity or relation prediction, as well as fine-tuning scenarios, across a diverse set of 57 datasets. Additionally, we conduct a comparison with large-context Large Language Models (LLMs) on zero-shot relation \revision{and entity} prediction tasks. Our results demonstrate that while LLMs achieve comparable performance to \ourmethod in-domain, their predictive capabilities degrade significantly out-of-domain (exhibiting high sensitivity to relation and entity permutations in the knowledge graph prompt), while \ourmethod maintains robust performance in these scenarios. This underscores the importance of further developing \revision{fully inductive models} to reason across domains. 
  The source code is available at \revision{\url{https://github.com/yuchengz99/TRIX}}.
\end{abstract}

\section{Introduction}
%\vspace{-5pt}
\section{Introduction}
\label{sec:introduction}
The business processes of organizations are experiencing ever-increasing complexity due to the large amount of data, high number of users, and high-tech devices involved \cite{martin2021pmopportunitieschallenges, beerepoot2023biggestbpmproblems}. This complexity may cause business processes to deviate from normal control flow due to unforeseen and disruptive anomalies \cite{adams2023proceddsriftdetection}. These control-flow anomalies manifest as unknown, skipped, and wrongly-ordered activities in the traces of event logs monitored from the execution of business processes \cite{ko2023adsystematicreview}. For the sake of clarity, let us consider an illustrative example of such anomalies. Figure \ref{FP_ANOMALIES} shows a so-called event log footprint, which captures the control flow relations of four activities of a hypothetical event log. In particular, this footprint captures the control-flow relations between activities \texttt{a}, \texttt{b}, \texttt{c} and \texttt{d}. These are the causal ($\rightarrow$) relation, concurrent ($\parallel$) relation, and other ($\#$) relations such as exclusivity or non-local dependency \cite{aalst2022pmhandbook}. In addition, on the right are six traces, of which five exhibit skipped, wrongly-ordered and unknown control-flow anomalies. For example, $\langle$\texttt{a b d}$\rangle$ has a skipped activity, which is \texttt{c}. Because of this skipped activity, the control-flow relation \texttt{b}$\,\#\,$\texttt{d} is violated, since \texttt{d} directly follows \texttt{b} in the anomalous trace.
\begin{figure}[!t]
\centering
\includegraphics[width=0.9\columnwidth]{images/FP_ANOMALIES.png}
\caption{An example event log footprint with six traces, of which five exhibit control-flow anomalies.}
\label{FP_ANOMALIES}
\end{figure}

\subsection{Control-flow anomaly detection}
Control-flow anomaly detection techniques aim to characterize the normal control flow from event logs and verify whether these deviations occur in new event logs \cite{ko2023adsystematicreview}. To develop control-flow anomaly detection techniques, \revision{process mining} has seen widespread adoption owing to process discovery and \revision{conformance checking}. On the one hand, process discovery is a set of algorithms that encode control-flow relations as a set of model elements and constraints according to a given modeling formalism \cite{aalst2022pmhandbook}; hereafter, we refer to the Petri net, a widespread modeling formalism. On the other hand, \revision{conformance checking} is an explainable set of algorithms that allows linking any deviations with the reference Petri net and providing the fitness measure, namely a measure of how much the Petri net fits the new event log \cite{aalst2022pmhandbook}. Many control-flow anomaly detection techniques based on \revision{conformance checking} (hereafter, \revision{conformance checking}-based techniques) use the fitness measure to determine whether an event log is anomalous \cite{bezerra2009pmad, bezerra2013adlogspais, myers2018icsadpm, pecchia2020applicationfailuresanalysispm}. 

The scientific literature also includes many \revision{conformance checking}-independent techniques for control-flow anomaly detection that combine specific types of trace encodings with machine/deep learning \cite{ko2023adsystematicreview, tavares2023pmtraceencoding}. Whereas these techniques are very effective, their explainability is challenging due to both the type of trace encoding employed and the machine/deep learning model used \cite{rawal2022trustworthyaiadvances,li2023explainablead}. Hence, in the following, we focus on the shortcomings of \revision{conformance checking}-based techniques to investigate whether it is possible to support the development of competitive control-flow anomaly detection techniques while maintaining the explainable nature of \revision{conformance checking}.
\begin{figure}[!t]
\centering
\includegraphics[width=\columnwidth]{images/HIGH_LEVEL_VIEW.png}
\caption{A high-level view of the proposed framework for combining \revision{process mining}-based feature extraction with dimensionality reduction for control-flow anomaly detection.}
\label{HIGH_LEVEL_VIEW}
\end{figure}

\subsection{Shortcomings of \revision{conformance checking}-based techniques}
Unfortunately, the detection effectiveness of \revision{conformance checking}-based techniques is affected by noisy data and low-quality Petri nets, which may be due to human errors in the modeling process or representational bias of process discovery algorithms \cite{bezerra2013adlogspais, pecchia2020applicationfailuresanalysispm, aalst2016pm}. Specifically, on the one hand, noisy data may introduce infrequent and deceptive control-flow relations that may result in inconsistent fitness measures, whereas, on the other hand, checking event logs against a low-quality Petri net could lead to an unreliable distribution of fitness measures. Nonetheless, such Petri nets can still be used as references to obtain insightful information for \revision{process mining}-based feature extraction, supporting the development of competitive and explainable \revision{conformance checking}-based techniques for control-flow anomaly detection despite the problems above. For example, a few works outline that token-based \revision{conformance checking} can be used for \revision{process mining}-based feature extraction to build tabular data and develop effective \revision{conformance checking}-based techniques for control-flow anomaly detection \cite{singh2022lapmsh, debenedictis2023dtadiiot}. However, to the best of our knowledge, the scientific literature lacks a structured proposal for \revision{process mining}-based feature extraction using the state-of-the-art \revision{conformance checking} variant, namely alignment-based \revision{conformance checking}.

\subsection{Contributions}
We propose a novel \revision{process mining}-based feature extraction approach with alignment-based \revision{conformance checking}. This variant aligns the deviating control flow with a reference Petri net; the resulting alignment can be inspected to extract additional statistics such as the number of times a given activity caused mismatches \cite{aalst2022pmhandbook}. We integrate this approach into a flexible and explainable framework for developing techniques for control-flow anomaly detection. The framework combines \revision{process mining}-based feature extraction and dimensionality reduction to handle high-dimensional feature sets, achieve detection effectiveness, and support explainability. Notably, in addition to our proposed \revision{process mining}-based feature extraction approach, the framework allows employing other approaches, enabling a fair comparison of multiple \revision{conformance checking}-based and \revision{conformance checking}-independent techniques for control-flow anomaly detection. Figure \ref{HIGH_LEVEL_VIEW} shows a high-level view of the framework. Business processes are monitored, and event logs obtained from the database of information systems. Subsequently, \revision{process mining}-based feature extraction is applied to these event logs and tabular data input to dimensionality reduction to identify control-flow anomalies. We apply several \revision{conformance checking}-based and \revision{conformance checking}-independent framework techniques to publicly available datasets, simulated data of a case study from railways, and real-world data of a case study from healthcare. We show that the framework techniques implementing our approach outperform the baseline \revision{conformance checking}-based techniques while maintaining the explainable nature of \revision{conformance checking}.

In summary, the contributions of this paper are as follows.
\begin{itemize}
    \item{
        A novel \revision{process mining}-based feature extraction approach to support the development of competitive and explainable \revision{conformance checking}-based techniques for control-flow anomaly detection.
    }
    \item{
        A flexible and explainable framework for developing techniques for control-flow anomaly detection using \revision{process mining}-based feature extraction and dimensionality reduction.
    }
    \item{
        Application to synthetic and real-world datasets of several \revision{conformance checking}-based and \revision{conformance checking}-independent framework techniques, evaluating their detection effectiveness and explainability.
    }
\end{itemize}

The rest of the paper is organized as follows.
\begin{itemize}
    \item Section \ref{sec:related_work} reviews the existing techniques for control-flow anomaly detection, categorizing them into \revision{conformance checking}-based and \revision{conformance checking}-independent techniques.
    \item Section \ref{sec:abccfe} provides the preliminaries of \revision{process mining} to establish the notation used throughout the paper, and delves into the details of the proposed \revision{process mining}-based feature extraction approach with alignment-based \revision{conformance checking}.
    \item Section \ref{sec:framework} describes the framework for developing \revision{conformance checking}-based and \revision{conformance checking}-independent techniques for control-flow anomaly detection that combine \revision{process mining}-based feature extraction and dimensionality reduction.
    \item Section \ref{sec:evaluation} presents the experiments conducted with multiple framework and baseline techniques using data from publicly available datasets and case studies.
    \item Section \ref{sec:conclusions} draws the conclusions and presents future work.
\end{itemize}

%\vspace{-5pt}
\section{Related Work}
%\vspace{-5pt}
\section{RELATED WORK}
\label{sec:relatedwork}
In this section, we describe the previous works related to our proposal, which are divided into two parts. In Section~\ref{sec:relatedwork_exoplanet}, we present a review of approaches based on machine learning techniques for the detection of planetary transit signals. Section~\ref{sec:relatedwork_attention} provides an account of the approaches based on attention mechanisms applied in Astronomy.\par

\subsection{Exoplanet detection}
\label{sec:relatedwork_exoplanet}
Machine learning methods have achieved great performance for the automatic selection of exoplanet transit signals. One of the earliest applications of machine learning is a model named Autovetter \citep{MCcauliff}, which is a random forest (RF) model based on characteristics derived from Kepler pipeline statistics to classify exoplanet and false positive signals. Then, other studies emerged that also used supervised learning. \cite{mislis2016sidra} also used a RF, but unlike the work by \citet{MCcauliff}, they used simulated light curves and a box least square \citep[BLS;][]{kovacs2002box}-based periodogram to search for transiting exoplanets. \citet{thompson2015machine} proposed a k-nearest neighbors model for Kepler data to determine if a given signal has similarity to known transits. Unsupervised learning techniques were also applied, such as self-organizing maps (SOM), proposed \citet{armstrong2016transit}; which implements an architecture to segment similar light curves. In the same way, \citet{armstrong2018automatic} developed a combination of supervised and unsupervised learning, including RF and SOM models. In general, these approaches require a previous phase of feature engineering for each light curve. \par

%DL is a modern data-driven technology that automatically extracts characteristics, and that has been successful in classification problems from a variety of application domains. The architecture relies on several layers of NNs of simple interconnected units and uses layers to build increasingly complex and useful features by means of linear and non-linear transformation. This family of models is capable of generating increasingly high-level representations \citep{lecun2015deep}.

The application of DL for exoplanetary signal detection has evolved rapidly in recent years and has become very popular in planetary science.  \citet{pearson2018} and \citet{zucker2018shallow} developed CNN-based algorithms that learn from synthetic data to search for exoplanets. Perhaps one of the most successful applications of the DL models in transit detection was that of \citet{Shallue_2018}; who, in collaboration with Google, proposed a CNN named AstroNet that recognizes exoplanet signals in real data from Kepler. AstroNet uses the training set of labelled TCEs from the Autovetter planet candidate catalog of Q1–Q17 data release 24 (DR24) of the Kepler mission \citep{catanzarite2015autovetter}. AstroNet analyses the data in two views: a ``global view'', and ``local view'' \citep{Shallue_2018}. \par


% The global view shows the characteristics of the light curve over an orbital period, and a local view shows the moment at occurring the transit in detail

%different = space-based

Based on AstroNet, researchers have modified the original AstroNet model to rank candidates from different surveys, specifically for Kepler and TESS missions. \citet{ansdell2018scientific} developed a CNN trained on Kepler data, and included for the first time the information on the centroids, showing that the model improves performance considerably. Then, \citet{osborn2020rapid} and \citet{yu2019identifying} also included the centroids information, but in addition, \citet{osborn2020rapid} included information of the stellar and transit parameters. Finally, \citet{rao2021nigraha} proposed a pipeline that includes a new ``half-phase'' view of the transit signal. This half-phase view represents a transit view with a different time and phase. The purpose of this view is to recover any possible secondary eclipse (the object hiding behind the disk of the primary star).


%last pipeline applies a procedure after the prediction of the model to obtain new candidates, this process is carried out through a series of steps that include the evaluation with Discovery and Validation of Exoplanets (DAVE) \citet{kostov2019discovery} that was adapted for the TESS telescope.\par
%



\subsection{Attention mechanisms in astronomy}
\label{sec:relatedwork_attention}
Despite the remarkable success of attention mechanisms in sequential data, few papers have exploited their advantages in astronomy. In particular, there are no models based on attention mechanisms for detecting planets. Below we present a summary of the main applications of this modeling approach to astronomy, based on two points of view; performance and interpretability of the model.\par
%Attention mechanisms have not yet been explored in all sub-areas of astronomy. However, recent works show a successful application of the mechanism.
%performance

The application of attention mechanisms has shown improvements in the performance of some regression and classification tasks compared to previous approaches. One of the first implementations of the attention mechanism was to find gravitational lenses proposed by \citet{thuruthipilly2021finding}. They designed 21 self-attention-based encoder models, where each model was trained separately with 18,000 simulated images, demonstrating that the model based on the Transformer has a better performance and uses fewer trainable parameters compared to CNN. A novel application was proposed by \citet{lin2021galaxy} for the morphological classification of galaxies, who used an architecture derived from the Transformer, named Vision Transformer (VIT) \citep{dosovitskiy2020image}. \citet{lin2021galaxy} demonstrated competitive results compared to CNNs. Another application with successful results was proposed by \citet{zerveas2021transformer}; which first proposed a transformer-based framework for learning unsupervised representations of multivariate time series. Their methodology takes advantage of unlabeled data to train an encoder and extract dense vector representations of time series. Subsequently, they evaluate the model for regression and classification tasks, demonstrating better performance than other state-of-the-art supervised methods, even with data sets with limited samples.

%interpretation
Regarding the interpretability of the model, a recent contribution that analyses the attention maps was presented by \citet{bowles20212}, which explored the use of group-equivariant self-attention for radio astronomy classification. Compared to other approaches, this model analysed the attention maps of the predictions and showed that the mechanism extracts the brightest spots and jets of the radio source more clearly. This indicates that attention maps for prediction interpretation could help experts see patterns that the human eye often misses. \par

In the field of variable stars, \citet{allam2021paying} employed the mechanism for classifying multivariate time series in variable stars. And additionally, \citet{allam2021paying} showed that the activation weights are accommodated according to the variation in brightness of the star, achieving a more interpretable model. And finally, related to the TESS telescope, \citet{morvan2022don} proposed a model that removes the noise from the light curves through the distribution of attention weights. \citet{morvan2022don} showed that the use of the attention mechanism is excellent for removing noise and outliers in time series datasets compared with other approaches. In addition, the use of attention maps allowed them to show the representations learned from the model. \par

Recent attention mechanism approaches in astronomy demonstrate comparable results with earlier approaches, such as CNNs. At the same time, they offer interpretability of their results, which allows a post-prediction analysis. \par



%\vspace{-15pt}
\section{Preliminaries}\label{sec:preliminary}
%\vspace{-8pt}
% Consider a lasso optimization procedure with potentially distinct regularization penalties:
% \begin{align}
%     \hat{\beta} = \arg\min_{\beta}\{\|y-X\beta\|^2_2+\sum_{i=1}^{N}\lambda_i|\beta_i|\}.
% \end{align}
\subsection{Supervised Data-Driven Learning}\label{subsec:supervised}
We consider a generic data-driven supervised learning procedure. Given a dataset \( \mathcal{D} \) consisting of \( n \) data points \( (x_i, y_i) \in \mathcal{X} \times \mathcal{Y} \) drawn from an underlying distribution \( p(\cdot|\theta) \), our goal is to estimate parameters \( \theta \in \Theta \) through a learning procedure, defined as \( f: (\mathcal{X} \times \mathcal{Y})^n \rightarrow \Theta \) 
that minimizes the predictive error on observed data. 
Specifically, the learning objective is defined as follows:
\begin{align}
\hat{\theta}_f := f(\mathcal{D}) = \arg\min_{\theta} \mathcal{L}(\theta, \mathcal{D}),
\end{align}
where \( \mathcal{L}(\cdot,\mathcal{D}) := \sum_{i=1}^{n} \mathcal{L}(\cdot, (x_i, y_i))\), and $\mathcal{L}$ is a loss function quantifying the error between predictions and true outcomes. 
Here, $\hat{\theta}_f$ is the parameter that best explains the observed data pairs \( (x_i, y_i) \) according to the chosen loss function \( \mathcal{L} (\cdot) \).

\paragraph{Feature Selection.}
Feature selection aims to improve model \( f \)'s predictive performance while minimizing redundancy. 
%Formally, given data \( X \), response \( y \), feature set \( \mathcal{F} \), loss function \( \mathcal{L}(\cdot) \), and a feature limit \( k \), the objective is:
% \begin{align}
% \mathcal{S}^* = \arg \min_{\mathcal{S} \subseteq \mathcal{F}, |\mathcal{S}| \leq k} \mathcal{L}(y, f(X_\mathcal{S})) + \lambda R(\mathcal{S}),
% \end{align}
% where \( X_\mathcal{S} \) is the submatrix of \( X \) for selected features \( \mathcal{S} \), \( \lambda \) is a regularization parameter, and \( R(\mathcal{S}) \) penalizes feature redundancy.
 State-of-the-art techniques fall into four categories: (i) filter methods, which rank features based on statistical properties like Fisher score \citep{duda2001pattern,song2012feature}; (ii) wrapper methods, which evaluate model performance on different feature subsets \citep{kohavi1997wrappers}; (iii) embedded methods, which integrate feature selection into the learning process using techniques like regularization \citep{tibshirani1996LASSO,lemhadri2021lassonet}; and (iv) hybrid methods, which combine elements of (i)-(iii) \citep{SINGH2021104396,li2022micq}. This paper focuses on embedded methods via Lasso, benchmarking against approaches from (i)-(iii).

\subsection{Language Modeling}
% The objective of language modeling is to learn a probability distribution \( p_{LM}(x) \) over sequences of text \( x = (X_1, \ldots, X_{|x|}) \), such that \( p_{LM}(x) \approx p_{text}(x) \), where \( p_{text}(x) \) represents the true distribution of natural language. This process involves estimating the likelihood of token sequences across variable lengths and diverse linguistic structures.
% Modern large language models (LLMs) are trained on vast datasets spanning encyclopedias, news, social media, books, and scientific papers \cite{gao2020pile}. This broad training enables them to generalize across domains, learn contextual knowledge, and perform zero-shot learning—tackling new tasks using only task descriptions without fine-tuning \cite{brown2020gpt3}.
Language modeling aims to approximate the true distribution of natural language \( p_{\text{text}}(x) \) by learning \( p_{\text{LM}}(x) \), a probability distribution over text sequences \( x = (X_1, \ldots, X_{|x|}) \). Modern large language models, trained on diverse datasets \citep{gao2020pile}, exhibit strong generalization across domains, acquire contextual knowledge, and perform zero-shot learning—solving new tasks using only task descriptions—or few-shot learning by leveraging a small number of demonstrations \citep{brown2020gpt3}.
\paragraph{Retrieval-Augmented Generation (RAG).} Retrieval-Augmented Generation (RAG) enhances the performance of generative language models by  integrating a domain-specific information retrieval process  \citep{lewis2020retrieval}. The RAG framework comprises two main components: \textit{retrieval}, which extracts relevant information from external knowledge sources, and \textit{generation}, where an LLM generates context-aware responses using the prompt combined with the retrieved context. Documents are indexed through various databases, such as relational, graph, or vector databases \citep{khattab2020colbert, douze2024faiss, peng2024graphretrievalaugmentedgenerationsurvey}, enabling efficient organization and retrieval via algorithms like semantic similarity search to match the prompt with relevant documents in the knowledge base. RAG has gained much traction recently due to its demonstrated ability to reduce incidence of hallucinations and boost LLMs' reliability as well as performance \citep{huang2023hallucination, zhang2023merging}. 
 
% image source: https://medium.com/@bindurani_22/retrieval-augmented-generation-815c1ae438d8
\begin{figure}
    \centering
\includegraphics[width=1.03\linewidth]{fig/fig1.pdf}
\vspace{-0.6cm}
\scriptsize 
    \caption{Retrieval Augmented Generation (RAG) based $\ell_1$-norm weights (penalty factors) for Lasso. Only feature names---no training data--- are included in LLM prompt.} 
    \label{fig:rag}
\end{figure}
% However, for the RAG model to be effective given the input token constraints of the LLM model used, we need to effectively process the retrieval documents through a procedure known as \textit{chunking}.

\subsection{Task-Specific Data-Driven Learning}
LLM-Lasso aims to bridge the gap between data-driven supervised learning and the predictive capabilities of LLMs trained on rich metadata. This fusion not only enhances traditional data-driven methods by incorporating key task-relevant contextual information often overlooked by such models, but can also be especially valuable in low-data regimes, where the learning algorithm $f:\mathcal{D}\rightarrow\Theta$ (seen as a map from datasets $\mathcal{D}$ to the space of decisions $\Theta$) is susceptible to overfitting.

The task-specific data-driven learning model $\tilde{f}:\mathcal{D}\times\mathcal{D}_\text{meta}\rightarrow\Theta$ can be described as a metadata-augmented version of $f$, where a link function $h(\cdot)$ integrates metadata (i.e. $\mathcal{D}_\text{meta}$) to refine the original learning process. This can be expressed as:
\[
\tilde{f}(\mathcal{D}, \mathcal{D}_\text{meta}) := \mathcal{T}(f(\mathcal{D}),  h(\mathcal{D}_{\text{meta}})),
\]
where the functional $\mathcal{T}$ takes the original learning algorithm $f(\mathcal{D})$ and transforms it into a task-specific learning algorithm $\tilde{f}(\mathcal{D}, \mathcal{D}_\text{meta})$ by incorporating the metadata $\mathcal{D}_\text{meta}$. 
% In particular, the link function $h(\mathcal{D}_{\text{meta}})$ provides a structured mechanism summarizing the contextual knowledge.

There are multiple approaches to formulate $\mathcal{T}$ and $h$.
%to ``inform" the data-driven model $f$ of %meta knowledge. 
For instance, LMPriors \citep{choi2022lmpriorspretrainedlanguagemodels} designed $h$ and $\mathcal{T}$ such that $h(\mathcal{D}_{\text{meta}})$ first specifies which features to retain (based on a probabilistic prior framework), and then $\mathcal{T}$ keeps the selected features and removes all the others from the original learning objective of $f$. 
Note that this approach inherently is restricted as it selects important features solely based on $\mathcal{D}_\text{meta}$ without seeing $\mathcal{D}$.

In contrast, we directly embed task-specific knowledge into the optimization landscape through regularization by introducing a structured inductive bias. This bias guides the learning process toward solutions that are consistent with metadata-informed insights, without relying on explicit probabilistic modeling. Abstractly, this can be expressed as:
\begin{align}
    \!\!\!\!\!\hat{\theta}_{\tilde{f}} := \tilde{f}(\mathcal{D},\mathcal{D}
    _\text{meta})= \arg\min_{\theta} \mathcal{L}(\theta, \mathcal{D}) + \lambda R(\theta, \mathcal{D}_{\text{meta}}),
\end{align}
where \( \lambda \) is a regularization parameter, \( R(\cdot) \) is a regularizer, and $\theta$ is the prediction parameter.
%We explain our framework with more details in the following section.


% Our research diverges from both aforementioned approaches by positioning the LLM not as a standalone feature selector but as an enhancement to data-driven models through an embedded feature selection method, L-LASSO. L-LASSO incorporates domain expertise—auxiliary natural language metadata about the task—via the LLM-informed LASSO penalty, which is then used in statistical models to enhance predictive performance. This method integrates the rich, context-sensitive insights of LLMs with the rigor and transparency of statistical modeling, bridging the gap between data-driven and knowledge-driven feature selection approaches. To approach this task, we need to tackle two key components: (i). train an LLM that is expert in the task-specific knowledge; (ii). inform data-driven feature selector LASSO with LLM knowledge.

% In practice, this involves combining techniques like prompt engineering and data engineering to develop an effective framework for integrating metadata into existing data-driven models. We will go through this in detail in Section \ref{mthd} and \ref{experiment}.



\section{\ourmethod Framework}
%\vspace{-5pt}
\section{Proposed Method}
\begin{figure}[t]
\begin{center}
\includegraphics[scale=0.4]{figure/converter_arc.png}
\end{center}
\caption{Converter architecture.}
\label{fig:converter_architecture}
\end{figure}

\subsection{Synvolution}
Let $\mathbf{A} = \mathbf{X}\mathbf{W}_{\text{Q}}(\mathbf{X}\mathbf{W}_{\text{K}})^{\mathrm{T}}/{\tau} \in \mathbb{R}^{n \times n}$ be an affinity matrix, the attention matrix $\mathcal{A} = \mathrm{softmax}(\mathbf{A})$ in Equation~\ref{eq:shsa} can be reformulated as a right stochastic normalized affinity form $\mathrm{SA}(\mathbf{X}) = \widetilde{\mathbf{D}}^{-1}\widetilde{\mathbf{A}}\mathbf{X}\mathbf{W}_{\text{V}}$, where $\widetilde{\mathbf{A}} = \exp(\mathbf{A}) \in \mathbb{R}^{n \times n}$ is defined as the affinity matrix after the element-wise exponentiation operation $\exp(\cdot)$, and $\widetilde{\mathbf{D}}_{u,u} = \sum_{v}\widetilde{\mathbf{A}}_{u,v} \in \mathbb{R}^{n \times n}$ is the corresponding degree matrix. When treating $\widetilde{\mathbf{A}}$ as a digraph adjacency matrix, we found that self-attention closely resembles digraph convolution. First, each element in either an attention matrix or a DGSO can be considered as a similarity from source entity to target entity. Second, both self-attention and digraph convolution can be degenerated to graph convolution form. For self-attention, this occurs when the query matrix is equal to the key matrix in each head, resulting in unidirectional symmetric self-attention. Similarly, for digraph convolution, the achievement of graph convolution can be implemented by symmetrizing the adjacency matrix of a digraph. Third, the softmax function in self-attention results in a row-wise normalized digraph adjacency form. 

Since digraph convolution closely resembles self-attention, we investigated replacing self-attention with digraph convolution. Under this hypothesis, a Transformer can be converted into a DGNN form. Based on this insight, we propose Converter. In this work, we decide to construct the DGSO directly. We develop a learnable unitary matrix as a DGSO through the inverse process of eigendecomposition. Our method consists of two phases. In the first phase, we synthesize the required eigenvalues through the following process.
\begin{equation}\label{eq:eigenvalue_gap}
\centering
\eu^{\ramuno\mathbf{\Lambda}}
= \exp\left[\ramuno\cdot\mathrm{diag}\left(\mathrm{pool}_{\text{avg}}\left[\mathrm{SIREN}(\mathbf{X})\right]\right)\right].
\end{equation}
Here, $\mathrm{SIREN}$ represents a 2-layer MLP with the sine function~\citep{NEURIPS2020_53c04118}, $\mathrm{pool}_{\text{avg}}(\cdot)$ is a 1D global average pooling, and $\mathrm{diag}(\cdot)$ is a diagonalize operation. We adopt the sine function because it demonstrates a remarkable ability in signal processing~\citep{NEURIPS2020_53c04118}.

In the second phase, we focus on constructing the necessary unitary eigenvector matrix through the inverse process of LQ factorization. Based on the Givens rotation method~\citep{doi:10.1137/0106004}, an arbitrary square matrix $\bm{\Phi} \in \mathbb{C}^{N \times N}$ can be decomposed into a product of a lower triangular matrix and Givens rotation matrices. Hence, we have
\begin{equation}\label{eq:givens_lq}
\centering
\bm{\Phi} 
= \mathbf{L}\mathbf{Q} 
= \mathbf{L}\left(\prod_{j=N}^{2}\prod_{i=j-1}^{1}\mathbf{G}_{i,j}\right),
\end{equation}
where $\mathbf{L} \in \mathbb{C}^{N \times N}$ is a lower triangular matrix, and $\mathbf{G}_{i,j} \in \mathbb{C}^{N \times N}$ is a Givens rotation matrix that resembles an identity matrix with the exception of the elements
\begin{equation}\label{eq:givens_rotation_matrix}
\centering
\begin{bmatrix}
G_{ii} & G_{ij} \\
G_{ji} & G_{jj}
\end{bmatrix} =
\begin{bmatrix}
\overline{c} & -s \\
\overline{s} & c
\end{bmatrix} =
\begin{bmatrix}
\eu^{-\ramuno(\frac{\alpha+\beta}{2})}\cos{(\frac{\gamma}{2})} 
& -\eu^{\ramuno(\frac{\alpha-\beta}{2})}\sin{(\frac{\gamma}{2})} \\
\eu^{-\ramuno(\frac{\alpha-\beta}{2})}\sin{(\frac{\gamma}{2})} 
& \eu^{\ramuno(\frac{\alpha+\beta}{2})}\cos{(\frac{\gamma}{2})}
\end{bmatrix},
\end{equation}
which characterized by parameters $\alpha$, $\beta$, and $\gamma \in [0, 2\pi]$. This methodology necessitates ${\left(N(N-1)\right)}/{2}$ pairs of Givens rotation matrices, i.e., it requires $\mathcal{O}(N^{2})$ space complexity. By reorganizing Givens rotation matrices, inserting permutation matrices, and repeating the patten, we have
\begin{equation}\label{eq:givens_lhhp}
\centering
\begin{split}
\bm{\Phi} 
&= \mathbf{L}\left(\prod_{l=1}^{L}\left(\prod_{i=N-1}^{1}\mathbf{G}^{(l)}_{i,i+1}\right)\left(\prod_{j=1}^{N-1}\mathbf{G}^{(l)}_{j,j+1}\right)\mathbf{P}^{(l)}\right)\\
&= \mathbf{L}\left(\prod_{l=1}^{L}\mathbf{H}^{(l)}_{\text{l}}\mathbf{H}^{(l)}_{\text{u}}\mathbf{P}^{(l)}\right).
\end{split}
\end{equation}
Here, $\mathbf{H}^{(l)}_{\text{l}} \in \mathbb{C}^{N \times N}$ is a lower unitary Hessenberg matrix, $\mathbf{H}^{(l)}_{\text{u}} \in \mathbb{C}^{N \times N}$ is an upper unitary Hessenberg matrix, and $\mathbf{P}^{(l)} \in \mathbb{R}^{N \times N}$ is a permutation matrix that either learnable~\citep{mena2018learning}, fixed~\citep{pmlr-v162-dao22a}, or even an identity matrix $\mathbf{I}_{N} \in \mathbb{R}^{N \times N}$.

We refer to Equation~\ref{eq:givens_lhhp} as the order-$L$ LHHP parametrization, $L$-LHHP for short, denoted by $\bm{\Phi}_{L-\text{LHHP}}$. In particular, when the lower triangular matrix $\mathbf{L}$ degenerates to a diagonal matrix $\mathbf{D}$, we term this pattern the order-$L$ DHHP parametrization, $L$-DHHP for short, denoted by $\bm{\Phi}_{L-\text{DHHP}}$. It requires $2L(N-1)$ pairs of Givens rotation matrices, which means the space complexity is $\mathcal{O}(LN)$. We observed that each unitary factor matrix resulting from the multiplication of lower and upper unitary Hessenberg matrices in the order-$L$ DHHP parametrization is dense rather than sparse, unlike the schemes proposed in \citep{KHALITOV2022160}. Since our method is based on the Givens rotation method, we make Assumption~\ref{asmp:dhhp_givens_num_upper_bound}. Under this assumption, we can establish the following propositions.

\begin{assumption}\label{asmp:dhhp_givens_num_upper_bound}
For constructing an arbitrary $N \times N$ dense unitary matrix, at most $\lceil\frac{N}{4}\rceil$ orders are sufficient for $L$-DHHP.
\end{assumption}

\begin{proposition}\label{prop:dhhp_universal}
$L$-DHHP captures the discrete unitary transforms, including discrete Fourier transform (DFT), the discrete Walsh–Hadamard transform (DWHT), the discrete cosine transform (DCT), the discrete sine transform (DST), and their inverses exactly.
\end{proposition}

\begin{proposition}\label{prop:dhhp_time_complexity}
Given an input signal $\mathbf{x} \in \mathbb{C}^{N}$ and an output signal $\mathbf{y} \in \mathbb{C}^{N}$, the time complexity of $L$-DHHP as a discrete unitary transform with the fast implementation as $\mathbf{y} = \bm{\Phi}\mathbf{x}$ is $\mathcal{O}(LN\log{N})$.
\end{proposition}

\begin{proposition}\label{prop:dhhp_full_rank}
$L$-DHHP is full-rank if and only if the diagonal matrix $\mathbf{D}$ is unitary.
\end{proposition}

We refer to this self-attention alternative as Synvolution:
\begin{equation}\label{eq:synvolution}
\centering
\mathrm{Synv}(\mathbf{X}\mathbf{W}_{\text{V}})
= \bm{\Phi}^{-1}\left[\exp(\ramuno\mathbf{\Lambda})\odot(\bm{\Phi}\mathbf{X}\mathbf{W}_{\text{V}})\right],
\end{equation}
where $\mathbf{X}\mathbf{W}_{\text{V}} \in \mathbb{C}^{N \times D}$ is denoted as the value matrix. Unlike FFT-based convolution~\citep{mathieu2013fast}, where the discrete unitary matrix is fixed and data-independent, the required parameters in $L$-DHHP are learnable and data-dependent. We adopt a similar processing method to that described in Equation~\ref{eq:eigenvalue_gap} to obtain the synthetic eigenvector matrix. For convenience, we set $L=1$, $\mathbf{P}^{(1)} = \mathbf{I}_{N}$, and $\mathbf{D}$ is unitary to obtain a dense unitary matrix that serves as the desired unitary eigenvector matrix. More details about the fast implementation of $1$-DHHP as a discrete unitary transform are provided in the appendix.

% We recognize that the core essence of self-attention can be characterized as a fast weight programmer~\citep{pmlr-v139-schlag21a}. In brief, it involves a non-linear projection of the input feature into a weighted tensor, known as a fast weight, which guides either the input feature itself or its linear projection. This operation is not unique to self-attention, but is also applicable to non-local operations~\citep{Wang_2018_CVPR}. For instance, channel attention~\citep{Hu_2018_CVPR}, a well-known instance of non-local operation, commonly employs the squeeze-and-excitation technique. The first squeeze operation aggregates the frequency information from the spatial extent, whereas the second excitation operation redistributes the gathered feature responses back to the local features. Another example is Involution~\citep{Li_2021_CVPR}.

\subsection{Kernelution}

\begin{figure*}[t]
\begin{center}
\includegraphics[scale=0.4]{figure/kernelution.png}
\end{center}
\caption{Illustration of the entire Kernelution process.}
\label{fig:kernelution}
\end{figure*}

\subsubsection{Chebyshev Polynomial Interpolation}
The multi-head operation, a common approach to enhance performance in Transformers, lacks solid theoretical support. In the contrast, FIR filters have a theoretical support in spectral graph theory~\citep{chung1997spectral}. Let $f(x)$ be the target function, then our goal is to approximate it with the smallest round-off error. Directly manipulating orthogonal polynomials to filter complex-valued signals is challenging, but using them to represent the argument function of signals is straightforward. To achieve it, we can choose an arbitrary orthogonal polynomial basis such as the Bernstein basis, Jacobi basis (including Chebyshev, Gegenbauer, Legendre, and Zernike bases), or even monomial basis. Consider the Chebyshev basis as an example. Given an arbitrary continuous function $f(x) \in C([-1,1])$ and a truncated Chebyshev polynomial $p$ with $K$ orders, then the target function $f(x)$ can be approximated as
\begin{equation}\label{eq:cpi}
\centering
f(x) \approx p(x) = \frac{1}{2}{\mu}_{0} + \sum_{k=1}^{K}{\mu}_{k}{T}_{k}(x),
\end{equation}
where $\mu_{k} \approx \frac{2}{K+1}\sum_{j=0}^{K}f(x_{j})T_{k}(x_{j})$ is the Chebyshev coefficient, and $x_{j}$ is the sampling Chebyshev node. This technique is termed the Chebyshev polynomial interpolation (CPI)~\citep{10.1137/1.9781611975949}. The operation on Chebyshev polynomial interpolation is considerably straightforward since Chebyshev polynomials are isomorphic with Fourier series. For differentiable or analytic functions, we have the following theorems.

\begin{theorem}[CPI for differentiable functions~\citep{10.1137/1.9781611975949}]\label{theorem:cpi_diff}
Let $\upsilon \geq 0$ and $\kappa > \upsilon$ be integers. Consider a function $f(x)$ whose derivatives up to order $\upsilon-1$ are absolutely continuous on $[-1, 1]$, and suppose $\lVert{\frac{\diff^{\upsilon}}{\diff{x}^{\upsilon}} f(x)}\rVert_{1} = \Upsilon$. For the $\kappa$-th degree Chebyshev interpolant $p(x)$, the following bounds hold: (1)~$\lVert{\mu_{\kappa}}\rVert \leq \frac{2\Upsilon}{\pi(\kappa-\upsilon)^{\upsilon+1}}$. (2)~$\lVert{f(x) - p(x)}\rVert \leq \frac{4\Upsilon}{\pi\upsilon(\kappa-\nu)^{\upsilon}}$.
\end{theorem}

\begin{theorem}[CPI for analytic functions~\citep{10.1137/1.9781611975949}]\label{theorem:cpi_ana}
Let $\kappa \geq 1$ be an integer and $f(x)$ an analytic function on $[-1, 1]$ that extends analytically to the open Bernstein ellipse $E_{\rho}$ with $\lVert{f(x)}\rVert \leq M$ for some $M$. For the $\kappa$-th degree Chebyshev interpolant $p(x)$, the following bounds hold: (1)~$\lVert{\mu_{0}}\rVert \leq M$. (2)~$\lVert{\mu_{\kappa}}\rVert \leq 2M\rho^{-\kappa}$. (3)~$\lVert{f(x)-p(x)}\rVert \leq \frac{4M\rho^{-\kappa}}{\rho-1}$.
\end{theorem}

Both Theorem~\ref{theorem:cpi_diff} and Theorem~\ref{theorem:cpi_ana} tell us that we can utilize the Chebyshev polynomial filter to approximate any continuous target function that lies in the range of $C[-1, 1]$ with a small round-off error.

\subsubsection{Kernel Polynomial Method}
In reality, the target function is probably discontinuous or singular in the polynomial interpolation interval. In this situation, the accuracy of the Chebyshev polynomial interpolation reduces to $\mathcal{O}(1)$ near discontinuities or singularities. Sufficiently far away from discontinuities or singularities, the convergence will be slowed to $\mathcal{O}(K^{-1})$. During the approximation process, oscillations will be present near discontinuities or singularities and they will not diminish as $K \to \infty$. This type of oscillation is termed the Gibbs oscillation, and this situation is known as the Gibbs phenomenon~\citep{Hewitt1979}.

To mitigate Gibbs oscillations, we apply a Gibbs damping factor $g_{k}$, which represented as a function of $\frac{k}{K+1}$, to each term of the Chebyshev polynomials. For any $f(x)$, we have
\begin{equation}\label{eq:kpm}
\centering
{f(x)}\approx{p}_{\text{KP}}(x) = \frac{1}{2}g_{0}\mu_{0} + \sum_{k=1}^{K}g_{k}\mu_{k}T_{k}(x).
\end{equation}
This modification of the Chebyshev coefficients is equivalent to the convolution of $p(x)$ with a kernel $\mathcal{K}(x,x_{0}) = \frac{2}{\pi\sqrt{1-x^{2}}}\left(\frac{1}{2}g_{0} + \sum_{k=1}^{K}g_{k}T_{k}(x)T_{k}(x_{0})\right)$ that $p_{\text{KP}}(x) = \int_{-1}^{1}\mathcal{K}(x,x_{0})f(x_{0})\diff{x_{0}}$. Thus, this method is also called the kernel polynomial method. It is widely employed in computational physics for calculating the density of states and other spectral properties of large quantum systems. 

Gibbs damping factors are a family of coefficients that satisfy three conditions: (1)~$g_{k} > 0$. (2)~$g_{0} = 1$. (3)~$\lim_{K \to \infty} {g_{1} \to 1}$. The conditions (1) and (2) are particularly valuable in real-world applications~\citep{RevModPhys.78.275,Weiße2008}. The first condition ensures that approximations of positive quantities remain positive, while the second conserves the integral of the expanded function $\int_{-1}^{1}p_{\text{KPM}}(x)\diff{x} = \int_{-1}^{1}f(x)\diff{x}$. Notably, $g_{k} = 1$ is the simplest Gibbs damping factor attributed to the Dirichlet kernel. More details about Gibbs damping factors are in the appendix.

Clearly, finding an appropriate kernel is crucial for approximation, as it determines whether the round-off error is minimized or not. As indicated in \citep{RevModPhys.78.275,Weiße2008}, kernel choices are data-dependent. More specifically, given a target function, we need to match an appropriate kernel and manually tune its hyperparameters (if the kernel has any) based on experience. Since the target function is unknown, we relax each $\mu_{k}$ with a learnable parameter $w_{k}$. The effectiveness of the Gibbs damping factors lie in their ability to reduce the weight of each term of the Chebyshev coefficients, thereby mitigating the contributions of higher-order terms. Based on this observation, and in order to prevent over-fitting, we propose the following loss function which is named the kernel polynomial loss (KPL):
\begin{equation}\label{eq:kpl}
\centering
\mathcal{L}_{\text{KP}} = \int_{-1}^{1}\abs{\frac{\diff{f(x)}}{\diff{x}}}^{2}\diff{x} \approx \sum_{k=1}^{K}{\pi}{k}^{2}\abs{w_{k}}^{2}.
\end{equation}
This results in an intuitive penalty applied to the Chebyshev coefficients, with higher order Chebyshev coefficients incurring greater penalties than the lower ones. It causes the Chebyshev polynomial interpolation with the kernel polynomial loss to simulate the kernel polynomial method with a learnable kernel. We apply the kernel polynomial method with Synolution, which turns out what we call Kernelution. The corresponding formula is defined as
\begin{equation}\label{eq:kernelution}
\centering
\mathrm{Kern}(\mathbf{X}\mathbf{W}_{\text{V}})
= \bm{\Phi}^{-1}\left[\exp\left(\ramuno\cdot{p}_{\text{KP}}(\mathbf{\Lambda})\right)\odot(\bm{\Phi}\mathbf{X}\mathbf{W}_{\text{V}})\right].
\end{equation}
It is worth noting that the kernel polynomial method is not the only operation compatible with Synvolution. Depending on practical requirements, Synvolution can also be made compatible with the multi-head operation, similar to other attention mechanisms. This means Synvolution can be equipped as a substitute for self-attention in Transformer-based models.

\subsection{Gated Feed-Forward Network and PostScaleNorm}
Both Synvolution and Kernelution effectively represent the direction and model the relationship between feature tokens in the spectral domain. A tricky problem is that the output of either Synvolution or Kernelution is complex-valued, whereas the labels are real-valued. This conflict motivates us to design a layer that maps a complex-valued tensor into a real-valued tensor. We propose a Gated Feed-Forward Network (GFNN) to solve this issue.
\begin{equation}\label{eq:gffn}
\centering
\mathrm{GFFN}(\mathbf{X}) = \left[\mathrm{softplus}(\Re(\mathbf{X})\mathbf{W}_{\Re})\odot\tanh(\Im(\mathbf{X})\mathbf{W}_{\Im})\right]\mathbf{W}_{\text{O}},
\end{equation}
where $\mathbf{W}_{\Re} \in \mathbb{R}^{D \times D_\text{hid}}$, $\mathbf{W}_{\Im} \in \mathbb{R}^{D \times D_\text{hid}}$ and $\mathbf{W}_{\text{O}} \in \mathbb{R}^{D_\text{hid} \times D}$ are trainable weight matrices. We let the real part to learn the magnitude, and the imaginary part to learn the sign. Besides, we apply the PostNorm architecture~\citep{wang-etal-2019-learning-deep} with ScaleNorm~\citep{nguyen-salazar-2019-transformers} across the whole model, namely PostScaleNorm. Specifically, we apply $\mathrm{ScaleNorm}(\mathbf{Z} + \zeta \cdot \Re(\mathbf{Z}) + (1 - \zeta) \cdot \Im(\mathbf{Z}))$ for a complex-valued signal $\mathbf{Z}$, where $\zeta \in [0, 1]$ is a learnable parameter.

\section{Experiments}
%\vspace{-5pt}
\section{Experiments}
\label{sec:exp}
Following the settings in Section \ref{sec:existing}, we evaluate \textit{NovelSum}'s correlation with the fine-tuned model performance across 53 IT datasets and compare it with previous diversity metrics. Additionally, we conduct a correlation analysis using Qwen-2.5-7B \cite{yang2024qwen2} as the backbone model, alongside previous LLaMA-3-8B experiments, to further demonstrate the metric's effectiveness across different scenarios. Qwen is used for both instruction tuning and deriving semantic embeddings. Due to resource constraints, we run each strategy on Qwen for two rounds, resulting in 25 datasets. 

\subsection{Main Results}

\begin{table*}[!t]
    \centering
    \resizebox{\linewidth}{!}{
    \begin{tabular}{lcccccccccc}
    \toprule
    \multirow{3}*{\textbf{Diversity Metrics}} & \multicolumn{10}{c}{\textbf{Data Selection Strategies}} \\
    \cmidrule(lr){2-11}
    & \multirow{2}*{\textbf{K-means}} & \multirow{2}*{\vtop{\hbox{\textbf{K-Center}}\vspace{1mm}\hbox{\textbf{-Greedy}}}}  & \multirow{2}*{\textbf{QDIT}} & \multirow{2}*{\vtop{\hbox{\textbf{Repr}}\vspace{1mm}\hbox{\textbf{Filter}}}} & \multicolumn{5}{c}{\textbf{Random}} & \multirow{2}{*}{\textbf{Duplicate}} \\ 
    \cmidrule(lr){6-10}
    & & & & & \textbf{$\mathcal{X}^{all}$} & ShareGPT & WizardLM & Alpaca & Dolly &  \\
    \midrule
    \rowcolor{gray!15} \multicolumn{11}{c}{\textit{LLaMA-3-8B}} \\
    Facility Loc. $_{\times10^5}$ & \cellcolor{BLUE!40} 2.99 & \cellcolor{ORANGE!10} 2.73 & \cellcolor{BLUE!40} 2.99 & \cellcolor{BLUE!20} 2.86 & \cellcolor{BLUE!40} 2.99 & \cellcolor{BLUE!0} 2.83 & \cellcolor{BLUE!30} 2.88 & \cellcolor{BLUE!0} 2.83 & \cellcolor{ORANGE!20} 2.59 & \cellcolor{ORANGE!30} 2.52 \\    
    DistSum$_{cosine}$  & \cellcolor{BLUE!30} 0.648 & \cellcolor{BLUE!60} 0.746 & \cellcolor{BLUE!0} 0.629 & \cellcolor{BLUE!50} 0.703 & \cellcolor{BLUE!10} 0.634 & \cellcolor{BLUE!40} 0.656 & \cellcolor{ORANGE!30} 0.578 & \cellcolor{ORANGE!10} 0.605 & \cellcolor{ORANGE!20} 0.603 & \cellcolor{BLUE!10} 0.634 \\
    Vendi Score $_{\times10^7}$ & \cellcolor{BLUE!30} 1.70 & \cellcolor{BLUE!60} 2.53 & \cellcolor{BLUE!10} 1.59 & \cellcolor{BLUE!50} 2.23 & \cellcolor{BLUE!20} 1.61 & \cellcolor{BLUE!30} 1.70 & \cellcolor{ORANGE!10} 1.44 & \cellcolor{ORANGE!20} 1.32 & \cellcolor{ORANGE!10} 1.44 & \cellcolor{ORANGE!30} 0.05 \\
    \textbf{NovelSum (Ours)} & \cellcolor{BLUE!60} 0.693 & \cellcolor{BLUE!50} 0.687 & \cellcolor{BLUE!30} 0.673 & \cellcolor{BLUE!20} 0.671 & \cellcolor{BLUE!40} 0.675 & \cellcolor{BLUE!10} 0.628 & \cellcolor{BLUE!0} 0.591 & \cellcolor{ORANGE!10} 0.572 & \cellcolor{ORANGE!20} 0.50 & \cellcolor{ORANGE!30} 0.461 \\
    \midrule    
    \textbf{Model Performance} & \cellcolor{BLUE!60}1.32 & \cellcolor{BLUE!50}1.31 & \cellcolor{BLUE!40}1.25 & \cellcolor{BLUE!30}1.05 & \cellcolor{BLUE!20}1.20 & \cellcolor{BLUE!10}0.83 & \cellcolor{BLUE!0}0.72 & \cellcolor{ORANGE!10}0.07 & \cellcolor{ORANGE!20}-0.14 & \cellcolor{ORANGE!30}-1.35 \\
    \midrule
    \midrule
    \rowcolor{gray!15} \multicolumn{11}{c}{\textit{Qwen-2.5-7B}} \\
    Facility Loc. $_{\times10^5}$ & \cellcolor{BLUE!40} 3.54 & \cellcolor{ORANGE!30} 3.42 & \cellcolor{BLUE!40} 3.54 & \cellcolor{ORANGE!20} 3.46 & \cellcolor{BLUE!40} 3.54 & \cellcolor{BLUE!30} 3.51 & \cellcolor{BLUE!10} 3.50 & \cellcolor{BLUE!10} 3.50 & \cellcolor{ORANGE!20} 3.46 & \cellcolor{BLUE!0} 3.48 \\ 
    DistSum$_{cosine}$ & \cellcolor{BLUE!30} 0.260 & \cellcolor{BLUE!60} 0.440 & \cellcolor{BLUE!0} 0.223 & \cellcolor{BLUE!50} 0.421 & \cellcolor{BLUE!10} 0.230 & \cellcolor{BLUE!40} 0.285 & \cellcolor{ORANGE!20} 0.211 & \cellcolor{ORANGE!30} 0.189 & \cellcolor{ORANGE!10} 0.221 & \cellcolor{BLUE!20} 0.243 \\
    Vendi Score $_{\times10^6}$ & \cellcolor{ORANGE!10} 1.60 & \cellcolor{BLUE!40} 3.09 & \cellcolor{BLUE!10} 2.60 & \cellcolor{BLUE!60} 7.15 & \cellcolor{ORANGE!20} 1.41 & \cellcolor{BLUE!50} 3.36 & \cellcolor{BLUE!20} 2.65 & \cellcolor{BLUE!0} 1.89 & \cellcolor{BLUE!30} 3.04 & \cellcolor{ORANGE!30} 0.20 \\
    \textbf{NovelSum (Ours)}  & \cellcolor{BLUE!40} 0.440 & \cellcolor{BLUE!60} 0.505 & \cellcolor{BLUE!20} 0.403 & \cellcolor{BLUE!50} 0.495 & \cellcolor{BLUE!30} 0.408 & \cellcolor{BLUE!10} 0.392 & \cellcolor{BLUE!0} 0.349 & \cellcolor{ORANGE!10} 0.336 & \cellcolor{ORANGE!20} 0.320 & \cellcolor{ORANGE!30} 0.309 \\
    \midrule
    \textbf{Model Performance} & \cellcolor{BLUE!30} 1.06 & \cellcolor{BLUE!60} 1.45 & \cellcolor{BLUE!40} 1.23 & \cellcolor{BLUE!50} 1.35 & \cellcolor{BLUE!20} 0.87 & \cellcolor{BLUE!10} 0.07 & \cellcolor{BLUE!0} -0.08 & \cellcolor{ORANGE!10} -0.38 & \cellcolor{ORANGE!30} -0.49 & \cellcolor{ORANGE!20} -0.43 \\
    \bottomrule
    \end{tabular}
    }
    \caption{Measuring the diversity of datasets selected by different strategies using \textit{NovelSum} and baseline metrics. Fine-tuned model performances (Eq. \ref{eq:perf}), based on MT-bench and AlpacaEval, are also included for cross reference. Darker \colorbox{BLUE!60}{blue} shades indicate higher values for each metric, while darker \colorbox{ORANGE!30}{orange} shades indicate lower values. While data selection strategies vary in performance on LLaMA-3-8B and Qwen-2.5-7B, \textit{NovelSum} consistently shows a stronger correlation with model performance than other metrics. More results are provided in Appendix \ref{app:results}.}
    \label{tbl:main}
    \vspace{-4mm}
\end{table*}


\begin{table}[t!]
\centering
\resizebox{\linewidth}{!}{
\begin{tabular}{lcccc}
\toprule
\multirow{2}*{\textbf{Diversity Metrics}} & \multicolumn{3}{c}{\textbf{LLaMA}} & \textbf{Qwen}\\
\cmidrule(lr){2-4} \cmidrule(lr){5-5} 
& \textbf{Pearson} & \textbf{Spearman} & \textbf{Avg.} & \textbf{Avg.} \\
\midrule
TTR & -0.38 & -0.16 & -0.27 & -0.30 \\
vocd-D & -0.43 & -0.17 & -0.30 & -0.31 \\
\midrule
Facility Loc. & 0.86 & 0.69 & 0.77 & 0.08 \\
Entropy & 0.93 & 0.80 & 0.86 & 0.63 \\
\midrule
LDD & 0.61 & 0.75 & 0.68 & 0.60 \\
KNN Distance & 0.59 & 0.80 & 0.70 & 0.67 \\
DistSum$_{cosine}$ & 0.85 & 0.67 & 0.76 & 0.51 \\
Vendi Score & 0.70 & 0.85 & 0.78 & 0.60 \\
DistSum$_{L2}$ & 0.86 & 0.76 & 0.81 & 0.51 \\
Cluster Inertia & 0.81 & 0.85 & 0.83 & 0.76 \\
Radius & 0.87 & 0.81 & 0.84 & 0.48 \\
\midrule
NovelSum & \textbf{0.98} & \textbf{0.95} & \textbf{0.97} & \textbf{0.90} \\
\bottomrule
\end{tabular}
}
\caption{Correlations between different metrics and model performance on LLaMA-3-8B and Qwen-2.5-7B.  “Avg.” denotes the average correlation (Eq. \ref{eq:cor}).}
\label{tbl:correlations}
\vspace{-2mm}
\end{table}

\paragraph{\textit{NovelSum} consistently achieves state-of-the-art correlation with model performance across various data selection strategies, backbone LLMs, and correlation measures.}
Table \ref{tbl:main} presents diversity measurement results on datasets constructed by mainstream data selection methods (based on $\mathcal{X}^{all}$), random selection from various sources, and duplicated samples (with only $m=100$ unique samples). 
Results from multiple runs are averaged for each strategy.
Although these strategies yield varying performance rankings across base models, \textit{NovelSum} consistently tracks changes in IT performance by accurately measuring dataset diversity. For instance, K-means achieves the best performance on LLaMA with the highest NovelSum score, while K-Center-Greedy excels on Qwen, also correlating with the highest NovelSum. Table \ref{tbl:correlations} shows the correlation coefficients between various metrics and model performance for both LLaMA and Qwen experiments, where \textit{NovelSum} achieves state-of-the-art correlation across different models and measures.

\paragraph{\textit{NovelSum} can provide valuable guidance for data engineering practices.}
As a reliable indicator of data diversity, \textit{NovelSum} can assess diversity at both the dataset and sample levels, directly guiding data selection and construction decisions. For example, Table \ref{tbl:main} shows that the combined data source $\mathcal{X}^{all}$ is a better choice for sampling diverse IT data than other sources. Moreover, \textit{NovelSum} can offer insights through comparative analyses, such as: (1) ShareGPT, which collects data from real internet users, exhibits greater diversity than Dolly, which relies on company employees, suggesting that IT samples from diverse sources enhance dataset diversity \cite{wang2024diversity-logD}; (2) In LLaMA experiments, random selection can outperform some mainstream strategies, aligning with prior work \cite{xia2024rethinking,diddee2024chasing}, highlighting gaps in current data selection methods for optimizing diversity.



\subsection{Ablation Study}


\textit{NovelSum} involves several flexible hyperparameters and variations. In our main experiments, \textit{NovelSum} uses cosine distance to compute $d(x_i, x_j)$ in Eq. \ref{eq:dad}. We set $\alpha = 1$, $\beta = 0.5$, and $K = 10$ nearest neighbors in Eq. \ref{eq:pws} and \ref{eq:dad}. Here, we conduct an ablation study to investigate the impact of these settings based on LLaMA-3-8B.

\begin{table}[ht!]
\centering
\resizebox{\linewidth}{!}{
\begin{tabular}{lccc}
\toprule
\textbf{Variants} & \textbf{Pearson} & \textbf{Spearman} & \textbf{Avg.} \\
\midrule
NovelSum & 0.98 & 0.96 & 0.97 \\
\midrule
\hspace{0.10cm} - Use $L2$ distance & 0.97 & 0.83 & 0.90\textsubscript{↓ 0.08} \\
\hspace{0.10cm} - $K=20$ & 0.98 & 0.96 & 0.97\textsubscript{↓ 0.00} \\
\hspace{0.10cm} - $\alpha=0$ (w/o proximity) & 0.79 & 0.31 & 0.55\textsubscript{↓ 0.42} \\
\hspace{0.10cm} - $\alpha=2$ & 0.73 & 0.88 & 0.81\textsubscript{↓ 0.16} \\
\hspace{0.10cm} - $\beta=0$ (w/o density) & 0.92 & 0.89 & 0.91\textsubscript{↓ 0.07} \\
\hspace{0.10cm} - $\beta=1$ & 0.90 & 0.62 & 0.76\textsubscript{↓ 0.21} \\
\bottomrule
\end{tabular}
}
\caption{Ablation Study for \textit{NovelSum}.}
\label{tbl:ablation}
\vspace{-2mm}
\end{table}

In Table \ref{tbl:ablation}, $\alpha=0$ removes the proximity weights, and $\beta=0$ eliminates the density multiplier. We observe that both $\alpha=0$ and $\beta=0$ significantly weaken the correlation, validating the benefits of the proximity-weighted sum and density-aware distance. Additionally, improper values for $\alpha$ and $\beta$ greatly reduce the metric's reliability, highlighting that \textit{NovelSum} strikes a delicate balance between distances and distribution. Replacing cosine distance with Euclidean distance and using more neighbors for density approximation have minimal impact, particularly on Pearson's correlation, demonstrating \textit{NovelSum}'s robustness to different distance measures.








%\vspace{-5pt}
\section{Conclusion}
%\vspace{-5pt}
In this paper we considered the fully inductive link prediction task in KGs. We identified the open challenges in existing fully inductive models, and proposed \ourmethod, a novel architecture designed to improve expressiveness and support efficient relation prediction tasks. Through comprehensive experiments spanning 57 diverse KGs datasets, we demonstrate that increased expressiveness translates into better performance. Additionally, our experimental study sheds light on the limitations of LLMs in exploiting graph information in new domains for entity and relation prediction tasks.


% \vspace{-10pt}
% \paragraph{Limitations and Impact Statement.}
% Despite the promising results of \ourmethod in the inductive settings we consider, there are still aspects that can be refined in future research.
% For example, while \ourmethod aims to enhance expressivity compared to existing methods, there might still be scenarios where its representation power is limited, leading to challenges in capturing complex relationships.
% Furthermore, the relation adjacency matrix size introduces additional computational complexity during training, potentially requiring more resources and time compared to simpler models. Finally, while not designed for malicious intent, there is a possibility of \ourmethod being repurposed for activities like uncovering sensitive information, raising questions about data privacy and responsible AI use.
% For natbib users:

\newpage
\clearpage
\section*{Acknowledgments}
This work was funded in part by the National Science Foundation (NSF) awards, CCF-1918483, CAREER IIS-1943364 and CNS-2212160, Amazon Research Award, AnalytiXIN, and the Wabash Heartland Innovation Network (WHIN), Ford, NVidia, CISCO, and Amazon. Computing infrastructure was supported in part by CNS-1925001 (CloudBank). This work was supported in part by AMD under the AMD HPC Fund program. %Any opinions, findings and conclusions or recommendations expressed in this material are those of the authors and do not necessarily reflect the views of the sponsors.

\bibliographystyle{unsrtnat}
\bibliography{ref}
% For bibLaTeX users:
% \printbibliography

\newpage
\appendix
\section{Pseudo Codes of Iterative Embedding Updates}
\Cref{entity_code} and \Cref{relation_code} show the iterative embedding updates in entity prediction task and relation prediction task respectively.

\begin{minipage}{\textwidth}
\begin{algorithm}[H]
\small
\caption{\ourmethod embedding updates for the entity prediction task}
\label{entity_code}
\begin{algorithmic}
\Require Query ($h$, $r$, ?); relation adjacency matrix $\mA_R$; entity adjacency matrix $\mA_V$; number of updates $L$
\Ensure Final entity embedding $\mX_{h,r}^{(L)}$
% \State $\mX_{h,r}^{(0)} = \text{INDICATOR}(h) \odot \mathbf{1}^{|V|\times d}$
% \State $\mZ_{h,r}^{(0)} = \text{INDICATOR}(r) \odot \mathbf{1}^{|R|\times d}$
\State $\mX_{h,r}^{(0)} = \text{INIT}_{V}(h)$ \Comment{Label $h$ with all-ones vector and the rest with all-zeros}
\State $\mZ_{h,r}^{(0)} = \text{INIT}_{R}(r)$ \Comment{Label $r$ with all-ones vector and the rest with all-zeros}
\For{$i \gets 1$ to $L$}
\For{$u \in V$}
\State $\mX_{h,r}^{(i)}(u) = \text{UP}_{V}^{(i)}\left(\mX_{h,r}^{(i-1)}(u), \text{AGG}_{V}^{(i)}\left(\text{MSG}_{V}^{(i)}(\mX_{h,r}^{(i-1)}(v), \mZ_{h,r}^{(i-1)}(r^\prime))|(u, r^\prime, v) \in \mA_V\right)\right)$
\EndFor
\For{$r^\prime \in R$}
\State $\mZ_{h,r}^{(i)}(r^\prime) = \text{UP}_{R}^{(i)}\left(\mZ_{h,r}^{(i-1)}(r^\prime), \text{AGG}_{R}^{(i)}\left(\text{MSG}_{R}^{(i)}(\mZ_{h,r}^{(i-1)}(r^{\prime\prime}), \mX_{h,r}^{(i)}(u))|(r^\prime,u,r^{\prime\prime})\in \mA_R\right)\right)$
\EndFor
\EndFor
\end{algorithmic}    
\end{algorithm}
\end{minipage}

\begin{minipage}{\textwidth}
\begin{algorithm}[H]
\small
\caption{\ourmethod embedding updates for the relation prediction task}
\label{relation_code}
\begin{algorithmic}
\Require Query ($h$, ?, $t$); relation adjacency matrix $\mA_R$; entity adjacency matrix $\mA_V$; number of updates $L$
\Ensure Final relation embedding $\mZ_{h,t}^{(L)}$
\State $\mZ_{h,t}^{(0)} = \mathbf{1}^{|R|\times d}$
\State $\mX_{h,t}^{(0)} = \text{INIT}(h, t)$ \Comment{Label $h$ with all-ones vector, $t$ with all-negative-ones and the rest with all-zeros}
% \State $\mX_{h,t}^{(0)} = \text{INDICATOR}(h, t) \odot \mathbf{1}^{|V|\times d}$
\For{$i \gets 1$ to $L$}
\For{$r \in R$}
\State $\mZ_{h,t}^{(i)}(r) = \text{UP}_{R}^{(i)}\left(\mZ_{h,t}^{(i-1)}(r), \text{AGG}_{R}^{(i)}\left(\text{MSG}_{R}^{(i)}(\mZ_{h,t}^{(i-1)}(r^\prime), \mX_{h,t}^{(i-1)}(u))|(r,u,r^\prime)\in \mA_R\right)\right)$
\EndFor
\For{$u \in V$}
\State $\mX_{h,t}^{(i)}(u) = \text{UP}_{V}^{(i)}\left(\mX_{h,t}^{(i-1)}(u), \text{AGG}_{V}^{(i)}\left(\text{MSG}_{V}^{(i)}(\mX_{h,t}^{(i-1)}(v), \mZ_{h,t}^{(i)}(r))|(u, r, v) \in \mA_V\right)\right)$
\EndFor
\EndFor
\end{algorithmic}    
\end{algorithm}
\end{minipage}

\section{Expressive Power}\label{app:expressive-power}
\section{Theoretical results}\label{sec:appendix_proof}




\subsection{Detailed notations}
\label{sec:apx_detailed_notation}

\begin{definition}[Embedding Layer]\label{defi:embedding_layer}
Given a finite vocabulary $\mathcal{V}$, embedding dimension $d\in\mathbb{N}^+$, token embedding parameter $\theta_{\tokenembedding}\in\mathbb{R}^{d\times \abs{\mathcal{V}}}$ and position embedding parameter $\theta_{\posencoding}\in \mathbb{R}^{d\times n_{\max}}$, we define the \emph{embedding layer} as a sequence-to-sequence map, denoted by $\embed_{\theta_\tokenembedding,\theta_\posencoding}:\mathcal{V}^n\to (\mathbb{R}^d)^n$ for any $1\le n\le n_{\max}$, where
\begin{equation}
    \embed_{\theta_\tokenembedding,\theta_\posencoding}(v_1,\ldots,v_n) = \left(\theta_\tokenembedding(v_1)+ \theta_\posencoding(1), \ldots, \theta_\tokenembedding(v_n)+ \theta_\posencoding(n)  \right). 
\end{equation}
\end{definition}

\paragraph{Multi-Head Self-Attention Mechanism:} Given attention parameters $\theta_\attn = \{W_Q, W_K, W_V, W_O\}$, where each $W_Q^m, W_K^m, W_V^m, W_O^m \in \mathbb{R}^{d_{\attn} \times d }$, we define the \emph{Self-Attention} layer with a causal mask for a decoder-only transformer in \Cref{alg:defi_attn}. We also define a \emph{Multi-Head Attention} layer as a collection of self-attention layer with non-shared parameters $\theta_\mha = \{\theta_{\attn}^{(h)}\}_{h=1}^H$, and its output is the sum of the outputs from each head. That is, $\mha_{\theta_\mha} = \sum_{h=1}^H \attn_{\theta_\attn^{(h)}}$.\footnote{Though in this paper we focus on attention with casual mask, our definition of looped transformer generalizes to the cases with other attention masks.}

\begin{algorithm}
    \caption{Causal Self-Attention, $\attn$}\label{alg:defi_attn}
    \begin{algorithmic}[1]
    \Require  Parameter $\theta_\attn = (W_Q,W_K,W_V,W_O)$, Input embedding $ x_1,\ldots, x_n)\in \left(\mathbb{R}^{d}\right)^n$.
    \Ensure Output embedding $x'= (x'_1,\ldots, x'_n) \triangleq \attn_{\theta_\attn}(x_1,\ldots, x_n)$.
    \State $q_i \triangleq W_Q  x_i, k_i \triangleq W_K x_i, v_i \triangleq W_V x_i, \forall i\in[n]$
    \State $s_i \triangleq \softmax(\inner{q_i}{k_1},\ldots,\inner{q_i}{k_i}) \| (0,\ldots, 0) $. 
    \State $h'_i \triangleq W_O^\top \sum_{j=1}^n (s_i)_j v_j$.
    \end{algorithmic}
    \end{algorithm}
\paragraph{Feed-Forward Network:} Given the parameters of the fully-connected feedforward network layer $\theta_\ff = (W_1, b_1, W_2, b_2) \in \mathbb{R}^{x_{\ff} \times d} \times \mathbb{R}^{d_{\ff}} \times \mathbb{R}^{d \times d_{\ff}} \times \mathbb{R}^{d_{\ff}}$, we define the feedforward layer $\ff_{\theta_\ff}: \mathbb{R}^{d} \to \mathbb{R}^{d}$ as $\ff_{\theta_{\ff}}(h) \triangleq W_2, \relu(W_1 h + b_1) + b_2$.

\iffalse
\begin{definition}[Transformer Block]\label{defi:transformer_block}
Given number of layers $L\in\mathbb{N}^+$ and parameter $\theta_\tfblock = (\theta^{(l)}_\mha,\theta^{(l)}_\ff )_{l=0}^{L-1}$, we define the $L$-layer transformer block $\tfblock_{\theta_\tfblock}:(\mathbb{R}^d)^n\to (\mathbb{R}^d)^n$ for any $n\in\mathbb{N}^+$ as \begin{equation}
    \tfblock_{\theta_\tfblock} \triangleq (\id+ \ff_{\theta^{(L-1)}_\ff})\circ (\id+ \mha_{\theta^{(L-1)}_\mha})\circ \cdots (\id+ \ff_{\theta^{(0)}_\ff})\circ (\id+ \mha_{\theta^{(0)}_\mha})
\end{equation}
\end{definition}
\fi

\begin{definition}[Output Layer]\label{defi:output_layer}
    Given parameter $\theta_\transoutput \in\mathbb{R}^{d\times |\mathcal{V}|}$, we denote the output layer as $\transoutput_{\theta_\transoutput}:(\mathbb{R}^d)^n\to \Delta^{|\mathcal{V}|-1}$, where 
    \begin{align}
        \transoutput_{\theta_\transoutput}(x_1,\ldots,x_n) \triangleq \softmax(x_n^\top \theta_{\transoutput})
    \end{align}
\end{definition}

Finally, we define the entire transformer model $p_{\theta}: \cup_{n\le n_{\max}} \mathcal{V}^n\to \Delta^{|\mathcal{V}|-1}$ as 
\begin{align}
p_{\theta}\triangleq \transoutput_{\theta_\transoutput}\circ \tfblock_{\theta_\tfblock}\circ \embed_{\theta_\tokenembedding,\theta_\posencoding}
\end{align}
for any $\theta = (\theta_\tfblock,\theta_\tokenembedding,\theta_\posencoding,\theta_\transoutput)$.For convenience, we also write $\left[p_{\theta}(v_1,\ldots,v_n)\right](v)$ as $p_{\theta}(v\mid v_1,\ldots,v_n)$.
In particular, we use $\transformer_{\theta}(v_1,\ldots,v_n) \triangleq \argmax_{v\in\mathcal{V}} p_{\theta}(v|v_1,\ldots,v_n)$ to denote the deterministic version of the transformer model. 

\iffalse
\begin{definition}[Looped Transformer]\label{defi:looped_transformer}
Given number of loops $T\in\mathbb{N}^+$,  parameters $\theta = (\theta_{\tfblock},\theta_\tokenembedding,\theta_\posencoding,\theta_\transoutput)$, where $\theta_{\transformer} = (\theta^{(l)}_\mha,\theta^{(l)}_\ff )_{l=0}^{L-1}$, we define the \emph{looped transformer} as  $p_{\theta,T}\triangleq \transoutput_{\theta_\transoutput}\circ \left(\tfblock_{\theta_\tfblock}\right)^T\circ \embed_{\theta_\tokenembedding,\theta_\posencoding}$ and the corresponding deterministic version as $\transformer_{\theta,T}(v_1,\ldots,v_n) \triangleq \argmax_{v\in\mathcal{V}} p_{\theta,T}(v|v_1,\ldots,v_n)$.
\end{definition}
\fi



\paragraph{Finite-precision Modeling:} In this paper we assume the transformer is of finite precision. More specifically, we follow the setting in \citet{li2024chain} and use the shorthand $\Floating_{s}\triangleq \{c\cdot k\cdot 2^{-s}\mid c\in \{-1,1\}, 0\le k\le 2^{2s}-1, k\in\mathbb{N}\}$ to denote fixed-point numbers of constant precision $s$ and rounding operation $\rds{\cdot}:\mathbb{R}\to \Floating_s$ to denote the correcting rounding, namely the mapping from $\mathbb{R}$ to the closest representable number in $\Floating_s$. (We break the tie by picking the number with smaller absolute value). We assume that (1). all the parameters of the transformer are in $\Floating_{s}$ and (2). correct rounding is performed after every binary operation in the forward pass of the transformer. We will refer the readers to \citet{li2024chain} for detailed discussion on such finite-precision modeling and only list important notations and lemmas that will be used in this paper below. 

We use $1_s$ to denote all-one vectors of length $s$. Similarly we define $\inner{\cdot}{\cdot}_s$, $\times_s$, and $\softmax_s$. We recall that for any $s\in \mathbb{N}^+$ and integer $0\le x\le 2^s-1$, we use $\bin_s(x)\in\{0,1\}^s$ to denote the usual binary encoding of integer $x$ using $s$ binary bits in the sense that $x = \sum_{i=1}^s 2^i (\bin_s(x))_i$ and $\sbin_s(x)\in\{-1,1\}^s$ to denote the signed binary encoding, which is $2\bin_s(x)-(1,\ldots,1)$. Finally we define $B_s = \max \Floating_s = 2^s-2^{-s}$.



%
%\begin{proof}[Proof of \Cref{lem:cot_1_is_squal_to_O1}]
%	This is a special case of \Cref{thm:cot_dpolyn_slogn_main}.
%\end{proof}

\begin{lemma}\label{lem:exp_rounding}[Lemma E.1, \citep{li2024chain}]
	For any $s\in\mathbb{N}^+$, it holds that $\rds{\exp(-B_s)} = 0$.
\end{lemma}

\begin{lemma}\label{lem:exp_rounding_up}[Lemma E.2, \citep{li2024chain}]
	For any $s\in\mathbb{N}^+$, it holds that $\rds{\exp(B_s)} = B_s$.
\end{lemma}

\begin{lemma}\label{lem:FB_simulating_boolean_gates}[Lemma E.5, \citep{li2024chain}]
Unlimited-fanin $\AND,\OR$ (resp. $\MAJORITY) :\{0,1\}^n\to \{0,1\}$ can be simulated by some 2-layer feedforward ReLU network with constant (resp. $\log n$) bits of precision constant hidden dimension and additional $n$ constant inputs of value 1. 

Mathematically, let $\ff[s(n)]$ be the set of functions $C:\{0,1\}^n\to\{0,1\}$ which can be a two-layer feedforward ReLU network with at most $s(n)$ bits of precision and constant hidden dimension $\ff_{\theta}:\{0,1\}^{2n}\to \{0,1\}, \ff_{\theta}(x') = W_2\times_s\relu(\rds{W_1\times_s x'+b_1})$, where $\theta = (W_2,W_1,b_1)$, such that for any $x\in\{0,1\}^n$, 
\begin{align}
\ff_\theta(x_1,1,x_2,1,\ldots,x_n,1) = C(x).	
\end{align}
We have unlimited-fanin $\AND,\OR\in \ff[1]$ and $\MAJORITY\in \ff[\log n]$.  
\end{lemma}

Given two vectors $x,y$ of the same length $s$, we use $\interleave{x}{y}$ to denote their interleaving, that is, $(\interleave{x}{y})_{2i-1} = x_i, (\interleave{x}{y})_{2i} = y_i$ for all $i\in [e]$. 
\begin{lemma}\label{lem:attention_rounding}[Lemma E.3, \citep{li2024chain}]
	For any $s\in\mathbb{N}^+$, let $q_i = \interleave{\sbin_s(i)}{1_s}$ and $k_i = B_s\cdot (\interleave{\sbin_s(i)}{(-1_s)})$ for all $i\in [2^s-1]$, it holds that $\rds{\exp(\inner{q_i}{k_j}_s)}=\indct{i=j}$ for all $i,j\in [2^s-1]$.
\end{lemma}

\subsection{Proofs}


\subsubsection{Looped models can simulate non-looped models}
\label{sec:apx_simulate_nonlooped}


\begin{proof}[Proof of \Cref{thm:main}]

We start by introduce some more notations. We will proof the theorem for any fixed sequence of $\vv = (v_1,\ldots, v_n)$. We use $\vx^{(l)}=(x_{i}^{(l)})_{i=1}^n$ to denote the intermediate embedding of $p_{\theta}$ in the $l$th layer. More specifically, we define 

\begin{align}
    \vx^{l} = (\id+ \ff_{\theta^{(l-1)}_\ff})\circ (\id+ \mha_{\theta^{(l-1)}_\mha})\circ \cdots (\id+ \ff_{\theta^{(0)}_\ff})\circ (\id+ \mha_{\theta^{(0)}_\mha})\circ \embed_{\theta_{\embed}}(\vv).
\end{align}

We also use $\vx^{(l+0.5)}=(x_{i}^{(l+0.5)})_{i=1}^n$ to denote the intermediate embedding of $p_{\theta}$ in the $l$th layer after the attention layer.
\begin{align}
    \vx^{l+0.5} =  (\id+ \mha_{\theta^{(l-1)}_\mha})(\vx^{l}).
\end{align}

Similarly, for the constructed looped transformer $p_{\theta,T}$, we use $\vv'=(\#,\vv_1,\ldots,\vv_n)$ to denote its input. For simplicity, we use the convention that $\#$ is at position $0$. The proof still works if $\#$ starts at position $1$ because we can just transfer the later tokens by $1$ position. We define $\vx'^{(l)}=(x'^{(l)}_0,x'^{(l)}_1,\ldots,x'^{(l)}_n)$ as the intermediate embedding of $p_{\theta}$ in the $l$th layer and $\vx'^{(l+0.5)}=(x'^{(l+0.5)}_0,x'^{(l+0.5)}_1,\ldots,x'^{(l+0.5)}_n)$ as the intermediate embedding of $p_{\theta}$ in the $l$th layer.

Below we first state the key properties that our construction will satisfy, which imply the correctness of the theorem and then we state our construction of $p_{\theta',T}$ and show the properties are actually satisfied:
\begin{itemize}
    \item $x'^{(l)}_{i} = (x^{(l)}_i, \bm{1}_{R} - e_{r(l)},l, \indct{i=0})$.
    \item $x'^{(l+0.5)}_{i} = (x^{(l+0.5)}_i, \bm{1}_{R} - e_{r(l)},l, \indct{i=0})$.
    \item $x^{(l)}_0 = x^{(l+0.5)}_0 = \bm{0}$.\footnote{Here we abuse the notation for simplicity of presentation. $x^{(l)}_0 = x^{(l+0.5)}_0$ are not defined in the original non-looped transformer. The key point here is that they are 0 vectors throughout the forward pass.}
\end{itemize}

% \begin{itemize}
%     \item $\theta'_{\ff} $ is essentially concatenation of $\theta^{(i)}_\ff$ for $i\in[r]$ in the hidden dimension of $\ff$. The goal is to make $\ff_{\theta'_{\ff}} = \sum_{i=1}^r \ff_{\theta^{(i)}_\ff}\indct{r(l)=i}$ at $l$th layer. 
%     \item  $\theta'_{\mha} $ is essentially concatenation of $\theta^{(i)}_\mha$ for $i\in[r]$ in the hidden dimension of $\mha$. The goal is to make $\mha_{\theta'_{\mha}} = \sum_{i=1}^r \mha_{\theta^{(i)}_\ff}\indct{r(l)=i}$ at $l$th layer.
%     \item Use extra dimension to count the depth $l$ and maintain $\bm{1} - e_{r(l)}$ using $R+1$ dimension on top of the original embedding size $d$ and use it to do switch. Such piecewise linear function can be approximated by a neural network with $O(L)$ hidden dimension.
%     \item use $\bm{1} - e_{r(l)}$ together with ReLU activation to do the select in MLP. Use $\indct - e_{r(l)}$ in attention to force all attention heads to attend to the dummy token $\#$ in the beginning of the sentence.
% \end{itemize}

To get the last coordinate $l$, which is a depth counter, we just need to add $1$ more hidden dimension in MLP. 

Next we show we can use two-layer with $L+2$ MLP to get the mapping from $\ell\mapsto \bm{1}_{R} - e_{r(l)}$. Let $(\theta^{(i)}_\ff,\theta^{(i)}_\mha)_{i=1}^R$ be the parameters of the $R$ distinct layers in $\theta$. We assume in the $l$th layer, $r(l)$'s parameters are used. This is because $e_{r(l)} = \sum_{i=1}^L e_{r(i)}0.5*( [l-i+1 ]_+-2[l-i]_+ + [l-i-1]_+)$.

Now we explain how to deactivate the undesired MLP neurons. In other words, our construction of $\theta'_{\ff} $ is essentially concatenation of $\theta^{(i)}_\ff$ for $i\in[r]$ in the hidden dimension of $\ff$, with the additional control that  $\ff_{\theta'_{\ff}}((x^{(l)}_i, \bm{1}_{R} - e_{r(l)},l),\indct{i=0}) = \sum_{i=1}^R \ff_{\theta^{(i)}_\ff}(x^{(l)}_i)\indct{r(l)=i,i\neq 0}$ at $l$th layer. This control can be done by subtracting $\bm{1} - e_{r(l)} + \indct{i=0}$ by a constant which is larger than the maximum pre-activation in the hidden layer.

Finally we explain how to deactivate the undesired attention. We will only use attention to update the first part of the embedding, which is $x^{(l+0.5)}_i$. A crucial step here is that we set the token embedding of $\#$ as $0$
We construct keys and queries as follows: 
\begin{enumerate}
    \item ${W'}_Q^{(r')}(x'^{(l)}_{i}) = (W_Q^{(r')}x^{(l)}_{i}, 1- \indct{r'=r(l)} $ for $r' \in [R]$ and $i=0,\ldots,n$
    \item ${W'}_K^{(r')}(x'^{(l)}_{i}) = (W_K^{(r')}x^{(l)}_{i}, -B\indct{i=0}) $ for $r' \in [R]$ and $i=0,\ldots,n$, where $B$ is some constant larger than the maximum previous inner product in attention, $\max_{l\in[L],i,j} \inner{(W_K x^{(l)}_{i}}{(W_Qx^{(l)}_{i}}$. 
    \item ${W'}_O^{(r')}{W'}_V^{(r')}(x'^{(l)}_{i}) = (W_O^{(r')}W_V^{(r')}x^{(l)}_{i}, \bm{0}, 0,0)$.
\end{enumerate}
This construction works because only the `desired' attention head $r=r(l)$ will be activated and behave as in the non-looped case, because otherwise all position in that attention head will be completely attended to position $0$ and returns a zero vector. (We can choose $B$ to be large enough and distribution calculated by the attention score is delta distribution) at position $0$, which yields a zero vector as its value. This completes the proof.
\end{proof}





\subsubsection{Group composition.}
\label{sec:apx_group_composition}






\begin{algorithm}[t]
\caption{Group Composition}\label{alg:group_composition}
\begin{algorithmic}[1]
\Require  Group elements $g_0,g_1,\ldots,g_n\in G$, where $g_0=e$.
\Ensure $g_0\circ g_1\circ \ldots g_n$.
\State $g^{(0)}_i = g_i$, $\forall0\le i\le n$.
\For{$l=1\to \lceil \log_2 n\rceil$}
\State $a^{(l)}_i = g^{(l-1)}_{[2i-n-1]_+},b^{(l)}_i = g^{(l-1)}_{[2i-n]_+}$  $\forall0\le i\le n$.
\State $g^{(l)}_i = a^{(l)}_i\circ b^{(l)}_i$, $\forall0\le i\le n$.
\EndFor
\State \Return $g^{(\lceil\log _2 n\rceil)}_{n}$.
\end{algorithmic}
\end{algorithm}



The landmark result in automata theory, Krohn-Rhodes Decompotision Theorem~\citep{krohn1965algebraic}, shows that all semi-automaton with solvable transformation group (which includes composition problem of solvable groups) can be simulated by a cascade of permutation-reset automata, which can be simulated by $\TC^0$ circuits. \citep{liu2022transformers} further showed that such automaton with solvable transformation group can be continuously simulated by constant-depth transformers. However, it is also shown~\citep{barrington1986bounded} that the composition problem of unsolvable groups are $\NC^1$-complete, for example, the composition of permutation group over $5$ elements, $S_5$. Under the common hardness assumption that $\NC^1\neq \TC^0)$,   constant depth transformer cannot solve composition of $S_5$ using a single forward pass~\citep{merrill2023parallelism,liu2022transformers,li2024chain}. But with CoT, very shallow transformers (depth equal to one or two) can simulate the composition problem of any group\citep{li2024chain,merrill2023expresssive}.



\begin{proof}[Proof of \Cref{thm:group_composition_log_depth}]
We will set the token embedding of $\#$ the same as that of $e$, which is the identity of $G$. In the following proof, we will just treat $\#$ as $e$. We will construct the transformer simulating group composition following the following algorithm~\Cref{alg:group_composition}, which gives the high-level idea of the construction. The correctness of \Cref{alg:group_composition} follows from the associativity of group composition. More concretely, we can verify by induction that $g^{l}_0\circ g^{l}_1\circ\ldots g^{l}_n$ is the same for all $l=0,\ldots, \lceil \log_2 n\rceil$ and in the final round, i.e., when $l=\lceil \log_2 n\rceil$, $g^{(l)}_i=e$  for all $i<n$. 



Below we show how to construct a transformer of the given sizes to simulate the above \Cref{alg:group_composition}.
We will embed each $g\in G$ as a different vector  $\overline{g} \in \{-1,1\}^{\lceil\log_2 |G|\rceil}$ and each position $0\le i\le n$ as its binary representation in $\overline{i} \in\{-1,1\}^{\lceil\log_2 n+1\rceil}$, which is a shorthand for $\sbin_s(i)$ with $s=\lceil\log_2 n+1\rceil$.
We concatenate them to get $\{x^{(0)}_i\}_{i=0}^n$, that is, $x^{(0)}_i = (\overline{g_i},\overline{i}, \overline{[2i-n-1]_+}, \overline{[2i-n-1]_+},0^{\lceil \log_2 |G|\rceil},0^{\lceil \log_2 |G|\rceil})$.
For convenience, we will drop the 0's in the end (also in the other proofs of the paper) and write it as $x^{(0)}_i = (\overline{g_i},\overline{i} , \overline{[2i-n-1]_+}, \overline{[2i-n-1]_+})$.
Below we show we can construct 1-layer transformer block with parameter $(\theta_\mha,\theta_\ff)$ satisfying that
\begin{enumerate}
    \item $\left[ \mha_{\theta_\mha}\left((\overline{g_i},\overline{i},\overline{[2i-n-1]_+}, \overline{[2i-n-1]_+})_{i=0}^n\right)\right]_k = (0^{\lceil\log_2 |G|\rceil +3\lceil\log_2 n+1\rceil}, \overline{g_{[2k-n-1]_+}}, \overline{g_{[2k-n]_+}})$ \\for all $g_0=e,g_i\in G \forall i\in[n]$, $k=0,\ldots,n$;
    \item $\ff_{\theta_\ff}(\overline{g},\overline{i}, \overline{j}, \overline{k},\overline{g'}, \overline{g''}) = (\overline{g'\circ g''}-\overline{g}, 0^{3\lceil\log_2 n+1\rceil},-\overline{g'}, -\overline{g''})$, for all $i,j,k=0,\ldots,n$, $g,g',g''\in G$. 
\end{enumerate}

The first claim is because we can use two attention heads to retrieve $ \overline{g_{[2k-n-1]_+}}$ and $ \overline{g_{[2k-n]_+}}$ respectively, where both of them use $\overline{k}$ as the key and use $-\overline{[2k-n-1]_+}$ and $-\overline{[2k-n]_+}$ as queries respectively. This is possible because all the required information are already in $x_i$. We further make attention temperature low enough so the probability returned by attention is a one-hot distribution at the position whose key is equal to the negative query after rounding.

Now we turn to the second claim about MLP. We will use $|G|^2$ neurons with ReLU activation and bias to simulate the product of $g'$ and $g''$. We can index each neuron 
by $(h,h')$ for $h,h'\in G$ and set its incoming weight $[W_1]_{(h,h'),:} = (\overline{h},\overline{h'})$ and set bias $(b_1)_{(h,h')} = - 2\lceil\log_2 |G|\rceil+1$, which ensures that the activation of neuron $(h,h')$ will only be $1$ when $g'=h,g''=h'$ and be $0$ otherwise. Then setting the outgoing weight of neuron $(h,h')$ as $\overline{h\circ h'}$ and the bias in the second layer to be $0$ finishes the construction for simulating the group composition. Finally we use the remaining $6\lceil\log_2 |G|\rceil$ to simulate negative identity mapping $x\to-x$ for the remaining $3\lceil\log_2 |G|\rceil$ embedding dimension. This completes the proof.
\end{proof}






% \iffalse


\subsection{Connection to chain-of-thought reasoning}
\label{sec:apx_cot_connection}


In this section, we establish a connection betwee looped models and CoT reasoning. We first define the recursion for CoT reasoning as follows:
$$
\transformer^{i}_\theta(v_1,\ldots,v_n)\triangleq \transformer^{i-1}_\theta(v_1,\ldots,v_n, \transformer_\theta(v_1,\ldots,v_n)),$$ for $i, n \in \mathbb{N}^+$ satisfying $i+n\le n_{\max}-1$ along with the base case of $\transformer^{1}_\theta(v_1,\ldots,v_n)\triangleq\transformer_\theta(v_1,\ldots,v_n)$. For all $0\le i\le n_{\max} - n-1$, the output with $i$ steps of CoT is
$v_{n+i+1}  = \transformer^{i+1}_\theta(v_1,\ldots,v_n) = \transformer_\theta(v_1,\ldots,v_n,v_{n+1},\ldots,v_{n+i})$.
% \ns{Connection to CoT reasoning} \sjr{this part is very unclear}

\iffalse
\begin{theorem}[Informal]\label{thm:cot_informal}
    For any $L$-layer non-looped transformer $\transformer_\theta$, there exists a looped transformer with parameter $\theta'$ and $L+\mathcal{O}(1)$ layers, such that for any input $\vx$ and integer $m$,  the output of non-looped transformer after $m$ steps of CoT is the same as that of the looped transformer on input $x$ concatenated by $m$ dummy tokens with $m$ loops.
\end{theorem}
\fi




We first give the formal statement below.
% of \Cref{thm:cot_informal}, which is \Cref{thm:cot_formal} below. 
\begin{theorem}[Looped transformer simulates CoT]\label{thm:cot_formal}
    For any $L$-layer non-looped transformer $\transformer_\theta$, there exists a looped transformer $\transformer_{\theta'}$ with $L+\mathcal{O}(1)$ layers, constantly many more dimensions in embedding, MLP and attention layers and constantly many more attention heads, such that for any input $\vv= (v_i)_{i=1}^n$ and integer $m$, the output of non-looped transformer after $m$ steps of CoT, $\transformer^m_\theta(\vv)$, is the same as that of the looped transformer on input $x$ concatenated by $m$ dummy tokens with $m$ loops, $\transformer_{\theta',m}(\vv,\#^m)$.
\end{theorem}


Below are some helping lemmas towards showing \Cref{thm:cot_formal} is at least as powerful as CoT.
\begin{lemma}[Simulating $\argmax$ using MLP]\label{lem:simulating_hard_argmax}
    For every $d\in \mathbb{N}$ and precision $s\in\mathbb{N}^+$, there exists a 3-layer network with $\relu$ activation and $d^2$ width $f$ with $s$ precision, such that for any $x\in\Floating_s^d$, if there is $k\in [d]$, such that $x_k >\max_{j\neq k,j\in [d]}x_j$, $f(x) = e_k$. 
\end{lemma}


\begin{proof}[Proof of \Cref{lem:simulating_hard_argmax}]
    Define $g_i = 2^s\cdot \relu(2^{-s} - \sum_{j\neq i}\relu(x_j-x_i))$ for each $i\in [n]$. We claim that if there is $k\in [d]$, such that $x_k -\max_{j\neq k,j\in [d]}x_j\ge 2^{-s}$, $g_i = 1$ iff $i=k$ for all $i\in [d]$. First $g_k =2^s\cdot \relu(2^{-s}) =1 $. Next for $i\neq k$, it clearly holds that  $\sum_{j\neq i}\relu(x_j-x_i)\ge 2^{-s}$ and thus $g_i\le 0$. 
    This construction can clearly be implemented by a 3-layer $\relu$ network with $s$ precision.
\end{proof}

\begin{lemma}[Simulating Decoding and Embedding using MLP]\label{lem:simulating_decoding_embedding}
    Given any $s$-precision $\theta_{\tokenembedding}\in\mathbb{R}^{d\times \Sigma}$ and $\theta_{\transoutput}$, there is a 5-layer network $f:\mathbb{R}^d\to \mathbb{R}^d$ with $\relu$ activation and $\max(|\Sigma|^2)$ width with $s$-precision, such that for all $s$-precision $x\in\mathbb{R}^d$ which admits unique $\argmax$ for $v\triangleq \argmax_{o\in\Sigma} (x^\top\theta_{\transoutput})(o)$, it holds that
    \begin{align*}
        f(x) =  \theta_{\tokenembedding}(v).
    \end{align*}
\end{lemma}
\begin{proof}[Proof of \Cref{lem:simulating_decoding_embedding}]
    This is a simple application of \Cref{lem:simulating_hard_argmax}.
\end{proof}

\begin{lemma}[Control Gate]\label{lem:control_gate}
    A 2-layer $\relu$ network with precision $s$ can implement $F:\Floating_s\times \Floating_s\times \{0,1\}, F(x,y,M) = Mx + (1-M)y $. 
\end{lemma}
\begin{proof}[Proof of \Cref{lem:control_gate}]
    Note that $F(x,y,M) = \relu(x-2^s\cdot (1-M)) - \relu(-x-2^s\cdot (1-M)) + \relu(y-2^s\cdot M) - \relu(-y-2^s\cdot M)$. The proof is completed.
\end{proof}

\begin{definition}[Transformer Block with Mask]\label{defi:embedding_layer_w_mask}
Given number of layers $L\in\mathbb{N}^+$, parameter $\theta_\tfblock = (\theta^{(l)}_\mha,\theta^{(l)}_\ff )_{l=0}^{L-1}$, and mask function $M:\mathbb{N}\to\{0,1\}$, we define the $L$-layer \emph{transformer block with mask}$\tfblock_{\theta_\tfblock,M}:(\mathbb{R}^d)^n\to (\mathbb{R}^d)^n$ for any $n\in\mathbb{N}^+$ as 
\begin{equation}
    [\tfblock_{\theta_\tfblock,M}(\vx)]_i \triangleq (1-M(i))x_i + M(i)[\tfblock_{\theta_\tfblock}(\vx)]_i
\end{equation}
\end{definition}

\begin{definition}[Looped Transformer with Mask]\label{defi:looped_transformer_w_mask}
Given number of loops $T\in\mathbb{N}^+$,  parameters $\theta = (\theta_{\tfblock},\theta_\tokenembedding,\theta_\posencoding,\theta_\transoutput)$, and mask functions $\{M^t\}_{t=1}^T$, where $\theta_{\transformer} = (\theta^{(l)}_\mha,\theta^{(l)}_\ff )_{l=0}^{L-1}$, we define the \emph{looped transformer with mask} as  $p_{\theta,T,M}\triangleq \transoutput_{\theta_\transoutput}\circ \tfblock_{\theta_\tfblock,M^T}\circ \cdots \tfblock_{\theta_\tfblock,M^1}\circ \embed_{\theta_\tokenembedding,\theta_\posencoding}$ and the corresponding deterministic version as $\transformer_{\theta,T,M}(v_1,\ldots,v_n) \triangleq \argmax_{v\in\mathcal{V}} p_{\theta,T,M}(v|v_1,\ldots,v_n)$.
\end{definition}

\begin{definition}[Shifting Layer]\label{defi:shifting_layer}
    We define the \emph{shifting layer} $\shift:(\mathbb{R}^d)^n\to (\mathbb{R}^d)^n$ as the following for any $d,n\in\mathbb{N}^+$ and $x_1,\ldots, x_n\in\mathbb{R}^{d}$:
    \begin{align}
        \shift(x_1,x_2,x_3, \ldots,x_n) = (x_1,x_1,x_2,x_3,\ldots,x_{n-1}).
    \end{align}
\end{definition}

\begin{lemma}\label{lem:shifting_layer}
    For input sequence length up to some integer $n$, $\shift$ could be implemented by a attention layer by concatenating each embedding $x_i$ with $(\sbin_s(i), \sbin_s(f(i)))$, where $n= \lceil \log_2 n +1\rceil$.
\end{lemma}

\begin{proof}[Proof of \Cref{lem:shifting_layer}]
    It is equivalent to show we can construct an attention heads which computes $x_{f(i)}$ at each position $i$. 

    To do so, we just need to invoke \Cref{lem:simulating_hard_argmax} and use that to set key and query, so position $i$ attends to position $f(i)$. We set value at position $i$ to be $x_i$. This completes the proof. 
\end{proof}


\begin{lemma}\label{lem:mlp_addition}
    For any positive integer $s>0$, there is a constant-depth MLP $F$ with $O(s)$ hidden neurons per layer and parameters in $\Floating_s$, such that for any input $\bin(x)\in\{-1,+1\}^s$, where $0\le x\le 2^s-1$, it holds that  
    \begin{align*}
        F(x) = \bin(x+1).
    \end{align*}
\end{lemma}

\begin{proof}[Proof of \Cref{lem:mlp_addition}]
    By \Cref{lem:FB_simulating_boolean_gates}, it suffices to show that we can simulate $\bin(x)\mapsto \bin(x)+1$ using $O(s)$ wide, constant-depth boolean circuits with $\AND,\OR,\NOT$ gates with unlimited fan-ins. This is immediate by noting that 
    \begin{align}
        [\bin(x+1)]_i = [\bin(x+1)]_i \oplus \bigwedge_{j=1}^{i-1}[\bin(x+1)]_j
    \end{align}
\end{proof}

\begin{lemma}\label{lem:mlp_comparison}
    For any positive integer $s>0$, there is a constant-depth MLP $F$ with $O(s)$ hidden neurons per layer and parameters in $\Floating_s$, such that for any input $(\bin(x),\bin(y))\in\{-1,+1\}^s$, where $0\le x,y\le 2^s-1$, it holds that  
    \begin{align*}
        F(x,y) = \indct{x > y}.
    \end{align*}
\end{lemma}

\begin{proof}[Proof of \Cref{lem:mlp_comparison}]
    By \Cref{lem:FB_simulating_boolean_gates}, it suffices to show that we can simulate $\bin(x),\bin(y)\mapsto \indct{x> y}$ using $O(s)$ wide, constant-depth boolean circuits with $\AND,\OR,\NOT$ gates with unlimited fan-ins. This is immediate by noting that 
    \begin{align}
\indct{x > y} = \bigvee\limits_{i\in[s]} \left( ([\bin(x)]_i = 1) \wedge ([\bin(y)]_i = 0) \wedge \bigwedge\limits_{1\le j< i} \left([\bin(x)]_j = [\bin(y)]_j\right) \right).
    \end{align}
\end{proof}
\begin{proof}[Proof of \Cref{thm:cot_formal}]

We consider mask $M^t(i) = \indct{i-t\ge n}$, which we call it CoT masking. Let $s = \lfloor \log (n+m) +1 \rfloor$, we use $\overline i$ to denote $\sbin_s(i)$ for $1\le i\le n+m$ for convenience. The embedding size of our newly constructed transformer is larger than the target transformer to be simulated by an additive constant of $3s$. We denote the new embedding at position $i$ after $t$th loop by $(x_i^{(t)}, p_i^{(t)})$, where $x_i^{(t)}$ will be the original embedding of the transformer to be simulated, and $p_i^{(t)}$ is of dimension $3s$ and only depends on $i$ and $t$. In particular, we can show that we can set $p_i^{(t)} \triangleq (\overline i,\overline{[i-2]_+ +1}, \overline{n+t})$ to save information about position and the number of loops --- $p_i^{(0)}$ is from the positional embedding and the update is due to \Cref{lem:mlp_addition}. The size of hidden dimension of MLP and attention (key, query, value) will also be increased by $O(s)$.

The proof contains two steps:
\begin{enumerate}
    \item To show that there is a transformer with CoT masks simulating the target transformer with $m$ steps of CoT and $L$ layers by looping its own $L+O(1)$ layer block $m$ times.
    \item To show that the above looped transformer with CoT masks can be simulated by a standard looped transformer without mask and with constantly many more layers.
\end{enumerate}

For the first claim, starting from the same parameter of the transformers with CoT $\theta$, we build a new looped model with parameter $\theta'$ with constantly many more layers in each transformer block and at most constantly many more heads per attention layer. First, we can add constantly many more layers to use MLP to simulate the decoding-encoding process using \Cref{lem:simulating_decoding_embedding}.
Next, we can add one more transformer layer in each block and use the attention layer to simulate the shifting layer by~\Cref{lem:shifting_layer}, since we have the additional position embedding $p_i^(t)$.
In particular, the embedding we get at position $n+t$ after $t$ loops, $x_{n+t}^{(t)}$,  now simulates the token embedding of $n+t$ of the CoT transformer.
By the way we define CoT mask $M$, for every $t\ge -n+1$, the embedding $\hat{x}_{n+t}^{(t')}$ will keep the same for all $t'\ge \max(t,0)$. 
In $t$th loop, the only embedding update that matters happens at $n+t$th position, because no updates happen at earlier positions, and updates at later positions $n+t'$ for some $t'>t$ will be overwritten eventually in the future loops $t'$, by some value which is independent of their value at the current loop $t$. In the end, we know the embedding $x_i^{(T)}$ in \Cref{defi:looped_transformer_w_mask} is exactly equal to that in CoT transformer, and so does the final output.

For the second claim, because CoT mask can be computed by a $O(\log(n+m))$ wide, constant-depth MLP~(\Cref{lem:mlp_comparison}), together with \Cref{lem:control_gate}, we know it suffices to increase the number of layers per transformer block and embedding size and hidden dimensions by constant to simulate the transformer with mask by a transformer without mask. 
\end{proof}


% \fi



% \ns{Connecting to our $p$-hop experiments}
\begin{theorem}[Restatement of Theorem 4.2, \citep{sanford2024transformers}]\label{thm:khop_log_depth}
    $p$-hop problem (\Cref{defi:khop}) can be solved by $\lfloor\log_2 p\rfloor+2$-layer non-looped transformer with $\log n$ bits of precision, at most $3$ different layers, $d=d_{\ff}=d_{\attn}=O(1)$ embedding size, hidden dimension for MLP and attention, $1$ attention head.
\end{theorem}




\section{Computational Resources}\label{appx:compute}
We implemented \ourmethod using PyTorch~\citep{paszke2019pytorch} (offered under BSD-3 Clause license) and the PyTorch Geometric library~\citep{fey2019fast} (offered under MIT license) for efficient processing of graph-structured data. All experiments were conducted on NVIDIA RTX A5000,  NVIDIA RTX A6000, and NVIDIA GeForce RTX 4090 GPUs, and on the Google's Gemini API. For hyperparameter tuning and model selection, we used the Weights and Biases (wandb) library~\citep{wandb}.

\section{LLM Experiment Details}
\label{appx:LLMexp}
%cite the paper / exactly same thing in the code / different orders of relations / comma and quotes / example of failure (different lanuages, 20 failures of each)

\subsection{Details for Task 1} \label{appx:T1}
\paragraph{Relation Prediction Prompt of Task 1} 
In the following task, you will be given background knowledge in the form of triplet (h, r, t) which means entity 'h' has relation 'r' with entity 't'. Then you will be asked some questions about the relationship between entities. Background knowledge: (Kris Kristofferson, occupation, guitarist); (Willow Smith, genre, indie pop);\dots What is the relationship between entity 'Gaspard Monge' and entity 'France'? Please choose one best answer from the following relations:|parent organization|studies|cause of death|architectural style|unmarried partner|industry|\dots|. You just need to give the relation and please do not give an explanation.

\paragraph{Entity Prediction Prompt of Task 1} 
In the following task, you will be given background knowledge in the form of triplet (h, r, t) which means entity 'h' has relation 'r' with entity 't'. Then you will be asked some questions about the relationship between entities. Background knowledge: (Kris Kristofferson, occupation, guitarist); (Willow Smith, genre, indie pop);\dots Predict the tail entity for triplet (Gaspard Monge, country of citizenship, ?). Please give the 10 most possible answers. You just need to give the names of the entities separated by commas and please do not give explanation.

\revisionlog{In the prompt for relation prediction for all the three tasks, all relations in the dataset are listed with '|' as the delimiter. For the sake of simplicity in the presentation, in the next subsections we use "\dots" to represent the rest of triplets and relations in the prompts.}

\Cref{llm-base} shows the Hits@1 of Gemini-1.5-pro on in-domain relation prediction task. \revision{\Cref{llm-base-entity} shows the Hits@10 of Gemini-1.5-flash on in-domain entity prediction task. We use Gemini-1.5-flash for entity prediction tasks for the sake of reducing costs of experiments.} We run the experiment 3 times with the same prompt in English to see if it can generate consistent answers. Gemini performs the task quite well and shows consistency across 3 runs. This indicates given background knowledge in the prompt, LLMs has the capacity to handle in-domain relation and entity predictions well. 

% We further tests its performance across different languages. \Cref{tab:llm-relation2} shows that the LLM has consistent good performance in different languages.

\begin{table*}[ht]
\centering
\caption{Task 1: In-domain LLM relation predictions Hits@1 on CoDEx-S.}
\begin{tabular}{lrrr |r}
\toprule
 & Run \#1  & Run \#2 & Run \#3  & Worst \\
Gemini-1.5-pro & 0.933 & 0.933 & 0.933 & 0.933\\
%gpt-4-0125-preview & N/A* & N/A* & N/A* & N/A* \\
ULTRA & 0.820 & 0.820 & 0.820 & 0.820 \\
\ourmethod & {\bf 0.935} & {\bf 0.935} & {\bf 0.935} & {\bf 0.935} \\
\hline
\end{tabular}
\label{llm-base}
\end{table*}

\begin{table*}[ht]
\centering
\caption{Task 1: In-domain LLM entity predictions Hits@10 on CoDEx-S.}
\begin{tabular}{lrrr |r}
\toprule
 & Run \#1  & Run \#2 & Run \#3  & Worst \\
Gemini-1.5-flash & 0.308 & 0.308 & 0.308 & 0.308\\
%gpt-4-0125-preview & N/A* & N/A* & N/A* & N/A* \\
ULTRA & 0.667 & 0.667 & 0.667 & 0.667 \\
\ourmethod & {\bf 0.670} & {\bf 0.670} & {\bf 0.670} & {\bf 0.670} \\
\hline
\end{tabular}
\label{llm-base-entity}
\end{table*}


% \begin{table*}[ht]
% \centering
% \caption{Task 1 on different languages: In-domain relation predictions of Gemini-1.5-pro on CoDEx-S in different languages with Prompt Template 1.N/A* means we were not able to perform the experiment due to API limitations on context size.}
% \begin{tabular}{l rrrrrr rr}
% \toprule
% & \multicolumn{6}{c}{Gemini-1.5-pro} & \multicolumn{2}{c}{K-GFMs}  \\
% \cmidrule(lr){2-7}
% \cmidrule(lr){8-9}
%  & \multicolumn{1}{c}{ar} & \multicolumn{1}{c}{de} & \multicolumn{1}{c}{en} & \multicolumn{1}{c}{es} & \multicolumn{1}{c}{ru} & \multicolumn{1}{c}{zh} & \multicolumn{1}{c}{ULTRA} & \multicolumn{1}{c}{\ourmethod} \\
% \midrule
% CoDEx-S & 0.903 & 0.935 & 0.933 & 0.933 & 0.903 & 0.933 & 0.820 & \textbf{0.935} \\
% CoDEx-M & N/A* & N/A* & N/A* & N/A* & N/A* & N/A* & 0.870 & \textbf{0.886} \\
% CoDEx-L & N/A* & N/A* & N/A* & N/A* & N/A* & N/A* & 0.824 & \textbf{0.837} \\
% \bottomrule
% \end{tabular}
% \label{tab:llm-relation2}
% \end{table*}


\subsection{Details for Task 2} \label{appx:T2}
\paragraph{Relation Prediction Prompt of Task 2} In the following task, you will first be given background knowledge in the form of triplet (h, r, t) which means entity 'h' has relation 'r' with entity 't'. Then you will be asked some questions about the relationship between entities. Please notice that some words are replaced with metasyntactic words in the following paragraph. Background knowledge: (foo, baz, guitarist); (Willow Smith, genre, bar);\dots What is the relationship between entity 'foo' and entity 'bar'? Please choose one best answer from the following relation IDs:|parent organization|studies|quux|baz|\dots|. You just need to give the relation and please do not give an explanation.

\paragraph{Entity Prediction Prompt of Task 2} In the following task, you will first be given background knowledge in the form of triplet (h, r, t) which means entity 'h' has relation 'r' with entity 't'. Then you will be asked some questions about the relationship between entities. Please notice that some words are replaced with metasyntactic words in the following paragraph. Background knowledge: (foo, baz, guitarist); (Willow Smith, genre, bar);\dots Predict the tail entity for triplet (foo, garply, ?). Please give the 10 most possible answers. You just need to give the names of the entities separated by commas and please do not give explanation.

\Cref{llm-metasyntactic} shows the Hits@1 of Gemini-1.5-pro on out-domain relation prediction task. \revision{\Cref{llm-metasyntactic-entity} shows the Hits@10 of Gemini-1.5-flash on out-domain entity prediction task.} The neighbor entities of the head entity and the relations that connect the head entity with its neighbors are replaced with metasyntactic words. We run the experiment 3 times with different metasyntactic words but the underlying structural pattern is exactly the same as in the in-domain task. The results indicate the LLM can not do the out-of-domain task well. This means the LLM relied more on the known semantic description of the words instead of the structural pattern of the graph so that when there are new entities and relations, it can not perform inductive reasoning on them.

\begin{table*}[t]
\centering
\caption{Task 2: Out-of-domain LLM relation predictions Hits@1 on CoDEx-S. Effects of Metasyntactic Words on Relation Predictions on CoDEx-S.\label{llm-metasyntactic}
}
\begin{tabular}{lrrr |r}
\toprule
 & Run \#1  & Run \#2 & Run \#3  & Worst \\
 \midrule
Gemini-1.5-pro & 0.667 & 0.667 & 0.633 & 0.633\\
ULTRA & 0.820 & 0.820 & 0.820 & 0.820 \\
\ourmethod & {\bf 0.935} & {\bf 0.935} & {\bf 0.935} & {\bf 0.935} \\
%gpt-4-0125-preview & N/A* & N/A* & N/A* & N/A* \\
\hline
\end{tabular}
\end{table*}

\begin{table*}[ht]
\centering
\caption{Task 2: Out-of-domain LLM entity predictions Hits@10 on CoDEx-S. Effects of Metasyntactic Words on Entity Predictions on CoDEx-S.\label{llm-metasyntactic-entity}
}
\begin{tabular}{lrrr |r}
\toprule
 & Run \#1  & Run \#2 & Run \#3  & Worst \\
 \midrule
Gemini-1.5-flash & 0.212  & 0.250 & 0.327 & 0.212\\
ULTRA & 0.667 & 0.667 & 0.667 & 0.667 \\
\ourmethod & {\bf 0.670} & {\bf 0.670} & {\bf 0.670} & {\bf 0.670} \\
%gpt-4-0125-preview & N/A* & N/A* & N/A* & N/A* \\
\hline
\end{tabular}
\end{table*}

\subsection{Details for Task 3}\label{appx:T3}
\paragraph{Relation Prediction Prompt of Task 3} In the following task, entities and relations will be expressed with their IDs. You will first be given the mapping from entities to their IDs and the mapping from relations to their IDs. Then you will be given background knowledge in the form of triplet (h, r, t) which means entity 'h' has relation 'r' with entity 't'. Finally you will be asked some questions about the relationship between entities. Entity mapping: Mireille Darc is entity '8831'; Breton is entity '20512'; Tomas Tranströmer is entity '1641'\dots Relation mapping: located in the administrative terroritorial entity is relation '15' \dots Background knowledge: (15443, 16, 1093); (21198, 16, 9387); (14854, 8, 10218)\dots What is the relationship between entity '18127' and entity '1799'? Please choose one best answer from the following relation IDs:|45|48|27|35|\dots|. You just need to give the ID of that relation and please do not give an explanation.

\paragraph{Entity Prediction Prompt of Task 3} In the following task, entities and relations will be expressed with their IDs. You will first be given the mapping from entities to their IDs and the mapping from relations to their IDs. Then you will be given background knowledge in the form of triplet (h, r, t) which means entity 'h' has relation 'r' with entity 't'. Finally you will be asked some questions about the relationship between entities. Entity mapping: Mireille Darc is entity '8831'; Breton is entity '20512'; Tomas Tranströmer is entity '1641'\dots Relation mapping: located in the administrative terroritorial entity is relation '15' \dots Background knowledge: (15443, 16, 1093); (21198, 16, 9387); (14854, 8, 10218)\dots Predict the tail entity for triplet (18127, 45, ?). Please give the 10 most possible answers. You just need to give the IDs of the entities separated by commas and please do not give explanation.


\Cref{llm-equivarariance} shows the Hits@1 of Gemini-1.5-pro on relation prediction in Task 3. \revision{\Cref{llm-equivarariance-entity} shows the Hits@10 of Gemini-1.5-flash on entity prediction in Task 3.} The entities and relations are expressed as IDs. We run the experiment 3 times with permutated IDs but the underlying structural pattern is exactly the same as in the in-domain task. The results demonstrate that the LLM is very sensitive to ID permutation so that its performance is inconsistent across 3 permutations.

\begin{table*}[t]
\centering
\caption{Task 3: Out-of-domain LLM relation predictions Hits@1 on CoDEx-S. Effects of Input Permutations on Relation Predictions on CoDEx-S.}
\begin{tabular}{lrrr |r}
\toprule
 & Permutation \#1  & Permutation \#2 & Permutation \#3  & Worst \\
 \midrule
Gemini-1.5-pro & 0.346 & 0.731 & 0.615 & 0.346\\
%gpt-4-0125-preview & N/A* & N/A* & N/A* & N/A* \\
ULTRA & 0.820 & 0.820 & 0.820 & 0.820 \\
\ourmethod & {\bf 0.935} & {\bf 0.935} & {\bf 0.935} & {\bf 0.935} \\
\hline
\end{tabular}
\label{llm-equivarariance}
\end{table*}

\begin{table*}[t]
\centering
\caption{Task 3: Out-of-domain LLM entity predictions Hits@10 on CoDEx-S. Effects of Input Permutations on Entity Predictions on CoDEx-S.}
\begin{tabular}{lrrr |r}
\toprule
 & Permutation \#1  & Permutation \#2 & Permutation \#3  & Worst \\
 \midrule
Gemini-1.5-flash & 0.212 & 0.250 & 0.231 & 0.212\\
%gpt-4-0125-preview & N/A* & N/A* & N/A* & N/A* \\
ULTRA & 0.667 & 0.667 & 0.667 & 0.667 \\
\ourmethod & {\bf 0.670} & {\bf 0.670} & {\bf 0.670} & {\bf 0.670} \\
\hline
\end{tabular}
\label{llm-equivarariance-entity}
\end{table*}

% \begin{table*}[ht]
% \centering
% \caption{Task 3: Out-of-domain LLM entity predictions Hits@10 on CoDEx-S. Effects of Input Permutations on Entity Predictions on CoDEx-S.}
% \begin{tabular}{lrrr |r}
% \toprule
%  & Permutation \#1  & Permutation \#2 & Permutation \#3  & Worst \\
%  \midrule
% Gemini-1.5-pro & 0.231 & 0.250 & 0.346 & 0.231\\
% %gpt-4-0125-preview & N/A* & N/A* & N/A* & N/A* \\
% ULTRA & 0.667 & 0.667 & 0.667 & 0.667 \\
% \ourmethod & {\bf 0.670} & {\bf 0.670} & {\bf 0.670} & {\bf 0.670} \\
% \hline
% \end{tabular}
% \label{llm-equivarariance-entity}
% \end{table*}

% \subsection{Capacity of LLMs in different languages}
% In this part we want to evaluate whether LLMs can make consistent queries in different languages. Due to token size limit of ChatGPT, we use prompt template 4 which directly queries the LLM without providing context from the knowledge graph~\cite{yao2023exploring}. While for Gemini, we use prompt template 1.

% \paragraph{Prompt Template 4 in English} What is the relationship between \{head entity\} and \{tail entity\}? Please choose one best answer from:|parent organization|studies|cause of death|\dots|genre|

% \begin{CJK}{UTF8}{gbsn}
% \paragraph{Prompt Template 4 in Chinese} \{头节点\}和\{尾节点\}之间的关系是什么?请从以下选项中选择一个最佳答案:|母组织|研究对象|死因|\dots|艺术流派|
% \end{CJK}


% \Cref{tab:llm-relation} shows the hits@1 results comparing gpt-4-0125-preview with the double equivariant graph models. Notably, when the knowledge graph is not used as part of the prompt, ChatGPT exhibits inferior performance compared to graph models, showcasing significant variability across different languages. Specifically, it achieves the highest hits@1 scores when queries are posed in Spanish, while its performance is notably poorer when queries are conducted in Russian. On the contrary, Gemini-1.5-pro performs quite well in different languages when the whole KG is provided in the prompt as shown in \Cref{tab:llm-relation2}.

% % this table to appendix
% \begin{table*}[ht]
% \centering
% \caption{Task 4: Relation Prediction hits@1 of ChatGPT(gpt-4-0125-preview) with prompt template 4 in different languages and double equivariant graph models.}
% \begin{tabular}{l rrrrrr rr}
% \toprule
% & \multicolumn{6}{c}{ChatGPT(gpt-4-0125-preview)} & \multicolumn{2}{c}{K-GFMs}  \\
% \cmidrule(lr){2-7}
% \cmidrule(lr){8-9}
%  & \multicolumn{1}{c}{ar} & \multicolumn{1}{c}{de} & \multicolumn{1}{c}{en} & \multicolumn{1}{c}{es} & \multicolumn{1}{c}{ru} & \multicolumn{1}{c}{zh} & \multicolumn{1}{c}{ULTRA} & \multicolumn{1}{c}{\ourmethod} \\
% \midrule
% CoDEx-S & 0.455 & 0.550 & 0.610 & 0.690 & 0.333 & 0.393 & 0.820 & \textbf{0.935} \\
% CoDEx-M & 0.452 & 0.505 & 0.620 & 0.667 & 0.381 & 0.425 & 0.870 & \textbf{0.886} \\
% CoDEx-L & 0.490 & 0.558 & 0.590 & 0.698 & 0.400 & 0.409 & 0.824 & \textbf{0.837} \\
% \bottomrule
% \end{tabular}
% \label{tab:llm-relation}
% \end{table*}

% \paragraph{Example of outputs of ChatGPT} In this part we will show some examples of answers given by ChatGPT (gpt-4-0125-preview). For the sake of saving space, we omit the options of relations in the questions since they are the same as in the template.

% Here are questions that are answered wrongly in both English and Chinese:

% \begin{enumerate}
% \item
% \begin{enumerate}
% \item 
% \textbf{Question in English: }What is the relationship between Charlie Wilson and African Americans? 

% \textbf{The true answer in English: } ethnic group

% \textbf{ChatGPT's answer in English: } member of political party

% \item 
% \begin{CJK}{UTF8}{gbsn}
% \textbf{Question in Chinese: }查理·威尔森 (歌手)和非裔美国人之间的关系是什么?

% \textbf{The true answer in Chinese: }族群

% \textbf{ChatGPT's answer in Chinese: }查理·威尔森(Charlie Wilson)是一名歌手,因此与上述选项中的"职业"最为相关。他作为一名艺术家,通过音乐与人们建立联系,而非裔美国人作为他的身份之一,对他的音乐风格、歌词内容以及他在音乐界的定位有着深刻的影响。因此,如果要从提供的选项中选择一个最佳答案来描述查理·威尔森和非裔美国人之间的关系,那么"职业"是最合适不过的了。
% \end{CJK}

% \end{enumerate}

% \end{enumerate}

% Here are questions that are answered correctly in English but wrongly in Chinese:
% \begin{enumerate}
% \item 
% \begin{enumerate}
%     \item 
%     \textbf{Question in English: }What is the relationship between J. J. Abrams and producer? 
    
%     \textbf{The true answer in English: } occupation
    
%     \textbf{ChatGPT's answer in English: } occupation
%     \item 
%     \begin{CJK}{UTF8}{gbsn}
%     \textbf{Question in Chinese: }杰弗里·雅各布·艾布拉姆斯和制作人之间的关系是什么?
    
%     \textbf{The true answer in Chinese: } 职业
    
%     \textbf{ChatGPT's answer in Chinese: }杰弗里·雅各布·艾布拉姆斯(J.J. Abrams)和制作人之间的关系最佳答案是:|导演|
%     \end{CJK}
% \end{enumerate}

% \item 
% \begin{enumerate}
%     \item 
%     \textbf{Question in English: }What is the relationship between Victor Ambartsumian and German Academy of Sciences Leopoldina?
    
%     \textbf{The true answer in English: } member of
    
%     \textbf{ChatGPT's answer in English: } member of
%     \item 
%     \begin{CJK}{UTF8}{gbsn}
%     \textbf{Question in Chinese: }维克托·安巴楚勉和利奥波第那科学院之间的关系是什么?
    
%     \textbf{The true answer in Chinese: } 所属组织
    
%     \textbf{ChatGPT's answer in Chinese: }维克托·安巴楚勉和利奥波第那科学院之间的关系最合适的选择是|创作者|。
%     \end{CJK}
% \end{enumerate}

% \item 
% \begin{enumerate}
%     \item 
%     \textbf{Question in English: }What is the relationship between Theodor W. Adorno and composer?
    
%     \textbf{The true answer in English: } occupation
    
%     \textbf{ChatGPT's answer in English: } occupation
%     \item 
%     \begin{CJK}{UTF8}{gbsn}
%     \textbf{Question in Chinese: }狄奧多·阿多诺和作曲家之间的关系是什么?
    
%     \textbf{The true answer in Chinese: } 职业
    
%     \textbf{ChatGPT's answer in Chinese: }狄奧多·阿多诺和作曲家之间的关系最佳选项是:**研究对象**。阿多诺是一位德国哲学家、社会学家、音乐学家和作曲家,对20世纪音乐,特别是十二音音乐和其背后的哲学和社会基础有深入研究,因此作曲家是他研究的主要对象之一。
%     \end{CJK}
% \end{enumerate}

% \item 
% \begin{enumerate}
%     \item 
%     \textbf{Question in English: }What is the relationship between Sri Lanka and International Centre for Settlement of Investment Disputes?
    
%     \textbf{The true answer in English: } member of
    
%     \textbf{ChatGPT's answer in English: } member of
%     \item 
%     \begin{CJK}{UTF8}{gbsn}
%     \textbf{Question in Chinese: }斯里兰卡和国际投资争端解决中心之间的关系是什么?
    
%     \textbf{The true answer in Chinese: } 所属组织
    
%     \textbf{ChatGPT's answer in Chinese: }斯里兰卡和国际投资争端解决中心之间的关系最恰当的选项是“法律形式”。这是因为国际投资争端解决中心(ICSID)主要处理投资者和国家间的争端解决,是根据法律协议和程序来操作的。当投资者认为他们的投资受到了违反国际投资协定的待遇时,他们可以选择通过ICSID对相关国家(比如斯里兰卡)提起诉讼。因此,在斯里兰卡与国际投资争端解决中心之间的关系中,法律形式是关键的组成部分。
%     \end{CJK}
% \end{enumerate}
% \end{enumerate}

\section{Datasets}
\label{appx:datasets}
The statistics of all 57 datasets used in the experiments in presented in Tables~\ref{tab:app_datasets_transd},\ref{tab:app_datasets_inde},\ref{tab:app_datasets_indr}. 
All datasets are publicly available under open licenses (MIT or CC-BY).

\begin{table*}[!t]
\centering
\caption{Transductive datasets (16) used in the experiments. Train, Valid, Test denote triples in the respective set. Task denotes the prediction task: \emph{h/t} is predicting both heads and tails, \emph{tails} is only predicting tails.}
\label{tab:app_datasets_transd}
%\scriptsize
\begin{adjustbox}{width=\textwidth}
\begin{tabular}{llrrrrrl}\toprule
Dataset & Reference &Entities &Rels &Train &Valid &Test &Task \\\midrule
CoDEx Small & \cite{safavi2020codex} &2034 &42 &32888 &1827 &1828 & h/t \\
WDsinger & \cite{dackgr} &10282 &135 &16142 &2163 &2203 & h/t  \\
FB15k237\_10 & \cite{dackgr} &11512 &237 &27211 &15624 &18150 & tails \\
FB15k237\_20 & \cite{dackgr} &13166 &237 &54423 &16963 &19776 & tails \\
FB15k237\_50 & \cite{dackgr} &14149 &237 &136057 &17449 &20324 & tails \\
FB15k237 & \cite{fb15k237} &14541 &237 &272115 &17535 &20466 & h/t \\
CoDEx Medium & \cite{safavi2020codex} &17050 &51 &185584 &10310 &10311 & h/t  \\
NELL23k & \cite{dackgr} &22925 &200 &25445 &4961 &4952 & h/t  \\
WN18RR & \cite{wn18rr} &40943 &11 &86835 &3034 &3134 & h/t \\
AristoV4 & \cite{ssl_rp} &44949 &1605 &242567 &20000 &20000 & h/t  \\
Hetionet & \cite{hetionet} &45158 &24 &2025177 &112510 &112510 & h/t  \\
NELL995 & \cite{nell995} &74536 &200 &149678 &543 &2818 & h/t \\
CoDEx Large & \cite{safavi2020codex} &77951 &69 &551193 &30622 &30622 & h/t \\
ConceptNet100k & \cite{cnet100k} &78334 &34 &100000 &1200 &1200 & h/t  \\
DBpedia100k & \cite{dbp100k} &99604 &470 &597572 &50000 &50000 & h/t  \\
YAGO310 & \cite{yago310} &123182 &37 &1079040 &5000 &5000 & h/t \\
\bottomrule
\end{tabular}
\end{adjustbox}
\end{table*}

\begin{table*}[!t]
\caption{Inductive entity $(e)$ datasets (18) used in the experiments. Triples denote the number of edges of the graph given at training, validation, or test. Valid and Test denote triples to be predicted in the validation and test sets in the respective validation and test graph.}
\label{tab:app_datasets_inde}
\begin{adjustbox}{width=\textwidth}
\begin{tabular}{lrrrrrrrrrr}\toprule
\multirow{2}{*}{Dataset} &\multirow{2}{*}{Rels} &\multicolumn{2}{c}{Training Graph} &\multicolumn{3}{c}{Validation Graph} &\multicolumn{3}{c}{Test Graph}  \\ \cmidrule(l){3-4} \cmidrule(l){5-7} \cmidrule(l){8-10} 
& &Entities &Triples &Entities &Triples &Valid  &Entities &Triples &Test  \\\midrule
FB v1~\cite{teru2020inductive} &180 &1594 &4245 &1594 &4245 &489 &1093 &1993 &411  \\
FB v2~\cite{teru2020inductive} &200 &2608 &9739 &2608 &9739 &1166 &1660 &4145 &947  \\
FB v3~\cite{teru2020inductive} &215 &3668 &17986 &3668 &17986 &2194 &2501 &7406 &1731  \\
FB v4~\cite{teru2020inductive} &219 &4707 &27203 &4707 &27203 &3352 &3051 &11714 &2840  \\
WN v1~\cite{teru2020inductive} &9 &2746 &5410 &2746 &5410 &630 &922 &1618 &373 \\
WN v2~\cite{teru2020inductive} &10 &6954 &15262 &6954 &15262 &1838 &2757 &4011 &852  \\
WN v3~\cite{teru2020inductive} &11 &12078 &25901 &12078 &25901 &3097 &5084 &6327 &1143 \\
WN v4~\cite{teru2020inductive} &9 &3861 &7940 &3861 &7940 &934 &7084 &12334 &2823  \\
NELL v1~\cite{teru2020inductive} &14 &3103 &4687 &3103 &4687 &414 &225 &833 &201 \\
NELL v2~\cite{teru2020inductive} &88 &2564 &8219 &2564 &8219 &922 &2086 &4586 &935 \\
NELL v3~\cite{teru2020inductive} &142 &4647 &16393 &4647 &16393 &1851 &3566 &8048 &1620 \\
NELL v4~\cite{teru2020inductive} &76 &2092 &7546 &2092 &7546 &876 &2795 &7073 &1447 \\
ILPC Small~\cite{ilpc} &48 &10230 &78616 &6653 &20960 &2908 &6653 &20960 &2902 \\
ILPC Large~\cite{ilpc} &65 &46626 &202446 &29246 &77044 &10179 &29246 &77044 &10184  \\
HM 1k~\cite{ham_bm} &11 &36237 &93364 &36311 &93364 &1771 &9899 &18638 &476 \\
HM 3k~\cite{ham_bm} &11 &32118 &71097 &32250 &71097 &1201 &19218 &38285 &1349 \\
HM 5k~\cite{ham_bm} &11 &28601 &57601 &28744 &57601 &900 &23792 &48425 &2124\\
IndigoBM~\cite{liu2021indigo} &229 &12721 &121601 &12797 &121601 &14121 &14775 &250195 &14904 \\
\bottomrule
\end{tabular}
\end{adjustbox}
\end{table*}

\begin{table*}[!t]
\caption{Inductive entity and relation $(e,r)$ datasets (23) used in the experiments. Triples denote the number of edges of the graph given at training, validation, or test. Valid and Test denote triples to be predicted in the validation and test sets in the respective validation and test graph.}
\label{tab:app_datasets_indr}
%\scriptsize
\begin{adjustbox}{width=\textwidth}
\begin{tabular}{lrrrrrrrrrrrrrr}\toprule
\multirow{2}{*}{Dataset} &\multicolumn{3}{c}{Training Graph} &\multicolumn{4}{c}{Validation Graph} &\multicolumn{4}{c}{Test Graph} \\ \cmidrule(l){2-4} \cmidrule(l){5-8} \cmidrule(l){9-12}
&Entities &Rels &Triples &Entities &Rels &Triples &Valid &Entities &Rels &Triples &Test \\\midrule
FB-25~\cite{lee2023ingram} &5190 &163 &91571 &4097 &216 &17147 &5716 &4097 &216 &17147 &5716  \\
FB-50~\cite{lee2023ingram} &5190 &153 &85375 &4445 &205 &11636 &3879 &4445 &205 &11636 &3879  \\
FB-75~\cite{lee2023ingram} &4659 &134 &62809 &2792 &186 &9316 &3106 &2792 &186 &9316 &3106  \\
FB-100~\cite{lee2023ingram} &4659 &134 &62809 &2624 &77 &6987 &2329 &2624 &77 &6987 &2329  \\
WK-25~\cite{lee2023ingram} &12659 &47 &41873 &3228 &74 &3391 &1130 &3228 &74 &3391 &1131  \\
WK-50~\cite{lee2023ingram} &12022 &72 &82481 &9328 &93 &9672 &3224 &9328 &93 &9672 &3225  \\
WK-75~\cite{lee2023ingram} &6853 &52 &28741 &2722 &65 &3430 &1143 &2722 &65 &3430 &1144  \\
WK-100~\cite{lee2023ingram} &9784 &67 &49875 &12136 &37 &13487 &4496 &12136 &37 &13487 &4496  \\
NL-0~\cite{lee2023ingram} &1814 &134 &7796 &2026 &112 &2287 &763 &2026 &112 &2287 &763  \\
NL-25~\cite{lee2023ingram} &4396 &106 &17578 &2146 &120 &2230 &743 &2146 &120 &2230 &744  \\
NL-50~\cite{lee2023ingram} &4396 &106 &17578 &2335 &119 &2576 &859 &2335 &119 &2576 &859  \\
NL-75~\cite{lee2023ingram} &2607 &96 &11058 &1578 &116 &1818 &606 &1578 &116 &1818 &607  \\
NL-100~\cite{lee2023ingram} &1258 &55 &7832 &1709 &53 &2378 &793 &1709 &53 &2378 &793  \\
\midrule
Metafam~\cite{mtdea} &1316 &28 &13821 &1316 &28 &13821 &590 &656 &28 &7257 &184 \\
FBNELL~\cite{mtdea} &4636 &100 &10275 &4636 &100 &10275 &1055 &4752 &183 &10685 &597  \\
Wiki MT1 tax~\cite{mtdea} &10000 &10 &17178 &10000 &10 &17178 &1908 &10000 &9 &16526 &1834  \\
Wiki MT1 health~\cite{mtdea} &10000 &7 &14371 &10000 &7 &14371 &1596 &10000 &7 &14110 &1566  \\
Wiki MT2 org~\cite{mtdea} &10000 &10 &23233 &10000 &10 &23233 &2581 &10000 &11 &21976 &2441  \\
Wiki MT2 sci~\cite{mtdea} &10000 &16 &16471 &10000 &16 &16471 &1830 &10000 &16 &14852 &1650  \\
Wiki MT3 art~\cite{mtdea} &10000 &45 &27262 &10000 &45 &27262 &3026 &10000 &45 &28023 &3113  \\
Wiki MT3 infra~\cite{mtdea} &10000 &24 &21990 &10000 &24 &21990 &2443 &10000 &27 &21646 &2405  \\
Wiki MT4 sci~\cite{mtdea} &10000 &42 &12576 &10000 &42 &12576 &1397 &10000 &42 &12516 &1388  \\
Wiki MT4 health~\cite{mtdea} &10000 &21 &15539 &10000 &21 &15539 &1725 &10000 &20 &15337 &1703  \\
\bottomrule
\end{tabular}
\end{adjustbox}
\end{table*}

\section{Detailed Experiment Results of Entity and Relation Prediction}
\label{appx:detailedresults}

\section{Overhead of \ourSystem.}
We report the size and inference time of the model for NeRF and \ourSystem in Table~\ref{table_overhead}.  
The results indicate that \ourSystem has a larger model size than NeRF, \ie 27.1 \vs 8.0\,MB.  
Correspondingly, \ourSystem exhibits a longer inference time, \ie 1.79 \vs 0.43\,s.  
Unlike NeRF, \ourSystem requires neighboring spectra as input. 
During inference, the target transmitter's neighbors are extracted from the training dataset, so \ourSystem does not incur additional data burdens.  
Moreover, since \ourSystem can operate in unseen scenes, it significantly reduces the requirement for a time-consuming training process.



\begin{table}[h]
\centering
\caption{Comparison of model size and inference time.}

\begin{tabular}{lC{0.8in}C{0.8in}}
\toprule
     & \nerft    & \ourSystem    
     \\ \midrule
Model size (MB) & 8.0  & 27.1   \\
Inference time (s) & 0.43    & 1.79  \\
\bottomrule
\end{tabular}
\label{table_overhead}
\end{table}






\end{document}
