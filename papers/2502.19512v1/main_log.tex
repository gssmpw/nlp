\documentclass{article}
\usepackage{log_2024}						% for camera-ready version
% \usepackage[review]{log_2024}				% for anonymous submission to proceedings track
% \usepackage[review,eatrack]{log_2024}		% for anonymous submission to extended abstract track
% \usepackage[preprint]{log_2024}			% for preprint version
% \usepackage[eatrack]{log_2024}				% for accepted extended abstracts

%%%%% NEW MATH DEFINITIONS %%%%%

% \usepackage{amsmath,amsfonts,bm}
\usepackage{amsmath,amsfonts}

\usepackage{pifont}


\newcommand{\R}{\mathbb{R}}


\def\va{{\mathbf{a}}}
\def\vg{{\mathbf{g}}}

% Sets
\def\sR{\mathbb{R}}
\def\sC{\mathbb{C}}
\def\sZ{\mathbb{Z}}
\def\sN{\mathbb{N}}
\def\sQ{\mathbb{Q}}

\def\sS{\mathcal{S}}



% Vectors
\def\vzero{{\mathbf{0}}}
\def\vone{{\mathbf{1}}}
\def\vmu{{\mathbf{\mu}}}
\def\vtheta{{\mathbf{\theta}}}
\def\va{{\mathbf{a}}}
\def\vb{{\mathbf{b}}}
\def\vc{{\mathbf{c}}}
\def\vd{{\mathbf{d}}}
\def\ve{{\mathbf{e}}}
\def\vf{{\mathbf{f}}}
\def\vg{{\mathbf{g}}}
\def\vh{{\mathbf{h}}}
\def\vi{{\mathbf{i}}}
\def\vj{{\mathbf{j}}}
\def\vk{{\mathbf{k}}}
\def\vl{{\mathbf{l}}}
\def\vm{{\mathbf{m}}}
\def\vn{{\mathbf{n}}}
\def\vo{{\mathbf{o}}}
\def\vp{{\mathbf{p}}}
\def\vq{{\mathbf{q}}}
\def\vr{{\mathbf{r}}}
\def\vs{{\mathbf{s}}}
\def\vt{{\mathbf{t}}}
\def\vu{{\mathbf{u}}}
\def\vv{{\mathbf{v}}}
\def\vw{{\mathbf{w}}}
\def\vx{{\mathbf{x}}}
\def\vy{{\mathbf{y}}}
\def\vz{{\mathbf{z}}}
\def\vzeta{{\mathbf{\zeta}}}

% Matrix
\def\mA{{\mathbf{A}}}
\def\mB{{\mathbf{B}}}
\def\mC{{\mathbf{C}}}
\def\mD{{\mathbf{D}}}
\def\mE{{\mathbf{E}}}
\def\mF{{\mathbf{F}}}
\def\mG{{\mathbf{G}}}
\def\mH{{\mathbf{H}}}
\def\mI{{\mathbf{I}}}
\def\mJ{{\mathbf{J}}}
\def\mK{{\mathbf{K}}}
\def\mL{{\mathbf{L}}}
\def\mM{{\mathbf{M}}}
\def\mN{{\mathbf{N}}}
\def\mO{{\mathbf{O}}}
\def\mP{{\mathbf{P}}}
\def\mQ{{\mathbf{Q}}}
\def\mR{{\mathbf{R}}}
\def\mS{{\mathbf{S}}}
\def\mT{{\mathbf{T}}}
\def\mU{{\mathbf{U}}}
\def\mV{{\mathbf{V}}}
\def\mW{{\mathbf{W}}}
\def\mX{{\mathbf{X}}}
\def\mY{{\mathbf{Y}}}
\def\mZ{{\mathbf{Z}}}
\def\mBeta{{\mathbf{\beta}}}
\def\mPhi{{\mathbf{\Phi}}}
\def\mLambda{{\mathbf{\Lambda}}}
\def\mSigma{{\mathbf{\Sigma}}}


% Expectation
% \def\eE{\mathop{\mathbb{E}}\limits}
\def\eE{\mathbb{E}}

% Probability
\def\pP{\mathbb{P}}

% Tilde
\def\tf{\tilde{f}}
\def\tS{\tilde{S}}
\def\wtF{\widetilde{\mathcal{F}}}
\def\whR{\widehat{R}}
\def\tvx{\tilde{\mathbf{x}}}
\def\ty{\tilde{y}}


\def\defeq{\overset{\textup{def}}{=}}
% \def\defeq{\overset{.}{=}}
\def\defone{\overset{\text{\ding{172}}}{=}}
\def\deftwo{\overset{\text{\ding{173}}}{=}}
\def\leqone{\overset{\text{\ding{172}}}{\leq}}
\def\leqtwo{\overset{\text{\ding{173}}}{\leq}}
\def\leqthree{\overset{\text{\ding{174}}}{\leq}}
\def\leqfour{\overset{\text{\ding{175}}}{\leq}}
\def\eqone{\overset{\text{\ding{172}}}{=}}
\def\eqtwo{\overset{\text{\ding{173}}}{=}}
\def\eqthree{\overset{\text{\ding{174}}}{=}}
\def\eqfour{\overset{\text{\ding{175}}}{=}}
\def\geqfive{\overset{\text{\ding{176}}}{\geq}}

\usepackage{booktabs}						% professional-quality tables
\usepackage{multirow}						% tabular cells spanning multiple rows
\usepackage{amsfonts}						% blackboard math symbols
\usepackage{graphicx}						% figures
\usepackage{duckuments}						% sample images

% If you want to use natbib:
\usepackage[numbers,compress,sort]{natbib}	% for numerical citations
% \usepackage[sort,round]{natbib}			% for textual citations

% If you want to use bibLaTeX, uncomment below:
% \usepackage[
%      backend=biber,
%      style=numeric-comp,
%      backref=true,
%      natbib=true]{biblatex}
% \addbibresource{reference.bib}

\usepackage[utf8]{inputenc} % allow utf-8 input
\usepackage[T1]{fontenc}    % use 8-bit T1 fonts
\usepackage[backref=page]{hyperref}       % hyperlinks
\usepackage{url}            % simple URL typesetting
\usepackage{multirow}
\usepackage{booktabs}       % professional-quality tables
\usepackage{amsfonts}       % blackboard math symbols
\usepackage{nicefrac}       % compact symbols for 1/2, etc.
\usepackage{microtype}      % microtypography
\usepackage{xcolor}         % colors
\usepackage{xspace}
\usepackage{adjustbox}
%\usepackage
\usepackage{xcolor}
% \hypersetup{
%     colorlinks,
%     linkcolor={red!50!black},
%     citecolor={blue!50!black},
%     urlcolor={blue!80!black}
% }
\usepackage{wrapfig}


\usepackage{bbm}
\usepackage{CJKutf8}

%%%%%% NEW MATH DEFINITIONS %%%%%

% \usepackage{amsmath,amsfonts,bm}
\usepackage{amsmath,amsfonts}

\usepackage{pifont}


\newcommand{\R}{\mathbb{R}}


\def\va{{\mathbf{a}}}
\def\vg{{\mathbf{g}}}

% Sets
\def\sR{\mathbb{R}}
\def\sC{\mathbb{C}}
\def\sZ{\mathbb{Z}}
\def\sN{\mathbb{N}}
\def\sQ{\mathbb{Q}}

\def\sS{\mathcal{S}}



% Vectors
\def\vzero{{\mathbf{0}}}
\def\vone{{\mathbf{1}}}
\def\vmu{{\mathbf{\mu}}}
\def\vtheta{{\mathbf{\theta}}}
\def\va{{\mathbf{a}}}
\def\vb{{\mathbf{b}}}
\def\vc{{\mathbf{c}}}
\def\vd{{\mathbf{d}}}
\def\ve{{\mathbf{e}}}
\def\vf{{\mathbf{f}}}
\def\vg{{\mathbf{g}}}
\def\vh{{\mathbf{h}}}
\def\vi{{\mathbf{i}}}
\def\vj{{\mathbf{j}}}
\def\vk{{\mathbf{k}}}
\def\vl{{\mathbf{l}}}
\def\vm{{\mathbf{m}}}
\def\vn{{\mathbf{n}}}
\def\vo{{\mathbf{o}}}
\def\vp{{\mathbf{p}}}
\def\vq{{\mathbf{q}}}
\def\vr{{\mathbf{r}}}
\def\vs{{\mathbf{s}}}
\def\vt{{\mathbf{t}}}
\def\vu{{\mathbf{u}}}
\def\vv{{\mathbf{v}}}
\def\vw{{\mathbf{w}}}
\def\vx{{\mathbf{x}}}
\def\vy{{\mathbf{y}}}
\def\vz{{\mathbf{z}}}
\def\vzeta{{\mathbf{\zeta}}}

% Matrix
\def\mA{{\mathbf{A}}}
\def\mB{{\mathbf{B}}}
\def\mC{{\mathbf{C}}}
\def\mD{{\mathbf{D}}}
\def\mE{{\mathbf{E}}}
\def\mF{{\mathbf{F}}}
\def\mG{{\mathbf{G}}}
\def\mH{{\mathbf{H}}}
\def\mI{{\mathbf{I}}}
\def\mJ{{\mathbf{J}}}
\def\mK{{\mathbf{K}}}
\def\mL{{\mathbf{L}}}
\def\mM{{\mathbf{M}}}
\def\mN{{\mathbf{N}}}
\def\mO{{\mathbf{O}}}
\def\mP{{\mathbf{P}}}
\def\mQ{{\mathbf{Q}}}
\def\mR{{\mathbf{R}}}
\def\mS{{\mathbf{S}}}
\def\mT{{\mathbf{T}}}
\def\mU{{\mathbf{U}}}
\def\mV{{\mathbf{V}}}
\def\mW{{\mathbf{W}}}
\def\mX{{\mathbf{X}}}
\def\mY{{\mathbf{Y}}}
\def\mZ{{\mathbf{Z}}}
\def\mBeta{{\mathbf{\beta}}}
\def\mPhi{{\mathbf{\Phi}}}
\def\mLambda{{\mathbf{\Lambda}}}
\def\mSigma{{\mathbf{\Sigma}}}


% Expectation
% \def\eE{\mathop{\mathbb{E}}\limits}
\def\eE{\mathbb{E}}

% Probability
\def\pP{\mathbb{P}}

% Tilde
\def\tf{\tilde{f}}
\def\tS{\tilde{S}}
\def\wtF{\widetilde{\mathcal{F}}}
\def\whR{\widehat{R}}
\def\tvx{\tilde{\mathbf{x}}}
\def\ty{\tilde{y}}


\def\defeq{\overset{\textup{def}}{=}}
% \def\defeq{\overset{.}{=}}
\def\defone{\overset{\text{\ding{172}}}{=}}
\def\deftwo{\overset{\text{\ding{173}}}{=}}
\def\leqone{\overset{\text{\ding{172}}}{\leq}}
\def\leqtwo{\overset{\text{\ding{173}}}{\leq}}
\def\leqthree{\overset{\text{\ding{174}}}{\leq}}
\def\leqfour{\overset{\text{\ding{175}}}{\leq}}
\def\eqone{\overset{\text{\ding{172}}}{=}}
\def\eqtwo{\overset{\text{\ding{173}}}{=}}
\def\eqthree{\overset{\text{\ding{174}}}{=}}
\def\eqfour{\overset{\text{\ding{175}}}{=}}
\def\geqfive{\overset{\text{\ding{176}}}{\geq}}
\usepackage{cleveref}
%\usepackage{graphicx}
\usepackage{subcaption}

\usepackage{algpseudocode}
\usepackage{algorithm}

\newcommand{\ourmethod}{TRIX\xspace}


\title{TRIX: A More Expressive \revision{Model for Zero-shot\\ 
Domain Transfer in Knowledge Graphs}}

\author[Y. Zhang et al.]{%
Yucheng Zhang\\
Purdue University\\
\email{zhan4332@purdue.edu}\And
Beatrice Bevilacqua\\
Purdue University\\
\email{bbevilac@purdue.edu}\AND
Mikhail Galkin\\
Intel AI Lab\\
\email{mikhail.galkin@intel.com}\And
Bruno Ribeiro\\
Purdue University\\
\email{ribeirob@purdue.edu}
}

\begin{document}

\maketitle

\begin{abstract}
\revision{Fully inductive knowledge graph models can be trained on multiple domains and subsequently perform zero-shot knowledge graph completion (KGC) in new unseen domains.
This is an important capability towards the goal of having foundation models for knowledge graphs.}
  In this work, we introduce a more expressive and capable \revision{fully inductive model}, dubbed \ourmethod, which not only yields strictly more expressive triplet embeddings (head entity, relation, tail entity)  compared to state-of-the-art methods, but also introduces a new capability: directly handling both entity and relation prediction tasks in inductive settings. 
  Empirically, we show that \ourmethod outperforms the state-of-the-art fully inductive models in zero-shot entity and relation predictions in new domains, and outperforms large-context LLMs in out-of-domain predictions.
  % Empirically, we show that \ourmethod\ surpasses state-of-the-art \revision{fully inductive models}, with an average improvement of up to 7.4\% in hits@10 accuracy in zero-shot entity or relation prediction, as well as fine-tuning scenarios, across a diverse set of 57 datasets. Additionally, we conduct a comparison with large-context Large Language Models (LLMs) on zero-shot relation \revision{and entity} prediction tasks. Our results demonstrate that while LLMs achieve comparable performance to \ourmethod in-domain, their predictive capabilities degrade significantly out-of-domain (exhibiting high sensitivity to relation and entity permutations in the knowledge graph prompt), while \ourmethod maintains robust performance in these scenarios. This underscores the importance of further developing \revision{fully inductive models} to reason across domains. 
  The source code is available at \revision{\url{https://github.com/yuchengz99/TRIX}}.
\end{abstract}

\section{Introduction}
%\vspace{-5pt}
\documentclass[../main.tex]{subfiles}
\graphicspath{{../images/}}
\makeatletter
\def\input@path{{../images/}}
\makeatother
\begin{document}
\section{Introduction}
\begin{figure}
\centering
\begin{tikzpicture}
\node[inner sep=0pt] (ws) at (0, 0) {
\includegraphics[height=.4\textwidth, trim={10cm 0 10cm 0},clip]{world_space.png}};
\node[inner sep=0pt] (cs) at (6,0) {\includegraphics[height=.4\textwidth, trim={10cm 1cm 10cm 4cm},clip]{conf_space.png}};
\end{tikzpicture}
\vspace{-5pt}
\label{fig:pbrm_intro}
\caption{\textbf{Left}: Shows world space obstacles as grey spheres. Robots start and goal configuration is colored red and green, respectively. Configurations along the computed path are colored transparent blue. \textbf{Right:} Mapped world space scenario to configuration space. Obstacle region is the grey mesh. Red spheres are collision-free regions computed by the neural SCDF. The optimized shortest path in the convex corridor is the blue curve.}
\vspace{-25pt}
\end{figure}
Motion planning is the problem of finding a collision-free trajectory that connects a given start and goal configuration. The planning takes place in the configuration space of the robot. For single body robots, like mobile robots or drones, the configuration space and the world space are usually the same. This simplifies the planning, since explicit obstacle representations are available which enables geometrical tools like separating hyperplanes, smallest distance to obstacles etc., to be used when designing motion planning algorithms. For multi-body robots like manipulators, the situation is completely different. The world space obstacles are usually mapped to non-convex regions, and to make the problem even harder, the mapping is usually not known. Forming explicit representations of the obstacle region in the configuration space is usually too expensive or intractable. Despite all of this, sampling based planners are used with great success, which mainly is due to their use of implicit representations of the obstacle region. The basic idea is to construct a graph in the configuration space that covers and connects the collision-free region. From this graph, a path can be extracted that connects a given start and goal configuration. The approach is computationally expensive, since the graph is constructed with the smallest geometrical building block available, points, which represents a collision-check. Furthermore, the extracted paths from the graph are non-smooth and jagged due to the stochastic nature of the approach. This adds an additional post-processing step to the process, where the paths are shortcutted and smoothened, before the path can be used for tracking. Clearly a lot of time is invested to form this graph and produce smooth paths. Thus, if the obstacles start to move, then all of this work is done in no use, since all points that make up this graph need to be re-verified, which is simply too time consuming to be done in real time.
\\\\
In this work, we want to address the existing drawbacks of the sampling based planners. Our main contribution is an improved motion planner where each vertex in the graph covers a collision-free region in the form of a sphere instead of a point and where the edges are formed with neighboring intersecting spheres. This representation has the advantage of instead of returning piecewise linear paths, returning a sequence of overlapping spheres, i.e. a convex corridor, that connects a given start and goal configuration, illustrated in Figure \ref{fig:pbrm_intro}. This convex corridor allows us to use convex optimization to produce smooth trajectories, instead of computationally expensive post-processing methods. The representation further allows us to estimate the coverage of the collision-free space, which gives us awareness and feedback in the offline roadmap construction phase. Finally, our representation is simple to adapt to moving obstacles, simply requery for the new radii and recheck for intersections. 
\\\\
The spherical collision-free regions are formed using a signed distance function (SDF), which is a function that returns the smallest distance from an arbitrary point to the boundary of an obstacle. As the name implies, the distance is signed, thus if the point is inside the obstacle it is negative otherwise positive. If the distance is positive, a sphere with radius equal to the distance is guaranteed to cover a collision-free region. Using an SDF in motion planning is not new, but what is novel about our approach is that we express the distance in the configuration space instead of the world space and by doing so allows us to form these convex collision-free regions. We refer to the resulting SDF as a signed configuration distance function (SCDF). Computing an SCDF analytically is non-trivial, our approach is therefore to parameterize the SCDF with a deep neural network and learn the mapping by supervised learning. Our resulting neural SCDF can compute distances for different parameter values of obstacle shapes and we also show how multiple distances can be combined, thus making our approach flexible.
\section{Related work}
Motion planning algorithms can roughly be divided into three families, grid-based, sampling based and optimization based methods. Grid-based methods (GBM) discretize the planning space from which a graph is then compiled. A standard search method is A$^\star$ \citep{a_star}, which is classified as an \textit{informed} search method, since it employs a heuristic function to speed up the search. A$^\star$ guarantees to return an optimal path at the level of discretization used. GBMs usually discretize the planning space by a regular lattice and this limits the GBMs to problems with low dimensionality due to the curse of dimensionality. Thus, GBMs are usually limited to single-body robots where the degrees of freedom (DOF) are low. To overcome the inherent scaling problem with the GBMs, stochastic methods are usually used for multi-body robots. These methods are termed as sampling-based methods (SBM) and core members within this family are the rapidly-exploring random trees (RRT) \citep{rrt} and the probabilistic roadmap (PRM) \citep{prm}. RRT grows a tree from the start configuration and explores the collision-free region in a rapid way until it is able to connect to the goal region. RRT is usually improved by bi-directional planning \citep{rrt_connect}, i.e. an additional tree is grown from the goal configuration and the trees are tested for connection after any tree has been expanded. RRT is a single-query method, thus it searches for a path from scratch each time it is queried. Contrary to this, PRM is a multi-query method, which solves for multiple queries without starting from scratch. PRM does this by creating a roadmap (graph) that covers the collision-free space as an offline step. The graph is then used to solve for multiple queries. PRMs are used in cases where the environment does not change since the extra offline step is too computationally costly and needs to be re-done if the environment is changed. In our work, we address this inherent issue by using a different roadmap representation. Our vertices in the graph cover a collision-free region in the form of spheres and we form the edges by checking for intersecting spheres. If something in the environment changes, we recompute the spheres radii and recheck the intersections, without relying on collision detection. We use a trained neural network to compute the sphere radius, therefore querying for the radius can be done fast, hence our representation enables the PRM for dynamic environments.
\\\\
In the recent decades, optimization based methods (OBM) \citep{chomp, schulman, itomp, stomp} have been introduced as an alternative to SBM for multi-body robots. Like the SBM, the OBMs scale well to higher dimensional problems and produce smoother motion. It is common to use a SDF in the optimization since it is a smooth function, thus enabling gradient-based methods. However, the standard way of expressing the SDF is in world space. The distance therefore needs to be mapped to the configuration space by the forward kinematics. This mapping makes the optimization problem a non-linear program (NLP), which is computationally expensive to solve. Recently, a different approach has been proposed. In \cite{mp_gcs} motion planning is formulated as a convex optimization problem by using the graph of convex sets framework \citep{gcs}. The underlying idea is to decompose the collision-free space into intersecting convex sets from which a convex optimization problem is formulated. In cases where an explicit representation of the obstacles in the configuration space exists, like for single-body robots, creating collision-free convex regions can be done fast \citep{iris}. For multi-body robots, this is non-trivial. Existing work does this successfully \citep{iris_nlp, iris_c} by an optimization based approach, but the methods are still too time consuming to be used in the presence of moving obstacles. Our approach is instead to use deep learning to learn an SDF expressed in the configuration space. With this, we can query for shortest distances to the collision boundary, which allows us to expand spherical regions which are collision-free. Our approach is fast and therefore enables our suggested roadmap planner to be used in dynamic environments.
\\\\
Recent research has focused on learning collision detection \citep{fk_kernel_distance, diffco, graphdistnet} by predicting the signed distance between the robot links and the surrounding obstacles in the world space. The learned SDF is used in trajectory optimization but since the distance is expressed in the world space, the problem becomes an NLP and therefore takes a long time to solve. We take a novel approach and suggest to instead express the signed distance in the configuration space. This allows us to improve the PRM at the same time as it enables convex optimization for trajectory optimization, which runs faster and is more reliable than NLP solvers. In \cite{cspf} a learned signed distance function in the configuration space is proposed similar to our approach. However, their approach is restricted to point cloud representations, while we propose to represent the obstacles as parameterized geometric shapes, e.g. spheres. Furthermore, we also show how to use our learned SCDF to improve an existing roadmap planner.
\section{Problem formulation}
A robot is located in the world space, $\W \subset \R^3 $. The unique location of the robot is given by its configuration $\q \in \C$, where $\C$ is the configuration space. The set of points covered by the robots bodies at a certain configuration is expressed as $\B(\q) \subset \W$. The robot is surrounded by $\NrObst$ obstacles $\O = \bigcup_{i=1}^{\NrObst} \O_i$, where  $\O_i \subset \W$. The representation of the obstacle in the configuration space is the set $\C\O_i = \{\q \in \C \: |\: \B(\q) \cap \O_i \neq \emptyset \}$. The obstacle space is formed as $\Co = \bigcup_{i=1}^{\NrObst} \C \O_i$. The complement is referred to as the free space, $\Cf = \C \setminus \Co$. The path planning problem is a tuple, ($\Cf$, $\qStart$, $\qGoal$), where we want to connect a query pair, consisting of a start, $\qStart$, and goal configuration, $\qGoal$, with a geometric path, $\q(s): [0, 1] \mapsto \Cf$, such that $\q(0)=\qStart$ and $\q(1)=\qGoal$, or report correctly when such a path does not exist.
\end{document}


%\vspace{-5pt}
\section{Related Work}
%\vspace{-5pt}
\section{Related Work}
% \subsection{Vision Language Model}
% 시각장애인에서 상황을 설명할 DB가 없으니 만들었다. 그리고 이를 VLM에 튜닝했다.
\subsection{Technical approaches for assisting the visually-impaired}


\subsection{Datasets for visual instruction tuning}


%\vspace{-15pt}
\section{Preliminaries}\label{sec:preliminary}
%\vspace{-8pt}
\section{Preliminaries}
\label{sec:preliminaries}
We first set up notations and mathematically formulate tasks.

\noindent\textbf{Language-Conditioned Imitation Learning (LC-IL)}. The task of LC-IL aims to train an agent to mimic expert behaviors from a given demonstration set $\mathcal{D}_d = \{(\mathbf{\tau}_i,l_i)\}_{i=1}^N$, where $l_i \in \mathcal{L} $ represents a task-specific language instruction. Each trajectory $\mathbf{\tau}_i\in\mathcal{T}$ consists of a sequence of state-action pairs $\mathbf{\tau}_i = \{(\mathbf{s}_j, \mathbf{a}_j)\}_{j=1}^T$ of the horizon length $T$. In robot manipulation tasks, action $\mathbf{a}_j\in\mathcal{A}$ corresponds to the control commands executed by the agent and state $\mathbf{s}_j = [\mathbf{p}_j; \mathbf{v}_j] \in\mathcal{S}$ records proprioceptive data $\mathbf{p}_j$ (\textit{e.g.,} joint positions, velocities) and visual inputs $\mathbf{o}_j\in\mathcal{O}$ (\textit{e.g.,} camera images) at the time step $j$. The objective of LC-IL is to find an optimal language-conditioned policy $\pi^*(\mathbf{a}|\mathbf{s},l): \mathcal{S}\times\mathcal{L}\mapsto\mathcal{A}$ via solving the supervised optimization as follows,
\begin{equation}\nonumber
    \pi^* \in \arg\min_{\pi} \mathbb{E}_{(\tau_i, l_i)\sim \mathcal{T}} \left[ \frac{1}{T} \sum_{(\mathbf{s}_j, \mathbf{a}_j) \sim \tau_i} \ell(\pi(\hat{\mathbf{a}}_j, \mathbf{s}_j|l_i),  \mathbf{a}_j)\right],
\end{equation}
where \(\ell(\cdot, \cdot)\) is a task-specific loss, such as mean squared error or cross-entropy. Training the policy \(\pi_\theta\) in an end-to-end fashion may require \textit{hundreds} of high-quality expert demonstrations to converge, primarily due to the high variance of visual inputs $\mathbf{o}$ and language instructions $l$.

% We study the problem of Language-Conditioned Imitation Learning ~\cite{rss21-gcil}, where the goal is to train an agent to perform tasks by conditioning its policy on both the state of the environment and language instruction. Formally, let \(\mathcal{O}\) be the observation space, \(\mathcal{A}\) the action space, and \(\mathcal{L}\) the language instruction space. The observation space \(\mathcal{O}\) typically includes visual or sensor data, such as images, that represent the partial observation of state \(\mathcal{S}\). The objective is to learn a policy \(\pi_\theta : \mathcal{O} \times \mathcal{L} \to \mathcal{A}\), parameterized by \(\theta\), that maps an observation \(o \in \mathcal{O}\) and a language instruction \(L \in \mathcal{L}\) to an action \(a \in \mathcal{A}\). We assume access to a dataset of expert demonstrations \(\mathcal{D}_{\operatorname{demo}} = \{(\{o_k^i, a_k^i\}_{i=1}^T, L_k)\}_{k=1}^N\), where each sample consists of a $T$-step observation-action trajectory and a corresponding language instruction \(L_k \in \mathcal{L}\). The goal is to train the policy \(\pi_\theta\) by minimizing the following loss function:
% \[
% \mathcal{L}(\theta) = \frac{1}{N} \sum_{k=1}^N \sum_{i=1}^T \ell(a_k^i, \pi_\theta(o_k^i, L_k)),
% \]
% where \(\ell(\cdot, \cdot)\) is a task-specific loss function, such as mean squared error or cross-entropy. 
\begin{table}
\centering
\caption{Comparison of different component designs in time contrast learning across mainstream vision-language pre-training. \vspace{1ex}
% The goal frame $o_g$ is typically set as the last frame $o_{T}$.
 }
\label{tab:comp}
\Large
\resizebox{\linewidth}{!}{ 
\begin{tabular}{llll}
\toprule
$\operatorname{Method}$      & \textcolor{black}{$\mathcal{P}(\mathcal{O}_{i})$}  & \textcolor{black}{$\mathcal{N}(\mathcal{O}_{i})$} & $\mathfrak{R}(\mathbf{v},\mathbf{l}_i)$  \\ \hline
$\operatorname{R3M}$         & $(o_0, o_{j>i})$      &  $(o_0,o_i,o_j^{\notin O_i})$   & $\operatorname{reward}(\mathbf{v},\mathbf{l}_i)$   \\    
$\operatorname{LIV}$         & $(o_T)$    &  $(o_T^{\notin O_i})$    & $\operatorname{cos}(\mathbf{v},\mathbf{l}_i)$  \\    
$\operatorname{DecisionNCE}$ & $(o_i,o_{j>i})$     &     $(o_i^{\notin O_i},o_{j>i}^{\notin O_i})$  & $\operatorname{cos}(\mathbf{v}_j-\mathbf{v}_i, \mathbf{l}_i)$  \\          
$\operatorname{AcTOL}$        & $(o_i,o_{j \in [T] \setminus \{i\}})$ & $(o_i,o_k: d_{i, k}>d_{i, j})$  & $-\Vert \operatorname{cos}(\mathbf{v}_i, \mathbf{l}_i)-\operatorname{cos}(\mathbf{v}_j, \mathbf{l}_i) \Vert_2 $     \\  \bottomrule                                                              
\end{tabular}
}
\end{table}

\paragraph{Vision-language Pre-training.}  Address such scalability issues can be achieved by leveraging large-scale, easily accessible human action video datasets $\mathcal{D}_p = \{(\mathcal{O}_i, l_i)\}_{i=1}^M$ \cite{corr18-epickitchen,cvpr22-ego4d}, where $\mathcal{O}_i=\{o_j\}_{j=1}^T$ represents a video clip with $T$ frames and $l_i$ the corresponding description. Pretraining on such datasets enables policies to rapidly learn visual-language correspondences with minimal expert demonstrations. Mainstream pretraining methods employ time contrastive learning \cite{icra18-tcn} to fine-tune a visual encoder $\mathcal{\phi}$ and a text encoder $\mathcal{\varphi}$, which project frames and descriptions into a shared $d$-dimensional embedding space, \textit{i.e.}, $\mathbf{v}_j = \phi(o_j)\in\mathbb{R}^d$ and $\mathbf{l}_i = \varphi(l_i)\in\mathbb{R}^d$. To provide a unified perspective on various pretraining approaches, we formulate them within the objective $\mathcal{L}_{\operatorname{tNCE}}(\phi, \varphi)$: \vspace{-2ex}
\begin{align}\nonumber\small
\mathcal{L}_{\operatorname{tNCE}}&=
-\mathbb{E}_{\substack{\scriptstyle o^+\sim\textcolor{black}{\mathcal{P}(\mathcal{O}_i)}}}
    \log  
    \frac{
        \exp(\mathfrak{R}(\mathbf{v}^+, \mathbf{l}_i))
    }{
        \mathbb{E}_{\scriptstyle o^- \sim \textcolor{black}{\mathcal{N}(\mathcal{O}_i)}}
        \exp(\mathfrak{R}(\mathbf{v}^-, \mathbf{l}_i))
    },
\end{align}

% \begin{align}\nonumber\small
% \mathcal{L}_{\operatorname{tNCE}}&=
% -\mathbb{E}_{\substack{\scriptstyle o\sim O_i \\ \scriptstyle o^+\sim\textcolor{black}{\mathcal{P}(o)}}}
%     \log  
%     \frac{
%         \exp(\mathfrak{R}(\mathbf{v}^+, \mathbf{v}, \mathbf{l}_i))
%     }{
%         \mathbb{E}_{\scriptstyle o^- \sim \textcolor{black}{\mathcal{N}(o)}}
%         \exp(\mathfrak{R}(\mathbf{v}, \mathbf{v}^-, \mathbf{l}_i))
%     },\vspace{-2ex}
% \end{align}
% where $\mathbf{v} = \phi(o)$, and 
where $\mathbf{v}^{+/-} = \phi(o^{+/-})$. Different pretraining strategies differ in their selection of (1) the positive frame set $\mathcal{P}(\mathcal{O}_i)$, (2) negative frame set $\mathcal{N}(\mathcal{O}_i)$; and (3) the semantic alignment scoring function $\mathfrak{R}(\mathbf{v}, \mathbf{l}_i)$ measuring the gap of VL similarities as detailed in Table \ref{tab:comp}. 

\noindent\textbf{Discussion.} As motivated by goal-conditioned RL \cite{nips17-her}, current approaches \textit{explicitly} select future frames (\textit{e.g.}, DecisionNCE) or the last frame (\textit{e.g.}, LIV) as the goal within the positive set, enforcing their visual embedding to align with the semantics. Likewise, the scoring functions $\mathfrak{R}$ are often designed to maximize this transition direction. However, the pretraining action videos are \textit{noisy} as actions may terminate early or include irrelevant subsequent actions, which may mislead the encoders and result in inaccurate vision-language association. As detecting precise action boundaries is non-trivial, we argue for a more flexible approach that leverages \textit{intrinsic} characteristics of actions to guide pretraining.



% we first pre-train a visual encoder \(\mathcal{\phi}: \mathcal{O} \to \mathbb{R}^d\) and a text encoder \(\mathcal{\varphi}: \mathcal{L} \to \mathbb{R}^d\) to learn mappings from the observation and the language instruction space to $d-$dimensional feature spaces. This pre-training can be done using large, less-expensive data without action annotation, such as human action videos . Then, with the frozen learned features \(\boldsymbol{v}\) and \(\boldsymbol{l}\) as input, we can only fine-tune a simple Multi-Layer Perceptron (MLP) with a few demonstrations to learn the map from the feature space \(\mathbb{R}^d \times \mathbb{R}^d\) to the action space \(\mathcal{A}\). Since both the observation space \(\mathcal{O}\) and the action space \(\mathcal{A}\) are continuous and ordered over time, we expect the representations learned through pre-training to also exhibit continuity and orderliness. This property in the representations allows for better learning of the continuous mapping between observations and actions. This property offers three significant benefits: First, the orderliness of the representation ensures that different states of the task, such as the start and end of an action, can be better captured and distinguished. Second, the continuity of the representation allows it to evolve smoothly as the task progresses, enabling the model to output stable actions based on the current state. Finally, we can demonstrate that even under small perturbations to the language instruction, these properties ensure the robustness of the learned representation. This robustness is crucial for maintaining performance in real-world scenarios where language instructions might contain minor ambiguities or variations.





% We consider a partially observable Markov Decision Process (POMDP) with language conditions, which models the interaction between an agent and an environment where observations are incomplete and actions are guided by natural language instructions. Formally, a POMDP is defined as a tuple $\langle \mathcal{S}, \mathcal{A}, \mathcal{O}, \mathcal{T}, \mathcal{R}, \mathcal{Z}, \gamma \rangle$, where $\mathcal{S}$ is the state space, $\mathcal{A}$ is the action space available to the agent. $\mathcal{O}$ is the observation space, which provides partial information about the environment. $\mathcal{T}(s' \mid s, a)$ is the state transition function. $\mathcal{R}(s, a)$ is the reward function. $\mathcal{Z}(o \mid s, a)$ is the observation function. $\gamma \in [0, 1)$ is the discount factor.

% To incorporate language instructions, we introduce a task description $L$, which specifies the agent's goal in natural language. The task description conditions the agent's policy $\pi(a \mid o, L)$, where $o$ is the agent's current observation. The agent aims to maximize the expected cumulative reward while adhering to the task described by $L$.

% Further, we assume the availability of a large-scale human action video dataset including $N$ video-instruction pairs, $\{(\{o_k^i\}_{i=1}^{t_k}, L_k)\}_{k=1}^N$, where each pair representing an action video with $t_k$ frames and its corresponding language description $L_k$. We pre-train the visual and language encoders on this dataset, with the visual features $\boldsymbol{v} = \operatorname{Enc}_v(o)$ and the language features $\boldsymbol{l} = \operatorname{Enc}_l(L)$. These pre-trained representations are then frozen and applied to train the policy $\pi$ in the aforementioned decision-making process, enabling the agent to better interpret and act upon language-conditioned tasks.

\section{\ourmethod Framework}
%\vspace{-5pt}
\section{Proposed Method}
\begin{figure}[t]
\begin{center}
\includegraphics[scale=0.4]{figure/converter_arc.png}
\end{center}
\caption{Converter architecture.}
\label{fig:converter_architecture}
\end{figure}

\subsection{Synvolution}
Let $\mathbf{A} = \mathbf{X}\mathbf{W}_{\text{Q}}(\mathbf{X}\mathbf{W}_{\text{K}})^{\mathrm{T}}/{\tau} \in \mathbb{R}^{n \times n}$ be an affinity matrix, the attention matrix $\mathcal{A} = \mathrm{softmax}(\mathbf{A})$ in Equation~\ref{eq:shsa} can be reformulated as a right stochastic normalized affinity form $\mathrm{SA}(\mathbf{X}) = \widetilde{\mathbf{D}}^{-1}\widetilde{\mathbf{A}}\mathbf{X}\mathbf{W}_{\text{V}}$, where $\widetilde{\mathbf{A}} = \exp(\mathbf{A}) \in \mathbb{R}^{n \times n}$ is defined as the affinity matrix after the element-wise exponentiation operation $\exp(\cdot)$, and $\widetilde{\mathbf{D}}_{u,u} = \sum_{v}\widetilde{\mathbf{A}}_{u,v} \in \mathbb{R}^{n \times n}$ is the corresponding degree matrix. When treating $\widetilde{\mathbf{A}}$ as a digraph adjacency matrix, we found that self-attention closely resembles digraph convolution. First, each element in either an attention matrix or a DGSO can be considered as a similarity from source entity to target entity. Second, both self-attention and digraph convolution can be degenerated to graph convolution form. For self-attention, this occurs when the query matrix is equal to the key matrix in each head, resulting in unidirectional symmetric self-attention. Similarly, for digraph convolution, the achievement of graph convolution can be implemented by symmetrizing the adjacency matrix of a digraph. Third, the softmax function in self-attention results in a row-wise normalized digraph adjacency form. 

Since digraph convolution closely resembles self-attention, we investigated replacing self-attention with digraph convolution. Under this hypothesis, a Transformer can be converted into a DGNN form. Based on this insight, we propose Converter. In this work, we decide to construct the DGSO directly. We develop a learnable unitary matrix as a DGSO through the inverse process of eigendecomposition. Our method consists of two phases. In the first phase, we synthesize the required eigenvalues through the following process.
\begin{equation}\label{eq:eigenvalue_gap}
\centering
\eu^{\ramuno\mathbf{\Lambda}}
= \exp\left[\ramuno\cdot\mathrm{diag}\left(\mathrm{pool}_{\text{avg}}\left[\mathrm{SIREN}(\mathbf{X})\right]\right)\right].
\end{equation}
Here, $\mathrm{SIREN}$ represents a 2-layer MLP with the sine function~\citep{NEURIPS2020_53c04118}, $\mathrm{pool}_{\text{avg}}(\cdot)$ is a 1D global average pooling, and $\mathrm{diag}(\cdot)$ is a diagonalize operation. We adopt the sine function because it demonstrates a remarkable ability in signal processing~\citep{NEURIPS2020_53c04118}.

In the second phase, we focus on constructing the necessary unitary eigenvector matrix through the inverse process of LQ factorization. Based on the Givens rotation method~\citep{doi:10.1137/0106004}, an arbitrary square matrix $\bm{\Phi} \in \mathbb{C}^{N \times N}$ can be decomposed into a product of a lower triangular matrix and Givens rotation matrices. Hence, we have
\begin{equation}\label{eq:givens_lq}
\centering
\bm{\Phi} 
= \mathbf{L}\mathbf{Q} 
= \mathbf{L}\left(\prod_{j=N}^{2}\prod_{i=j-1}^{1}\mathbf{G}_{i,j}\right),
\end{equation}
where $\mathbf{L} \in \mathbb{C}^{N \times N}$ is a lower triangular matrix, and $\mathbf{G}_{i,j} \in \mathbb{C}^{N \times N}$ is a Givens rotation matrix that resembles an identity matrix with the exception of the elements
\begin{equation}\label{eq:givens_rotation_matrix}
\centering
\begin{bmatrix}
G_{ii} & G_{ij} \\
G_{ji} & G_{jj}
\end{bmatrix} =
\begin{bmatrix}
\overline{c} & -s \\
\overline{s} & c
\end{bmatrix} =
\begin{bmatrix}
\eu^{-\ramuno(\frac{\alpha+\beta}{2})}\cos{(\frac{\gamma}{2})} 
& -\eu^{\ramuno(\frac{\alpha-\beta}{2})}\sin{(\frac{\gamma}{2})} \\
\eu^{-\ramuno(\frac{\alpha-\beta}{2})}\sin{(\frac{\gamma}{2})} 
& \eu^{\ramuno(\frac{\alpha+\beta}{2})}\cos{(\frac{\gamma}{2})}
\end{bmatrix},
\end{equation}
which characterized by parameters $\alpha$, $\beta$, and $\gamma \in [0, 2\pi]$. This methodology necessitates ${\left(N(N-1)\right)}/{2}$ pairs of Givens rotation matrices, i.e., it requires $\mathcal{O}(N^{2})$ space complexity. By reorganizing Givens rotation matrices, inserting permutation matrices, and repeating the patten, we have
\begin{equation}\label{eq:givens_lhhp}
\centering
\begin{split}
\bm{\Phi} 
&= \mathbf{L}\left(\prod_{l=1}^{L}\left(\prod_{i=N-1}^{1}\mathbf{G}^{(l)}_{i,i+1}\right)\left(\prod_{j=1}^{N-1}\mathbf{G}^{(l)}_{j,j+1}\right)\mathbf{P}^{(l)}\right)\\
&= \mathbf{L}\left(\prod_{l=1}^{L}\mathbf{H}^{(l)}_{\text{l}}\mathbf{H}^{(l)}_{\text{u}}\mathbf{P}^{(l)}\right).
\end{split}
\end{equation}
Here, $\mathbf{H}^{(l)}_{\text{l}} \in \mathbb{C}^{N \times N}$ is a lower unitary Hessenberg matrix, $\mathbf{H}^{(l)}_{\text{u}} \in \mathbb{C}^{N \times N}$ is an upper unitary Hessenberg matrix, and $\mathbf{P}^{(l)} \in \mathbb{R}^{N \times N}$ is a permutation matrix that either learnable~\citep{mena2018learning}, fixed~\citep{pmlr-v162-dao22a}, or even an identity matrix $\mathbf{I}_{N} \in \mathbb{R}^{N \times N}$.

We refer to Equation~\ref{eq:givens_lhhp} as the order-$L$ LHHP parametrization, $L$-LHHP for short, denoted by $\bm{\Phi}_{L-\text{LHHP}}$. In particular, when the lower triangular matrix $\mathbf{L}$ degenerates to a diagonal matrix $\mathbf{D}$, we term this pattern the order-$L$ DHHP parametrization, $L$-DHHP for short, denoted by $\bm{\Phi}_{L-\text{DHHP}}$. It requires $2L(N-1)$ pairs of Givens rotation matrices, which means the space complexity is $\mathcal{O}(LN)$. We observed that each unitary factor matrix resulting from the multiplication of lower and upper unitary Hessenberg matrices in the order-$L$ DHHP parametrization is dense rather than sparse, unlike the schemes proposed in \citep{KHALITOV2022160}. Since our method is based on the Givens rotation method, we make Assumption~\ref{asmp:dhhp_givens_num_upper_bound}. Under this assumption, we can establish the following propositions.

\begin{assumption}\label{asmp:dhhp_givens_num_upper_bound}
For constructing an arbitrary $N \times N$ dense unitary matrix, at most $\lceil\frac{N}{4}\rceil$ orders are sufficient for $L$-DHHP.
\end{assumption}

\begin{proposition}\label{prop:dhhp_universal}
$L$-DHHP captures the discrete unitary transforms, including discrete Fourier transform (DFT), the discrete Walsh–Hadamard transform (DWHT), the discrete cosine transform (DCT), the discrete sine transform (DST), and their inverses exactly.
\end{proposition}

\begin{proposition}\label{prop:dhhp_time_complexity}
Given an input signal $\mathbf{x} \in \mathbb{C}^{N}$ and an output signal $\mathbf{y} \in \mathbb{C}^{N}$, the time complexity of $L$-DHHP as a discrete unitary transform with the fast implementation as $\mathbf{y} = \bm{\Phi}\mathbf{x}$ is $\mathcal{O}(LN\log{N})$.
\end{proposition}

\begin{proposition}\label{prop:dhhp_full_rank}
$L$-DHHP is full-rank if and only if the diagonal matrix $\mathbf{D}$ is unitary.
\end{proposition}

We refer to this self-attention alternative as Synvolution:
\begin{equation}\label{eq:synvolution}
\centering
\mathrm{Synv}(\mathbf{X}\mathbf{W}_{\text{V}})
= \bm{\Phi}^{-1}\left[\exp(\ramuno\mathbf{\Lambda})\odot(\bm{\Phi}\mathbf{X}\mathbf{W}_{\text{V}})\right],
\end{equation}
where $\mathbf{X}\mathbf{W}_{\text{V}} \in \mathbb{C}^{N \times D}$ is denoted as the value matrix. Unlike FFT-based convolution~\citep{mathieu2013fast}, where the discrete unitary matrix is fixed and data-independent, the required parameters in $L$-DHHP are learnable and data-dependent. We adopt a similar processing method to that described in Equation~\ref{eq:eigenvalue_gap} to obtain the synthetic eigenvector matrix. For convenience, we set $L=1$, $\mathbf{P}^{(1)} = \mathbf{I}_{N}$, and $\mathbf{D}$ is unitary to obtain a dense unitary matrix that serves as the desired unitary eigenvector matrix. More details about the fast implementation of $1$-DHHP as a discrete unitary transform are provided in the appendix.

% We recognize that the core essence of self-attention can be characterized as a fast weight programmer~\citep{pmlr-v139-schlag21a}. In brief, it involves a non-linear projection of the input feature into a weighted tensor, known as a fast weight, which guides either the input feature itself or its linear projection. This operation is not unique to self-attention, but is also applicable to non-local operations~\citep{Wang_2018_CVPR}. For instance, channel attention~\citep{Hu_2018_CVPR}, a well-known instance of non-local operation, commonly employs the squeeze-and-excitation technique. The first squeeze operation aggregates the frequency information from the spatial extent, whereas the second excitation operation redistributes the gathered feature responses back to the local features. Another example is Involution~\citep{Li_2021_CVPR}.

\subsection{Kernelution}

\begin{figure*}[t]
\begin{center}
\includegraphics[scale=0.4]{figure/kernelution.png}
\end{center}
\caption{Illustration of the entire Kernelution process.}
\label{fig:kernelution}
\end{figure*}

\subsubsection{Chebyshev Polynomial Interpolation}
The multi-head operation, a common approach to enhance performance in Transformers, lacks solid theoretical support. In the contrast, FIR filters have a theoretical support in spectral graph theory~\citep{chung1997spectral}. Let $f(x)$ be the target function, then our goal is to approximate it with the smallest round-off error. Directly manipulating orthogonal polynomials to filter complex-valued signals is challenging, but using them to represent the argument function of signals is straightforward. To achieve it, we can choose an arbitrary orthogonal polynomial basis such as the Bernstein basis, Jacobi basis (including Chebyshev, Gegenbauer, Legendre, and Zernike bases), or even monomial basis. Consider the Chebyshev basis as an example. Given an arbitrary continuous function $f(x) \in C([-1,1])$ and a truncated Chebyshev polynomial $p$ with $K$ orders, then the target function $f(x)$ can be approximated as
\begin{equation}\label{eq:cpi}
\centering
f(x) \approx p(x) = \frac{1}{2}{\mu}_{0} + \sum_{k=1}^{K}{\mu}_{k}{T}_{k}(x),
\end{equation}
where $\mu_{k} \approx \frac{2}{K+1}\sum_{j=0}^{K}f(x_{j})T_{k}(x_{j})$ is the Chebyshev coefficient, and $x_{j}$ is the sampling Chebyshev node. This technique is termed the Chebyshev polynomial interpolation (CPI)~\citep{10.1137/1.9781611975949}. The operation on Chebyshev polynomial interpolation is considerably straightforward since Chebyshev polynomials are isomorphic with Fourier series. For differentiable or analytic functions, we have the following theorems.

\begin{theorem}[CPI for differentiable functions~\citep{10.1137/1.9781611975949}]\label{theorem:cpi_diff}
Let $\upsilon \geq 0$ and $\kappa > \upsilon$ be integers. Consider a function $f(x)$ whose derivatives up to order $\upsilon-1$ are absolutely continuous on $[-1, 1]$, and suppose $\lVert{\frac{\diff^{\upsilon}}{\diff{x}^{\upsilon}} f(x)}\rVert_{1} = \Upsilon$. For the $\kappa$-th degree Chebyshev interpolant $p(x)$, the following bounds hold: (1)~$\lVert{\mu_{\kappa}}\rVert \leq \frac{2\Upsilon}{\pi(\kappa-\upsilon)^{\upsilon+1}}$. (2)~$\lVert{f(x) - p(x)}\rVert \leq \frac{4\Upsilon}{\pi\upsilon(\kappa-\nu)^{\upsilon}}$.
\end{theorem}

\begin{theorem}[CPI for analytic functions~\citep{10.1137/1.9781611975949}]\label{theorem:cpi_ana}
Let $\kappa \geq 1$ be an integer and $f(x)$ an analytic function on $[-1, 1]$ that extends analytically to the open Bernstein ellipse $E_{\rho}$ with $\lVert{f(x)}\rVert \leq M$ for some $M$. For the $\kappa$-th degree Chebyshev interpolant $p(x)$, the following bounds hold: (1)~$\lVert{\mu_{0}}\rVert \leq M$. (2)~$\lVert{\mu_{\kappa}}\rVert \leq 2M\rho^{-\kappa}$. (3)~$\lVert{f(x)-p(x)}\rVert \leq \frac{4M\rho^{-\kappa}}{\rho-1}$.
\end{theorem}

Both Theorem~\ref{theorem:cpi_diff} and Theorem~\ref{theorem:cpi_ana} tell us that we can utilize the Chebyshev polynomial filter to approximate any continuous target function that lies in the range of $C[-1, 1]$ with a small round-off error.

\subsubsection{Kernel Polynomial Method}
In reality, the target function is probably discontinuous or singular in the polynomial interpolation interval. In this situation, the accuracy of the Chebyshev polynomial interpolation reduces to $\mathcal{O}(1)$ near discontinuities or singularities. Sufficiently far away from discontinuities or singularities, the convergence will be slowed to $\mathcal{O}(K^{-1})$. During the approximation process, oscillations will be present near discontinuities or singularities and they will not diminish as $K \to \infty$. This type of oscillation is termed the Gibbs oscillation, and this situation is known as the Gibbs phenomenon~\citep{Hewitt1979}.

To mitigate Gibbs oscillations, we apply a Gibbs damping factor $g_{k}$, which represented as a function of $\frac{k}{K+1}$, to each term of the Chebyshev polynomials. For any $f(x)$, we have
\begin{equation}\label{eq:kpm}
\centering
{f(x)}\approx{p}_{\text{KP}}(x) = \frac{1}{2}g_{0}\mu_{0} + \sum_{k=1}^{K}g_{k}\mu_{k}T_{k}(x).
\end{equation}
This modification of the Chebyshev coefficients is equivalent to the convolution of $p(x)$ with a kernel $\mathcal{K}(x,x_{0}) = \frac{2}{\pi\sqrt{1-x^{2}}}\left(\frac{1}{2}g_{0} + \sum_{k=1}^{K}g_{k}T_{k}(x)T_{k}(x_{0})\right)$ that $p_{\text{KP}}(x) = \int_{-1}^{1}\mathcal{K}(x,x_{0})f(x_{0})\diff{x_{0}}$. Thus, this method is also called the kernel polynomial method. It is widely employed in computational physics for calculating the density of states and other spectral properties of large quantum systems. 

Gibbs damping factors are a family of coefficients that satisfy three conditions: (1)~$g_{k} > 0$. (2)~$g_{0} = 1$. (3)~$\lim_{K \to \infty} {g_{1} \to 1}$. The conditions (1) and (2) are particularly valuable in real-world applications~\citep{RevModPhys.78.275,Weiße2008}. The first condition ensures that approximations of positive quantities remain positive, while the second conserves the integral of the expanded function $\int_{-1}^{1}p_{\text{KPM}}(x)\diff{x} = \int_{-1}^{1}f(x)\diff{x}$. Notably, $g_{k} = 1$ is the simplest Gibbs damping factor attributed to the Dirichlet kernel. More details about Gibbs damping factors are in the appendix.

Clearly, finding an appropriate kernel is crucial for approximation, as it determines whether the round-off error is minimized or not. As indicated in \citep{RevModPhys.78.275,Weiße2008}, kernel choices are data-dependent. More specifically, given a target function, we need to match an appropriate kernel and manually tune its hyperparameters (if the kernel has any) based on experience. Since the target function is unknown, we relax each $\mu_{k}$ with a learnable parameter $w_{k}$. The effectiveness of the Gibbs damping factors lie in their ability to reduce the weight of each term of the Chebyshev coefficients, thereby mitigating the contributions of higher-order terms. Based on this observation, and in order to prevent over-fitting, we propose the following loss function which is named the kernel polynomial loss (KPL):
\begin{equation}\label{eq:kpl}
\centering
\mathcal{L}_{\text{KP}} = \int_{-1}^{1}\abs{\frac{\diff{f(x)}}{\diff{x}}}^{2}\diff{x} \approx \sum_{k=1}^{K}{\pi}{k}^{2}\abs{w_{k}}^{2}.
\end{equation}
This results in an intuitive penalty applied to the Chebyshev coefficients, with higher order Chebyshev coefficients incurring greater penalties than the lower ones. It causes the Chebyshev polynomial interpolation with the kernel polynomial loss to simulate the kernel polynomial method with a learnable kernel. We apply the kernel polynomial method with Synolution, which turns out what we call Kernelution. The corresponding formula is defined as
\begin{equation}\label{eq:kernelution}
\centering
\mathrm{Kern}(\mathbf{X}\mathbf{W}_{\text{V}})
= \bm{\Phi}^{-1}\left[\exp\left(\ramuno\cdot{p}_{\text{KP}}(\mathbf{\Lambda})\right)\odot(\bm{\Phi}\mathbf{X}\mathbf{W}_{\text{V}})\right].
\end{equation}
It is worth noting that the kernel polynomial method is not the only operation compatible with Synvolution. Depending on practical requirements, Synvolution can also be made compatible with the multi-head operation, similar to other attention mechanisms. This means Synvolution can be equipped as a substitute for self-attention in Transformer-based models.

\subsection{Gated Feed-Forward Network and PostScaleNorm}
Both Synvolution and Kernelution effectively represent the direction and model the relationship between feature tokens in the spectral domain. A tricky problem is that the output of either Synvolution or Kernelution is complex-valued, whereas the labels are real-valued. This conflict motivates us to design a layer that maps a complex-valued tensor into a real-valued tensor. We propose a Gated Feed-Forward Network (GFNN) to solve this issue.
\begin{equation}\label{eq:gffn}
\centering
\mathrm{GFFN}(\mathbf{X}) = \left[\mathrm{softplus}(\Re(\mathbf{X})\mathbf{W}_{\Re})\odot\tanh(\Im(\mathbf{X})\mathbf{W}_{\Im})\right]\mathbf{W}_{\text{O}},
\end{equation}
where $\mathbf{W}_{\Re} \in \mathbb{R}^{D \times D_\text{hid}}$, $\mathbf{W}_{\Im} \in \mathbb{R}^{D \times D_\text{hid}}$ and $\mathbf{W}_{\text{O}} \in \mathbb{R}^{D_\text{hid} \times D}$ are trainable weight matrices. We let the real part to learn the magnitude, and the imaginary part to learn the sign. Besides, we apply the PostNorm architecture~\citep{wang-etal-2019-learning-deep} with ScaleNorm~\citep{nguyen-salazar-2019-transformers} across the whole model, namely PostScaleNorm. Specifically, we apply $\mathrm{ScaleNorm}(\mathbf{Z} + \zeta \cdot \Re(\mathbf{Z}) + (1 - \zeta) \cdot \Im(\mathbf{Z}))$ for a complex-valued signal $\mathbf{Z}$, where $\zeta \in [0, 1]$ is a learnable parameter.

\section{Experiments}
%\vspace{-5pt}
\section{Experiments}\label{sec_exp}
%\hp{Accelerating IM simulation~\cite{tang2015influence}}

% \begin{itemize}
%     \item 6.1. Problem setting of three COPs, including the general model and three specific CO problems 
%     \item 6.2. Experiment Setting (hyperparameters, details of training, evaluation, and test) 写在appendix里吧
%     \item 6.3. Performance analysis 这个要占半页
% \end{itemize}

%\hp{need to think of a way to compress these tables / visuals.} 

%\hp{\cancel{Baselines}; hyperparamters; \cancel{metrics}; etc.}

With theoretical guarantees on the existence and convergence of NE for ACCES games, we are also interested in how our proposed algorithm CCDO-RL works empirically. To evaluate this, we conduct experiments of CCDO-RL on three distinct ACCES game instances introduced in Section \ref{sub_exp_ins} and analyze the performance of CCDO-RL in Section \ref{sub_train_eval}. Section 6.2.1 aims to empirically demonstrate the convergence (Figures \ref{fig_exploit_20} and \ref{fig_exploit_50}) of the algorithm CCDO-RL over realistic CO problems, and show its consistency with Theorem \ref{CCDOA}. Section 6.2.2 intends to show the average reward (to seen training graphs) as well as the generalizability (to unseen test graphs) of the combinatorial player in real-world ACCES games (shown in Tables \ref{tab_aver}, and \ref{tab_gene}).

\subsection{Three Instances of ACCES Games} \label{sub_exp_ins}
% \hp{This para does not make much sense. Need to follow the framework in the Preliminaries section.}
% For combinatorial optimization problems in real-world applications, situations are more complicated and intractable due to changeable environmental or physical parameters. The form of parameter sets is very crucial because different types have different solvability and computation complexity. Forms of parameter sets mainly contain discrete sets, interval sets \cite{buchheim2018robust} like polyhedral and ellipsoid, probability distributions \cite{carlsson2018wasserstein}, and variable functions \cite{krause2008robust}.

% In reality, these parameters are often impacted by some common factors, such as conditions of weather, transportation, and individual personalities. \cite{kalimeris2019robust} proposed an assumption that real instances (e.g. demands in CVRP, coverages in CSP) 
%Considering affected or attacked COPs, the real instance $\{\theta_{i}\}$ always relied on the estimated value $\{\hat{\theta}_{i}$\} and the variation determined by independent factors $\{g_{i}\}$ and environment/physical parameters/attacker actions $\{\eta\}$. The concrete parameter influence model is stated as follows:

We consider a certain COP which is parameterized with $\{\theta_{i}\}$, where $i$ is the index of nodes (such as a target in security games) -- e.g., such parameters can be interpreted as attack probability of targets.
%coverage radius, customer's demands, or attack probability of targets. 
In real-world applications, we often need to estimate such parameters before solving the COPs. Unfortunately, the estimation $\{\hat{\theta}_{i}\}$ often bears a gap to the true value $\{\theta_{i}\}$, which derives from e.g. environment (aleatoric) uncertainty, model (epistemic) uncertainty, or an attacker trying to manipulate the defender's utility. We use a generic model to formulate this gap:
\begin{equation}\label{linrob}
    \theta_{i} = \hat{\theta}_{i} + y \cdot \tau_{i},
\end{equation}
where $y$ represents the strategy of the nature/attacker, $\tau_{i}$ is the environment factors like weather and transportation conditions, or human subjective factors like the preference of the attacker. 
Such abstraction can represent a wide range of ACCES games, such as facility location covering problems \cite{an2020battery, TIRKOLAEE2020340}, CVRP \cite{vehiclerouting.ch8,dinh2018exact, FLORIO20231081}, security patrolling (OP) \citep{xu2021robust}, and influence maximization problem \cite{kalimeris2019robust}. We describe three instances of ACCES games based on the model (\ref{linrob}).%Based on this model (\ref{linrob}), we focus on three combinatorial optimization problems with attacks or environmental/physical influence.

% \hp{Hard to follow. We should point out what are the two players, what are X, Y, u etc}

\textbf{Adversarial Covering Salesman Problem (ACSP):} In a map of cities, every city $i$ has a coverage $\theta_{i}$. A salesman finds the shortest path such that all cities are visited or covered, with $\theta_{i}$ influenced by physical factors $\tau_i$ and transportation parameters $y$ based on Eq.(\ref{linrob}). The salesman is Player 1 where $X$ consists of the feasible paths of the salesman. Nature is Player 2 with $Y$ = $[0, 1]^K \ni y, K \in \mathbb{N}$. The utility function of Player 1 $u$ is the opposite of the total traveling distance.

\textbf{Adversarial Capacitated Vehicle Routing Problem (ACVRP):} A vehicle with a constrained capacity of goods finds the shortest path under the worst case with the $i_{th}$ customer's demand $\theta_i$ changed by environmental factors $\tau_i$ and weather parameter $y$ on Eq.(\ref{linrob}). The vehicle is Player 1 where $X$ is the set of the feasible path $x$. Nature is Player 2 where $Y$ is $[0, 1]^K \ni y, K \in \mathbb{N}$. The utility function of Player 1  $u$ is the opposite of total delivery distance satisfying all the demands of customers.


\textbf{Patrolling Game (PG):} The patrolling game is described in the introduction.

For all the problem instances, we run our algorithm on two problem sizes: 20 nodes and 50 nodes. The detailed description and problem parameters of the three game instances are in Appendix \ref{app_ex_para_set}.

% Similarly, in the vehicle route problem (VRP), conditions with correlated parameters arouse broad attention from scholars \cite{vehiclerouting.ch8,dinh2018exact,FLORIO20231081}. \cite{dinh2018exact} considered the demand correlation by geographical proximity of nodes, described by some independent random variables in the fractional form. \cite{FLORIO20231081} utilized 'external factors' to stand for unknown covariates affecting all demands and presented a Bayesian model to learn correlations. Further more, about IM problems, \cite{kalimeris2019robust} combined node features and uncertain hyperparameters to fit the influence probability on each edge.

% \subsection{Training CCDO-RL}

% For all the problems, CCDO-RL adopts the REINFORCE algorithm with an attention-based encoder-decoder framework \cite{kool2018attention} (used as an inductive graph representation component) to learn a (generalizable) COP solver for one player (protagonist), and PPO \cite{schulman2017proximal} to train a policy for the other player (adversary) whose strategy space is continuous. CCDO-RL is trained with 50 epochs on a set of 10,000 graphs (with 20 or 50 nodes). The hyperparameters of CCDO-RL are specified in Appendix \ref{app_ex_para_set} (Table \ref{tab_hyper_ccdorl}). Our code is included as supplementary material for ease of reproduction. 
% % \hp{need to specify hyperparas}

\subsection{Performance of CCDO-RL}\label{sub_train_eval}

Two aspects are evaluated for the performance of CCDO-RL, i.e., i) Convergence to NE (Section \ref{sub_per_conver}) exploring whether CCDO-RL can compute the NE, and ii) Protagonist policy's average reward and generalizability (Section \ref{sub_per_rob}). Generalizability refers to the ability of RL models trained on previously seen graphs (problem instances), to perform well on a new set of unseen test graphs. The model’s usability is enhanced by generalizability, rather than focusing solely on the average reward, which is a critical motivation in the literature on RL for COPs \citep{khalil2017learning, kool2018attention}.

For all the problems, CCDO-RL adopts the REINFORCE algorithm with an attention-based encoder-decoder framework \citep{kool2018attention} (used as an inductive graph representation component) to learn a generalizable COP solver for Player 1 (protagonist), and PPO to train a policy for Player 2 (adversary) whose strategy space is continuous. CCDO-RL is trained on a set of 10,000 graphs (with 20 or 50 nodes). The hyperparameters of CCDO-RL are specified in Appendix \ref{app_ex_para_set} (Table \ref{tab_hyper_ccdorl}). Our code is included as supplementary material and will be open-sourced for ease of reproduction. 

% \textbf{Training.} For all the problems, CCDO-RL adopts the REINFORCE algorithm with attention-based encoder-decoder framework \cite{kool2018attention} (used as an inductive graph representation component) to learn a (generalizable) COP solver for one player (protagonist), and PPO \cite{schulman2017proximal} to train a policy for the other player (adversary) whose strategy space is continuous. CCDO-RL is trained with 50 epochs on a set of 10,000 graphs (with 20 or 50 nodes). 

% \hp{We should first present results about convergence as it is mostly aligned with the theory.}

\subsubsection{Convergence to NE} \label{sub_per_conver}

Exploitability is a common metric to describe the closeness to true NE by calculating the sum of performance distances between each new best response and subgame NE, i.e. $\sum_{i=1,2} U(\pi_{i,k}^{br}, \sigma_{-i,k}) - U(\sigma)$ in the general two-player game. Since our game is zero-sum, the calculation is as follows:
\begin{equation*}
   \text{Exploitability}(\sigma) = \max_{\pi_1 \in \Sigma_1} U(\pi_1, \sigma_{2}) - \min_{\pi_2 \in \Sigma_2} U(\sigma_1, \pi_2).
\end{equation*}
From Figure \ref{fig_exploit_20}, we can see that CCDO-RL can converge to approximate NE in 25 iterations or less (in the PG setting), reaching 0.05 in ACSP, 0.10 in ACVRP, and 0.03 in PG with 20 nodes. Similar results are observed in problems with 50 nodes (see Figure \ref{fig_exploit_50} in Appendix \ref{app_exp}). These results validate the effectiveness of CCDO-RL in finding the NE for various types of games.

%Similarly, the exploitability of three COPs in 50 nodes is provided in the appendix \ref{app_exp}.
\vspace{-\baselineskip}
\begin{figure}[htbp]
	\centering
    \subfigure[ACSP20]{
    \label{csp20_nashconv}
    \includegraphics[scale=0.20]{Figures/nashconv_log_csp20_sm_7.eps}
    }
    \subfigure[ACVRP20]{
    \label{cvrp20_nashconv}%文中引用该图片代号
    \includegraphics[scale=0.20]{Figures/nashconv_log_svrp20_sm_7.eps}
    }
    \subfigure[PG20]{
    \label{opsa20_nashconv}
    \includegraphics[scale=0.20]{Figures/nashconv_log_pg20_sm_7.eps}
    }
    \caption{Exploitability curve of CCDO-RL on three games of 20 nodes}
    \label{fig_exploit_20}
\end{figure}
\vspace{-\baselineskip}
\subsubsection{Average reward and Generalizability of Combinatorial player} \label{sub_per_rob}
% \subsubsection{Robustness and Generalizability of Protagonist Policy} \label{sub_per_rob}
%\hp{CCDO-RL being better in these following metrics is only kind of a by-product.}

% \textbf{Evaluation.} The learned policies are then tested on 200 graphs, where 100 of them are randomly selected from the 10,000 training graphs, and the other 100 are unseen graphs. 
% We use two metrics to evaluate the performance of different policies for the protagonist player: \textbf{Average proportional loss} $R-$ describes the policy overfitting degree \citep{lanctot2017unified}; \textbf{Reward} evaluates the performance of the protagonist with the adversary under three COPs.  
% \begin{eqnarray}
%         &R- = (\hat{D} - \hat{O}) / \hat{D}.
% \end{eqnarray}
% in which $\hat{D}$ is the mean value of the diagonals and $\hat{O}$ is the mean value of the off-diagonals in the payoff matrix provided in the Appendix \ref{app_exp}.

% Because the protagonist policy is trained against a powerful adversary under our ACCES game setting, the obtained policy is naturally robust against adversarial perturbations. This subsection sheds a bit of light on this perspective and quantifies the extent of robustness of CCDO-RL as well as the ability of RL to generalize to unseen test graphs.

\textbf{Evaluation.} The learned policies are tested on 200 graphs, with 100 being randomly selected from the 10,000 training graphs (to show the average reward), and the other 100 being unseen graphs (to test policy generalization). We evaluate the performance of the protagonist with the adversary under three COPs. For each COP, the performance is considered both on the 20-node and 50-node map.
% We use two metrics to evaluate the performance of different policies for the protagonist player: \textbf{Average proportional loss} $R-$ describes the policy overfitting degree \citep{lanctot2017unified}; \textbf{Reward} evaluates the performance of the protagonist with the adversary under three COPs.

\textbf{Baselines.} There are heuristic algorithms for each game instance (Heuristic in Table \ref{tab_aver} and \ref{tab_gene}) and a single-player RL algorithm. For ACVRP, we adopt the Tabu Search algorithm (Tabu) \citep{li2020improved} as the heuristic algorithm, which is widely applied in the routing problem. For ACSP, the common benchmark local search algorithm, LS2 \citep{golden2012generalized}, is used. For PG, we choose the greedy algorithm as the baseline. The "RL against Stoc" algorithm in Tables \ref{tab_aver} and \ref{tab_gene} is identical to the protagonist model in CCDO-RL but trained in environments with stochastic adversarial perturbations.

% \textbf{Baselines.} There are a heuristic algorithms for each game instance {\color{red} (Heuristic mentioned in the Table \ref{tab_aver} and \ref{tab_gene})} and a single-player RL algorithm. For ACVRP, we adopt the Clarke-Wright (CW) algorithm \citep{pichpibul2013heuristic} and the Tabu Search algorithm (Tabu) \citep{li2020improved} as heuristics, which are applied widely in the routing problem. For ACSP, two common benchmark local search algorithms, LS1 and LS2 \citep{golden2012generalized}, are used. For PG, we choose a local search algorithm \citep{vansteenwegen2009iterated} and the greedy algorithm as the heuristic baselines. {\color{red} The "RL  against Stoc" algorithm referred to Tables \ref{tab_aver} and \ref{tab_gene}} is identical to the protagonist model in CCDO-RL {\color{red} but trained on environments with stochastic adversarial perturbations.} 

\textbf{Average Reward.}  As illustrated in Table \ref{tab_aver}, our algorithm achieves a better average reward than baselines (10.08\% improvement on average of all settings against two baselines), regardless of CO instance or problem size, when confronting the adversary trained by CCDO-RL. In the setting of CSP-20 nodes, the average reward is improved by 46.98\% compared to the heuristic and by 7.14\% compared with the RL against Stoc. For the 50-node setting, the improvements are 45.91\% and 5.28\% respectively. Similarly, the improvements in contrast to Heuristic and RL against Stoc are as follows: 1.72\% and 3.01\%  for CVRP-20 nodes, 0.75\% and 4.46\% for CVRP-50 nodes, 4.17\% and 1.48\% for PG-20 nodes, and 10.60\% and 4.38\% for PG-50 nodes.

\textbf{Generalizability.} From Table \ref{tab_gene}, CCDO-RL continues to achieve a better average reward when facing the adversary, demonstrating that the learned RL policies generalize well to unseen graphs. Even though the non-RL baselines do have access to the graph structures and other problem information of the unseen problem instances, CCDO-RL can obtain comparable performances without re-training on the new problem instances. The improvements versus Heuristic and RL against Stoc are 46.61\% and 7.02\% for CSP-20 nodes, 42.24\% and 3.94\% for CSP-50 nodes, 1.12\% and 1.56\% for CVRP-20 nodes, 0.90\% and 5.05\% for CVRP-50 nodes, 5.35\% and 2.40\% for PG-20 nodes, and 12.17\% and 10.33\% for PG-50 nodes. Even when confronting the stochastic adversary, CCDO shows superior generalizability compared to two baselines across three COPs, with average improvements of 6.31\%, 3.42\%, and 3.95\% respectively. Detailed results are provided in Appendix \ref{app_exp} (Tables \ref{tab_csp_full_20} - \ref{tab_op_full_50}). 
% The model’s usability is enhanced by the ability to generalize rather than focusing solely on the average reward, which is a critical motivation of the RL for combinatorial optimization literature \citep{khalil2017learning, kool2018attention}.  

\begin{remark}
    In CO problems (or more broadly, operations research and economics), it is known that achieving solution quality improvements against strong baselines (e.g., the RL methods trained with a stochastic adversary) is very challenging, and the margins are usually small \citep{kool2018attention}, sometimes even less than 1\%. However, these “tiny” marginal improvements in profits keep small business owners in the real world alive. Last, the improvement depends a lot on the problem settings, and we show that sometimes the improvement can be much more significant.
\end{remark}
\vspace{-\baselineskip}
% \textbf{Performance analysis.} The robustness results of CCDO-RL for ACSP are shown in Table \ref{tab_csp}. We have the following observations: 1) On both of the 100 seen/unseen graphs, single-player RL performs better than heuristic algorithms no matter whether attacked or not. (2) When confronting the adversary trained by CCDO-RL, CCDO-RL exceeds RL by 0.25 and 0.24 on the training set, and by 0.25 and 0.18 on the test set, respectively under the 20-node and 50-node graphs. This demonstrates the robustness of CCDO-RL. 3) Compared to the performance of the training set with that of the test set, we can see that RL and CCDO-RL both maintain a certain degree of generalization. Similar results for ACVRP (Table \ref{tab_cvrp}) and SPG (Table \ref{tab_op}) are provided in Appendix \ref{app_exp}. 

\begin{table}[ht]
  \caption{Average reward against CCDO-RL's adversary (on seen graphs)}
  \vspace{\baselineskip}
  \label{tab_aver}
  \centering
  \small
  \begin{tabular}{lllllll}
    \toprule
    \multirow{2}{*}{method} & \multicolumn{2}{c}{ACSP (Mean$\pm$Std)} & \multicolumn{2}{c}{ACVRP (Mean$\pm$Std)} & \multicolumn{2}{c}{PG (Mean$\pm$Std)} \\
    \cmidrule(r){2-3} \cmidrule{4-5} \cmidrule(r){6-7}
                            & 20 nodes & 50 nodes & 20 nodes & 50 nodes & 20 nodes & 50 nodes\\
    \midrule
    Heuristic & 6.13$\pm$1.20 & 7.55$\pm$1.42 & 7.65$\pm$1.23  & 13.38$\pm$1.70 & 2.64$\pm$1.03 & 4.53$\pm$1.84   \\
    RL against Stoc    & 3.50$\pm$0.47  & 4.55$\pm$0.62  & 7.55$\pm$1.16  & 13.90$\pm$1.63 & 2.71$\pm$0.90 & 4.80$\pm$2.18   \\
    CCDO-RL   & $\pmb{3.25}$$\pm$0.42 & $\pmb{4.31}$$\pm$0.51  & $\pmb{7.42}$$\pm$1.21  & $\pmb{13.28}$$\pm$1.52 &  $\pmb{2.75}$$\pm$0.87 & $\pmb{5.01}$$\pm$1.91  \\
    \bottomrule
  \end{tabular}
\end{table}
\vspace{-\baselineskip}

\begin{table}[htp]
  \caption{Generalizability against CCDO-RL's adversary (on unseen graphs)}
  \vspace{\baselineskip}
  \label{tab_gene}
  \centering
  \small
  \begin{threeparttable}
  \begin{tabular}{lllllll}
    \toprule
    \multirow{2}{*}{method} & \multicolumn{2}{c}{ACSP (Mean$\pm$Std)} & \multicolumn{2}{c}{ACVRP (Mean$\pm$Std)} & \multicolumn{2}{c}{PG (Mean$\pm$Std)} \\
    \cmidrule(r){2-3} \cmidrule{4-5} \cmidrule(r){6-7}
                            & 20 nodes & 50 nodes & 20 nodes & 50 nodes & 20 nodes & 50 nodes\\
    \midrule
    Heuristic & 6.20$\pm$1.33 & 7.60$\pm$1.37   & 7.64$\pm$1.30  & 13.27$\pm$1.87 & 2.43$\pm$0.98 & 4.19$\pm$1.69    \\
    RL against Stoc  & 3.56$\pm$0.37  & 4.57$\pm$0.58  & 7.67$\pm$1.30  & 13.85$\pm$1.53 &  2.50$\pm$0.95 & 4.26$\pm$2.17 \\
    CCDO-RL   & $\pmb{3.31}$$\pm$0.35 & $\pmb{4.39}$$\pm$0.52  & $\pmb{7.55}$$\pm$1.28  & $\pmb{13.15}$$\pm$1.59 & $\pmb{2.56}$$\pm$0.92 & $\pmb{4.70}$$\pm$1.94\\

    \bottomrule
  \end{tabular}
  \begin{tablenotes}
      \footnotesize
      \item[1] For the average reward of ACSP and ACVRP, smaller is better while for that of PG larger is better.
  \end{tablenotes}
  \end{threeparttable}
\end{table}
\vspace{-\baselineskip}
% two heuristics and one RL
% \begin{table}[ht]
%   \caption{{\color{red} Average reward of CCDO-RL (on seen graphs). For the value of CSP and CVRP, larger is better while for that of PG smaller is better.}}
%   \label{tab_aver}
%   \centering
%   \small
%   \begin{tabular}{lllllll}
%     \toprule
%     \multirow{2}{*}{method} & \multicolumn{2}{c}{CSP (Mean$\pm$Std)} & \multicolumn{2}{c}{CVRP (Mean$\pm$Std)} & \multicolumn{2}{c}{PG (Mean$\pm$Std)} \\
%     \cmidrule(r){2-3} \cmidrule{4-5} \cmidrule(r){6-7}
%                             & 20 nodes & 50 nodes & 20 nodes & 50 nodes & 20 nodes & 50 nodes\\
%     \midrule
%     Baseline 1 & 4.52$\pm$0.71  & 5.98$\pm$0.94 & 7.64$\pm$1.56  & 13.49$\pm$2.10 & 2.71$\pm$1.10 & 1.82$\pm$1.40   \\
%     Baseline 2 & 6.13$\pm$1.20 & 7.55$\pm$1.42   & 7.65$\pm$1.23  & 13.38$\pm$1.70 & 2.64$\pm$1.03 & 1.47$\pm$0.99  \\
%     RL {\color{red}against Stoc}    & 3.50$\pm$0.47  & 4.55$\pm$0.62  & 7.55$\pm$1.16  & 13.90$\pm$1.63 & 2.71$\pm$0.90 & 1.54$\pm$1.03   \\
%     CCDO-RL   & $\pmb{3.25}$$\pm$0.42 & $\pmb{4.31}$$\pm$0.51  & $\pmb{7.42}$$\pm$1.21  & $\pmb{13.28}$$\pm$1.52 &  $\pmb{2.75}$$\pm$0.87 & $\pmb{1.87}$$\pm$1.22  \\
%     \bottomrule
%   \end{tabular}
% \end{table}


% \begin{table}[htp]
%   \caption{{\color{red}Generalizability of CCDO-RL (on unseen graphs)}}
%   \label{tab_gene}
%   \centering
%   \small
%   \begin{threeparttable}
%   \begin{tabular}{lllllll}
%     \toprule
%     \multirow{2}{*}{method} & \multicolumn{2}{c}{CSP (Mean$\pm$Std)} & \multicolumn{2}{c}{CVRP (Mean$\pm$Std)} & \multicolumn{2}{c}{PG (Mean$\pm$Std)} \\
%     \cmidrule(r){2-3} \cmidrule{4-5} \cmidrule(r){6-7}
%                             & 20 nodes & 50 nodes & 20 nodes & 50 nodes & 20 nodes & 50 nodes\\
%     \midrule
%     Baseline 1 & 4.53$\pm$0.79  & 5.95$\pm$0.96 & 7.55$\pm$1.39  & 13.35$\pm$2.04 & 2.52$\pm$1.08 & $\pmb{1.86}$$\pm$1.44  \\
%     Baseline 2 & 6.20$\pm$1.33 & 7.60$\pm$1.37   & 7.64$\pm$1.3  & 13.27$\pm$1.87 & 2.43$\pm$0.98 & 1.52$\pm$1.20    \\
%     RL {\color{red}against Stoc}  & 3.56$\pm$0.37  & 4.57$\pm$0.58  & 7.67$\pm$1.30  & 13.85$\pm$1.53 &  2.50$\pm$0.95 & 1.03$\pm$5.05 \\
%     CCDO-RL   & $\pmb{3.31}$$\pm$0.35 & $\pmb{4.39}$$\pm$0.52  & $\pmb{7.55}$$\pm$1.28  & $\pmb{13.15}$$\pm$1.59 & $\pmb{2.56}$$\pm$0.92 & 1.35$\pm$5.09\\

%     \bottomrule
%   \end{tabular}
%   \begin{tablenotes}
%       \footnotesize
%       \item[1] For the value of CSP and CVRP, larger is better while for that of PG smaller is better.
%   \end{tablenotes}
%   \end{threeparttable}
% \end{table}



%\vspace{-5pt}
\section{Conclusion}
%\vspace{-5pt}
In this paper we considered the fully inductive link prediction task in KGs. We identified the open challenges in existing fully inductive models, and proposed \ourmethod, a novel architecture designed to improve expressiveness and support efficient relation prediction tasks. Through comprehensive experiments spanning 57 diverse KGs datasets, we demonstrate that increased expressiveness translates into better performance. Additionally, our experimental study sheds light on the limitations of LLMs in exploiting graph information in new domains for entity and relation prediction tasks.


% \vspace{-10pt}
% \paragraph{Limitations and Impact Statement.}
% Despite the promising results of \ourmethod in the inductive settings we consider, there are still aspects that can be refined in future research.
% For example, while \ourmethod aims to enhance expressivity compared to existing methods, there might still be scenarios where its representation power is limited, leading to challenges in capturing complex relationships.
% Furthermore, the relation adjacency matrix size introduces additional computational complexity during training, potentially requiring more resources and time compared to simpler models. Finally, while not designed for malicious intent, there is a possibility of \ourmethod being repurposed for activities like uncovering sensitive information, raising questions about data privacy and responsible AI use.
% For natbib users:

\newpage
\clearpage
\section*{Acknowledgments}
This work was funded in part by the National Science Foundation (NSF) awards, CCF-1918483, CAREER IIS-1943364 and CNS-2212160, Amazon Research Award, AnalytiXIN, and the Wabash Heartland Innovation Network (WHIN), Ford, NVidia, CISCO, and Amazon. Computing infrastructure was supported in part by CNS-1925001 (CloudBank). This work was supported in part by AMD under the AMD HPC Fund program. %Any opinions, findings and conclusions or recommendations expressed in this material are those of the authors and do not necessarily reflect the views of the sponsors.

\bibliographystyle{unsrtnat}
\bibliography{ref}
% For bibLaTeX users:
% \printbibliography

\newpage
\appendix
\section{Pseudo Codes of Iterative Embedding Updates}
\Cref{entity_code} and \Cref{relation_code} show the iterative embedding updates in entity prediction task and relation prediction task respectively.

\begin{minipage}{\textwidth}
\begin{algorithm}[H]
\small
\caption{\ourmethod embedding updates for the entity prediction task}
\label{entity_code}
\begin{algorithmic}
\Require Query ($h$, $r$, ?); relation adjacency matrix $\mA_R$; entity adjacency matrix $\mA_V$; number of updates $L$
\Ensure Final entity embedding $\mX_{h,r}^{(L)}$
% \State $\mX_{h,r}^{(0)} = \text{INDICATOR}(h) \odot \mathbf{1}^{|V|\times d}$
% \State $\mZ_{h,r}^{(0)} = \text{INDICATOR}(r) \odot \mathbf{1}^{|R|\times d}$
\State $\mX_{h,r}^{(0)} = \text{INIT}_{V}(h)$ \Comment{Label $h$ with all-ones vector and the rest with all-zeros}
\State $\mZ_{h,r}^{(0)} = \text{INIT}_{R}(r)$ \Comment{Label $r$ with all-ones vector and the rest with all-zeros}
\For{$i \gets 1$ to $L$}
\For{$u \in V$}
\State $\mX_{h,r}^{(i)}(u) = \text{UP}_{V}^{(i)}\left(\mX_{h,r}^{(i-1)}(u), \text{AGG}_{V}^{(i)}\left(\text{MSG}_{V}^{(i)}(\mX_{h,r}^{(i-1)}(v), \mZ_{h,r}^{(i-1)}(r^\prime))|(u, r^\prime, v) \in \mA_V\right)\right)$
\EndFor
\For{$r^\prime \in R$}
\State $\mZ_{h,r}^{(i)}(r^\prime) = \text{UP}_{R}^{(i)}\left(\mZ_{h,r}^{(i-1)}(r^\prime), \text{AGG}_{R}^{(i)}\left(\text{MSG}_{R}^{(i)}(\mZ_{h,r}^{(i-1)}(r^{\prime\prime}), \mX_{h,r}^{(i)}(u))|(r^\prime,u,r^{\prime\prime})\in \mA_R\right)\right)$
\EndFor
\EndFor
\end{algorithmic}    
\end{algorithm}
\end{minipage}

\begin{minipage}{\textwidth}
\begin{algorithm}[H]
\small
\caption{\ourmethod embedding updates for the relation prediction task}
\label{relation_code}
\begin{algorithmic}
\Require Query ($h$, ?, $t$); relation adjacency matrix $\mA_R$; entity adjacency matrix $\mA_V$; number of updates $L$
\Ensure Final relation embedding $\mZ_{h,t}^{(L)}$
\State $\mZ_{h,t}^{(0)} = \mathbf{1}^{|R|\times d}$
\State $\mX_{h,t}^{(0)} = \text{INIT}(h, t)$ \Comment{Label $h$ with all-ones vector, $t$ with all-negative-ones and the rest with all-zeros}
% \State $\mX_{h,t}^{(0)} = \text{INDICATOR}(h, t) \odot \mathbf{1}^{|V|\times d}$
\For{$i \gets 1$ to $L$}
\For{$r \in R$}
\State $\mZ_{h,t}^{(i)}(r) = \text{UP}_{R}^{(i)}\left(\mZ_{h,t}^{(i-1)}(r), \text{AGG}_{R}^{(i)}\left(\text{MSG}_{R}^{(i)}(\mZ_{h,t}^{(i-1)}(r^\prime), \mX_{h,t}^{(i-1)}(u))|(r,u,r^\prime)\in \mA_R\right)\right)$
\EndFor
\For{$u \in V$}
\State $\mX_{h,t}^{(i)}(u) = \text{UP}_{V}^{(i)}\left(\mX_{h,t}^{(i-1)}(u), \text{AGG}_{V}^{(i)}\left(\text{MSG}_{V}^{(i)}(\mX_{h,t}^{(i-1)}(v), \mZ_{h,t}^{(i)}(r))|(u, r, v) \in \mA_V\right)\right)$
\EndFor
\EndFor
\end{algorithmic}    
\end{algorithm}
\end{minipage}

\section{Expressive Power}\label{app:expressive-power}
\section{Proofs}\label{sec:proofs}
\subsection{Propagation of chaos for MFLD (Section \ref{sec:main_results})}\label{subsec:poc_proof}
\begin{proof}[Proof of Lemma \ref{lemma:clsi}]
    The first equality of the assertion was proved by \citet{nitanda2024improved}. We here prove the inequality by utilizing the argument of conditional and marginal distribution of $\pow[\mu,N]$ \cite{chen2022uniform}.
    
    For $\vX \sim \pow[\mu,N]$, we denote by $\pow[\mu,N]_{i|-i}(\cdot | \vx^{-i})$ and $\pow[\mu,N]_{-i}$ the conditional distribution of $X^i$ conditioned by $\vX^{-i}=\vx^{-i}$ and the marginal distribution of $\vX^{-i}$, respectively.
    It holds that 
    \begin{align} 
        &\bE_{\vx \sim \pow[\mu,N]}\left[ \left\| \nabla \log \frac{\rd \pow[\mu,N]}{\rd \pow[\mu,N]_*}(\vX) \right\|_2^2\right] \notag\\
        &= \sum_{i=1}^N \bE_{\vx \sim \pow[\mu,N]}\left[ \left\| \nabla_{x^i} \log \frac{\rd \pow[\mu,N]}{\rd \vx}(\vX) + \frac{N}{\lambda}\nabla_{x^i}F(\mu_\vx)\right\|_2^2\right] \notag\\
        &= \sum_{i=1}^N \bE_{\vX^{-i} \sim \pow[\mu,N]_{-i}} \left[ \bE_{X^i \sim \pow[\mu,N]_{i|-i}(\cdot | \vX^{-i})}\left[ \left\| \nabla_{x^i} \log \frac{\rd \pow[\mu,N]}{\rd \vx}(\vX) + \frac{N}{\lambda}\nabla_{x^i}F(\mu_\vx)\right\|_2^2 \right]\right]. \label{eq:fisher_div}
    \end{align}

    We write $p_{-i}(\vx^{-i})=\frac{\rd \pow[\mu,N]_{-i}}{\rd \vx^{-i}}(\vx^{-i})$ and $p_{i|-i}(x|\vx^{-i})=\frac{\rd \pow[\mu,N]_{i|-i}(\cdot|\vx^{-i})}{\rd x}(x)$.
    Since $\frac{\rd \pow[\mu,N]}{\rd \vx}(\vx) = p_{-i}(\vx^{-i}) p_{i|-i}(x^i|\vx^{-i})$, we get the following equation:
    \begin{equation*}
        \nabla_{x^i} \log \frac{\rd \pow[\mu,N]}{\rd \vx}(\vx)
        = \frac{\nabla_{x^i}(p_{-i}(\vx^{-i}) p_{i|-i}(x^{i}|\vx^{-i}))}{p_{-i}(\vx^{-i}) p_{i|-i}(x^{i}|\vx^{-i})}
        = \frac{\nabla_{x^i}p_{i|-i}(x^i|\vx^{-i})}{p_{i|-i}(x^i|\vx^{-i})}
        = \nabla \log p_{i|-i}(x^i|\vx^{-i}).
    \end{equation*}
    Hence, Eq.~\eqref{eq:fisher_div} can be further bounded by the LSI on the conditional Gibbs distribution (Assumption \ref{assumption:uniform_directional_lsi}) as follows: 
    \begin{align}
        &\sum_{i=1}^N \bE_{\vX^{-i} \sim \pow[\mu,N]_{-i}} \left[ \bE_{X^i \sim \pow[\mu,N]_{i|-i}(\cdot |\vX^{-i})}\left[ \left\| \nabla \log p_{i|-i}(X^i|\vX^{-i}) + \frac{N}{\lambda}\nabla_{x^i}F(\mu_\vx)\right\|_2^2 \right]\right] \notag \\
        &=\sum_{i=1}^N \bE_{\vX^{-i} \sim \pow[\mu,N]_{-i}} \left[ \bE_{X^i \sim \pow[\mu,N]_{i|-i}(\cdot|\vX^{-i})}\left[ \left\| \nabla \log \frac{\rd \pow[\mu,N]_{i|-i}}{\rd \nu_{i|-i}}(X^i|\vX^{-i}) \right\|_2^2 \right]\right] \notag \\
        &\geq 2\alpha  \sum_{i=1}^N \bE_{\vX^{-i} \sim \pow[\mu,N]_{-i}} \left[ \KL( \pow[\mu,N]_{i|-i}(\cdot|\vX^{-i}) \| \nu_{i|-i}(\cdot|\vX^{-i})) \right].  \label{eq:fisher_div_eval}
    \end{align}

    Let $\nu$ be the probability distribution on $\bR^d$ with the density $\frac{\rd \nu}{\rd x}(x) \propto \exp(-r(x)/\lambda)$. Here, notice that the conditional Gibbs distribution $\nu_{i|-i}(\cdot|\vx^{-i})$ is the minimizer of the following objective: for $\vx^{-i}$
    \begin{align*} 
        \nu_{i|-i}(\cdot|\vx^{-i}) 
        &= \argmin_{\mu \in \cP_2(\bR^d)}\left\{
        \int NF_0(\rho_{x\cup\vx^{-i}})\mu(\rd x) 
        + \bE_{X\sim \mu}[r(X)] + \lambda \Ent(\mu)
        \right\} \\
        &= \argmin_{\mu \in \cP_2(\bR^d)}\left\{
        \int NF_0(\rho_{x\cup\vx^{-i}})\mu(\rd x) 
        + \lambda\KL(\mu\|\nu)
        \right\}.
    \end{align*}

    Because of the optimality of the conditional Gibbs distribution, we have 
    \begin{align}
        &\lambda\KL( \pow[\mu,N]_{i|-i}(\cdot|\vX^{-i}) \| \nu_{i|-i}(\cdot|\vX^{-i})) \notag\\
        &=\lambda \Ent(\pow[\mu,N]_{i|-i}(\cdot|\vX^{-i}))
        + \int NF(\rho_{x \cup \vX^{-i}}) \pow[\mu,N]_{i|-i}(\rd x|\vX^{-i})
        + \lambda \log \int \exp\left(-\frac{N}{\lambda}F(\rho_{x\cup \vX^{-i}})\right) \rd x \notag \\
        &=\lambda \KL\left(\pow[\mu,N]_{i|-i}(\cdot|\vX^{-i} )\| \nu \right)
        + \int NF_0(\rho_{x \cup \vX^{-i}}) \left(\pow[\mu,N]_{i|-i}(\rd x|\vX^{-i})
        - \nu_{i|-i}(\rd x|\vX^{-i})(\rd x)\right)  - \lambda \KL\left(\nu_{i|-i}(\cdot|\vX^{-i}) \| \nu\right) \notag \\
        &\geq \lambda \KL\left(\pow[\mu,N]_{i|-i}(\cdot|\vX^{-i} )\| \nu \right)
        + \int NF_0(\rho_{x \cup \vX^{-i}}) \left(\pow[\mu,N]_{i|-i}(\rd x|\vX^{-i})
        - \mu_*(\rd x)\right)  - \lambda \KL\left(\mu_* \| \nu\right). \label{eq:fisher_div_eval2}
    \end{align}

    The expectation of the second term of the right-hand side can be futher evaluated as 
    \begin{align}
        &N \sum_{i=1}^N \bE_{\vX \sim \pow[\mu,N]} \left[ \int F_0(\rho_{x \cup \vX^{-i}}) \left(\pow[\mu,N]_{i|-i}(\rd x|\vX^{-i})- \mu_*(\rd x)\right) \right] \notag \\
        &= N \sum_{i=1}^N \bE_{\vX \sim \pow[\mu,N]} \left[ F_0(\rho_\vX) - \int F_0(\rho_{x \cup \vX^{-i}})\mu_*(\rd x) \right] \notag \\
        &= -N \sum_{i=1}^N \bE_{\vX \sim \pow[\mu,N]} \left[\int \left\{ B_{F_0}(\rho_{x \cup \vX^{-i}},\rho_\vX) + \pd< \frac{\delta F_0(\rho_\vX)}{\delta \mu}, \rho_{x \cup \vX^{-i}}-\rho_\vX> \right\} \mu_*(\rd x)\right] \notag \\
        &= -N \sum_{i=1}^N \bE_{\vX \sim \pow[\mu,N]} \left[\int B_{F_0}(\rho_{x \cup \vX^{-i}},\rho_\vX) \mu_*(\rd x)\right] \notag \\
        &~~~~ + \sum_{i=1}^N \bE_{\vX \sim \pow[\mu,N]} \left[\int \left\{ \frac{\delta F_0(\rho_\vX)}{\delta \mu}(X^i) - \frac{\delta F_0(\rho_\vX)}{\delta \mu}(x) \right\} \mu_*(\rd x)\right] \notag \\
        &= -N \sum_{i=1}^N \bE_{\vX \sim \pow[\mu,N]} \left[\int B_{F_0}(\rho_{x \cup \vX^{-i}},\rho_\vX) \mu_*(\rd x)\right] \notag \\
        &~~~~ + N \bE_{\vX \sim \pow[\mu,N]} \left[\int \frac{\delta F_0(\rho_\vX)}{\delta \mu}(x) \left(\rho_{\vX}(\rd x) - \mu_*(\rd x)\right) \right] \notag \\
        &\geq - B 
        + N \left(\bE_{\vX \sim \pow[\mu,N]} \left[ F_0(\rho_\vX)\right] - F_0(\mu_*) \right), \label{eq:fisher_div_eval3}
    \end{align}
    where the last inequality is due to the convexity of $F_0$ and Assumption \ref{assumption:nonlinearity}.

    By the information inequality (Lemma 5.1 of \citet{chen2022uniform}), the first term of Eq.~\eqref{eq:fisher_div_eval2} of the right-hand side can be evaluated as 
    \begin{align}
        \sum_{i=1}^N \bE_{\vX \sim \pow[\mu,N]}\left[ \KL\left(\pow[\mu,N]_{i|-i}(\cdot|\vX^{-i} )\| \nu \right) \right] 
        \geq \KL(\pow[\mu,N] \| \tensor[\nu,N]).
    \end{align}
    
    Combining all of them, we get
    \begin{align*}
        &\bE_{\vx \sim \pow[\mu,N]}\left[ \left\| \nabla \log \frac{\rd \pow[\mu,N]}{\rd \pow[\mu,N]_*}(\vX) \right\|_2^2\right] \\
        &\geq \frac{2\alpha}{\lambda} \left\{ - B + N \bE_{\vX \sim \pow[\mu,N]} \left[ F_0(\rho_\vX) \right] + \lambda \KL(\pow[\mu,N] \| \tensor[\nu,N]) 
        - N(F_0(\mu_*) + \lambda \KL\left(\mu_* \| \nu\right)) \right\} \\
        &= \frac{2\alpha}{\lambda} \left\{ - B + N \bE_{\vX \sim \pow[\mu,N]} \left[ F(\rho_\vX) \right] + \lambda \Ent(\pow[\mu,N]) - N(F(\mu_*) +\lambda\Ent(\mu_*)) \right\} \\
        &= \frac{2\alpha}{\lambda} \left( - B + \pow[\cL,N](\pow[\mu,N]) - N \cL(\mu_*) \right).
    \end{align*}
    This concludes the proof.
\end{proof}

\begin{proof}[Proof of Theorem \ref{theorem:mfld_convergence}]
    We here prove the convergence of MFLD in the discrete-time by using the one-step interpolation argument \cite{nitanda2024improved,suzuki2023convergence}. 
    
    We construct the one-step interpolation for $k$-th iteration: $X_{k+1}^i = X_k^i - \eta \nabla \frac{\delta F(\rho_{\vX_k})}{\delta \mu}(X_k^i) + \sqrt{2\lambda \eta} \xi_k^i,~(i\in \{1,2,\ldots,d\})$.
    as follows: for $i\in \{1,2,\ldots,d\}$,
    \begin{equation}\label{eq:noisy-GD_dynamics}
        \rd Y_t^i = - \nabla \frac{\delta F(\rho_{\vY_0})}{\delta \mu}(Y_0^i)\rd t + \sqrt{2\lambda}\rd W_t,
    \end{equation}
    where $\vY_0 = (Y_0^1,\ldots,Y_0^d) = (X_k^1,\ldots,X_k^d)$ and $W_t$ is the standard Brownian motion in $\bR^d$ with $W_0 = 0$.
    We denote by $\nu_t$ the distributions of $\vY_t$. 
    Then, $\nu_0 = \pow[\mu,N]_k (= \mathrm{Law}(\vX_k))$, $\nu_\eta = \pow[\mu,N]_{k+1} (=\mathrm{Law}(\vX_{k+1}))$ (i.e., $\vY_{\eta} \disteq \vX_{k+1}$).
    In this proof, we identify the probability distribution with its density function with respect to the Lebesgure measure for notational simplicity. For instance, we denote by $\pow[\mu,N]_*(\vy)$ the density of $\pow[\mu,N]_*$. 

    By the proof of Theorem 2 in \citet{nitanda2024improved}, we see for $t \in [0,\eta]$,
    \begin{align}
        \frac{\rd \pow[\cL,N]}{\rd t}(\nu_t)
        &\leq - \frac{\lambda^2}{2} \int \nu_t(\vy) \left\| \nabla \log \frac{\nu_t}{\pow[\mu,N]_*}(\vy) \right\|_2^2 \rd \vy 
        + N \delta_\eta, \label{eq:one_step_decrease}
    \end{align}
    where $\delta_\eta = 8\eta( C_2^2 + \lambda^{\prime 2}) (\eta C_1^2 + \lambda d) 
    + 32 \eta^2 \lambda'^2( C_2^2 + \lambda^{\prime 2}) \left( \frac{1}{N}\bE\left[ \left\| \vX_0 \right\|_2^2 \right] + \frac{1}{\lambda'}\left(\frac{C_1^2}{4\lambda'} + \lambda d\right) \right)$.

    Combining Lemma \ref{lemma:clsi} with the above inequality, we get
    \begin{align*}
        &\frac{\rd \pow[\cL,N]}{\rd t}(\nu_t)
        \leq  - \alpha \lambda \left( \pow[\cL,N](\nu_t) - N \cL(\mu_*) - B \right)
        + N \delta_\eta. \\
        \Longleftrightarrow 
        ~~~~&\frac{\rd}{\rd t}\left( \pow[\cL,N](\nu_t) - N \cL(\mu_*) - B - \frac{N\delta_\eta}{\alpha \lambda} \right)
        \leq  - \alpha \lambda \left( \pow[\cL,N](\nu_t) - N \cL(\mu_*) - B - \frac{N\delta_\eta}{\alpha \lambda} \right).
    \end{align*}    

    Noting $\nu_\eta = \pow[\mu,N]_{k+1}$ and $\nu_0 = \pow[\mu,N]_k$, the Gr\"{o}nwall’s inequality leads to 
    \[ 
        \pow[\cL,N](\pow[\mu,N]_{k+1}) -  N\cL(\mu_*) - B - \frac{ N\pow[\delta,N]_{\eta}}{\alpha \lambda} 
        \leq \exp( -\alpha\lambda\eta )\left(  \pow[\cL,N](\pow[\mu,N]_k) -  N\cL(\mu_*) - B- \frac{N\pow[\delta,N]_{\eta}}{\alpha \lambda} \right). 
    \]
    
This inequality holds at every iteration of (\ref{eq:noisy-GD_dynamics}). Hence, we arrive at the desired result,  
\begin{align*}
    \frac{1}{N}\pow[\cL,N](\pow[\mu,N]_k) - \cL(\mu_*)  
    &\leq \frac{B}{N} + \frac{ \pow[\delta,N]_{\eta}}{\alpha \lambda} 
    + \exp( -\alpha\lambda\eta k )\left( \frac{1}{N}\pow[\cL,N](\pow[\mu,N]_0) - \cL(\mu_*) - \frac{B}{N} - \frac{ \pow[\delta,N]_{\eta}}{\alpha \lambda} \right) \\
    &\leq \frac{B}{N} + \frac{ \pow[\delta,N]_{\eta}}{\alpha \lambda} 
    + \exp( -\alpha\lambda\eta k )\left( \frac{1}{N}\pow[\cL,N](\pow[\mu,N]_0) - \cL(\mu_*) \right).
\end{align*}
\end{proof}

\subsection{Point-wise model approximation error (Section \ref{subsec:pw_model_error})}\label{subsec:pw_model_error_proof}
\begin{proof}[Proof of Lemma \ref{lemma:han_inequality}]
    It follows that by Han's inequality \citep{dembo1991information},
    \[ \frac{1}{s \binom{N}{s}} \sum_{|S|=s} \int \pow[\mu,N]_S(\rd \vx_S) \log \frac{\rd \pow[\mu,N]_S}{\rd \vx_S}(\vx_S) \leq \frac{1}{N}\int \pow[\mu,N](\rd x) \log \frac{\rd \pow[\mu,N]}{\rd \vx}(\vx). \]
    Moreover, we see
    \begin{align*} 
        \sum_{|S|=s} \int \pow[\mu,N]_S(\rd \vx_S) \log \frac{\rd \tensor[\mu_*,k]}{\rd \vx_S}(\vx_S)
        &= \sum_{|S|=s} \sum_{i \in S}\int \pow[\mu,N]_i(\rd x^i) \log \frac{\rd \mu_*}{\rd x}(x^i)  \\
        &= \binom{N-1}{s-1}\sum_{i=1}^N\int \pow[\mu,N]_i(\rd x^i) \log \frac{\rd \mu_*}{\rd x}(x^i) \\
        &= \binom{N-1}{s-1} \int \pow[\mu,N](\rd \vx) \log \frac{\rd \tensor[\mu_*,N]}{\rd \vx}(\vx). 
    \end{align*}
    Noticing $\binom{N-1}{s-1} = \frac{s}{N}\binom{N}{s}$, we conclude the first statement which immediately implies the second statement.
\end{proof}


\begin{proposition}\label{prop:pw_poc_merge}
    Suppose $\pow[\mu,N]$ is exchangeable and $\vX_{j} \sim \tensor[\mu,N]$ $(j=1,2,\ldots,M)$. Then, it follows that for any $z \in \cZ$,
    \begin{align*} 
        \bE_{\{\vX_j\}_{j=1}^M}\left[ \left( \frac{1}{M}\sum_{j=1}^M\bE_{\rho_{\vX_j}}[ h(X,z)] - \bE_{\mu_*}[h(X,z)]\right)^2 \right] 
        \leq \frac{4R^2}{N M} + \frac{8R^2}{M}\sqrt{ \frac{\KL(\pow[\mu,N]\|\tensor[\mu_*,N])}{N}} 
    + \frac{2R^2 \KL(\pow[\mu,N] \| \tensor[\mu,N]_*)}{N}.
    \end{align*}
\end{proposition}
\begin{proof}[Proof of Proposition \ref{prop:pw_poc_merge}]
    \begin{align*}
        \bE_{\{\vX_j\}_{j=1}^M}&\left[ \left( \frac{1}{M}\sum_{j=1}^M\bE_{X \sim \rho_{\vX_j}}[ h(X,z)] - \bE_{X\sim\mu_*}[h(X,z)]\right)^2 \right] \\
        &= \frac{1}{M^2}\bE_{\{\vX_j\}_{j=1}^M}\left[ \sum_{j=1}^M\left(\bE_{X \sim \rho_{\vX_j}}[ h(X,z)] - \bE_{X\sim\mu_*}[h(X,z)]\right)^2 \right] \\
        &+ \frac{1}{M^2}\bE_{\{\vX_j\}_{j=1}^M}\left[ \sum_{j\neq k}\left( \bE_{X \sim \rho_{\vX_j}}[ h(X,z)] - \bE_{X\sim\mu_*}[h(X,z)]\right)\left( \bE_{X \sim \rho_{\vX_k}}[ h(X,z)] - \bE_{X\sim\mu_*}[h(X,z)]\right) \right].
    \end{align*}
    Using Proposition \ref{prop:pw_model_approximation}, we can upper bound the first term by $\frac{4R^2}{Ms'} + \frac{8R^2}{M}\sqrt{ \frac{\KL(\pow[\mu,N]\|\tensor[\mu_*,N])}{N}}$. 
    The second term can be evaluated as follows. Set $H(\vX_j) = \bE_{X \sim \mu_{\vX_j}}[ h(X,z)] - \bE_{X\sim\mu_*}[h(X,z)]$. Then for $j\neq k$,
    \begin{align*}
        \bE_{\{\vX_j\}_{j=1}^M}\left[ H(\vX_j)H(\vX_k) \right]
        &= \left( \bE_{\vX_j}\left[ H(\vX_j) \right] \right)^2 \\
        &= \left( \bE_{\vX_j}\left[  \frac{1}{s'} \sum_{i=1}^{s'} h(X_j^i,z) \right] - \bE_{X\sim\mu_*}[h(X,z)] \right)^2 \\
        &= \left( \bE_{X \sim \pow[\mu,N]_1}\left[  h(X,z) \right] - \bE_{X\sim\mu_*}[h(X,z)] \right)^2 \\
        &\leq 4R^2 \TV^2(\pow[\mu,N]_1,\mu_*) \\
        &\leq 2R^2 \KL(\pow[\mu,N]_1 \| \mu_*) \\
        &\leq \frac{2R^2}{N}\KL(\pow[\mu,N] \| \tensor[\mu,N]_*).
    \end{align*}
    This concludes the proof.
\end{proof}

%%%%%
\subsection{Uniform model approximation error (Section \ref{subsec:uniform_model_error})}\label{subsec:uniform_model_error_proof}
We evaluate the empirical Rademacher complexity $\hat{\cR}_{N,M}(\cF)$ by using Dudley's entropy integral. We define the metric $\|f\|_{N,M,2} = \sqrt{\frac{1}{MN}\sum_{j=1}^M\sum_{i=1}^N |f(X^i_j)|^2}$.
We denote by $\cN(\cF,\epsilon,\|\cdot\|_{N,M,2})$ the $\epsilon$-covering number of $\cF$ with respect to the $\|\cdot\|_{N,M,2}$-norm.
\begin{lemma}[Dudley's entropy integral]\label{lemma:dudley}
    Given a function class $\cF$ on $\bR^d$, we suppose $R = \sup_{f\in\cF}\|f\|_{N,M,2} < \infty$. Then,
    \begin{equation*}
        \hat{\cR}_{N,M}(\cF) 
        \leq \inf_{\delta > 0} 
        \left\{
            4\delta 
            + \frac{12}{\sqrt{MN}} \int_{\delta}^{R} \sqrt{\log 2 \cN(\cF,\epsilon,\|\cdot\|_{N,M,2})} \rd\epsilon
        \right\}.
    \end{equation*}
\end{lemma}

\begin{proposition}\label{proposition:rademacher_complexity_bound}
    Suppose Assumption \ref{assumption:model_constraint} holds and $\vX_j \sim \pow[\mu,N]$ $(j=1,2,\ldots,M)$ are independent. Then, we get
    \[ \bE_{\{\vX_j\}_{j=1}^M} \left[\hat{\cR}_{N,M}(\cF) \right] 
    \leq 4 R\sqrt{\frac{d}{MN}}
        + 12R\sqrt{\frac{1}{MN} \left(\log 2 + d\log \left( 1 + 2\beta MR^{-1}\sqrt{MNd^{-1}}\bE_{\vX\sim\pow[\mu,N]}[\|\vX\|_2]\right)\right)}. \]
\end{proposition}
\begin{proof}
    Since $\| f \|_{N,M,2} \leq \| f \|_{N,M,\infty} = \max_{i,j}|f(X^i_j)|$, it is sufficient evaluate the $\epsilon$-covering number of $\cF$ with respect to $\| \cdot \|_{N,M,\infty}$. 
    We write $r=\max_{i,j}\|X^i_j\|_2$. By Assumption \ref{assumption:model_constraint}, for any $z, z' \in \cZ$,
    \[ \max_{i,j}| h(X^i_j,z) - h(X^i_j,z')| 
    \leq \max_{i,j} \beta \|X^i_j\|_2  \| z - z' \|_2 
    =  \beta r \| z - z' \|_2, \]
    we see $\cN(\cF, \epsilon,\|\cdot\|_{N,M,\infty}) \leq \cN\left(\cZ, \epsilon/(\beta r), \|\cdot\|_{2}\right) = \left( 1 + \frac{2 \beta r}{\epsilon}\right)^{d}$.

    Therefore, by Lemma \ref{lemma:dudley} with $\delta = R\sqrt{d(MN)^{-1}}$, we get
    \begin{align*}
        \hat{\cR}_{N,M}(\cF) 
        &\leq 4R \sqrt{\frac{d}{MN}}
        + 12R\sqrt{\frac{1}{MN}\log 2 \cN(\cF,R\sqrt{d(MN)^{-1}},\|\cdot\|_{N,M,\infty})} \\
        &=4R \sqrt{\frac{d}{MN}}
        + 12R\sqrt{\frac{1}{MN} \left(\log 2 + d\log \left( 1 + 2 \beta r R^{-1}\sqrt{MNd^{-1}}\right)\right)} \\
        &\leq 4R \sqrt{\frac{d}{MN}}
        + 12R\sqrt{\frac{1}{MN} \left(\log 2 + d\log \left( 1 + 2 \beta R^{-1}\sqrt{MNd^{-1}} \sum_{j=1}^M \|\vX_j\|_2\right)\right)},
    \end{align*}
    where we used $r \leq \sum_{j=1}^M \|\vX_j\|_2$.
    Finally, Jensen's inequality yields
    \[ \bE_{\{\vX_j\}_{j=1}^M} \left[\hat{\cR}_{N,M}(\cF) \right] 
    \leq 4R \sqrt{\frac{d}{MN}}
        + 12R\sqrt{\frac{1}{MN} \left(\log 2 + d\log \left( 1 + 2\beta MR^{-1}\sqrt{MNd^{-1}}\bE_{\vX\sim\pow[\mu,N]}[\|\vX\|_2] \right)\right)}.\]
\end{proof}

Here, we give the complete version of the uniform model approximation bound.
\begin{theorem}[Complete version of Theorem \ref{theorem:uniform_approximation_multiple_mfld}]\label{theorem:uniform_approximation_multiple_mfld_complete}
    Suppose Assumption \ref{assumption:model_constraint} and the same conditions as in Theorem \ref{theorem:mfld_convergence} hold. Run $M$-parallel MFLD in the discrete time independently, with $\eta \lambda' < 1/2$ and $\vX_{j,0} \sim \tensor[\mu,N]_0 (j=1,2,\ldots,M)$. Then,
    \begin{align*}
        \bE_{\{\vX_{j,k}\}}&\left[\left\| \frac{1}{M}\sum_{j=1}^M \bE_{X\sim \rho_{\vx_{j,k}}}[h(X,\cdot)] - \bE_{X\sim \mu_*} [h(X,\cdot)]\right\|_\infty\right] \\
        &\leq \frac{5CR}{4}\sqrt{\frac{d}{MN} + \frac{dB}{\lambda N}}
        + CR\sqrt{\frac{d\lambda}{MN(\lambda + MB)}}\log\left( C'\sqrt{\left(\lambda+MB\right)\frac{\pi}{\lambda}} \right) \\
        &+ CR\sqrt{\frac{d\lambda M N}{\lambda + MB}} \left(\frac{ \delta_{\eta}}{\alpha\lambda^2} 
        + \frac{1}{\lambda}\exp( -\alpha\lambda\eta k )\pow[\Delta,N]_0 \right),\\
        %&=\tilde{O}\left( R\sqrt{\frac{d}{MN} + \frac{dB}{\lambda N}
        %+ \frac{d\lambda}{MN(\lambda + MB)}} \right) \\
        %&+ O\left( R\sqrt{\frac{d\lambda M N}{\lambda + MB}} \left(\frac{ \delta_{\eta}}{\alpha\lambda^2} 
        %+ \frac{1}{\lambda}\exp( -\alpha\lambda\eta k )\pow[\Delta,N]_0 \right) \right),
    \end{align*}
    where $C'=1 + 2\beta MR^{-1}\sqrt{MNd^{-1}}\bE_{\vX\sim\tensor[\mu,N]_*}[\|\vX\|_2]$.
\end{theorem}
\begin{proof}
    For $\vx_1,\ldots,\vx_M \in \bR^{dN}$, we set 
    $g(\vx_1,\ldots,\vx_M) = \sup_{z \in \cZ} 
    \left| \frac{1}{M}\sum_{j=1}^M \bE_{X\sim \rho_{\vx_j}}[h(X,z)] - \bE_{X\sim \mu_*}[h(X,z)]\right|$.
    By the variational formulation of KL-divergence (e.g., Corollary 4.15 in \citet{boucheron2013concentration}), we get
    \begin{align}\label{eq:variational_kl}
        \bE_{\mu^{(N)\otimes M}}[g]
        &\leq \frac{1}{\gamma} \log \bE_{\mu^{\otimes N M}_*}[\exp(\gamma g)] + \frac{\KL(\mu^{(N)\otimes M}\|\mu^{\otimes N M}_*)}{\gamma} \notag \\
        &\leq \frac{1}{\gamma} \log \bE_{\mu^{\otimes N M}_*}[\exp(\gamma g)] + \frac{M\KL(\mu^{(N)}\|\mu^{\otimes N}_*)}{\gamma} 
    \end{align}

    For independent random variables $\vX_j \sim \tensor[\mu,N]_* (j=1,2,\ldots,M)$, by Lemma \ref{lemma:uniform_lln} and \ref{proposition:rademacher_complexity_bound}, it follows that with high probability $1-\delta$,
    \begin{align*}
        &g(\vX_1,\ldots,\vX_M) \\
        &\leq 2\bE_{\{\vX_j\}_{j=1}^M}\left[\hat{\cR}_{N,M}(\cF)\right] 
        + R \sqrt{ \frac{2\log(1/\delta)}{MN}} \\
        &\leq 8 R\sqrt{\frac{d}{MN}}
        + 24R\sqrt{\frac{1}{MN} \left(\log 2 + d\log \left( 1 + 2\beta MR^{-1}\sqrt{MNd^{-1}}\bE_{\vX\sim\tensor[\mu,N]_*}[\|\vX\|_2]\right)\right)}
        + R \sqrt{ \frac{2\log(1/\delta)}{MN}}\\
        &\leq CR\sqrt{\frac{d\log(C'/\delta)}{MN}},
    \end{align*}
    where $C$ is a uniform constant and $C'=1 + 2\beta MR^{-1}\sqrt{MNd^{-1}}\bE_{\vX\sim\tensor[\mu,N]_*}[\|\vX\|_2]$.
    This means
    \begin{align*}
        &\bP_{\mu^{\otimes N M}_*}\left[ g(\vX_1,\ldots,\vX_M) > CR\sqrt{\frac{d\log(C'/\delta)}{MN}} \right] \leq \delta \\
        \iff
        &\bP_{\mu^{\otimes N M}_*}\left[ g(\vX_1,\ldots,\vX_M) > t \right] \leq C'\exp\left( - \frac{MNt^2}{dC^2R^2}\right) \\
        \iff
        &\bP_{\mu^{\otimes N M}_*}\left[ g(\vX_1,\ldots,\vX_M) > \frac{1}{\gamma}\log t \right] \leq C'\exp\left( - \frac{MN(\log t)^2}{dC^2R^2\gamma^2}\right).
    \end{align*}
    Using this tail bound, 
    \begin{align*}
        \bE_{\mu^{\otimes N M}_*}[\exp(\gamma g)]
        &=\int_0^\infty \bP_{\mu^{\otimes N M}_*}\left[ \exp(\gamma g(\vX_1,\ldots,\vX_M)) > t \right] \rd t \\
        &=\int_0^\infty \bP_{\mu^{\otimes N M}_*}\left[ g(\vX_1,\ldots,\vX_M) > \frac{1}{\gamma} \log t \right] \rd t \\
        &=\int_0^\infty C'\exp\left( - \frac{MN(\log t)^2}{dC^2R^2\gamma^2} \right) \rd t \\
        &=C'CR\gamma\sqrt{\frac{\pi d}{MN}}\exp\left( \frac{dC^2R^2\gamma^2}{4MN} \right).
    \end{align*}
    Therefore, we get
    \begin{equation*}
        \bE_{\mu^{(N)\otimes M}}[g]
        \leq \frac{dC^2R^2\gamma}{4MN} + \frac{1}{\gamma}\log\left( C'CR\gamma\sqrt{\frac{\pi d}{MN}} \right)
        + \frac{M\KL(\mu^{(N)}\|\mu^{\otimes N}_*)}{\gamma}.
    \end{equation*}
    Moreover, by applying Lemma \ref{lemma:clsi} Theorem \ref{theorem:mfld_convergence} to Eq.~\eqref{eq:variational_kl}, we get
    \begin{align*}
        \bE_{\{\vX_{j,k}\}}&\left[\left\| \frac{1}{M}\sum_{j=1}^M \bE_{X\sim \rho_{\vx_{j,k}}}[h(X,\cdot)] - \bE_{X\sim \mu_*} [h(X,\cdot)]\right\|_\infty\right] \\
        &\leq \frac{dC^2R^2\gamma}{4MN}
        + \frac{1}{\gamma}\log\left( C'CR\gamma\sqrt{\frac{\pi d}{MN}} \right)
        + \frac{M}{\gamma} \left(  \frac{B}{\lambda} + \frac{ N\delta_{\eta}}{\alpha\lambda^2} 
        + \frac{N}{\lambda}\exp( -\alpha\lambda\eta k )\pow[\Delta,N]_0 \right).
    \end{align*}

    Finally, by seting $\gamma = \frac{1}{CR}\sqrt{\frac{MN}{d}\left(1+\frac{MB}{\lambda}\right)}$, we get
    \begin{align*}
        \bE_{\{\vX_{j,k}\}}&\left[\left\| \frac{1}{M}\sum_{j=1}^M \bE_{X\sim \rho_{\vx_{j,k}}}[h(X,\cdot)] - \bE_{X\sim \mu_*} [h(X,\cdot)]\right\|_\infty\right] \\
        &\leq \frac{5CR}{4}\sqrt{\frac{d}{MN} + \frac{dB}{\lambda N}}
        + CR\sqrt{\frac{d\lambda}{MN(\lambda + MB)}}\log\left( C'\sqrt{\left(\lambda+MB\right)\frac{\pi}{\lambda}} \right) \\
        &+ CR\sqrt{\frac{d\lambda M N}{\lambda + MB}} \left(\frac{ \delta_{\eta}}{\alpha\lambda^2} 
        + \frac{1}{\lambda}\exp( -\alpha\lambda\eta k )\pow[\Delta,N]_0 \right).
    \end{align*}
\end{proof}

%\begin{proof}[New proof]
%    Let $\vX_j=(X_j^1,\ldots,X_j^N) \sim \pow[\mu,N]~(j=1,2,\ldots,M)$ be $M$ independent random variables.
%    We set
%    \begin{align*}  
%        &g_N(z) = \frac{1}{MN}\sum_{j=1}^M\sum_{i=1}^N h(X_{j}^i,z) - \bE_{X \sim \pow[\mu,N]_1}[h(X,z)], \\
%        &S_N(z) = |g_N(z)| - \bE_{\{\vX_j\}_{j=1}^M}[ | g_N(z) | ].
%    \end{align*}
%    We define the metric $d$ defined in $\cZ$ as follows: for any $z,z'\in\cZ$,
%    \begin{align*}
%        d(z,z') = \sqrt{ \bE_{\{\vX_j\}_{j=1}^M}\left[ (S_N(z) - S_N(z'))^2 \right] }.
%    \end{align*}


%    Given $\{\vX_j\}_{j=1}^M$, we set 
%    \[ S_{M,N}(z) = \frac{1}{M}\sum_{j=1}^M \bE_{X\sim \rho_{\vx_{j}}}[h(X,z)] - \bE_{X\sim \mu_*} [h(X,z)],~~~~(z \in \cZ). \]

%    We endow the data space $\cZ$ with the metric $d(z,z') = |S_{M,N}(z) - S_{M,N}(z')|$ ($\forall z,z' \in \cZ$). We denote by $\cN(\cZ,\epsilon,d)$ the $\epsilon$-covering number of $\cZ$ with respect to $d$. We also denote by $\cZ_\epsilon \subset \cZ$ a finite subset that provides $\epsilon$-covering of $\cF$ corresponding to $\cN(\cZ,\epsilon,d)$. For any $z \in \cZ$, there exists $\pi_\epsilon(z) \in \cZ_\epsilon$ such that $d(z,z')\leq \epsilon$.

%    Noting that for any $z,z'\in \cZ$,
%    \begin{align*}
%        d(z,z')
%        &= \left| \frac{1}{MN}\sum_{j=1}^M\sum_{i=1}^N \left( h(X_j^i,z) - h(X_j^i,z') \right)
%        + \bE_{X\sim\mu_*}[h(X,z)-h(X,z')] \right| \\
%        &\leq \frac{1}{MN}\sum_{j=1}^M\sum_{i=1}^N \left| h(X_j^i,z) - h(X_j^i,z') \right|
%        + \bE_{X\sim\mu_*}\left[ \left| h(X,z)-h(X,z') \right| \right] \\
%        &\leq \beta \left( \frac{1}{MN}\sum_{j=1}^M\sum_{i=1}^N\|X_j^i\|_2
%        + \bE_{X\sim\mu_*}[\|X\|_2]\right)\|z-z'\|_2 \\
%        &= \beta r \|z-z'\|_2,
%    \end{align*}
%    where we set $r = \frac{1}{MN}\sum_{j=1}^M\sum_{i=1}^N\|X_j^i\|_2$, we see $\cN(\cZ, \epsilon, d) \leq \cN\left(\cZ, \epsilon/(\beta r), \|\cdot\|_{2}\right) = \left( 1 + \frac{2 \beta r}{\epsilon}\right)^{d}$.
    
%    Let $\vX_j=(X_j^1,\ldots,X_j^N) \sim \pow[\mu,N]~(j=1,2,\ldots,M)$ be $M$ independent random variables. Then, we get
%    \begin{align*}
%        \bE_{\{\vX_j\}_{j=1}^M}\left[ 
%            \sup_{z\in\cZ} S_{M,N}^2(z)
%        \right]
%        &= \bE_{\{\vX_j\}_{j=1}^M}\left[ 
%            \sup_{z\in\cZ} (S_{M,N}(z) - S_{M,N}(\pi(z)) + S_{M,N}(\pi(z)))^2
%        \right] \\
%        &\leq 2\bE_{\{\vX_j\}_{j=1}^M}\left[ \sup_{z\in\cZ} \left(S_{M,N}(z) - S_{M,N}(\pi(z))\right)^2 
%            + \sup_{z\in\cZ} S_{M,N}^2(\pi(z))
%        \right]\\
%       &\leq 2\bE_{\{\vX_j\}_{j=1}^M}\left[ \epsilon^2 
%           + \sup_{z\in\cZ} S_{M,N}^2(\pi(z))
%        \right]\\
%        &\leq 2\epsilon^2 + 2\bE_{\{\vX_j\}_{j=1}^M}\left[ 
%            \max_{z\in\cZ_{\epsilon}} S_{M,N}^2(z)
%        \right]
%    \end{align*}
    
%    Because of the boundedness $\sup_{x \in \bR^d, z\in\cZ}|h(x,z)| \leq R$, $S^2_{M,N}(z) - \bE[S^2_{M,N}(z)]$ is a sub-Gaussian with variance factor $4R^4$. Therefore, by the maximal inequality for sub-Gaussian distribution (e.g., see \citet{boucheron2013concentration}), we get
%    \begin{align*}
%        &\max_{z\in\cZ_{\epsilon}} \left\{ S_{M,N}^2(z) - \bE[S^2_{M,N}(z)]\right\}
%        \leq 2\sqrt{2}R^2 \log \cN(\cZ,\epsilon,d)
%        \leq 2\sqrt{2}R^2 d \log \left( 1 + \frac{2\beta r}{\epsilon}\right) \\
%        \iff &\max_{z\in\cZ_{\epsilon}} S_{M,N}^2(z)
%        \leq \max_{z\in\cZ_{\epsilon}} \bE[S^2_{M,N}(z)]
%        + 2\sqrt{2}R^2 d \log \left( 1 + \frac{2\beta r}{\epsilon}\right).
%    \end{align*}
%\end{proof}

\section{Computational Resources}\label{appx:compute}
We implemented \ourmethod using PyTorch~\citep{paszke2019pytorch} (offered under BSD-3 Clause license) and the PyTorch Geometric library~\citep{fey2019fast} (offered under MIT license) for efficient processing of graph-structured data. All experiments were conducted on NVIDIA RTX A5000,  NVIDIA RTX A6000, and NVIDIA GeForce RTX 4090 GPUs, and on the Google's Gemini API. For hyperparameter tuning and model selection, we used the Weights and Biases (wandb) library~\citep{wandb}.

\section{LLM Experiment Details}
\label{appx:LLMexp}
%cite the paper / exactly same thing in the code / different orders of relations / comma and quotes / example of failure (different lanuages, 20 failures of each)

\subsection{Details for Task 1} \label{appx:T1}
\paragraph{Relation Prediction Prompt of Task 1} 
In the following task, you will be given background knowledge in the form of triplet (h, r, t) which means entity 'h' has relation 'r' with entity 't'. Then you will be asked some questions about the relationship between entities. Background knowledge: (Kris Kristofferson, occupation, guitarist); (Willow Smith, genre, indie pop);\dots What is the relationship between entity 'Gaspard Monge' and entity 'France'? Please choose one best answer from the following relations:|parent organization|studies|cause of death|architectural style|unmarried partner|industry|\dots|. You just need to give the relation and please do not give an explanation.

\paragraph{Entity Prediction Prompt of Task 1} 
In the following task, you will be given background knowledge in the form of triplet (h, r, t) which means entity 'h' has relation 'r' with entity 't'. Then you will be asked some questions about the relationship between entities. Background knowledge: (Kris Kristofferson, occupation, guitarist); (Willow Smith, genre, indie pop);\dots Predict the tail entity for triplet (Gaspard Monge, country of citizenship, ?). Please give the 10 most possible answers. You just need to give the names of the entities separated by commas and please do not give explanation.

\revisionlog{In the prompt for relation prediction for all the three tasks, all relations in the dataset are listed with '|' as the delimiter. For the sake of simplicity in the presentation, in the next subsections we use "\dots" to represent the rest of triplets and relations in the prompts.}

\Cref{llm-base} shows the Hits@1 of Gemini-1.5-pro on in-domain relation prediction task. \revision{\Cref{llm-base-entity} shows the Hits@10 of Gemini-1.5-flash on in-domain entity prediction task. We use Gemini-1.5-flash for entity prediction tasks for the sake of reducing costs of experiments.} We run the experiment 3 times with the same prompt in English to see if it can generate consistent answers. Gemini performs the task quite well and shows consistency across 3 runs. This indicates given background knowledge in the prompt, LLMs has the capacity to handle in-domain relation and entity predictions well. 

% We further tests its performance across different languages. \Cref{tab:llm-relation2} shows that the LLM has consistent good performance in different languages.

\begin{table*}[ht]
\centering
\caption{Task 1: In-domain LLM relation predictions Hits@1 on CoDEx-S.}
\begin{tabular}{lrrr |r}
\toprule
 & Run \#1  & Run \#2 & Run \#3  & Worst \\
Gemini-1.5-pro & 0.933 & 0.933 & 0.933 & 0.933\\
%gpt-4-0125-preview & N/A* & N/A* & N/A* & N/A* \\
ULTRA & 0.820 & 0.820 & 0.820 & 0.820 \\
\ourmethod & {\bf 0.935} & {\bf 0.935} & {\bf 0.935} & {\bf 0.935} \\
\hline
\end{tabular}
\label{llm-base}
\end{table*}

\begin{table*}[ht]
\centering
\caption{Task 1: In-domain LLM entity predictions Hits@10 on CoDEx-S.}
\begin{tabular}{lrrr |r}
\toprule
 & Run \#1  & Run \#2 & Run \#3  & Worst \\
Gemini-1.5-flash & 0.308 & 0.308 & 0.308 & 0.308\\
%gpt-4-0125-preview & N/A* & N/A* & N/A* & N/A* \\
ULTRA & 0.667 & 0.667 & 0.667 & 0.667 \\
\ourmethod & {\bf 0.670} & {\bf 0.670} & {\bf 0.670} & {\bf 0.670} \\
\hline
\end{tabular}
\label{llm-base-entity}
\end{table*}


% \begin{table*}[ht]
% \centering
% \caption{Task 1 on different languages: In-domain relation predictions of Gemini-1.5-pro on CoDEx-S in different languages with Prompt Template 1.N/A* means we were not able to perform the experiment due to API limitations on context size.}
% \begin{tabular}{l rrrrrr rr}
% \toprule
% & \multicolumn{6}{c}{Gemini-1.5-pro} & \multicolumn{2}{c}{K-GFMs}  \\
% \cmidrule(lr){2-7}
% \cmidrule(lr){8-9}
%  & \multicolumn{1}{c}{ar} & \multicolumn{1}{c}{de} & \multicolumn{1}{c}{en} & \multicolumn{1}{c}{es} & \multicolumn{1}{c}{ru} & \multicolumn{1}{c}{zh} & \multicolumn{1}{c}{ULTRA} & \multicolumn{1}{c}{\ourmethod} \\
% \midrule
% CoDEx-S & 0.903 & 0.935 & 0.933 & 0.933 & 0.903 & 0.933 & 0.820 & \textbf{0.935} \\
% CoDEx-M & N/A* & N/A* & N/A* & N/A* & N/A* & N/A* & 0.870 & \textbf{0.886} \\
% CoDEx-L & N/A* & N/A* & N/A* & N/A* & N/A* & N/A* & 0.824 & \textbf{0.837} \\
% \bottomrule
% \end{tabular}
% \label{tab:llm-relation2}
% \end{table*}


\subsection{Details for Task 2} \label{appx:T2}
\paragraph{Relation Prediction Prompt of Task 2} In the following task, you will first be given background knowledge in the form of triplet (h, r, t) which means entity 'h' has relation 'r' with entity 't'. Then you will be asked some questions about the relationship between entities. Please notice that some words are replaced with metasyntactic words in the following paragraph. Background knowledge: (foo, baz, guitarist); (Willow Smith, genre, bar);\dots What is the relationship between entity 'foo' and entity 'bar'? Please choose one best answer from the following relation IDs:|parent organization|studies|quux|baz|\dots|. You just need to give the relation and please do not give an explanation.

\paragraph{Entity Prediction Prompt of Task 2} In the following task, you will first be given background knowledge in the form of triplet (h, r, t) which means entity 'h' has relation 'r' with entity 't'. Then you will be asked some questions about the relationship between entities. Please notice that some words are replaced with metasyntactic words in the following paragraph. Background knowledge: (foo, baz, guitarist); (Willow Smith, genre, bar);\dots Predict the tail entity for triplet (foo, garply, ?). Please give the 10 most possible answers. You just need to give the names of the entities separated by commas and please do not give explanation.

\Cref{llm-metasyntactic} shows the Hits@1 of Gemini-1.5-pro on out-domain relation prediction task. \revision{\Cref{llm-metasyntactic-entity} shows the Hits@10 of Gemini-1.5-flash on out-domain entity prediction task.} The neighbor entities of the head entity and the relations that connect the head entity with its neighbors are replaced with metasyntactic words. We run the experiment 3 times with different metasyntactic words but the underlying structural pattern is exactly the same as in the in-domain task. The results indicate the LLM can not do the out-of-domain task well. This means the LLM relied more on the known semantic description of the words instead of the structural pattern of the graph so that when there are new entities and relations, it can not perform inductive reasoning on them.

\begin{table*}[t]
\centering
\caption{Task 2: Out-of-domain LLM relation predictions Hits@1 on CoDEx-S. Effects of Metasyntactic Words on Relation Predictions on CoDEx-S.\label{llm-metasyntactic}
}
\begin{tabular}{lrrr |r}
\toprule
 & Run \#1  & Run \#2 & Run \#3  & Worst \\
 \midrule
Gemini-1.5-pro & 0.667 & 0.667 & 0.633 & 0.633\\
ULTRA & 0.820 & 0.820 & 0.820 & 0.820 \\
\ourmethod & {\bf 0.935} & {\bf 0.935} & {\bf 0.935} & {\bf 0.935} \\
%gpt-4-0125-preview & N/A* & N/A* & N/A* & N/A* \\
\hline
\end{tabular}
\end{table*}

\begin{table*}[ht]
\centering
\caption{Task 2: Out-of-domain LLM entity predictions Hits@10 on CoDEx-S. Effects of Metasyntactic Words on Entity Predictions on CoDEx-S.\label{llm-metasyntactic-entity}
}
\begin{tabular}{lrrr |r}
\toprule
 & Run \#1  & Run \#2 & Run \#3  & Worst \\
 \midrule
Gemini-1.5-flash & 0.212  & 0.250 & 0.327 & 0.212\\
ULTRA & 0.667 & 0.667 & 0.667 & 0.667 \\
\ourmethod & {\bf 0.670} & {\bf 0.670} & {\bf 0.670} & {\bf 0.670} \\
%gpt-4-0125-preview & N/A* & N/A* & N/A* & N/A* \\
\hline
\end{tabular}
\end{table*}

\subsection{Details for Task 3}\label{appx:T3}
\paragraph{Relation Prediction Prompt of Task 3} In the following task, entities and relations will be expressed with their IDs. You will first be given the mapping from entities to their IDs and the mapping from relations to their IDs. Then you will be given background knowledge in the form of triplet (h, r, t) which means entity 'h' has relation 'r' with entity 't'. Finally you will be asked some questions about the relationship between entities. Entity mapping: Mireille Darc is entity '8831'; Breton is entity '20512'; Tomas Tranströmer is entity '1641'\dots Relation mapping: located in the administrative terroritorial entity is relation '15' \dots Background knowledge: (15443, 16, 1093); (21198, 16, 9387); (14854, 8, 10218)\dots What is the relationship between entity '18127' and entity '1799'? Please choose one best answer from the following relation IDs:|45|48|27|35|\dots|. You just need to give the ID of that relation and please do not give an explanation.

\paragraph{Entity Prediction Prompt of Task 3} In the following task, entities and relations will be expressed with their IDs. You will first be given the mapping from entities to their IDs and the mapping from relations to their IDs. Then you will be given background knowledge in the form of triplet (h, r, t) which means entity 'h' has relation 'r' with entity 't'. Finally you will be asked some questions about the relationship between entities. Entity mapping: Mireille Darc is entity '8831'; Breton is entity '20512'; Tomas Tranströmer is entity '1641'\dots Relation mapping: located in the administrative terroritorial entity is relation '15' \dots Background knowledge: (15443, 16, 1093); (21198, 16, 9387); (14854, 8, 10218)\dots Predict the tail entity for triplet (18127, 45, ?). Please give the 10 most possible answers. You just need to give the IDs of the entities separated by commas and please do not give explanation.


\Cref{llm-equivarariance} shows the Hits@1 of Gemini-1.5-pro on relation prediction in Task 3. \revision{\Cref{llm-equivarariance-entity} shows the Hits@10 of Gemini-1.5-flash on entity prediction in Task 3.} The entities and relations are expressed as IDs. We run the experiment 3 times with permutated IDs but the underlying structural pattern is exactly the same as in the in-domain task. The results demonstrate that the LLM is very sensitive to ID permutation so that its performance is inconsistent across 3 permutations.

\begin{table*}[t]
\centering
\caption{Task 3: Out-of-domain LLM relation predictions Hits@1 on CoDEx-S. Effects of Input Permutations on Relation Predictions on CoDEx-S.}
\begin{tabular}{lrrr |r}
\toprule
 & Permutation \#1  & Permutation \#2 & Permutation \#3  & Worst \\
 \midrule
Gemini-1.5-pro & 0.346 & 0.731 & 0.615 & 0.346\\
%gpt-4-0125-preview & N/A* & N/A* & N/A* & N/A* \\
ULTRA & 0.820 & 0.820 & 0.820 & 0.820 \\
\ourmethod & {\bf 0.935} & {\bf 0.935} & {\bf 0.935} & {\bf 0.935} \\
\hline
\end{tabular}
\label{llm-equivarariance}
\end{table*}

\begin{table*}[t]
\centering
\caption{Task 3: Out-of-domain LLM entity predictions Hits@10 on CoDEx-S. Effects of Input Permutations on Entity Predictions on CoDEx-S.}
\begin{tabular}{lrrr |r}
\toprule
 & Permutation \#1  & Permutation \#2 & Permutation \#3  & Worst \\
 \midrule
Gemini-1.5-flash & 0.212 & 0.250 & 0.231 & 0.212\\
%gpt-4-0125-preview & N/A* & N/A* & N/A* & N/A* \\
ULTRA & 0.667 & 0.667 & 0.667 & 0.667 \\
\ourmethod & {\bf 0.670} & {\bf 0.670} & {\bf 0.670} & {\bf 0.670} \\
\hline
\end{tabular}
\label{llm-equivarariance-entity}
\end{table*}

% \begin{table*}[ht]
% \centering
% \caption{Task 3: Out-of-domain LLM entity predictions Hits@10 on CoDEx-S. Effects of Input Permutations on Entity Predictions on CoDEx-S.}
% \begin{tabular}{lrrr |r}
% \toprule
%  & Permutation \#1  & Permutation \#2 & Permutation \#3  & Worst \\
%  \midrule
% Gemini-1.5-pro & 0.231 & 0.250 & 0.346 & 0.231\\
% %gpt-4-0125-preview & N/A* & N/A* & N/A* & N/A* \\
% ULTRA & 0.667 & 0.667 & 0.667 & 0.667 \\
% \ourmethod & {\bf 0.670} & {\bf 0.670} & {\bf 0.670} & {\bf 0.670} \\
% \hline
% \end{tabular}
% \label{llm-equivarariance-entity}
% \end{table*}

% \subsection{Capacity of LLMs in different languages}
% In this part we want to evaluate whether LLMs can make consistent queries in different languages. Due to token size limit of ChatGPT, we use prompt template 4 which directly queries the LLM without providing context from the knowledge graph~\cite{yao2023exploring}. While for Gemini, we use prompt template 1.

% \paragraph{Prompt Template 4 in English} What is the relationship between \{head entity\} and \{tail entity\}? Please choose one best answer from:|parent organization|studies|cause of death|\dots|genre|

% \begin{CJK}{UTF8}{gbsn}
% \paragraph{Prompt Template 4 in Chinese} \{头节点\}和\{尾节点\}之间的关系是什么?请从以下选项中选择一个最佳答案:|母组织|研究对象|死因|\dots|艺术流派|
% \end{CJK}


% \Cref{tab:llm-relation} shows the hits@1 results comparing gpt-4-0125-preview with the double equivariant graph models. Notably, when the knowledge graph is not used as part of the prompt, ChatGPT exhibits inferior performance compared to graph models, showcasing significant variability across different languages. Specifically, it achieves the highest hits@1 scores when queries are posed in Spanish, while its performance is notably poorer when queries are conducted in Russian. On the contrary, Gemini-1.5-pro performs quite well in different languages when the whole KG is provided in the prompt as shown in \Cref{tab:llm-relation2}.

% % this table to appendix
% \begin{table*}[ht]
% \centering
% \caption{Task 4: Relation Prediction hits@1 of ChatGPT(gpt-4-0125-preview) with prompt template 4 in different languages and double equivariant graph models.}
% \begin{tabular}{l rrrrrr rr}
% \toprule
% & \multicolumn{6}{c}{ChatGPT(gpt-4-0125-preview)} & \multicolumn{2}{c}{K-GFMs}  \\
% \cmidrule(lr){2-7}
% \cmidrule(lr){8-9}
%  & \multicolumn{1}{c}{ar} & \multicolumn{1}{c}{de} & \multicolumn{1}{c}{en} & \multicolumn{1}{c}{es} & \multicolumn{1}{c}{ru} & \multicolumn{1}{c}{zh} & \multicolumn{1}{c}{ULTRA} & \multicolumn{1}{c}{\ourmethod} \\
% \midrule
% CoDEx-S & 0.455 & 0.550 & 0.610 & 0.690 & 0.333 & 0.393 & 0.820 & \textbf{0.935} \\
% CoDEx-M & 0.452 & 0.505 & 0.620 & 0.667 & 0.381 & 0.425 & 0.870 & \textbf{0.886} \\
% CoDEx-L & 0.490 & 0.558 & 0.590 & 0.698 & 0.400 & 0.409 & 0.824 & \textbf{0.837} \\
% \bottomrule
% \end{tabular}
% \label{tab:llm-relation}
% \end{table*}

% \paragraph{Example of outputs of ChatGPT} In this part we will show some examples of answers given by ChatGPT (gpt-4-0125-preview). For the sake of saving space, we omit the options of relations in the questions since they are the same as in the template.

% Here are questions that are answered wrongly in both English and Chinese:

% \begin{enumerate}
% \item
% \begin{enumerate}
% \item 
% \textbf{Question in English: }What is the relationship between Charlie Wilson and African Americans? 

% \textbf{The true answer in English: } ethnic group

% \textbf{ChatGPT's answer in English: } member of political party

% \item 
% \begin{CJK}{UTF8}{gbsn}
% \textbf{Question in Chinese: }查理·威尔森 (歌手)和非裔美国人之间的关系是什么?

% \textbf{The true answer in Chinese: }族群

% \textbf{ChatGPT's answer in Chinese: }查理·威尔森(Charlie Wilson)是一名歌手,因此与上述选项中的"职业"最为相关。他作为一名艺术家,通过音乐与人们建立联系,而非裔美国人作为他的身份之一,对他的音乐风格、歌词内容以及他在音乐界的定位有着深刻的影响。因此,如果要从提供的选项中选择一个最佳答案来描述查理·威尔森和非裔美国人之间的关系,那么"职业"是最合适不过的了。
% \end{CJK}

% \end{enumerate}

% \end{enumerate}

% Here are questions that are answered correctly in English but wrongly in Chinese:
% \begin{enumerate}
% \item 
% \begin{enumerate}
%     \item 
%     \textbf{Question in English: }What is the relationship between J. J. Abrams and producer? 
    
%     \textbf{The true answer in English: } occupation
    
%     \textbf{ChatGPT's answer in English: } occupation
%     \item 
%     \begin{CJK}{UTF8}{gbsn}
%     \textbf{Question in Chinese: }杰弗里·雅各布·艾布拉姆斯和制作人之间的关系是什么?
    
%     \textbf{The true answer in Chinese: } 职业
    
%     \textbf{ChatGPT's answer in Chinese: }杰弗里·雅各布·艾布拉姆斯(J.J. Abrams)和制作人之间的关系最佳答案是:|导演|
%     \end{CJK}
% \end{enumerate}

% \item 
% \begin{enumerate}
%     \item 
%     \textbf{Question in English: }What is the relationship between Victor Ambartsumian and German Academy of Sciences Leopoldina?
    
%     \textbf{The true answer in English: } member of
    
%     \textbf{ChatGPT's answer in English: } member of
%     \item 
%     \begin{CJK}{UTF8}{gbsn}
%     \textbf{Question in Chinese: }维克托·安巴楚勉和利奥波第那科学院之间的关系是什么?
    
%     \textbf{The true answer in Chinese: } 所属组织
    
%     \textbf{ChatGPT's answer in Chinese: }维克托·安巴楚勉和利奥波第那科学院之间的关系最合适的选择是|创作者|。
%     \end{CJK}
% \end{enumerate}

% \item 
% \begin{enumerate}
%     \item 
%     \textbf{Question in English: }What is the relationship between Theodor W. Adorno and composer?
    
%     \textbf{The true answer in English: } occupation
    
%     \textbf{ChatGPT's answer in English: } occupation
%     \item 
%     \begin{CJK}{UTF8}{gbsn}
%     \textbf{Question in Chinese: }狄奧多·阿多诺和作曲家之间的关系是什么?
    
%     \textbf{The true answer in Chinese: } 职业
    
%     \textbf{ChatGPT's answer in Chinese: }狄奧多·阿多诺和作曲家之间的关系最佳选项是:**研究对象**。阿多诺是一位德国哲学家、社会学家、音乐学家和作曲家,对20世纪音乐,特别是十二音音乐和其背后的哲学和社会基础有深入研究,因此作曲家是他研究的主要对象之一。
%     \end{CJK}
% \end{enumerate}

% \item 
% \begin{enumerate}
%     \item 
%     \textbf{Question in English: }What is the relationship between Sri Lanka and International Centre for Settlement of Investment Disputes?
    
%     \textbf{The true answer in English: } member of
    
%     \textbf{ChatGPT's answer in English: } member of
%     \item 
%     \begin{CJK}{UTF8}{gbsn}
%     \textbf{Question in Chinese: }斯里兰卡和国际投资争端解决中心之间的关系是什么?
    
%     \textbf{The true answer in Chinese: } 所属组织
    
%     \textbf{ChatGPT's answer in Chinese: }斯里兰卡和国际投资争端解决中心之间的关系最恰当的选项是“法律形式”。这是因为国际投资争端解决中心(ICSID)主要处理投资者和国家间的争端解决,是根据法律协议和程序来操作的。当投资者认为他们的投资受到了违反国际投资协定的待遇时,他们可以选择通过ICSID对相关国家(比如斯里兰卡)提起诉讼。因此,在斯里兰卡与国际投资争端解决中心之间的关系中,法律形式是关键的组成部分。
%     \end{CJK}
% \end{enumerate}
% \end{enumerate}

\section{Datasets}
\label{appx:datasets}
The statistics of all 57 datasets used in the experiments in presented in Tables~\ref{tab:app_datasets_transd},\ref{tab:app_datasets_inde},\ref{tab:app_datasets_indr}. 
All datasets are publicly available under open licenses (MIT or CC-BY).

\begin{table*}[!t]
\centering
\caption{Transductive datasets (16) used in the experiments. Train, Valid, Test denote triples in the respective set. Task denotes the prediction task: \emph{h/t} is predicting both heads and tails, \emph{tails} is only predicting tails.}
\label{tab:app_datasets_transd}
%\scriptsize
\begin{adjustbox}{width=\textwidth}
\begin{tabular}{llrrrrrl}\toprule
Dataset & Reference &Entities &Rels &Train &Valid &Test &Task \\\midrule
CoDEx Small & \cite{safavi2020codex} &2034 &42 &32888 &1827 &1828 & h/t \\
WDsinger & \cite{dackgr} &10282 &135 &16142 &2163 &2203 & h/t  \\
FB15k237\_10 & \cite{dackgr} &11512 &237 &27211 &15624 &18150 & tails \\
FB15k237\_20 & \cite{dackgr} &13166 &237 &54423 &16963 &19776 & tails \\
FB15k237\_50 & \cite{dackgr} &14149 &237 &136057 &17449 &20324 & tails \\
FB15k237 & \cite{fb15k237} &14541 &237 &272115 &17535 &20466 & h/t \\
CoDEx Medium & \cite{safavi2020codex} &17050 &51 &185584 &10310 &10311 & h/t  \\
NELL23k & \cite{dackgr} &22925 &200 &25445 &4961 &4952 & h/t  \\
WN18RR & \cite{wn18rr} &40943 &11 &86835 &3034 &3134 & h/t \\
AristoV4 & \cite{ssl_rp} &44949 &1605 &242567 &20000 &20000 & h/t  \\
Hetionet & \cite{hetionet} &45158 &24 &2025177 &112510 &112510 & h/t  \\
NELL995 & \cite{nell995} &74536 &200 &149678 &543 &2818 & h/t \\
CoDEx Large & \cite{safavi2020codex} &77951 &69 &551193 &30622 &30622 & h/t \\
ConceptNet100k & \cite{cnet100k} &78334 &34 &100000 &1200 &1200 & h/t  \\
DBpedia100k & \cite{dbp100k} &99604 &470 &597572 &50000 &50000 & h/t  \\
YAGO310 & \cite{yago310} &123182 &37 &1079040 &5000 &5000 & h/t \\
\bottomrule
\end{tabular}
\end{adjustbox}
\end{table*}

\begin{table*}[!t]
\caption{Inductive entity $(e)$ datasets (18) used in the experiments. Triples denote the number of edges of the graph given at training, validation, or test. Valid and Test denote triples to be predicted in the validation and test sets in the respective validation and test graph.}
\label{tab:app_datasets_inde}
\begin{adjustbox}{width=\textwidth}
\begin{tabular}{lrrrrrrrrrr}\toprule
\multirow{2}{*}{Dataset} &\multirow{2}{*}{Rels} &\multicolumn{2}{c}{Training Graph} &\multicolumn{3}{c}{Validation Graph} &\multicolumn{3}{c}{Test Graph}  \\ \cmidrule(l){3-4} \cmidrule(l){5-7} \cmidrule(l){8-10} 
& &Entities &Triples &Entities &Triples &Valid  &Entities &Triples &Test  \\\midrule
FB v1~\cite{teru2020inductive} &180 &1594 &4245 &1594 &4245 &489 &1093 &1993 &411  \\
FB v2~\cite{teru2020inductive} &200 &2608 &9739 &2608 &9739 &1166 &1660 &4145 &947  \\
FB v3~\cite{teru2020inductive} &215 &3668 &17986 &3668 &17986 &2194 &2501 &7406 &1731  \\
FB v4~\cite{teru2020inductive} &219 &4707 &27203 &4707 &27203 &3352 &3051 &11714 &2840  \\
WN v1~\cite{teru2020inductive} &9 &2746 &5410 &2746 &5410 &630 &922 &1618 &373 \\
WN v2~\cite{teru2020inductive} &10 &6954 &15262 &6954 &15262 &1838 &2757 &4011 &852  \\
WN v3~\cite{teru2020inductive} &11 &12078 &25901 &12078 &25901 &3097 &5084 &6327 &1143 \\
WN v4~\cite{teru2020inductive} &9 &3861 &7940 &3861 &7940 &934 &7084 &12334 &2823  \\
NELL v1~\cite{teru2020inductive} &14 &3103 &4687 &3103 &4687 &414 &225 &833 &201 \\
NELL v2~\cite{teru2020inductive} &88 &2564 &8219 &2564 &8219 &922 &2086 &4586 &935 \\
NELL v3~\cite{teru2020inductive} &142 &4647 &16393 &4647 &16393 &1851 &3566 &8048 &1620 \\
NELL v4~\cite{teru2020inductive} &76 &2092 &7546 &2092 &7546 &876 &2795 &7073 &1447 \\
ILPC Small~\cite{ilpc} &48 &10230 &78616 &6653 &20960 &2908 &6653 &20960 &2902 \\
ILPC Large~\cite{ilpc} &65 &46626 &202446 &29246 &77044 &10179 &29246 &77044 &10184  \\
HM 1k~\cite{ham_bm} &11 &36237 &93364 &36311 &93364 &1771 &9899 &18638 &476 \\
HM 3k~\cite{ham_bm} &11 &32118 &71097 &32250 &71097 &1201 &19218 &38285 &1349 \\
HM 5k~\cite{ham_bm} &11 &28601 &57601 &28744 &57601 &900 &23792 &48425 &2124\\
IndigoBM~\cite{liu2021indigo} &229 &12721 &121601 &12797 &121601 &14121 &14775 &250195 &14904 \\
\bottomrule
\end{tabular}
\end{adjustbox}
\end{table*}

\begin{table*}[!t]
\caption{Inductive entity and relation $(e,r)$ datasets (23) used in the experiments. Triples denote the number of edges of the graph given at training, validation, or test. Valid and Test denote triples to be predicted in the validation and test sets in the respective validation and test graph.}
\label{tab:app_datasets_indr}
%\scriptsize
\begin{adjustbox}{width=\textwidth}
\begin{tabular}{lrrrrrrrrrrrrrr}\toprule
\multirow{2}{*}{Dataset} &\multicolumn{3}{c}{Training Graph} &\multicolumn{4}{c}{Validation Graph} &\multicolumn{4}{c}{Test Graph} \\ \cmidrule(l){2-4} \cmidrule(l){5-8} \cmidrule(l){9-12}
&Entities &Rels &Triples &Entities &Rels &Triples &Valid &Entities &Rels &Triples &Test \\\midrule
FB-25~\cite{lee2023ingram} &5190 &163 &91571 &4097 &216 &17147 &5716 &4097 &216 &17147 &5716  \\
FB-50~\cite{lee2023ingram} &5190 &153 &85375 &4445 &205 &11636 &3879 &4445 &205 &11636 &3879  \\
FB-75~\cite{lee2023ingram} &4659 &134 &62809 &2792 &186 &9316 &3106 &2792 &186 &9316 &3106  \\
FB-100~\cite{lee2023ingram} &4659 &134 &62809 &2624 &77 &6987 &2329 &2624 &77 &6987 &2329  \\
WK-25~\cite{lee2023ingram} &12659 &47 &41873 &3228 &74 &3391 &1130 &3228 &74 &3391 &1131  \\
WK-50~\cite{lee2023ingram} &12022 &72 &82481 &9328 &93 &9672 &3224 &9328 &93 &9672 &3225  \\
WK-75~\cite{lee2023ingram} &6853 &52 &28741 &2722 &65 &3430 &1143 &2722 &65 &3430 &1144  \\
WK-100~\cite{lee2023ingram} &9784 &67 &49875 &12136 &37 &13487 &4496 &12136 &37 &13487 &4496  \\
NL-0~\cite{lee2023ingram} &1814 &134 &7796 &2026 &112 &2287 &763 &2026 &112 &2287 &763  \\
NL-25~\cite{lee2023ingram} &4396 &106 &17578 &2146 &120 &2230 &743 &2146 &120 &2230 &744  \\
NL-50~\cite{lee2023ingram} &4396 &106 &17578 &2335 &119 &2576 &859 &2335 &119 &2576 &859  \\
NL-75~\cite{lee2023ingram} &2607 &96 &11058 &1578 &116 &1818 &606 &1578 &116 &1818 &607  \\
NL-100~\cite{lee2023ingram} &1258 &55 &7832 &1709 &53 &2378 &793 &1709 &53 &2378 &793  \\
\midrule
Metafam~\cite{mtdea} &1316 &28 &13821 &1316 &28 &13821 &590 &656 &28 &7257 &184 \\
FBNELL~\cite{mtdea} &4636 &100 &10275 &4636 &100 &10275 &1055 &4752 &183 &10685 &597  \\
Wiki MT1 tax~\cite{mtdea} &10000 &10 &17178 &10000 &10 &17178 &1908 &10000 &9 &16526 &1834  \\
Wiki MT1 health~\cite{mtdea} &10000 &7 &14371 &10000 &7 &14371 &1596 &10000 &7 &14110 &1566  \\
Wiki MT2 org~\cite{mtdea} &10000 &10 &23233 &10000 &10 &23233 &2581 &10000 &11 &21976 &2441  \\
Wiki MT2 sci~\cite{mtdea} &10000 &16 &16471 &10000 &16 &16471 &1830 &10000 &16 &14852 &1650  \\
Wiki MT3 art~\cite{mtdea} &10000 &45 &27262 &10000 &45 &27262 &3026 &10000 &45 &28023 &3113  \\
Wiki MT3 infra~\cite{mtdea} &10000 &24 &21990 &10000 &24 &21990 &2443 &10000 &27 &21646 &2405  \\
Wiki MT4 sci~\cite{mtdea} &10000 &42 &12576 &10000 &42 &12576 &1397 &10000 &42 &12516 &1388  \\
Wiki MT4 health~\cite{mtdea} &10000 &21 &15539 &10000 &21 &15539 &1725 &10000 &20 &15337 &1703  \\
\bottomrule
\end{tabular}
\end{adjustbox}
\end{table*}

\section{Detailed Experiment Results of Entity and Relation Prediction}
\label{appx:detailedresults}
\newpage
\section{Omitted Details in \pref{sec: experiment}}\label{app: experiment}
For completeness, we include the pseudo code for the benchmark used in our experiment, that is, \texttt{LinUCB} using only the observed feedback;
see \pref{alg:linUCB}.

\begin{algorithm}[h]
\caption{LinUCB with Delayed Feedback}\label{alg:linUCB}
Input: action set $\calA$, a parameter $\lambda>0$.

Initialize: $\wh{\theta}_1$ arbitrarily, $\beta_t = \sqrt{\lambda} + \sqrt{2\log T+n\log(1+\frac{t}{n\lambda})}$ for all $t\in[T]$, $H_1 = \lambda I$.

\For{$t=1,2,\dots,T$}{
     Pick 
     \begin{align*}
         a_t=
         \begin{cases}
            \argmin_{a\in\calA} \inner{a,\wh{\theta}_t} - \beta \|a\|_{H_t}^{-1}, &\mbox{in the loss case,} \\
            \argmax_{a\in\calA} \inner{a,\wh{\theta}_t} + \beta \|a\|_{H_t}^{-1}, &\mbox{in the reward case.}
        \end{cases}
     \end{align*}
     
     Observe the payoff $u_\tau$ for all $\tau$ such that $\tau+d_{\tau}\in (t-1,t]$.

     Update $H_{t+1} = H_t + \sum_{\tau:\tau+d_{\tau}\in(t-1,t]}a_{\tau}a_{\tau}^\top$ and $\wh{\theta}_{t+1}=H_{t+1}^{-1}\sum_{\tau:\tau+d_{\tau}\leq t}a_{\tau}u_{\tau}$.
     
}
\end{algorithm}


\end{document}
