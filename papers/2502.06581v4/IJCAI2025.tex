\documentclass{article}
\pdfpagewidth=8.5in
\pdfpageheight=11in
\usepackage{ijcai25}
\usepackage{times}
\usepackage{soul}
\usepackage{tikz}
\usetikzlibrary{shapes.geometric, arrows, positioning,fit,calc}

\usepackage[edges]{forest}
\definecolor{hidden-draw}{RGB}{0,0,0}
\definecolor{hidden-pink}{rgb}{0.98, 0.94, 0.75}
% \definecolor{level0}{rgb}{0.67, 0.88, 0.69}
\definecolor{level3}{RGB}{216,226,234}
\definecolor{level1}{rgb}{0.882, 0.835, 0.906}
\definecolor{level2}{rgb}{0.8, 0.8, 1.0}
\definecolor{level0}{rgb}{1.0, 0.71, 0.76}
\definecolor{level4}{rgb}{0.98, 0.92, 0.84}

%%% mycolor
\definecolor{myred}{RGB}{247,226,231}
\definecolor{myblue}{RGB}{216,226,234}
\definecolor{myyellow}{RGB}{252,238,221}
\definecolor{mypurple}{RGB}{233,229,241}

\usepackage{url}
\usepackage[hidelinks]{hyperref}
\usepackage[utf8]{inputenc}
\usepackage[small]{caption}
\usepackage{graphicx}
\usepackage{amsthm}
\usepackage{amsmath,amssymb,amsfonts}
\usepackage{booktabs}
\usepackage{subcaption}
\usepackage{algorithm}
\usepackage{algorithmic}
\urlstyle{same}
\usepackage{multirow}
\usepackage{tabularx}
\usepackage{enumitem}
\usepackage{bbm}
\usepackage{balance}
\usepackage[flushleft]{threeparttable}
\newtheorem{example}{Example}
\newtheorem{theorem}{Theorem}
\newtheorem{problem}{Problem}
\newtheorem{definition}{Definition}
\DeclareMathOperator*{\argmax}{arg\,max}
\DeclareMathOperator*{\argmin}{arg\,min}
\usepackage{color, colortbl}
\definecolor{Gray}{gray}{0.9}
\usepackage{setspace}
\usepackage{forest}
\usepackage{titlesec}
\titlespacing*{\paragraph}{0pt}{0.7ex}{0.8ex} % 调整左缩进、前间距和后间距,按需修改数值
\urlstyle{same}
\title{A Survey on Video Analytics in Cloud-Edge-Terminal Collaborative Systems}
\author{
% Jin Liu$^{1}$\and 
Linxiao Gong$^{1,3}$\and
% Yang Liu$^{1}$\and
Hao Yang$^1$\and
Gaoyun Fang$^4$\and
Bobo Ju$^1$\and
Juncen Guo$^1$\and\\
Xiaoguang Zhu$^5$\and
Xiping Hu$^6$\and
Yan Wang$^1$\and
Peng Sun$^2$\and
Azzedine Boukerche$^7$\\
\affiliations
$^1$Fudan University,\quad
$^2$Duke Kunshan University,\quad
$^3$University of Toronto,\quad\\
$^4$Imperial College London,\quad
$^5$University of California, Davis,\quad
$^6$SMBU,\quad
$^7$University of Ottawa\quad
% $^1$The University of Toronto, Toronto, Canada,\quad
% $^2$Fudan University, Shanghai, China, \\
% $^3$Imperial College London, London, UK,\quad
% $^4$University of California, Davis, USA,\\
% $^5$SMBU, Shenzhen, China, \quad
% $^6$Duke Kunshan University, Suzhou, China,\\
% $^7$University of Ottawa, Ottawa, Canada\\
\emails % todo check email
% linxiao.gong@mail.utoronto.ca, \{haoyang24, bbju21, yanwang19\}@fudan.edu.cn, p.fang23@imperial.ac.uk, xgzhu@ucdavis.edu, huxp@bit.edu.cn, peng.sun568@duke.edu, aboukerc@uottawa.ca
\{lxgong21, haoyang24, bbju21, guojc23, yanwang19\}@fudan.edu.cn, p.fang23@imperial.ac.uk, xgzhu@ucdavis.edu, huxp@smbu.edu.cn, peng.sun568@duke.edu, aboukerc@uottawa.ca
}
% \author{
%     % Author list to be finalized
%     Authors TBD
%     \affiliations
%      Institutions TBD
% }
\begin{document}
\maketitle
\begin{abstract}
The explosive growth of video data has driven the development of distributed video analytics in cloud-edge-terminal collaborative (CETC) systems, enabling efficient video processing, real-time inference, and privacy-preserving analysis. Among multiple advantages, CETC systems can distribute video processing tasks and enable adaptive analytics across cloud, edge, and terminal devices, leading to breakthroughs in video surveillance, autonomous driving, and smart cities. In this survey, we first analyze fundamental architectural components, including hierarchical, distributed, and hybrid frameworks, alongside edge computing platforms and resource management mechanisms. Building upon these foundations, edge-centric approaches emphasize on-device processing, edge-assisted offloading, and edge intelligence, while cloud-centric methods leverage powerful computational capabilities for complex video understanding and model training. Our investigation also covers hybrid video analytics incorporating adaptive task offloading and resource-aware scheduling techniques that optimize performance across the entire system. Beyond conventional approaches, recent advances in large language models and multimodal integration reveal both opportunities and challenges in platform scalability, data protection, and system reliability. Future directions also encompass explainable systems, efficient processing mechanisms, and advanced video analytics, offering valuable insights for researchers and practitioners in this dynamic field.
\end{abstract}
\section{Introduction}\label{sec1}
Widespread deployment of surveillance cameras, smartphones, and IoT devices has generated unprecedented volumes of video data, driving innovations across traffic monitoring, smart cities, and industrial automation~\cite{zhang2021serverless}. Video analytics has become increasingly critical, driven by the growing need for real-time insights across diverse scenarios~\cite{lu2024samedge}, from predictive maintenance systems~\cite{ji2024task} and autonomous vehicles demanding instantaneous decisions~\cite{zhu2024edge}, to advanced interactive applications like augmented and virtual reality platforms requiring responsive video analysis for compelling user experiences~\cite{zhou2024survey}. Consequently, cloud-edge-terminal collaborative (CETC) systems have emerged as a promising solution to manage this computational landscape by combining cloud servers with edge and terminal devices~\cite{gao2024edgevision}, while resource management strategies adapt to varying content complexity and processing demands~\cite{lu2023turbo}, as evidenced by the growing publication and citation trends, shown in Fig.~\ref{fig:pub_stats}.

Modern cloud-based video analytics, while leveraging substantial computational resources, encounters fundamental operational challenges in bandwidth utilization, response latency, and network reliability~\cite{gu2024aienhanced}. Specifically, large-scale video data transmission to centralized servers creates significant network bottlenecks, especially evident in deployments with multiple high-resolution camera streams. Such network congestion directly impacts real-time applications like traffic monitoring and autonomous driving systems, where processing delays can critically affect decision-making capabilities~\cite{ji2024task}. Moreover, the dependence on consistent cloud connectivity poses significant challenges for system deployment in areas with unstable or constrained network infrastructure~\cite{vo2022edge}, thereby affecting the overall system reliability and effectiveness.
\begin{figure}[t]
    \centering
    \includegraphics[width=0.48\textwidth]{fig1_video_cetc_stats.pdf}
    \vspace{-20px}
    \caption{Publication and citation statistics over the last decade.}
    \label{fig:pub_stats}
    \vspace{-15px}
\end{figure}

Although existing surveys have studied video analytics and edge-cloud computing, there is still a lack of systematic analysis of video analytics in CETC systems. Previous surveys either focus on specific aspects like edge computing \cite{zhu2024edge}, video streaming \cite{zhou2024survey}, or generative AI \cite{xu2024unleashing}, without systematically examining the integration of video analytics across cloud, edge, and terminal tiers. While \cite{gu2024aienhanced} provides insights into AI-enhanced cloud-edge-terminal networks, it does not specifically address the unique challenges and opportunities in video analytics applications. We aim to bridge this gap by providing a systematic review of video analytics in CETC systems, with particular attention to the interplay between different tiers and their roles in efficient video processing and analysis. The taxonomy of this survey is illustrated in Fig.~\ref{fig:taxonomy}, which categorizes existing works into edge-centric, cloud-centric, and hybrid video analytics. The main contributions of this survey are fourfold: \textbf{1)} \textit{A systematic taxonomy}. To the best of our knowledge, we are the first to present a structured taxonomy that categorizes existing works in video analytics across cloud, edge, and terminal tiers, covering architectural paradigms, processing strategies, and optimization techniques. \textbf{2)} \textit{A comprehensive review}. We provide an analysis of current research progress in edge-centric, cloud-centric, and hybrid video analytics, examining their advantages, limitations, and practical implementations. \textbf{3)} \textit{Novel insights}. We identify emerging trends and challenges in CETC video analytics, including platform integration, system scalability, data protection, and the integration of advanced technologies like LLMs and multimodal learning. \textbf{4)} \textit{Future directions}. We discuss potential research opportunities and outline promising directions for advancing video analytics in CETC systems.
% \footnote{More details on \href{https://github.com/linxiaogong/CETC}{GitHub}: \underline{\url{http://github.com/linxiaogong/CETC}}.}
\tikzstyle{my-box}=[
    rectangle,
    draw=hidden-draw,
    rounded corners,
    text opacity=1,
    minimum height=1.5em,
    minimum width=5em,
    align=center,
    fill opacity=.5,
    line width=0.8pt,
    font=\fontsize{11}{15}\selectfont,
    text=black,
]

\tikzstyle{tklevel0}=[my-box,
    fill=level0!50,
    font=\LARGE,
]

\tikzstyle{tklevel1}=[my-box,
    fill=level1,
    font=\LARGE,
]

\tikzstyle{tklevel2}=[my-box,
    fill=level2!60,
]
\tikzstyle{tklevel3}=[my-box,
    fill=level3!60,
]

\tikzstyle{leaf}=[my-box,
    fill=level4!50,
]

\begin{figure}[t]
    \centering
    \resizebox{0.48\textwidth}{!}
    {
        \begin{forest}
            forked edges,
            for tree={
                fill=level0!50,
                grow=east,
                reversed=true,
                anchor=base west,
                parent anchor=east,
                child anchor=west,
                base=center,
                font=\large,
                rectangle,
                draw=hidden-draw,
                rounded corners,
                align=center,
                minimum width=4em,
                edge+={darkgray, line width=1pt},
                s sep=3pt,
                line width=0.8pt,
                ver/.style={rotate=90, child anchor=north, parent anchor=south, anchor=center},
            },
            where level=0{ver,tklevel0}{},
            where level=1{ver,tklevel1}{},
            where level=2{text width=7em,tklevel2,}{},
            where level=3{text width=10.3em,tklevel3,}{},
            [Video Analytics in CETC Systems
                [Edge-centric Analytics
                    [On-device\\Processing,align=center
                        [REVAMP2T \cite{neff2020revamp2t}{,}~Joshi \textit{et al.}~\shortcite{joshi2024integration}{,}\\SAMEdge \cite{lu2024samedge}{,}~Gao \textit{et al.}~\shortcite{gao2024edgevision}, leaf,text width=42.8em,]
                    ]
                    [Edge-assisted\\Offloading,align=center
                        [Turbo \cite{lu2023turbo}{,}~Ji \textit{et al.}~\shortcite{ji2024task}{,} Bandwidth Optimization \cite{wang2023shoggoth}, leaf,text width=42.8em,]
                    ]
                    [Edge\\Intelligence
                        [MARL \cite{gao2024edgevision}{,} Online Distillation \cite{wang2023shoggoth}{,} \cite{bhardwaj2022ekya}, leaf,text width=42.8em,]
                    ]
                    [Edge-based\\Processing
                        [Video Frame Filtering
                            [DTMP \cite{xue2021denoisingbased}{,} Zhu \textit{et al.}~\shortcite{zhu2022open}{,}\\ DeGraF \cite{stephenson2019degrafflow}{,}~Wu \textit{et al.}~\shortcite{wu2020descriptive}, leaf,text width=30.6em,]
                        ]
                        [ROI Extraction
                            [Lu \textit{et al.}~\shortcite{lu2023turbo}{,} Superpixel Motion \cite{xu2019robust}{,}\\
                            REVAMP2T \cite{neff2020revamp2t}{,}~Nikouei \textit{et al.}~\shortcite{nikouei2018realtime}, leaf,text width=30.6em,]
                        ]
                        [Feature Encoding
                            [perf4sight~\cite{rajagopal2021perf4sight}{,}TVNet~\cite{fan2018endtoend}
                            \\TransFlow \cite{lu2023transflow}{,} DLOF \cite{tran2024deeplearning}, leaf,text width=30.6em,]
                        ]
                    ]
                    [Optimization\\Strategies
                        [Model Compression
                            [HQ \cite{liu2023hyperspherical}{,} SBS \cite{liu2023singlepath}{,}\\DoReFa-Net \cite{guerra2021automatic}, leaf,text width=30.6em,]
                        ]
                        [Hardware Acceleration
                            [iGniter Xu \textit{et al.}~\shortcite{xu2023igniter}{,} Ollivier \textit{et al.}~\shortcite{ollivier2023sustainable}{,}\\~Rajagopal \textit{et al.}~\shortcite{rajagopal2021perf4sight}, leaf,text width=30.6em,]
                        ]
                        [Adaptive Filtering 
                            [Ekya \cite{bhardwaj2022ekya}{,} EdgeDASH \cite{bayhan2021edgedash}{,}\\~Xiao \textit{et al.}~\shortcite{xiao2024transcodingenabled}{,}~Lin \textit{et al.}~\shortcite{lin2022efficient}, leaf,text width=30.6em,]
                        ]
                    ]
                ]
                [Cloud-centric Analytics
                    [Cloud\\Processing
                        [Gao \textit{et al.}~\shortcite{gao2024edgevision}{,} Centralized Management \cite{wang2023shoggoth}{,}~Zhang \textit{et al.}~\shortcite{zhang2021serverless}, leaf,text width=42.8em,]
                    ]
                    [Training\\Strategies
                        [Supervised Learning
                            [VCReg \cite{zhu2024variancecovariance}{,} Video-LaVIT \cite{jin2024videolavit}{,}\\~Zhang \textit{et al.}~\shortcite{zhang2021serverless}, leaf,text width=30.6em,]
                        ]
                        [Unsupervised Learning
                            [DyAnNet \cite{thakare2023dyannet}{,}  Meissen \textit{et al.}~\shortcite{meissen2023unsupervised}{,}\\GMM-DAE~\cite{ouyang2021video}, leaf,text width=30.6em,]
                        ]
                        [Transfer Learning
                            [Co-finetuning \cite{arnab2022transfer}{,} VCReg \cite{zhu2024variancecovariance}{,}\\ Caccia \textit{et al.}~\shortcite{caccia2023computeoptimal}, leaf,text width=30.6em,]
                        ]
                    ]
                    [Inference\\Optimization
                        [Model Parallelism
                            [MPVN \cite{carreira2018massively}{,}~Xu \textit{et al.}~\shortcite{xu2023igniter}{,}\\CROSSBOW \cite{koliousis2019crossbow}, leaf,text width=30.6em,]
                        ]
                        [Batch Processing
                            [Llama \cite{romero2021llama}{,} VPaaS~\cite{zhang2021serverless}{,}\\~\cite{carreira2018massively}, leaf,text width=30.6em,]
                        ]
                        [Model Serving
                            [Liu \textit{et al.}~\shortcite{liu2023hyperspherical}{,} Koliousis \textit{et al.}~\shortcite{koliousis2019crossbow}{,}\\ Hyper~\cite{buniatyan2019hyper}, leaf,text width=30.6em,]
                        ]
                    ]
                ]
                [Hybrid Analytics,align=center,
                    [Adaptive Task\\Offloading
                        [Tian \textit{et al.}~\shortcite{tian2024dynamic}{,} ~Silva \textit{et al.}~\shortcite{silva2021energyaware}{,} ECIVA~\cite{ji2024task}{,}~Li \textit{et al.}~\shortcite{li2024twotimescale}{,}\\ DMQTO~\cite{sharma2024deep}{,} ECAO~\cite{zhang2024energy}{,} ~Xiao \textit{et al.}~\shortcite{xiao2024transcodingenabled},leaf,text width=42.8em,]
                    ]
                    [Resource-aware\\Scheduling
                        [Zhang \textit{et al.}~\shortcite{zhang2024hybrid}{,} ~Li \textit{et al.}~\shortcite{li2024twotimescale}{,} A2C-DRL~\cite{lu2024a2cdrl}{,}\\EdgeVision~\cite{gao2024edgevision},leaf,text width=42.8em,]
                    ]
                ]
            ]
        \end{forest}
    }    
\vspace{-16px}
\caption{Taxonomy of video analytics in CETC systems.}
\label{fig:taxonomy}
\vspace{-15pt}
\end{figure}


\section{Foundation of CETC}\label{sec2}
\subsection{Architectural Paradigms}\label{sec2.1}
% Architectures in CETC systems adopt three core paradigms differentiated by resource coordination mechanisms and processing hierarchies.
Architectures in CETC systems adopt three core paradigms differentiated by resource coordination mechanisms.
\paragraph{Hierarchical Architectures.}\label{sec2.1.1}
Hierarchical architectures form the backbone of contemporary CETC video analytics systems, enabling distributed processing across multiple computational tiers~\cite{joshi2024integration}. Within this framework, the cloud serves as the central orchestrator, managing complex computations while edge devices positioned near data sources handle intermediate computations and feature extraction. Specifically, given raw video data $V_t$ originates from terminal devices, undergoing initial processing $f_t$ locally before edge-level transformations $f_e$ generate processed data $D_e$ for cloud analysis $f_c$ and subsequent control signal $C$ generation. Resource optimization occurs through strategic task allocation $t_i$, which minimizes latency $L_j$ based on available computational resources $R_j$ across deployment tiers~\cite{tian2024dynamic}. Edge-level filtering and aggregation mechanisms effectively reduce data volumes, while centralized management streamlines system control and enhances security through restricted raw data access~\cite{vo2022edge}. Notable challenges include potential vulnerability to cloud-level failures and network disruptions~\cite{sedlak2024equilibrium}.
\paragraph{Distributed Architectures.}\label{sec2.1.2}
Collaborative processing paradigms form the foundation of modern distributed video analytics, enabling autonomous edge and terminal device cooperation while reducing cloud dependencies. Fundamental to this approach is the implementation of federated learning~\cite{guo2023fedfsnet}, which aggregates local model updates $w_i^t$ from $N$ participating devices into a global model $W_t = \frac{1}{N} \sum_{i=1}^{N} w_i^t$ without raw data exchange. Resource utilization optimization involves careful consideration of task offloading decisions through utility functions $\min \sum_{i=1}^{N} \sum_{j=1}^{M} x_{ij} (c_{ij} + l_{ij})$ with task-to-tier allocation variable $x_{ij}$, balancing communication costs $c_{ij}$ and execution costs $l_{ij}$ under system constraints~\cite{ji2024task}. Advanced scheduling mechanisms~\cite{kuswiradyo2024optimizing} complement these efforts through dynamic resource allocation. Additionally, the computing continuum paradigm further bridges edge-cloud tiers for adaptive resource sharing and workload distribution, while WebAssembly~\cite{sedlak2024equilibrium} enables cross-platform compatibility through unified runtime environments.

\paragraph{Hybrid Architectures.}\label{sec2.1.3}
Adaptive resource allocation mechanisms characterize hybrid architectures, enabling dynamic workload distribution across cloud, edge, and terminal tiers based on operational parameters~\cite{bayhan2021edgedash}. To optimize this distribution, a cost-minimization framework guides task execution decisions, where $C_i(R_j)$ represents the resource-specific cost for task $T_i$ on computing unit $R_j$. Within this framework, communication efficiency optimization focuses on minimizing latency $L_i$, expressed as $D_i/B_j$ for data volume $D_i$ and available bandwidth $B_j$~\cite{silva2021energyaware}. System performance benefits from content-aware video encoding strategies that adjust rates $E_i$ based on bandwidth availability and content complexity~\cite{zhang2021serverless}. Additionally, hierarchical learning, with initial training in the cloud and subsequent fine-tuning at the edge, enables efficient model updates \cite{xu2024unleashing}. To enhance computational efficiency, transfer learning approaches facilitate model adaptation through cloud-initialized training refined at edge nodes~\cite{gao2024edgevision}. Finally, performance optimization leverages resource-aware scheduling algorithms that consider device-specific capabilities.


\subsection{Edge Computing Platforms}\label{sec2.2.2}
Modern platforms including NVIDIA Jetson and AWS Wavelength leverage containerization technologies (e.g., Docker and Kubernetes) for streamlined deployment~\cite{lu2024samedge}, with selection criteria driven by computational and power requirements~\cite{bhardwaj2022ekya}. Resource-optimized computing platforms enable efficient deployment of video analytics models on constrained devices like single-board computers and embedded systems.  By applying queueing theory principles, system architects model resource allocation across $N_s$ identical servers with arrival rate $\lambda$ and service rate $\mu$, maintaining system utilization $\rho = \frac{\lambda}{N_s\mu}$ below unity to ensure processing stability~\cite{ji2024task}. Real-time performance metrics employ Little's Law, where average waiting time $W = \frac{q_l}{\lambda}$ reflects queue length $q_l$~\cite{gao2024edgevision}. Likewise, Hardware acceleration with graphics processing units (GPUs) and field-programmable gate arrays (FPGAs) offers significant performance improvements, often quantified by the speedup ratio $S = \frac{T_{\text{CPU}}}{T_{\text{accelerator}}}$, while power optimization necessitates a careful balance between static and dynamic power consumption $P = P_{\text{static}} + P_{\text{dynamic}}$.

\subsection{Resource Management}\label{sec2.2.4}
Resource management in CETC systems presents multifaceted challenges of optimizing computational efficiency across distributed infrastructure tiers.
\paragraph{Latency Minimization.} Resource allocation in CETC video analytics systems efficiently distributes computing, storage, and network resources, aiming to minimize system latency $L_{sys}$, which encompasses processing $L_{proc}$, transmission $L_{trans}$, and queuing $L_{queue}$ latencies, i.e., $L_{sys} = L_{proc} + L_{trans} + L_{queue}$. Recent works demonstrate innovative approaches to optimize each component: 1) For $L_{trans}$, \cite{zhang2021serverless} proposes a cloud-fog video streaming protocol that reduces bandwidth usage by 21\% through adaptive quality selection and fog-assisted correction; 2) $L_{queue}$ is addressed by A2C-DRL \cite{lu2024a2cdrl} that dynamically schedules tasks across edge-cloud nodes, achieving RTT reduction through decentralized reinforcement learning.

\paragraph{Energy Efficiency and Throughput Maximization.} Minimizing energy consumption $E_{total}$ is crucial, particularly at the edge and terminal devices. The total energy consumption across $N_d$ devices is calculated by summing individual device energy $E_i$ based on allocated computing resources $r_i$ and processing time $t_i$: $E_{total} = \sum_{i=1}^{N_d} E_i(r_i, t_i)$. \cite{hua2024energyefficient} explores energy-efficient allocation in heterogeneous edge-cloud computing, considering user mobility. In parallel, network throughput $T$ is maximized through efficient bandwidth allocation, computed as the sum product of allocated bandwidth $B_{jk}$ and achieved data rate $R_{jk}$ across $M$ links and $K$ users: $T = \sum_{j=1}^{M} \sum_{k=1}^{K} B_{jk} \cdot R_{jk}$. Furthermore, continuous learning of video analytics models on edge servers can optimize resource utilization and improve overall network performance \cite{bhardwaj2022ekya}.

\paragraph{Quality of Experience and Dynamic Adaptation.}  Quality of experience (QoE), often represented by a utility function $U(q)$, is a critical factor in video analytics. For example, EdgeDASH \cite{bayhan2021edgedash} framework explores network-assisted adaptive video streaming for edge caching, optimizing QoE by facilitating the use of cached video content and improving cache hit rates and video quality. Reinforcement learning (RL) offers a promising approach, with \cite{ji2024task} introducing a GRU-enhanced deep Q-Network (GEDQN) for task offloading. Building upon this foundation, hybrid approaches combining deep reinforcement learning with multi-objective optimization demonstrate superior performance in balancing task completion time, cost, energy consumption, and system reliability \cite{zhang2024hybrid}. Furthermore, opportunistic data enhancement using idle GPU resources \cite{lu2023turbo}, showcases innovative strategies for resource utilization optimization.
\section{Edge-centric Video Analytics}\label{sec3}
\subsection{On-device Processing}\label{sec3.2}
On-device video analytics in CETC systems offers advantages such as reduced latency for real-time applications like autonomous driving and security surveillance \cite{neff2020revamp2t}, and minimized bandwidth consumption, particularly crucial in bandwidth-constrained environments \cite{joshi2024integration}.  Additionally, on-device processing enhances privacy by localizing sensitive video data and improves system scalability and resilience through distributed workload processing \cite{gao2024edgevision}. However, challenges include limited computational resources and power budgets on edge devices, restricting model complexity and performance \cite{sedlak2024equilibrium}.  The heterogeneity of edge devices also introduces integration and management complexities, while ensuring reliability and robustness against device failures or network disruptions remains a concern \cite{vo2022edge}.  Optimizing model size and computational efficiency for resource-constrained devices while maintaining accuracy is crucial, necessitating techniques like model compression, hardware acceleration, and efficient task offloading strategies \cite{ji2024task}.  Some works like REVAMP$^2$T \cite{neff2020revamp2t} and SAMEdge \cite{lu2024samedge} demonstrate the feasibility of on-device processing for real-time, privacy-aware pedestrian tracking and supporting complex models on edge devices.
\subsection{Edge-assisted Offloading}\label{sec3.3}
Edge servers are crucial for pre-processing and filtering video data in CETC systems \cite{kuswiradyo2024optimizing}, reducing the volume of data transmitted to the cloud and thus minimizing bandwidth consumption and enhancing system efficiency.  Specifically, they perform initial processing, such as Turbo \cite{lu2023turbo} and ROI extraction \cite{ji2024task}, to reduce redundancy and extract key information before cloud transmission.  Such edge pre-processing contributes to lower latency by performing computations closer to the data source \cite{buniatyan2019hyper}, enabling near real-time analytics even with limited bandwidth \cite{wang2023shoggoth}.  Additionally, edge-assisted offloading improves resource utilization by distributing the computational workload, optimizing the use of both edge and cloud resources for more efficient and scalable video analytics systems \cite{hua2024energyefficient}.
\subsection{Edge Intelligence}\label{sec3.4}
Advanced computational methodologies in CETC systems leverage intelligent decision-making capabilities at network edges, enabling real-time insights with minimal cloud dependencies~\cite{sharma2024deep}. Specifically, online knowledge distillation techniques effectively adapt to dynamic scenes while mitigating data drift, complemented by adaptive training approaches utilizing small batches and selective frame sampling~\cite{wang2023shoggoth}. Integrating AI at the network edge successfully addresses stability concerns and device overload while optimizing energy-latency trade-offs, enhanced by incremental learning with real-time data streams~\cite{joshi2024integration}. In addition, multiagent reinforcement learning (MARL) like EdgeVision~\cite{gao2024edgevision} also facilitate autonomous learning across edge nodes, while direct deployment allows for continuous learning and adaptation of video analytics models on edge servers, addressing the challenges of data drift and ensuring long-term model accuracy \cite{bhardwaj2022ekya}.

\subsection{Edge-based Video Processing}\label{sec3.5}
Edge computing plays a crucial role in CETC systems by enabling video processing near the data source. This section explores key techniques for edge-based video processing.
\paragraph{Video Frame Filtering.}\label{sec3.5.1}
% Edge video frame filtering (VFF) reduces downstream computational load and cloud transmission volume by discarding redundant or irrelevant frames. Keyframe extraction identifies and retains the most informative frames within a video sequence, while background subtraction removes the static background, isolating moving foreground objects for analysis \cite{wu2020descriptive}.  Several techniques enhance background subtraction efficiency and robustness.  For instance, background initialization algorithms using superpixel motion detection, effectively separate foreground from background \cite{xu2019robust}.  Methods based on sparse representation and outlier removal, along with robust online matrix factorization, model the background and segment foreground objects, adapting to background changes~\cite{tran2024deeplearning}.  Approaches like the weakly supervised method for open set video anomaly detection (OpenVAD), which integrates evidential deep learning and normalizing flows \cite{zhu2022open}, enhance robustness in challenging scenarios by effectively identifying both known and novel anomalies.  Additionally, denoising-based turbo message passing (DTMP) \cite{xue2021denoisingbased} optimizes background subtraction in compressed video streams, while convolutional neural networks (CNNs) provide efficient filtering for depth videos, particularly in complex lighting conditions, the dense gradient based features (DeGraF) \cite{stephenson2019degrafflow} further enhances optical flow estimation by offering uniformly distributed and tunable feature points. Consequently, by leveraging lightweight models and employing model selection strategies at the edge, real-time video analytics can be achieved even with limited resources and bandwidth \cite{gao2024edgevision}.

Edge video frame filtering (VFF) optimizes computational efficiency by discarding redundant frames and extracting key information through intelligent frame selection and background modeling. Keyframe extraction retains informative frames while background subtraction isolates moving objects for downstream analysis \cite{wu2020descriptive}. Advanced background modeling techniques, from superpixel motion detection \cite{xu2019robust} to sparse representation with outlier removal \cite{tran2024deeplearning}, enable robust foreground segmentation under dynamic conditions. Recent innovations like weakly supervised OpenVAD which integrates evidential deep learning and normalizing flows \cite{zhu2022open}, enhance robustness in challenging scenarios by effectively identifying both known and novel anomalies.  Additionally, denoising-based turbo message passing (DTMP) \cite{xue2021denoisingbased} optimizes background subtraction in compressed video streams, while CNNs provide efficient filtering for depth videos, particularly in complex lighting conditions. DeGraF \cite{stephenson2019degrafflow} further enhances optical flow estimation by offering uniformly distributed and tunable feature points. Consequently, by leveraging lightweight models and employing model selection strategies at the edge, real-time video analytics can be achieved even with limited resources and bandwidth \cite{gao2024edgevision}.

\paragraph{Region of Interest Extraction.}\label{sec3.5.2}
ROIE is crucial for edge-centric video analytics, identifying relevant areas within video frames at the edge to reduce computational burden and bandwidth requirements for downstream processing.  Efficient ROIE is paramount in edge-centric video analytics, employing diverse techniques like object detection~\cite{lu2023turbo} can identify informative regions for subsequent processing.  Additionally, motion detection algorithms, such as those employed in the blockchain-enabled surveillance video indexing scheme \cite{nikouei2018realtime}, effectively identify regions exhibiting significant motion, enabling efficient extraction of motion information and object segmentation. Superpixel-based motion detection \cite{xu2019robust} combines spatial and temporal information for robust extraction under challenging conditions. Interactive segmentation techniques, such as REVAMP$^2$T \cite{neff2020revamp2t}, enable users to refine ROI selection through clicks, thereby enhancing the accuracy and efficiency of video analysis tasks.

\paragraph{Feature Encoding.}\label{sec3.5.3}
Edge computing layers perform critical visual abstraction through compact feature extraction, where convolutional networks like perf4sight~\cite{rajagopal2021perf4sight} establish hierarchical spatial representations. Temporal dynamics are captured via optical flow estimation techniques, with DeGraF-Flow~\cite{stephenson2019degrafflow} leveraging dense gradient fields and TransFlow~\cite{lu2023transflow} introducing transformer-based global dependency modeling. Complementing these approaches, TVNet~\cite{fan2018endtoend} and DLOF~\cite{tran2024deeplearning} employ deep architectures to learn flow-like motion patterns for enhanced video understanding. Additionally, feature encoding exploits temporal context through two-stream architectures \cite{ji2024task} and self-supervised contrastive learning \cite{noguchi2023egovehicle} to enhance video analytics performance.

\subsection{Optimization Strategies}\label{sec3.6}
Edge video analytics optimization strategies encompass model compression, hardware acceleration, and adaptive filtering for computational efficiency.
\paragraph{Model Compression.}\label{sec3.6.1}
Optimizing deep neural networks for edge deployment necessitates innovative approaches to reduce computational overhead while maintaining analytical accuracy through pruning and quantization \cite{joshi2024integration}. Structured pruning eliminates redundant filters or channels to enhance inference speed on generic hardware, whereas unstructured pruning targets sparse weight removal for potential accuracy preservation at the cost of specialized accelerator requirements. Quantization reduces the precision of model parameters and activations, representing them with fewer bits, as exemplified by binarized neural networks and DoReFa-Net \cite{guerra2021automatic}, drastically reducing model size and computational cost. Hyperspherical quantization methods like hyperspherical quantization (HQ) \cite{liu2023hyperspherical} improve the accuracy of quantized models by minimizing the distance between full-precision and quantized weights. Joint pruning and quantization techniques like single-path bit sharing (SBS) \cite{liu2023singlepath}, offer a unified approach for higher compression ratios. To optimize memory usage during training, techniques like entropy-based pruning can be employed to minimize the size of intermediate activations, making it possible to train larger models or utilize devices with limited memory capacity \cite{caccia2023computeoptimal}.

\paragraph{Hardware Acceleration.}\label{sec3.6.2} % could short
Edge hardware acceleration utilizes specialized hardware such as GPUs \cite{xu2023igniter} and FPGAs \cite{ollivier2023sustainable} to improve the efficiency of video processing at the edge. The utilization of GPUs in edge computing environments, as demonstrated by the perf4sight \cite{rajagopal2021perf4sight}, highlights their effectiveness in accelerating computationally intensive tasks such as CNN-based object detection and video analysis.  In contrast, FPGAs provide flexibility and customizability, enabling the implementation of optimized hardware architectures for specific video processing tasks and achieving high performance and energy efficiency.  Their reconfigurability also allows adaptation to evolving video analytics algorithms and application requirements.  In heterogeneous embedded systems, the use of such specialized hardware further enhances performance and energy efficiency \cite{lin2022efficient}.  Considering sustainable AI, evaluating hardware acceleration options based on factors like embodied energy and carbon footprint becomes crucial, with technologies like Racetrack memory PIM offering potential advantages \cite{ollivier2023sustainable}.

\paragraph{Adaptive Filtering.}\label{sec3.6.3}
Adaptive filtering in edge-centric video analytics dynamically adjusts filtering parameters based on both video content characteristics and real-time edge resource availability to optimize performance in resource-constrained environments.  For example, in complex scenes or when motion is high, more intensive filtering may be necessary for effective feature extraction, while under simpler content or low resource availability, less intensive filtering can conserve power and minimize latency. Such dynamic adaptation aligns with opportunistic resource utilization \cite{lu2023turbo}, and resonates with quality adaptation mechanisms in adaptive video streaming (e.g., EdgeDASH \cite{bayhan2021edgedash}).  Adaptive filtering can also optimize VFF and ROIE by dynamically selecting keyframes, adjusting background subtraction parameters, or refining object/motion detection parameters based on video content dynamics.  Systems like Ekya \cite{bhardwaj2022ekya} leverage this approach to balance retraining and inference accuracy on edge servers, complementing resource-aware scheduling \cite{xiao2024transcodingenabled}.  In addition, parameter selection can be informed by video content variability for consistent quality, and the system in \cite{lin2022efficient} demonstrates the importance of optimizing data flows across heterogeneous compute units, which can further enhance adaptive filtering's effectiveness.

\section{Cloud-centric Video Analytics}\label{sec4}
\subsection{Cloud-based Processing}\label{sec4.2}
Powerful computational capabilities transform complex video analytics through cloud-based infrastructures, leveraging vast computational resources and storage capacity \cite{gao2024edgevision}.  The abundant computing power enables execution of computationally intensive deep learning models for tasks like object detection and action recognition, crucial for handling high-resolution video and complex analytical pipelines.  Correspondingly, the virtually unlimited storage accommodates large datasets generated by video analytics, facilitating long-term trend analysis and pattern recognition. Additionally, centralized cloud resources facilitate training sophisticated video analytics models using large datasets, leading to improved accuracy and generalization through training paradigms including supervised, unsupervised, and transfer learning \cite{zhang2021serverless}. Cloud platforms also offer scalability and flexibility, adapting to fluctuating workloads and varying processing demands, essential for supporting real-time video processing from multiple sources \cite{lu2023turbo}.  Centralized management simplifies deployment, updates, and monitoring, streamlining system administration \cite{wang2023shoggoth}. Consequently, cloud processing allows edge and terminal devices to focus on data acquisition and preprocessing, while the cloud handles model training and complex inference, optimizing resource utilization across the system \cite{lu2024samedge}.
\subsection{Training Strategies}\label{sec4.4}
In this section, we review training strategies for cloud-centric video analytics, including supervised, unsupervised, and transfer learning.
\paragraph{Supervised Learning.}\label{sec4.4.1}
Deep learning models have been widely adopted in cloud-centric video analytics, demonstrating outstanding performance in pattern and behavior recognition, with capabilities in accurate anomaly detection and behavior analysis \cite{xu2024unleashing}. Cloud infrastructure enables training of complex CNNs that learn spatio-temporal representations, which can be deployed across CETC systems for tasks ranging from ego-vehicle action recognition to unintentional action identification \cite{noguchi2023egovehicle}. Additionally, model performance relies on representation diversity and feature transferability, with techniques like variance-covariance regularization (VCReg) promoting high-variance, low-covariance representations across diverse tasks \cite{zhu2024variancecovariance}. Moreover, in open-set video anomaly detection, where models need to generalize to unseen anomaly types, the choice of training data becomes crucial for ensuring robust performance, while the cost of labeling diverse anomaly types can be substantial \cite{zhu2022open}.

\paragraph{Unsupervised Learning.}\label{sec4.4.2}
Novel pattern discovery and anomaly identification emerge through unlabeled video data analysis, offering particular value when labeled data proves scarce or costly \cite{liu2024generalized}. For example, deep probabilistic models and clustering techniques like GMM-DAE \cite{ouyang2021video}, enable outlier detection in latent feature spaces. Recently approaches integrate evidential deep learning with normalizing flows \cite{zhu2022open} to address open-set video anomaly detection, while physical law discovery from distorted video leverages autoencoders and symbolic regression \cite{udrescu2021symbolic}. DyAnNet \cite{thakare2023dyannet} uses unsupervised clustering and scene dynamicity to generate preliminary anomaly scores, subsequently refined by a cross-branch feed-forward network.  Similarly, structural feature-autoencoders \cite{meissen2023unsupervised} address imperfect reconstruction in medical images by transforming input images into a feature space for anomaly detection along different discriminative feature maps.
\paragraph{Transfer Learning.}\label{sec4.4.3}
Pre-trained model adaptation leverages edge-cloud collaborative architectures to enhance performance and efficiency in video analytics tasks, utilizing distributed computational resources and optimized model deployment strategies \cite{xu2024unleashing}. The success of transfer learning in video analytics is strongly influenced by the alignment between the source and target domains. For example, Video-LaVIT \cite{jin2024videolavit} leverages unified visual-motional tokenization, and has shown promising results in transferring knowledge across diverse video-language tasks. Transferring knowledge from action recognition in generic videos to recognizing surgical procedures in medical videos could be effective due to shared features like human motion, while vastly different source and target tasks may hinder performance \cite{caccia2023computeoptimal}. In addition, the size and quality of target dataset also influence fine-tuning outcomes, while techniques like co-finetuning enhance rare class performance \cite{arnab2022transfer}. Similarly, variance-covariance regularization (VCReg) \cite{zhu2024variancecovariance} promotes feature diversity, complemented by structured pruning approaches that optimize computational efficiency.

\subsection{Inference Optimization}\label{sec4.5}
Emerging cloud-based inference optimization techniques address the challenges of large-scale video analytics through parallel processing, distributed computation, and efficient model serving strategies.
\paragraph{Model Parallelism.}\label{sec4.5.1}
Distributed computation across multiple cloud instances enables efficient execution of large-scale video analytics models through parallel processing, particularly benefiting complex architectures that exceed single-machine capabilities~\cite{tian2024dynamic}. It benefits complex video models that are too large for single machines or require excessive inference times. For example, massively parallel video networks (MPVN) \cite{carreira2018massively} use operation pipelining and multi-rate clocks for parallel computation across multiple GPUs, minimizing latency and maximizing throughput for tasks like action recognition. Similarly, CROSSBOW \cite{koliousis2019crossbow} employs synchronous model averaging (SMA) to improve training time by allowing replicas to explore the solution space independently while maintaining consistency, and systems like iGniter \cite{xu2023igniter} address GPU sharing challenges through proactive, interference-aware resource provisioning for predictable DNN inference and cost-efficient allocation.

\paragraph{Batch Processing.}\label{sec4.5.2}
Concurrent processing of multiple video streams enables significant efficiency gains through cloud-based computational resources, optimizing both processing time and cost-effectiveness compared to sequential approaches. For example, platforms like Llama~\cite{romero2021llama} leverage heterogeneous serverless frameworks for dynamic resource allocation, while systems such as MPVN \cite{carreira2018massively} integrate model parallelism techniques (e.g., VPaaS \cite{zhang2021serverless}) to maximize throughput and minimize latency. However, effective batch processing requires careful resource allocation and scheduling to avoid bottlenecks, which analyze aggregating functions in serverless clouds for video streaming \cite{wu2020descriptive}.

\paragraph{Model Serving.}\label{sec4.5.3}

Deploying video analytics models in the cloud requires accessible APIs for applications and users. Cloud platforms provide robust infrastructure for efficient model management, scaling, and monitoring, ensuring high availability through dynamic resource allocation. Tools like TensorFlow Serving and AWS SageMaker enable seamless deployment and updates of models, facilitating reliable service delivery \cite{vo2022edge}. For example, Hyper~\cite{buniatyan2019hyper} offers a hybrid distributed cloud solution for large-scale deep learning tasks, encompassing both model deployment and inference. Additionally, model quantization like HQ \cite{liu2023hyperspherical} can reduce model size and improve accuracy, leading to more efficient model serving.  In addition, CROSSBOW \cite{koliousis2019crossbow}, a system for training deep learning models on multi-GPU servers, has the potential to be extended for optimized model serving in similar cloud environments.
\section{Hybrid Video Analytics}\label{sec5}
\begin{figure}[t]
    \centering
    \includegraphics[width=0.48\textwidth]{fig3_example.pdf}
    \vspace{-20px}
    \caption{Example of CETC video analytics workflow.}
    \label{fig:workflow}
    \vspace{-12pt}
\end{figure}
\subsection{Adaptive Task Offloading}\label{sec5.1}
Dynamic workload distribution across CETC video analytics systems responds intelligently to network bandwidth fluctuations, resource availability, and application demands \cite{tian2024dynamic}. Under constrained network conditions, latency-sensitive operations migrate to edge nodes to circumvent transmission bottlenecks \cite{silva2021energyaware}, whereas cloud resources activate when edge capacity thresholds are exceeded. The Shoggoth architecture \cite{wang2023shoggoth} dynamically adjusts task allocation between edge and cloud based on these factors, optimizing inference performance. Contemporary research diverges into multiple optimization paradigms: token bucket regulated task scheduling in ECIVA \cite{ji2024task} contrasts with hierarchical approaches like dual-timescale service caching \cite{li2024twotimescale} and meta-reinforcement learning architectures, DMQTO \cite{sharma2024deep}, leverage online and reinforcement learning to optimize offloading decisions. Building upon these foundations, task dependency modeling in human digital twin systems, ECAO \cite{zhang2024energy}, introduces context-aware orchestration, while transcoding-aware caching strategies \cite{xiao2024transcodingenabled} optimize content delivery through adaptive bitrate adjustments, as illustrated in Fig.~\ref{fig:workflow}.

\subsection{Resource-aware Scheduling}\label{sec5.2}
Resource-aware scheduling intelligently distributes video processing tasks across the heterogeneous cloud-edge-terminal infrastructure, considering diverse capabilities and constraints to optimize system performance \cite{zhang2024hybrid}.  Specifically, these approaches aim to minimize latency and energy consumption while maximizing resource utilization and meeting quality-of-service requirements. For example, computationally intensive tasks can be offloaded to the resource-rich cloud, while latency-sensitive tasks can be executed on edge devices \cite{ji2024task}.  Several key works address this challenge. \cite{li2024twotimescale} propose a two-timescale framework for joint service caching and resource allocation, using Lyapunov-based optimization to maximize long-term network performance under energy constraints.  In a different vein,~\cite{lu2024a2cdrl} introduce A2C-DRL, a dynamic scheduling technique leveraging Advantage Actor-Critic and deep reinforcement learning for efficient resource utilization and load balancing in stochastic edge-cloud environments. Additionally, EdgeVision \cite{gao2024edgevision} employs multiagent reinforcement learning for resource-aware scheduling, enabling autonomous edge nodes to learn optimal policies for workload distribution and request dispatching. Similarly,~\cite{zhang2024hybrid} combines deep reinforcement learning and multi-objective optimization, considering task completion time, cost, energy consumption, and system reliability for balanced performance.
\section{Challenges and Future Directions}\label{sec6}
Despite significant progress in video analytics, challenges persist in CETC systems, and we identify and discuss key issues and future directions in this section.
\paragraph{Platform Integration.}\label{sec6.1}
Heterogeneous hardware and diverse software environments pose significant challenges to seamless interoperability in CETC video analytics systems.  Specifically, achieving efficient data exchange, consistent model deployment, and coordinated task execution across these disparate platforms introduces complexities. To address these challenges, standardized interfaces and APIs can abstract hardware and software differences, enabling portable and efficient code development.  In addition, containerization and virtualization technologies simplify deployment and management across diverse platforms, while optimized communication protocols and data serialization formats minimize overhead and maximize performance.
\paragraph{System Scalability.}\label{sec6.2}
Edge video analytics systems face significant scalability challenges due to the growing number of edge devices and video streams, where distributed collaboration complexities and device heterogeneity introduce communication bottlenecks and inefficient resource allocation \cite{vo2022edge}. To address these challenges, decentralized management strategies via MARL enable autonomous resource allocation and task offloading \cite{gao2024edgevision}, while efficient communication protocols minimize overhead through hierarchical or peer-to-peer sharing \cite{ji2024task}. Furthermore, optimizing data processing pipelines via adaptive online learning \cite{wang2023shoggoth,lu2023turbo}, task-specific discrimination, and incorporating active inference mechanisms \cite{sedlak2024equilibrium} further enhance system scalability.
\paragraph{Data Protection.}\label{sec6.3}
Privacy and security considerations in distributed cloud-edge-terminal environments demand robust protection mechanisms for sensitive video data. To address this, federated learning emerges as a key solution, enabling collaborative model training without raw data sharing through frameworks like Fed-FSNet~\cite{guo2023fedfsnet}. For better data protection, HierSFL~\cite{quan2023hiersfl} incorporates local differential privacy with noise-added updates. Device heterogeneity poses significant challenges to efficiency and fairness in these systems. To address these issues, fuzzy networks facilitate synthetic i.i.d. data generation to mitigate distribution biases. Additionally, complementary approaches such as secure multi-party computation, homomorphic encryption, and differential privacy provide additional safeguards despite computational overhead~\cite{guo2023fedfsnet}.

\paragraph{System Resilience.}\label{sec6.4}
Video analytics in CETC systems require robust fault tolerance due to potential disruptions like network failures and device outages. For example, load rebalancing within a device cluster can restore individual edge device performance after network failures \cite{sedlak2024equilibrium}. The dynamic nature of these environments necessitates sophisticated adaptation mechanisms, particularly in task offloading frameworks that leverage enhanced deep Q-networks (EDQN) \cite{ji2024task} to optimize resource allocation during system failures. In addtion, resource-aware scheduling \cite{gao2024edgevision} further bolsters resilience through dynamic resource allocation based on real-time conditions. Redundancy in hardware and software, including backup servers and data replication, and blockchain technology for secure index authentication \cite{nikouei2018realtime} also enhances resilience. Additionally, graceful degradation strategies, which prioritize essential tasks and reduce the quality of non-critical processes during disruptions, ensure a baseline service level.
\paragraph{Explainable and Efficient Systems.}\label{sec6.5}
Explainable video analytics models enhance trust and transparency by providing insights into their decision-making processes, while efficient system design and specialized hardware accelerate edge processing through techniques like heterogeneous computing-based video segmentation~\cite{lin2022efficient} and turbo opportunistic enhancement~\cite{lu2023turbo} for latency-preserving accuracy improvements. Emerging large language model-multi-agent (LLM-MA) systems~\cite{ning2023videobench} advance video understanding and content generation~\cite{zhou2024survey} by combining multiple modalities~\cite{lu2024samedge}, though platform integration and scalability challenges~\cite{romero2021llama} must be addressed to realize the full potential of efficient and explainable LLM-MA systems in collaborative environments~\cite{bhardwaj2022ekya}.

\paragraph{Advanced Video Analytics.}\label{sec6.6}
Recent advances in video analytics have been revolutionized by LLMs and multimodal integration approaches. Video-LLMs and Video-LLaVA demonstrate sophisticated video understanding capabilities, while specialized benchmarks (e.g., Video-Bench and VLM-Eval) enable systematic evaluation~\cite{ning2023videobench}. Resource-efficient solutions leverage image LLM priors, and advanced architectures such as Video-LaVIT~\cite{jin2024videolavit} employ decoupled visual-motional tokenization for unified video-language pre-training. Integrating video data with multimodal integration of audio and text and enhanced time series analysis capabilities, offer opportunities for comprehensive video understanding and analysis~\cite{zhou2024survey}.
\vspace{-6pt}
\section{Conclusion}\label{sec7}
In this survey, we present an overview of video analytics in CETC systems, covering architectural paradigms, key technologies, and optimization strategies. Moreover, we highlight the benefits and challenges of edge-centric and cloud-centric approaches, emphasizing the importance of hybrid methods. Despite significant progress, challenges in platform integration, system scalability, data protection, and system resilience still require further research and innovative solutions.
\bibliographystyle{named}
\begin{spacing}{0.82} 
% {\fontsize{9pt}{10pt}\selectfont
\bibliography{IJCAI2025.bib}
% }
\end{spacing}
\end{document}