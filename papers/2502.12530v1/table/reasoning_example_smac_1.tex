
\begin{tcolorbox}[title=An example that a negative explanation classified
by the \rlm~is corrected by the rectified flow model $\varphi$ of on $\mathtt{SMAC}$]

    \textbf{Q} = Here are 5 allies and 6 enemies in SMAC. The goal of each ally is to attack all the enemies together with other allies. At every step, each ally can choose one action to take from the Action set. The trajectory of each ally consists of a series of State, Action, Value pairs. Here are 4 consecutive State, Action, Value pairs where the last state-action pair misses the Action: xxx. Please help me concisely summarize and analyze this part of the trajectory for the missing Action prediction.\\
    
    \textbf{Action set $\mathcal{A}$} = [`DEAD', `STOP', `NORTH', `SOUTH', `EAST', `WEST', \textcolor{outputcolor} {`Attack Enemy 0' (Correct)}, `Attack Enemy 1', `Attack Enemy 2', `Attack Enemy 3', `Attack Enemy 4', `Attack Enemy 5']\\
    % In the provided trajectory, each ally demonstrates a cautious yet proactive approach to engaging enemies. 
    
    \textbf{{Explanation:}} ...In the first state, Ally opts to move EAST, likely positioning itself xxx. Ally’s next action, “Attack Enemy 3”, shows a shift to offensive behavior, ... \textcolor{outputcolor}{Ally’s following action, “Attack Enemy 0”, further prioritizes offense, which might reflect a high-risk engagement. Ally’s last action solidifies this pattern, further engaging Enemy 0, likely due to the imminent engagement's hazards. Throughout these actions, ... The allies prioritize attacking specific foes based on their availability and proximity.$\checkmark$}  \\
    
    \textbf{{Distribution $p$ from the \rlm:}} [0.0074, 0.0031, 0.0260, 0.0062, 0.0055, 0.0096, 0.1069, 0.2148, \textbf{0.2812}, 0.1089, 0.1260, 0.1045] $\rightarrow$ `Attack Enemy 2’ {\color{red}$\boldsymbol{\times}$}\\\
    
    \textbf{{Distribution $\hat{p}$ from the rectified flow model:}} 
[0.0084, 0.0040, 0.0084, 0.0061, 0.0064, 0.0099, \textbf{0.3254}, 0.1110, 0.1137, 0.1771, 0.1065, 0.1218] $\rightarrow$ `Attack Enemy 0’ \textcolor{outputcolor} {$\checkmark$}\\

\end{tcolorbox}



