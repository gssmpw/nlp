
\begin{tcolorbox}[title=An example on $\mathtt{SMAC}$ that a negative explanation generated by the \rlm~is optimized to a positive explanation after training for two rounds]

\textbf{Q} = Here are 5 allies and 6 enemies in $\mathtt{SMAC}$. The goal of each ally is to attack all the enemies together with other allies. At every step, each ally can choose one action to take from the Action set. The trajectory of each ally consists of a series of State, Action, Value pairs. Here are 4 consecutive State, Action, Value pairs where the last state-action pair misses the Action: xxx. Please help me concisely summarize and analyze this part of the trajectory for the missing Action prediction.\\

    \textbf{Actual Action} = NORTH\\
    % In the provided trajectory, each ally demonstrates a cautious yet proactive approach to engaging enemies. 
    
    \textbf{{Explanation:}} In the first state, ally 0,1,2, and 3 all move south, ... The following action, STOP, ... Subsequently, ally 0 and 2 move WEST, ... Throughout these steps, allies prioritize movement over direct engagement, likely because no enemies are in view. Given the current state, where allies are arranged with their backs touching and no Enemy is in sight, the next logical action is EAST. This directional movement likely aims to realign the formation back to its original or a more advantageous state, ensuring that allies are positioned optimally for future engagements. {\color{red}This EAST movement contrasts with WEST, which may have been a short-term repositioning to balance the formation }{\color{red}(EAST, $\boldsymbol{\times}$)}, and NORTH and SOUTH, which were likely attempts to regroup or reposition relative to other allies. \\
    
    \textbf{{Improved Explanation (Ours):}}   In the first state, the ally chooses to move SOUTH, ... In the second state, the ally STOPs, ... Moving WEST in the third state likely aims to cover more ground or flank the enemies, ... In the latest state, with no visible enemies and all allies maintaining full health and no weapon cooldowns, the ally opts to move NORTH. This directional movement decision likely stems from a strategic need to reposition for a better vantage point or to avoid potential threats, \textcolor{outputcolor} {given the NORTHward movement may offer a more advantageous stance compared to other directions.$\checkmark$} Choosing NORTH over EAST, for instance, may be due to factors such as better cover or better line-of-sight, enhancing the ally's readiness to engage when enemies become visible.  \\

\end{tcolorbox}



