\begin{table}[]
    \centering
    \caption{An example of a true negative sample. The explanation is becomes reasonable after training with our method.}
    \vspace{-1em}
    \begin{tabular}{p{1.0\linewidth}}
    \toprule
        \textbf{Context} (shortened): A 67-year-old woman... had a pulmonary embolism... She had a retroperitoneal hemorrhage; anticoagulant therapy was temporarily discontinued, and she underwent placement of an inferior vena cava (IVC) filter. She had a hematoma that was resolving... Today, she says she has had a persistent sensation of tingling and numbness of her left thigh... the sensation has improved... Her only medication is warfarin. Vital signs are within normal limits. Examination of the skin shows no abnormalities. Muscle strength is normal. Sensation to light touch is decreased.. on the lateral aspect of the left anterior thigh. Which of the following is the most likely cause of this patient's decreased sensation?
 
    $\bm{\mathcal{A}}$: [A: Cerebral infarction during the hospitalization; B: Complication of the IVC filter placement; \textbf{C: Compression of the lateral femoral cutaneous nerve {\color{red}(Correct)}}; D: Hematoma of the left thigh.]
    
    \textbf{Explanation}: ... The patient's symptoms began after hospitalization, aligning with the placement of the IVC filter. Placement of such devices can sometimes cause ... {\color{red}The improvement over time supports a reversible cause related to the filter placement rather than other options like nerve compression from trauma or progressive neurological conditions.$\boldsymbol{\times}$}  Additionally, warfarin therapy alone typically ... 

        \textbf{\textcolor{cotcolor}{Improved Explanation (Ours):}} ...involving nerves near sites of cannula placement or previous bleeding sites. The Bezold personalization test assesses the discriminatory ability to touchantsia. Here, the sensation loss around the left thigh \textcolor{outputcolor}{aligns with nerve compression near the sites of recent interventions.$\checkmark$} 
    \bottomrule
    \end{tabular}
    \label{tab:my_label}
\end{table}

% \begin{figure}[]
% \begin{tcolorbox}
%     Q = 

%     $\mathcal{A}$ = \\
%     % In the provided trajectory, each ally demonstrates a cautious yet proactive approach to engaging enemies. 
    
%     \textbf{\textcolor{usercolor}{Plain Explanation ($\mathtt{SFT}$):}} \\
    

% \end{tcolorbox}
% \caption{XXX}
% \end{figure}