%%%%%%%% ICML 2025 EXAMPLE LATEX SUBMISSION FILE %%%%%%%%%%%%%%%%%

\documentclass{article}

% Recommended, but optional, packages for figures and better typesetting:
\usepackage{microtype}
\usepackage{graphicx}
\usepackage{subfigure}
\usepackage{booktabs} % for professional tables

% hyperref makes hyperlinks in the resulting PDF.
% If your build breaks (sometimes temporarily if a hyperlink spans a page)
% please comment out the following usepackage line and replace
% \usepackage{icml2025} with \usepackage[nohyperref]{icml2025} above.
\usepackage{hyperref}
\usepackage{dashrule}
\usepackage{arydshln}


% Attempt to make hyperref and algorithmic work together better:
\newcommand{\theHalgorithm}{\arabic{algorithm}}

% Use the following line for the initial blind version submitted for review:
% \usepackage{icml2025}

% If accepted, instead use the following line for the camera-ready submission:
\usepackage[accepted]{icml2025}

% For theorems and such
\usepackage{amsmath}
\usepackage{amssymb}
\usepackage{mathtools}
\usepackage{amsthm}
\usepackage{tabularray}
\usepackage{multirow} 
\usepackage{threeparttable}
\usepackage[most]{tcolorbox}
\usepackage{caption}
\usepackage{wrapfig}
\usepackage{amssymb}
\usepackage{xcolor}
% \usepackage{multicol}
% if you use cleveref..
\usepackage[capitalize,noabbrev]{cleveref}


%%%%%%%%%%%%%%%%%%%%%%%%%%%%%%%%
% THEOREMS
%%%%%%%%%%%%%%%%%%%%%%%%%%%%%%%%
\theoremstyle{plain}
\newtheorem{theorem}{Theorem}[section]
\newtheorem{proposition}[theorem]{Proposition}
\newtheorem{lemma}[theorem]{Lemma}
\newtheorem{corollary}[theorem]{Corollary}
\theoremstyle{definition}
\newtheorem{definition}[theorem]{Definition}
\newtheorem{assumption}[theorem]{Assumption}
\theoremstyle{remark}
\newtheorem{remark}[theorem]{Remark}

% Todonotes is useful during development; simply uncomment the next line
%    and comment out the line below the next line to turn off comments
%\usepackage[disable,textsize=tiny]{todonotes}
\usepackage[textsize=tiny]{todonotes}

\newcount\Comments  % 0 suppresses notes to selves in text
\Comments = 1
\newcommand{\kibitz}[2]{\ifnum\Comments=1{\color{#1}{#2}}\fi}
\newcommand{\tw}[1]{\kibitzAdd{tw}{[Tonghan: #1]}}
\definecolor{tw}{rgb}{0.0, 0.0, 0.5}
\newcommand{\twadd}[1]{\kibitz{blue}{#1}}


%%%%% NEW MATH DEFINITIONS %%%%%

\usepackage{amsmath,amsfonts,bm}
\usepackage{derivative}
% Mark sections of captions for referring to divisions of figures
\newcommand{\figleft}{{\em (Left)}}
\newcommand{\figcenter}{{\em (Center)}}
\newcommand{\figright}{{\em (Right)}}
\newcommand{\figtop}{{\em (Top)}}
\newcommand{\figbottom}{{\em (Bottom)}}
\newcommand{\captiona}{{\em (a)}}
\newcommand{\captionb}{{\em (b)}}
\newcommand{\captionc}{{\em (c)}}
\newcommand{\captiond}{{\em (d)}}

% Highlight a newly defined term
\newcommand{\newterm}[1]{{\bf #1}}

% Derivative d 
\newcommand{\deriv}{{\mathrm{d}}}

% Figure reference, lower-case.
\def\figref#1{figure~\ref{#1}}
% Figure reference, capital. For start of sentence
\def\Figref#1{Figure~\ref{#1}}
\def\twofigref#1#2{figures \ref{#1} and \ref{#2}}
\def\quadfigref#1#2#3#4{figures \ref{#1}, \ref{#2}, \ref{#3} and \ref{#4}}
% Section reference, lower-case.
\def\secref#1{section~\ref{#1}}
% Section reference, capital.
\def\Secref#1{Section~\ref{#1}}
% Reference to two sections.
\def\twosecrefs#1#2{sections \ref{#1} and \ref{#2}}
% Reference to three sections.
\def\secrefs#1#2#3{sections \ref{#1}, \ref{#2} and \ref{#3}}
% Reference to an equation, lower-case.
\def\eqref#1{equation~\ref{#1}}
% Reference to an equation, upper case
\def\Eqref#1{Equation~\ref{#1}}
% A raw reference to an equation---avoid using if possible
\def\plaineqref#1{\ref{#1}}
% Reference to a chapter, lower-case.
\def\chapref#1{chapter~\ref{#1}}
% Reference to an equation, upper case.
\def\Chapref#1{Chapter~\ref{#1}}
% Reference to a range of chapters
\def\rangechapref#1#2{chapters\ref{#1}--\ref{#2}}
% Reference to an algorithm, lower-case.
\def\algref#1{algorithm~\ref{#1}}
% Reference to an algorithm, upper case.
\def\Algref#1{Algorithm~\ref{#1}}
\def\twoalgref#1#2{algorithms \ref{#1} and \ref{#2}}
\def\Twoalgref#1#2{Algorithms \ref{#1} and \ref{#2}}
% Reference to a part, lower case
\def\partref#1{part~\ref{#1}}
% Reference to a part, upper case
\def\Partref#1{Part~\ref{#1}}
\def\twopartref#1#2{parts \ref{#1} and \ref{#2}}

\def\ceil#1{\lceil #1 \rceil}
\def\floor#1{\lfloor #1 \rfloor}
\def\1{\bm{1}}
\newcommand{\train}{\mathcal{D}}
\newcommand{\valid}{\mathcal{D_{\mathrm{valid}}}}
\newcommand{\test}{\mathcal{D_{\mathrm{test}}}}

\def\eps{{\epsilon}}


% Random variables
\def\reta{{\textnormal{$\eta$}}}
\def\ra{{\textnormal{a}}}
\def\rb{{\textnormal{b}}}
\def\rc{{\textnormal{c}}}
\def\rd{{\textnormal{d}}}
\def\re{{\textnormal{e}}}
\def\rf{{\textnormal{f}}}
\def\rg{{\textnormal{g}}}
\def\rh{{\textnormal{h}}}
\def\ri{{\textnormal{i}}}
\def\rj{{\textnormal{j}}}
\def\rk{{\textnormal{k}}}
\def\rl{{\textnormal{l}}}
% rm is already a command, just don't name any random variables m
\def\rn{{\textnormal{n}}}
\def\ro{{\textnormal{o}}}
\def\rp{{\textnormal{p}}}
\def\rq{{\textnormal{q}}}
\def\rr{{\textnormal{r}}}
\def\rs{{\textnormal{s}}}
\def\rt{{\textnormal{t}}}
\def\ru{{\textnormal{u}}}
\def\rv{{\textnormal{v}}}
\def\rw{{\textnormal{w}}}
\def\rx{{\textnormal{x}}}
\def\ry{{\textnormal{y}}}
\def\rz{{\textnormal{z}}}

% Random vectors
\def\rvepsilon{{\mathbf{\epsilon}}}
\def\rvphi{{\mathbf{\phi}}}
\def\rvtheta{{\mathbf{\theta}}}
\def\rva{{\mathbf{a}}}
\def\rvb{{\mathbf{b}}}
\def\rvc{{\mathbf{c}}}
\def\rvd{{\mathbf{d}}}
\def\rve{{\mathbf{e}}}
\def\rvf{{\mathbf{f}}}
\def\rvg{{\mathbf{g}}}
\def\rvh{{\mathbf{h}}}
\def\rvu{{\mathbf{i}}}
\def\rvj{{\mathbf{j}}}
\def\rvk{{\mathbf{k}}}
\def\rvl{{\mathbf{l}}}
\def\rvm{{\mathbf{m}}}
\def\rvn{{\mathbf{n}}}
\def\rvo{{\mathbf{o}}}
\def\rvp{{\mathbf{p}}}
\def\rvq{{\mathbf{q}}}
\def\rvr{{\mathbf{r}}}
\def\rvs{{\mathbf{s}}}
\def\rvt{{\mathbf{t}}}
\def\rvu{{\mathbf{u}}}
\def\rvv{{\mathbf{v}}}
\def\rvw{{\mathbf{w}}}
\def\rvx{{\mathbf{x}}}
\def\rvy{{\mathbf{y}}}
\def\rvz{{\mathbf{z}}}

% Elements of random vectors
\def\erva{{\textnormal{a}}}
\def\ervb{{\textnormal{b}}}
\def\ervc{{\textnormal{c}}}
\def\ervd{{\textnormal{d}}}
\def\erve{{\textnormal{e}}}
\def\ervf{{\textnormal{f}}}
\def\ervg{{\textnormal{g}}}
\def\ervh{{\textnormal{h}}}
\def\ervi{{\textnormal{i}}}
\def\ervj{{\textnormal{j}}}
\def\ervk{{\textnormal{k}}}
\def\ervl{{\textnormal{l}}}
\def\ervm{{\textnormal{m}}}
\def\ervn{{\textnormal{n}}}
\def\ervo{{\textnormal{o}}}
\def\ervp{{\textnormal{p}}}
\def\ervq{{\textnormal{q}}}
\def\ervr{{\textnormal{r}}}
\def\ervs{{\textnormal{s}}}
\def\ervt{{\textnormal{t}}}
\def\ervu{{\textnormal{u}}}
\def\ervv{{\textnormal{v}}}
\def\ervw{{\textnormal{w}}}
\def\ervx{{\textnormal{x}}}
\def\ervy{{\textnormal{y}}}
\def\ervz{{\textnormal{z}}}

% Random matrices
\def\rmA{{\mathbf{A}}}
\def\rmB{{\mathbf{B}}}
\def\rmC{{\mathbf{C}}}
\def\rmD{{\mathbf{D}}}
\def\rmE{{\mathbf{E}}}
\def\rmF{{\mathbf{F}}}
\def\rmG{{\mathbf{G}}}
\def\rmH{{\mathbf{H}}}
\def\rmI{{\mathbf{I}}}
\def\rmJ{{\mathbf{J}}}
\def\rmK{{\mathbf{K}}}
\def\rmL{{\mathbf{L}}}
\def\rmM{{\mathbf{M}}}
\def\rmN{{\mathbf{N}}}
\def\rmO{{\mathbf{O}}}
\def\rmP{{\mathbf{P}}}
\def\rmQ{{\mathbf{Q}}}
\def\rmR{{\mathbf{R}}}
\def\rmS{{\mathbf{S}}}
\def\rmT{{\mathbf{T}}}
\def\rmU{{\mathbf{U}}}
\def\rmV{{\mathbf{V}}}
\def\rmW{{\mathbf{W}}}
\def\rmX{{\mathbf{X}}}
\def\rmY{{\mathbf{Y}}}
\def\rmZ{{\mathbf{Z}}}

% Elements of random matrices
\def\ermA{{\textnormal{A}}}
\def\ermB{{\textnormal{B}}}
\def\ermC{{\textnormal{C}}}
\def\ermD{{\textnormal{D}}}
\def\ermE{{\textnormal{E}}}
\def\ermF{{\textnormal{F}}}
\def\ermG{{\textnormal{G}}}
\def\ermH{{\textnormal{H}}}
\def\ermI{{\textnormal{I}}}
\def\ermJ{{\textnormal{J}}}
\def\ermK{{\textnormal{K}}}
\def\ermL{{\textnormal{L}}}
\def\ermM{{\textnormal{M}}}
\def\ermN{{\textnormal{N}}}
\def\ermO{{\textnormal{O}}}
\def\ermP{{\textnormal{P}}}
\def\ermQ{{\textnormal{Q}}}
\def\ermR{{\textnormal{R}}}
\def\ermS{{\textnormal{S}}}
\def\ermT{{\textnormal{T}}}
\def\ermU{{\textnormal{U}}}
\def\ermV{{\textnormal{V}}}
\def\ermW{{\textnormal{W}}}
\def\ermX{{\textnormal{X}}}
\def\ermY{{\textnormal{Y}}}
\def\ermZ{{\textnormal{Z}}}

% Vectors
\def\vzero{{\bm{0}}}
\def\vone{{\bm{1}}}
\def\vmu{{\bm{\mu}}}
\def\vtheta{{\bm{\theta}}}
\def\vphi{{\bm{\phi}}}
\def\va{{\bm{a}}}
\def\vb{{\bm{b}}}
\def\vc{{\bm{c}}}
\def\vd{{\bm{d}}}
\def\ve{{\bm{e}}}
\def\vf{{\bm{f}}}
\def\vg{{\bm{g}}}
\def\vh{{\bm{h}}}
\def\vi{{\bm{i}}}
\def\vj{{\bm{j}}}
\def\vk{{\bm{k}}}
\def\vl{{\bm{l}}}
\def\vm{{\bm{m}}}
\def\vn{{\bm{n}}}
\def\vo{{\bm{o}}}
\def\vp{{\bm{p}}}
\def\vq{{\bm{q}}}
\def\vr{{\bm{r}}}
\def\vs{{\bm{s}}}
\def\vt{{\bm{t}}}
\def\vu{{\bm{u}}}
\def\vv{{\bm{v}}}
\def\vw{{\bm{w}}}
\def\vx{{\bm{x}}}
\def\vy{{\bm{y}}}
\def\vz{{\bm{z}}}

% Elements of vectors
\def\evalpha{{\alpha}}
\def\evbeta{{\beta}}
\def\evepsilon{{\epsilon}}
\def\evlambda{{\lambda}}
\def\evomega{{\omega}}
\def\evmu{{\mu}}
\def\evpsi{{\psi}}
\def\evsigma{{\sigma}}
\def\evtheta{{\theta}}
\def\eva{{a}}
\def\evb{{b}}
\def\evc{{c}}
\def\evd{{d}}
\def\eve{{e}}
\def\evf{{f}}
\def\evg{{g}}
\def\evh{{h}}
\def\evi{{i}}
\def\evj{{j}}
\def\evk{{k}}
\def\evl{{l}}
\def\evm{{m}}
\def\evn{{n}}
\def\evo{{o}}
\def\evp{{p}}
\def\evq{{q}}
\def\evr{{r}}
\def\evs{{s}}
\def\evt{{t}}
\def\evu{{u}}
\def\evv{{v}}
\def\evw{{w}}
\def\evx{{x}}
\def\evy{{y}}
\def\evz{{z}}

% Matrix
\def\mA{{\bm{A}}}
\def\mB{{\bm{B}}}
\def\mC{{\bm{C}}}
\def\mD{{\bm{D}}}
\def\mE{{\bm{E}}}
\def\mF{{\bm{F}}}
\def\mG{{\bm{G}}}
\def\mH{{\bm{H}}}
\def\mI{{\bm{I}}}
\def\mJ{{\bm{J}}}
\def\mK{{\bm{K}}}
\def\mL{{\bm{L}}}
\def\mM{{\bm{M}}}
\def\mN{{\bm{N}}}
\def\mO{{\bm{O}}}
\def\mP{{\bm{P}}}
\def\mQ{{\bm{Q}}}
\def\mR{{\bm{R}}}
\def\mS{{\bm{S}}}
\def\mT{{\bm{T}}}
\def\mU{{\bm{U}}}
\def\mV{{\bm{V}}}
\def\mW{{\bm{W}}}
\def\mX{{\bm{X}}}
\def\mY{{\bm{Y}}}
\def\mZ{{\bm{Z}}}
\def\mBeta{{\bm{\beta}}}
\def\mPhi{{\bm{\Phi}}}
\def\mLambda{{\bm{\Lambda}}}
\def\mSigma{{\bm{\Sigma}}}

% Tensor
\DeclareMathAlphabet{\mathsfit}{\encodingdefault}{\sfdefault}{m}{sl}
\SetMathAlphabet{\mathsfit}{bold}{\encodingdefault}{\sfdefault}{bx}{n}
\newcommand{\tens}[1]{\bm{\mathsfit{#1}}}
\def\tA{{\tens{A}}}
\def\tB{{\tens{B}}}
\def\tC{{\tens{C}}}
\def\tD{{\tens{D}}}
\def\tE{{\tens{E}}}
\def\tF{{\tens{F}}}
\def\tG{{\tens{G}}}
\def\tH{{\tens{H}}}
\def\tI{{\tens{I}}}
\def\tJ{{\tens{J}}}
\def\tK{{\tens{K}}}
\def\tL{{\tens{L}}}
\def\tM{{\tens{M}}}
\def\tN{{\tens{N}}}
\def\tO{{\tens{O}}}
\def\tP{{\tens{P}}}
\def\tQ{{\tens{Q}}}
\def\tR{{\tens{R}}}
\def\tS{{\tens{S}}}
\def\tT{{\tens{T}}}
\def\tU{{\tens{U}}}
\def\tV{{\tens{V}}}
\def\tW{{\tens{W}}}
\def\tX{{\tens{X}}}
\def\tY{{\tens{Y}}}
\def\tZ{{\tens{Z}}}


% Graph
\def\gA{{\mathcal{A}}}
\def\gB{{\mathcal{B}}}
\def\gC{{\mathcal{C}}}
\def\gD{{\mathcal{D}}}
\def\gE{{\mathcal{E}}}
\def\gF{{\mathcal{F}}}
\def\gG{{\mathcal{G}}}
\def\gH{{\mathcal{H}}}
\def\gI{{\mathcal{I}}}
\def\gJ{{\mathcal{J}}}
\def\gK{{\mathcal{K}}}
\def\gL{{\mathcal{L}}}
\def\gM{{\mathcal{M}}}
\def\gN{{\mathcal{N}}}
\def\gO{{\mathcal{O}}}
\def\gP{{\mathcal{P}}}
\def\gQ{{\mathcal{Q}}}
\def\gR{{\mathcal{R}}}
\def\gS{{\mathcal{S}}}
\def\gT{{\mathcal{T}}}
\def\gU{{\mathcal{U}}}
\def\gV{{\mathcal{V}}}
\def\gW{{\mathcal{W}}}
\def\gX{{\mathcal{X}}}
\def\gY{{\mathcal{Y}}}
\def\gZ{{\mathcal{Z}}}

% Sets
\def\sA{{\mathbb{A}}}
\def\sB{{\mathbb{B}}}
\def\sC{{\mathbb{C}}}
\def\sD{{\mathbb{D}}}
% Don't use a set called E, because this would be the same as our symbol
% for expectation.
\def\sF{{\mathbb{F}}}
\def\sG{{\mathbb{G}}}
\def\sH{{\mathbb{H}}}
\def\sI{{\mathbb{I}}}
\def\sJ{{\mathbb{J}}}
\def\sK{{\mathbb{K}}}
\def\sL{{\mathbb{L}}}
\def\sM{{\mathbb{M}}}
\def\sN{{\mathbb{N}}}
\def\sO{{\mathbb{O}}}
\def\sP{{\mathbb{P}}}
\def\sQ{{\mathbb{Q}}}
\def\sR{{\mathbb{R}}}
\def\sS{{\mathbb{S}}}
\def\sT{{\mathbb{T}}}
\def\sU{{\mathbb{U}}}
\def\sV{{\mathbb{V}}}
\def\sW{{\mathbb{W}}}
\def\sX{{\mathbb{X}}}
\def\sY{{\mathbb{Y}}}
\def\sZ{{\mathbb{Z}}}

% Entries of a matrix
\def\emLambda{{\Lambda}}
\def\emA{{A}}
\def\emB{{B}}
\def\emC{{C}}
\def\emD{{D}}
\def\emE{{E}}
\def\emF{{F}}
\def\emG{{G}}
\def\emH{{H}}
\def\emI{{I}}
\def\emJ{{J}}
\def\emK{{K}}
\def\emL{{L}}
\def\emM{{M}}
\def\emN{{N}}
\def\emO{{O}}
\def\emP{{P}}
\def\emQ{{Q}}
\def\emR{{R}}
\def\emS{{S}}
\def\emT{{T}}
\def\emU{{U}}
\def\emV{{V}}
\def\emW{{W}}
\def\emX{{X}}
\def\emY{{Y}}
\def\emZ{{Z}}
\def\emSigma{{\Sigma}}

% entries of a tensor
% Same font as tensor, without \bm wrapper
\newcommand{\etens}[1]{\mathsfit{#1}}
\def\etLambda{{\etens{\Lambda}}}
\def\etA{{\etens{A}}}
\def\etB{{\etens{B}}}
\def\etC{{\etens{C}}}
\def\etD{{\etens{D}}}
\def\etE{{\etens{E}}}
\def\etF{{\etens{F}}}
\def\etG{{\etens{G}}}
\def\etH{{\etens{H}}}
\def\etI{{\etens{I}}}
\def\etJ{{\etens{J}}}
\def\etK{{\etens{K}}}
\def\etL{{\etens{L}}}
\def\etM{{\etens{M}}}
\def\etN{{\etens{N}}}
\def\etO{{\etens{O}}}
\def\etP{{\etens{P}}}
\def\etQ{{\etens{Q}}}
\def\etR{{\etens{R}}}
\def\etS{{\etens{S}}}
\def\etT{{\etens{T}}}
\def\etU{{\etens{U}}}
\def\etV{{\etens{V}}}
\def\etW{{\etens{W}}}
\def\etX{{\etens{X}}}
\def\etY{{\etens{Y}}}
\def\etZ{{\etens{Z}}}

% The true underlying data generating distribution
\newcommand{\pdata}{p_{\rm{data}}}
\newcommand{\ptarget}{p_{\rm{target}}}
\newcommand{\pprior}{p_{\rm{prior}}}
\newcommand{\pbase}{p_{\rm{base}}}
\newcommand{\pref}{p_{\rm{ref}}}

% The empirical distribution defined by the training set
\newcommand{\ptrain}{\hat{p}_{\rm{data}}}
\newcommand{\Ptrain}{\hat{P}_{\rm{data}}}
% The model distribution
\newcommand{\pmodel}{p_{\rm{model}}}
\newcommand{\Pmodel}{P_{\rm{model}}}
\newcommand{\ptildemodel}{\tilde{p}_{\rm{model}}}
% Stochastic autoencoder distributions
\newcommand{\pencode}{p_{\rm{encoder}}}
\newcommand{\pdecode}{p_{\rm{decoder}}}
\newcommand{\precons}{p_{\rm{reconstruct}}}

\newcommand{\laplace}{\mathrm{Laplace}} % Laplace distribution

\newcommand{\E}{\mathbb{E}}
\newcommand{\Ls}{\mathcal{L}}
\newcommand{\R}{\mathbb{R}}
\newcommand{\emp}{\tilde{p}}
\newcommand{\lr}{\alpha}
\newcommand{\reg}{\lambda}
\newcommand{\rect}{\mathrm{rectifier}}
\newcommand{\softmax}{\mathrm{softmax}}
\newcommand{\sigmoid}{\sigma}
\newcommand{\softplus}{\zeta}
\newcommand{\KL}{D_{\mathrm{KL}}}
\newcommand{\Var}{\mathrm{Var}}
\newcommand{\standarderror}{\mathrm{SE}}
\newcommand{\Cov}{\mathrm{Cov}}
% Wolfram Mathworld says $L^2$ is for function spaces and $\ell^2$ is for vectors
% But then they seem to use $L^2$ for vectors throughout the site, and so does
% wikipedia.
\newcommand{\normlzero}{L^0}
\newcommand{\normlone}{L^1}
\newcommand{\normltwo}{L^2}
\newcommand{\normlp}{L^p}
\newcommand{\normmax}{L^\infty}

\newcommand{\parents}{Pa} % See usage in notation.tex. Chosen to match Daphne's book.

\DeclareMathOperator*{\argmax}{arg\,max}
\DeclareMathOperator*{\argmin}{arg\,min}

\DeclareMathOperator{\sign}{sign}
\DeclareMathOperator{\Tr}{Tr}
\let\ab\allowbreak

\definecolor{usercolor}{RGB}{0, 0, 128}       % Navy Blue for User
\definecolor{cotcolor}{RGB}{128, 0, 0}        % Maroon for o1 CoT
\definecolor{outputcolor}{RGB}{0, 128, 0}     % Green for o1 Output
\newcommand{\shortn}{\textup{\texttt{-}}}
\newcommand{\shorte}{\textup{\texttt{=}}}
\newcommand{\shortp}{\textup{\texttt{+}}}
\newcommand{\shortl}{\textup{\texttt{<}}}
\newcommand{\shortg}{\textup{\texttt{>}}}
\newcommand{\ie}{\textit{i}.\textit{e}.}
\newcommand{\eg}{\textit{e}.\textit{g}.}
\newcommand{\etal}{\textit{et al}.}
\newcommand{\etc}{\textit{etc}.}
\newcommand{\Tau}{\mathrm{T}}
\newcommand{\rlm}{\textsc{Guidance LLM}}
\newcommand{\elm}{\textsc{Explanation LLM}}
\newcommand{\caquerylm}[1]{Q^{\textsc{LM}}_{#1}}
\newcommand{\caqueryflow}[1]{Q^{\textsc{Flow}}_{#1}}
\newcommand{\name}{\textsc{OrdinaryNet}}
\NewDocumentCommand{\idx}{o o o o}{\ensuremath{
{#1}
\IfValueT{#2}{\IfBlankTF{#2}{}{_{#2}}}
\IfValueT{#3}{\IfBlankTF{#3}{}{^{(#3)}}}
\IfValueT{#4}{\IfBlankTF{#4}{}{(#4)}}
}}

% The \icmltitle you define below is probably too long as a header.
% Therefore, a short form for the running title is supplied here:
\icmltitlerunning{}%Train LLMs with Diffusion Rewards

\begin{document}

\twocolumn[
\icmltitle{Policy-to-Language:\\ Train LLMs to Explain Decisions with Flow-Matching Generated Rewards
%Train LLMs with Diffusion Rewards for Explaining Decisions
}

% It is OKAY to include author information, even for blind
% submissions: the style file will automatically remove it for you
% unless you've provided the [accepted] option to the icml2025
% package.

% List of affiliations: The first argument should be a (short)
% identifier you will use later to specify author affiliations
% Academic affiliations should list Department, University, City, Region, Country
% Industry affiliations should list Company, City, Region, Country

% You can specify symbols, otherwise they are numbered in order.
% Ideally, you should not use this facility. Affiliations will be numbered
% in order of appearance and this is the preferred way.
\icmlsetsymbol{equal}{*}

\begin{icmlauthorlist}
\icmlauthor{Xinyi Yang}{yyy}
\icmlauthor{Liang Zeng}{xxx}
\icmlauthor{Heng Dong}{xxx}
\icmlauthor{Chao Yu}{yyy}
\icmlauthor{Xiaoran Wu}{zzz}
\icmlauthor{Huazhong Yang}{yyy}
\icmlauthor{Yu Wang}{yyy}
%\icmlauthor{}{sch}
\icmlauthor{Milind Tambe}{sch}
\icmlauthor{Tonghan Wang}{sch}
%\icmlauthor{}{sch}
%\icmlauthor{}{sch}
\end{icmlauthorlist}

\icmlaffiliation{yyy}{Department of Electronic Engineering, Tsinghua University, Beijing, China}
\icmlaffiliation{xxx}{Institute for Interdisciplinary Information Sciences, Tsinghua University, Beijing, China}
% \icmlaffiliation{comp}{Skywork AI, Beijing, China}
\icmlaffiliation{zzz}{Department of Computer Science and Technology, Tsinghua University, Beijing, China}
\icmlaffiliation{sch}{Harvard John A. Paulson School of Engineering and Applied Sciences, Harvard University, Cambridge, USA}

% \icmlcorrespondingauthor{Xinyi Yang}{first1.last1@xxx.edu}
\icmlcorrespondingauthor{Tonghan Wang}{twang1@g.harvard.edu}

% You may provide any keywords that you
% find helpful for describing your paper; these are used to populate
% the "keywords" metadata in the PDF but will not be shown in the document
\icmlkeywords{Machine Learning, ICML}

\vskip 0.3in
]

% this must go after the closing bracket ] following \twocolumn[ ...

% This command actually creates the footnote in the first column
% listing the affiliations and the copyright notice.
% The command takes one argument, which is text to display at the start of the footnote.
% The \icmlEqualContribution command is standard text for equal contribution.
% Remove it (just {}) if you do not need this facility.

\printAffiliationsAndNotice{}  % leave blank if no need to mention equal contribution
% \printAffiliationsAndNotice{\icmlEqualContribution} % otherwise use the standard text.

\begin{abstract}
Recent advancements in 3D multi-object tracking (3D MOT) have predominantly relied on tracking-by-detection pipelines. However, these approaches often neglect potential enhancements in 3D detection processes, leading to high false positives (FP), missed detections (FN), and identity switches (IDS), particularly in challenging scenarios such as crowded scenes, small-object configurations, and adverse weather conditions. Furthermore, limitations in data preprocessing, association mechanisms, motion modeling, and life-cycle management hinder overall tracking robustness. To address these issues, we present \textbf{Easy-Poly}, a real-time, filter-based 3D MOT framework for multiple object categories. Our contributions include: (1) An \textit{Augmented Proposal Generator} utilizing multi-modal data augmentation and refined SpConv operations, significantly improving mAP and NDS on nuScenes; (2) A \textbf{Dynamic Track-Oriented (DTO)} data association algorithm that effectively manages uncertainties and occlusions through optimal assignment and multiple hypothesis handling; (3) A \textbf{Dynamic Motion Modeling (DMM)} incorporating a confidence-weighted Kalman filter and adaptive noise covariances, enhancing MOTA and AMOTA in challenging conditions; and (4) An extended life-cycle management system with adjustive thresholds to reduce ID switches and false terminations. Experimental results show that Easy-Poly outperforms state-of-the-art methods such as Poly-MOT and Fast-Poly~\cite{li2024fast}, achieving notable gains in mAP (e.g., from 63.30\% to 64.96\% with LargeKernel3D) and AMOTA (e.g., from 73.1\% to 74.5\%), while also running in real-time. These findings highlight Easy-Poly's adaptability and robustness in diverse scenarios, making it a compelling choice for autonomous driving and related 3D MOT applications. The source code of this paper will be published upon acceptance.

% 3D Multi-Object Tracking (MOT) is essential for autonomous driving systems, contributing significantly to vehicle safety and navigation. Despite recent advancements, existing 3D tracking methods still face significant challenges in accuracy, particularly when dealing with small objects, crowded environments, and adverse weather conditions. To overcome these challenges, we propose \textbf{Easy-Poly}, a novel and efficient multi-modal 3D MOT framework. \textbf{Easy-Poly} employs the Focal Sparse Convolution (\textbf{FocalsConv}) model for object detection. By optimizing convolution operations and augmenting data with multiple modalities, we significantly enhance detection precision.
% \textbf{Easy-Poly} introduces several key innovations: (1) an optimized Kalman filter in the pre-processing stage, (2) integration of the Dynamic Track-Oriented (\textbf{DTO}) Data Association algorithm with confidence-weighted motion models for data association, (3) Dynamic Motion Modeling (\textbf{DMM}) with Adaptive Noise Covariances, and (4) enhanced trajectory management throughout the tracking life-cycle. These improvements increase the robustness and efficiency of tracking, especially in complex scenarios such as crowded scenes and challenging weather conditions. Experimental results on the \textbf{nuScenes} dataset demonstrate that in the pre-processing stage of \textbf{Easy-Poly}, the optimized \textbf{FocalsConv} model achieves a mean Average Precision (mAP) of \textbf{64.96\%} for object detection. Furthermore, the multi-object tracking performance reaches a high AMOTA of \textbf{75.0\%}, surpassing existing methods across multiple performance metrics.
 
% Code and data are available at \textcolor{blue}{\textit{\url{https://github.com/zhangpengtom/FocalsConvPlus}}} and  \textcolor{blue}
%  \textit{\url{https://github.com/zhangpengtom/EasyPoly}.}
%  } 

\end{abstract}
\section{Introduction}

Deep Reinforcement Learning (DRL) has emerged as a transformative paradigm for solving complex sequential decision-making problems. By enabling autonomous agents to interact with an environment, receive feedback in the form of rewards, and iteratively refine their policies, DRL has demonstrated remarkable success across a diverse range of domains including games (\eg Atari~\citep{mnih2013playing,kaiser2020model}, Go~\citep{silver2018general,silver2017mastering}, and StarCraft II~\citep{vinyals2019grandmaster,vinyals2017starcraft}), robotics~\citep{kalashnikov2018scalable}, communication networks~\citep{feriani2021single}, and finance~\citep{liu2024dynamic}. These successes underscore DRL's capability to surpass traditional rule-based systems, particularly in high-dimensional and dynamically evolving environments.

Despite these advances, a fundamental challenge remains: DRL agents typically rely on deep neural networks, which operate as black-box models, obscuring the rationale behind their decision-making processes. This opacity poses significant barriers to adoption in safety-critical and high-stakes applications, where interpretability is crucial for trust, compliance, and debugging. The lack of transparency in DRL can lead to unreliable decision-making, rendering it unsuitable for domains where explainability is a prerequisite, such as healthcare, autonomous driving, and financial risk assessment.

To address these concerns, the field of Explainable Deep Reinforcement Learning (XRL) has emerged, aiming to develop techniques that enhance the interpretability of DRL policies. XRL seeks to provide insights into an agent’s decision-making process, enabling researchers, practitioners, and end-users to understand, validate, and refine learned policies. By facilitating greater transparency, XRL contributes to the development of safer, more robust, and ethically aligned AI systems.

Furthermore, the increasing integration of Reinforcement Learning (RL) with Large Language Models (LLMs) has placed RL at the forefront of natural language processing (NLP) advancements. Methods such as Reinforcement Learning from Human Feedback (RLHF)~\citep{bai2022training,ouyang2022training} have become essential for aligning LLM outputs with human preferences and ethical guidelines. By treating language generation as a sequential decision-making process, RL-based fine-tuning enables LLMs to optimize for attributes such as factual accuracy, coherence, and user satisfaction, surpassing conventional supervised learning techniques. However, the application of RL in LLM alignment further amplifies the explainability challenge, as the complex interactions between RL updates and neural representations remain poorly understood.

This survey provides a systematic review of explainability methods in DRL, with a particular focus on their integration with LLMs and human-in-the-loop systems. We first introduce fundamental RL concepts and highlight key advances in DRL. We then categorize and analyze existing explanation techniques, encompassing feature-level, state-level, dataset-level, and model-level approaches. Additionally, we discuss methods for evaluating XRL techniques, considering both qualitative and quantitative assessment criteria. Finally, we explore real-world applications of XRL, including policy refinement, adversarial attack mitigation, and emerging challenges in ensuring interpretability in modern AI systems. Through this survey, we aim to provide a comprehensive perspective on the current state of XRL and outline future research directions to advance the development of interpretable and trustworthy DRL models.
\section{Preliminaries}

\textbf{Sealed-Bid Combinatorial Auction}. We consider sealed-bid CAs with a single bidder and $m$ items, $M=\{1,\ldots,m\}$.
The bidder has a {\em valuation function}, $v: 2^M \rightarrow \mathbb{R}_{\ge 0}$. Valuation $v$ is drawn independently from a distribution $F$ defined on the space of possible valuation functions $V$, determining how valuable each bundle $S\in 2^M$ is for the bidder. We consider bounded valuation functions: $v(S)\in[0, v_{\max}]$, $S\subset 2^M$, with $v_{\max}>0$, and they are normalized so that $v(\varnothing)=0$.
%
%\forall v_i\in \supp(F_i)$} \dcp{i just realized that we need a bounded domain $V_i$ for a grid to be well defined, right? comment on this.}
% We use $\vv$ to denote the value of the bidder for each of the $2^m$ bundles. 
The auctioneer  knows distribution $F$ but not the  valuation  $v$. The bidder reports their valuation function, perhaps untruthfully, as their {\em bid (function)}, $b\in V$. 

In CAs, a suitable {\em bidding language} is critical to allow a bidder to report their
bid without needing to enumerate a value for every possible bundle.  There
are many ways to do this, but a  common approach is to use the {\em XOR bidding language}, which allows bidders to submit bid prices for each of multiple bundles under an exclusive-or condition; in effect, only one bid price on a bundle can be accepted. Popular CA testbeds such as CATS~\citep{leyton2000towards} and SATS~\citep{weiss2017sats} employ this bidding language extensively.\footnote{When representing the values of multiple bidders these testbeds often also introduce so-called dummy items for distinguishing the bids of different 
bidders. Still, the semantics for a single bidder is, in effect, that of the XOR language.}
The semantics of the XOR bidding language is that the value on a bundle $S$ is the maximum bid price on
any bundle $S'$, submitted as part of the XOR bid, and for which $S'\subseteq S$. XOR bids are 
succinct for valuation functions in which the bidder is only interested in a bounded number
of possible bundles.

We seek an auction $(g,p)$ that maximizes expected revenue. Here, $g: V\rightarrow \mathcal{X}$ is the {\em allocation rule},  where $\mathcal{X}$ is the space of feasible allocations (i.e., no item allocated more than once), so that $g(b)\subseteq M$ denotes the set of items (perhaps empty) allocated to the bidder at bid $b$.
%
Also, $p: V\rightarrow \mathbb{R}_{\ge 0}$ is the {\em payment rule},
specifying the price associated with allocation $g(b)$. 
 %
The utility to the bidder with valuation function $v$ at bid  $b$ is $u(v;b)=v(g(b))-p(b)$, which is
the standard model of quasi-linearity so that values are in effect quantified in monetary units, say dollars.
%
 In full generality, the allocation and payment rules may be \emph{randomized}, with
 the bidder assumed to be risk neutral  and seeking
to maximize their 
expected utility.


In a \emph{dominant-strategy incentive compatible} (DSIC) auction, or {\em strategy-proof (SP)} auction, the bidder's utility is  maximized by bidding their true valuation $v$, whatever this valuation is; i.e., $u(v; v)\ge u(v;b)$, for $\forall v\in V, \forall b\in V$.
An auction is \emph{individually rational} (IR) if the bidder receives a non-negative utility when participating and truthfully reporting: $u(v;v)\ge 0$, for $\forall v\in V$.
Following the revelation principle, it is without loss of generality to focus on  
SP
auctions, as any auction that achieves a particular expected revenue in a dominant-strategy equilibrium
 can be transformed into an SP auction with the same expected revenue.
 %
 Optimal auction design therefore seeks to identify an SP and IR auction that maximizes the expected revenue, i.e., $\mathbb{E}_{v\sim \bm F}[p(v)]$. 

\textbf{Menu-Based CAs}. In a {\em menu-based auction}, allocation and payment rules  are
represented through a menu, $B$, consisting of
$K\ge 1$  {\em menu elements}.
%
We write $B=(B^{(1)},\ldots,B^{(K)})$, 
and the $k$th \emph{menu element}, $B^{(k)}$,
 specifies allocation probabilities on bundles,
 $\alpha^{(k)}: 2^M\rightarrow [0,1]$, and a {\em price}, $\beta^{(k)}\in \mathbb{R}$.
%
Here, we allow randomization, where  $\alpha^{(k)}(S)\in[0,1]$ denotes the
  probability that bundle $S\in 2^M$ is assigned to the bidder in menu element $k$. 
  % this assignment made independently of the 
  % assignment of some other item $j'\neq j$ (conditioned on the choice of
  % menu element $k$).
  %
   % \dcp{i'm wondering if this independent distribution is wlog for additive and unit-demand valuations---you can ask Sai and Michael if they know about anything.}
   %
  %
We refer to the menu $B$ as corresponding to a {\em menu-based representation 
of an auction.} The bidder with bid $b$ is assigned the element from menu $B$ that maximizes their utility according to the reported valuation: $k^*\in \arg\max_k \sum_{S\in 2^M}\alpha^{(k)}(S)b(S)-\beta^{(k)}$. We denote this optimal element by $(\alpha^*(b), \beta^*(b))$. 
The use of menu-based representations for auction design 
is without loss of generality and DSIC~\cite{hammond1979straightforward}.
%\dcp{need to explain how the allocation and payment rule is defined, via arg max given bid b and the menu} \dcp{need a bit more here, defining $\beta^*$ as the optimal element from menu $B$}
%
%In the context of menu-based auction design, the 
The optimal auction design problem is to find a menu-based representation that 
maximizes  expected revenue, i.e., $\mathbb{E}_{v\sim F}[ \beta^{*}(v)]$. Deep menu-based methods~\cite{dutting2024optimal,shen2019automated} in the differentiable economics literature~\cite{zheng2022ai,finocchiaro2021bridging,wang2023deep,ivanov2024principal,zhang2024position,hossain2024multi,rahme2020auction,ivanov2022optimal,curry2022differentiable,duan2023scalable} learn to generate such menus by neural networks.
%dcp cut among all  menus of a certain size.

% \tw{which means the number of elements in a menu?} \dcp{yes; now I say `among all menus` not `among all menu representations`. ok?}.  


\textbf{Diffusion Models and Continuous Normalizing Flow}. Diffusion models have  emerged as a powerful class of generative AI methods, spurring notable advances in a wide range of tasks such as image generation~\cite{rombach2022high,esser2024scaling}, video generation~\cite{ho2022video,ceylan2023pix2video,ho2022imagen}, molecular design~\cite{gruver2024protein}, text generation~\cite{lou2024discrete}, and multi-agent learning~\cite{wang2024diffusion}. At their core, these models perform a {\em forward noising process} in which noise is incrementally added to training data over multiple steps, gradually corrupting the original sample.
%\dcp{is `signal` right? or `sample`?} \tw{My opinion is that these works are rooted in the information theory, so they use "signal" a lot.} it's only used here in our paper, so I have dropped `signal`
%
A {\em reverse diffusion process} is then learned to iteratively remove noise, thereby reconstructing data from near-random initial states. In our setting, instead of reconstructing data, we extend the diffusion process to develop a tractable and differentiable method that optimizes a high-dimensional distribution.
%\tw{Do we need this here?}
%dcp -- yes, I like it

In particular, {\em score-based diffusion models} enjoy strong mathematical and physical underpinnings. The forward noising process is an {\em Itô stochastic differential equation} (SDE),
\begin{align}
    d\vx = \vf(\vx,t)dt + h(t)d\vw,
\end{align}
%
where $\vx(t) \in\mathbb{R}^\ell$ is the {\em state} at time $t$, for some $\ell\in \mathbb{Z}_{>0}$, $\vf(\cdot,t):\mathbb{R}^\ell\rightarrow\mathbb{R}^\ell$ is the {\em drift coefficient}, $h(\cdot):\mathbb{R}\rightarrow\mathbb{R}$ is the {\em diffusion coefficient}, and $\vw$ is the {\em standard Wiener process} (Brownian motion). Different forward processes are designed by specifying functional forms for $\vf(\cdot,t)$ and $h(\cdot)$. The generation of data is then based on the reverse process, which is  a diffusion process  given by the {\em reverse-time SDE}~\cite{anderson1982reverse},
%
\begin{align}
    d\vx = [\vf(\vx,t)-h(t)^2\nabla_\vx\log q_t(\vx)]dt + h(t)d\bar{\vw},\label{equ:r-sde}
\end{align}
%
where $dt$ is an infinitesimal negative timestep,  $\bar{\vw}$ is the {\em standard Brownian motion with reversed time flow},  and {\em $q_t(\vx)$ is the distribution of state  $\vx(t)$
at time $t$}.
The principal task in diffusion models is to learn the {\em score function}, $\nabla_\vx\log q_t(\vx)$, which has been effectively achieved using neural networks in recent work. This
enables solving the reverse-time SDE and  generating new data samples.  Notably, in the diffusion model  (and more broadly, generative AI) literature, $q_0(\vx)$ is typically a known target distribution over data samples from a pre-training dateset.
%\dcp{as $\vx_T$ given samples $\vs_0$?}

The reverse-time SDE (Eq.~\ref{equ:r-sde}) can be mathematically intricate, motivating the study of an equivalent, \emph{deterministic reverse process} modeled by an ordinary differential equation (ODE),
%
\begin{align}
    d\vx = [\vf(\vx,t)-\frac{1}{2}h(t)^2\nabla_\vx\log q_t(\vx)]dt,\label{equ:r-ode}
\end{align}
%
which  preserves the same marginal probability densities $\{q_t(\vx)\}_{t=0}^T$ as the SDE in Eq.~\ref{equ:r-sde}~\cite{song2021score}.
%
% Drawbacks of the normalizing flow include that we require the function $f$ to be invertible, and working its Jacobian might be expensive. We can consider a continuous version of normalizing flow where the function $f$ is applied an infinite time. 
%
Eq.~\ref{equ:r-ode} also highlights the connection between diffusion models and \emph{continuous normalizing flow}: each of them learns to transform and manipulate distributions by an ODE. Intuitively, a {\em continuous normalizing flow} transports an input $\vx_0\in \mathbb{R}^\ell$ to $\vx_t=\phi(t, \vx_0)$ at timestep $t\in[0,T]$.
%
Here, $\phi(t, \cdot):\mathbb{R}^\ell\rightarrow\mathbb{R}^\ell$ is the  \emph{flow}, and is governed by the ODE,
%
\begin{align}
    \frac{d}{dt}\vx_t = \varphi\left(t, \vx_t\right),\label{equ:f-ode}
\end{align}
%
where the vector field $\varphi: [0,T]\times \mathbb{R}^\ell\rightarrow \mathbb{R}^\ell$ specifies the  rate of
change of the state $\vx$.
%
Continuous normalizing flow \citep{chen2018neural} suggests to represent vector field $\varphi$ with a neural network. The flow $\phi$ transforms an initial distribution $p_0(\vx)$ to a final distribution $p_T(\vx)$ an time $T$.

% and the final distribution $p(z)=p_1(z_1)$ can be obtained by solving .

\textbf{Rectified Flow}. A bottleneck that restricts the use of continuous normalizing flow in large-scale problems is that the ODE (Eq.~\ref{equ:f-ode})
is hard to solve when the vector field $\varphi$ is complex.  The {\em rectified flow}~\cite{liu2022flow} addresses this by encouraging the flow to follow the linear path:
\begin{align}
    \min_\varphi \int_0^T \mathbb{E}_{\vx_0\sim p_0(\vx),\vx_T\sim p_T(\vx)}\left[\|(\vx_T-\vx_0)-\varphi(t, \vx_t)\|^2\right]dt, \ \ \ \vx_t = t\vx_T + (1-t)\vx_0.\label{equ:rf}
\end{align}

Here, the target distribution $p_T(\vx)$ (from which $\vx_T$ are sampled) and the initial distribution $p_0(\vx)$ (from which $\vx_0$ are sampled) are known. $\vx_t,t\in[0,T]$ is the interpolated point between $\vx_T$ and $\vx_0$, and the rectified flow encourages the vector field to align as closely as possible with the straight line $\vx_T-\vx_0$.

As discussed in the introduction, the application of diffusion models or continuous normalizing flow in generative AI tasks relies on access to a known target distribution $p_T(\vx)$, but in our optimal CA design task, $p_T(\vx)$ is unknown and needs to be optimized. 


% \dcp{same comment, do you want int from $0$ to $T$, and $Z_T$ not $Z_1$?}

% A rectified flow converting $X_0\sim p_0(\vx)$ to $X_T\sim p_1(z_1)$ \dcp{do you want $Z_T, p_T, z_T$ here?} is an ODE
% %
% \begin{align}
%     dZ_t = \varphi(t, Z_t) dt,
% \end{align}
%
% where $\varphi(t, Z_t)$ \dcp{earlier $\varphi(t, z_t)$}
% is trained to drive  $(Z_1-Z_0)$:

\section{\underline{V}ision \underline{L}anguage \underline{D}isinformation Detection \underline{Bench}mark}  
\label{method}  
\textsf{\textbf{\textsc{VLDBench}}} (Figure~\ref{fig:vlbias}) is a comprehensive classification multimodal benchmark for disinformation detection in news articles. It comprises 31,339 articles and visual samples curated from 58 news sources ranging from the Financial Times, CNN, and New York Times to Axios and Wall Street Journal as shown in Figure \ref{fig:news_sources_distribution}. \textsf{\textbf{\textsc{VLDBench}}} spans 13 unique categories (Figure \ref{fig:news_categories}) : \textit{National, Business and Finance, International, Entertainment, Local/Regional, Opinion/Editorial, Health, Sports, Politics, Weather and Environment, Technology, Science, and Other} —adding depth to the disinformation domains. We present the further statistical details in Appendix \ref{app:data-analysis}.
 
\subsection{Task Definition}
\textbf{Disinformation Detection:}  
The core task is to determine whether a text–image news article contains disinformation. We adopt the following definition:  

\emph{“False, misleading, or manipulated information—textual or visual—intentionally created or disseminated to deceive, harm, or influence individuals, groups, or public opinion.”}  

This definition aligns with established social science research \cite{benkler2018network} and governance frameworks \cite{unesco2023journalism}. We specifically focus on the `intent' behind disinformation, which remains relevant over time but has broader effects beyond just being factually incorrect.
\begin{figure*}
    \centering
    \includegraphics[width=0.95\textwidth]{figures/qualitative_figure.pdf}
    \vspace{-1em}
\caption{Disinformation Trends Across News Categories: We analyze the likelihood of disinformation across different categories, based on disinformation narratives and confidence levels generated by GPT-4o.}
    \vspace{-1em}
    \label{fig:disinfo-analysis}
\end{figure*}
\subsection{Data Pipeline}  
\textbf{Dataset Collection:}  
From May 6, 2023, to September 6, 2023, we aggregated data via Google RSS feeds from diverse news sources (Table~\ref{tab:sources}), adhering to Google’s terms of service \cite{google_tos}. We carefully curated  high-quality visual samples from these news sources to ensure a diverse representation of topics. All data collection complied with ethical guidelines \cite{uwaterloo_ethics_review}, regarding intellectual property and privacy protection. 

\textbf{Quality Assurance.}  
Collected articles underwent a rigorous human review and pre-processing phase. First, we removed entries with incomplete text, low-resolution or missing images, duplicates, and media-focused URLs (e.g., \texttt{/video}, \texttt{/gallery}). Articles with fewer than 20 sentences were discarded to ensure textual depth. For each article, the first image was selected to represent the visual context. We periodically reviewed the quality of the curated data to ensure the API returned valid and consistent results. These steps yielded over 31k curated text-image news articles that are moved to the annotation pipeline.

\subsection{Annotation Pipeline}  
To the best of our knowledge, \textsf{\textbf{\textsc{VLDBench}}} is the largest and most comprehensive humanly verified disinformation detection benchmark with over 300 hours of human verification. Figure \ref{fig:vlbias} shows our semi-annotated, data collection and annotation pipeline. After quality assurance, each article was prompted and categorized by GPT-4o as either \texttt{Likely} or \texttt{Unlikely} to contain disinformation, a binary choice designed to balance nuance with manageability. To ensure reliability, GPT-4o assessed text-image alignment three times per sample—first, to minimize random variance in its responses, and second, to resolve potential ties in classification, with an odd number of evaluations ensuring a definitive outcome. GPT-4o was chosen for this task due to its demonstrated effectiveness in both textual \cite{kim2024meganno+} and visual reasoning tasks \cite{shahriar2024putting}. An example is shown in Figure \ref{fig:disinfo-analysis}.

We categorized our data into 13 unique news categories (Figure \ref{fig:news_categories}) by providing image-text pairs to GPT-4o, drawing inspiration from AllSlides \cite{allsides_mediabiaschart}  and frameworks like Media Cloud \cite{media_cloud}. The dataset statistics are given in Table~\ref{tab:dataset_statistics}.

\begin{figure}[ht]
    \centering
    \includegraphics[width=0.48\textwidth]{figures/news_categories_distribution.pdf}
    \caption{Category distribution with overlaps. Total unique articles = 31,339. Percentages sum to \(>100\%\) due to multi-category articles.}
    \label{fig:news_categories}
    \vspace{-1em}
\end{figure}

To ensure high-quality benchmarking, a team of 22 domain experts (Appendix~\ref{app:team}) systematically reviewed the GPT-4o labels and rationales, assessing their accuracy, consistency, and alignment with human judgment. This process included a rigorous structured reconciliation phase, refining annotation guidelines and finalizing the labels. The evaluation resulted in a Cohen’s $\kappa$ of 0.78, indicating strong inter-annotator agreement.


\paragraph{Stability of Automatic Annotations:} To assess the reliability of automated annotations, we conducted a controlled experiment comparing GPT-4o labels with those of human annotators. We randomly selected 1,000 GPT-4o-annotated samples from the previous step, provided annotation guidelines, and asked domain experts (without showing the GPT-4o labels) to manually annotate them. Comparing both sets of labels, GPT-4o achieved an F1 score of 0.89 and an MCC of 0.77, while human annotators scored F1 = 0.92 and MCC = 0.81 (Figure~\ref{fig:alignment_metrics}). These results demonstrate the effectiveness of our semi-annotated pipeline, aligning well with human judgment and ensuring reliable automated labeling.

\begin{table*}[!t]
    \centering
    \resizebox{0.8\textwidth}{!}{
    \begin{tabular}{@{}l|c|c|c|c|c||c@{}}
        \toprule
        & \makecell{MATRES} & \makecell{TB-Dense} & \makecell{TCR} & \makecell{TDD-Manual} & \makecell{NarrativeTime} & \makecell{\textbf{\App{}}} \\
        \midrule
        \multicolumn{7}{c}{\textbf{Datasets Statistics}} \\
        \midrule
        Documents & 275 & 36 & 25 & 34 & 36 & 30 \\
        Events & 6,099 & 1,498 & 1,134 & 1,101 & 1,715 & 470 \\
        \midrule
        \textit{before} & 6,852 (50) & 1,361 (21) & 1,780 (67) & 1,561 (25) & 17,011 (22) & 1,540 (44) \\
        \textit{after} & 4,752 (35) & 1,182 (19) & 862 (33) & 1,054 (17) & 18,366 (23) & 1,347 (39) \\
        \textit{equal} & 448 (4) & 237 (4) & 4 (0) & 140 (2) & 5,298 (7) & 150 (4) \\
        \textit{vague} & 1,525 (11) & 2,837 (45) & -- & -- & 25,679 (33) & 446 (13) \\
        \textit{includes} & -- & 305 (5) & -- & 2,008 (33) & 5,781 (7) & -- \\
        \textit{is-included} & -- & 383 (6) & -- & 1,387 (23) & 6,639 (8) & -- \\
        \textit{overlaps} & -- & -- & -- & -- & 227 (0) & -- \\
        \midrule
        Total Relations & 13,577 & 6,305 & 2,646 & 6,150 & 79,001 & 3,483 \\
        \midrule
        \multicolumn{7}{c}{\textbf{Per Document Average Annotation Sparsity}} \\
        \midrule
        Events & 22.2 & 41.6 & 45.4 & 32.4 & 47.6 & 15.6 \\
        Actual Relations & 49.4 & 183.7 & 105.8 & 180.9 & 1,110.1 & 114.9 \\
        Expected Relations & 234.8 & 844.5 & 1,006.1 & 508.1 & 1,110.1 & 114.9 \\
        \midrule
        Missing Relations & 79\% & 78.3\% & 89.5\% & 64.4\% & 0\% & 0\% \\
        \bottomrule
    \end{tabular}}
    \caption{The upper part of the table presents the statistics of notable datasets for the temporal relation extraction task alongside \App{}. In parentheses, the values indicate the percentage of each relation type relative to the total relations in the dataset. The bottom part of the table summarizes the average percentage of missing relations per document, calculated as the ratio of actual annotated relations to a complete relation coverage, referred to as \textit{Expected Relations}.}
    \label{tab:stats_all}
\end{table*}


% \begin{table*}[!t]
%     \centering
%     \resizebox{0.8\textwidth}{!}{
%     \begin{tabular}{@{}l|c|c|c|c|c|c@{}}
%         \toprule
%         & \makecell{MATRES} & \makecell{TBD} & \makecell{TCR} & \makecell{TDD-Man} & \makecell{NarrativeTime} & \makecell{\App{}} \\
%         \midrule
%         Docs & 275 & 36 & 25 & 34 & 36 & 30 \\
%         Events & 6,099 & 1,498 & 1,134 & 1,101 & 1,715 & 470 \\
%         \midrule
%         Before (\%) & 6,852 (50) & 1,361 (21) & 1,780 (67) & 1,561 (25) & 17,011 (22) & 1,540 (44) \\
%         After (\%) & 4,752 (35) & 1,182 (19) & 862 (33) & 1,054 (17) & 18,366 (23) & 1,347 (39) \\
%         Equal (\%) & 448 (4) & 237 (4) & 4 (0) & 140 (2) & 5,298 (7) & 150 (4) \\
%         Vague (\%) & 1,525 (11) & 2,837 (45) & -- & -- & 25,679 (33) & 446 (13) \\
%         Includes (\%) & -- & 305 (5) & -- & 2,008 (33) & 5,781 (7) & -- \\
%         IsIncluded (\%) & -- & 383 (6) & -- & 1,387 (23) & 6,639 (8) & -- \\
%         Overlaps (\%) & -- & -- & -- & -- & 227 (0) & -- \\
%         \midrule
%         Total Rels & 13,577 & 6,305 & 2,646 & 6,150 & 79,001 & 3,483 \\
%         \bottomrule
%     \end{tabular}}
%     \caption{Statistics of notable datasets for the temporal relation extraction task.}
%     \label{tab:stats}
% \end{table*}



\section{Related Works}

\textbf{LLM explanations}. 
Previous work leveraging LLMs to generate explanations can be categorized into two approaches. For post-hoc natural language explanations, methods such as AMPLIFY\cite{krishna2024post}, Self-Explain\cite{rajagopal2021selfexplain}, and Summarize and Score (SASC)~\cite{singh2023explaining} generate concise rationales based on agent decisions, sometimes accompanied by an explanation score to assess reliability. For ad-hoc methods, Chain-of-Thought (CoT) prompting~\cite{wei2022chain} is a widely adopted in-context learning technique that relies on step-by-step explanations or reasoning to enhance decision-making. Self-Taught Reasoner (STaR)~\cite{zelikman2022star} introduces an iterative refinement method, where a model improves its own explanations through self-generated rationales. While these methods are prompting-based and do not require additional training, optimization-based CoT methods like ReFT~\cite{trung2024reft} have has been developed. Our method falls within the domain of ad-hoc natural language explanations. Without knowing agent decisions, we train an LLM to generate informative and reliable explanations using a generative flow matching model. We compare against CoT and ReFT in our experiments. 
% demonstrating the potential of self-improvement for enhancing explanation quality

% which seeks to provide insights into the causes of model decisions and make them understandable for humans
% While much of the research in this field has focused on supervised learning,
\textbf{Explainable AI}. Our method is suited within  the domain of explainable AI~\cite{arrieta2020explainable,carvalho2019machine,ehsan2019automated,gunning2017explainable,ras2018explanation,gilpin2018explaining} and draws particular parallels with explainable RL (XRL). Post-hoc XRL methods focus on relating inputs and outputs of a trained RL policy in an interpretable way, using an interpretable \emph{surrogate} model as policy approximation. Examples of surrogate models include imitation learning~\cite{abbeel2004apprenticeship}, learning from demonstration~\cite{argall2009survey}, and finite state machines~\cite{koul2018learning, danesh2021re}. However, in order to be interpretable, surrogate models are designed as simple as possible. More related works are in Appendix~\ref{appx:related_work}.

Generating natural language explanations for RL models is appealing, but previous work mainly focuses on specific scenarios like self-driving~\cite{cai2024driving}, recommender systems~\cite{lubos2024llm}, stock prediction~\cite{koa2024learning}, robotics~\cite{lu2023closer}, autonomous navigation~\cite{trigg2024natural}, and network slicing~\cite{ameur2024leveraging}, leaving a general policy-to-language method underexplored. 

\textbf{Diffusion in Transformer} (DiT,~\citet{yang2023diffusion}) leverages the strengths of self-attention of Transformers to improve the performance of diffusion models across a range of tasks, including image and text generation~\cite{cao2024survey}. \citet{dhariwal2021diffusion} demonstrate how Transformer-based architectures can optimize the denoising process in diffusion models, resulting in high-quality image synthesis. \citet{ulhaq2022efficient} explore efficient implementations for diffusion within Transformer. These works are related to our work, as we embed flow matching into the last layer of an LLM. \textbf{Cross-attention} is a popular technique for processing information across multiple modalities~\citep{radford2021learning, alayrac2022flamingo, li2023blip}. Approaches such as T2I-Adapter~\citep{mou2024t2i} and VMix~\citep{wu2024vmix} use cross-attention mechanisms between text encoders (an LLM) and diffusion models to enhance the generation of high-quality images from textual descriptions. More generally, cross-attention has helped solve tasks that require both vision and language understanding~\citep{hatamizadeh2025diffit, cao2024survey}.  Different from previous work on DiT and cross-attention-based image/video generation, to our best knowledge, the proposed method is the first to use generative models and cross-attention to generate rewards for RL-based LLM training.

% They make decision trees differentiable by replacing the Boolean decisions with sigmoid activation functions.  represented by tree nodes~ are developed to iConsequently, the representational capacity of these models typically cannot support them to interpretable all the decisions made by the original model. However, letting RL agents explain their actions in natural language is challenging because of the simultaneous learning of policy and language, and the alignment between them. \citet{ehsan2018rationalization} and~\citet{wang2019verbal} solve this problem by proposing a supervised learning framework using action-explanation pairs annotated by humans. However, since the explanations are provided by humans, these methods are actually learning how humans perceive.
% Diffusion models~\cite{ho2020denoising,song2019generative} have recently emerged as a powerful class of generative AI methods, spurring notable advances in a wide range of tasks such as image generation~\cite{rombach2022high,esser2024scaling}, video generation~\cite{ho2022video,ceylan2023pix2video,ho2022imagen}, molecular design~\cite{gruver2024protein}, and text generation~\cite{lou2024discrete}. At their core, these models perform a forward noising process in which noise is incrementally added to training data over multiple steps, gradually corrupting the original samples. Then, a reverse diffusion process is learned to iteratively remove noise, thereby reconstructing data from near-random initial states. 

% \subsection{Generative Models by Flow Matching}

% Flow matching~\cite{chen2018neural,lipman2022flow} is related to score-based diffusion models~\cite{song2019generative} and enjoys solid mathematical underpinnings. This line of research leverages \emph{continuous normalizing flows} to transform a simple distribution such as Gaussian noise to a complex distribution such as those of natural images. A bottleneck that restricts the use of continuous normalizing flow in large-scale problems is that the ODE is hard to solve. The {\em rectified flow}~\cite{liu2022flow} simplifies the ODE by encouraging vector fields to be represented by straight lines.
% is an innovative approach that uses Transformer architectures  diffusion models with . The core idea is to

% 
\section{Experiment}
We evaluate our proposed method with strong baselines and further analyze contributions of different components, and the impact of key parameters.

\subsection{Experiment Setup}
\textbf{Dataset.}
We evaluate all the methods on Inter-X dataset, which consists about 9K training samples and 1,708 test samples. Each sample is an action-reaction sequence and three corresponding textual description.
As supplementation, we mix our pre-training data with single person motion-text dataset HumanML3D~\citep{humanml3d}, which consists more than 23K annotated motion sequences.
We uniformly sample frames for both datasets to 30 FPS. 

\textbf{Evaluation Metrics.}
Following single-person motion generation~\citep{t2mgpt}, we adopt the these metrics to quantitatively evaluate the generated motion: R-Precision measures the ranking of Euclidean distances between motion and text features. Accuracy (Acc.) assesses how likely a generated motion could be successfully recognized as its interaction label, like ``high-five''. Frechet Inception Distance~\citep{fid} (FID) evaluates the similarity in feature space between predicted and ground-truth motion. Multimodal Distance (MMDist.) calculates the average Euclidean distance between generated motion and the corresponding text description. Diversity (Div.) measures the feature diversity within generated motions. All the metrics reported are calculated with batch size set to 32, and accumulated across the test dataset, and we evaluate each method for 20 times with different seeds to calculate the final results at 95\% confidence interval.

\textbf{Evaluation Model.} \label{sec:eval}
Every metric mentioned above requires an encoder $\mathcal{M}$ to extract motion feature.
For single person text-to-motion generation tasks, a motion-text matching model are commonly trained as human motion feature extractor.
A simple way to transfer this method to interaction domain is to directly train an interaction-to-text matching model $\mathcal{M}(\mathbf{a}, \hat{\mathbf{b}}, text)$, where action sequence $\mathbf{a}$ and predicted reaction sequence $\hat{\mathbf{b}}$ together is regarded as a generated interaction sequence, or a reaction-to-text match model $\mathcal{M}(\hat{\mathbf{b}}, text)$.
However, the former one may focus too much on the ground-truth action input, leading insufficient discriminative power of $\hat{\mathbf{b}}$'s quality, while the latter one lacks semantics provided by action, thus leading to subpar matching capability.

To address the issue, we simply uniformly mask off a large portion of $\mathbf{a}$, obtaining down-sampled action motion sequence $\mathbf{a}'$ (downsampled to 1 FPS in our setting), which serves as a semantic hint for the matching process while not introducing too much emphasis on input action sequence.
The final evaluation model consists of an masked interaction encoder and a text encoder.
We use contrastive loss following CLIP~\citep{clip}, which encourages paired motion and text features to be close geometrically.
In addition, we add a classification head after the predicted motion features, to simultaneously predict interaction labels, such as ``high-five''.

\textbf{Baselines.} To evaluate the performance of our method \ModelAbbr~on online and unconstrained setting, we compare \ModelAbbr~with the following baselines:
1) \textbf{InterFormer}~\citep{interformer} is a transformer based action-to-reaction generation model that leverages human skeleton as prior knowledge for efficient attention process.
2) \textbf{MotionGPT}~\citep{motiongpt} is a motion-language model that leverages an LLM for motion and text generation. We extend the motion tokenizer of MotionGPT to encode multi-person motion, while keeping other settings unchanged.
3) \textbf{InterGen}~\citep{intergen} proposes a mutual attention mechanism within diffusion process for human interaction generation, we reproduce and adapt IngerGen to action-to-reaction generation.
4) \textbf{ReGenNet}~\citep{regennet} is latest state-of-the-art model on action-to-reaction generation. It adopts a transformer decoder based diffusion model, which directly predicts human reaction given action input in unconstrained and online manner as ours.


\textbf{Implementation Details.}
For the LLM, we adopt Flan-T5-base~\citep{flan,t5} as our base model, with extended vocabulary. We warm up the learning rate for 1,000 steps, peaking at 1e-4 for the pre-training phase, and use the same learning rate for fine-tuning.
Both the pre-training and fine-tuning phases are trained on a single machine with 8 Tesla V100 GPUs. The training batch size is set to 32 for the LLM and we monitor the validation loss and reaction generation metrics for early-stopping, resulting about 100K pre-training steps and 40K fine-tuning steps.
We set the re-thinking interval $N_r$ to 4 tokens and divide each space signal into $N_b=10$ bins.

\begin{table}[t]
\centering
\tiny
\caption{Comparison to state-of-the-art baselines and ablation studies of our method on Inter-X dataset. $\uparrow$ or $\downarrow$ denotes a higher or lower value is better, and $\rightarrow$ means that the value closer to real is better. We use $\pm$ to represent 95\% confidence interval and highlight the best results in \textbf{bold}. For ablation methods (in grey), PT, M, P, S, and SP are abbreviations for pre-training, motion, pose, space, and single-person data, respectively.}
\label{tab:main}
\begin{tabular}{l|ccccccc}
\toprule
\multirow{2}{*}{Methods} &  \multicolumn{3}{c}{R-Precision$\uparrow$} & \multirow{2}{*}{Acc.$\uparrow$}& \multirow{2}{*}{FID$\downarrow$}         & \multirow{2}{*}{MMDist$\downarrow$} & \multirow{2}{*}{Div.$\rightarrow$} \\
               & Top-1       & Top-2       & Top-3     &     &         &                               &                       \\ \midrule
Real                    & $0.511^{\pm.003}$ & $0.682^{\pm.002}$ & $0.776^{\pm.002}$ & $0.463^{\pm.000}$  & $0.000^{\pm.000}$         & $5.348^{\pm.002}$         & $2.498^{\pm.005}$           \\ \midrule
InterFormer             & $0.172^{\pm.012}$ & $0.292^{\pm.013}$ & $0.343^{\pm.012}$ & $0.171^{\pm.009}$ & $10.468^{\pm.021}$        &  $7.831^{\pm.018}$         & $3.505^{\pm.023}$           \\
MotionGPT &            $0.238^{\pm.003}$        &     $0.354^{\pm.004}$        &     $0.441^{\pm.003}$   &    $0.186^{\pm.002}$            &     $5.823^{\pm.048}$               &      $6.211^{\pm.005}$           &      $2.615^{\pm.007}$ \\
InterGen                & $0.326^{\pm.036}$ & $0.423^{\pm.063}$ & $0.525^{\pm.053}$ & $0.254^{\pm.019}$  & $5.506^{\pm.257}$         & $6.182^{\pm.038}$         & $2.284^{\pm.009}$           \\
ReGenNet                & $0.384^{\pm.005}$ & $0.483^{\pm.002}$ & $0.572^{\pm.003}$ & $0.297^{\pm.004}$  & $3.988^{\pm.048}$         & $5.867^{\pm.009}$         & $\mathbf{2.502^{\pm.001}}$           \\ \midrule
% \rowcolor[HTML]{EFEFEF}
\ModelAbbr~(Ours)       & $\mathbf{0.423^{\pm.005}}$ & $\mathbf{0.599^{\pm.003}}$ & $\mathbf{0.693^{\pm.003}}$ & $\mathbf{0.318^{\pm.003}}$  & $\mathbf{1.942^{\pm.017}}$         & $\mathbf{5.643^{\pm.003}}$         & $2.629^{\pm.006}$           \\
\rowcolor[HTML]{EFEFEF}
w/o Think           & $0.367^{\pm.003}$ & $0.491^{\pm.027}$ & $0.584^{\pm.008}$ & $0.230^{\pm.036}$ & $3.828^{\pm.016}$         & $6.186^{\pm.055}$         & $2.609^{\pm.006}$           \\
\rowcolor[HTML]{EFEFEF}
w/o All PT.         & $0.398^{\pm.007}$ & $0.531^{\pm.002}$ & $0.628^{\pm.003}$ & $0.288^{\pm.002}$ & $3.467^{\pm.113}$         & $5.822^{\pm.003}$         & $2.909^{\pm.053}$           \\
\rowcolor[HTML]{EFEFEF}
w/o M-M PT. & $0.408^{\pm.005}$ & $0.563^{\pm.004}$ & $0.646^{\pm.005}$ & $0.293^{\pm.002}$ & $2.874^{\pm.020}$         & $5.736^{\pm.003}$         & $2.553^{\pm.006}$           \\
\rowcolor[HTML]{EFEFEF}
w/o P-S PT. & $0.417^{\pm.004}$ & $0.582^{\pm.004}$ & $0.664^{\pm.004}$ & $0.308^{\pm.003}$ & $2.685^{\pm.024}$         & $5.699^{\pm.004}$         & $2.859^{\pm.007}$           \\
\rowcolor[HTML]{EFEFEF}
w/o M-T PT. & $0.406^{\pm.003}$ & $0.557^{\pm.004}$ & $0.637^{\pm.004}$ & $0.304^{\pm.003}$ & $2.580^{\pm.021}$         & $5.822 ^{\pm.003}$         & $2.889^{\pm.005}$           \\
\rowcolor[HTML]{EFEFEF}
w/o SP Data     & $0.414^{\pm.004}$ & $0.592^{\pm.005}$ & $0.685^{\pm.003}$ & $0.315^{\pm.004}$ & $2.007^{\pm.015}$         & $5.667^{\pm.003}$         & $2.611^{\pm.005}$           \\
\bottomrule
\end{tabular}
\end{table}



\begin{figure}
    \centering
    \includegraphics[width=\linewidth]{figs/tsne.pdf}
    \caption{Visualization of a person's motion sequences in Inter-X dataset and HumanML3D dataset.}
    \label{fig:tsne}
\end{figure}

\subsection{Comparison to Baselines}\label{sec:sota}
As shown in the upper side of Table~\ref{tab:main}, our method \ModelAbbr~significantly outperforms baseline methods in terms of ranking, accuracy, FID and multimodal distance, showing superior human reaction generation quality.
Compared to MotionGPT, which adopts a similar motion-language architecture, \ModelAbbr~expresses stronger performance, which we attribute to our unified representation of motion via space and pose tokenizers, enabling effective individual pose and inter-person spatial relationship representation.
\ModelAbbr~also surpasses the diffusion-based methods, InterGen and ReGenNet, with our think-then-react architecture, improving generated motions by describing observed action and reasoning what reaction is expected on semantic level. In addition, ReGenNet and MotionGPT get closer diversity to the real than our model. We mainly attribute to that, \ModelAbbr~may conduct multiple re-thinking processes during inference, and the inferred semantics may bring a higher diversity.


\subsection{Ablation Study of Key Components}
To evaluate the effectiveness of our proposed key designs, we conduct detailed ablation studies by removing each of them to observe how much drop compared to the full version of our \ModelAbbr~method. The larger drop indicates more contribution. The results are shown in gray lines of Table~\ref{tab:main}. According to the drops in FID, all designs, including thinking, pre-training tasks and using single person data in pre-training, have positive contributions to the final performance, and thinking contributes the most. Some detailed findings and analyses are as follows.

First, we skip \textbf{thinking} stage during inference, and find the performance drops significantly in FID from 1.9 to 3.8. This supports the necessity of our proposed thinking process before reacting. We also notice decreasing diversity of generated samples, as the model relies solely on input action, and cannot explicitly capture and infer action's intent, thus leading to more rigid motion in some cases.

Second, to evaluate the effectiveness of \textbf{pre-training}, we omit the pre-training stage, and directly train our model \ModelAbbr~for thinking and reacting tasks. As shown in Table~\ref{tab:main}, our model's performance deteriorates without a fine-grained pre-training phase from 1.9 to 3.4 in FID. This indicates that pre-training can effectively adapt a language model (Flan-T5-base) into a motion and language model. We further removing three kinds of pre-training tasks: motion-motion (M-M PT.), pose-space (P-S PT.), and motion-text (M-T PT.). The results show that the without any task, the performance obviously gets worse, from 1.9 to 2.5 - 2.8 in FID, indicating their positive contribution to the final performance and complementary values to each other.

Third, to see how much \textbf{single-person data} helps reaction generation, we remove single person motion-text data, i.e., the data from HumanML3D dataset, from our training set. The result (w/o SP Data) shows that the model performs worse without training on HumanML3D, which proves that our unified motion encoder and motion-language architecture can leverage both single- and multi-person data, alleviating the insufficiency of training data. However, the benefit from single-person data is not as large as we expect. 

% What's more, we evaluate the necessity of \textbf{decoupled space-motion tokenizer}, and the results are shown in Table~\ref{tab:vqvae}. We design a plain motion VQ-VAE with unnormalized action and reaction as input, maintaining absolute space and pose features. With the trained motion VQ-VAE, we encode action/reaction into tokens, which are then fed into TTR for reaction prediction task. First, without normalized motion as input, the reconstruction FID significantly rises from 0.262 to 0.983, showing deteriorated reconstruction performance due to insufficient utilization of codebook. Second, in the reaction generation phase, TTR's performance drops dramatically, as the badly constructed codebook leads to inaccurate action understanding and reaction prediction, highlighting the necessity of decoupling token representation of space and pose features in multi-person scenario.

\begin{figure}
    \centering
    \includegraphics[width=\linewidth]{figs/case_study.pdf}
    \caption{Visualized cases of our predicted reactions (in green) to input action (in blue) and corresponding thinking results. We also provide a failure case in figure (d), where TTR misunderstands the input action as ``wrestling'', which should be ``embracing''.}
    \label{fig:case_study}
\end{figure}


\subsection{Analysis on Overlapping between Single- and Multi-Person Motions}
To investigate the reason of small contribution from single-person data, we further visualize motion sequences of single-person motion (HumanML3D), two-person action (Inter-X Action) and reaction (Inter-X Reaction) in the same space, as presented in Figure~\ref{fig:tsne}. Specifically, we use t-SNE tool~\cite{tsne} to project motion token sequence features into two-dimension. As shown in Figure~\ref{fig:tsne}, the single- and two-person motion sequences have little overlap. When doing case studies, we find that most two-person motion are unique, e.g., massage and being pulled, and will never be used in single-person motion. Similarly, most single-person motions are unique too, e.g., T-pose, and seldom appear in multi-person interaction. There are only a few overlapped motions, e.g., standing still. In addition, when comparing action and reaction sequences in multi-person interaction, we have some interesting findings. When reactions are close to actions, the motion usually belongs to symmetrical interactions, e.g., pulling or being pulled; whereas, when actions are far from reactions, the motion usually belongs to asymmetrical interaction, e.g., massage.


\subsection{Impact of Down-Sampling Parameter in Matching Model for Evaluation}

As described in Section~\ref{sec:eval}, we propose downsampling action motion sequence to avoid matching models for evaluation pay too much attention to input action rather than output reaction. We conduct an experiment to change the downsampling parameter frame rate and calculate the difference between taking ground-truth action and random action as the input of $\mathcal{M}$, in terms of summed ranking scores (Top-1, Top-2, Top-3 and Acc.). As presented in Figure~\ref{fig:discriminative}, 
difference is lowest when FPS equals to 0, which meaning we only match generated reaction motion with text. It goes up to the peak when FPS equals 1 and quickly goes down to low values, even close to the lowest when FPS is about 15. This indicates that it is necessary to concatenate input action with generated reaction to compose a meaningful interaction in evaluation, otherwise the motion-text matching model cannot well recognize the interaction. However, only 1 FPS is enough. With larger FPS, the matching models will be disturbed by input action rather than the generated reaction. Thus, we choose 1 FPS, corresponding to the largest difference, as our final setting.

\subsection{Impact of Re-thinking Interval}
% Our aim is to generate real-time reaction online, and thus time interval is an important parameter to generation quality. 
We change the re-thinking interval $N_r$ from about 1 to 100 timesteps (about 0.1 to 10 seconds) and observe how it impacts generative quality measure FID. As shown in Figure~\ref{fig:latency}, FID falls down first until $N_r=4$ (about 0.5 second) and then continues rising up. This indicate that the best time interval is about 0.5 second. When the time interval is too short, our \ModelAbbr~model cannot get enough information to re-think what the input action means and will bring some randomness into predicting appropriate reaction. When the time interval gets too long, our \ModelAbbr~model give slow responses to the input action sequences and generates coarse-grained reaction.

We also evaluate the average inference time per step (AITS) with respect to the re-thinking interval. As shown in Figure~\ref{fig:latency}, the inference time significantly decreases as the re-thinking interval increases, eventually converging to approximately 10 milliseconds per step (100 FPS). In our setup, we opt to re-think every four steps, resulting in an inference time of less than 50 milliseconds, which meets the requirements for a real-time system.

\subsection{User Study}
To further evaluate our model qualitatively, we conduct a user study on TTR vs. the latest SOTA method ReGenNet, and the results are shown in Figure~\ref{fig:user_study}. We randomly sample 100 action sequences from Inter-X dataset, which are fed into TTR and ReGenNet to predict reactions, and ask four real human to choose the better ones. It can be seen that TTR surpasses ReGenNet on all the duration range, and the winning rate rises significantly when motion duration is longer. We mainly contribute this to our explicit thinking and re-thinking procedure, which ensures semantics matching and alleviates accumulated errors. 

\begin{figure}[t]
    \centering
    \begin{minipage}[b]{0.3\textwidth}
        \centering
        \includegraphics[width=\textwidth]{figs/discriminative_power.pdf}
        \caption{Impact of input action FPS to summed ranking score differences.}
        \label{fig:discriminative}
    \end{minipage}
    \hfill
    \begin{minipage}[b]{0.35\textwidth}
        \centering
        \includegraphics[width=\textwidth]{figs/latency.pdf}
        \caption{Impact of re-thinking interval to FID and average inference time per step (AITS).}
        \label{fig:latency}
    \end{minipage}
    \hfill
    \begin{minipage}[b]{0.3\textwidth}
        \centering
        \includegraphics[width=\linewidth]{figs/user_study.pdf}
        \caption{User preference between TTR and ReGenNet on different motion duration.}
        \label{fig:user_study}
    \end{minipage}
\end{figure}

\section{Conclusion}

We justify that a flow matching generative model can produce dense and reliable rewards for training LLMs to explain the decisions of RL agents and other LLMs. 
Looking into the future, we envision extending this method to a general LLM training approach, automatically generating high-quality dense rewards, and ultimately reducing the reliance on human feedback. 

% Our method has the potential to facilitate human-AI collaboration applications, such as transportation, education, and security defense.

\newpage
\section{Impact Statements}
This paper presents work whose goal is to advance the field of machine learning by developing a model-agnostic explanation generator for intelligent agents, enhancing transparency and interpretability in agent decision prediction. The ability to generate effective and interpretable explanations has the potential to foster trust in AI systems, improving effectiveness in high-stakes applications such as healthcare, finance, and autonomous systems. Overall, we believe our work contributes positively to the broader AI ecosystem by promoting more explainable and trustworthy AI.


% In the unusual situation where you want a paper to appear in the
% references without citing it in the main text, use \nocite
% \nocite{langley00}

\bibliography{root}
\bibliographystyle{icml2025}


%%%%%%%%%%%%%%%%%%%%%%%%%%%%%%%%%%%%%%%%%%%%%%%%%%%%%%%%%%%%%%%%%%%%%%%%%%%%%%%
%%%%%%%%%%%%%%%%%%%%%%%%%%%%%%%%%%%%%%%%%%%%%%%%%%%%%%%%%%%%%%%%%%%%%%%%%%%%%%%
% APPENDIX
%%%%%%%%%%%%%%%%%%%%%%%%%%%%%%%%%%%%%%%%%%%%%%%%%%%%%%%%%%%%%%%%%%%%%%%%%%%%%%%
%%%%%%%%%%%%%%%%%%%%%%%%%%%%%%%%%%%%%%%%%%%%%%%%%%%%%%%%%%%%%%%%%%%%%%%%%%%%%%%
\newpage
\appendix
\onecolumn
\subsection{Lloyd-Max Algorithm}
\label{subsec:Lloyd-Max}
For a given quantization bitwidth $B$ and an operand $\bm{X}$, the Lloyd-Max algorithm finds $2^B$ quantization levels $\{\hat{x}_i\}_{i=1}^{2^B}$ such that quantizing $\bm{X}$ by rounding each scalar in $\bm{X}$ to the nearest quantization level minimizes the quantization MSE. 

The algorithm starts with an initial guess of quantization levels and then iteratively computes quantization thresholds $\{\tau_i\}_{i=1}^{2^B-1}$ and updates quantization levels $\{\hat{x}_i\}_{i=1}^{2^B}$. Specifically, at iteration $n$, thresholds are set to the midpoints of the previous iteration's levels:
\begin{align*}
    \tau_i^{(n)}=\frac{\hat{x}_i^{(n-1)}+\hat{x}_{i+1}^{(n-1)}}2 \text{ for } i=1\ldots 2^B-1
\end{align*}
Subsequently, the quantization levels are re-computed as conditional means of the data regions defined by the new thresholds:
\begin{align*}
    \hat{x}_i^{(n)}=\mathbb{E}\left[ \bm{X} \big| \bm{X}\in [\tau_{i-1}^{(n)},\tau_i^{(n)}] \right] \text{ for } i=1\ldots 2^B
\end{align*}
where to satisfy boundary conditions we have $\tau_0=-\infty$ and $\tau_{2^B}=\infty$. The algorithm iterates the above steps until convergence.

Figure \ref{fig:lm_quant} compares the quantization levels of a $7$-bit floating point (E3M3) quantizer (left) to a $7$-bit Lloyd-Max quantizer (right) when quantizing a layer of weights from the GPT3-126M model at a per-tensor granularity. As shown, the Lloyd-Max quantizer achieves substantially lower quantization MSE. Further, Table \ref{tab:FP7_vs_LM7} shows the superior perplexity achieved by Lloyd-Max quantizers for bitwidths of $7$, $6$ and $5$. The difference between the quantizers is clear at 5 bits, where per-tensor FP quantization incurs a drastic and unacceptable increase in perplexity, while Lloyd-Max quantization incurs a much smaller increase. Nevertheless, we note that even the optimal Lloyd-Max quantizer incurs a notable ($\sim 1.5$) increase in perplexity due to the coarse granularity of quantization. 

\begin{figure}[h]
  \centering
  \includegraphics[width=0.7\linewidth]{sections/figures/LM7_FP7.pdf}
  \caption{\small Quantization levels and the corresponding quantization MSE of Floating Point (left) vs Lloyd-Max (right) Quantizers for a layer of weights in the GPT3-126M model.}
  \label{fig:lm_quant}
\end{figure}

\begin{table}[h]\scriptsize
\begin{center}
\caption{\label{tab:FP7_vs_LM7} \small Comparing perplexity (lower is better) achieved by floating point quantizers and Lloyd-Max quantizers on a GPT3-126M model for the Wikitext-103 dataset.}
\begin{tabular}{c|cc|c}
\hline
 \multirow{2}{*}{\textbf{Bitwidth}} & \multicolumn{2}{|c|}{\textbf{Floating-Point Quantizer}} & \textbf{Lloyd-Max Quantizer} \\
 & Best Format & Wikitext-103 Perplexity & Wikitext-103 Perplexity \\
\hline
7 & E3M3 & 18.32 & 18.27 \\
6 & E3M2 & 19.07 & 18.51 \\
5 & E4M0 & 43.89 & 19.71 \\
\hline
\end{tabular}
\end{center}
\end{table}

\subsection{Proof of Local Optimality of LO-BCQ}
\label{subsec:lobcq_opt_proof}
For a given block $\bm{b}_j$, the quantization MSE during LO-BCQ can be empirically evaluated as $\frac{1}{L_b}\lVert \bm{b}_j- \bm{\hat{b}}_j\rVert^2_2$ where $\bm{\hat{b}}_j$ is computed from equation (\ref{eq:clustered_quantization_definition}) as $C_{f(\bm{b}_j)}(\bm{b}_j)$. Further, for a given block cluster $\mathcal{B}_i$, we compute the quantization MSE as $\frac{1}{|\mathcal{B}_{i}|}\sum_{\bm{b} \in \mathcal{B}_{i}} \frac{1}{L_b}\lVert \bm{b}- C_i^{(n)}(\bm{b})\rVert^2_2$. Therefore, at the end of iteration $n$, we evaluate the overall quantization MSE $J^{(n)}$ for a given operand $\bm{X}$ composed of $N_c$ block clusters as:
\begin{align*}
    \label{eq:mse_iter_n}
    J^{(n)} = \frac{1}{N_c} \sum_{i=1}^{N_c} \frac{1}{|\mathcal{B}_{i}^{(n)}|}\sum_{\bm{v} \in \mathcal{B}_{i}^{(n)}} \frac{1}{L_b}\lVert \bm{b}- B_i^{(n)}(\bm{b})\rVert^2_2
\end{align*}

At the end of iteration $n$, the codebooks are updated from $\mathcal{C}^{(n-1)}$ to $\mathcal{C}^{(n)}$. However, the mapping of a given vector $\bm{b}_j$ to quantizers $\mathcal{C}^{(n)}$ remains as  $f^{(n)}(\bm{b}_j)$. At the next iteration, during the vector clustering step, $f^{(n+1)}(\bm{b}_j)$ finds new mapping of $\bm{b}_j$ to updated codebooks $\mathcal{C}^{(n)}$ such that the quantization MSE over the candidate codebooks is minimized. Therefore, we obtain the following result for $\bm{b}_j$:
\begin{align*}
\frac{1}{L_b}\lVert \bm{b}_j - C_{f^{(n+1)}(\bm{b}_j)}^{(n)}(\bm{b}_j)\rVert^2_2 \le \frac{1}{L_b}\lVert \bm{b}_j - C_{f^{(n)}(\bm{b}_j)}^{(n)}(\bm{b}_j)\rVert^2_2
\end{align*}

That is, quantizing $\bm{b}_j$ at the end of the block clustering step of iteration $n+1$ results in lower quantization MSE compared to quantizing at the end of iteration $n$. Since this is true for all $\bm{b} \in \bm{X}$, we assert the following:
\begin{equation}
\begin{split}
\label{eq:mse_ineq_1}
    \tilde{J}^{(n+1)} &= \frac{1}{N_c} \sum_{i=1}^{N_c} \frac{1}{|\mathcal{B}_{i}^{(n+1)}|}\sum_{\bm{b} \in \mathcal{B}_{i}^{(n+1)}} \frac{1}{L_b}\lVert \bm{b} - C_i^{(n)}(b)\rVert^2_2 \le J^{(n)}
\end{split}
\end{equation}
where $\tilde{J}^{(n+1)}$ is the the quantization MSE after the vector clustering step at iteration $n+1$.

Next, during the codebook update step (\ref{eq:quantizers_update}) at iteration $n+1$, the per-cluster codebooks $\mathcal{C}^{(n)}$ are updated to $\mathcal{C}^{(n+1)}$ by invoking the Lloyd-Max algorithm \citep{Lloyd}. We know that for any given value distribution, the Lloyd-Max algorithm minimizes the quantization MSE. Therefore, for a given vector cluster $\mathcal{B}_i$ we obtain the following result:

\begin{equation}
    \frac{1}{|\mathcal{B}_{i}^{(n+1)}|}\sum_{\bm{b} \in \mathcal{B}_{i}^{(n+1)}} \frac{1}{L_b}\lVert \bm{b}- C_i^{(n+1)}(\bm{b})\rVert^2_2 \le \frac{1}{|\mathcal{B}_{i}^{(n+1)}|}\sum_{\bm{b} \in \mathcal{B}_{i}^{(n+1)}} \frac{1}{L_b}\lVert \bm{b}- C_i^{(n)}(\bm{b})\rVert^2_2
\end{equation}

The above equation states that quantizing the given block cluster $\mathcal{B}_i$ after updating the associated codebook from $C_i^{(n)}$ to $C_i^{(n+1)}$ results in lower quantization MSE. Since this is true for all the block clusters, we derive the following result: 
\begin{equation}
\begin{split}
\label{eq:mse_ineq_2}
     J^{(n+1)} &= \frac{1}{N_c} \sum_{i=1}^{N_c} \frac{1}{|\mathcal{B}_{i}^{(n+1)}|}\sum_{\bm{b} \in \mathcal{B}_{i}^{(n+1)}} \frac{1}{L_b}\lVert \bm{b}- C_i^{(n+1)}(\bm{b})\rVert^2_2  \le \tilde{J}^{(n+1)}   
\end{split}
\end{equation}

Following (\ref{eq:mse_ineq_1}) and (\ref{eq:mse_ineq_2}), we find that the quantization MSE is non-increasing for each iteration, that is, $J^{(1)} \ge J^{(2)} \ge J^{(3)} \ge \ldots \ge J^{(M)}$ where $M$ is the maximum number of iterations. 
%Therefore, we can say that if the algorithm converges, then it must be that it has converged to a local minimum. 
\hfill $\blacksquare$


\begin{figure}
    \begin{center}
    \includegraphics[width=0.5\textwidth]{sections//figures/mse_vs_iter.pdf}
    \end{center}
    \caption{\small NMSE vs iterations during LO-BCQ compared to other block quantization proposals}
    \label{fig:nmse_vs_iter}
\end{figure}

Figure \ref{fig:nmse_vs_iter} shows the empirical convergence of LO-BCQ across several block lengths and number of codebooks. Also, the MSE achieved by LO-BCQ is compared to baselines such as MXFP and VSQ. As shown, LO-BCQ converges to a lower MSE than the baselines. Further, we achieve better convergence for larger number of codebooks ($N_c$) and for a smaller block length ($L_b$), both of which increase the bitwidth of BCQ (see Eq \ref{eq:bitwidth_bcq}).


\subsection{Additional Accuracy Results}
%Table \ref{tab:lobcq_config} lists the various LOBCQ configurations and their corresponding bitwidths.
\begin{table}
\setlength{\tabcolsep}{4.75pt}
\begin{center}
\caption{\label{tab:lobcq_config} Various LO-BCQ configurations and their bitwidths.}
\begin{tabular}{|c||c|c|c|c||c|c||c|} 
\hline
 & \multicolumn{4}{|c||}{$L_b=8$} & \multicolumn{2}{|c||}{$L_b=4$} & $L_b=2$ \\
 \hline
 \backslashbox{$L_A$\kern-1em}{\kern-1em$N_c$} & 2 & 4 & 8 & 16 & 2 & 4 & 2 \\
 \hline
 64 & 4.25 & 4.375 & 4.5 & 4.625 & 4.375 & 4.625 & 4.625\\
 \hline
 32 & 4.375 & 4.5 & 4.625& 4.75 & 4.5 & 4.75 & 4.75 \\
 \hline
 16 & 4.625 & 4.75& 4.875 & 5 & 4.75 & 5 & 5 \\
 \hline
\end{tabular}
\end{center}
\end{table}

%\subsection{Perplexity achieved by various LO-BCQ configurations on Wikitext-103 dataset}

\begin{table} \centering
\begin{tabular}{|c||c|c|c|c||c|c||c|} 
\hline
 $L_b \rightarrow$& \multicolumn{4}{c||}{8} & \multicolumn{2}{c||}{4} & 2\\
 \hline
 \backslashbox{$L_A$\kern-1em}{\kern-1em$N_c$} & 2 & 4 & 8 & 16 & 2 & 4 & 2  \\
 %$N_c \rightarrow$ & 2 & 4 & 8 & 16 & 2 & 4 & 2 \\
 \hline
 \hline
 \multicolumn{8}{c}{GPT3-1.3B (FP32 PPL = 9.98)} \\ 
 \hline
 \hline
 64 & 10.40 & 10.23 & 10.17 & 10.15 &  10.28 & 10.18 & 10.19 \\
 \hline
 32 & 10.25 & 10.20 & 10.15 & 10.12 &  10.23 & 10.17 & 10.17 \\
 \hline
 16 & 10.22 & 10.16 & 10.10 & 10.09 &  10.21 & 10.14 & 10.16 \\
 \hline
  \hline
 \multicolumn{8}{c}{GPT3-8B (FP32 PPL = 7.38)} \\ 
 \hline
 \hline
 64 & 7.61 & 7.52 & 7.48 &  7.47 &  7.55 &  7.49 & 7.50 \\
 \hline
 32 & 7.52 & 7.50 & 7.46 &  7.45 &  7.52 &  7.48 & 7.48  \\
 \hline
 16 & 7.51 & 7.48 & 7.44 &  7.44 &  7.51 &  7.49 & 7.47  \\
 \hline
\end{tabular}
\caption{\label{tab:ppl_gpt3_abalation} Wikitext-103 perplexity across GPT3-1.3B and 8B models.}
\end{table}

\begin{table} \centering
\begin{tabular}{|c||c|c|c|c||} 
\hline
 $L_b \rightarrow$& \multicolumn{4}{c||}{8}\\
 \hline
 \backslashbox{$L_A$\kern-1em}{\kern-1em$N_c$} & 2 & 4 & 8 & 16 \\
 %$N_c \rightarrow$ & 2 & 4 & 8 & 16 & 2 & 4 & 2 \\
 \hline
 \hline
 \multicolumn{5}{|c|}{Llama2-7B (FP32 PPL = 5.06)} \\ 
 \hline
 \hline
 64 & 5.31 & 5.26 & 5.19 & 5.18  \\
 \hline
 32 & 5.23 & 5.25 & 5.18 & 5.15  \\
 \hline
 16 & 5.23 & 5.19 & 5.16 & 5.14  \\
 \hline
 \multicolumn{5}{|c|}{Nemotron4-15B (FP32 PPL = 5.87)} \\ 
 \hline
 \hline
 64  & 6.3 & 6.20 & 6.13 & 6.08  \\
 \hline
 32  & 6.24 & 6.12 & 6.07 & 6.03  \\
 \hline
 16  & 6.12 & 6.14 & 6.04 & 6.02  \\
 \hline
 \multicolumn{5}{|c|}{Nemotron4-340B (FP32 PPL = 3.48)} \\ 
 \hline
 \hline
 64 & 3.67 & 3.62 & 3.60 & 3.59 \\
 \hline
 32 & 3.63 & 3.61 & 3.59 & 3.56 \\
 \hline
 16 & 3.61 & 3.58 & 3.57 & 3.55 \\
 \hline
\end{tabular}
\caption{\label{tab:ppl_llama7B_nemo15B} Wikitext-103 perplexity compared to FP32 baseline in Llama2-7B and Nemotron4-15B, 340B models}
\end{table}

%\subsection{Perplexity achieved by various LO-BCQ configurations on MMLU dataset}


\begin{table} \centering
\begin{tabular}{|c||c|c|c|c||c|c|c|c|} 
\hline
 $L_b \rightarrow$& \multicolumn{4}{c||}{8} & \multicolumn{4}{c||}{8}\\
 \hline
 \backslashbox{$L_A$\kern-1em}{\kern-1em$N_c$} & 2 & 4 & 8 & 16 & 2 & 4 & 8 & 16  \\
 %$N_c \rightarrow$ & 2 & 4 & 8 & 16 & 2 & 4 & 2 \\
 \hline
 \hline
 \multicolumn{5}{|c|}{Llama2-7B (FP32 Accuracy = 45.8\%)} & \multicolumn{4}{|c|}{Llama2-70B (FP32 Accuracy = 69.12\%)} \\ 
 \hline
 \hline
 64 & 43.9 & 43.4 & 43.9 & 44.9 & 68.07 & 68.27 & 68.17 & 68.75 \\
 \hline
 32 & 44.5 & 43.8 & 44.9 & 44.5 & 68.37 & 68.51 & 68.35 & 68.27  \\
 \hline
 16 & 43.9 & 42.7 & 44.9 & 45 & 68.12 & 68.77 & 68.31 & 68.59  \\
 \hline
 \hline
 \multicolumn{5}{|c|}{GPT3-22B (FP32 Accuracy = 38.75\%)} & \multicolumn{4}{|c|}{Nemotron4-15B (FP32 Accuracy = 64.3\%)} \\ 
 \hline
 \hline
 64 & 36.71 & 38.85 & 38.13 & 38.92 & 63.17 & 62.36 & 63.72 & 64.09 \\
 \hline
 32 & 37.95 & 38.69 & 39.45 & 38.34 & 64.05 & 62.30 & 63.8 & 64.33  \\
 \hline
 16 & 38.88 & 38.80 & 38.31 & 38.92 & 63.22 & 63.51 & 63.93 & 64.43  \\
 \hline
\end{tabular}
\caption{\label{tab:mmlu_abalation} Accuracy on MMLU dataset across GPT3-22B, Llama2-7B, 70B and Nemotron4-15B models.}
\end{table}


%\subsection{Perplexity achieved by various LO-BCQ configurations on LM evaluation harness}

\begin{table} \centering
\begin{tabular}{|c||c|c|c|c||c|c|c|c|} 
\hline
 $L_b \rightarrow$& \multicolumn{4}{c||}{8} & \multicolumn{4}{c||}{8}\\
 \hline
 \backslashbox{$L_A$\kern-1em}{\kern-1em$N_c$} & 2 & 4 & 8 & 16 & 2 & 4 & 8 & 16  \\
 %$N_c \rightarrow$ & 2 & 4 & 8 & 16 & 2 & 4 & 2 \\
 \hline
 \hline
 \multicolumn{5}{|c|}{Race (FP32 Accuracy = 37.51\%)} & \multicolumn{4}{|c|}{Boolq (FP32 Accuracy = 64.62\%)} \\ 
 \hline
 \hline
 64 & 36.94 & 37.13 & 36.27 & 37.13 & 63.73 & 62.26 & 63.49 & 63.36 \\
 \hline
 32 & 37.03 & 36.36 & 36.08 & 37.03 & 62.54 & 63.51 & 63.49 & 63.55  \\
 \hline
 16 & 37.03 & 37.03 & 36.46 & 37.03 & 61.1 & 63.79 & 63.58 & 63.33  \\
 \hline
 \hline
 \multicolumn{5}{|c|}{Winogrande (FP32 Accuracy = 58.01\%)} & \multicolumn{4}{|c|}{Piqa (FP32 Accuracy = 74.21\%)} \\ 
 \hline
 \hline
 64 & 58.17 & 57.22 & 57.85 & 58.33 & 73.01 & 73.07 & 73.07 & 72.80 \\
 \hline
 32 & 59.12 & 58.09 & 57.85 & 58.41 & 73.01 & 73.94 & 72.74 & 73.18  \\
 \hline
 16 & 57.93 & 58.88 & 57.93 & 58.56 & 73.94 & 72.80 & 73.01 & 73.94  \\
 \hline
\end{tabular}
\caption{\label{tab:mmlu_abalation} Accuracy on LM evaluation harness tasks on GPT3-1.3B model.}
\end{table}

\begin{table} \centering
\begin{tabular}{|c||c|c|c|c||c|c|c|c|} 
\hline
 $L_b \rightarrow$& \multicolumn{4}{c||}{8} & \multicolumn{4}{c||}{8}\\
 \hline
 \backslashbox{$L_A$\kern-1em}{\kern-1em$N_c$} & 2 & 4 & 8 & 16 & 2 & 4 & 8 & 16  \\
 %$N_c \rightarrow$ & 2 & 4 & 8 & 16 & 2 & 4 & 2 \\
 \hline
 \hline
 \multicolumn{5}{|c|}{Race (FP32 Accuracy = 41.34\%)} & \multicolumn{4}{|c|}{Boolq (FP32 Accuracy = 68.32\%)} \\ 
 \hline
 \hline
 64 & 40.48 & 40.10 & 39.43 & 39.90 & 69.20 & 68.41 & 69.45 & 68.56 \\
 \hline
 32 & 39.52 & 39.52 & 40.77 & 39.62 & 68.32 & 67.43 & 68.17 & 69.30  \\
 \hline
 16 & 39.81 & 39.71 & 39.90 & 40.38 & 68.10 & 66.33 & 69.51 & 69.42  \\
 \hline
 \hline
 \multicolumn{5}{|c|}{Winogrande (FP32 Accuracy = 67.88\%)} & \multicolumn{4}{|c|}{Piqa (FP32 Accuracy = 78.78\%)} \\ 
 \hline
 \hline
 64 & 66.85 & 66.61 & 67.72 & 67.88 & 77.31 & 77.42 & 77.75 & 77.64 \\
 \hline
 32 & 67.25 & 67.72 & 67.72 & 67.00 & 77.31 & 77.04 & 77.80 & 77.37  \\
 \hline
 16 & 68.11 & 68.90 & 67.88 & 67.48 & 77.37 & 78.13 & 78.13 & 77.69  \\
 \hline
\end{tabular}
\caption{\label{tab:mmlu_abalation} Accuracy on LM evaluation harness tasks on GPT3-8B model.}
\end{table}

\begin{table} \centering
\begin{tabular}{|c||c|c|c|c||c|c|c|c|} 
\hline
 $L_b \rightarrow$& \multicolumn{4}{c||}{8} & \multicolumn{4}{c||}{8}\\
 \hline
 \backslashbox{$L_A$\kern-1em}{\kern-1em$N_c$} & 2 & 4 & 8 & 16 & 2 & 4 & 8 & 16  \\
 %$N_c \rightarrow$ & 2 & 4 & 8 & 16 & 2 & 4 & 2 \\
 \hline
 \hline
 \multicolumn{5}{|c|}{Race (FP32 Accuracy = 40.67\%)} & \multicolumn{4}{|c|}{Boolq (FP32 Accuracy = 76.54\%)} \\ 
 \hline
 \hline
 64 & 40.48 & 40.10 & 39.43 & 39.90 & 75.41 & 75.11 & 77.09 & 75.66 \\
 \hline
 32 & 39.52 & 39.52 & 40.77 & 39.62 & 76.02 & 76.02 & 75.96 & 75.35  \\
 \hline
 16 & 39.81 & 39.71 & 39.90 & 40.38 & 75.05 & 73.82 & 75.72 & 76.09  \\
 \hline
 \hline
 \multicolumn{5}{|c|}{Winogrande (FP32 Accuracy = 70.64\%)} & \multicolumn{4}{|c|}{Piqa (FP32 Accuracy = 79.16\%)} \\ 
 \hline
 \hline
 64 & 69.14 & 70.17 & 70.17 & 70.56 & 78.24 & 79.00 & 78.62 & 78.73 \\
 \hline
 32 & 70.96 & 69.69 & 71.27 & 69.30 & 78.56 & 79.49 & 79.16 & 78.89  \\
 \hline
 16 & 71.03 & 69.53 & 69.69 & 70.40 & 78.13 & 79.16 & 79.00 & 79.00  \\
 \hline
\end{tabular}
\caption{\label{tab:mmlu_abalation} Accuracy on LM evaluation harness tasks on GPT3-22B model.}
\end{table}

\begin{table} \centering
\begin{tabular}{|c||c|c|c|c||c|c|c|c|} 
\hline
 $L_b \rightarrow$& \multicolumn{4}{c||}{8} & \multicolumn{4}{c||}{8}\\
 \hline
 \backslashbox{$L_A$\kern-1em}{\kern-1em$N_c$} & 2 & 4 & 8 & 16 & 2 & 4 & 8 & 16  \\
 %$N_c \rightarrow$ & 2 & 4 & 8 & 16 & 2 & 4 & 2 \\
 \hline
 \hline
 \multicolumn{5}{|c|}{Race (FP32 Accuracy = 44.4\%)} & \multicolumn{4}{|c|}{Boolq (FP32 Accuracy = 79.29\%)} \\ 
 \hline
 \hline
 64 & 42.49 & 42.51 & 42.58 & 43.45 & 77.58 & 77.37 & 77.43 & 78.1 \\
 \hline
 32 & 43.35 & 42.49 & 43.64 & 43.73 & 77.86 & 75.32 & 77.28 & 77.86  \\
 \hline
 16 & 44.21 & 44.21 & 43.64 & 42.97 & 78.65 & 77 & 76.94 & 77.98  \\
 \hline
 \hline
 \multicolumn{5}{|c|}{Winogrande (FP32 Accuracy = 69.38\%)} & \multicolumn{4}{|c|}{Piqa (FP32 Accuracy = 78.07\%)} \\ 
 \hline
 \hline
 64 & 68.9 & 68.43 & 69.77 & 68.19 & 77.09 & 76.82 & 77.09 & 77.86 \\
 \hline
 32 & 69.38 & 68.51 & 68.82 & 68.90 & 78.07 & 76.71 & 78.07 & 77.86  \\
 \hline
 16 & 69.53 & 67.09 & 69.38 & 68.90 & 77.37 & 77.8 & 77.91 & 77.69  \\
 \hline
\end{tabular}
\caption{\label{tab:mmlu_abalation} Accuracy on LM evaluation harness tasks on Llama2-7B model.}
\end{table}

\begin{table} \centering
\begin{tabular}{|c||c|c|c|c||c|c|c|c|} 
\hline
 $L_b \rightarrow$& \multicolumn{4}{c||}{8} & \multicolumn{4}{c||}{8}\\
 \hline
 \backslashbox{$L_A$\kern-1em}{\kern-1em$N_c$} & 2 & 4 & 8 & 16 & 2 & 4 & 8 & 16  \\
 %$N_c \rightarrow$ & 2 & 4 & 8 & 16 & 2 & 4 & 2 \\
 \hline
 \hline
 \multicolumn{5}{|c|}{Race (FP32 Accuracy = 48.8\%)} & \multicolumn{4}{|c|}{Boolq (FP32 Accuracy = 85.23\%)} \\ 
 \hline
 \hline
 64 & 49.00 & 49.00 & 49.28 & 48.71 & 82.82 & 84.28 & 84.03 & 84.25 \\
 \hline
 32 & 49.57 & 48.52 & 48.33 & 49.28 & 83.85 & 84.46 & 84.31 & 84.93  \\
 \hline
 16 & 49.85 & 49.09 & 49.28 & 48.99 & 85.11 & 84.46 & 84.61 & 83.94  \\
 \hline
 \hline
 \multicolumn{5}{|c|}{Winogrande (FP32 Accuracy = 79.95\%)} & \multicolumn{4}{|c|}{Piqa (FP32 Accuracy = 81.56\%)} \\ 
 \hline
 \hline
 64 & 78.77 & 78.45 & 78.37 & 79.16 & 81.45 & 80.69 & 81.45 & 81.5 \\
 \hline
 32 & 78.45 & 79.01 & 78.69 & 80.66 & 81.56 & 80.58 & 81.18 & 81.34  \\
 \hline
 16 & 79.95 & 79.56 & 79.79 & 79.72 & 81.28 & 81.66 & 81.28 & 80.96  \\
 \hline
\end{tabular}
\caption{\label{tab:mmlu_abalation} Accuracy on LM evaluation harness tasks on Llama2-70B model.}
\end{table}

%\section{MSE Studies}
%\textcolor{red}{TODO}


\subsection{Number Formats and Quantization Method}
\label{subsec:numFormats_quantMethod}
\subsubsection{Integer Format}
An $n$-bit signed integer (INT) is typically represented with a 2s-complement format \citep{yao2022zeroquant,xiao2023smoothquant,dai2021vsq}, where the most significant bit denotes the sign.

\subsubsection{Floating Point Format}
An $n$-bit signed floating point (FP) number $x$ comprises of a 1-bit sign ($x_{\mathrm{sign}}$), $B_m$-bit mantissa ($x_{\mathrm{mant}}$) and $B_e$-bit exponent ($x_{\mathrm{exp}}$) such that $B_m+B_e=n-1$. The associated constant exponent bias ($E_{\mathrm{bias}}$) is computed as $(2^{{B_e}-1}-1)$. We denote this format as $E_{B_e}M_{B_m}$.  

\subsubsection{Quantization Scheme}
\label{subsec:quant_method}
A quantization scheme dictates how a given unquantized tensor is converted to its quantized representation. We consider FP formats for the purpose of illustration. Given an unquantized tensor $\bm{X}$ and an FP format $E_{B_e}M_{B_m}$, we first, we compute the quantization scale factor $s_X$ that maps the maximum absolute value of $\bm{X}$ to the maximum quantization level of the $E_{B_e}M_{B_m}$ format as follows:
\begin{align}
\label{eq:sf}
    s_X = \frac{\mathrm{max}(|\bm{X}|)}{\mathrm{max}(E_{B_e}M_{B_m})}
\end{align}
In the above equation, $|\cdot|$ denotes the absolute value function.

Next, we scale $\bm{X}$ by $s_X$ and quantize it to $\hat{\bm{X}}$ by rounding it to the nearest quantization level of $E_{B_e}M_{B_m}$ as:

\begin{align}
\label{eq:tensor_quant}
    \hat{\bm{X}} = \text{round-to-nearest}\left(\frac{\bm{X}}{s_X}, E_{B_e}M_{B_m}\right)
\end{align}

We perform dynamic max-scaled quantization \citep{wu2020integer}, where the scale factor $s$ for activations is dynamically computed during runtime.

\subsection{Vector Scaled Quantization}
\begin{wrapfigure}{r}{0.35\linewidth}
  \centering
  \includegraphics[width=\linewidth]{sections/figures/vsquant.jpg}
  \caption{\small Vectorwise decomposition for per-vector scaled quantization (VSQ \citep{dai2021vsq}).}
  \label{fig:vsquant}
\end{wrapfigure}
During VSQ \citep{dai2021vsq}, the operand tensors are decomposed into 1D vectors in a hardware friendly manner as shown in Figure \ref{fig:vsquant}. Since the decomposed tensors are used as operands in matrix multiplications during inference, it is beneficial to perform this decomposition along the reduction dimension of the multiplication. The vectorwise quantization is performed similar to tensorwise quantization described in Equations \ref{eq:sf} and \ref{eq:tensor_quant}, where a scale factor $s_v$ is required for each vector $\bm{v}$ that maps the maximum absolute value of that vector to the maximum quantization level. While smaller vector lengths can lead to larger accuracy gains, the associated memory and computational overheads due to the per-vector scale factors increases. To alleviate these overheads, VSQ \citep{dai2021vsq} proposed a second level quantization of the per-vector scale factors to unsigned integers, while MX \citep{rouhani2023shared} quantizes them to integer powers of 2 (denoted as $2^{INT}$).

\subsubsection{MX Format}
The MX format proposed in \citep{rouhani2023microscaling} introduces the concept of sub-block shifting. For every two scalar elements of $b$-bits each, there is a shared exponent bit. The value of this exponent bit is determined through an empirical analysis that targets minimizing quantization MSE. We note that the FP format $E_{1}M_{b}$ is strictly better than MX from an accuracy perspective since it allocates a dedicated exponent bit to each scalar as opposed to sharing it across two scalars. Therefore, we conservatively bound the accuracy of a $b+2$-bit signed MX format with that of a $E_{1}M_{b}$ format in our comparisons. For instance, we use E1M2 format as a proxy for MX4.

\begin{figure}
    \centering
    \includegraphics[width=1\linewidth]{sections//figures/BlockFormats.pdf}
    \caption{\small Comparing LO-BCQ to MX format.}
    \label{fig:block_formats}
\end{figure}

Figure \ref{fig:block_formats} compares our $4$-bit LO-BCQ block format to MX \citep{rouhani2023microscaling}. As shown, both LO-BCQ and MX decompose a given operand tensor into block arrays and each block array into blocks. Similar to MX, we find that per-block quantization ($L_b < L_A$) leads to better accuracy due to increased flexibility. While MX achieves this through per-block $1$-bit micro-scales, we associate a dedicated codebook to each block through a per-block codebook selector. Further, MX quantizes the per-block array scale-factor to E8M0 format without per-tensor scaling. In contrast during LO-BCQ, we find that per-tensor scaling combined with quantization of per-block array scale-factor to E4M3 format results in superior inference accuracy across models. 

%%%%%%%%%%%%%%%%%%%%%%%%%%%%%%%%%%%%%%%%%%%%%%%%%%%%%%%%%%%%%%%%%%%%%%%%%%%%%%%
%%%%%%%%%%%%%%%%%%%%%%%%%%%%%%%%%%%%%%%%%%%%%%%%%%%%%%%%%%%%%%%%%%%%%%%%%%%%%%%


\end{document}


% This document was modified from the file originally made available by
% Pat Langley and Andrea Danyluk for ICML-2K. This version was created
% by Iain Murray in 2018, and modified by Alexandre Bouchard in
% 2019 and 2021 and by Csaba Szepesvari, Gang Niu and Sivan Sabato in 2022.
% Modified again in 2023 and 2024 by Sivan Sabato and Jonathan Scarlett.
% Previous contributors include Dan Roy, Lise Getoor and Tobias
% Scheffer, which was slightly modified from the 2010 version by
% Thorsten Joachims & Johannes Fuernkranz, slightly modified from the
% 2009 version by Kiri Wagstaff and Sam Roweis's 2008 version, which is
% slightly modified from Prasad Tadepalli's 2007 version which is a
% lightly changed version of the previous year's version by Andrew
% Moore, which was in turn edited from those of Kristian Kersting and
% Codrina Lauth. Alex Smola contributed to the algorithmic style files.
