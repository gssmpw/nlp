% This must be in the first 5 lines to tell arXiv to use pdfLaTeX, which is strongly recommended.
\pdfoutput=1
% In particular, the hyperref package requires pdfLaTeX in order to break URLs across lines.

\documentclass[11pt]{article}

% Change "review" to "final" to generate the final (sometimes called camera-ready) version.
% Change to "preprint" to generate a non-anonymous version with page numbers.
\usepackage[preprint]{acl}

% Standard package includes
\usepackage{times}
\usepackage{latexsym}

% For proper rendering and hyphenation of words containing Latin characters (including in bib files)
\usepackage[T1]{fontenc}
% For Vietnamese characters
% \usepackage[T5]{fontenc}
% See https://www.latex-project.org/help/documentation/encguide.pdf for other character sets

% This assumes your files are encoded as UTF8
\usepackage[utf8]{inputenc}

% This is not strictly necessary, and may be commented out,
% but it will improve the layout of the manuscript,
% and will typically save some space.
\usepackage{microtype}

% This is also not strictly necessary, and may be commented out.
% However, it will improve the aesthetics of text in
% the typewriter font.
\usepackage{inconsolata}

%Including images in your LaTeX document requires adding
%additional package(s)
\usepackage{graphicx}
\usepackage{booktabs}
\usepackage{amssymb}
\usepackage{pifont}
\usepackage{amsmath}
\usepackage{multirow}
\usepackage{colortbl}
\usepackage[most]{tcolorbox}
\usepackage{xcolor}
\usepackage{fvextra}
\usepackage{fancyvrb}
\DefineVerbatimEnvironment{CustomVerbatim}{Verbatim}{formatcom=\normalfont, breaklines=true, breaksymbolleft={}, breaksymbolright={}}
\usepackage{ragged2e}
\usepackage[normalem]{ulem}
\useunder{\uline}{\ul}{}


% If the title and author information does not fit in the area allocated, uncomment the following
%
%\setlength\titlebox{<dim>}
%
% and set <dim> to something 5cm or larger.

\title{LR$^2$Bench: Evaluating Long-chain Reflective Reasoning Capabilities of Large Language Models via Constraint Satisfaction Problems}


% Author information can be set in various styles:
% For several authors from the same institution:
% \author{Author 1 \and ... \and Author n \\
%         Address line \\ ... \\ Address line}
% if the names do not fit well on one line use
%         Author 1 \\ {\bf Author 2} \\ ... \\ {\bf Author n} \\
% For authors from different institutions:
% \author{Author 1 \\ Address line \\  ... \\ Address line
%         \And  ... \And
%         Author n \\ Address line \\ ... \\ Address line}
% To start a separate ``row'' of authors use \AND, as in
% \author{Author 1 \\ Address line \\  ... \\ Address line
%         \AND
%         Author 2 \\ Address line \\ ... \\ Address line \And
%         Author 3 \\ Address line \\ ... \\ Address line}

% \author{First Author \\
%   Affiliation / Address line 1 \\
%   Affiliation / Address line 2 \\
%   Affiliation / Address line 3 \\
%   \texttt{email@domain} \\\And
%   Second Author \\
%   Affiliation / Address line 1 \\
%   Affiliation / Address line 2 \\
%   Affiliation / Address line 3 \\
%   \texttt{email@domain} \\}

\author{
    Jianghao Chen\textsuperscript{1,2,3}, \ 
    Zhenlin Wei\textsuperscript{1,2},\ 
    Zhenjiang Ren\textsuperscript{1,2}, \
    Ziyong Li\textsuperscript{1,2}, \
    Jiajun Zhang\textsuperscript{1,2,4,5}\thanks{\ \ Corresponding Author}   \\
    \textsuperscript{1}Institute of Automation, Chinese Academy of Sciences\\
    \textsuperscript{2}School of Artificial Intelligence, University of Chinese Academy of Sciences\\
    \textsuperscript{3}Zhongguancun Academy, Beijing, China\\
    \textsuperscript{4}Wuhan AI Research, 
    \textsuperscript{5}Shanghai Artificial Intelligence Laboratory, Shanghai, China\\
    \texttt{\{chenjianghao2022, weizhenlin2025, renzhenjiang2024, liziyong2023\}@ia.ac.cn}\\
    \texttt{jjzhang@nlpr.ia.ac.cn} \\
}


%\author{
%  \textbf{First Author\textsuperscript{1}},
%  \textbf{Second Author\textsuperscript{1,2}},
%  \textbf{Third T. Author\textsuperscript{1}},
%  \textbf{Fourth Author\textsuperscript{1}},
%\\
%  \textbf{Fifth Author\textsuperscript{1,2}},
%  \textbf{Sixth Author\textsuperscript{1}},
%  \textbf{Seventh Author\textsuperscript{1}},
%  \textbf{Eighth Author \textsuperscript{1,2,3,4}},
%\\
%  \textbf{Ninth Author\textsuperscript{1}},
%  \textbf{Tenth Author\textsuperscript{1}},
%  \textbf{Eleventh E. Author\textsuperscript{1,2,3,4,5}},
%  \textbf{Twelfth Author\textsuperscript{1}},
%\\
%  \textbf{Thirteenth Author\textsuperscript{3}},
%  \textbf{Fourteenth F. Author\textsuperscript{2,4}},
%  \textbf{Fifteenth Author\textsuperscript{1}},
%  \textbf{Sixteenth Author\textsuperscript{1}},
%\\
%  \textbf{Seventeenth S. Author\textsuperscript{4,5}},
%  \textbf{Eighteenth Author\textsuperscript{3,4}},
%  \textbf{Nineteenth N. Author\textsuperscript{2,5}},
%  \textbf{Twentieth Author\textsuperscript{1}}
%\\
%\\
%  \textsuperscript{1}Affiliation 1,
%  \textsuperscript{2}Affiliation 2,
%  \textsuperscript{3}Affiliation 3,
%  \textsuperscript{4}Affiliation 4,
%  \textsuperscript{5}Affiliation 5
%\\
%  \small{
%    \textbf{Correspondence:} \href{mailto:email@domain}{email@domain}
%  }
%}

\begin{document}
\maketitle
\begin{abstract}
Recent progress in o1-like models has significantly enhanced the reasoning abilities of Large Language Models (LLMs), empowering them to tackle increasingly complex tasks through reflection capabilities, such as making assumptions, backtracking, and self-refinement. However, effectively evaluating such reflection capabilities remains challenging due to the lack of appropriate benchmarks. To bridge this gap, we introduce LR$^2$Bench, a novel benchmark designed to evaluate the \textbf{L}ong-chain \textbf{R}eflective \textbf{R}easoning capabilities of LLMs. LR$^2$Bench comprises 850 samples across six Constraint Satisfaction Problems (CSPs) where reflective reasoning is crucial for deriving solutions that meet all given constraints. Each type of task focuses on distinct constraint patterns, such as knowledge-based, logical, and spatial constraints, providing a comprehensive evaluation of diverse problem-solving scenarios. We conduct extensive evaluation on both conventional models and o1-like models. Our experimental results reveal that even the most advanced reasoning-specific models, such as DeepSeek-R1 and OpenAI o1-preview, struggle with tasks in LR$^2$Bench, achieving an average Exact Match score of only 20.0\% and 23.6\%, respectively. These findings underscore the significant room for improvement in the reflective reasoning capabilities of current LLMs. The leaderboard of our benchmark is available at \url{https://huggingface.co/spaces/UltraRonin/LR2Bench}.
\end{abstract}



\section{Introduction}



Recent advancements in Large Language Models (LLMs), exemplified by o1-like models \citep{o1, qwq-32b-preview, guo2025deepseek}, have demonstrated substantial progress in their reasoning capabilities. These models exhibit more human-like behaviors, such as making assumptions, verification, backtracking, and self-correction, enabling them to address increasingly complex challenges across various domains \citep{zhong2024evaluation, wu2024comparative, gao2024omni, wang2024planning}.

\begin{figure}[t]
  \centering
  \includegraphics[width=0.9\columnwidth]{fig/sample.drawio.pdf}
  \caption{The response demonstrates the model's reflective reasoning process while solving a Sudoku problem.}
  \label{motivation}
  \vspace{-4mm}
\end{figure}


\section{Overview}

\revision{In this section, we first explain the foundational concept of Hausdorff distance-based penetration depth algorithms, which are essential for understanding our method (Sec.~\ref{sec:preliminary}).
We then provide a brief overview of our proposed RT-based penetration depth algorithm (Sec.~\ref{subsec:algo_overview}).}



\section{Preliminaries }
\label{sec:Preliminaries}

% Before we introduce our method, we first overview the important basics of 3D dynamic human modeling with Gaussian splatting. Then, we discuss the diffusion-based 3d generation techniques, and how they can be applied to human modeling.
% \ZY{I stopp here. TBC.}
% \subsection{Dynamic human modeling with Gaussian splatting}
\subsection{3D Gaussian Splatting}
3D Gaussian splatting~\cite{kerbl3Dgaussians} is an explicit scene representation that allows high-quality real-time rendering. The given scene is represented by a set of static 3D Gaussians, which are parameterized as follows: Gaussian center $x\in {\mathbb{R}^3}$, color $c\in {\mathbb{R}^3}$, opacity $\alpha\in {\mathbb{R}}$, spatial rotation in the form of quaternion $q\in {\mathbb{R}^4}$, and scaling factor $s\in {\mathbb{R}^3}$. Given these properties, the rendering process is represented as:
\begin{equation}
  I = Splatting(x, c, s, \alpha, q, r),
  \label{eq:splattingGA}
\end{equation}
where $I$ is the rendered image, $r$ is a set of query rays crossing the scene, and $Splatting(\cdot)$ is a differentiable rendering process. We refer readers to Kerbl et al.'s paper~\cite{kerbl3Dgaussians} for the details of Gaussian splatting. 



% \ZY{I would suggest move this part to the method part.}
% GaissianAvatar is a dynamic human generation model based on Gaussian splitting. Given a sequence of RGB images, this method utilizes fitted SMPLs and sampled points on its surface to obtain a pose-dependent feature map by a pose encoder. The pose-dependent features and a geometry feature are fed in a Gaussian decoder, which is employed to establish a functional mapping from the underlying geometry of the human form to diverse attributes of 3D Gaussians on the canonical surfaces. The parameter prediction process is articulated as follows:
% \begin{equation}
%   (\Delta x,c,s)=G_{\theta}(S+P),
%   \label{eq:gaussiandecoder}
% \end{equation}
%  where $G_{\theta}$ represents the Gaussian decoder, and $(S+P)$ is the multiplication of geometry feature S and pose feature P. Instead of optimizing all attributes of Gaussian, this decoder predicts 3D positional offset $\Delta{x} \in {\mathbb{R}^3}$, color $c\in\mathbb{R}^3$, and 3D scaling factor $ s\in\mathbb{R}^3$. To enhance geometry reconstruction accuracy, the opacity $\alpha$ and 3D rotation $q$ are set to fixed values of $1$ and $(1,0,0,0)$ respectively.
 
%  To render the canonical avatar in observation space, we seamlessly combine the Linear Blend Skinning function with the Gaussian Splatting~\cite{kerbl3Dgaussians} rendering process: 
% \begin{equation}
%   I_{\theta}=Splatting(x_o,Q,d),
%   \label{eq:splatting}
% \end{equation}
% \begin{equation}
%   x_o = T_{lbs}(x_c,p,w),
%   \label{eq:LBS}
% \end{equation}
% where $I_{\theta}$ represents the final rendered image, and the canonical Gaussian position $x_c$ is the sum of the initial position $x$ and the predicted offset $\Delta x$. The LBS function $T_{lbs}$ applies the SMPL skeleton pose $p$ and blending weights $w$ to deform $x_c$ into observation space as $x_o$. $Q$ denotes the remaining attributes of the Gaussians. With the rendering process, they can now reposition these canonical 3D Gaussians into the observation space.



\subsection{Score Distillation Sampling}
Score Distillation Sampling (SDS)~\cite{poole2022dreamfusion} builds a bridge between diffusion models and 3D representations. In SDS, the noised input is denoised in one time-step, and the difference between added noise and predicted noise is considered SDS loss, expressed as:

% \begin{equation}
%   \mathcal{L}_{SDS}(I_{\Phi}) \triangleq E_{t,\epsilon}[w(t)(\epsilon_{\phi}(z_t,y,t)-\epsilon)\frac{\partial I_{\Phi}}{\partial\Phi}],
%   \label{eq:SDSObserv}
% \end{equation}
\begin{equation}
    \mathcal{L}_{\text{SDS}}(I_{\Phi}) \triangleq \mathbb{E}_{t,\epsilon} \left[ w(t) \left( \epsilon_{\phi}(z_t, y, t) - \epsilon \right) \frac{\partial I_{\Phi}}{\partial \Phi} \right],
  \label{eq:SDSObservGA}
\end{equation}
where the input $I_{\Phi}$ represents a rendered image from a 3D representation, such as 3D Gaussians, with optimizable parameters $\Phi$. $\epsilon_{\phi}$ corresponds to the predicted noise of diffusion networks, which is produced by incorporating the noise image $z_t$ as input and conditioning it with a text or image $y$ at timestep $t$. The noise image $z_t$ is derived by introducing noise $\epsilon$ into $I_{\Phi}$ at timestep $t$. The loss is weighted by the diffusion scheduler $w(t)$. 
% \vspace{-3mm}

\subsection{Overview of the RTPD Algorithm}\label{subsec:algo_overview}
Fig.~\ref{fig:Overview} presents an overview of our RTPD algorithm.
It is grounded in the Hausdorff distance-based penetration depth calculation method (Sec.~\ref{sec:preliminary}).
%, similar to that of Tang et al.~\shortcite{SIG09HIST}.
The process consists of two primary phases: penetration surface extraction and Hausdorff distance calculation.
We leverage the RTX platform's capabilities to accelerate both of these steps.

\begin{figure*}[t]
    \centering
    \includegraphics[width=0.8\textwidth]{Image/overview.pdf}
    \caption{The overview of RT-based penetration depth calculation algorithm overview}
    \label{fig:Overview}
\end{figure*}

The penetration surface extraction phase focuses on identifying the overlapped region between two objects.
\revision{The penetration surface is defined as a set of polygons from one object, where at least one of its vertices lies within the other object. 
Note that in our work, we focus on triangles rather than general polygons, as they are processed most efficiently on the RTX platform.}
To facilitate this extraction, we introduce a ray-tracing-based \revision{Point-in-Polyhedron} test (RT-PIP), significantly accelerated through the use of RT cores (Sec.~\ref{sec:RT-PIP}).
This test capitalizes on the ray-surface intersection capabilities of the RTX platform.
%
Initially, a Geometry Acceleration Structure (GAS) is generated for each object, as required by the RTX platform.
The RT-PIP module takes the GAS of one object (e.g., $GAS_{A}$) and the point set of the other object (e.g., $P_{B}$).
It outputs a set of points (e.g., $P_{\partial B}$) representing the penetration region, indicating their location inside the opposing object.
Subsequently, a penetration surface (e.g., $\partial B$) is constructed using this point set (e.g., $P_{\partial B}$) (Sec.~\ref{subsec:surfaceGen}).
%
The generated penetration surfaces (e.g., $\partial A$ and $\partial B$) are then forwarded to the next step. 

The Hausdorff distance calculation phase utilizes the ray-surface intersection test of the RTX platform (Sec.~\ref{sec:RT-Hausdorff}) to compute the Hausdorff distance between two objects.
We introduce a novel Ray-Tracing-based Hausdorff DISTance algorithm, RT-HDIST.
It begins by generating GAS for the two penetration surfaces, $P_{\partial A}$ and $P_{\partial B}$, derived from the preceding step.
RT-HDIST processes the GAS of a penetration surface (e.g., $GAS_{\partial A}$) alongside the point set of the other penetration surface (e.g., $P_{\partial B}$) to compute the penetration depth between them.
The algorithm operates bidirectionally, considering both directions ($\partial A \to \partial B$ and $\partial B \to \partial A$).
The final penetration depth between the two objects, A and B, is determined by selecting the larger value from these two directional computations.

%In the Hausdorff distance calculation step, we compute the Hausdorff distance between given two objects using a ray-surface-intersection test. (Sec.~\ref{sec:RT-Hausdorff}) Initially, we construct the GAS for both $\partial A$ and $\partial B$ to utilize the RT-core effectively. The RT-based Hausdorff distance algorithms then determine the Hausdorff distance by processing the GAS of one object (e.g. $GAS_{\partial A}$) and set of the vertices of the other (e.g. $P_{\partial B}$). Following the Hausdorff distance definition (Eq.~\ref{equation:hausdorff_definition}), we compute the Hausdorff distance to both directions ($\partial A \to \partial B$) and ($\partial B \to \partial A$). As a result, the bigger one is the final Hausdorff distance, and also it is the penetration depth between input object $A$ and $B$.


%the proposed RT-based penetration depth calculation pipeline.
%Our proposed methods adopt Tang's Hausdorff-based penetration depth methods~\cite{SIG09HIST}. The pipeline is divided into the penetration surface extraction step and the Hausdorff distance calculation between the penetration surface steps. However, since Tang's approach is not suitable for the RT platform in detail, we modified and applied it with appropriate methods.

%The penetration surface extraction step is extracting overlapped surfaces on other objects. To utilize the RT core, we use the ray-intersection-based PIP(Point-In-Polygon) algorithms instead of collision detection between two objects which Tang et al.~\cite{SIG09HIST} used. (Sec.~\ref{sec:RT-PIP})
%RT core-based PIP test uses a ray-surface intersection test. For purpose this, we generate the GAS(Geometry Acceleration Structure) for each object. RT core-based PIP test takes the GAS of one object (e.g. $GAS_{A}$) and a set of vertex of another one (e.g. $P_{B}$). Then this computes the penetrated vertex set of another one (e.g. $P_{\partial B}$). To calculate the Hausdorff distance, these vertex sets change to objects constructed by penetrated surface (e.g. $\partial B$). Finally, the two generated overlapped surface objects $\partial A$ and $\partial B$ are used in the Hausdorff distance calculation step.

Despite this progress, the reflection capabilities of LLMs remain largely unexplored. Reflection can be defined as the process of engaging in attentive, critical, exploratory, and iterative self-interactions with one's thoughts, actions, and underlying conceptual frameworks \citep{nguyen2014reflection}. Existing research predominantly focuses on evaluating LLMs' abilities to utilize explicit feedback for self-criticism or self-refinement \citep{li2024hindsight, lan2024criticeval, gou2024critic, sun2024critique, lin-etal-2024-criticbench}. However, these approaches overlook a fundamental question: \textbf{\textit{Can LLMs spontaneously engage in the whole reflection process to solve more complex tasks?}} Reflection is not simply about reacting to feedback. It also encompasses exploring possible solutions, assessing and adjusting strategies, and adapting when confronted with contradiction. To better understand how LLMs might engage in these capabilities, we consider Constraint Satisfaction Problems (CSPs).
% These capabilities become particularly crucial when addressing multi-step complex tasks like Constraint Satisfaction Problems (CSPs). 
CSPs are defined as a set of variables whose state must satisfy specific constraints, which inherently require iterative exploration in a large search space without predefined solution paths \citep{dechter2003constraint}. As shown in Figure \ref{motivation}, the response of QwQ-32B-Preview \citep{qwq-32b-preview} to Sudoku, a classic CSP, exemplifies a typical reflection process. This task presents a scenario with row, column, and grid constraints, requiring iterative trial-and-error to determine valid values for each cell. The model initially makes an assumption, then identifies a conflict, and finally backtracks to correct its initial guess. This demonstrates an effective reflective reasoning process often absent from standard NLP tasks, such as summarization, translation, and question-answering.


Therefore, to systematically investigate reflective reasoning in LLMs, we propose LR$^2$Bench, a novel benchmark for evaluating the \textbf{L}ong-chain \textbf{R}eflective \textbf{R}easoning capabilities of LLMs. LR$^2$Bench consists of six CSPs: Crossword, Acrostic, Logic Puzzle, Cryptogram, Sudoku, and Drop Quote. Each task necessitates reflection processes and emphasizes specific constraint patterns, such as knowledge-based constraints, logical constraints, and spatial constraints, ensuring a comprehensive assessment of LLMs' reflective reasoning capabilities across diverse problem-solving scenarios. We manually collect and annotate a total of 850 examples spanning multiple difficulty levels across all task types. Through extensive experiments on both convention and o1-like models,
we observe that even current top-performed reasoning LLMs (e.g., DeepSeek-R1 and o1-preview) struggle to complete tasks in LR$^2$Bench, underscoring substantial room for improvement in this critical area. Furthermore, most conventional models exhibit inferior performance than o1-like models and achieve near-zero sample-level accuracy in LR$^2$Bench, highlighting the essential role of reflective reasoning skills in tackling the challenges posed by our benchmark.

% 2) While open-source reasoning LLMs (e.g.,  QwQ-32B-Preview) show better performance than conventional models in some tasks, they fail in non-reasoning domains like instruction following.

Our contributions are summarized as follows: 1) We introduce LR$^2$Bench, a novel benchmark with six tasks and diverse difficulty levels, designed to evaluate the long-chain reflective reasoning capabilities of LLMs. 2) We conduct a comprehensive evaluation of reflective reasoning capabilities in both conventional and reasoning-specific LLMs. 3) We present a detailed analysis of the limitations exhibited by current LLMs on LR$^2$Bench, establishing a foundation for future research on the development of advanced reasoning models.


\begin{figure*}[t]
  \includegraphics[width=\linewidth]{fig/collection.drawio_new.pdf}
  \caption{The overall pipeline of LR$^2$Bench, illustrated with the Sudoku task.}
  \label{framework}
  \vspace{-4mm}
\end{figure*}


\section{LR$^2$Bench}
In this section, we introduce the construction process of LR$^2$Bench. Section \ref{sub1} provides an overview of each task. Section \ref{sub2} discusses the diverse scenarios targeted by LR${}^2$Bench. Section \ref{sub3} shows the annotation process and data statistics, and Section \ref{sub4} outlines the evaluation metrics. Figure \ref{framework} illustrates the data collection, annotation, and evaluation pipeline of our LR$^2$Bench.


\subsection{Task Overview}
\label{sub1}
We first introduce the task descriptions and collection methods for all tasks in LR$^2$Bench. For detailed information, please refer to Appendix \ref{app_example}.

\paragraph{Crossword} The Crossword task requires inferring correct words from given clues and filling them into a grid. A key challenge lies in satisfying the constraint of shared letter intersections between horizontal and vertical words. We collected 150 Crossword samples published in 2024 from Los Angeles Times\footnote{\url{https://www.latimes.com}} and Vulture\footnote{\url{https://www.vulture.com}} in three sizes: $5\times5$, $10\times10$, and $15\times15$, with 50 ones for each size.

\paragraph{Acrostic} The Acrostic task involves word clues like Crossword, but its objective is to form a hidden quotation or sentence from the answers to the clues. This requires that the answer words not only satisfy the corresponding clues but also effectively integrate to construct the ultimate hidden message. We collected 50 easy and 50 hard Acrostic samples from Printable Puzzles\footnote{\url{https://www.printable-puzzles.com/printable-acrostic-puzzles.php}} with timestamps ranging from September 2024 to December 2024.

\paragraph{Logic Puzzle} The Logic Puzzle task constitutes a problem that necessitates logical reasoning to deduce relationships between a set of entities based on the given constraints and clues. The objective is to systematically analyze the given information, employing techniques such as hypothesis formation, elimination, and deductive inference, to determine a unique solution that satisfies all given constraints. We collected 50 puzzles for each of the four sizes ($4\times4$, $4\times5$, $4\times6$, and $4\times7$) from Printable Puzzles\footnote{\url{https://www.printable-puzzles.com/printable-logic-puzzles.php}}, with timestamps ranging from September 2024 to December 2024.

\paragraph{Cryptogram} The Cryptogram task involves the decryption of an encrypted quotation or sentence, where each letter of an original text is substituted with another, resulting in an apparently nonsense text. Decryption requires identifying patterns, common letter frequencies, and word structures to deduce the letter-to-letter correspondences, ultimately reconstructing the original content. We collected 50 easy and 50 hard samples from Printable Puzzles\footnote{\url{https://www.printable-puzzles.com/printable-cryptograms.php}} with timestamps ranging from September 2024 to December 2024.

\paragraph{Sudoku} The Sudoku task consists of filling a $n^2 \times n^2$ grid with digits from 1 to $n^2$, subject to the constraint that each row, column, and $n \times n$ subgrid contains all digits from 1 to $n^2$ without repetition. Success in Sudoku relies on logical deduction and careful consideration of the existing digits to determine valid placements for the remaining numbers. From 1sudoku\footnote{\url{https://1sudoku.com}}, we collected 200 Sudoku samples in total: 50 easy and 50 hard samples for both $4\times4$ and $9\times9$ sizes.

\paragraph{Drop Quote} The Drop Quote task comprises a grid of multiple rows and columns, with each column providing a set of candidate letters. The task requires determining the correct row for letters in each column, effectively "dropping" it into target place to reveal the hidden quotation. We created 50 easy samples by manually compiling common quotations, and collected 50 hard samples from Printable Puzzles\footnote{\url{https://www.printable-puzzles.com/printable-drop-quotes.php}}, with timestamps ranging from September 2024 to December 2024.

% \subsection{What Can LR$^2$Bench Tell?}
\subsection{Diverse Scenarios}
\label{sub2}

Each type of task within LR$^2$Bench focuses on different constraint patterns, providing a comprehensive framework to evaluate models’ reflective reasoning capabilities across diverse scenarios. We further explore the varying capabilities required for completing the tasks in LR$^2$Bench.

\paragraph{Reflection} Reflection is the most fundamental capability for tackling the complex tasks in LR$^2$Bench. Unlike simple problems with short-cut solutions, these tasks are inherently iterative, demanding the exploration of multiple possibilities, identification of dead ends, and adaptive revision of initial hypotheses. Such reflective reasoning capabilities enable a thorough analysis and refinement of strategies, ultimately leading to more robust and effective solutions.

\paragraph{Long-chain Generation} LR$^2$Bench incorporates tasks that necessitate long-chain generation, a crucial capability for LLMs to tackle complex reasoning problems. Unlike tasks with simple and isolated answers, these tasks require LLMs to generate a long chain of steps or decisions that build upon each other toward a final solution.

\paragraph{Knowledge-based Reasoning} Both Crossword and Acrostic tasks demand broad world knowledge and commonsense reasoning abilities since the clue answers often hinge on cultural references, idiomatic expressions, and diverse factual domains. Additionally, the Cryptogram and Drop Quote tasks require knowledge of typical phrase structures and common linguistic patterns to decode messages or reconstruct quotations effectively.

\paragraph{Logical Reasoning} The logical reasoning ability is essential across various tasks, especially evident in the Logic Puzzles and Sudoku. These tasks involve information integration and systematic application of deductive reasoning to solve problems constrained by specific rules.

\paragraph{Spatial Reasoning} While not the primary focus for all tasks, spatial reasoning also emerges as a critical capability within LR$^2$Bench, particularly in tasks with grids. For Crossword, considering letter intersections across horizontal and vertical placements is crucial. Similarly, in Sudoku, effective digit placement requires reasoning about row, column, and subgrid constraints, all of which involve spatial relationships within the grid.


\subsection{Data Annotation and Statistics}
\label{sub3}
For PDF data collected from websites, we manually convert key elements in task samples into a structured text format suitable for LLMs' inputs. Then, we manually craft task-specific instructions, including problem definitions and rules, to provide LLMs with the necessary guidance to process each task effectively. To further control the output format of LLMs, we manually construct two simple shots for each type of task to facilitate subsequent answer extraction and evaluation. Table \ref{tab:overview} presents all tasks in LR$^2$Bench along with their attributes. Appendix \ref{app_example} shows the text-format task examples. Appendix \ref{app_prompt} shows the detailed instructions and few-shot examples for each type of task.

\subsection{Evaluation Metrics}
\label{sub4}
Since all the tasks in LR$^2$Bench consist of multiple subtasks (e.g., words inference in Crossword, cells completion in Sudoku), we employ fine-grained, subtask-level evaluation metrics. Given a problem with $N$ subtasks, let $G = \{ g_1, \ldots, g_N \}$ and $P = \{ p_1, \ldots, p_N \}$ denote the ground truth and LLM-generated answers for each subtask, respectively. We define the following evaluation metrics:


\paragraph{Completion Ratio} The Completion Ratio (CR) metric measures the proportion of subtasks within a given problem that LLMs successfully complete, regardless of the correctness of the answers. 
CR is calculated as follows:
\begin{equation}\label{cr}
% \small \text{CR} = \frac{\text{Number of Completed Samples}}{\text{Total Samples Number}}
\text{CR} = \frac{1}{N} \sum_{i=1}^{N} \mathbb{I}(p_i \neq \varnothing)
\end{equation}
where $\mathbb{I}(p_i \neq \varnothing)$ equals 1 if subtask $i$ is completed (i.e., $p_i$ is not empty) and 0 otherwise. In our experiments, we observe that some models struggle to produce a complete reasoning chain and fail to reach the final answer for each subtask. Therefore, we propose CR to measure the long reasoning chain generation capability of LLMs.


\paragraph{Exact Match} The Exact Match (EM) metric employs a strict correctness criterion: for each subtask, the generated answer by LLMs must exactly match the ground truth. EM is calculated as follows:
\begin{equation}\label{em}
% \small \text{EM}= \frac{\text{Number of Exactly Matched Samples}}{\text{Total Samples Number}}
% \text{EM} = \mathbb{I}\left( \bigwedge_{i=1}^{N} (p_i = g_i) \right)
\text{EM} = \mathbb{I}\left( \frac{1}{N} \sum_{i=1}^{N} \mathbb{I}(p_i = g_i) = 1.0 \right)
\end{equation}
This metric is crucial for evaluating the absolute accuracy on individual samples. For example, in a Sudoku task, EM would require that the completed grid be identical to the ground truth, with no errors in any row, column, or subgrid.

\paragraph{Partial Match} To mitigate the strictness of EM, we employ a Partial Match metric (PM-0.5). This metric relaxes the requirement for a perfect match: one LLM-generated answer is considered correct under PM-0.5 if at least 50\% of its subtasks are correct.
PM-0.5 is calculated as follows:
\begin{equation}\label{pm05}
% \small \text{PM-0.5} = \frac{\text{Number of Samples with more than 50\% Match}}{\text{Total Samples Number}}
\text{PM-0.5} = \mathbb{I}\left( \frac{1}{N} \sum_{i=1}^{N} \mathbb{I}(p_i = g_i) \geq 0.5 \right)
\end{equation}


\paragraph{Subtask Accuracy} For more fine-grained evaluation, we propose Subtask Accuracy (S-Acc), which calculates the proportion of correctly solved subtasks compared to the ground truth:
\begin{equation}\label{s-acc}
\text{S-Acc} = \frac{1}{N} \sum_{i=1}^{N} \mathbb{I}(p_i = g_i)
\end{equation}
By considering both S-Acc and EM, we can gain a more comprehensive understanding of model performance across complex reasoning tasks.


\section{Experiments}

\subsection{Experimental Setup}
% todo
\paragraph{Evaluated Models} We evaluate 11 open-source models and 5 closed-source models. Specifically, we include leading o1-like models: QwQ-32B-Preview \citep{qwq-32b-preview}, DeepSeek-R1 \citep{guo2025deepseek}, Gemini-2.0-flash-thinking \citep{deepmind_gemini_flash_thinking}, OpenAI o1-mini and o1-preview \citep{o1}. Please refer to Appendix \ref{app_models} for detailed information of all selected models.

\paragraph{Implementation Details} We utilize the default prompt templates for all LLMs. Appendix \ref{app_example} and \ref{app_prompt} show detailed problem formats, instructions, and few-shot examples for each type of task. To facilitate evaluation, we wrap answers within specific tags (e.g., \texttt{<Answer>} and \texttt{</Answer>}) in the few-shot examples. This enables precise answer extraction from the model responses. We use the vLLM inference framework \citep{kwon2023efficient} and employ greedy sampling with temperature = 0 to minimize randomness, except for o1-mini and o1-preview which have an inherent temperature of 1. The maximum sequence length is set to the default maximum value for each model.

\subsection{Main Results}


Table \ref{tab:main_exp_sum} presents the average performance across six tasks on LR$^2$Bench, with individual task results detailed in Table \ref{tab:main_exp}.

\begin{table}[!h]
    \centering
    \resizebox{1.0\columnwidth}{!}{
    % \setlength{\tabcolsep}{3pt}
    \begin{tabular}{l|ccccc}
        \toprule
        \textbf{Model} & \textbf{CR} & \textbf{S-Acc} & \textbf{EM} & \textbf{PM-0.5} & \textbf{Tokens} \\
        \midrule
        \rowcolor{gray!15} \multicolumn{6}{l}{\textit{\textbf{Open-source LLMs}}}\\
        \midrule 
        Llama-3.1-8B-Instruct & 42.6 & 9.9 & 0.0 & 3.8 & 2,478 \\
        Llama-3.1-70B-Instruct & 71.8 & 27.4 & 0.5 & 21.9 & 2,090 \\
        Llama-3.3-70B-Instruct & 92.4 & 33.1 & 1.3 & 25.8 & 1,842 \\
        Mistral-7B-Instruct-v0.3 & 85.8 & 12.1 & 0.0 & 2.3 & 2,736 \\
        Mistral-Small-Instruct-2409 & 91.0 & 23.1 & 0.2 & 13.3 & 2,273 \\
        Mistral-Large-Instruct-2411 & 96.1 & \underline{36.4} & 2.5 & \underline{30.0} & 2,313 \\
        Qwen2.5-7B-Instruct & 85.1 & 17.7 & 0.3 & 5.1 & 2,086 \\
        Qwen2.5-32B-Instruct & \underline{96.2} & 29.9 & 0.6 & 14.8 & 1,924 \\
        Qwen2.5-72B-Instruct & 95.0 & 33.9 & 0.9 & 20.8 & 2,359 \\
        QwQ-32B-Preview & 65.0 & 26.6 & \underline{8.5} & 19.3 & 6,709 \\
        DeepSeek-R1 & \textbf{100.0} & \textbf{58.4} & \textbf{20.0} & \textbf{62.0} & 9,856 \\
        \midrule
        \rowcolor{blue!15} \multicolumn{6}{l}{\textit{\textbf{Closed-source LLMs}}}\\
        \midrule
        Gemini-2.0-flash & 81.1 & 37.0 & 2.4 & 34.5 & 2,637 \\
        Gemini-2.0-flash-thinking & 88.2 & 39.4 & 4.3 & 35.0 & 3,725 \\
        GPT-4o & \textbf{99.8} & \underline{43.7} & 3.2 & \underline{41.7} & 1,486 \\
        o1-mini & \underline{97.7} & 41.3 & \underline{9.1} & 32.8 & 9,576 \\
        o1-preview & 96.3 & \textbf{58.7} & \textbf{23.6} & \textbf{61.7} & 11,436 \\ 
        \bottomrule
    \end{tabular}}    
    \caption{Average performance (\%) across six tasks on LR${}^{2}$Bench. The best and second-best results are highlighted in \textbf{bold} and \underline{underlined}, respectively. "Tokens" denotes the average number of generated tokens.}
    \label{tab:main_exp_sum}
    % \vspace{-2mm}
\end{table}
% This must be in the first 5 lines to tell arXiv to use pdfLaTeX, which is strongly recommended.
\pdfoutput=1
% In particular, the hyperref package requires pdfLaTeX in order to break URLs across lines.

\documentclass[11pt]{article}

% Change "review" to "final" to generate the final (sometimes called camera-ready) version.
% Change to "preprint" to generate a non-anonymous version with page numbers.
\usepackage{acl}

% Standard package includes
\usepackage{times}
\usepackage{latexsym}

% Draw tables
\usepackage{booktabs}
\usepackage{multirow}
\usepackage{xcolor}
\usepackage{colortbl}
\usepackage{array} 
\usepackage{amsmath}

\newcolumntype{C}{>{\centering\arraybackslash}p{0.07\textwidth}}
% For proper rendering and hyphenation of words containing Latin characters (including in bib files)
\usepackage[T1]{fontenc}
% For Vietnamese characters
% \usepackage[T5]{fontenc}
% See https://www.latex-project.org/help/documentation/encguide.pdf for other character sets
% This assumes your files are encoded as UTF8
\usepackage[utf8]{inputenc}

% This is not strictly necessary, and may be commented out,
% but it will improve the layout of the manuscript,
% and will typically save some space.
\usepackage{microtype}
\DeclareMathOperator*{\argmax}{arg\,max}
% This is also not strictly necessary, and may be commented out.
% However, it will improve the aesthetics of text in
% the typewriter font.
\usepackage{inconsolata}

%Including images in your LaTeX document requires adding
%additional package(s)
\usepackage{graphicx}
% If the title and author information does not fit in the area allocated, uncomment the following
%
%\setlength\titlebox{<dim>}
%
% and set <dim> to something 5cm or larger.

\title{Wi-Chat: Large Language Model Powered Wi-Fi Sensing}

% Author information can be set in various styles:
% For several authors from the same institution:
% \author{Author 1 \and ... \and Author n \\
%         Address line \\ ... \\ Address line}
% if the names do not fit well on one line use
%         Author 1 \\ {\bf Author 2} \\ ... \\ {\bf Author n} \\
% For authors from different institutions:
% \author{Author 1 \\ Address line \\  ... \\ Address line
%         \And  ... \And
%         Author n \\ Address line \\ ... \\ Address line}
% To start a separate ``row'' of authors use \AND, as in
% \author{Author 1 \\ Address line \\  ... \\ Address line
%         \AND
%         Author 2 \\ Address line \\ ... \\ Address line \And
%         Author 3 \\ Address line \\ ... \\ Address line}

% \author{First Author \\
%   Affiliation / Address line 1 \\
%   Affiliation / Address line 2 \\
%   Affiliation / Address line 3 \\
%   \texttt{email@domain} \\\And
%   Second Author \\
%   Affiliation / Address line 1 \\
%   Affiliation / Address line 2 \\
%   Affiliation / Address line 3 \\
%   \texttt{email@domain} \\}
% \author{Haohan Yuan \qquad Haopeng Zhang\thanks{corresponding author} \\ 
%   ALOHA Lab, University of Hawaii at Manoa \\
%   % Affiliation / Address line 2 \\
%   % Affiliation / Address line 3 \\
%   \texttt{\{haohany,haopengz\}@hawaii.edu}}
  
\author{
{Haopeng Zhang$\dag$\thanks{These authors contributed equally to this work.}, Yili Ren$\ddagger$\footnotemark[1], Haohan Yuan$\dag$, Jingzhe Zhang$\ddagger$, Yitong Shen$\ddagger$} \\
ALOHA Lab, University of Hawaii at Manoa$\dag$, University of South Florida$\ddagger$ \\
\{haopengz, haohany\}@hawaii.edu\\
\{yiliren, jingzhe, shen202\}@usf.edu\\}



  
%\author{
%  \textbf{First Author\textsuperscript{1}},
%  \textbf{Second Author\textsuperscript{1,2}},
%  \textbf{Third T. Author\textsuperscript{1}},
%  \textbf{Fourth Author\textsuperscript{1}},
%\\
%  \textbf{Fifth Author\textsuperscript{1,2}},
%  \textbf{Sixth Author\textsuperscript{1}},
%  \textbf{Seventh Author\textsuperscript{1}},
%  \textbf{Eighth Author \textsuperscript{1,2,3,4}},
%\\
%  \textbf{Ninth Author\textsuperscript{1}},
%  \textbf{Tenth Author\textsuperscript{1}},
%  \textbf{Eleventh E. Author\textsuperscript{1,2,3,4,5}},
%  \textbf{Twelfth Author\textsuperscript{1}},
%\\
%  \textbf{Thirteenth Author\textsuperscript{3}},
%  \textbf{Fourteenth F. Author\textsuperscript{2,4}},
%  \textbf{Fifteenth Author\textsuperscript{1}},
%  \textbf{Sixteenth Author\textsuperscript{1}},
%\\
%  \textbf{Seventeenth S. Author\textsuperscript{4,5}},
%  \textbf{Eighteenth Author\textsuperscript{3,4}},
%  \textbf{Nineteenth N. Author\textsuperscript{2,5}},
%  \textbf{Twentieth Author\textsuperscript{1}}
%\\
%\\
%  \textsuperscript{1}Affiliation 1,
%  \textsuperscript{2}Affiliation 2,
%  \textsuperscript{3}Affiliation 3,
%  \textsuperscript{4}Affiliation 4,
%  \textsuperscript{5}Affiliation 5
%\\
%  \small{
%    \textbf{Correspondence:} \href{mailto:email@domain}{email@domain}
%  }
%}

\begin{document}
\maketitle
\begin{abstract}
Recent advancements in Large Language Models (LLMs) have demonstrated remarkable capabilities across diverse tasks. However, their potential to integrate physical model knowledge for real-world signal interpretation remains largely unexplored. In this work, we introduce Wi-Chat, the first LLM-powered Wi-Fi-based human activity recognition system. We demonstrate that LLMs can process raw Wi-Fi signals and infer human activities by incorporating Wi-Fi sensing principles into prompts. Our approach leverages physical model insights to guide LLMs in interpreting Channel State Information (CSI) data without traditional signal processing techniques. Through experiments on real-world Wi-Fi datasets, we show that LLMs exhibit strong reasoning capabilities, achieving zero-shot activity recognition. These findings highlight a new paradigm for Wi-Fi sensing, expanding LLM applications beyond conventional language tasks and enhancing the accessibility of wireless sensing for real-world deployments.
\end{abstract}

\section{Introduction}

In today’s rapidly evolving digital landscape, the transformative power of web technologies has redefined not only how services are delivered but also how complex tasks are approached. Web-based systems have become increasingly prevalent in risk control across various domains. This widespread adoption is due their accessibility, scalability, and ability to remotely connect various types of users. For example, these systems are used for process safety management in industry~\cite{kannan2016web}, safety risk early warning in urban construction~\cite{ding2013development}, and safe monitoring of infrastructural systems~\cite{repetto2018web}. Within these web-based risk management systems, the source search problem presents a huge challenge. Source search refers to the task of identifying the origin of a risky event, such as a gas leak and the emission point of toxic substances. This source search capability is crucial for effective risk management and decision-making.

Traditional approaches to implementing source search capabilities into the web systems often rely on solely algorithmic solutions~\cite{ristic2016study}. These methods, while relatively straightforward to implement, often struggle to achieve acceptable performances due to algorithmic local optima and complex unknown environments~\cite{zhao2020searching}. More recently, web crowdsourcing has emerged as a promising alternative for tackling the source search problem by incorporating human efforts in these web systems on-the-fly~\cite{zhao2024user}. This approach outsources the task of addressing issues encountered during the source search process to human workers, leveraging their capabilities to enhance system performance.

These solutions often employ a human-AI collaborative way~\cite{zhao2023leveraging} where algorithms handle exploration-exploitation and report the encountered problems while human workers resolve complex decision-making bottlenecks to help the algorithms getting rid of local deadlocks~\cite{zhao2022crowd}. Although effective, this paradigm suffers from two inherent limitations: increased operational costs from continuous human intervention, and slow response times of human workers due to sequential decision-making. These challenges motivate our investigation into developing autonomous systems that preserve human-like reasoning capabilities while reducing dependency on massive crowdsourced labor.

Furthermore, recent advancements in large language models (LLMs)~\cite{chang2024survey} and multi-modal LLMs (MLLMs)~\cite{huang2023chatgpt} have unveiled promising avenues for addressing these challenges. One clear opportunity involves the seamless integration of visual understanding and linguistic reasoning for robust decision-making in search tasks. However, whether large models-assisted source search is really effective and efficient for improving the current source search algorithms~\cite{ji2022source} remains unknown. \textit{To address the research gap, we are particularly interested in answering the following two research questions in this work:}

\textbf{\textit{RQ1: }}How can source search capabilities be integrated into web-based systems to support decision-making in time-sensitive risk management scenarios? 
% \sq{I mention ``time-sensitive'' here because I feel like we shall say something about the response time -- LLM has to be faster than humans}

\textbf{\textit{RQ2: }}How can MLLMs and LLMs enhance the effectiveness and efficiency of existing source search algorithms? 

% \textit{\textbf{RQ2:}} To what extent does the performance of large models-assisted search align with or approach the effectiveness of human-AI collaborative search? 

To answer the research questions, we propose a novel framework called Auto-\
S$^2$earch (\textbf{Auto}nomous \textbf{S}ource \textbf{Search}) and implement a prototype system that leverages advanced web technologies to simulate real-world conditions for zero-shot source search. Unlike traditional methods that rely on pre-defined heuristics or extensive human intervention, AutoS$^2$earch employs a carefully designed prompt that encapsulates human rationales, thereby guiding the MLLM to generate coherent and accurate scene descriptions from visual inputs about four directional choices. Based on these language-based descriptions, the LLM is enabled to determine the optimal directional choice through chain-of-thought (CoT) reasoning. Comprehensive empirical validation demonstrates that AutoS$^2$-\ 
earch achieves a success rate of 95–98\%, closely approaching the performance of human-AI collaborative search across 20 benchmark scenarios~\cite{zhao2023leveraging}. 

Our work indicates that the role of humans in future web crowdsourcing tasks may evolve from executors to validators or supervisors. Furthermore, incorporating explanations of LLM decisions into web-based system interfaces has the potential to help humans enhance task performance in risk control.






\section{Related Work}
\label{sec:relatedworks}

% \begin{table*}[t]
% \centering 
% \renewcommand\arraystretch{0.98}
% \fontsize{8}{10}\selectfont \setlength{\tabcolsep}{0.4em}
% \begin{tabular}{@{}lc|cc|cc|cc@{}}
% \toprule
% \textbf{Methods}           & \begin{tabular}[c]{@{}c@{}}\textbf{Training}\\ \textbf{Paradigm}\end{tabular} & \begin{tabular}[c]{@{}c@{}}\textbf{$\#$ PT Data}\\ \textbf{(Tokens)}\end{tabular} & \begin{tabular}[c]{@{}c@{}}\textbf{$\#$ IFT Data}\\ \textbf{(Samples)}\end{tabular} & \textbf{Code}  & \begin{tabular}[c]{@{}c@{}}\textbf{Natural}\\ \textbf{Language}\end{tabular} & \begin{tabular}[c]{@{}c@{}}\textbf{Action}\\ \textbf{Trajectories}\end{tabular} & \begin{tabular}[c]{@{}c@{}}\textbf{API}\\ \textbf{Documentation}\end{tabular}\\ \midrule 
% NexusRaven~\citep{srinivasan2023nexusraven} & IFT & - & - & \textcolor{green}{\CheckmarkBold} & \textcolor{green}{\CheckmarkBold} &\textcolor{red}{\XSolidBrush}&\textcolor{red}{\XSolidBrush}\\
% AgentInstruct~\citep{zeng2023agenttuning} & IFT & - & 2k & \textcolor{green}{\CheckmarkBold} & \textcolor{green}{\CheckmarkBold} &\textcolor{red}{\XSolidBrush}&\textcolor{red}{\XSolidBrush} \\
% AgentEvol~\citep{xi2024agentgym} & IFT & - & 14.5k & \textcolor{green}{\CheckmarkBold} & \textcolor{green}{\CheckmarkBold} &\textcolor{green}{\CheckmarkBold}&\textcolor{red}{\XSolidBrush} \\
% Gorilla~\citep{patil2023gorilla}& IFT & - & 16k & \textcolor{green}{\CheckmarkBold} & \textcolor{green}{\CheckmarkBold} &\textcolor{red}{\XSolidBrush}&\textcolor{green}{\CheckmarkBold}\\
% OpenFunctions-v2~\citep{patil2023gorilla} & IFT & - & 65k & \textcolor{green}{\CheckmarkBold} & \textcolor{green}{\CheckmarkBold} &\textcolor{red}{\XSolidBrush}&\textcolor{green}{\CheckmarkBold}\\
% LAM~\citep{zhang2024agentohana} & IFT & - & 42.6k & \textcolor{green}{\CheckmarkBold} & \textcolor{green}{\CheckmarkBold} &\textcolor{green}{\CheckmarkBold}&\textcolor{red}{\XSolidBrush} \\
% xLAM~\citep{liu2024apigen} & IFT & - & 60k & \textcolor{green}{\CheckmarkBold} & \textcolor{green}{\CheckmarkBold} &\textcolor{green}{\CheckmarkBold}&\textcolor{red}{\XSolidBrush} \\\midrule
% LEMUR~\citep{xu2024lemur} & PT & 90B & 300k & \textcolor{green}{\CheckmarkBold} & \textcolor{green}{\CheckmarkBold} &\textcolor{green}{\CheckmarkBold}&\textcolor{red}{\XSolidBrush}\\
% \rowcolor{teal!12} \method & PT & 103B & 95k & \textcolor{green}{\CheckmarkBold} & \textcolor{green}{\CheckmarkBold} & \textcolor{green}{\CheckmarkBold} & \textcolor{green}{\CheckmarkBold} \\
% \bottomrule
% \end{tabular}
% \caption{Summary of existing tuning- and pretraining-based LLM agents with their training sample sizes. "PT" and "IFT" denote "Pre-Training" and "Instruction Fine-Tuning", respectively. }
% \label{tab:related}
% \end{table*}

\begin{table*}[ht]
\begin{threeparttable}
\centering 
\renewcommand\arraystretch{0.98}
\fontsize{7}{9}\selectfont \setlength{\tabcolsep}{0.2em}
\begin{tabular}{@{}l|c|c|ccc|cc|cc|cccc@{}}
\toprule
\textbf{Methods} & \textbf{Datasets}           & \begin{tabular}[c]{@{}c@{}}\textbf{Training}\\ \textbf{Paradigm}\end{tabular} & \begin{tabular}[c]{@{}c@{}}\textbf{\# PT Data}\\ \textbf{(Tokens)}\end{tabular} & \begin{tabular}[c]{@{}c@{}}\textbf{\# IFT Data}\\ \textbf{(Samples)}\end{tabular} & \textbf{\# APIs} & \textbf{Code}  & \begin{tabular}[c]{@{}c@{}}\textbf{Nat.}\\ \textbf{Lang.}\end{tabular} & \begin{tabular}[c]{@{}c@{}}\textbf{Action}\\ \textbf{Traj.}\end{tabular} & \begin{tabular}[c]{@{}c@{}}\textbf{API}\\ \textbf{Doc.}\end{tabular} & \begin{tabular}[c]{@{}c@{}}\textbf{Func.}\\ \textbf{Call}\end{tabular} & \begin{tabular}[c]{@{}c@{}}\textbf{Multi.}\\ \textbf{Step}\end{tabular}  & \begin{tabular}[c]{@{}c@{}}\textbf{Plan}\\ \textbf{Refine}\end{tabular}  & \begin{tabular}[c]{@{}c@{}}\textbf{Multi.}\\ \textbf{Turn}\end{tabular}\\ \midrule 
\multicolumn{13}{l}{\emph{Instruction Finetuning-based LLM Agents for Intrinsic Reasoning}}  \\ \midrule
FireAct~\cite{chen2023fireact} & FireAct & IFT & - & 2.1K & 10 & \textcolor{red}{\XSolidBrush} &\textcolor{green}{\CheckmarkBold} &\textcolor{green}{\CheckmarkBold}  & \textcolor{red}{\XSolidBrush} &\textcolor{green}{\CheckmarkBold} & \textcolor{red}{\XSolidBrush} &\textcolor{green}{\CheckmarkBold} & \textcolor{red}{\XSolidBrush} \\
ToolAlpaca~\cite{tang2023toolalpaca} & ToolAlpaca & IFT & - & 4.0K & 400 & \textcolor{red}{\XSolidBrush} &\textcolor{green}{\CheckmarkBold} &\textcolor{green}{\CheckmarkBold} & \textcolor{red}{\XSolidBrush} &\textcolor{green}{\CheckmarkBold} & \textcolor{red}{\XSolidBrush}  &\textcolor{green}{\CheckmarkBold} & \textcolor{red}{\XSolidBrush}  \\
ToolLLaMA~\cite{qin2023toolllm} & ToolBench & IFT & - & 12.7K & 16,464 & \textcolor{red}{\XSolidBrush} &\textcolor{green}{\CheckmarkBold} &\textcolor{green}{\CheckmarkBold} &\textcolor{red}{\XSolidBrush} &\textcolor{green}{\CheckmarkBold}&\textcolor{green}{\CheckmarkBold}&\textcolor{green}{\CheckmarkBold} &\textcolor{green}{\CheckmarkBold}\\
AgentEvol~\citep{xi2024agentgym} & AgentTraj-L & IFT & - & 14.5K & 24 &\textcolor{red}{\XSolidBrush} & \textcolor{green}{\CheckmarkBold} &\textcolor{green}{\CheckmarkBold}&\textcolor{red}{\XSolidBrush} &\textcolor{green}{\CheckmarkBold}&\textcolor{red}{\XSolidBrush} &\textcolor{red}{\XSolidBrush} &\textcolor{green}{\CheckmarkBold}\\
Lumos~\cite{yin2024agent} & Lumos & IFT  & - & 20.0K & 16 &\textcolor{red}{\XSolidBrush} & \textcolor{green}{\CheckmarkBold} & \textcolor{green}{\CheckmarkBold} &\textcolor{red}{\XSolidBrush} & \textcolor{green}{\CheckmarkBold} & \textcolor{green}{\CheckmarkBold} &\textcolor{red}{\XSolidBrush} & \textcolor{green}{\CheckmarkBold}\\
Agent-FLAN~\cite{chen2024agent} & Agent-FLAN & IFT & - & 24.7K & 20 &\textcolor{red}{\XSolidBrush} & \textcolor{green}{\CheckmarkBold} & \textcolor{green}{\CheckmarkBold} &\textcolor{red}{\XSolidBrush} & \textcolor{green}{\CheckmarkBold}& \textcolor{green}{\CheckmarkBold}&\textcolor{red}{\XSolidBrush} & \textcolor{green}{\CheckmarkBold}\\
AgentTuning~\citep{zeng2023agenttuning} & AgentInstruct & IFT & - & 35.0K & - &\textcolor{red}{\XSolidBrush} & \textcolor{green}{\CheckmarkBold} & \textcolor{green}{\CheckmarkBold} &\textcolor{red}{\XSolidBrush} & \textcolor{green}{\CheckmarkBold} &\textcolor{red}{\XSolidBrush} &\textcolor{red}{\XSolidBrush} & \textcolor{green}{\CheckmarkBold}\\\midrule
\multicolumn{13}{l}{\emph{Instruction Finetuning-based LLM Agents for Function Calling}} \\\midrule
NexusRaven~\citep{srinivasan2023nexusraven} & NexusRaven & IFT & - & - & 116 & \textcolor{green}{\CheckmarkBold} & \textcolor{green}{\CheckmarkBold}  & \textcolor{green}{\CheckmarkBold} &\textcolor{red}{\XSolidBrush} & \textcolor{green}{\CheckmarkBold} &\textcolor{red}{\XSolidBrush} &\textcolor{red}{\XSolidBrush}&\textcolor{red}{\XSolidBrush}\\
Gorilla~\citep{patil2023gorilla} & Gorilla & IFT & - & 16.0K & 1,645 & \textcolor{green}{\CheckmarkBold} &\textcolor{red}{\XSolidBrush} &\textcolor{red}{\XSolidBrush}&\textcolor{green}{\CheckmarkBold} &\textcolor{green}{\CheckmarkBold} &\textcolor{red}{\XSolidBrush} &\textcolor{red}{\XSolidBrush} &\textcolor{red}{\XSolidBrush}\\
OpenFunctions-v2~\citep{patil2023gorilla} & OpenFunctions-v2 & IFT & - & 65.0K & - & \textcolor{green}{\CheckmarkBold} & \textcolor{green}{\CheckmarkBold} &\textcolor{red}{\XSolidBrush} &\textcolor{green}{\CheckmarkBold} &\textcolor{green}{\CheckmarkBold} &\textcolor{red}{\XSolidBrush} &\textcolor{red}{\XSolidBrush} &\textcolor{red}{\XSolidBrush}\\
API Pack~\cite{guo2024api} & API Pack & IFT & - & 1.1M & 11,213 &\textcolor{green}{\CheckmarkBold} &\textcolor{red}{\XSolidBrush} &\textcolor{green}{\CheckmarkBold} &\textcolor{red}{\XSolidBrush} &\textcolor{green}{\CheckmarkBold} &\textcolor{red}{\XSolidBrush}&\textcolor{red}{\XSolidBrush}&\textcolor{red}{\XSolidBrush}\\ 
LAM~\citep{zhang2024agentohana} & AgentOhana & IFT & - & 42.6K & - & \textcolor{green}{\CheckmarkBold} & \textcolor{green}{\CheckmarkBold} &\textcolor{green}{\CheckmarkBold}&\textcolor{red}{\XSolidBrush} &\textcolor{green}{\CheckmarkBold}&\textcolor{red}{\XSolidBrush}&\textcolor{green}{\CheckmarkBold}&\textcolor{green}{\CheckmarkBold}\\
xLAM~\citep{liu2024apigen} & APIGen & IFT & - & 60.0K & 3,673 & \textcolor{green}{\CheckmarkBold} & \textcolor{green}{\CheckmarkBold} &\textcolor{green}{\CheckmarkBold}&\textcolor{red}{\XSolidBrush} &\textcolor{green}{\CheckmarkBold}&\textcolor{red}{\XSolidBrush}&\textcolor{green}{\CheckmarkBold}&\textcolor{green}{\CheckmarkBold}\\\midrule
\multicolumn{13}{l}{\emph{Pretraining-based LLM Agents}}  \\\midrule
% LEMUR~\citep{xu2024lemur} & PT & 90B & 300.0K & - & \textcolor{green}{\CheckmarkBold} & \textcolor{green}{\CheckmarkBold} &\textcolor{green}{\CheckmarkBold}&\textcolor{red}{\XSolidBrush} & \textcolor{red}{\XSolidBrush} &\textcolor{green}{\CheckmarkBold} &\textcolor{red}{\XSolidBrush}&\textcolor{red}{\XSolidBrush}\\
\rowcolor{teal!12} \method & \dataset & PT & 103B & 95.0K  & 76,537  & \textcolor{green}{\CheckmarkBold} & \textcolor{green}{\CheckmarkBold} & \textcolor{green}{\CheckmarkBold} & \textcolor{green}{\CheckmarkBold} & \textcolor{green}{\CheckmarkBold} & \textcolor{green}{\CheckmarkBold} & \textcolor{green}{\CheckmarkBold} & \textcolor{green}{\CheckmarkBold}\\
\bottomrule
\end{tabular}
% \begin{tablenotes}
%     \item $^*$ In addition, the StarCoder-API can offer 4.77M more APIs.
% \end{tablenotes}
\caption{Summary of existing instruction finetuning-based LLM agents for intrinsic reasoning and function calling, along with their training resources and sample sizes. "PT" and "IFT" denote "Pre-Training" and "Instruction Fine-Tuning", respectively.}
\vspace{-2ex}
\label{tab:related}
\end{threeparttable}
\end{table*}

\noindent \textbf{Prompting-based LLM Agents.} Due to the lack of agent-specific pre-training corpus, existing LLM agents rely on either prompt engineering~\cite{hsieh2023tool,lu2024chameleon,yao2022react,wang2023voyager} or instruction fine-tuning~\cite{chen2023fireact,zeng2023agenttuning} to understand human instructions, decompose high-level tasks, generate grounded plans, and execute multi-step actions. 
However, prompting-based methods mainly depend on the capabilities of backbone LLMs (usually commercial LLMs), failing to introduce new knowledge and struggling to generalize to unseen tasks~\cite{sun2024adaplanner,zhuang2023toolchain}. 

\noindent \textbf{Instruction Finetuning-based LLM Agents.} Considering the extensive diversity of APIs and the complexity of multi-tool instructions, tool learning inherently presents greater challenges than natural language tasks, such as text generation~\cite{qin2023toolllm}.
Post-training techniques focus more on instruction following and aligning output with specific formats~\cite{patil2023gorilla,hao2024toolkengpt,qin2023toolllm,schick2024toolformer}, rather than fundamentally improving model knowledge or capabilities. 
Moreover, heavy fine-tuning can hinder generalization or even degrade performance in non-agent use cases, potentially suppressing the original base model capabilities~\cite{ghosh2024a}.

\noindent \textbf{Pretraining-based LLM Agents.} While pre-training serves as an essential alternative, prior works~\cite{nijkamp2023codegen,roziere2023code,xu2024lemur,patil2023gorilla} have primarily focused on improving task-specific capabilities (\eg, code generation) instead of general-domain LLM agents, due to single-source, uni-type, small-scale, and poor-quality pre-training data. 
Existing tool documentation data for agent training either lacks diverse real-world APIs~\cite{patil2023gorilla, tang2023toolalpaca} or is constrained to single-tool or single-round tool execution. 
Furthermore, trajectory data mostly imitate expert behavior or follow function-calling rules with inferior planning and reasoning, failing to fully elicit LLMs' capabilities and handle complex instructions~\cite{qin2023toolllm}. 
Given a wide range of candidate API functions, each comprising various function names and parameters available at every planning step, identifying globally optimal solutions and generalizing across tasks remains highly challenging.



\section{Preliminaries}
\label{Preliminaries}
\begin{figure*}[t]
    \centering
    \includegraphics[width=0.95\linewidth]{fig/HealthGPT_Framework.png}
    \caption{The \ourmethod{} architecture integrates hierarchical visual perception and H-LoRA, employing a task-specific hard router to select visual features and H-LoRA plugins, ultimately generating outputs with an autoregressive manner.}
    \label{fig:architecture}
\end{figure*}
\noindent\textbf{Large Vision-Language Models.} 
The input to a LVLM typically consists of an image $x^{\text{img}}$ and a discrete text sequence $x^{\text{txt}}$. The visual encoder $\mathcal{E}^{\text{img}}$ converts the input image $x^{\text{img}}$ into a sequence of visual tokens $\mathcal{V} = [v_i]_{i=1}^{N_v}$, while the text sequence $x^{\text{txt}}$ is mapped into a sequence of text tokens $\mathcal{T} = [t_i]_{i=1}^{N_t}$ using an embedding function $\mathcal{E}^{\text{txt}}$. The LLM $\mathcal{M_\text{LLM}}(\cdot|\theta)$ models the joint probability of the token sequence $\mathcal{U} = \{\mathcal{V},\mathcal{T}\}$, which is expressed as:
\begin{equation}
    P_\theta(R | \mathcal{U}) = \prod_{i=1}^{N_r} P_\theta(r_i | \{\mathcal{U}, r_{<i}\}),
\end{equation}
where $R = [r_i]_{i=1}^{N_r}$ is the text response sequence. The LVLM iteratively generates the next token $r_i$ based on $r_{<i}$. The optimization objective is to minimize the cross-entropy loss of the response $\mathcal{R}$.
% \begin{equation}
%     \mathcal{L}_{\text{VLM}} = \mathbb{E}_{R|\mathcal{U}}\left[-\log P_\theta(R | \mathcal{U})\right]
% \end{equation}
It is worth noting that most LVLMs adopt a design paradigm based on ViT, alignment adapters, and pre-trained LLMs\cite{liu2023llava,liu2024improved}, enabling quick adaptation to downstream tasks.


\noindent\textbf{VQGAN.}
VQGAN~\cite{esser2021taming} employs latent space compression and indexing mechanisms to effectively learn a complete discrete representation of images. VQGAN first maps the input image $x^{\text{img}}$ to a latent representation $z = \mathcal{E}(x)$ through a encoder $\mathcal{E}$. Then, the latent representation is quantized using a codebook $\mathcal{Z} = \{z_k\}_{k=1}^K$, generating a discrete index sequence $\mathcal{I} = [i_m]_{m=1}^N$, where $i_m \in \mathcal{Z}$ represents the quantized code index:
\begin{equation}
    \mathcal{I} = \text{Quantize}(z|\mathcal{Z}) = \arg\min_{z_k \in \mathcal{Z}} \| z - z_k \|_2.
\end{equation}
In our approach, the discrete index sequence $\mathcal{I}$ serves as a supervisory signal for the generation task, enabling the model to predict the index sequence $\hat{\mathcal{I}}$ from input conditions such as text or other modality signals.  
Finally, the predicted index sequence $\hat{\mathcal{I}}$ is upsampled by the VQGAN decoder $G$, generating the high-quality image $\hat{x}^\text{img} = G(\hat{\mathcal{I}})$.



\noindent\textbf{Low Rank Adaptation.} 
LoRA\cite{hu2021lora} effectively captures the characteristics of downstream tasks by introducing low-rank adapters. The core idea is to decompose the bypass weight matrix $\Delta W\in\mathbb{R}^{d^{\text{in}} \times d^{\text{out}}}$ into two low-rank matrices $ \{A \in \mathbb{R}^{d^{\text{in}} \times r}, B \in \mathbb{R}^{r \times d^{\text{out}}} \}$, where $ r \ll \min\{d^{\text{in}}, d^{\text{out}}\} $, significantly reducing learnable parameters. The output with the LoRA adapter for the input $x$ is then given by:
\begin{equation}
    h = x W_0 + \alpha x \Delta W/r = x W_0 + \alpha xAB/r,
\end{equation}
where matrix $ A $ is initialized with a Gaussian distribution, while the matrix $ B $ is initialized as a zero matrix. The scaling factor $ \alpha/r $ controls the impact of $ \Delta W $ on the model.

\section{HealthGPT}
\label{Method}


\subsection{Unified Autoregressive Generation.}  
% As shown in Figure~\ref{fig:architecture}, 
\ourmethod{} (Figure~\ref{fig:architecture}) utilizes a discrete token representation that covers both text and visual outputs, unifying visual comprehension and generation as an autoregressive task. 
For comprehension, $\mathcal{M}_\text{llm}$ receives the input joint sequence $\mathcal{U}$ and outputs a series of text token $\mathcal{R} = [r_1, r_2, \dots, r_{N_r}]$, where $r_i \in \mathcal{V}_{\text{txt}}$, and $\mathcal{V}_{\text{txt}}$ represents the LLM's vocabulary:
\begin{equation}
    P_\theta(\mathcal{R} \mid \mathcal{U}) = \prod_{i=1}^{N_r} P_\theta(r_i \mid \mathcal{U}, r_{<i}).
\end{equation}
For generation, $\mathcal{M}_\text{llm}$ first receives a special start token $\langle \text{START\_IMG} \rangle$, then generates a series of tokens corresponding to the VQGAN indices $\mathcal{I} = [i_1, i_2, \dots, i_{N_i}]$, where $i_j \in \mathcal{V}_{\text{vq}}$, and $\mathcal{V}_{\text{vq}}$ represents the index range of VQGAN. Upon completion of generation, the LLM outputs an end token $\langle \text{END\_IMG} \rangle$:
\begin{equation}
    P_\theta(\mathcal{I} \mid \mathcal{U}) = \prod_{j=1}^{N_i} P_\theta(i_j \mid \mathcal{U}, i_{<j}).
\end{equation}
Finally, the generated index sequence $\mathcal{I}$ is fed into the decoder $G$, which reconstructs the target image $\hat{x}^{\text{img}} = G(\mathcal{I})$.

\subsection{Hierarchical Visual Perception}  
Given the differences in visual perception between comprehension and generation tasks—where the former focuses on abstract semantics and the latter emphasizes complete semantics—we employ ViT to compress the image into discrete visual tokens at multiple hierarchical levels.
Specifically, the image is converted into a series of features $\{f_1, f_2, \dots, f_L\}$ as it passes through $L$ ViT blocks.

To address the needs of various tasks, the hidden states are divided into two types: (i) \textit{Concrete-grained features} $\mathcal{F}^{\text{Con}} = \{f_1, f_2, \dots, f_k\}, k < L$, derived from the shallower layers of ViT, containing sufficient global features, suitable for generation tasks; 
(ii) \textit{Abstract-grained features} $\mathcal{F}^{\text{Abs}} = \{f_{k+1}, f_{k+2}, \dots, f_L\}$, derived from the deeper layers of ViT, which contain abstract semantic information closer to the text space, suitable for comprehension tasks.

The task type $T$ (comprehension or generation) determines which set of features is selected as the input for the downstream large language model:
\begin{equation}
    \mathcal{F}^{\text{img}}_T =
    \begin{cases}
        \mathcal{F}^{\text{Con}}, & \text{if } T = \text{generation task} \\
        \mathcal{F}^{\text{Abs}}, & \text{if } T = \text{comprehension task}
    \end{cases}
\end{equation}
We integrate the image features $\mathcal{F}^{\text{img}}_T$ and text features $\mathcal{T}$ into a joint sequence through simple concatenation, which is then fed into the LLM $\mathcal{M}_{\text{llm}}$ for autoregressive generation.
% :
% \begin{equation}
%     \mathcal{R} = \mathcal{M}_{\text{llm}}(\mathcal{U}|\theta), \quad \mathcal{U} = [\mathcal{F}^{\text{img}}_T; \mathcal{T}]
% \end{equation}
\subsection{Heterogeneous Knowledge Adaptation}
We devise H-LoRA, which stores heterogeneous knowledge from comprehension and generation tasks in separate modules and dynamically routes to extract task-relevant knowledge from these modules. 
At the task level, for each task type $ T $, we dynamically assign a dedicated H-LoRA submodule $ \theta^T $, which is expressed as:
\begin{equation}
    \mathcal{R} = \mathcal{M}_\text{LLM}(\mathcal{U}|\theta, \theta^T), \quad \theta^T = \{A^T, B^T, \mathcal{R}^T_\text{outer}\}.
\end{equation}
At the feature level for a single task, H-LoRA integrates the idea of Mixture of Experts (MoE)~\cite{masoudnia2014mixture} and designs an efficient matrix merging and routing weight allocation mechanism, thus avoiding the significant computational delay introduced by matrix splitting in existing MoELoRA~\cite{luo2024moelora}. Specifically, we first merge the low-rank matrices (rank = r) of $ k $ LoRA experts into a unified matrix:
\begin{equation}
    \mathbf{A}^{\text{merged}}, \mathbf{B}^{\text{merged}} = \text{Concat}(\{A_i\}_1^k), \text{Concat}(\{B_i\}_1^k),
\end{equation}
where $ \mathbf{A}^{\text{merged}} \in \mathbb{R}^{d^\text{in} \times rk} $ and $ \mathbf{B}^{\text{merged}} \in \mathbb{R}^{rk \times d^\text{out}} $. The $k$-dimension routing layer generates expert weights $ \mathcal{W} \in \mathbb{R}^{\text{token\_num} \times k} $ based on the input hidden state $ x $, and these are expanded to $ \mathbb{R}^{\text{token\_num} \times rk} $ as follows:
\begin{equation}
    \mathcal{W}^\text{expanded} = \alpha k \mathcal{W} / r \otimes \mathbf{1}_r,
\end{equation}
where $ \otimes $ denotes the replication operation.
The overall output of H-LoRA is computed as:
\begin{equation}
    \mathcal{O}^\text{H-LoRA} = (x \mathbf{A}^{\text{merged}} \odot \mathcal{W}^\text{expanded}) \mathbf{B}^{\text{merged}},
\end{equation}
where $ \odot $ represents element-wise multiplication. Finally, the output of H-LoRA is added to the frozen pre-trained weights to produce the final output:
\begin{equation}
    \mathcal{O} = x W_0 + \mathcal{O}^\text{H-LoRA}.
\end{equation}
% In summary, H-LoRA is a task-based dynamic PEFT method that achieves high efficiency in single-task fine-tuning.

\subsection{Training Pipeline}

\begin{figure}[t]
    \centering
    \hspace{-4mm}
    \includegraphics[width=0.94\linewidth]{fig/data.pdf}
    \caption{Data statistics of \texttt{VL-Health}. }
    \label{fig:data}
\end{figure}
\noindent \textbf{1st Stage: Multi-modal Alignment.} 
In the first stage, we design separate visual adapters and H-LoRA submodules for medical unified tasks. For the medical comprehension task, we train abstract-grained visual adapters using high-quality image-text pairs to align visual embeddings with textual embeddings, thereby enabling the model to accurately describe medical visual content. During this process, the pre-trained LLM and its corresponding H-LoRA submodules remain frozen. In contrast, the medical generation task requires training concrete-grained adapters and H-LoRA submodules while keeping the LLM frozen. Meanwhile, we extend the textual vocabulary to include multimodal tokens, enabling the support of additional VQGAN vector quantization indices. The model trains on image-VQ pairs, endowing the pre-trained LLM with the capability for image reconstruction. This design ensures pixel-level consistency of pre- and post-LVLM. The processes establish the initial alignment between the LLM’s outputs and the visual inputs.

\noindent \textbf{2nd Stage: Heterogeneous H-LoRA Plugin Adaptation.}  
The submodules of H-LoRA share the word embedding layer and output head but may encounter issues such as bias and scale inconsistencies during training across different tasks. To ensure that the multiple H-LoRA plugins seamlessly interface with the LLMs and form a unified base, we fine-tune the word embedding layer and output head using a small amount of mixed data to maintain consistency in the model weights. Specifically, during this stage, all H-LoRA submodules for different tasks are kept frozen, with only the word embedding layer and output head being optimized. Through this stage, the model accumulates foundational knowledge for unified tasks by adapting H-LoRA plugins.

\begin{table*}[!t]
\centering
\caption{Comparison of \ourmethod{} with other LVLMs and unified multi-modal models on medical visual comprehension tasks. \textbf{Bold} and \underline{underlined} text indicates the best performance and second-best performance, respectively.}
\resizebox{\textwidth}{!}{
\begin{tabular}{c|lcc|cccccccc|c}
\toprule
\rowcolor[HTML]{E9F3FE} &  &  &  & \multicolumn{2}{c}{\textbf{VQA-RAD \textuparrow}} & \multicolumn{2}{c}{\textbf{SLAKE \textuparrow}} & \multicolumn{2}{c}{\textbf{PathVQA \textuparrow}} &  &  &  \\ 
\cline{5-10}
\rowcolor[HTML]{E9F3FE}\multirow{-2}{*}{\textbf{Type}} & \multirow{-2}{*}{\textbf{Model}} & \multirow{-2}{*}{\textbf{\# Params}} & \multirow{-2}{*}{\makecell{\textbf{Medical} \\ \textbf{LVLM}}} & \textbf{close} & \textbf{all} & \textbf{close} & \textbf{all} & \textbf{close} & \textbf{all} & \multirow{-2}{*}{\makecell{\textbf{MMMU} \\ \textbf{-Med}}\textuparrow} & \multirow{-2}{*}{\textbf{OMVQA}\textuparrow} & \multirow{-2}{*}{\textbf{Avg. \textuparrow}} \\ 
\midrule \midrule
\multirow{9}{*}{\textbf{Comp. Only}} 
& Med-Flamingo & 8.3B & \Large \ding{51} & 58.6 & 43.0 & 47.0 & 25.5 & 61.9 & 31.3 & 28.7 & 34.9 & 41.4 \\
& LLaVA-Med & 7B & \Large \ding{51} & 60.2 & 48.1 & 58.4 & 44.8 & 62.3 & 35.7 & 30.0 & 41.3 & 47.6 \\
& HuatuoGPT-Vision & 7B & \Large \ding{51} & 66.9 & 53.0 & 59.8 & 49.1 & 52.9 & 32.0 & 42.0 & 50.0 & 50.7 \\
& BLIP-2 & 6.7B & \Large \ding{55} & 43.4 & 36.8 & 41.6 & 35.3 & 48.5 & 28.8 & 27.3 & 26.9 & 36.1 \\
& LLaVA-v1.5 & 7B & \Large \ding{55} & 51.8 & 42.8 & 37.1 & 37.7 & 53.5 & 31.4 & 32.7 & 44.7 & 41.5 \\
& InstructBLIP & 7B & \Large \ding{55} & 61.0 & 44.8 & 66.8 & 43.3 & 56.0 & 32.3 & 25.3 & 29.0 & 44.8 \\
& Yi-VL & 6B & \Large \ding{55} & 52.6 & 42.1 & 52.4 & 38.4 & 54.9 & 30.9 & 38.0 & 50.2 & 44.9 \\
& InternVL2 & 8B & \Large \ding{55} & 64.9 & 49.0 & 66.6 & 50.1 & 60.0 & 31.9 & \underline{43.3} & 54.5 & 52.5\\
& Llama-3.2 & 11B & \Large \ding{55} & 68.9 & 45.5 & 72.4 & 52.1 & 62.8 & 33.6 & 39.3 & 63.2 & 54.7 \\
\midrule
\multirow{5}{*}{\textbf{Comp. \& Gen.}} 
& Show-o & 1.3B & \Large \ding{55} & 50.6 & 33.9 & 31.5 & 17.9 & 52.9 & 28.2 & 22.7 & 45.7 & 42.6 \\
& Unified-IO 2 & 7B & \Large \ding{55} & 46.2 & 32.6 & 35.9 & 21.9 & 52.5 & 27.0 & 25.3 & 33.0 & 33.8 \\
& Janus & 1.3B & \Large \ding{55} & 70.9 & 52.8 & 34.7 & 26.9 & 51.9 & 27.9 & 30.0 & 26.8 & 33.5 \\
& \cellcolor[HTML]{DAE0FB}HealthGPT-M3 & \cellcolor[HTML]{DAE0FB}3.8B & \cellcolor[HTML]{DAE0FB}\Large \ding{51} & \cellcolor[HTML]{DAE0FB}\underline{73.7} & \cellcolor[HTML]{DAE0FB}\underline{55.9} & \cellcolor[HTML]{DAE0FB}\underline{74.6} & \cellcolor[HTML]{DAE0FB}\underline{56.4} & \cellcolor[HTML]{DAE0FB}\underline{78.7} & \cellcolor[HTML]{DAE0FB}\underline{39.7} & \cellcolor[HTML]{DAE0FB}\underline{43.3} & \cellcolor[HTML]{DAE0FB}\underline{68.5} & \cellcolor[HTML]{DAE0FB}\underline{61.3} \\
& \cellcolor[HTML]{DAE0FB}HealthGPT-L14 & \cellcolor[HTML]{DAE0FB}14B & \cellcolor[HTML]{DAE0FB}\Large \ding{51} & \cellcolor[HTML]{DAE0FB}\textbf{77.7} & \cellcolor[HTML]{DAE0FB}\textbf{58.3} & \cellcolor[HTML]{DAE0FB}\textbf{76.4} & \cellcolor[HTML]{DAE0FB}\textbf{64.5} & \cellcolor[HTML]{DAE0FB}\textbf{85.9} & \cellcolor[HTML]{DAE0FB}\textbf{44.4} & \cellcolor[HTML]{DAE0FB}\textbf{49.2} & \cellcolor[HTML]{DAE0FB}\textbf{74.4} & \cellcolor[HTML]{DAE0FB}\textbf{66.4} \\
\bottomrule
\end{tabular}
}
\label{tab:results}
\end{table*}
\begin{table*}[ht]
    \centering
    \caption{The experimental results for the four modality conversion tasks.}
    \resizebox{\textwidth}{!}{
    \begin{tabular}{l|ccc|ccc|ccc|ccc}
        \toprule
        \rowcolor[HTML]{E9F3FE} & \multicolumn{3}{c}{\textbf{CT to MRI (Brain)}} & \multicolumn{3}{c}{\textbf{CT to MRI (Pelvis)}} & \multicolumn{3}{c}{\textbf{MRI to CT (Brain)}} & \multicolumn{3}{c}{\textbf{MRI to CT (Pelvis)}} \\
        \cline{2-13}
        \rowcolor[HTML]{E9F3FE}\multirow{-2}{*}{\textbf{Model}}& \textbf{SSIM $\uparrow$} & \textbf{PSNR $\uparrow$} & \textbf{MSE $\downarrow$} & \textbf{SSIM $\uparrow$} & \textbf{PSNR $\uparrow$} & \textbf{MSE $\downarrow$} & \textbf{SSIM $\uparrow$} & \textbf{PSNR $\uparrow$} & \textbf{MSE $\downarrow$} & \textbf{SSIM $\uparrow$} & \textbf{PSNR $\uparrow$} & \textbf{MSE $\downarrow$} \\
        \midrule \midrule
        pix2pix & 71.09 & 32.65 & 36.85 & 59.17 & 31.02 & 51.91 & 78.79 & 33.85 & 28.33 & 72.31 & 32.98 & 36.19 \\
        CycleGAN & 54.76 & 32.23 & 40.56 & 54.54 & 30.77 & 55.00 & 63.75 & 31.02 & 52.78 & 50.54 & 29.89 & 67.78 \\
        BBDM & {71.69} & {32.91} & {34.44} & 57.37 & 31.37 & 48.06 & \textbf{86.40} & 34.12 & 26.61 & {79.26} & 33.15 & 33.60 \\
        Vmanba & 69.54 & 32.67 & 36.42 & {63.01} & {31.47} & {46.99} & 79.63 & 34.12 & 26.49 & 77.45 & 33.53 & 31.85 \\
        DiffMa & 71.47 & 32.74 & 35.77 & 62.56 & 31.43 & 47.38 & 79.00 & {34.13} & {26.45} & 78.53 & {33.68} & {30.51} \\
        \rowcolor[HTML]{DAE0FB}HealthGPT-M3 & \underline{79.38} & \underline{33.03} & \underline{33.48} & \underline{71.81} & \underline{31.83} & \underline{43.45} & {85.06} & \textbf{34.40} & \textbf{25.49} & \underline{84.23} & \textbf{34.29} & \textbf{27.99} \\
        \rowcolor[HTML]{DAE0FB}HealthGPT-L14 & \textbf{79.73} & \textbf{33.10} & \textbf{32.96} & \textbf{71.92} & \textbf{31.87} & \textbf{43.09} & \underline{85.31} & \underline{34.29} & \underline{26.20} & \textbf{84.96} & \underline{34.14} & \underline{28.13} \\
        \bottomrule
    \end{tabular}
    }
    \label{tab:conversion}
\end{table*}

\noindent \textbf{3rd Stage: Visual Instruction Fine-Tuning.}  
In the third stage, we introduce additional task-specific data to further optimize the model and enhance its adaptability to downstream tasks such as medical visual comprehension (e.g., medical QA, medical dialogues, and report generation) or generation tasks (e.g., super-resolution, denoising, and modality conversion). Notably, by this stage, the word embedding layer and output head have been fine-tuned, only the H-LoRA modules and adapter modules need to be trained. This strategy significantly improves the model's adaptability and flexibility across different tasks.


\section{Experiment}
\label{s:experiment}

\subsection{Data Description}
We evaluate our method on FI~\cite{you2016building}, Twitter\_LDL~\cite{yang2017learning} and Artphoto~\cite{machajdik2010affective}.
FI is a public dataset built from Flickr and Instagram, with 23,308 images and eight emotion categories, namely \textit{amusement}, \textit{anger}, \textit{awe},  \textit{contentment}, \textit{disgust}, \textit{excitement},  \textit{fear}, and \textit{sadness}. 
% Since images in FI are all copyrighted by law, some images are corrupted now, so we remove these samples and retain 21,828 images.
% T4SA contains images from Twitter, which are classified into three categories: \textit{positive}, \textit{neutral}, and \textit{negative}. In this paper, we adopt the base version of B-T4SA, which contains 470,586 images and provides text descriptions of the corresponding tweets.
Twitter\_LDL contains 10,045 images from Twitter, with the same eight categories as the FI dataset.
% 。
For these two datasets, they are randomly split into 80\%
training and 20\% testing set.
Artphoto contains 806 artistic photos from the DeviantArt website, which we use to further evaluate the zero-shot capability of our model.
% on the small-scale dataset.
% We construct and publicly release the first image sentiment analysis dataset containing metadata.
% 。

% Based on these datasets, we are the first to construct and publicly release metadata-enhanced image sentiment analysis datasets. These datasets include scenes, tags, descriptions, and corresponding confidence scores, and are available at this link for future research purposes.


% 
\begin{table}[t]
\centering
% \begin{center}
\caption{Overall performance of different models on FI and Twitter\_LDL datasets.}
\label{tab:cap1}
% \resizebox{\linewidth}{!}
{
\begin{tabular}{l|c|c|c|c}
\hline
\multirow{2}{*}{\textbf{Model}} & \multicolumn{2}{c|}{\textbf{FI}}  & \multicolumn{2}{c}{\textbf{Twitter\_LDL}} \\ \cline{2-5} 
  & \textbf{Accuracy} & \textbf{F1} & \textbf{Accuracy} & \textbf{F1}  \\ \hline
% (\rownumber)~AlexNet~\cite{krizhevsky2017imagenet}  & 58.13\% & 56.35\%  & 56.24\%& 55.02\%  \\ 
% (\rownumber)~VGG16~\cite{simonyan2014very}  & 63.75\%& 63.08\%  & 59.34\%& 59.02\%  \\ 
(\rownumber)~ResNet101~\cite{he2016deep} & 66.16\%& 65.56\%  & 62.02\% & 61.34\%  \\ 
(\rownumber)~CDA~\cite{han2023boosting} & 66.71\%& 65.37\%  & 64.14\% & 62.85\%  \\ 
(\rownumber)~CECCN~\cite{ruan2024color} & 67.96\%& 66.74\%  & 64.59\%& 64.72\% \\ 
(\rownumber)~EmoVIT~\cite{xie2024emovit} & 68.09\%& 67.45\%  & 63.12\% & 61.97\%  \\ 
(\rownumber)~ComLDL~\cite{zhang2022compound} & 68.83\%& 67.28\%  & 65.29\% & 63.12\%  \\ 
(\rownumber)~WSDEN~\cite{li2023weakly} & 69.78\%& 69.61\%  & 67.04\% & 65.49\% \\ 
(\rownumber)~ECWA~\cite{deng2021emotion} & 70.87\%& 69.08\%  & 67.81\% & 66.87\%  \\ 
(\rownumber)~EECon~\cite{yang2023exploiting} & 71.13\%& 68.34\%  & 64.27\%& 63.16\%  \\ 
(\rownumber)~MAM~\cite{zhang2024affective} & 71.44\%  & 70.83\% & 67.18\%  & 65.01\%\\ 
(\rownumber)~TGCA-PVT~\cite{chen2024tgca}   & 73.05\%  & 71.46\% & 69.87\%  & 68.32\% \\ 
(\rownumber)~OEAN~\cite{zhang2024object}   & 73.40\%  & 72.63\% & 70.52\%  & 69.47\% \\ \hline
(\rownumber)~\shortname  & \textbf{79.48\%} & \textbf{79.22\%} & \textbf{74.12\%} & \textbf{73.09\%} \\ \hline
\end{tabular}
}
\vspace{-6mm}
% \end{center}
\end{table}
% 

\subsection{Experiment Setting}
% \subsubsection{Model Setting.}
% 
\textbf{Model Setting:}
For feature representation, we set $k=10$ to select object tags, and adopt clip-vit-base-patch32 as the pre-trained model for unified feature representation.
Moreover, we empirically set $(d_e, d_h, d_k, d_s) = (512, 128, 16, 64)$, and set the classification class $L$ to 8.

% 

\textbf{Training Setting:}
To initialize the model, we set all weights such as $\boldsymbol{W}$ following the truncated normal distribution, and use AdamW optimizer with the learning rate of $1 \times 10^{-4}$.
% warmup scheduler of cosine, warmup steps of 2000.
Furthermore, we set the batch size to 32 and the epoch of the training process to 200.
During the implementation, we utilize \textit{PyTorch} to build our entire model.
% , and our project codes are publicly available at https://github.com/zzmyrep/MESN.
% Our project codes as well as data are all publicly available on GitHub\footnote{https://github.com/zzmyrep/KBCEN}.
% Code is available at \href{https://github.com/zzmyrep/KBCEN}{https://github.com/zzmyrep/KBCEN}.

\textbf{Evaluation Metrics:}
Following~\cite{zhang2024affective, chen2024tgca, zhang2024object}, we adopt \textit{accuracy} and \textit{F1} as our evaluation metrics to measure the performance of different methods for image sentiment analysis. 



\subsection{Experiment Result}
% We compare our model against the following baselines: AlexNet~\cite{krizhevsky2017imagenet}, VGG16~\cite{simonyan2014very}, ResNet101~\cite{he2016deep}, CECCN~\cite{ruan2024color}, EmoVIT~\cite{xie2024emovit}, WSCNet~\cite{yang2018weakly}, ECWA~\cite{deng2021emotion}, EECon~\cite{yang2023exploiting}, MAM~\cite{zhang2024affective} and TGCA-PVT~\cite{chen2024tgca}, and the overall results are summarized in Table~\ref{tab:cap1}.
We compare our model against several baselines, and the overall results are summarized in Table~\ref{tab:cap1}.
We observe that our model achieves the best performance in both accuracy and F1 metrics, significantly outperforming the previous models. 
This superior performance is mainly attributed to our effective utilization of metadata to enhance image sentiment analysis, as well as the exceptional capability of the unified sentiment transformer framework we developed. These results strongly demonstrate that our proposed method can bring encouraging performance for image sentiment analysis.

\setcounter{magicrownumbers}{0} 
\begin{table}[t]
\begin{center}
\caption{Ablation study of~\shortname~on FI dataset.} 
% \vspace{1mm}
\label{tab:cap2}
\resizebox{.9\linewidth}{!}
{
\begin{tabular}{lcc}
  \hline
  \textbf{Model} & \textbf{Accuracy} & \textbf{F1} \\
  \hline
  (\rownumber)~Ours (w/o vision) & 65.72\% & 64.54\% \\
  (\rownumber)~Ours (w/o text description) & 74.05\% & 72.58\% \\
  (\rownumber)~Ours (w/o object tag) & 77.45\% & 76.84\% \\
  (\rownumber)~Ours (w/o scene tag) & 78.47\% & 78.21\% \\
  \hline
  (\rownumber)~Ours (w/o unified embedding) & 76.41\% & 76.23\% \\
  (\rownumber)~Ours (w/o adaptive learning) & 76.83\% & 76.56\% \\
  (\rownumber)~Ours (w/o cross-modal fusion) & 76.85\% & 76.49\% \\
  \hline
  (\rownumber)~Ours  & \textbf{79.48\%} & \textbf{79.22\%} \\
  \hline
\end{tabular}
}
\end{center}
\vspace{-5mm}
\end{table}


\begin{figure}[t]
\centering
% \vspace{-2mm}
\includegraphics[width=0.42\textwidth]{fig/2dvisual-linux4-paper2.pdf}
\caption{Visualization of feature distribution on eight categories before (left) and after (right) model processing.}
% 
\label{fig:visualization}
\vspace{-5mm}
\end{figure}

\subsection{Ablation Performance}
In this subsection, we conduct an ablation study to examine which component is really important for performance improvement. The results are reported in Table~\ref{tab:cap2}.

For information utilization, we observe a significant decline in model performance when visual features are removed. Additionally, the performance of \shortname~decreases when different metadata are removed separately, which means that text description, object tag, and scene tag are all critical for image sentiment analysis.
Recalling the model architecture, we separately remove transformer layers of the unified representation module, the adaptive learning module, and the cross-modal fusion module, replacing them with MLPs of the same parameter scale.
In this way, we can observe varying degrees of decline in model performance, indicating that these modules are indispensable for our model to achieve better performance.

\subsection{Visualization}
% 


% % 开始使用minipage进行左右排列
% \begin{minipage}[t]{0.45\textwidth}  % 子图1宽度为45%
%     \centering
%     \includegraphics[width=\textwidth]{2dvisual.pdf}  % 插入图片
%     \captionof{figure}{Visualization of feature distribution.}  % 使用captionof添加图片标题
%     \label{fig:visualization}
% \end{minipage}


% \begin{figure}[t]
% \centering
% \vspace{-2mm}
% \includegraphics[width=0.45\textwidth]{fig/2dvisual.pdf}
% \caption{Visualization of feature distribution.}
% \label{fig:visualization}
% % \vspace{-4mm}
% \end{figure}

% \begin{figure}[t]
% \centering
% \vspace{-2mm}
% \includegraphics[width=0.45\textwidth]{fig/2dvisual-linux3-paper.pdf}
% \caption{Visualization of feature distribution.}
% \label{fig:visualization}
% % \vspace{-4mm}
% \end{figure}



\begin{figure}[tbp]   
\vspace{-4mm}
  \centering            
  \subfloat[Depth of adaptive learning layers]   
  {
    \label{fig:subfig1}\includegraphics[width=0.22\textwidth]{fig/fig_sensitivity-a5}
  }
  \subfloat[Depth of fusion layers]
  {
    % \label{fig:subfig2}\includegraphics[width=0.22\textwidth]{fig/fig_sensitivity-b2}
    \label{fig:subfig2}\includegraphics[width=0.22\textwidth]{fig/fig_sensitivity-b2-num.pdf}
  }
  \caption{Sensitivity study of \shortname~on different depth. }   
  \label{fig:fig_sensitivity}  
\vspace{-2mm}
\end{figure}

% \begin{figure}[htbp]
% \centerline{\includegraphics{2dvisual.pdf}}
% \caption{Visualization of feature distribution.}
% \label{fig:visualization}
% \end{figure}

% In Fig.~\ref{fig:visualization}, we use t-SNE~\cite{van2008visualizing} to reduce the dimension of data features for visualization, Figure in left represents the metadata features before model processing, the features are obtained by embedding through the CLIP model, and figure in right shows the features of the data after model processing, it can be observed that after the model processing, the data with different label categories fall in different regions in the space, therefore, we can conclude that the Therefore, we can conclude that the model can effectively utilize the information contained in the metadata and use it to guide the model for classification.

In Fig.~\ref{fig:visualization}, we use t-SNE~\cite{van2008visualizing} to reduce the dimension of data features for visualization.
The left figure shows metadata features before being processed by our model (\textit{i.e.}, embedded by CLIP), while the right shows the distribution of features after being processed by our model.
We can observe that after the model processing, data with the same label are closer to each other, while others are farther away.
Therefore, it shows that the model can effectively utilize the information contained in the metadata and use it to guide the classification process.

\subsection{Sensitivity Analysis}
% 
In this subsection, we conduct a sensitivity analysis to figure out the effect of different depth settings of adaptive learning layers and fusion layers. 
% In this subsection, we conduct a sensitivity analysis to figure out the effect of different depth settings on the model. 
% Fig.~\ref{fig:fig_sensitivity} presents the effect of different depth settings of adaptive learning layers and fusion layers. 
Taking Fig.~\ref{fig:fig_sensitivity} (a) as an example, the model performance improves with increasing depth, reaching the best performance at a depth of 4.
% Taking Fig.~\ref{fig:fig_sensitivity} (a) as an example, the performance of \shortname~improves with the increase of depth at first, reaching the best performance at a depth of 4.
When the depth continues to increase, the accuracy decreases to varying degrees.
Similar results can be observed in Fig.~\ref{fig:fig_sensitivity} (b).
Therefore, we set their depths to 4 and 6 respectively to achieve the best results.

% Through our experiments, we can observe that the effect of modifying these hyperparameters on the results of the experiments is very weak, and the surface model is not sensitive to the hyperparameters.


\subsection{Zero-shot Capability}
% 

% (1)~GCH~\cite{2010Analyzing} & 21.78\% & (5)~RA-DLNet~\cite{2020A} & 34.01\% \\ \hline
% (2)~WSCNet~\cite{2019WSCNet}  & 30.25\% & (6)~CECCN~\cite{ruan2024color} & 43.83\% \\ \hline
% (3)~PCNN~\cite{2015Robust} & 31.68\%  & (7)~EmoVIT~\cite{xie2024emovit} & 44.90\% \\ \hline
% (4)~AR~\cite{2018Visual} & 32.67\% & (8)~Ours (Zero-shot) & 47.83\% \\ \hline


\begin{table}[t]
\centering
\caption{Zero-shot capability of \shortname.}
\label{tab:cap3}
\resizebox{1\linewidth}{!}
{
\begin{tabular}{lc|lc}
\hline
\textbf{Model} & \textbf{Accuracy} & \textbf{Model} & \textbf{Accuracy} \\ \hline
(1)~WSCNet~\cite{2019WSCNet}  & 30.25\% & (5)~MAM~\cite{zhang2024affective} & 39.56\%  \\ \hline
(2)~AR~\cite{2018Visual} & 32.67\% & (6)~CECCN~\cite{ruan2024color} & 43.83\% \\ \hline
(3)~RA-DLNet~\cite{2020A} & 34.01\%  & (7)~EmoVIT~\cite{xie2024emovit} & 44.90\% \\ \hline
(4)~CDA~\cite{han2023boosting} & 38.64\% & (8)~Ours (Zero-shot) & 47.83\% \\ \hline
\end{tabular}
}
\vspace{-5mm}
\end{table}

% We use the model trained on the FI dataset to test on the artphoto dataset to verify the model's generalization ability as well as robustness to other distributed datasets.
% We can observe that the MESN model shows strong competitiveness in terms of accuracy when compared to other trained models, which suggests that the model has a good generalization ability in the OOD task.

To validate the model's generalization ability and robustness to other distributed datasets, we directly test the model trained on the FI dataset, without training on Artphoto. 
% As observed in Table 3, compared to other models trained on Artphoto, we achieve highly competitive zero-shot performance, indicating that the model has good generalization ability in out-of-distribution tasks.
From Table~\ref{tab:cap3}, we can observe that compared with other models trained on Artphoto, we achieve competitive zero-shot performance, which shows that the model has good generalization ability in out-of-distribution tasks.


%%%%%%%%%%%%
%  E2E     %
%%%%%%%%%%%%


\section{Conclusion}
In this paper, we introduced Wi-Chat, the first LLM-powered Wi-Fi-based human activity recognition system that integrates the reasoning capabilities of large language models with the sensing potential of wireless signals. Our experimental results on a self-collected Wi-Fi CSI dataset demonstrate the promising potential of LLMs in enabling zero-shot Wi-Fi sensing. These findings suggest a new paradigm for human activity recognition that does not rely on extensive labeled data. We hope future research will build upon this direction, further exploring the applications of LLMs in signal processing domains such as IoT, mobile sensing, and radar-based systems.

\section*{Limitations}
While our work represents the first attempt to leverage LLMs for processing Wi-Fi signals, it is a preliminary study focused on a relatively simple task: Wi-Fi-based human activity recognition. This choice allows us to explore the feasibility of LLMs in wireless sensing but also comes with certain limitations.

Our approach primarily evaluates zero-shot performance, which, while promising, may still lag behind traditional supervised learning methods in highly complex or fine-grained recognition tasks. Besides, our study is limited to a controlled environment with a self-collected dataset, and the generalizability of LLMs to diverse real-world scenarios with varying Wi-Fi conditions, environmental interference, and device heterogeneity remains an open question.

Additionally, we have yet to explore the full potential of LLMs in more advanced Wi-Fi sensing applications, such as fine-grained gesture recognition, occupancy detection, and passive health monitoring. Future work should investigate the scalability of LLM-based approaches, their robustness to domain shifts, and their integration with multimodal sensing techniques in broader IoT applications.


% Bibliography entries for the entire Anthology, followed by custom entries
%\bibliography{anthology,custom}
% Custom bibliography entries only
\bibliography{main}
\newpage
\appendix

\section{Experiment prompts}
\label{sec:prompt}
The prompts used in the LLM experiments are shown in the following Table~\ref{tab:prompts}.

\definecolor{titlecolor}{rgb}{0.9, 0.5, 0.1}
\definecolor{anscolor}{rgb}{0.2, 0.5, 0.8}
\definecolor{labelcolor}{HTML}{48a07e}
\begin{table*}[h]
	\centering
	
 % \vspace{-0.2cm}
	
	\begin{center}
		\begin{tikzpicture}[
				chatbox_inner/.style={rectangle, rounded corners, opacity=0, text opacity=1, font=\sffamily\scriptsize, text width=5in, text height=9pt, inner xsep=6pt, inner ysep=6pt},
				chatbox_prompt_inner/.style={chatbox_inner, align=flush left, xshift=0pt, text height=11pt},
				chatbox_user_inner/.style={chatbox_inner, align=flush left, xshift=0pt},
				chatbox_gpt_inner/.style={chatbox_inner, align=flush left, xshift=0pt},
				chatbox/.style={chatbox_inner, draw=black!25, fill=gray!7, opacity=1, text opacity=0},
				chatbox_prompt/.style={chatbox, align=flush left, fill=gray!1.5, draw=black!30, text height=10pt},
				chatbox_user/.style={chatbox, align=flush left},
				chatbox_gpt/.style={chatbox, align=flush left},
				chatbox2/.style={chatbox_gpt, fill=green!25},
				chatbox3/.style={chatbox_gpt, fill=red!20, draw=black!20},
				chatbox4/.style={chatbox_gpt, fill=yellow!30},
				labelbox/.style={rectangle, rounded corners, draw=black!50, font=\sffamily\scriptsize\bfseries, fill=gray!5, inner sep=3pt},
			]
											
			\node[chatbox_user] (q1) {
				\textbf{System prompt}
				\newline
				\newline
				You are a helpful and precise assistant for segmenting and labeling sentences. We would like to request your help on curating a dataset for entity-level hallucination detection.
				\newline \newline
                We will give you a machine generated biography and a list of checked facts about the biography. Each fact consists of a sentence and a label (True/False). Please do the following process. First, breaking down the biography into words. Second, by referring to the provided list of facts, merging some broken down words in the previous step to form meaningful entities. For example, ``strategic thinking'' should be one entity instead of two. Third, according to the labels in the list of facts, labeling each entity as True or False. Specifically, for facts that share a similar sentence structure (\eg, \textit{``He was born on Mach 9, 1941.''} (\texttt{True}) and \textit{``He was born in Ramos Mejia.''} (\texttt{False})), please first assign labels to entities that differ across atomic facts. For example, first labeling ``Mach 9, 1941'' (\texttt{True}) and ``Ramos Mejia'' (\texttt{False}) in the above case. For those entities that are the same across atomic facts (\eg, ``was born'') or are neutral (\eg, ``he,'' ``in,'' and ``on''), please label them as \texttt{True}. For the cases that there is no atomic fact that shares the same sentence structure, please identify the most informative entities in the sentence and label them with the same label as the atomic fact while treating the rest of the entities as \texttt{True}. In the end, output the entities and labels in the following format:
                \begin{itemize}[nosep]
                    \item Entity 1 (Label 1)
                    \item Entity 2 (Label 2)
                    \item ...
                    \item Entity N (Label N)
                \end{itemize}
                % \newline \newline
                Here are two examples:
                \newline\newline
                \textbf{[Example 1]}
                \newline
                [The start of the biography]
                \newline
                \textcolor{titlecolor}{Marianne McAndrew is an American actress and singer, born on November 21, 1942, in Cleveland, Ohio. She began her acting career in the late 1960s, appearing in various television shows and films.}
                \newline
                [The end of the biography]
                \newline \newline
                [The start of the list of checked facts]
                \newline
                \textcolor{anscolor}{[Marianne McAndrew is an American. (False); Marianne McAndrew is an actress. (True); Marianne McAndrew is a singer. (False); Marianne McAndrew was born on November 21, 1942. (False); Marianne McAndrew was born in Cleveland, Ohio. (False); She began her acting career in the late 1960s. (True); She has appeared in various television shows. (True); She has appeared in various films. (True)]}
                \newline
                [The end of the list of checked facts]
                \newline \newline
                [The start of the ideal output]
                \newline
                \textcolor{labelcolor}{[Marianne McAndrew (True); is (True); an (True); American (False); actress (True); and (True); singer (False); , (True); born (True); on (True); November 21, 1942 (False); , (True); in (True); Cleveland, Ohio (False); . (True); She (True); began (True); her (True); acting career (True); in (True); the late 1960s (True); , (True); appearing (True); in (True); various (True); television shows (True); and (True); films (True); . (True)]}
                \newline
                [The end of the ideal output]
				\newline \newline
                \textbf{[Example 2]}
                \newline
                [The start of the biography]
                \newline
                \textcolor{titlecolor}{Doug Sheehan is an American actor who was born on April 27, 1949, in Santa Monica, California. He is best known for his roles in soap operas, including his portrayal of Joe Kelly on ``General Hospital'' and Ben Gibson on ``Knots Landing.''}
                \newline
                [The end of the biography]
                \newline \newline
                [The start of the list of checked facts]
                \newline
                \textcolor{anscolor}{[Doug Sheehan is an American. (True); Doug Sheehan is an actor. (True); Doug Sheehan was born on April 27, 1949. (True); Doug Sheehan was born in Santa Monica, California. (False); He is best known for his roles in soap operas. (True); He portrayed Joe Kelly. (True); Joe Kelly was in General Hospital. (True); General Hospital is a soap opera. (True); He portrayed Ben Gibson. (True); Ben Gibson was in Knots Landing. (True); Knots Landing is a soap opera. (True)]}
                \newline
                [The end of the list of checked facts]
                \newline \newline
                [The start of the ideal output]
                \newline
                \textcolor{labelcolor}{[Doug Sheehan (True); is (True); an (True); American (True); actor (True); who (True); was born (True); on (True); April 27, 1949 (True); in (True); Santa Monica, California (False); . (True); He (True); is (True); best known (True); for (True); his roles in soap operas (True); , (True); including (True); in (True); his portrayal (True); of (True); Joe Kelly (True); on (True); ``General Hospital'' (True); and (True); Ben Gibson (True); on (True); ``Knots Landing.'' (True)]}
                \newline
                [The end of the ideal output]
				\newline \newline
				\textbf{User prompt}
				\newline
				\newline
				[The start of the biography]
				\newline
				\textcolor{magenta}{\texttt{\{BIOGRAPHY\}}}
				\newline
				[The ebd of the biography]
				\newline \newline
				[The start of the list of checked facts]
				\newline
				\textcolor{magenta}{\texttt{\{LIST OF CHECKED FACTS\}}}
				\newline
				[The end of the list of checked facts]
			};
			\node[chatbox_user_inner] (q1_text) at (q1) {
				\textbf{System prompt}
				\newline
				\newline
				You are a helpful and precise assistant for segmenting and labeling sentences. We would like to request your help on curating a dataset for entity-level hallucination detection.
				\newline \newline
                We will give you a machine generated biography and a list of checked facts about the biography. Each fact consists of a sentence and a label (True/False). Please do the following process. First, breaking down the biography into words. Second, by referring to the provided list of facts, merging some broken down words in the previous step to form meaningful entities. For example, ``strategic thinking'' should be one entity instead of two. Third, according to the labels in the list of facts, labeling each entity as True or False. Specifically, for facts that share a similar sentence structure (\eg, \textit{``He was born on Mach 9, 1941.''} (\texttt{True}) and \textit{``He was born in Ramos Mejia.''} (\texttt{False})), please first assign labels to entities that differ across atomic facts. For example, first labeling ``Mach 9, 1941'' (\texttt{True}) and ``Ramos Mejia'' (\texttt{False}) in the above case. For those entities that are the same across atomic facts (\eg, ``was born'') or are neutral (\eg, ``he,'' ``in,'' and ``on''), please label them as \texttt{True}. For the cases that there is no atomic fact that shares the same sentence structure, please identify the most informative entities in the sentence and label them with the same label as the atomic fact while treating the rest of the entities as \texttt{True}. In the end, output the entities and labels in the following format:
                \begin{itemize}[nosep]
                    \item Entity 1 (Label 1)
                    \item Entity 2 (Label 2)
                    \item ...
                    \item Entity N (Label N)
                \end{itemize}
                % \newline \newline
                Here are two examples:
                \newline\newline
                \textbf{[Example 1]}
                \newline
                [The start of the biography]
                \newline
                \textcolor{titlecolor}{Marianne McAndrew is an American actress and singer, born on November 21, 1942, in Cleveland, Ohio. She began her acting career in the late 1960s, appearing in various television shows and films.}
                \newline
                [The end of the biography]
                \newline \newline
                [The start of the list of checked facts]
                \newline
                \textcolor{anscolor}{[Marianne McAndrew is an American. (False); Marianne McAndrew is an actress. (True); Marianne McAndrew is a singer. (False); Marianne McAndrew was born on November 21, 1942. (False); Marianne McAndrew was born in Cleveland, Ohio. (False); She began her acting career in the late 1960s. (True); She has appeared in various television shows. (True); She has appeared in various films. (True)]}
                \newline
                [The end of the list of checked facts]
                \newline \newline
                [The start of the ideal output]
                \newline
                \textcolor{labelcolor}{[Marianne McAndrew (True); is (True); an (True); American (False); actress (True); and (True); singer (False); , (True); born (True); on (True); November 21, 1942 (False); , (True); in (True); Cleveland, Ohio (False); . (True); She (True); began (True); her (True); acting career (True); in (True); the late 1960s (True); , (True); appearing (True); in (True); various (True); television shows (True); and (True); films (True); . (True)]}
                \newline
                [The end of the ideal output]
				\newline \newline
                \textbf{[Example 2]}
                \newline
                [The start of the biography]
                \newline
                \textcolor{titlecolor}{Doug Sheehan is an American actor who was born on April 27, 1949, in Santa Monica, California. He is best known for his roles in soap operas, including his portrayal of Joe Kelly on ``General Hospital'' and Ben Gibson on ``Knots Landing.''}
                \newline
                [The end of the biography]
                \newline \newline
                [The start of the list of checked facts]
                \newline
                \textcolor{anscolor}{[Doug Sheehan is an American. (True); Doug Sheehan is an actor. (True); Doug Sheehan was born on April 27, 1949. (True); Doug Sheehan was born in Santa Monica, California. (False); He is best known for his roles in soap operas. (True); He portrayed Joe Kelly. (True); Joe Kelly was in General Hospital. (True); General Hospital is a soap opera. (True); He portrayed Ben Gibson. (True); Ben Gibson was in Knots Landing. (True); Knots Landing is a soap opera. (True)]}
                \newline
                [The end of the list of checked facts]
                \newline \newline
                [The start of the ideal output]
                \newline
                \textcolor{labelcolor}{[Doug Sheehan (True); is (True); an (True); American (True); actor (True); who (True); was born (True); on (True); April 27, 1949 (True); in (True); Santa Monica, California (False); . (True); He (True); is (True); best known (True); for (True); his roles in soap operas (True); , (True); including (True); in (True); his portrayal (True); of (True); Joe Kelly (True); on (True); ``General Hospital'' (True); and (True); Ben Gibson (True); on (True); ``Knots Landing.'' (True)]}
                \newline
                [The end of the ideal output]
				\newline \newline
				\textbf{User prompt}
				\newline
				\newline
				[The start of the biography]
				\newline
				\textcolor{magenta}{\texttt{\{BIOGRAPHY\}}}
				\newline
				[The ebd of the biography]
				\newline \newline
				[The start of the list of checked facts]
				\newline
				\textcolor{magenta}{\texttt{\{LIST OF CHECKED FACTS\}}}
				\newline
				[The end of the list of checked facts]
			};
		\end{tikzpicture}
        \caption{GPT-4o prompt for labeling hallucinated entities.}\label{tb:gpt-4-prompt}
	\end{center}
\vspace{-0cm}
\end{table*}
% \section{Full Experiment Results}
% \begin{table*}[th]
    \centering
    \small
    \caption{Classification Results}
    \begin{tabular}{lcccc}
        \toprule
        \textbf{Method} & \textbf{Accuracy} & \textbf{Precision} & \textbf{Recall} & \textbf{F1-score} \\
        \midrule
        \multicolumn{5}{c}{\textbf{Zero Shot}} \\
                Zero-shot E-eyes & 0.26 & 0.26 & 0.27 & 0.26 \\
        Zero-shot CARM & 0.24 & 0.24 & 0.24 & 0.24 \\
                Zero-shot SVM & 0.27 & 0.28 & 0.28 & 0.27 \\
        Zero-shot CNN & 0.23 & 0.24 & 0.23 & 0.23 \\
        Zero-shot RNN & 0.26 & 0.26 & 0.26 & 0.26 \\
DeepSeek-0shot & 0.54 & 0.61 & 0.54 & 0.52 \\
DeepSeek-0shot-COT & 0.33 & 0.24 & 0.33 & 0.23 \\
DeepSeek-0shot-Knowledge & 0.45 & 0.46 & 0.45 & 0.44 \\
Gemma2-0shot & 0.35 & 0.22 & 0.38 & 0.27 \\
Gemma2-0shot-COT & 0.36 & 0.22 & 0.36 & 0.27 \\
Gemma2-0shot-Knowledge & 0.32 & 0.18 & 0.34 & 0.20 \\
GPT-4o-mini-0shot & 0.48 & 0.53 & 0.48 & 0.41 \\
GPT-4o-mini-0shot-COT & 0.33 & 0.50 & 0.33 & 0.38 \\
GPT-4o-mini-0shot-Knowledge & 0.49 & 0.31 & 0.49 & 0.36 \\
GPT-4o-0shot & 0.62 & 0.62 & 0.47 & 0.42 \\
GPT-4o-0shot-COT & 0.29 & 0.45 & 0.29 & 0.21 \\
GPT-4o-0shot-Knowledge & 0.44 & 0.52 & 0.44 & 0.39 \\
LLaMA-0shot & 0.32 & 0.25 & 0.32 & 0.24 \\
LLaMA-0shot-COT & 0.12 & 0.25 & 0.12 & 0.09 \\
LLaMA-0shot-Knowledge & 0.32 & 0.25 & 0.32 & 0.28 \\
Mistral-0shot & 0.19 & 0.23 & 0.19 & 0.10 \\
Mistral-0shot-Knowledge & 0.21 & 0.40 & 0.21 & 0.11 \\
        \midrule
        \multicolumn{5}{c}{\textbf{4 Shot}} \\
GPT-4o-mini-4shot & 0.58 & 0.59 & 0.58 & 0.53 \\
GPT-4o-mini-4shot-COT & 0.57 & 0.53 & 0.57 & 0.50 \\
GPT-4o-mini-4shot-Knowledge & 0.56 & 0.51 & 0.56 & 0.47 \\
GPT-4o-4shot & 0.77 & 0.84 & 0.77 & 0.73 \\
GPT-4o-4shot-COT & 0.63 & 0.76 & 0.63 & 0.53 \\
GPT-4o-4shot-Knowledge & 0.72 & 0.82 & 0.71 & 0.66 \\
LLaMA-4shot & 0.29 & 0.24 & 0.29 & 0.21 \\
LLaMA-4shot-COT & 0.20 & 0.30 & 0.20 & 0.13 \\
LLaMA-4shot-Knowledge & 0.15 & 0.23 & 0.13 & 0.13 \\
Mistral-4shot & 0.02 & 0.02 & 0.02 & 0.02 \\
Mistral-4shot-Knowledge & 0.21 & 0.27 & 0.21 & 0.20 \\
        \midrule
        
        \multicolumn{5}{c}{\textbf{Suprevised}} \\
        SVM & 0.94 & 0.92 & 0.91 & 0.91 \\
        CNN & 0.98 & 0.98 & 0.97 & 0.97 \\
        RNN & 0.99 & 0.99 & 0.99 & 0.99 \\
        % \midrule
        % \multicolumn{5}{c}{\textbf{Conventional Wi-Fi-based Human Activity Recognition Systems}} \\
        E-eyes & 1.00 & 1.00 & 1.00 & 1.00 \\
        CARM & 0.98 & 0.98 & 0.98 & 0.98 \\
\midrule
 \multicolumn{5}{c}{\textbf{Vision Models}} \\
           Zero-shot SVM & 0.26 & 0.25 & 0.25 & 0.25 \\
        Zero-shot CNN & 0.26 & 0.25 & 0.26 & 0.26 \\
        Zero-shot RNN & 0.28 & 0.28 & 0.29 & 0.28 \\
        SVM & 0.99 & 0.99 & 0.99 & 0.99 \\
        CNN & 0.98 & 0.99 & 0.98 & 0.98 \\
        RNN & 0.98 & 0.99 & 0.98 & 0.98 \\
GPT-4o-mini-Vision & 0.84 & 0.85 & 0.84 & 0.84 \\
GPT-4o-mini-Vision-COT & 0.90 & 0.91 & 0.90 & 0.90 \\
GPT-4o-Vision & 0.74 & 0.82 & 0.74 & 0.73 \\
GPT-4o-Vision-COT & 0.70 & 0.83 & 0.70 & 0.68 \\
LLaMA-Vision & 0.20 & 0.23 & 0.20 & 0.09 \\
LLaMA-Vision-Knowledge & 0.22 & 0.05 & 0.22 & 0.08 \\

        \bottomrule
    \end{tabular}
    \label{full}
\end{table*}




\end{document}


As shown in Table \ref{tab:main_exp_sum}, a significant gap persists between EM and other evaluation metrics across all models. This observation indicates that while LLMs can address specific aspects of a given task, achieving a complete and accurate solution remains a significant challenge. Moreover, closed-source models generally exhibit superior performance across key metrics (S-Acc, EM, and PM-0.5) to open-source models, with o1-preview achieving the best EM score of 23.6\%. DeepSeek-R1 demonstrates the most advanced reasoning capabilities among open-source models and even achieves a comparable EM score of 20.0\% to o1-preview. Notably, there also exists a great performance gap between conventional models and o1-like models. Although GPT-4o shows the second-highest S-Acc, it lags significantly behind the top-performing model o1-preview in EM. This highlights the effectiveness of the "slow thinking" approaches which trade better performance with more inference tokens.



\section{Analysis}
\begin{figure}[!h]
  \centering
  % \vspace*{1mm}
  \includegraphics[width=0.8\columnwidth]{fig/difficulty.pdf}
  \caption{\label{difficulty}Performance trend of GPT-4o and o1-preview under varying difficulty levels.}
\end{figure}
\subsection{Task Difficulty}

We present the detailed model performance across varying difficulty for all six tasks in Appendix \ref{app_difficulty}. Here, we focus on the Logic Puzzle, illustrating the impact of increasing problem complexity. Figure \ref{difficulty} shows the performance of GPT-4o and o1-preview under different difficulty levels. The substantial performance gap between the two models demonstrates o1-preview's superior logical deduction and reflective reasoning abilities. As the solution space expands from $4\times4$ to $4\times7$, while models can still solve part of the problem, they fail to find the complete solution. This highlights the challenge of exploring exponentially increasing solution spaces for complex reasoning tasks.



\subsection{Long Reasoning Chain Generation}

Tasks in LR$^2$Bench often involve multiple subtasks (e.g., solving all clues for Crossword, inferring the digits for each cell for Sudoku). Such inherent complexity requires LLMs to continuously explore quite large solution spaces through reflective reasoning processes thus presenting a significant challenge for LLMs in generating long reasoning chains. As shown in Table \ref{tab:main_exp_sum}, many models failed to completely generate the entire reasoning process, resulting in low CR scores. Through analysis of incomplete model responses, we find that a key obstacle to this phenomenon is the tendency of LLMs to generate repetitive content, ultimately reaching the maximum sequence length. This redundancy primarily occurs when LLMs encounter contradictions (see Section \ref{qualitative} for a detailed analysis). Such redundancy wastes valuable context window size, preventing the model from exploring the full solution space and completing further reasoning processes. To quantify this, we calculate the average 10-gram redundancy ratio of the models' responses across all tasks in LR$^2$Bench, excluding Sudoku due to its inherently repetitive cell-by-cell reasoning strategy. Figure \ref{redundant} reveals a strong negative correlation between redundancy and CR, suggesting that redundant generation is a key factor limiting the long-chain reasoning capability of LLMs. Notably, QwQ-32B-Preview shows lower redundancy but still fails in completion due to its endless trial-and-error without reaching a meaningful conclusion.


\subsection{Conventional Models vs. o1-like Models}
% different tasks, difficulty
As shown in Table \ref{tab:main_exp}, for Crossword and Acrostic tasks, we observe that conventional models can achieve high S-Acc and PM-0.5, but nearly zero EM. This suggests that LLMs with strong commonsense reasoning abilities can correctly infer parts of words from the provided clues. However, some clues may have multiple valid answers, requiring LLMs to determine the final answers based on the constraint of shared letters with other words. Therefore, the lack of reflective reasoning prevents these models from achieving perfect solutions. Besides, GPT-4o performs better than o1-mini on these tasks, suggesting that only models possessing both capabilities can achieve high EM scores. 

\begin{figure}[t!]
  \centering
  \includegraphics[width=1.0\columnwidth]{fig/r.pdf}
  \caption{The relationship between redundancy and CR across different models.}
  \label{redundant}
  \vspace{-4mm}
\end{figure}

For tasks that heavily rely on reflection capabilities, such as Logic Puzzle and Sudoku, the performance gap between o1-like models and conventional models is the most obvious. Notably, o1-preview outperforms GPT-4o by 37.5\% and 35.5\% and QwQ-32B-Preview outperforms Qwen-2.5-70B-Instruct by 19.5\% and 26.0\% on EM for these two tasks, respectively. These substantial gains highlight the significant advantage of o1-like models in tasks that require verifying multiple constraints and employing backtracking mechanisms. In contrast, conventional models often exhibit limitations with their single-pass reasoning paradigm. This approach proves inadequate for scenarios requiring exploration of alternative solution paths or correction of initial assumptions.

Cryptogram and Drop Quote pose the most significant challenges for all evaluated models. We attribute this difficulty to the expansive search space and limited helpful clues inherent in these problems. In the Cryptogram, each letter has theoretically 26 possible substitutions. Similarly, in Drop Quote, each letter within a given column can be dropped to any row. This contrasts with Logic Puzzles and Sudoku, which offer strong constraints that effectively reduce the search space. Even the most advanced model o1-preview struggles to complete these tasks with only a 13\% EM score.


\subsection{Qualitative Analysis}
\label{qualitative}

We analyze several typical behaviors of LLMs leading to the failure in completing our benchmark and provide detailed cases in Appendix \ref{app_case}.

\paragraph{Lack of Reflection Mechanism} This deficiency mainly occurs in conventional models. Taking the Logic Puzzle for example, our analysis of GPT-4o's responses in Appendix \ref{lack} reveals that although the model can effectively break down individual clues and generate initial deductions, it fails to perform the iterative cross-checking to ensure consistency with all established constraints. These shortages indicate that conventional models lack the reflective reasoning capabilities necessary to solve complex constraint satisfaction problems.

\paragraph{Stuck in Contradictions} We observe that LLMs often struggle with complex reasoning tasks when confronted with contradictions. As shown in Appendix \ref{stuck}, this phenomenon mainly manifests as looping within similar sentences, repeatedly stating conflicting information without making progress toward a coherent resolution. Instead of backtracking and revising previous assumptions upon encountering a contradiction, they tend to focus only on the conflicting points. This narrow focus prevents them from exploring alternative solution paths.

\paragraph{Give-up Moment} As illustrated in Appendix \ref{giveup}, the occurrence of a "Give-up Moment" is observed when reasoning-specific LLMs struggle with complex problems due to the time (inference tokens) constraint. We focus on one of the worst-performing tasks Cryptogram to quantify this give-up ratio in QwQ-32B and DeepSeek-R1 responses, respectively. We manually review 50 responses from each model on Cryptogram. For QwQ-32B, 15 out of 50 responses are incomplete due to the maximum context window size of 32k. 26 out of 50 responses show model's give-up and only provide part of the answers. Similarly, for DeepSeek-R1, 39 out of 50 responses compromise to give the most possible answers given the time constraints. This highlights the challenges posed by problems requiring extensive reflective reasoning and suggests a potential bottleneck in the current capabilities of even the leading reasoning LLMs.

\subsection{Future Directions for o1-like LLMs}
Addressing the limitations highlighted in our qualitative analysis suggests several key directions for future o1-like LLM development:

\paragraph{Reasoning Paradigm Transformation} Conventional LLMs fail in solving complex CSPs in LR$^2$Bench due to the lack of reflection mechanisms. This suggests the importance of enabling LLMs to perform iterative, multi-step reflective reasoning processes. Future research would shift the focus from linear reasoning chains to dynamic, reflective thinking paradigms.

\paragraph{Reasoning Adaptation} Current LLMs often employ a uniform reasoning strategy regardless of problem complexity or type. Future research should prioritize developing adaptive mechanisms that allow LLMs to dynamically adjust their reasoning processes based on specific problems. Simpler problems may only require a straightforward approach, while complex problems necessitate a more deliberate, multi-step, and reflective process. Furthermore, different tasks may demand distinct reasoning approaches. LLMs should be able to determine the type of reasoning required (e.g., logical, mathematical, commonsense) and adjust their strategies accordingly.

\paragraph{Enhancing Reflection Mechanisms} Although leading o1-like LLMs have possessed reflective reasoning capabilities, they struggle to effectively utilize them when confronted with contradictions. Future development should focus on enabling models to recognize inconsistencies and pivot towards alternative hypotheses, emphasizing flexibility and exploration of different reasoning trajectories rather than fixation on conflicting information.

\paragraph{Improving Reasoning Efficiency} Current o1-like models tend to generate lengthy reasoning chains when tackling complex problems, leading to significant inference costs. Future work could prioritize optimizing these reasoning chains to achieve greater efficiency without sacrificing accuracy. 

\section{Related Work}
% \subsection{Reasoning with LLMs}
% \subsection{Reflective Reasoning Evaluation}
\subsection{Reflection Capabilities of LLMs}
The impressive reasoning capabilities of LLMs have naturally led to increased interest in reflection, a more sophisticated and human-like form of reasoning. Previous studies primarily focus on evaluating LLMs' abilities to rectify their response based on explicit feedback, including self-improvement with critique prompts \citep{lan2024criticeval, li2024reflection, lin-etal-2024-criticbench, li2024hindsight, madaan2024self}, leveraging external tools such as code interpreters or search engines \citep{gou2024critic, chen2024teaching, shinn2024reflexion}, and engaging in multi-LLM interaction through debating \citep{liang-etal-2024-encouraging, huang2024large}. However, these works mainly evaluate LLMs' behaviors in response to feedback. They fail to assess LLMs' capabilities to spontaneously engage in the complete reflection process for complex reasoning tasks. Our proposed LR$^2$Bench provides scenarios necessitating capabilities, such as making assumptions, verification, backtracking, and self-refinement, thus filling a critical gap in evaluating LLMs' intrinsic reflective reasoning abilities.

% \subsection{Complex Reasoning Tasks for LLMs}
\subsection{Puzzle-solving for LLMs}
Puzzle-solving \citep{giadikiaroglou2024puzzle} offers valuable insight for evaluating the complex reasoning capabilities of LLMs across diverse scenarios. \citet{ishay2023leveraging} explore Sudoku solving strategies with answer set programming. \citet{ding2023everything} leverages reinforcement learning and Monte Carlo Tree Search to solve problems like Game of 24, 8-Puzzle, and Pocket Cube. \citet{yao2024tree} introduces "Tree of Thought" to enable self-evaluating and backtracking for Game of 24 and Crosswords. \citet{mittal2024puzzlebench} combines LLMs with symbolic solvers and program interpreters to complete first-order combinatorial reasoning problems. \citet{tyagi2024step} focuses on grid puzzles to evaluate the generated reasoning chains of LLMs. Moreover, existing studies have also investigate Board game \citep{kazemi2024boardgameqa}, Chess \citep{feng2024chessgpt} and social games \citep{light2023avalonbench, wang2023avalon, xu2023exploring}. However, these studies primarily leverage external tools or specialized algorithms to develop task-specific solutions within limited puzzle domains. In contrast, LR$^2$Bench provides diverse tasks and difficulty levels and focuses on evaluating the intrinsic reflective reasoning capabilities of LLMs.


\section{Conclusion}
This paper introduces LR$^2$Bench, a novel benchmark to comprehensively evaluate the reflection capabilities of LLMs in long-chain reasoning. LR$^2$Bench comprises six tasks with varying difficulty levels, providing a thorough analysis across diverse scenarios. The experimental results show that o1-like models outperform conventional models, demonstrating their superior performance on reflective reasoning. Our findings also highlight the limitation of current reasoning LLMs and reveal that even the most advanced reasoning models fall short of achieving satisfactory performance, suggesting significant room for enhancement in reflective reasoning capabilities.

\section*{Limitations}
The limitations of our work can be summarized as follows: Firstly, due to the scarcity of well-defined real-world constraint satisfaction problems, we relied on puzzle-like data for evaluating LLMs' reflective reasoning capabilities. Secondly, the inherent complexity and verbosity of LLM-generated responses to these complex reasoning tasks posed challenges for more fine-grained analysis. We only analyze several typical phenomena of current leading models rather than conducting a more detailed analysis of specific reflective reasoning processes.

% Bibliography entries for the entire Anthology, followed by custom entries
%\bibliography{anthology,custom}
% Custom bibliography entries only
\bibliography{custom}

\clearpage

\appendix

\section{Task Example}
\label{app_example}
In this section, we provide specific examples of each type of task to facilitate a better understanding of data samples in LR$^2$Bench.

\subsection{Crossword}
\begin{tcolorbox}[breakable, title=Crossword Example, colframe=blue!60]
\begin{CustomVerbatim}
Grid:
#   1   2   3   #
4   ?   ?   ?   5
6   ?   ?   ?   ?
7   ?   ?   ?   ?
8   ?   ?   ?   #

Clues:
Across:
1: "When They See Us" director DuVernay (3)
4: WNBA team based in Seattle (5)
6: Locations for orations (5)
7: Signed (5)
8: Lipton products (4)
Down:
1: Be penitent (5)
2: Black Russian ingredient (5)
3: First sign of the zodiac (5)
4: Barbecue rod (4)
5: Fuming (3)

Answer:
#   A   V   A   #
S   T   O   R   M
P   O   D   I   A
I   N   K   E   D
T   E   A   S   #
\end{CustomVerbatim}
\end{tcolorbox}

\subsection{Acrostic}
\begin{tcolorbox}[breakable, title=Acrostic Example, colframe=blue!60]
\begin{CustomVerbatim}
Grid:
1 2 3 4 # 5 6 7 # 8 9 10 11 # 12 13 14 # 15 16 17 18 19 20 21 22 # 23 24 25 26 27 28 # 29 30 31 32 33 34 # 35 # 36 37 38 39 40 41 ' 42 # 43 44 45 46 . 47 48 49 # 50 51 52 # 53 54 55 56 57 58 # 59 60 # 61 62 # 63 64 65 66 , 67 68 # 69 70 71 72 73 74 # 75 76 # 77 78 # 79 80 81 82 . 83 84 ' 85 # 86 # 87 88 89 90 # 91 92 93 94 95 96 .

Clues:
A
57 18 88 9 33 80 78
Gaynor hit \"I Will ____\" (1 word, 7 letters)
B
29 70 67 82 21
Wheat or milk beginning (1 word, 5 letters)
C
5 7 24 11 35 75 90
Extol (1 word, 7 letters)
D
30 50 10 91 46 14
Wintry projection (1 word, 6 letters)
E
52 26 73 58 94
Second-generation Japanese (1 word, 5 letters)
F
59 93 45 95 54 43 84 65 41 27
Standard (1 word, 10 letters)
G
92 72 60 4 37 66
Like some sweatshirts (1 word, 6 letters)
H
74 34 53 64 28 89 22
Scrambles a message (1 word, 7 letters)
I
36 25 2 61 86 23 83 19 81 31 47
Odds (1 word, 11 letters)
J
38 96 63 15 40 6 85
The Arctic and the Antarctic, e.g. (1 word, 7 letters)
K
71 8 49 87 20
Parisian eggs (1 word, 5 letters)
L
1 56 76 77 79 13
Peanut (1 word, 6 letters)
M
68 48 55 39
Aussie jumpers (1 word, 4 letters)
N
12 17 44 42 62
Strike ___ (what models do) (2 words, 5 letters)
O
16 51 69 32 3
Far from wimpy (1 word, 5 letters)

Answer:
"Good and evil are impulses buried within a person's soul. You can choose to be good, or choose to be evil. It's a free choice." - James Swain, The Program
\end{CustomVerbatim}
\end{tcolorbox}

\subsection{Logic Puzzle}
\begin{tcolorbox}[breakable, title=Logic Puzzle Example, colframe=blue!60]
\begin{CustomVerbatim}
Variables:
{
    "anniversaries": [
        "January 28",
        "March 6",
        "November 2",
        "November 23"
    ],
    "first names": [
        "Asher",
        "Bradley",
        "Kayla",
        "Malia"
    ],
    "conveyances": [
        "10-speed bike",
        "mountain bike",
        "segway",
        "skateboard"
    ],
    "actors": [
        "Liam Neeson",
        "Morgan Freeman",
        "Robert Duvall",
        "Tom Cruise"
    ]
}

Clues:
1. The commuter who rides a 10-speed bike is not Bradley.
2. Kayla isn't related to Robert Duvall and doesn't ride a segway.
3. Robert Duvall's cousin has an anniversary after the commuter who rides a segway.
4. Morgan Freeman's cousin is not Asher and doesn't ride a segway.
5. Liam Neeson's cousin is Malia.
6. Liam Neeson's cousin uses a mountain bike to get to work.
7. Of Malia and the commuter who rides a 10-speed bike, one has an anniversary on January 28 and the other has an anniversary on November 23.
8. Either the person whose anniversary is on March 6 or the person whose anniversary is on November 2 is Morgan Freeman's cousin.
9. Tom Cruise's cousin has an anniversary after the commuter who rides a skateboard.

Answer:
[
    {
        "anniversaries": "January 28",
        "first names": "Malia",
        "conveyances": "Mountain bike",
        "actors": "Liam Neeson"
    },
    {
        "anniversaries": "March 6",
        "first names": "Kayla",
        "conveyances": "Skateboard",
        "actors": "Morgan Freeman"
    },
    {
        "anniversaries": "November 2",
        "first names": "Bradley",
        "conveyances": "Segway",
        "actors": "Tom Cruise"
    },
    {
        "anniversaries": "November 23",
        "first names": "Asher",
        "conveyances": "10-speed bike",
        "actors": "Robert Duvall"
    }
]
\end{CustomVerbatim}
\end{tcolorbox}

\subsection{Cryptogram}
\begin{tcolorbox}[breakable, title=Cryptogram Example, colframe=blue!60]
\begin{CustomVerbatim}
Cryptogram:
VYQ'O CY HGYWQV THMFQC OKB LYGUV YLBT MYW H UFJFQC. OKB LYGUV YLBT MYW QYOKFQC. FO LHT KBGB RFGTO. - Mark Twain

Answer:
Don't go around saying the world owes you a living. The world owes you nothing. It was here first. - Mark Twain
\end{CustomVerbatim}
\end{tcolorbox}

\subsection{Sudoku}
\begin{tcolorbox}[breakable, title=Sudoku Example, colframe=blue!60]
\begin{CustomVerbatim}
Sudoku:
[[4, 0, 0, 0],
 [0, 3, 2, 0],
 [0, 4, 1, 0],
 [0, 0, 0, 2]]

Answer:
[[4, 2, 3, 1],
 [1, 3, 2, 4],
 [2, 4, 1, 3],
 [3, 1, 4, 2]]
\end{CustomVerbatim}
\end{tcolorbox}

\subsection{Drop Quote}
\begin{tcolorbox}[breakable, title=Drop Quote Example, colframe=blue!60]
\begin{CustomVerbatim}
Grid:
1  2  #  4  5  6  7  8  9   #   11 12 13 14 15
#  2  3  4  5  6  7  8  9  10  11  #  13 14 15
1  2  #  4  5   '  7  #  9  10  11 12  .    #   #

Column 1: I I
Column 2: I L T
Column 3: M
Column 4: A I P
Column 5: L O T
Column 6: S W
Column 7: A S S
Column 8: I Y
Column 9: B D S
Column 10: L O
Column 11: E N S
Column 12: E E
Column 13: E U
Column 14: M N
Column 15: S T

Answer:
It always seems impossible until it's done. - Nelson Mandela
\end{CustomVerbatim}
\end{tcolorbox}


\section{Prompt Templates}
\label{app_prompt}
We provide detailed prompt templates for each type of task in LR$^2$Bench, including task descriptions, examples, and solution guidelines.
\subsection{Crossword}
\begin{tcolorbox}[breakable, title=Prompt for Crossword]
\small
\textbf{[Task Description]} \\

A Crossword puzzle is a word game that consists of a grid, with clues given for words that fit into the grid both across (horizontally) and down (vertically). Your goal is to fill in the grid with words based on the clues provided. Here's a detailed explanation of how the game works: \\

1. Understand the Grid Layout \\
The grid is made up of numbers, hashtags ("\#"), and question marks ("?"). \\

Hashtag ("\#") acts as a separator between words. \\

The number represents the starting points of Across and Down words. \\

Question mark ("?") represents part of words but don't start a new word. \\

2. Read the Clues \\
Clues are provided for each word to be filled into the grid, split into two categories: Across clues (these are for words that go horizontally in the grid) and Down clues (these are for words that go vertically in the grid). \\

The number in brackets after the clue indicates the length of the word. \\

Clues are often short definitions, synonyms, or phrases related to the word. \\

Some clues may involve wordplay, anagrams, or puns, depending on the puzzle's difficulty and style. \\

3. Solve the Puzzle \\
Think of words that fit the clue and match the number of letters specified. \\

For example, if a clue says "Animal that barks (3)", you might guess "DOG" because it has 3 letters. When getting the answer, double-check that it fits the clue, both in meaning and the number of letters. If you make a mistake, just try again! \\

As you fill in answers, they will help you solve other clues, since words intersect and share letters in the grid. This cross-checking mechanism helps in verifying correct answers. \\

\textbf{[Example 1]}

\begin{CustomVerbatim}
Here is a Crossword puzzle:
GRID:
# 1 2 3 4
5 ? ? ? ?
6 ? ? ? ?
7 ? ? ? ?
# 8 ? ? #

CLUES:
Across:
1 "More ___ you'll ever know" (4)
5 Divine favor, in Christianity (5)
6 Sisters who might be sitters (5)
7 Portly-sounding beer (5)
8 Cool and stylish (3)

Down:
1 "The ___ will set you free. But not until it is finished with you": David Foster Wallace (5)
2 Capital of Vietnam (5)
3 Misbehave (5)
4 Prepare for a new baby, say (4)
5 Fuel for some stoves (3)
\end{CustomVerbatim}

\textbf{[Solution 1]}

\begin{CustomVerbatim}
Let's start solving the Crossword puzzle step by step.

Across Clues:

<1-Across>
1-Across: "More ___ you'll ever know" (4)

The clue is a reference to a popular phrase, "More than you'll ever know," suggesting the answer is "than".
Answer: THAN.
</1-Across>

<5-Across>
5-Across: Divine favor, in Christianity (5)

The clue refers to a concept of God's blessing or grace, and the answer is "grace".
Answer: GRACE.
</5-Across>

... ...

Here is the final grid after finishing all the clues:
<Answer>
{"1-Across": "THAN", "5-Across": "GRACE", "6-Across": "AUNTS", "7-Across": "STOUT", "8-Across": "HIP", "1-Down": "TRUTH", "2-Down": "HANOI", "3-Down": "ACTUP", "4-Down": "NEST", "5-Down": "GAS"}
</Answer>
\end{CustomVerbatim}

\textbf{[Example 2]} \\

<Example 2> \\

\textbf{[Solution 2]} \\

<Solution 2> \\

\textbf{[Solution Guideline]} \\

Please solve the Crossword Puzzle according to the provided rules. Please also follow the requests below to present your analysis and solutions: \\

1. Analyze each clue carefully to understand its meaning and potential word associations. Be open to the possibility of wordplay or puns that might lead to unexpected solutions. The filled-in words will help you infer the answers to the subsequent clues. \\

2. Provide the final answer for each clue. The final answer should be presented after "Answer:". \\

3. Please wrap all the analysis of each clue with <n-Across> and </n-Across> or <n-Down> and </n-Down> tags, where n is the clue number. For example, the Across clue 1 should be wrapped with <1-Across> and </1-Across> tags. The Down clue 1 should be wrapped with <1-Down> and </1-Down> tags, and so on. \\

4. After solving all the clues, please summarize all the answer words in following json format and warp them with <Answer> and </Answer> tags: \\
<Answer> \\
\{"1-Across": "ANSWER1", "2-Across": "ANSWER2", ... "1-Down": "ANSWER3", "2-Down": "ANSWER4", ...\} \\
</Answer> \\

5. Please generate your response without redundant and repeating content. \\

\textbf{[Question]} \\

<Question>
\end{tcolorbox}





\subsection{Acrostic}
\begin{tcolorbox}[breakable, title=Prompt for Acrostic]
\small
\textbf{[Task Description]} \\

An Acrostic puzzle is a word game that consists of two main parts: a grid and a set of clues. The objective is to fill in the grid with letters from the answers to the clues, forming a hidden quotation or sentence. Here's a detailed explanation of how the game works: \\

1. Understand the Grid Layout \\
The grid is made up of numbers, hashtags ("\#"), and punctuations. Hashtag ("\#") acts as space separator between words. \\

Each number corresponds to a specific letter in a word. The hidden quotation or sentence is formed by filling in the grid with the correct letters. \\

2. Read the Clues \\
Each clue is made up of a string of numbers and a clue text. \\

The clue text is usually a short definition, synonym or phrase related to the answer word or phrase. \\

The number of words and letters of the answer are given in brackets after the clue text. \\

The string of numbers provided indicates the specific position of each letter of the answer in the grid. \\

3. Solve the Puzzle \\
Think of answers that fit the clue and match the number of words and letters specified. \\

For example, if a clue says "Animal that barks (1 word, 3 letters)", you might guess "DOG" because it has 3 letters. When getting the answer, double-check that it fits the clue, both in meaning and the number of letters. If you make a mistake, just try again! \\

As you fill in the letters of answers, these letters can be part of the hidden word in the grid, thus providing hints for the left letters of this word. This mechanism can help both verify your answers and solve difficult clues. \\

For example, if the puzzle looks like this: "... \# 5 6 7 \# ..." and you have already got \{5: 'A', 7: 'D'\}, you can guess that the word is "AND" and get \{6: 'N'\} without its corresponding clue. Don't be afraid to skip a tough clue and come back to it later. \\

As you fill in the grid, you will reveal part of the hidden words. Use this information to help solve the remaining clues. \\

    
\textbf{[Example 1]}

\begin{CustomVerbatim}
Here is an Acrostic puzzle:
GRID:
1 2 3 4 5 # 6 7 8 9 10 # 11 12 # 13 14 15 # 16 17 18 19 20 # 21 22 23 24 25 26 27 28 29 , 30 31 32 # 33 34 35 # 36 37 38 # 39 40 41 42 43 # 44 45 46 47 # 48 49 50 # 51 52 53 54 55 56 57 # 58 59 60 61 # 62 63 64 65 66 67 68 69 70 # 71 72 73 74 # 75 # 76 77 78 79 # 80 81 82 # 83 84 85 # 86 87 88 89 90 91 92 .

CLUES:
A
58 86 92 3 60 49 57 46 70 40
Cosmetics magnate (3 words, 10 letters)
B
52 34 44 54 84 67 11 36 25 18 62
Nonphysical assets (1 word, 11 letters)
C
76 45 35 88 56 81 43 41 68 71
Religious schools of opinion (1 word, 10 letters)
D 
48 37 16 24 82 85 30 63 73
Cancer stick (1 word, 9 letters)
E
10 32 2 74 22 23 75 79 39 5 64
Establish a home base (1 word, 11 letters)
F
59 9 14 69 83 53 28
"____ is only a bench covered in velvet" (quote by Napoleon) (2 words, 7 letters)
G
55 26 38 31 20 21 87 91 6
___ up (became less serious) (1 word, 9 letters)
H
4 17 61 65 13 90 77 29
Rest (1 word, 8 letters)
I
33 80 51 15 66 27 7 12
Infractions (1 word, 8 letters)
J
1 19 50 72 8 89 47 78 42
Whisky-vermouth cocktail (1 word, 9 letters)
\end{CustomVerbatim}

\textbf{[Solution 1]}

\begin{CustomVerbatim}
Let's start solving the Acrostic puzzle step by step.

<A-CLUE>
A
58 86 92 3 60 49 57 46 70 40
Cosmetics magnate (3 words, 10 letters)

The clue "Cosmetics magnate" succinctly encapsulates Mary Kay Ash's pioneering role in establishing a direct sales empire in the beauty industry.
Answer: MARYKAYASH.
Mapping: {'58': 'M', '86': 'A', '92': 'R', '3': 'Y', '60': 'K', '49': 'A', '57': 'Y', '46': 'A', '70': 'S', '40': 'H'}
</A-CLUE>

<B-CLUE>
B
52 34 44 54 84 67 11 36 25 18 62
Nonphysical assets (1 word, 11 letters)

"Nonphysical assets" aptly refers to intangibles, highlighting assets like intellectual property and goodwill that lack physical form but possess significant business value.
Answer: INTANGIBLES.
Mapping: {'52': 'I', '34': 'N', '44': 'T', '54': 'A', '84': 'N', '67': 'G', '11': 'I', '36': 'B', '25': 'L', '18': 'E', '62': 'S'}
</B-CLUE>

... ...

Now let's gather all the mappings in order:
<Answer>
{'1': 'M', '2': 'A', '3': 'Y', '4': 'B', '5': 'E', '6': 'D', '7': 'E', '8': 'A', '9': 'T', '10': 'H', '11': 'I', '12': 'S', '13': 'T', '14': 'H', '15': 'E', '16': 'G', '17': 'R', '18': 'E', '19': 'A', '20': 'T', '21': 'E', '22': 'Q', '23': 'U', '24': 'A', '25': 'L', '26': 'I', '27': 'S', '28': 'E', '29': 'R', '30': 'T', '31': 'H', '32': 'E', '33': 'O', '34': 'N', '35': 'E', '36': 'B', '37': 'I', '38': 'G', '39': 'T', '40': 'H', '41': 'I', '42': 'N', '43': 'G', '44': 'T', '45': 'H', '46': 'A', '47': 'T', '48': 'C', '49': 'A', '50': 'N', '51': 'F', '52': 'I', '53': 'N', '54': 'A', '55': 'L', '56': 'L', '57': 'Y', '58': 'M', '59': 'A', '60': 'K', '61': 'E', '62': 'S', '63': 'T', '64': 'R', '65': 'A', '66': 'N', '67': 'G', '68': 'E', '69': 'R', '70': 'S', '71': 'S', '72': 'H', '73': 'E', '74': 'D', '75': 'A', '76': 'T', '77': 'E', '78': 'A', '79': 'R', '80': 'F', '81': 'O', '82': 'R', '83': 'O', '84': 'N', '85': 'E', '86': 'A', '87': 'N', '88': 'O', '89': 'T', '90': 'H', '91': 'E', '92': 'R'}
</Answer>
\end{CustomVerbatim}


\textbf{[Example 2]} \\

<Example 2> \\

\textbf{[Solution 2]} \\

<Solution 2> \\

\textbf{[Solution Guideline]} \\

Please solve the Acrostic Puzzle according to the provided rules. Please also follow the requests below to present your analysis and solutions: \\

1. Analyze the clue carefully to understand its meaning and potential word associations. Be open to the possibility of wordplay or puns that might lead to unexpected solutions. The filled-in words will help you infer the answers to the subsequent clues. \\

2. Provide the final answer for each clue. The final answer should be presented after "Answer:". \\

3. Create a python dictionary mapping in a single line that links the number positions in the clues to the corresponding letters in the final answer. Note that the blank spaces and punctuation should be omitted in the mapping. The python dictionary should be presented after "Mapping:". \\

4. Please wrap all the analysis of each clue with <n-CLUE> and </n-CLUE> tags, where n is the label of the clue. For example, the first clue should be wrapped with <A-CLUE> and </A-CLUE> tags. The second clue should be wrapped with <B-CLUE> and </B-CLUE> tags, and so on. \\

5. After solving all the clues, please gather all the mappings and place them in order between the <Answer> and </Answer> tags. The gathered mapping should look like \{'1': 'LETTER1', '2': 'LETTER2', ...\}. \\

6. Please generate your response without redundant and repeating content. \\

\textbf{[Question]} \\

<Question>
\end{tcolorbox}

\subsection{Logic Puzzle}
\begin{tcolorbox}[breakable, title=Prompt for Logic Puzzle]
\small
\textbf{[Task Description]} \\

Logic puzzles require the solver to deduce the relationships between different people, places and things based on a limited number of clues given in the puzzle. Remember: every item belongs to one and only one person, no item will ever be shared. Using only the clues provided and simple deductive logic and reasoning. \\

\textbf{[Example 1]}

\begin{CustomVerbatim}
Here is a logic puzzle:
{
    "variables": {
        "names": [
            "Fiona",
            "Tobias",
            "Leanne"
        ],
        "main dishes": [
            "Beef Wellington",
            "Gado Gado",
            "Shrimp Ceviche"
        ],
        "desserts": [
            "Creme Brulee",
            "Raspberry Tiramisu",
            "French Apple Cake"
        ]
    },
    "clues": {
        "1": "The three people are Tobias, the one who brought the Gado Gado, and the one who brought the Raspberry Tiramisu.",
        "2": "Fiona did not bring the Beef Wellington.",
        "3": "The Shrimp Ceviche was brought by either Tobias or the person who brought the Creme Brulee.",
    }
}

Please complete the following blank based on your final solution:
[
    {
        "names": "Fiona",
        "main dishes": "____",
        "desserts": "____",
    },
    {
        "names": "Tobias",
        "main dishes": "____",
        "desserts": "____",
    },
    {
        "names": "Leanne",
        "main dishes": "____",
        "desserts": "____",
    }
]
\end{CustomVerbatim}

\textbf{[Solution 1]}

\begin{CustomVerbatim}
Let's start solving the logic puzzle step by step.

Participants and Dishes
Names: Fiona, Tobias, Leanne
Main Dishes: Beef Wellington, Gado Gado, Shrimp Ceviche
Desserts: Crème Brûlée, Raspberry Tiramisu, French Apple Cake


Interpreting Clue 1:
    Roles:
        Tobias: Did not bring Gado Gado or Raspberry Tiramisu.
        Gado Gado Bringer: Not Tobias.
        Raspberry Tiramisu Bringer: Not Tobias.

    Conclusion:
        Fiona and Leanne must be the ones who brought Gado Gado and Raspberry Tiramisu, respectively (or vice versa).

... ...

Final Answer in JSON Format:
<Answer>
[
    {
        "names": "Fiona",
        "main dishes": "Gado Gado",
        "desserts": "Creme Brulee",
    },
    {
        "names": "Tobias",
        "main dishes": "Shrimp Ceviche",
        "desserts": "French Apple Cake",
    },
    {
        "names": "Leanne",
        "main dishes": "Beef Wellington",
        "desserts": "Raspberry Tiramisu",
    }
]
</Answer>
\end{CustomVerbatim}

\textbf{[Example 2]} \\

<Example 2> \\

\textbf{[Solution 2]} \\

<Solution 2> \\

\textbf{[Solution Guideline]} \\

Please solve the Logic Puzzle according to the provided rules. Please also follow the requests below to present your analysis and solutions: \\

1. Read and understand each clue in the context of the puzzle. Apply each clue one by one to deduce the correct arrangement of different variables. \\

2. Use logical reasoning to figure out the relationships between the variables based on the clues provided. \\

3. After solving the puzzle, present your final solution using JSON Format and wrap it with the <Answer> and </Answer> tags. For example:

\begin{CustomVerbatim}
<Answer>
[
    {
        "Variable1": "Value1",
        "Variable2": "Value2",
        ...
    },
    {
        "Variable1": "Value3",
        "Variable2": "Value4",
        ...
    },
    ...
]
</Answer>
\end{CustomVerbatim}

4. Please generate your response without redundant and repeating content. \\

\textbf{[Question]} \\

<Question>
\end{tcolorbox}



\subsection{Cryptogram}
\begin{tcolorbox}[breakable, title=Prompt for Cryptogram]
\small
\textbf{[Task Description]} \\

A cryptogram is a type of puzzle that consists of a short piece of encrypted text. You will decode a cryptogram that uses a simple substitution cipher, where each letter in the message has been substituted with another letter. The goal is to decode the hidden quote along with its author's name. You may use the author's name to help reduce the difficulty of the decoding process by focusing on common words, letter patterns, and stylistic cues from the author. \\

\textbf{[Example 1]} \\

Think step-by-step to decipher the cryptogram: \\
VOBB XSDO QK IOFFOU FGHD VOBB KHQX. \\
— Benjamin Franklin \\

\textbf{[Solution 1]} \\

Let's start solving the cryptogram puzzle step by step. \\

Step 1: Use the Author's Name as a Clue Benjamin Franklin is known for his concise and wise sayings. Recognizing his style helps anticipate common structures and themes, such as practical advice or moral lessons. \\

Step 2: Analyze the Cryptogram The cryptogram is: \\
VOBB XSDO QK IOFFOU FGHD VOBB KHQX. -Benjamin Franklin \\

Notable features: \\
"VOBB" appears twice, suggesting a common word with double letters. \\
The structure of the sentence hints at a proverb or a well-known quote. \\

... ... \\

\begin{CustomVerbatim}
Mapping:
<Mapping>
{ "V": "W", "O": "E", "B": "L", "X": "D", "S": "O", "D": "N", "Q": "I", "K": "S", "I": "B", "F": "T", "U": "R", "G": "H", "H": "A"}
</Mapping>

Answer:
<Answer>
Well done is better than well said.
-Benjamin Franklin
</Answer>
\end{CustomVerbatim}

\textbf{[Example 2]} \\

<Example 2> \\

\textbf{[Solution 2]} \\

<Solution 2> \\

\textbf{[Solution Guideline]} \\

Please follow these steps to solve the Cryptogram: \\

1. Use the author's name as a clue: Knowing the author's name can help you predict common words or letter combinations typical for this author. For instance, if the author is "Shakespeare," you might anticipate archaic or common Shakespearean phrases (like "thou," "thee," etc.). \\

2. Analyze the cryptogram: Look at the frequency of letters and common letter patterns, such as double letters or common suffixes and prefixes. Focus on the parts of the cryptogram that seem to match the author's typical writing style or famous phrases. \\

3. Map common words: If you recognize a word in the cryptogram that matches the author's typical vocabulary, substitute letters based on that. \\

4. Make educated guesses: Use common English words (such as "the," "and," "of," etc.) and letter pairs (like "th," "he," "in", etc.) to identify possible substitutions. If one assumption doesn't work, try another. \\

5. Verification: After generating the decoded message, check if the quote and the author's name make logical sense. If needed, revisit the assumptions and adjust the letter mappings. \\

6. Please provide step by step analysis and create a python dictionary mapping of the fully substitutioin wrapping it between <Mapping> and </Mapping> tag. \\

7. Please provide the final decoded quote and wrap it between <Answer> and </Answer> tag. \\

8. Please generate your response without redundant and repeating content. \\

\textbf{[Question]} \\

<Question>
\end{tcolorbox}





\subsection{Sudoku}
\begin{tcolorbox}[breakable, title=Prompt for Sudoku]
\small
\textbf{[Task Description]} \\

Sudoku is a logic-based, combinatorial number-placement puzzle. The puzzle consists of an $n^2 \times n^2$ grid, partially filled with numbers from 1 to $n^2$ and 0 (empty cells). The objective is to fill the grid so that each row, each column, and each $n \times n$ subgrid must contain every number from 1 to $n^2$ exactly once. \\

\textbf{[Example 1]}

\begin{CustomVerbatim}
Here is a 4 x 4 sudoku puzzle:
[[2, 0, 0, 0],
 [0, 0, 3, 0],
 [0, 4, 0, 0],
 [0, 0, 0, 1]]
\end{CustomVerbatim}


\textbf{[Solution 1]}

\begin{CustomVerbatim}
Let's start solving the sudoku puzzle step by step.

Step1: Analyze Empty Cell in Row 1
Cell (1,2):
    Eliminated Numbers: 2 (already in row), 4 (already in column)
    Possible Numbers: 1, 3
Cell (1,3):
    Eliminated Numbers: 2 (already in row), 3 (already in column)
    Possible Numbers: 1, 4
Cell (1,4):
    Eliminated Numbers: 2 (already in row), 1 (already in column), 3 (already in subgrid)
    Only possible number: 4

... ...

Step 5: Final Answer
<Answer>
[[2, 3, 1, 4],
 [4, 1, 3, 2],
 [1, 4, 2, 3],
 [3, 2, 4, 1]]
</Answer>
\end{CustomVerbatim}

\textbf{[Example 2]} \\

<Example 2> \\

\textbf{[Solution 2]} \\

<Solution 2> \\

\textbf{[Solution Guideline]} \\

Please solve the Sudoku Puzzle according to the provided rules. Please also follow the requests below to present your analysis and solutions: \\

1. For each empty cell, try to deduce which numbers are allowed based on the existing numbers in its row, column, and subgrid. \\

2. If you reach a point where no obvious choices are available, you may need to backtrack. Try filling in a number and see if it leads to a valid solution. If it causes a contradiction, backtrack to a previous decision and try a different path. \\

3. Please provide step by step analysis and present the final answer as the same json format of the input grid. Wrapping the final answer with <Answer> and </Answer> tags. \\

4. Please generate your response without redundant and repeating content. \\

\textbf{[Question]} \\

<Question>
\end{tcolorbox}





\subsection{Drop Quote}
\begin{tcolorbox}[breakable, title=Prompt for Drop Quote]
\small
\textbf{[Task Description]} \\

A Drop Quote Puzzle consists of two parts: a grid and a set of given letters that "drop" into specific columns of the grid. The goal is to fill in the grid with a meaningful quote or statement. \\

Key Features: \\
1. The grid contains numbers, "\#" symbols (representing spaces between words), and already placed punctuation. \\

2. Words may break across lines, starting at the end of one line and continuing to the beginning of the next. Words end only at "\#" blocks or the end of the puzzle. \\

3. The letters are organized by column number and are given in alphabetical order. These letters need to be placed into the grid. \\

4. Each letter drops into a specific column, and your task is to identify the correct placement for each letter based on the grid structure. Do not simply fill the columns in the order they are listed. Instead, consider where each letter fits contextually within the puzzle to form words that make sense in the final quote or statement. \\

Rules: \\
1. The "\#" symbols indicate where spaces between words should be. \\

2. Column labels (e.g., Column 1: O T) indicate the letters that will drop into the grid for Column 1, listed in alphabetical order. For example, in Column 1, the letters "O" and "T" must be placed in the appropriate grid positions. \\

3. Fill the grid with the letters in such a way that they form a meaningful statement or quotation. \\

\textbf{[Example 1]}

\begin{CustomVerbatim}
Here is a drop quote puzzle:
[['1', '2', '3', '4', '#', '6', '7'],
 ['#', '2', '3', '4', '5', '6', '.']]

Column 1: T
Column 2: I M
Column 3: M O
Column 4: E N
Column 5: E
Column 6: I Y
Column 7: S
\end{CustomVerbatim}


\textbf{[Solution 1]}

\begin{CustomVerbatim}
Let's start solving the sudoku puzzle step by step.

We first focus on Columns with single letters:
    "T" for Row 1, Column 1
    "E" for Row 2, Column 5
    "S" for Row 1, Column 7

Then, we check the word ['6', '7'] in Row 1:
    The candidates for Column 6 are "I" and "Y", and for Column 7, it is "S". The only word that fits is "IS". So, place "I" in Row 1, Column 6 and "Y" in Row 2, Column 6.

Let's check the current grid:
[['T', '2', '3', '4', '#', 'I', 'S'],
 ['#', '2', '3', '4', 'E', 'Y', '.']]

... ...

Here is the final answer gird:
<Answer>
[['T', 'I', 'M', 'E', '#', 'I', 'S'],
 ['#', 'M', 'O', 'N', 'E', 'Y', '.']]
</Answer>
\end{CustomVerbatim}

\textbf{[Example 2]} \\

<Example 2> \\

\textbf{[Solution 2]} \\

<Solution 2> \\

\textbf{[Solution Guideline]} \\

Please solve the Drop Quote Puzzle according to the provided rules. Please also follow the requests below to present your analysis and solutions: \\

1. Isolated boxes are usually, but not always, the words A or I. Similarly, when you have a 2-letter word, see what words you can make and then see what letters that leaves you over with for other lines. The most common 2-letter words are: OF, TO, IN, IT, IS, BE, AS, AT, SO, WE, HE, BY, OR, ON, DO, IF, ME, MY, UP, AN, GO, NO, US, AM. \\

2. The letters in a column with fewer letters can be placed quickly by noticing which letters must be consonants and vowels, or by eliminating the possibility of a letter appearing in a certain spot. \\

3. If you have an uncommon letter such as a J, K, Q, X, or Z, think about what letters might go before or after it. Similarly, look for common consonant pairs such as TH, ND, NT, ST, as well as doubled letters. \\

4. Be alert for common prefixes such as DE-, MIS-, RE-, and UN-, as well as common suffixes such as -ABLE, -ED, -ING, -LY, -NESS, and -TION. \\

5. Every time you fill in a letter, it means that you've eliminated some letters that can go elsewhere. Frequently, this means that you can immediately fill in some additional letters. \\

6. Please provide step by step analysis and present the final answer as the same json format of the input grid. Wrapping the final answer in <Answer> and </Answer> tags. \\

7. Please generate your response without redundant and repeating content. \\

\textbf{[Question]} \\

<Question>
\end{tcolorbox}


\section{Models}
\label{app_models}
As shown in Table \ref{models}, we list the key details for eleven open-source models and five closed-source models in our evaluation.

\section{Task Difficulty}
\label{app_difficulty}
Table \ref{space} analyzes the difficulty of each type of task through the size of solution space. We approximately calculate the solution space by multiplying the number of elements to be filled (e.g., grid size for Crossword) by the number of possible answers for each element (e.g., 26 letters for Crossword). Detailed performance across varying difficulty levels for each task is presented in Tables \ref{level_crossword}, \ref{level_acrostic_cryptogram}, \ref{level_logic}, \ref{level_sudoku}, and \ref{level_drop}. Our findings indicate that the CR score does not significantly decrease with increasing task difficulty. This suggests that the long-chain generation capability is likely an inherent property of these models, rather than being strongly task-dependent. However, as the solution space expands with increasing difficulty, we observe a consistent performance decline across all models. In the most challenging settings, even the top-performing model achieves an EM score of less than 10\%, highlighting the difficulty of our benchmark for evaluating LLMs' reflective reasoning capabilities.
\begin{table}[t]
\centering
\resizebox{1.0\columnwidth}{!}{\begin{tabular}{llc}
\toprule
\textbf{Task} &  
\textbf{Difficulty} &

\begin{tabular}[c]{@{}c@{}}\hspace{0.5em} \textbf{Solution Space} \hspace{0.5em} \\ \textbf{Upper Bound} \end{tabular} \\
\midrule
\multirow{3}{*}{Crossword} & $5\times5$ & $26\times5\times5$ \\
& $10\times10$ & $26\times10\times10$ \\
& $15\times15$ & $26\times15\times15$ \\
\midrule
\multirow{2}{*}{Acrostic} & Easy & $26\times110$ \\
& Hard & $26\times214$ \\
\midrule
\multirow{4}{*}{Logic Puzzle} & $4\times4$ & ${4!}^3$ \\
& $4\times5$ & ${4!}^4$ \\
& $4\times6$ & ${4!}^5$ \\
& $4\times7$ & ${4!}^6$ \\
\midrule
\multirow{2}{*}{Cryptogram} & Easy & $26!$ \\
& Hard & $26!$ \\
\midrule
\multirow{4}{*}{Sudoku} & $4\times4$, Easy & $4^{16}$ \\
& $4\times4$, Hard & $4^{16}$ \\
& $9\times9$, Easy & $9^{81}$ \\
& $9\times9$, Hard & $9^{81}$ \\
\midrule
\multirow{2}{*}{Drop Quote} & Easy & $15\times3!$ \\
& Hard & $20\times5!$ \\
\bottomrule
\end{tabular}}
\caption{The solution space upper bound of each type of task.}
\label{space}
\end{table}


% \begin{table}[t]
%     \centering
%     \resizebox{1.0\columnwidth}{!}{
%     \begin{tabular}{lcc}
%         \toprule
%         \textbf{Task} & \textbf{Difficulty} & \textbf{Solution Space Upper Bound}\\
%         \midrule
%         \multirow{3}{*}{Crossword} & $5\times5$ & $26\times5\times5$ \\
%         & $10\times10$ & $26\times10\times10$ \\
%         & $15\times15$ & $26\times15\times15$ \\
%         \midrule
%         \multirow{2}{*}{Acrostic} & Easy & $26\times110$ \\
%         & Hard & $26\times214$ \\
%         \midrule
%         \multirow{4}{*}{Logic Puzzle} & $4\times4$ & ${4!}^3$ \\
%         & $4\times5$ & ${4!}^4$ \\
%         & $4\times6$ & ${4!}^5$ \\
%         & $4\times7$ & ${4!}^6$ \\
%         \midrule
%         \multirow{2}{*}{Cryptogram} & Easy & $26!$ \\
%         & Hard & $26!$ \\
%         \midrule
%         \multirow{4}{*}{Sudoku} & $4\times4$, Easy & ${4!}^3$ \\
%         & $4\times4$, Hard & ${4!}^4$ \\
%         & $9\times9$, Easy & ${4!}^5$ \\
%         & $9\times9$, Hard & ${4!}^6$ \\
%         \midrule
        

%     \end{tabular}}    
%     \caption{.}
%     \label{tab:main_exp_sum}
% \end{table}





\section{Response Cases}
\label{app_case}

\subsection{Lack of Reflective Reasoning}
\label{lack}
The following example shows that GPT-4o fails to solve the logic problem due to the lack of reflective reasoning capabilities. The model only performs a single turn of deduction of all the clues and outputs the final answers with insufficient evidence.

\begin{tcolorbox}[breakable, colback=blue!8, title=Example 1: Response of GPT-4o to Logic Puzzle]
\small
\begin{Verbatim}[formatcom=\normalfont, breaklines=true, breaksymbolleft={}, breaksymbolright={}, commandchars=\\\{\}]
To solve the logic puzzle, let's analyze the clues and deduce the relationships between the variables.

### Step-by-Step Analysis:

1. **Clue 1**: The commuter who rides a 10-speed bike is not Bradley.
   - Bradley does not ride a 10-speed bike.

2. **Clue 2**: Kayla isn't related to Robert Duvall and doesn't ride a segway.
   - Kayla's actor is not Robert Duvall, and her conveyance is not a segway.

3. **Clue 3**: Robert Duvall's cousin has an anniversary after the commuter who rides a segway.
   - The segway rider's anniversary is before Robert Duvall's cousin.

4. **Clue 4**: Morgan Freeman's cousin is not Asher and doesn't ride a segway.
   - Morgan Freeman's cousin is not Asher and does not ride a segway.

5. **Clue 5**: Liam Neeson's cousin is Malia.
   - Malia's actor is Liam Neeson.

6. **Clue 6**: Liam Neeson's cousin uses a mountain bike to get to work.
   - Malia rides a mountain bike.

7. **Clue 7**: Of Malia and the commuter who rides a 10-speed bike, one has an anniversary on January 28 and the other has an anniversary on November 23.
   - Malia or the 10-speed bike rider has an anniversary on January 28 or November 23.

8. **Clue 8**: Either the person whose anniversary is on March 6 or the person whose anniversary is on November 2 is Morgan Freeman's cousin.
   - Morgan Freeman's cousin has an anniversary on March 6 or November 2.

9. **Clue 9**: Tom Cruise's cousin has an anniversary after the commuter who rides a skateboard.
   - The skateboard rider's anniversary is before Tom Cruise's cousin.

### Deductions:

- From Clue 5 and Clue 6, Malia rides a mountain bike and is Liam Neeson's cousin.

- From Clue 7, Malia's anniversary is either January 28 or November 23.

- From Clue 8, Morgan Freeman's cousin has an anniversary on March 6 or November 2.

- From Clue 9, the skateboard rider's anniversary is before Tom Cruise's cousin.

### Assignments:

- **Malia**: Since Malia rides a mountain bike and is Liam Neeson's cousin, she cannot be Morgan Freeman's cousin. Therefore, Malia's anniversary is not March 6 or November 2. Thus, Malia's anniversary is January 28 or November 23.

- **Kayla**: Kayla is not related to Robert Duvall and doesn't ride a segway. She could be Morgan Freeman's cousin with an anniversary on March 6 or November 2.

- **Bradley**: Bradley does not ride a 10-speed bike. He could be the skateboard rider.

- **Asher**: Asher is not Morgan Freeman's cousin, so he could be Tom Cruise's cousin.

### Final Solution:

Based on the deductions and assignments, the final solution is:

<Answer>
[
    \{
        "anniversaries": "January 28",
        "first names": "Malia",
        "conveyances": "mountain bike",
        "actors": "Liam Neeson"
    \},
    \{
        "anniversaries": "March 6",
        "first names": "Kayla",
        \textcolor{red}{"conveyances": "10-speed bike", \ding{55}}
        "actors": "Morgan Freeman"
    \},
    \{
        "anniversaries": "November 2",
        "first names": "Bradley",
        \textcolor{red}{"conveyances": "skateboard", \ding{55}}
        \textcolor{red}{"actors": "Robert Duvall \ding{55}}
    \},
    \{
        "anniversaries": "November 23",
        "first names": "Asher",
        \textcolor{red}{"conveyances": "segway", \ding{55}}
        \textcolor{red}{"actors": "Tom Cruise" \ding{55}}
    \}
]
</Answer>
\normalsize \textcolor{red}{\textit{\textbf{
- Only single deduction
- Insufficient evidence
- Lack of iterative deduction to ensure consistency with all constraints}}}
\end{Verbatim}
\end{tcolorbox}

\subsection{Stuck in Contradictions}
\label{stuck}
The following example shows that Llama-3.1-70B falls into repetition of the same sentences when discovering contradictions. Instead of backtracking and revising its assumptions when a contradiction is found (e.g., Cassidy cannot finish at both 3 min 2 sec and 3 min 8 sec), the model appears to get stuck in a loop, repeatedly stating the conflicting information without making progress.
\begin{tcolorbox}[breakable, colback=blue!8, title=Example 2: Response of Llama-3.1-70B to Logic Puzzle]
\small
\begin{Verbatim}[formatcom=\normalfont, breaklines=true, breaksymbolleft={}, breaksymbolright={}, commandchars=\\\{\}]
Let's start solving the logic puzzle step by step.

Participants and Attributes
Times: 2 min, 59 sec, 3 min, 2 sec, 3 min, 8 sec, 3 min, 14 sec
First Names: Cassidy, Eva, Grace, Savannah
Breads: challah, flatbread, focaccia, matzo
Fruits: currants, kiwis, oranges, pineapples

Applying Clue 1:
    Either the person who bought the matzo or Cassidy finished at 2 min, 59 sec and the other grows pineapples.
    Clue 8 states that Savannah finished at 2 min, 59 sec.
    Therefore, Cassidy grows pineapples and did not buy the matzo.

Applying Clue 8:
    Savannah finished at 2 min, 59 sec.
    Clue 6 states that Savannah finished before the person who bought the focaccia.
    Therefore, the person who bought the focaccia finished at 3 min, 2 sec, 3 min, 8 sec, or 3 min, 14 sec.

Applying Clue 4:
    The person who grows currants bought the focaccia.
    Clue 2 states that either the runner who finished at 3 min, 2 sec or 3 min, 8 sec grows currants.
    Therefore, the person who bought the focaccia finished at 3 min, 2 sec or 3 min, 8 sec.

Applying Clue 9:
    The person who grows currants finished before Grace.
    Since the person who bought the focaccia finished at 3 min, 2 sec or 3 min, 8 sec, Grace finished at 3 min, 14 sec.

Applying Clue 5:
    The runner who finished at 3 min, 14 sec did not buy the challah.
    Therefore, Grace bought the matzo.

Applying Clue 3:
    Cassidy did not finish at 3 min, 2 sec and did not buy the flatbread.
    Since Cassidy did not buy the matzo (Clue 1) and did not buy the flatbread, Cassidy bought the challah.

Applying Clue 7:
    The person who grows oranges is not Grace.
    Since Cassidy grows pineapples, the person who grows oranges is either Eva or Savannah.

Assigning Remaining Attributes:
    Eva or Savannah grows oranges.
    Since Savannah finished at 2 min, 59 sec and the person who bought the focaccia finished at 3 min, 2 sec or 3 min, 8 sec, Eva grows oranges and bought the flatbread.
    Savannah grows kiwis and bought the focaccia is incorrect because the person who bought the focaccia finished at 3 min, 2 sec or 3 min, 8 sec.
    Therefore, Savannah bought the challah is incorrect, and Eva bought the flatbread and grows oranges.
    The person who bought the focaccia finished at 3 min, 2 sec or 3 min, 8 sec.
    Since the person who grows currants finished before Grace, the person who bought the focaccia finished at 3 min, 2 sec.
    Therefore, the person who finished at 3 min, 8 sec grows pineapples is incorrect, and Cassidy finished at 3 min, 8 sec is incorrect.
    \textcolor{red}{Cassidy finished at 3 min, 14 sec is incorrect because Grace finished at 3 min, 14 sec.}
    \textcolor{red}{Therefore, Cassidy finished at 3 min, 2 sec is incorrect, and Cassidy finished at 3 min, 8 sec is incorrect.}
    \textcolor{red}{Cassidy finished at 3 min, 14 sec is incorrect because Grace finished at 3 min, 14 sec.}
    \textcolor{red}{Therefore, Cassidy finished at 3 min, 2 sec is incorrect, and Cassidy finished at 3 min, 8 sec is incorrect.}
    \textcolor{red}{... ...}
    \normalsize \textcolor{red}{\textit{\textbf{
    - Looping within the same sentence}}}

\end{Verbatim}
\end{tcolorbox}

The following example indicates that QwQ-32B struggles with contraction involving multiple interconnected variables. The primary issue with the reasoning is its disorganized and cyclical behavior. The model jumps between different deductions and assumptions without systematically eliminating possibilities or locking in firm conclusions.

\begin{tcolorbox}[breakable, colback=blue!8, title=Example 3: Response of QwQ-32B to Logic Puzzle]
\small
\begin{Verbatim}[formatcom=\normalfont, breaklines=true, breaksymbolleft={}, breaksymbolright={}, commandchars=\\\{\}]
Let's tackle this logic puzzle step by step. It's a bit complex with multiple variables, but I'll try to break it down systematically.

First, let's list out all the variables we have:

... ...

Now, let's look at the anniversaries.

... ...

\textcolor{red}{So, if Bradley has Mar 6}, Tom Cruise's cousin must have Nov 2.

\textcolor{red}{If Bradley has Nov 2}, Tom Cruise's cousin must have a later date, but Nov 23 is after Nov 2, but Nov 23 is already assigned to either Malia or Kayla.

Wait, anniversaries are Jan 28, Mar 6, Nov 2, Nov 23.

... ...

\textcolor{red}{So, if Bradley has Mar 6}, Tom Cruise's cousin has Nov 2.

\textcolor{red}{If Bradley has Nov 2}, Tom Cruise's cousin has Nov 23.

But Nov 23 is assigned to either Malia or Kayla.

... ...

So, \textcolor{red}{if Bradley has Nov 2}, Tom Cruise's cousin must have Nov 23, but Nov 23 is assigned to either Malia or Kayla.

Wait, that can't be, because Tom Cruise's cousin must be someone else.

... ...

\textcolor{red}{If Bradley has Mar 6}, then Tom Cruise's cousin has Nov 2.

... ...

\textcolor{red}{But if Bradley has Nov 2}, Tom Cruise's cousin must have Nov 23, which is assigned to Malia or Kayla.

This seems conflicting.

Wait, perhaps I need to consider the possible combinations.

Let's consider two scenarios:

\textcolor{red}{**Scenario 1: Bradley has Mar 6**}

... ...

Therefore, Scenario 1 is invalid.

\textcolor{red}{Let's consider **Scenario 2: Bradley has Nov 2**}

... ...

Therefore, this assignment is also invalid.

... ...

Given the complexity, I'll have to conclude with the best possible answer based on the steps I've taken.

... ...
\normalsize \textcolor{red}{\textit{\textbf{
- Stuck in a reasoning cycle
- Jumping between different deductions and assumptions}}}

\end{Verbatim}
\end{tcolorbox}


\subsection{Give-up Moment}
\label{giveup}
The following examples show the "Give-up Moment" in the responses of QwQ-32B and DeepSeek-R1 to the Cryptogram task. Both models, faced with the complexity of the problem and the constraints of inference time, reach a point where they give up further decryption attempts.


\begin{tcolorbox}[breakable, colback=blue!8, title=Example 4: Response of QwQ-32B to Cryptogram]
\small
\begin{Verbatim}[formatcom=\normalfont, breaklines=true, breaksymbolleft={}, breaksymbolright={}, commandchars=\\\{\}]
I'm going to try to solve this cryptogram step by step. The cryptogram is:

HNG UGEJ LROUKG RT EJXRLH EJJ ROU KUCXGL, CT HNG HUROAJG CL HEWGF HR HUEKG HNGX HR E KRXXRF RUCICF, YCJJ AG TROFQ HR AG CF CQJGFGLL.

— Walter Gaston Shotwell

First, I need to understand that this is a substitution cipher, where each letter in the original text is replaced by another letter. 

My goal is to figure out which letter corresponds to which in the original text.

... ...

But that might not be accurate.

Alternatively, "KRXXRF" could be "SELVES", so X = L, R = V, F = E.

But again, just a guess.

This is getting too speculative.

\textcolor{red}{Given the time constraints, I think I'll stop here and try to provide the best possible mapping and decryption based on my attempts.}

<Mapping>
\{"H": "T", "N": "H", "G": "E", "U": "O", "J": "A", "L": "R", "R": "B", "E": "I", "T": "S", "O": "C", "K": "D", "C": "M", "X": "P", "F": "U", "Q": "W", "A": "F", "W": "G", "S": "Y", "Y": "K", "I": "Z"\}
</Mapping>

<Answer>
THE OIA ARBORE BIS IAUPRH IAA COU MPORHE, CY THE THUOAFM CI HIWGF HB HUEKG THNG HB I KRXPPU MUCICF, YAJJ AF TROFQ HB AF CF CQJGFGLL. — Walter Gaston Shotwell
</Answer>
\normalsize \textcolor{red}{\textit{\textbf{
- Occurrence of "Give-up Moment"}}}

\end{Verbatim}
\end{tcolorbox}


\begin{tcolorbox}[breakable, colback=blue!8, title=Example 5: Response of DeepSeek-R1 to Cryptogram]
\small
\begin{Verbatim}[formatcom=\normalfont, breaklines=true, breaksymbolleft={}, breaksymbolright={}, commandchars=\\\{\}]
<think>
Okay, let's tackle this cryptogram. The author is Charles Baudelaire, a French poet known for his symbolic and often darkly romantic style. That might not immediately help with common phrases, but maybe some themes like beauty, art, or suffering. Let's look at the cryptogram:

AM AR YIPYIMMSQWI MNSM, SLDBP MNI YAPNMR DV LSB, MNI YAPNM DV HDBMYSFAHMABP DBIRIWV NSR QIIB VDYPDMMIB. — Charles Baudelaire

First, I'll note the structure. The first part ends with a comma, then another phrase, and a longer sentence. Let's break it down into parts.

... ...

But "corruption" is 10 letters, YIPYIMMSQWI is 11 letters. So maybe with a comma: "Art is the corruption, the soul..."

But this is not helpful.

\textcolor{red}{Given the time I've spent and the need to provide an answer, I'll proceed to hypothesize the final answer based on common Baudelaire themes and possible mappings.}
</think>

<Mapping>
{'A': 'T', 'M': 'H', 'R': 'E', 'Y': 'B', 'I': 'U', 'P': 'A', 'S': 'L', 'Q': 'I', 'W': 'G', 'N': 'S', 'D': 'O', 'B': 'R', 'V': 'M', 'L': 'F', 'F': 'Y', 'H': 'P', 'G': 'C', 'C': 'D', 'K': 'K', 'X': 'X', 'Z': 'Z'}
</Mapping>

<Answer>
The beautiful is the shadow of the true.
— Charles Baudelaire
</Answer>
\normalsize \textcolor{red}{\textit{\textbf{
- Occurrence of "Give-up Moment"}}}

\end{Verbatim}
\end{tcolorbox}



% \setlength{\tabcolsep}{4pt}
\begin{table*}[!h]
    \centering
    {\resizebox{0.8\textwidth}{!}{\begin{tabular}{lcrl}
        \toprule
        \textbf{Model} & \textbf{Architecture} & \textbf{Size} & \textbf{Hugginggface/API} \\
        \midrule
        \rowcolor{gray!15} \multicolumn{4}{l}{\textit{\textbf{Open-source LLMs}}}\\
        \midrule 
        Llama-3.1-8B-Instruct \citep{dubey2024llama} & Dense & 8B & meta-llama/Llama-3.1-8B-Instruct \\
        Llama-3.1-70B-Instruct \citep{dubey2024llama} & Dense & 70B & meta-llama/Llama-3.1-70B-Instruct \\
        Llama-3.3-70B-Instruct \citep{dubey2024llama} & Dense & 70B & meta-llama/Llama-3.3-70B-Instruct \\
        Mistral-7B-Instruct-v0.3 \citep{jiang2023mistral} & Dense & 7B & mistralai/Mistral-7B-Instruct-v0.3 \\
        Mistral-Small-Instruct-2409 \citep{jiang2023mistral} & Dense & 22B & mistralai/Mistral-Small-Instruct-2409 \\
        Mistral-Large-Instruct-2411 \citep{jiang2023mistral} & Dense & 123B & mistralai/Mistral-Large-Instruct-2411 \\
        Qwen2.5-7B-Instruct \citep{yang2024qwen2} & Dense & 7B & Qwen/Qwen2.5-7B-Instruct \\
        Qwen2.5-32B-Instruct \citep{yang2024qwen2} & Dense & 32B & Qwen/Qwen2.5-32B-Instruct \\
        Qwen2.5-72B-Instruct \citep{yang2024qwen2} & Dense & 72B & Qwen/Qwen2.5-72B-Instruct \\
        QwQ-32B-Preview \citep{qwq-32b-preview} & Dense & 32B & Qwen/QwQ-32B-Preview \\
        DeepSeek-R1 \citep{guo2025deepseek} & MoE & 671B & deepseek-ai/DeepSeek-R1 \\
        \midrule
        \rowcolor{blue!15} \multicolumn{4}{l}{\textit{\textbf{Closed-source LLMs}}} \\
        Gemini-2.0-flash \citep{deepmind_gemini_flash} & - & - & \texttt{gemini-2.0-flash-exp} \\
        Gemini-2.0-flash-thinking \citep{deepmind_gemini_flash_thinking} & - & - & \texttt{gemini-2.0-flash-thinking-exp-1219} \\
        GPT-4o \citep{gpt4o} & - & - & \texttt{gpt-4o-2024-08-06} \\
        o1-mini \citep{o1} & - & - & \texttt{o1-mini-2024-09-12} \\
        o1-preview \citep{o1} & - & - & \texttt{o1-preview-2024-09-12} \\
        \bottomrule
    \end{tabular}}}
    \caption{\label{models}Detailed information of all evaluated models in LR${}^{2}$Bench.} 
\end{table*}
\input{table/level_Crossword}
\input{table/level_Acrostic_cryptogram}
\begin{table*}[t]
    \centering
    \resizebox{1.0\textwidth}{!}{
    \setlength{\tabcolsep}{3pt}
    \begin{tabular}{l|ccccc|ccccc|ccccc|ccccc}
        \toprule
        \multirow{2}{*}{\textbf{Model}} &  \multicolumn{5}{c|}{\textbf{Logic Puzzle - $4\times4$}} & \multicolumn{5}{c|}{\textbf{Logic Puzzle - $4\times5$}}  & \multicolumn{5}{c}{\textbf{Logic Puzzle - $4\times6$}} & \multicolumn{5}{c}{\textbf{Logic Puzzle - $4\times7$}} \\
        \cmidrule(l){2-21}
        & CR & S-Acc & EM & PM-0.5 & Tokens & CR & S-Acc & EM & PM-0.5 & Tokens & CR & S-Acc & EM & PM-0.5 & Tokens & CR & S-Acc & EM & PM-0.5 & Tokens \\
        \midrule
        \rowcolor{gray!15} \multicolumn{21}{l}{\textit{\textbf{Open-source LLMs}}}\\
        \midrule 
Llama-3.1-8B-Instruct & 86.0 & 30.7 & 0.0 & 28.0 & 1049 & 52.0 & 12.8 & 0.0 & 0.0 & 1214 & 46.0 & 10.6 & 0.0 & 2.0 & 1225 & 44.0 & 9.8 & 0.0 & 2.0 & 1683 \\
Llama-3.1-70B-Instruct & 66.0 & 35.2 & 8.0 & 42.0 & 989 & 58.0 & 20.0 & 0.0 & 10.0 & 1126 & 54.0 & 18.9 & 0.0 & 12.0 & 1109 & 46.0 & 17.1 & 0.0 & 8.0 & 1435 \\
Llama-3.3-70B-Instruct & 88.0 & 46.7 & 4.0 & 52.0 & 1258 & 82.0 & 35.6 & 0.0 & 26.0 & 1576 & 90.0 & 25.4 & 0.0 & 12.0 & 1931 & 62.0 & 21.2 & 0.0 & 10.0 & 2188 \\
Mistral-7B-Instruct-v0.3 & \textbf{100.0} & 26.8 & 0.0 & 16.0 & 1289 & {\ul 98.0} & 18.0 & 0.0 & 2.0 & 1590 & 96.0 & 15.0 & 0.0 & 0.0 & 1846 & {\ul 94.0} & 16.8 & 0.0 & 0.0 & 1746 \\
Mistral-Small-Instruct-2409 & \textbf{100.0} & 38.7 & 2.0 & 34.0 & 1358 & \textbf{100.0} & 33.6 & 0.0 & 12.0 & 1424 & {\ul 98.0} & 23.7 & 0.0 & 2.0 & 1554 & \textbf{100.0} & 26.7 & 0.0 & 2.0 & 1721 \\
Mistral-Large-Instruct-2411 & \textbf{100.0} & 53.2 & 10.0 & 62.0 & 1293 & \textbf{100.0} & 40.4 & 2.0 & 30.0 & 1532 & \textbf{100.0} & {\ul 30.6} & 0.0 & 18.0 & 1772 & \textbf{100.0} & {\ul 29.1} & 0.0 & 12.0 & 1950 \\
Qwen2.5-7B-Instruct & \textbf{100.0} & 33.5 & 0.0 & 20.0 & 1133 & {\ul 98.0} & 28.8 & 0.0 & 14.0 & 1254 & 96.0 & 21.0 & 0.0 & 0.0 & 1494 & 92.0 & 19.7 & 0.0 & 0.0 & 1701 \\
Qwen2.5-32B-Instruct & \textbf{100.0} & 45.5 & 0.0 & 54.0 & 1070 & 92.0 & 34.3 & 0.0 & 20.0 & 1137 & 94.0 & 24.1 & 0.0 & 6.0 & 1241 & 86.0 & 25.0 & 0.0 & 10.0 & 1383 \\
Qwen2.5-72B-Instruct & {\ul 94.0} & 48.3 & 0.0 & 56.0 & 1490 & {\ul 98.0} & 34.8 & 0.0 & 16.0 & 1641 & 92.0 & 26.3 & 0.0 & 14.0 & 1931 & 90.0 & 26.7 & 0.0 & 6.0 & 2177 \\
QwQ-32B-Preview & {\ul 94.0} & {\ul 76.3} & {\ul 56.0} & {\ul 80.0} & 4766 & 86.0 & {\ul 52.7} & {\ul 14.0} & {\ul 64.0} & 8966 & 68.0 & 30.2 & {\ul 4.0} & {\ul 28.0} & 11292 & 66.0 & 25.8 & {\ul 4.0} & {\ul 20.0} & 13070 \\
DeepSeek-R1 & \textbf{100.0} & \textbf{94.2} & \textbf{90.0} & \textbf{94.0} & 4724 & \textbf{100.0} & \textbf{70.5} & \textbf{36.0} & \textbf{68.0} & 8907 & \textbf{100.0} & \textbf{50.2} & \textbf{16.0} & \textbf{50.0} & 11860 & \textbf{100.0} & \textbf{62.6} & \textbf{28.0} & \textbf{60.0} & 11329 \\
        \midrule
        \rowcolor{blue!15} \multicolumn{21}{l}{\textit{\textbf{Closed-source LLMs}}}\\
        \midrule
Gemini-2.0-flash & 70.0 & 35.3 & 6.0 & 36.0 & 1569 & 56.0 & 22.8 & 2.0 & 16.0 & 2283 & 62.0 & 23.8 & 0.0 & 18.0 & 2254 & 44.0 & 14.8 & 0.0 & 10.0 & 2308 \\
Gemini-2.0-flash-thinking & \textbf{100.0} & 63.8 & 28.0 & 66.0 & 3310 & {\ul 98.0} & 44.0 & 2.0 & 34.0 & 4073 & \textbf{100.0} & 37.8 & 0.0 & 30.0 & 4311 & {\ul 98.0} & 38.1 & 2.0 & 20.0 & 4458 \\
GPT-4o & \textbf{100.0} & 55.0 & 14.0 & 62.0 & 871 & \textbf{100.0} & 41.9 & 0.0 & 30.0 & 907 & \textbf{100.0} & 29.9 & 0.0 & 14.0 & 974 & \textbf{100.0} & 30.3 & 0.0 & 12.0 & 1061 \\
o1-mini & {\ul 98.0} & {\ul 79.5} & {\ul 66.0} & {\ul 82.0} & 5572 & \textbf{100.0} & {\ul 60.5} & {\ul 16.0} & {\ul 62.0} & 9522 & {\ul 98.0} & {\ul 42.9} & {\ul 4.0} & {\ul 36.0} & 12121 & \textbf{100.0} & {\ul 45.9} & {\ul 8.0} & {\ul 34.0} & 13753 \\
o1-preview & \textbf{100.0} & \textbf{96.7} & \textbf{92.0} & \textbf{98.0} & 6199 & {\ul 98.0} & \textbf{74.5} & \textbf{48.0} & \textbf{74.0} & 9129 & \textbf{100.0} & \textbf{53.8} & \textbf{14.0} & \textbf{56.0} & 11140 & {\ul 98.0} & \textbf{50.1} & \textbf{10.0} & \textbf{46.0} & 11330 \\
        \bottomrule
    \end{tabular}}    
    \caption{Performance (\%) of LLMs on Logic Puzzle across all difficulty levels. The best and second-best results are highlighted in \textbf{bold} and \underline{underlined}, respectively. "Tokens" denotes the average number of generated tokens.}
    \label{level_logic}
\end{table*}
\begin{table*}[t]
    \centering
    \resizebox{1.0\textwidth}{!}{
    \setlength{\tabcolsep}{3pt}
    \begin{tabular}{l|ccccc|ccccc|ccccc|ccccc}
        \toprule
        \multirow{2}{*}{\textbf{Model}} &  \multicolumn{5}{c|}{\textbf{Sudoku - $4\times4$, Easy}} & \multicolumn{5}{c|}{\textbf{Sudoku - $4\times4$, Hard}}  & \multicolumn{5}{c}{\textbf{Sudoku - $9\times9$, Easy}} & \multicolumn{5}{c}{\textbf{Sudoku - $9\times9$, Hard}} \\
        \cmidrule(l){2-21}
        & CR & S-Acc & EM & PM-0.5 & Tokens & CR & S-Acc & EM & PM-0.5 & Tokens & CR & S-Acc & EM & PM-0.5 & Tokens & CR & S-Acc & EM & PM-0.5 & Tokens \\
        \midrule
        \rowcolor{gray!15} \multicolumn{21}{l}{\textit{\textbf{Open-source LLMs}}}\\
        \midrule 
Llama-3.1-8B-Instruct & 4.0 & 0.8 & 0.0 & 0.0 & 931 & 6.0 & 1.3 & 0.0 & 0.0 & 1468 & 12.0 & 1.6 & 0.0 & 0.0 & 5051 & 8.0 & 1.2 & 0.0 & 0.0 & 3678 \\
Llama-3.1-70B-Instruct & 68.0 & 32.4 & 0.0 & 38.0 & 1142 & 60.0 & 31.5 & 4.0 & 32.0 & 1343 & 74.0 & 16.4 & 0.0 & 0.0 & 2514 & 76.0 & 16.7 & 0.0 & 0.0 & 2761 \\
Llama-3.3-70B-Instruct & 84.0 & 44.2 & 10.0 & 48.0 & 1253 & 90.0 & 48.9 & 18.0 & 42.0 & 1200 & \textbf{100.0} & 24.7 & 0.0 & 0.0 & 917 & \textbf{100.0} & 21.5 & 0.0 & 0.0 & 877 \\
Mistral-7B-Instruct-v0.3 & 96.0 & 19.6 & 0.0 & 4.0 & 1333 & {\ul 92.0} & 17.5 & 0.0 & 2.0 & 1413 & 78.0 & 6.0 & 0.0 & 0.0 & 4725 & 70.0 & 4.4 & 0.0 & 0.0 & 4962 \\
Mistral-Small-Instruct-2409 & \textbf{100.0} & 33.2 & 0.0 & 18.0 & 1351 & \textbf{100.0} & 30.2 & 2.0 & 12.0 & 1272 & 80.0 & 9.5 & 0.0 & 0.0 & 2400 & 76.0 & 9.1 & 0.0 & 0.0 & 2850 \\
Mistral-Large-Instruct-2411 & 94.0 & {\ul 68.2} & 26.0 & {\ul 78.0} & 1204 & {\ul 92.0} & 54.2 & 14.0 & 54.0 & 1235 & 86.0 & 21.0 & 0.0 & 2.0 & 2910 & 70.0 & 14.5 & 0.0 & 0.0 & 2472 \\
Qwen2.5-7B-Instruct & {\ul 98.0} & 42.0 & 2.0 & 42.0 & 937 & \textbf{100.0} & 36.2 & 4.0 & 16.0 & 951 & {\ul 96.0} & 25.5 & 0.0 & 2.0 & 2022 & 84.0 & 17.1 & 0.0 & 0.0 & 2034 \\
Qwen2.5-32B-Instruct & \textbf{100.0} & 53.8 & 12.0 & 60.0 & 1095 & \textbf{100.0} & 50.2 & 2.0 & 46.0 & 1158 & \textbf{100.0} & {\ul 34.9} & 0.0 & {\ul 8.0} & 1435 & \textbf{100.0} & {\ul 32.5} & 0.0 & {\ul 8.0} & 1119 \\
Qwen2.5-72B-Instruct & \textbf{100.0} & 60.8 & 16.0 & 72.0 & 1380 & \textbf{100.0} & 55.6 & 6.0 & 58.0 & 1520 & 94.0 & 28.4 & 0.0 & 2.0 & 2543 & {\ul 96.0} & 27.3 & 0.0 & 4.0 & 2610 \\
QwQ-32B-Preview & 66.0 & 63.8 & {\ul 60.0} & 64.0 & 3217 & 76.0 & {\ul 72.0} & {\ul 66.0} & {\ul 72.0} & 5119 & 44.0 & 15.3 & 0.0 & 6.0 & 12575 & 32.0 & 9.5 & 0.0 & 0.0 & 12613 \\
DeepSeek-R1 & \textbf{100.0} & \textbf{100.0} & \textbf{100.0} & \textbf{100.0} & 2878 & \textbf{100.0} & \textbf{100.0} & \textbf{100.0} & \textbf{100.0} & 3620 & \textbf{100.0} & \textbf{46.6} & 0.0 & \textbf{42.0} & 14491 & \textbf{100.0} & \textbf{34.5} & 0.0 & \textbf{14.0} & 12117 \\
        \midrule
        \rowcolor{blue!15} \multicolumn{21}{l}{\textit{\textbf{Closed-source LLMs}}}\\
        \midrule
Gemini-2.0-flash & {\ul 94.0} & 57.8 & 24.0 & 66.0 & 1437 & 92.0 & 58.9 & 26.0 & 60.0 & 1454 & {\ul 94.0} & 33.8 & 0.0 & 10.0 & 4194 & {\ul 92.0} & 30.6 & 0.0 & {\ul 14.0} & 4287 \\
Gemini-2.0-flash-thinking & 70.0 & 62.8 & 44.0 & 68.0 & 2009 & 66.0 & 50.0 & 22.0 & 54.0 & 2350 & 92.0 & \textbf{41.7} & 0.0 & \textbf{36.0} & 5511 & 90.0 & {\ul 31.3} & 0.0 & 6.0 & 5541 \\
GPT-4o & \textbf{100.0} & 68.0 & 26.0 & 84.0 & 1015 & \textbf{100.0} & 63.3 & 32.0 & 64.0 & 1039 & \textbf{100.0} & {\ul 39.7} & 0.0 & {\ul 26.0} & 1234 & \textbf{100.0} & \textbf{37.9} & 0.0 & \textbf{18.0} & 1127 \\
o1-mini & \textbf{100.0} & {\ul 85.4} & {\ul 66.0} & {\ul 88.0} & 3059 & {\ul 96.0} & {\ul 68.6} & {\ul 42.0} & {\ul 72.0} & 3403 & \textbf{100.0} & 32.9 & 0.0 & 10.0 & 4428 & \textbf{100.0} & 26.7 & 0.0 & 2.0 & 4953 \\
o1-preview & \textbf{100.0} & \textbf{100.0} & \textbf{100.0} & \textbf{100.0} & 5616 & \textbf{100.0} & \textbf{100.0} & \textbf{100.0} & \textbf{100.0} & 6191 & 86.0 & 34.5 & 0.0 & 16.0 & 10328 & 80.0 & 25.7 & 0.0 & 6.0 & 10111 \\
        \bottomrule
    \end{tabular}}    
    \caption{Performance (\%) of LLMs on Sudoku across all difficulty levels. The best and second-best results are highlighted in \textbf{bold} and \underline{underlined}, respectively. "Tokens" denotes the average number of generated tokens.}
    \label{level_sudoku}
\end{table*}
\begin{table*}[t]
    \centering
    \resizebox{0.7\textwidth}{!}{
    \setlength{\tabcolsep}{3pt}
    \begin{tabular}{l|ccccc|ccccc}
        \toprule
        \multirow{2}{*}{\textbf{Model}} &  \multicolumn{5}{c|}{\textbf{Drop Quote - Easy}} & \multicolumn{5}{c|}{\textbf{Drop Quote - Hard}} \\
        \cmidrule(l){2-11}
        & CR & S-Acc & EM & PM-0.5 & Tokens & CR & S-Acc & EM & PM-0.5 & Tokens \\
        \midrule
        \rowcolor{gray!15} \multicolumn{11}{l}{\textit{\textbf{Open-source LLMs}}}\\
        \midrule 
Llama-3.1-8B-Instruct & 38.0 & 9.6 & 0.0 & 2.0 & 1692 & 50.0 & 12.8 & 0.0 & 0.0 & 2553 \\
Llama-3.1-70B-Instruct & 86.0 & 32.7 & 0.0 & 22.0 & 1298 & 78.0 & 22.7 & 0.0 & {\ul 2.0} & 1698 \\
Llama-3.3-70B-Instruct & {\ul 98.0} & 33.1 & 0.0 & 24.0 & 900 & \textbf{100.0} & 24.9 & 0.0 & {\ul 2.0} & 936 \\
Mistral-7B-Instruct-v0.3 & 72.0 & 8.9 & 0.0 & 2.0 & 2193 & 60.0 & 4.3 & 0.0 & 0.0 & 2480 \\
Mistral-Small-Instruct-2409 & {\ul 98.0} & 30.5 & 0.0 & 12.0 & 1357 & 96.0 & 23.3 & 0.0 & 0.0 & 1873 \\
Mistral-Large-Instruct-2411 & {\ul 98.0} & 29.4 & 0.0 & 18.0 & 1429 & {\ul 98.0} & 20.0 & 0.0 & 0.0 & 1702 \\
Qwen2.5-7B-Instruct & \textbf{100.0} & 24.9 & 0.0 & 8.0 & 1500 & 96.0 & 18.9 & 0.0 & 0.0 & 2204 \\
Qwen2.5-32B-Instruct & {\ul 98.0} & 33.5 & 0.0 & {\ul 26.0} & 1084 & 92.0 & 23.3 & 0.0 & {\ul 2.0} & 1310 \\
Qwen2.5-72B-Instruct & 96.0 & \textbf{35.5} & 0.0 & 24.0 & 1505 & 92.0 & {\ul 26.2} & 0.0 & {\ul 2.0} & 2009 \\
QwQ-32B-Preview & 32.0 & 9.6 & 0.0 & 14.0 & 5987 & 34.0 & 5.4 & 0.0 & {\ul 2.0} & 6169 \\
DeepSeek-R1 & \textbf{100.0} & \textbf{54.6} & \textbf{14.0} & \textbf{58.0} & 11202 & \textbf{100.0} & \textbf{40.1} & 0.0 & \textbf{26.0} & 11643 \\
        \midrule
        \rowcolor{blue!15} \multicolumn{11}{l}{\textit{\textbf{Closed-source LLMs}}}\\
        \midrule
Gemini-2.0-flash & 92.0 & 37.2 & 0.0 & 28.0 & 2149 & 92.0 & \textbf{31.3} & 0.0 & 6.0 & 3286 \\
Gemini-2.0-flash-thinking & {\ul 96.0} & 38.8 & 0.0 & {\ul 38.0} & 3621 & {\ul 96.0} & {\ul 30.0} & 0.0 & {\ul 8.0} & 3150 \\
GPT-4o & \textbf{98.0} & 34.8 & 0.0 & 24.0 & 1125 & \textbf{100.0} & 27.3 & 0.0 & 4.0 & 1205 \\
o1-mini & \textbf{98.0} & {\ul 41.4} & {\ul 4.0} & 36.0 & 14130 & 94.0 & 27.1 & 0.0 & 6.0 & 12380 \\
o1-preview & \textbf{98.0} & \textbf{49.6} & \textbf{18.0} & \textbf{56.0} & 13064 & {\ul 96.0} & 27.9 & \textbf{8.0} & \textbf{20.0} & 14126 \\
        \bottomrule
    \end{tabular}}    
    \caption{Performance (\%) of LLMs on Drop Quote across all difficulty levels. The best and second-best results are highlighted in \textbf{bold} and \underline{underlined}, respectively. "Tokens" denotes the average number of generated tokens.}
    \label{level_drop}
\end{table*}


\end{document}
