% This must be in the first 5 lines to tell arXiv to use pdfLaTeX, which is strongly recommended.
\pdfoutput=1
% In particular, the hyperref package requires pdfLaTeX in order to break URLs across lines.

\documentclass[11pt]{article}

% Change "review" to "final" to generate the final (sometimes called camera-ready) version.
% Change to "preprint" to generate a non-anonymous version with page numbers.
\usepackage[preprint]{acl}

% Standard package includes
\usepackage{times}
\usepackage{latexsym}

% For proper rendering and hyphenation of words containing Latin characters (including in bib files)
\usepackage[T1]{fontenc}
% For Vietnamese characters
% \usepackage[T5]{fontenc}
% See https://www.latex-project.org/help/documentation/encguide.pdf for other character sets

% This assumes your files are encoded as UTF8
\usepackage[utf8]{inputenc}

% This is not strictly necessary, and may be commented out,
% but it will improve the layout of the manuscript,
% and will typically save some space.
\usepackage{microtype}

% This is also not strictly necessary, and may be commented out.
% However, it will improve the aesthetics of text in
% the typewriter font.
\usepackage{inconsolata}

%Including images in your LaTeX document requires adding
%additional package(s)
\usepackage{graphicx}
\usepackage{booktabs}
\usepackage{amssymb}
\usepackage{pifont}
\usepackage{amsmath}
\usepackage{multirow}
\usepackage{colortbl}
\usepackage[most]{tcolorbox}
\usepackage{xcolor}
\usepackage{fvextra}
\usepackage{fancyvrb}
\DefineVerbatimEnvironment{CustomVerbatim}{Verbatim}{formatcom=\normalfont, breaklines=true, breaksymbolleft={}, breaksymbolright={}}
\usepackage{ragged2e}
\usepackage[normalem]{ulem}
\useunder{\uline}{\ul}{}


% If the title and author information does not fit in the area allocated, uncomment the following
%
%\setlength\titlebox{<dim>}
%
% and set <dim> to something 5cm or larger.

\title{LR$^2$Bench: Evaluating Long-chain Reflective Reasoning Capabilities of Large Language Models via Constraint Satisfaction Problems}


% Author information can be set in various styles:
% For several authors from the same institution:
% \author{Author 1 \and ... \and Author n \\
%         Address line \\ ... \\ Address line}
% if the names do not fit well on one line use
%         Author 1 \\ {\bf Author 2} \\ ... \\ {\bf Author n} \\
% For authors from different institutions:
% \author{Author 1 \\ Address line \\  ... \\ Address line
%         \And  ... \And
%         Author n \\ Address line \\ ... \\ Address line}
% To start a separate ``row'' of authors use \AND, as in
% \author{Author 1 \\ Address line \\  ... \\ Address line
%         \AND
%         Author 2 \\ Address line \\ ... \\ Address line \And
%         Author 3 \\ Address line \\ ... \\ Address line}

% \author{First Author \\
%   Affiliation / Address line 1 \\
%   Affiliation / Address line 2 \\
%   Affiliation / Address line 3 \\
%   \texttt{email@domain} \\\And
%   Second Author \\
%   Affiliation / Address line 1 \\
%   Affiliation / Address line 2 \\
%   Affiliation / Address line 3 \\
%   \texttt{email@domain} \\}

\author{
    Jianghao Chen\textsuperscript{1,2,3}, \ 
    Zhenlin Wei\textsuperscript{1,2},\ 
    Zhenjiang Ren\textsuperscript{1,2}, \
    Ziyong Li\textsuperscript{1,2}, \
    Jiajun Zhang\textsuperscript{1,2,4,5}\thanks{\ \ Corresponding Author}   \\
    \textsuperscript{1}Institute of Automation, Chinese Academy of Sciences\\
    \textsuperscript{2}School of Artificial Intelligence, University of Chinese Academy of Sciences\\
    \textsuperscript{3}Zhongguancun Academy, Beijing, China\\
    \textsuperscript{4}Wuhan AI Research, 
    \textsuperscript{5}Shanghai Artificial Intelligence Laboratory, Shanghai, China\\
    \texttt{\{chenjianghao2022, weizhenlin2025, renzhenjiang2024, liziyong2023\}@ia.ac.cn}\\
    \texttt{jjzhang@nlpr.ia.ac.cn} \\
}


%\author{
%  \textbf{First Author\textsuperscript{1}},
%  \textbf{Second Author\textsuperscript{1,2}},
%  \textbf{Third T. Author\textsuperscript{1}},
%  \textbf{Fourth Author\textsuperscript{1}},
%\\
%  \textbf{Fifth Author\textsuperscript{1,2}},
%  \textbf{Sixth Author\textsuperscript{1}},
%  \textbf{Seventh Author\textsuperscript{1}},
%  \textbf{Eighth Author \textsuperscript{1,2,3,4}},
%\\
%  \textbf{Ninth Author\textsuperscript{1}},
%  \textbf{Tenth Author\textsuperscript{1}},
%  \textbf{Eleventh E. Author\textsuperscript{1,2,3,4,5}},
%  \textbf{Twelfth Author\textsuperscript{1}},
%\\
%  \textbf{Thirteenth Author\textsuperscript{3}},
%  \textbf{Fourteenth F. Author\textsuperscript{2,4}},
%  \textbf{Fifteenth Author\textsuperscript{1}},
%  \textbf{Sixteenth Author\textsuperscript{1}},
%\\
%  \textbf{Seventeenth S. Author\textsuperscript{4,5}},
%  \textbf{Eighteenth Author\textsuperscript{3,4}},
%  \textbf{Nineteenth N. Author\textsuperscript{2,5}},
%  \textbf{Twentieth Author\textsuperscript{1}}
%\\
%\\
%  \textsuperscript{1}Affiliation 1,
%  \textsuperscript{2}Affiliation 2,
%  \textsuperscript{3}Affiliation 3,
%  \textsuperscript{4}Affiliation 4,
%  \textsuperscript{5}Affiliation 5
%\\
%  \small{
%    \textbf{Correspondence:} \href{mailto:email@domain}{email@domain}
%  }
%}

\begin{document}
\maketitle
\begin{abstract}
Recent progress in o1-like models has significantly enhanced the reasoning abilities of Large Language Models (LLMs), empowering them to tackle increasingly complex tasks through reflection capabilities, such as making assumptions, backtracking, and self-refinement. However, effectively evaluating such reflection capabilities remains challenging due to the lack of appropriate benchmarks. To bridge this gap, we introduce LR$^2$Bench, a novel benchmark designed to evaluate the \textbf{L}ong-chain \textbf{R}eflective \textbf{R}easoning capabilities of LLMs. LR$^2$Bench comprises 850 samples across six Constraint Satisfaction Problems (CSPs) where reflective reasoning is crucial for deriving solutions that meet all given constraints. Each type of task focuses on distinct constraint patterns, such as knowledge-based, logical, and spatial constraints, providing a comprehensive evaluation of diverse problem-solving scenarios. We conduct extensive evaluation on both conventional models and o1-like models. Our experimental results reveal that even the most advanced reasoning-specific models, such as DeepSeek-R1 and OpenAI o1-preview, struggle with tasks in LR$^2$Bench, achieving an average Exact Match score of only 20.0\% and 23.6\%, respectively. These findings underscore the significant room for improvement in the reflective reasoning capabilities of current LLMs. The leaderboard of our benchmark is available at \url{https://huggingface.co/spaces/UltraRonin/LR2Bench}.
\end{abstract}



\section{Introduction}



Recent advancements in Large Language Models (LLMs), exemplified by o1-like models \citep{o1, qwq-32b-preview, guo2025deepseek}, have demonstrated substantial progress in their reasoning capabilities. These models exhibit more human-like behaviors, such as making assumptions, verification, backtracking, and self-correction, enabling them to address increasingly complex challenges across various domains \citep{zhong2024evaluation, wu2024comparative, gao2024omni, wang2024planning}.

\begin{figure}[t]
  \centering
  \includegraphics[width=0.9\columnwidth]{fig/sample.drawio.pdf}
  \caption{The response demonstrates the model's reflective reasoning process while solving a Sudoku problem.}
  \label{motivation}
  \vspace{-4mm}
\end{figure}


\begin{figure*}[t]
\begin{center}
\includegraphics[width=.85\linewidth]{fig_overview_v3.pdf}
\end{center}
\caption{
FastAtlas Overview: In each frame, we compute charts spanning fully or partially visible triangles (a), determine texture space bounding boxes for the visible portions of the view-space projections of each chart, and tightly pack these boxes into atlases (b, here $2K \times 2K$). We simultaneously bijectively parameterize and shade the charts into their atlas boxes, obtaining high quality texture space shading (c), and use this shading to render the shaded frames (d).}
\label{fig:overview}
\label{fig:alg_overview}
\end{figure*}

\section{Overview}
\label{sec:overview}
Our work has two core contributions: a real-time, GPU-based algorithm for tight packing of general parameterized charts into compact atlases; and a real-time TSS method that
utilizes this packing.  

\paragraph*{FastAtlas Packing.}
FastAtlas runs entirely on the GPU as a series of compute shaders. It takes the bounding boxes of parameterized charts as input, and packs them into an atlas (Fig~\ref{fig:overview}b, Sec.~\ref{sec:pack}). As such, the only input it requires are the dimensions of the bounding boxes.
Its outputs are deterministic; identical input charts are packed into identical atlases. This is critical for TSS and similar applications, as it ensures that consecutive frames taken from the same camera view have the same shading. Even minute shading differences across such frames can cause sampling jitter, leading to undesirable flicker \cite{baker2012rock}. 
While prior methods such as \cite{mueller2018shading,hladky2019tessellated,hladky2021snakebinning,Neff2022MSA} cap the dimensions of the charts that can be packed as-is for a given atlas size, and scale down all charts that exceed these dimensions, we scale all charts by the same factor, and do so only when strictly necessary to achieve packing success (Figs~\ref{fig:atlas},~\ref{fig:sas_issues}). 

\paragraph*{TSS using FastAtlas.}
Our end-to-end TSS atlas generation method combines the packing method above with a novel approach for computing seamless per-frame charts. 
We define our charts as the connected components of the visible surfaces in each frame (Fig.~\ref{fig:overview}a), and efficiently compute them using a parallel union-find algorithm (Sec.~\ref{sec:visible}). Since the boundaries of these charts coincide with the contours of the rendered surface, they are {\em invisible} to the viewer. This approach 
eliminates the artifacts caused by shading discontinuities along visible seams (Fig.~\ref{fig:seams}). 

\begin{parWithWrapFigure}
\begin{wrapfigure}{l}{.27\columnwidth}%
\includegraphics[width=\linewidth]{fig_inset_view_plane.pdf}%
\end{wrapfigure}
We bijectively parametrize the {\em visible portions} of our charts by projecting them to view space (inset). This maps a constant number of texels to each pixel in the final rendered output, evenly distributing residual undersampling error across all image pixels. While conceptually straightforward, efficiently parameterizing charts containing partially visible triangles using viewspace projection is non-trivial, as the visible portions may no longer be triangular (e.g. green triangle in the inset); applying naive projection to triangles with vertices behind the camera may produce ill-posed results. Clipping triangles before projection is both computationally expensive and significantly complicates downstream operations. We avoid explicit clipping by observing that all that is required for atlas packing is the dimensions of, potentially conservative, bounding boxes of these projected visible portions. We compute such bounding boxes without explicit chart clipping by adapting a conservative screen coverage estimator \shortcite{Blinn:CalculatingScreenCoverage} (Sec.~\ref{sec:box}). We then pack the computed boxes using FastAtlas. 
\end{parWithWrapFigure}

Finally, we shade the visible portion of each chart into its corresponding atlas bounding box (Fig~\ref{fig:overview}c). 
The resulting texture is then used during rasterization as a standard texture map (Fig. ~\ref{fig:overview}d). 
Our framework is compatible with all existing approaches for texture space shading, including forward shading, raytraced illumination, or deferred shading in texture space \cite{baker:2016}. In the examples shown, we use the standard forward shading based rendering pipeline included in the G3D Innovation Engine \cite{G3D17}, a commercial grade renderer.


Despite this progress, the reflection capabilities of LLMs remain largely unexplored. Reflection can be defined as the process of engaging in attentive, critical, exploratory, and iterative self-interactions with one's thoughts, actions, and underlying conceptual frameworks \citep{nguyen2014reflection}. Existing research predominantly focuses on evaluating LLMs' abilities to utilize explicit feedback for self-criticism or self-refinement \citep{li2024hindsight, lan2024criticeval, gou2024critic, sun2024critique, lin-etal-2024-criticbench}. However, these approaches overlook a fundamental question: \textbf{\textit{Can LLMs spontaneously engage in the whole reflection process to solve more complex tasks?}} Reflection is not simply about reacting to feedback. It also encompasses exploring possible solutions, assessing and adjusting strategies, and adapting when confronted with contradiction. To better understand how LLMs might engage in these capabilities, we consider Constraint Satisfaction Problems (CSPs).
% These capabilities become particularly crucial when addressing multi-step complex tasks like Constraint Satisfaction Problems (CSPs). 
CSPs are defined as a set of variables whose state must satisfy specific constraints, which inherently require iterative exploration in a large search space without predefined solution paths \citep{dechter2003constraint}. As shown in Figure \ref{motivation}, the response of QwQ-32B-Preview \citep{qwq-32b-preview} to Sudoku, a classic CSP, exemplifies a typical reflection process. This task presents a scenario with row, column, and grid constraints, requiring iterative trial-and-error to determine valid values for each cell. The model initially makes an assumption, then identifies a conflict, and finally backtracks to correct its initial guess. This demonstrates an effective reflective reasoning process often absent from standard NLP tasks, such as summarization, translation, and question-answering.


Therefore, to systematically investigate reflective reasoning in LLMs, we propose LR$^2$Bench, a novel benchmark for evaluating the \textbf{L}ong-chain \textbf{R}eflective \textbf{R}easoning capabilities of LLMs. LR$^2$Bench consists of six CSPs: Crossword, Acrostic, Logic Puzzle, Cryptogram, Sudoku, and Drop Quote. Each task necessitates reflection processes and emphasizes specific constraint patterns, such as knowledge-based constraints, logical constraints, and spatial constraints, ensuring a comprehensive assessment of LLMs' reflective reasoning capabilities across diverse problem-solving scenarios. We manually collect and annotate a total of 850 examples spanning multiple difficulty levels across all task types. Through extensive experiments on both convention and o1-like models,
we observe that even current top-performed reasoning LLMs (e.g., DeepSeek-R1 and o1-preview) struggle to complete tasks in LR$^2$Bench, underscoring substantial room for improvement in this critical area. Furthermore, most conventional models exhibit inferior performance than o1-like models and achieve near-zero sample-level accuracy in LR$^2$Bench, highlighting the essential role of reflective reasoning skills in tackling the challenges posed by our benchmark.

% 2) While open-source reasoning LLMs (e.g.,  QwQ-32B-Preview) show better performance than conventional models in some tasks, they fail in non-reasoning domains like instruction following.

Our contributions are summarized as follows: 1) We introduce LR$^2$Bench, a novel benchmark with six tasks and diverse difficulty levels, designed to evaluate the long-chain reflective reasoning capabilities of LLMs. 2) We conduct a comprehensive evaluation of reflective reasoning capabilities in both conventional and reasoning-specific LLMs. 3) We present a detailed analysis of the limitations exhibited by current LLMs on LR$^2$Bench, establishing a foundation for future research on the development of advanced reasoning models.


\begin{figure*}[t]
  \includegraphics[width=\linewidth]{fig/collection.drawio_new.pdf}
  \caption{The overall pipeline of LR$^2$Bench, illustrated with the Sudoku task.}
  \label{framework}
  \vspace{-4mm}
\end{figure*}


\section{LR$^2$Bench}
In this section, we introduce the construction process of LR$^2$Bench. Section \ref{sub1} provides an overview of each task. Section \ref{sub2} discusses the diverse scenarios targeted by LR${}^2$Bench. Section \ref{sub3} shows the annotation process and data statistics, and Section \ref{sub4} outlines the evaluation metrics. Figure \ref{framework} illustrates the data collection, annotation, and evaluation pipeline of our LR$^2$Bench.


\subsection{Task Overview}
\label{sub1}
We first introduce the task descriptions and collection methods for all tasks in LR$^2$Bench. For detailed information, please refer to Appendix \ref{app_example}.

\paragraph{Crossword} The Crossword task requires inferring correct words from given clues and filling them into a grid. A key challenge lies in satisfying the constraint of shared letter intersections between horizontal and vertical words. We collected 150 Crossword samples published in 2024 from Los Angeles Times\footnote{\url{https://www.latimes.com}} and Vulture\footnote{\url{https://www.vulture.com}} in three sizes: $5\times5$, $10\times10$, and $15\times15$, with 50 ones for each size.

\paragraph{Acrostic} The Acrostic task involves word clues like Crossword, but its objective is to form a hidden quotation or sentence from the answers to the clues. This requires that the answer words not only satisfy the corresponding clues but also effectively integrate to construct the ultimate hidden message. We collected 50 easy and 50 hard Acrostic samples from Printable Puzzles\footnote{\url{https://www.printable-puzzles.com/printable-acrostic-puzzles.php}} with timestamps ranging from September 2024 to December 2024.

\paragraph{Logic Puzzle} The Logic Puzzle task constitutes a problem that necessitates logical reasoning to deduce relationships between a set of entities based on the given constraints and clues. The objective is to systematically analyze the given information, employing techniques such as hypothesis formation, elimination, and deductive inference, to determine a unique solution that satisfies all given constraints. We collected 50 puzzles for each of the four sizes ($4\times4$, $4\times5$, $4\times6$, and $4\times7$) from Printable Puzzles\footnote{\url{https://www.printable-puzzles.com/printable-logic-puzzles.php}}, with timestamps ranging from September 2024 to December 2024.

\paragraph{Cryptogram} The Cryptogram task involves the decryption of an encrypted quotation or sentence, where each letter of an original text is substituted with another, resulting in an apparently nonsense text. Decryption requires identifying patterns, common letter frequencies, and word structures to deduce the letter-to-letter correspondences, ultimately reconstructing the original content. We collected 50 easy and 50 hard samples from Printable Puzzles\footnote{\url{https://www.printable-puzzles.com/printable-cryptograms.php}} with timestamps ranging from September 2024 to December 2024.

\paragraph{Sudoku} The Sudoku task consists of filling a $n^2 \times n^2$ grid with digits from 1 to $n^2$, subject to the constraint that each row, column, and $n \times n$ subgrid contains all digits from 1 to $n^2$ without repetition. Success in Sudoku relies on logical deduction and careful consideration of the existing digits to determine valid placements for the remaining numbers. From 1sudoku\footnote{\url{https://1sudoku.com}}, we collected 200 Sudoku samples in total: 50 easy and 50 hard samples for both $4\times4$ and $9\times9$ sizes.

\paragraph{Drop Quote} The Drop Quote task comprises a grid of multiple rows and columns, with each column providing a set of candidate letters. The task requires determining the correct row for letters in each column, effectively "dropping" it into target place to reveal the hidden quotation. We created 50 easy samples by manually compiling common quotations, and collected 50 hard samples from Printable Puzzles\footnote{\url{https://www.printable-puzzles.com/printable-drop-quotes.php}}, with timestamps ranging from September 2024 to December 2024.

% \subsection{What Can LR$^2$Bench Tell?}
\subsection{Diverse Scenarios}
\label{sub2}

Each type of task within LR$^2$Bench focuses on different constraint patterns, providing a comprehensive framework to evaluate models’ reflective reasoning capabilities across diverse scenarios. We further explore the varying capabilities required for completing the tasks in LR$^2$Bench.

\paragraph{Reflection} Reflection is the most fundamental capability for tackling the complex tasks in LR$^2$Bench. Unlike simple problems with short-cut solutions, these tasks are inherently iterative, demanding the exploration of multiple possibilities, identification of dead ends, and adaptive revision of initial hypotheses. Such reflective reasoning capabilities enable a thorough analysis and refinement of strategies, ultimately leading to more robust and effective solutions.

\paragraph{Long-chain Generation} LR$^2$Bench incorporates tasks that necessitate long-chain generation, a crucial capability for LLMs to tackle complex reasoning problems. Unlike tasks with simple and isolated answers, these tasks require LLMs to generate a long chain of steps or decisions that build upon each other toward a final solution.

\paragraph{Knowledge-based Reasoning} Both Crossword and Acrostic tasks demand broad world knowledge and commonsense reasoning abilities since the clue answers often hinge on cultural references, idiomatic expressions, and diverse factual domains. Additionally, the Cryptogram and Drop Quote tasks require knowledge of typical phrase structures and common linguistic patterns to decode messages or reconstruct quotations effectively.

\paragraph{Logical Reasoning} The logical reasoning ability is essential across various tasks, especially evident in the Logic Puzzles and Sudoku. These tasks involve information integration and systematic application of deductive reasoning to solve problems constrained by specific rules.

\paragraph{Spatial Reasoning} While not the primary focus for all tasks, spatial reasoning also emerges as a critical capability within LR$^2$Bench, particularly in tasks with grids. For Crossword, considering letter intersections across horizontal and vertical placements is crucial. Similarly, in Sudoku, effective digit placement requires reasoning about row, column, and subgrid constraints, all of which involve spatial relationships within the grid.


\subsection{Data Annotation and Statistics}
\label{sub3}
For PDF data collected from websites, we manually convert key elements in task samples into a structured text format suitable for LLMs' inputs. Then, we manually craft task-specific instructions, including problem definitions and rules, to provide LLMs with the necessary guidance to process each task effectively. To further control the output format of LLMs, we manually construct two simple shots for each type of task to facilitate subsequent answer extraction and evaluation. Table \ref{tab:overview} presents all tasks in LR$^2$Bench along with their attributes. Appendix \ref{app_example} shows the text-format task examples. Appendix \ref{app_prompt} shows the detailed instructions and few-shot examples for each type of task.

\subsection{Evaluation Metrics}
\label{sub4}
Since all the tasks in LR$^2$Bench consist of multiple subtasks (e.g., words inference in Crossword, cells completion in Sudoku), we employ fine-grained, subtask-level evaluation metrics. Given a problem with $N$ subtasks, let $G = \{ g_1, \ldots, g_N \}$ and $P = \{ p_1, \ldots, p_N \}$ denote the ground truth and LLM-generated answers for each subtask, respectively. We define the following evaluation metrics:


\paragraph{Completion Ratio} The Completion Ratio (CR) metric measures the proportion of subtasks within a given problem that LLMs successfully complete, regardless of the correctness of the answers. 
CR is calculated as follows:
\begin{equation}\label{cr}
% \small \text{CR} = \frac{\text{Number of Completed Samples}}{\text{Total Samples Number}}
\text{CR} = \frac{1}{N} \sum_{i=1}^{N} \mathbb{I}(p_i \neq \varnothing)
\end{equation}
where $\mathbb{I}(p_i \neq \varnothing)$ equals 1 if subtask $i$ is completed (i.e., $p_i$ is not empty) and 0 otherwise. In our experiments, we observe that some models struggle to produce a complete reasoning chain and fail to reach the final answer for each subtask. Therefore, we propose CR to measure the long reasoning chain generation capability of LLMs.


\paragraph{Exact Match} The Exact Match (EM) metric employs a strict correctness criterion: for each subtask, the generated answer by LLMs must exactly match the ground truth. EM is calculated as follows:
\begin{equation}\label{em}
% \small \text{EM}= \frac{\text{Number of Exactly Matched Samples}}{\text{Total Samples Number}}
% \text{EM} = \mathbb{I}\left( \bigwedge_{i=1}^{N} (p_i = g_i) \right)
\text{EM} = \mathbb{I}\left( \frac{1}{N} \sum_{i=1}^{N} \mathbb{I}(p_i = g_i) = 1.0 \right)
\end{equation}
This metric is crucial for evaluating the absolute accuracy on individual samples. For example, in a Sudoku task, EM would require that the completed grid be identical to the ground truth, with no errors in any row, column, or subgrid.

\paragraph{Partial Match} To mitigate the strictness of EM, we employ a Partial Match metric (PM-0.5). This metric relaxes the requirement for a perfect match: one LLM-generated answer is considered correct under PM-0.5 if at least 50\% of its subtasks are correct.
PM-0.5 is calculated as follows:
\begin{equation}\label{pm05}
% \small \text{PM-0.5} = \frac{\text{Number of Samples with more than 50\% Match}}{\text{Total Samples Number}}
\text{PM-0.5} = \mathbb{I}\left( \frac{1}{N} \sum_{i=1}^{N} \mathbb{I}(p_i = g_i) \geq 0.5 \right)
\end{equation}


\paragraph{Subtask Accuracy} For more fine-grained evaluation, we propose Subtask Accuracy (S-Acc), which calculates the proportion of correctly solved subtasks compared to the ground truth:
\begin{equation}\label{s-acc}
\text{S-Acc} = \frac{1}{N} \sum_{i=1}^{N} \mathbb{I}(p_i = g_i)
\end{equation}
By considering both S-Acc and EM, we can gain a more comprehensive understanding of model performance across complex reasoning tasks.


\section{Experiments}

\subsection{Experimental Setup}
% todo
\paragraph{Evaluated Models} We evaluate 11 open-source models and 5 closed-source models. Specifically, we include leading o1-like models: QwQ-32B-Preview \citep{qwq-32b-preview}, DeepSeek-R1 \citep{guo2025deepseek}, Gemini-2.0-flash-thinking \citep{deepmind_gemini_flash_thinking}, OpenAI o1-mini and o1-preview \citep{o1}. Please refer to Appendix \ref{app_models} for detailed information of all selected models.

\paragraph{Implementation Details} We utilize the default prompt templates for all LLMs. Appendix \ref{app_example} and \ref{app_prompt} show detailed problem formats, instructions, and few-shot examples for each type of task. To facilitate evaluation, we wrap answers within specific tags (e.g., \texttt{<Answer>} and \texttt{</Answer>}) in the few-shot examples. This enables precise answer extraction from the model responses. We use the vLLM inference framework \citep{kwon2023efficient} and employ greedy sampling with temperature = 0 to minimize randomness, except for o1-mini and o1-preview which have an inherent temperature of 1. The maximum sequence length is set to the default maximum value for each model.

\subsection{Main Results}


Table \ref{tab:main_exp_sum} presents the average performance across six tasks on LR$^2$Bench, with individual task results detailed in Table \ref{tab:main_exp}.

\begin{table}[!h]
    \centering
    \resizebox{1.0\columnwidth}{!}{
    % \setlength{\tabcolsep}{3pt}
    \begin{tabular}{l|ccccc}
        \toprule
        \textbf{Model} & \textbf{CR} & \textbf{S-Acc} & \textbf{EM} & \textbf{PM-0.5} & \textbf{Tokens} \\
        \midrule
        \rowcolor{gray!15} \multicolumn{6}{l}{\textit{\textbf{Open-source LLMs}}}\\
        \midrule 
        Llama-3.1-8B-Instruct & 42.6 & 9.9 & 0.0 & 3.8 & 2,478 \\
        Llama-3.1-70B-Instruct & 71.8 & 27.4 & 0.5 & 21.9 & 2,090 \\
        Llama-3.3-70B-Instruct & 92.4 & 33.1 & 1.3 & 25.8 & 1,842 \\
        Mistral-7B-Instruct-v0.3 & 85.8 & 12.1 & 0.0 & 2.3 & 2,736 \\
        Mistral-Small-Instruct-2409 & 91.0 & 23.1 & 0.2 & 13.3 & 2,273 \\
        Mistral-Large-Instruct-2411 & 96.1 & \underline{36.4} & 2.5 & \underline{30.0} & 2,313 \\
        Qwen2.5-7B-Instruct & 85.1 & 17.7 & 0.3 & 5.1 & 2,086 \\
        Qwen2.5-32B-Instruct & \underline{96.2} & 29.9 & 0.6 & 14.8 & 1,924 \\
        Qwen2.5-72B-Instruct & 95.0 & 33.9 & 0.9 & 20.8 & 2,359 \\
        QwQ-32B-Preview & 65.0 & 26.6 & \underline{8.5} & 19.3 & 6,709 \\
        DeepSeek-R1 & \textbf{100.0} & \textbf{58.4} & \textbf{20.0} & \textbf{62.0} & 9,856 \\
        \midrule
        \rowcolor{blue!15} \multicolumn{6}{l}{\textit{\textbf{Closed-source LLMs}}}\\
        \midrule
        Gemini-2.0-flash & 81.1 & 37.0 & 2.4 & 34.5 & 2,637 \\
        Gemini-2.0-flash-thinking & 88.2 & 39.4 & 4.3 & 35.0 & 3,725 \\
        GPT-4o & \textbf{99.8} & \underline{43.7} & 3.2 & \underline{41.7} & 1,486 \\
        o1-mini & \underline{97.7} & 41.3 & \underline{9.1} & 32.8 & 9,576 \\
        o1-preview & 96.3 & \textbf{58.7} & \textbf{23.6} & \textbf{61.7} & 11,436 \\ 
        \bottomrule
    \end{tabular}}    
    \caption{Average performance (\%) across six tasks on LR${}^{2}$Bench. The best and second-best results are highlighted in \textbf{bold} and \underline{underlined}, respectively. "Tokens" denotes the average number of generated tokens.}
    \label{tab:main_exp_sum}
    % \vspace{-2mm}
\end{table}
\documentclass{MITstyle}

%\usepackage[table]{xcolor}
\usepackage{chngcntr}
\usepackage{hyperref}
\usepackage{microtype}

\title{A Lightweight and Extensible Cell Segmentation and Classification Model for Whole Slide Images}

\author{Nikita Shvetsov~$^{1, }$\footnote{Correspondence e-mail: nikita.shvetsov@uit.no}, Thomas K. Kilvaer~$^{2, 3}$, Masoud Tafavvoghi~$^{4}$, Anders Sildnes~$^{1}$, \\ Kajsa Møllersen~$^{4}$, Lill-Tove Rasmussen Busund~$^{5, 6}$, Lars Ailo Bongo~$^{1}$ \\
%
\vspace{1em} % Space between authors and afilliations
%
\normalfont{\small $^{1}$Department of Computer Science, UiT The Arctic University of Norway}\\
\normalfont{\small $^{2}$Department of Oncology, University Hospital of North Norway}\\
\normalfont{\small $^{3}$Department of Clinical Medicine, UiT The Arctic University of Norway}\\
\normalfont{\small $^{4}$Department of Community Medicine, UiT The Arctic University of Norway}\\
\normalfont{\small $^{5}$Department of Medical Biology, UiT The Arctic University of Norway} \\
\normalfont{\small $^{6}$Department of Clinical Pathology, University Hospital of North Norway} %\vspace{2em}
}

\begin{document}
\maketitle

\section*{Abstract}

% \begin{abstract}
% Developing clinically useful cell-level analysis tools in digital pathology remains challenging due to limitations in dataset granularity, inconsistent annotations, computational demands of advanced models, and difficulties in integrating new technologies into clinical workflows. To address these challenges, we propose a multi-faceted solution that enhances data quality, model performance, and usability to create a lightweight and extensible cell segmentation and classification model.

% First, we update data labels by employing a cross-relabeling process that refines the labels of two existing datasets, PanNuke and MoNuSAC, to create a new unified dataset with enhanced granularity, encompassing seven distinct cell types. Second, we leverage the H-Optimus foundation model as a fixed encoder to improve feature representation for simultaneous cell segmentation and classification tasks. Third, to address the computational demands of foundation models, we employ knowledge distillation to reduce model size and complexity while maintaining comparable performance. Finally, to facilitate integration into clinical workflows, we integrate the distilled model into the QuPath software, a widely used open-source platform in digital pathology.

% Our results demonstrate improvements in cell segmentation and classification performance using the H‑Optimus-based model compared to a CNN-based model. Specifically, the average $R^2$ improved from 0.575 to 0.871, and the average $PQ$ score improved from 0.450 to 0.492, indicating better alignment with actual cell counts and enhanced segmentation and classification quality. Furthermore, the distilled student model maintains performance comparable to the larger foundation model while reducing the parameter count by a factor of 48.
% Overall, by reducing computational complexity and integrating it into existing workflows, the proposed approach may significantly impact diagnostic processes, reduce the workload of pathologists, and contribute to improved patient outcomes. Though our approach shows potential enhancements in efficiency and usability of cell segmentation and classification models in digital pathology, extensive validation is needed to deploy these models in clinical practice.
% \end{abstract}

%%% shortened abstract
\begin{abstract}
Developing clinically useful cell-level analysis tools in digital pathology remains challenging due to limitations in dataset granularity, inconsistent annotations, high computational demands, and difficulties integrating new technologies into workflows. To address these issues, we propose a solution that enhances data quality, model performance, and usability by creating a lightweight, extensible cell segmentation and classification model. 

First, we update data labels through cross-relabeling to refine annotations of PanNuke and MoNuSAC, producing a unified dataset with seven distinct cell types. Second, we leverage the H-Optimus foundation model as a fixed encoder to improve feature representation for simultaneous segmentation and classification tasks. Third, to address foundation models' computational demands, we distill knowledge to reduce model size and complexity while maintaining comparable performance. Finally, we integrate the distilled model into QuPath, a widely used open-source digital pathology platform. 

Results demonstrate improved segmentation and classification performance using the H-Optimus-based model compared to a CNN-based model. Specifically, average $R^2$ improved from 0.575 to 0.871, and average $PQ$ score improved from 0.450 to 0.492, indicating better alignment with actual cell counts and enhanced segmentation quality. The distilled model maintains comparable performance while reducing parameter count by a factor of 48. By reducing computational complexity and integrating into workflows, this approach may significantly impact diagnostics, reduce pathologist workload, and improve outcomes. Although the method shows promise, extensive validation is necessary prior to clinical deployment.
\end{abstract}
\clearpage

\section{Introduction}
In digital pathology, accurate segmentation and classification of cells are crucial for many diagnostic, prognostic, and predictive analyses \cite{Jaber_Beziaeva_etal._2019,Lin_Pan_etal._2022,Park_Ock_etal._2022,Shen_Choi_etal._2024}. Nowadays, developments in computational pathology offer multiple solutions \cite{H._Qu_P._Wu_etal._2020,Javed_Mahmood_etal._2020} to utilize cell-level datasets to train machine learning models that solve these problems. The quality and specificity of training datasets are critical for robust and accurate models. Adhering to the principle of "garbage in, garbage out", it is essential to ensure that these datasets are extensively and accurately labeled with distinct classes that reflect the diverse biological characteristics of different cell types. Unfortunately, the number of open-source datasets comprising such high-quality annotations is limited. Existing cell segmentation datasets \cite{Gamper_Koohbanani_etal._2019,Graham_Vu_etal._2019,Verma_Kumar_etal._2021} may offer extensive annotations for certain cell types while providing more general labels for others. For example, in PanNuke, which is one of the largest open-source datasets comprising labeled cells, various types of morphologically and functionally different inflammatory cells like macrophages and lymphocytes are clustered in a broad "inflammatory" class. Consequently, these classes are frequently omitted from analyses or aggregated into broader meta-classes \cite{Gamper_Koohbanani_etal._2020} and likely interfere with other cell classes included in the dataset. This and similar inconsistencies in annotation granularity limit the ability of machine learning models to learn the comprehensive and nuanced features necessary for accurate cell segmentation and classification. To address these challenges, methods for refining and standardizing dataset annotations are essential to enhance the quality of training data.

A complementary approach to mitigate the absence of high-quality training data is the use of foundation models. Foundation models as encoders are defined as large-scale, versatile networks pre-trained on vast, diverse datasets using self-supervised learning, contrasting with convolutional neural network (CNN) pre-trained encoders that rely on supervised learning with labeled data. In practice, foundation models leverage enormous amounts of weakly or unlabeled data from millions of whole slide images (WSIs) and employ self-attention mechanisms to capture long-range dependencies and global context \cite{Chen_Ding_etal._2024,Saillard_Jenatton_etal._2024,Vorontsov_Bozkurt_etal._2024,Xu_Usuyama_etal._2024}. As a consequence, foundation models are able to produce transferable feature representations across different cell types and tissue environments. The feature representations can be leveraged by decoder networks to produce segmentation masks and pixel-level classifications. Because foundation models have comprehensive feature representations, they can be effectively fine-tuned using much smaller amounts of cell-level data compared to the large datasets needed to train models from scratch. Furthermore, foundation models incorporate adversarial training elements or contrastive learning \cite{Chen_Ding_etal._2024,Xu_Usuyama_etal._2024}, enhancing their resilience and adaptability by exposing them to challenging and varied scenarios during training. This may result in more generalizable models, often making them well-suited for diverse and complex tasks in digital pathology.

Despite the inherent advantages of foundation models, their deployment for practical use faces its own obstacles. In particular, they require substantial computational power, financial investments and rigorous testing to ensure reliability and efficacy for a given task \cite{Akkus_Dangott_etal._2022,Dragomir_Cocuz_etal._2022,Go_2022,Jafri_Farooqui_etal._2024}. Moreover, while foundation models enhance feature representation and performance, they depend on the quality of available annotations for decoder fine-tuning and, like any other model, cannot resolve existing inconsistencies or ambiguities in data labels. Therefore, there remains a critical need for solutions that address both data quality and practical deployment considerations.
Further, integrating new technologies into existing clinical workflows often encounters resistance, as it necessitates adjustments to established diagnostic processes. So, there is a need to develop solutions that could be integrated into current practices, minimizing the burden on medical professionals to adopt new tools \cite{King_Williams_etal._2023}.

Existing solutions \cite{Goldsborough_Philps_etal._2024,Hörst_Rempe_etal._2024}, while addressing some aspects of these challenges, fall short in providing a comprehensive approach. To address the data quality and clinical deployment issues, we propose a multi-faceted solution that encompasses data refinement, model optimization, and integration with existing pathology tools (\hyperref[fig:fig1]{Figure 1}). The outcome is a lightweight cell segmentation and classification model that can be integrated into digital pathology workflows for practical clinical use.

\begin{figure}[h!]
    \centering
    \includegraphics[width=\textwidth, height=0.82\textheight, keepaspectratio]{images/Figure_1.pdf}
    \caption{Overview of the proposed solution, including 1) Data refinement using cross-relabeling, 2) Teacher model development and fine tuning, 3) Student model optimization with knowledge distillation and 4) Student model and QuPath integration}
    \label{fig:fig1}
\end{figure}
\clearpage

Our approach begins with preparing the data for the fine-tuning and training of the machine learning models. We create a refined dataset, acquired via cross-relabeling two cell-level datasets, enhancing annotation specificity and consistency of the labeled data. Subsequently, we create a cell segmentation and classification model based on the foundation model. We leverage the foundation model as a fixed encoder and fine-tune a decoder using the refined dataset to improve generalization across diverse tissue- and cell types.
To ensure that the model remains lightweight and deployable in a possibly resource-constrained environment, we employ knowledge distillation to approximate the functionality of the foundation model. Finally, to facilitate the practical application of our model in digital pathology workflows, we integrate it with the QuPath \cite{Bankhead_Loughrey_etal._2017} application. Each methodological component contributes to the overarching goal of enhancing model performance, generalizability, and usability in clinical settings.

The primary contributions of this paper are:
\begin{enumerate}
    \item \textit{Data labels refinement through cross-relabeling:}
    
    We propose a new method for refining labels of cell-level datasets through cross-relabeling. This method employs classification models to re-label broad and ambiguous instances, resulting in a more diverse dataset. Our evaluation demonstrates that these classification models achieve high accuracy on test subsets, indicating the reliability of the method for label refinement.

    \item \textit{Enhanced model performance via foundation models:}
    
    We employ a foundation model as a feature extractor for the cell segmentation and classification task. In comparison with training a CNN model from scratch, the foundation model backbone only needs fine-tuning, which significantly reduces training time, computational resources and data requirements. We show that using a foundation model encoder leads to better performance in cell segmentation and classification networks than using a CNN-based encoder. This improvement may enable the model to generalize more effectively across various tissue types and imaging methods.
    
    \item \textit{Model optimization through knowledge distillation:}
    
    We show that a smaller student model trained using knowledge distillation on the refined dataset obtained via our cross-relabeling approach from a foundation model achieves comparable performance in cell segmentation and quantification tasks. As a result, this model is more suitable for deployment in environments without high-performance computing resources.
    
    \item \textit{Integration with QuPath:}
    
    We integrate the distilled cell segmentation and classification model into QuPath, a widely used open-source digital pathology platform, to accelerate clinical adaptation by enabling pathologists to more easily incorporate advanced computational tools into their existing workflows.
\end{enumerate}

Through these methodological steps, we aim to bridge the gap between advanced machine learning techniques and practical clinical applications, making accurate and efficient digital pathology accessible in a broader range of healthcare settings.

\section{Refining Existing Datasets Using Cross-Relabeling}
To address the limitations of sparse and ambiguous labeling of cell-level datasets, we propose a generalizable cross-relabeling strategy that can be applied to any dataset containing broadly categorized or imprecisely labeled cell types. This approach involves training and subsequently leveraging classification models to refine broad categories into more specific or biologically relevant classes.
When applied to cell-level data, the methodology includes extracting individual cell images from the dataset patches, preprocessing these images to standardize the size and accommodate partial cells, and then training deep learning classifiers capable of distinguishing between the finer cell subtypes within the coarser categories. 
To illustrate our approach, we focus on the PanNuke \cite{Gamper_Koohbanani_etal._2020, Gamper_Koohbanani_etal._2019} and MoNuSAC \cite{Verma_Kumar_etal._2021} datasets that we have used to train models for cell quantification in our previous works \cite{Shvetsov_Grønnesby_etal._2022,Shvetsov_Sildnes_etal._2024}. We find that for better cell differentiation we have to introduce more granular labels. PanNuke includes a broad classification of "inflammatory" cells, encompassing lymphocytes, macrophages, and neutrophils. Each cell type differs significantly in structure, function, and clinical relevance. Conversely, MoNuSAC uses the label "epithelial" for a class that comprises both benign epithelial cells and malignant neoplastic cells. This practice makes it challenging to differentiate between benign and malignant epithelial cells in the dataset, which is a critical distinction when identifying tumor areas within tissue samples. To address these issues, we implement a cross-relabeling strategy as shown in \hyperref[fig:fig2]{Figure 2}. The key components are two classification models: one is trained on singular cell images from PanNuke data to classify the epithelial meta-class into epithelial and neoplastic classes. The other is trained on MoNuSAC to refine the inflammatory class into lymphocytes, neutrophils, and macrophages.

\begin{figure}[h!]
    \centering
    \includegraphics[width=\textwidth]{images/Figure_2.pdf}
    \caption{Refined dataset generation via cross relabeling}
    \label{fig:fig2}
\end{figure}

The refining approach consists of three consecutive steps. The first is the preprocessing step, in which we extract individual cells from both datasets (\hyperref[fig:fig3]{Figure 3}). The specifics of PanNuke and MoNuSAC patch preparation before cell preprocessing are provided in \hyperref[chap:S1]{Appendix S1}.

\begin{figure}[h!]
    \centering
    \includegraphics[width=\textwidth]{images/Figure_3.pdf}
    \caption{Cell instances preprocessing including (1) cell map extraction, (2) bounding box delineation, (3) adjusting cell boxes and (4) cropping and resizing of cell images}
    \label{fig:fig3}
\end{figure}

During preprocessing, we extract cell type maps from the ground truth label mask and calculate bounding boxes around each cell instance. To accommodate partial cells at patch borders, a common issue in cropped patch images, we employ mirror padding and extend the field of view of the cell label by 15 pixels to capture adjacent cells. We then crop and resize the identified regions to $64 \times 64$ pixels using bicubic interpolation.

The preprocessed PanNuke dataset comprises 68,031 neoplastic and 23,207 epithelial cell images, while MoNuSAC comprises  33,104 lymphocytes, 1,252 neutrophils, and 1,695 macrophages, which we subsequently use in training cell classification models and classifying the cell image data \hyperref[fig:S2]{Appendix Figure S2 (1)}. 

The next step is to train two distinct ResNet50-based classifiers tailored to address the specific labeling challenges inherent in each dataset. We use ResNet50 for classification models due to its proven effectiveness for image classification tasks in histopathology \cite{pan2022reviewmachinelearningapproaches}, and its compatibility with small images. For the PanNuke dataset, we design the classifier, trained on MoNuSAC data, to disaggregate the heterogeneous "inflammatory" cell category into distinct subtypes: lymphocytes, macrophages, and neutrophils. Similarly, for the MoNuSAC dataset, the classifier is trained on PanNuke data and distinguishes between benign and malignant epithelial cells within the overarching "epithelial" label. By applying these targeted classifiers to their respective datasets, we assign more specific labels to individual cell instances, thus enabling us to create a unified dataset.
To ensure a balanced representation of classes, we train both models on datasets that had been equalized to match the size of the least represented class. Thus, we obtain datasets comprising 23,207 samples per class for PanNuke and 1,252 samples per class for MoNuSAC data. Next, we partition both of them into training (70\%), validation (20\%), and testing (10\%) subsets. To mitigate the risk of overfitting, we use a single dropout layer with a rate of p=0.5 in both models and data augmentation using randomized color perturbations, rotation, and horizontal and vertical flipping. We employ AdamW optimizer and the cross-entropy loss function for the training criterion.

To evaluate the two trained models, we measure the classification accuracy on the respective test subsets. The accuracies on the test subset for both classifiers are presented in \hyperref[tab:1]{Table 1}. The PanNuke model achieves an average accuracy of 93.57\%, with higher accuracy for neoplastic cells (96.06\%) compared to epithelial cells (86.26\%). The confusion matrix in Figure A3.1 shows that the model predominantly distinguishes accurately between epithelial and neoplastic tissues, with a substantial number of correct classifications and relatively few misclassifications. The MoNuSAC model demonstrates an average accuracy of 98.92\%, excelling in classifying lymphocytes (99.67\%) and macrophages (94.12\%), with lower performance for neutrophils (85.71\%). The confusion matrix in Figure A3.2 shows that the model identifies lymphocytes and performs reasonably well with macrophages and neutrophils.

\begin{table}[h!]
\renewcommand{\arraystretch}{1.5}
  \centering
  \caption{Cell classification results for PanNuke and MoNuSAC trained models (CI 95\%).}
  \label{tab:1}
  \begin{tabular}{|l|c|c|}
   \hline
   %\rowcolor{gray!30}
    Accuracy               & PanNuke model              & MoNuSAC model              \\
    \hline
    Average      & 0.936 (0.931--0.941)         & 0.989 (0.986--0.993)        \\
    \hline
    Neoplastic   & 0.961 (0.956--0.965)         & -                          \\
    \hline
    Epithelial   & 0.863 (0.849--0.877)         & -                          \\
    \hline
    Lymphocytes  & -                          & 0.997 (0.995--0.999)        \\
    \hline
    Neutrophils  & -                          & 0.857 (0.796--0.918)        \\
    \hline
    Macrophages  & -                          & 0.941 (0.906--0.976)        \\
    \hline
  \end{tabular}
\end{table}

Finally, during the last step, we use the model trained on PanNuke data for epithelial cells in MoNuSAC and the model trained on MoNuSAC for the inflammatory cells class in PanNuke. Specifically, we use classifier models to relabel epithelial cells in MoNuSAC and inflammatory cells in PanNuke data. Then we combine cells with refined labels and the rest of the cells in both datasets to create a refined dataset (\hyperref[fig:S2]{Appendix Figure S2 (2)}). The process of relabeling cells and visualizing them on a patch is shown in \hyperref[fig:fig4]{Figure 4}. The cell counts in the refined dataset are provided in \hyperref[tab:S4]{Appendix Table S4}.

\begin{figure}[h!]
    \centering
    \includegraphics[width=\textwidth, height=0.42\textheight, keepaspectratio]{images/Figure_4.pdf}
    \caption{Cell relabeling procedure for epithelial and inflammatory cell classes}
    \label{fig:fig4}
\end{figure}

%\hfill

Relabeling and combining datasets have been explored in a prior study \cite{Parulekar_Kanwat_etal._2023}, where consecutive fine-tuning on multiple datasets was employed to account for hierarchical class label structures. While the method presented in \cite{Parulekar_Kanwat_etal._2023} is intuitive, it often lacks consistency and requires multiple fine-tuning runs, which can be cumbersome and time-consuming. 
In contrast, cross-relabeling simplifies this process by using specialized classification models tailored to each dataset's specific labeling challenges. This approach provides better transparency and produces a unified dataset encompassing seven distinct cell types across multiple tissue samples, enhancing data diversity for further model training or fine-tuning.

Despite these improvements, cross-relabeling does not entirely resolve issues related to poor labeling quality or the amount of labeled data. Specifically, our results show lower accuracies persist for underrepresented classes, such as macrophages, which may stem from a limited sample availability and intrinsic challenges in distinguishing these cells based solely on H\&E staining. Furthermore, while our method enhances label specificity, it relies on the initial quality of the broad labels; thus, any fundamental inaccuracies in the original annotations can propagate through the relabeling process. Addressing the overall problem of limited data labels may require integrating additional data sources or utilizing complementary immunohistochemical staining methods.
Although the reported performance metrics are obtained from evaluations on the native test sets of each dataset, it is important to note that the primary application of these classifiers is to perform cross-relabeling, where a model trained on one dataset (e.g., PanNuke) is applied to another (e.g., MoNuSAC) and vice versa. We acknowledge that a more systematic evaluation of cross-dataset generalization is needed and could be performed in future work.

Overall, the refined dataset produced by our approach can enhance the supervised training or fine-tuning of cell segmentation and classification models, especially those that utilize pre-trained foundation models to improve feature extraction robustness. In addition, these models can detect nuanced classes that enable researchers to conduct more detailed analyses of biological processes in computational pathology.

\section{Foundation models for robust cell segmentation and classification}

Accurate cell segmentation and classification in digital pathology are hindered by limited labeled data and the fact that conventional CNNs are unable to capture global contextual information due to their local receptive field constraints \cite{Gheflati_Rivaz_2022,Yang_Marcus_etal.}. Traditional approaches in cell quantification have predominantly relied on CNN encoders, such as ResNet50, given their proven effectiveness in semantic segmentation tasks \cite{Deshmane_2023,Graham_Vu_etal._2019,Mukasheva_Koishiyeva_etal._2024,Stringer_Wang_etal._2021}. However, approaches that include fine-tuning of pretrained CNNs, data augmentation, and stain normalization to partially increase data variability and address staining differences often fail to achieve the necessary generalization and robustness across diverse tissue types and staining conditions \cite{G._Wang_W._Li_etal._2018,Gao_Bagci_etal._2018,Karim_El_Khoury_Martin_Fockedey_etal._2021}.

To overcome these challenges, we leverage an encoder-decoder network that uses a foundation model as the encoder and a CNN upsampling decoder (\hyperref[fig:fig5]{Figure 5}) for simultaneous cell segmentation and classification in 2D patches extracted from WSIs. Foundation models with transformer-based architectures are viable alternatives to CNN-based encoders \cite{Shamshad_Khan_etal._2023,Sourget_2023}. They enable the creation of more advanced architectures that can decode or transform learned features more effectively \cite{Chen_Duan_etal._2023,Cheng_Misra_etal._2022,Xie_Wang_etal._2021}.

\begin{figure}[h!]
    \centering
    \includegraphics[width=\textwidth]{images/Figure_5.pdf}
    \caption{UNETR-like model with foundational model as backbone}
    \label{fig:fig5}
\end{figure}

By utilizing a transformer-based encoder, we incorporate global contextual information into the feature extraction process, which is a key advantage of such architectures \cite{Chen_Lu_etal._2021}. This foundation model integration facilitates accurate pixel-wise segmentation and classification without the need for extensive encoder training, thereby potentially improving generalization across varied cellular structures and tissue types.
In our implementation, we employ a modified UNETR \cite{Hatamizadeh_Tang_etal._2021} architecture that combines a vision transformer (ViT) \cite{Dosovitskiy_Beyer_etal._2021} encoder with a CNN-based decoder. The encoder utilizes the pretrained H-Optimus foundation model, which contains 1.1 billion parameters and is trained on over 500,000 H\&E stained WSIs \cite{Saillard_Jenatton_etal._2024}. We extract outputs from four evenly spaced transformer blocks $Z_i$, where $i \in [1, 14, 26, 38]$, to serve as residual connections for the CNN decoder. We select these blocks based on our observation that features from non-adjacent levels of the encoder lead to better overall performance on the test subset.

The CNN decoder upsamples the feature representations, acquired from the transformer blocks, to generate an intermediate vector that is handled by two task-specific layers that generate cell segmentation and classification masks. The first task-specific layer is the ‘Cellpose head’,  which is used to delineate cell instances. The layer generates horizontal and vertical gradient maps to form vector fields that are refined through gradient tracking in a post-processing step using the Cellpose algorithm \cite{Stringer_Wang_etal._2021}, known for its efficacy in cell segmentation tasks and generalizability across multiple domains \cite{Pachitariu_Stringer_2022,Stringer_Pachitariu_2024}. The second task-specific layer is the "Cell type head", which assigns labels to individual pixels. In the post-processing step, we determine the output classification label of each segmented cell instance by majority voting over the labeled pixels that comprise the cell in the segmentation map.

To evaluate model performance and measure the impact of adding a foundation model as backbone, we compare it to a ResNet50-based model. ResNet50 is a widely used solution for encoders in segmentation architectures in the medical domain \cite{Deshmane_2023,Graham_Vu_etal._2019,Mukasheva_Koishiyeva_etal._2024,Stringer_Wang_etal._2021}. For the H-Optimus-based model, we utilize frozen weights for the encoder and only fine-tune the decoder to take advantage of the extensive pre-training of the foundation model. For the ResNet50-based model we start with ImageNet \cite{Deng_Dong_etal.} weights and train both encoder and decoder parts. Hyperparameters for the training step are set to be identical, where possible, for comparable evaluation. 
For this evaluation, we deliberately use the PanNuke dataset to provide a standardized and controlled comparison between the H‑Optimus and ResNet50-based models (\hyperref[fig:S2]{Appendix Figure S2 (3)}). Specifically, we use two of the default PanNuke dataset splits (66\%) for training and validation, and reserve the third split (33\%) for testing.

To address the challenge of cell class imbalance in the PanNuke dataset, which is a common characteristic in most cell-level H\&E patch datasets, both models’ training processes employ a weighted loss function comprising cross-entropy and focal loss \cite{Lin_Goyal_etal._2018}. The focal loss component is adjusted with coefficients derived from each cell class' instance frequency, emphasizing learning from underrepresented classes and enhancing the model's sensitivity to rare but significant cellular patterns. The cross-entropy loss is augmented with spectral decoupling regularization \cite{Pezeshki_Kaba_etal._2021,Pohjonen_Stürenberg_etal._2022} and spatially varying label smoothing \cite{Islam_Glocker_2021}, which potentially stabilizes training and improves generalization in case of complex tissue morphologies. For optimization, we employ the \textit{AdamW} \cite{Loshchilov_Hutter_2019} to counter unbalanced class scenarios, with cosine annealing learning rate scheduler.

We utilize the scikit-learn library \cite{Van_der_Walt_Schönberger_etal._2014} and HoVer-Net \cite{Graham_Vu_etal._2019} implementations of $R^2$ (the coefficient of determination) and $PQ$ (panoptic quality) to evaluate our experiments. Complete mathematical formulations and detailed explanations of these metrics are provided in \hyperref[chap:S5]{Appendix S5}. To compute confidence intervals, we use nonparametric bootstrapping, where after calculating the metric on the full sample, we generated 1000 bootstrap replicates by resampling with replacement and then determined the 95\% confidence intervals as the 2.5th and 97.5th percentiles of the resulting empirical distribution.

%\hfill

The model comparisons are summarized in \hyperref[tab:2]{Table 2}. The H‑Optimus-based model achieves higher $R^2$ across all cell classes compared to the ResNet50-based model, which means that its predictions are more closely aligned with the PanNuke cell counts, indicating a stronger correlation with the observed data. Notably, the improvement of $R^2_{dead}$ may be an indicator of better global contextual representations provided by the foundation model backbone. In terms of segmentation and classification quality combined, measured by the PQ score, the H‑Optimus-based model demonstrates notable improvements across most cell classes. Overall, the average $R^2$ improved from 0.575 to 0.871, while the average $PQ$ score improved from 0.450 to 0.492, demonstrating better performance of the H-Optimus-based model.

\begin{table}[h!]
\renewcommand{\arraystretch}{1.5}
  \centering
  \caption{Cell quantification metrics for baseline and proposed models (CI 95\%).}
  \label{tab:2}
  \begin{tabular}{|l|c|c|}
    \hline
    %\rowcolor{gray!30}
    Metric             & Resnet50-based            & H-optimus-based              \\
    \hline
    $R^2_{neoplastic}$    & 0.681 (0.576--0.769)       & \textbf{0.941 (0.917--0.960)} \\
    \hline
    $R^2_{inflammatory}$  & 0.863 (0.778--0.903)       & \textbf{0.949 (0.918--0.966)} \\
    \hline
    $R^2_{connective}$    & 0.600 (0.488--0.698)       & 0.609 (0.436--0.772)          \\
    \hline
    $R^2_{dead}$          & 0.097 (-11.389--0.669)     & 0.925 (0.404--0.982)          \\
    \hline
    $R^2_{epithelial}$    & 0.635 (0.490--0.747)       & \textbf{0.930 (0.886--0.964)} \\
    \hline
    $PQ_{neoplastic}$       & 0.517 (0.499--0.535)       & \textbf{0.589 (0.575--0.604)} \\
    \hline
    $PQ_{inflammatory}$     & 0.455 (0.429--0.482)       & \textbf{0.528 (0.507--0.549)} \\
    \hline
    $PQ_{connective}$       & 0.416 (0.400--0.431)       & \textbf{0.451 (0.436--0.465)} \\
    \hline
    $PQ_{dead}$             & 0.374 (0.342--0.408)       & 0.292 (0.209--0.365)          \\
    \hline
    $PQ_{epithelial}$       & 0.488 (0.460--0.519)       & \textbf{0.599 (0.579--0.618)} \\
    \hline
  \end{tabular}
\end{table}

Our results  show that integrating the H‑Optimus foundation model within the UNETR architecture enhances the model's ability to segment and classify cells across diverse tissues from PanNuke data. The pretrained transformer encoder provides robust feature representations, resulting in higher average $R^2$ and $PQ$ scores compared to the CNN-based model. This leads to more reliable cell quantification and more accurate downstream analysis. Additionally, the streamlined fine-tuning process reduces computational overhead and training time, making the model more adaptable for new data.

Despite these advancements, the foundation model-based approach does not fully resolve all challenges related to cell segmentation and classification. We observe lower metric scores for underrepresented classes in the training data. Furthermore, foundation models typically encompass billions of parameters, resulting in substantial computational and memory requirements. It therefore poses challenges for deployment in resource-constrained environments, limiting their practical applicability in certain clinical settings.

\section{Model optimization via Knowledge Distillation}

To address the limitations posed by the extensive size of foundation models, we implement knowledge distillation — a model compression technique that leverages the teacher-student paradigm \cite{Hinton_Vinyals_etal._2015}. By training a smaller, more efficient student model to replicate the output of a larger, pre-trained teacher model, we retain performance while significantly reducing the model's complexity and resource requirements (\hyperref[fig:fig6]{Figure 6}).

\begin{figure}[h!]
    \centering
    \includegraphics[width=\textwidth, height=0.45\textheight, keepaspectratio]{images/Figure_6.pdf}
    \caption{Knowledge distillation framework for training a student model using a pre-trained teacher}
    \label{fig:fig6}
\end{figure}

We employ knowledge distillation to compress the H‑Optimus-based teacher model into a more efficient student model. The teacher model is the modified UNETR architecture with the H‑Optimus foundation model described in the previous chapter. The student model is based on a UNet architecture augmented with residual connections and incorporates a smaller ViT encoder with 9 million parameters \cite{Steiner_Kolesnikov_etal._2022,Wightman_2019}. 

First, we fine-tune the teacher model using the refined dataset from the cross-relabeling procedure (Section 2). Initially we train the decoder of the teacher model while keeping the encoder weights frozen. We split the refined dataset into train (70\%), validation (20\%) and test (10\%) subsets (\hyperref[fig:S2]{Appendix Figure S2 (4)}). During fine-tuning, we use the train and validation subsets, while leaving the test subset for model evaluation. We set the training procedure and model hyperparameters to be identical to those that were used to demonstrate the utility of foundation models for the simultaneous cell segmentation and classification task.

Next, we perform knowledge distillation from teacher to student using the refined dataset used to fine-tune the teacher model. The student model is trained to replicate the teacher model's outputs. We utilize a specialized loss function that aligns the student's predicted probability distribution with the teacher's, incorporating the teacher's class probability distribution derived from the output. Following the methodology of Hinton et al. \cite{Hinton_Vinyals_etal._2015}, we experiment with various hyperparameter settings for the temperature ($T$) and the balancing coefficients ($\alpha$ and $\beta$) in the loss function. We vary $T$ from 1 to 20 and adjust $\alpha$ and $\beta$ to balance the distillation and student losses. Through iterative tuning and evaluation, we identify that setting $T=14$, $\alpha=0.3$, and $\beta=0.7$ yields a configuration that converges and closely approximates the teacher model's performance during training.

Finally, we assess the performance of both models using the $R^2$ and $PQ$ (defined in \hyperref[chap:S5]{Appendix S5}) on the test set of the refined dataset (\hyperref[tab:3]{Table 3}). We observe that the 95\% confidence intervals overlap for most cell types, so we cannot claim statistically significant performance differences between the teacher and student models. One exception appears in the neoplastic class. The teacher model produces an $R^2$ of 0.919, while the student model shows an $R^2$ of 0.852. In addition, the student model achieves higher $PQ$ values for the neoplastic and connective classes, though the confidence intervals show overlap.

\begin{table}[h!]
\renewcommand{\arraystretch}{1.5}
  \centering
  \caption{Cell quantification metrics for teacher and distilled student models (CI 95\%).}
  \label{tab:3}
  \begin{tabular}{|l|c|c|}
    \hline
    %\rowcolor{gray!30}
    Metric & Teacher & Student \\
    \hline
    $R^2_{neoplastic}$    & \textbf{0.919} (0.898--0.939) & 0.852 (0.800--0.891) \\
    \hline
    $R^2_{lymphocyte}$    & 0.969 (0.956--0.977)         & 0.969 (0.956--0.978) \\
    \hline
    $R^2_{connective}$    & 0.694 (0.548--0.809)         & 0.618 (0.469--0.741) \\
    \hline
    $R^2_{dead}$          & 0.755 (0.400--0.908)         & 0.424 (0.100--0.731) \\
    \hline
    $R^2_{epithelial}$    & 0.922 (0.870--0.958)         & 0.843 (0.738--0.917) \\
    \hline
    $R^2_{macrophage}$    & 0.384 (-0.369--0.724)        & 0.704 (0.352--0.859) \\
    \hline
    $R^2_{neutrofil}$     & 0.854 (0.578--0.929)         & 0.833 (0.502--0.925) \\
    \hline
    $PQ_{neoplastic}$       & 0.581 (0.569--0.593)         & 0.601 (0.588--0.613) \\
    \hline
    $PQ_{lymphocyte}$       & 0.536 (0.520--0.553)         & 0.563 (0.544--0.579) \\
    \hline
    $PQ_{connective}$       & 0.436 (0.421--0.451)         & 0.457 (0.441--0.474) \\
    \hline
    $PQ_{dead}$             & 0.272 (0.235--0.315)         & 0.279 (0.201--0.369) \\
    \hline
    $PQ_{epithelial}$       & 0.522 (0.500--0.545)         & 0.530 (0.506--0.555) \\
    \hline
    $PQ_{macrophage}$       & 0.524 (0.459--0.588)         & 0.474 (0.405--0.543) \\
    \hline
    $PQ_{neutrofil}$        & 0.541 (0.490--0.592)         & 0.565 (0.522--0.607) \\
    \hline
  \end{tabular}
\end{table}


We further decompose the $PQ$ metric into its $SQ$ and $DQ$ components (\hyperref[tab:S6]{Appendix Table S6}). Both models produce nearly identical $SQ$ values, which indicates that they predict instance boundaries with similar precision. Although the student model shows some improvement in $DQ$ scores for certain classes, the confidence intervals overlap and do not confirm a statistically significant difference.

We observe that the student and teacher models yield comparable detection performance despite the student model using a much smaller and simpler architecture. A model with fewer parameters reduces the risk of overfitting when training data are scarce relative to the model’s complexity \cite{Farias_Ludermir_etal._2022}. The knowledge distillation process also encourages the student model to focus on the most generalizable detection features learned from the teacher. These factors enable the student model to achieve similar detection performance across different cell types.

Additionally, considering the model sizes reported in \hyperref[tab:4]{Table 4}, the distilled model achieves a significant reduction compared to the teacher model, with a 48-fold decrease in parameter count and a 5.5-fold reduction in on-disk size. In inference mode, the teacher model requires 16 GB of VRAM for a batch size of 32, while the distilled model only needs 3 GB of VRAM for the same batch size. These reductions make the distilled model significantly more practical for fine-tuning and deployment in resource-constrained environments.

\begin{table}[h!]
\renewcommand{\arraystretch}{1.5}
  \centering
  \caption{Parameter counts and size of teacher and distilled model}
  \label{tab:4}
  \adjustbox{max width=\textwidth}{%
  \begin{tabular}{|l|c|c|c|}
    \hline
    %\rowcolor{gray!30}
    Metric & H-optimus-based (Teacher) & mobileViT-based (Student) & Magnitude of difference \\
    \hline
    Parameters count       & 1,158,917,906   & \textbf{24,093,393}   & \textbf{48x}  \\
    \hline
    Estimated Total Size (MB) & 87,912       & \textbf{15,935}    & \textbf{5.5x} \\
    \hline
  \end{tabular}%
}
\end{table}

%\hfill

With recent advancements in complex network architectures and the use of pretrained encoders to achieve state-of-the-art performance \cite{Baumann_Dislich_etal._2024,Hörst_Rempe_etal._2024} in cell segmentation and classification tasks, model size, computational complexity, and processing times have increased. This limits the scalability and accessibility of these models. As we demonstrate, this may be mitigated using knowledge distillation. Studies in the field of natural language processing have demonstrated the efficacy of knowledge distillation in retaining the capabilities of the teacher model while achieving significant reductions in size and complexity \cite{Huangpu_Gao_2024,Sun_Yu_etal.}. 

We demonstrate the feasibility of knowledge distillation in digital pathology, specifically for cell segmentation and classification tasks. Moreover, we achieve this performance while also significantly reducing the parameter count. In addressing the challenge of knowledge transfer, we found that distillation from a transformer-based model to a smaller transformer is more straightforward than attempting to map transformer features to CNN blocks. In our experiments, using a CNN-based network as a student results in worse cell quantification performance due to the structural constraints of CNN feature space dimensions. 

Although our primary approach relies on a transformer-based student model that performs well, it can be further optimized to incorporate advantages from CNN architectures. For example, employing alternative techniques such as using ViT adapters \cite{Chen_Duan_etal._2023} or $1 \times 1$ convolutions to adjust feature map sizes may be beneficial for harnessing CNN advantages like enhanced local feature extraction. Moreover, if additional performance improvements are desired, the process can be further enhanced by applying supplementary knowledge distillation techniques, such as self-distillation \cite{Zhang_Song_etal._2019} or online distillation \cite{Houyon_Cioppa_etal._2023}.

Despite these promising results, further validation on independent datasets is necessary to fully understand the model's limitations. Underrepresented classes may pose challenges when addressing complex cases. Pathologists need to validate these models to adopt them in clinical settings. While the distilled models are smaller and more deployable, a technological gap persists because pathologists traditionally rely on established methods for inspecting WSIs and diagnosing diseases. Addressing the complexities involved in deploying models for inference and supporting pathologists in adopting new tools is essential for integrating these models into clinical workflows.

\section{Model integration with QuPath}
Digital pathology tools with graphical user interfaces are essential for visualizing and analyzing WSIs. To make our student model useful in clinical pathology workflows, it needs to be integrated into a tool that enables inspecting regions, creating annotations, and providing quantitative analyses of biomarkers. Therefore, we integrate the trained student model from the previous chapter into the QuPath open‑source platform \cite{Bankhead_Loughrey_etal._2017}. QuPath provides the required annotation, visualization, and analysis tools to interpret complex histological data, including workflows for cell segmentation, classification, and quantification (\hyperref[fig:fig7]{Figure 7}). 

\begin{figure}[h!]
    \centering
    \includegraphics[width=\textwidth]{images/Figure_7.pdf}
    \caption{Visualization of model-generated cell quantification annotations (left) and the corresponding unannotated slide (right) in QuPath}
    \label{fig:fig7}
\end{figure}

To identify the regions in a WSI critical for prognosticating tumor development, such as specific tumor areas or border regions without overlapping healthy tissue, the pathologist uses QuPath to outline these regions. Then, the pathologist initiates a cell segmentation and classification script through the QuPath interface for the selected regions. The resulting annotations and quantified cell information are then directly overlaid onto the WSI in the QuPath interface. Additional design and implementation details are in \hyperref[chap:S7]{Appendix S7}. 

Two common approaches for integrating deep learning models into QuPath are Java‑based native QuPath extensions \cite{Goldsborough_Philps_etal._2024} and the execution of RESTful API requests to a model server coupled with handling the response via an extension, as demonstrated in the application of cell segmentation models applied to immunofluorescence images \cite{Sugawara_2023}. While the community is actively working on these integration strategies, there is currently no universal solution that fully addresses all integration and performance requirements.

Extensions may offer better integration with QuPath, allowing slightly improved performance and more widespread usage of the built-in QuPath models, but they lack the flexibility to customize models and modify their behavior. For example, the newest version of QuPath includes models such as StarDist \cite{Weigert_Schmidt} and InstanSeg \cite{Goldsborough_Philps_etal._2024} that can perform cell segmentation. Both models pose limitations when applied to simultaneous cell segmentation and classification. StarDist performs well only on convex, round shapes by design, whereas some neoplastic, inflammatory, and connective cells exhibit complex and non-convex shapes. InstanSeg provides only semantic segmentation without assigning classes to the segmented cells.

%\hfill

In contrast, our approach offers an alternative integration strategy. It utilizes the paquo library to directly interact with QuPath’s internal application programming interface from within Python. This enables data exchange and processing without the need for intermediate conversion steps and provides greater control over model customization, retraining, and the incorporation of custom processing steps.

The integration of our custom model with QuPath underscores its potential to significantly enhance the diagnostic process by reducing the time burden on pathologists and enabling them to focus on more complex interpretative tasks using familiar software. Leveraging a tool that is already well-established among pathologists increases the likelihood of its adoption into daily clinical workflows. The quantitative data generated through the automated workflow is critical for both clinical decision-making and research, facilitating more accurate biomarker analysis, enabling robust statistical evaluations, and supporting hypothesis generation and testing. Additionally, by streamlining cell segmentation and classification, the tool enhances the scalability and reproducibility of pathological assessments, ultimately contributing to improved diagnostic accuracy and patient outcomes.

\section{Conclusion and future work}

In this study, we address critical challenges in digital pathology and tackle the usability and deployment issues of the developed models in standard computing environments without the need for high-performance computing systems. Our multi-faceted approach encompasses data refinement through cross-relabeling, leveraging foundation models for robust cell segmentation and classification, optimizing model performance via knowledge distillation, and integrating the optimized model into the QuPath software for practical application. This approach is used to construct a capable, versatile, and adjustable model for cell segmentation and classification, with enhanced performance and usability.

\begin{sloppypar}
While our approach shows potential in the field of computational pathology, certain limitations persist. 
For example, our implementation currently exhibits lower performance in detecting macrophages. 
This serves as an instance of the broader challenge of accurately identifying complex cell types. In order to address this issue, extending our approach to incorporate additional data sources, exploring alternative modeling approaches, and integrating other imaging modalities such as immunohistochemical staining may help improve detection accuracy. Moreover, although the distilled model reduces computational demands, integrating advanced deep learning models into clinical practice requires addressing technological gaps and potential resistance to adopting new tools within established diagnostic processes.
\end{sloppypar}

Future work could focus on several key areas to refine the proposed approach and facilitate its adoption in clinical environments. Enhancing the cell-relabeling process with additional datasets \cite{Graham_Jahanifar_etal._2021} could improve the representation of underrepresented cell types and enhance overall model performance. Also, incorporating additional data sources, such as multi-modal imaging or complementary staining methods, may address limitations related to cell type differentiation and class imbalance. Exploring other foundation models \cite{Vorontsov_Bozkurt_etal._2024,Zimmermann_Vorontsov_etal._2024} or introducing additional modalities \cite{Ding_Wagner_etal._2024,Vaidya_Zhang_etal._2025} may provide alternative architectures better suited to specific tasks or offer improved efficiency. Implementing more complex knowledge distillation techniques \cite{Houyon_Cioppa_etal._2023,Zhang_Song_etal._2019} could further optimize the model's performance and adaptability. Additionally, deeper integration with QuPath or other digital pathology software could provide pathologists more control over cell quantification analysis directly within the QuPath interface, thereby increasing accessibility and usability. Such enhancements would not only refine model performance but also ensure greater adaptability and scalability within various clinical environments. Finally, extensive validation of the model by pathologists and benchmarking against independent datasets are essential steps toward establishing the model's reliability and fostering confidence in its clinical utility.

\section*{Acknowledgments} 
This work was funded in part by the Research Council of Norway grant no. 309439 SFI Visual Intelligence, and the North Norwegian Health Authority grant no. HNF1521-20.

\bibliographystyle{IEEEtran}
\begin{sloppypar}
\begin{thebibliography}{99}

\bibitem{chaplot2020neural} Chaplot, Devendra Singh, et al. "Neural topological slam for visual navigation." Proceedings of the IEEE/CVF conference on computer vision and pattern recognition. 2020.

\bibitem{maksymets2021thda} Maksymets, Oleksandr, et al. "Thda: Treasure hunt data augmentation for semantic navigation." Proceedings of the IEEE/CVF International Conference on Computer Vision. 2021.

\bibitem{mezghan2022memory} Mezghan, Lina, et al. "Memory-augmented reinforcement learning for image-goal navigation." 2022 IEEE/RSJ International Conference on Intelligent Robots and Systems (IROS). IEEE, 2022.

\bibitem{al2022zero} Al-Halah, Ziad, Santhosh Kumar Ramakrishnan, and Kristen Grauman. "Zero experience required: Plug \& play modular transfer learning for semantic visual navigation." Proceedings of the IEEE/CVF Conference on Computer Vision and Pattern Recognition. 2022.

\bibitem{ye2021auxiliary} Ye, Joel, et al. "Auxiliary tasks and exploration enable objectgoal navigation." Proceedings of the IEEE/CVF international conference on computer vision. 2021.

\bibitem{chaplot2020object} Chaplot, Devendra Singh, et al. "Object goal navigation using goal-oriented semantic exploration." Advances in Neural Information Processing Systems 33 (2020)

\bibitem{ramakrishnan2022poni} Ramakrishnan, Santhosh Kumar, et al. "Poni: Potential functions for objectgoal navigation with interaction-free learning." Proceedings of the IEEE/CVF Conference on Computer Vision and Pattern Recognition. 2022.

\bibitem{ramrakhya2022habitat} Ramrakhya, Ram, et al. "Habitat-web: Learning embodied object-search strategies from human demonstrations at scale." Proceedings of the IEEE/CVF Conference on Computer Vision and Pattern Recognition. 2022.

\bibitem{mousavian2019visual} Mousavian, Arsalan, et al. "Visual representations for semantic target driven navigation." 2019 International Conference on Robotics and Automation (ICRA). IEEE, 2019.

\bibitem{dhariwal2021diffusion} Dhariwal, Prafulla, and Alexander Nichol. "Diffusion models beat gans on image synthesis." Advances in neural information processing systems 34 (2021)

\bibitem{ho2022classifier} Ho, Jonathan, and Tim Salimans. "Classifier-free diffusion guidance." arXiv preprint arXiv:2207.12598 (2022).

\bibitem{nichol2021glide} Nichol, Alex, et al. "Glide: Towards photorealistic image generation and editing with text-guided diffusion models." arXiv preprint arXiv:2112.10741 (2021)

\bibitem{brooks2023instructpix2pix} Brooks, Tim, Aleksander Holynski, and Alexei A. Efros. "Instructpix2pix: Learning to follow image editing instructions." Proceedings of the IEEE/CVF Conference on Computer Vision and Pattern Recognition. 2023.

\bibitem{fu2023guiding} Fu, Tsu-Jui, et al. "Guiding instruction-based image editing via multimodal large language models." arXiv preprint arXiv:2309.17102 (2023).

\bibitem{geng2024instructdiffusion} Geng, Zigang, et al. "Instructdiffusion: A generalist modeling interface for vision tasks." Proceedings of the IEEE/CVF Conference on Computer Vision and Pattern Recognition. 2024.

\bibitem{zhou2024minedreamer} Zhou, Enshen, et al. "Minedreamer: Learning to follow instructions via chain-of-imagination for simulated-world control." arXiv preprint arXiv:2403.12037 (2024).

\bibitem{zhou2023esc} Zhou, Kaiwen, et al. "Esc: Exploration with soft commonsense constraints for zero-shot object navigation." International Conference on Machine Learning. PMLR, 2023.

\bibitem{yu2023l3mvn} Yu, Bangguo, Hamidreza Kasaei, and Ming Cao. "L3mvn: Leveraging large language models for visual target navigation." 2023 IEEE/RSJ International Conference on Intelligent Robots and Systems (IROS). IEEE, 2023.

\bibitem{gadre2023cows} Gadre, Samir Yitzhak, et al. "Cows on pasture: Baselines and benchmarks for language-driven zero-shot object navigation." Proceedings of the IEEE/CVF Conference on Computer Vision and Pattern Recognition. 2023.

\bibitem{shah2023navigation} Shah, Dhruv, et al. "Navigation with large language models: Semantic guesswork as a heuristic for planning." Conference on Robot Learning. PMLR, 2023.

\bibitem{cai2024bridging} Cai, Wenzhe, et al. "Bridging zero-shot object navigation and foundation models through pixel-guided navigation skill." 2024 IEEE International Conference on Robotics and Automation (ICRA). IEEE, 2024.

\bibitem{yu2023co} Yu, Bangguo, Hamidreza Kasaei, and Ming Cao. "Co-NavGPT: Multi-robot cooperative visual semantic navigation using large language models." arXiv preprint arXiv:2310.07937 (2023).

\bibitem{wu2024voronav} Wu, Pengying, et al. "Voronav: Voronoi-based zero-shot object navigation with large language model." arXiv preprint arXiv:2401.02695 (2024).

\bibitem{qin2023mp5} Qin, Yiran, et al. "Mp5: A multi-modal open-ended embodied system in minecraft via active perception." arXiv preprint arXiv:2312.07472 (2023).

\bibitem{du2024learning} Du, Yilun, et al. "Learning universal policies via text-guided video generation." Advances in Neural Information Processing Systems 36 (2024).

\bibitem{ajay2024compositional} Ajay, Anurag, et al. "Compositional foundation models for hierarchical planning." Advances in Neural Information Processing Systems 36 (2024).

\bibitem{liang2024skilldiffuser} Liang, Zhixuan, et al. "Skilldiffuser: Interpretable hierarchical planning via skill abstractions in diffusion-based task execution." Proceedings of the IEEE/CVF Conference on Computer Vision and Pattern Recognition. 2024.

\bibitem{heusel2017gans} Heusel, Martin, et al. "Gans trained by a two time-scale update rule converge to a local nash equilibrium." Advances in neural information processing systems 30 (2017).

\bibitem{zhang2018unreasonable} Zhang, Richard, et al. "The unreasonable effectiveness of deep features as a perceptual metric." Proceedings of the IEEE conference on computer vision and pattern recognition. 2018.

\bibitem{brown2020language} Brown, Tom B. "Language models are few-shot learners." arXiv preprint arXiv:2005.14165 (2020).

\bibitem{podell2023sdxl} Podell, Dustin, et al. "Sdxl: Improving latent diffusion models for high-resolution image synthesis." arXiv preprint arXiv:2307.01952 (2023).

\bibitem{brohan2022rt} Brohan, Anthony, et al. "Rt-1: Robotics transformer for real-world control at scale." arXiv preprint arXiv:2212.06817 (2022).

\bibitem{brohan2023rt} Brohan, Anthony, et al. "Rt-2: Vision-language-action models transfer web knowledge to robotic control." arXiv preprint arXiv:2307.15818 (2023).

\bibitem{li2024manipllm} Li, Xiaoqi, et al. "Manipllm: Embodied multimodal large language model for object-centric robotic manipulation." Proceedings of the IEEE/CVF Conference on Computer Vision and Pattern Recognition. 2024.

\bibitem{shah2023vint} Shah, Dhruv, et al. "ViNT: A foundation model for visual navigation." arXiv preprint arXiv:2306.14846 (2023).

\bibitem{liu2024visual} Liu, Haotian, et al. "Visual instruction tuning." Advances in neural information processing systems 36 (2024).

\bibitem{hu2021lora} Hu, Edward J., et al. "Lora: Low-rank adaptation of large language models." arXiv preprint arXiv:2106.09685 (2021).

\bibitem{qin2023supfusion} Qin, Yiran, et al. "SupFusion: Supervised LiDAR-camera fusion for 3D object detection." Proceedings of the IEEE/CVF International Conference on Computer Vision. 2023.

\bibitem{qin2024worldsimbench} Qin, Yiran, et al. "Worldsimbench: Towards video generation models as world simulators." arXiv preprint arXiv:2410.18072 (2024).

\bibitem{yu2025gamefactory} Yu, Jiwen, et al. "GameFactory: Creating New Games with Generative Interactive Videos." arXiv preprint arXiv:2501.08325 (2025).

\bibitem{zhou2024code} Zhou, Enshen, et al. "Code-as-Monitor: Constraint-aware Visual Programming for Reactive and Proactive Robotic Failure Detection." arXiv preprint arXiv:2412.04455 (2024).

\bibitem{zhang2024ad} Zhang, Zaibin, et al. "AD-H: Autonomous Driving with Hierarchical Agents." arXiv preprint arXiv:2406.03474 (2024).

\bibitem{wang2024toward} Wang, Chaoqun, et al. "Toward Accurate Camera-based 3D Object Detection via Cascade Depth Estimation and Calibration." arXiv preprint arXiv:2402.04883 (2024).

\bibitem{huang2024story3d} Huang, Yuzhou, et al. "Story3d-agent: Exploring 3d storytelling visualization with large language models." arXiv preprint arXiv:2408.11801 (2024).

\bibitem{savinov2018semi} Savinov, Nikolay, Alexey Dosovitskiy, and Vladlen Koltun. "Semi-parametric topological memory for navigation." arXiv preprint arXiv:1803.00653 (2018).

\bibitem{majumdar2022zson} Majumdar, Arjun, et al. "Zson: Zero-shot object-goal navigation using multimodal goal embeddings." Advances in Neural Information Processing Systems 35 (2022): 32340-32352.

\bibitem{yadav2023offline} Yadav, Karmesh, et al. "Offline visual representation learning for embodied navigation." Workshop on Reincarnating Reinforcement Learning at ICLR 2023. 2023.

\bibitem{yadav2023ovrl} Yadav, Karmesh, et al. "Ovrl-v2: A simple state-of-art baseline for imagenav and objectnav." arXiv preprint arXiv:2303.07798 (2023).

\bibitem{sun2024fgprompt} Sun, Xinyu, et al. "FGPrompt: fine-grained goal prompting for image-goal navigation." Advances in Neural Information Processing Systems 36 (2024).

\bibitem{zhu2017target} Zhu, Yuke, et al. "Target-driven visual navigation in indoor scenes using deep reinforcement learning." 2017 IEEE international conference on robotics and automation (ICRA). IEEE, 2017.

\bibitem{koh2024generating} Koh, Jing Yu, Daniel Fried, and Russ R. Salakhutdinov. "Generating images with multimodal language models." Advances in Neural Information Processing Systems 36 (2024).

\bibitem{krantz2022instance} Krantz, Jacob, et al. "Instance-specific image goal navigation: Training embodied agents to find object instances." arXiv preprint arXiv:2211.15876 (2022).

\bibitem{schulman2017proximal} Schulman, John, et al. "Proximal policy optimization algorithms." arXiv preprint arXiv:1707.06347 (2017).

\bibitem{anderson2018evaluation} Anderson, Peter, et al. "On evaluation of embodied navigation agents." arXiv preprint arXiv:1807.06757 (2018).

\bibitem{lin2024navcot} Lin, Bingqian, et al. "NavCoT: Boosting LLM-Based Vision-and-Language Navigation via Learning Disentangled Reasoning." arXiv preprint arXiv:2403.07376 (2024).

\bibitem{NavGPT} Zhou, Gengze, Yicong Hong, and Qi Wu. "Navgpt: Explicit reasoning in vision-and-language navigation with large language models." Proceedings of the AAAI Conference on Artificial Intelligence.

\bibitem{hahn2021no} Hahn, Meera, et al. "No rl, no simulation: Learning to navigate without navigating." Advances in Neural Information Processing Systems 34 (2021): 26661-26673.

\bibitem{li2025t2isafety} Li, Lijun, et al. "T2ISafety: Benchmark for Assessing Fairness, Toxicity, and Privacy in Image Generation." arXiv preprint arXiv:2501.12612 (2025).

\bibitem{an2024agfsync} An, Jingkun, et al. "AGFSync: Leveraging AI-Generated Feedback for Preference Optimization in Text-to-Image Generation." arXiv preprint arXiv:2403.13352 (2024).


\end{thebibliography}
\end{sloppypar}

\clearpage
\beginsupplement
\section*{Appendix}
\renewcommand{\thesubsection}{S\arabic{subsection}}

\subsection{\label{chap:S1}PanNuke and MoNuSAC preprocessing}
The PanNuke dataset comprises a set of 7,901 RGB patches, each with dimensions of $256 \times 256$ pixels, which we set as the standard patch size for our analysis. In contrast, the MoNuSAC dataset encompasses 294 images of heterogeneous dimensions. To standardize the MoNuSAC images with our experiments, we implement a standardization protocol. Specifically, for images exceeding the dimensions of $256 \times 256$ pixels, we segment them into equal-sized patches and apply mirror padding to the remaining portions to avoid information loss at the peripherals. Patches with dimensions less than $128 \times 128$ pixels are excluded from the dataset due to the insufficient resolution to capture relevant cellular details. For patches where either dimension falls between 128 and 256 pixels, we employ upsampling to achieve the standard patch size. As a result, we obtain a total of 2,823 RGB patches derived from the MoNuSAC dataset for subsequent analysis. For additional details on the MoNuSAC data preparation process, refer to the source code \cite{Shvetsov_2025a}.
\clearpage

\subsection{\label{chap:S2}Data usage for the methodology}

\counterwithin{figure}{subsection}
\renewcommand{\thefigure}{S\arabic{subsection}}

\begin{figure}[h!]
    \centering
    \includegraphics[width=\textwidth, height=0.85\textheight, keepaspectratio]{images/A2.pdf}
    \caption{Overview of the methodology for cross-labeling, dataset refinement, and model comparison. (1) Cross-relabeling - training and testing cell classification models, (2) Cross-relabeling - using cell classification models to create refined dataset, (3) Fine-tuning and training models for comparison, (4) Student knowledge distillation with refined dataset}
    \label{fig:S2}
\end{figure}
\clearpage

\subsection{\label{chap:S3}Confusion matrices for classification models}
\counterwithin{figure}{subsection}
\renewcommand{\thefigure}{S\arabic{subsection}.\arabic{figure}}

\begin{figure}[h!]
    \centering
    \includegraphics[width=\textwidth, height=0.4\textheight, keepaspectratio]{images/A3_1.pdf}
    \caption{Confusion matrix for PanNuke trained model}
    \label{fig:S3.1}
\end{figure}

\begin{figure}[h!]
    \centering
    \includegraphics[width=\textwidth, height=0.4\textheight, keepaspectratio]{images/A3_2.pdf}
    \caption{Confusion matrix for MoNuSAC trained model}
    \label{fig:S3.2}
\end{figure}

\clearpage

\subsection{\label{chap:S4}Datasets cell counts}

\counterwithin{table}{subsection}
\renewcommand{\thetable}{S\arabic{subsection}}

\begin{table}[h!]
\renewcommand{\arraystretch}{2.0}
\centering
\caption{\label{tab:S4}Cell counts for PanNuke, MoNuSAC and refined datasets. Numbers in parentheses indicate preprocessed cell counts for cell classifier models training and testing.}
%\adjustbox{max width=\textwidth}{%
\begin{tabular}{|l|c|c|c|}
\hline
%\rowcolor{gray!30}
Cell type & PanNuke & MoNuSAC & Refined \\
\hline
Neoplastic & 77,403 (68,031) & - & 105,451 \\
\hline
Epithelial & 26,572 (23,207) & - & 29,926 \\
\hline
Epithelial (benign and malignant) & - & 31,402 & - \\
\hline
Inflammatory & 32,276 & - & - \\
\hline
Lymphocytes & - & 37,045 (33,104) & 65,275 \\
\hline
Neutrophils & - & 1,355 (1,252) & 3,833 \\
\hline
Macrophage & - & 1,842 (1,695) & 3,410 \\
\hline
Dead & 2,908 & - & 2,908 \\
\hline
Connective & 50,585 & - & 50,585 \\
\hline
\end{tabular}
%
%}
\end{table}



\clearpage

\subsection{\label{chap:S5}Definition of validation metrics}
\counterwithin{equation}{subsection}
\renewcommand{\theequation}{\arabic{equation}}

\subsubsection{\label{chap:S5.1}R\textsuperscript{2}}
The coefficient of determination, denoted as $R^2$, is a statistical measure that represents the proportion of variance in the dependent variable that is predictable from the independent variables. In the context of cell quantification in pathology, $R^2$ is used to assess how well the predicted quantities of different cell types in a patch align with the actual quantities observed in the ground truth data, with higher values representing more accurate quantification. $R^2$ is defined as
\begin{equation*}
R^2 = 1 - \frac{\sum_{i=1}^n (y_i - \hat{y}_i)^2}{\sum_{i=1}^n (y_i - \bar{y})^2},
\end{equation*}
where $y_i$ represents the actual number of cells of a specific type in the $i$-th image, $\hat{y}_i$ represents the predicted number of cells of that type in the $i$-th image, $\bar{y}$ is the mean of the actual numbers across all images, and $n$ is the total number of images in the dataset.

The $R^2$ metric has a range of $(-\infty, 1]$. An $R^2$ of 1 indicates perfect prediction, where all predicted values exactly match the actual values. An $R^2$ of 0 suggests that the model explains none of the variability of the response data around its mean. If $R^2$ is negative, it indicates that the model performs worse than a model that simply predicts the mean of the actual values for all observations.

\subsubsection{\label{chap:S5.2}PQ}
Panoptic Quality ($PQ$) is a comprehensive metric used to evaluate the performance of segmentation models in tasks that require both instance segmentation and classification. $PQ$ provides a single score that encapsulates both the detection accuracy (i.e., how many objects were correctly identified) and the segmentation quality (i.e., how accurately the objects' boundaries were delineated). This metric is particularly useful in multiclass scenarios where each pixel is classified into distinct categories, such as different cell types in pathology images.

$PQ$ is calculated as the product of two terms: Detection Quality ($DQ$) and Segmentation Quality ($SQ$). It can be expressed as
\begin{equation*}
PQ = DQ \cdot SQ,
\end{equation*}
where
\begin{equation*}
DQ = \frac{TP}{TP + 0.5\, FP + 0.5\, FN},
\end{equation*}
\begin{equation*}
SQ = \frac{\sum_{(p, g) \in \mathcal{M}} IoU(p, g)}{TP}.
\end{equation*}
In these formulas, $TP$ denotes the number of correctly matched instances between ground truth and prediction, $FP$ denotes the predicted instances that have no corresponding ground truth, $FN$ denotes the ground truth instances that were not detected, $IoU(p, g)$ is the Intersection over Union for a pair of matched instances $p$ (prediction) and $g$ (ground truth), and $\mathcal{M}$ is the set of matched pairs.

The $PQ$ metric is calculated for each class and is averaged across classes to provide a global performance measure.

The $PQ$ score has a range of $[0, 1.0]$, where a higher score indicates better performance in both detecting and segmenting the instances correctly. A $PQ$ of 1 signifies perfect identification and segmentation of all instances, whereas a $PQ$ of 0 indicates that no instances were correctly identified and segmented.

\clearpage

\subsection{\label{chap:S6}Segmentation and Detection quality metrics for teacher and student models}

\begin{table}[h!]
\renewcommand{\arraystretch}{2.0}
\centering
\caption{Segmentation and detection quality for student and teacher models (CI 95\%)}
\label{tab:S6}
%\adjustbox{max width=\textwidth}{%
\begin{tabular}{|l|c|c|}
\hline
%\rowcolor{gray!30}
Metric & Teacher & Student \\
\hline
$SQ_{neoplastic}$ & 0.819 (0.815--0.823) & 0.824 (0.819--0.828) \\
\hline
$SQ_{lymphocyte}$ & 0.795 (0.788--0.802) & 0.790 (0.783--0.796) \\
\hline
$SQ_{connective}$ & 0.770 (0.762--0.776) & 0.780 (0.772--0.786) \\
\hline
$SQ_{dead}$ & 0.659 (0.623--0.688) & 0.657 (0.624--0.695) \\
\hline
$SQ_{epithelial}$ & 0.780 (0.770--0.790) & 0.788 (0.779--0.797) \\
\hline
$SQ_{macrophage}$ & 0.788 (0.760--0.810) & 0.757 (0.730--0.783) \\
\hline
$SQ_{neutrofil}$ & 0.782 (0.761--0.801) & 0.775 (0.759--0.792) \\
\hline
$DQ_{neoplastic}$ & 0.706 (0.692--0.719) & 0.727 (0.712--0.741) \\
\hline
$DQ_{lymphocyte}$ & 0.675 (0.656--0.698) & 0.713 (0.691--0.734) \\
\hline
$DQ_{connective}$ & 0.566 (0.546--0.584) & 0.583 (0.565--0.602) \\
\hline
$DQ_{dead}$ & 0.410 (0.361--0.465) & 0.435 (0.306--0.561) \\
\hline
$DQ_{epithelial}$ & 0.668 (0.639--0.694) & 0.673 (0.644--0.702) \\
\hline
$DQ_{macrophage}$ & 0.657 (0.583--0.727) & 0.615 (0.531--0.703) \\
\hline
$DQ_{neutrofil}$ & 0.691 (0.625--0.753) & 0.729 (0.679--0.778) \\
\hline
\end{tabular}
%
%}
\end{table}

\clearpage

\subsection{\label{chap:S7}QuPath integration method}
We adopt an integration strategy leveraging the paquo \cite{Bayer_AG} library, a Python package that enables direct interaction with QuPath’s internal API, thereby facilitating seamless data exchange without intermediate conversion steps. The data processing pipeline (\hyperref[fig:S7]{Appendix Figure S7}) begins with the acquisition of WSIs and their associated annotations from QuPath, which are represented as Shapely \cite{Gillies_Wel_etal._2024} polygons. Utilizing paquo, we directly read, create, and modify these annotations and detections within a QuPath project in the Python environment. Images are then cropped using these polygons and processed by cell segmentation and classification models employing standard vision processing toolkits such as OpenCV, pyvips, and PyTorch. Additionally, QuPath employs Groovy scripts to initiate a Python process that starts the entire pipeline from QuPath graphical interface: fetching polygons, extracting images from them, and running deep learning model inference on the cropped images. 
The results are returned to QuPath, leveraging paquo's Python bindings to manipulate QuPath data while minimizing the computational overhead typically associated with cross-environment communication.

\counterwithin{figure}{subsection}
\renewcommand{\thefigure}{S\arabic{subsection}}

\begin{figure}[h!]
    \centering
    \includegraphics[width=\textwidth]{images/A7.pdf}
    \caption{QuPath integration workflow using Python environment}
    \label{fig:S7}
\end{figure}

Compared to traditional workflows that involve exporting annotations as GeoJSON, classifying them in Python, and reimporting them into QuPath, our approach offers several advantages. We eliminate the need to switch between programming languages, providing a cohesive and streamlined development process entirely within QuPath software and removing the necessity to use other tools. Meanwhile, we avoid storing annotations as intermediate JSON files unless required for external use or archiving. By conducting the entire inference and post-processing workflow within the Python environment, we leverage the power and flexibility of Python libraries for image processing and machine learning. This approach also enables adjustments to any set of labels and models, thereby improving its applicability.

%\hfill

The distilled model and QuPath integration code are packaged into a Docker container, enabling streamlined execution with the Docker engine. Detailed integration code and deployment instructions can be found in the GitHub repository \cite{Shvetsov_2025b}.

Despite these benefits, we acknowledge that the paquo library is a proof‑of‑concept project in its early development stage and has not been tested across all versions of QuPath.

\clearpage

\subsection{\label{chap:S8}Data and code availability statement}
All datasets, models, and code used in this study are publicly available and can be obtained from the repositories listed below. 
The PanNuke \cite{Gamper_Koohbanani_etal._2019} and MoNuSAC \cite{Verma_Kumar_etal._2021} datasets are publicly accessible, and download information along with detailed descriptions can be found in their respective articles. Preprocessing scripts for PanNuke and MoNuSAC data, as well as individual cell extraction scripts, are available on GitHub \cite{Shvetsov_2025a}. The H-Optimus foundation model used in our experiments can be downloaded from the HuggingFace repository \cite{hoptimus2024}, and model information is available on GitHub \cite{Saillard_Jenatton_etal._2024}. In addition, the integration code for QuPath and the distilled model packaged in a Docker container are provided in the repository \cite{Shvetsov_2025b}, and paquo Python library is available from the authors GitHub repository \cite{Bayer_AG}.
\clearpage

\end{document}


As shown in Table \ref{tab:main_exp_sum}, a significant gap persists between EM and other evaluation metrics across all models. This observation indicates that while LLMs can address specific aspects of a given task, achieving a complete and accurate solution remains a significant challenge. Moreover, closed-source models generally exhibit superior performance across key metrics (S-Acc, EM, and PM-0.5) to open-source models, with o1-preview achieving the best EM score of 23.6\%. DeepSeek-R1 demonstrates the most advanced reasoning capabilities among open-source models and even achieves a comparable EM score of 20.0\% to o1-preview. Notably, there also exists a great performance gap between conventional models and o1-like models. Although GPT-4o shows the second-highest S-Acc, it lags significantly behind the top-performing model o1-preview in EM. This highlights the effectiveness of the "slow thinking" approaches which trade better performance with more inference tokens.



\section{Analysis}
\begin{figure}[!h]
  \centering
  % \vspace*{1mm}
  \includegraphics[width=0.8\columnwidth]{fig/difficulty.pdf}
  \caption{\label{difficulty}Performance trend of GPT-4o and o1-preview under varying difficulty levels.}
\end{figure}
\subsection{Task Difficulty}

We present the detailed model performance across varying difficulty for all six tasks in Appendix \ref{app_difficulty}. Here, we focus on the Logic Puzzle, illustrating the impact of increasing problem complexity. Figure \ref{difficulty} shows the performance of GPT-4o and o1-preview under different difficulty levels. The substantial performance gap between the two models demonstrates o1-preview's superior logical deduction and reflective reasoning abilities. As the solution space expands from $4\times4$ to $4\times7$, while models can still solve part of the problem, they fail to find the complete solution. This highlights the challenge of exploring exponentially increasing solution spaces for complex reasoning tasks.



\subsection{Long Reasoning Chain Generation}

Tasks in LR$^2$Bench often involve multiple subtasks (e.g., solving all clues for Crossword, inferring the digits for each cell for Sudoku). Such inherent complexity requires LLMs to continuously explore quite large solution spaces through reflective reasoning processes thus presenting a significant challenge for LLMs in generating long reasoning chains. As shown in Table \ref{tab:main_exp_sum}, many models failed to completely generate the entire reasoning process, resulting in low CR scores. Through analysis of incomplete model responses, we find that a key obstacle to this phenomenon is the tendency of LLMs to generate repetitive content, ultimately reaching the maximum sequence length. This redundancy primarily occurs when LLMs encounter contradictions (see Section \ref{qualitative} for a detailed analysis). Such redundancy wastes valuable context window size, preventing the model from exploring the full solution space and completing further reasoning processes. To quantify this, we calculate the average 10-gram redundancy ratio of the models' responses across all tasks in LR$^2$Bench, excluding Sudoku due to its inherently repetitive cell-by-cell reasoning strategy. Figure \ref{redundant} reveals a strong negative correlation between redundancy and CR, suggesting that redundant generation is a key factor limiting the long-chain reasoning capability of LLMs. Notably, QwQ-32B-Preview shows lower redundancy but still fails in completion due to its endless trial-and-error without reaching a meaningful conclusion.


\subsection{Conventional Models vs. o1-like Models}
% different tasks, difficulty
As shown in Table \ref{tab:main_exp}, for Crossword and Acrostic tasks, we observe that conventional models can achieve high S-Acc and PM-0.5, but nearly zero EM. This suggests that LLMs with strong commonsense reasoning abilities can correctly infer parts of words from the provided clues. However, some clues may have multiple valid answers, requiring LLMs to determine the final answers based on the constraint of shared letters with other words. Therefore, the lack of reflective reasoning prevents these models from achieving perfect solutions. Besides, GPT-4o performs better than o1-mini on these tasks, suggesting that only models possessing both capabilities can achieve high EM scores. 

\begin{figure}[t!]
  \centering
  \includegraphics[width=1.0\columnwidth]{fig/r.pdf}
  \caption{The relationship between redundancy and CR across different models.}
  \label{redundant}
  \vspace{-4mm}
\end{figure}

For tasks that heavily rely on reflection capabilities, such as Logic Puzzle and Sudoku, the performance gap between o1-like models and conventional models is the most obvious. Notably, o1-preview outperforms GPT-4o by 37.5\% and 35.5\% and QwQ-32B-Preview outperforms Qwen-2.5-70B-Instruct by 19.5\% and 26.0\% on EM for these two tasks, respectively. These substantial gains highlight the significant advantage of o1-like models in tasks that require verifying multiple constraints and employing backtracking mechanisms. In contrast, conventional models often exhibit limitations with their single-pass reasoning paradigm. This approach proves inadequate for scenarios requiring exploration of alternative solution paths or correction of initial assumptions.

Cryptogram and Drop Quote pose the most significant challenges for all evaluated models. We attribute this difficulty to the expansive search space and limited helpful clues inherent in these problems. In the Cryptogram, each letter has theoretically 26 possible substitutions. Similarly, in Drop Quote, each letter within a given column can be dropped to any row. This contrasts with Logic Puzzles and Sudoku, which offer strong constraints that effectively reduce the search space. Even the most advanced model o1-preview struggles to complete these tasks with only a 13\% EM score.


\subsection{Qualitative Analysis}
\label{qualitative}

We analyze several typical behaviors of LLMs leading to the failure in completing our benchmark and provide detailed cases in Appendix \ref{app_case}.

\paragraph{Lack of Reflection Mechanism} This deficiency mainly occurs in conventional models. Taking the Logic Puzzle for example, our analysis of GPT-4o's responses in Appendix \ref{lack} reveals that although the model can effectively break down individual clues and generate initial deductions, it fails to perform the iterative cross-checking to ensure consistency with all established constraints. These shortages indicate that conventional models lack the reflective reasoning capabilities necessary to solve complex constraint satisfaction problems.

\paragraph{Stuck in Contradictions} We observe that LLMs often struggle with complex reasoning tasks when confronted with contradictions. As shown in Appendix \ref{stuck}, this phenomenon mainly manifests as looping within similar sentences, repeatedly stating conflicting information without making progress toward a coherent resolution. Instead of backtracking and revising previous assumptions upon encountering a contradiction, they tend to focus only on the conflicting points. This narrow focus prevents them from exploring alternative solution paths.

\paragraph{Give-up Moment} As illustrated in Appendix \ref{giveup}, the occurrence of a "Give-up Moment" is observed when reasoning-specific LLMs struggle with complex problems due to the time (inference tokens) constraint. We focus on one of the worst-performing tasks Cryptogram to quantify this give-up ratio in QwQ-32B and DeepSeek-R1 responses, respectively. We manually review 50 responses from each model on Cryptogram. For QwQ-32B, 15 out of 50 responses are incomplete due to the maximum context window size of 32k. 26 out of 50 responses show model's give-up and only provide part of the answers. Similarly, for DeepSeek-R1, 39 out of 50 responses compromise to give the most possible answers given the time constraints. This highlights the challenges posed by problems requiring extensive reflective reasoning and suggests a potential bottleneck in the current capabilities of even the leading reasoning LLMs.

\subsection{Future Directions for o1-like LLMs}
Addressing the limitations highlighted in our qualitative analysis suggests several key directions for future o1-like LLM development:

\paragraph{Reasoning Paradigm Transformation} Conventional LLMs fail in solving complex CSPs in LR$^2$Bench due to the lack of reflection mechanisms. This suggests the importance of enabling LLMs to perform iterative, multi-step reflective reasoning processes. Future research would shift the focus from linear reasoning chains to dynamic, reflective thinking paradigms.

\paragraph{Reasoning Adaptation} Current LLMs often employ a uniform reasoning strategy regardless of problem complexity or type. Future research should prioritize developing adaptive mechanisms that allow LLMs to dynamically adjust their reasoning processes based on specific problems. Simpler problems may only require a straightforward approach, while complex problems necessitate a more deliberate, multi-step, and reflective process. Furthermore, different tasks may demand distinct reasoning approaches. LLMs should be able to determine the type of reasoning required (e.g., logical, mathematical, commonsense) and adjust their strategies accordingly.

\paragraph{Enhancing Reflection Mechanisms} Although leading o1-like LLMs have possessed reflective reasoning capabilities, they struggle to effectively utilize them when confronted with contradictions. Future development should focus on enabling models to recognize inconsistencies and pivot towards alternative hypotheses, emphasizing flexibility and exploration of different reasoning trajectories rather than fixation on conflicting information.

\paragraph{Improving Reasoning Efficiency} Current o1-like models tend to generate lengthy reasoning chains when tackling complex problems, leading to significant inference costs. Future work could prioritize optimizing these reasoning chains to achieve greater efficiency without sacrificing accuracy. 

\section{Related Work}
% \subsection{Reasoning with LLMs}
% \subsection{Reflective Reasoning Evaluation}
\subsection{Reflection Capabilities of LLMs}
The impressive reasoning capabilities of LLMs have naturally led to increased interest in reflection, a more sophisticated and human-like form of reasoning. Previous studies primarily focus on evaluating LLMs' abilities to rectify their response based on explicit feedback, including self-improvement with critique prompts \citep{lan2024criticeval, li2024reflection, lin-etal-2024-criticbench, li2024hindsight, madaan2024self}, leveraging external tools such as code interpreters or search engines \citep{gou2024critic, chen2024teaching, shinn2024reflexion}, and engaging in multi-LLM interaction through debating \citep{liang-etal-2024-encouraging, huang2024large}. However, these works mainly evaluate LLMs' behaviors in response to feedback. They fail to assess LLMs' capabilities to spontaneously engage in the complete reflection process for complex reasoning tasks. Our proposed LR$^2$Bench provides scenarios necessitating capabilities, such as making assumptions, verification, backtracking, and self-refinement, thus filling a critical gap in evaluating LLMs' intrinsic reflective reasoning abilities.

% \subsection{Complex Reasoning Tasks for LLMs}
\subsection{Puzzle-solving for LLMs}
Puzzle-solving \citep{giadikiaroglou2024puzzle} offers valuable insight for evaluating the complex reasoning capabilities of LLMs across diverse scenarios. \citet{ishay2023leveraging} explore Sudoku solving strategies with answer set programming. \citet{ding2023everything} leverages reinforcement learning and Monte Carlo Tree Search to solve problems like Game of 24, 8-Puzzle, and Pocket Cube. \citet{yao2024tree} introduces "Tree of Thought" to enable self-evaluating and backtracking for Game of 24 and Crosswords. \citet{mittal2024puzzlebench} combines LLMs with symbolic solvers and program interpreters to complete first-order combinatorial reasoning problems. \citet{tyagi2024step} focuses on grid puzzles to evaluate the generated reasoning chains of LLMs. Moreover, existing studies have also investigate Board game \citep{kazemi2024boardgameqa}, Chess \citep{feng2024chessgpt} and social games \citep{light2023avalonbench, wang2023avalon, xu2023exploring}. However, these studies primarily leverage external tools or specialized algorithms to develop task-specific solutions within limited puzzle domains. In contrast, LR$^2$Bench provides diverse tasks and difficulty levels and focuses on evaluating the intrinsic reflective reasoning capabilities of LLMs.


\section{Conclusion}
This paper introduces LR$^2$Bench, a novel benchmark to comprehensively evaluate the reflection capabilities of LLMs in long-chain reasoning. LR$^2$Bench comprises six tasks with varying difficulty levels, providing a thorough analysis across diverse scenarios. The experimental results show that o1-like models outperform conventional models, demonstrating their superior performance on reflective reasoning. Our findings also highlight the limitation of current reasoning LLMs and reveal that even the most advanced reasoning models fall short of achieving satisfactory performance, suggesting significant room for enhancement in reflective reasoning capabilities.

\section*{Limitations}
The limitations of our work can be summarized as follows: Firstly, due to the scarcity of well-defined real-world constraint satisfaction problems, we relied on puzzle-like data for evaluating LLMs' reflective reasoning capabilities. Secondly, the inherent complexity and verbosity of LLM-generated responses to these complex reasoning tasks posed challenges for more fine-grained analysis. We only analyze several typical phenomena of current leading models rather than conducting a more detailed analysis of specific reflective reasoning processes.

% Bibliography entries for the entire Anthology, followed by custom entries
%\bibliography{anthology,custom}
% Custom bibliography entries only
\bibliography{custom}

\clearpage

\appendix

\section{Task Example}
\label{app_example}
In this section, we provide specific examples of each type of task to facilitate a better understanding of data samples in LR$^2$Bench.

\subsection{Crossword}
\begin{tcolorbox}[breakable, title=Crossword Example, colframe=blue!60]
\begin{CustomVerbatim}
Grid:
#   1   2   3   #
4   ?   ?   ?   5
6   ?   ?   ?   ?
7   ?   ?   ?   ?
8   ?   ?   ?   #

Clues:
Across:
1: "When They See Us" director DuVernay (3)
4: WNBA team based in Seattle (5)
6: Locations for orations (5)
7: Signed (5)
8: Lipton products (4)
Down:
1: Be penitent (5)
2: Black Russian ingredient (5)
3: First sign of the zodiac (5)
4: Barbecue rod (4)
5: Fuming (3)

Answer:
#   A   V   A   #
S   T   O   R   M
P   O   D   I   A
I   N   K   E   D
T   E   A   S   #
\end{CustomVerbatim}
\end{tcolorbox}

\subsection{Acrostic}
\begin{tcolorbox}[breakable, title=Acrostic Example, colframe=blue!60]
\begin{CustomVerbatim}
Grid:
1 2 3 4 # 5 6 7 # 8 9 10 11 # 12 13 14 # 15 16 17 18 19 20 21 22 # 23 24 25 26 27 28 # 29 30 31 32 33 34 # 35 # 36 37 38 39 40 41 ' 42 # 43 44 45 46 . 47 48 49 # 50 51 52 # 53 54 55 56 57 58 # 59 60 # 61 62 # 63 64 65 66 , 67 68 # 69 70 71 72 73 74 # 75 76 # 77 78 # 79 80 81 82 . 83 84 ' 85 # 86 # 87 88 89 90 # 91 92 93 94 95 96 .

Clues:
A
57 18 88 9 33 80 78
Gaynor hit \"I Will ____\" (1 word, 7 letters)
B
29 70 67 82 21
Wheat or milk beginning (1 word, 5 letters)
C
5 7 24 11 35 75 90
Extol (1 word, 7 letters)
D
30 50 10 91 46 14
Wintry projection (1 word, 6 letters)
E
52 26 73 58 94
Second-generation Japanese (1 word, 5 letters)
F
59 93 45 95 54 43 84 65 41 27
Standard (1 word, 10 letters)
G
92 72 60 4 37 66
Like some sweatshirts (1 word, 6 letters)
H
74 34 53 64 28 89 22
Scrambles a message (1 word, 7 letters)
I
36 25 2 61 86 23 83 19 81 31 47
Odds (1 word, 11 letters)
J
38 96 63 15 40 6 85
The Arctic and the Antarctic, e.g. (1 word, 7 letters)
K
71 8 49 87 20
Parisian eggs (1 word, 5 letters)
L
1 56 76 77 79 13
Peanut (1 word, 6 letters)
M
68 48 55 39
Aussie jumpers (1 word, 4 letters)
N
12 17 44 42 62
Strike ___ (what models do) (2 words, 5 letters)
O
16 51 69 32 3
Far from wimpy (1 word, 5 letters)

Answer:
"Good and evil are impulses buried within a person's soul. You can choose to be good, or choose to be evil. It's a free choice." - James Swain, The Program
\end{CustomVerbatim}
\end{tcolorbox}

\subsection{Logic Puzzle}
\begin{tcolorbox}[breakable, title=Logic Puzzle Example, colframe=blue!60]
\begin{CustomVerbatim}
Variables:
{
    "anniversaries": [
        "January 28",
        "March 6",
        "November 2",
        "November 23"
    ],
    "first names": [
        "Asher",
        "Bradley",
        "Kayla",
        "Malia"
    ],
    "conveyances": [
        "10-speed bike",
        "mountain bike",
        "segway",
        "skateboard"
    ],
    "actors": [
        "Liam Neeson",
        "Morgan Freeman",
        "Robert Duvall",
        "Tom Cruise"
    ]
}

Clues:
1. The commuter who rides a 10-speed bike is not Bradley.
2. Kayla isn't related to Robert Duvall and doesn't ride a segway.
3. Robert Duvall's cousin has an anniversary after the commuter who rides a segway.
4. Morgan Freeman's cousin is not Asher and doesn't ride a segway.
5. Liam Neeson's cousin is Malia.
6. Liam Neeson's cousin uses a mountain bike to get to work.
7. Of Malia and the commuter who rides a 10-speed bike, one has an anniversary on January 28 and the other has an anniversary on November 23.
8. Either the person whose anniversary is on March 6 or the person whose anniversary is on November 2 is Morgan Freeman's cousin.
9. Tom Cruise's cousin has an anniversary after the commuter who rides a skateboard.

Answer:
[
    {
        "anniversaries": "January 28",
        "first names": "Malia",
        "conveyances": "Mountain bike",
        "actors": "Liam Neeson"
    },
    {
        "anniversaries": "March 6",
        "first names": "Kayla",
        "conveyances": "Skateboard",
        "actors": "Morgan Freeman"
    },
    {
        "anniversaries": "November 2",
        "first names": "Bradley",
        "conveyances": "Segway",
        "actors": "Tom Cruise"
    },
    {
        "anniversaries": "November 23",
        "first names": "Asher",
        "conveyances": "10-speed bike",
        "actors": "Robert Duvall"
    }
]
\end{CustomVerbatim}
\end{tcolorbox}

\subsection{Cryptogram}
\begin{tcolorbox}[breakable, title=Cryptogram Example, colframe=blue!60]
\begin{CustomVerbatim}
Cryptogram:
VYQ'O CY HGYWQV THMFQC OKB LYGUV YLBT MYW H UFJFQC. OKB LYGUV YLBT MYW QYOKFQC. FO LHT KBGB RFGTO. - Mark Twain

Answer:
Don't go around saying the world owes you a living. The world owes you nothing. It was here first. - Mark Twain
\end{CustomVerbatim}
\end{tcolorbox}

\subsection{Sudoku}
\begin{tcolorbox}[breakable, title=Sudoku Example, colframe=blue!60]
\begin{CustomVerbatim}
Sudoku:
[[4, 0, 0, 0],
 [0, 3, 2, 0],
 [0, 4, 1, 0],
 [0, 0, 0, 2]]

Answer:
[[4, 2, 3, 1],
 [1, 3, 2, 4],
 [2, 4, 1, 3],
 [3, 1, 4, 2]]
\end{CustomVerbatim}
\end{tcolorbox}

\subsection{Drop Quote}
\begin{tcolorbox}[breakable, title=Drop Quote Example, colframe=blue!60]
\begin{CustomVerbatim}
Grid:
1  2  #  4  5  6  7  8  9   #   11 12 13 14 15
#  2  3  4  5  6  7  8  9  10  11  #  13 14 15
1  2  #  4  5   '  7  #  9  10  11 12  .    #   #

Column 1: I I
Column 2: I L T
Column 3: M
Column 4: A I P
Column 5: L O T
Column 6: S W
Column 7: A S S
Column 8: I Y
Column 9: B D S
Column 10: L O
Column 11: E N S
Column 12: E E
Column 13: E U
Column 14: M N
Column 15: S T

Answer:
It always seems impossible until it's done. - Nelson Mandela
\end{CustomVerbatim}
\end{tcolorbox}


\section{Prompt Templates}
\label{app_prompt}
We provide detailed prompt templates for each type of task in LR$^2$Bench, including task descriptions, examples, and solution guidelines.
\subsection{Crossword}
\begin{tcolorbox}[breakable, title=Prompt for Crossword]
\small
\textbf{[Task Description]} \\

A Crossword puzzle is a word game that consists of a grid, with clues given for words that fit into the grid both across (horizontally) and down (vertically). Your goal is to fill in the grid with words based on the clues provided. Here's a detailed explanation of how the game works: \\

1. Understand the Grid Layout \\
The grid is made up of numbers, hashtags ("\#"), and question marks ("?"). \\

Hashtag ("\#") acts as a separator between words. \\

The number represents the starting points of Across and Down words. \\

Question mark ("?") represents part of words but don't start a new word. \\

2. Read the Clues \\
Clues are provided for each word to be filled into the grid, split into two categories: Across clues (these are for words that go horizontally in the grid) and Down clues (these are for words that go vertically in the grid). \\

The number in brackets after the clue indicates the length of the word. \\

Clues are often short definitions, synonyms, or phrases related to the word. \\

Some clues may involve wordplay, anagrams, or puns, depending on the puzzle's difficulty and style. \\

3. Solve the Puzzle \\
Think of words that fit the clue and match the number of letters specified. \\

For example, if a clue says "Animal that barks (3)", you might guess "DOG" because it has 3 letters. When getting the answer, double-check that it fits the clue, both in meaning and the number of letters. If you make a mistake, just try again! \\

As you fill in answers, they will help you solve other clues, since words intersect and share letters in the grid. This cross-checking mechanism helps in verifying correct answers. \\

\textbf{[Example 1]}

\begin{CustomVerbatim}
Here is a Crossword puzzle:
GRID:
# 1 2 3 4
5 ? ? ? ?
6 ? ? ? ?
7 ? ? ? ?
# 8 ? ? #

CLUES:
Across:
1 "More ___ you'll ever know" (4)
5 Divine favor, in Christianity (5)
6 Sisters who might be sitters (5)
7 Portly-sounding beer (5)
8 Cool and stylish (3)

Down:
1 "The ___ will set you free. But not until it is finished with you": David Foster Wallace (5)
2 Capital of Vietnam (5)
3 Misbehave (5)
4 Prepare for a new baby, say (4)
5 Fuel for some stoves (3)
\end{CustomVerbatim}

\textbf{[Solution 1]}

\begin{CustomVerbatim}
Let's start solving the Crossword puzzle step by step.

Across Clues:

<1-Across>
1-Across: "More ___ you'll ever know" (4)

The clue is a reference to a popular phrase, "More than you'll ever know," suggesting the answer is "than".
Answer: THAN.
</1-Across>

<5-Across>
5-Across: Divine favor, in Christianity (5)

The clue refers to a concept of God's blessing or grace, and the answer is "grace".
Answer: GRACE.
</5-Across>

... ...

Here is the final grid after finishing all the clues:
<Answer>
{"1-Across": "THAN", "5-Across": "GRACE", "6-Across": "AUNTS", "7-Across": "STOUT", "8-Across": "HIP", "1-Down": "TRUTH", "2-Down": "HANOI", "3-Down": "ACTUP", "4-Down": "NEST", "5-Down": "GAS"}
</Answer>
\end{CustomVerbatim}

\textbf{[Example 2]} \\

<Example 2> \\

\textbf{[Solution 2]} \\

<Solution 2> \\

\textbf{[Solution Guideline]} \\

Please solve the Crossword Puzzle according to the provided rules. Please also follow the requests below to present your analysis and solutions: \\

1. Analyze each clue carefully to understand its meaning and potential word associations. Be open to the possibility of wordplay or puns that might lead to unexpected solutions. The filled-in words will help you infer the answers to the subsequent clues. \\

2. Provide the final answer for each clue. The final answer should be presented after "Answer:". \\

3. Please wrap all the analysis of each clue with <n-Across> and </n-Across> or <n-Down> and </n-Down> tags, where n is the clue number. For example, the Across clue 1 should be wrapped with <1-Across> and </1-Across> tags. The Down clue 1 should be wrapped with <1-Down> and </1-Down> tags, and so on. \\

4. After solving all the clues, please summarize all the answer words in following json format and warp them with <Answer> and </Answer> tags: \\
<Answer> \\
\{"1-Across": "ANSWER1", "2-Across": "ANSWER2", ... "1-Down": "ANSWER3", "2-Down": "ANSWER4", ...\} \\
</Answer> \\

5. Please generate your response without redundant and repeating content. \\

\textbf{[Question]} \\

<Question>
\end{tcolorbox}





\subsection{Acrostic}
\begin{tcolorbox}[breakable, title=Prompt for Acrostic]
\small
\textbf{[Task Description]} \\

An Acrostic puzzle is a word game that consists of two main parts: a grid and a set of clues. The objective is to fill in the grid with letters from the answers to the clues, forming a hidden quotation or sentence. Here's a detailed explanation of how the game works: \\

1. Understand the Grid Layout \\
The grid is made up of numbers, hashtags ("\#"), and punctuations. Hashtag ("\#") acts as space separator between words. \\

Each number corresponds to a specific letter in a word. The hidden quotation or sentence is formed by filling in the grid with the correct letters. \\

2. Read the Clues \\
Each clue is made up of a string of numbers and a clue text. \\

The clue text is usually a short definition, synonym or phrase related to the answer word or phrase. \\

The number of words and letters of the answer are given in brackets after the clue text. \\

The string of numbers provided indicates the specific position of each letter of the answer in the grid. \\

3. Solve the Puzzle \\
Think of answers that fit the clue and match the number of words and letters specified. \\

For example, if a clue says "Animal that barks (1 word, 3 letters)", you might guess "DOG" because it has 3 letters. When getting the answer, double-check that it fits the clue, both in meaning and the number of letters. If you make a mistake, just try again! \\

As you fill in the letters of answers, these letters can be part of the hidden word in the grid, thus providing hints for the left letters of this word. This mechanism can help both verify your answers and solve difficult clues. \\

For example, if the puzzle looks like this: "... \# 5 6 7 \# ..." and you have already got \{5: 'A', 7: 'D'\}, you can guess that the word is "AND" and get \{6: 'N'\} without its corresponding clue. Don't be afraid to skip a tough clue and come back to it later. \\

As you fill in the grid, you will reveal part of the hidden words. Use this information to help solve the remaining clues. \\

    
\textbf{[Example 1]}

\begin{CustomVerbatim}
Here is an Acrostic puzzle:
GRID:
1 2 3 4 5 # 6 7 8 9 10 # 11 12 # 13 14 15 # 16 17 18 19 20 # 21 22 23 24 25 26 27 28 29 , 30 31 32 # 33 34 35 # 36 37 38 # 39 40 41 42 43 # 44 45 46 47 # 48 49 50 # 51 52 53 54 55 56 57 # 58 59 60 61 # 62 63 64 65 66 67 68 69 70 # 71 72 73 74 # 75 # 76 77 78 79 # 80 81 82 # 83 84 85 # 86 87 88 89 90 91 92 .

CLUES:
A
58 86 92 3 60 49 57 46 70 40
Cosmetics magnate (3 words, 10 letters)
B
52 34 44 54 84 67 11 36 25 18 62
Nonphysical assets (1 word, 11 letters)
C
76 45 35 88 56 81 43 41 68 71
Religious schools of opinion (1 word, 10 letters)
D 
48 37 16 24 82 85 30 63 73
Cancer stick (1 word, 9 letters)
E
10 32 2 74 22 23 75 79 39 5 64
Establish a home base (1 word, 11 letters)
F
59 9 14 69 83 53 28
"____ is only a bench covered in velvet" (quote by Napoleon) (2 words, 7 letters)
G
55 26 38 31 20 21 87 91 6
___ up (became less serious) (1 word, 9 letters)
H
4 17 61 65 13 90 77 29
Rest (1 word, 8 letters)
I
33 80 51 15 66 27 7 12
Infractions (1 word, 8 letters)
J
1 19 50 72 8 89 47 78 42
Whisky-vermouth cocktail (1 word, 9 letters)
\end{CustomVerbatim}

\textbf{[Solution 1]}

\begin{CustomVerbatim}
Let's start solving the Acrostic puzzle step by step.

<A-CLUE>
A
58 86 92 3 60 49 57 46 70 40
Cosmetics magnate (3 words, 10 letters)

The clue "Cosmetics magnate" succinctly encapsulates Mary Kay Ash's pioneering role in establishing a direct sales empire in the beauty industry.
Answer: MARYKAYASH.
Mapping: {'58': 'M', '86': 'A', '92': 'R', '3': 'Y', '60': 'K', '49': 'A', '57': 'Y', '46': 'A', '70': 'S', '40': 'H'}
</A-CLUE>

<B-CLUE>
B
52 34 44 54 84 67 11 36 25 18 62
Nonphysical assets (1 word, 11 letters)

"Nonphysical assets" aptly refers to intangibles, highlighting assets like intellectual property and goodwill that lack physical form but possess significant business value.
Answer: INTANGIBLES.
Mapping: {'52': 'I', '34': 'N', '44': 'T', '54': 'A', '84': 'N', '67': 'G', '11': 'I', '36': 'B', '25': 'L', '18': 'E', '62': 'S'}
</B-CLUE>

... ...

Now let's gather all the mappings in order:
<Answer>
{'1': 'M', '2': 'A', '3': 'Y', '4': 'B', '5': 'E', '6': 'D', '7': 'E', '8': 'A', '9': 'T', '10': 'H', '11': 'I', '12': 'S', '13': 'T', '14': 'H', '15': 'E', '16': 'G', '17': 'R', '18': 'E', '19': 'A', '20': 'T', '21': 'E', '22': 'Q', '23': 'U', '24': 'A', '25': 'L', '26': 'I', '27': 'S', '28': 'E', '29': 'R', '30': 'T', '31': 'H', '32': 'E', '33': 'O', '34': 'N', '35': 'E', '36': 'B', '37': 'I', '38': 'G', '39': 'T', '40': 'H', '41': 'I', '42': 'N', '43': 'G', '44': 'T', '45': 'H', '46': 'A', '47': 'T', '48': 'C', '49': 'A', '50': 'N', '51': 'F', '52': 'I', '53': 'N', '54': 'A', '55': 'L', '56': 'L', '57': 'Y', '58': 'M', '59': 'A', '60': 'K', '61': 'E', '62': 'S', '63': 'T', '64': 'R', '65': 'A', '66': 'N', '67': 'G', '68': 'E', '69': 'R', '70': 'S', '71': 'S', '72': 'H', '73': 'E', '74': 'D', '75': 'A', '76': 'T', '77': 'E', '78': 'A', '79': 'R', '80': 'F', '81': 'O', '82': 'R', '83': 'O', '84': 'N', '85': 'E', '86': 'A', '87': 'N', '88': 'O', '89': 'T', '90': 'H', '91': 'E', '92': 'R'}
</Answer>
\end{CustomVerbatim}


\textbf{[Example 2]} \\

<Example 2> \\

\textbf{[Solution 2]} \\

<Solution 2> \\

\textbf{[Solution Guideline]} \\

Please solve the Acrostic Puzzle according to the provided rules. Please also follow the requests below to present your analysis and solutions: \\

1. Analyze the clue carefully to understand its meaning and potential word associations. Be open to the possibility of wordplay or puns that might lead to unexpected solutions. The filled-in words will help you infer the answers to the subsequent clues. \\

2. Provide the final answer for each clue. The final answer should be presented after "Answer:". \\

3. Create a python dictionary mapping in a single line that links the number positions in the clues to the corresponding letters in the final answer. Note that the blank spaces and punctuation should be omitted in the mapping. The python dictionary should be presented after "Mapping:". \\

4. Please wrap all the analysis of each clue with <n-CLUE> and </n-CLUE> tags, where n is the label of the clue. For example, the first clue should be wrapped with <A-CLUE> and </A-CLUE> tags. The second clue should be wrapped with <B-CLUE> and </B-CLUE> tags, and so on. \\

5. After solving all the clues, please gather all the mappings and place them in order between the <Answer> and </Answer> tags. The gathered mapping should look like \{'1': 'LETTER1', '2': 'LETTER2', ...\}. \\

6. Please generate your response without redundant and repeating content. \\

\textbf{[Question]} \\

<Question>
\end{tcolorbox}

\subsection{Logic Puzzle}
\begin{tcolorbox}[breakable, title=Prompt for Logic Puzzle]
\small
\textbf{[Task Description]} \\

Logic puzzles require the solver to deduce the relationships between different people, places and things based on a limited number of clues given in the puzzle. Remember: every item belongs to one and only one person, no item will ever be shared. Using only the clues provided and simple deductive logic and reasoning. \\

\textbf{[Example 1]}

\begin{CustomVerbatim}
Here is a logic puzzle:
{
    "variables": {
        "names": [
            "Fiona",
            "Tobias",
            "Leanne"
        ],
        "main dishes": [
            "Beef Wellington",
            "Gado Gado",
            "Shrimp Ceviche"
        ],
        "desserts": [
            "Creme Brulee",
            "Raspberry Tiramisu",
            "French Apple Cake"
        ]
    },
    "clues": {
        "1": "The three people are Tobias, the one who brought the Gado Gado, and the one who brought the Raspberry Tiramisu.",
        "2": "Fiona did not bring the Beef Wellington.",
        "3": "The Shrimp Ceviche was brought by either Tobias or the person who brought the Creme Brulee.",
    }
}

Please complete the following blank based on your final solution:
[
    {
        "names": "Fiona",
        "main dishes": "____",
        "desserts": "____",
    },
    {
        "names": "Tobias",
        "main dishes": "____",
        "desserts": "____",
    },
    {
        "names": "Leanne",
        "main dishes": "____",
        "desserts": "____",
    }
]
\end{CustomVerbatim}

\textbf{[Solution 1]}

\begin{CustomVerbatim}
Let's start solving the logic puzzle step by step.

Participants and Dishes
Names: Fiona, Tobias, Leanne
Main Dishes: Beef Wellington, Gado Gado, Shrimp Ceviche
Desserts: Crème Brûlée, Raspberry Tiramisu, French Apple Cake


Interpreting Clue 1:
    Roles:
        Tobias: Did not bring Gado Gado or Raspberry Tiramisu.
        Gado Gado Bringer: Not Tobias.
        Raspberry Tiramisu Bringer: Not Tobias.

    Conclusion:
        Fiona and Leanne must be the ones who brought Gado Gado and Raspberry Tiramisu, respectively (or vice versa).

... ...

Final Answer in JSON Format:
<Answer>
[
    {
        "names": "Fiona",
        "main dishes": "Gado Gado",
        "desserts": "Creme Brulee",
    },
    {
        "names": "Tobias",
        "main dishes": "Shrimp Ceviche",
        "desserts": "French Apple Cake",
    },
    {
        "names": "Leanne",
        "main dishes": "Beef Wellington",
        "desserts": "Raspberry Tiramisu",
    }
]
</Answer>
\end{CustomVerbatim}

\textbf{[Example 2]} \\

<Example 2> \\

\textbf{[Solution 2]} \\

<Solution 2> \\

\textbf{[Solution Guideline]} \\

Please solve the Logic Puzzle according to the provided rules. Please also follow the requests below to present your analysis and solutions: \\

1. Read and understand each clue in the context of the puzzle. Apply each clue one by one to deduce the correct arrangement of different variables. \\

2. Use logical reasoning to figure out the relationships between the variables based on the clues provided. \\

3. After solving the puzzle, present your final solution using JSON Format and wrap it with the <Answer> and </Answer> tags. For example:

\begin{CustomVerbatim}
<Answer>
[
    {
        "Variable1": "Value1",
        "Variable2": "Value2",
        ...
    },
    {
        "Variable1": "Value3",
        "Variable2": "Value4",
        ...
    },
    ...
]
</Answer>
\end{CustomVerbatim}

4. Please generate your response without redundant and repeating content. \\

\textbf{[Question]} \\

<Question>
\end{tcolorbox}



\subsection{Cryptogram}
\begin{tcolorbox}[breakable, title=Prompt for Cryptogram]
\small
\textbf{[Task Description]} \\

A cryptogram is a type of puzzle that consists of a short piece of encrypted text. You will decode a cryptogram that uses a simple substitution cipher, where each letter in the message has been substituted with another letter. The goal is to decode the hidden quote along with its author's name. You may use the author's name to help reduce the difficulty of the decoding process by focusing on common words, letter patterns, and stylistic cues from the author. \\

\textbf{[Example 1]} \\

Think step-by-step to decipher the cryptogram: \\
VOBB XSDO QK IOFFOU FGHD VOBB KHQX. \\
— Benjamin Franklin \\

\textbf{[Solution 1]} \\

Let's start solving the cryptogram puzzle step by step. \\

Step 1: Use the Author's Name as a Clue Benjamin Franklin is known for his concise and wise sayings. Recognizing his style helps anticipate common structures and themes, such as practical advice or moral lessons. \\

Step 2: Analyze the Cryptogram The cryptogram is: \\
VOBB XSDO QK IOFFOU FGHD VOBB KHQX. -Benjamin Franklin \\

Notable features: \\
"VOBB" appears twice, suggesting a common word with double letters. \\
The structure of the sentence hints at a proverb or a well-known quote. \\

... ... \\

\begin{CustomVerbatim}
Mapping:
<Mapping>
{ "V": "W", "O": "E", "B": "L", "X": "D", "S": "O", "D": "N", "Q": "I", "K": "S", "I": "B", "F": "T", "U": "R", "G": "H", "H": "A"}
</Mapping>

Answer:
<Answer>
Well done is better than well said.
-Benjamin Franklin
</Answer>
\end{CustomVerbatim}

\textbf{[Example 2]} \\

<Example 2> \\

\textbf{[Solution 2]} \\

<Solution 2> \\

\textbf{[Solution Guideline]} \\

Please follow these steps to solve the Cryptogram: \\

1. Use the author's name as a clue: Knowing the author's name can help you predict common words or letter combinations typical for this author. For instance, if the author is "Shakespeare," you might anticipate archaic or common Shakespearean phrases (like "thou," "thee," etc.). \\

2. Analyze the cryptogram: Look at the frequency of letters and common letter patterns, such as double letters or common suffixes and prefixes. Focus on the parts of the cryptogram that seem to match the author's typical writing style or famous phrases. \\

3. Map common words: If you recognize a word in the cryptogram that matches the author's typical vocabulary, substitute letters based on that. \\

4. Make educated guesses: Use common English words (such as "the," "and," "of," etc.) and letter pairs (like "th," "he," "in", etc.) to identify possible substitutions. If one assumption doesn't work, try another. \\

5. Verification: After generating the decoded message, check if the quote and the author's name make logical sense. If needed, revisit the assumptions and adjust the letter mappings. \\

6. Please provide step by step analysis and create a python dictionary mapping of the fully substitutioin wrapping it between <Mapping> and </Mapping> tag. \\

7. Please provide the final decoded quote and wrap it between <Answer> and </Answer> tag. \\

8. Please generate your response without redundant and repeating content. \\

\textbf{[Question]} \\

<Question>
\end{tcolorbox}





\subsection{Sudoku}
\begin{tcolorbox}[breakable, title=Prompt for Sudoku]
\small
\textbf{[Task Description]} \\

Sudoku is a logic-based, combinatorial number-placement puzzle. The puzzle consists of an $n^2 \times n^2$ grid, partially filled with numbers from 1 to $n^2$ and 0 (empty cells). The objective is to fill the grid so that each row, each column, and each $n \times n$ subgrid must contain every number from 1 to $n^2$ exactly once. \\

\textbf{[Example 1]}

\begin{CustomVerbatim}
Here is a 4 x 4 sudoku puzzle:
[[2, 0, 0, 0],
 [0, 0, 3, 0],
 [0, 4, 0, 0],
 [0, 0, 0, 1]]
\end{CustomVerbatim}


\textbf{[Solution 1]}

\begin{CustomVerbatim}
Let's start solving the sudoku puzzle step by step.

Step1: Analyze Empty Cell in Row 1
Cell (1,2):
    Eliminated Numbers: 2 (already in row), 4 (already in column)
    Possible Numbers: 1, 3
Cell (1,3):
    Eliminated Numbers: 2 (already in row), 3 (already in column)
    Possible Numbers: 1, 4
Cell (1,4):
    Eliminated Numbers: 2 (already in row), 1 (already in column), 3 (already in subgrid)
    Only possible number: 4

... ...

Step 5: Final Answer
<Answer>
[[2, 3, 1, 4],
 [4, 1, 3, 2],
 [1, 4, 2, 3],
 [3, 2, 4, 1]]
</Answer>
\end{CustomVerbatim}

\textbf{[Example 2]} \\

<Example 2> \\

\textbf{[Solution 2]} \\

<Solution 2> \\

\textbf{[Solution Guideline]} \\

Please solve the Sudoku Puzzle according to the provided rules. Please also follow the requests below to present your analysis and solutions: \\

1. For each empty cell, try to deduce which numbers are allowed based on the existing numbers in its row, column, and subgrid. \\

2. If you reach a point where no obvious choices are available, you may need to backtrack. Try filling in a number and see if it leads to a valid solution. If it causes a contradiction, backtrack to a previous decision and try a different path. \\

3. Please provide step by step analysis and present the final answer as the same json format of the input grid. Wrapping the final answer with <Answer> and </Answer> tags. \\

4. Please generate your response without redundant and repeating content. \\

\textbf{[Question]} \\

<Question>
\end{tcolorbox}





\subsection{Drop Quote}
\begin{tcolorbox}[breakable, title=Prompt for Drop Quote]
\small
\textbf{[Task Description]} \\

A Drop Quote Puzzle consists of two parts: a grid and a set of given letters that "drop" into specific columns of the grid. The goal is to fill in the grid with a meaningful quote or statement. \\

Key Features: \\
1. The grid contains numbers, "\#" symbols (representing spaces between words), and already placed punctuation. \\

2. Words may break across lines, starting at the end of one line and continuing to the beginning of the next. Words end only at "\#" blocks or the end of the puzzle. \\

3. The letters are organized by column number and are given in alphabetical order. These letters need to be placed into the grid. \\

4. Each letter drops into a specific column, and your task is to identify the correct placement for each letter based on the grid structure. Do not simply fill the columns in the order they are listed. Instead, consider where each letter fits contextually within the puzzle to form words that make sense in the final quote or statement. \\

Rules: \\
1. The "\#" symbols indicate where spaces between words should be. \\

2. Column labels (e.g., Column 1: O T) indicate the letters that will drop into the grid for Column 1, listed in alphabetical order. For example, in Column 1, the letters "O" and "T" must be placed in the appropriate grid positions. \\

3. Fill the grid with the letters in such a way that they form a meaningful statement or quotation. \\

\textbf{[Example 1]}

\begin{CustomVerbatim}
Here is a drop quote puzzle:
[['1', '2', '3', '4', '#', '6', '7'],
 ['#', '2', '3', '4', '5', '6', '.']]

Column 1: T
Column 2: I M
Column 3: M O
Column 4: E N
Column 5: E
Column 6: I Y
Column 7: S
\end{CustomVerbatim}


\textbf{[Solution 1]}

\begin{CustomVerbatim}
Let's start solving the sudoku puzzle step by step.

We first focus on Columns with single letters:
    "T" for Row 1, Column 1
    "E" for Row 2, Column 5
    "S" for Row 1, Column 7

Then, we check the word ['6', '7'] in Row 1:
    The candidates for Column 6 are "I" and "Y", and for Column 7, it is "S". The only word that fits is "IS". So, place "I" in Row 1, Column 6 and "Y" in Row 2, Column 6.

Let's check the current grid:
[['T', '2', '3', '4', '#', 'I', 'S'],
 ['#', '2', '3', '4', 'E', 'Y', '.']]

... ...

Here is the final answer gird:
<Answer>
[['T', 'I', 'M', 'E', '#', 'I', 'S'],
 ['#', 'M', 'O', 'N', 'E', 'Y', '.']]
</Answer>
\end{CustomVerbatim}

\textbf{[Example 2]} \\

<Example 2> \\

\textbf{[Solution 2]} \\

<Solution 2> \\

\textbf{[Solution Guideline]} \\

Please solve the Drop Quote Puzzle according to the provided rules. Please also follow the requests below to present your analysis and solutions: \\

1. Isolated boxes are usually, but not always, the words A or I. Similarly, when you have a 2-letter word, see what words you can make and then see what letters that leaves you over with for other lines. The most common 2-letter words are: OF, TO, IN, IT, IS, BE, AS, AT, SO, WE, HE, BY, OR, ON, DO, IF, ME, MY, UP, AN, GO, NO, US, AM. \\

2. The letters in a column with fewer letters can be placed quickly by noticing which letters must be consonants and vowels, or by eliminating the possibility of a letter appearing in a certain spot. \\

3. If you have an uncommon letter such as a J, K, Q, X, or Z, think about what letters might go before or after it. Similarly, look for common consonant pairs such as TH, ND, NT, ST, as well as doubled letters. \\

4. Be alert for common prefixes such as DE-, MIS-, RE-, and UN-, as well as common suffixes such as -ABLE, -ED, -ING, -LY, -NESS, and -TION. \\

5. Every time you fill in a letter, it means that you've eliminated some letters that can go elsewhere. Frequently, this means that you can immediately fill in some additional letters. \\

6. Please provide step by step analysis and present the final answer as the same json format of the input grid. Wrapping the final answer in <Answer> and </Answer> tags. \\

7. Please generate your response without redundant and repeating content. \\

\textbf{[Question]} \\

<Question>
\end{tcolorbox}


\section{Models}
\label{app_models}
As shown in Table \ref{models}, we list the key details for eleven open-source models and five closed-source models in our evaluation.

\section{Task Difficulty}
\label{app_difficulty}
Table \ref{space} analyzes the difficulty of each type of task through the size of solution space. We approximately calculate the solution space by multiplying the number of elements to be filled (e.g., grid size for Crossword) by the number of possible answers for each element (e.g., 26 letters for Crossword). Detailed performance across varying difficulty levels for each task is presented in Tables \ref{level_crossword}, \ref{level_acrostic_cryptogram}, \ref{level_logic}, \ref{level_sudoku}, and \ref{level_drop}. Our findings indicate that the CR score does not significantly decrease with increasing task difficulty. This suggests that the long-chain generation capability is likely an inherent property of these models, rather than being strongly task-dependent. However, as the solution space expands with increasing difficulty, we observe a consistent performance decline across all models. In the most challenging settings, even the top-performing model achieves an EM score of less than 10\%, highlighting the difficulty of our benchmark for evaluating LLMs' reflective reasoning capabilities.
\begin{table}[t]
\centering
\resizebox{1.0\columnwidth}{!}{\begin{tabular}{llc}
\toprule
\textbf{Task} &  
\textbf{Difficulty} &

\begin{tabular}[c]{@{}c@{}}\hspace{0.5em} \textbf{Solution Space} \hspace{0.5em} \\ \textbf{Upper Bound} \end{tabular} \\
\midrule
\multirow{3}{*}{Crossword} & $5\times5$ & $26\times5\times5$ \\
& $10\times10$ & $26\times10\times10$ \\
& $15\times15$ & $26\times15\times15$ \\
\midrule
\multirow{2}{*}{Acrostic} & Easy & $26\times110$ \\
& Hard & $26\times214$ \\
\midrule
\multirow{4}{*}{Logic Puzzle} & $4\times4$ & ${4!}^3$ \\
& $4\times5$ & ${4!}^4$ \\
& $4\times6$ & ${4!}^5$ \\
& $4\times7$ & ${4!}^6$ \\
\midrule
\multirow{2}{*}{Cryptogram} & Easy & $26!$ \\
& Hard & $26!$ \\
\midrule
\multirow{4}{*}{Sudoku} & $4\times4$, Easy & $4^{16}$ \\
& $4\times4$, Hard & $4^{16}$ \\
& $9\times9$, Easy & $9^{81}$ \\
& $9\times9$, Hard & $9^{81}$ \\
\midrule
\multirow{2}{*}{Drop Quote} & Easy & $15\times3!$ \\
& Hard & $20\times5!$ \\
\bottomrule
\end{tabular}}
\caption{The solution space upper bound of each type of task.}
\label{space}
\end{table}


% \begin{table}[t]
%     \centering
%     \resizebox{1.0\columnwidth}{!}{
%     \begin{tabular}{lcc}
%         \toprule
%         \textbf{Task} & \textbf{Difficulty} & \textbf{Solution Space Upper Bound}\\
%         \midrule
%         \multirow{3}{*}{Crossword} & $5\times5$ & $26\times5\times5$ \\
%         & $10\times10$ & $26\times10\times10$ \\
%         & $15\times15$ & $26\times15\times15$ \\
%         \midrule
%         \multirow{2}{*}{Acrostic} & Easy & $26\times110$ \\
%         & Hard & $26\times214$ \\
%         \midrule
%         \multirow{4}{*}{Logic Puzzle} & $4\times4$ & ${4!}^3$ \\
%         & $4\times5$ & ${4!}^4$ \\
%         & $4\times6$ & ${4!}^5$ \\
%         & $4\times7$ & ${4!}^6$ \\
%         \midrule
%         \multirow{2}{*}{Cryptogram} & Easy & $26!$ \\
%         & Hard & $26!$ \\
%         \midrule
%         \multirow{4}{*}{Sudoku} & $4\times4$, Easy & ${4!}^3$ \\
%         & $4\times4$, Hard & ${4!}^4$ \\
%         & $9\times9$, Easy & ${4!}^5$ \\
%         & $9\times9$, Hard & ${4!}^6$ \\
%         \midrule
        

%     \end{tabular}}    
%     \caption{.}
%     \label{tab:main_exp_sum}
% \end{table}





\section{Response Cases}
\label{app_case}

\subsection{Lack of Reflective Reasoning}
\label{lack}
The following example shows that GPT-4o fails to solve the logic problem due to the lack of reflective reasoning capabilities. The model only performs a single turn of deduction of all the clues and outputs the final answers with insufficient evidence.

\begin{tcolorbox}[breakable, colback=blue!8, title=Example 1: Response of GPT-4o to Logic Puzzle]
\small
\begin{Verbatim}[formatcom=\normalfont, breaklines=true, breaksymbolleft={}, breaksymbolright={}, commandchars=\\\{\}]
To solve the logic puzzle, let's analyze the clues and deduce the relationships between the variables.

### Step-by-Step Analysis:

1. **Clue 1**: The commuter who rides a 10-speed bike is not Bradley.
   - Bradley does not ride a 10-speed bike.

2. **Clue 2**: Kayla isn't related to Robert Duvall and doesn't ride a segway.
   - Kayla's actor is not Robert Duvall, and her conveyance is not a segway.

3. **Clue 3**: Robert Duvall's cousin has an anniversary after the commuter who rides a segway.
   - The segway rider's anniversary is before Robert Duvall's cousin.

4. **Clue 4**: Morgan Freeman's cousin is not Asher and doesn't ride a segway.
   - Morgan Freeman's cousin is not Asher and does not ride a segway.

5. **Clue 5**: Liam Neeson's cousin is Malia.
   - Malia's actor is Liam Neeson.

6. **Clue 6**: Liam Neeson's cousin uses a mountain bike to get to work.
   - Malia rides a mountain bike.

7. **Clue 7**: Of Malia and the commuter who rides a 10-speed bike, one has an anniversary on January 28 and the other has an anniversary on November 23.
   - Malia or the 10-speed bike rider has an anniversary on January 28 or November 23.

8. **Clue 8**: Either the person whose anniversary is on March 6 or the person whose anniversary is on November 2 is Morgan Freeman's cousin.
   - Morgan Freeman's cousin has an anniversary on March 6 or November 2.

9. **Clue 9**: Tom Cruise's cousin has an anniversary after the commuter who rides a skateboard.
   - The skateboard rider's anniversary is before Tom Cruise's cousin.

### Deductions:

- From Clue 5 and Clue 6, Malia rides a mountain bike and is Liam Neeson's cousin.

- From Clue 7, Malia's anniversary is either January 28 or November 23.

- From Clue 8, Morgan Freeman's cousin has an anniversary on March 6 or November 2.

- From Clue 9, the skateboard rider's anniversary is before Tom Cruise's cousin.

### Assignments:

- **Malia**: Since Malia rides a mountain bike and is Liam Neeson's cousin, she cannot be Morgan Freeman's cousin. Therefore, Malia's anniversary is not March 6 or November 2. Thus, Malia's anniversary is January 28 or November 23.

- **Kayla**: Kayla is not related to Robert Duvall and doesn't ride a segway. She could be Morgan Freeman's cousin with an anniversary on March 6 or November 2.

- **Bradley**: Bradley does not ride a 10-speed bike. He could be the skateboard rider.

- **Asher**: Asher is not Morgan Freeman's cousin, so he could be Tom Cruise's cousin.

### Final Solution:

Based on the deductions and assignments, the final solution is:

<Answer>
[
    \{
        "anniversaries": "January 28",
        "first names": "Malia",
        "conveyances": "mountain bike",
        "actors": "Liam Neeson"
    \},
    \{
        "anniversaries": "March 6",
        "first names": "Kayla",
        \textcolor{red}{"conveyances": "10-speed bike", \ding{55}}
        "actors": "Morgan Freeman"
    \},
    \{
        "anniversaries": "November 2",
        "first names": "Bradley",
        \textcolor{red}{"conveyances": "skateboard", \ding{55}}
        \textcolor{red}{"actors": "Robert Duvall \ding{55}}
    \},
    \{
        "anniversaries": "November 23",
        "first names": "Asher",
        \textcolor{red}{"conveyances": "segway", \ding{55}}
        \textcolor{red}{"actors": "Tom Cruise" \ding{55}}
    \}
]
</Answer>
\normalsize \textcolor{red}{\textit{\textbf{
- Only single deduction
- Insufficient evidence
- Lack of iterative deduction to ensure consistency with all constraints}}}
\end{Verbatim}
\end{tcolorbox}

\subsection{Stuck in Contradictions}
\label{stuck}
The following example shows that Llama-3.1-70B falls into repetition of the same sentences when discovering contradictions. Instead of backtracking and revising its assumptions when a contradiction is found (e.g., Cassidy cannot finish at both 3 min 2 sec and 3 min 8 sec), the model appears to get stuck in a loop, repeatedly stating the conflicting information without making progress.
\begin{tcolorbox}[breakable, colback=blue!8, title=Example 2: Response of Llama-3.1-70B to Logic Puzzle]
\small
\begin{Verbatim}[formatcom=\normalfont, breaklines=true, breaksymbolleft={}, breaksymbolright={}, commandchars=\\\{\}]
Let's start solving the logic puzzle step by step.

Participants and Attributes
Times: 2 min, 59 sec, 3 min, 2 sec, 3 min, 8 sec, 3 min, 14 sec
First Names: Cassidy, Eva, Grace, Savannah
Breads: challah, flatbread, focaccia, matzo
Fruits: currants, kiwis, oranges, pineapples

Applying Clue 1:
    Either the person who bought the matzo or Cassidy finished at 2 min, 59 sec and the other grows pineapples.
    Clue 8 states that Savannah finished at 2 min, 59 sec.
    Therefore, Cassidy grows pineapples and did not buy the matzo.

Applying Clue 8:
    Savannah finished at 2 min, 59 sec.
    Clue 6 states that Savannah finished before the person who bought the focaccia.
    Therefore, the person who bought the focaccia finished at 3 min, 2 sec, 3 min, 8 sec, or 3 min, 14 sec.

Applying Clue 4:
    The person who grows currants bought the focaccia.
    Clue 2 states that either the runner who finished at 3 min, 2 sec or 3 min, 8 sec grows currants.
    Therefore, the person who bought the focaccia finished at 3 min, 2 sec or 3 min, 8 sec.

Applying Clue 9:
    The person who grows currants finished before Grace.
    Since the person who bought the focaccia finished at 3 min, 2 sec or 3 min, 8 sec, Grace finished at 3 min, 14 sec.

Applying Clue 5:
    The runner who finished at 3 min, 14 sec did not buy the challah.
    Therefore, Grace bought the matzo.

Applying Clue 3:
    Cassidy did not finish at 3 min, 2 sec and did not buy the flatbread.
    Since Cassidy did not buy the matzo (Clue 1) and did not buy the flatbread, Cassidy bought the challah.

Applying Clue 7:
    The person who grows oranges is not Grace.
    Since Cassidy grows pineapples, the person who grows oranges is either Eva or Savannah.

Assigning Remaining Attributes:
    Eva or Savannah grows oranges.
    Since Savannah finished at 2 min, 59 sec and the person who bought the focaccia finished at 3 min, 2 sec or 3 min, 8 sec, Eva grows oranges and bought the flatbread.
    Savannah grows kiwis and bought the focaccia is incorrect because the person who bought the focaccia finished at 3 min, 2 sec or 3 min, 8 sec.
    Therefore, Savannah bought the challah is incorrect, and Eva bought the flatbread and grows oranges.
    The person who bought the focaccia finished at 3 min, 2 sec or 3 min, 8 sec.
    Since the person who grows currants finished before Grace, the person who bought the focaccia finished at 3 min, 2 sec.
    Therefore, the person who finished at 3 min, 8 sec grows pineapples is incorrect, and Cassidy finished at 3 min, 8 sec is incorrect.
    \textcolor{red}{Cassidy finished at 3 min, 14 sec is incorrect because Grace finished at 3 min, 14 sec.}
    \textcolor{red}{Therefore, Cassidy finished at 3 min, 2 sec is incorrect, and Cassidy finished at 3 min, 8 sec is incorrect.}
    \textcolor{red}{Cassidy finished at 3 min, 14 sec is incorrect because Grace finished at 3 min, 14 sec.}
    \textcolor{red}{Therefore, Cassidy finished at 3 min, 2 sec is incorrect, and Cassidy finished at 3 min, 8 sec is incorrect.}
    \textcolor{red}{... ...}
    \normalsize \textcolor{red}{\textit{\textbf{
    - Looping within the same sentence}}}

\end{Verbatim}
\end{tcolorbox}

The following example indicates that QwQ-32B struggles with contraction involving multiple interconnected variables. The primary issue with the reasoning is its disorganized and cyclical behavior. The model jumps between different deductions and assumptions without systematically eliminating possibilities or locking in firm conclusions.

\begin{tcolorbox}[breakable, colback=blue!8, title=Example 3: Response of QwQ-32B to Logic Puzzle]
\small
\begin{Verbatim}[formatcom=\normalfont, breaklines=true, breaksymbolleft={}, breaksymbolright={}, commandchars=\\\{\}]
Let's tackle this logic puzzle step by step. It's a bit complex with multiple variables, but I'll try to break it down systematically.

First, let's list out all the variables we have:

... ...

Now, let's look at the anniversaries.

... ...

\textcolor{red}{So, if Bradley has Mar 6}, Tom Cruise's cousin must have Nov 2.

\textcolor{red}{If Bradley has Nov 2}, Tom Cruise's cousin must have a later date, but Nov 23 is after Nov 2, but Nov 23 is already assigned to either Malia or Kayla.

Wait, anniversaries are Jan 28, Mar 6, Nov 2, Nov 23.

... ...

\textcolor{red}{So, if Bradley has Mar 6}, Tom Cruise's cousin has Nov 2.

\textcolor{red}{If Bradley has Nov 2}, Tom Cruise's cousin has Nov 23.

But Nov 23 is assigned to either Malia or Kayla.

... ...

So, \textcolor{red}{if Bradley has Nov 2}, Tom Cruise's cousin must have Nov 23, but Nov 23 is assigned to either Malia or Kayla.

Wait, that can't be, because Tom Cruise's cousin must be someone else.

... ...

\textcolor{red}{If Bradley has Mar 6}, then Tom Cruise's cousin has Nov 2.

... ...

\textcolor{red}{But if Bradley has Nov 2}, Tom Cruise's cousin must have Nov 23, which is assigned to Malia or Kayla.

This seems conflicting.

Wait, perhaps I need to consider the possible combinations.

Let's consider two scenarios:

\textcolor{red}{**Scenario 1: Bradley has Mar 6**}

... ...

Therefore, Scenario 1 is invalid.

\textcolor{red}{Let's consider **Scenario 2: Bradley has Nov 2**}

... ...

Therefore, this assignment is also invalid.

... ...

Given the complexity, I'll have to conclude with the best possible answer based on the steps I've taken.

... ...
\normalsize \textcolor{red}{\textit{\textbf{
- Stuck in a reasoning cycle
- Jumping between different deductions and assumptions}}}

\end{Verbatim}
\end{tcolorbox}


\subsection{Give-up Moment}
\label{giveup}
The following examples show the "Give-up Moment" in the responses of QwQ-32B and DeepSeek-R1 to the Cryptogram task. Both models, faced with the complexity of the problem and the constraints of inference time, reach a point where they give up further decryption attempts.


\begin{tcolorbox}[breakable, colback=blue!8, title=Example 4: Response of QwQ-32B to Cryptogram]
\small
\begin{Verbatim}[formatcom=\normalfont, breaklines=true, breaksymbolleft={}, breaksymbolright={}, commandchars=\\\{\}]
I'm going to try to solve this cryptogram step by step. The cryptogram is:

HNG UGEJ LROUKG RT EJXRLH EJJ ROU KUCXGL, CT HNG HUROAJG CL HEWGF HR HUEKG HNGX HR E KRXXRF RUCICF, YCJJ AG TROFQ HR AG CF CQJGFGLL.

— Walter Gaston Shotwell

First, I need to understand that this is a substitution cipher, where each letter in the original text is replaced by another letter. 

My goal is to figure out which letter corresponds to which in the original text.

... ...

But that might not be accurate.

Alternatively, "KRXXRF" could be "SELVES", so X = L, R = V, F = E.

But again, just a guess.

This is getting too speculative.

\textcolor{red}{Given the time constraints, I think I'll stop here and try to provide the best possible mapping and decryption based on my attempts.}

<Mapping>
\{"H": "T", "N": "H", "G": "E", "U": "O", "J": "A", "L": "R", "R": "B", "E": "I", "T": "S", "O": "C", "K": "D", "C": "M", "X": "P", "F": "U", "Q": "W", "A": "F", "W": "G", "S": "Y", "Y": "K", "I": "Z"\}
</Mapping>

<Answer>
THE OIA ARBORE BIS IAUPRH IAA COU MPORHE, CY THE THUOAFM CI HIWGF HB HUEKG THNG HB I KRXPPU MUCICF, YAJJ AF TROFQ HB AF CF CQJGFGLL. — Walter Gaston Shotwell
</Answer>
\normalsize \textcolor{red}{\textit{\textbf{
- Occurrence of "Give-up Moment"}}}

\end{Verbatim}
\end{tcolorbox}


\begin{tcolorbox}[breakable, colback=blue!8, title=Example 5: Response of DeepSeek-R1 to Cryptogram]
\small
\begin{Verbatim}[formatcom=\normalfont, breaklines=true, breaksymbolleft={}, breaksymbolright={}, commandchars=\\\{\}]
<think>
Okay, let's tackle this cryptogram. The author is Charles Baudelaire, a French poet known for his symbolic and often darkly romantic style. That might not immediately help with common phrases, but maybe some themes like beauty, art, or suffering. Let's look at the cryptogram:

AM AR YIPYIMMSQWI MNSM, SLDBP MNI YAPNMR DV LSB, MNI YAPNM DV HDBMYSFAHMABP DBIRIWV NSR QIIB VDYPDMMIB. — Charles Baudelaire

First, I'll note the structure. The first part ends with a comma, then another phrase, and a longer sentence. Let's break it down into parts.

... ...

But "corruption" is 10 letters, YIPYIMMSQWI is 11 letters. So maybe with a comma: "Art is the corruption, the soul..."

But this is not helpful.

\textcolor{red}{Given the time I've spent and the need to provide an answer, I'll proceed to hypothesize the final answer based on common Baudelaire themes and possible mappings.}
</think>

<Mapping>
{'A': 'T', 'M': 'H', 'R': 'E', 'Y': 'B', 'I': 'U', 'P': 'A', 'S': 'L', 'Q': 'I', 'W': 'G', 'N': 'S', 'D': 'O', 'B': 'R', 'V': 'M', 'L': 'F', 'F': 'Y', 'H': 'P', 'G': 'C', 'C': 'D', 'K': 'K', 'X': 'X', 'Z': 'Z'}
</Mapping>

<Answer>
The beautiful is the shadow of the true.
— Charles Baudelaire
</Answer>
\normalsize \textcolor{red}{\textit{\textbf{
- Occurrence of "Give-up Moment"}}}

\end{Verbatim}
\end{tcolorbox}



% \setlength{\tabcolsep}{4pt}
\begin{table*}[!h]
    \centering
    {\resizebox{0.8\textwidth}{!}{\begin{tabular}{lcrl}
        \toprule
        \textbf{Model} & \textbf{Architecture} & \textbf{Size} & \textbf{Hugginggface/API} \\
        \midrule
        \rowcolor{gray!15} \multicolumn{4}{l}{\textit{\textbf{Open-source LLMs}}}\\
        \midrule 
        Llama-3.1-8B-Instruct \citep{dubey2024llama} & Dense & 8B & meta-llama/Llama-3.1-8B-Instruct \\
        Llama-3.1-70B-Instruct \citep{dubey2024llama} & Dense & 70B & meta-llama/Llama-3.1-70B-Instruct \\
        Llama-3.3-70B-Instruct \citep{dubey2024llama} & Dense & 70B & meta-llama/Llama-3.3-70B-Instruct \\
        Mistral-7B-Instruct-v0.3 \citep{jiang2023mistral} & Dense & 7B & mistralai/Mistral-7B-Instruct-v0.3 \\
        Mistral-Small-Instruct-2409 \citep{jiang2023mistral} & Dense & 22B & mistralai/Mistral-Small-Instruct-2409 \\
        Mistral-Large-Instruct-2411 \citep{jiang2023mistral} & Dense & 123B & mistralai/Mistral-Large-Instruct-2411 \\
        Qwen2.5-7B-Instruct \citep{yang2024qwen2} & Dense & 7B & Qwen/Qwen2.5-7B-Instruct \\
        Qwen2.5-32B-Instruct \citep{yang2024qwen2} & Dense & 32B & Qwen/Qwen2.5-32B-Instruct \\
        Qwen2.5-72B-Instruct \citep{yang2024qwen2} & Dense & 72B & Qwen/Qwen2.5-72B-Instruct \\
        QwQ-32B-Preview \citep{qwq-32b-preview} & Dense & 32B & Qwen/QwQ-32B-Preview \\
        DeepSeek-R1 \citep{guo2025deepseek} & MoE & 671B & deepseek-ai/DeepSeek-R1 \\
        \midrule
        \rowcolor{blue!15} \multicolumn{4}{l}{\textit{\textbf{Closed-source LLMs}}} \\
        Gemini-2.0-flash \citep{deepmind_gemini_flash} & - & - & \texttt{gemini-2.0-flash-exp} \\
        Gemini-2.0-flash-thinking \citep{deepmind_gemini_flash_thinking} & - & - & \texttt{gemini-2.0-flash-thinking-exp-1219} \\
        GPT-4o \citep{gpt4o} & - & - & \texttt{gpt-4o-2024-08-06} \\
        o1-mini \citep{o1} & - & - & \texttt{o1-mini-2024-09-12} \\
        o1-preview \citep{o1} & - & - & \texttt{o1-preview-2024-09-12} \\
        \bottomrule
    \end{tabular}}}
    \caption{\label{models}Detailed information of all evaluated models in LR${}^{2}$Bench.} 
\end{table*}
\input{table/level_Crossword}
\input{table/level_Acrostic_cryptogram}
\begin{table*}[t]
    \centering
    \resizebox{1.0\textwidth}{!}{
    \setlength{\tabcolsep}{3pt}
    \begin{tabular}{l|ccccc|ccccc|ccccc|ccccc}
        \toprule
        \multirow{2}{*}{\textbf{Model}} &  \multicolumn{5}{c|}{\textbf{Logic Puzzle - $4\times4$}} & \multicolumn{5}{c|}{\textbf{Logic Puzzle - $4\times5$}}  & \multicolumn{5}{c}{\textbf{Logic Puzzle - $4\times6$}} & \multicolumn{5}{c}{\textbf{Logic Puzzle - $4\times7$}} \\
        \cmidrule(l){2-21}
        & CR & S-Acc & EM & PM-0.5 & Tokens & CR & S-Acc & EM & PM-0.5 & Tokens & CR & S-Acc & EM & PM-0.5 & Tokens & CR & S-Acc & EM & PM-0.5 & Tokens \\
        \midrule
        \rowcolor{gray!15} \multicolumn{21}{l}{\textit{\textbf{Open-source LLMs}}}\\
        \midrule 
Llama-3.1-8B-Instruct & 86.0 & 30.7 & 0.0 & 28.0 & 1049 & 52.0 & 12.8 & 0.0 & 0.0 & 1214 & 46.0 & 10.6 & 0.0 & 2.0 & 1225 & 44.0 & 9.8 & 0.0 & 2.0 & 1683 \\
Llama-3.1-70B-Instruct & 66.0 & 35.2 & 8.0 & 42.0 & 989 & 58.0 & 20.0 & 0.0 & 10.0 & 1126 & 54.0 & 18.9 & 0.0 & 12.0 & 1109 & 46.0 & 17.1 & 0.0 & 8.0 & 1435 \\
Llama-3.3-70B-Instruct & 88.0 & 46.7 & 4.0 & 52.0 & 1258 & 82.0 & 35.6 & 0.0 & 26.0 & 1576 & 90.0 & 25.4 & 0.0 & 12.0 & 1931 & 62.0 & 21.2 & 0.0 & 10.0 & 2188 \\
Mistral-7B-Instruct-v0.3 & \textbf{100.0} & 26.8 & 0.0 & 16.0 & 1289 & {\ul 98.0} & 18.0 & 0.0 & 2.0 & 1590 & 96.0 & 15.0 & 0.0 & 0.0 & 1846 & {\ul 94.0} & 16.8 & 0.0 & 0.0 & 1746 \\
Mistral-Small-Instruct-2409 & \textbf{100.0} & 38.7 & 2.0 & 34.0 & 1358 & \textbf{100.0} & 33.6 & 0.0 & 12.0 & 1424 & {\ul 98.0} & 23.7 & 0.0 & 2.0 & 1554 & \textbf{100.0} & 26.7 & 0.0 & 2.0 & 1721 \\
Mistral-Large-Instruct-2411 & \textbf{100.0} & 53.2 & 10.0 & 62.0 & 1293 & \textbf{100.0} & 40.4 & 2.0 & 30.0 & 1532 & \textbf{100.0} & {\ul 30.6} & 0.0 & 18.0 & 1772 & \textbf{100.0} & {\ul 29.1} & 0.0 & 12.0 & 1950 \\
Qwen2.5-7B-Instruct & \textbf{100.0} & 33.5 & 0.0 & 20.0 & 1133 & {\ul 98.0} & 28.8 & 0.0 & 14.0 & 1254 & 96.0 & 21.0 & 0.0 & 0.0 & 1494 & 92.0 & 19.7 & 0.0 & 0.0 & 1701 \\
Qwen2.5-32B-Instruct & \textbf{100.0} & 45.5 & 0.0 & 54.0 & 1070 & 92.0 & 34.3 & 0.0 & 20.0 & 1137 & 94.0 & 24.1 & 0.0 & 6.0 & 1241 & 86.0 & 25.0 & 0.0 & 10.0 & 1383 \\
Qwen2.5-72B-Instruct & {\ul 94.0} & 48.3 & 0.0 & 56.0 & 1490 & {\ul 98.0} & 34.8 & 0.0 & 16.0 & 1641 & 92.0 & 26.3 & 0.0 & 14.0 & 1931 & 90.0 & 26.7 & 0.0 & 6.0 & 2177 \\
QwQ-32B-Preview & {\ul 94.0} & {\ul 76.3} & {\ul 56.0} & {\ul 80.0} & 4766 & 86.0 & {\ul 52.7} & {\ul 14.0} & {\ul 64.0} & 8966 & 68.0 & 30.2 & {\ul 4.0} & {\ul 28.0} & 11292 & 66.0 & 25.8 & {\ul 4.0} & {\ul 20.0} & 13070 \\
DeepSeek-R1 & \textbf{100.0} & \textbf{94.2} & \textbf{90.0} & \textbf{94.0} & 4724 & \textbf{100.0} & \textbf{70.5} & \textbf{36.0} & \textbf{68.0} & 8907 & \textbf{100.0} & \textbf{50.2} & \textbf{16.0} & \textbf{50.0} & 11860 & \textbf{100.0} & \textbf{62.6} & \textbf{28.0} & \textbf{60.0} & 11329 \\
        \midrule
        \rowcolor{blue!15} \multicolumn{21}{l}{\textit{\textbf{Closed-source LLMs}}}\\
        \midrule
Gemini-2.0-flash & 70.0 & 35.3 & 6.0 & 36.0 & 1569 & 56.0 & 22.8 & 2.0 & 16.0 & 2283 & 62.0 & 23.8 & 0.0 & 18.0 & 2254 & 44.0 & 14.8 & 0.0 & 10.0 & 2308 \\
Gemini-2.0-flash-thinking & \textbf{100.0} & 63.8 & 28.0 & 66.0 & 3310 & {\ul 98.0} & 44.0 & 2.0 & 34.0 & 4073 & \textbf{100.0} & 37.8 & 0.0 & 30.0 & 4311 & {\ul 98.0} & 38.1 & 2.0 & 20.0 & 4458 \\
GPT-4o & \textbf{100.0} & 55.0 & 14.0 & 62.0 & 871 & \textbf{100.0} & 41.9 & 0.0 & 30.0 & 907 & \textbf{100.0} & 29.9 & 0.0 & 14.0 & 974 & \textbf{100.0} & 30.3 & 0.0 & 12.0 & 1061 \\
o1-mini & {\ul 98.0} & {\ul 79.5} & {\ul 66.0} & {\ul 82.0} & 5572 & \textbf{100.0} & {\ul 60.5} & {\ul 16.0} & {\ul 62.0} & 9522 & {\ul 98.0} & {\ul 42.9} & {\ul 4.0} & {\ul 36.0} & 12121 & \textbf{100.0} & {\ul 45.9} & {\ul 8.0} & {\ul 34.0} & 13753 \\
o1-preview & \textbf{100.0} & \textbf{96.7} & \textbf{92.0} & \textbf{98.0} & 6199 & {\ul 98.0} & \textbf{74.5} & \textbf{48.0} & \textbf{74.0} & 9129 & \textbf{100.0} & \textbf{53.8} & \textbf{14.0} & \textbf{56.0} & 11140 & {\ul 98.0} & \textbf{50.1} & \textbf{10.0} & \textbf{46.0} & 11330 \\
        \bottomrule
    \end{tabular}}    
    \caption{Performance (\%) of LLMs on Logic Puzzle across all difficulty levels. The best and second-best results are highlighted in \textbf{bold} and \underline{underlined}, respectively. "Tokens" denotes the average number of generated tokens.}
    \label{level_logic}
\end{table*}
\begin{table*}[t]
    \centering
    \resizebox{1.0\textwidth}{!}{
    \setlength{\tabcolsep}{3pt}
    \begin{tabular}{l|ccccc|ccccc|ccccc|ccccc}
        \toprule
        \multirow{2}{*}{\textbf{Model}} &  \multicolumn{5}{c|}{\textbf{Sudoku - $4\times4$, Easy}} & \multicolumn{5}{c|}{\textbf{Sudoku - $4\times4$, Hard}}  & \multicolumn{5}{c}{\textbf{Sudoku - $9\times9$, Easy}} & \multicolumn{5}{c}{\textbf{Sudoku - $9\times9$, Hard}} \\
        \cmidrule(l){2-21}
        & CR & S-Acc & EM & PM-0.5 & Tokens & CR & S-Acc & EM & PM-0.5 & Tokens & CR & S-Acc & EM & PM-0.5 & Tokens & CR & S-Acc & EM & PM-0.5 & Tokens \\
        \midrule
        \rowcolor{gray!15} \multicolumn{21}{l}{\textit{\textbf{Open-source LLMs}}}\\
        \midrule 
Llama-3.1-8B-Instruct & 4.0 & 0.8 & 0.0 & 0.0 & 931 & 6.0 & 1.3 & 0.0 & 0.0 & 1468 & 12.0 & 1.6 & 0.0 & 0.0 & 5051 & 8.0 & 1.2 & 0.0 & 0.0 & 3678 \\
Llama-3.1-70B-Instruct & 68.0 & 32.4 & 0.0 & 38.0 & 1142 & 60.0 & 31.5 & 4.0 & 32.0 & 1343 & 74.0 & 16.4 & 0.0 & 0.0 & 2514 & 76.0 & 16.7 & 0.0 & 0.0 & 2761 \\
Llama-3.3-70B-Instruct & 84.0 & 44.2 & 10.0 & 48.0 & 1253 & 90.0 & 48.9 & 18.0 & 42.0 & 1200 & \textbf{100.0} & 24.7 & 0.0 & 0.0 & 917 & \textbf{100.0} & 21.5 & 0.0 & 0.0 & 877 \\
Mistral-7B-Instruct-v0.3 & 96.0 & 19.6 & 0.0 & 4.0 & 1333 & {\ul 92.0} & 17.5 & 0.0 & 2.0 & 1413 & 78.0 & 6.0 & 0.0 & 0.0 & 4725 & 70.0 & 4.4 & 0.0 & 0.0 & 4962 \\
Mistral-Small-Instruct-2409 & \textbf{100.0} & 33.2 & 0.0 & 18.0 & 1351 & \textbf{100.0} & 30.2 & 2.0 & 12.0 & 1272 & 80.0 & 9.5 & 0.0 & 0.0 & 2400 & 76.0 & 9.1 & 0.0 & 0.0 & 2850 \\
Mistral-Large-Instruct-2411 & 94.0 & {\ul 68.2} & 26.0 & {\ul 78.0} & 1204 & {\ul 92.0} & 54.2 & 14.0 & 54.0 & 1235 & 86.0 & 21.0 & 0.0 & 2.0 & 2910 & 70.0 & 14.5 & 0.0 & 0.0 & 2472 \\
Qwen2.5-7B-Instruct & {\ul 98.0} & 42.0 & 2.0 & 42.0 & 937 & \textbf{100.0} & 36.2 & 4.0 & 16.0 & 951 & {\ul 96.0} & 25.5 & 0.0 & 2.0 & 2022 & 84.0 & 17.1 & 0.0 & 0.0 & 2034 \\
Qwen2.5-32B-Instruct & \textbf{100.0} & 53.8 & 12.0 & 60.0 & 1095 & \textbf{100.0} & 50.2 & 2.0 & 46.0 & 1158 & \textbf{100.0} & {\ul 34.9} & 0.0 & {\ul 8.0} & 1435 & \textbf{100.0} & {\ul 32.5} & 0.0 & {\ul 8.0} & 1119 \\
Qwen2.5-72B-Instruct & \textbf{100.0} & 60.8 & 16.0 & 72.0 & 1380 & \textbf{100.0} & 55.6 & 6.0 & 58.0 & 1520 & 94.0 & 28.4 & 0.0 & 2.0 & 2543 & {\ul 96.0} & 27.3 & 0.0 & 4.0 & 2610 \\
QwQ-32B-Preview & 66.0 & 63.8 & {\ul 60.0} & 64.0 & 3217 & 76.0 & {\ul 72.0} & {\ul 66.0} & {\ul 72.0} & 5119 & 44.0 & 15.3 & 0.0 & 6.0 & 12575 & 32.0 & 9.5 & 0.0 & 0.0 & 12613 \\
DeepSeek-R1 & \textbf{100.0} & \textbf{100.0} & \textbf{100.0} & \textbf{100.0} & 2878 & \textbf{100.0} & \textbf{100.0} & \textbf{100.0} & \textbf{100.0} & 3620 & \textbf{100.0} & \textbf{46.6} & 0.0 & \textbf{42.0} & 14491 & \textbf{100.0} & \textbf{34.5} & 0.0 & \textbf{14.0} & 12117 \\
        \midrule
        \rowcolor{blue!15} \multicolumn{21}{l}{\textit{\textbf{Closed-source LLMs}}}\\
        \midrule
Gemini-2.0-flash & {\ul 94.0} & 57.8 & 24.0 & 66.0 & 1437 & 92.0 & 58.9 & 26.0 & 60.0 & 1454 & {\ul 94.0} & 33.8 & 0.0 & 10.0 & 4194 & {\ul 92.0} & 30.6 & 0.0 & {\ul 14.0} & 4287 \\
Gemini-2.0-flash-thinking & 70.0 & 62.8 & 44.0 & 68.0 & 2009 & 66.0 & 50.0 & 22.0 & 54.0 & 2350 & 92.0 & \textbf{41.7} & 0.0 & \textbf{36.0} & 5511 & 90.0 & {\ul 31.3} & 0.0 & 6.0 & 5541 \\
GPT-4o & \textbf{100.0} & 68.0 & 26.0 & 84.0 & 1015 & \textbf{100.0} & 63.3 & 32.0 & 64.0 & 1039 & \textbf{100.0} & {\ul 39.7} & 0.0 & {\ul 26.0} & 1234 & \textbf{100.0} & \textbf{37.9} & 0.0 & \textbf{18.0} & 1127 \\
o1-mini & \textbf{100.0} & {\ul 85.4} & {\ul 66.0} & {\ul 88.0} & 3059 & {\ul 96.0} & {\ul 68.6} & {\ul 42.0} & {\ul 72.0} & 3403 & \textbf{100.0} & 32.9 & 0.0 & 10.0 & 4428 & \textbf{100.0} & 26.7 & 0.0 & 2.0 & 4953 \\
o1-preview & \textbf{100.0} & \textbf{100.0} & \textbf{100.0} & \textbf{100.0} & 5616 & \textbf{100.0} & \textbf{100.0} & \textbf{100.0} & \textbf{100.0} & 6191 & 86.0 & 34.5 & 0.0 & 16.0 & 10328 & 80.0 & 25.7 & 0.0 & 6.0 & 10111 \\
        \bottomrule
    \end{tabular}}    
    \caption{Performance (\%) of LLMs on Sudoku across all difficulty levels. The best and second-best results are highlighted in \textbf{bold} and \underline{underlined}, respectively. "Tokens" denotes the average number of generated tokens.}
    \label{level_sudoku}
\end{table*}
\begin{table*}[t]
    \centering
    \resizebox{0.7\textwidth}{!}{
    \setlength{\tabcolsep}{3pt}
    \begin{tabular}{l|ccccc|ccccc}
        \toprule
        \multirow{2}{*}{\textbf{Model}} &  \multicolumn{5}{c|}{\textbf{Drop Quote - Easy}} & \multicolumn{5}{c|}{\textbf{Drop Quote - Hard}} \\
        \cmidrule(l){2-11}
        & CR & S-Acc & EM & PM-0.5 & Tokens & CR & S-Acc & EM & PM-0.5 & Tokens \\
        \midrule
        \rowcolor{gray!15} \multicolumn{11}{l}{\textit{\textbf{Open-source LLMs}}}\\
        \midrule 
Llama-3.1-8B-Instruct & 38.0 & 9.6 & 0.0 & 2.0 & 1692 & 50.0 & 12.8 & 0.0 & 0.0 & 2553 \\
Llama-3.1-70B-Instruct & 86.0 & 32.7 & 0.0 & 22.0 & 1298 & 78.0 & 22.7 & 0.0 & {\ul 2.0} & 1698 \\
Llama-3.3-70B-Instruct & {\ul 98.0} & 33.1 & 0.0 & 24.0 & 900 & \textbf{100.0} & 24.9 & 0.0 & {\ul 2.0} & 936 \\
Mistral-7B-Instruct-v0.3 & 72.0 & 8.9 & 0.0 & 2.0 & 2193 & 60.0 & 4.3 & 0.0 & 0.0 & 2480 \\
Mistral-Small-Instruct-2409 & {\ul 98.0} & 30.5 & 0.0 & 12.0 & 1357 & 96.0 & 23.3 & 0.0 & 0.0 & 1873 \\
Mistral-Large-Instruct-2411 & {\ul 98.0} & 29.4 & 0.0 & 18.0 & 1429 & {\ul 98.0} & 20.0 & 0.0 & 0.0 & 1702 \\
Qwen2.5-7B-Instruct & \textbf{100.0} & 24.9 & 0.0 & 8.0 & 1500 & 96.0 & 18.9 & 0.0 & 0.0 & 2204 \\
Qwen2.5-32B-Instruct & {\ul 98.0} & 33.5 & 0.0 & {\ul 26.0} & 1084 & 92.0 & 23.3 & 0.0 & {\ul 2.0} & 1310 \\
Qwen2.5-72B-Instruct & 96.0 & \textbf{35.5} & 0.0 & 24.0 & 1505 & 92.0 & {\ul 26.2} & 0.0 & {\ul 2.0} & 2009 \\
QwQ-32B-Preview & 32.0 & 9.6 & 0.0 & 14.0 & 5987 & 34.0 & 5.4 & 0.0 & {\ul 2.0} & 6169 \\
DeepSeek-R1 & \textbf{100.0} & \textbf{54.6} & \textbf{14.0} & \textbf{58.0} & 11202 & \textbf{100.0} & \textbf{40.1} & 0.0 & \textbf{26.0} & 11643 \\
        \midrule
        \rowcolor{blue!15} \multicolumn{11}{l}{\textit{\textbf{Closed-source LLMs}}}\\
        \midrule
Gemini-2.0-flash & 92.0 & 37.2 & 0.0 & 28.0 & 2149 & 92.0 & \textbf{31.3} & 0.0 & 6.0 & 3286 \\
Gemini-2.0-flash-thinking & {\ul 96.0} & 38.8 & 0.0 & {\ul 38.0} & 3621 & {\ul 96.0} & {\ul 30.0} & 0.0 & {\ul 8.0} & 3150 \\
GPT-4o & \textbf{98.0} & 34.8 & 0.0 & 24.0 & 1125 & \textbf{100.0} & 27.3 & 0.0 & 4.0 & 1205 \\
o1-mini & \textbf{98.0} & {\ul 41.4} & {\ul 4.0} & 36.0 & 14130 & 94.0 & 27.1 & 0.0 & 6.0 & 12380 \\
o1-preview & \textbf{98.0} & \textbf{49.6} & \textbf{18.0} & \textbf{56.0} & 13064 & {\ul 96.0} & 27.9 & \textbf{8.0} & \textbf{20.0} & 14126 \\
        \bottomrule
    \end{tabular}}    
    \caption{Performance (\%) of LLMs on Drop Quote across all difficulty levels. The best and second-best results are highlighted in \textbf{bold} and \underline{underlined}, respectively. "Tokens" denotes the average number of generated tokens.}
    \label{level_drop}
\end{table*}


\end{document}
