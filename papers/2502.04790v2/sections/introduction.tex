\section{Introduction}
\label{sec:introduction}
Large language models (LLMs) have shown exceptional capabilities across a variety of natural language processing (NLP) tasks \citep{openai2024gpt4,brown2020language,bubeck2023sparks,radford2018improving,radford2019language,touvron2023llama,touvron2023llama2,anil2023palm,chowdhery2023palm}. 
%
However, even the most advanced LLMs exhibit limitations in complex mathematical reasoning and logical inference scenarios \citep{liu2023evaluating}. 
%

 \begin{figure*}[t]
    \centering
    \includegraphics[width=1\linewidth]{./figs/2.pdf}
     \caption{\textbf{Redundant Viewpoints Exchange between Agents.} The perspectives of Agent 1 and Agent 3 demonstrate a notable similarity. Throughout the debate, these viewpoints are exchanged with Agent 2, who receives these akin and repetitive viewpoints.
 }
     \label{fig:examples}
     \vspace{-1.0 em}
 \end{figure*}
 
To address these challenges, researchers have introduced techniques such as Chain-of-Thought (CoT) reasoning \citep{wei2023chainofthought} which decomposes complex problems into sequential steps, and self-consistency (SC) mechanisms \citep{wang2023selfconsistency}, along with self-correction strategies \citep{welleck2022generating,madaan2024self,shinn2024reflexion}. 

%
Despite these innovations, studies have shown that LLMs still struggle to improve through self-correction alone \citep{huang2023large, valmeekam2023can, stechly2023gpt}.
%
An emerging alternative is the Multi-agent Debate (MAD) framework, in which multiple independent agents propose and critique their own answers through rounds of debate, ultimately converging on a more robust consensus \citep{sun2023corex}. 
%
MAD has demonstrated promise in addressing the limitations of LLM self-correction by leveraging diverse agent perspectives to refine answers over iterative discussions \citep{chan2023chateval,du2023improving,liang2023encouraging}. 
%
However, as the number of agents and debate rounds increase, the token cost escalates significantly, limiting the scalability of MAD, especially in resource-constrained environments \citep{li2024improving,liu2024GroupDebate}.
%
To alleviate the token cost problem in multi-agent debates, researchers have proposed several strategies. 
%
For instance, \citep{du2023improving} summarizes agents' outputs at the end of each round, while \citep{sun2023corex} introduces a "forgetting" mechanism, where only the previous round’s outputs is retained for future rounds. 
Another approach, Sparse-MAD (S-MAD) \citep{li2024improving}, reduces communication overhead by limiting information exchange to neighboring agents. 
GroupDebate (GD) \citep{liu2024GroupDebate} further reduces token cost by clustering agents into smaller debate groups that exchange intermediate results between groups.

Although the reduction in token cost have achieved by the aforementioned approaches, our experiment reveals a substantial presence of redundancy and duplicate information in the inter-agent information exchange. As depicted in Figure \ref{fig:examples}, Agent 1 and Agent 3 exhibit repetitive viewpoints, leading to exacerbate the issue of token cost due to redundant duplication during the inter-agent information exchange. The issue of redundancy and duplication primarily stems from two potential factors: the limited solution space inherent in complex reasoning tasks, and the tendency of large language models to generate repetitive responses when faced with similar inputs \citep{holtzman2019curious,xu2022learning,yan2023understanding}.

To address these limitations, we propose a novel approach \textbf{S}elective \textbf{S}parse \textbf{M}ulti-\textbf{A}gent \textbf{D}ebate (\({\text{S}^2\text{-MAD}}\)), as shown in Figure \ref{fig:framework}. This approach utilizes a Decision-Making Mechanism to determine whether to participate in the debate, thereby further reducing token cost within multi-agent debates. Specifically, based on a grouping strategy, \({\text{S}^2\text{-MAD}}\) first generates initial viewpoints for the agents. In each round of debate, the Decision-Making Mechanism enables agents to selectively incorporate non-redundant responses that differ from their current viewpoints for answer checking and updating. The agents have the option to selectively engage in both intra-group and inter-group discussions, enabling them to actively participate in debates. The process concludes either when consensus is reached among the agents or when a final answer is obtained through majority voting.

To validate the effectiveness of \({\text{S}^2\text{-MAD}}\), we conduct a theoretical analysis of total token cost and perform extensive experiments across five tasks using different models. These experiments compare \({\text{S}^2\text{-MAD}}\) with existing multi-agent debate strategy as well as single-agent reasoning approaches, demonstrating its capability to significantly reduce token counts while maintaining comparable accuracy. Specifically, \({\text{S}^2\text{-MAD}}\) reduces token cost by up to 94.5\% compared to MAD, 90.2\% compared to MAD-Sparse, and 87.0\% compared to GD, while also significantly reducing token cost by up to 81.7\% compared to CoT-SC.  Importantly, these reductions come with a performance degradation of less than 2.0\%, demonstrating that \({\text{S}^2\text{-MAD}}\) maintains high accuracy while minimizing communication overhead. 

The main contributions of this paper are as follows:
\begin{enumerate}
    \item We propose \({\text{S}^2\text{-MAD}}\), an innovative sparse multi-agent debate strategy with Decision-Making Mechanism that reduces redundant information and inefficient debate. 
    \item We theoretically demonstrate the token cost advantages of \({\text{S}^2\text{-MAD}}\) over MAD, S-MAD, and GD.
    \item We validate the effectiveness of \({\text{S}^2\text{-MAD}}\) across five datasets using commercial and open-source models, demonstrating a significant reduction in token cost with minimal performance loss.

    %We validate the effectiveness of \( \mathbf{S^2\text{-MAD}} \) across five datasets using commercial or open-source models, demonstrating a significant reduction in token costs with minimal performance loss.

\end{enumerate}

\begin{figure*}[t]
    \centering
    \includegraphics[width=1\linewidth]{./figs/1.pdf}
     \caption{\textbf{Process of \({\text{S}^2\text{-MAD}}\).} The \({\text{S}^2\text{-MAD}}\) includes three stages: all agents generate initial responses independently at the first round and participate in group discussions to reach consensus under a Decision-Making Mechanism, which comprises: (1) Similarity calculation module accesses the similarity of responses either between or within groups. (2) Redundancy filter module filters redundant information, retaining only unique information that differs from the agent's own perspective. (3) Conditional participation module decide to participate in debate or not.
}
    \label{fig:framework}
    \vspace{-1.0 em}
\end{figure*}