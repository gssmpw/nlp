%Language Models are trained on the next-token prediction task, but can they be used to infer properties about the entire sequence of tokens they will eventually generate?
Autoregressive Language Models output text by sequentially predicting the next token to generate, with modern methods like Chain-of-Thought (CoT) prompting achieving state-of-the-art reasoning capabilities by scaling the number of generated tokens. However, are there times when we can infer how the model will behave (e.g. abstain from answering a question) early in the computation, making generation unnecessary? We show that internal representation of input tokens alone can often precisely predict, not just the next token, but eventual behavior over the entire output sequence. We leverage this capacity and learn probes on internal states to create early warning (and exit) systems. Specifically, if the probes can confidently estimate the way the LM is going to behave, then the system will avoid generating tokens altogether and return the estimated behavior instead. On 27 text classification datasets spanning five different tasks, we apply this method to estimate the eventual answer of an LM under CoT prompting, reducing inference costs by 65\% (average) while suffering an accuracy loss of no more than 1.4\% (worst case). We demonstrate the potential of this method to pre-emptively identify when a model will abstain from answering a question, fail to follow output format specifications, or give a low-confidence response. We explore the limits of this capability, showing that probes generalize to unseen datasets, but perform worse when LM outputs are longer and struggle to predict properties that require access to knowledge that the models themselves lack. Encouragingly, performance scales with model size, suggesting applicability to the largest of models.~\footnote{\url{https://github.com/DhananjayAshok/LMBehaviorEstimation}}