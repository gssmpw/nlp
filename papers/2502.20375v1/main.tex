\documentclass[11pt]{article}
\newcommand{\thought}[1]{{\color[rgb]{0.2,0.39,0.66}(#1)}}
\newcommand{\todo}[1]{{\color[rgb]{1.0,0.0,0.0}(#1)}}
\newcommand{\hsh}[1]{{\color{green!50!black} Henrik: #1}}
\newcommand{\st}[1]{{\color{red!50!black} Sebastian: #1}}

\newcommand{\ulm}[1]{_{\scaleto{\mathrm{#1}}{3pt}}}
\newcommand\at[2]{\left.#1\right|_{#2}}











\newtheorem{assumption}{Assumption}

\DeclareMathOperator*{\argmax}{arg\,max}
\DeclareMathOperator*{\argmin}{arg\,min}

\newcommand{\swname}[1]{\texttt{#1}}
\newcommand{\ie}{i\/.\/e\/.,\/~}
\newcommand{\eg}{e\/.\/g\/.,\/~}
\newcommand{\cf}{cf\/.\/~}

\newcommand{\fig}{Fig\/.\/~}
\newcommand{\defn}{Def\/.\/~}
\newcommand{\sect}{Sec\/.\/~}
\newcommand{\tabl}{Tab\/.\/~}
\newcommand{\algo}{Algorithm~}
\newcommand{\theo}{Theorem~}

\newcommand{\bnnl}{3 hidden layers}
\newcommand{\bnnn}{50 neurons}
\newcommand{\bnna}{tanh activations}

\newcommand{\capt}[1]{\mdseries{\emph{#1}}}

\newcommand{\videolink}{at \url{https://youtu.be/_d7AqTRjz6g}}
\newcommand{\codelink}{\url{https://github.com/wheelbot/mini-wheelbot}}

\newcommand{\fakepar}[1]{\vspace{0mm}\noindent\textbf{#1.}}

\newcommand{\needref}{\textcolor{red}{[REF]}}

\newcommand{\plotfontsize}{9pt}


\title{When does a predictor know its own loss?}
\author{Aravind Gollakota\\
Apple\\
\and
Parikshit Gopalan\\
Apple\\
\and
Aayush Karan\\
Harvard University\thanks{Work done  while interning at Apple. Authors in alphabetical order.}\\
\and 
Charlotte Peale\\
Stanford University\footnotemark[1]\\
\and
Udi Wieder\\
Apple
}


\begin{document}
\date{February 18, 2025}

\maketitle
\begin{abstract}

    Given a predictor and a loss function, how well can we predict the loss that the predictor will incur on an input? This is the problem of loss prediction, a key computational task associated with uncertainty estimation for a predictor. In a classification setting, a predictor will typically predict a distribution over labels and hence have its own estimate of the loss that it will incur, given by the entropy of the predicted distribution. Should we trust this estimate? In other words, when does the predictor know what it knows and what it does not know?

    In this work we study the theoretical foundations of loss prediction.
    Our main contribution is to establish tight connections between nontrivial loss prediction and certain forms of multicalibration \cite{hebert2018multicalibration}, a multigroup fairness notion that asks for calibrated predictions across computationally identifiable subgroups.
    Formally, we show that a loss predictor that is able to improve on the self-estimate of a predictor yields a witness to a failure of multicalibration, and vice versa. This has the implication that nontrivial loss prediction is in effect no easier or harder than auditing for multicalibration. We support our theoretical results with experiments that show a robust positive correlation between the multicalibration error of a predictor and the efficacy of training a loss predictor. 
\end{abstract}


\section{Introduction}
\label{sec:intro}

\begin{figure*}[tb]
    \centering
    \includegraphics[width=0.848\linewidth]{figs/circuitnn.pdf} 
    \caption{Illustration of differentiable CircuitNN. CircuitNN is designed based on differentiable NAND gates. After DAS is guided by PI and PO pairs of the truth table, CircuitNN can get the precise circuit architecture logic equivalent to the truth table.}
    \label{fig:circuitnn}
\end{figure*}

% 1. Describe the importance of logic synthesis
% 2. Existing Problems
% (a) Neural Architecture Search: Unstable, Predefined Setting, etc.
% (b) Circuit Generation: Probabilistic Model, Logic Equivalence

With the rapid advancement of technology, the scale of integrated circuits (ICs) has expanded exponentially. 
This expansion has introduced significant challenges in chip manufacturing, particularly concerning power and area metrics.
A primary objective in IC design is achieving the same circuit function with fewer transistors, thereby reducing power usage and area occupancy.

Logic synthesis~\cite{hachtel2005logicsynth}, a critical step in electronic design automation (EDA), transforms behavioral-level circuit designs into optimized gate-level circuits, ultimately yielding the final IC layout. 
The primary goal of logic synthesis is to identify the physical implementation with the fewest gates for a given circuit function. 
This task constitutes a challenging NP-hard combinatorial optimization problem. 
Current logic synthesis tools~\cite{brayton2010abc, wolf2013yosys} rely on human-designed heuristics, often leading to sub-optimal outcomes.

Differentiable architecture search (DAS) techniques~\cite{liu2018darts, chu2020darts} offer novel perspectives on addressing challenges in this problem.
Circuit functions can be represented through truth tables, which map binary inputs to their corresponding outputs. 
Truth tables provide a precise representation of input-output relationships, ensuring the design of functionally equivalent circuits.
Inspired by this, researchers~\cite{deepmind2024ai4sys, wang2024tnet} have begun exploring the application of DAS to synthesize circuits directly from truth tables.
Specifically, \citet{deepmind2024ai4sys} proposed CircuitNN, a framework that learns differentiable connection structures with logic gates, enabling the automatic generation of logic circuits from truth tables.
This approach significantly reduces the complexity of traditional circuit generation. 
Building on this, \citet{wang2024tnet} introduced T-Net, a triangle-shaped variant of CircuitNN, incorporating regularization techniques to enhance the efficiency of DAS.

Despite these advancements, several challenges remain. 
The computational complexity of DAS grows quadratically with the number of gates, posing scalability issues.
Although triangle-shaped architecture~\cite{wang2024tnet} partially mitigates this problem, redundancy persists. 
%Additionally, DAS is susceptible to converging to local optima, limiting the ability to search architectures that satisfy the given truth tables~\cite{liu2018darts}. 
%Furthermore, hyperparameters (network depth and layer width) require extensive searches, introducing complexity and prolonging the synthesis process. 
Additionally, DAS is susceptible to converging to local optima~\cite{liu2018darts} and hyperparameters (network depth and layer width) require extensive searches. 
The challenges arise from the vast search space in DAS. 
% Even with predefined settings for CircuitNN, finding a configuration that meets the truth table requires extensive trial and error during the DAS process. 
Intuitively, limiting the search space through predefined parameters (network depth, gates per layer, and connection probabilities) can significantly reduce the complexity.

Recent advances~\cite{openai2023gpt4, abramson2024alphafold3, esser2024sd3, li2024mar} in conditional generative models have demonstrated remarkable performance across language, vision, and graph generation tasks. 
Motivated by these developments, we propose a novel approach to circuit generation that generates preliminary circuit structures to guide DAS in generating refined circuits matching specified truth tables. 
Firstly, we introduce CircuitVQ, a tokenizer with a discrete codebook for circuit tokenization. 
Built upon our Circuit AutoEncoder framework~\cite{hou2022graphmae,li2023maskgae,wu2025mgvga}, CircuitVQ is trained through a circuit reconstruction task. 
Specifically, the CircuitVQ encoder encodes input circuits into discrete tokens using a learnable codebook, while the decoder reconstructs the circuit adjacency matrix based on these tokens.
Subsequently, the CircuitVQ encoder serves as a circuit tokenizer for CircuitAR pretraining, which employs a masked autoregressive modeling paradigm~\cite{chang2022maskgit, li2023mage}. 
In this process, the discrete codes function as supervision signals. 
After training, CircuitAR can generate discrete tokens progressively, which can be decoded into initial circuit structures by the decoder of the CircuitVQ. 
These prior insights can guide DAS in producing refined circuits that match the target truth tables precisely.

Our key contributions can be summarized as follows:
\begin{itemize}
\item We introduce CircuitVQ, a circuit tokenizer that facilitates graph autoregressive modeling for circuit generation, based on our Circuit AutoEncoder framework;
\item Develop CircuitAR, a model trained using masked autoregressive modeling, which generates initial circuit structures conditioned on given truth tables;
\item Propose a refinement framework that integrates differentiable architecture search to produce functionally equivalent circuits guided by target truth tables;
\item Comprehensive experiments demonstrating the scalability and capability emergence of our CircuitAR and the superior performance of the proposed circuit generation approach.
\end{itemize}

% Motivation
% (a) Diffusion (Vision, Graph), Autoregressive (Language, Vision)
% (b) Circuit Generation for Predefined Setting
% (c) Neural Architecture Search for Strict Logic Equivalence

% Contribution
% (a) Circuit Tokenizer (new transformer arch, training strategy)
% (b) CircuitAR (train and gen strategies, post-ar strategy)
% (c) Extensive Evaluation including BitD (Bit Distance) for Scalability

\section{Loss prediction}
\label{sec:lp}

We consider binary classification, with a distribution $\mD$ on  $\X \times \zo$. 
A predictor is a function $p:\X \to [0,1]$. The Bayes optimal predictor is defined as $p^*(x) = \ex[y|x]$. Given $p$, we define the simulated distribution $\mD(p)$ on $\X \times \zo$ where $x$ is drawn as in $\mD$, and $y|x \sim \Ber(p(x))$. Let $\ell: \zo \times [0,1] \to [0,1]$ be a proper loss function.\footnote{ The case of general losses reduces to the proper loss case; please see \cref{sec:non-proper} for details. We also assume for technical convenience that the loss is bounded. Losses that are not strictly bounded, such as cross entropy, can be handled with some further care and constraints on predicted probabilities.} We will use the following characterization of proper losses.

\begin{lemma}[\cite{gneiting2007strictly}] For every proper loss $\ell$,  there exists a concave function $H_{\ell}: [0, 1] \rightarrow \R$ so that
\[ \ell(y, v) = H_{\ell}(v) + (y - v)H_{\ell}'(v).\]
where $H_{\ell}'(v)$ is a ``superderivative'' of $H_{\ell}$, i.e. for any $v, w \in [0, 1]$, $H_{\ell}(v) \leq H_{\ell}(w) + (v - w)H_{\ell}'(w)$. 
\end{lemma}
When $H_{\ell}(v)$ is differentiable at all $v \in [0, 1]$, the superderivative is unique, and is just the derivative.  From the definition it follows that 
\begin{align*} 
H_{\ell}(v) &= \ex_{y \sim \Ber(v)}[\ell(y,v)] \in [0,1]\\
H_{\ell}'(v) &= \ell(1, v) - \ell(0,v) \in [-1,1]
\end{align*}
Let $L(p^*; p) = \ex_{y \sim \Ber(p^*)}[\ell(y, p)]$ denote the expected loss when $y \sim \Ber(p^*)$ but we predict $p$. Then
\begin{align}
\label{eq:exp-loss}
 L(p^*;p) = H_{\ell}(p) + (p^* - p)H_{\ell}'(p) \geq H_{\ell}(p^*) = L(p^*; p^*) 
\end{align}
where the inequality follows from the concavity of $H_{\ell}$, and is equivalent to the loss $\ell$ being proper.

We now define the notion of a loss predictor. 

\begin{definition}[Loss predictor]
\label{def:lp}
    Let $p$ be a predictor and $\ell$ be a proper loss. Let $\Phi$ be an abstract feature space, which we will make concrete shortly. 
    A \emph{loss predictor} is a function $\lossPred: \Phi \to \R$, which takes as input some features $\phi(p,x) \in \Phi$ related to a point $x$ and its prediction using $p$, and returns an estimate $\lossPred(\phi(p,x))$ of the expected loss $\ex[\ell(y, p(x))|x]$ suffered by $p$ at the point $x$. We define a hierarchy of loss predictors of increasing strength, depending on the information contained in $\phi(p,x)$:
    \begin{enumerate}
        \item \emph{Prediction-only loss predictors} only have access to $p$'s prediction at a point $x$, i.e. $\phi(p, x) = p(x)$. The most natural choice for a prediction-only loss predictor is given by the self-entropy predictor, which we will define in Definition~\ref{def:can-loss-pred}.
        \item \emph{Input-aware loss predictors} have access to the input features $\inp(x)$ used to train the model $p$, as well as its prediction, i.e. $\phi(p, x) = (\inp(x), p(x))$. 
        \item \emph{Representation-aware loss predictors} have access to $\phi(p, x) = ( p(x), \inp(x), r(x))$, where $r(x)$ is some representation of $x$. We distinguish between two kinds of representations:
        \begin{itemize}
            \item Internal representations: The representation $r(x)= r_p(x)$ consists of features that are explicitly computed by the predictor $p$ in the course of computing $p(x)$. For instance, they could consist of the embedding of $x$ produced by the last few layers of a DNN.  
            \item External representations: The representation $r(x) = r_e(x)$ consists of features that are not explicitly computed by the  predictor $p$. For instance, they could be the representation of $x$ obtained from a different model, or by consulting human experts.         
        \end{itemize}
     \end{enumerate}
\end{definition}

A few comments on the definition: 

\begin{itemize}
\item Two-headed architectures that simultaneously train both the predictor and the loss-predictor (such as \cite{yoo2019learning,kirillov2023segment}) are a class of internal representation-aware predictors. In contrast, loss-predictors that use an embedding produced by a foundation model (such as \cite{jain2022distilling}, which audits the errors of the predictor) are external representation-aware.  

\item   If we allow the loss predictor to be significantly more complex than the predictor $p$, then it could compute $r_p(x)$ from $\inp(x)$ using the model $p$. So the gap between input-aware and representation-aware loss predictors diminishes as the loss-predictor becomes more computationally powerful. But in the (important) setting where the loss predictor is less computationally powerful than the predictor, there could be a gap.

\item In contrast, external representations might contain auxiliary information  that cannot be computed using $\inp(x)$, regardless of the computational power of the loss predictor. 
\end{itemize}

The loss predictor can be trained using standard regression, given access to a training set of points $(\phi(p, x), y)$ where $(x, y)$ are drawn from the distribution $\mD$.  One can measure the performance of our loss predictor as we would with any regression problem. We formulate it using the squared loss, as $\ex[(\ell(y, p(x)) - \lossPred(\phi(p, x))^2]$. It follows from Equation \eqref{eq:exp-loss} that the Bayes optimal loss predictor is given by
$\lossPred^*(x) = L(p^*(x); p(x))$. But computing this requires knowing the Bayes optimal predictor $p^*$, and is likely to be infeasible in most settings. Rather, we will compare our loss predictor to a canonical baseline which we describe next.

\paragraph{The self-entropy predictor.}

Following \cite{OI}, given a predictor $p$, we define the simulated distribution $\mD(p)$  on pairs $(x, \ty) \in \X \times \zo$,  where $x \sim \mD$ and $\ex[\ty|x] = p(x)$. The predictor hypothesizes that this how labels are being generated. Hence for each $x \in \X$, the self-entropy predictor predicts the expected loss according to this distribution. 

\begin{definition}[Self-entropy predictor]\label{def:can-loss-pred}
    Given a proper loss $\ell$ and predictor $\pred$,  the \emph{self-entropy predictor} is the prediction-only loss predictor $\clp: [0, 1] \to \R$  that predicts the expected loss when $\ty \sim \Ber(\pred(x))$ at each $x$; that is
    \[\clp(\pred(x)) = \ex_{\ty \sim \Ber(p(x))}[\ell(\ty, p(x))] = H_{\ell}(\pred(x)).\]
\end{definition}


We use the self-entropy predictor as our baseline. Hence the question is when can we learn a loss predictor with significantly lower squared loss than the self-entropy predictor. We formalize this using the notion of advantage of a loss predictor over the self-entropy predictor.

\begin{definition}[Advantage of a loss predictor]\label{def:lp-advantage}
    Define the advantage of a loss predictor $\lossPred$ over the self-entropy predictor to be the difference in the squared error
    \[ \adv(\lossPred) = \ex[(\ell(y, p(x)) - \clp(p(x)))^2] - \ex[(\ell(y, p(x)) - \lossPred(\phi(p, x))^2]. \]
\end{definition}%

We want loss predictors whose advantage is positive and as large as possible.
Our goal is understand under what conditions we can hope to learn such a predictor. 

\paragraph{On non-proper losses.} So far we have assumed that we a trying to predict the proper loss incurred by a predictor. We can generalize this to a setting where we have a hypothesis $h:\X \to \mA$ (for instance $h$ might be a binary classifier), and a loss function $\ell:\zo \times \mA \to \R$. It turns out that our theory extends seamlessly to the non-proper setting, under rather mild assumptions on the hypothesis $h$. We present this extension in Appendix \ref{sec:non-proper}.





\section{Multicalibration}
\label{sec:mc}

Having defined our notion of a loss predictor, we next introduce the framework of multicalibration proposed by~\cite{hebert2018multicalibration}. Our definition is most similar to the presentation used in~\cite{kim2022universal}.

\begin{definition}[Multicalibration]
\label{def:mc}
    Let $\phi(p, x) \in \Phi$ be some auxiliary set of features related to the computation of $p(x)$, which we define concretely below. Let $\calC$ be a class of weight functions $c: \Phi \rightarrow [-1, 1]$, and $p: \calX \rightarrow [0, 1]$ a binary predictor for a target distribution $\calD$ over $\calX \times \{0, 1\}$. Then, the multicalibration error of $p$ with respect to $\calC$ is defined as 
    \[\MCE(\calC, p) := \max_{c \in \calC} \left|\ex_{x, y \sim \calD}[(y - p(x))c(\phi(p, x))]\right|.\]
    The information contained in $\phi(p, x)$ gives rise to a hierarchy of multicalibration notions of increasing strength:
    \begin{enumerate}
        \item \emph{Calibration} corresponds to the setting where  $\phi(p, x)= p(x)$, and test functions can only depend on $p$'s prediction. 
        \item \emph{Multicalibration} corresponds to the case where test functions can additionally depend on the input features, i.e. $\phi(p, x) = (p(x), \inp(x))$.
        \item \emph{Representation-aware multicalibration} is a strengthening of multicalibration where test functions can additionally depend on some representation $r(x)$ of $x$ i.e., $\phi(p, x) = (p(x), \inp(x), r(x))$. We distinguish between internal representations $r_p(x)$ and external representations $r_e(x)$ as with loss predictors (Definition \ref{def:lp}).
    \end{enumerate}
\end{definition}

The first two levels in this hierarchy, calibration and multicalibration, have been extensively studied in previous works. 
In standard multicalibration, we require that a predictor $p(x)$ be well-calibrated under a broad class of test functions, $\calC$, that depend only on $\inp(x)$ and $p(x)$. The literature on multicalibration typically identifies $\inp(x)$ with $x$ itself. The last level of the hierarchy, representation-aware multicalibration, is a strengthening of multicalibration that naturally extends the multicalibration framework of ~\cite{hebert2018multicalibration}. As in the case of loss-predictors, the gap between internal representations $r_p(x)$ and $\inp(x)$ is computational; whereas the gap between external representations $r(x)$ and $\inp(x)$ could be information-theoretic. 








\begin{definition}[Multicalibration violation witness]
    We say that a function $c: \Phi \times [0,1] \rightarrow [-1, 1]$ is a witness for a multicalibration violation of magnitude $\alpha$ for a predictor $p$ if 
    \[\left|\ex_{x, y \sim \calD}[(y - p(x))c(\phi(p, x))]\right| > \alpha.\]
\end{definition}

\cite{hebert2018multicalibration} showed that if we find such a witness, we can use it to improve the predictor $p$ in a way that reduces the squared loss. While their argument is stated for the input-aware setting where $\phi(p,x) = (p(x), \inp(x))$, it applies to the representation-aware setting as well. 





    



\section{Loss prediction advantage and multicalibration auditing}
\label{sec:lp-mc}

In this section, we establish the relationship between learning loss predictors with good advantage, and auditing for multicalibration, i.e. finding a $c$ that witnesses a large multicalibration violation. The main result of our section is the following theorem, which establishes the correspondence between various levels of loss predictors and multicalibration requirements, when instantiated with the appropriate values $\phi(p, x)$:

\begin{theorem}\label{thm:mc-conv-vs-loss-pred}
    Let $\calF$ be a class of loss predictors $f: \Phi \rightarrow [0, 1]$. 
    Let $\calF' \supseteq \calF$ be the augmented function class defined as 
    \[\calF' = \{\Pi_{[0,1]}((1 - \beta)H_{\ell}(p(x)) + \beta f(\phi(p,x))) : \beta \in [-1, 1], f \in \calF\}.\]
    Let $\calC$ be a class of weight functions defined as 
    \[\calC = \{ (f(\phi(p, x)) - H_{\ell}(p(x)))H_{\ell}'(p(x)) : f \in \calF\}.\]
    Then, 
    \[\frac{1}{2}\max_{\lossPred \in \calF} \adv(\lossPred) \leq \MCE(\calC, p) \leq \sqrt{\max_{\lossPred \in \calF'} \adv(\lossPred)}.\]
\end{theorem}

The proof of Theorem~\ref{thm:mc-conv-vs-loss-pred} can be found in Appendix~\ref{sec:mc-conv-cs-loss-pred-pf} and follows from two key lemmas. Lemma~\ref{lem:adv-implies-mc-err} establishes the left-hand inequality by showing how a loss predictor with good advantage can be used to construct a witness of large multicalibration error. Conversely, Lemma~\ref{lem:mc-err-implies-adv} establishes the right-hand inequality by showing how to leverage a witness for large multicalibration error to construct a loss predictor with large advantage. 

Before presenting our main lemmas, we introduce two auxiliary claims that are well-known in the literature on boosting and gradient descent. We provide proofs here for completeness and notational consistency.


Let $\mD'$ be a distribution over $(x, z) \in X \times [0,1]$. Let $h_1, h_2: \X \to [0,1]$ be two hypotheses. Under what conditions does $h_2$ improve on $h_1$? The following lemma gives a necessary condition: the update $\delta(x) = h_2(x) - h_1(x)$ must be correlated with the residual errors $z - h_1(x)$ of the hypothesis $h_1$ under the distribution $\mD'$. 

 
\begin{claim}
\label{lem:improve-1}
    For two hypotheses $h_1, h_2$, 
    \[ \E_{\mD'}[(h_1(x) - z)^2] - \E[(h_2(x) - z)^2] \leq 2\E[(h_2(x) - h_1(x))(z - h_1(x))]. \]
\end{claim}
\begin{proof}
    Let us write $\delta(x) = h_2(x) - h_1(x)$. Then we have
    \begin{align*}
        \E[(h_1(x) - z)^2] - \E[(h_2(x) - z)^2] &= \E_{\mD'}[(h_1(x) -z)^2 - (h_1(x)  -z + \delta(x) )^2]\\
        &= -2\E[(h_1(x) - z)\delta(x)] - \E[\delta(x)^2]\\
        & \leq 2\E_{\mD}[(z - h_1(x))\delta(x)].
    \end{align*}
\end{proof}

Conversely, if we find an update $\delta(x)$ which is correlated with the residuals, we can perform a gradient descent update to reduce the squared error. We let $\Pi_{[0,1]}:\R \to [0,1]$ denote the projection operator onto the unit interval.

\begin{claim}
\label{lem:improve-2}    
If there exists $\delta:\X \to [-1,1]$ such that $\E_{\mD'}[\delta(x)(z - h_1(x))] \geq \beta \geq 0$, then setting $h_2(x) = \Pi_{[0,1]}(h_1(x)+ \beta \delta(x))$ gives
    \[ \E_{\mD'}[(h_1(x) - z)^2] - \E[(h_2(x) - z)^2] \geq \beta^2.\]
\end{claim}

\begin{proof}
    Without projection, we can write the gap in squared error as 
    \begin{align*}
        \E_{\mD'}[(h_1(x) - z)^2] - \E[(h_1(x) + \beta\delta(x) - z)^2] &=  2\beta \E_{\mD'}[(z - h_1(x))\delta(x)] - \beta^2\E[\delta(x)^2]\\
        &\geq 2\beta^2 - \beta^2 = \beta^2.
    \end{align*}
    While $h_1(x) + \beta \delta(x)$ may not be bounded in $[0,1]$, projection onto the interval can only further reduce the squared error.
\end{proof}

With these in hand, we show that any loss predictor with a non-trivial advantage points us to a failure of multicalibration. 

\begin{lemma}\label{lem:adv-implies-mc-err}
    Assume that $\lossPred$ achieves advantage $\alpha \geq 0$ over the self-entropy predictor. Then the function $\delta(\phi(p, x)) = \lossPred(\phi(p, x)) - \clp(p(x))$ satisfies
    \[ \E[\delta(\phi(p, x))H_{\ell}'(p(x))(y - p(x))] \geq \alpha/2.\]
    In other words, $\lossPred$ can be used to construct a witness $c(\phi(p, x)) = \delta(\phi(p, x))H_{\ell}'(p(x))$ for a multicalibration violation of magnitude $\alpha/2$. 
\end{lemma}
\begin{proof}
    Consider the loss regression problem, where we draw $(x,y) \in \X \times \zo \sim \mD$ and then return the pair $(x, z = \ell(y, p(x))$. We will use Claim \ref{lem:improve-1}, where we take $h_1 = \clp$ to be the self-entropy predictor and $h_2 = \lossPred$. 
    We can estimate the residual error of the self-entropy predictor as
    \begin{align}
    \label{eq:rewrite}
        \ell(y, p(x)) - \clp(p(x)) &= H_{\ell}(p(x)) + (y - p(x))H_{\ell}'(p(x)) - H_{\ell}(p(x))\notag\\
        &= (y - p(x))H_{\ell}'(p(x)).
    \end{align}
    By Claim \ref{lem:improve-1}, we have
    \begin{align*}
        \alpha &= \ex[(\ell(y, p(x)) - \clp(p(x)))^2] - \ex[(\ell(y, p(x)) - \lossPred(\phi(p, x))^2]\\
        &\leq 2\E[(\lossPred(\phi(p, x)) - \clp(p(x)))(\ell(y, p(x)) - \clp(p(x)))\\ &= 2\E[\delta(\phi(p, x))H_{\ell}'(p(x))(y - p(x))]
    \end{align*}
\end{proof}

Conversely to the result of Lemma~\ref{lem:adv-implies-mc-err}, we show that we can leverage certain types of multicalibration failures to predict loss with an advantage over the self-entropy predictor.

\begin{lemma}\label{lem:mc-err-implies-adv}
    Assume there exists a function $\delta:\Phi \to [-1,1]$ such that
    \[ \E[\delta(\phi(p, x))H_{\ell}'(p(x))(y - p(x))] \geq \beta \geq 0.\]
    i.e., the function $c(\phi(p, x)) = \delta(\phi(p, x))H_{\ell}'(p(x))$ is a witness for a multicalibration violation of magnitude $\beta$. Define the loss predictor
    \[ \lossPred(\phi(p, x)) = \Pi_{[0,1]}(\clp(p(x)) + \beta 
    \delta(\phi(p, x))).\] 
    Then $\adv(\lossPred) \geq \beta^2$. 
\end{lemma}
\begin{proof}
    We again consider the loss regression problem, 
    We now apply Lemma \ref{lem:improve-2} with $z = \ell(y, p(x))$, $h_1 = \clp$. The correlation condition we require is
    \[  \E[\delta(\phi(p, x))(\ell(y, p(x) - \clp(p(x)))] \geq \beta, \]
    By Equation \eqref{eq:rewrite}, we have
    \[ \E[\delta(\phi(p, x))(\ell(y, p(x)) - \clp(p(x)))] = \E[\delta(\phi(p, x))H_{\ell}'(p(x))(y - p(x))] \]
    which is at least $\beta$ by our assumption. Hence Claim \ref{lem:improve-2} implies that $h_2 = \lossPred$ has advantage $\beta^2$ over $\clp$.     
\end{proof}


\section{Loss prediction for multiple losses}
\label{sec:multiple-loss}

Up to this point, our discussion has focused on loss prediction for a single, predetermined loss function. However, in real-world applications, multiple stakeholders may use a predictor, each with unique objectives and priorities that correspond to different loss functions. This scenario would require training separate loss predictors for each user to meet their individual needs.

The self-entropy predictor offers a key advantage: it can be computed for any loss function using only the predictions $p(x)$, eliminating the need for additional training. Moreover, by extending the result of Theorem~\ref{thm:mc-conv-vs-loss-pred}, we can define a class test functions $\calC$ such that when $p$ is multicalibrated with respect to $\calC$, its self-entropy predictions simultaneously compete with the best-in-class loss predictor for each loss in a rich class of losses $\calL$, rather than just a fixed loss. We formalize this in the following lemma, which we prove in Appendix~\ref{sec:many-losses-mc-pf}:

\begin{lemma}\label{lem:many-losses-mc}
    Let $\calF$ be a class of loss predictors $f: \Phi \rightarrow [0, 1]$. Let $\calL$ be a class of bounded proper losses $\ell: \{0, 1\} \times [0, 1] \rightarrow [0, 1]$ with associated concave entropy functions $H_{\ell}: [0, 1] \rightarrow [0,1]$, and let $\calC_{\calL}$ be the class of test functions 
    \[\calC_{\calL} = \{(f(\phi(p, x)) - H_{\ell}(p(x)))H'_{\ell}(p(x)) : f \in \calF, \ell \in \calL\}.\]
    Then,
    \[\max_{\ell \in \calL}\max_{\lossPred \in \calF} \adv(\lossPred) \leq 2\MCE(\calC_{\calL}, p).\]
    I.e., no loss predictor from $\calF$ for any loss $\ell \in \calL$ can obtain better advantage than $2\MCE(\calC_{\calL}, p)$ over the self-entropy predictor. 
\end{lemma}

When $\calL$ is the set of all proper losses, the form of multicalibration imposed by $\calC_{\calL}$ can be thought of as the extension to multicalibration of the notion of \emph{proper calibration}, recently proposed by~\cite{OKK25}. The proper calibration error of a predictor $p$ is defined as 

\[\text{PCE}(p) = \max_{\ell \in \calL_{\text{prop}}}\left|\ex[H'_{\ell}(p(x))(y - p(x))]\right|\]
where $\calL_{\text{prop}}$ denotes the set of proper losses. Our condition can be thought of as ``proper multicalibration'' where each test function consists of $H_{\ell}'(p(x))$ multiplied with an additional test function $\delta(\phi(p, x))$, that may depend on other features in addition to the prediction value. 

\subsection{Achieving efficient multicalibration for many losses}

As the class of losses we consider expands, training an effective loss predictor for each individual loss becomes increasingly challenging. This section demonstrates that in certain scenarios, it is possible to efficiently produce a multicalibrated predictor with respect to the class of tests outlined in Lemma~\ref{lem:many-losses-mc}, even for some infinite classes of losses. This approach allows us to learn a single predictor $p$ whose self-entropy estimates can compete with the best $\lossPred \in \calF$ for every 
$\ell \in \calL$, thus eliminating the need to train separate predictors for each loss.

This result relies on the existence of a ``finite approximate basis'' (Definition~\ref{def:finite-approx-basis}) for the class of functions $\{H_{\ell}'\}_{\ell \in \calL}$, and is inspired by the techniques of~\cite{OKK25}, who use a similar approach to show the efficiency of proper calibration when $\{H_{\ell}'\}_{\ell \in \calL}$ has a finite approximate basis. 

We show a general version of this result in Theorem~\ref{thm:general-approx-basis-mc}, and instantiate it here for the class of 1-Lipschitz proper losses, $\calL_{Lip}$. 

The instantiation relies on a result proved by \cite{OKK25}, who show that $\{H_{\ell}'\}_{\ell \in \calL_{Lip}}$ has such a finite basis. We show efficiency in terms of oracle access to a weak-agnostic-learner for $\calF$, the class of loss predictors. We motivate this assumption by observing that if we care about learning a loss predictor from the class $\calF$, it's reasonable to assume that we have access to a weak agnostic learner for $\calF$. We formally define a weak agnostic learner as follows. 

\begin{definition}[Weak agnostic learner]\label{def:weak-agnostic-learner}
    Let $\alpha \geq 0$, $\delta \geq 0$. An $\alpha$-weak agnostic learner for $\calF \subseteq \{f: \Phi \rightarrow [-1, 1]\}$, closed under negation, with sample complexity $n$ and failure parameter $\delta$ is an algorithm that when given $n$ samples from a distribution $\calU$ over $\Phi \times [-1, 1]$, outputs $f \in \calF \cup \{\bot\}$ such that with probability at least $1 - \delta$ over the samples from $\calU$ and the randomness in the algorithm itself, if $\max_{f \in \calF} \ex_{(\phi, z) \sim \calU}[f(\phi)z] \geq \alpha,$
    the algorithm returns a $f \in \calF$ such that 
    $\ex_{(\phi, z) \sim \calU}[f(\phi)z] \geq \alpha/2.$
    Otherwise, if for all $f \in \calF$, 
    $\ex_{(\phi, z) \sim \calU}[f(\phi)z] \leq \alpha,$ the algorithm either returns $f = \bot$ or $f \in \calF$ such that $\ex_{(\phi, z) \sim \calU}[f(\phi)z] \geq \alpha/2.$
\end{definition}

With this definition in hand, we are ready to present the main theorem of this section. The proof can be found in Appendix~\ref{sec:1-lip-mc-pred-pf}.

\begin{theorem}\label{thm:1-lip-mc-pred}
    Fix $\delta, \epsilon > 0$. Let $\calL_{Lip}$ be the class of proper 1-Lipschitz losses $\ell:\{0, 1\} \times [0, 1] \rightarrow [0, 1]$, and let $\calF$ be a class of loss predictors $\calF: \Phi \rightarrow [-1, 1]$ that is closed under negation and contains the class of self entropy predictors, $\calH_{\calL_{Lip}} = \{H_{\ell}\}_{\ell \in \calL_{Lip}}$. Further assume that we have access to an $\alpha$-weak-agnostic-learner for $\calF$ with sample complexity $n$ and failure parameter $\beta \leq \frac{\alpha^2\delta}{4\lceil 2/\epsilon + 1\rceil}$. 

    Then, there exists an algorithm that, given $m = O(n/\alpha^2)$ samples, with probability at least $1 - \delta$ outputs a predictor $p$ such that 
    \[\max_{\ell \in \calL_{Lip}} \max_{\lossPred \in \calF} \adv(\lossPred) \leq 16\alpha + 4\epsilon.\]
\end{theorem}

In other words, our learned $p$'s self-entropy predictions compete with the best-in-class loss predictor with \emph{every} $\ell \in \calL_{Lip}$, up to an error of $16\alpha + 4\epsilon$. 








\section{Experiments: Planning outperforms Heuristics}
\label{sec:experiment}

We begin our empirical demonstrations by showcasing the effectiveness of our planning framework on both synthetic and real datasets. We focus on the simplest planning algorithm, 1-step lookaheads (Algorithm~\ref{alg:complete}), and show that even basic planning can hold great promise. 
We illustrate our framework using two uncertainty quantification modules---GPs and 
\ensembles/ \ensembleplus. 

Throughout this section, we focus on evaluating the mean squared error of 
a regression model $\model$,  and develop adaptive policies that minimize uncertainty on $g(f)$ defined in~\eqref{eqn:l2-g-f}.
When GPs provide a valid model of uncertainty, 
our experiments show that our planning framework significantly outperforms other baselines. 
We further demonstrate that our conceptual framework extends to deep learning-based uncertainty quantification methods such as  \ensembleplus while highlighting computational challenges that need to be resolved in order to scale our ideas. 
For simplicity, we assume a naive predictor, i.e., $\psi(\cdot) \equiv 0$. However, we emphasize that this problem is just as complex as if we were using a sophisticated model $\psi(.)$. The performance gap between the algorithms 
primarily depends
on the level  of uncertainty in our prior beliefs.

To evaluate the performance of our algorithm, we benchmark it against several baselines. 
%Active learning baselines use an acquisition function $\ac$ to select points that have the highest   function value: $X\opt_t \in \argmax_{X \in \xpoolj{t}} \ac({X})$ at every step $t$. These methods may also need an UQ module, which we simply use the same UQ module as in our algorithm, and it  outputs $V(X)$ that measures the the uncertainty of each point $X \in \xpoolj{t}$.
Our first set of baselines are from active learning~\citep{AggarwalKoGuHaPh14}:
\\ % \noindent\textbf{Active Learning Heuristics:} 
\textbf{(1)} 
\textsf{Uncertainty Sampling (Static):}  In this approach, we query the samples for which the model is least certain about. Specifically, we estimate the variance of the latent output $f(X)$ for each $X \in \xpool$ using the UQ module and select the top-$K$ points with the highest uncertainty. \\
\textbf{(2)} \textsf{Uncertainty Sampling (Sequential):} This is a greedy heuristic that sequentially selects the points with the highest uncertainty within a batch, while updating the posterior beliefs using pseudo labels from the current posterior state. Unlike \textsf{Uncertainty Sampling (Static)}, this method takes into account the information gained from each point within batch, and hence tries to diversify the selected points within a batch. 

 
We also compare our approach to the  \textbf{(3)} \textsf{Random Sampling}, which selects each batch uniformly at random from the pool. Additionally, we compare solving the planning problem using  \textsf{REINFORCE}-based policy gradients with   $\mathsf{Smoothed\text{-}Autodiff}$ policy gradients.\footnote{Our code repository is available at
  \url{https://github.com/namkoong-lab/adaptive-labeling}.}
%Detailed experimental setups are provided in Section \ref{sec:details-experiments}.

%We repeat all experiments with 10 random seeds.




\begin{figure}[t]
\centering
\begin{minipage}[b]{0.49\textwidth}
\centering
\includegraphics[width=\textwidth, height=5cm]{figures/original_scale/Var_of_l_2_loss.pdf}
\caption{(Synthetic data) Variance of mean squared loss evaluated through the posterior belief $\mu_t$ at each horizon $t$. This is the objective that policy gradient methods like \textsf{REINFORCE} and $\ouralgo$ optimizes. 1-step lookaheads are surprisingly effective even in long horizons.}
\label{fig:var-l2-sim}
\end{minipage}
\hfill
\begin{minipage}[b]{0.49\textwidth}
\centering \includegraphics[width=\textwidth, height=5cm]{figures/original_scale/Error_of_estimated_model_l_2_loss.pdf}
\caption{(Synthetic data) Error between MSE calculated based on collected data $\mc{D}^{0:T}$ vs. population oracle MSE over $\mc{D}_{\rm eval} \sim P_X$. Reducing uncertainty over posteriors directly leads to better OOD evaluations. 1-step lookaheads significantly outperform active learning heuristics in small horizons.}
\label{fig:mean-l2-sim}
\end{minipage}
%\caption{Simulated data for GPs}
%\label{fig:both_plots}
\end{figure}

\subsection{Planning with Gaussian processes}
\label{sec:experiment-plan-GP}
We now briefly describe the data generation process for the GP experiments,  deferring a more detailed discussion of the dataset generation to Section~\ref{sec:details-experiments}. 
We use both the synthetic data and the real data to test our methodology.
For the \emph{simulated data},  we construct a setting where the general population is distributed across \emph{51 non-overlapping clusters} while the initial labeled data $\dtrain$ just comes from one cluster. In contrast, both $\dpool \defeq (\xpool,\ypool),\deval \defeq (\xeval,\yeval)$ are generated   from all the clusters. 
We begin with a low-dimensional scenario, generating a one-dimensional regression setting using a GP. %Gaussian Process (GP).
Although the data-generating process is not known to the algorithms,  we assume that the GP hyperparameters are known to all the algorithms
to ensure fair comparisons. This can be viewed as a setting where our prior is well-specified, allowing us to isolate the effects
of different policy optimization approaches
 without any concerns about the misspecified priors. We select $10$ batches, each of size $K=5$ across $T = 10$ time horizons.

To examine the robustness of our method against the distributional assumptions made  in the simulated case, we then move to a real dataset where the correct prior is not known. We simulate selection bias from the eICU dataset~\citep{PollardJoRaCeMaBa18}, which contains real-world patient data with in-hospital mortality outcomes. 
We conduct a $k$-means clustering to generate 51 clusters and then select data from those clusters. We view this to be a credible replication of practice, as severe distribution shifts are common due to selection bias in clinical labels.  To convert the binary mortality labels into a regression setting, we train a  random forest classifier and fit a GP on predicted scores, which serves as the UQ module for all the algorithms. As before, the task is to select 10 batches, each consisting of 5 samples, across 10 time horizons.

 In Figures~\ref{fig:var-l2-sim} and~\ref{fig:mean-l2-sim}, we present results for the simulated data. 
Figure~\ref{fig:var-l2-sim} shows the variance of $\ell_2$ loss, and Figure~\ref{fig:mean-l2-sim} presents the error in the estimated $\ell_2$ loss using $\mu_t$ (relative to true $\ell_2$ loss, that is unknown to the algorithm). 
As we can see from these plots, our method one-step lookahead  gives substantial improvements  over active learning baselines and random sampling. In addition,
compared to the one-step lookahead planning approach using \textsf{REINFORCE}-based policy gradients, 
we observe that $\mathsf{Smoothed\text{-}Autodiff}$-based policy gradients provide significantly more robust performance over all horizons.

In Figures~\ref{fig:var-l2-real}~and~\ref{fig:mean-l2-real}, we observe similar findings on the eICU data. We see that planning policies (\textsf{REINFORCE} and $\mathsf{Smoothed\text{-}Autodiff}$) consistently outperform other heuristics by a large margin.  Active learning baselines perform poorly in these small-horizon batched problems and can sometimes be even worse than the random search baselines.  Overall, our results show the importance of careful planning in adaptive labeling for reliable model evaluation. 

We offer some intuition as to why one-step lookahead planning may outperform other heuristic algorithms. 
 First,  \textsf{Uncertainty sampling (Static)} while myopically selects the
 top-$K$ inputs with the highest uncertainty, it fails to consider 
the overlap in information content among the ``best” instances; see \citep{AggarwalKoGuHaPh14} for more details. 
In other words,  it might acquire points from the same region with high uncertainty while failing to induce diversity among the batch.
Although \textsf{Uncertainty Sampling (Sequential)} somewhat addresses the issue of information overlap, a significant drawback of 
this algorithm
is the disconnect between the objective we aim to optimize and the algorithm. For example, it might sample from a region with high uncertainty but very low density. 

\begin{figure}[t]
\centering
\begin{minipage}[b]{0.48\textwidth}
\centering
\includegraphics[width=\textwidth, height=5cm]{figures/original_scale/Var_of_l_2_loss_real.pdf}
\caption{(Real-world eICU data) Variance of mean squared loss evaluated through the posterior belief $\mu_t$ at each horizon $t$. Even 1-step lookaheads are extremely effective planners, and auto-differentiation-based pathwise policy gradients provide a reliable optimization algorithm based on low-variance gradient estimates.}
\label{fig:var-l2-real}
\end{minipage}
\hfill
\begin{minipage}[b]{0.48\textwidth}
\centering \includegraphics[width=\textwidth, height=5cm]{figures/original_scale/Error_of_estimated_model_l_2_loss_real.pdf}
\caption{(Real-world eICU data) Error between MSE calculated based on collected data $\mc{D}^{0:T}$ vs. population oracle MSE over $\mc{D}_{\rm eval} \sim P_X$. Reducing uncertainty over posteriors directly leads to better OOD evaluations. Our method significantly outperforms active learning-based heuristics, and random sampling.}
\label{fig:mean-l2-real}
\end{minipage}
%\caption{Real data for GPs}
\end{figure}
 
%\vspace{-1.5cm}
% \begin{wrapfigure}{r}{.32\columnwidth}
%   \vspace{-.5cm} 
%   \centering
% \includegraphics[scale=.29]{figures/Var of l2l_2 loss.pdf}
%   \vspace{-0.2cm}
%   \caption{Results of GP}
% \label{fig:var-l2-gp}
%   \vspace{-0.1cm}
% \end{wrapfigure}


% Attempts have been made  in the past to address these  drawbacks heuristically  (see \citep{AggarwalKoGuHaPh14}). We give a unified computational framework while approaching the problem in a more principled manner and solving it more optimally.




\subsection{Planning with  neural network-based uncertainty quantification methods ($\ensembleplus$)}


We now provide a proof-of-concept that shows the generalizability of our conceptual framework  to the deep learning-based UQ modules, specifically focusing on $\ensembleplus$ due to their previously observed superior performance~\citep{OsbandWenAsDwIbLuRo23}. Recall that implementing our framework with deep learning-based UQ modules  requires us to retrain the model across multiple possible random actions $\bm{a}(\theta)$ sampled from the current policy $\pi_\theta$.
This requires significant computational resources, in sharp contrast to the GPs where the posteriors are in closed form and can be readily updated and differentiated. 

Due to the computational constraints, we test $\ensembleplus$ on a toy setting to demonstrate the generalizability of our framework. We consider a setting where the general population consists of four clusters, while the initial labeled data only comes from one cluster. Again we generate data using GPs.  The task is to select a batch of 2 points in one horizon. We detail the $\ensembleplus$ architecture in Section \ref{sec:details-experiments}, and we assume prior uncertainty to be large (depends on the scaling of the prior generating functions). 
The results are summarized in the Table~\ref{tab:UQ_ensemble}.

% \begin{table}[H]
% \vspace{-10pt}
% \caption{Performance under \ensembleplus as UQ module}
%     \centering
%     \begin{tabular}{|m{3cm}|m{2.5cm}|m{2cm}|} 
%     \hline
%       Algorithm   & Variance of $\loss_2$ loss estimate & Error of $\loss_2$ loss estimate  \\ \hline Random Sampling 
%          & $1710.9 \pm 1352.1$ & $8.67\pm6.62$ 
%       \\ \hline \ouralgo & $1.30 \pm 0.68$ & $0.91\pm0.25$ \\ \hline
%     \end{tabular}
%     \label{tab:UQ_ensemble}
%     %\vspace{-10pt}
% \end{table}




\begin{table}[h]
\vspace{-10pt}
\caption{Performance under \ensembleplus as the UQ module}
\centering
\begin{tabular}{|l|l|l|}
\hline
Algorithm   & Variance of $\loss_2$ loss estimate & Error of $\loss_2$ loss estimate  \\
\hline
\textsf{Random sampling} & 7129.8 $\pm$ 1027.0 & 136.2 $\pm$ 8.28 \\ \hline
\textsf{Uncertainty sampling (Static)} & 10852 $\pm$ 0.0 & 162.156 $\pm$ 0.0 \\ \hline
\textsf{Uncertainty sampling (Sequential)} & 8585.5 $\pm$ 898.9 & 144 $\pm$ 6.93 \\ \hline
\textsf{REINFORCE} & 1697.1 $\pm$ 0.0 & 45.27 $\pm$ 0.0 \\ \hline
\ouralgo & 1697.1 $\pm$ 0.0 & 45.27 $\pm$ 0.0 \\ \hline
\end{tabular}
%\caption{Comparison of different algorithms based on variance   and   error in $\ell_2$ loss estimation with Ensemble $+$ as the UQ module. Our results demonstrate that {\ouralgo} and REINFORCE outperformthe other active learning based heuristics, confirming the benefits of our MDP formulation for the adaptive labeling problem, as also demonstrated in Section 4.\\
%\footnotesize{Experimental details: We use Gaussian Processes as our data generating process, GP parameters are the same as in Section D.3.  The task is to select a batch of 2 points along one horizon.The marginal distribution $p_X$ has 4 \textit{non-overlapping} clusters. Initial data comes from one cluster, while pool and evaluation points comes from all the clusters. We have $20$ initial labeled data points, $10$ pool points, and $252$ evaluation points.  Training procedures are similar to the one in Section D.3.} }
\label{tab:UQ_ensemble}
\end{table}



% We faced  issues in scaling up these experiments which will be our focus in the future. 





% \begin{itemize}
%     \item Posteriors should be consistent. Two dimensions: even with less training,  
%     \item the inference should be  fast enough
% \end{itemize}


% Potential research directions for uncertainty quantification

% In this section we consider a simple setting We consider a simpler setting and 


% For synthetic dataset generation, we use ...... For real datasets, we use ...... We compare our methodolgy to several baselines ()    This Section is structured as follows:
% \begin{itemize}
%     \item \textbf{GPs, square loss objective} (Section \ref{}): 
%     %the broad aim of the experiments  in this section is to isolate the performance of our methodology without any concerns for the inefficiencies induced due to a mis-specified prior or imperfect posterior inference. To accomplish this we generate synthetic datasets using GPs (detailed later). We use the well specified prior (GPs - with same hyperparameter setting) as our UQ module.   
%      As GPs provide differentaible posterior inference - any errors induced due to imperfect posterior updates are also isolated. We note that under this setting
%      \item In Section\ref{} we demonstrate why our methodology performs better than other baselines - by devising various synthetic experiments ()
%     \item  \textbf{UQ Benchmarking }(Section \ref{}): Before diving into the experiments using $\ensembleplus$ and ENNs,  we showcase our benchmarking experiments in Section \ref{}. We use real datasets We observe that ENNs perform better
%      \item \textbf{Ensemble $+$}, objective: recall, accuracy
%     \item \textbf{ENN}, objective: recall, accuracy
% \end{itemize}




% In Section {}, we test 
% \subsection{Experimental details}

% \begin{itemize}
%     \item UQ methodologies - GPs, ENNs
%     \item Objectives - Recall,  ATE
%     \item Datasets - ATE-synthetic datasets, Recall-synthetic, real datasets
%     \item Baselines - 
%     \begin{itemize}
%         \item Random sampling
%         \item Active learning - Uncertainty based sampling - In regression setting almost all of the 
%         \item Myopic greedy - Greedy Batch based sampling
%         \item Policy Gradient
%     \end{itemize}
    
% \end{itemize}

% \subsection{Experiments}
%     \begin{itemize}
%     \item GPs with square loss
%     \item Benchmarking ENN
%         \item ENNs with ATE
%         \item ENNs with Recall
%     \end{itemize}

% \subsection{Benefits over other algorithms - intuition and experiments}

%Active learning - Myopic greedy / Don't rely on the objective rather some entropy version.


%%% Local Variables:
%%% mode: latex
%%% TeX-master: "main"
%%% End:


\section{Related Work} \label{sec:related}

% \textbf{Adversarial Attack}
\textbf{Attacks on SLAM.} 
%With the rise of machine learning, 
The robustness of computer vision systems is being actively investigated. With the emergence of adversarial images in the digital domain by adding optimized noise directly to images~\cite{szegedy2013intriguing,carlini2017towards}, researchers find that such attacks also exist physically in the real world \cite{eykholt2018robust,song2018physical,zhao2019seeing}. To fill the gap between attacks in the digital and physical worlds, recent studies have demonstrated that attacks on real-world computer vision systems are practical \cite{eykholt2018robust,li2019adversarial,man2020ghostimage,sharif2016accessorize,zhao2019seeing,zhou2018invisible}. However, attacks on traditional computer vision methods such as SLAM are relatively less explored. \cite{yoshida2022adversarial} proposes an attack against the scan matching algorithm in LiDAR-based SLAM, while most SLAMs in AR/VR devices rely on different sensors like RGB/depth cameras and IMUs. \cite{ikram2022perceptual} and \cite{chen2024adversary} mislead visual SLAM by poisoning the images with special patterns, and \cite{wang2021can} causes the camera to fail using infrared light. In our work, we demonstrate attacks on Visual-Inertial SLAM (VI-SLAM) by perturbing the IMU readings, rather than cameras, and showing its impact on XR user experience. 

\textbf{Acoustic Injection Attacks.} Among various physical attacks, acoustic injection attacks are attractive due to their low cost. Son~\etal~\cite{son2015rocking} were the first to introduce acoustic attacks on MEMS gyroscopes, demonstrating how these attacks could lead to sensor denial-of-service and result in drone crashes. WALNUT~\cite{trippel2017walnut} expanded on this by developing output biasing and control attacks that enable precise manipulation of MEMS accelerometer outputs using modulated sound waves. Wang et al.~\cite{wang2017sonic} demonstrated a sonic gun, showcasing the vulnerability of various smart devices (\eg drones and self-balancing vehicles) to acoustic attacks. Tu et al. \cite{tu2018injected} designed side-swing and switching attacks to alter the outputs of MEMS gyroscopes and accelerometers. Furthermore, Ji et al. \cite{ji2021poltergeist} fool the object detectors by applying acoustic attack to the image stabilizers commonly used in modern cameras. However, none of the existing works study the relationship between the acoustic injections and SLAM outputs on recent XR devices. 

% \zijian{Do we need one session about security in AR/VR?}
% \yicheng{TODO}
%\jiasi{cite the AIVR paper (UMass Amherst?) paper is we have not already. They add IMU perturbation but w/o SLAM, iirc} \yicheng{Cited}

\textbf{XR Security and Privacy.} 
%Security and privacy concerns in XR systems have gained significant attention. 
For single-user XR systems, researchers have demonstrated various side-channel attacks to extract sensitive information (\eg keystrokes) through video feeds~\cite{ling2019know}, head movements~\cite{nair2023unique, slocum2023going}, architectural hints~\cite{zhang2023its,shang2020arspy}, power usage~\cite{li2024dangers}, and EM side-channel leakages~\cite{al2021vr}. In multi-user XR systems, Su et al.~\cite{su2024remote} use avatar motion data to infer keystrokes in shared VR environments. Slocum et al.~\cite{slocum2024doesn} reveal vulnerabilities in the shared state frameworks of multi-user AR. Similarly, Lebeck et al.~\cite{lebeck2017securing} highlight risks like deceptive virtual objects and emphasize access control for managing shared physical and virtual spaces. Ruth et al.~\cite{ruth2019secure} further propose a secure multi-user AR framework focusing on content sharing and permissions.
Chandio et al.~\cite{chandio2024stealthy} %introduced a multi-modal spatiotemporal attack that 
simultaneously manipulated visual and inertial sensors to disrupt XR pose estimation. However, their study evaluated the attack using offline datasets and assumed the attacker's capability to manipulate IMU data streams through acoustic means, without real experiments. Ours is the first to demonstrate acoustic injection attacks on recent XR devices, like the Hololens 2, in the real world.
 


\section*{Acknowledgments}
{\textcopyright}2025 All rights reserved. The research described in this paper was carried out at the Jet Propulsion Laboratory, California Institute of Technology, under a contract with the National Aeronautics and Space Administration (80NM0018D0004).



\bibliographystyle{alpha}
\bibliography{ref}
\appendix
\section{Handling non-proper losses}
\label{sec:non-proper}

We consider an abstract action space $\mA$; examples are the discrete setting where $\mA = [k]$, and the continuous setting where $\mA = \R$. 
A hypothesis is a function $h: \X \to \mA$. 
A loss function is a function $\ell: \zo \times \mA \to [0,1]$. The expected loss of hypothesis $h$ at the point $x$ is given by $\ex[\ell(y, h(x))|x]$. The goal of a loss predictor is to learn a function $\lossPred: \Phi \to \R$ that gives pointwise estimates of this quantity. As in definition \ref{def:lp}, we can define a hierarchy of loss predictors based on the features available to them.


For any loss $\ell$, if the labels are drawn according to $y \sim \Ber(p)$ for any $p \in [0, 1]$, then the optimal prediction that minimizes the loss, $k_{\ell}(p) \in [0, 1]$ is defined by 
\[k_{\ell}(p) = \argmin_{v \in [0, 1]}\ex_{y \sim \Ber(\pbayes(x))}[\ell(y, v)]
\footnote{In the event that there is no unique minimum, we allow $k_\ell$ to output a subset of $\mA$, so it is strictly speaking a relation rather than a function. We blur this distinction for simplicity}.\]












If there exist a {\em latent} predictor $p_h:\X \to [0,1]$ so that $h = k_\ell \circ p_h$ is obtained by best-responding to its predictions, then we can reduce to the setting of proper losses, since 
\[ \ex[\ell(y, h(x))] = \ex[\ell(y, k_\ell(p(x)) = \ex [\ell \circ k_\ell(y, p(x))] \]
and we have the following result of \cite{kleinberg2023u}.

\begin{lemma}[\cite{kleinberg2023u}]
\label{lem:klst}
    For any loss $\ell:\zo \times \mA \to [0,1]$, the loss function $\ell \circ k_\ell: \zo \times [0,1] \to \R$ is a proper loss.
\end{lemma}

But under what conditions on $h$ does there exist such a predictor $p_h$? And is it easy to estimate its predictions given access to $h$? 
 
To answer the first question, we show that it is equivalent to assuming that the hypothesis satisfies a simple optimality condition for the loss.

\begin{definition}
    The hypothesis $h:\X \to [0,1]$ is swap-optimal for $\mD$ if for every function $\kappa:\mA \to \mA$, it holds that $\ex[\ell(y, h(x))] \leq \ex[\ell(y, \kappa(h(x)))]$.
\end{definition}
Swap optimality is a weak optimality condition that can be easily achieved by post-processing. It is quite reasonable to assume that a well-trained model optimized to minimize loss satisfies this guarantee. For instance, a well-trained image classifier should not improve if every time it predicts {\em cat}, we say {\em dog} instead. 
For a swap optimal hypothesis $h$, we show that is indeed easy to identify a latent predictor $p_h$ so that $h$ is obtained by best-responding to its predictions. This theorem lets us extend our theory of loss prediction for proper losses to arbitrary loss functions under the rather weak assumption that $h$ is swap-optimal. 

\begin{theorem}
\label{thm:swap-proper}
Given a hypothesis $h:\X \to \mA$ and a distribution $\mD$, define the predictor $p_h: \X \to [0,1]$ by $p_h(x) = \ex_\mD[y|h(x)]$. The hypothesis $h$ is swap optimal iff $h(x)= k_\ell \circ p_h(x)$ for all $x \in \X$.\footnote{Strictly speaking, $k_\ell$ is not a function as there can be many optimal actions. However we interpret this equation as saying $h(x)$ is in the set of optimal actions for $p_h(x)$.}
\end{theorem}
\begin{proof}
    Assume that $h$ is not swap-optimal, so there exist $\kappa$ such that $\ex[\ell(y, \kappa(h(x)))] < \ex[\ell(y, h(x))]$. There must exist a specific choice of $h(x) = a \in \mA$ conditioned on which the inequality still holds, hence 
    \[ \ex[\ell(y, \kappa(a))|h(x) = a]  \leq \ex[\ell(y, a)|h(x) =a]. \]
    Let $\ex[y|h(x) =a] = v$. But this shows that when $y \sim \Ber(v)$, $\ex[\ell(y, \kappa(a)] < \ex[\ell(y,a)]$, so $a \neq k_\ell(v)$. Hence for all $x \in h^{-1}(a)$, we have $h(x) = a \neq k_\ell(v) = k_\ell(p_h(x))$. 

    Conversely, if $h$ is indeed swap optimal, then it must be the case that every action $a \in \mA$ is a best response to $\E[y|h(x) =a] = p_h(x)$, which means we have $h(x) = k_\ell(p_h(x))$.
\end{proof}




\section{Proofs from Section~\ref{sec:lp-mc}}

\subsection{Proof of Theorem~\ref{thm:mc-conv-vs-loss-pred}}\label{sec:mc-conv-cs-loss-pred-pf}

\begin{proof}[Proof of Theorem~\ref{thm:mc-conv-vs-loss-pred}]
    The inequality on the left follows from Theorem~\ref{lem:adv-implies-mc-err}, while the inequality on the right follows from Lemma~\ref{lem:mc-err-implies-adv}. We prove each in turn, starting with the left-hand inequality. 

    By Theorem~\ref{lem:adv-implies-mc-err}, if there exists a $f \in \calF$ such that setting $\lossPred = f$ gives $\adv(\lossPred) = \alpha$, then this implies that 
    \[\ex[(\lossPred(\phi(p, x) - \clp(p(x)))H_{\ell}'(p(x))(y - p(x))] \geq \alpha/2.\]
    We observe that because $\lossPred = f \in \calF$, the witness of this multicalibration violation, $(\lossPred(\phi(p, x) - \clp(p(x)))H_{\ell}'(p(x))$ lies in $\calC$, and thus 
    \begin{align*}
        \MCE(\calC, p) &= \max_{c \in \calC}\left| \ex[c(\phi(p, x))(y - p(x))]\right|\\
        & \geq \ex[(\lossPred(\phi(p, x) - \clp(p(x)))H_{\ell}'(p(x))(y - p(x))]\\
        &\geq \alpha/2\\
        &= \frac{1}{2}\adv(\lossPred)
    \end{align*}

    The inequality follows by taking the maximum over all $\lossPred \in \calF$, as $\lossPred$ was chosen arbitrarily. 

    We now move on to proving the inequality on the right, i.e., the upper bound on $\MCE(\calC, p)$. 

    By definition of the multicalibration error and $\calC$, there exists some $c \in \calC$ that witnesses a multicalibration error of magnitude $\MCE(\calC, p)$, i.e. for some $f \in \calF$, 

    \[\MCE(\calC, p) = \left|\underbrace{\ex[(f(\phi(p, x)) - H_{\ell}(p(x)))H_{\ell}'(p(x))(y - p(x))]}_{:= E_f}\right|. \]

    Thus, if we define $\delta: \Phi \rightarrow [-1, 1]$ as 
    \[\delta(\phi(p, x)) = \sgn(E_f)(f(\phi(p, x)) - H_{\ell}(p(x))),\]
    it follows that 
    \begin{align*}
        \ex[\delta(\phi(p, x))H_{\ell}'(p(x))(y - p(x))] &= \MCE(\calC, p).
    \end{align*}

    Applying Lemma~\ref{lem:mc-err-implies-adv} for this $\delta$ implies that for the loss predictor defined by $\lossPred(\phi(p, x)) = \Pi_{[0,1]}(\clp(p(x)) + \MCE(\calC, p) 
    \delta(\phi(p, x)))$ satisfies 

    \[\adv(\lossPred(\phi(p, x))) \geq \MCE(\calC, p)^2.\]

    The proof of the inequality follows by observing that $\lossPred \in \calF'$, because 
    \begin{align*}
        \lossPred(\phi(p, x)) &= \Pi_{[0,1]}(\clp(p(x)) + \MCE(\calC, p) 
    \delta(\phi(p, x)))\\
    &= \Pi_{[0,1]}(H_{\ell}(p(x)) + \underbrace{\MCE(\calC, p) 
    \sgn(E_f)}_{:= \beta}(f(\phi(p, x)) - H_{\ell}(p(x))))\\
    &= \Pi_{[0,1]}((1 - \beta)H_{\ell}(p(x)) + \beta f(\phi(p, x)))
    \end{align*}

    Where $\beta = \sgn(E_f)\MCE(\calC, p) \in [-1, 1]$, because $\MCE(\calC, p) \in [0, 1]$. 

    Thus, $\lossPred \in \calF'$, and so we conclude that 

    \[\max_{\lossPred \in \calF'}\adv(\lossPred(\phi(p, x))) \geq \MCE(\calC, p)^2.\]

    We get the right-hand inequality from the statement after taking square root of both sides. 
\end{proof}


\section{Extended discussion and proofs from Section~\ref{sec:multiple-loss}}


\subsection{Proof of Lemma~\ref{lem:many-losses-mc}}\label{sec:many-losses-mc-pf}
\begin{proof}[Proof of Lemma~\ref{lem:many-losses-mc}]
    For a fixed loss $\ell \in \calL$, let 
    \[\calC_{\ell} = \{(f(\phi(p, x)) - H_{\ell}(p(x)))H'_{\ell}(p(x)) : f \in \calF\}.\]

    By Theorem~\ref{thm:mc-conv-vs-loss-pred}, we can guarantee that 
    \[\max_{\lossPred \in \calF} \adv(\lossPred) \leq 2\MCE(\calC_{\ell}, p).\]
    Taking the max over $\calL$ for both sides, we get
    \[\max_{\ell \in \calL}\max_{\lossPred \in \calF} \adv(\lossPred) \leq \max_{\ell \in \calL} 2\MCE(\calC_{\ell}, p) \leq 2\MCE(\calC_{\calL}, p).\]

    Where the right-most inequality follows from the fact that $\calC_{\calL} = \bigcup_{\ell \in \calL} \calC_{\ell}$. This proves the desired inequality.
\end{proof}

\subsection{Multicalibration for product classes}

In this section, we introduce some useful notation that we will use to refer to and relate certain classes of multicalibration test functions. 

\begin{definition}\label{def:prod-class}
    Let $\calA \subseteq \{a: \Phi \rightarrow [-1, 1]\}$ and $\calB \subseteq \{b: \Phi \rightarrow [-1, 1]\}$ be two classes of functions. We denote the product class of test functions with respect to $\calA$ and $\calB$ as $\calC_{\calA, \calB}$, and define it as 
    \[\calC_{\calA, \calB} = \{a(\phi)b(\phi) : a \in \calA, b \in \calB\}.\]
\end{definition}

We use this notation in the following lemma, which shows that multicalibration with respect to the test functions $\calC_{\calL}$ defined in Lemma~\ref{lem:many-losses-mc} is implied by multicalibration with respect to a product class of test functions:

\begin{lemma}\label{lem:loss-to-prod-class-mc}
    Let $\calF$ be a class of loss predictors $f: \Phi \rightarrow [0, 1]$. Let $\calL$ be a class of bounded proper losses $\ell: \{0, 1\} \times [0, 1] \rightarrow [0, 1]$ with associated concave entropy functions $H_{\ell}: [0, 1] \rightarrow [0,1]$, and let $\calC_{\calL}$ be the class of test functions 
    \[\calC_{\calL} = \{(f(\phi(p, x)) - H_{\ell}(p(x)))H'_{\ell}(p(x)) : f \in \calF, \ell \in \calL\}.\]

    Denote $\calH_{\calL} = \{H_{\ell} : \ell \in \calL\}$, and $\calH'_{\calL} = \{H'_{\ell}: \ell \in \calL\}$.

    Then, 

    \[\MCE(\calC_{\calL},p) \leq \MCE(\calC_{\calF, \calH'_{\calL}}, p) + \MCE(\calC_{\calH_{\calL}, \calH'_{\calL}}, p) \leq 2\MCE(\calC_{\calF\cup \calH_{\calL}, \calH'_{\calL}}, p).\]
\end{lemma}

\begin{proof}
    By definition of $\calC_{\calL}$, we can readily decompose the multicalibration error into the desired terms:

    \begin{align*}
        \MCE(\calC_{\calL},p) &= \max_{f \in \calF, \ell \in \calL}\left|\ex[(f(\phi(p, x)) - H_{\ell}(p(x)))H'_{\ell}(p(x))(y - p(x))]\right|\\
        &\leq \max_{f \in \calF, \ell \in \calL}\left|\ex[f(\phi(p, x))H'_{\ell}(p(x))(y - p(x))]\right| + \max_{\ell \in \calL}\left|\ex[H_{\ell}(p(x))H'_{\ell}(p(x))(y - p(x))]\right|\\
        &\leq \max_{f \in \calF, \ell \in \calL}\left|\ex[f(\phi(p, x))H'_{\ell}(p(x))(y - p(x))]\right| + \max_{h \in \calH_{\calL}, h' \in \calH'_{\calL}}\left|\ex[h(p(x))h'(p(x))(y - p(x))]\right|\\
        &= \MCE(\calC_{\calF, \calH'_{\calL}}, p) + \MCE(\calC_{\calH_{\calL}, \calH'_{\calL}}, p).
    \end{align*}

    This proves the left-hand inequality. The right-hand inequality follows from the observation that \[\MCE(\calC_{\calF \cup \calH_{\calL}, \calH'_{\calL}}, p) = \max\{\MCE(\calC_{\calF, \calH'_{\calL}}, p), \MCE(\calC_{\calH_{\calL}, \calH'_{\calL}}, p)\}.\]
\end{proof}

An important property of product classes is that given two classes $\calA$ and $\calB$, whenever we have a weak learner for $\calA$ and $\calB$ is finite, we can efficiently learn a multicalibrated predictor with respect to $\calC_{\calA, \calB}$. Our approach closely follows that of~\cite{gopalan2022low}, who show how to learn multicalibrated predictors for product classes where one class depends only on $x$, and the other depends only on $p(x)$. Despite this choice in setup, their particular algorithm and results naturally generalize to the case where the two function classes can have richer input spaces. 

For completeness, we translate their algorithm and results to our setting. The algorithm for product-class multicalibration can be found in Algorithm~\ref{alg:product-class} (c.f. Algorithm 1 of~\cite{gopalan2022low}).

The following lemmas prove correctness and sample complexity of the algorithm. 

\begin{lemma}[Correctness of Algorithm~\ref{alg:product-class}]
    If Algorithm~\ref{alg:product-class} returns a predictor $p : \mathcal{X} \to [0, 1]$, then $p$ satisfies 
    \[\MCE(\calC_{\calA, \calB}, p) \leq \alpha.\]
\end{lemma}

\begin{proof}
    Observe that Algorithm~\ref{alg:product-class} only returns a predictor $p_t$ if, in the $t$th iteration, for every $b \in \calB$, the call to the weak agnostic learner $\text{WAL}_{\calA}$ returns $\bot$. By the weak agnostic learning property (Definition~\ref{def:weak-agnostic-learner}), returning $\bot$ in every call indicates that for all $b \in \calB$ and for all $a \in \calA$,
    
    \[
    \ex\left[a(\phi(p_t, x))b(\phi(p_t, x))(y - p_t(x)) \right] \leq \alpha.
    \]
    
    By definition, this means that $\MCE(\calC_{\calA, \calB}, p_t) \leq \alpha$
\end{proof}

\begin{lemma}[Termination of Algorithm~\ref{alg:product-class}]
    Algorithm~\ref{alg:product-class} is guaranteed to terminate and return a $p_T$ after $T \leq 4/\alpha^2$ iterations.
\end{lemma}

\begin{proof}
    We show that the number of iterations is bounded via a potential argument, where the potential function is the squared error of the current predictor $p$ from the bayes-optimal predictor $p^*(x) = \ex[y|x]$: $\sqLoss(p) := \ex[(p(x) - p^*(x))^2]$. 

    Note that by definition, $\sqLoss(p_0) \leq 1$, and for all $p: \calX \rightarrow [0, 1]$, $\sqLoss(p) \geq 0$. 

    The change in potential after the $t$th update can be computed as follows. Note that due to the guarantee of the weak agnostic learner, because $a_{t + 1} \neq \bot$, we are guaranteed that for the $b \in \calB$ used in the update, 
    \[\ex[b(\phi(p, x))a(\phi(p, x))(y - p(x))] \geq \alpha/2.\]

    Thus, following the same proof as Lemma~\ref{lem:improve-2}, we conclude that 
    \begin{align*}
        \sqLoss(p_t) - \sqLoss(p_{t + 1}) &\geq \ex[(p^*(x) - p_t(x))^2] - \ex[(p^*(x) - p_t(x) -\frac{\alpha}{2}\delta_{t + 1}(x))^2]\\
        &= \alpha \ex[(p^*(x) - p_t(x))\delta_{t + 1}(x)] - \frac{\alpha^2}{4}\ex[\delta_{t + 1}(x)^2]\\
        &= \alpha \ex[(p^*(x) - p_t(x))a_{t + 1}(\phi(p_t, x))b(\phi(p_t, x))] - \frac{\alpha^2}{4}\ex[a_{t + 1}(\phi(p_t, x))^2b(\phi(p_t, x))^2]\\
        &\geq \alpha^2/2 - \alpha^2/4\\
        &= \alpha^2/4.
    \end{align*}

    Thus, the potential function decreases by at least $\alpha^2/4$ in each round, and since $\sqLoss(p_0) \leq 1$ and $\sqLoss(p_t) \geq 0$ for all $t$, the total number of iterations is bounded by $T < 4/\alpha^2$.
    \end{proof}

We finally turn to the sample complexity and success probability of Algorithm~\ref{alg:product-class}. 

\begin{lemma}\label{lem:finite-B-mc-alg}
    Let $\alpha, \delta > 0$. Let $\calA \subseteq \{a: \Phi \rightarrow [-1, 1]\}$ and $\calB \subseteq \{b: \Phi \rightarrow [-1, 1]\}$ be two classes of functions, where we assume $\calA$ is closed under negation, and $\calB$ is finite. Suppose we have access to an $\alpha$-weak-agnostic-learner for $\calA$ with sample complexity $n$ and failure parameter $\beta \leq \frac{\alpha^2\delta}{4|\calB|}$. 

    Then, given $m = O(n/\alpha^2)$ samples, with probability at least $1 - \delta$ Algorithm~\ref{alg:product-class} outputs a predictor $p$ with $\MCE(\calC_{\calA, \calB}, p) \leq \alpha$. 
\end{lemma}

\begin{proof}
    We assume that we use a fresh sample for the weak agnostic learner at each iteration of size $n$. Because the algorithm terminates in at most $4/\alpha^2$ iterations, we thus need at most $4n/\alpha^2 = O(n/\alpha^2)$ fresh samples. 

    For the failure bound, note that we make at most $4|\calB|/\alpha^2$ calls to the weak agnostic learner. Via a union bound, because we assume that $\beta \leq \frac{\alpha^2\delta}{4|\calB|}$, we conclude that the probability that at least one of the calls to the weak learner fails is bounded by $\delta$, as desired.   
\end{proof}

\begin{algorithm}
    \caption{Product-Class Multicalibration}
    \begin{algorithmic}[1]\label{alg:product-class}
        \STATE \textbf{Input:} training data $\{(x_i, y_i)\}_{i=1}^m$
        \STATE \textbf{Function classes:} \\
        \quad $\calA \subseteq \{a: \Phi \to [-1, 1]\}$ (closed under negation), \\
         \quad $\calB \subseteq \{b: \Phi \to [-1, 1]\}$ (finite)
        \STATE \textbf{$\alpha$-Weak Agnostic Learner for $\calA$:} $\text{WAL}_{\calA}$
        \STATE approximation $\alpha > 0$
        \STATE \textbf{Output:} $(\calC_{\calA, \calB},\alpha)$-multicalibrated predictor $p: \mathcal{X} \to [0,1]$
        \STATE $p_0(\cdot) \gets 1/2$
        \STATE $mc \gets \text{false}$
        \STATE $t \gets 0$
        \WHILE{$\neg mc$}
            \STATE $mc \gets \text{true}$
            \FOR{each $b \in \calB$}
                \STATE $a_{t+1} \gets \text{WAL}_{\calA}(\{(\phi(p, x_i), b(\phi(p_t, x_i))(y_i - p_t(x_i)))\}_{i=1}^m)$
                \IF{$a_{t+1} = \bot$}
                    \STATE \textbf{continue}
                \ELSE
                    \STATE $\delta_{t+1}(\cdot) \gets b(\phi(p_t, \cdot)) a_{t+1}(\phi(p_t, \cdot))$
                    \STATE $p_{t+1}(\cdot) \gets \Pi_{[0,1]}(p_t(\cdot) + \frac{\alpha}{2} \delta_{t+1}(\cdot)) $ 
                    \STATE $mc \gets \text{false}$
                    \STATE $t \gets t + 1$
                    \STATE \textbf{break}
                \ENDIF
            \ENDFOR
        \ENDWHILE
        \STATE \textbf{return} $p_t$
    \end{algorithmic}
\end{algorithm}

\subsection{Multicalibration for classes with approximate bases}

In this section, we show that Algorithm~\ref{alg:product-class} can also be used to guarantee multicalibration for product classes when $\calB$ is not finite, but has a finite approximate basis. 

\begin{definition}[Finite Approximate Basis]\label{def:finite-approx-basis}
     Let $\Gamma$ be a set and $\mathcal{F} = \{f : \Gamma \to [-1, 1]\}$ a class of functions on $\Gamma$. We say that a set $\mathcal{G} = \{g : \Gamma \to [-1, 1]\}$ is a finite $\epsilon$-basis for $\mathcal{F}$ of size $d$ and coefficient norm $\lambda$, if $\calG = \{g_1, ..., g_d\}$, and for every function $f \in \mathcal{F}$, there exist coefficients $\alpha_1, \alpha_2, \ldots, \alpha_d \in [-1,1]$ satisfying
\begin{equation}
    \forall x \in \Gamma \quad \left| f(x) - \sum_{i=1}^{d} \alpha_i g_i(x) \right| \leq \epsilon \quad \text{and} \quad \sum_{i=1}^{d} |\alpha_i| \leq \lambda. 
\end{equation}
\end{definition}

\begin{lemma}\label{lem:basis-mce-bound}
    Let $\calA \subseteq \{a: \Phi \rightarrow [-1, 1]\}$ and $\calB \subseteq \{b: \Phi \rightarrow [-1, 1]\}$ be two classes of functions. Suppose that $\calB$ has a finite approximate $\epsilon$-basis $\calG$ with coefficient norm $\lambda$. Then, for any predictor $p: \calX \rightarrow [0, 1]$, 
    \[\MCE(\calC_{\calA, \calB}, p) \leq \lambda\MCE(\calC_{\calA, \calG}, p) + \epsilon\]
\end{lemma}

\begin{proof}
    The upper bound quickly follows from expanding the definition of $\MCE(\calC_{\calA, \calB}, p)$. Note that 

    \begin{align*}
        \MCE(\calC_{\calA, \calB}, p) &= \max_{a \in \calA, b \in \calB}\left|\ex[a(\phi(p, x))b(\phi(p, x))(y - p(x))]\right|\\
        &\leq \max_{a \in \calA, b \in \calB}\left|\ex[a(\phi(p, x))\left(\sum_{i = 1}^d g_i(\phi(p, x))\alpha_i(b)\right)(y - p(x))]\right| + \epsilon\\
        &\leq \max_{a \in \calA, g \in \calG}\lambda\left|\ex[a(\phi(p, x))g(\phi(p, x))(y - p(x))]\right| + \epsilon\\
        &= \lambda\MCE(\calC_{\calA, \calG}, p) + \epsilon
    \end{align*}
\end{proof}

Thus, we get the following immediate Corollary of Lemmas~\ref{lem:finite-B-mc-alg} and \ref{lem:basis-mce-bound}:

\begin{corollary}\label{cor:approx-basis-alg}
    Let $\alpha, \delta > 0$. Let $\calA \subseteq \{a: \Phi \rightarrow [-1, 1]\}$ and $\calB \subseteq \{b: \Phi \rightarrow [-1, 1]\}$ be two classes of functions, where we assume $\calA$ is closed under negation, and $\calB$ has a finite approximate $\epsilon$-basis of size $d$ and coefficient norm $\lambda$. Suppose we have access to an $\alpha$-weak-agnostic-learner for $\calA$ with sample complexity $n$ and failure parameter $\beta \leq \frac{\alpha^2\delta}{4d}$. 

    Then, there exists an algorithm that, given $m = O(n/\alpha^2)$ samples, with probability at least $1 - \delta$ outputs a predictor $p$ with $\MCE(\calC_{\calA, \calB}, p) \leq \lambda\alpha + \epsilon$. 
\end{corollary}

\subsection{Instantiating multicalibration for loss prediction}

We are finally ready to show that we can use multicalibration to learn a predictor with accurate self-entropy predictions for any loss in a rich class. 

\begin{theorem}\label{thm:general-approx-basis-mc}
    Fix $\delta \geq 0$. Let $\calL$ be some class of bounded proper losses, and let $\calF$ be a class of loss predictors $\calF: \Phi \rightarrow [-1, 1]$ that is closed under negation and contains the class of self entropy predictors, $\calH_{\calL} = \{H_{\ell}\}_{\ell \in \calL}$. Suppose that the class of functions $\calH'_{\calL} =\{H'_{\ell}\}_{\ell \in \calL}$ has a finite approximate $\epsilon$-basis of size $d$ and coefficient norm $\lambda$. Further assume that we have access to an $\alpha$-weak-agnostic-learner for $\calF$ with sample complexity $n$ and failure parameter $\beta \leq \frac{\alpha^2\delta}{4d}$. 

    Then, there exists an algorithm that, given $m = O(n/\alpha^2)$ samples, with probability at least $1 - \delta$ outputs a predictor $p$ such that 
    \[\max_{\ell \in \calL} \max_{\lossPred \in \calF} \adv(\lossPred) \leq 4\lambda\alpha + 4\epsilon.\]
\end{theorem}

\begin{proof}
    The proof combines the helper lemmas we have proved in this section. 

    First, note that by Lemma~\ref{lem:many-losses-mc}, we are guaranteed that 
    \[\max_{\ell \in \calL} \max_{\lossPred \in \calF} \adv(\lossPred) \leq 2\MCE(\calC_{\calL}, p),\]
    where $\calC_{\calL}$ is defined as in Lemma~\ref{lem:many-losses-mc}. 

    By Lemma~\ref{lem:loss-to-prod-class-mc}, we can further guarantee that 
    \[\max_{\ell \in \calL} \max_{\lossPred \in \calF} \adv(\lossPred) \leq 4\MCE(\calC_{\calF \cup \calH_{\calL}, \calH'_{\calL}}, p) = 4\MCE(\calC_{\calF, \calH'_{\calL}}, p),\]
    where the last equality follows from our assumption that $\calF$ contains $\calH_{\calL}$.

    From here, we can now apply the result of Corollary~\ref{cor:approx-basis-alg}, which guarantees that given $m = O(n/\alpha^2)$ samples, we can output a predictor satisfying 
    $\MCE(\calC_{\calF, \calH'_{\calL}}, p) \leq \lambda\alpha + \epsilon$. 

    Substituting this bound into our upper bound, we conclude that we get a predictor satisfying 
    \[\max_{\ell \in \calL} \max_{\lossPred \in \calF} \adv(\lossPred) \leq 4\lambda\alpha + 4\epsilon.\]

    This completes the proof. 
\end{proof}

\subsection{Proof of Theorem~\ref{thm:1-lip-mc-pred}}\label{sec:1-lip-mc-pred-pf}

Theorem~\ref{thm:1-lip-mc-pred} instantiates the general result of Theorem~\ref{thm:general-approx-basis-mc} for a class $\calL$ that has a finite approximate basis: the class of all 1-Lipschitz proper losses. 

We appeal to a result proved by~\cite{OKK25}, which proves the existence of such a basis for this class:

\begin{lemma}[\cite{OKK25}, Lemma 5.4]\label{lem:1-lip-basis}
    Let $\epsilon > 0$, and let $\calL_{Lip}$ be the class of proper 1-Lipschitz loss functions $\ell: \{0, 1\} \times [0, 1] \rightarrow [0,1]$. The class $\{H'_{\ell}\}_{\ell \in \calL_{Lip}}$ admits an $\epsilon$-approximate basis of size $\lceil \frac{2}{\epsilon} + 1 \rceil$ and coefficient norm 4. 
\end{lemma}

with this Lemma in hand, we are ready to prove the theorem. 

\begin{proof}[Proof of Theorem~\ref{thm:1-lip-mc-pred}]
    We instantiate the basis from Lemma~\ref{lem:1-lip-basis}, and use this as the input to Theorem~\ref{thm:general-approx-basis-mc}.

    Because $d = \lceil 2/\epsilon + 1\rceil$, the bound on the weak-agnostic-learner's error probability becomes $\beta \leq \frac{\alpha^2\delta}{4\lceil 2/\epsilon + 1\rceil}$, and the resulting error guarantee gives us a predictor $p$ satisfying 

    \[\max_{\ell \in \calL_{Lip}} \max_{\lossPred \in \calF} \adv(\lossPred) \leq 16\alpha + 4\epsilon\]
    with probability at least $1 - \delta$. 
\end{proof}




\end{document}
