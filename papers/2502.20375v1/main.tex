\documentclass[11pt]{article}

%
\setlength\unitlength{1mm}
\newcommand{\twodots}{\mathinner {\ldotp \ldotp}}
% bb font symbols
\newcommand{\Rho}{\mathrm{P}}
\newcommand{\Tau}{\mathrm{T}}

\newfont{\bbb}{msbm10 scaled 700}
\newcommand{\CCC}{\mbox{\bbb C}}

\newfont{\bb}{msbm10 scaled 1100}
\newcommand{\CC}{\mbox{\bb C}}
\newcommand{\PP}{\mbox{\bb P}}
\newcommand{\RR}{\mbox{\bb R}}
\newcommand{\QQ}{\mbox{\bb Q}}
\newcommand{\ZZ}{\mbox{\bb Z}}
\newcommand{\FF}{\mbox{\bb F}}
\newcommand{\GG}{\mbox{\bb G}}
\newcommand{\EE}{\mbox{\bb E}}
\newcommand{\NN}{\mbox{\bb N}}
\newcommand{\KK}{\mbox{\bb K}}
\newcommand{\HH}{\mbox{\bb H}}
\newcommand{\SSS}{\mbox{\bb S}}
\newcommand{\UU}{\mbox{\bb U}}
\newcommand{\VV}{\mbox{\bb V}}


\newcommand{\yy}{\mathbbm{y}}
\newcommand{\xx}{\mathbbm{x}}
\newcommand{\zz}{\mathbbm{z}}
\newcommand{\sss}{\mathbbm{s}}
\newcommand{\rr}{\mathbbm{r}}
\newcommand{\pp}{\mathbbm{p}}
\newcommand{\qq}{\mathbbm{q}}
\newcommand{\ww}{\mathbbm{w}}
\newcommand{\hh}{\mathbbm{h}}
\newcommand{\vvv}{\mathbbm{v}}

% Vectors

\newcommand{\av}{{\bf a}}
\newcommand{\bv}{{\bf b}}
\newcommand{\cv}{{\bf c}}
\newcommand{\dv}{{\bf d}}
\newcommand{\ev}{{\bf e}}
\newcommand{\fv}{{\bf f}}
\newcommand{\gv}{{\bf g}}
\newcommand{\hv}{{\bf h}}
\newcommand{\iv}{{\bf i}}
\newcommand{\jv}{{\bf j}}
\newcommand{\kv}{{\bf k}}
\newcommand{\lv}{{\bf l}}
\newcommand{\mv}{{\bf m}}
\newcommand{\nv}{{\bf n}}
\newcommand{\ov}{{\bf o}}
\newcommand{\pv}{{\bf p}}
\newcommand{\qv}{{\bf q}}
\newcommand{\rv}{{\bf r}}
\newcommand{\sv}{{\bf s}}
\newcommand{\tv}{{\bf t}}
\newcommand{\uv}{{\bf u}}
\newcommand{\wv}{{\bf w}}
\newcommand{\vv}{{\bf v}}
\newcommand{\xv}{{\bf x}}
\newcommand{\yv}{{\bf y}}
\newcommand{\zv}{{\bf z}}
\newcommand{\zerov}{{\bf 0}}
\newcommand{\onev}{{\bf 1}}

% Matrices

\newcommand{\Am}{{\bf A}}
\newcommand{\Bm}{{\bf B}}
\newcommand{\Cm}{{\bf C}}
\newcommand{\Dm}{{\bf D}}
\newcommand{\Em}{{\bf E}}
\newcommand{\Fm}{{\bf F}}
\newcommand{\Gm}{{\bf G}}
\newcommand{\Hm}{{\bf H}}
\newcommand{\Id}{{\bf I}}
\newcommand{\Jm}{{\bf J}}
\newcommand{\Km}{{\bf K}}
\newcommand{\Lm}{{\bf L}}
\newcommand{\Mm}{{\bf M}}
\newcommand{\Nm}{{\bf N}}
\newcommand{\Om}{{\bf O}}
\newcommand{\Pm}{{\bf P}}
\newcommand{\Qm}{{\bf Q}}
\newcommand{\Rm}{{\bf R}}
\newcommand{\Sm}{{\bf S}}
\newcommand{\Tm}{{\bf T}}
\newcommand{\Um}{{\bf U}}
\newcommand{\Wm}{{\bf W}}
\newcommand{\Vm}{{\bf V}}
\newcommand{\Xm}{{\bf X}}
\newcommand{\Ym}{{\bf Y}}
\newcommand{\Zm}{{\bf Z}}

% Calligraphic

\newcommand{\Ac}{{\cal A}}
\newcommand{\Bc}{{\cal B}}
\newcommand{\Cc}{{\cal C}}
\newcommand{\Dc}{{\cal D}}
\newcommand{\Ec}{{\cal E}}
\newcommand{\Fc}{{\cal F}}
\newcommand{\Gc}{{\cal G}}
\newcommand{\Hc}{{\cal H}}
\newcommand{\Ic}{{\cal I}}
\newcommand{\Jc}{{\cal J}}
\newcommand{\Kc}{{\cal K}}
\newcommand{\Lc}{{\cal L}}
\newcommand{\Mc}{{\cal M}}
\newcommand{\Nc}{{\cal N}}
\newcommand{\nc}{{\cal n}}
\newcommand{\Oc}{{\cal O}}
\newcommand{\Pc}{{\cal P}}
\newcommand{\Qc}{{\cal Q}}
\newcommand{\Rc}{{\cal R}}
\newcommand{\Sc}{{\cal S}}
\newcommand{\Tc}{{\cal T}}
\newcommand{\Uc}{{\cal U}}
\newcommand{\Wc}{{\cal W}}
\newcommand{\Vc}{{\cal V}}
\newcommand{\Xc}{{\cal X}}
\newcommand{\Yc}{{\cal Y}}
\newcommand{\Zc}{{\cal Z}}

% Bold greek letters

\newcommand{\alphav}{\hbox{\boldmath$\alpha$}}
\newcommand{\betav}{\hbox{\boldmath$\beta$}}
\newcommand{\gammav}{\hbox{\boldmath$\gamma$}}
\newcommand{\deltav}{\hbox{\boldmath$\delta$}}
\newcommand{\etav}{\hbox{\boldmath$\eta$}}
\newcommand{\lambdav}{\hbox{\boldmath$\lambda$}}
\newcommand{\epsilonv}{\hbox{\boldmath$\epsilon$}}
\newcommand{\nuv}{\hbox{\boldmath$\nu$}}
\newcommand{\muv}{\hbox{\boldmath$\mu$}}
\newcommand{\zetav}{\hbox{\boldmath$\zeta$}}
\newcommand{\phiv}{\hbox{\boldmath$\phi$}}
\newcommand{\psiv}{\hbox{\boldmath$\psi$}}
\newcommand{\thetav}{\hbox{\boldmath$\theta$}}
\newcommand{\tauv}{\hbox{\boldmath$\tau$}}
\newcommand{\omegav}{\hbox{\boldmath$\omega$}}
\newcommand{\xiv}{\hbox{\boldmath$\xi$}}
\newcommand{\sigmav}{\hbox{\boldmath$\sigma$}}
\newcommand{\piv}{\hbox{\boldmath$\pi$}}
\newcommand{\rhov}{\hbox{\boldmath$\rho$}}
\newcommand{\upsilonv}{\hbox{\boldmath$\upsilon$}}

\newcommand{\Gammam}{\hbox{\boldmath$\Gamma$}}
\newcommand{\Lambdam}{\hbox{\boldmath$\Lambda$}}
\newcommand{\Deltam}{\hbox{\boldmath$\Delta$}}
\newcommand{\Sigmam}{\hbox{\boldmath$\Sigma$}}
\newcommand{\Phim}{\hbox{\boldmath$\Phi$}}
\newcommand{\Pim}{\hbox{\boldmath$\Pi$}}
\newcommand{\Psim}{\hbox{\boldmath$\Psi$}}
\newcommand{\Thetam}{\hbox{\boldmath$\Theta$}}
\newcommand{\Omegam}{\hbox{\boldmath$\Omega$}}
\newcommand{\Xim}{\hbox{\boldmath$\Xi$}}


% Sans Serif small case

\newcommand{\Gsf}{{\sf G}}

\newcommand{\asf}{{\sf a}}
\newcommand{\bsf}{{\sf b}}
\newcommand{\csf}{{\sf c}}
\newcommand{\dsf}{{\sf d}}
\newcommand{\esf}{{\sf e}}
\newcommand{\fsf}{{\sf f}}
\newcommand{\gsf}{{\sf g}}
\newcommand{\hsf}{{\sf h}}
\newcommand{\isf}{{\sf i}}
\newcommand{\jsf}{{\sf j}}
\newcommand{\ksf}{{\sf k}}
\newcommand{\lsf}{{\sf l}}
\newcommand{\msf}{{\sf m}}
\newcommand{\nsf}{{\sf n}}
\newcommand{\osf}{{\sf o}}
\newcommand{\psf}{{\sf p}}
\newcommand{\qsf}{{\sf q}}
\newcommand{\rsf}{{\sf r}}
\newcommand{\ssf}{{\sf s}}
\newcommand{\tsf}{{\sf t}}
\newcommand{\usf}{{\sf u}}
\newcommand{\wsf}{{\sf w}}
\newcommand{\vsf}{{\sf v}}
\newcommand{\xsf}{{\sf x}}
\newcommand{\ysf}{{\sf y}}
\newcommand{\zsf}{{\sf z}}


% mixed symbols

\newcommand{\sinc}{{\hbox{sinc}}}
\newcommand{\diag}{{\hbox{diag}}}
\renewcommand{\det}{{\hbox{det}}}
\newcommand{\trace}{{\hbox{tr}}}
\newcommand{\sign}{{\hbox{sign}}}
\renewcommand{\arg}{{\hbox{arg}}}
\newcommand{\var}{{\hbox{var}}}
\newcommand{\cov}{{\hbox{cov}}}
\newcommand{\Ei}{{\rm E}_{\rm i}}
\renewcommand{\Re}{{\rm Re}}
\renewcommand{\Im}{{\rm Im}}
\newcommand{\eqdef}{\stackrel{\Delta}{=}}
\newcommand{\defines}{{\,\,\stackrel{\scriptscriptstyle \bigtriangleup}{=}\,\,}}
\newcommand{\<}{\left\langle}
\renewcommand{\>}{\right\rangle}
\newcommand{\herm}{{\sf H}}
\newcommand{\trasp}{{\sf T}}
\newcommand{\transp}{{\sf T}}
\renewcommand{\vec}{{\rm vec}}
\newcommand{\Psf}{{\sf P}}
\newcommand{\SINR}{{\sf SINR}}
\newcommand{\SNR}{{\sf SNR}}
\newcommand{\MMSE}{{\sf MMSE}}
\newcommand{\REF}{{\RED [REF]}}

% Markov chain
\usepackage{stmaryrd} % for \mkv 
\newcommand{\mkv}{-\!\!\!\!\minuso\!\!\!\!-}

% Colors

\newcommand{\RED}{\color[rgb]{1.00,0.10,0.10}}
\newcommand{\BLUE}{\color[rgb]{0,0,0.90}}
\newcommand{\GREEN}{\color[rgb]{0,0.80,0.20}}

%%%%%%%%%%%%%%%%%%%%%%%%%%%%%%%%%%%%%%%%%%
\usepackage{hyperref}
\hypersetup{
    bookmarks=true,         % show bookmarks bar?
    unicode=false,          % non-Latin characters in AcrobatÕs bookmarks
    pdftoolbar=true,        % show AcrobatÕs toolbar?
    pdfmenubar=true,        % show AcrobatÕs menu?
    pdffitwindow=false,     % window fit to page when opened
    pdfstartview={FitH},    % fits the width of the page to the window
%    pdftitle={My title},    % title
%    pdfauthor={Author},     % author
%    pdfsubject={Subject},   % subject of the document
%    pdfcreator={Creator},   % creator of the document
%    pdfproducer={Producer}, % producer of the document
%    pdfkeywords={keyword1} {key2} {key3}, % list of keywords
    pdfnewwindow=true,      % links in new window
    colorlinks=true,       % false: boxed links; true: colored links
    linkcolor=red,          % color of internal links (change box color with linkbordercolor)
    citecolor=green,        % color of links to bibliography
    filecolor=blue,      % color of file links
    urlcolor=blue           % color of external links
}
%%%%%%%%%%%%%%%%%%%%%%%%%%%%%%%%%%%%%%%%%%%



\title{When does a predictor know its own loss?}
\author{Aravind Gollakota\\
Apple\\
\and
Parikshit Gopalan\\
Apple\\
\and
Aayush Karan\\
Harvard University\thanks{Work done  while interning at Apple. Authors in alphabetical order.}\\
\and 
Charlotte Peale\\
Stanford University\footnotemark[1]\\
\and
Udi Wieder\\
Apple
}


\begin{document}
\date{February 18, 2025}

\maketitle
\begin{abstract}

    Given a predictor and a loss function, how well can we predict the loss that the predictor will incur on an input? This is the problem of loss prediction, a key computational task associated with uncertainty estimation for a predictor. In a classification setting, a predictor will typically predict a distribution over labels and hence have its own estimate of the loss that it will incur, given by the entropy of the predicted distribution. Should we trust this estimate? In other words, when does the predictor know what it knows and what it does not know?

    In this work we study the theoretical foundations of loss prediction.
    Our main contribution is to establish tight connections between nontrivial loss prediction and certain forms of multicalibration \cite{hebert2018multicalibration}, a multigroup fairness notion that asks for calibrated predictions across computationally identifiable subgroups.
    Formally, we show that a loss predictor that is able to improve on the self-estimate of a predictor yields a witness to a failure of multicalibration, and vice versa. This has the implication that nontrivial loss prediction is in effect no easier or harder than auditing for multicalibration. We support our theoretical results with experiments that show a robust positive correlation between the multicalibration error of a predictor and the efficacy of training a loss predictor. 
\end{abstract}


\section{Introduction}


\begin{figure}[t]
\centering
\includegraphics[width=0.6\columnwidth]{figures/evaluation_desiderata_V5.pdf}
\vspace{-0.5cm}
\caption{\systemName is a platform for conducting realistic evaluations of code LLMs, collecting human preferences of coding models with real users, real tasks, and in realistic environments, aimed at addressing the limitations of existing evaluations.
}
\label{fig:motivation}
\end{figure}

\begin{figure*}[t]
\centering
\includegraphics[width=\textwidth]{figures/system_design_v2.png}
\caption{We introduce \systemName, a VSCode extension to collect human preferences of code directly in a developer's IDE. \systemName enables developers to use code completions from various models. The system comprises a) the interface in the user's IDE which presents paired completions to users (left), b) a sampling strategy that picks model pairs to reduce latency (right, top), and c) a prompting scheme that allows diverse LLMs to perform code completions with high fidelity.
Users can select between the top completion (green box) using \texttt{tab} or the bottom completion (blue box) using \texttt{shift+tab}.}
\label{fig:overview}
\end{figure*}

As model capabilities improve, large language models (LLMs) are increasingly integrated into user environments and workflows.
For example, software developers code with AI in integrated developer environments (IDEs)~\citep{peng2023impact}, doctors rely on notes generated through ambient listening~\citep{oberst2024science}, and lawyers consider case evidence identified by electronic discovery systems~\citep{yang2024beyond}.
Increasing deployment of models in productivity tools demands evaluation that more closely reflects real-world circumstances~\citep{hutchinson2022evaluation, saxon2024benchmarks, kapoor2024ai}.
While newer benchmarks and live platforms incorporate human feedback to capture real-world usage, they almost exclusively focus on evaluating LLMs in chat conversations~\citep{zheng2023judging,dubois2023alpacafarm,chiang2024chatbot, kirk2024the}.
Model evaluation must move beyond chat-based interactions and into specialized user environments.



 

In this work, we focus on evaluating LLM-based coding assistants. 
Despite the popularity of these tools---millions of developers use Github Copilot~\citep{Copilot}---existing
evaluations of the coding capabilities of new models exhibit multiple limitations (Figure~\ref{fig:motivation}, bottom).
Traditional ML benchmarks evaluate LLM capabilities by measuring how well a model can complete static, interview-style coding tasks~\citep{chen2021evaluating,austin2021program,jain2024livecodebench, white2024livebench} and lack \emph{real users}. 
User studies recruit real users to evaluate the effectiveness of LLMs as coding assistants, but are often limited to simple programming tasks as opposed to \emph{real tasks}~\citep{vaithilingam2022expectation,ross2023programmer, mozannar2024realhumaneval}.
Recent efforts to collect human feedback such as Chatbot Arena~\citep{chiang2024chatbot} are still removed from a \emph{realistic environment}, resulting in users and data that deviate from typical software development processes.
We introduce \systemName to address these limitations (Figure~\ref{fig:motivation}, top), and we describe our three main contributions below.


\textbf{We deploy \systemName in-the-wild to collect human preferences on code.} 
\systemName is a Visual Studio Code extension, collecting preferences directly in a developer's IDE within their actual workflow (Figure~\ref{fig:overview}).
\systemName provides developers with code completions, akin to the type of support provided by Github Copilot~\citep{Copilot}. 
Over the past 3 months, \systemName has served over~\completions suggestions from 10 state-of-the-art LLMs, 
gathering \sampleCount~votes from \userCount~users.
To collect user preferences,
\systemName presents a novel interface that shows users paired code completions from two different LLMs, which are determined based on a sampling strategy that aims to 
mitigate latency while preserving coverage across model comparisons.
Additionally, we devise a prompting scheme that allows a diverse set of models to perform code completions with high fidelity.
See Section~\ref{sec:system} and Section~\ref{sec:deployment} for details about system design and deployment respectively.



\textbf{We construct a leaderboard of user preferences and find notable differences from existing static benchmarks and human preference leaderboards.}
In general, we observe that smaller models seem to overperform in static benchmarks compared to our leaderboard, while performance among larger models is mixed (Section~\ref{sec:leaderboard_calculation}).
We attribute these differences to the fact that \systemName is exposed to users and tasks that differ drastically from code evaluations in the past. 
Our data spans 103 programming languages and 24 natural languages as well as a variety of real-world applications and code structures, while static benchmarks tend to focus on a specific programming and natural language and task (e.g. coding competition problems).
Additionally, while all of \systemName interactions contain code contexts and the majority involve infilling tasks, a much smaller fraction of Chatbot Arena's coding tasks contain code context, with infilling tasks appearing even more rarely. 
We analyze our data in depth in Section~\ref{subsec:comparison}.



\textbf{We derive new insights into user preferences of code by analyzing \systemName's diverse and distinct data distribution.}
We compare user preferences across different stratifications of input data (e.g., common versus rare languages) and observe which affect observed preferences most (Section~\ref{sec:analysis}).
For example, while user preferences stay relatively consistent across various programming languages, they differ drastically between different task categories (e.g. frontend/backend versus algorithm design).
We also observe variations in user preference due to different features related to code structure 
(e.g., context length and completion patterns).
We open-source \systemName and release a curated subset of code contexts.
Altogether, our results highlight the necessity of model evaluation in realistic and domain-specific settings.





\section{Loss prediction}
\label{sec:lp}

We consider binary classification, with a distribution $\mD$ on  $\X \times \zo$. 
A predictor is a function $p:\X \to [0,1]$. The Bayes optimal predictor is defined as $p^*(x) = \ex[y|x]$. Given $p$, we define the simulated distribution $\mD(p)$ on $\X \times \zo$ where $x$ is drawn as in $\mD$, and $y|x \sim \Ber(p(x))$. Let $\ell: \zo \times [0,1] \to [0,1]$ be a proper loss function.\footnote{ The case of general losses reduces to the proper loss case; please see \cref{sec:non-proper} for details. We also assume for technical convenience that the loss is bounded. Losses that are not strictly bounded, such as cross entropy, can be handled with some further care and constraints on predicted probabilities.} We will use the following characterization of proper losses.

\begin{lemma}[\cite{gneiting2007strictly}] For every proper loss $\ell$,  there exists a concave function $H_{\ell}: [0, 1] \rightarrow \R$ so that
\[ \ell(y, v) = H_{\ell}(v) + (y - v)H_{\ell}'(v).\]
where $H_{\ell}'(v)$ is a ``superderivative'' of $H_{\ell}$, i.e. for any $v, w \in [0, 1]$, $H_{\ell}(v) \leq H_{\ell}(w) + (v - w)H_{\ell}'(w)$. 
\end{lemma}
When $H_{\ell}(v)$ is differentiable at all $v \in [0, 1]$, the superderivative is unique, and is just the derivative.  From the definition it follows that 
\begin{align*} 
H_{\ell}(v) &= \ex_{y \sim \Ber(v)}[\ell(y,v)] \in [0,1]\\
H_{\ell}'(v) &= \ell(1, v) - \ell(0,v) \in [-1,1]
\end{align*}
Let $L(p^*; p) = \ex_{y \sim \Ber(p^*)}[\ell(y, p)]$ denote the expected loss when $y \sim \Ber(p^*)$ but we predict $p$. Then
\begin{align}
\label{eq:exp-loss}
 L(p^*;p) = H_{\ell}(p) + (p^* - p)H_{\ell}'(p) \geq H_{\ell}(p^*) = L(p^*; p^*) 
\end{align}
where the inequality follows from the concavity of $H_{\ell}$, and is equivalent to the loss $\ell$ being proper.

We now define the notion of a loss predictor. 

\begin{definition}[Loss predictor]
\label{def:lp}
    Let $p$ be a predictor and $\ell$ be a proper loss. Let $\Phi$ be an abstract feature space, which we will make concrete shortly. 
    A \emph{loss predictor} is a function $\lossPred: \Phi \to \R$, which takes as input some features $\phi(p,x) \in \Phi$ related to a point $x$ and its prediction using $p$, and returns an estimate $\lossPred(\phi(p,x))$ of the expected loss $\ex[\ell(y, p(x))|x]$ suffered by $p$ at the point $x$. We define a hierarchy of loss predictors of increasing strength, depending on the information contained in $\phi(p,x)$:
    \begin{enumerate}
        \item \emph{Prediction-only loss predictors} only have access to $p$'s prediction at a point $x$, i.e. $\phi(p, x) = p(x)$. The most natural choice for a prediction-only loss predictor is given by the self-entropy predictor, which we will define in Definition~\ref{def:can-loss-pred}.
        \item \emph{Input-aware loss predictors} have access to the input features $\inp(x)$ used to train the model $p$, as well as its prediction, i.e. $\phi(p, x) = (\inp(x), p(x))$. 
        \item \emph{Representation-aware loss predictors} have access to $\phi(p, x) = ( p(x), \inp(x), r(x))$, where $r(x)$ is some representation of $x$. We distinguish between two kinds of representations:
        \begin{itemize}
            \item Internal representations: The representation $r(x)= r_p(x)$ consists of features that are explicitly computed by the predictor $p$ in the course of computing $p(x)$. For instance, they could consist of the embedding of $x$ produced by the last few layers of a DNN.  
            \item External representations: The representation $r(x) = r_e(x)$ consists of features that are not explicitly computed by the  predictor $p$. For instance, they could be the representation of $x$ obtained from a different model, or by consulting human experts.         
        \end{itemize}
     \end{enumerate}
\end{definition}

A few comments on the definition: 

\begin{itemize}
\item Two-headed architectures that simultaneously train both the predictor and the loss-predictor (such as \cite{yoo2019learning,kirillov2023segment}) are a class of internal representation-aware predictors. In contrast, loss-predictors that use an embedding produced by a foundation model (such as \cite{jain2022distilling}, which audits the errors of the predictor) are external representation-aware.  

\item   If we allow the loss predictor to be significantly more complex than the predictor $p$, then it could compute $r_p(x)$ from $\inp(x)$ using the model $p$. So the gap between input-aware and representation-aware loss predictors diminishes as the loss-predictor becomes more computationally powerful. But in the (important) setting where the loss predictor is less computationally powerful than the predictor, there could be a gap.

\item In contrast, external representations might contain auxiliary information  that cannot be computed using $\inp(x)$, regardless of the computational power of the loss predictor. 
\end{itemize}

The loss predictor can be trained using standard regression, given access to a training set of points $(\phi(p, x), y)$ where $(x, y)$ are drawn from the distribution $\mD$.  One can measure the performance of our loss predictor as we would with any regression problem. We formulate it using the squared loss, as $\ex[(\ell(y, p(x)) - \lossPred(\phi(p, x))^2]$. It follows from Equation \eqref{eq:exp-loss} that the Bayes optimal loss predictor is given by
$\lossPred^*(x) = L(p^*(x); p(x))$. But computing this requires knowing the Bayes optimal predictor $p^*$, and is likely to be infeasible in most settings. Rather, we will compare our loss predictor to a canonical baseline which we describe next.

\paragraph{The self-entropy predictor.}

Following \cite{OI}, given a predictor $p$, we define the simulated distribution $\mD(p)$  on pairs $(x, \ty) \in \X \times \zo$,  where $x \sim \mD$ and $\ex[\ty|x] = p(x)$. The predictor hypothesizes that this how labels are being generated. Hence for each $x \in \X$, the self-entropy predictor predicts the expected loss according to this distribution. 

\begin{definition}[Self-entropy predictor]\label{def:can-loss-pred}
    Given a proper loss $\ell$ and predictor $\pred$,  the \emph{self-entropy predictor} is the prediction-only loss predictor $\clp: [0, 1] \to \R$  that predicts the expected loss when $\ty \sim \Ber(\pred(x))$ at each $x$; that is
    \[\clp(\pred(x)) = \ex_{\ty \sim \Ber(p(x))}[\ell(\ty, p(x))] = H_{\ell}(\pred(x)).\]
\end{definition}


We use the self-entropy predictor as our baseline. Hence the question is when can we learn a loss predictor with significantly lower squared loss than the self-entropy predictor. We formalize this using the notion of advantage of a loss predictor over the self-entropy predictor.

\begin{definition}[Advantage of a loss predictor]\label{def:lp-advantage}
    Define the advantage of a loss predictor $\lossPred$ over the self-entropy predictor to be the difference in the squared error
    \[ \adv(\lossPred) = \ex[(\ell(y, p(x)) - \clp(p(x)))^2] - \ex[(\ell(y, p(x)) - \lossPred(\phi(p, x))^2]. \]
\end{definition}%

We want loss predictors whose advantage is positive and as large as possible.
Our goal is understand under what conditions we can hope to learn such a predictor. 

\paragraph{On non-proper losses.} So far we have assumed that we a trying to predict the proper loss incurred by a predictor. We can generalize this to a setting where we have a hypothesis $h:\X \to \mA$ (for instance $h$ might be a binary classifier), and a loss function $\ell:\zo \times \mA \to \R$. It turns out that our theory extends seamlessly to the non-proper setting, under rather mild assumptions on the hypothesis $h$. We present this extension in Appendix \ref{sec:non-proper}.





\section{Multicalibration}
\label{sec:mc}

Having defined our notion of a loss predictor, we next introduce the framework of multicalibration proposed by~\cite{hebert2018multicalibration}. Our definition is most similar to the presentation used in~\cite{kim2022universal}.

\begin{definition}[Multicalibration]
\label{def:mc}
    Let $\phi(p, x) \in \Phi$ be some auxiliary set of features related to the computation of $p(x)$, which we define concretely below. Let $\calC$ be a class of weight functions $c: \Phi \rightarrow [-1, 1]$, and $p: \calX \rightarrow [0, 1]$ a binary predictor for a target distribution $\calD$ over $\calX \times \{0, 1\}$. Then, the multicalibration error of $p$ with respect to $\calC$ is defined as 
    \[\MCE(\calC, p) := \max_{c \in \calC} \left|\ex_{x, y \sim \calD}[(y - p(x))c(\phi(p, x))]\right|.\]
    The information contained in $\phi(p, x)$ gives rise to a hierarchy of multicalibration notions of increasing strength:
    \begin{enumerate}
        \item \emph{Calibration} corresponds to the setting where  $\phi(p, x)= p(x)$, and test functions can only depend on $p$'s prediction. 
        \item \emph{Multicalibration} corresponds to the case where test functions can additionally depend on the input features, i.e. $\phi(p, x) = (p(x), \inp(x))$.
        \item \emph{Representation-aware multicalibration} is a strengthening of multicalibration where test functions can additionally depend on some representation $r(x)$ of $x$ i.e., $\phi(p, x) = (p(x), \inp(x), r(x))$. We distinguish between internal representations $r_p(x)$ and external representations $r_e(x)$ as with loss predictors (Definition \ref{def:lp}).
    \end{enumerate}
\end{definition}

The first two levels in this hierarchy, calibration and multicalibration, have been extensively studied in previous works. 
In standard multicalibration, we require that a predictor $p(x)$ be well-calibrated under a broad class of test functions, $\calC$, that depend only on $\inp(x)$ and $p(x)$. The literature on multicalibration typically identifies $\inp(x)$ with $x$ itself. The last level of the hierarchy, representation-aware multicalibration, is a strengthening of multicalibration that naturally extends the multicalibration framework of ~\cite{hebert2018multicalibration}. As in the case of loss-predictors, the gap between internal representations $r_p(x)$ and $\inp(x)$ is computational; whereas the gap between external representations $r(x)$ and $\inp(x)$ could be information-theoretic. 








\begin{definition}[Multicalibration violation witness]
    We say that a function $c: \Phi \times [0,1] \rightarrow [-1, 1]$ is a witness for a multicalibration violation of magnitude $\alpha$ for a predictor $p$ if 
    \[\left|\ex_{x, y \sim \calD}[(y - p(x))c(\phi(p, x))]\right| > \alpha.\]
\end{definition}

\cite{hebert2018multicalibration} showed that if we find such a witness, we can use it to improve the predictor $p$ in a way that reduces the squared loss. While their argument is stated for the input-aware setting where $\phi(p,x) = (p(x), \inp(x))$, it applies to the representation-aware setting as well. 





    



\section{Loss prediction advantage and multicalibration auditing}
\label{sec:lp-mc}

In this section, we establish the relationship between learning loss predictors with good advantage, and auditing for multicalibration, i.e. finding a $c$ that witnesses a large multicalibration violation. The main result of our section is the following theorem, which establishes the correspondence between various levels of loss predictors and multicalibration requirements, when instantiated with the appropriate values $\phi(p, x)$:

\begin{theorem}\label{thm:mc-conv-vs-loss-pred}
    Let $\calF$ be a class of loss predictors $f: \Phi \rightarrow [0, 1]$. 
    Let $\calF' \supseteq \calF$ be the augmented function class defined as 
    \[\calF' = \{\Pi_{[0,1]}((1 - \beta)H_{\ell}(p(x)) + \beta f(\phi(p,x))) : \beta \in [-1, 1], f \in \calF\}.\]
    Let $\calC$ be a class of weight functions defined as 
    \[\calC = \{ (f(\phi(p, x)) - H_{\ell}(p(x)))H_{\ell}'(p(x)) : f \in \calF\}.\]
    Then, 
    \[\frac{1}{2}\max_{\lossPred \in \calF} \adv(\lossPred) \leq \MCE(\calC, p) \leq \sqrt{\max_{\lossPred \in \calF'} \adv(\lossPred)}.\]
\end{theorem}

The proof of Theorem~\ref{thm:mc-conv-vs-loss-pred} can be found in Appendix~\ref{sec:mc-conv-cs-loss-pred-pf} and follows from two key lemmas. Lemma~\ref{lem:adv-implies-mc-err} establishes the left-hand inequality by showing how a loss predictor with good advantage can be used to construct a witness of large multicalibration error. Conversely, Lemma~\ref{lem:mc-err-implies-adv} establishes the right-hand inequality by showing how to leverage a witness for large multicalibration error to construct a loss predictor with large advantage. 

Before presenting our main lemmas, we introduce two auxiliary claims that are well-known in the literature on boosting and gradient descent. We provide proofs here for completeness and notational consistency.


Let $\mD'$ be a distribution over $(x, z) \in X \times [0,1]$. Let $h_1, h_2: \X \to [0,1]$ be two hypotheses. Under what conditions does $h_2$ improve on $h_1$? The following lemma gives a necessary condition: the update $\delta(x) = h_2(x) - h_1(x)$ must be correlated with the residual errors $z - h_1(x)$ of the hypothesis $h_1$ under the distribution $\mD'$. 

 
\begin{claim}
\label{lem:improve-1}
    For two hypotheses $h_1, h_2$, 
    \[ \E_{\mD'}[(h_1(x) - z)^2] - \E[(h_2(x) - z)^2] \leq 2\E[(h_2(x) - h_1(x))(z - h_1(x))]. \]
\end{claim}
\begin{proof}
    Let us write $\delta(x) = h_2(x) - h_1(x)$. Then we have
    \begin{align*}
        \E[(h_1(x) - z)^2] - \E[(h_2(x) - z)^2] &= \E_{\mD'}[(h_1(x) -z)^2 - (h_1(x)  -z + \delta(x) )^2]\\
        &= -2\E[(h_1(x) - z)\delta(x)] - \E[\delta(x)^2]\\
        & \leq 2\E_{\mD}[(z - h_1(x))\delta(x)].
    \end{align*}
\end{proof}

Conversely, if we find an update $\delta(x)$ which is correlated with the residuals, we can perform a gradient descent update to reduce the squared error. We let $\Pi_{[0,1]}:\R \to [0,1]$ denote the projection operator onto the unit interval.

\begin{claim}
\label{lem:improve-2}    
If there exists $\delta:\X \to [-1,1]$ such that $\E_{\mD'}[\delta(x)(z - h_1(x))] \geq \beta \geq 0$, then setting $h_2(x) = \Pi_{[0,1]}(h_1(x)+ \beta \delta(x))$ gives
    \[ \E_{\mD'}[(h_1(x) - z)^2] - \E[(h_2(x) - z)^2] \geq \beta^2.\]
\end{claim}

\begin{proof}
    Without projection, we can write the gap in squared error as 
    \begin{align*}
        \E_{\mD'}[(h_1(x) - z)^2] - \E[(h_1(x) + \beta\delta(x) - z)^2] &=  2\beta \E_{\mD'}[(z - h_1(x))\delta(x)] - \beta^2\E[\delta(x)^2]\\
        &\geq 2\beta^2 - \beta^2 = \beta^2.
    \end{align*}
    While $h_1(x) + \beta \delta(x)$ may not be bounded in $[0,1]$, projection onto the interval can only further reduce the squared error.
\end{proof}

With these in hand, we show that any loss predictor with a non-trivial advantage points us to a failure of multicalibration. 

\begin{lemma}\label{lem:adv-implies-mc-err}
    Assume that $\lossPred$ achieves advantage $\alpha \geq 0$ over the self-entropy predictor. Then the function $\delta(\phi(p, x)) = \lossPred(\phi(p, x)) - \clp(p(x))$ satisfies
    \[ \E[\delta(\phi(p, x))H_{\ell}'(p(x))(y - p(x))] \geq \alpha/2.\]
    In other words, $\lossPred$ can be used to construct a witness $c(\phi(p, x)) = \delta(\phi(p, x))H_{\ell}'(p(x))$ for a multicalibration violation of magnitude $\alpha/2$. 
\end{lemma}
\begin{proof}
    Consider the loss regression problem, where we draw $(x,y) \in \X \times \zo \sim \mD$ and then return the pair $(x, z = \ell(y, p(x))$. We will use Claim \ref{lem:improve-1}, where we take $h_1 = \clp$ to be the self-entropy predictor and $h_2 = \lossPred$. 
    We can estimate the residual error of the self-entropy predictor as
    \begin{align}
    \label{eq:rewrite}
        \ell(y, p(x)) - \clp(p(x)) &= H_{\ell}(p(x)) + (y - p(x))H_{\ell}'(p(x)) - H_{\ell}(p(x))\notag\\
        &= (y - p(x))H_{\ell}'(p(x)).
    \end{align}
    By Claim \ref{lem:improve-1}, we have
    \begin{align*}
        \alpha &= \ex[(\ell(y, p(x)) - \clp(p(x)))^2] - \ex[(\ell(y, p(x)) - \lossPred(\phi(p, x))^2]\\
        &\leq 2\E[(\lossPred(\phi(p, x)) - \clp(p(x)))(\ell(y, p(x)) - \clp(p(x)))\\ &= 2\E[\delta(\phi(p, x))H_{\ell}'(p(x))(y - p(x))]
    \end{align*}
\end{proof}

Conversely to the result of Lemma~\ref{lem:adv-implies-mc-err}, we show that we can leverage certain types of multicalibration failures to predict loss with an advantage over the self-entropy predictor.

\begin{lemma}\label{lem:mc-err-implies-adv}
    Assume there exists a function $\delta:\Phi \to [-1,1]$ such that
    \[ \E[\delta(\phi(p, x))H_{\ell}'(p(x))(y - p(x))] \geq \beta \geq 0.\]
    i.e., the function $c(\phi(p, x)) = \delta(\phi(p, x))H_{\ell}'(p(x))$ is a witness for a multicalibration violation of magnitude $\beta$. Define the loss predictor
    \[ \lossPred(\phi(p, x)) = \Pi_{[0,1]}(\clp(p(x)) + \beta 
    \delta(\phi(p, x))).\] 
    Then $\adv(\lossPred) \geq \beta^2$. 
\end{lemma}
\begin{proof}
    We again consider the loss regression problem, 
    We now apply Lemma \ref{lem:improve-2} with $z = \ell(y, p(x))$, $h_1 = \clp$. The correlation condition we require is
    \[  \E[\delta(\phi(p, x))(\ell(y, p(x) - \clp(p(x)))] \geq \beta, \]
    By Equation \eqref{eq:rewrite}, we have
    \[ \E[\delta(\phi(p, x))(\ell(y, p(x)) - \clp(p(x)))] = \E[\delta(\phi(p, x))H_{\ell}'(p(x))(y - p(x))] \]
    which is at least $\beta$ by our assumption. Hence Claim \ref{lem:improve-2} implies that $h_2 = \lossPred$ has advantage $\beta^2$ over $\clp$.     
\end{proof}


\section{Loss prediction for multiple losses}
\label{sec:multiple-loss}

Up to this point, our discussion has focused on loss prediction for a single, predetermined loss function. However, in real-world applications, multiple stakeholders may use a predictor, each with unique objectives and priorities that correspond to different loss functions. This scenario would require training separate loss predictors for each user to meet their individual needs.

The self-entropy predictor offers a key advantage: it can be computed for any loss function using only the predictions $p(x)$, eliminating the need for additional training. Moreover, by extending the result of Theorem~\ref{thm:mc-conv-vs-loss-pred}, we can define a class test functions $\calC$ such that when $p$ is multicalibrated with respect to $\calC$, its self-entropy predictions simultaneously compete with the best-in-class loss predictor for each loss in a rich class of losses $\calL$, rather than just a fixed loss. We formalize this in the following lemma, which we prove in Appendix~\ref{sec:many-losses-mc-pf}:

\begin{lemma}\label{lem:many-losses-mc}
    Let $\calF$ be a class of loss predictors $f: \Phi \rightarrow [0, 1]$. Let $\calL$ be a class of bounded proper losses $\ell: \{0, 1\} \times [0, 1] \rightarrow [0, 1]$ with associated concave entropy functions $H_{\ell}: [0, 1] \rightarrow [0,1]$, and let $\calC_{\calL}$ be the class of test functions 
    \[\calC_{\calL} = \{(f(\phi(p, x)) - H_{\ell}(p(x)))H'_{\ell}(p(x)) : f \in \calF, \ell \in \calL\}.\]
    Then,
    \[\max_{\ell \in \calL}\max_{\lossPred \in \calF} \adv(\lossPred) \leq 2\MCE(\calC_{\calL}, p).\]
    I.e., no loss predictor from $\calF$ for any loss $\ell \in \calL$ can obtain better advantage than $2\MCE(\calC_{\calL}, p)$ over the self-entropy predictor. 
\end{lemma}

When $\calL$ is the set of all proper losses, the form of multicalibration imposed by $\calC_{\calL}$ can be thought of as the extension to multicalibration of the notion of \emph{proper calibration}, recently proposed by~\cite{OKK25}. The proper calibration error of a predictor $p$ is defined as 

\[\text{PCE}(p) = \max_{\ell \in \calL_{\text{prop}}}\left|\ex[H'_{\ell}(p(x))(y - p(x))]\right|\]
where $\calL_{\text{prop}}$ denotes the set of proper losses. Our condition can be thought of as ``proper multicalibration'' where each test function consists of $H_{\ell}'(p(x))$ multiplied with an additional test function $\delta(\phi(p, x))$, that may depend on other features in addition to the prediction value. 

\subsection{Achieving efficient multicalibration for many losses}

As the class of losses we consider expands, training an effective loss predictor for each individual loss becomes increasingly challenging. This section demonstrates that in certain scenarios, it is possible to efficiently produce a multicalibrated predictor with respect to the class of tests outlined in Lemma~\ref{lem:many-losses-mc}, even for some infinite classes of losses. This approach allows us to learn a single predictor $p$ whose self-entropy estimates can compete with the best $\lossPred \in \calF$ for every 
$\ell \in \calL$, thus eliminating the need to train separate predictors for each loss.

This result relies on the existence of a ``finite approximate basis'' (Definition~\ref{def:finite-approx-basis}) for the class of functions $\{H_{\ell}'\}_{\ell \in \calL}$, and is inspired by the techniques of~\cite{OKK25}, who use a similar approach to show the efficiency of proper calibration when $\{H_{\ell}'\}_{\ell \in \calL}$ has a finite approximate basis. 

We show a general version of this result in Theorem~\ref{thm:general-approx-basis-mc}, and instantiate it here for the class of 1-Lipschitz proper losses, $\calL_{Lip}$. 

The instantiation relies on a result proved by \cite{OKK25}, who show that $\{H_{\ell}'\}_{\ell \in \calL_{Lip}}$ has such a finite basis. We show efficiency in terms of oracle access to a weak-agnostic-learner for $\calF$, the class of loss predictors. We motivate this assumption by observing that if we care about learning a loss predictor from the class $\calF$, it's reasonable to assume that we have access to a weak agnostic learner for $\calF$. We formally define a weak agnostic learner as follows. 

\begin{definition}[Weak agnostic learner]\label{def:weak-agnostic-learner}
    Let $\alpha \geq 0$, $\delta \geq 0$. An $\alpha$-weak agnostic learner for $\calF \subseteq \{f: \Phi \rightarrow [-1, 1]\}$, closed under negation, with sample complexity $n$ and failure parameter $\delta$ is an algorithm that when given $n$ samples from a distribution $\calU$ over $\Phi \times [-1, 1]$, outputs $f \in \calF \cup \{\bot\}$ such that with probability at least $1 - \delta$ over the samples from $\calU$ and the randomness in the algorithm itself, if $\max_{f \in \calF} \ex_{(\phi, z) \sim \calU}[f(\phi)z] \geq \alpha,$
    the algorithm returns a $f \in \calF$ such that 
    $\ex_{(\phi, z) \sim \calU}[f(\phi)z] \geq \alpha/2.$
    Otherwise, if for all $f \in \calF$, 
    $\ex_{(\phi, z) \sim \calU}[f(\phi)z] \leq \alpha,$ the algorithm either returns $f = \bot$ or $f \in \calF$ such that $\ex_{(\phi, z) \sim \calU}[f(\phi)z] \geq \alpha/2.$
\end{definition}

With this definition in hand, we are ready to present the main theorem of this section. The proof can be found in Appendix~\ref{sec:1-lip-mc-pred-pf}.

\begin{theorem}\label{thm:1-lip-mc-pred}
    Fix $\delta, \epsilon > 0$. Let $\calL_{Lip}$ be the class of proper 1-Lipschitz losses $\ell:\{0, 1\} \times [0, 1] \rightarrow [0, 1]$, and let $\calF$ be a class of loss predictors $\calF: \Phi \rightarrow [-1, 1]$ that is closed under negation and contains the class of self entropy predictors, $\calH_{\calL_{Lip}} = \{H_{\ell}\}_{\ell \in \calL_{Lip}}$. Further assume that we have access to an $\alpha$-weak-agnostic-learner for $\calF$ with sample complexity $n$ and failure parameter $\beta \leq \frac{\alpha^2\delta}{4\lceil 2/\epsilon + 1\rceil}$. 

    Then, there exists an algorithm that, given $m = O(n/\alpha^2)$ samples, with probability at least $1 - \delta$ outputs a predictor $p$ such that 
    \[\max_{\ell \in \calL_{Lip}} \max_{\lossPred \in \calF} \adv(\lossPred) \leq 16\alpha + 4\epsilon.\]
\end{theorem}

In other words, our learned $p$'s self-entropy predictions compete with the best-in-class loss predictor with \emph{every} $\ell \in \calL_{Lip}$, up to an error of $16\alpha + 4\epsilon$. 








\section{Experiments}
\label{sec:experiments}
The experiments are designed to address two key research questions.
First, \textbf{RQ1} evaluates whether the average $L_2$-norm of the counterfactual perturbation vectors ($\overline{||\perturb||}$) decreases as the model overfits the data, thereby providing further empirical validation for our hypothesis.
Second, \textbf{RQ2} evaluates the ability of the proposed counterfactual regularized loss, as defined in (\ref{eq:regularized_loss2}), to mitigate overfitting when compared to existing regularization techniques.

% The experiments are designed to address three key research questions. First, \textbf{RQ1} investigates whether the mean perturbation vector norm decreases as the model overfits the data, aiming to further validate our intuition. Second, \textbf{RQ2} explores whether the mean perturbation vector norm can be effectively leveraged as a regularization term during training, offering insights into its potential role in mitigating overfitting. Finally, \textbf{RQ3} examines whether our counterfactual regularizer enables the model to achieve superior performance compared to existing regularization methods, thus highlighting its practical advantage.

\subsection{Experimental Setup}
\textbf{\textit{Datasets, Models, and Tasks.}}
The experiments are conducted on three datasets: \textit{Water Potability}~\cite{kadiwal2020waterpotability}, \textit{Phomene}~\cite{phomene}, and \textit{CIFAR-10}~\cite{krizhevsky2009learning}. For \textit{Water Potability} and \textit{Phomene}, we randomly select $80\%$ of the samples for the training set, and the remaining $20\%$ for the test set, \textit{CIFAR-10} comes already split. Furthermore, we consider the following models: Logistic Regression, Multi-Layer Perceptron (MLP) with 100 and 30 neurons on each hidden layer, and PreactResNet-18~\cite{he2016cvecvv} as a Convolutional Neural Network (CNN) architecture.
We focus on binary classification tasks and leave the extension to multiclass scenarios for future work. However, for datasets that are inherently multiclass, we transform the problem into a binary classification task by selecting two classes, aligning with our assumption.

\smallskip
\noindent\textbf{\textit{Evaluation Measures.}} To characterize the degree of overfitting, we use the test loss, as it serves as a reliable indicator of the model's generalization capability to unseen data. Additionally, we evaluate the predictive performance of each model using the test accuracy.

\smallskip
\noindent\textbf{\textit{Baselines.}} We compare CF-Reg with the following regularization techniques: L1 (``Lasso''), L2 (``Ridge''), and Dropout.

\smallskip
\noindent\textbf{\textit{Configurations.}}
For each model, we adopt specific configurations as follows.
\begin{itemize}
\item \textit{Logistic Regression:} To induce overfitting in the model, we artificially increase the dimensionality of the data beyond the number of training samples by applying a polynomial feature expansion. This approach ensures that the model has enough capacity to overfit the training data, allowing us to analyze the impact of our counterfactual regularizer. The degree of the polynomial is chosen as the smallest degree that makes the number of features greater than the number of data.
\item \textit{Neural Networks (MLP and CNN):} To take advantage of the closed-form solution for computing the optimal perturbation vector as defined in (\ref{eq:opt-delta}), we use a local linear approximation of the neural network models. Hence, given an instance $\inst_i$, we consider the (optimal) counterfactual not with respect to $\model$ but with respect to:
\begin{equation}
\label{eq:taylor}
    \model^{lin}(\inst) = \model(\inst_i) + \nabla_{\inst}\model(\inst_i)(\inst - \inst_i),
\end{equation}
where $\model^{lin}$ represents the first-order Taylor approximation of $\model$ at $\inst_i$.
Note that this step is unnecessary for Logistic Regression, as it is inherently a linear model.
\end{itemize}

\smallskip
\noindent \textbf{\textit{Implementation Details.}} We run all experiments on a machine equipped with an AMD Ryzen 9 7900 12-Core Processor and an NVIDIA GeForce RTX 4090 GPU. Our implementation is based on the PyTorch Lightning framework. We use stochastic gradient descent as the optimizer with a learning rate of $\eta = 0.001$ and no weight decay. We use a batch size of $128$. The training and test steps are conducted for $6000$ epochs on the \textit{Water Potability} and \textit{Phoneme} datasets, while for the \textit{CIFAR-10} dataset, they are performed for $200$ epochs.
Finally, the contribution $w_i^{\varepsilon}$ of each training point $\inst_i$ is uniformly set as $w_i^{\varepsilon} = 1~\forall i\in \{1,\ldots,m\}$.

The source code implementation for our experiments is available at the following GitHub repository: \url{https://anonymous.4open.science/r/COCE-80B4/README.md} 

\subsection{RQ1: Counterfactual Perturbation vs. Overfitting}
To address \textbf{RQ1}, we analyze the relationship between the test loss and the average $L_2$-norm of the counterfactual perturbation vectors ($\overline{||\perturb||}$) over training epochs.

In particular, Figure~\ref{fig:delta_loss_epochs} depicts the evolution of $\overline{||\perturb||}$ alongside the test loss for an MLP trained \textit{without} regularization on the \textit{Water Potability} dataset. 
\begin{figure}[ht]
    \centering
    \includegraphics[width=0.85\linewidth]{img/delta_loss_epochs.png}
    \caption{The average counterfactual perturbation vector $\overline{||\perturb||}$ (left $y$-axis) and the cross-entropy test loss (right $y$-axis) over training epochs ($x$-axis) for an MLP trained on the \textit{Water Potability} dataset \textit{without} regularization.}
    \label{fig:delta_loss_epochs}
\end{figure}

The plot shows a clear trend as the model starts to overfit the data (evidenced by an increase in test loss). 
Notably, $\overline{||\perturb||}$ begins to decrease, which aligns with the hypothesis that the average distance to the optimal counterfactual example gets smaller as the model's decision boundary becomes increasingly adherent to the training data.

It is worth noting that this trend is heavily influenced by the choice of the counterfactual generator model. In particular, the relationship between $\overline{||\perturb||}$ and the degree of overfitting may become even more pronounced when leveraging more accurate counterfactual generators. However, these models often come at the cost of higher computational complexity, and their exploration is left to future work.

Nonetheless, we expect that $\overline{||\perturb||}$ will eventually stabilize at a plateau, as the average $L_2$-norm of the optimal counterfactual perturbations cannot vanish to zero.

% Additionally, the choice of employing the score-based counterfactual explanation framework to generate counterfactuals was driven to promote computational efficiency.

% Future enhancements to the framework may involve adopting models capable of generating more precise counterfactuals. While such approaches may yield to performance improvements, they are likely to come at the cost of increased computational complexity.


\subsection{RQ2: Counterfactual Regularization Performance}
To answer \textbf{RQ2}, we evaluate the effectiveness of the proposed counterfactual regularization (CF-Reg) by comparing its performance against existing baselines: unregularized training loss (No-Reg), L1 regularization (L1-Reg), L2 regularization (L2-Reg), and Dropout.
Specifically, for each model and dataset combination, Table~\ref{tab:regularization_comparison} presents the mean value and standard deviation of test accuracy achieved by each method across 5 random initialization. 

The table illustrates that our regularization technique consistently delivers better results than existing methods across all evaluated scenarios, except for one case -- i.e., Logistic Regression on the \textit{Phomene} dataset. 
However, this setting exhibits an unusual pattern, as the highest model accuracy is achieved without any regularization. Even in this case, CF-Reg still surpasses other regularization baselines.

From the results above, we derive the following key insights. First, CF-Reg proves to be effective across various model types, ranging from simple linear models (Logistic Regression) to deep architectures like MLPs and CNNs, and across diverse datasets, including both tabular and image data. 
Second, CF-Reg's strong performance on the \textit{Water} dataset with Logistic Regression suggests that its benefits may be more pronounced when applied to simpler models. However, the unexpected outcome on the \textit{Phoneme} dataset calls for further investigation into this phenomenon.


\begin{table*}[h!]
    \centering
    \caption{Mean value and standard deviation of test accuracy across 5 random initializations for different model, dataset, and regularization method. The best results are highlighted in \textbf{bold}.}
    \label{tab:regularization_comparison}
    \begin{tabular}{|c|c|c|c|c|c|c|}
        \hline
        \textbf{Model} & \textbf{Dataset} & \textbf{No-Reg} & \textbf{L1-Reg} & \textbf{L2-Reg} & \textbf{Dropout} & \textbf{CF-Reg (ours)} \\ \hline
        Logistic Regression   & \textit{Water}   & $0.6595 \pm 0.0038$   & $0.6729 \pm 0.0056$   & $0.6756 \pm 0.0046$  & N/A    & $\mathbf{0.6918 \pm 0.0036}$                     \\ \hline
        MLP   & \textit{Water}   & $0.6756 \pm 0.0042$   & $0.6790 \pm 0.0058$   & $0.6790 \pm 0.0023$  & $0.6750 \pm 0.0036$    & $\mathbf{0.6802 \pm 0.0046}$                    \\ \hline
%        MLP   & \textit{Adult}   & $0.8404 \pm 0.0010$   & $\mathbf{0.8495 \pm 0.0007}$   & $0.8489 \pm 0.0014$  & $\mathbf{0.8495 \pm 0.0016}$     & $0.8449 \pm 0.0019$                    \\ \hline
        Logistic Regression   & \textit{Phomene}   & $\mathbf{0.8148 \pm 0.0020}$   & $0.8041 \pm 0.0028$   & $0.7835 \pm 0.0176$  & N/A    & $0.8098 \pm 0.0055$                     \\ \hline
        MLP   & \textit{Phomene}   & $0.8677 \pm 0.0033$   & $0.8374 \pm 0.0080$   & $0.8673 \pm 0.0045$  & $0.8672 \pm 0.0042$     & $\mathbf{0.8718 \pm 0.0040}$                    \\ \hline
        CNN   & \textit{CIFAR-10} & $0.6670 \pm 0.0233$   & $0.6229 \pm 0.0850$   & $0.7348 \pm 0.0365$   & N/A    & $\mathbf{0.7427 \pm 0.0571}$                     \\ \hline
    \end{tabular}
\end{table*}

\begin{table*}[htb!]
    \centering
    \caption{Hyperparameter configurations utilized for the generation of Table \ref{tab:regularization_comparison}. For our regularization the hyperparameters are reported as $\mathbf{\alpha/\beta}$.}
    \label{tab:performance_parameters}
    \begin{tabular}{|c|c|c|c|c|c|c|}
        \hline
        \textbf{Model} & \textbf{Dataset} & \textbf{No-Reg} & \textbf{L1-Reg} & \textbf{L2-Reg} & \textbf{Dropout} & \textbf{CF-Reg (ours)} \\ \hline
        Logistic Regression   & \textit{Water}   & N/A   & $0.0093$   & $0.6927$  & N/A    & $0.3791/1.0355$                     \\ \hline
        MLP   & \textit{Water}   & N/A   & $0.0007$   & $0.0022$  & $0.0002$    & $0.2567/1.9775$                    \\ \hline
        Logistic Regression   &
        \textit{Phomene}   & N/A   & $0.0097$   & $0.7979$  & N/A    & $0.0571/1.8516$                     \\ \hline
        MLP   & \textit{Phomene}   & N/A   & $0.0007$   & $4.24\cdot10^{-5}$  & $0.0015$    & $0.0516/2.2700$                    \\ \hline
       % MLP   & \textit{Adult}   & N/A   & $0.0018$   & $0.0018$  & $0.0601$     & $0.0764/2.2068$                    \\ \hline
        CNN   & \textit{CIFAR-10} & N/A   & $0.0050$   & $0.0864$ & N/A    & $0.3018/
        2.1502$                     \\ \hline
    \end{tabular}
\end{table*}

\begin{table*}[htb!]
    \centering
    \caption{Mean value and standard deviation of training time across 5 different runs. The reported time (in seconds) corresponds to the generation of each entry in Table \ref{tab:regularization_comparison}. Times are }
    \label{tab:times}
    \begin{tabular}{|c|c|c|c|c|c|c|}
        \hline
        \textbf{Model} & \textbf{Dataset} & \textbf{No-Reg} & \textbf{L1-Reg} & \textbf{L2-Reg} & \textbf{Dropout} & \textbf{CF-Reg (ours)} \\ \hline
        Logistic Regression   & \textit{Water}   & $222.98 \pm 1.07$   & $239.94 \pm 2.59$   & $241.60 \pm 1.88$  & N/A    & $251.50 \pm 1.93$                     \\ \hline
        MLP   & \textit{Water}   & $225.71 \pm 3.85$   & $250.13 \pm 4.44$   & $255.78 \pm 2.38$  & $237.83 \pm 3.45$    & $266.48 \pm 3.46$                    \\ \hline
        Logistic Regression   & \textit{Phomene}   & $266.39 \pm 0.82$ & $367.52 \pm 6.85$   & $361.69 \pm 4.04$  & N/A   & $310.48 \pm 0.76$                    \\ \hline
        MLP   &
        \textit{Phomene} & $335.62 \pm 1.77$   & $390.86 \pm 2.11$   & $393.96 \pm 1.95$ & $363.51 \pm 5.07$    & $403.14 \pm 1.92$                     \\ \hline
       % MLP   & \textit{Adult}   & N/A   & $0.0018$   & $0.0018$  & $0.0601$     & $0.0764/2.2068$                    \\ \hline
        CNN   & \textit{CIFAR-10} & $370.09 \pm 0.18$   & $395.71 \pm 0.55$   & $401.38 \pm 0.16$ & N/A    & $1287.8 \pm 0.26$                     \\ \hline
    \end{tabular}
\end{table*}

\subsection{Feasibility of our Method}
A crucial requirement for any regularization technique is that it should impose minimal impact on the overall training process.
In this respect, CF-Reg introduces an overhead that depends on the time required to find the optimal counterfactual example for each training instance. 
As such, the more sophisticated the counterfactual generator model probed during training the higher would be the time required. However, a more advanced counterfactual generator might provide a more effective regularization. We discuss this trade-off in more details in Section~\ref{sec:discussion}.

Table~\ref{tab:times} presents the average training time ($\pm$ standard deviation) for each model and dataset combination listed in Table~\ref{tab:regularization_comparison}.
We can observe that the higher accuracy achieved by CF-Reg using the score-based counterfactual generator comes with only minimal overhead. However, when applied to deep neural networks with many hidden layers, such as \textit{PreactResNet-18}, the forward derivative computation required for the linearization of the network introduces a more noticeable computational cost, explaining the longer training times in the table.

\subsection{Hyperparameter Sensitivity Analysis}
The proposed counterfactual regularization technique relies on two key hyperparameters: $\alpha$ and $\beta$. The former is intrinsic to the loss formulation defined in (\ref{eq:cf-train}), while the latter is closely tied to the choice of the score-based counterfactual explanation method used.

Figure~\ref{fig:test_alpha_beta} illustrates how the test accuracy of an MLP trained on the \textit{Water Potability} dataset changes for different combinations of $\alpha$ and $\beta$.

\begin{figure}[ht]
    \centering
    \includegraphics[width=0.85\linewidth]{img/test_acc_alpha_beta.png}
    \caption{The test accuracy of an MLP trained on the \textit{Water Potability} dataset, evaluated while varying the weight of our counterfactual regularizer ($\alpha$) for different values of $\beta$.}
    \label{fig:test_alpha_beta}
\end{figure}

We observe that, for a fixed $\beta$, increasing the weight of our counterfactual regularizer ($\alpha$) can slightly improve test accuracy until a sudden drop is noticed for $\alpha > 0.1$.
This behavior was expected, as the impact of our penalty, like any regularization term, can be disruptive if not properly controlled.

Moreover, this finding further demonstrates that our regularization method, CF-Reg, is inherently data-driven. Therefore, it requires specific fine-tuning based on the combination of the model and dataset at hand.
\putsec{related}{Related Work}

\noindent \textbf{Efficient Radiance Field Rendering.}
%
The introduction of Neural Radiance Fields (NeRF)~\cite{mil:sri20} has
generated significant interest in efficient 3D scene representation and
rendering for radiance fields.
%
Over the past years, there has been a large amount of research aimed at
accelerating NeRFs through algorithmic or software
optimizations~\cite{mul:eva22,fri:yu22,che:fun23,sun:sun22}, and the
development of hardware
accelerators~\cite{lee:cho23,li:li23,son:wen23,mub:kan23,fen:liu24}.
%
The state-of-the-art method, 3D Gaussian splatting~\cite{ker:kop23}, has
further fueled interest in accelerating radiance field
rendering~\cite{rad:ste24,lee:lee24,nie:stu24,lee:rho24,ham:mel24} as it
employs rasterization primitives that can be rendered much faster than NeRFs.
%
However, previous research focused on software graphics rendering on
programmable cores or building dedicated hardware accelerators. In contrast,
\name{} investigates the potential of efficient radiance field rendering while
utilizing fixed-function units in graphics hardware.
%
To our knowledge, this is the first work that assesses the performance
implications of rendering Gaussian-based radiance fields on the hardware
graphics pipeline with software and hardware optimizations.

%%%%%%%%%%%%%%%%%%%%%%%%%%%%%%%%%%%%%%%%%%%%%%%%%%%%%%%%%%%%%%%%%%%%%%%%%%
\myparagraph{Enhancing Graphics Rendering Hardware.}
%
The performance advantage of executing graphics rendering on either
programmable shader cores or fixed-function units varies depending on the
rendering methods and hardware designs.
%
Previous studies have explored the performance implication of graphics hardware
design by developing simulation infrastructures for graphics
workloads~\cite{bar:gon06,gub:aam19,tin:sax23,arn:par13}.
%
Additionally, several studies have aimed to improve the performance of
special-purpose hardware such as ray tracing units in graphics
hardware~\cite{cho:now23,liu:cha21} and proposed hardware accelerators for
graphics applications~\cite{lu:hua17,ram:gri09}.
%
In contrast to these works, which primarily evaluate traditional graphics
workloads, our work focuses on improving the performance of volume rendering
workloads, such as Gaussian splatting, which require blending a huge number of
fragments per pixel.

%%%%%%%%%%%%%%%%%%%%%%%%%%%%%%%%%%%%%%%%%%%%%%%%%%%%%%%%%%%%%%%%%%%%%%%%%%
%
In the context of multi-sample anti-aliasing, prior work proposed reducing the
amount of redundant shading by merging fragments from adjacent triangles in a
mesh at the quad granularity~\cite{fat:bou10}.
%
While both our work and quad-fragment merging (QFM)~\cite{fat:bou10} aim to
reduce operations by merging quads, our proposed technique differs from QFM in
many aspects.
%
Our method aims to blend \emph{overlapping primitives} along the depth
direction and applies to quads from any primitive. In contrast, QFM merges quad
fragments from small (e.g., pixel-sized) triangles that \emph{share} an edge
(i.e., \emph{connected}, \emph{non-overlapping} triangles).
%
As such, QFM is not applicable to the scenes consisting of a number of
unconnected transparent triangles, such as those in 3D Gaussian splatting.
%
In addition, our method computes the \emph{exact} color for each pixel by
offloading blending operations from ROPs to shader units, whereas QFM
\emph{approximates} pixel colors by using the color from one triangle when
multiple triangles are merged into a single quad.


\section{Acknowledgements}




\bibliographystyle{alpha}
\bibliography{ref}
\appendix
\section{Handling non-proper losses}
\label{sec:non-proper}

We consider an abstract action space $\mA$; examples are the discrete setting where $\mA = [k]$, and the continuous setting where $\mA = \R$. 
A hypothesis is a function $h: \X \to \mA$. 
A loss function is a function $\ell: \zo \times \mA \to [0,1]$. The expected loss of hypothesis $h$ at the point $x$ is given by $\ex[\ell(y, h(x))|x]$. The goal of a loss predictor is to learn a function $\lossPred: \Phi \to \R$ that gives pointwise estimates of this quantity. As in definition \ref{def:lp}, we can define a hierarchy of loss predictors based on the features available to them.


For any loss $\ell$, if the labels are drawn according to $y \sim \Ber(p)$ for any $p \in [0, 1]$, then the optimal prediction that minimizes the loss, $k_{\ell}(p) \in [0, 1]$ is defined by 
\[k_{\ell}(p) = \argmin_{v \in [0, 1]}\ex_{y \sim \Ber(\pbayes(x))}[\ell(y, v)]
\footnote{In the event that there is no unique minimum, we allow $k_\ell$ to output a subset of $\mA$, so it is strictly speaking a relation rather than a function. We blur this distinction for simplicity}.\]












If there exist a {\em latent} predictor $p_h:\X \to [0,1]$ so that $h = k_\ell \circ p_h$ is obtained by best-responding to its predictions, then we can reduce to the setting of proper losses, since 
\[ \ex[\ell(y, h(x))] = \ex[\ell(y, k_\ell(p(x)) = \ex [\ell \circ k_\ell(y, p(x))] \]
and we have the following result of \cite{kleinberg2023u}.

\begin{lemma}[\cite{kleinberg2023u}]
\label{lem:klst}
    For any loss $\ell:\zo \times \mA \to [0,1]$, the loss function $\ell \circ k_\ell: \zo \times [0,1] \to \R$ is a proper loss.
\end{lemma}

But under what conditions on $h$ does there exist such a predictor $p_h$? And is it easy to estimate its predictions given access to $h$? 
 
To answer the first question, we show that it is equivalent to assuming that the hypothesis satisfies a simple optimality condition for the loss.

\begin{definition}
    The hypothesis $h:\X \to [0,1]$ is swap-optimal for $\mD$ if for every function $\kappa:\mA \to \mA$, it holds that $\ex[\ell(y, h(x))] \leq \ex[\ell(y, \kappa(h(x)))]$.
\end{definition}
Swap optimality is a weak optimality condition that can be easily achieved by post-processing. It is quite reasonable to assume that a well-trained model optimized to minimize loss satisfies this guarantee. For instance, a well-trained image classifier should not improve if every time it predicts {\em cat}, we say {\em dog} instead. 
For a swap optimal hypothesis $h$, we show that is indeed easy to identify a latent predictor $p_h$ so that $h$ is obtained by best-responding to its predictions. This theorem lets us extend our theory of loss prediction for proper losses to arbitrary loss functions under the rather weak assumption that $h$ is swap-optimal. 

\begin{theorem}
\label{thm:swap-proper}
Given a hypothesis $h:\X \to \mA$ and a distribution $\mD$, define the predictor $p_h: \X \to [0,1]$ by $p_h(x) = \ex_\mD[y|h(x)]$. The hypothesis $h$ is swap optimal iff $h(x)= k_\ell \circ p_h(x)$ for all $x \in \X$.\footnote{Strictly speaking, $k_\ell$ is not a function as there can be many optimal actions. However we interpret this equation as saying $h(x)$ is in the set of optimal actions for $p_h(x)$.}
\end{theorem}
\begin{proof}
    Assume that $h$ is not swap-optimal, so there exist $\kappa$ such that $\ex[\ell(y, \kappa(h(x)))] < \ex[\ell(y, h(x))]$. There must exist a specific choice of $h(x) = a \in \mA$ conditioned on which the inequality still holds, hence 
    \[ \ex[\ell(y, \kappa(a))|h(x) = a]  \leq \ex[\ell(y, a)|h(x) =a]. \]
    Let $\ex[y|h(x) =a] = v$. But this shows that when $y \sim \Ber(v)$, $\ex[\ell(y, \kappa(a)] < \ex[\ell(y,a)]$, so $a \neq k_\ell(v)$. Hence for all $x \in h^{-1}(a)$, we have $h(x) = a \neq k_\ell(v) = k_\ell(p_h(x))$. 

    Conversely, if $h$ is indeed swap optimal, then it must be the case that every action $a \in \mA$ is a best response to $\E[y|h(x) =a] = p_h(x)$, which means we have $h(x) = k_\ell(p_h(x))$.
\end{proof}




\section{Proofs from Section~\ref{sec:lp-mc}}

\subsection{Proof of Theorem~\ref{thm:mc-conv-vs-loss-pred}}\label{sec:mc-conv-cs-loss-pred-pf}

\begin{proof}[Proof of Theorem~\ref{thm:mc-conv-vs-loss-pred}]
    The inequality on the left follows from Theorem~\ref{lem:adv-implies-mc-err}, while the inequality on the right follows from Lemma~\ref{lem:mc-err-implies-adv}. We prove each in turn, starting with the left-hand inequality. 

    By Theorem~\ref{lem:adv-implies-mc-err}, if there exists a $f \in \calF$ such that setting $\lossPred = f$ gives $\adv(\lossPred) = \alpha$, then this implies that 
    \[\ex[(\lossPred(\phi(p, x) - \clp(p(x)))H_{\ell}'(p(x))(y - p(x))] \geq \alpha/2.\]
    We observe that because $\lossPred = f \in \calF$, the witness of this multicalibration violation, $(\lossPred(\phi(p, x) - \clp(p(x)))H_{\ell}'(p(x))$ lies in $\calC$, and thus 
    \begin{align*}
        \MCE(\calC, p) &= \max_{c \in \calC}\left| \ex[c(\phi(p, x))(y - p(x))]\right|\\
        & \geq \ex[(\lossPred(\phi(p, x) - \clp(p(x)))H_{\ell}'(p(x))(y - p(x))]\\
        &\geq \alpha/2\\
        &= \frac{1}{2}\adv(\lossPred)
    \end{align*}

    The inequality follows by taking the maximum over all $\lossPred \in \calF$, as $\lossPred$ was chosen arbitrarily. 

    We now move on to proving the inequality on the right, i.e., the upper bound on $\MCE(\calC, p)$. 

    By definition of the multicalibration error and $\calC$, there exists some $c \in \calC$ that witnesses a multicalibration error of magnitude $\MCE(\calC, p)$, i.e. for some $f \in \calF$, 

    \[\MCE(\calC, p) = \left|\underbrace{\ex[(f(\phi(p, x)) - H_{\ell}(p(x)))H_{\ell}'(p(x))(y - p(x))]}_{:= E_f}\right|. \]

    Thus, if we define $\delta: \Phi \rightarrow [-1, 1]$ as 
    \[\delta(\phi(p, x)) = \sgn(E_f)(f(\phi(p, x)) - H_{\ell}(p(x))),\]
    it follows that 
    \begin{align*}
        \ex[\delta(\phi(p, x))H_{\ell}'(p(x))(y - p(x))] &= \MCE(\calC, p).
    \end{align*}

    Applying Lemma~\ref{lem:mc-err-implies-adv} for this $\delta$ implies that for the loss predictor defined by $\lossPred(\phi(p, x)) = \Pi_{[0,1]}(\clp(p(x)) + \MCE(\calC, p) 
    \delta(\phi(p, x)))$ satisfies 

    \[\adv(\lossPred(\phi(p, x))) \geq \MCE(\calC, p)^2.\]

    The proof of the inequality follows by observing that $\lossPred \in \calF'$, because 
    \begin{align*}
        \lossPred(\phi(p, x)) &= \Pi_{[0,1]}(\clp(p(x)) + \MCE(\calC, p) 
    \delta(\phi(p, x)))\\
    &= \Pi_{[0,1]}(H_{\ell}(p(x)) + \underbrace{\MCE(\calC, p) 
    \sgn(E_f)}_{:= \beta}(f(\phi(p, x)) - H_{\ell}(p(x))))\\
    &= \Pi_{[0,1]}((1 - \beta)H_{\ell}(p(x)) + \beta f(\phi(p, x)))
    \end{align*}

    Where $\beta = \sgn(E_f)\MCE(\calC, p) \in [-1, 1]$, because $\MCE(\calC, p) \in [0, 1]$. 

    Thus, $\lossPred \in \calF'$, and so we conclude that 

    \[\max_{\lossPred \in \calF'}\adv(\lossPred(\phi(p, x))) \geq \MCE(\calC, p)^2.\]

    We get the right-hand inequality from the statement after taking square root of both sides. 
\end{proof}


\section{Extended discussion and proofs from Section~\ref{sec:multiple-loss}}


\subsection{Proof of Lemma~\ref{lem:many-losses-mc}}\label{sec:many-losses-mc-pf}
\begin{proof}[Proof of Lemma~\ref{lem:many-losses-mc}]
    For a fixed loss $\ell \in \calL$, let 
    \[\calC_{\ell} = \{(f(\phi(p, x)) - H_{\ell}(p(x)))H'_{\ell}(p(x)) : f \in \calF\}.\]

    By Theorem~\ref{thm:mc-conv-vs-loss-pred}, we can guarantee that 
    \[\max_{\lossPred \in \calF} \adv(\lossPred) \leq 2\MCE(\calC_{\ell}, p).\]
    Taking the max over $\calL$ for both sides, we get
    \[\max_{\ell \in \calL}\max_{\lossPred \in \calF} \adv(\lossPred) \leq \max_{\ell \in \calL} 2\MCE(\calC_{\ell}, p) \leq 2\MCE(\calC_{\calL}, p).\]

    Where the right-most inequality follows from the fact that $\calC_{\calL} = \bigcup_{\ell \in \calL} \calC_{\ell}$. This proves the desired inequality.
\end{proof}

\subsection{Multicalibration for product classes}

In this section, we introduce some useful notation that we will use to refer to and relate certain classes of multicalibration test functions. 

\begin{definition}\label{def:prod-class}
    Let $\calA \subseteq \{a: \Phi \rightarrow [-1, 1]\}$ and $\calB \subseteq \{b: \Phi \rightarrow [-1, 1]\}$ be two classes of functions. We denote the product class of test functions with respect to $\calA$ and $\calB$ as $\calC_{\calA, \calB}$, and define it as 
    \[\calC_{\calA, \calB} = \{a(\phi)b(\phi) : a \in \calA, b \in \calB\}.\]
\end{definition}

We use this notation in the following lemma, which shows that multicalibration with respect to the test functions $\calC_{\calL}$ defined in Lemma~\ref{lem:many-losses-mc} is implied by multicalibration with respect to a product class of test functions:

\begin{lemma}\label{lem:loss-to-prod-class-mc}
    Let $\calF$ be a class of loss predictors $f: \Phi \rightarrow [0, 1]$. Let $\calL$ be a class of bounded proper losses $\ell: \{0, 1\} \times [0, 1] \rightarrow [0, 1]$ with associated concave entropy functions $H_{\ell}: [0, 1] \rightarrow [0,1]$, and let $\calC_{\calL}$ be the class of test functions 
    \[\calC_{\calL} = \{(f(\phi(p, x)) - H_{\ell}(p(x)))H'_{\ell}(p(x)) : f \in \calF, \ell \in \calL\}.\]

    Denote $\calH_{\calL} = \{H_{\ell} : \ell \in \calL\}$, and $\calH'_{\calL} = \{H'_{\ell}: \ell \in \calL\}$.

    Then, 

    \[\MCE(\calC_{\calL},p) \leq \MCE(\calC_{\calF, \calH'_{\calL}}, p) + \MCE(\calC_{\calH_{\calL}, \calH'_{\calL}}, p) \leq 2\MCE(\calC_{\calF\cup \calH_{\calL}, \calH'_{\calL}}, p).\]
\end{lemma}

\begin{proof}
    By definition of $\calC_{\calL}$, we can readily decompose the multicalibration error into the desired terms:

    \begin{align*}
        \MCE(\calC_{\calL},p) &= \max_{f \in \calF, \ell \in \calL}\left|\ex[(f(\phi(p, x)) - H_{\ell}(p(x)))H'_{\ell}(p(x))(y - p(x))]\right|\\
        &\leq \max_{f \in \calF, \ell \in \calL}\left|\ex[f(\phi(p, x))H'_{\ell}(p(x))(y - p(x))]\right| + \max_{\ell \in \calL}\left|\ex[H_{\ell}(p(x))H'_{\ell}(p(x))(y - p(x))]\right|\\
        &\leq \max_{f \in \calF, \ell \in \calL}\left|\ex[f(\phi(p, x))H'_{\ell}(p(x))(y - p(x))]\right| + \max_{h \in \calH_{\calL}, h' \in \calH'_{\calL}}\left|\ex[h(p(x))h'(p(x))(y - p(x))]\right|\\
        &= \MCE(\calC_{\calF, \calH'_{\calL}}, p) + \MCE(\calC_{\calH_{\calL}, \calH'_{\calL}}, p).
    \end{align*}

    This proves the left-hand inequality. The right-hand inequality follows from the observation that \[\MCE(\calC_{\calF \cup \calH_{\calL}, \calH'_{\calL}}, p) = \max\{\MCE(\calC_{\calF, \calH'_{\calL}}, p), \MCE(\calC_{\calH_{\calL}, \calH'_{\calL}}, p)\}.\]
\end{proof}

An important property of product classes is that given two classes $\calA$ and $\calB$, whenever we have a weak learner for $\calA$ and $\calB$ is finite, we can efficiently learn a multicalibrated predictor with respect to $\calC_{\calA, \calB}$. Our approach closely follows that of~\cite{gopalan2022low}, who show how to learn multicalibrated predictors for product classes where one class depends only on $x$, and the other depends only on $p(x)$. Despite this choice in setup, their particular algorithm and results naturally generalize to the case where the two function classes can have richer input spaces. 

For completeness, we translate their algorithm and results to our setting. The algorithm for product-class multicalibration can be found in Algorithm~\ref{alg:product-class} (c.f. Algorithm 1 of~\cite{gopalan2022low}).

The following lemmas prove correctness and sample complexity of the algorithm. 

\begin{lemma}[Correctness of Algorithm~\ref{alg:product-class}]
    If Algorithm~\ref{alg:product-class} returns a predictor $p : \mathcal{X} \to [0, 1]$, then $p$ satisfies 
    \[\MCE(\calC_{\calA, \calB}, p) \leq \alpha.\]
\end{lemma}

\begin{proof}
    Observe that Algorithm~\ref{alg:product-class} only returns a predictor $p_t$ if, in the $t$th iteration, for every $b \in \calB$, the call to the weak agnostic learner $\text{WAL}_{\calA}$ returns $\bot$. By the weak agnostic learning property (Definition~\ref{def:weak-agnostic-learner}), returning $\bot$ in every call indicates that for all $b \in \calB$ and for all $a \in \calA$,
    
    \[
    \ex\left[a(\phi(p_t, x))b(\phi(p_t, x))(y - p_t(x)) \right] \leq \alpha.
    \]
    
    By definition, this means that $\MCE(\calC_{\calA, \calB}, p_t) \leq \alpha$
\end{proof}

\begin{lemma}[Termination of Algorithm~\ref{alg:product-class}]
    Algorithm~\ref{alg:product-class} is guaranteed to terminate and return a $p_T$ after $T \leq 4/\alpha^2$ iterations.
\end{lemma}

\begin{proof}
    We show that the number of iterations is bounded via a potential argument, where the potential function is the squared error of the current predictor $p$ from the bayes-optimal predictor $p^*(x) = \ex[y|x]$: $\sqLoss(p) := \ex[(p(x) - p^*(x))^2]$. 

    Note that by definition, $\sqLoss(p_0) \leq 1$, and for all $p: \calX \rightarrow [0, 1]$, $\sqLoss(p) \geq 0$. 

    The change in potential after the $t$th update can be computed as follows. Note that due to the guarantee of the weak agnostic learner, because $a_{t + 1} \neq \bot$, we are guaranteed that for the $b \in \calB$ used in the update, 
    \[\ex[b(\phi(p, x))a(\phi(p, x))(y - p(x))] \geq \alpha/2.\]

    Thus, following the same proof as Lemma~\ref{lem:improve-2}, we conclude that 
    \begin{align*}
        \sqLoss(p_t) - \sqLoss(p_{t + 1}) &\geq \ex[(p^*(x) - p_t(x))^2] - \ex[(p^*(x) - p_t(x) -\frac{\alpha}{2}\delta_{t + 1}(x))^2]\\
        &= \alpha \ex[(p^*(x) - p_t(x))\delta_{t + 1}(x)] - \frac{\alpha^2}{4}\ex[\delta_{t + 1}(x)^2]\\
        &= \alpha \ex[(p^*(x) - p_t(x))a_{t + 1}(\phi(p_t, x))b(\phi(p_t, x))] - \frac{\alpha^2}{4}\ex[a_{t + 1}(\phi(p_t, x))^2b(\phi(p_t, x))^2]\\
        &\geq \alpha^2/2 - \alpha^2/4\\
        &= \alpha^2/4.
    \end{align*}

    Thus, the potential function decreases by at least $\alpha^2/4$ in each round, and since $\sqLoss(p_0) \leq 1$ and $\sqLoss(p_t) \geq 0$ for all $t$, the total number of iterations is bounded by $T < 4/\alpha^2$.
    \end{proof}

We finally turn to the sample complexity and success probability of Algorithm~\ref{alg:product-class}. 

\begin{lemma}\label{lem:finite-B-mc-alg}
    Let $\alpha, \delta > 0$. Let $\calA \subseteq \{a: \Phi \rightarrow [-1, 1]\}$ and $\calB \subseteq \{b: \Phi \rightarrow [-1, 1]\}$ be two classes of functions, where we assume $\calA$ is closed under negation, and $\calB$ is finite. Suppose we have access to an $\alpha$-weak-agnostic-learner for $\calA$ with sample complexity $n$ and failure parameter $\beta \leq \frac{\alpha^2\delta}{4|\calB|}$. 

    Then, given $m = O(n/\alpha^2)$ samples, with probability at least $1 - \delta$ Algorithm~\ref{alg:product-class} outputs a predictor $p$ with $\MCE(\calC_{\calA, \calB}, p) \leq \alpha$. 
\end{lemma}

\begin{proof}
    We assume that we use a fresh sample for the weak agnostic learner at each iteration of size $n$. Because the algorithm terminates in at most $4/\alpha^2$ iterations, we thus need at most $4n/\alpha^2 = O(n/\alpha^2)$ fresh samples. 

    For the failure bound, note that we make at most $4|\calB|/\alpha^2$ calls to the weak agnostic learner. Via a union bound, because we assume that $\beta \leq \frac{\alpha^2\delta}{4|\calB|}$, we conclude that the probability that at least one of the calls to the weak learner fails is bounded by $\delta$, as desired.   
\end{proof}

\begin{algorithm}
    \caption{Product-Class Multicalibration}
    \begin{algorithmic}[1]\label{alg:product-class}
        \STATE \textbf{Input:} training data $\{(x_i, y_i)\}_{i=1}^m$
        \STATE \textbf{Function classes:} \\
        \quad $\calA \subseteq \{a: \Phi \to [-1, 1]\}$ (closed under negation), \\
         \quad $\calB \subseteq \{b: \Phi \to [-1, 1]\}$ (finite)
        \STATE \textbf{$\alpha$-Weak Agnostic Learner for $\calA$:} $\text{WAL}_{\calA}$
        \STATE approximation $\alpha > 0$
        \STATE \textbf{Output:} $(\calC_{\calA, \calB},\alpha)$-multicalibrated predictor $p: \mathcal{X} \to [0,1]$
        \STATE $p_0(\cdot) \gets 1/2$
        \STATE $mc \gets \text{false}$
        \STATE $t \gets 0$
        \WHILE{$\neg mc$}
            \STATE $mc \gets \text{true}$
            \FOR{each $b \in \calB$}
                \STATE $a_{t+1} \gets \text{WAL}_{\calA}(\{(\phi(p, x_i), b(\phi(p_t, x_i))(y_i - p_t(x_i)))\}_{i=1}^m)$
                \IF{$a_{t+1} = \bot$}
                    \STATE \textbf{continue}
                \ELSE
                    \STATE $\delta_{t+1}(\cdot) \gets b(\phi(p_t, \cdot)) a_{t+1}(\phi(p_t, \cdot))$
                    \STATE $p_{t+1}(\cdot) \gets \Pi_{[0,1]}(p_t(\cdot) + \frac{\alpha}{2} \delta_{t+1}(\cdot)) $ 
                    \STATE $mc \gets \text{false}$
                    \STATE $t \gets t + 1$
                    \STATE \textbf{break}
                \ENDIF
            \ENDFOR
        \ENDWHILE
        \STATE \textbf{return} $p_t$
    \end{algorithmic}
\end{algorithm}

\subsection{Multicalibration for classes with approximate bases}

In this section, we show that Algorithm~\ref{alg:product-class} can also be used to guarantee multicalibration for product classes when $\calB$ is not finite, but has a finite approximate basis. 

\begin{definition}[Finite Approximate Basis]\label{def:finite-approx-basis}
     Let $\Gamma$ be a set and $\mathcal{F} = \{f : \Gamma \to [-1, 1]\}$ a class of functions on $\Gamma$. We say that a set $\mathcal{G} = \{g : \Gamma \to [-1, 1]\}$ is a finite $\epsilon$-basis for $\mathcal{F}$ of size $d$ and coefficient norm $\lambda$, if $\calG = \{g_1, ..., g_d\}$, and for every function $f \in \mathcal{F}$, there exist coefficients $\alpha_1, \alpha_2, \ldots, \alpha_d \in [-1,1]$ satisfying
\begin{equation}
    \forall x \in \Gamma \quad \left| f(x) - \sum_{i=1}^{d} \alpha_i g_i(x) \right| \leq \epsilon \quad \text{and} \quad \sum_{i=1}^{d} |\alpha_i| \leq \lambda. 
\end{equation}
\end{definition}

\begin{lemma}\label{lem:basis-mce-bound}
    Let $\calA \subseteq \{a: \Phi \rightarrow [-1, 1]\}$ and $\calB \subseteq \{b: \Phi \rightarrow [-1, 1]\}$ be two classes of functions. Suppose that $\calB$ has a finite approximate $\epsilon$-basis $\calG$ with coefficient norm $\lambda$. Then, for any predictor $p: \calX \rightarrow [0, 1]$, 
    \[\MCE(\calC_{\calA, \calB}, p) \leq \lambda\MCE(\calC_{\calA, \calG}, p) + \epsilon\]
\end{lemma}

\begin{proof}
    The upper bound quickly follows from expanding the definition of $\MCE(\calC_{\calA, \calB}, p)$. Note that 

    \begin{align*}
        \MCE(\calC_{\calA, \calB}, p) &= \max_{a \in \calA, b \in \calB}\left|\ex[a(\phi(p, x))b(\phi(p, x))(y - p(x))]\right|\\
        &\leq \max_{a \in \calA, b \in \calB}\left|\ex[a(\phi(p, x))\left(\sum_{i = 1}^d g_i(\phi(p, x))\alpha_i(b)\right)(y - p(x))]\right| + \epsilon\\
        &\leq \max_{a \in \calA, g \in \calG}\lambda\left|\ex[a(\phi(p, x))g(\phi(p, x))(y - p(x))]\right| + \epsilon\\
        &= \lambda\MCE(\calC_{\calA, \calG}, p) + \epsilon
    \end{align*}
\end{proof}

Thus, we get the following immediate Corollary of Lemmas~\ref{lem:finite-B-mc-alg} and \ref{lem:basis-mce-bound}:

\begin{corollary}\label{cor:approx-basis-alg}
    Let $\alpha, \delta > 0$. Let $\calA \subseteq \{a: \Phi \rightarrow [-1, 1]\}$ and $\calB \subseteq \{b: \Phi \rightarrow [-1, 1]\}$ be two classes of functions, where we assume $\calA$ is closed under negation, and $\calB$ has a finite approximate $\epsilon$-basis of size $d$ and coefficient norm $\lambda$. Suppose we have access to an $\alpha$-weak-agnostic-learner for $\calA$ with sample complexity $n$ and failure parameter $\beta \leq \frac{\alpha^2\delta}{4d}$. 

    Then, there exists an algorithm that, given $m = O(n/\alpha^2)$ samples, with probability at least $1 - \delta$ outputs a predictor $p$ with $\MCE(\calC_{\calA, \calB}, p) \leq \lambda\alpha + \epsilon$. 
\end{corollary}

\subsection{Instantiating multicalibration for loss prediction}

We are finally ready to show that we can use multicalibration to learn a predictor with accurate self-entropy predictions for any loss in a rich class. 

\begin{theorem}\label{thm:general-approx-basis-mc}
    Fix $\delta \geq 0$. Let $\calL$ be some class of bounded proper losses, and let $\calF$ be a class of loss predictors $\calF: \Phi \rightarrow [-1, 1]$ that is closed under negation and contains the class of self entropy predictors, $\calH_{\calL} = \{H_{\ell}\}_{\ell \in \calL}$. Suppose that the class of functions $\calH'_{\calL} =\{H'_{\ell}\}_{\ell \in \calL}$ has a finite approximate $\epsilon$-basis of size $d$ and coefficient norm $\lambda$. Further assume that we have access to an $\alpha$-weak-agnostic-learner for $\calF$ with sample complexity $n$ and failure parameter $\beta \leq \frac{\alpha^2\delta}{4d}$. 

    Then, there exists an algorithm that, given $m = O(n/\alpha^2)$ samples, with probability at least $1 - \delta$ outputs a predictor $p$ such that 
    \[\max_{\ell \in \calL} \max_{\lossPred \in \calF} \adv(\lossPred) \leq 4\lambda\alpha + 4\epsilon.\]
\end{theorem}

\begin{proof}
    The proof combines the helper lemmas we have proved in this section. 

    First, note that by Lemma~\ref{lem:many-losses-mc}, we are guaranteed that 
    \[\max_{\ell \in \calL} \max_{\lossPred \in \calF} \adv(\lossPred) \leq 2\MCE(\calC_{\calL}, p),\]
    where $\calC_{\calL}$ is defined as in Lemma~\ref{lem:many-losses-mc}. 

    By Lemma~\ref{lem:loss-to-prod-class-mc}, we can further guarantee that 
    \[\max_{\ell \in \calL} \max_{\lossPred \in \calF} \adv(\lossPred) \leq 4\MCE(\calC_{\calF \cup \calH_{\calL}, \calH'_{\calL}}, p) = 4\MCE(\calC_{\calF, \calH'_{\calL}}, p),\]
    where the last equality follows from our assumption that $\calF$ contains $\calH_{\calL}$.

    From here, we can now apply the result of Corollary~\ref{cor:approx-basis-alg}, which guarantees that given $m = O(n/\alpha^2)$ samples, we can output a predictor satisfying 
    $\MCE(\calC_{\calF, \calH'_{\calL}}, p) \leq \lambda\alpha + \epsilon$. 

    Substituting this bound into our upper bound, we conclude that we get a predictor satisfying 
    \[\max_{\ell \in \calL} \max_{\lossPred \in \calF} \adv(\lossPred) \leq 4\lambda\alpha + 4\epsilon.\]

    This completes the proof. 
\end{proof}

\subsection{Proof of Theorem~\ref{thm:1-lip-mc-pred}}\label{sec:1-lip-mc-pred-pf}

Theorem~\ref{thm:1-lip-mc-pred} instantiates the general result of Theorem~\ref{thm:general-approx-basis-mc} for a class $\calL$ that has a finite approximate basis: the class of all 1-Lipschitz proper losses. 

We appeal to a result proved by~\cite{OKK25}, which proves the existence of such a basis for this class:

\begin{lemma}[\cite{OKK25}, Lemma 5.4]\label{lem:1-lip-basis}
    Let $\epsilon > 0$, and let $\calL_{Lip}$ be the class of proper 1-Lipschitz loss functions $\ell: \{0, 1\} \times [0, 1] \rightarrow [0,1]$. The class $\{H'_{\ell}\}_{\ell \in \calL_{Lip}}$ admits an $\epsilon$-approximate basis of size $\lceil \frac{2}{\epsilon} + 1 \rceil$ and coefficient norm 4. 
\end{lemma}

with this Lemma in hand, we are ready to prove the theorem. 

\begin{proof}[Proof of Theorem~\ref{thm:1-lip-mc-pred}]
    We instantiate the basis from Lemma~\ref{lem:1-lip-basis}, and use this as the input to Theorem~\ref{thm:general-approx-basis-mc}.

    Because $d = \lceil 2/\epsilon + 1\rceil$, the bound on the weak-agnostic-learner's error probability becomes $\beta \leq \frac{\alpha^2\delta}{4\lceil 2/\epsilon + 1\rceil}$, and the resulting error guarantee gives us a predictor $p$ satisfying 

    \[\max_{\ell \in \calL_{Lip}} \max_{\lossPred \in \calF} \adv(\lossPred) \leq 16\alpha + 4\epsilon\]
    with probability at least $1 - \delta$. 
\end{proof}




\end{document}
