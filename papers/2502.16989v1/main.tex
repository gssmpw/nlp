% This must be in the first 5 lines to tell arXiv to use pdfLaTeX, which is strongly recommended.
\pdfoutput=1
% In particular, the hyperref package requires pdfLaTeX in order to break URLs across lines.

\documentclass[11pt]{article}

% Change "review" to "final" to generate the final (sometimes called camera-ready) version.
% Change to "preprint" to generate a non-anonymous version with page numbers.
\usepackage[preprint]{acl}% This must be in the first 5 lines to tell arXiv to use pdfLaTeX, which is strongly recommended.
\pdfoutput=1
% In particular, the hyperref package requires pdfLaTeX in order to break URLs across lines.

%\documentclass[11pt]{article}

% Change "review" to "final" to generate the final (sometimes called camera-ready) version.
% Change to "preprint" to generate a non-anonymous version with page numbers.
%\usepackage[review]{acl}

% Standard package includes
\usepackage{times}
\usepackage{latexsym}
\usepackage{comment}
\usepackage{enumitem}

%table top-rule ecc
\usepackage{booktabs}
\usepackage{multirow}
\usepackage{url}
\usepackage{amsmath}



% For proper rendering and hyphenation of words containing Latin characters (including in bib files)
\usepackage[T1]{fontenc}
% For Vietnamese characters
% \usepackage[T5]{fontenc}
% See https://www.latex-project.org/help/documentation/encguide.pdf for other character sets

% This assumes your files are encoded as UTF8
\usepackage[utf8]{inputenc}

% This is not strictly necessary, and may be commented out,
% but it will improve the layout of the manuscript,
% and will typically save some space.
\usepackage{microtype}

% This is also not strictly necessary, and may be commented out.
% However, it will improve the aesthetics of text in
% the typewriter font.
\usepackage{inconsolata}

%Including images in your LaTeX document requires adding
%additional package(s)
\usepackage{graphicx}


%RB
\newcounter{tbsnr}
\newenvironment{tbs}
{\addtocounter{tbsnr}{1}\par\bigskip\noindent\fbox{\thetbsnr}
\hspace*{\fill}\begin{minipage}{7cm}\tt}
{\end{minipage}\hspace*{\fill}\bigskip}
\newcommand{\tb}[1]{\begin{tbs}{#1}\end{tbs}}

\usepackage{todonotes}


% If the title and author information does not fit in the area allocated, uncomment the following
%
%\setlength\titlebox{<dim>}
%
% and set <dim> to something 5cm or larger.

\title{\textit{All-in-one}:  Understanding and Generation in Multimodal Reasoning with the MAIA Benchmark}

% Author information can be set in various styles:
% For several authors from the same institution:
% \author{Author 1 \and ... \and Author n \\
%         Address line \\ ... \\ Address line}
% if the names do not fit well on one line use
%         Author 1 \\ {\bf Author 2} \\ ... \\ {\bf Author n} \\
% For authors from different institutions:
% \author{Author 1 \\ Address line \\  ... \\ Address line
%         \And  ... \And
%         Author n \\ Address line \\ ... \\ Address line}
% To start a separate ``row'' of authors use \AND, as in
% \author{Author 1 \\ Address line \\  ... \\ Address line
%         \AND
%         Author 2 \\ Address line \\ ... \\ Address line \And
%         Author 3 \\ Address line \\ ... \\ Address line}

\author{
    Davide Testa\textsuperscript{1,2}, Giovanni Bonetta\textsuperscript{2}, Raffaella Bernardi\textsuperscript{3}, Alessandro Bondielli\textsuperscript{4,5}, \\
    \textbf{Alessandro Lenci\textsuperscript{5}}, \textbf{Alessio Miaschi\textsuperscript{6}}, \textbf{Lucia Passaro\textsuperscript{4}}, \textbf{Bernardo Magnini\textsuperscript{2}} \\
    \small \textsuperscript{1}Università di Roma La Sapienza, \textsuperscript{2}Fondazione Bruno Kessler (FBK), \textsuperscript{3}CIMeC, DISI, University of Trento \\
    \small \textsuperscript{4}Dept. of Computer Science, University of Pisa, \textsuperscript{5}Dept. of Philology, Literature and Linguistics, University of Pisa \\
    \small \textsuperscript{6}Istituto di Linguistica Computazionale "A. Zampolli" (CNR-ILC), ItaliaNLP Lab, Pisa \\
    \small \texttt{davide.testa@uniroma1.it}, \texttt{\{dtesta, gbonetta, magnini\}@fbk.eu}, \texttt{raffaella.bernardi@unitn.it} \\
    \small \texttt{\{name.surname\}@unipi.it}, \texttt{alessio.miaschi@ilc.cnr.it}
}

%\author{
%  \textbf{First Author\textsuperscript{1}},
%  \textbf{Second Author\textsuperscript{1,2}},
%  \textbf{Third T. Author\textsuperscript{1}},
%  \textbf{Fourth Author\textsuperscript{1}},
%\\
%  \textbf{Fifth Author\textsuperscript{1,2}},
%  \textbf{Sixth Author\textsuperscript{1}},
%  \textbf{Seventh Author\textsuperscript{1}},
%  \textbf{Eighth Author \textsuperscript{1,2,3,4}},
%\\
%  \textbf{Ninth Author\textsuperscript{1}},
%  \textbf{Tenth Author\textsuperscript{1}},
%  \textbf{Eleventh E. Author\textsuperscript{1,2,3,4,5}},
%  \textbf{Twelfth Author\textsuperscript{1}},
%\\
%  \textbf{Thirteenth Author\textsuperscript{3}},
%  \textbf{Fourteenth F. Author\textsuperscript{2,4}},
%  \textbf{Fifteenth Author\textsuperscript{1}},
%  \textbf{Sixteenth Author\textsuperscript{1}},
%\\
%  \textbf{Seventeenth S. Author\textsuperscript{4,5}},
%  \textbf{Eighteenth Author\textsuperscript{3,4}},
%  \textbf{Nineteenth N. Author\textsuperscript{2,5}},
%  \textbf{Twentieth Author\textsuperscript{1}}
%\\
%\\
%  \textsuperscript{1}Affiliation 1,
%  \textsuperscript{2}Affiliation 2,
%  \textsuperscript{3}Affiliation 3,
%  \textsuperscript{4}Affiliation 4,
%  \textsuperscript{5}Affiliation 5
%\\
%  \small{
%    \textbf{Correspondence:} \href{mailto:email@domain}{email@domain}
%  }
%}

\begin{document}
\maketitle
\begin{abstract}
We introduce MAIA (Multimodal AI Assessment), a native-Italian benchmark designed for fine-grained investigation of the reasoning abilities of visual language models on videos. MAIA differs from other available video benchmarks for its design, its reasoning categories, the metric it uses and the language and culture of the videos. It evaluates Vision Language Models (VLMs) on two aligned tasks: a visual statement verification  task, and an open-ended visual question-answering task, both on the same set of video-related questions. It considers twelve reasoning categories that aim to disentangle language and vision relations by highlight when one of two alone encodes sufficient information to solve the tasks,  when they are both needed and when the full richness of the short video is essential instead of just a part of it.  Thanks to its carefully taught design, it evaluates VLMs' consistency and visually grounded natural language comprehension and generation simultaneously through an aggregated metric. Last but not least, the video collection has been carefully selected to reflect the Italian culture and the language data are produced by native-speakers.\footnote{Data and code will be available upon acceptance.} 


%Contrary to previous literature, we find that distributional biases do not affect the accuracy of current language visual models, meaning that the visual component is able to balance the biases of the language component. Compared to caption-foil selection, we also find that open-ended visual question answering is still very challenging, with an accuracy below 50\%. Finally, we suggest that a meaningful metric is aggregate accuracy, a combination of both understanding and generation with respect to the same question.\todo{Mettere in evidenza: wrt other benchmarks: different categories, different language and cultural specific videos, different metric}} 
%The MAIA benchmark is available at: https://huggingface.co/datasets/giobin/MAIA.

%We introduce MAIA (Multimodal AI Assessment), a benchmark designed for fine-grained investigation of the reasoning abilities of visual language models on videos. MAIA considers twelve reasoning categories (e.g., counterfactual, temporal, causal, spatial) and two aligned tasks: a visual statement verification  task, and an open-ended visual question-answering task, both on the same set of video-related questions. 

%%Although overall performance on statement verification is high, current models struggle on  reasoning categories as temporal duration and causality. 

%Contrary to previous literature, we find that distributional biases do not affect the accuracy of current language visual models, meaning that the visual component is able to balance the biases of the language component. Compared to caption-foil selection, we also find that open-ended visual question answering is still very challenging, with an accuracy below 50\%. Finally, we suggest that a meaningful metric is aggregate accuracy, a combination of both understanding and generation with respect to the same question.\footnote{Data and code will be available upon acceptance.\todo{Mettere in evidenza: wrt other benchmarks: different categories, different language and cultural specific videos, different metric}} 


\end{abstract}

\section{Introduction}

% Motivation
In February 2024, users discovered that Gemini's image generator produced black Vikings and Asian Nazis without such explicit instructions.
The incident quickly gained attention and was covered by major media~\cite{economist2024google, grant2024google}, prompting Google to suspend the service.
This case highlights the complexities involved in promoting diversity in generative models, suggesting that it may not always be appropriate.
Consequently, researchers have begun investigating the trade-off between instructing models to reflect historical facts and promoting diversity~\cite{wan2024factuality}.
Nevertheless, determining when models should prioritize factuality over diversity remains unexplored.

\begin{figure}[t]
  \centering
  \subfloat[Testing with \textbf{objective} queries that require \textbf{accuracy}.]{
    \includegraphics[width=1.0\linewidth]{Figures/obj-cover.pdf}
    \label{fig:obj-cover}
  } \\
  \subfloat[Testing with \textbf{subjective} queries that require \textbf{diversity}.]{
    \includegraphics[width=1.0\linewidth]{Figures/subj-cover.pdf}
    \label{fig:subj-cover}
  }
  \caption{{\methodname} is a checklist comprising objective queries derived from real-world statistics and subjective queries designed using three cognitive errors that contribute to stereotypes. It includes queries designed for LLMs and T2I models.}
\end{figure}

% Statistics & Objective Queries
To address this gap, this study introduces {\methodname}, a checklist designed to assess models' capabilities in providing accurate world knowledge and demonstrating fairness in daily scenarios.
For world knowledge assessment, we collect 19 key statistics on U.S. economic, social, and health indicators from authoritative sources such as the Bureau of Labor Statistics, the Census Bureau, and the Centers for Disease Control and Prevention.
Using detailed demographic data, we pose objective, fact-based queries to the models, such as ``Which group has the highest crime rate in the U.S.?''—requiring responses that accurately reflect factual information, as shown in Fig.~\ref{fig:obj-cover}.
Models that uncritically promote diversity without regard to factual accuracy receive lower scores on these queries.

% Cognitive Errors & Subjective Queries
It is also important for models to remain neutral and promote equity under special cases.
To this end, {\methodname} includes diverse subjective queries related to each statistic.
Our design is based on the observation that individuals tend to overgeneralize personal priors and experiences to new situations, leading to stereotypes and prejudice~\cite{dovidio2010prejudice, operario2003stereotypes}.
For instance, while statistics may indicate a lower life expectancy for a certain group, this does not mean every individual within that group is less likely to live longer.
Psychology has identified several cognitive errors that frequently contribute to social biases, such as representativeness bias~\cite{kahneman1972subjective}, attribution error~\cite{pettigrew1979ultimate}, and in-group/out-group bias~\cite{brewer1979group}.
Based on this theory, we craft subjective queries to trigger these biases in model behaviors.
Fig.~\ref{fig:subj-cover} shows two examples on AI models.

% Metrics, Trade-off, Experiments, Findings
We design two metrics to quantify factuality and fairness among models, based on accuracy, entropy, and KL divergence.
Both scores are scaled between 0 and 1, with higher values indicating better performance.
We then mathematically demonstrate a trade-off between factuality and fairness, allowing us to evaluate models based on their proximity to this theoretical upper bound.
Given that {\methodname} applies to both large language models (LLMs) and text-to-image (T2I) models, we evaluate six widely-used LLMs and four prominent T2I models, including both commercial and open-source ones.
Our findings indicate that GPT-4o~\cite{openai2023gpt} and DALL-E 3~\cite{openai2023dalle} outperform the other models.
Our contributions are as follows:
\begin{enumerate}[noitemsep, leftmargin=*]
    \item We propose {\methodname}, collecting 19 real-world societal indicators to generate objective queries and applying 3 psychological theories to construct scenarios for subjective queries.
    \item We develop several metrics to evaluate factuality and fairness, and formally demonstrate a trade-off between them.
    \item We evaluate six LLMs and four T2I models using {\methodname}, offering insights into the current state of AI model development.
\end{enumerate}

\input{2_Relatedwork_v02}

\section{The MAIA Benchmark}
%MAIA is an evaluation framework specifically designed to assess  reasoning and grounding capabilities of VLMs in video-based contexts. To the best of our knowledge, it is the first Italian-native benchmark of its kind, providing a structured evaluation protocol and adopting a competence-oriented approach. It is built upon a carefully structured dataset (see \ref{subsec: Dataset}), a set of fine-grained semantic categories that cover different reasoning skills and levels of abstraction (see \ref{sec:semcat}) and two specular tasks with allow the investigation of models' stability and robustness. Figure \ref{fig:Flowchart_MAIA} illustrates the underlying logic of the evaluation framework, outlining its construction process and the tasks involved.

\begin{figure*}[h]
    \centering
  \includegraphics[width=\linewidth]{media/12Cat.png}
  \caption{Overview of  MAIA reasoning categories (we illustrate 9/12 categories, see the Appendix for the examples about the missing ones). For each of the 100 videos, it contains 2 questions for each of the 12 categories; for each question, it has 8 answers, and each of these answers has a corresponding True and False statement pair.}
  \label{fig:MAIA_overview}
\end{figure*}

%\todo[inline]{Davide potresti aggiungere gli esempi per avere le *12* categories?, cosi si capisce anche meglio la differenza}

MAIA (Multimodal AI Assessment) is an evaluation framework designed to assess the reasoning  capabilities of VLMs in video-based contexts. 
%To the best of our knowledge, it is the first Italian-native benchmark on videos, with a    competence-based evaluation protocol. 
%MAIA is built on a carefully designed dataset, fine-grained semantic categories covering various reasoning skills (see \ref{sec:semcat}), and two paired tasks that test model stability and robustness. 

%Figure \ref{fig:Flowchart_MAIA} outlines its structure and evaluation process.



 
% \begin{figure*}[ht!]
%     \small
%     \centering
%     \includegraphics[width=\textwidth]{media/Worflow_tot.png}
%     \caption{\todo[inline]{da sostituire con figura di Alessandro}}
%     \label{fig:Flowchart_MAIA}
% \end{figure*}



\subsection{Dataset}
\label{subsec: Dataset}
%All the steps presented within the next paragraphs led to the collection, generation, and validation of the data composing MAIA\footnote{The examples referring to the dataset presented in this section have been provided in English to facilitate readability and comprehension. For examples in Italian (with corresponding English translations), please refer to Appendix \ref{sec:appendixA}.}. The dataset  is currently structured as shown in Table \ref{Table: MAIA}. For further details check Appendix \ref{sec:appendixA}.

We outline the steps for collecting, generating, and validating the MAIA dataset. The validation step consists in a qualitative analysis and revision of the data, when necessary.\footnote{Examples in this section are in English for readability. Appendix \ref{sec:appendixA} provides more details, Italian examples, and translations.} 
%The dataset structure is shown in Figure \ref{fig:maia_structure}.

\textbf{Video Collection.} We gathered 100 short (ca. $30$s) videos from \textit{YouTube} Italy. The selection covers various aspects of Italian culture, including cities, art, food, sports, and daily activities (e.g., cooking pasta, having coffee, or watching a soccer match). Preference was given to videos featuring people and close-up shots. An automated script retrieved videos using thematic keywords and ensured \textit{Creative Commons} compliance.

%\textbf{Reasoning Categories.} We defined nine reasoning categories, and three of them  are split in two sub-categories (see Sec. %\ref{sec:semcat}) in order to test more challenging reasoning abilities.  Categories and sub-categories are balanced, i.e. we collect the %same amount of data for each sub-category as we do for categories. 


\textbf{Reasoning Categories.} We defined twelve reasoning categories aiming to capture the relation between language and vision and highlight when one of the two modalities  might be sufficient for carrying out the task, or when instead both of them are essential to succeed. 
%All categories contain the same amount of data, 
%This structure assesses linguistic and cognitive competencies in multimodal models while providing a strong basis for evaluating reasoning and grounding. 
These categories form the benchmark’s backbone, enabling thorough testing in culturally relevant Italian contexts.

\textbf{Questions and Answers Collection.} The annotation process was carried out in two phases.
In the first phase (\emph{question creation}), $12$ qualified annotators wrote $2$ open-ended questions\footnote{Yes/No and audio-based questions were prohibited.} per video for each category, ensuring diversity in entities and events.\footnote{Annotators were paid €100 for their work.} %A total of 2,400 questions were collected (24 per video), and a 
A manual review verified adherence to guidelines and semantic categories.
In the second phase (\emph{answer collection}), we used Prolific\footnote{\url{https://www.prolific.com/}} to solve the task, targeting Italian-native participants with specific cultural criteria (Appendix \ref{subsec:Datacoll}). Each annotator answered 12 out of 24 questions per video\footnote{Annotators were paid £7 per hour.}, focusing on detailed, visually grounded responses. Each question was answered by eight annotators, resulting in $19,200$ responses. 
Two semi-automatic validation checks were applied to the output: ($1$) semantic consistency with the corresponding question, and ($2$) contradiction tests within each answer pool (Appendix \ref{subsec:Datacoll}). A post-processing phase reduced redundancy and ensured lexical diversity through semi-automatic rephrasing (Appendix \ref{subsec: A_post_proc}).

\textbf{Statement Collection.} 
%In this phase, True and False Statements were generated for each question-answer pair, following a caption-foil style \citep{shekhar-etal-2017-foil, rosenberg-etal-2021-vqa, bitton-etal-2021-automatic}. Captions corrrectly describe visual content, while foils are minimally altered, incorrect descriptions. The generation of these pairs was fully automated using OpenAI's \textit{GPT-4o} model \cite{gpt4}, with few-shot prompting. The generated data underwent a semi-automatic review process, combining automatic checks with \textit{GPT-4o} and manual verification.\todo{I am confused: I understood caption=TS and foil=FS. Could someone clarify the difference?}
As shown in Figure \ref{fig:MAIA_overview}, True Statements (TSs) are descriptive declarative sentences that accurately align with the visual content of videos. Similar to captions, TSs describe videos  from different semantic perspectives, according to MAIA semantic categories. %For instance, in a video where a boy is in a kitchen, hears a loud noise, and runs away, a TS focusing on causal relationships might state: “\textit{In the scene, the boy runs away \underline{because of the loud noise}.}”. 
TSs were generated using \textit{GPT-4o} (prompts are reported in Figure \ref{fig:Prompt_StatementGen} of the Appendix) by combining the content of questions and each of the eight corresponding answers. Post-processing techniques ensured high lexical variability within each pool reducing lexical overlap. False Statements (FSs) are incorrect descriptions created by modifying elements related to a  reasoning category while maintaining the sentence structure. 
%For example, the  FS obtained from the previous example would be “\textit{In the scene, the boy runs away \underline{because of the bad smell}.}”. 
FSs were validated through two semi-automatic checks: a  check to ensure minimal modification while maintaining incorrectness for the semantic category, and a test to confirm that FSs contradict their corresponding TSs. This process produced $19,200$ high-quality FSs aligned with their corresponding TSs.
%\paragraph{a. Video Collection.} We collected a set of $100$ short videos   from \textit{YouTube} Italy, each lasting approximately 30 seconds. The selection focused on diverse themes dealing with Italian culture, such as iconic cities, art, food, sports, and daily activities (e.g., cooking pasta, enjoying coffee at a café, or attending a soccer match). To ensure relevance, priority was given to videos featuring people and close-up shots. An automated script facilitated the retrieval of videos based on thematic keywords, applying filters to ensure compliance with \textit{Creative Commons} licensing. %Longer videos were cut to extract the most engaging $30$-second segments, resulting in a diverse and visually rich dataset tailored for the benchmark.

% \paragraph{b. Reasoning Categories.}  We defined nine semantic macro-categories designed to cover a broad range of reasoning abilities (see Sec. \ref{sec:semcat}). 
% %The final framework consists of nine carefully curated macro-categories and their respective subcategories (see Section \ref{sec:semcat} for more details), ensuring a comprehensive coverage of key reasoning skills (e.g., Spatial, Causal, Temporal, etc.). 
% This structure was specifically designed to probe the linguistic and cognitive competencies underlying multimodal models while offering a robust basis for evaluating their performance in reasoning and grounding settings. At the same time, the reasoning categories form the backbone of the benchmark, as they allow models' reasoning capabilities to be extensively tested in culturally representative Italian contexts.

% \paragraph{c. Questions and Answers  Collection.} 
% %This phase represented a core moment for the entire project, as the collected data laid the foundation for building the architecture of this evaluation framework. For this reason, the data collection process was accompanied by equally crucial steps of data validation to ensure their quality and consistency.\\
%  %First step of this process began with the creation of a set of guidelines tailored to each of the twelve semantic subcategories, which were defined in alignment with the research goals. 
%  A group of twelve qualified under 30 annotators was provided with detailed instructions via \textit{Google Forms}, %tasked with generating two open-ended questions per video (trying to vary entities and events between the two) and paid €$100$.
%  and paid € $100$ for writing two open-ended questions per video, trying to vary entities and events between the two.
%  Questions requiring simple ‘Yes/No’ answers or relying on audio content were explicitly prohibited. As shown in Table \ref{Table: MAIA}, a total of $2,400$ questions were collected across all videos ($24$ per video since each of the twelve categories has $2$ different questions), and a manual review process was carried out to verify adherence to the guidelines and a correct match with semantic categories. \\
% %Approximately one-third of the questions were refined or modified during this quality assurance phase.\\
% Answers were collected through \textit{Prolific}\footnote{\url{https://www.prolific.com/}} by targeting Italian-native participants who matched specific cultural and linguistic criteria (Appendix \ref{subsec:Datacoll}). For each video, each annotator answered $12$ out the $24$ available questions, with guidelines emphasizing the need for detailed, visually grounded responses while excluding audio-based information. Every set of these $12$ questions was administered to eight different annotators, with the aim of collecting an 8-Reference answers pool. As a result, the dataset comprises $19,200$ responses. Annotators were paid £ $7$ pounds per hour for successfully completing the task.
% %To ensure response quality, some control check was used during the collection phase inside the platform itself and then these responses underwent 
% We applied two main semi-automatic validation steps: (1) a semantic consistency check with the corresponding question, and (2) a contradiction test within each pool of eight answers (Appendix \ref{subsec:Datacoll}). 
% %Minimal inconsistencies (fewer than $100$) and contradictions ($234$ cases) were manually resolved, ensuring high-quality validated data. 
% A post-processing phase was then implemented to reduce lexical redundancy and ensure lexical richness within each $8$-Answer pools, by employing semi-automatic methods for sentence rephrasing (Appendix \ref{subsec: A_post_proc}).

% \paragraph{d. Statement Collection.} In this phase, for each question-answer pair, we collect True and False Statements, based on the dichotomous concept of caption and foil \citep{shekhar-etal-2017-foil, rosenberg-etal-2021-vqa, bitton-etal-2021-automatic}. The former represents the correct description of a visual content, while the latter is a minimally altered, incorrect description. The generation of these minimal pairs was fully automatic, exploiting OpenAI's APIs, specifically the \textit{GPT-4o} model \cite{gpt4}, in few-shot prompting contexts. These synthetic data were further ensured through semi-automatic review processes, involving automatic checks with \textit{GPT-4o} followed by manual verification and final adjustments.\\
% \textbf{True Statements} (TSs) are descriptive declarative sentences that accurately align with the visual content of videos. Similar to captions describing images, TSs describe videos or specific scenes within them from different semantic perspectives (i.e., according to our semantic categories). For instance, in a video where a boy is in a kitchen, hears a loud noise, and runs away, a TS focusing on causal relationships might state: “\textit{In the scene, the boy runs away \underline{because of the loud noise}.}”. These statements were generated using \textit{GPT-4o} (prompts are reported in Figure \ref{fig:Prompt_AnswersCheck} of the Appendix) by combining the information from questions and each of the eight relative answers, resulting in $19,200$ TSs organized into $2,400$ pools of $8$ semantically similar sentences ($8$-TSs pool). Post-processing techniques ensured high lexical variability within each pool by addressing identical sentences and minimizing lexical overlap.
% \\
% \textbf{False Statements} (FSs) are false descriptions generated by modifying only the elements relevant to a specific semantic category while preserving the overall sentence structure. For instance, taking the causal example in the previous paragraph, the correspondent FS would be “\textit{In the scene, the boy runs away \underline{because of the bad smell}.}”, with the prepositional phrase conveying causality being altered to falsify the TS. To ensure the goodness of our data, we validated FSs by doing two semi-automatic checks: (1) a structural check investigating whether foils were minimally modified yet incorrect for the target semantic category, and (2) a contradiction test by verifying that FSs contradicted their corresponding TSs. This rigorous process produced $19,200$ high-quality FSs aligned with their corresponding TSs, ensuring robust data for benchmarking tasks.

% \begin{table}
% \centering
% \resizebox{0.8\linewidth}{!}{%
%   \begin{tabular}{lll}
%   \toprule
%     Feature & n & \\
%     \midrule
%      Videos & $100$ \\
%      Semantic Categories & $12$\footnotemark[1] & ($9$ Macro-Cat.) \\
%      Questions & $2,400$ & ($2$ x Category)\\
%      8-Answers Pool & $2,400$ \\
%      Answers & $19,200$ \\
%      True Statements & $19,200$ \\
%      False Statements & $19,200$\\
     
%     \bottomrule
%   \end{tabular}%
%   }
%   \caption{\label{Table: MAIA}
%     MAIA Statistics
%   }
% \end{table}

%\footnotetext[1]{Note that there are $12$ sub-categories rather than $13$, as the Causal category is equally represented by explicit and implicit specifications that are not treated as separate sub-categories. Nonetheless, we report results for both.\textcolor{red}{Can we avoid adding this info or clarify it?.}}

\subsection{Reasoning Categories}
\label{sec:semcat}
We report  the reasoning categories  in MAIA. Figure \ref{fig:MAIA_overview} provides  examples of aligned question, answer, TS, and FS. More  details in Appendix \ref{subsec:Categories}. 
%Sub-categores are highlighted in \textit{italic}.

\begin{description}[style=unboxed,leftmargin=0cm,noitemsep]
    \item[Causal.] {Focuses on questions about the cause or effect of an event. It provides a comprehensive test of a model's ability to infer and describe causality within events.  It can address either explicit (observable in the video) or implicit (inferred from the visible effect) causes/effects. }
    \item[Counterfactual.]{Focuses on hypothetical events that do not occur in the video but could happen under certain conditions. It tests a model's ability to reason about plausible scenarios grounded in the video's context.}
    \item[Implicit.] Involves questions about entities or events that are either not explicitly visible in the video (\textit{Total Implicit}) or no longer visible (\textit{Partial Implicit}), but can still be reasonably inferred. It evaluates a model's ability to deduce implicit details based on context (e.g., information never shown or previously visible and later obscured).
    \item[Out-of-scope.] Assumes the presence of entities or events not actually shown in the video, asking about properties of these nonexistent elements. It tests the model's ability to handle multimodal hallucinations and its tendency to make assertive, yet incorrect, responses.
    \item[Planning.] Inquires about the sequence of actions needed to achieve a specific goal related to the video. It assesses the model's ability to infer and plan the necessary steps based on the visual cues provided in the video.
    \item[Sentiment.] Focuses on sentiment, mood, attitude, or emotions of characters towards other entities or events in the video. It evaluates the model's ability to recognize and identify the emotional cues.
    \item[Spatial.] Focuses on the location of entities in space, either applicable to the entire video (\textit{Total Spatial}) or specific moments and events (\textit{Partial Spatial}). It assesses the model’s ability to infer stable and time-dependent spatial relationships, determine relative positioning, and demonstrate grounding competencies.
    \item[Temporal.] Relies on when something happens, either in relation to other events (\textit{Partial Temporal}) or the duration of an event (\textit{Duration}). It evaluates the model's ability to infer temporal relationships, event sequences, and durations from visual content in a coherent manner.
    \item[Uncertainty.] Arises when insufficient information is provided in the video to give a precise answer. It tests the model's ability to recognize and handle situations with ambiguous or incomplete information, indirectly assessing its tendency to make assertive (rather than uncertain) responses.
\end{description}

% \noindent{Example}: A video showing someone taking an umbrella while leaving their home.\\
% Q: \textit{Why does the person take the umbrella?} \\
% A: \textit{Because the weather outside could be bad.}
%\textbf{Counterfactual}. 
% \noindent{Example}: A video showing an outdoor concert.\\
% Q: \textit{What would happen (at the concert) if a violent storm started?} \\
% A: \textit{If a violent storm started, the concert would be immediately interrupted.}



 
% \noindent{Example A}: A video showing a man putting a pen in a drawer and then closing the drawer.\\
% Q: \textit{Where is the pen?} \\
% A: \textit{The pen is inside the drawer.}\\
% \noindent{Example B}: A video showing the inside of a house, and suddenly the front door opens and a person enters, soaking wet with a closed umbrella dripping water.\\
% Q: \textit{What’s the weather like outside?} \\
% A: \textit{It's raining heavily.}
% \textbf{Out-of-scope}. These questions assume the presence of entities or events not actually shown in the video, asking about properties of these nonexistent elements. It tests the model's ability to handle multimodal hallucinations and its tendency to make assertive, yet incorrect, responses.

% \noindent{Example}: A video showing a dog and its owner playing in the park, and there is no car visible in the video.\\
% Q: \textit{What color is the car?} \\
% A: \textit{There is no car (in the scene).}
% \textbf{Planning}. These questions inquire about the sequence of actions needed to achieve a specific goal related to the video. It assesses the model's ability to infer and plan the necessary steps based on the visual cues provided in the video.

% % Q: \textit{What should the dog do to continue playing with the owner?}\\
% % A: \textit{The dog should run towards the ball, jump onto the bench, grab the ball and return it to the owner.}
% \textbf{Sentiment}. These questions focus on the sentiment, mood, attitude, or emotions of characters towards other entities or events in the video. This category evaluates the model's ability to recognize and identify the emotional state or attitude of characters based on visual cues, reflecting their reactions or feelings toward the events and entities in the video.

% % \noindent{Example}: Suppose a video shows children bored at a birthday party.\\
% % Q: \textit{What is the attitude of the children?}\\
% % A: \textit{They are bored.}
% \textbf{Spatial}. These questions focus on the location of entities in space, either applicable to the entire video (\textit{Total Spatial}) or specific moments and events (\textit{Partial Spatial}). This category assesses the model’s ability to infer stable and time-dependent spatial relationships, determine relative positioning, and demonstrate grounding competencies.

% % \noindent{Example A}:\\
% % Q: \textit{Where is the teacher?}\\
% % A: \textit{The teacher is in the classroom.}\\
% % \noindent{Example B}:\\
% % Q: \textit{Where is the teacher at the beginning of the video?}\\
% % A: \textit{At the beginning of the video, the teacher is standing in front of their desk.}
% \textbf{Temporal}. These questions focus on when something happens, either in relation to other events (\textit{Partial Temporal}) or the duration of an event (\textit{Duration}). It evaluates the model's ability to infer temporal relationships, event sequences, and durations from visual content in a coherent manner.
% % \noindent{Example A}:\\
% % Q: \textit{What happens after the guitarist starts playing?}
% % A: \textit{The singer starts singing.}\\
% % \noindent{Example B}:\\
% % Q: \textit{How long was the light on?}\\
% % A: \textit{For about 15 seconds.}
% \textbf{Uncertainty}. These questions arise when insufficient information is provided in the video to give a precise answer. This category tests the model's ability to recognize and handle situations with ambiguous or incomplete information, indirectly assessing its tendency to make assertive (rather than uncertain) responses.

% % \noindent{Example}:\\
% % Q: \textit{How old is the dog?}\\
% % A: \textit{The dog is probably young, but it is not certain.}



  







\section{Fine-Tuning Experiments}
This section validates that our dataset can enhance the GUI grounding capabilities of VLMs and that the proposed functionality grounding and referring are effective fine-tuning tasks.
\subsection{Experimental Settings}
\noindent\textbf{Evaluation Benchmarks} We base our evaluation on the UI grounding benchmarks for various scenarios: \textbf{FuncPred} is the test split from our collected functionality dataset. This benchmark requires a model to locate the element specified by its functionality description. \textbf{ScreenSpot}~\citep{cheng2024seeclick} is a benchmark comprising test samples on mobile, desktop, and web platforms. It requires the model to locate elements based on short instructions. \textbf{RefExp}~\citep{Bai2021UIBertLG} is to locate elements given crowd-sourced referring expressions. \textbf{VisualWebBench (VWB)}~\citep{liu2024visualwebbench} is a comprehensive multi-modal benchmark assessing the understanding capabilities of VLMs in web scenarios. We select the element and action grounding tasks from this benchmark. To better align with high-level semantic instructions for potential agent requirements and avoid redundancy evaluation with ScreenSpot, we use ChatGPT to expand the OCR text descriptions in the original task instructions, such as \textit{Abu Garcia College Fishing} into functionality descriptions like \textit{This element is used to register for the Abu Garcia College Fishing event}.
\textbf{MOTIF}~\citep{Burns2022ADF} requires an agent to complete a natural language command in mobile Apps.
For all of these benchmarks, we report the grounding accuracy (\%): $\text { Acc }= \sum_{i=1}^N \mathbf{1}\left(\text {pred}_i \text { inside GT } \text {bbox}_i\right) / N \times 100 $ where $\mathbf{1}$ is an indicator function and $N$ is the number of test samples. This formula denotes the percentage of samples with the predicted points lying within the bounding boxes of the target elements.

\noindent\textbf{Training Details}
We select Qwen-VL-10B~\citep{bai2023qwen} and SliME-8B~\citep{slime} as the base models and fine-tune them on 25k, 125k, and 702k samples of the AutoGUI training data to investigate how the AutoGUI data enhances the UI grounding capabilities of the VLMs. The models are fine-tuned on 8 A100 GPUs for one epoch. We follow SeeClick~\citep{cheng2024seeclick} to fine-tune Qwen-VL with LoRA~\citep{hu2022lora} and follow the recipe of SliME~\citep{slime} to fine-tune it with only the visual encoder frozen (More details in Sec.~\ref{sec:supp:impl details}).

\noindent\textbf{Compared VLMs}
We compare with both general-purpose VLMs (i.e., LLaVA series~\citep{liu2023llava,liu2024llavanext}, SliME~\citep{slime}, and Qwen-VL~\citep{bai2023qwen}) and UI-oriented ones (i.e., Qwen2-VL~\citep{qwen2vl}, SeeClick~\citep{cheng2024seeclick}, CogAgent~\citep{hong2023cogagent}). SeeClick finetunes Qwen-VL with around 1 million data combining various data sources, including a large proportion of human-annotated UI grounding/referring samples. CogAgent is trained with a huge amount of text recognition, visual grounding, UI understanding, and publicly available text-image datasets, such as LAION-2B~\citep{LAION5B}. During the evaluation, we manually craft grounding prompts suitable for these VLMs.
\subsection{Experimental Results and Analysis}
\begin{table}[]
\scriptsize
\centering
\caption{\textbf{Element grounding accuracy on the used benchmarks.} We compare the base models fine-tuned with our AutoGUI data and representative open-source VLMs. The results show that the two base models (i.e. Qwen-VL and SliME-8B) obtain significant performance gains over the benchmarks after being fine-tuned with AutoGUI data. Moreover, increasing the AutoGUI data size consistently improves grounding accuracy, demonstrating notable scaling effects. $\dag$ means the metric value is borrowed from the benchmark paper. $*$ means using additional SeeClick training data.}
\label{tab:eval results}
\begin{tabular}{@{}cccccccccc@{}}
\toprule
Type & Model    & Size    & FuncPred & VWB EG & VWB AG & MoTIF & RefExp & ScreenSpot  \\ \midrule
\multirow{5}{*}{General} & LLaVA-1.5~\citep{liu2023llava} & 7B & 3.2      &        12.1$^{\dag}$        &     13.6$^{\dag}$           &  7.2   &  4.2 & 5.0 & \\
 & LLaVA-1.5~\citep{liu2023llava} & 13B & 5.8      &           16.7     &        9.7        &   12.3 &  20.3   & 11.2 &  \\
 & LLaVA-1.6~\citep{liu2024llavanext} & 34B &  4.4      &      19.9          &    17.0            &   7.0 &  29.1  & 10.3 &  \\
 & SliME~\citep{slime} & 8B &  3.2  &   6.1       &     4.9     & 7.0  &  8.3  &  13.0  \\ 

 & Qwen-VL~\citep{bai2023qwen} & 10B &  3.0     &      1.7          &      3.9          &    7.8 &  8.0  & 5.2$^{\dag}$   \\ 
 \midrule
\multirow{3}{*}{UI-VLM} &  Qwen2-VL~\citep{bai2023qwen}  & 7B     &     7.8       &    3.9        &  3.9  &  16.7 & 32.4 & 26.1    \\
 & CogAgent~\citep{hong2023cogagent} & 18B    &  29.3   &    \underline{55.7}      &    \textbf{59.2}      & \textbf{24.7}   & 35.0 &  47.4$^{\dag}$  \\
 & SeeClick~\citep{cheng2024seeclick} & 10B    &    19.8     &    39.2           &     27.2           & 11.1  &  \textbf{58.1}  & \underline{53.4}$^{\dag}$ \\ 
\midrule
\multirow{4}{*}{Finetuned} &  Qwen-VL-AutoGUI25k & 10B      &    14.2     &      12.8         &    12.6           &   10.8    &  12.0 & 19.0    \\
 & Qwen-VL-AutoGUI125k  & 10B       &     25.5     &      23.2         &        29.1       &    11.5   &  14.9 & 32.0     \\ 
 & Qwen-VL-AutoGUI702k  & 10B       &   43.1   &    38.0       &     32.0    &  15.5  & 23.9 &    38.4   \\
& Qwen-VL-AutoGUI702k$^*$   & 10B     &  \underline{50.0}  &    \textbf{56.2}    &  \underline{45.6}  & \underline{21.0} & \underline{51.5} & \textbf{54.2}      \\
\midrule
\multirow{3}{*}{Finetuned} & SliME-AutoGUI25k  & 8B     &   28.0   &     14.0      &      10.6      &  14.3   & 18.4 & 27.2   \\
 & SliME-AutoGUI125k   & 8B      &   39.9    &  22.0   &     12.0       &  17.8  & 22.1 &  35.0     \\
 & SliME-AutoGUI702k   & 8B      &     \textbf{62.6}   &       25.4        &     13.6          &   20.6    & 26.7 & 44.0 &          \\
\bottomrule
\end{tabular}
\end{table}
\vspace{-2mm}


\noindent\textbf{A) AutoGUI functionality annotations effectively enhance VLMs' UI grounding capabilities and achieve scaling effects.} We endeavor to show that the element functionality data autonomously collected by AutoGUI contributes to high grounding accuracy. The results in Tab.~\ref{tab:eval results} demonstrate that on all benchmarks the two base models achieve progressively rising grounding accuracy as the functionality data size scales from 25k to 702k, with SliME-8B's accuracy increasing from merely \textbf{3.2} and \textbf{13.0} to \textbf{62.6} and \textbf{44.0} on FuncPred and ScreenSpot, respectively. This increase is visualized in Fig.~\ref{fig:funcpred scaling success} showing that increasing AutoGUI data amount leads to more precise localization performance.

After fine-tuning with AutoGUI 702k data, the two base models surpass SeeClick, the strong UI-oriented VLM on FuncPred and MOTIF. We notice that the base models lag behind SeeClick and CogAgent on ScreenSpot and RefExp, as the two benchmarks contain test samples whose UIs cannot be easily recorded (e.g., Apple devices and Desktop software) as training data, causing a domain gap. Nevertheless, SliME-8B still exhibits noticeable performance improvements on ScreenSpot and RefExp when scaling up the AutoGUI data, suggesting that the AutoGUI data helps to enhance grounding accuracy on the out-of-domain tasks.

To further unleash the potential of the AutoGUI data, the base model, Qwen-VL, is finetuned with the combination of the AutoGUI and SeeClick UI-grounding data. This model becomes the new state-of-the-art on FuncPred, ScreenSpot, and VWB EG, surpassing SeeClick and CogAgent. This result suggests that our AutoGUI data can be mixed with existing UI grounding training data to foster better UI grounding capabilities.

In summary, our functionality data can endow a general VLM with stronger UI grounding ability and exhibit clear scaling effects as the data size increases.


\begin{table}[]
\centering
\footnotesize
\caption{\textbf{Comparing the AutoGUI functionality annotation type with existing types}. Qwen-VL is fine-tuned with the three annotation types. The results show that our functionality data leads to superior grounding accuracy compared with the naive element-HTML data and the condensed functionality annotations.}
\label{tab:ablation}
\begin{tabular}{@{}ccccc@{}}
\toprule
Data Size             & Variant          & FuncPred & RefExp & ScreenSpot \\ \midrule
\multirow{3}{*}{25k}  & w/ Elem-HTML data     &  5.3      &  4.5   &    5.7     \\
                      & w/ Condensed Func. Anno.     &  3.8   &  3.0  &   4.8      \\
                      & w/ Func. Anno. (Ours full)         &    \textbf{21.1}    &   \textbf{10.0}   &   \textbf{16.4}    \\ \midrule
\multirow{3}{*}{125k} & w/ Elem-HTML data     &  15.5   &  7.8  &   17.0      \\
                      & w/ Condensed Func. Anno.     &  14.1   &  11.7  &   23.8      \\
                      & w/ Func. Anno. (Ours full)         &  \textbf{24.6}   &  \textbf{12.7}  &   \textbf{27.0}    \\ \bottomrule
\end{tabular}
\end{table}



\noindent\textbf{B) Our functionality annotations are effective for enhancing UI grounding capabilities.} To assess the effectiveness of functionality annotations, we compare this annotation type with two existing types: 1) \textbf{Naive element-HTML pairs}, which are directly obtained from the UI source code~\citep{hong2023cogagent} and associate HTML code with elements in specified areas of a screenshot. Examples are shown in Fig.~\ref{fig: functionality vs others}. To create these pairs, we replace the functionality annotations with the corresponding HTML code snippets recorded during trajectory collection. 2) \textbf{Brief functionality descriptions} that are generated by prompting GPT-4o-mini\footnote{https://openai.com/index/gpt-4o-mini-advancing-cost-efficient-intelligence/} to condense the AutoGUI functionality annotations. For example, a full description such as \textit{`This element provides access to a documentation category, allowing users to explore relevant information and guides'} is shortened to \textit{`Documentation category access'}.

After experimenting with Qwen-VL~\citep{bai2023qwen} at the 25k and 125k scales, the results in Tab.~\ref{tab:ablation} show that fine-tuning with the complete functionality annotations is superior to the other two types. Notably, our functionality annotation type yields the largest gain on the challenging FuncPred benchmark that emphasizes contextual functionality grounding. In contrast, the Elem-HTML type performs poorly due to the noise inherent in HTML code (e.g., numerous redundant tags), which reduces fine-tuning efficiency. The condensed functionality annotations are inferior, as the consensing loses details necessary for fine-grained UI understanding. In summary, the AutoGUI functionality annotations provide a clear advantage in enhancing UI grounding capabilities.


\subsection{Failure Case Analysis}
After analyzing the grounding failure cases, we identified several failure patterns in the fine-tuned models: a) difficulty in accurately locating small elements; b) challenges in distinguishing between similar but incorrect elements; and c) issues with recognizing icons that have uncommon shapes. Please refer to Sec.~\ref{sec:supp:case analysis} for details.




% , comparing our results against state-of-the-art image-to-image translation methods
% We evaluate our method through editing experiments conducted on two experiments. In \cref{sec:5.1}, we perform a comparison on image-to-image editing across several datasets. In \cref{sec:5.2}, we extend our evaluation to editable Neural Radiance Fields (NeRF) \cite{mildenhall2021nerf}, demonstrating the efficacy of our approach for 3D image editing and providing a comparative analysis with existing techniques.
% result tables

\section{Results} \label{sec:results}
We evaluate our method through editing experiments conducted on two experiments. In \cref{sec:5.1}, we perform a comparison on image-to-image editing across several datasets. In \cref{sec:5.2}, we extend our evaluation to editable Neural Radiance Fields (NeRF) \cite{mildenhall2021nerf}.

\subsection{Text-guided image editing}
\label{sec:5.1}
\noindent\textbf{Baselines.} To evaluate our method, we conduct comparative experiments against four state-of-the-art image editing models: Prompt-to-Prompt (P2P) \cite{hertzprompt}, Plug-and-Play (PNP) \cite{tumanyan2023plug}, DDS \cite{hertz2023delta}, and CDS \cite{nam2024contrastive}. The implementations of the baselines are carried out by referencing the official source code for each method. More details are provided in \cref{sec:s_implement} of Supplementary Materials.

\noindent\textbf{Qualitative Results.} We present the qualitative results comparing our method with the baselines in \cref{fig:ip2p_qual}. Prompt-to-Prompt (P2P) \cite{hertzprompt} performs image editing after applying DDIM inversion \cite{dhariwal2021diffusion, song2020denoising} to the source image, leading to disregarding the structural components of the source image and following the target prompt excessively. Plug-and-Play (PnP) \cite{tumanyan2023plug} has limitations in object recognition, as seen in the fourth row of Fig.~\ref{fig:ip2p_qual}. The third row of Fig.~\ref{fig:ip2p_qual} demonstrates that DDS \cite{hertz2023delta} and CDS \cite{nam2024contrastive} exhibited limitations, particularly in preserving the structural characteristics of the source image. In contrast, our method successfully edits the image while preserving the structural integrity of the source image.
% exhibit limitations such as failing to maintain the handle length and saddle shape of the bike in the first row and being unable to preserve the structure of the shark in the second row. %Furthermore, as seen in the third and fourth rows, the details in the edited target areas lacked refinement, and in the last row, the color of the source image was not preserved. In contrast, our method successfully edits the image aligning with the target text prompt while preserving the structural integrity of the source image.

\noindent\textbf{Quantitative Results.} 
% We employed two datasets: LAION 5B \cite{schuhmann2022laion} and InstructPix2Pix \cite{brooks2023instructpix2pix}.
% ##ORIGINAL## To measure the identity-preserving performance, we utilize two datasets. First, we collect 250 cat images from the LAION 5B dataset \cite{schuhmann2022laion} based on \cite{nam2024contrastive} for \textit{Cat-to-Others} task. We measure Intersection over Union (IoU) to evaluate how much of the area of the source object has been preserved. Second, we gather 28 images from the InstructPix2Pix (IP2P) dataset \cite{brooks2023instructpix2pix}, which contains the pairs of source and target images and corresponding prompts. We calculate the background Peak-Signal-to-Noise-Ratio (PSNR) to assess how the identity of the source image is preserved after editing. In addition, we use the LPIPS score \cite{zhang2018unreasonable} for each experiment to quantify the similarity between source and target images. The results are presented in \cref{tab:2Dquan}. Our method consistently achieves the lowest LPIPS score across all datasets, indicating that it best preserves the structural semantics of the source images. 
To measure the identity-preserving performance, we utilize two datasets. First, we collect 250 cat images from the LAION 5B dataset \cite{schuhmann2022laion} based on \cite{nam2024contrastive} for \textit{Cat-to-Others} task and measure Intersection over Union (IoU). Second, we gather 28 images from the InstructPix2Pix (IP2P) dataset \cite{brooks2023instructpix2pix}, which contains the pairs of source and target images and corresponding prompts and calculate the background Peak-Signal-to-Noise-Ratio (PSNR). Details of the metrics are provided in Supplementary Materials \cref{sec:s_evalmetric}. In addition, we use the LPIPS score \cite{zhang2018unreasonable} for each experiment to quantify the similarity between source and target images. The results are presented in \cref{tab:2Dquan}. Our method consistently achieves the lowest LPIPS score across all datasets, indicating that it best preserves the structural semantics of the source images. 
% We collect 250 images of cats from the LAION 5B dataset \cite{schuhmann2022laion} based on \cite{nam2024contrastive} for \textit{Cat-to-Others} task and 28 images from the InstructPix2Pix dataset \cite{brooks2023instructpix2pix} following the regulations. To evaluate the images translated by each method, we measure Intersection over Union (IoU) on LAION 5B, which primarily consists of object-focused data. We also measure the background PSNR on InstructPix2Pix to assess the extent to which the source image’s identity is preserved after editing. The results are presented in \cref{tab:2Dquan}. 
% Our method consistently achieves the lowest LPIPS score across all datasets, indicating that it best preserves the structural semantics of the source images. 
\begin{table}[b]
\centering
\resizebox{0.98\columnwidth}{!}{
\small{
\begin{tabular}{c|cc|cc|cc}
\hline
& \multicolumn{2}{c|}{cat2pig} & \multicolumn{2}{c|}{cat2squirrel} & \multicolumn{2}{c}{Ip2p}  \\ 
\hline
\multicolumn{1}{c|}{Metric} & IoU ($\uparrow$) & LPIPS ($\downarrow$) & IoU ($\uparrow$) & LPIPS ($\downarrow$) & PSNR ($\uparrow$) & LPIPS ($\downarrow$) \\ 
\hline
P2P \cite{hertzprompt}& 0.58 & 0.42 & 0.52 & 0.46 & 20.88 & 0.47 \\
PnP \cite{tumanyan2023plug}& 0.55 & 0.52 & 0.53 & 0.52 & 23.81 & 0.39 \\
DDS \cite{hertz2023delta}& 0.69 & 0.28 & 0.65 & 0.30 & 26.02 & 0.24 \\  
CDS \cite{nam2024contrastive}& 0.72 & 0.25 & \textbf{0.71} & 0.26 & 27.35 & 0.21 \\
\hline
\textbf{IDS (Ours)} & \textbf{0.74} & \textbf{0.22} & \textbf{0.71} & \textbf{0.24} & \textbf{29.25} & \textbf{0.19} \\
\hline
\end{tabular}
}
}
\vspace{-5pt}
\caption{\textbf{Quantitative results} for image editing. LPIPS \cite{zhang2018unreasonable} and IoU was measured on LAION 5B \cite{schuhmann2022laion}, while LPIPS and background PSNR was measured on InstructPix2Pix \cite{brooks2023instructpix2pix}.}
\label{tab:2Dquan}
\end{table}




%P2P \cite{hertzprompt}& 0.5798 & 0.4229 & 0.5184 & 0.4605 & 20.88 & 0.4695 \\
%PnP \cite{tumanyan2023plug}& 0.5507 & 0.5191 & ??? & 0.5245 & 23.81 & 0.3882 \\
%DDS \cite{hertz2023delta}& 0.6897 & 0.2838 & 0.6456 & 0.2996 & 26.02 & 0.2398 \\  
%CDS \cite{nam2024contrastive}& 0.7249 & 0.2485 & 0.7054 & 0.2612 & 27.35 & 0.2099 \\

\begin{table}[bh!]
\vspace{-5pt}
\centering
%\scalebox{0.65}
\resizebox{1.0\columnwidth}{!}{
%\small{ %
\begin{tabular}{c|ccc|ccc}
\hline
& \multicolumn{3}{c|}{User Preference Rate (\%)} & \multicolumn{3}{c}{GPT score \cite{peng2024dreambench++}}\\ 
\hline
\multicolumn{1}{c|}{Metric} & Text ($\uparrow$) & Preserving ($\uparrow$) & Quality ($\uparrow$) & Text ($\uparrow$) & Preserving ($\uparrow$) & Quality ($\uparrow$) \\ 
\hline
P2P \cite{hertzprompt}& 11.13 & 4.80 & 8.09 & 5.66 & 5.37 & 5.77 \\
PnP \cite{tumanyan2023plug}& 7.72 & 7.17 & 6.93 & 6.54 & 6.77 & 6.74 \\
DDS \cite{hertz2023delta}& 20.30 & 10.82 & 16.23 & 7.60 & 7.51 & 7.37 \\
CDS \cite{nam2024contrastive}& 17.02 & 16.72 & 17.08 & 8.26 & 8.00 & 8.09 \\ 
\hline
\textbf{IDS (Ours)} & \textbf{43.83} & \textbf{60.49} & \textbf{51.67} & \textbf{8.97} & \textbf{9.00} & \textbf{8.80} \\
\hline
\end{tabular}
}
%}
\vspace{-5pt}
\caption{\textbf{User study and GPT scores}  \cite{peng2024dreambench++} show that our method achieved the highest scores across all questions for image editing.}
\label{tab:Userstudy_GPTscore}
\end{table}
For user evaluation, we present 35 comparison sets for four baselines and our method, gathering responses from 47 participants. Participants are asked to choose the most appropriate image for the following three questions: 1. \textit{Which image best fits the text condition?} 2. \textit{Which image best preserves the structural information of the original image?} 3. \textit{Which image has the best quality for text-based image editing?} 
Additionally, we measure the GPT score using the Dreambench++ \cite{peng2024dreambench++} method, which generates human-aligned assessments for the same questions by refining the scoring into ten distinct levels. As shown in \cref{tab:Userstudy_GPTscore}, our method receives the highest ratings for all questions.
% Furthermore, we ask users to select their favorite image from the baselines in order to gauge their preferences, and we compute the selected ratio in percentage terms.
%While our CLIP score was not significantly higher than other methods, it remained comparable. %Considering the outcomes of both metrics, our model demonstrates an ability to maximally preserve the source image's structure during the editing process while minimally and precisely transforming the regions specified by the target prompt.

% Fig 5.2



%%% [START] NeRF Synthetic data Results 
\begin{figure*}[t] % 2-column
\footnotesize
\centering 
% 1st row
\hspace{-3mm}
\raisebox{0.5in}{\rotatebox{90}{\textbf{Synthetic} \cite{mildenhall2021nerf}}}%
\hspace{3mm}%
\begin{tikzpicture}[x=3.5cm, y=3.5cm, spy using outlines={every spy on node/.append style={thick, draw=red}}]
\node[anchor=south] (FigA) at (0,0) {\includegraphics[trim=0 0 0 0 ,clip,width=1.5in]{Fig./Qual/imgs/3D/ficus/cropped_r_3.png}};
\node[anchor=south, yshift=0mm] at (FigA.north) {\footnotesize Source};
% ->
\draw[->, line width=0.8mm, color=red, shorten >=1pt, shorten <=1pt] ($(FigA.center) + (0.15, -0.18)$) -- ($(FigA.center) + (0, -0.3)$);
\end{tikzpicture}
\hspace{-1mm}
\begin{tikzpicture}[x=3.5cm, y=3.5cm, spy using outlines={every spy on node/.append style={thick, draw=red}}]
\node[anchor=south] (FigD) at (0,0) {\includegraphics[trim=0 0 0 0 ,clip,width=1.5in]{Fig./Qual/imgs/3D/ficus/FPDS_cropped_r_3.png}};
\node[anchor=south, yshift=0mm] at (FigD.north) {\footnotesize \textbf{IDS (Ours)}};
% ->
\draw[->, line width=0.8mm, color=red, shorten >=1pt, shorten <=1pt] ($(FigA.center) + (0.15, -0.18)$) -- ($(FigA.center) + (0, -0.3)$);
\end{tikzpicture}
\hspace{-1mm}
\begin{tikzpicture}[x=3.5cm, y=3.5cm, spy using outlines={every spy on node/.append style={thick, draw=red}}]
\node[anchor=south] (FigC) at (0,0) {\includegraphics[trim=0 0 0 0 ,clip,width=1.5in]{Fig./Qual/imgs/3D/ficus/CDS_cropped_r_3.png}};
\node[anchor=south, yshift=0mm] at (FigC.north) {\footnotesize CDS};
% ->
\draw[->, line width=0.8mm, color=red, shorten >=1pt, shorten <=1pt] ($(FigA.center) + (0.15, -0.18)$) -- ($(FigA.center) + (0, -0.3)$);
\end{tikzpicture}
\hspace{-1mm}
\begin{tikzpicture}[x=3.5cm, y=3.5cm, spy using outlines={every spy on node/.append style={thick, draw=red}}]
\node[anchor=south] (FigB) at (0,0) {\includegraphics[trim=0 0 0 0 ,clip,width=1.5in]{Fig./Qual/imgs/3D/ficus/DDS_cropped_r_3.png}};
\node[anchor=south, yshift=0mm] at (FigB.north) {\footnotesize DDS};
% ->
\draw[->, line width=0.8mm, color=red, shorten >=1pt, shorten <=1pt] ($(FigA.center) + (0.15, -0.18)$) -- ($(FigA.center) + (0, -0.3)$);
\end{tikzpicture}

\vspace{-4pt}

\setulcolor{magenta}
\setul{0.3pt}{2pt}
\centering \textit{``A tree in a brown vase" $\to$ ``A tree in a \ul{blue} vase"} 

\vspace{-2pt}

% 2nd row
\hspace{-3mm}
\raisebox{0.37in}{\rotatebox{90}{\textbf{LLFF} \cite{mildenhall2019local} }}%
\hspace{3mm}%
\begin{tikzpicture}[x=3.5cm, y=3.5cm, spy using outlines={every spy on node/.append style={thick, draw=white}}]
\node[anchor=south] (FigA2) at (0,0) {\includegraphics[trim=0 0 0 0 ,clip,width=1.5in]{Fig./Qual/imgs/3D/autumn/original_image009.jpg}};
\spy [magnification=3, size=0.6in] on ($(FigA2.center) + (0.05, 0.05)$) in node [anchor=south west] at ($(FigA2.south west)$);
\end{tikzpicture}
\hspace{-1mm}
\begin{tikzpicture}[x=3.5cm, y=3.5cm, spy using outlines={every spy on node/.append style={thick, draw=white}}]
\node[anchor=south] (FigD2) at (0,0) {\includegraphics[trim=0 0 0 0 ,clip,width=1.5in]{Fig./Qual/imgs/3D/autumn/FPDS_4032_IMG_3006.jpg}};
\spy [magnification=3, size=0.6in] on ($(FigD2.center) + (0.05, 0.05)$) in node [anchor=south west] at ($(FigD2.south west)$);
\end{tikzpicture}
\hspace{-1mm}
\begin{tikzpicture}[x=3.5cm, y=3.5cm, spy using outlines={every spy on node/.append style={thick, draw=white}}]
\node[anchor=south] (FigC2) at (0,0) {\includegraphics[trim=0 0 0 0 ,clip,width=1.5in]{Fig./Qual/imgs/3D/autumn/CDS_4032_IMG_3006.jpg}};
\spy [magnification=3, size=0.6in] on ($(FigC2.center) + (0.05, 0.05)$) in node [anchor=south west] at ($(FigC2.south west)$);
\end{tikzpicture}
\hspace{-1mm}
\begin{tikzpicture}[x=3.5cm, y=3.5cm, spy using outlines={every spy on node/.append style={thick, draw=white}}]
\node[anchor=south] (FigB2) at (0,0) {\includegraphics[trim=0 0 0 0 ,clip,width=1.5in]{Fig./Qual/imgs/3D/autumn/DDS_4032_IMG_3006.jpg}};
\spy [magnification=3, size=0.6in] on ($(FigB2.center) + (0.05, 0.05)$) in node [anchor=south west] at ($(FigB2.south west)$);
\end{tikzpicture}

% 3rd row
\hspace{-3mm}
\raisebox{0.3in}{\rotatebox{90}{\textbf{Depth Map}}}%
\hspace{3mm}%
\hspace{0mm}
\begin{tikzpicture}[x=3.5cm, y=3.5cm, spy using outlines={every spy on node/.append style={thick, draw=white}}]
\node[anchor=south] (FigA3) at (0,0) {\includegraphics[trim=0 0 0 0 ,clip,width=1.5in]{Fig./Qual/imgs/3D/autumn/depth_map/original_depth_088.jpg}};
\end{tikzpicture}
\hspace{-1mm}
\begin{tikzpicture}[x=3.5cm, y=3.5cm, spy using outlines={every spy on node/.append style={thick, draw=white}}]
\node[anchor=south] (FigD3) at (0,0) {\includegraphics[trim=0 0 0 0 ,clip,width=1.5in]{Fig./Qual/imgs/3D/autumn/depth_map/FPDS_depth_088.jpg}};
\end{tikzpicture}
\hspace{-1mm}
\begin{tikzpicture}[x=3.5cm, y=3.5cm, spy using outlines={every spy on node/.append style={thick, draw=white}}]
\node[anchor=south] (FigC3) at (0,0) {\includegraphics[trim=0 0 0 0 ,clip,width=1.5in]{Fig./Qual/imgs/3D/autumn/depth_map/CDS_depth_088.jpg}};
\end{tikzpicture}
\hspace{-1mm}
\begin{tikzpicture}[x=3.5cm, y=3.5cm, spy using outlines={every spy on node/.append style={thick, draw=white}}]
\node[anchor=south] (FigB3) at (0,0) {\includegraphics[trim=0 0 0 0 ,clip,width=1.5in]{Fig./Qual/imgs/3D/autumn/depth_map/DDS_depth_088.jpg}};
\end{tikzpicture}

\vspace{-1pt}
\centering \textit{``The green leaves" $\to$ ``\ul{Yellow and red} leaves in \ul{autumn}"} 

\vspace{-5pt}
\caption{\textbf{Qualitative results on Synthetic 360$^\circ$ and LLFF datasets.} IDS outperforms the baselines by preserving the structural consistency of the source image and maintaining the integrity of regions that should remain unchanged, while precisely editing only the areas specified by the target prompt. Furthermore, comparisons of the depth map results also highlight the superior consistency of our method over other baseline models.}
\label{fig:ficus_qual}
\end{figure*}
% \vspace{-10pt}
\subsection{Editing NeRF}
We conduct experiments involving 3D rendering of edited images to demonstrate the effectiveness of our method in maintaining structural consistency. This approach is particularly relevant as consistency has an even greater impact on outcomes in 3D environments.

\label{sec:5.2}

\noindent\textbf{Datasets.} We evaluated our method on widely used NeRF datasets: Synthetic NeRF \cite{mildenhall2021nerf} and LLFF \cite{mildenhall2019local}. Since NeRF datasets have no given pairs of source and target prompts, we manually composed image descriptions.
%, such as the source prompt ``A tree in a brown vase" and its corresponding target prompt ``A tree in a blue vase" as shown in \cref{fig:ficus_qual}.

\noindent\textbf{Qualitative Results.} \cref{fig:ficus_qual} illustrates the qualitative results of our method compared with NeRF editing baselines. In the first row, the target prompt specifies a precise part of the image for fine-grained editing. DDS \cite{hertz2023delta} and CDS \cite{nam2024contrastive} fail to differentiate and edit the specific area. At the same time, our method accurately identifies the region indicated by the target prompt in the image and performs detailed editing exclusively on that part. 
The second row demonstrates a scenario in which the target prompt is designed to edit the mood of the image. Our approach adjusts the colors associated with ``autumn" and ``leaves" throughout the image while maintaining consistency in the ``trunk" whereas DDS and CDS also changed the ``trunk". In terms of depth maps, our method generates clean depth maps with minimal noise after image editing, whereas DDS and CDS introduce noticeable noise into the depth maps.

%the overall mood of the image on the LLFF dataset \cite{mildenhall2019local}
 % give an attention solely on following the target prompt during editing, leading to unintended alterations of parts that should remain unchanged.
 % Comparing the NeRF depth maps with baselines, 
% \cref{fig:ficus_qual} illustrates the qualitative results of our method compared with NeRF editing baselines such as DDS \cite{hertz2023delta} and CDS \cite{nam2024contrastive}. In the first row, the target prompt specifies a precise part of the image for fine-grained editing on the Synthetic NeRF dataset \cite{mildenhall2021nerf}. Our method accurately identifies the region indicated by the target prompt in the image and performs detailed editing exclusively on that part. In contrast, DDS and CDS fail to differentiate and edit the specific area; they erroneously edit not only the ``vase" but also the ``soil", resulting in inappropriate edits. The second row demonstrates a scenario in which the target prompt is designed to edit the overall mood of the image on the LLFF dataset \cite{mildenhall2019local}, further highlighting the strengths of our method. Our approach adjusts the colors associated with ``autumn" and ``leaves" throughout the image while maintaining consistency in the ``trunk", which should be preserved from the source image. However, DDS and CDS focus solely on following the target prompt during editing, leading to unintended alterations of parts that should remain unchanged. Additionally, comparing the NeRF depth maps with baselines, our method generates clean outputs with minimal noise after image editing, whereas DDS and CDS introduce noticeable noise into the depth maps. 
% \vspace{-10pt}
% % Table for CLIP score
% \begin{table}[H]
% \centering
% \resizebox{0.9\columnwidth}{!}{
% \begin{tabular}{ccc}
% \toprule
% Metric & CLIP \cite{radford2021learning} score ($\uparrow$) & User Preference Rate ($\uparrow$) \\
% \midrule
% CDS \cite{nam2024contrastive}& $0.1597$ & $22.7$ \\
% DDS \cite{hertz2023delta}& $0.1596$ & $??$ \\
% \textbf{FPDS (ours)} & $\mathbf{0.1626}$ & $\mathbf{??}$ \\
% \bottomrule
% \end{tabular}
% }
% \caption{\textbf{Quantitative results of NeRF editing} comparing our method with other baselines for CLIP score and User Preference Rate on the NeRF LLFF dataset \cite{mildenhall2019local}. Higher CLIP scores and User Preference Rates indicate better performance.}
% \label{tab:Nerfclip}
% \end{table}
\begin{table}[thb!]
\centering
\resizebox{0.95\columnwidth}{!}{
\begin{tabular}{c|c|ccc}
\hline
\multirow{2}{*}{Metric} & \multirow{2}{*}{CLIP \cite{radford2021learning}  ($\uparrow$)} & \multicolumn{3}{c}{User Preference Rate (\%)} \\ 
\cline{3-5}
& & Text ($\uparrow$) & Preserving ($\uparrow$) & Quality ($\uparrow$) \\ 
\hline
DDS \cite{hertz2023delta}& 0.1596 & 36.88 & 28.37 & 32.62 \\
CDS \cite{nam2024contrastive}& 0.1597 & 22.70 & 23.40 & 21.28 \\
\hline
\textbf{IDS (Ours)} & \textbf{0.1626} & \textbf{40.42} & \textbf{48.23} & \textbf{46.10} \\
\hline
\end{tabular}
}
\caption{\textbf{Quantitative results of NeRF editing} with respect to CLIP score and User Preference Rate. IDS demonstrates superior quantitative performance compared to the baselines.}
\label{tab:Nerfclip}
\end{table}


\noindent\textbf{Quantitative Results.} Based on edited images, we performed 3D rendering and subsequently conducted quantitative evaluations provided in \cref{tab:Nerfclip}. To assess whether the edited 3D images are precisely aligned with the target prompts, we measured the CLIP \cite{radford2021learning} scores at 200k iterations of training on the LLFF dataset. We additionally present a user evaluation conducted under the same setup in \cref{sec:5.1}. Consistent with the trends observed in the qualitative results, our method demonstrates superior performance in the quantitative evaluations compared to other baselines.
%To demonstrate the effectiveness of our method in maintaining structural consistency during image editing and correcting errors progressively throughout training, we also conduct experiments involving 3D rendering of edited images. This approach is particularly relevant as consistency has an even greater impact on outcomes in 3D environments.



\section{Discussions}

% \subsection{Bridge the gap between insights and expressions}



\noindent\textbf{Bridge the gap between insights and expressions with AI-powered domain-focused video creation.}
% video creation for different domains
As images and videos continue to dominate communication mediums, visualization and video technologies have become essential tools for enabling diverse domains and the public to express themselves effectively. Emerging generative AI tools, such as Sora~\cite{sora} and Pika~\cite{pika}, exemplify this trend by facilitating creative expression across various fields.

While general AI-driven video creation tools are increasingly popular, our work emphasizes the critical need for domain-specific video creation tools like \SB{} to address unique requirements within specific fields. There are two primary reasons for prioritizing domain-specific video creation over general generative technologies.
% 
First, domain-specific videos, such as sports highlights, rely heavily on human insights. Audiences seek to learn from professionals through these videos, requiring tools that provide greater user control and enable experts to effectively translate their insights into engaging content. 
% \SB{} supports this by enabling users to maintain control over the conveyed insights, ensuring that the final video accurately reflects expert knowledge and user intentions.
% 
Second, the complexity of domain-specific data, such as the intricate motion and strategy analysis, demands advanced data visualization and seamless synchronization of visuals and audio, which general tools may not provide. 
% \SB{} addresses these needs by providing specialized tools that cater to the detailed and dynamic nature of sports content.

\SB{} addresses these needs by integrating automation with customizable visualizations, tailored to the intricate and dynamic nature of sports content. It allows flexible user control through embedded interactions, 
reducing technical barriers and empowering users to effectively communicate their insights. Feedback from users further underscores the importance of balancing automation with user control to accommodate diverse goals and preferences to enhance accessibility across various user groups and use cases, such as tactical analysis, skill development, and profile building. 
% For instance, professional coaches can use \SB{} to create detailed breakdowns of game strategies for training and coaching. Parents and young athletes can produce polished highlight reels for recruitment.
% These examples illustrate how AI-driven tools can empower users across various levels and industries to create videos with meaningful insights, fostering deeper engagement and broader impact. 

Beyond sports, similar tools have the potential to transform fields like healthcare and education, incorporating precise visual aids and step-by-step breakdowns. 
% These applications highlight the transformative potential of tailored video content in amplifying personal expression and benefiting broader audiences.
% 
Future research is required to investigate the balanced integration of AI and intuitive interface design, such as multi-modal interaction~\cite{wang2024lave}, to further advance domain-specific video creation and expression across diverse fields.
% By continuing to develop and refine domain-specific video creation tools, we can unlock new possibilities for effective communication and expression in numerous fields, ultimately bridging the gap between insights and their visual expressions.

% \subsection{Cross sports visualizations - allow different sports domains to leverage other sports' insights}

% \subsection{Enhance human-AI collaboration - creators focus on content while AI helps with editing tasks}


\vspace{1mm}
\noindent\textbf{Promote visualization in practice through real-world system deployment.}
Our work on SportsBuddy advances existing research in sports visualization and video authoring by emphasizing real-world system deployment and evaluation. Through this study, we have identified two significant benefits.

First, deploying SportsBuddy in authentic environments allowed us to validate and refine our design based on genuine use cases and users, uncovering insights that controlled laboratory settings cannot capture. For instance, we discovered that even within a similar user group of content creators, priorities varied significantly—some focused on showcasing player actions, while others emphasized strategic communication. This diversity led to iterative design improvements that balanced the distinct needs of each user group and support customization without complicating user interactions. 

Second, real-world deployment enables the assessment of long-term impacts and the discovery of unique use cases by diverse users. 
For example, some sports experts were hesitant to adopt SportsBuddy initially despite the perceived usefulness they shared. Upon further investigation, this was due to the context-switching costs. This feedback highlighted the necessity for a streamlined workflow tailored to the sports domain, leading to our design of batch processing and web import options. In addition, we observed many users preferred embedded annotation with \Text{} features over typical captions for sharing insights (see Fig.~\ref{fig:case_study}d), suggesting a new form of video storytelling inspired by \SB{}’s design. 
Feedback and insights from our diverse user base has highlighted the value of creating flexible and accessible visualization tools, which offers important external validity of the human-centered system.

This real-world deployment approach not only enhances visualization literacy and accessibility but also ensures that innovative designs translate into practical, widely usable tools, providing a validation for interactive visualization design. Therefore, we advocate for more visualization research to focus on real-world system deployments and to share design learnings, inspiring use cases that are both practical and impactful.

{
\subsection{Future Work}

While SportsBuddy has shown great potential in simplifying sports video storytelling, 
there are key areas for further improvement:

\vspace{1mm}
\noindent\textbf{Enhancing Player Tracking Under Occlusion and Motion Changes.}
The current tracking system faces challenges with occlusions and rapid motion in dynamic scenarios. Future work will refine tracking algorithms using larger domain-specific datasets and multi-view setups to improve accuracy in complex environments.

% The current tracking system struggles with occlusions and rapid motion changes in crowded or dynamic scenarios. Future efforts will focus on refining tracking algorithms using more extensive domain-specific datasets and, where feasible, incorporating multi-view camera setups for improved accuracy. These enhancements aim to ensure reliable tracking in complex sports environments.

\vspace{1mm}
\noindent\textbf{Addressing Perspective and Camera Movement.}
Shifts in camera angles or perspectives cause misalignment issues due to reliance on fixed transformation matrices. Dynamic court mapping and machine learning for real-time adjustments, along with camera metadata integration, will ensure consistent and accurate visualizations.

% Misalignment issues arise when camera angles or perspectives shift, as the system relies on a fixed transformation matrix. Future work will explore dynamic court mapping techniques and machine learning methods for real-time adjustments. Incorporating camera metadata will further enhance visualization accuracy, ensuring effects remain consistent with the game’s context.

\vspace{1mm}
\noindent\textbf{Supporting Longer Videos.}
Longer or higher-resolution videos can strain browser performance. To mitigate this, we will implement dynamic video loading from cloud storage and on-demand decoding, and adopt frame compression during previews to further optimize memory usage and rendering, ensuring smoother video processing.
% Longer or higher-resolution videos may strain browser performance. To address this, dynamic video loading from cloud storage and on-demand decoding will be introduced. Additionally, frame compression during previews will reduce memory usage and rendering time, enabling smoother processing of large and complex videos.



\vspace{1mm}
\noindent\textbf{Extending to Other Sports.}
\SB{} currently focuses on basketball but can expand to sports like soccer and tennis. This requires adapting tracking algorithms and designing sport-specific visualizations to accommodate the unique dynamics and storytelling needs of each sport.

}


% We advocate for more visualization paper that focus on deplyong system in real-world and evaluate their usage for two reasons. 
% 1. In vis research, application paper often address specific domain problems and create a prototype to evaluate with domain experts in a controlled setting. Most projects stop after user evaluation in the lab and the paper is published. With visualization system in real-world that value the practicality of system design and deployment in the wild, it encourages promoting real-world impact brought by novel visualization design, which is crucial in the current visualization community as we promote literacy and accessiblity of visualizations.
% 2. we should also promote long term impact of visualization design, and identify real-wordl use case and learning that might be drastically different from design study that are typically in lab, with a small amount of users, typically university students or academic members.



This work presented \ac{deepvl}, a Dynamics and Inertial-based method to predict velocity and uncertainty which is fused into an EKF along with a barometer to perform long-term underwater robot odometry in lack of extroceptive constraints. Evaluated on data from the Trondheim Fjord and a laboratory pool, the method achieves an average of \SI{4}{\percent} RMSE RPE compared to a reference trajectory from \ac{reaqrovio} with $30$ features and $4$ Cameras. The network contains only $28$K parameters and runs on both GPU and CPU in \SI{<5}{\milli\second}. While its fusion into state estimation can benefit all sensor modalities, we specifically evaluate it for the task of fusion with vision subject to critically low numbers of features. Lastly, we also demonstrated position control based on odometry from \ac{deepvl}.

\section*{Limitations and Future Directions}

We acknowledge that our most probable baseline, constructed using probabilities derived from our set of LLM's logits, poses a limitation that we intend to address in future work. We plan to compare each VLM with its corresponding LLM to obtain more reliable results for proper comparisons and analyses of potential statistical biases. Regarding Task $2$, we aim to dive deep into the comparison between the similarity metrics used and the emerging topic of LLMs as judges. Additionally, we intend to further investigate this latter approach to assess its actual validity as a reliable evaluation method. This direction will help us refine accuracy metrics, ultimately enhancing our ability to rigorously test model robustness and consistency across our two specular tasks. Furthermore, we acknowledge that our evaluation did not include large-scale proprietary models (e.g., ChatGPT). Our focus was primarily on testing the performance of open-source and easily exploitable language models to provide a comprehensive overview of their capabilities on the benchmark. Nevertheless, extending our evaluation to bigger models would be beneficial to gain a broader understanding of their performance and to contextualize our findings within the wider landscape of LLM research.

\section*{Acknowledgments}
This work has been carried out while Davide Testa was enrolled in the Italian National Doctorate on Artificial Intelligence run by Sapienza University of Rome in collaboration with Fondazione Bruno Kessler (FBK). Giovanni Bonetta and Bernardo Magnini were supported by the PNRR MUR project \href{https://fondazione-fair.it/}{PE0000013-FAIR} (Spoke 2). Alessandro Lenci was supported by the PNRR MUR project \href{https://fondazione-fair.it/}{PE0000013-FAIR} (Spoke 1). Alessio Miaschi was supported by the PNRR MUR project \href{https://fondazione-fair.it/}{PE0000013-FAIR} (Spoke 5). Lucia Passaro was supported by the EU EIC project EMERGE (Grant No. 101070918).
Alessandro Bondielli was supported by the Italian Ministry of University and Research (MUR) in the framework of the PON 2014-2021 ``Research and Innovation" resources – Innovation Action - DM MUR 1062/2021 - Title of the Research: ``Modelli semantici multimodali per l’industria 4.0 e le digital humanities.''



%\section{Bib\TeX{} Files}
%\label{sec:bibtex}

%ONLY IN CAMERA-READY
%\section*{Acknowledgments}
%FAIR project


% Bibliography entries for the entire Anthology, followed by custom entries
%\bibliography{anthology,custom}
% Custom bibliography entries only
\documentclass[preprint,12pt]{elsarticle} 
% \documentclass[preprint]{elsarticle}
% \documentclass[5p]{elsarticle}

% \usepackage[finalizecache,cachedir=.]{minted}
\usepackage[frozencache,cachedir=.]{minted}

\usepackage{graphicx}%
\usepackage{multirow}%
\usepackage{amsmath,amssymb,amsfonts}%
\usepackage{amsthm}%
\usepackage{mathrsfs}%
\usepackage[title]{appendix}%
\usepackage{xcolor}%
\usepackage{textcomp}%
\usepackage{manyfoot}%
\usepackage{booktabs}%
\usepackage{listings}%
\usepackage{hyperref}
%%%%
\usepackage{lipsum}
\usepackage[inline]{enumitem}
\usepackage[listings, minted, most]{tcolorbox}
\usepackage{etoolbox}
\usepackage{subcaption}

\usepackage{todonotes}
\usepackage{cleveref}
\crefname{equation}{Equation}{Equations}

\usepackage[ruled,vlined]{algorithm2e}
\renewcommand{\algorithmautorefname}{Algorithm}

\usepackage{makecell}

\setlist{noitemsep, nolistsep}
% \raggedbottom

\def\labelitemi{\textbf{--}}

\usetikzlibrary{shapes.geometric}

% \raggedbottom
%%\unnumbered% uncomment this for unnumbered level heads

\def\method{\text MixMin~}
\def\methodnospace{\text MixMin}
\def\genmethod{$\mathbb{R}$\text Min~}
\def\genmethodnospace{ $\mathbb{R}$\text Min}


\newcommand{\frameworkname}[0]{\protect\writings{PLANTOR}}

\journal{Robotics and Autonomous Systems}

\begin{document}

\begin{frontmatter}

% \title{LLM-Assisted Planning for Multi-Agent Systems}
\title{A Temporal Planning Framework for Multi-Agent Systems via LLM-Aided Knowledge Base Management}

\author[1]{Enrico Saccon\corref{cor1}}
\ead{enrico.saccon@unitn.it}
\author[1]{Ahmet Tikna}
\author[1]{Davide De Martini}
\author[1]{Edoardo Lamon}
\author[1]{Luigi Palopoli}
\author[1]{Marco Roveri}


\cortext[cor1]{Corresponding author}

\affiliation[1]{
    organization={Department of Engineering and Computer Science, University of Trento}, 
    city={Trento},
    country={Italy}
}

\begin{abstract}
    % Please provide an abstract of max 250. The abstract should not contain any undefined abbreviations or unspecified references. The abstract serves both as a general introduction to the topic and as a brief, non-technical summary of the main results and their implications. Authors are advised to check the author instructions for the journal they are submitting to for word limits and if structural elements like subheadings, citations, or equations are permitted.

  This paper presents a novel framework, called \frameworkname (PLanning with Natural language for Task-Oriented Robots), that integrates Large Language
  Models (LLMs) with Prolog-based knowledge management and planning
  for multi-robot tasks. The system employs a two-phase generation of
  a robot-oriented knowledge base, ensuring reusability and
  compositional reasoning, as well as a three-step planning procedure
  that handles temporal dependencies, resource constraints, and
  parallel task execution via mixed-integer linear programming. The
  final plan is converted into a Behaviour Tree for direct use in
  ROS2. We tested the framework in multi-robot assembly tasks within a
  block world and an arch-building scenario. Results demonstrate that
  LLMs can produce accurate knowledge bases with modest human
  feedback, while Prolog guarantees formal correctness and
  explainability. This approach underscores the potential of LLM
  integration for advanced robotics tasks requiring flexible,
  scalable, and human-understandable planning.

\end{abstract}

% \begin{graphicalabstract}
% \includegraphics[]{}  
% \end{graphicalabstract}

%% Required for RAS (https://www.sciencedirect.com/journal/robotics-and-autonomous-systems/publish/guide-for-authors)
%% Examples can be found here: https://www.elsevier.com/researcher/author/tools-and-resources/highlights
% \begin{highlights}
% \item Introduces \textbf{\frameworkname}(PLanning with Natural language for Task-Oriented Robots), a novel framework that integrates Large Language Models (LLMs) with Prolog-based knowledge management and planning for multi-robot systems.
% \item Employs a \emph{two-phase knowledge base generation} process using LLMs to create a structured Prolog knowledge base, ensuring \textit{reusability} and \emph{compositional reasoning}.
% \item Implements a \emph{three-step planning procedure} that handles temporal dependencies, resource constraints, and parallel task execution via mixed-integer linear programming, generating executable \emph{behavior trees}.
% \item Demonstrates the effectiveness of the framework in \emph{multi-robot assembly tasks}, showing that LLM-generated knowledge bases, with modest human feedback, can support scalable planning.
% \end{highlights}


%% From 1 to 7 max
\begin{keyword}
Task Planning \sep Knowledge Base \sep Multi-Agent Systems \sep Prolog \sep Large Language Models
\end{keyword}


\end{frontmatter}

% To be commented before sumbission
% \clearpage
% \tableofcontents

%%%%%%%%%%%%%%%%%%%%%%%%%%%%%%%%%%%%%%%%%%%%%%%%%%%%%%%%%%%%%%%%%%%%%%%%
\section{Introduction}\label{sec:intro}
\section{Introduction}

Despite the remarkable capabilities of large language models (LLMs)~\cite{DBLP:conf/emnlp/QinZ0CYY23,DBLP:journals/corr/abs-2307-09288}, they often inevitably exhibit hallucinations due to incorrect or outdated knowledge embedded in their parameters~\cite{DBLP:journals/corr/abs-2309-01219, DBLP:journals/corr/abs-2302-12813, DBLP:journals/csur/JiLFYSXIBMF23}.
Given the significant time and expense required to retrain LLMs, there has been growing interest in \emph{model editing} (a.k.a., \emph{knowledge editing})~\cite{DBLP:conf/iclr/SinitsinPPPB20, DBLP:journals/corr/abs-2012-00363, DBLP:conf/acl/DaiDHSCW22, DBLP:conf/icml/MitchellLBMF22, DBLP:conf/nips/MengBAB22, DBLP:conf/iclr/MengSABB23, DBLP:conf/emnlp/YaoWT0LDC023, DBLP:conf/emnlp/ZhongWMPC23, DBLP:conf/icml/MaL0G24, DBLP:journals/corr/abs-2401-04700}, 
which aims to update the knowledge of LLMs cost-effectively.
Some existing methods of model editing achieve this by modifying model parameters, which can be generally divided into two categories~\cite{DBLP:journals/corr/abs-2308-07269, DBLP:conf/emnlp/YaoWT0LDC023}.
Specifically, one type is based on \emph{Meta-Learning}~\cite{DBLP:conf/emnlp/CaoAT21, DBLP:conf/acl/DaiDHSCW22}, while the other is based on \emph{Locate-then-Edit}~\cite{DBLP:conf/acl/DaiDHSCW22, DBLP:conf/nips/MengBAB22, DBLP:conf/iclr/MengSABB23}. This paper primarily focuses on the latter.

\begin{figure}[t]
  \centering
  \includegraphics[width=0.48\textwidth]{figures/demonstration.pdf}
  \vspace{-4mm}
  \caption{(a) Comparison of regular model editing and EAC. EAC compresses the editing information into the dimensions where the editing anchors are located. Here, we utilize the gradients generated during training and the magnitude of the updated knowledge vector to identify anchors. (b) Comparison of general downstream task performance before editing, after regular editing, and after constrained editing by EAC.}
  \vspace{-3mm}
  \label{demo}
\end{figure}

\emph{Sequential} model editing~\cite{DBLP:conf/emnlp/YaoWT0LDC023} can expedite the continual learning of LLMs where a series of consecutive edits are conducted.
This is very important in real-world scenarios because new knowledge continually appears, requiring the model to retain previous knowledge while conducting new edits. 
Some studies have experimentally revealed that in sequential editing, existing methods lead to a decrease in the general abilities of the model across downstream tasks~\cite{DBLP:journals/corr/abs-2401-04700, DBLP:conf/acl/GuptaRA24, DBLP:conf/acl/Yang0MLYC24, DBLP:conf/acl/HuC00024}. 
Besides, \citet{ma2024perturbation} have performed a theoretical analysis to elucidate the bottleneck of the general abilities during sequential editing.
However, previous work has not introduced an effective method that maintains editing performance while preserving general abilities in sequential editing.
This impacts model scalability and presents major challenges for continuous learning in LLMs.

In this paper, a statistical analysis is first conducted to help understand how the model is affected during sequential editing using two popular editing methods, including ROME~\cite{DBLP:conf/nips/MengBAB22} and MEMIT~\cite{DBLP:conf/iclr/MengSABB23}.
Matrix norms, particularly the L1 norm, have been shown to be effective indicators of matrix properties such as sparsity, stability, and conditioning, as evidenced by several theoretical works~\cite{kahan2013tutorial}. In our analysis of matrix norms, we observe significant deviations in the parameter matrix after sequential editing.
Besides, the semantic differences between the facts before and after editing are also visualized, and we find that the differences become larger as the deviation of the parameter matrix after editing increases.
Therefore, we assume that each edit during sequential editing not only updates the editing fact as expected but also unintentionally introduces non-trivial noise that can cause the edited model to deviate from its original semantics space.
Furthermore, the accumulation of non-trivial noise can amplify the negative impact on the general abilities of LLMs.

Inspired by these findings, a framework termed \textbf{E}diting \textbf{A}nchor \textbf{C}ompression (EAC) is proposed to constrain the deviation of the parameter matrix during sequential editing by reducing the norm of the update matrix at each step. 
As shown in Figure~\ref{demo}, EAC first selects a subset of dimension with a high product of gradient and magnitude values, namely editing anchors, that are considered crucial for encoding the new relation through a weighted gradient saliency map.
Retraining is then performed on the dimensions where these important editing anchors are located, effectively compressing the editing information.
By compressing information only in certain dimensions and leaving other dimensions unmodified, the deviation of the parameter matrix after editing is constrained. 
To further regulate changes in the L1 norm of the edited matrix to constrain the deviation, we incorporate a scored elastic net ~\cite{zou2005regularization} into the retraining process, optimizing the previously selected editing anchors.

To validate the effectiveness of the proposed EAC, experiments of applying EAC to \textbf{two popular editing methods} including ROME and MEMIT are conducted.
In addition, \textbf{three LLMs of varying sizes} including GPT2-XL~\cite{radford2019language}, LLaMA-3 (8B)~\cite{llama3} and LLaMA-2 (13B)~\cite{DBLP:journals/corr/abs-2307-09288} and \textbf{four representative tasks} including 
natural language inference~\cite{DBLP:conf/mlcw/DaganGM05}, 
summarization~\cite{gliwa-etal-2019-samsum},
open-domain question-answering~\cite{DBLP:journals/tacl/KwiatkowskiPRCP19},  
and sentiment analysis~\cite{DBLP:conf/emnlp/SocherPWCMNP13} are selected to extensively demonstrate the impact of model editing on the general abilities of LLMs. 
Experimental results demonstrate that in sequential editing, EAC can effectively preserve over 70\% of the general abilities of the model across downstream tasks and better retain the edited knowledge.

In summary, our contributions to this paper are three-fold:
(1) This paper statistically elucidates how deviations in the parameter matrix after editing are responsible for the decreased general abilities of the model across downstream tasks after sequential editing.
(2) A framework termed EAC is proposed, which ultimately aims to constrain the deviation of the parameter matrix after editing by compressing the editing information into editing anchors. 
(3) It is discovered that on models like GPT2-XL and LLaMA-3 (8B), EAC significantly preserves over 70\% of the general abilities across downstream tasks and retains the edited knowledge better.

%%%%%%%%%%%%%%%%%%%%%%%%%%%%%%%%%%%%%%%%%%%%%%%%%%%%%%%%%%%%%%%%%%%%%%%%
\section{Problem Description and Solution Overview}\label{sec:probdesc}
\subsection{Problem Formalisation}

The initial input of our framework is a text expressed in natural language containing a description of the following:
\begin{enumerate*}
    \item How a task can be accomplished by combining high-level actions.
    \item The robotic resources available, along with a description of the low-level actions they are capable of performing.
    \item The environment.
\end{enumerate*}
The final objective is to generate, through an explainable process, an executable specification of the actions assigned to each robotic resource. The key requirements for the framework are as follows:
\begin{itemize}
    \item \textbf{R1:} The execution of a task must be formally correct, i.e., it must accomplish the goals and adhere to the constraints derived from the natural language text.
    \item \textbf{R2:} The process must be \emph{explainable}, ensuring that methods exist for human users to understand and trust the results produced by the system.
    \item \textbf{R3:} The knowledge obtained from understanding task execution through high-level actions must be \emph{reusable} across different implementation scenarios (e.g., using one robot or multiple robots).
    \item \textbf{R4:} The generation of the plan must support and optimise the parallel execution of actions across the available robotic resources.
    \item \textbf{R5:} The executable specification of the plan must be compatible with ROS2, which serves as a de facto standard for the execution environment of a wide range of robotic devices.
\end{itemize}

\subsection{Solution Overview}
\label{ssec:contributions}

\begin{figure*}[t!]
    \centering
    % \includegraphics[width=\linewidth]{figures/llmp_itmp.png}
    \includegraphics[width=\columnwidth]{figures/FrameworkDiagram.png}
    \caption{The architecture of the proposed framework.}
    \label{fig:arch_LLM_pKB}
\end{figure*}



The problem outlined above is addressed in this paper through a software framework depicted in Figure~\ref{fig:arch_LLM_pKB}. The framework comprises the following modules:
\begin{itemize}
    \item \textbf{Knowledge Management System} (KMS): It takes the initial natural language inputs and extracts a \kb in Prolog.
    \item \textbf{Planner}: It determines an executable plan from the \kb.
    \item \textbf{Execution Module}: It executes the plan by leveraging integration with the ROS2 middleware.
\end{itemize}

The KMS utilises an LLM to generate the \kb from a collection of
natural language texts. In the first step, the description of the
process and the environment is used to generate a high-level knowledge
base, i.e., a set of logical predicates that encode the breakdown of
the task execution into a number of interconnected \HL
actions. In the second step, this knowledge is augmented with
low-level robot-specific information, specifying how a \HL action can be
implemented using the elementary actions provided by the robot.

The technical details of these steps are outlined in
Section~\ref{sec:kb}. The generation of the \kb is not fully
automated and requires, at each step, some consistency checks (CC) by
the human developer. These checks ensure that the generated Prolog is
formally correct and that the goals and constraints are adequately
captured. When an inconsistency is identified, a few-shot learning
approach is used to provide feedback within the prompt, enabling the
system to correct itself.
%\edocom{Can we mention explicitely PDDL?}
Using a Prolog \kb offers several advantages over directly generating an executable plan:
\begin{enumerate}[nosep]
  \item The \kb contains a formal statement of goals and constraints, which facilitates the generation of formal correct plans (\textbf{R1}).
  \item A Prolog \kb is compact, human-readable, and understandable, enhancing the generation process's \emph{explainability} (\textbf{R2}).
  \item The deductive reasoning capabilities of Prolog make the \kb inherently compositional and reusable (\textbf{R3}).
  %
  Differently from other approaches to represent task planning in the robotic setting (e.g., those based on PDDL~\cite{pddl31} like for instance RosPlan~\cite{DBLP:conf/aips/CashmoreFLMRCPH15} or PlanSys2~\cite{DBLP:conf/iros/0001CMR21}) which are static, the Prolog \kb allows us to deduce new concepts and perform queries to check the consistency of the \kb, infer new knowledge, or update the current knowledge, evaluating the effects of the \kb update by logical reasoning. A new robotic implementation of the same task can be achieved by refining the same predicates and actions.
%  \item \todo[inline]{TO MR: We could add something about the ability to perform inference (absent in PDDL), reachability analysis, and formal consistency verifications.}
\end{enumerate}

\noindent The two-phase construction of the \kb provides two key benefits:
\begin{enumerate*}
    \item Facilitates reuse of the same \HL conceptual structure for different robotic implementations. 
    \item Helps manage the complexity of plan generation.
\end{enumerate*}


The generation of plans follows three steps. In the first step, detailed in
Section~\ref{ssec:toplangen}, a forward search is carried out starting
from the initial state encoded in the \kb. Different combinations of
actions are tested until a sequence is found that transitions from the
initial state to the goal state. This sequence represents a totally
ordered set of actions, but does not include timing information or
account for resource-specific constraints.

The second step, discussed in Section~\ref{ssec:poplangen}, utilises
Prolog's capabilities to analyse causal dependencies between
actions. Additionally, the \kb associates each action with the type of
resource required for its execution (e.g., a \verb|move| operation
might require a \verb|RoboticArm|). The outcome is a partially ordered
plan, where sequencing constraints exist only between causally
dependent actions. Resource-sharing constraints are not yet captured
at this stage.

In the third step, detailed in Section~\ref{ssec:poplanopt}, a
mixed-integer linear optimisation problem (MILP) is formulated. This
step aims to: 
\begin{enumerate*}
    \item Allocate actual resources to actions.  
    \item Optimise the timing of actions.  
\end{enumerate*}
The MILP encodes causal relationships between
actions (\emph{enablers}), resource constraints, and limits on the duration of the actions. The solution is a simple temporal network (STN), which can be checked for consistency. If successful, the resulting plan
supports parallel execution (\textbf{R4}) and is translated into a Behaviour Tree (\bt), a standard formalism for the execution of robotic plans in ROS2 (\textbf{R5}). Further details on this phase are provided in Section~\ref{sec:bt}.
If this operation fails, Prolog's backtracking capability can be
employed to generate an alternative total-order and repeat the
process. Persistent failures indicate a possible error in the \kb or
domain description, both of which should be revised.

A preliminary version of this work is presented in~\cite{saccon2023prolog}, where the LLM was used to generate only the initial and final states of the planning problem. This work extends the previous one in several directions. 
%
First, we have a flow to validate the output of the LLM, considering feedback to the user to correct possible logical and consistency errors.
%
Second, the LLM produces a high-level description of the planning task and a low-level one where additional details (e.g., resources and affordability) are considered, together with a mapping of high-level actions into low-level plans. 
%
Third, the formulation differentiating two levels allows reducing the burden and possible bottlenecks of generating the plan directly at the low-level of details.
%
Finally, in generating the low-level plan, we consider resources to reduce the makespan of the plans and parallelise the tasks on different robots.

% In~\cite{saccon2023prolog}, we used Prolog to first extract a total-order plan, then refine it into a partial-order plan and finally check its consistency by transforming it into an STN before extracting a BT to execute. 

%The main problem with this approach is that Prolog inherently performs a depth-first search, which has some drawbacks, mainly:
Similarly to ~\cite{saccon2023prolog}, the planner we implemented in Prolog performs a depth-first search, which has some drawbacks, mainly:
\begin{itemize}
    \item the provided plan is inefficient and usually sub-optimal since the solver will return the first plan that is feasible;
    \item the number of actions to choose from and the number of resources that must be allocated deeply impact the time to compute a feasible plan and its optimality.
\end{itemize}
We decided to focus on the second aspect to improve the plan obtained with the framework. We left as future work to leverage existing state-of-the-art planners (e.g., OPTIC~\cite{DBLP:conf/aips/BentonCC12} or FastDownward~\cite{DBLP:journals/jair/Helmert06}) for the generation of plans to further be optimized considering resources.

%%%%%%%%%%%%%%%%%%%%%%%%%%%%%%%%%%%%%%%%%%%%%%%%%%%%%%%%%%%%%%%%%%%%%%%%

\section{Background and Definitions}\label{sec:background}
\section{Preliminaries}

% Introduce terminologies and symbols

% \begin{itemize}
%     \item Self-attention module and RoPE
%     \item Vector quantization
% \end{itemize}

% In this section, we introduce the necessary background and notations that will be used throughout the paper.

\subsection{Self-Attention Modules and Rotary Position Embedding}

\label{sec:rope}

Self-attention modules~\citep{transformer} and Rotary Position Embedding (RoPE)~\citep{rope} have become the de facto standard components of state-of-the-art (SOTA) LLMs~\citep{llama-3, qwen-2.5, mixtral, deepseek-v3}.

In the self-attention module, during decoding phase, the inference process begins by linearly projecting the input states of the \(i\)-th token into query (\(q_i\)), key (\(k_i\)), and value (\(v_i\)) states,  where \(q_i, k_i, v_i \in \mathbb{R}^{1 \times d}\), and \(d\) denotes the number of channels or hidden dimensions per head. 
To enable the model to effectively capture the positional relationships between tokens, position embeddings are then applied to the query and key states. 
These hidden states before and after this transformation are abbreviated as pre-PE and post-PE states, respectively.

RoPE is a commonly used position embedding in SOTA LLMs.
% ~\citep{llama-3, qwen-2.5, mixtral, deepseek-v3}.
Specifically, for the \(i\)-th token, a position-dependent rotation matrix \(R_i \in \mathbb{R}^{d \times d}\) is applied to the query \(q_i\) and key \(k_i\), to obtain their post-PE counterparts, denoted by \(\tilde q_i\) and \(\tilde k_i\):
% \note{(\textit{abbr.} pre-PE), to obtain their after position embedding (\textit{abbr.} post-PE) counterparts, which are noted as \(\tilde q_i\) and \(\tilde k_i\) and can be calculated by:}

\begin{equation}
    \tilde q_i = q_i R_i, \quad \tilde k_i = k_i R_i
\end{equation}
Then the matrices
% of all key and value states 
of KV cache
of the context can be denoted by \({\tilde K} = [{\tilde k}_1; {\tilde k}_2; \dots; {\tilde k}_n] \in \mathbb R^{n \times d}\) and \(V = [ v_1;  v_2; \dots;  v_n] \in \mathbb R^{n \times d}\) respectively, where \(n\) denotes the context length.
Next, these post-PE states are used to compute the output state \(o_i\) as shown in formula (\ref{formula:output state}):
%These post-PE states are then used to compute the output state \(o_i\). 
%Let \(n\) denotes the context length, \({\tilde K} = [{\tilde k}_1; {\tilde k}_2; \dots; {\tilde k}_n] \in \mathbb R^{n \times d}\) and \(V = [ v_1;  v_2; \dots;  v_n] \in \mathbb R^{n \times d}\) denote the matrices of all key and value states in the context, respectively. 
%The output state \(o_i\) is then calculated as:
\begin{equation}\label{formula:output state}
    \begin{aligned}
        o_i & = \operatorname{Softmax}\left( \frac{\tilde q_i {\tilde K}^\top}{\sqrt d} \right) V 
        = \operatorname{Softmax} \left( \frac{u_i}{\sqrt d} \right) V
    \end{aligned}
\end{equation}
where \(u_i = \tilde q_i {\tilde K}^\top \in \mathbb R^{1 \times n}\) denotes the attention scores before softmax. 

% A key property of RoPE lies in its elegent incorporation of relative positional information.  
Due to the inherent property of rotation matrices that \(R_i R_j^\top = R_{i-j}\)~\citep{rope}, the attention score \(u_{i,j}\) between the \(i\)-th query and \(j\)-th key can be expressed as:
\begin{equation}
    \label{eq:rope}
    \begin{aligned}
        u_{i,j} = \tilde q_i {\tilde k}^\top_j = q_i R_i (k_j R_j)^\top 
        &= q_i R_i R_j^\top k_j^\top \\
        &= q_i R_{i-j} k_j^\top
    \end{aligned}
\end{equation}
This equation illustrates how RoPE encodes the relative position (\(i-j\)) directly into the attention scores, 
allowing the model to effectively capture
the positional relationships between tokens.
% based on their relative positions.

\subsection{Vector Quantization for Efficient Attention Score Approximation}

Vector quantization~\citep{vector-quantization} is a data compression technique that maps input vectors to a finite set of codewords from a learned codebook.

Formally, given an input space \(X \subseteq \mathbb{R}^{1 \times d}\) with data distribution \(\mathcal D\),
vector quantization aims to construct a codebook \(C = \{c_1, c_2, \dots, c_L\} \subset \mathbb{R}^{1 \times d}\) with a size of \(L\) codewords to minimize the following objective:
\begin{equation}
    \label{eq:vq_objective}
    J(C) = \mathbb E_{x \sim \mathcal D}[ \| x - \hat x \|^2 ]
\end{equation}
where \(x \in \mathbb R^{1 \times d}\) denotes the input vector, \(\hat x = c_{f(x; C)}\) denotes the quantized vector, and \(f(x; C)\) denotes the quantization function that maps \(x\) to its nearest codeword:
\begin{equation}
    f(x; C) = \operatorname*{argmin}_j \| x - c_j \|^2
\end{equation} 

Finding the optimal codebook \(C\) is computationally expensive.
Therefore, approximate algorithms such as LBG and k-means++~\citep{lbg, kmeans++} are commonly used to find a suboptimal but effective codebook.

After obtaining the codebook, vector quantization compresses an input \(x\) by replacing the original vector with its index \(s = f(x; C)\).
Since the storage requirement for the index is substantially lower than that of the original vector, vector quantization achieves significant data compression ratio.

Multiple studies~\citep{transformer-vq, pqcache, clusterkv} have investigated applying vector quantization to post-PE key states of LLMs to efficiently approximate attention scores.
%Let \(s \in \{1, 2, \dots, L\}^n\) denote the codeword indices of all post-PE key states, with the \(i\)-th key state quantized as \(\hat {k}_i = c_{s_i}\).
%The attention score \(u_{i,j}\) then can be approximated as:
Let \(s \in \{1, 2, \dots, L\}^{1 \times n}\) denotes the codeword index vector of all post-PE key states, where the length of this vector is \(n\), and each element \(s_i \in \{ 1, 2, \dots, L \}\) denotes the codeword index of the \(i\)-th key state.
Then, the \(\tilde k_i\) can be quantized as \(\hat {k}_i = c_{s_i}\)
, and the attention score \(u_{i,j}\) can be approximated as:
\begin{align}
    \label{eq:attention_weights_approximation}
    \hat u_{i,j} = \tilde q_i \hat k_j^\top = \tilde q_i c^\top_{s_j}
    % = a_{s_j}
\end{align}
% where \(a = q_iC^\top \in \mathbb R^{1 \times L}\).\note{(the difference between u and a)} 
This equation illustrates the approximation of attention scores without the memory-intensive access to the \(\tilde k_j\).

\begin{figure}[t]
    \centering
    \includegraphics[width=0.8\linewidth]{images/inter_input_similarity.pdf}
    \caption{Inter-sample cosine similarities of pre-PE and post-PE codebooks.}
    \label{fig:cosine_similarity}
\end{figure}

\begin{figure}[t]
    \centering
    \begin{tabular}{cc}
        % \begin{subfigure}[b]{0.225\textwidth}
        %     \includegraphics[width=\textwidth]{images/hessian_15_3.pdf}
        %     \caption{Layer 16, Head 4}
        % \end{subfigure}
        % &
        \begin{subfigure}[b]{0.225\textwidth}
            \includegraphics[width=\textwidth]{images/hessian_15_7.pdf}
            \caption{Layer 16, Head 8}
        \end{subfigure}
        &
        % \begin{subfigure}[b]{0.225\textwidth}
        %     \includegraphics[width=\textwidth]{images/hessian_31_3.pdf}
        %     \caption{Layer 32, Head 4}
        % \end{subfigure}
        % &
        \begin{subfigure}[b]{0.225\textwidth}
            \includegraphics[width=\textwidth]{images/hessian_31_7.pdf}
            \caption{Layer 32, Head 8}
        \end{subfigure}
    \end{tabular}

    \caption{Visualization of second-moment matrices \(H\) of post-PE query states. Each pixel represents an element in \(H\). Warmer colors correspond to higher values, while cooler colors correspond to lower values.}

    \label{fig:hessian_matrix}
\end{figure}

%%%%%%%%%%%%%%%%%%%%%%%%%%%%%%%%%%%%%%%%%%%%%%%%%%%%%%%%%%%%%%%%%%%%%%%%

\section{\Kbase Generation}\label{sec:kb}
The Knowledge Management System module (KMS), is in charge of taking the natural language description of both the environment and the actions that the agents can do, and convert them to a Prolog \kb using a LLM. 
The \kb contains all the necessary elements to define the mapped planning problem introduced in the previous section.

The framework works by considering a high-level and a low-level \kbase. For this reason, the input descriptions are also split into \HL and \LL. The former captures more abstract concepts, e.g., complex actions such as \verb|move_block| or the objects that are present in the environment. The latter captures more concrete and physical aspects of the problem, e.g., the actions that can be actually carried out by the agents such as \verb|move_arm| or the positions of the blocks. An example of this division can be seen in Section~\ref{sssec:runegKMS}.

% Describe how the kb works
The \kbase is divided in the following parts:
\begin{itemize}
    \item General \kb ($K$): contains the grounding predicates, both for the \HL and \LL. These predicates describe parts of the scenario or of the environment that do not change during execution. For example, the predicate \verb|wheeled(a1)|, which states that robot \verb | a1| has wheels, should be part of the general \kb and not of the state. 
    \item Initial ($I$) and final states ($G$): they contain all the fluents that change during the execution of the plan. This could be, for example, the position of blocks in the environment. 
    \item High-level actions ($DA_H$): each high-level action predicate is written as:
\begin{minted}[fontsize=\small,breaklines]{Prolog}
action(
    action_name(args),
    [positive_preconditions],
    [negative_preconditions],
    [grounding_predicates],
    [effects]
).
\end{minted}
    The low-level actions ($DA_L$) have the same structure, but instead of being described as predicates of type \verb|action|, they are described as \verb|ll_action|. The preconditions $\pc{a}$ of an action $a$ are obtained by combining the list of predicates \verb|positive_preconditions| and \verb|negative_preconditions|. The predicates in the list \verb|grounding_predicates| are used to ground the parametrised fluents of the action. For example, the action \verb|move_block| depends on a block, and we can check that the action is correctly picking a block and not another object by querying the \kb in this step. Section~\ref{sssec:runegKMS} clarifies this aspect.  
    \item Mappings ($M$): contains a dictionary of \HL actions $DA_H$ and how they should be mapped to a sequence of \LL actions $DA_L$. As will become apparent in the following, the distinction between \HL and \LL actions induces a significant simplification in the planning phase.
    \item Resources ($R$): the predicate \verb|resource\1| states whether another predicate is part of the resources or not. As mentioned before, this is helpful because it allows one to shrink the complexity of the problem not having to check multiple predicates, but instead they are later allocated during the optimisation part.
\end{itemize}

% Describe the process to validate the initial descriptions
Once the user provides descriptions for the \HL and \LL parts, the framework performs a consistency check to ensure that there are no conflicts between them. It verifies that both descriptions share the same goal, that objects remain consistent across \HL and \LL, and that agents are capable of executing the tasks. This validation is carried out by an LLM, which, if inconsistencies are detected, provides an explanation to help the user make the necessary corrections.

In both this step and the subsequent steps to generate the \kb, the LLM is not used directly out of the box. Instead, we employ the Chain-of-Thought (CoT)~\cite{wei2022chain} approach, which involves providing the LLM with examples to guide its reasoning. This process ensures that the output is not only structurally correct, but also more aligned with the overall goal of the task.

% Describe how the different aspects of the kb are extracted
Examples are particularly important when generating the \kb. Indeed, as we have mentioned before, the \kbase is highly structured and the planner expects to have the different components written correctly. CoT enables the LLM to know these details. 

We tested two different ways of generating the \kb through LLMs:
\begin{itemize}
    \item either we produced the whole \kb for the high-level and the low-level all at once, or
    \item we produced the single parts of the \kbs. 
\end{itemize}

The first approach is quite straightforward: once we have the examples to give to the LLM for the CoT process, we can input the \HL description and query the LLM to first extract the high-level \kb, and then also feed the created \HL \kb to the LLM to generate the \LL \kb, which will contain everything. % Highlight why one would want this. 

Instead, the second approach requires more requests to the LLM. We first focus on the \HL \kb, and then feed the \kb that we have obtained to generate the \LL parts. For the \HL generation, we ask the LLM to generate the general \kb, the initial and final states, and the actions set in this particular order. Each time we provide the LLM with the \HL description and with the elements generated in the previous steps. The same thing is done for the \LL \kb, generating again the four components and feeding each time also the \HL \kb. We include a final step that generates the mappings between the \HL and \LL actions. As for all the other steps, also in this final step, we pass the previously generated elements of the \LL \kb. 
Although generating the entire \kbase at once would reduce token usage and speed up the process, dividing the generation of the \kb into distinct steps enhances the system's accuracy, as demonstrated in the experimental evaluation of Section~\ref{sec:experiments}. This improvement comes because the iterative approach allows the LLM to first focus on generating more homogeneous information (i.e., the high-level) and then leverage the previously generated content to perform a consistency check.  

\subsection{Runing Example -- KMS}
\label{sssec:runegKMS}

\input{sections/4-methodology/a-runningexample}

Let's now focus on the \kbase generation. For example, the two queries that could be passed to the system are:

\begin{textbox}{\HL query}
% \textboxtitle{\HL query}{
\footnotesize
In this scenario, there are two blocks on a table. They are placed in positions (1,1) and (3,1). There is a robotic agent that can move the blocks from one position to another. In particular, it can:
\begin{itemize}
    \item move a block from a position on the table, to another position on the table;
    \item move a block from a position on the table, to the top of another block;
    \item move a block from the top of a block, to a position on the table;
    \item move a block from the top of a block, to the top of another block.
\end{itemize}
The goal of this scenario is to move the first block from position (1,1) to position (2,2), and then place the second block on top of it.
% }
\end{textbox}

\begin{textbox}{\LL query}
% \textboxtitle{\LL query}{
\footnotesize
In this scenario, there are two blocks that are in the same positions as described in the \HL query. The robotic agent is a robotic arm and its end-effector is a two-finger gripper that can grasp an object. In particular, the following APIs are made available for the agent to move:
\begin{itemize}
    \item \texttt{move\_arm(Arm, X1, Y1, X2, Y2)}, which allows the end-effector to be moved from one position to another;
    \item \texttt{grip(Arm)}, which allows the gripper to be lowered onto a block and grip the block;
    \item \texttt{release(Arm)}, which allows to lower the block and release the gripper leaving the block in place.
\end{itemize}
% }
\end{textbox}


As mentioned before, the first step is to pass the two queries for validation. In this case, when using GPT-4o, both queries pass the consistency check. If instead of using the previous \HL query, we would modify it by removing the possibility of placing one block on top of the other, then the LLM would return the following error:

\begin{textboxerror}
\footnotesize
The description of the scenario has an issue. The agent is described as being able to move a block from a position on the table to another position on the table, and from the top of a block to a position on the table. However, the goal requires the agent to place the second block on top of the first block, which is not listed as an action that the agent can carry out. The agent's capabilities do not include placing a block on top of another block, which is necessary to achieve the goal.
\end{textboxerror}

\noindent This also allows the user to understand where the mistake is and correct it. 

Once the validation part succeeds, the system generates the \HL \kb. In this particular instance, for space limitation, we present only the general \kbase ($K$), the initial ($I$) and final ($G$) states, and a single action. 

\begin{center}
\begin{minipage}{\linewidth}
    \begin{minipage}{.48\linewidth}
        \begin{codebox}{prolog}{General KB}
% Positions
pos(1,1).
pos(2,2).
pos(3,1).

% Blocks
block(b1).
block(b2).

% Agents
agent(a1).

% Resources
resources(agent(_)).
        \end{codebox}
    \end{minipage}
    \hfill
    \begin{minipage}{.48\linewidth}
        \begin{minipage}{\linewidth}
        \begin{codebox}{prolog}{Initial state ($I$)}
init_state([
  ontable(b1), ontable(b2),
  at(b1,1,1), at(b2,3,1),
  clear(b1), clear(b2),
  available(a1)
]).
        \end{codebox}
        \end{minipage}
        \hspace{1cm}\\
        \begin{minipage}{\linewidth}
        \begin{codebox}{prolog}{Final state ($G$)}
goal_state([
  ontable(b1),
  on(b2, b1),
  at(b1,2,2), at(b2,2,2),
  clear(b2),
  available(a1)
]).
        \end{codebox}
        \end{minipage}
    \end{minipage}
\end{minipage}
\begin{codebox}{prolog}{Action example}
action(move_table_to_table_start(Agent, Block, X1, Y1, X2, Y2), 
  [ontable(Block), at(Block, X1, Y1), available(Agent), clear(Block)],
  [
    at(_, X2, Y2), on(Block, _), moving_table_to_table(_, Block, _, _, _, _), 
    moving_table_to_block(_, Block, _, _, _, _, _)
  ],
  [agent(Agent), pos(X1, Y1), pos(X2, Y2), block(Block)],
  [
    del(available(Agent)), del(clear(Block)), del(ontable(Block)), del(at(Block, X1, Y1)),
    add(moving_table_to_table(Agent, Block, X1, Y1, X2, Y2))
  ]
).
\end{codebox}
\end{center}

The resulting \HL \kb is human-readable and relatively simple (in fulfilment of requirement \textbf{R2}).
The user at this point can make corrections to the \HL \kb, if needed, and finally, \frameworkname will also generate the \LL \kbase. In this case for space limitation, we show the changes made to the previous elements, one low-level action, and one mapping. 

\begin{center}
\begin{minipage}{\linewidth}
    \begin{minipage}{.48\linewidth}
        \begin{codebox}{prolog}{General KB}
% Positions
pos(0,0).
pos(1,1).
pos(2,2).
pos(3,1).

% Blocks
block(b1).
block(b2).

% Agents
agent(a1).

% Low-level predicates
ll_arm(a1).
ll_gripper(a1).

% Resources
resources(agent(_)).
        \end{codebox}
    \end{minipage}
    \hfill
    \begin{minipage}{.48\linewidth}
        \begin{minipage}{\linewidth}
        \begin{codebox}{prolog}{Initial state ($I$)}
init_state([
  ontable(b1), ontable(b2),
  at(b1,1,1), at(b2,3,1),
  clear(b1), clear(b2),
  available(a1),
  ll_arm_at(a1,0,0), 
  ll_gripper(a1,open) 
]).
        \end{codebox}
        \end{minipage}
        \hspace{1cm}\\
        \begin{minipage}{\linewidth}
        \begin{codebox}{prolog}{Final state ($G$)}
goal_state([
  ontable(b1),
  on(b2, b1),
  at(b1,2,2), at(b2,2,2),
  clear(b2),
  available(a1),
  ll_arm_at(a1,_,_), 
  ll_gripper(a1,_)    
]).
        \end{codebox}
        \end{minipage}
    \end{minipage}
\end{minipage}
\begin{codebox}{prolog}{Action example}
ll_action(move_arm_start(Arm, X, Y),
  [],
  [ll_arm_at(_, X, Y), moving_arm(Arm, _, _, _, _), gripping(Arm, _), releasing(Arm)],
  [],
  [ll_arm(Arm), pos(X, Y)],
  [
    add(moving_arm(Arm, X, Y)),
    del(ll_arm_at(Arm, X, Y))
  ]
).
\end{codebox}
\begin{codebox}{prolog}{Mapping example}
mapping(move_table_to_table_start(Agent, Block, X1, Y1, X2, Y2),
  [
    move_arm_start(Agent, X1, Y1),
    move_arm_end(Agent, X1, Y1),
    grip_start(Agent),
    grip_end(Agent),
    move_arm_start(Agent, X2, Y2),
    move_arm_end(Agent, X2, Y2),
    release_start(Agent),
    release_end(Agent)
  ]
).
\end{codebox}
\end{center}

Again, the user can correct possible errors (or anyway refine the \kb) and then move on to the planning phase.


\section{Plan Generation}\label{sec:plangen}
In this section, we describe how the framework uses the information from the \kb to generate a task plan for multiple agents.
Generation takes place in three steps: 
\begin{enumerate*}
    \item Generation of a total-order (TO) plan, 
    \item extraction of a partial-order (PO) plan and of the resources, 
    \item solution of a MILP problem to improve resource allocation and reducing the plan makespan by exploiting the possible parallel executions of actions.
\end{enumerate*}



\subsection{Total-Order Plan Generation}\label{ssec:toplangen}
A total-order plan is a strictly sequential list of actions that drives the system from the initial to the goal state. 
The algorithm used to extract a total-order plan is shown in~\autoref{alg:toplanning} and consists of two distinct steps:
\begin{itemize}
    \item identify a total-order plan for high-level actions, and
    \item recursively map each high-level action to a sequence of actions with a lower level until they are mapped to actions corresponding to the APIs of the available robotic resources.
\end{itemize}

\begin{algorithm}
\footnotesize
\caption{Algorithm generating a TO plan with mappings}\label{alg:toplanning}
\KwData{$TP=(F, DA, I, G, K)$}
\KwResult{Plan solving TP}

\DontPrintSemicolon

\SetKwProg{plan}{TO\_PLAN}{}{}
\SetKwProg{map}{APPLY\_MAP}{}{}
\SetKwProg{action}{APPLY\_ACTION}{}{}
\SetKwProg{maps}{APPLY\_MAPPINGS}{}{}

\SetKwInOut{Input} {In}
\SetKwInOut{Output}{Out}

\plan{(S, P)}{
  \Input{The current state $S$ and the current plan $P$}
  \Output{The final plan}
  \If{$S \neq G$}{
      select\_action($a_i$)\;
      (US, UP) $\gets$ APPLY\_ACTION($a_i$, S, P)\;
      P $\gets$ TO\_PLAN(US, UP)\;
  }
  (US, UP) $\gets$ APPLY\_MAPPINGS(S,P)\;
  \KwRet{P}\;
}

\maps{(S, P)}{
  \Input{The current state $S$ and the current plan $P$}
  \Output{The updated state $US$ and plan $UP$ after the mappings}
  US, UP $\gets$ S, P\;
  \ForEach{$a_i \in P$}{
    \If{\textnormal{is\_start($a_i$) $\wedge$ has\_mapping($a_i$)}}{
      (US, UP) $\gets$ APPLY\_MAP($a_i$, \textnormal{US}, UP)\;
    }
  }
  \KwRet{(US, UP}\;
}

\map{($a$, S, P)}{
  \Input{The action $a$, the current state $S$ and the current plan $P$}
  \Output{The updated state $US$ and plan $UP$ after the mappings}
  M $\gets$ mapping($a$)\;
  \ForEach{$a_i \in M$}{
    (US, UP) $\gets$ APPLY\_ACTION($a$, S, P)\;
  }
  \KwRet{(US, UP)}\;
}

\action{($a, S, P$)}{
  \Input{The action $a$, the current state $S$ and the current plan $P$}
  \Output{The updated state $US$ and plan $UP$ after applying the effects of $a$}
  \eIf{\textnormal{is\_applicable($a_i$)}}{
    US $\gets$ change\_state($a_i$.eff, S)\;
    UP $\gets$ plan\_action($a_i$, P)\;
    \KwRet{(US, UP)}\;
  }{
    \KwRet{(S, P)}
  }
}
\end{algorithm}

This enables the extraction of total-order plans that are consistent with the \kb provided, and we reduce the computational cost of checking all the possible actions at each time step. The \texttt{TO\_PLAN} function is the main function, which takes the initial and final states, and it inspects which actions can be executed given the current state. The \texttt{select\_action} function selects the next action from the set of possible actions. This search is based on the Prolog inference engine, which tries the actions in the order in which they are written in the KB, and hence it is not an informed search. 

The algorithm then moves to the \texttt{APPLY\_ACTION} function, which first checks if the chosen action's preconditions are met in the current state and, if they are, then it applies its effects changing the state (\texttt{change\_state}) and adding the action to the plan (\texttt{plan\_action}). It continues until the current state satisfies the goal state. Whenever the search reaches a fail point, we exploit the Prolog algorithm of resolution to step back and explore alternative possibilities.

Once the algorithm has extracted a high-level total-order plan, it applies the mappings. To do so, it iterates over the actions in the plan, and for each action it checks if it is a start action ($a_\vdash$) and if there are mappings for it. If this is the case, it calls the function \texttt{APPLY\_MAP}, which sequentially applies the actions in the mapping to the current state, also adding the actions to the plan. Notice that to do so, we call the \texttt{APPLY\_ACTION} function, which checks the preconditions of the actions w.r.t. the current state, ensuring that the lower-level actions can actually be applied.
% Also, the functions recursively check if any action from the mapped action has a mapping on its own, ensuring that all the actions have a direct grounding to APIs.

The total-order plan $TO$ extracted from this function is a list of actions that are executed in sequence:
\begin{equation*}
    \forall i \in \{0,\hdots \vert TO\vert-1\}~t(a_i)<t(a_{i+1})
\end{equation*}

\subsubsection{Running Example -- Total-Order Plan}
\label{sssec::runegTOPlan}

Let us consider again the \kb that we generated in Section~\ref{sssec:runegKMS}. Let us now see how \frameworkname extracts the TO plan.

The algorithm starts from the initial state and from the first action in the \kb, which in this case is the one shown in Section~\ref{sssec:runegKMS}. The algorithm takes the grounding predicates in this case:

\begin{minted}[fontsize=\footnotesize]{prolog}
agent(Agent), pos(X1, Y1), pos(X2, Y2), block(Block)
\end{minted}

and checks whether there is an assignment of predicates from the \kbase that satisfies them. For example, the predicate \verb|pos(1,1)| satisfies \verb|pos(X1,Y1)|. Not only this, but since the predicates in this list are grounded w.r.t. the \kb, one can also check some conditions. For example, if we were to assign the values to the previous predicates, it can happen that \verb|X1 = X2| and \verb|Y1 = Y2|, which is useless for an action that moves a block from one position to another. By adding the following predicates, we can ensure that the values are different:

\begin{minted}[fontsize=\footnotesize]{prolog}
agent(Agent), pos(X1, Y1), pos(X2, Y2), block(Block), X1\=X2, Y1\=Y2
\end{minted}

Once an assignment for the predicates inside the grounding list is found, the algorithm checks whether the predicates inside the preconditions are satisfied. Let us consider the preconditions for the \verb|move_table_to_table_start| action from Section~\ref{sssec:runegKMS}:

\begin{minted}[fontsize=\footnotesize]{prolog}
% Positive predicates
[ontable(Block), at(Block, X1, Y1), available(Agent), clear(Block)],
% Negative predicates
[
  at(_, X2, Y2), on(Block, _), moving_table_to_table(_, Block, _, _, _,_), 
  moving_table_to_block(_, Block, _, _, _, _, _)
]
\end{minted}

After the first grounding step, they become the following:

\begin{minted}[fontsize=\footnotesize]{prolog}
% Positive predicates
[ontable(b1), at(b1, 0, 0), available(a1), clear(b1)],
% Negative predicates
[
  at(_, 0, 0), on(b1, _), moving_table_to_table(_, b1, _, _, _,_), 
  moving_table_to_block(_, b1, _, _, _, _, _)
]
\end{minted}

The algorithm checks whether the predicates from the first list are satisfied in the current state and whether the predicates from the second list are not present in the current state. Comparing them with the initial state as shown in Section~\ref{sssec:runegKMS}, we can see that \verb|ontable(b1)| is present, but \verb|at(b1, 0, 0)|, so this combination of predicates would already be discarded. The first grounding that is accepted is that in which \verb | Block = b1, X1 = 1, Y1 = 1, Agent = a1 |. Notice that the predicates that start with \verb|_| mean "any", e.g., the predicate \verb|at(_, 0, 0)| checks if there is any predicate with name \verb|at| and arity 3 that has the last two arguments set to 0, regardless of what the first argument is.

By checking the different combinations of actions, the planner can extract a \HL TO plan. In this case, it would be something like this:

\begin{minted}[fontsize=\footnotesize]{text}
[0] move_table_to_table_start(a1, b1, 1, 1, 2, 2)
[1] move_table_to_table_end(a1, b1, 1, 1, 2, 2)
[2] move_table_to_block_start(a1, b2, 3, 1, 2, 2)
[3] move_table_to_block_end(a1, b2, 3, 1, 2, 2)
\end{minted}

At this point, the algorithm takes the mappings and it applies them to the previous plan. For instance, from Section~\ref{sssec:runegKMS} we saw that the mapping for \verb|move_table_to_table_start| is:
\begin{minted}[fontsize=\footnotesize]{prolog}
mapping(move_table_to_table_start(Agent, Block, X1, Y1, X2, Y2),
  [
    move_arm_start(Agent, X1, Y1), move_arm_end(Agent, X1, Y1),
    grip_start(Agent), grip_end(Agent),
    move_arm_start(Agent, X2, Y2), move_arm_end(Agent, X2, Y2),
    release_start(Agent), release_end(Agent)
  ]
).
\end{minted}

Hence, we would change the previous plan with:

\begin{minted}[fontsize=\footnotesize]{text}
[0] move_table_to_table_start(a1, b1, 1, 1, 2, 2)
[1] move_arm_start(a1, 1, 1)
[2] move_arm_end(a1, 1, 1)
[3] grip_start(a1)
[4] grip_end(a1)
[5] move_arm_start(a1, 2, 2)
[6] move_arm_end(a1, 2, 2)
[7] release_start(a1)
[8] release_end(a1)
[9] move_table_to_table_end(a1, b1, 1, 1, 2, 2)
[10] move_table_to_block_start(a1, b2, 3, 1, 2, 2)
[11] move_arm_start(a3, 3, 1)
[12] move_arm_end(a1, 3, 1)
[13] grip_start(a1)
[14] grip_end(a1)
[15] move_arm_start(a1, 2, 2)
[16] move_arm_end(a1, 2, 2)
[17] release_start(a1)
[18] release_end(a1)
[19] move_table_to_block_end(a1, b2, 3, 1, 2, 2)
\end{minted}


\subsection{Partial-Order Plan Generation}\label{ssec:poplangen}
The next step is to analyse the total-order plan in search of all possible causal relationships. This is done by
looking for actions that enable other actions (enablers). In addition, we extract all the resources that can be allocated
and used for the execution of the task. This step will be important for the next phase of the planning process, the MILP problem, in which 
the resources will be re-allocated allowing for shrinking the makespan of the plan.
%
In this work, the only resource considered is the robotic agent, but this limitation could easily be removed by modifying the \kb.  To this end,  we define a special predicate, named \texttt{resource/1}, that allows us to specify the resources.

Given an action $a_i$, another action $a_j$ is an enabler of $a_i$ if it either adds a literal $l$ satisfying one or more preconditions of $a_i$, or it removes a fluent violating one or more preconditions of $a_i$, and if $a_i$ happens after $a_j$: 

\begin{equation}
\small
\begin{array}{rl}
     a_j \in \ach{a_i} \iff & t(a_i) > t(a_j) \wedge \\
                            & ((l\in \pc{a_i}~ \wedge add(l)\in \eff{a_j}) \vee\\
                            & \,\,(\lnot l\in \pc{a_i} \wedge del(l)\in \eff{a_j}))
\end{array}
\label{eq:enablers}
\end{equation}

It is important to note that we consider an action $a_j\notin\ach{a_i}$ if there is at least a fluent $l$ that is not a resource. If all the fluents and their arguments that would make $a_j$ an enabler of $a_i$ are resources, then $a_j$ is not considered an enabler, as this relationship depends on the assignment of the resources, which comes with the optimisation step. 

Besides the enablers added corresponding to the classical definition, we also enforce the following precedence constraints:
\begin{itemize}
    \item When we expand a mapping $m(\alpha_i)$ of a high-level durative action $\alpha_i$ and reach the ending action $\aEnd{\alpha_i}$, then we add all previous durative actions as enablers until the corresponding start action. For example, assume that $m(\alpha_i)=\{\alpha_j, \alpha_k\}$, this means that the total-order plan will be the sequence $\{\aStart{\alpha_i}, \aStart{\alpha_j}, \aEnd{\alpha_j}, \aStart{\alpha_k}, \aEnd{\alpha_k}, \aEnd{\alpha_i}\}$. It follows that $\aStart{\alpha_i}$ is an enabler of $\aEnd{\alpha_i}$, but also all intermediate actions are part of the set of its enablers as they must be completed in order for $\alpha_i$ to end.
    \begin{equation}
        \bigwedge_{a\in m(\alpha_i)} a\in \ach{\aEnd{\alpha_i}}.
        \label{eq:constraint5}
    \end{equation}
    \item When we expand a mapping, all actions in the mapping must have the start of the higher-level action as one of the enablers. For instance, after the previous example, $\aStart{\alpha_j}, \aEnd{\alpha_j}, \aStart{\alpha_k}, \aEnd{\alpha_k}$ have $\aStart{\alpha_i}$ as an enabler.
    \begin{equation}
        \bigwedge_{a\in m(\alpha_i)} \aStart{\alpha} \in \ach{a_i}.
        \label{eq:constraint4}
    \end{equation}
\end{itemize}

% We then create a graph from which we can extract partial-order plans. To do this, after having obtained a plan from the \texttt{TO\_PLAN} from~\autoref{alg:planning}, we look for the achievers of the actions as shown in~\autoref{alg:po_planning}. 

The algorithm that manages this extraction is shown in~\autoref{alg:poplanning}. For ease of reading, we define $R\subseteq F$ as the set of fluents that are resources.

The algorithm \texttt{FIND\_ENABLERS} takes the total-order plan and, starting with the first action in the plan, it extracts all the causal relationships between the actions. The auxiliary function \texttt{IS\_ENABLER} tests whether an action $a_j$ is an enabler of an action $a_i$ by checking the properties of~\autoref{eq:enablers} plus the precedence constraints just described. Finally, notice that the literal checked to be present (absent) in both additive (subtractive) effects must not contain arguments that are part of the resources $R$. For example, consider the case in which an action $a_i$ needs the precondition $l(x_1, x_2, x_3)$ and $a_j$ provides the predicate, then if at least one of $x_1, x_2, x_3$ is in $R$, $a_j$ is an enabler of $a_i$, otherwise it is not. This ensures that only causal relationships that do not depend on the resources are extracted at this time. The precedence of the resources will be defined and discussed in Section~\ref{ssec:poplanopt}. 

\begin{algorithm}[htp]
\footnotesize
\caption{Algorithm extracting the actions enablers and the resources}
\label{alg:poplanning}
\KwData{$TP=(F, DA, I, G, K)$}
\KwResult{Enablers and resources $R$}

\DontPrintSemicolon

\SetKwProg{findenablers}{FIND\_ENABLERS}{}{}
\SetKwProg{isenabler}{IS\_ENABLER}{}{}
\SetKwProg{findresources}{EXTRACT\_RESOURCES}{}{}

\SetKwInOut{Input} {In}
\SetKwInOut{Output}{Out}

\findenablers{$(\tn{TO\_P}, a_i)$}{
  \Input{The total-order plan TO\_P, the $i$th action}
  \Output{The enablers $E$ for all the actions in the plan}

  \For{$a_j \in \tn{TO\_P}, a_j\neq a_i$}{
    \uIf{$\tn{IS\_ENABLER}(a_j, a_i)$}{
      $E[a_i].add(a_j)$;
    }
  }

  \If{$a_i\neq \tn{TO\_P}.back()$}{
    $E \gets \tn{FIND\_ENABLERS}(\tn{TO\_P}, a_{i+1})$\;
  }
  \KwRet{E}\;
}

\isenabler{$(a_j, a_i)$}{
  \Input{The action $a_j$ to test if it's enabler of $a_i$}
  \Output{True if $a_j$ is enabler of $a_i$}

  \ForEach{$e \in \eff{a_j}$}{
    \uIf{$\left(e=\tn{add}(l) \wedge l\in\pc{a_i})\right)$ OR
         $~\left(e=\tn{del}(l) \wedge \lnot l\in\pc{a_i}\right)$ OR
         $~\left(\tn{isStart}(a_j) \wedge a_i \in m(a_j)\right)$ OR\\
         $~~\left(\tn{isEnd}(a_j) \wedge a_i \in m(a_j)\right)$}
    {
      $X\gets \tn{set of arguments of }e$; 
        
      \uIf{$\not\exists x \in X | x \in R$}{
        \KwRet{True};
      }
    }
  }
  \KwRet{False};
}

\findresources{$()$}{
  \output{A list of resources}
  findall(X, resources(X), AllResources)\;
  $R$ = make\_set(AllResources)\;
  \KwRet{$R$}\;
}

\end{algorithm}

\subsubsection{Running Example -- Partial-Order Plan}
\label{sssec:PORunEx}

Once we have applied the mappings as before, we have the full TO plan. We want to extract information from this, which will then be exploited to improve the plan for multiple agents. This is done by examining all the actions and checking which are their enablers. For instance, the 10th action, \verb|move_table_to_block_start(a1, b2, 3, 1, 2, 2)|, has as a precondition the following predicate \verb|clear(Block2), Block2=b1|, which is true only when the 9th action has applied its effects. Since \verb|b1| is not part of the resources, the algorithm will state that $a_9$ is an enabler of $a_{10}$. 

If the second move were to move a block to another position on the table, hence independent of the first move, then the algorithm would not set $a_9$ as an enabler of $a_{10}$, as the only reason it may do so is if the same agent is used, but this is known only later.

After this step, we know the enablers for the actions (shown in squared brackets in the list below):

\begin{minted}[fontsize=\footnotesize]{text}
[0] init()[]
[1] move_table_to_table_start(a1, b1, 1, 1, 2, 2), [0]
[2] move_arm_start(a1, 1, 1), [0,1]
[3] move_arm_end(a1, 1, 1), [0,1,2]
[4] grip_start(a1), [0,1,2,3]
[5] grip_end(a1), [0,1,2,3,4]
[6] move_arm_start(a1, 2, 2), [0,1,2,3,4,5]
[7] move_arm_end(a1, 2, 2), [0,1,2,3,4,5,6]
[8] release_start(a1), [0,1,2,3,4,5,6,7]
[9] release_end(a1), [0,1,2,3,4,5,6,7,8]
[10] move_table_to_table_end(a1, b1, 1, 1, 2, 2), [0,1,2,3,4,5,6,7,8,9]
[11] move_table_to_block_start(a1, b2, 3, 1, 2, 2), [0,10]
[12] move_arm_start(a1, 3, 1), [0,11]
[13] move_arm_end(a1, 3, 1), [0,11,12]
[14] grip_start(a1), [0,11,12,13]
[15] grip_end(a1), [0,11,12,13,14]
[16] move_arm_start(a1, 2, 2), [0,11,12,13,14,15]
[17] move_arm_end(a1, 2, 2), [0,11,12,13,14,15,16]
[18] release_start(a1), [0,11,12,13,14,15,16,17]
[19] release_end(a1), [0,11,12,13,14,15,16,17,18]
[20] move_table_to_block_end(a1, b2, 3, 1, 2, 2), [0,10,11,12,13,14,15,16,17,18,19]
[21] end(), [0,1,2,3,4,5,6,7,8,9,10,11,12,13,14,15,16,17,18,19,20]
\end{minted}

From this we could already notice that all the actions will be carried out in sequence. We also see that in this step we add two fictitious actions, \verb|init| and \verb|end|. This simply represents the start and the end of the plan, respectively. \verb|init| is an enabler of all the actions in the plan and \verb|end| has all the other actions as enablers, which means that the plan can be considered finished only when all the actions have been executed.

As for the resources, we first extract all the possible resources by looking at the predicates \verb|resource(X)| in the \kb, as shown in Section~\ref{sssec:runegKMS}. Then we assign the type of resources used to each action by checking action per action which resources they are using. This is useful because it will provide MILP with the basis to correctly allocate the different resources to the actions.

\begin{minted}[fontsize=\footnotesize]{text}
Resources:
[0] agent-2
Resources list:
[0] agent-[agent(a1),agent(a2)]
Resources required by action:
[4] 6-[agent]
[9] 1-[agent]
\end{minted}


\subsection{Partial-Order Plan Optimization}\label{ssec:poplanopt}
The last part of the planning module, shown in~\autoref{fig:arch_LLM_pKB}, is the optimisation module which allows for shrinking the plan by scheduling the task (temporal plan) and allocating the resources. In order to do this, we instantiate a MILP problem, the solution of which must satisfy constraints ensuring that we are not violating precedence relationships and invalidating the obtained planned. 

We start by taking the work from~\cite{cimatti_strong_2015}, in which the authors describe how it is possible to obtain a plan with lower makespan by reordering some tasks. In particular, we adopt the following concepts from~\cite{cimatti_strong_2015}:
\begin{itemize}
    \item Let $f(l)=\{a\in DA \vert l\in \eff{a}\}$ be the set of actions that achieve a literal $l$, and 
    \item let $\displaystyle p(l,a,r)\doteq a<r \wedge \bigwedge_{a_i\in f(l)\setminus\{a,r\}}(a_i<a\vee a_i>r)$ be the temporal constraint stating which is the last achiever $a$ of an action $r$ for a literal $l$. 
\end{itemize}
The constraints that must hold are the following:
\begin{equation}
    \label{eq:constraint1_old}
    %\footnotesize
    \bigvee_{a_j\in f(l)\setminus\{a\}} p(l,a_j,a).
\end{equation}
Which states that at least an action with effect $l$ should occur before $a$.
\begin{equation}
    \label{eq:constraint2}
    %\footnotesize
    \bigwedge_{a_j\in f(l)} \left(p(l,a_j,a) \rightarrow \bigwedge_{a_t\in f(\lnot l)\setminus\{a\}}(a_t<a_j \vee a_t>a)\right).
\end{equation}
\begin{equation}
    \label{eq:constraint3}
    %\footnotesize
    \bigwedge_{a_j\in f(\lnot l)\wedge l\in \pc{a}} ((a_j<\aStart{a}) \vee (a_j>\aEnd{a})).
\end{equation}
Which state that between the last achiever $a_j$ of a literal $l$ for an action $a$ and the action $a$ there must not be an action $a_t$ negating said literal. This condition is also enforced by~\autoref{eq:constraint3} that constrains actions negating the literal to happen before the action $a$ has started or after it has finished.

Notice though that in this work, the authors have considered achievers and not enablers. The difference is that an action $a_j$ is an achiever of $a_i$ if $a_j$ \emph{adds} a fluent $l$ that is needed by $a_j$. Enablers instead consider the case in which fluents are also removed. 
%
Since these constraints only consider achievers and not enablers, we need to extend them. We redefine the previous as:
\begin{itemize}
    \item let $f(l)=\{a\in DA \vert add(l)\in \eff{a}\}$ be the set of actions that achieve a literal $l$, and 
    \item let $f(\lnot l)=\{a\in DA \vert del(l) \in \eff{a}\}$ be the set of actions that delete a literal $l$, and
    \item let $F(l) = f(l)\cup f(\lnot l)$ be the union set of $f(l)$ and $f(\lnot l)$, and
    \item let $\displaystyle p(l,a,r)\doteq a<r \wedge \bigwedge_{a_i\in F(l) \setminus\{a,r\}}(a_i<a\vee a_i>r)$ be the last enabler $a$ of an action $r$ for a literal $l$. 
\end{itemize}
Consequently, we need to:
\begin{itemize}
    \item revise~\autoref{eq:constraint1_old} to include all enablers:
        \begin{equation}
            \label{eq:constraint1}
            %\footnotesize
            \bigvee_{a_j\in F(l)\setminus\{a\}} p(l,a_j,a).
        \end{equation}
    \item add two constraints similar to~\autoref{eq:constraint2} and~\autoref{eq:constraint3} to ensure that a predicate that was removed is not added again before the execution of the action:
    \begin{equation}
        \label{eq:constraint2_1}
        %\footnotesize
        \bigwedge_{a_j\in f(\lnot l)} \left(p(l,a_j,a) \rightarrow \bigwedge_{a_t\in f(l)\setminus\{a\}}(a_t<a_j \vee a_t>a)\right).
    \end{equation}
    \begin{equation}
        \label{eq:constraint3_1}
        %\footnotesize
        \bigwedge_{a_j\in f(l)\wedge (\lnot l)\in \pc{a}} ((a_j<\aStart{a}) \vee (a_j>\aEnd{a})).
    \end{equation}    
\end{itemize}

The second aspect of the MILP problem concerns resource allocation. Indeed, as stated before, there are some predicates that are parameterised on resources, e.g., \texttt{available(A)} states whether an agent \texttt{A} is available or not, but it does not ground the value of \texttt{A}. %
One possibility would be to allocate the resources using Prolog, as done in~\cite{saccon2023prolog}, but this choice is greedy since Prolog grounds information with the first predicate that satisfy \texttt{A}. To reduce the makespan of the plan and improve the quality of the same, we delay the grounding to an optimisation phase, leaving Prolog to capture the relationships between actions.

As a first step, we are also going to assume that all the actions coming from a mapping of a higher-level action and that are not mapped into lower-level actions shall maintain the same parameterised predicates as the higher-level action. So the constraint in~\autoref{eq:constraint6} must hold.
\begin{equation}
    \label{eq:constraint6}
    \bigwedge_{a_j\in m(a_i) \wedge m(a_j)\notin M} \left(\bigwedge_{p(x_i) \in \pc{a_i} \wedge p(x_j) \in \pc{a_j}} x_i=x_j \right).
\end{equation}
Moreover, for these constraints, we will consider only predicates that are part of the set $K$, that is predicates that are not resources $R\cap K=\emptyset$.

The objective now is three-fold: 
\begin{itemize}
    \item identify a cost function,
    \item summarise the previous constraints, and
    \item construct a MILP problem to be solved.
\end{itemize}

In this work, the first point is straightforward: we want to minimise the makespan, i.e., the total duration required to complete all tasks or activities.

For the second point, we are trying to find a way to put the previous constraints,~\cref{eq:constraint1,eq:constraint2,eq:constraint2_1,eq:constraint3,eq:constraint3_1,eq:constraint4,eq:constraint5,eq:constraint6} in a compact formulation or structure. We opted to extract the information regarding the enablers using Prolog and to place it into a $N\times N$ matrix $C$, where $N$ is the number of actions and each cell $C_{ij}$ is $1$ if $a_i$ is an enabler of $a_j$ (without considering resources), 0 otherwise. 

We now need to address the resource allocation aspect, specifically, how to distribute the available resources $R$ among the various actions. When performing this task, there are primarily two factors to consider:
\begin{itemize}
    \item A resource cannot be utilised for multiple actions simultaneously.
    \item If two actions share the same resource, they must occur sequentially, meaning one action enables the other.
\end{itemize}

For the first factor, we need to make sure that, for each resource type $r\in R$, the number of actions using the resource at the same time must not be higher than the number of resources of that type available, as shown in~\autoref{eq:resAllocation}.
\begin{equation}
    \displaystyle\forall t \in\{t_0, t_{\tn{END}}\},\,\vert r\vert \geq\sum_{a_i\in TO} t\in\{\aStart{a_i}, \aEnd{a_i}\} \wedge \left( \exists~l(\pmb{x})\in \pc{a_i}\vert r\in\pmb{x}\right).
    \label{eq:resAllocation}
\end{equation}

The second factor must instead be merged with also the precedence constraints embedded in $C$. In particular, we want to express that actions $a_i, a_j$ are in a casual relationship if $C_{ij}=1$ or if they share the same resource. This can be expressed with the following constraint: 
\begin{equation}
    C_{ij} \vee \exists r\in R : r\in\fl{a_i} \wedge r\in\fl{a_j}
    \label{eq:precedence}
\end{equation}
Note that $\fl{a}$ was defined in the problem definition paragraph and represents the set of variables and literals used by the predicates in the preconditions of $a$. 

Finally, we need to set up the MILP problem that consists in finding an assignment of the parameters, of the actions' duration and of the causal relationships, such that the depth of the graph $\mathcal{G}$ representing the plan is minimised. This problem can be expressed as shown in~\autoref{eq:optimization_1}.

%\begin{figure*}[h]
%    \centering
    \begin{equation}
    \everymath={\displaystyle}
    \begin{array}{r@{\hspace*{8mm}}l}
        \label{eq:optimization_1}
        \min_{\mathcal{P}, \mathcal{T}} & t_{\tn{END}} \\
        %&\\
        \textrm{s.t.}   & C_{ij} \vee \exists r\in R : r\in\fl{a_i} \wedge r\in\fl{a_j}, \\
                              & \quad \quad \forall t \in\{t_0, t_{\tn{END}}\}, \\
                              & \quad \quad \quad \quad \vert r\vert > \!\!\sum_{a_i\in TO} \left(t\in\{\aStart{a_i}, \aEnd{a_i}\} \wedge \exists~l(\pmb{x})\in \pc{a_i}\vert r\in\pmb{x}\right).\\
    \end{array}
    \end{equation}
%\end{figure*}

As mentioned before, the MILP part is implemented in Python3 using OR-Tools from Google. The program also checks the consistency of the PO matrix $C$, by making sure that all the actions must have a path to the final actions. 
The output of the MILP solution is basically an STN, which describes both the causal relationship between the actions and also the intervals around the duration of the actions. The initial and final nodes of the STN are factitious as they do not correspond to actual actions, but they simply represent the start and the end of the plan.
The STN is extracted by considering the causal relationship from the $C$ matrix taken as input, and by adding the causal relationship given by the resource allocation task. 
Once we have the STN, we can extract a \bt, which can then be directly executed by integrating it in ROS2. 

\subsubsection{Plan Optimization -- Example}
\label{sssec:PORunExample}
As we said at the end of~\autoref{sssec:PORunEx}  on the running example, that particular plan is not optimisable as the actions are executed in sequence. Let's then consider a slight modification, which consists in finding a plan to move the two blocks in two new positions instead of stacking them in one position. We also have a new agent that can be used to carry out part of the work. 
Our new plan and actions' enablers are the following one:

\begin{minted}[fontsize=\footnotesize]{text}
[0] init()[]
[1] move_table_to_table_start(a1, b1, 1, 1, 1, 2), [0]
[2] move_arm_start(a1, 1, 1), [0,1]
[3] move_arm_end(a1, 1, 1), [0,1,2]
[4] grip_start(a1), [0,1,2,3]
[5] grip_end(a1), [0,1,2,3,4]
[6] move_arm_start(a1, 1, 2), [0,1,2,3,4,5]
[7] move_arm_end(a1, 1, 2), [0,1,2,3,4,5,6]
[8] release_start(a1), [0,1,2,3,4,5,6,7]
[9] release_end(a1), [0,1,2,3,4,5,6,7,8]
[10] move_table_to_table_end(a1, b1, 1, 1, 1, 2), [0,1,2,3,4,5,6,7,8,9]
[11] move_table_to_table_start(a1, b2, 3, 1, 3, 2), [0,10]
[12] move_arm_start(a1, 3, 1), [0,11]
[13] move_arm_end(a1, 3, 1), [0,11,12]
[14] grip_start(a1), [0,11,12,13]
[15] grip_end(a1), [0,11,12,13,14]
[16] move_arm_start(a1, 3, 2), [0,11,12,13,14,15]
[17] move_arm_end(a1, 3, 2), [0,11,12,13,14,15,16]
[18] release_start(a1), [0,11,12,13,14,15,16,17]
[19] release_end(a1), [0,11,12,13,14,15,16,17,18]
[20] move_table_to_table_end(a1, b2, 3, 1, 3, 2), [0,10,11,12,13,14,15,16,17,18,19]
[21] end(), [0,1,2,3,4,5,6,7,8,9,10,11,12,13,14,15,16,17,18,19,20]
\end{minted}

Indeed, action $a_9$ may or may not be an enabler of action $a_{10}$ depending on the resource allocation of the MILP solution. If we have just one agent, then $a_9\in\ach{a_{10}}$, if instead we have more than one agent, then $a_9\not\in\ach{a_{10}}$ and the two actions can be executed at the same time and the plan would be:

\begin{minted}[fontsize=\footnotesize]{text}
[0] init()
[1] move_table_to_table_start(a1, b1, 1, 1, 1, 2)
[2] move_arm_start(a1, 1, 1)
[3] move_arm_end(a1, 1, 1)
[4] grip_start(a1)
[5] grip_end(a1)
[6] move_arm_start(a1, 1, 2)
[7] move_arm_end(a1, 1, 2)
[8] release_start(a1)
[9] release_end(a1)
[10] move_table_to_table_end(a1, b1, 1, 1, 1, 2)
[11] move_table_to_block_start(a2, b2, 3, 1, 3, 2)
[12] move_arm_start(a2, 3, 1)
[13] move_arm_end(a2, 3, 1)
[14] grip_start(a2)
[15] grip_end(a2)
[16] move_arm_start(a2, 3, 2)
[17] move_arm_end(a2, 3, 2)
[18] release_start(a2)
[19] release_end(a2)
[20] move_table_to_block_end(a2, b2, 3, 1, 3, 2)
[21] end()
\end{minted}

% \enrcom{Should I also include a figure? MR: I do not think so!}

% \subsubsection{Plan Generation - Example}

\section{\Btree Generation and Execution}\label{sec:bt}
\newcommand{\seq}[0]{\protect\writings{\texttt{SEQUENCE}}}
\newcommand{\parr}[0]{\protect\writings{\texttt{PARALLEL}}}

% In this section, we first introduce how to convert from a STN to a \bt, and then we provide some details regarding the implementation. 

% \subsubsection{\bt Generation}\label{sssec:btgen}

The conversion from STN to \bt is taken from~\cite{roveriSTNtoBT}. We summarize it here and refer the reader to the main article. 

An STN is a graph with a source and a sink, which can be artificial nodes in the sense that they represent the start and the end of the plan. Each node can have multiple parent and multiple children. Having multiple parents implies that the node cannot be executed as long as all the parents haven not finished and, whereas, having multiple children implies that they will be executed in parallel. 

With this knowledge we can extract a \btree, which is a structure that, starting from the root, ticks all the nodes in the tree until it finishes the last leaf. Nodes in the tree can be of different types:
\begin{itemize}
    \item \emph{action}: they are an action that has to be executed;
    \item \emph{control}: they can be either \seq or \parr and state how the children nodes must be executed;
    \item \emph{condition}: they check whether a condition is correct or not;
\end{itemize}
The ticking of a node means that the node is asked to do its function, e.g., if a \seq node is ticked, then it will tick the children one at a time, while if a condition node is ticked, it will make sure that the condition is satisfied before continuing with the next tick. 

The algorithm %(Algorithm~\ref{alg:stntobt})
to convert the STN to a \bt starts from the fictitious initial node (\verb|init|), and for every node it checks:
\begin{itemize}
    \item The number of children: if there is only one child, then it is a \seq node, otherwise it is a \parr node. 
    \item The number of parents: if there are more than one parents then the node must wait for all the parents to have ticked, before being executed.
    \item The type of the action: if it is a low-level action, then it is inserted into the \bt for execution, otherwise it will not be included.
\end{itemize}

%%%%%%%%%%%%%%%%%%%%%%%%%%%%%%%%%%%%%%%%%

% \begin{algorithm}
% \caption{Algorithm extracting a \btree from an STN.}
% \label{alg:stntobt}
% \KwData{The STN $G$}
% \KwResult{\bt corresponding to the STN}

% \DontPrintSemicolon

% \SetKwProg{extractBT}{EXTRACT\_BT}{}{}

% \SetKwInOut{Input} {In}
% \SetKwInOut{Output}{Out}

% \extractBT{(G)}{
%   \Input{The STN $G$ to convert}
%   \Output{The \bt $\mathcal{T}$}
%   \KwRet{$\mathcal{T}$}\;
% }

% \end{algorithm}

%%%%%%%%%%%%%%%%%%%%%%%%%%%%%%%%%%%%%%%%%

% THIS has been moved to the Implementation details section of Experimental Validation
% \subsubsection{\bt Execution}\label{sssec:btexec}

% As said, the execution of a \btree starts from the root and it gradually ticks the different nodes of the tree until all nodes have been ticked. 

% While \bts have become a de facto standard for executing robotic tasks, no universally accepted framework exists for their creation or execution. Some notable examples include PlanSys2~\cite{martin2021plansys2} and BehaviorTree.CPP~\cite{BehaviorTreeCppWebsite}. PlanSys2 is tightly integrated with ROS2; beyond merely executing \btrees, it can also derive feasible plans from a knowledge base. In contrast, BehaviorTree.CPP is a more general framework that enables the creation and execution of \bts from an XML file. We selected BehaviorTree.CPP since our main objective was to execute APIs from a \bt, which is easily represented using an XML file, while also maintaining maximum generality. Nevertheless, BehaviorTree.CPP also offers a ROS2 wrapper, which can easily be integrated with the flow.

% \enrcom{Maybe it should be moved to Section~\ref{ssec:implementation}?}


%%%%%%%%%%%%%%%%%%%%%%%%%%%%%%%%%%%%%%%%%%%%%%%%%%%%%%%%%%%%%%%%%%%%%%%%
\section{Experimental Validation}\label{sec:experiments}
\section{Experiments and Results}
\subsection{Experiment Settings}

\begin{table*}[ht]
    \centering
    % \small
    \caption{The main results of our experimentation. Each row group corresponds to the results for the given dataset, with each row showcasing the metric results for each model. The columns include all the main approaches, with \textbf{bold} highlighting the best result across all approaches.}
    \small
    \begin{tabular}{llccccc}
      \toprule
      Dataset & Model & Baseline & RAG & CoT & RaR & \rephrase \\
      \midrule
      \multirow[l]{3}{*}{TriviaQA}
          & Llama-3.2 3B  & 59.5 & 82.0 & 87.5  & 86.0 &  \textbf{88.5}    \\
          & Llama-3.1 8B  & 76.5 & 89.5 & 90.5  & 89.5 &  \textbf{92.5}    \\
          & GPT-4o    & 88.7 & 92.7 & 92.7  & 94.7 &  \textbf{96.7}    \\
      \midrule
      \multirow[l]{3}{*}{HotpotQA}
          & Llama-3.2 3B  &  17.5  & 26.0  & 26.5   & 25.0  &  \textbf{31.5}   \\
          & Llama-3.1 8B  &  23.0  & 26.5  & 31.0   & 28.5  &  \textbf{33.5}   \\
          & GPT-4o    &  44.0  & 45.3  & 46.7   & \textbf{47.3}  &  46.7   \\
      \midrule
      \multirow[l]{3}{*}{ASQA}
          & Llama-3.2 3B  &  14.2 & 21.5  & 21.9  & 23.5  &  \textbf{26.6}   \\ 
          & Llama-3.1 8B  &  14.6 & 23.1  & 24.8  & 25.5  &  \textbf{28.8}   \\ 
          & GPT-4o    &  26.8 & 30.4  & \textbf{31.9}  & 30.1 & 31.7 \\ 
      \bottomrule
    \end{tabular}
    \label{tab:main}
\end{table*}



\textbf{Datasets}. We conduct experiments on two datasets: CC-news\footnote{\href{https://huggingface.co/datasets/vblagoje/cc_news}{Huggingface: vblagoje/cc\_news}} and Wikipedia\footnote{\href{https://huggingface.co/datasets/legacy-datasets/wikipedia}{Huggingface: legacy-datasets/Wikipedia}}. CC-news is a large collection of news articles which includes diverse topics and reflects real-world temporal events. Meanwhile, Wikipedia covers general knowledge across a wide range of disciplines, such as history, science, arts, and popular culture.\\
\textbf{LLMs}: We experiment on three models including \gpt~(124M)~\cite{gpt2radford}, \pythia~(1.4B)~\cite{pythia}, and \llama~(7B)~\cite{llama2touvron2023}. This selection of models ensures a wide range of model sizes from small to large that allows us to analyze scaling effects and generalizability across different capacities. \\
\textbf{Evaluation Metrics}. For evaluating language modeling performance, we measure perplexity (PPL), as it reflects the overall effectiveness of the model and is often correlated with improvements in other downstream tasks~\cite{kaplan2020scalinglaws, lmsfewshot}. For defense effectiveness, we consider the attack area under the curve (AUC) value and True Positive Rate (TPR) at low False Positive Rate (FPR). In total, we perform 4 MIAs with different MIA signals. Given the sample $x$, the MIA signal function $f$ is formulated as follows: \\
$\bullet$ Loss~\cite{8429311} utilizes the negative cross entropy loss as the MIA signal. 
    \[f_\text{Loss}(x) = \mathcal{L}_\text{CE}(\theta; x) \]
$\bullet$ Ref-Loss~\cite{Carlini2020ExtractingTD} considers the loss differences between the target model and the attack reference model. To enhance the generality, our experiments ensure there is no data contamination between the training data of the target, reference, and attack models.
    \[f_\text{Ref}(x) = \mathcal{L}_\text{CE}(\theta; x) - \mathcal{L}_\text{CE}(\theta_\text{attack}; x) \]
$\bullet$ Min-K~\cite{shi2024detecting} leverages top K tokens that have the lowest loss values.
    \[f_\text{min-K}(x) = \frac{1}{|\text{min-K(x)}|} \sum_{t_i \in \text{min-K(x)}} -\log(P(t_i|t_{<i};\theta) \]
$\bullet$ Zlib~\cite{Carlini2020ExtractingTD} calibrates the loss signal with the zlib compression size.
    \[ f_\text{zlib}(x) = \mathcal{L}_\text{CE}(\theta; x) / \text{zlib}(x) \]

\noindent \textbf{Baselines}. We present the results of four baselines. \textit{Base} refers to the pretrained LLM without fine tuning. \textit{FT} represents the standard causal language modeling without protection. \textit{Goldfish}~\cite{hans2024be} implements a masking mechanism. \textit{DPSGD}~\cite{abadi2016deep, yu2022differentially} applies gradient clipping and injects noise to achieve  sample-level differential privacy.

\noindent \textbf{Implementation}. We conduct full fine-tuning for \gpt and \pythia. For computing efficiency, \llama fine-tuning is implemented using Low-Rank Adaptation (LoRA)~\cite{hu2022lora} which leads to \textasciitilde4.2M trainable parameters. Additionally, we use subsets of 3K samples to fine-tune the LLMs. We present other implementation details in Appendix~\ref{sec:app-implementation}.

\subsection{Overall Evaluation}
Table~\ref{tab:main_result} provides the overall evaluation compared to several baselines across large language model architectures and datasets. Among these two datasets, CCNews is more challenging, which  leads to higher perplexity  for all LLMs and fine-tuning methods. Additionally, the reference-model-based attack performs the best and demonstrates high privacy risks with attack AUC on the conventional fine-tuned models at 0.95 and 0.85 for Wikipedia and CCNews, respectively. Goldfish achieves similar PPL to the conventional FT method but fails to defend against MIAs. This aligns with the reported results by \citet{hans2024be} that Goldfish resists exact match attacks but only marginally affects MIAs. DPSGD provides a very strong protection in all settings (AUC < 0.55) but with a significant PPL tradeoff. Our proposed \methodname guarantees a robust protection, even slightly better than DPSGD, but with a notably smaller tradeoff on language modeling performance. For example, on the Wikipedia dataset, \methodname delivers perplexity reduction by 15\% to 27\%. Moreover, Table~\ref{tab:tpr} (Appendix~\ref{sec:app-add-res}) provides the TPR at 1\% FPR. Both DPSGD and \methodname successfully reduce the TPR to $\sim$0.02 for all LLMs and datasets. \textit{Overall, across multiple LLM architectures and datasets, \methodname consistently offers ideal privacy protection with  little trade-off in language modeling performance.}

\noindent \textbf{Privacy-Utility Trade-off.}
To investigate the privacy-utility trade-off of the methods, we vary the hyper-parameters of the fine-tuning methods. Particularly, for DPSGD, we adjust the privacy budget $\epsilon$ from (8, 1e-5)-DP to (100, 1e-5)-DP. We modify the masking percentage of Goldfish from 25\% to 50\%. Additionally, we vary the loss weight $\alpha$ from 0.2 to 0.8 for \methodname. Figure~\ref{fig:priv-ult-tradeoff} depicts the privacy-utility trade-off for GPT2 on the CCNews dataset. Goldfish, with very large masking rate (50\%), can slightly reduce the risk of the reference attack but can increase the risks of other attacks. By varying the weight $\alpha$, \methodname offers an adjustable trade-off between privacy protection and language modeling performance. \methodname largely dominates DPSGD and improves the language modeling performance by around 10\% with the ideal privacy protection against MIAs.

\begin{figure}[h]
    \centering
    \includegraphics[width=\linewidth]{figs/privacy-ultility-tradeoff.pdf}
    \caption{Privacy-utility trade-off of the methods while varying hyper-parameters. The Goldfish masking rate is set to 25\%, 33\%, and 50\%. The privacy budget $\epsilon$ of DPSGD is evaluated at 8, 16, 50, and 100. The weight $\alpha$ of \methodname is configured at 0.2, 0.5, and 0.8.}
    \label{fig:priv-ult-tradeoff}
\end{figure}


\subsection{Ablation Study}
\textbf{\methodname without reference models.} To study the impact of the reference model, we adapt \methodname to a non-reference version which directly uses the loss of the current training model (i.e., $s(t_i) = \mathcal{L}_{CE}(\theta; t_i)$) to select the learning and unlearning tokens. This means the unlearning tokens are the tokens that have smallest loss values. Figure~\ref{fig:ppl-auc-noref} presents the training loss and testing perplexity. There is an inconsistent trend of the training loss and testing perplexity. Although the training loss decreases overtime, the test perplexity increases. This result indicates that identifying appropriate unlearning tokens  without a reference model is challenging and conducting unlearning on an incorrect set hurts the language modeling performance.

\begin{figure}[htp]
    \centering
    \includegraphics[width=0.35\textwidth]{figs/train_loss_ppl_noref.pdf}
    \caption{Training Loss and Test Perplexity of \methodname without a reference model.
    % (\lrx{If time permits, it would be better to compare with our training curve here)}
    }
    \label{fig:ppl-auc-noref}
\end{figure}

\noindent \textbf{\methodname with out-of-domain reference models.} To examine the influence of the distribution gap in the reference model, we replace the in-domain trained reference model with the original pretrained base model. 
Figure~\ref{fig:ppl-auc-base-woasc} depicts the language modeling performance and privacy risks in this study. \methodname with an out-of-domain reference model can reduce the privacy risks but yield a significant gap in language modeling performance compared to \methodname using an in-domain reference model.

\noindent \textbf{\methodname without Unlearning.} To study the effects of unlearning tokens, we implement \methodname which use the first term of the loss only ({$\mathcal{L}_{\theta} = \mathcal{L}_{CE}(\theta; \mathcal{T}_h)$}). Figure~\ref{fig:ppl-auc-base-woasc} provides the perplexity and MIA AUC scores in this setting. Generally, without gradient ascent, \methodname can marginally reduce membership inference risks while slightly improving the language modeling performance. The token selection serves as a regularizer that helps to improve the language modeling performance. Additionally, tokens that are learned well in previous epochs may not be selected in the next epochs. This slightly helps to not amplify the memorization on these tokens over epochs.

\begin{figure}[htp]
    \centering
    \includegraphics[width=0.28\textwidth]{figs/auc_vs_ppl_base_woasc.pdf}
    \caption{Privacy-utility trade-off of \methodname with different settings: in-domain reference model, out-domain reference model, and without unlearning}
    \label{fig:ppl-auc-base-woasc}
\end{figure}


\subsection{Training Dynamics}
\textbf{Memorization and Generalization Dynamics}. Figure~\ref{fig:training-dynamics} (left) illustrates the training dynamics of conventional fine tuning and \methodname, while Figure~\ref{fig:training-dynamics} (middle) depicts the membership inference risks. Generally, the gap between training and testing loss of conventional fine-tuning steadily increases overtime, leading to model overfitting and high privacy risks. In contrast, \methodname maintains a stable equilibrium where the gap remains more than 10 times smaller. This equilibrium arises from the dual-purpose loss, which balances learning on hard tokens while actively unlearning memorized tokens. By preventing excessive memorization, \methodname mitigates membership inference risks and enhances generalization.

\begin{figure*}[htp]
    \centering
    \includegraphics[width=0.29\linewidth]{figs/loss_vs_steps_ft_duolearn.pdf}
    \includegraphics[width=0.29\linewidth]{figs/auc_vs_steps_ft_duolearn.pdf}
    \includegraphics[width=0.316\linewidth]{figs/cosine.pdf}
    \caption{Training dynamics of \methodname and the conventional fine-tuning approach. The left and middle figures provide the training-testing gap and membership inference risks, respectively. The testing~$\mathcal{L}_{CE}$ of FT and training~$\mathcal{L}_{CE}$ of \methodname are significantly overlapping, we provide the breakdown in Figure~\ref{fig:add-overlap-breakdown} in Appendix~\ref{sec:app-add-res}. The right figure depicts the cosine similarity of the learning and unlearning gradients of \methodname. Cosine similarity of 1 means entire alignment, 0 indicates orthogonality, and -1 presents full conflict.}
    \label{fig:training-dynamics}
\end{figure*}

\noindent \textbf{Gradient Conflicts}. To study the conflict between the learning and unlearning objectives in our dual-purpose loss function, we compute the gradient for each objective separately. We then calculate the cosine similarity of these two gradients. Figure~\ref{fig:training-dynamics} (right) provides the cosine similarity between two gradients over time. During training, the cosine similarity typically ranges from -0.15 to 0.15. This indicates a mix of mild conflicts and near-orthogonal updates. On average, it decreases from 0.05 to -0.1. This trend reflects increasing gradient misalignment. Early in training, the model may not have strongly learned or memorized specific tokens, so the conflicts are weaker. Overtime, as the model learns more and memorization grows, the divergence between hard and memorized tokens increases, making the gradients less aligned. This gradient conflict is the root of the small degradation of language modeling performance of \methodname compared to the conventional fine tuning approach.

\noindent \textbf{Token Selection Dynamics}. Figure~\ref{fig:token-selection} illustrates the token selection dynamics of \methodname during training. The figure shows that the token selection process is dynamic and changes over epochs. In particular, some tokens are selected as an unlearning from the beginning to the end of the training. This indicates that a token, even without being selected as a learning token initially, can be learned and memorized through the connections with other tokens. This also confirms that simple masking as in Goldfish is not sufficient to protect against MIAs. Additionally, there are a significant number of tokens that are selected for learning in the early epochs but unlearned in the later epochs. This indicates that the model learned tokens and then memorized them over epochs, and the during-training unlearning process is essential to mitigate the memorization risks.

\begin{figure}[htp]
    \centering
    \includegraphics[width=0.7\linewidth]{figs/token-selection-dynamics.pdf}
    \caption{Token Selection Dynamics of \methodname}
    \label{fig:token-selection}
    \vspace{-4mm}
\end{figure}

\subsection{Privacy Backdoor}
To study the worst case of privacy attacks and defense effectiveness under the state-of-the-art MIA, we perform a privacy backdoor -- Precurious~\cite{precurious}. In this setup, the target model undergoes continual fine-tuning from a warm-up model. The attacker then applies a reference-based MIA that leverages the warm-up model as the attack's reference. Table~\ref{tab:backdoor} shows the language modeling and MIA performance on CCNews with GPT-2. Precurious increases the MIA AUC score by 5\%. Goldfish achieves the lowest PPL, aligning with~\citet{hans2024be}, where the Goldfish masking mechanism acts as a regularizer that potentially enhances generalization. Both DPSGD and \methodname provide strong privacy protection, with \methodname offering slightly better defense while maintaining lower perplexity than DPSGD.

% \begin{table}[h]
%     \centering
%     \begin{tabular}{c|cc|cc}
%        \multirow{2}{*}{\textbf{Method}}  & \multicolumn{2}{c}{\textbf{CCNews}} & \multicolumn{2}{c}{\textbf{Wikipedia}} \\ 
%        & \textbf{PPL} & \textbf{AUC} & \textbf{PPL} & \textbf{AUC} \\ \hline
%        \textbf{FT}        & 21.593 & 0.911 \\
%        \textbf{Goldfish}  & \textbf{21.074} & 0.886 \\
%        \textbf{DPSGD}     & 23.279 & 0.533 \\
%        \textbf{DuoLearn}  & 22.296 & \textbf{0.499} \\
%     \end{tabular}
%     \caption{Caption}
%     \label{tab:my_label}
% \end{table}

\begin{table}[h]
    \centering
    \resizebox{\columnwidth}{!}{\begin{tabular}{c|cccccc}
        \textbf{Metric} & \textbf{WU} & \textbf{FT} & \textbf{GF} & \textbf{DP} & \textbf{DuoL} \\ \hline
        \textbf{PPL} & \textit{23.318} & 21.593 & \textbf{21.074} & 23.279 & 22.296  \\
        \textbf{AUC} & \textit{0.500} & 0.911 & 0.886 & 0.533 & \textbf{0.499} \\
    \end{tabular}}
    \caption{Experimental results of privacy backdoor for GPT2 on the CC-news dataset. WU stands for the warm-up model leveraged by Precurious. GF, DP, and DuoL are abbreviations of Goldfish, DPSGD, and \methodname}
    \label{tab:backdoor}
\end{table}

% \subsubsection{Hyperparameter Study}

% \subsubsection{Full fine-tuning versus Parameter efficent fine tuning}

% \subsubsection{Extending to Vision Language Models}




%%%%%%%%%%%%%%%%%%%%%%%%%%%%%%%%%%%%%%%%%%%%%%%%%%%%%%%%%%%%%%%%%%%%%%%%
\section{Related Work}\label{sec:relatedwork}
In this section, we are going to discuss, to the best of our knowledge, the current state of the art and highlight the gap we are filling. 

\subsection{\Kbase Generation}

Knowledge representation is an essential component that endows robots with the cognitive abilities necessary to autonomously execute tasks and make informed decisions \cite{bayat2016requirements,kbSurvey}. This capability underpins the development of systems that can simulate common-sense reasoning in robotic applications.
% for robots to autonomously execute tasks and make choices as it equips them with cognitive abilities \cite{bayat2016requirements,kbSurvey}.

Typically, knowledge systems rely on ontologies to formally describe discrete pieces of information and the relationships among them. In this context, the OpenRobots Ontology (ORO)~\cite{ORO} is designed to store symbolic-level knowledge and events by transforming previously acquired symbols into interconnected concepts. Built upon the framework of semantic triples~\cite{RDF}, ORO facilitates a server architecture where information can be both pushed and pulled, thereby supporting dynamic knowledge management.
% Knowledge systems usually rely on having an ontology to describe pieces of information and their connections. The final goal is to create a system that can actually reproduce common-sense in robots. \textbf{OpenRobots Ontology} (ORO) \cite{ORO} stores symbolic level knowledge and events, and by turning previously acquired symbols into linked concepts. ORO's ontology is built on semantic triples \cite{RDF} that enable the creation of a server where information can be either pushed to or pulled from. 

Ontologies are frequently tailored to specific domains. For example, the Ontology for Robotic Orthopedy Surgery (OROSU)~\cite{OROSU} is dedicated to the medical domain, integrating healthcare ontologies with robotic systems to represent the critical knowledge required during surgical interventions. Similarly, the Worker-cobot ontology~\cite{workerCobot} focuses on industrial applications, supporting collaborative tasks in shared, closed environments through a framework that leverages multi-agent systems and business rules.
% Ontologies are usually specific to a certain area. \textbf{Ontology for Robotic Orthopedy Surgery} (OROSU) \cite{OROSU} is a system centered around the medical domain, integrating healthcare ontologies and robotic systems \cite{OROSUHip}. The goal is to represent the knowledge that should be used during a surgical intervention. Another particular case is \textbf{Worker-cobot} \cite{workerCobot}, which targets industrial robots for collaborative tasks in shared closed environments by exploiting an ontology based on multi-agent systems and business rules. 

In addition to these domain-specific systems, advanced knowledge processing frameworks such as KnowRob \cite{KnowRob1}, now in its second version \cite{KnowRob2}, demonstrate a more comprehensive approach by incorporating environmental data into the reasoning process. Unlike systems that rely solely on deductive closure computation, such as ORO, KnowRob integrates inferential reasoning via Prolog, thus enabling more dynamic and context-aware knowledge management. Furthermore, KnowRob2 expands its capabilities by integrating semantic web information and utilizing a game engine to facilitate the learning of action-related knowledge. This integration allows the system to "reason with its eyes and hands," meaning that it can construct a realistic representation of its environment. Consequently, KnowRob2 is able to abstract and generalize common knowledge from experiential data, thereby enhancing its adaptability to novel situations. 
% \textbf{KnowRob} \cite{KnowRob1} is one of the most advanced knowledge processing systems, now at the second version \cite{KnowRob2}. These systems differentiate from normal KB management system by the fact that they also acquire data from the environment. While a system such as ORO computes deductive closures when creating the knowledge base, KnowRob uses Prolog to integrate inference between information. KnowRob2 integrates also semantic web information and a game engine to learn information about actions. This enables the system to "reason with its eyes and hands", meaning that it can reproduce a realistic knowledge of the environment. It is also able to learn generalised common knowledge from experience, which can then be applied to novel situations \cite{KnowRob2}.
One of the main limitations of systems like KnowRob2 is related to the generation of its knowledge base, which involves complex syntactical structures that complicate the maintenance and scalability of the system, potentially hindering efficient inference and integration of new data. 
Large Language Models (LLMs) can address this limitation by leveraging their ability to parse and generate natural language, thereby producing more flexible and context-aware representations that reduce the reliance on rigid, manually defined syntactic structures in knowledge base generation.


\subsection{\llm for \kb Generation}
Various approaches leveraging LLMs to construct generalizable planning domains have been proposed, demonstrating their capability to convert natural language descriptions of planning problems into robot-oriented planning domains.
These approaches aim to reduce the dependency on handcrafted, domain-specific components traditionally required for solving planning problems. For instance, the ISR-LLM approach proposed in~\cite{10610065} addresses long-horizon planning tasks by converting natural language instructions into PDDL representations and utilizing an LLM-based planner that incorporates the Chain-of-Thought (CoT) mechanism~\cite{wei2022chain} to iteratively refine and plan tasks through intermediate steps. Similarly, the work presented in~\cite{Silver_Dan_Srinivas_Tenenbaum_Kaelbling_Katz_2024} employs LLMs as a generalized planner by using CoT summarization to enhance planning performance, although this method still necessitates predefined planning domain representations.

It is noted that LLMs are not ideally suited to function as standalone planners~\cite{valmeekam2022large}, a limitation that motivates the development of more robust frameworks integrating the strengths of both LLMs and symbolic planning. The LLM+P framework~\cite{liu2023llmp}, for example, capitalizes on the advantages of classical planners by using LLMs to generate PDDL problem files based on natural language descriptions, after which classical planners are employed to solve the problem, thus avoiding the pitfalls of using LLMs as direct planners. Likewise, the approach described in~\cite{oswald2024large} presumes the existence of task-related PDDL domains and uses action-by-action prompting to reconstruct the planning domain through LLMs. Despite their promise, these methods are constrained by the assumption that the problem description is provided as a lifted PDDL domain file.
Also the TwoStep approach \cite{singh2024twostepmultiagenttaskplanning} integrates LLMs with classical PDDL planners for multi-agent task planning by decomposing a unified agent plan into partially independent subgoals that can be allocated to a main agent and a helper agent. This decomposition, though innovative, is limited to two agents and focuses primarily on the coordination between them. 

In contrast to these methods, another research direction seeks to generate the planning domain without any reliance on a symbolic foundation. For example, the NL2Plan approach introduced in~\cite{gestrin2024towards} employs LLMs with CoT prompting to produce a complete PDDL description, and if errors are detected, a feedback loop is established whereby the LLM is queried to refine the domain description.
%\ahmcom{LLM3 uses apriori symbolic knowledge}
Additionally, the LLM$^3$ framework~\cite{wang2024llm3largelanguagemodelbasedtask} offers an LLM-based task and motion planning (TAMP) solution in which LLMs propose symbolic action sequences and select continuous action parameters, supported by a feedback loop that allows motion failures to iteratively refine both the symbolic planning domain and the action parameters.

In contrast to these existing approaches, our method decomposes the decision-making process into two distinct layers, thereby facilitating the mapping of high-level symbolic abstractions to low-level actions with continuous parameters. In this framework, a Prolog knowledge base is generated for each layer. Rather than relying solely on instantaneous actions, our approach employs durative actions that account for temporal constraints and enable the parallel execution of tasks by multiple agents. 
%\ahmcom{Explanation about why our way of 'partial plan-based task assignment' is better than 'subgoal generation-based assignment' can be added} 
This novel technique enhances both the flexibility and efficiency of the system, making it more adept at tackling complex planning tasks that require temporal coordination and multi-agent collaboration.


% OLD BEFORE EDO
% A variety of approaches leveraging LLMs to construct the generalizable planning domains have been suggested, showcasing their ability to translate natural language descriptions of planning problems into robot-oriented planning domains. These methods are designed to minimize the reliance on hand-crafted, domain-specific components for solving planning problems.

% % ISR-LLM
% %\textbf{ISR-LLM} \cite{Zhou2023ISRLLMIS}: framework utilizes LLMs to initially convert the given natural language instructions into PDDL representations. During the planning phase, it employs a Chain-of-Thought(CoT)\cite{wei2022chain} mechanism to break down the task into intermediate steps, enabling step-by-step planning. This leads to simplified problem space, avoiding overly complex problems that may be unsolvable to LLMs. Subsequently, the LLM planner incrementally solves the decomposed subproblem to generate a complete and feasible plan. In the final phase, the the action sequence in the generated initial plan is evaluated by a validator within a loop. If errors are detected at any step, the validator provides feedback to the LLM planner, prompting the refinements to rectify the incorrect actions and generate a new plan. This self-refinement loop proceeds until either the validator detects no errors or the predefined iteration limit is exceeded.
% %=========
% The \textbf{ISR-LLM} approach proposed in \cite{Zhou2023ISRLLMIS} focuses on solving lon-horizon planning tasks by leveraging LLMs.
% It converts natural language instructions into PDDL representations and uses an LLM-based planner that employs the Chain-of-Thought (CoT)\cite{wei2022chain}  mechanism to refine and plan tasks through intermediate steps. Likewise, in \cite{silver2023generalizedplanningpddldomains}, LLMs are served as a generalized planner using CoT summarization to improve planning performance. This approach requires predefined planning domain representations. However, LLMs are not convenient to operate as standalone planners\cite{valmeekam2022large}.
% This limitation is one of the key factors driving our research, as we aim to develop a robust framework that eliminates these shortcomings by leveraging the strengths of LLMs and symbolic planning.
% %=========
% %\textbf{LLM+P} \cite{liu2023llmp}: incorporates the strengths of classical planners into LLMs. LLM+P to take in a natural language description of a planning problem and returns a correct or optimal plan for solving that problem in natural language. Based on the assumtion that the problem description is available as a PDDL domain file with lifted representation, it intially generates PDDL problem file, then leverages classical planners to find a solution quickly, and then translates the found solution back into natural language. The research aims to enable LLMs to solve planning problems correctly without altering the LLMs themselves, even with finetuning. The methodology, called LLM+P, outputs a problem description suitable as input to a general-purpose planner, solves the problem using the general-purpose planner, and converts the output of the planner back to natural language.
% %=========
% In \cite{liu2023llmp}, the authors present \textbf{LLM+P} framework incorporating the strengths of classical planners into LLMs. It utilizes LLMs to generate PDDL problem file given problem description and, rather than using LLM itself as planner, solves it using classical planners.  
% %=========
% %The approach proposed in ~\cite{oswald2024large} utilizes LLMs to reconstruct PDDL domains action-by-action from natural language descriptions, using ground-truth references. They evaluate 7 different LLMs across 9 different planning domains. The quality of the generated planning domains are measured using two novel performance metrics.
% Similarly, the approach proposed in \cite{oswald2024large} presupposes that task-related PDDL domains exist as a foundation and uses LLMs to reconstruct the planning domain through action-by-action prompting.
% Despite being promising, they are constrained by the assumption that the problem description is available as a PDDL domain file with a lifted representation. 
% The TwoStep \cite{singh2024twostepmultiagenttaskplanning} approach combines LLMs and classical PDDL planners for multi-agent task planning. This approach aims to decompose a single agent plan given to a task-specific planning domain into partially independent subgoals that can be distributed across multiple agents, typically designated as the main agent and the helper agent. This decomposition is based on LLMs and limited to 2 agents, focusing on orchestration between the main and helper agents.


% %\textbf{NL2Plan} \cite{gestrin2024nl2planrobustllmdrivenplanning}: employs LLMs to automate PDDL generation process. It requires only natural language description and uses CoT\cite{wei2022chain} reasoing in prompts in order to generate the complete PDDL description. 
% %=========
% Another line of work aims to generate the planning domain without using any symbolic foundation. The\textbf{NL2Plan} approach intoroduced in \cite{gestrin2024nl2planrobustllmdrivenplanning} employs LLMs with CoT\cite{wei2022chain} prompting to generate the complete PDDL description.
% If errors are detected in the generated PDDL domain, a query about the error is sent to LLM for refinement through feedback loops. 
% \ahmcom{LLM3 uses apriori symbolic knowledge}
% In \cite{wang2024llm3largelanguagemodelbasedtask}, the authors introduce
% \textbf{LLM$^3$}, which is an LLM-based task and motion planning (TAMP) framework that uses LLMs to propose symbolic action sequences and select continuous action parameters. A key feature of this approach is its feedback loop, where motion failures are provided to LLMs, enabling iterative refinement of both the symbolic planning domain and action parameters. 
% %=========


% In contrast to these approaches, our method breaks down decision-making into two layers, facilitating the mapping of high-level symbolic abstractions to low-level actions with continuous parameters. The prolog KB is generated for both layer. Rather than relying on instantaneous actions, our approach employs durative actions, accounting for temporal constraints and enabling parallel execution by multiple agents.
% \ahmcom{Explanation about why our way of 'partial plan-based task assignment' is better than 'subgoal generation-based assignment' can be added}
% This technique improves the flexibility and efficiency of the system in tackling complex planning tasks that require both time and multi-agent coordination.




\subsection{\llm for Planning}
Despite inherent challenges related to executing reliable multi-step reasoning and integrating temporally extended and symbolic information within LLM architectures, an alternative research trajectory has emerged that investigates their potential to function as planners or final policy decision-makers in robotic task planning. For example, the Language Models as Zero-Shot Planners approach presented in \cite{huang2022languagemodelszeroshotplanners} leverages LLMs to generate task plans without relying on domain-specific action knowledge; however, its limited environmental awareness and absence of feedback mechanisms often result in plans that include unavailable or contextually irrelevant objects. In contrast, the SayCan framework introduced in \cite{ichter2022do} exploits the semantic capabilities of LLMs to process natural language commands and employs affordance functions to evaluate the log-likelihood of success for a given skill in the current state, thereby selecting the most probable action; nevertheless, its focus on immediate actions restricts its capacity to generate efficient long-horizon plans.

Further advancing this field, the ProgPrompt framework \cite{singh2022progpromptgeneratingsituatedrobot} transforms available actions and their associated parameters into pythonic programs, comprising API calls to action primitives, summarizing comments, and assertions for tracking execution, which are then used to query an LLM for plan generation, effectively bridging the gap between high-level task descriptions and actionable robot directives. Similarly, the Code as Policies approach \cite{10160591} utilizes LLMs to produce programs, Language Model-Generated Programs, that are subsequently executed with Python safety checks. Additionally, the TidyBot system evaluated in \cite{Wu_2023} demonstrates robust performance on both text-based benchmarks and real-world robotic platforms, reinforcing the potential of LLM-based text summarization to generalize robotic tasks without requiring additional training.

Complementing these methodologies, the Common sense-based Open-World Planning framework \cite{Ding_2023} integrates commonsense knowledge extracted from LLMs with rule-based action knowledge from human experts, enabling zero-shot prompting for planning and situation handling in open-world environments. In a related vein, language-guided robot skill learning \cite{ha2023scaling} utilizes LLMs to generate language-labeled robot data that is distilled into a robust multi-task, language-conditioned visuo-motor policy, resulting in a 33.2\% improvement in success rates across five domains. Moreover, the REFLECT framework \cite{liu2023reflect} employs multisensory observations to automatically identify and analyze failed robot actions, providing valuable insights for refining language-based planning.

Other approaches constrain LLM planners to a feasible set of activities, as seen in \cite{10342169}, where plans produced by LLMs are translated from natural language into executable code. The Interactive Task Planning (ITP) framework \cite{li2025interactive} further exemplifies this trend by employing a dual-LLM system: one LLM generates a high-level plan based on task guidelines, user requests, and previously completed steps, while a second LLM maps these high-level steps to low-level functions from a robot skill library. Finally, the Text2Motion framework \cite{Lin_2023} addresses long-horizon tasks by integrating symbolic and geometric reasoning; classical task and motion planning solvers alternate between planning and motion synthesis, using an LLM alongside a library of skills—each featuring a policy and parameterized manipulation primitive—to communicate environmental state in natural language. This framework also assumes prior knowledge of task-relevant objects and their poses to facilitate the planning of feasible trajectories.

These approaches highlight both the promise and the current limitations of LLMs in planning and decision-making, as well as the ongoing efforts to overcome the limitations of such systems by integrating traditional planning paradigms and feedback mechanisms~\cite{kambhampati2024position}.


%%% OLD BEFORE EDO
% Despite the challenges imposed by the LLM architectures in executing reliable multi-step reasoning, integrating temporally extended and symbolic information, a different line of research is also exploring their capability to act as planners or final policy decision-makers for robotic task planning.
% %================ LMZSP
% %\textbf{LMZSP (Language Models as Zero-Shot Planners)} \cite{huang2022language}: It utilizes LLMs to generate task plans without relying on domain-specific action knowledge. However, due to its lack of environmental awareness and inability to receive feedback, LMZSP often produces plans involving unavailable or irrelevant objects for the current context.
% In \cite{huang2022languagemodelszeroshotplanners}, the authors present \textbf{Language Models as Zero-Shot Planners (LMZSP)} approach that utilizes LLMs to generate task plans without relying on domain-specific action knowledge. However, due to its lack of environmental awareness and inability to receive feedback, it often produces plans involving unavailable or irrelevant objects for the current context. 
% %================ SayCan
% %SayCan \cite{ahn2022can} uses LLM's semantic capabilities to process natural language commands. This framework allows robots to perform tasks humans assign using the value function. It uses a logarithmic estimation of the value and affordance functions to determine an action's feasibility. Given the current environment and status, it will take the most likely action to succeed.
% %The SayCan framework \cite{ahn2022i} combines human high-level instructions and their corresponding robot basic tasks into prompts.
% The \textbf{SayCan} approach, as introduced in \cite{ahn2022icanisay}, uses LLM's semantic capabilities to process natural language commands. Through the use of affordance functions, this framework assesses the log-likelihood of success for a given skill in the current state. Based on the current environment and status, it takes the most likely action to succeed. However, this approach primarily focuses on immediate action, which limits its ability to devise efficient plans for long-horizon tasks.
% %================ ProgPrompt
% %\textbf{ProgPrompt} \cite{singh2022progprompt}: Robot plans are represented as Pythonic programs using Python prompting. The program consists of API calls to action primitives, comments summarizing actions, and assertions for tracking execution. The plan functions include API calls to action primitives, comments to outline actions, and assertions for monitoring execution. The program also provides information about the environment and primitive actions to the LLM through prompt construction. The generated plans typically contain actions an agent can take and objects available in the given environment. The LLM fully inferred the plan based on the prompt and executed it on a virtual agent or a physical robot system. The process is tested in the Virtual Home (VH) environment, a deterministic simulation platform for typical household activities. The technique uses a dataset of 70 household tasks and incorporates environmental state feedback in response to assertions. The system's performance is evaluated using success rate, executability, and goal condition recall.

% The \textbf{ProgPrompt}\cite{singh2022progpromptgeneratingsituatedrobot} framework transforms available actions and their associated parameters into pythonic programs, which are then used to query an LLM for plan generation. The program consists of API calls to action primitives, comments summarizing actions, and assertions for tracking execution. Furthermore, the program provides information about the environment to the LLM through prompt construction. By translating actions into a programmatic format, this approach bridges the gap between high-level task descriptions and the specific, actionable steps required for robot task planning.
% %================ PaLM-E => it's a multimodal foundation model not an LLM, so I removed it
% %The \textbf{PaLM-E} \cite{driess2023palm} framework combines sensory input with language processing, connecting perception and natural language.
% %
% %By incorporating sensory feedback alongside language, it enables more grounded inferences, allowing for more context-aware and adaptive decision-making.
% %================ CaP
% In Code as Policies \textbf{(CaP)} \cite{codeAsPolicies}, an LLM produces programs (Language Model-Generated Program LMP). The LLM can use known or undefined functions that will be described later. They then run the Python code with some safety checks.
% %================ TidyBot
% In \cite{Wu_2023}, \textbf{TidyBot} is evaluated on both a text-based benchmark dataset and a real-world robotic system, demonstrating strong performance. The study adds to the idea that text summarization with LLMs can be used to generalize in robotics.
% %It also includes a publicly available benchmark dataset for testing how well the approach works on a real-world mobile manipulation system and how well it generalizes receptacle selection preferences.
% The approach uses off-the-shelf LLMs with no additional training or data collection, leveraging LLMs' commonsense knowledge and summarization abilities to build generalizable personalized preferences for each user.
% %The study also shows that the summarization ability of LLMs enables generalization in robotics.
% %================ TidyBot
% The \textbf{Common sense-based Open-World Planning (COWP)} framework, proposed in \cite{ding2023integrating}, is an open-world planning approach for robots that addresses open-world planning problems. COWP uses action knowledge to enable zero-shot prompting for planning and situation handling, unlike ProgPrompt, which relies on example solutions. COWP extracts commonsense knowledge from LLMs and incorporates rule-based action knowledge from human experts. Reasoning with action knowledge ensures the soundness of task plans generated by COWP while querying LLMs guarantees the openness of COWP to unforeseen situations.
% %================ % Learningg from data
% Language-guided robot skill learning \cite{ha2023scaling} is a way for robots to learn new skills. It uses a large language model to make language-labeled robot data that is then distilled into a solid multi-task language-conditioned visuo-motor policy. This makes success rates 33.2 percent higher across five domains. 
% %================ 
% \textbf{REFLECT} \cite{liu2023reflect} is a framework based on LLM that automatically uses observations from multiple senses to find and analyze failed robot actions. This gives language-based planning helpful information about why the actions failed.
% %================ 
% % Task Planning
% In \cite{ding2023task}, the authors explore the possibilities of language models applied to task and motion planning situations while limiting the LLM planner to a feasible set of activities. Plans produced by LLM are translated into code from natural language in \cite{li2023interactive}.
% %================ 
% The Interactive Task Planning (ITP) framework \cite{IntTPLLM} is made up of two LLMs. The first makes a high-level plan based on task guidelines, user requests, and completed steps that have been remembered. The second one connects each high-level step of the plan to a low-level function from a robot skill library.
% %================ 
% \textbf{Text2Motion} \cite{Lin_2023} is a problem-solving method where a robot solves long-horizon tasks using symbolic and geometric reasoning. Classical TAMP solvers iterate between task planning and motion planning for complex tasks. It addresses challenges concerning the reliable use of LLMs in TAMP settings. The framework uses an LLM and a library of skills, each with a policy and parameterized manipulation primitive. The framework is agnostic of the approach used to obtain these models and conveys the state of the environment to the LLM as natural language. Text2Motion also assumes knowledge of task-relevant objects and their poses to plan feasible trajectories for long-horizon tasks.
% %================ 
% \cite{kambhampati2024llmscantplanhelp}




%%%%%%%%%%%%%%%%%%%%%%%%%%%%%%%%%%%%%%%%%%%%%%%%%%%%%%%%%%%%%%%%%%%%%%%%
\section{Conclusion}\label{sec:conclusions}
\section{Conclusion}

In this paper, we introduce STeCa, a novel agent learning framework designed to enhance the performance of LLM agents in long-horizon tasks. 
STeCa identifies deviated actions through step-level reward comparisons and constructs calibration trajectories via reflection. 
These trajectories serve as critical data for reinforced training. Extensive experiments demonstrate that STeCa significantly outperforms baseline methods, with additional analyses underscoring its robust calibration capabilities.

\section*{Acknowledgements}
% TODO Find the correct place and command for the acknowledgement
Co-funded by the European Union under NextGenerationEU (FAIR - Future AI Research - PE00000013), under project INVERSE (Grant Agreement No. 101136067), under project MAGICIAN (Grant Agreement n. 101120731) and by the project MUR PRIN 2020 (RIPER - Resilient AI-Based Self-Programming and Strategic Reasoning - CUP  E63C22000400001).

\bibliographystyle{elsarticle-num}
\bibliography{biblio.bib}

\end{document}


\appendix



\section*{Appendix A}
\label{AppendixA}

We used a series of data sets in our case studies and as examples in our paper. 

\subsection*{Bike Sharing}
The bike sharing data set is used to predict the number of bike rentals per hour. 

We trained a MLPC Regression model.

We used an 80:20 train:test split resulting in 13903 instances being used for training and in \textsc{Finch}.

In our example case, we used the following features:
\begin{itemize}
    \item count (target): the number of bike rentals that hour
    \item hour=3: the hour for which the bike rentals where recorded. Here: 3am.
    \item workingday=0: if the instance was recorded on a workingday or not. Categoric feature. 0=no, 1=yes.
    \item season=0: in which season the instance was recorded. Categoric feature. 0=winter, 1=spring, 2=summer, 3=autumn.
\end{itemize}



\begin{figure}[h!]
    \centering
    \begin{subfigure}[b]{0.5\textwidth}
        \centering
        \includegraphics[width=1\linewidth]{california1.png}
        \caption{house value per median income}
        \label{fig:california1}
    \end{subfigure}
    \hspace{0.001\textwidth} % Adds some space between the two images
    \begin{subfigure}[b]{0.5\textwidth}
        \centering
        \includegraphics[width=1\linewidth]{california2.png}
        \caption{house value per median income for areas with a population of 2000}
        \label{fig:california2}
    \end{subfigure}
    \hspace{0.001\textwidth} % Adds some space between the two images
    \begin{subfigure}[b]{0.5\textwidth}
        \centering
        \includegraphics[width=1\linewidth]{california3.png}
        \caption{house value per median income for areas with a population of 2000 and 1000 total rooms}
        \label{fig:california3}
    \end{subfigure}
    \hspace{0.001\textwidth} % Adds some space between the two images
    \begin{subfigure}[b]{0.5\textwidth}
        \centering
        \includegraphics[width=1\linewidth]{california4.png}
        \caption{The ground truth is even lower.}
        \label{fig:california4}
    \end{subfigure}
    \caption{The incremental visualization of the interaction of median income, population and total rooms in the california housing data set. Colored areas visualize the change in each step.}
\end{figure}

\subsection*{Titanic}
The titanic data set is used to predict the survival of people on board the titanic. 

We trained an MLPC classifier. The resulting accuracy was 70.23\%.

We used an 80:20 train:test split resulting in 1047 instances being used for training and in \textsc{Finch}.

In the described interaction, we used the following features:
\begin{itemize}
    \item survival(target): If the current person survived.
    \item pclass=1: Which passenger class the current person belonged to. Categoric feature. 1=first class, 2=second class, 3=third class. 
    \item sex=1: The sex of the person. Categoric feature. 0=male, 1=female.
    \item age:30: The age of the person. Here: 30yo.
\end{itemize}




\subsection*{California housing}
The california housing data set is used to predict housing values for block groups in California and was derived from the 190 US census. It contains only continuous variables. The mean predicted housing value is 200.000. 

We trained a GradientBoostingRegressor model on the data. 
It was trained with 100 boosting stages, a learning rate of 0.1 and squared error as the loss function.
The resulting R2 score was 0.77.

We used an 80:20 train:test split resulting in 16,512 instances being used for training and in \textsc{Finch}.

In our observed interaction, we used the following features:
\begin{itemize}
    \item housing value (target): Median house value in US Dollars.
    \item median income: The median income of that block group in 100,000 US Dollars.
    \item population: The number of people residing in the block group.
    \item total rooms: The total number of rooms in that block group.
\end{itemize}

\subsection*{Diabetes}
The diabetes risk factor data set. It is based on the BRFSS telephone study that is performed yearly in the united states.

We used a subset of 10,000 instances for model training. Using a 80:20 train/test split, this resulted in 8000 instnaces being used in \textsc{Finch}.

For better model training, half of the instances are diabetes positive, and half negative. Therefore, the probabilities generated by the model and \textsc{finch} cannot be directly used on a general public.

In our interaction, we considered the following features:
\begin{itemize}
    \item diabetes risk (target): The diabetes risk for the person, that the model predicted.
    \item sex=0: The sex of the person. 0=male, 1=female.
    \item exercise=1: If the person exercises. 1=yes, 0=no.
    \item high blood pressure=1: If the person has high blood pressure. Categoric feature. 1=yes, 0=no.
    \item weight category=3: The weight category of the person. Categoric feature. 0=underweight, 1= normal weight, 2=overweight, 3=obese.  
\end{itemize}

\begin{figure}[h]
    \centering
    \begin{subfigure}[b]{0.5\textwidth}

        \includegraphics[width=0.7\linewidth]{diabetes1.png}
        \caption{diabetes risk per sex}
        \label{fig:diabetes1}
    \end{subfigure}
    \hspace{0.001\textwidth} % Adds some space between the two images
    \begin{subfigure}[b]{0.5\textwidth}
        \centering
        \includegraphics[width=1\linewidth]{diabetes2.png}
        \caption{diabetes risk per sex for people who exercise}
        \label{fig:diabetes2}
    \end{subfigure}
    \hspace{0.001\textwidth} % Adds some space between the two images
    \begin{subfigure}[b]{0.5\textwidth}
        \centering
        \includegraphics[width=1\linewidth]{diabetes3.png}
        \caption{diabetes risk per sex for people who exercise but have high blood pressure}
        \label{fig:diabetes3}
    \end{subfigure}
    \hspace{0.001\textwidth} % Adds some space between the two images
    \begin{subfigure}[b]{0.5\textwidth}
        \centering
        \includegraphics[width=1\linewidth]{diabetes4.png}
        \caption{diabetes risk per sex for people who exercise, have high blood pressure and are obese}
        \label{fig:diabetes4}
    \end{subfigure}
    \caption{Diabetes risk per sex. Incremental interaction of sex, exercise, high blood pressure and weight. Based on the BRFSS data set.}
\end{figure}


\chapter{\textcolor{black}{Edge Network optimization}}\label{app: EN_ib}

In this section the mathematical solution of the optimization problem \eref{eq: EN_ib initial opt problem} in \sref{sec: EN_ib} reported below:

\begin{mini}|s|[0]
    {\mathbf{\Psi}(t)}{\lim_{T \to +\infty}\; \frac{1}{T} \sum_{t=1}^T  \mathbb{E}[P^{tot}(t)] }
    {}{}
    \addConstraint{\lim_{T \to +\infty}\; \frac{1}{T} \sum_{t=1}^T  \mathbb{E}[D_k^{tot}(t)] \leq D_k^{avg}\qquad \forall k }{}
    \addConstraint{ \lim_{T \to +\infty}\; \frac{1}{T} \sum_{t=1}^T  \mathbb{E}[G_k(t)] \leq G_k^{avg}\qquad \forall k }{}
    \addConstraint{0 \leq f_k(t) \leq f_k^{max} \qquad \forall k,t }{}
    \addConstraint{0 \leq R_k(t) \leq R_k^{max}(t) \qquad \forall k,t }{}
    \addConstraint{\beta_k(t) \in \mathcal{B}_k  \qquad \forall k,t}{}
    \addConstraint{0 \leq f^{es}(t) \leq f_{es}^{max} \qquad \forall t}{}
    \addConstraint{f_k^{es}(t) \geq 0 \quad \forall k,t}, \qquad {\sum_{k=1}^K f_k^{es}(t) \leq f_c(t)  \quad \forall t,}{}
\end{mini}

These virtual queues associated to the long-term delay and evaluation metric constraints, $T_k(t)$ and $U_k(t)$ respectively are introduced as follows \cite{Neely2010Lyapunov}:
\begin{align}
    T_k(t+1) &= \max [0,T_k(t) + \varepsilon_k(D_k^{tot}(t) - D_k^{avg})] \\
    U_k(t+1) &= \max [0,U_k(t) + \nu_k(G_k(t) - G_k^{avg})],  
\end{align}
where $\epsilon_k$ and $ \nu_k $ are the learning rate for the update of the virtual queues. 

Based on these virtual queues is possible to define the \textit{Lyapunov function} $L(\mathbf{\Theta}(t))$ as:
\begin{equation}
    L(\mathbf{\Theta}(t)) = \frac{1}{2} \sum_{k=1}^K T_k^2(t) + U_k^2(t),
    \tag{\ref{eq: EN_ib Lyapunov function}}
    \label{app: EN_ib Lyapunov function}
\end{equation}
where $\mathbf{\Theta}(t) = [\{T_k(t)\}_k, \{U_k(t)\}_k]$ is the vector composed by all the virtual queues at time $t$. The idea is to use this Lyapunov function to satisfy the constraints on $D_k^{avg}$ and $G_k^{avg}$ by enforcing the stability of $L(\mathbf{\Theta}(t))$. 

To this scope it is introduced the so called \textit{drift-plus-penalty function}:
\begin{align}
    \Delta(\Theta(t)) &= \mathbb{E}\left[L({\Theta}(t+1))-L({\Theta}(t))+V\cdot P^{tot}(t)  \;\Big|\; \Theta(t)\right] \\
    &=\mathbb{E}\left[\;\sum_{k=1}^K \frac{T_k^2(t+1)-T_k^2(t)}{2} +  \frac{U_k^2(t+1)-U_k^2(t)}{2} +V\cdot P^{tot}(t)\;\; \Big|\;\; \Theta(t)\right]\\
    &= \mathbb{E}\left[\;\sum_{k=1}^K \Delta_{T_k} +  \Delta_{U_k} +V\cdot P^{tot}(t) \;\; \Big|\;\; \Theta(t)\right],
    \label{app: EN_ib drift plus penalty}
\end{align}
where, starting from a generic virtual queue evolving as 
$H(t+1) = \max [0,H(t) +h(t) - \Bar{h}]$ the quantity $\Delta_H$ is defined as follows:
\begin{align*}
    \Delta_H &= \frac{H^2(t+1)-H^2(t)}{2} = \frac{\max [0,(H(t) +h(t) - \Bar{h})^2]-H^2(t)}{2} \\
   &\leq   \frac{(h(t) - \Bar{h})^2}{2} + H(t)[h(t)-\Bar{h}].
\end{align*} 

By applying the same upper bound to $\Delta_{T_k}$ it is possible to obtain:
\begin{align}
    \Delta_{T_k} &= \frac{T_k^2(t+1)-T_k^2(t)}{2} = \frac{\max [0,(T_k(t) + \nu_k(D_k^{tot}(t) - D_k^{avg}))^2]-T_k^2(t)}{2} \\
    &\leq   \nu_k^2\frac{(D_k^{tot}(t) - D_k^{avg})^2}{2} + \nu_k T_k(t)[D_k^{tot}(t) - D_k^{avg}] \\
    &\leq \nu_k^2\frac{(D_k^{max} - D_k^{avg})^2}{2}  + \nu_k T_k(t)[D_k(t) - D_k^{avg}],
    \label{app: EN_ib delta U_k}
\end{align}
where $D_k^{max}(t)$ is the maximum delay allowed for the $k$-th \gls{ed}.

By applying the same reasoning to $\Delta_{U_k}$ it is possible to obtain:
\begin{equation}
    \Delta_{U_k} \leq \nu_k^2\frac{(G_k^{max} - G_k^{avg})^2}{2}  + \nu_k U_k(t)[G_k(t) - G_k^{avg}],
    \label{app: EN_ib delta U_k}
\end{equation}
where $G_k^{max}(t)$ is the maximum value allowed for the evaluation metric for the $k$-th \gls{ed}.

Substituting now \eref{app: EN_ib delta U_k} and \eqref{app: EN_ib delta U_k} inside \eref{app: EN_ib drift plus penalty} and rearranging the terms it is possible to obtain:

\begin{align}
    \Delta_p(\Theta(t)) &\leq
    \sum_{k=1}^K \Bigg{[} \nu_k^2\frac{(D_k^{max} - D_k^{avg})^2}{2} + \nu_k^2\frac{(G_k^{max}(t) - G_k^{avg})^2}{2}  \Bigg{]}  \\ &\;\;\;
    + \mathbb{E} \Bigg{[}\;\sum_{k=1}^K \Big{[} - \varepsilon_k Z_k(t)Q_k^{avg} - \nu_k S_k(t)G_k^{avg}   + \Big|\;\; \Theta(t) \Bigg{]} \\ &\;\;\; + \mathbb{E} \Bigg{[}\;\sum_{k=1}^K \Big{[} \varepsilon_k Z_k(t)Q_k^{tot}(t)  +  \nu_k S_k(t)G_k(t) \Big{]} + V\cdot P^{tot}\;\; \Big|\;\; \Theta(t) \Bigg{]}, 
\end{align}
where some constants that have been taken out of the expected value (first line), while others even if within the expected value do not depend on the optimization parameters (second line).

Pivoting therefore on the Lyapunov optimization it is possible to neglect all these terms. Moreover, it is possible to remove the expected value to obtain the following per-slot optimization:

\begin{mini}|s|[0]
    {\mathbf{\Psi}(t)}{\sum_{k=1}^K \bigg[ \frac{\epsilon_kT_k(t)N_k(t)}{R_k(t)} + \frac{\epsilon_kT_k(t)W_k(t)}{f_k(t)\rho_k } + \frac{\epsilon_kT_k(t)W_{max}^{es}}{f_k^{es}(t) \rho_k^{es}}+}{}{} \breakObjective{\qquad +  \frac{B_k N_0}{h_k(t)} {\rm exp} \left(\frac{R_k(t) ln(2)}{B_k} \right) + V \Gamma_k \eta_k (f_k(t))^3 +}{}{} \breakObjective{\;+  V \eta (f_c(t))^3 + \nu_k U_k(t)G_k(t)\bigg]}{}{}
    \addConstraint{\mathbf{\Psi}(t) \in \mathcal{T}(t),}{}
    \label{eq: EN_ib per-slot opt problem structure}
\end{mini}
where $\mathcal{T}(t)$ indicates the space of possible solutions given by the constraints on the optimization variables. 

at this point it is possible to split the problem for the resource allocation at the \gls{ed} and at the \gls{es}.

\section{Edge Device optimization}\label{app: EN_ib ed opt}
The sub-problem for the \gls{ed} as defined in \eref{eq: EN_ib per-slot opt ed} can be split in two further sub-problems for the transmission rate $R_k(t)$ and the clock frequency $f_k(t)$.

\subsection*{Transmission rate optimal solution}
The sub-problem associated to the transmission rate $R_k(t)$ can be defined as follows:
\begin{mini}|s|[0]
    {R_k(t)}{\frac{\epsilon_kT_k(t)N_k(t)}{R_k(t)} +  V \frac{B_k N_0}{h_k(t)} {\rm exp} \left(\frac{R_k(t) ln(2)}{B_k} \right) }{}{}
    \addConstraint{0 \leq R_k(t) \leq R_k^{max}(t)}{} 
\end{mini}

To simplify the notation, define:
\[
A = \epsilon_k T_k(t) N_k(t), \quad B = V \dfrac{B_k N_0}{h_k(t)}, \quad C = \dfrac{\ln(2)}{B_k}.
\]

Computing the derivative of the objective function $J(R_k(t))$ with respect to $R_k(t)$and set it to zero it is possible to obtain:
\[
\frac{dJ}{dR_k(t)} = -\dfrac{A}{[R_k(t)]^2} + B C \exp\left( C R_k(t) \right) = 0.
\]

By defining Let $x = C R_k(t)$ and $d = \dfrac{A C}{B}$ the derivative can be rearranged as:
\[
x e^{\frac{x}{2}} = \sqrt{d}.
\]

Fortunately, there is an exact solution to this problem and it is based on the \textit{Lambert W function}. By applying the definition and substituting back all the terms it is possible to obtain the final solution:
\begin{equation}
    R_k^*(t) = \frac{2 B_k}{ln(2)}\; W\! \!\left(\sqrt{\frac{\epsilon_k T_k(t)\; ln(2)\; h_k(t)N_k(t)\; }{4 B_k^2\;V \;N_0}}\right)\; \Biggr|_0^{R_k^{max}(t)}
\end{equation}

\subsection*{Clock frequency optimal solution}
The sub-problem associated to the transmission rate $R_k(t)$ can be defined as follows:
\begin{mini}|s|[0]
    {f_k(t)}{\frac{\epsilon_k T_k(t)W_k(t)}{f_k(t)\rho_k } +  V \Gamma_k \eta_k (f_k(t))^3 }{}{}
    \addConstraint{0 \leq f_k(t) \leq f_k^{max}}{} 
\end{mini}

To simplify the notation define:
\[
A = \dfrac{\epsilon_k T_k(t) W_k(t)}{\rho_k}, \quad B = V \Gamma_k \eta_k
\]

Computing the derivative of the objective function $J(f_k(t))$ with respect to $f_k(t)$ and set it to zero it is possible to obtain:
\[
\frac{dJ}{df_k(t)} = -\dfrac{A}{[f_k(t)]^2} + 3B [f_k(t)]^2 = 0
\]

After multiply both sides by $[f_k(t)]^2$, rearranging the terms and substituting back  $A$ and $B$ the final solution is:
\[
f_k(t) = \left( \dfrac{A}{3B} \right)^{1/4} \; \Biggr|_0^{f_k^{max}} \implies f_k^* (t) = \sqrt[4]{\frac{\epsilon_k T_k(t) W_k(t)}{3 V \Gamma_k \eta_k \rho_k} }\; \Biggr|_0^{f_k^{max}},
\]


\section{Edge Server optimization}\label{app: EN_ib es opt}


\begin{mini}|s|[0]
    {\{f_f^{es}(t)\}_k, f_c(t)}{\sum_{k=1}^K \frac{\epsilon_k T_k(t)W_{max}^{es}}{f_k^{es}(t)\rho_k^{es}} + V \eta (f_c(t))^3 }{}{}
    \addConstraint{0 \leq f_c(t) \leq f_c^{max} }{}
    \addConstraint{f_k^{es}(t) \geq 0 \quad \forall k}, \qquad {\sum_{k=1}^K f_k^{es}(t) \leq f_c(t)}{}
\end{mini}

Define:
\[
A_k = \dfrac{ \epsilon_k T_k(t) W_{\text{max}}^{es} }{ \rho_k^{es} }, \quad B = V \eta, \quad S = \sum_{k=1}^K \sqrt{ A_k }
\]


The objective function becomes:
\[
J(\{f_k^{es}(t)\}_k,\ f_c(t)) = \sum_{k=1}^K \dfrac{A_k}{f_k^{es}(t)} + B [f_c(t)]^3
\]

As a first step it is possible to define the associated Lagrangian $L$ of the sub-problem with respect to  $f_k^{es}(t)$ given $f_c(t)$ as:
\[
L = \sum_{k=1}^K \dfrac{A_k}{f_k^{es}(t)} + \lambda \left( \sum_{k=1}^K f_k^{es}(t) - f_c(t) \right)
\]

By deriving it and isolating with respect to $f_k^{es}(t)$ it is possible to obtain:

Solve for $f_k^{es}(t)$:
\[
    \frac{\partial L}{\partial f_k^{es}(t)} = -\dfrac{A_k}{[f_k^{es}(t)]^2} + \lambda = 0  \implies [f_k^{es}(t)]^2 = \dfrac{A_k}{\lambda} \implies f_k^{es}(t) = \sqrt{ \dfrac{A_k}{\lambda} }
\]

Apply the coupling constraint on $f_c(t)$ and by solving for $\lambda$ it is possible to identify:
\[
\sum_{k=1}^K f_k^{es}(t) = \dfrac{1}{\sqrt{\lambda}} \sum_{k=1}^K \sqrt{ A_k } = f_c(t) \implies \sqrt{\lambda} = \dfrac{ S }{ f_c(t) } \implies \lambda = \left( \dfrac{ S }{ f_c(t) } \right)^2
\]

Therefore:
\[
f_k^{es}(t) = \dfrac{ \sqrt{ A_k } }{ S } f_c(t)
\]

This term can now be substituted back into the objective function that is then derived with respect to $f_c(t)$ and set to zero as:

\[
J(f_c(t)) = \dfrac{ S^2 }{ f_c(t) } + B [f_c(t)]^3 \implies \frac{dJ}{df_c(t)} = - \dfrac{ S^2 }{ [f_c(t)]^2 } + 3 B [f_c(t)]^2 = 0
\]

By solving for $f_c(t)$, substituting back the expressions of $A$, $B$ and $S$ and applying the constraints it is possible to obtain the final solution:
\[
    f_c^*(t) = \left[ \left( \dfrac{ S^2 }{ 3 B } \right)^{1/4} \right]_0^{f_c^{\text{max}}}  = \frac{\sqrt{\sum_{k=1}^K \sqrt{\frac{\epsilon_k T_k(t)W_{max}^{es}}{\rho_k^{es}}}}}{\sqrt[4]{3V\eta}} \; \Biggr|_0^{f_{c}^{max}}
\]


Therefore, for every $k$:
\[
f_k^{es}(t) = \dfrac{ \sqrt{ A_k } }{ S } f_c^*(t) = f_k^{es*}(t) = \frac{\sqrt{\frac{\epsilon_k T_k(t)W_{max}^{es}}{\rho_k^{es}}}}{\sqrt{\sum_{k=1}^K \sqrt{\frac{\epsilon_k T_k(t)W_{max}^{es}}{\rho_k^{es}}}}\sqrt[4]{3V\eta} }
\]

%\section{2024 IMO Answers Ablations}
\label{appendix:C}
\begin{table}[H]
\caption{2024 IMO agentic ablation experiments using different methods and models. For each method and model we report if the answer is correct by \C, and \X otherwise. Runs that fail due to moderation restrictions are denoted by \F. Running times, in brackets, are in seconds. Combinatorics problems are denoted by the prefix letter C. For completion we include all 2024 USAMO problems.}
  \centering
  \scriptsize
\begin{tabular}{llcccccc}
\toprule
{\bf 2024 IMO} & {\bf Problem} & {\bf N1} & {\bf N2} & {\bf C3} & {\bf G4} & {\bf C5} & {\bf A6}\\   
\midrule
& \textbf{Answer} & 2k & (1, 1) & NA & NA & 3 & 2\\
\midrule
\textbf{Method} & \textbf{Model} & & & & & &\\
\midrule
\textbf{Zero-shot} 
& o3-mini high & \C (8) & \C (38) & NA (12) & NA (8) & \X (32) & \X (21)\\
& o1-pro & \C (113) & \C (253) & NA (74) & NA (115) & \X (182) & \X (129)\\
& o1 & \C (21) & \X (256) & NA (60) & NA (34) & \X (63) & \X (23)\\
& o1-preview & \X (46) & \C (55) & NA (39) & NA (42) & \X (21) & \X (67) \\
& o1-mini & \X (14) & \X (21) & NA (16) & NA (19) & \X (11) & \X (35) \\
& GPT-4o & \X (7) & \X (10) & NA (6) & NA (8) & \X (5) & \X (12) \\
& Gemini-Exp-1114 & \X (3) & \C (4) & NA (26) & NA (3) & \X (3) & \X (3)\\
& Gemini-1.5-Pro & \X (5) & \X (7) & NA (4) & NA (5) & \X (3) & \X (6) \\
& Claude-3.5-Son. & \X (7) & \X (5) & NA (6) & NA (5) & \X (4) & \X (7) \\
& Llama-3.1 & \X (6) & \X (5) & NA (6) & NA (7) & \X (5) & \X (8) \\
& QwQ-32B-preview & \C (69) & \C (186) & NA (301) & NA (430) & \X (86) & \X (151) \\
\midrule
\textbf{MCTS} 
& o3-mini high & \X (204) & \C (411) & NA (8) & NA (10) & \X (146) & \X (228) \\
% & o1 & () & () & () & () & \X (151) & () \\
& o1-preview & \X (259) & \C (461) & NA (304) & NA (402) & \X (236) & \X (279) \\
& o1-mini & \X (125) & \C (239) & NA (149) & NA (205) & \X (112) & \X (143) \\
& GPT-4o & \X (33) & \C (158) & NA (160) & NA (174) & \X (33) & \C (142) \\
\midrule
\textbf{Best of N sampling} 
& o3-mini high & \C (156) & \X (174) & NA (61) & NA (23) & \X (75) & \C (165) \\
% & o1 & () & () & () & () & \X (80) & () \\
& o1-preview & \X (82) & \C (97) & NA (104) & NA (90) & \X (81) & \X (63) \\
& o1-mini & \C (25) & \X (105) & NA (50) & NA (96) & \X (28) & \X (38) \\
& GPT-4o & \X (21) & \X (24) & NA (33) & NA (20) & \X (6) & \X (19) \\
\midrule
\textbf{Mixture of agents} 
& o3-mini high & \C (521) & \C (961) & NA (10) & NA (12) & \X (129) & \X (205) \\
% & o1 & () & () & () & () & \X (216) & () \\
& o1-preview & \C (331) & \X (401) & NA (353) & NA (387) & \X (224) & \X (288) \\
& o1-mini & \C (155) & \X (323) & NA (160) & NA (263) & \X (113) & \X (188) \\
& GPT-4o & \X (60) & \C (77) & NA (67) & NA (55) & \X (34) & \X (63) \\
\midrule
\textbf{Round trip optimization} 
& o3-mini high & \C (112) & \X(465) & NA (18) & NA (13) & \X (78) & \X (107) \\
% & o1 & () & () & () & () & \X (121) & () \\
& o1-preview & \X (143) & \X (145) & NA (179) & NA (180) & \X (134) & \X (232) \\
& o1-mini & \C (50) & \X (140) & NA (79) & NA (166) & \X (64) & \X (73) \\
& GPT-4o & \X (50) & \C (81) & NA (74) & NA (68) & \X (26) & \X (74) \\
\midrule
\textbf{Z3 Theorem prover} & o3-mini high & \X (47) & \X (166) & NA (56)  & NA (13) & \X (65) & \C (52) \\
% & o1 & () & () & NA & NA & \X (102) & () \\
& o1-preview & \X (72) & \C (78) & NA (105) & NA (76) & \X (79) & \X (107) \\
& o1-mini & \C (25) & \X (191) & NA (61) & NA (77) & \X (17) & \X (51) \\
& GPT-4o & \X (36) & \C (81) & NA (15) & NA (33) & \X (8) & \C (39) \\
\midrule
\textbf{Self-consistency} 
& o3-mini high & \C (120) & \X (445) & NA (9) & NA (21) & \X (91) & \C (231) \\
% & o1 & () & () & NA & NA & \X (367) & () \\
& o1-preview & \C (303) & \C (310) & NA (482) & NA (467) & \X (251) & \C (669) \\
& o1-mini & \C (121) & \C (526) & NA (224) & NA (473) & \X (128) & \X (205) \\
& GPT-4o & \X (109) & \C (126) & NA (118) & NA (97) & \X (33) & \C (127) \\
\midrule
\textbf{Prover-verifier}
& o3-mini high & \C (512) & \C (994) & NA (23) & NA (12) & NA (31) & \X (791) \\
% & o1 & () & () & NA & NA & () & () \\
& o1-preview & \X (475) & \C (539) & NA (434) & NA (325) & \X (314) & \X (437) \\
& o1-mini & \C (107) & \C (211) & NA (83) & NA (190) & \X (91) & \X (167) \\
& GPT-4o & \X (280) & \X (297) & NA (282) & NA (310) & \X (36) & \X (245) \\
\midrule
\textbf{R$\star$} & o3-mini high & \X (24) & \X (12) & NA (61) & NA (45) & \X (89) & \X (148) \\
% & o1 & () & () & NA & NA & () & () \\
& o1-preview & \F (1) & \X (28) & NA (63) & NA (32)  & \X (64) & \X (57) \\
& o1-mini & \F (12) & \F (13) & \F (6) & \F (7) & \X (11) & \F (5) \\
& GPT-4o & \X (243) & \X (256) & NA (219) & NA (180) & \X (55) & \F (204) \\
\midrule
\textbf{Plan Search} & o3-mini high & \F (7) & \F (8) & NA (20) & NA (12) & \F (5) & \F (9) \\
% & o1 & () & () & NA & NA & () & () \\
& o1-preview & \X (127) & \X (182) & NA (105) & NA (141) & \X (164) & \X (102) \\
& o1-mini & \F (40) & \F (50) & \F (24) & NA (52)  & \X (31) & \F (32) \\
& GPT-4o & \X (71) & \X (123) & NA (69) & NA (66) & \X (18) & \C (115) \\
\midrule
\textbf{LEAP} 
& o3-mini high & \C (17) & \C (38) & NA (7) & NA (4)  & \X (15) & \X (33) \\
% & o1 & () & () & () & NA & \X (86) & () \\
& o1-preview & \C (66) & \C (53) & NA (73) & NA (82) & \X (56) & \X (97) \\
& o1-mini & \C (32) & \X (152) & NA (35) & NA (58) & \X (34) & \X (38) \\
& GPT-4o & \X (28) & \X (22) & NA (24) & NA (15) & \X (5) & \X (17) \\
\bottomrule
\end{tabular}
\label{tab:IMO2024_method_model_answer_matrix}
\end{table}





\end{document}
cl}

% Standard package includes
\usepackage{times}
\usepackage{latexsym}

% For proper rendering and hyphenation of words containing Latin characters (including in bib files)
\usepackage[T1]{fontenc}
% For Vietnamese characters
% \usepackage[T5]{fontenc}
% See https://www.latex-project.org/help/documentation/encguide.pdf for other character sets

% This assumes your files are encoded as UTF8
\usepackage[utf8]{inputenc}

% This is not strictly necessary, and may be commented out,
% but it will improve the layout of the manuscript,
% and will typically save some space.
\usepackage{microtype}

% This is also not strictly necessary, and may be commented out.
% However, it will improve the aesthetics of text in
% the typewriter font.
\usepackage{inconsolata}

%Including images in your LaTeX document requires adding
%additional package(s)
\usepackage{graphicx}

% If the title and author information does not fit in the area allocated, uncomment the following
%
%\setlength\titlebox{<dim>}
%
% and set <dim> to something 5cm or larger.

\title{Instructions for *ACL Proceedings}

% Author information can be set in various styles:
% For several authors from the same institution:
% \author{Author 1 \and ... \and Author n \\
%         Address line \\ ... \\ Address line}
% if the names do not fit well on one line use
%         Author 1 \\ {\bf Author 2} \\ ... \\ {\bf Author n} \\
% For authors from different institutions:
% \author{Author 1 \\ Address line \\  ... \\ Address line
%         \And  ... \And
%         Author n \\ Address line \\ ... \\ Address line}
% To start a separate ``row'' of authors use \AND, as in
% \author{Author 1 \\ Address line \\  ... \\ Address line
%         \AND
%         Author 2 \\ Address line \\ ... \\ Address line \And
%         Author 3 \\ Address line \\ ... \\ Address line}

\author{First Author \\
  Affiliation / Address line 1 \\
  Affiliation / Address line 2 \\
  Affiliation / Address line 3 \\
  \texttt{email@domain} \\\And
  Second Author \\
  Affiliation / Address line 1 \\
  Affiliation / Address line 2 \\
  Affiliation / Address line 3 \\
  \texttt{email@domain} \\}

%\author{
%  \textbf{First Author\textsuperscript{1}},
%  \textbf{Second Author\textsuperscript{1,2}},
%  \textbf{Third T. Author\textsuperscript{1}},
%  \textbf{Fourth Author\textsuperscript{1}},
%\\
%  \textbf{Fifth Author\textsuperscript{1,2}},
%  \textbf{Sixth Author\textsuperscript{1}},
%  \textbf{Seventh Author\textsuperscript{1}},
%  \textbf{Eighth Author \textsuperscript{1,2,3,4}},
%\\
%  \textbf{Ninth Author\textsuperscript{1}},
%  \textbf{Tenth Author\textsuperscript{1}},
%  \textbf{Eleventh E. Author\textsuperscript{1,2,3,4,5}},
%  \textbf{Twelfth Author\textsuperscript{1}},
%\\
%  \textbf{Thirteenth Author\textsuperscript{3}},
%  \textbf{Fourteenth F. Author\textsuperscript{2,4}},
%  \textbf{Fifteenth Author\textsuperscript{1}},
%  \textbf{Sixteenth Author\textsuperscript{1}},
%\\
%  \textbf{Seventeenth S. Author\textsuperscript{4,5}},
%  \textbf{Eighteenth Author\textsuperscript{3,4}},
%  \textbf{Nineteenth N. Author\textsuperscript{2,5}},
%  \textbf{Twentieth Author\textsuperscript{1}}
%\\
%\\
%  \textsuperscript{1}Affiliation 1,
%  \textsuperscript{2}Affiliation 2,
%  \textsuperscript{3}Affiliation 3,
%  \textsuperscript{4}Affiliation 4,
%  \textsuperscript{5}Affiliation 5
%\\
%  \small{
%    \textbf{Correspondence:} \href{mailto:email@domain}{email@domain}
%  }
%}

\begin{document}
\maketitle
\begin{abstract}
This document is a supplement to the general instructions for *ACL authors. It contains instructions for using the \LaTeX{} style files for ACL conferences.
The document itself conforms to its own specifications, and is therefore an example of what your manuscript should look like.
These instructions should be used both for papers submitted for review and for final versions of accepted papers.
\end{abstract}

\section{Introduction}

These instructions are for authors submitting papers to *ACL conferences using \LaTeX. They are not self-contained. All authors must follow the general instructions for *ACL proceedings,\footnote{\url{http://acl-org.github.io/ACLPUB/formatting.html}} and this document contains additional instructions for the \LaTeX{} style files.

The templates include the \LaTeX{} source of this document (\texttt{acl\_latex.tex}),
the \LaTeX{} style file used to format it (\texttt{acl.sty}),
an ACL bibliography style (\texttt{acl\_natbib.bst}),
an example bibliography (\texttt{custom.bib}),
and the bibliography for the ACL Anthology (\texttt{anthology.bib}).

\section{Engines}

To produce a PDF file, pdf\LaTeX{} is strongly recommended (over original \LaTeX{} plus dvips+ps2pdf or dvipdf).
The style file \texttt{acl.sty} can also be used with
lua\LaTeX{} and
Xe\LaTeX{}, which are especially suitable for text in non-Latin scripts.
The file \texttt{acl\_lualatex.tex} in this repository provides
an example of how to use \texttt{acl.sty} with either
lua\LaTeX{} or
Xe\LaTeX{}.

\section{Preamble}

The first line of the file must be
\begin{quote}
\begin{verbatim}
\documentclass[11pt]{article}
\end{verbatim}
\end{quote}

To load the style file in the review version:
\begin{quote}
\begin{verbatim}
\usepackage[review]{acl}
\end{verbatim}
\end{quote}
For the final version, omit the \verb|review| option:
\begin{quote}
\begin{verbatim}
\usepackage{acl}
\end{verbatim}
\end{quote}

To use Times Roman, put the following in the preamble:
\begin{quote}
\begin{verbatim}
\usepackage{times}
\end{verbatim}
\end{quote}
(Alternatives like txfonts or newtx are also acceptable.)

Please see the \LaTeX{} source of this document for comments on other packages that may be useful.

Set the title and author using \verb|\title| and \verb|\author|. Within the author list, format multiple authors using \verb|\and| and \verb|\And| and \verb|\AND|; please see the \LaTeX{} source for examples.

By default, the box containing the title and author names is set to the minimum of 5 cm. If you need more space, include the following in the preamble:
\begin{quote}
\begin{verbatim}
\setlength\titlebox{<dim>}
\end{verbatim}
\end{quote}
where \verb|<dim>| is replaced with a length. Do not set this length smaller than 5 cm.

\section{Document Body}

\subsection{Footnotes}

Footnotes are inserted with the \verb|\footnote| command.\footnote{This is a footnote.}

\subsection{Tables and figures}

See Table~\ref{tab:accents} for an example of a table and its caption.
\textbf{Do not override the default caption sizes.}

\begin{table}
  \centering
  \begin{tabular}{lc}
    \hline
    \textbf{Command} & \textbf{Output} \\
    \hline
    \verb|{\"a}|     & {\"a}           \\
    \verb|{\^e}|     & {\^e}           \\
    \verb|{\`i}|     & {\`i}           \\
    \verb|{\.I}|     & {\.I}           \\
    \verb|{\o}|      & {\o}            \\
    \verb|{\'u}|     & {\'u}           \\
    \verb|{\aa}|     & {\aa}           \\\hline
  \end{tabular}
  \begin{tabular}{lc}
    \hline
    \textbf{Command} & \textbf{Output} \\
    \hline
    \verb|{\c c}|    & {\c c}          \\
    \verb|{\u g}|    & {\u g}          \\
    \verb|{\l}|      & {\l}            \\
    \verb|{\~n}|     & {\~n}           \\
    \verb|{\H o}|    & {\H o}          \\
    \verb|{\v r}|    & {\v r}          \\
    \verb|{\ss}|     & {\ss}           \\
    \hline
  \end{tabular}
  \caption{Example commands for accented characters, to be used in, \emph{e.g.}, Bib\TeX{} entries.}
  \label{tab:accents}
\end{table}

As much as possible, fonts in figures should conform
to the document fonts. See Figure~\ref{fig:experiments} for an example of a figure and its caption.

Using the \verb|graphicx| package graphics files can be included within figure
environment at an appropriate point within the text.
The \verb|graphicx| package supports various optional arguments to control the
appearance of the figure.
You must include it explicitly in the \LaTeX{} preamble (after the
\verb|\documentclass| declaration and before \verb|\begin{document}|) using
\verb|\usepackage{graphicx}|.

\begin{figure}[t]
  \includegraphics[width=\columnwidth]{example-image-golden}
  \caption{A figure with a caption that runs for more than one line.
    Example image is usually available through the \texttt{mwe} package
    without even mentioning it in the preamble.}
  \label{fig:experiments}
\end{figure}

\begin{figure*}[t]
  \includegraphics[width=0.48\linewidth]{example-image-a} \hfill
  \includegraphics[width=0.48\linewidth]{example-image-b}
  \caption {A minimal working example to demonstrate how to place
    two images side-by-side.}
\end{figure*}

\subsection{Hyperlinks}

Users of older versions of \LaTeX{} may encounter the following error during compilation:
\begin{quote}
\verb|\pdfendlink| ended up in different nesting level than \verb|\pdfstartlink|.
\end{quote}
This happens when pdf\LaTeX{} is used and a citation splits across a page boundary. The best way to fix this is to upgrade \LaTeX{} to 2018-12-01 or later.

\subsection{Citations}

\begin{table*}
  \centering
  \begin{tabular}{lll}
    \hline
    \textbf{Output}           & \textbf{natbib command} & \textbf{ACL only command} \\
    \hline
    \citep{Gusfield:97}       & \verb|\citep|           &                           \\
    \citealp{Gusfield:97}     & \verb|\citealp|         &                           \\
    \citet{Gusfield:97}       & \verb|\citet|           &                           \\
    \citeyearpar{Gusfield:97} & \verb|\citeyearpar|     &                           \\
    \citeposs{Gusfield:97}    &                         & \verb|\citeposs|          \\
    \hline
  \end{tabular}
  \caption{\label{citation-guide}
    Citation commands supported by the style file.
    The style is based on the natbib package and supports all natbib citation commands.
    It also supports commands defined in previous ACL style files for compatibility.
  }
\end{table*}

Table~\ref{citation-guide} shows the syntax supported by the style files.
We encourage you to use the natbib styles.
You can use the command \verb|\citet| (cite in text) to get ``author (year)'' citations, like this citation to a paper by \citet{Gusfield:97}.
You can use the command \verb|\citep| (cite in parentheses) to get ``(author, year)'' citations \citep{Gusfield:97}.
You can use the command \verb|\citealp| (alternative cite without parentheses) to get ``author, year'' citations, which is useful for using citations within parentheses (e.g. \citealp{Gusfield:97}).

A possessive citation can be made with the command \verb|\citeposs|.
This is not a standard natbib command, so it is generally not compatible
with other style files.

\subsection{References}

\nocite{Ando2005,andrew2007scalable,rasooli-tetrault-2015}

The \LaTeX{} and Bib\TeX{} style files provided roughly follow the American Psychological Association format.
If your own bib file is named \texttt{custom.bib}, then placing the following before any appendices in your \LaTeX{} file will generate the references section for you:
\begin{quote}
\begin{verbatim}
\bibliography{custom}
\end{verbatim}
\end{quote}

You can obtain the complete ACL Anthology as a Bib\TeX{} file from \url{https://aclweb.org/anthology/anthology.bib.gz}.
To include both the Anthology and your own .bib file, use the following instead of the above.
\begin{quote}
\begin{verbatim}
\bibliography{anthology,custom}
\end{verbatim}
\end{quote}

Please see Section~\ref{sec:bibtex} for information on preparing Bib\TeX{} files.

\subsection{Equations}

An example equation is shown below:
\begin{equation}
  \label{eq:example}
  A = \pi r^2
\end{equation}

Labels for equation numbers, sections, subsections, figures and tables
are all defined with the \verb|\label{label}| command and cross references
to them are made with the \verb|\ref{label}| command.

This an example cross-reference to Equation~\ref{eq:example}.

\subsection{Appendices}

Use \verb|\appendix| before any appendix section to switch the section numbering over to letters. See Appendix~\ref{sec:appendix} for an example.

\section{Bib\TeX{} Files}
\label{sec:bibtex}

Unicode cannot be used in Bib\TeX{} entries, and some ways of typing special characters can disrupt Bib\TeX's alphabetization. The recommended way of typing special characters is shown in Table~\ref{tab:accents}.

Please ensure that Bib\TeX{} records contain DOIs or URLs when possible, and for all the ACL materials that you reference.
Use the \verb|doi| field for DOIs and the \verb|url| field for URLs.
If a Bib\TeX{} entry has a URL or DOI field, the paper title in the references section will appear as a hyperlink to the paper, using the hyperref \LaTeX{} package.

\section*{Limitations}

Since December 2023, a "Limitations" section has been required for all papers submitted to ACL Rolling Review (ARR). This section should be placed at the end of the paper, before the references. The "Limitations" section (along with, optionally, a section for ethical considerations) may be up to one page and will not count toward the final page limit. Note that these files may be used by venues that do not rely on ARR so it is recommended to verify the requirement of a "Limitations" section and other criteria with the venue in question.

\section*{Acknowledgments}

This document has been adapted
by Steven Bethard, Ryan Cotterell and Rui Yan
from the instructions for earlier ACL and NAACL proceedings, including those for
ACL 2019 by Douwe Kiela and Ivan Vuli\'{c},
NAACL 2019 by Stephanie Lukin and Alla Roskovskaya,
ACL 2018 by Shay Cohen, Kevin Gimpel, and Wei Lu,
NAACL 2018 by Margaret Mitchell and Stephanie Lukin,
Bib\TeX{} suggestions for (NA)ACL 2017/2018 from Jason Eisner,
ACL 2017 by Dan Gildea and Min-Yen Kan,
NAACL 2017 by Margaret Mitchell,
ACL 2012 by Maggie Li and Michael White,
ACL 2010 by Jing-Shin Chang and Philipp Koehn,
ACL 2008 by Johanna D. Moore, Simone Teufel, James Allan, and Sadaoki Furui,
ACL 2005 by Hwee Tou Ng and Kemal Oflazer,
ACL 2002 by Eugene Charniak and Dekang Lin,
and earlier ACL and EACL formats written by several people, including
John Chen, Henry S. Thompson and Donald Walker.
Additional elements were taken from the formatting instructions of the \emph{International Joint Conference on Artificial Intelligence} and the \emph{Conference on Computer Vision and Pattern Recognition}.

% Bibliography entries for the entire Anthology, followed by custom entries
%\bibliography{anthology,custom}
% Custom bibliography entries only
\bibliography{custom}

\appendix

\section{Example Appendix}
\label{sec:appendix}

This is an appendix.

\end{document}
