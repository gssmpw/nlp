\section*{Limitations and Future Directions}

We acknowledge that our most probable baseline, constructed using probabilities derived from our set of LLM's logits, poses a limitation that we intend to address in future work. We plan to compare each VLM with its corresponding LLM to obtain more reliable results for proper comparisons and analyses of potential statistical biases. Regarding Task $2$, we aim to dive deep into the comparison between the similarity metrics used and the emerging topic of LLMs as judges. Additionally, we intend to further investigate this latter approach to assess its actual validity as a reliable evaluation method. This direction will help us refine accuracy metrics, ultimately enhancing our ability to rigorously test model robustness and consistency across our two specular tasks. Furthermore, we acknowledge that our evaluation did not include large-scale proprietary models (e.g., ChatGPT). Our focus was primarily on testing the performance of open-source and easily exploitable language models to provide a comprehensive overview of their capabilities on the benchmark. Nevertheless, extending our evaluation to bigger models would be beneficial to gain a broader understanding of their performance and to contextualize our findings within the wider landscape of LLM research.

\section*{Acknowledgments}
This work has been carried out while Davide Testa was enrolled in the Italian National Doctorate on Artificial Intelligence run by Sapienza University of Rome in collaboration with Fondazione Bruno Kessler (FBK). Giovanni Bonetta and Bernardo Magnini were supported by the PNRR MUR project \href{https://fondazione-fair.it/}{PE0000013-FAIR} (Spoke 2). Alessandro Lenci was supported by the PNRR MUR project \href{https://fondazione-fair.it/}{PE0000013-FAIR} (Spoke 1). Alessio Miaschi was supported by the PNRR MUR project \href{https://fondazione-fair.it/}{PE0000013-FAIR} (Spoke 5). Lucia Passaro was supported by the EU EIC project EMERGE (Grant No. 101070918).
Alessandro Bondielli was supported by the Italian Ministry of University and Research (MUR) in the framework of the PON 2014-2021 ``Research and Innovation" resources – Innovation Action - DM MUR 1062/2021 - Title of the Research: ``Modelli semantici multimodali per l’industria 4.0 e le digital humanities.''