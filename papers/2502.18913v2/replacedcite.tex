\section{Related Work}
Existing Mandarin-English CS speech datasets can be broadly categorized into read speech and spontaneous speech corpora. Read speech datasets typically contain pre-defined sentences that participants are instructed to read aloud, offering controlled phonetic and linguistic variations but lacking the spontaneity of natural conversations.  

The CECOS dataset____ is one of the earliest Mandarin-English CS corpora, comprising 12.1 hours of read speech from 77 speakers at National Cheng Kung University in Taiwan. While it includes code-switching utterances, it lacks publicly available transcriptions and contains non-native speaker accents. OC16-CE80____ significantly expands the scale, offering 80 hours of read speech from over 1400 speakers, with transcriptions available but not open-sourced. The ASRU dataset____, developed for an ASR challenge, contains 240 hours of predominantly Mandarin speech interspersed with some English. Although transcriptions exist, the dataset is not publicly accessible.  

More recent datasets, such as DOTA-ME-CS____, offer open-source transcriptions and introduce AI-based augmentation techniques (e.g., timbre synthesis, speed variation, and noise addition) to enhance diversity. However, its scale remains relatively small, with only 18.54 hours from 34 speakers. TALCS____ provides a much larger dataset, comprising 587 hours of speech from online teaching scenarios. While it is open-source and valuable for acoustic modeling, its domain-specific nature introduces biases in discourse structure, grammar, and lexical choices, making it less representative of everyday spontaneous conversations.  

Spontaneous CS datasets, in contrast, are essential for modeling real-world language use but present greater challenges in collection and annotation. SEAME____ provides approximately 30 hours of spontaneous Mandarin-English conversations from 92 speakers in Singapore and Malaysia. It includes word-level transcriptions with time-aligned language boundaries, making it a valuable resource for code-switching research. However, the dataset is not freely accessible and requires a purchase, which may limit its availability for broader research applications.
____ compiled 36 hours of spontaneous CS speech across various settings, including conversational meetings and student interviews, but only part-of-speech data is transcribed, limiting its usability for ASR research. ASCEND____ provides a smaller (10.62 hours) yet fully transcribed and open-source dataset of spontaneous, multi-turn CS conversations recorded in Hong Kong, featuring 23 bilingual speakers.  

Despite these advancements, most existing datasets exhibit limitations in scale, availability, or annotation completeness. Many either focus on isolated code-switching utterances rather than full dialogues, or remain inaccessible to the research community. In contrast, our dataset aims to bridge these gaps by providing 104 hours of spontaneous Mandarin-English CS speech, featuring full-length dialogue recordings with comprehensive transcriptions. It captures naturalistic code-switching patterns within extended conversations, making it a valuable resource for both ASR research and broader linguistic analysis.