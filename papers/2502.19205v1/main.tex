%\documentclass{article}

% VLDB template version of 2020-08-03 enhances the ACM template, version 1.7.0:
% https://www.acm.org/publications/proceedings-template
% The ACM Latex guide provides further information about the ACM template

\documentclass[sigconf, nonacm]{acmart}

%% The following content must be adapted for the final version
% paper-specific
\newcommand\vldbdoi{XX.XX/XXX.XX}
\newcommand\vldbpages{XXX-XXX}
% issue-specific
\newcommand\vldbvolume{14}
\newcommand\vldbissue{1}
\newcommand\vldbyear{2020}
% should be fine as it is
\newcommand\vldbauthors{\authors}
\newcommand\vldbtitle{\shorttitle} 
% leave empty if no availability url should be set
\newcommand\vldbavailabilityurl{URL_TO_YOUR_ARTIFACTS}
% whether page numbers should be shown or not, use 'plain' for review versions, 'empty' for camera ready
\newcommand\vldbpagestyle{plain} 

\usepackage[utf8]{inputenc}
\usepackage[english]{babel}
%\usepackage[a4paper]{geometry}
%\usepackage[a4paper,hmargin=2cm]{geometry}
\usepackage[T1]{fontenc}
\usepackage{comment}

% Se importi questo sparisce il simbolo della sommatoria da tutto il pdf... Capisco che il font ora è un po' più brutto però a parte che dobbiamo rispettare i font che ci impongono, il simbolo di sommatoria ci serve pre forza hahaha
%\usepackage{mathpazo}

\usepackage{graphicx} % Required for inserting images
%\usepackage{amsmath,amsfonts,amssymb}
\usepackage{amsmath,amsfonts}
\usepackage{amsthm}
\usepackage{xcolor}
\usepackage[ruled,vlined,boxed,linesnumbered]{algorithm2e}
\usepackage{caption}
\usepackage{subcaption}
\usepackage{pifont}
\usepackage{multirow}
\usepackage{booktabs}
\usepackage{todonotes}
\usepackage{cleveref}
\usepackage{mathtools}
\usepackage{multirow}
\usepackage{makecell}

%\usepackage{ulem}
% \usepackage[pdfencoding=auto]{hyperref}



%\newtheorem{lemma}[definition]{Lemma}
%\newtheorem{claim}[definition]{Claim}
%\newtheorem{fact}[definition]{Fact}
%\newtheorem{theorem}[definition]{Theorem}
%\newtheorem{proposition}[definition]{Proposition}
%\newtheorem{corollary}[definition]{Corollary}
%\newtheorem{observation}[definition]{Observation}
%\newtheorem{example}[definition]{Example}
%\newtheorem{conj}[definition]{Conjecture}



%\title{Friends-of-Friends Similarity Estimation:  Lazy Algorithms for  Incremental Graphs}
\title{Approximate $2$-hop neighborhoods on incremental graphs: \\ An efficient lazy approach }



%\author{}
\date{\today}


%
\setlength\unitlength{1mm}
\newcommand{\twodots}{\mathinner {\ldotp \ldotp}}
% bb font symbols
\newcommand{\Rho}{\mathrm{P}}
\newcommand{\Tau}{\mathrm{T}}

\newfont{\bbb}{msbm10 scaled 700}
\newcommand{\CCC}{\mbox{\bbb C}}

\newfont{\bb}{msbm10 scaled 1100}
\newcommand{\CC}{\mbox{\bb C}}
\newcommand{\PP}{\mbox{\bb P}}
\newcommand{\RR}{\mbox{\bb R}}
\newcommand{\QQ}{\mbox{\bb Q}}
\newcommand{\ZZ}{\mbox{\bb Z}}
\newcommand{\FF}{\mbox{\bb F}}
\newcommand{\GG}{\mbox{\bb G}}
\newcommand{\EE}{\mbox{\bb E}}
\newcommand{\NN}{\mbox{\bb N}}
\newcommand{\KK}{\mbox{\bb K}}
\newcommand{\HH}{\mbox{\bb H}}
\newcommand{\SSS}{\mbox{\bb S}}
\newcommand{\UU}{\mbox{\bb U}}
\newcommand{\VV}{\mbox{\bb V}}


\newcommand{\yy}{\mathbbm{y}}
\newcommand{\xx}{\mathbbm{x}}
\newcommand{\zz}{\mathbbm{z}}
\newcommand{\sss}{\mathbbm{s}}
\newcommand{\rr}{\mathbbm{r}}
\newcommand{\pp}{\mathbbm{p}}
\newcommand{\qq}{\mathbbm{q}}
\newcommand{\ww}{\mathbbm{w}}
\newcommand{\hh}{\mathbbm{h}}
\newcommand{\vvv}{\mathbbm{v}}

% Vectors

\newcommand{\av}{{\bf a}}
\newcommand{\bv}{{\bf b}}
\newcommand{\cv}{{\bf c}}
\newcommand{\dv}{{\bf d}}
\newcommand{\ev}{{\bf e}}
\newcommand{\fv}{{\bf f}}
\newcommand{\gv}{{\bf g}}
\newcommand{\hv}{{\bf h}}
\newcommand{\iv}{{\bf i}}
\newcommand{\jv}{{\bf j}}
\newcommand{\kv}{{\bf k}}
\newcommand{\lv}{{\bf l}}
\newcommand{\mv}{{\bf m}}
\newcommand{\nv}{{\bf n}}
\newcommand{\ov}{{\bf o}}
\newcommand{\pv}{{\bf p}}
\newcommand{\qv}{{\bf q}}
\newcommand{\rv}{{\bf r}}
\newcommand{\sv}{{\bf s}}
\newcommand{\tv}{{\bf t}}
\newcommand{\uv}{{\bf u}}
\newcommand{\wv}{{\bf w}}
\newcommand{\vv}{{\bf v}}
\newcommand{\xv}{{\bf x}}
\newcommand{\yv}{{\bf y}}
\newcommand{\zv}{{\bf z}}
\newcommand{\zerov}{{\bf 0}}
\newcommand{\onev}{{\bf 1}}

% Matrices

\newcommand{\Am}{{\bf A}}
\newcommand{\Bm}{{\bf B}}
\newcommand{\Cm}{{\bf C}}
\newcommand{\Dm}{{\bf D}}
\newcommand{\Em}{{\bf E}}
\newcommand{\Fm}{{\bf F}}
\newcommand{\Gm}{{\bf G}}
\newcommand{\Hm}{{\bf H}}
\newcommand{\Id}{{\bf I}}
\newcommand{\Jm}{{\bf J}}
\newcommand{\Km}{{\bf K}}
\newcommand{\Lm}{{\bf L}}
\newcommand{\Mm}{{\bf M}}
\newcommand{\Nm}{{\bf N}}
\newcommand{\Om}{{\bf O}}
\newcommand{\Pm}{{\bf P}}
\newcommand{\Qm}{{\bf Q}}
\newcommand{\Rm}{{\bf R}}
\newcommand{\Sm}{{\bf S}}
\newcommand{\Tm}{{\bf T}}
\newcommand{\Um}{{\bf U}}
\newcommand{\Wm}{{\bf W}}
\newcommand{\Vm}{{\bf V}}
\newcommand{\Xm}{{\bf X}}
\newcommand{\Ym}{{\bf Y}}
\newcommand{\Zm}{{\bf Z}}

% Calligraphic

\newcommand{\Ac}{{\cal A}}
\newcommand{\Bc}{{\cal B}}
\newcommand{\Cc}{{\cal C}}
\newcommand{\Dc}{{\cal D}}
\newcommand{\Ec}{{\cal E}}
\newcommand{\Fc}{{\cal F}}
\newcommand{\Gc}{{\cal G}}
\newcommand{\Hc}{{\cal H}}
\newcommand{\Ic}{{\cal I}}
\newcommand{\Jc}{{\cal J}}
\newcommand{\Kc}{{\cal K}}
\newcommand{\Lc}{{\cal L}}
\newcommand{\Mc}{{\cal M}}
\newcommand{\Nc}{{\cal N}}
\newcommand{\nc}{{\cal n}}
\newcommand{\Oc}{{\cal O}}
\newcommand{\Pc}{{\cal P}}
\newcommand{\Qc}{{\cal Q}}
\newcommand{\Rc}{{\cal R}}
\newcommand{\Sc}{{\cal S}}
\newcommand{\Tc}{{\cal T}}
\newcommand{\Uc}{{\cal U}}
\newcommand{\Wc}{{\cal W}}
\newcommand{\Vc}{{\cal V}}
\newcommand{\Xc}{{\cal X}}
\newcommand{\Yc}{{\cal Y}}
\newcommand{\Zc}{{\cal Z}}

% Bold greek letters

\newcommand{\alphav}{\hbox{\boldmath$\alpha$}}
\newcommand{\betav}{\hbox{\boldmath$\beta$}}
\newcommand{\gammav}{\hbox{\boldmath$\gamma$}}
\newcommand{\deltav}{\hbox{\boldmath$\delta$}}
\newcommand{\etav}{\hbox{\boldmath$\eta$}}
\newcommand{\lambdav}{\hbox{\boldmath$\lambda$}}
\newcommand{\epsilonv}{\hbox{\boldmath$\epsilon$}}
\newcommand{\nuv}{\hbox{\boldmath$\nu$}}
\newcommand{\muv}{\hbox{\boldmath$\mu$}}
\newcommand{\zetav}{\hbox{\boldmath$\zeta$}}
\newcommand{\phiv}{\hbox{\boldmath$\phi$}}
\newcommand{\psiv}{\hbox{\boldmath$\psi$}}
\newcommand{\thetav}{\hbox{\boldmath$\theta$}}
\newcommand{\tauv}{\hbox{\boldmath$\tau$}}
\newcommand{\omegav}{\hbox{\boldmath$\omega$}}
\newcommand{\xiv}{\hbox{\boldmath$\xi$}}
\newcommand{\sigmav}{\hbox{\boldmath$\sigma$}}
\newcommand{\piv}{\hbox{\boldmath$\pi$}}
\newcommand{\rhov}{\hbox{\boldmath$\rho$}}
\newcommand{\upsilonv}{\hbox{\boldmath$\upsilon$}}

\newcommand{\Gammam}{\hbox{\boldmath$\Gamma$}}
\newcommand{\Lambdam}{\hbox{\boldmath$\Lambda$}}
\newcommand{\Deltam}{\hbox{\boldmath$\Delta$}}
\newcommand{\Sigmam}{\hbox{\boldmath$\Sigma$}}
\newcommand{\Phim}{\hbox{\boldmath$\Phi$}}
\newcommand{\Pim}{\hbox{\boldmath$\Pi$}}
\newcommand{\Psim}{\hbox{\boldmath$\Psi$}}
\newcommand{\Thetam}{\hbox{\boldmath$\Theta$}}
\newcommand{\Omegam}{\hbox{\boldmath$\Omega$}}
\newcommand{\Xim}{\hbox{\boldmath$\Xi$}}


% Sans Serif small case

\newcommand{\Gsf}{{\sf G}}

\newcommand{\asf}{{\sf a}}
\newcommand{\bsf}{{\sf b}}
\newcommand{\csf}{{\sf c}}
\newcommand{\dsf}{{\sf d}}
\newcommand{\esf}{{\sf e}}
\newcommand{\fsf}{{\sf f}}
\newcommand{\gsf}{{\sf g}}
\newcommand{\hsf}{{\sf h}}
\newcommand{\isf}{{\sf i}}
\newcommand{\jsf}{{\sf j}}
\newcommand{\ksf}{{\sf k}}
\newcommand{\lsf}{{\sf l}}
\newcommand{\msf}{{\sf m}}
\newcommand{\nsf}{{\sf n}}
\newcommand{\osf}{{\sf o}}
\newcommand{\psf}{{\sf p}}
\newcommand{\qsf}{{\sf q}}
\newcommand{\rsf}{{\sf r}}
\newcommand{\ssf}{{\sf s}}
\newcommand{\tsf}{{\sf t}}
\newcommand{\usf}{{\sf u}}
\newcommand{\wsf}{{\sf w}}
\newcommand{\vsf}{{\sf v}}
\newcommand{\xsf}{{\sf x}}
\newcommand{\ysf}{{\sf y}}
\newcommand{\zsf}{{\sf z}}


% mixed symbols

\newcommand{\sinc}{{\hbox{sinc}}}
\newcommand{\diag}{{\hbox{diag}}}
\renewcommand{\det}{{\hbox{det}}}
\newcommand{\trace}{{\hbox{tr}}}
\newcommand{\sign}{{\hbox{sign}}}
\renewcommand{\arg}{{\hbox{arg}}}
\newcommand{\var}{{\hbox{var}}}
\newcommand{\cov}{{\hbox{cov}}}
\newcommand{\Ei}{{\rm E}_{\rm i}}
\renewcommand{\Re}{{\rm Re}}
\renewcommand{\Im}{{\rm Im}}
\newcommand{\eqdef}{\stackrel{\Delta}{=}}
\newcommand{\defines}{{\,\,\stackrel{\scriptscriptstyle \bigtriangleup}{=}\,\,}}
\newcommand{\<}{\left\langle}
\renewcommand{\>}{\right\rangle}
\newcommand{\herm}{{\sf H}}
\newcommand{\trasp}{{\sf T}}
\newcommand{\transp}{{\sf T}}
\renewcommand{\vec}{{\rm vec}}
\newcommand{\Psf}{{\sf P}}
\newcommand{\SINR}{{\sf SINR}}
\newcommand{\SNR}{{\sf SNR}}
\newcommand{\MMSE}{{\sf MMSE}}
\newcommand{\REF}{{\RED [REF]}}

% Markov chain
\usepackage{stmaryrd} % for \mkv 
\newcommand{\mkv}{-\!\!\!\!\minuso\!\!\!\!-}

% Colors

\newcommand{\RED}{\color[rgb]{1.00,0.10,0.10}}
\newcommand{\BLUE}{\color[rgb]{0,0,0.90}}
\newcommand{\GREEN}{\color[rgb]{0,0.80,0.20}}

%%%%%%%%%%%%%%%%%%%%%%%%%%%%%%%%%%%%%%%%%%
\usepackage{hyperref}
\hypersetup{
    bookmarks=true,         % show bookmarks bar?
    unicode=false,          % non-Latin characters in AcrobatÕs bookmarks
    pdftoolbar=true,        % show AcrobatÕs toolbar?
    pdfmenubar=true,        % show AcrobatÕs menu?
    pdffitwindow=false,     % window fit to page when opened
    pdfstartview={FitH},    % fits the width of the page to the window
%    pdftitle={My title},    % title
%    pdfauthor={Author},     % author
%    pdfsubject={Subject},   % subject of the document
%    pdfcreator={Creator},   % creator of the document
%    pdfproducer={Producer}, % producer of the document
%    pdfkeywords={keyword1} {key2} {key3}, % list of keywords
    pdfnewwindow=true,      % links in new window
    colorlinks=true,       % false: boxed links; true: colored links
    linkcolor=red,          % color of internal links (change box color with linkbordercolor)
    citecolor=green,        % color of links to bibliography
    filecolor=blue,      % color of file links
    urlcolor=blue           % color of external links
}
%%%%%%%%%%%%%%%%%%%%%%%%%%%%%%%%%%%%%%%%%%%



\begin{document}

%%
%% The "author" command and its associated commands are used to define the authors and their affiliations.
\author{Luca Becchetti}
\affiliation{%
  \institution{\textit{Sapienza} University of Rome}
   \country{Italy}
   }
\email{becchetti@diag.uniroma1.it}

\author{Andrea Clementi}
\affiliation{%
  \institution{ \textit{Tor Vergata} University of Rome}
   \country{Italy}
   }
   \email{clementi@mat.uniroma2.it}
%\orcid{0000-0002-1825-0097}
 
% \institution{The Th{\o}rv{\"a}ld Group}
   
\author{Luciano Gualà}
%\orcid{0000-0001-5109-3700}
\affiliation{%
  \institution{ \textit{Tor Vergata} University of Rome}
   \country{Italy}
   }
   \email{guala@mat.uniroma2.it}

\author{Luca Pep\'e Sciarria}
\affiliation{%
  \institution{ \textit{Tor Vergata} University of Rome}
   \country{Italy}
   }
   \email{luca.pepesciarria@gmail.com}

\author{Alessandro Straziota}
\affiliation{%
  \institution{\textit{Tor Vergata} University of Rome}
   \country{Italy}
   }
    \email{alessandro.straziota@uniroma2.it}

\author{ Matteo Stromieri}
\affiliation{%
  \institution{ \textit{Tor Vergata} University of Rome}
   \country{Italy}
   }
   \email{matteo.stromieri@students.uniroma2.eu}

%%
%% The abstract is a short summary of the work to be presented in the
%% article.
\begin{abstract}
 \begin{abstract}  
Test time scaling is currently one of the most active research areas that shows promise after training time scaling has reached its limits.
Deep-thinking (DT) models are a class of recurrent models that can perform easy-to-hard generalization by assigning more compute to harder test samples.
However, due to their inability to determine the complexity of a test sample, DT models have to use a large amount of computation for both easy and hard test samples.
Excessive test time computation is wasteful and can cause the ``overthinking'' problem where more test time computation leads to worse results.
In this paper, we introduce a test time training method for determining the optimal amount of computation needed for each sample during test time.
We also propose Conv-LiGRU, a novel recurrent architecture for efficient and robust visual reasoning. 
Extensive experiments demonstrate that Conv-LiGRU is more stable than DT, effectively mitigates the ``overthinking'' phenomenon, and achieves superior accuracy.
\end{abstract}  
\end{abstract}
\maketitle

%%% do not modify the following VLDB block %%
%%% VLDB block start %%%
\pagestyle{\vldbpagestyle}
\begingroup\small\noindent\raggedright\textbf{PVLDB Reference Format:}\\
\vldbauthors. \vldbtitle. PVLDB, \vldbvolume(\vldbissue): \vldbpages, \vldbyear.\\
\href{https://doi.org/\vldbdoi}{doi:\vldbdoi}
\endgroup
\begingroup
\renewcommand\thefootnote{}\footnote{\noindent
This work is licensed under the Creative Commons BY-NC-ND 4.0 International License. Visit \url{https://creativecommons.org/licenses/by-nc-nd/4.0/} to view a copy of this license. For any use beyond those covered by this license, obtain permission by emailing \href{mailto:info@vldb.org}{info@vldb.org}. Copyright is held by the owner/author(s). Publication rights licensed to the VLDB Endowment. \\
\raggedright Proceedings of the VLDB Endowment, Vol. \vldbvolume, No. \vldbissue\ %
ISSN 2150-8097. \\
\href{https://doi.org/\vldbdoi}{doi:\vldbdoi} \\
}\addtocounter{footnote}{-1}\endgroup
%%% VLDB block end %%%

%%% do not modify the following VLDB block %%
%%% VLDB block start %%%
\ifdefempty{\vldbavailabilityurl}{}{
\vspace{.3cm}
\begingroup\small\noindent\raggedright\textbf{PVLDB Artifact Availability:}\\
The source code, data, and/or other artifacts have been made available at \url{https://github.com/Gnumlab/graph_ball}.
%\url{\vldbavailabilityurl}.
\endgroup
}
%%% VLDB block end %%%

\section{Introduction}
In this paper, we consider the task of processing a possibly large, dynamic graph $G(V,E)$, incrementally provided as a stream of edge insertions, so that at any point of the stream it is possible to efficiently evaluate different queries that involve functions of the \textit{$h$-hop neighborhoods} of its vertices. For a vertex $v\in V$, its $h$-hop neighborhood is simply the \emph{set} of vertices that are within $h$ hops from $v$. In the remainder, $h$-hop neighborhoods are called \textit{$h$-balls} for brevity. As concrete examples of query types we consider, one might want to estimate the size of the $2$-ball at a given vertex, or the Jaccard similarity between the $2$-balls centered at any given two vertices, or other indices of a similar flavor that depend on the intersection or union between $1$-balls and/or $2$-balls, just to mention a few.

Neighborhood-based indices are common in key mining tasks, such as link prediction in social \cite{liben2003link} and biological networks \cite{wang2023assessment} or to describe statistical properties of large social graphs \cite{becchetti2008efficient}. For example, $2$-hop neighborhoods are important in social network analysis and similarity-based link prediction \cite{zhou2021experimental,zareie2020similarity,Sim-Nodes_Survey_2024}, while accurate approximations of $h$-balls' sizes are used to estimate key statistical properties of (very) large social networks \cite{boldi2011hyperanf,backstrom2012four}, or as link-based features in classifiers for Web spam detection \cite{becchetti2008link}. 
 
When the graph is static, an effective approach to this general task is to treat $h$-balls as subsets of the vertices of the graph, suitably represented using approximate summaries or sketches \cite{agarwal2013mergeable}. This line of attack has proved successful, for example in the efficient and scalable evaluation of important neighborhood-based queries on massive graphs that in part or mostly reside on secondary storage \cite{feigenbaum2005graph,mcgregor2014graph,becchetti2008efficient,boldi2011hyperanf}.
%and can only be accessed over consecutive, sequential passes
  

Nowadays, standard applications in social network analysis often entail dynamic scenarios in which  input graphs \textit{evolve over time}, under a sequence of  edge insertions and possibly deletions \cite{aggarwal2014evolutionary}. 



With respect to a static scenario, the dynamic case poses new and significant challenges even in the incremental setting, as soon as $h > 1$.\footnote{The case $h = 1$ is considerably simpler and it \textit{barely relates to graphs}: adding or removing one edge $(u, v)$ simply requires updating the $1$-balls of $u$ and $v$ accordingly, i.e., updating two corresponding set sketches by adding or removing one item. This has been the focus of extensive work in the recent past that we discuss in \Cref{subse:related}.} To see this, it may be useful to briefly sketch the cost of maintaining $1$- and $2$-balls exactly under a sequence of edge insertions, as we discuss in more detail in Section \ref{sec:detalgo}. 
When $h = 2$, each \texttt{Insert}($u, v$) operation entails (see Algorithm \ref{algo:naive} and Figure \ref{fig:basic-example}): i) updating the $2$-ball of $u$ to its union with the $1$-ball of $v$ and viceversa (what we call a \textit{heavy} update); ii) adding $v$ to the $2$-ball of \emph{each} neighbor of $u$ and viceversa (what we call a \textit{light} update). Both heavy and light updates can result in high computational costs per edge insertion: a heavy update can be expensive if at least one of the neighborhoods to merge is large; on the other hand, light updates are relatively inexpensive, but they can be numerous when large neighborhoods are involved, again resulting in a high overall cost per edge insertion. Unfortunately, $h$-balls can grow extremely fast with $h$ in many social networks, already as one switches from $h = 1$ to $h = 2$ \cite{becchetti2008link,backstrom2012four}. For the same reason, maintaining lossless representations of $2$-balls for each vertex of such networks might require considerable memory resources and might negatively impact the cost of serving neighborhood-based queries that involve moderately or highly central vertices.


To address the aforementioned issues for graphs that reside in main memory, one might want to trade some degree of accuracy for the following broad goals: 1) designing algorithms with low update costs, possibly $O(1)$ amortized per edge insertion; 2) minimizing memory footprint beyond what is needed to store the graph; 3) maintaining $1$- and $2$-balls using data structures that afford efficient, real-time computation of queries as the ones mentioned earlier with minimal memory footprint.

\iffalse
In particular (see Algorithm \ref{algo:naive}), if we are to maintain $1$- and $2$-balls exactly, every \texttt{Insert}($u, v$) operation entails (among others) the following updates (Algorithm \ref{algo:naive} and Figure \ref{fig:basic-example}): i) $\ball_2(u)\leftarrow\ball_2(u)\cup\ball_1(v)$; ii) $\ball_2(v)\leftarrow\ball_2(v)\cup\ball_1(u)$; iii) $\ball_2(x)\leftarrow\ball_2(x)\cup\{v\}$ for every $x\in\neigh(u)$; iv) $\ball_2(y)\leftarrow\ball_2(y)\cup\{u\}$ for every $y\in\neigh(v)$. Each single update i) and ii) (heavy updates) can be computationally expensive when it involves large neighborhoods, while updates iii) and iv) (light updates) are relatively inexpensive but they can be numerous, again in the case of large neighborhoods. Overall, processing edge insertions can be very expensive in some graphs, such as social networks, where $|\ball_h(u)|$ can grow extremely fast, already switching from $h = 1$ to $h = 2$ \cite{becchetti2008link,backstrom2012four}. Hence, while affording exact answering of neighborhood-based queries, maintaining $1$- and $2$-balls exactly may not be feasible in practice because of the associated memory footprint and, to a lesser extent, because of the costs of neighborhood-based queries that involve moderately or highly central vertices. Ideally, in the quest for scalable solutions for possibly large graphs residing in main memory, one might want to trade some degree of accuracy for the following broad goals: 1) designing algorithms with low update costs, possibly $\bigO(1)$ amortized per edge insertion; 2) minimizing memory footprint beyond what is needed to store the graph;\rem{Forse qui dovremmo mettere una footnote per dire che assumiamo una rappresentazione efficiente ma standard del grafo in memoria principale, essendo una sua rappresentazione compressa "beyond the scope ..."}; 3) whatever the data structures used to maintain $1$- and $2$-balls, these should afford efficient, real-time computation of queries as the ones mentioned earlier with minimal memory footprint.
\fi

Heavy updates are natural and well-known candidates for efficient (albeit approximate) implementation using compact, sketch-based data structures \cite{gibbons2001estimating,broder2001completeness,broder2000identifying,agarwal2013mergeable,trevisan/646978.711822}. However, sketches alone are of no avail in handling light updates, whose sheer potential number requires a novel approach.
The literature on efficient data structures that handle insertions and often deletions over dynamic graphs is rich. However, efficient solutions to implement neighborhood-based queries on dynamic edge streams are only known for $1$-balls \cite{BSS20,MROS,VOS,CGPS24}, nor do approaches devised for other dynamic problems adapt to our setting in any obvious way, something we elaborate more upon in Section \ref{subse:related}. 



\subsection{Our Contribution}
In this paper, we propose an approach that trades some degree of accuracy for a substantial improvement in the average number of light updates. In a nutshell, upon an edge insertion, our algorithm performs the (two) corresponding heavy updates, but in general only a subset of the required light updates, according to a scheme that combines a threshold-based mechanism and a randomized, batch-update policy. Hence, for every vertex $u$, we only keep an approximation (a subset to be specific) of $u$'s $2$-ball. If $1$- and $2$-balls are represented with suitable data sketches, our approach affords constant average update cost per edge insertion.\footnote{The particular sketch used depends on the neighborhood queries we want to be able to serve. When sketches are used, the cost of merging two neighborhoods corresponds to the cost of combining the corresponding sketches, which is typically a constant that depends on the desired approximation guarantees. For example, if we are interested in the Jaccard similarity between pairs of $1$- and/or $2$-balls, this cost will be proportional to the (constant) number of minhash values we use to represent each neighborhood.} While the behavior and accuracy guarantees of most sketching techniques are well understood, the estimation error induced by lazy updates can be arbitrarily high in some cases. The main focus of this paper is on the latter aspect, which is absent in the static case but critical in the dynamic setting. Accordingly, we assume lossless representations of $1$-balls and approximate $2$-balls in our theoretical analyses in Sections \ref{sec:detalgo} and \ref{sec:gammaok}, while we use  standard sketching techniques to represent $1$- and $2$-balls in the actual implementations of the algorithms and baselines we consider in the experimental analysis discussed in Section \ref{sec:exp}.



\paragraph{Almost-optimal performance on random sequences.} We prove in Section \ref{subse:rand_perm} that even a simplified, deterministic variant of our lazy-update \Cref{alg:det_thresh} achieves asymptotically optimal expected performance when the  sequence of edge insertions is a random, uniform permutation over an \textit{arbitrary} set of edges. In other words, our lazy approach is robust to adversarial topologies as long as the edge sequence follows a random order.
Formally, we prove that, for any desired $0 < \varepsilon < 1$, our algorithm only performs $O(\frac{1}{\varepsilon})$ (amortized) updates per edge insertion, while at any time $t$ and for every vertex $v$, the estimated size of $v$'s $2$-ball is, in expectation,  at most a factor $\varepsilon$ away from its true value. We further prove that this approximation result holds with a probability that exponentially increases with the true size of the $2$-ball itself (\Cref{thm:random_seq_quality}). 
Thanks to this analysis in concentration,  our results can be extended to other functions of $2$-balls, including union, intersection and Jaccard similarity (see \Cref{cor:jacc} for this less obvious case). 

As positive as this result may sound, it begs the following questions from a careful reader: 1) Are the results above robust to adversarial sequences? 2) Is a performance analysis under random sequences representative of practical scenarios? More generally, does our lazy scheme offer significant practical advantages?

\paragraph{Performance analysis on adversarial inputs.} While our results for random sequences are optimal regardless of the underlying graph's topology, one might wonder about the ability of an adversary to design \textit{worst-case, adaptive sequences} that force our approach to behave poorly and, in this case, whether any conditions on the graph topology are \textit{necessary} for this to happen. We investigate this issues in Section \ref{sec:gammaok}, where we first show that it is possible to design  worst-case sequences of edge insertions that force our algorithm to perform arbitrarily worse than the random setting (\Cref{thm:lower}).  However, as a further contribution, we also prove that worst-case input sequences exist \textit{only if} the \textit{girth}  \cite{diestel2024graph} of the final graph is at most $4$.
More precisely, we show that a randomized, special case of \Cref{alg:det_thresh} achieves asymptotically optimal performance on a class of graphs that contains all graphs with girth at least $5$, \footnote{The  class is in fact more general since it also includes graphs with  a ``bounded'' number of cycles of length at most 4. See \Cref{def:gammaok}, for a formal definition of this class.} even when the input sequence is chosen by an adaptive adversary. 

\smallskip


\iffalse 

\paragraph{Experimental Analysis.}
As for the second question above, an analysis under random permutation sequences as the one in Section \ref{subse:rand_perm} is relatively common in the literature on dynamic edge streams and data streams  \cite{buriol2006counting,kapralov2014approximating,peng2018estimating,Hanauer22DynamicSurvey}. Yet, one might rightly wonder about its practical significance for the task considered in this paper. We investigate this question in Section \ref{sec:exp}, where we conduct experiments on small, medium and large-sized, incremental graphs (whose main properties are summarized in Table \ref{tab:summary_dynamic_dataset}). At least on the diverse sample of real networks we consider, experimental results on the estimation of key queries such as size and Jaccard similarity are consistent with the theoretical findings from Section \ref{subse:rand_perm}. 
Again in agreement with the analysis performed there, results highlight considerable savings in computational cost, compared to baselines that maintain a consistent view of the entire edge sequence. 
We finally remark that the datasets we consider are samples of real social networks. As such, they have relatively large local and global clustering coefficients\footnote{At least the undirected ones.} and thus low girth. Hence, our experimental analysis further supports the robustness of our theoretical findings: forcing our algorithm(s) into a worst-case behavior not only requires topologies characterized by a low girth, but also carefully crafted input sequences that are unlikely to occur in practice.
\fi

\paragraph{Experimental analysis.} 
As for the second question above, an analysis under random permutation sequences as the one in Section \ref{subse:rand_perm} is relatively common in the literature on dynamic edge streams and data streams  \cite{buriol2006counting,kapralov2014approximating,peng2018estimating,Hanauer22DynamicSurvey}. Yet, one might rightly wonder about its practical significance for the task considered in this paper. We investigate this question in Section \ref{sec:exp}, where we conduct experiments on small, medium and large-sized, incremental graphs (whose main properties are summarized in Table \ref{tab:summary_dynamic_dataset}). At least on the diverse sample of real networks we consider, experimental results on the estimation of key queries such as size and Jaccard similarity are consistent with the theoretical findings from Section \ref{subse:rand_perm}. The main take-away is that, when using sketches to represent the $1$- and $2$-balls, the errors obtained with our lazy update policy are similar and fully comparable to those of the baseline, which performs all necessary light updates. At the same time, our algorithm proves to be significantly faster than the baseline, sometimes achieving a speedup of up to $90\times$. 

We finally remark that the datasets we consider are samples of real social networks. As such, they have relatively large local and global clustering coefficients\footnote{At least the undirected ones.} and thus low girth. Hence, our experimental analysis further supports the robustness of our theoretical findings: forcing our algorithm(s) into a worst-case behavior not only requires topologies characterized by a low girth, but also carefully crafted input sequences that are unlikely to occur in practice.




%Another important remark is that the accuracy obtained by our lazy scheme on \emph{real} graphs and \emph{real} edge-insertion sequences is fully consistent with our theoretical findings for \emph{random permutation} sequences. This, combined with the fact that the graphs we considered have relatively large local and global clustering coefficients\footnote{At least the undirected ones.} and thus low girth, might suggest that uniform random permutations are a reasonable theoretical proxy of real sequences and that pathological worst-case graphs and worst-case sequences are pretty rare in practice.  





%\rem{Remind that results apply to directed graphs and multiple insertions but we give %analysis for undirected case for the sake of simplicity.}

\paragraph{Remark.}
We finally stress that although we present them for the undirected case for ease of presentation and for the sake of brevity, our algorithms apply to directed graphs as well,\footnote{Of course, in this case we have directed $h$-balls, i.e., sets of a vertices that can be reached in at most $h$ hops from a given vertex, or from which it is possible to reach the vertex under consideration in at most $h$ hops.} while our analysis extends to the directed case with minor modifications.

\subsection{Further related work}\label{subse:related}
Efficient data structures for queries that involve $h$-balls of a dynamic graph turn out to be useful in different network applications. Besides those we  mentioned earlier   \cite{becchetti2008link,zareie2020similarity,Sim-Nodes_Survey_2024}, we cite here the work \cite{cavallo20222}, where the notion of \textit{2-hop Neighbor Class Similarity} (2NCS) is proposed: this is  a new quantitative graph structural property that correlates with \textit{Graph Neural Networks} (GNN) \cite{scarselli2008graph,wu2022graph} performance more strongly and consistently than alternative metrics. 2NCS considers two-hop neighborhoods as a theoretically derived consequence of the two-step label propagation process governing GCN’s training-inference process.

%\item \textbf{Efficient solutions for \textit{all our queries} on  1-Balls on dynamic graphs.} 

As remarked in the introduction, efficient solutions for Jaccard similarity queries on $1$-balls   have been proposed for different dynamic graph models: all of them share the use of suitable data sketches to manage insertion and deletion of elements from sets. In particular,  \cite{BSS20,MROS,VOS} proposes and compares different  approaches that work in  the fully-dynamic streaming model, while an efficient solution, based on a buffered version of the $k$-min-hashing scheme is proposed in \cite{CGPS24}. This works in the fully-dynamic streaming model and allows recovery actions when certain ``worst-case'' edge deletion events occur. A further algorithm is presented in \cite{zhang2022effective}, where \textit{bottom-$k$ sketches} \cite{cohen2007summarizing} are used to perform dynamic graph clustering based on Jaccard similarity among vertices' neighborhoods. We remark that 
none of these previous approaches include ideas or tools that can be adapted to efficiently manage the $2$-ball update-operations we need to implement in this work.



As for other queries that might be "related" or "useful" in our setting, a considerable amount of work on data structures that support edge insertions and deletions exists for several queries,  such as connectivity or reachability, (exact or approximate) distances, minimum spanning tree, (approximate) \textit{betweenness centrality}, and so on. We refer the reader to \cite{HanauerHS22} for a nice survey on experimental and theoretical results on the topic. To the best of our knowledge however, none of these approaches can be obviously adapted to handle the types of queries we consider in this work. For example, a natural idea would be using an incremental data structure to dynamically maintain the first $h$ levels of a BFS tree, such as \cite{EvenS81,RodittyZ11}, that achieve $O(h)$ amortized update time. However, let alone effectiveness in efficiently serving queries as the ones we consider here, the data structure uses $\Omega(n)$ space per BFS. This is prohibitive in our setting, where we would need to instantiate one such data structure for each vertex, with total space $\Omega(n^2)$. Moreover, since in a degree-$\Delta$ graph $\Theta(\Delta)$ BFS trees can change following a single edge insertion, the corresponding amortized time per edge insertion could be as high as $\Theta(\Delta)$, which is basically the same cost of the baseline solution we discuss at the beginning of Section \ref{sec:detalgo}.



% !TEX root =  ../main.tex
\section{Background on causality and abstraction}\label{sec:preliminaries}

This section provides the notation and key concepts related to causal modeling and abstraction theory.

\spara{Notation.} The set of integers from $1$ to $n$ is $[n]$.
The vectors of zeros and ones of size $n$ are $\zeros_n$ and $\ones_n$.
The identity matrix of size $n \times n$ is $\identity_n$. The Frobenius norm is $\frob{\mathbf{A}}$.
The set of positive definite matrices over $\reall^{n\times n}$ is $\pd^n$. The Hadamard product is $\odot$.
Function composition is $\circ$.
The domain of a function is $\dom{\cdot}$ and its kernel $\ker$.
Let $\mathcal{M}(\mathcal{X}^n)$ be the set of Borel measures over $\mathcal{X}^n \subseteq \reall^n$. Given a measure $\mu^n \in \mathcal{M}(\mathcal{X}^n)$ and a measurable map $\varphi^{\V}$, $\mathcal{X}^n \ni \mathbf{x} \overset{\varphi^{\V}}{\longmapsto} \V^\top \mathbf{x} \in \mathcal{X}^m$, we denote by $\varphi^{\V}_{\#}(\mu^n) \coloneqq \mu^n(\varphi^{\V^{-1}}(\mathbf{x}))$ the pushforward measure $\mu^m \in \mathcal{M}(\mathcal{X}^m)$. 


We now present the standard definition of SCM.

\begin{definition}[SCM, \citealp{pearl2009causality}]\label{def:SCM}
A (Markovian) structural causal model (SCM) $\scm^n$ is a tuple $\langle \myendogenous, \myexogenous, \myfunctional, \zeta^\myexogenous \rangle$, where \emph{(i)} $\myendogenous = \{X_1, \ldots, X_n\}$ is a set of $n$ endogenous random variables; \emph{(ii)} $\myexogenous =\{Z_1,\ldots,Z_n\}$ is a set of $n$ exogenous variables; \emph{(iii)} $\myfunctional$ is a set of $n$ functional assignments such that $X_i=f_i(\parents_i, Z_i)$, $\forall \; i \in [n]$, with $ \parents_i \subseteq \myendogenous \setminus \{ X_i\}$; \emph{(iv)} $\zeta^\myexogenous$ is a product probability measure over independent exogenous variables $\zeta^\myexogenous=\prod_{i \in [n]} \zeta^i$, where $\zeta^i=P(Z_i)$. 
\end{definition}
A Markovian SCM induces a directed acyclic graph (DAG) $\mathcal{G}_{\scm^n}$ where the nodes represent the variables $\myendogenous$ and the edges are determined by the structural functions $\myfunctional$; $ \parents_i$ constitutes then the parent set for $X_i$. Furthermore, we can recursively rewrite the set of structural function $\myfunctional$ as a set of mixing functions $\mymixing$ dependent only on the exogenous variables (cf. \cref{app:CA}). A key feature for studying causality is the possibility of defining interventions on the model:
\begin{definition}[Hard intervention, \citealp{pearl2009causality}]\label{def:intervention}
Given SCM $\scm^n = \langle \myendogenous, \myexogenous, \myfunctional, \zeta^\myexogenous \rangle$, a (hard) intervention $\iota = \operatorname{do}(\myendogenous^{\iota} = \mathbf{x}^{\iota})$, $\myendogenous^{\iota}\subseteq \myendogenous$,
is an operator that generates a new post-intervention SCM $\scm^n_\iota = \langle \myendogenous, \myexogenous, \myfunctional_\iota, \zeta^\myexogenous \rangle$ by replacing each function $f_i$ for $X_i\in\myendogenous^{\iota}$ with the constant $x_i^\iota\in \mathbf{x}^\iota$. 
Graphically, an intervention mutilates $\mathcal{G}_{\mathsf{M}^n}$ by removing all the incoming edges of the variables in $\myendogenous^{\iota}$.
\end{definition}

Given multiple SCMs describing the same system at different levels of granularity, CA provides the definition of an $\alpha$-abstraction map to relate these SCMs:
\begin{definition}[$\abst$-abstraction, \citealp{rischel2020category}]\label{def:abstraction}
Given low-level $\mathsf{M}^\ell$ and high-level $\mathsf{M}^h$ SCMs, an $\abst$-abstraction is a triple $\abst = \langle \Rset, \amap, \alphamap{} \rangle$, where \emph{(i)} $\Rset \subseteq \datalow$ is a subset of relevant variables in $\mathsf{M}^\ell$; \emph{(ii)} $\amap: \Rset \rightarrow \datahigh$ is a surjective function between the relevant variables of $\mathsf{M}^\ell$ and the endogenous variables of $\mathsf{M}^h$; \emph{(iii)} $\alphamap{}: \dom{\Rset} \rightarrow \dom{\datahigh}$ is a modular function $\alphamap{} = \bigotimes_{i\in[n]} \alphamap{X^h_i}$ made up by surjective functions $\alphamap{X^h_i}: \dom{\amap^{-1}(X^h_i)} \rightarrow \dom{X^h_i}$ from the outcome of low-level variables $\amap^{-1}(X^h_i) \in \datalow$ onto outcomes of the high-level variables $X^h_i \in \datahigh$.
\end{definition}
Notice that an $\abst$-abstraction simultaneously maps variables via the function $\amap$ and values through the function $\alphamap{}$. The definition itself does not place any constraint on these functions, although a common requirement in the literature is for the abstraction to satisfy \emph{interventional consistency} \cite{rubenstein2017causal,rischel2020category,beckers2019abstracting}. An important class of such well-behaved abstractions is \emph{constructive linear abstraction}, for which the following properties hold. By constructivity, \emph{(i)} $\abst$ is interventionally consistent; \emph{(ii)} all low-level variables are relevant $\Rset=\datalow$; \emph{(iii)} in addition to the map $\alphamap{}$ between endogenous variables, there exists a map ${\alphamap{}}_U$ between exogenous variables satisfying interventional consistency \cite{beckers2019abstracting,schooltink2024aligning}. By linearity, $\alphamap{} = \V^\top \in \reall^{h \times \ell}$ \cite{massidda2024learningcausalabstractionslinear}. \cref{app:CA} provides formal definitions for interventional consistency, linear and constructive abstraction.
\section{Lazy-Update Algorithms} \label{sec:detalgo}


After giving some preliminaries we will use through all this paper, in \Cref{ssec:algos} we describe the lazy-update algorithmic scheme, while in \Cref{ssec:detalgo-time-wc}, we provide a general bound on its amortized update cost that holds for  arbitrary sequences of edge insertions.

\subsubsection*{Preliminaries and notations}
The dynamic (incremental) graph model we study can be defined as a sequence
    $\dynG = \{ G^{(0)}(V,E^{(0)}), \ldots, $ $ G^{(t)}(V,E^{(t)}),  \ldots G^{(T)}(V,E^{(T)}) \}$,   where: (i) the set of vertices $V = \{1, \ldots , n \}$ is fixed, (ii) $T \leq \binom{n}{2}$ is the final graph,  while (iii)  $E^{(t)}$ is the subset of edges at time $t$. Note that this  changes in every time step $t \geq 1$, as a new edge $e^{(t)}$ is inserted, so that $E^{(t+1)} = E^{(t)} \cup \{e^{(t)}\} $. 
We remark that  our analysis and all our results can be easily adapted to a more general model that includes any combination of the following variants: (i)  growing vertex sets, (ii)  multiple insertions of the same edge, and (iii) directed edges (thus yielding directed graphs).  However,  the corresponding  adaptations of our analysis  would require  significantly heavier notation and some technicalities that  we decided to avoid for the sake of clarity and space.

Our goal is to design algorithms that, at every time step $t \geq 1$, are able to efficiently compute queries over the current $2$-balls of $G^{(t)}$. As mentioned in the introduction, our focus is on queries that are typical in graph mining such as: (i) given a vertex $u$, estimate the size of $\ball_2(u)$, and (ii) given two vertices $u,v \in V$, estimate the Jaccard similarity of the corresponding $2$-balls:
\[ 
    \jacc(\ball_2(u),\ball_2(v)) \ = \ \frac{|\ball_2(u) \cap \ball_2(u) |}{|\ball_2(u) \cup \ball_2(u) |}  \, . 
\]
Both the  theoretical  and experimental analysis of  our   lazy-update algorithms  consider the following key performance  measures: the \textit{amortized update time} per edge insertion and the \textit{approximation ratio} of our algorithms on the quantities $|\ball_2(u)|$ and $\jacc(\ball_2(u),\ball_2(v))$, for any choice of the input vertices. Intuitively, the amortized update time is the average time it takes to process a new edge, a more formal definition is deferred to \Cref{sec:detalgo}, after  a detailed description of the   algorithms  we consider.
    
We next summarize notation that is extensively used in the remainder of the paper. For a vertex $v \in V$ of a graph $G(V,E)$, we define:
    
    
    \begin{description} 
     % \item[$ \neigh^{(t)}(u)$:]    neighborhood  of $u$ as    = \{ v \in V^{(t)} \, : \,   (u,v) \in E^{(t)} \} $
  % \item[$\bd_v$:] the heavy degree;
    % \item[$\rd_v$:] the light degree;
    \item[$\neigh(v)$:] the set of neighborhoods of the vertex $v$.
    \item[$\deg_v$:] the degree of $v$. Notice that $\deg_v = \vert \neigh(v) \vert$;
    \item[$\lset_h(v)$:] set of vertices at distance exactly $h$ from $v$;
    \item[$\ball_h(v)$:] set of vertices at distance at most $h$ from $v$. 
\end{description}

The reader may have noticed that, in our notation above, the term $t$ does not appear: this is due to the fact that our analysis holds at any (arbitrarily-fixed) time step, which  is always clear from context. 



 

\subsection{Algorithm description} \label{ssec:algos}

%\subsection{A threshold-based deterministic algorithm}\label{subse:threshold}
Consider the addition of a new edge $(u,v)$ to $G$. Clearly, the only $2$-balls that are affected are those centered at $u$, $v$, and at every vertex $w \in \neigh(u) \cup \neigh(v)$. A \emph{baseline} strategy, given as \Cref{algo:naive} for the sake of reference, tracks changes exactly and thus updates all $2$-balls that are affected by an edge insertion. 


\begin{algorithm}[h!]
\SetAlgoLined
\DontPrintSemicolon
%\KwData{Undirected Graph $G = (V, E)$}
\SetKwFunction{FMain}{Insert}
\SetKwProg{Fn}{Function}{:}{end}
\Fn{\FMain{$u, v$}}{
    \ForEach{$x \in \neigh(u) \cup \{ u \}$}{
        $\ball_2(v) \gets \ball_2(v) \cup \{x\}$\;
        $\ball_2(x) \gets \ball_2(x) \cup \{v\}$\;
    }
    \ForEach{$x \in \neigh(v) \cup \{v\}$}{
        $\ball_2(u) \gets \ball_2(u) \cup \{x\}$\;
        $\ball_2(x) \gets \ball_2(x) \cup \{u\}$\;
    }
}
\caption{Baseline algorithm.}
\label{algo:naive}
\end{algorithm}
The magnitude of the changes (and the associated computational costs) induced by \texttt{Insert}$(u, v)$ vary. In particular, $\ball_2(u)$ can change significantly, as all vertices in $\ball_1(v)$ will be included in $\ball_2(u)$ (we refer to this as a \emph{heavy} update). Instead, for any vertex $w \in \neigh(u)\setminus\{v\}$, $\ball_2(w)$ will grow by at most one element, namely $v$ (this is referred to as a \emph{light} update). 
Symmetrically, the same holds for $v$ and for every $w \in \neigh(v) \setminus \{u\}$. 

A key observation at this point is that, while heavy updates can be addressed using (possibly, approximate) data structures that allow efficient merging of $1$- and $2$-balls, this line of attack fails with light updates, whose cost derives from their potential number, which can be large in many real cases, as we noted in the introduction.

A first idea to reduce the average number of updates per edge insertion is to perform heavy updates immediately, instead processing light updates in batches that are performed occasionally. More precisely, when a new edge $(u,v)$ arrives, it is initially marked as a \emph{red edge}. Whenever the number of red edges incident to a vertex $u$ exceeds a certain threshold, all the corresponding light updates are processed, and the state of red edges is updated to \emph{black}. See \Cref{fig:basic-example} for an example. 

\begin{figure}[h]
    \centering
    \includegraphics[width=0.45\linewidth]{img/basic-example.pdf}
    \caption{Example of insertion of a new edge $(u,v)$. The algorithm merges the $1$-ball of $v$ with the $2$-ball of $u$ (heavy update), while it does not immediately add vertex $v$ to the $2$-ball of vertex $w$ or any other of $u$'s neighbors. 
    %This is not the only vertex $w$ is not aware of (for instance $w'$ is another vertex %belonging to the 2-hop ball of $w$ that $w$ is not aware of).
    }
    \label{fig:basic-example}
\end{figure}

The idea behind the threshold-based approach is to maintain a balance between the number of black and red edges for every vertex. While useful when edge insertions appear in a random order, this approach may fail when red edges considerably expand the original size of the $2$-ball of some vertex $u$. In order to mitigate this problem, our algorithm  uses a second ingredient: upon each edge insertion $(u,v)$, the algorithm selects $k$ vertices from $\neigh(u)$ and $k$ from $\neigh(v)$ uniformly at random and performs a batch of light updates for the selected vertices, even if the threshold has not been reached yet.

\iffalse
Notice that, if the number of red edges did not reach the threshold yet, there might be some vertex $w \in \neigh(u) \setminus \{v\}$ that is not aware of all the vertices contained in its own $2$-ball. In order to mitigate this, our algorithm will use another ingredient. At each edge insertion $(u,v)$, the algorithm selects uniformly at random $k$ vertices from $\neigh(u)$ and $k$ from $\neigh(v)$, and performs the batch of light updates for the selected vertices, even if the threshold has not yet been reached.
\fi

These ideas are formalized in \Cref{alg:det_thresh}. For each vertex $v$, our algorithm maintains two sets $\apxball_1(v)$ and $\apxball_2(v)$, as well as the \emph{black degree} $\bd_v$ and \emph{red degree} $\rd_v$ of $v$.
Our algorithm guarantees that $\apxball_1(v)$ is exactly $\ball_1(v)$, while $\apxball_2(v)$ is in general a subset of $\ball_2(v)$. The algorithm uses two global parameters, namely a \textit{threshold} $\varphi \in [0,1]$, and an integer $k$. The role of the parameter $\varphi$ can be understood as follows: when $\varphi$ is set to $0$, the algorithm performs all heavy and light updates for every edge insertion, ensuring that $\apxball_2(v)$ always matches $\ball_2(v)$. As $\varphi$ increases, the update function becomes lazier: light updates are not always executed, and $\apxball_2(v)$ is typically a proper subset of $\ball_2(v)$. For instance, when $\varphi = 1$, light updates are performed in batches every time the degree of a vertex doubles. Parameter $k$ specifies the number of neighbors of $v$ that are randomly selected for update of their $2$-balls whenever an edge insertion involving $v$ occurs. This mechanism corresponds to \Cref{line:random_selection,line:random_selection_for,line:random_selection_for_inside} of \Cref{alg:det_thresh}. 

We call \lazyscheme$(\varphi,k)$ the algorithm that runs \Cref{alg:init} on an initial graph $G^{(0)}$ and then processes a sequence $S$ of edge insertions by running \Cref{alg:det_thresh} on each edge of $S$.  

\begin{algorithm}[h]
\SetAlgoLined
\DontPrintSemicolon
\KwData{An undirected graph $G=(V,E)$, a threshold parameter $0 \leq \varphi \leq 1$, and an integer $k \geq 0$.}
set $\varphi$ and $k$ as global parameters\;
\ForEach{vertex $u \in V$}{
    $\delta_u \gets 0$\;
    $\Delta_u \gets \deg_u$\;
    $\apxball_1(u) \gets \ball_1(u)$\;
    $\apxball_2(u) \gets \ball_2(u)$\;
}
\caption{\texttt{Init} operation}\label{alg:init}
\end{algorithm}

%Due to the nature of this algorithm, which triggers batch updates once a certain threshold %is surpassed, we have named it \emph{Threshold-Batching Update}.

\iffalse
When a new edge is added to the graph, the algorithm updates the above information as detailed in the pseudo-code given in \Cref{alg:det_thresh}. The role of the parameter $\varphi$ can be understood as follows: when $\varphi$ is set to $0$, the algorithm performs all the heavy and light updates for every edge insertion, ensuring that $\apxball_2(v)$ always matches the current ball $\ball_2(v)$. As $\varphi$ increases, the update function becomes lazier: light updates are not always executed, and $\apxball_2(v)$ may become a strict subset of $\ball_2(v)$. For instance, when $\varphi = 1$, light updates are performed in batches every time the degree of a vertex doubles. \Cref{line:random_selection,line:random_selection_for,line:random_selection_for_inside} specify the random selection explained above.

Due to the nature of this algorithm, which triggers batch updates once a certain threshold is surpassed, we have named it \emph{Threshold-Batching Update}.
\fi


\begin{algorithm}[h]
\SetAlgoLined
\DontPrintSemicolon
%\KwData{An undirected graph $G=(V,E)$, a threshold factor $\varphi$.}
\SetKwFunction{FMain}{Insert}
\SetKwProg{Fn}{Function}{:}{end}
\Fn{\FMain{$(u, v)$}}{
    \For{$x \in \{u,v\}$}{
        let $y \in \{u,v\} \setminus \{x\}$\;
        $\apxball_1(x) \gets \apxball_1(x) \cup \{y\}$\; \label{line:simple_union}
        \tcp{heavy update}
        $\apxball_2(x) \gets \apxball_2(x) \cup \apxball_1(y)$\; \label{line:heavy_update}
        $\delta_x \gets \delta_x + 1$\;
    
        \uIf{$\delta_x \geq \varphi \cdot \Delta_x$}{ \label{line:threshold_check}
            $\Delta_x \gets \Delta_x + \delta_x$\;
            $\delta_x \gets 0$\;
            \ForEach{$z \in \neigh(x)$}{ \label{line:propagate}
                \tcp{batch of light updates}
                $\apxball_2(z) \gets \apxball_2(z) \cup \apxball_1(x)$\; \label{line:light_updates_for}
            }
      }
      \Else {
        select $k$ vertices $w_1, \dots, w_k \in \neigh(x)$ u.a.r.\;\label{line:random_selection}
        \For{$i = 1, \dots, k$}{ \label{line:random_selection_for}
                \tcp{batch of light updates}
                $\apxball_2(w_i) \gets \apxball_2(w_i) \cup \apxball_1(x)$\; \label{line:random_selection_for_inside}
            } 
      }
    }
}

\caption{ \textsc{Insert} }\label{alg:det_thresh}
\end{algorithm}
\paragraph{A note on neighborhood representation.}
As we mentioned in the introduction, we treat  $\apxball_1(v)$ and $\apxball_2(v)$ as sets of vertices in this section and in Section \ref{sec:gammaok}. We remark that this only serves the purpose of analyzing the error introduced by our lazy update policies: lossless representations of $1$- and $2$-balls may be unfeasible for medium or large graphs and compact data sketches are typically used to represent them in such cases. The choice of the actual sketch strongly depends on the type of query (or queries) one wants to support, such as $1$- or $2$-ball sizes \cite{flajolet1985probabilistic,boldi2011hyperanf} or Jaccard similarity between $2$-balls \cite{broder2000identifying,cohen2007summarizing,becchetti2008efficient}. 
All sketches used for typical neighborhood queries are well-understood and come with strong accuracy guarantees. Moreover, they allow to perform the union of $1$- and $2$-balls we are interested in time proportional to the sketch size, which is independent of the sizes of the balls to merge \cite{agarwal2013mergeable}. 

\subsection{Cost analysis for arbitrary sequences} \label{ssec:detalgo-time-wc}
Consistently to what we remarked above, our cost analysis focuses on the number of 
\emph{set-union} operations: This performance measure in fact dominates
the computational cost of \Cref{alg:det_thresh}.  More in detail, given any sequence $S$ of edge insertions, starting from an initial graph $G^{(0)}$,  we denote by $\cost(S)$ the overall number of union operations performed in \Cref{line:simple_union,line:heavy_update,line:light_updates_for,line:random_selection_for_inside} of \Cref{alg:det_thresh} on the input sequence $S$.


We observe that a trivial upper bound to $\cost(S)$ is $O(\Delta |S|)$, since each insertion can cost $O(\Delta)$ union operations where $\Delta$ is the maximum degree of the current graph. However, this trivial argument turns out to be too pessimistic: in what follows,  we  provide a  more refined analysis of the amortized cost\footnote{The \emph{amortized analysis} is a well-known method originally introduced in \cite{Tarjan_amortized} that allows to compute tight bounds on the cost of a \textit{sequence} of operations, rather than the worst-case cost of an individual operation. In more detail,  we average the cost of a \emph{worst case} sequence of operations to obtain a more meaningful cost per operation.} per edge insertion. We say that an algorithm has \emph{amortized cost} $\hat{c}$ per edge insertion if, for any sequence $S$ of edge insertions, we have $\cost(S) \le \hat{c} |S|$.

\begin{lemma}
    \label{lm:amortized_det_alg}
    Given any initial graph $G^{(0)}$ and any sequence $S$ of edge insertions, the amortized update cost of \Cref{alg:det_thresh} is $O(\frac{1}{\varphi}+k)$ per edge insertion.
\end{lemma}
\begin{proof}
    Let us first consider the case $k=0$, i.e., when the random selection and the consequent instructions in \Cref{line:random_selection,line:random_selection_for,line:random_selection_for_inside} are never performed.  Our amortized analysis makes use of the \textit{accounting method} \cite{Tarjan_amortized}. The idea is  paying  the cost of any batch of light updates by charging it to previous edge insertions. More precisely, we assign \emph{credits} to each edge insertion that we will use to pay the cost of subsequent batches of light updates. Formally, the \emph{amortized} cost of an edge insertion is defined as the \emph{actual} cost of the operation, plus the credits we assign to it, minus the credits (accumulated from previous operations) we spend for it. We need to carefully define such credits in order to guarantee that the sum of the amortized costs is an upper bound to the sum of the actual costs, i.e. we always have enough credits to pay for costly batch light updates.
    
    We proceed as follows. When we insert the edge $(u,v)$, we put $2/\varphi$ credits on $u$ and $2/\varphi$ credits on $v$. Now we bound the actual and amortized cost of each insertion. 
    
    First, consider an edge insertion $(u,v)$ that does not trigger a batch of light updates. Its actual cost is 4 union operations (those in \Cref{line:simple_union,line:heavy_update}, 2 for each endpoint of $(u,v)$). Then its amortized cost is upper-bounded by $4+4/\varphi=O(1/\varphi)$. Now consider the case in which the insertion causes a batch of light updates for $u$, or $v$, or both. We show that the credits accumulated by previous insertions are sufficient to pay for its cost. To see this, consider a batch of light updates involving vertex $x \in \{u,v\}$. And let $\bd_x$ and $\rd_x$ be the current black and red degrees of $x$ at that time (just before \Cref{line:threshold_check} is evaluated). It is clear that for vertex $x$ we have accumulated  $\rd_x \cdot 2/\varphi$ credits that now we use to pay for the cost of \Cref{line:propagate,line:light_updates_for}. This cost equals to $\deg_x$ union operations, where $\deg_x$ is the current degree of vertex $x$. Since the batch of light updates has just been triggered, we have that $\rd_x \ge \varphi \bd_x$, and hence we have at least $\rd_x \cdot 2/\varphi \ge \varphi \bd_x \cdot 2/\varphi=2 \bd_x$ credits to pay for the $\deg_x=\bd_x+\rd_x=\bd_x+\varphi\bd_x \le 2 \bd_x$ union operations. This concludes the proof. 

    Finally, to obtain the claim when $k>0$, we notice that, in this case, every edge insertion causes $O(k)$ additional union operations.
\end{proof}

%\subsection{Approximation analysis over random  sequences of arbitrary graphs}
\section{Random edge sequences}\label{subse:rand_perm}
In this section, we analyze the accuracy of our lazy-update algorithm(s) over an arbitrary dynamic graph, whose edges are given in input as a uniformly sampled, random permutation over its edge set.
Dynamic graphs resulting from random sequences of edge insertions have been an effective tool to provide theoretical insights that have often proved robust to empirical validation in various dynamic scenarios \cite{monemizadeh2017testable,kapralov2014approximating,peng2018estimating,mcgregor2014graph,chakrabarti2008robust}.
In more detail, assume $G = (V, E)$, with $|E| = t$, is the graph observed up to some time $t$ of interest. Following \cite{monemizadeh2017testable,peng2018estimating}, we assume that the sequence of edges up to time $t$ is chosen uniformly at random from the set of all permutations over $E$.\footnote{It should be noted that this includes the general case in which $t$ is any intermediate point of a longer stream that possibly extends well beyond $t$. In this case, it is well-known and easy to see that, conditioned on the subset $E$ of the edges released up to time $t$, their sequence is just a permutation of $E$.} The following fact is an immediate consequence of well-known and intuitive properties of random permutations. We state it informally for the sake of completeness, avoiding any further, unnecessary notation.
\begin{fact}\label{fa:perm}
    Consider a dynamic graph $G = (V, E)$, whose edges are observed sequentially according to a permutation over $E$ chosen uniformly at random. Then, for every $E'\subseteq E$, the sequence in which edges in $E'$ are observed is itself a uniformly chosen, random permutation over $E'$.
\end{fact}



In the remainder, for an arbitrary vertex $v$, we analyze how well the output $\apxball_2(v)$ of \Cref{alg:det_thresh} approximates $\ball_2(v)$ at any round $t$ in terms of its \emph{coverage}:

\begin{definition}\label{def:coverage}
    We say that the output  $\apxball_2(v)$  of \lazyscheme$(\varphi,k)$ is a $(1-\varepsilon)$-\textit{covering} of $\ball_2(v)$ if the following holds: i) $\apxball_2(v) \subseteq \ball_2(v)$; ii) $\Expec{}{\vert \apxball_2(v) \vert} \geq (1-\varepsilon) \vert \ball_2(v) \vert$, where expectation is taken over the randomness of the algorithm and/or the input sequence. When the algorithm produces a $(1-\varepsilon)$-covering $\apxball_2(v)$ of $\ball_2(v)$ for every $v$, we say it has \emph{approximation ratio} $\frac{1}{(1-\varepsilon)}$.
\end{definition}
Our main result in this section is formalized in the following 

\begin{theorem}\label{thm:random_seq_quality}
    Let  $\varepsilon \in (0,1)$,  and fix  parameters $k = 0$ and $\varphi = \frac{\varepsilon}{1-\varepsilon}$. Consider any  graph $G(V,E)$ submitted as  a  uniform  random permutation of its edge set $E$ to \lazyscheme$(\varphi,k)$. Then, at every time step $t\le |E|$, the algorithm has approximation ratio $\frac{1}{1-\varepsilon}$. Moreover, for every $\alpha > 0$ and every vertex $v  \in V$, we have:
    \[
        \Prob{}{|\apxball_2(v)| < \frac{1 - \alpha}{1 + \varphi}|L_2(v)|}\le e^{-\frac{2\alpha^2|L_2(v)|}{(1 + \varphi)^2}}.
    \]
\end{theorem}
\begin{proof}
Fix a vertex $v \in V$ and a round $t \geq 1$. In the remainder of this proof, all quantities are taken at time $t$. We are interested in how close $|\apxball_2(v)|$ is to $|\ball_2(v)|$. To begin, we note that the following relationship holds deterministically:
\begin{equation}\label{eq:apxball_det}
    |\apxball_2(v)| = 1 + |L_1(v)| + |\apxball_2(v)\cap L_2(v)|,
\end{equation}
where the only random variable on the right hand side is the last term. 
\begin{figure}[h!]
    \centering
    \includegraphics[width=0.7\linewidth]{img/example_partition.pdf}
    \caption{Example of a partition of $L_2(v)$ into three sets $C_1, C_2, C_3$. Edges connecting vertices $w \in L_2(v)$ to their respective partitions are thicker.}
    \label{fig:L2_partition}
\end{figure}
We next define a partition $\mathcal{C} = \{C_{u}: u\in L_1(v)\}$ of $L_2(v)$ as follows: for each $w \in L_2(v)$, we choose a vertex $u \in L_1(v) \cap \neigh(w)$ and assign $w$ to $C_u$. This way, each vertex $w\in L_2(v)$ is associated to exactly \emph{one} edge connecting one vertex in $L_1(v)$ to $w$ (see \Cref{fig:L2_partition}, where the edges in question are thick in the picture). Let $E_v$ denote the set of such edges and note that i) $E_v$ is a subset of the edges connecting vertices in $L_1(v)$ to those in $L_2(v)$, ii) $|E_v| = |L_2(v)|$ by definition and iii) $|C_u|\le \deg_u - 1$ for every $u\in L_1(v)$, given that $C_u$ contains a subset of $u$'s neighbors and $(v, u)$ is always present. Moreover, \Cref{alg:det_thresh} guarantees that $|\apxball_2(v)\cap L_2(v)|$ is at least the number of edges in $E_v$ that are black. 
These considerations allow us to conclude that
\[
    |\apxball_2(v)\cap L_2(v)| \ge |\{e\in E_v:\text{ $e$ is black}\}|.
\]
A key observation at this point is that \Cref{alg:det_thresh} implies that for every $x\in V$, $\rd_x\le\left\lfloor\frac{\varphi}{1 + \varphi}\deg_x\right\rfloor$. As a consequence, if some $e = (u, w)\in E_v$ was not among the last $\left\lfloor\frac{\varphi}{1 + \varphi}\deg_u\right\rfloor$ edges incident in $u$ that were released within time $t$, it is necessarily black. For $e = (u, w)\in E_v$, with $u\in L_1(v)$ and $w\in L_2(v)$, let $X_e = 1$ if $e$ was among the first $\deg_u - \left\lfloor\frac{\varphi}{1 + \varphi}\deg_u\right\rfloor$ edges incident in $u$ that were released up to time $t$ and let $X_e = 0$ otherwise. Following the argument above, the event $( X_e = 1 )$ implies the event $\text{"$e$ is black"}$, whence:
\begin{equation}\label{eq:apx_balls}
    |\apxball_2(v)\cap L_2(v)| \ge |\{e\in E_v:\text{ $e$ is black}\}|\ge \sum_{e\in E_v}X_e.
\end{equation}
Next, we are interested in bounds on $\Prob{}{X_e = 1}$. Assume $e$ is incident in $u$ and let $S$ be the set of edges incident in $u$ observed up to time $t$. 
Then, from \Cref{fa:perm}, the sequence in which these edges are observed is just a random permutation of $S$. 
This immediately implies that, if $e$ is incident to vertex $u\in L_1(v)$, then  
\[
    \Prob{}{X_e = 1} = \frac{\deg_u - \left\lfloor\frac{\varphi}{1 + \varphi}\deg_u\right\rfloor}{\deg_u}\ge\frac{1}{1 + \varphi}.
\]
Together with \eqref{eq:apx_balls} this yields:
\[
    \Expec{}{|\apxball_2(v)|}\ge 1 + |L_1(v)| + \frac{1}{1 + \varphi}|L_2(v)| \geq \frac{1}{1+\varphi}\vert \ball_2(v) \vert.
\]
We next show that $\sum_{e\in E_v}X_e$ is concentrated around its expectation when $|L_2(v)|$ is large enough, which implies that $|\apxball_2(v)|$ is concentrated around a value close to $|\ball_2(v)|$ in this case. The main technical hurdle here is that the $X_e$'s are correlated (albeit mildly, as we shall see). To prove concentration, we resort to Martingale properties of random edge sequences to apply the method of (Average) Bounded Differences \cite{dubhashi2009concentration}. In order to do this, we need bounds on $\Prob{}{X_e = 1\vert X_f = 1}$ and $\Prob{}{X_e = 1\vert X_f = 0}$, for $e, f\in E_v$. Assume again that $e$ is incident in $u\in L_1(v)$ and that $S$ is the set of edges incident in $u$ observed up to time $t$. Assume first that $f$ is also incident in $u$ and that, without loss of generality, $f$ is the $i$-th edge to appear among those in $S$. $X_f = 1$ implies $i\le\deg_u - \left\lfloor\frac{\varphi}{1 + \varphi}\deg_u\right\rfloor$. On the other hand, for any such choice for $f$'s position in the sequence, Fact \ref{fa:perm} implies that $e$ will appear in a position $j$ that is sampled uniformly at random from the remaining ones, so that $\Prob{}{X_e = 1\vert X_f = 1} = \frac{\deg_u - \left\lfloor\frac{\varphi}{1 + \varphi}\deg_u\right\rfloor - 1}{\deg_u - 1}$ in this case. With a similar argument, it can be seen that $\Prob{}{X_e = 1\vert X_f = 0} = \frac{\deg_u - \left\lfloor\frac{\varphi}{1 + \varphi}\deg_u\right\rfloor}{\deg_u - 1}$. Intuitively and unsurprisingly, the events $(X_e = 1)$ and $(X_f = 1)$ are negatively correlated, while $(X_e = 1)$ and $(X_f = 0)$ are positively correlated. This allows us to conclude that $\Prob{}{X_e = 1\vert X_f = 1}\le \Prob{}{X_e = 1\vert X_f = 0}$ and 

\[
    \Prob{}{X_e = 1\vert X_f = 0} - \Prob{}{X_e = 1\vert X_f = 1}\le\frac{1}{\deg_u - 1}.
\]
Assume next that $f$ is not incident in $u$. Again from Fact \ref{fa:perm}, in this case $f$ has no bearing on the relative order in which edges incident in $u$ appear, so that $\Prob{}{X_e = 1\vert X_f = 0} = \Prob{}{X_e = 1\vert X_f = 1} = \Prob{}{X_e = 1}$. Now, without loss of generality, suppose that $f = (z, w)$, with $z\in L_1(v)$, so that $w\in C_{z}$. Denote by $E_v(z)$ the subset of edges in $E_v$ with one end point in $C_{z}$. Moving to conditional expectations we have
\begin{align*}
    &\Expec{}{\sum_{e\in E_v}X_e \, \vert\,  X_f = 0} - \Expec{}{\sum_{e\in E_v}X_e \, \vert\, X_f = 1} \\
    &= \sum_{e\in E_v}\left(\Prob{}{X_e = 1 \, \vert\, X_f = 0} - \Prob{}{X_e = 1 \, \vert\, X_f = 1}\right)\\
    &= \sum_{e\in E_v \setminus E_v(z)}\left(\Prob{}{X_e = 1 \, \vert\, X_f = 0} - \Prob{}{X_e = 1 \, \vert\, X_f = 1}\right) \\
    &+ \sum_{e\in E_v(z)}\left(\Prob{}{X_e = 1 \, \vert\, X_f = 0} - \Prob{}{X_e = 1 \, \vert \, X_f = 1}\right)\\
    &\le\frac{|C_{z}|}{\deg_z - 1}\le 1,
\end{align*}
where the third inequality follows from the definition of $C_z$, since $f$ is incident in $z$, while the last inequality follows since $|C_z|\le \deg_z - 1$ for every $z\in L_1(v)$, because one of the edges incident in $z$ is by definition the one shared with $v$.

We can therefore apply \cite[Definition 5.5 and Corollary 5.1]{dubhashi2009concentration}, with $c\le |L_2(v)|$ to obtain, for every $\alpha > 0$:
\begin{align*}
    &\Prob{}{\Expec{}{\sum_{e\in E_v}X_e} - \sum_{e\in E_v}X_e > \alpha\Expec{}{\sum_{e\in E_v}X_e}}\le e^{-\frac{2\alpha^2|L_2(v)|}{(1 + \varphi)^2}},
\end{align*}
where in the right hand side we also used the bound $\Expec{}{\sum_{e\in E_v}X_e}\ge\frac{1}{1 + \varphi}|L_2(v)|$ we showed earlier.
Finally, we recall \eqref{eq:apxball_det} and \eqref{eq:apx_balls} to conclude that $|\apxball_2(v)|\ge \frac{1 - \alpha}{1 + \varphi}|L_2(v)|$ with (at least) the same probability.
\end{proof}

\Cref{thm:random_seq_quality} easily implies approximation bounds on indices that depend on the union and/or intersection of $2$-balls. For example, we immediately have the following approximation bound on the Jaccard similarity between any pair of $2$-balls.

\begin{corollary}\label{cor:jacc}
  Under the same assumptions as \Cref{thm:random_seq_quality}, at any time step $t \geq 1$ and for any pair of vertices $u,v \in V$, \lazyscheme$(\varphi,k)$ guarantees the following approximation of the Jaccard similarity between $\ball_2(u)$ and $\ball_2(v)$ with probability at least $1 - e^{-\frac{2\alpha^2|L_2(u)|}{(1 + \varphi)^2}} - e^{-\frac{2\alpha^2|L_2(v)|}{(1 + \varphi)^2}}$: 
    \begin{equation}\label{eq:jacc_apx}
        \textstyle \dfrac{\jacc(\ball_2(u),\ball_2(v))}{1-2\varepsilon'} \geq \jacc(\apxball_2(u), \apxball_2(v)) \geq (1-\varepsilon')\jacc(\ball_2(u),\ball_2(v)) - \varepsilon',
    \end{equation}
    where $\varepsilon' = \frac{\varphi + \alpha}{1 + \varphi}$.
\end{corollary}
\begin{proof}
It is easy to see that $|\apxball_2(u)|\ge (1 - \varepsilon')|\ball_2(u)|$ and $|\apxball_2(v)|\ge (1 - \varepsilon')|\ball_2(v)|$ together imply \eqref{eq:jacc_apx} deterministically. The result then immediately follows from \Cref{thm:random_seq_quality} and a union bound on the events $(|\apxball_2(u)| < (1 - \varepsilon')|L_2(u)|)$ and $(|\apxball_2(v)| < (1 - \varepsilon')|L_2(v)|)$.
\end{proof}


\section{Adversarial edge  sequences}\label{sec:gammaok}
We next study our lazy-update algorithm in an adversarial framework. We  show  in \Cref{ssec:lowerbound} that if the adversary can \textit{both}: i) choose a worst-case  graph $G$ \textit{and} ii)   submit $G$ according to an \textit{adaptive} sequence of edge insertions, then it is possible to prove a strong lower bound on the achievable update-time/approximation trade-off of the whole parameterized scheme $\lazyscheme (\varphi,k)$.

On the other hand, in \Cref{ssec:gammaok} we provide a \textit{necessary} condition for the adversarial, worst-case framework above: the \textit{girth} \cite{diestel2024graph} of $G$ must be smaller than 5. More precisely, $G$ must contain an unbounded number of triangles and cycles of length 4. We do this by showing that for a suitable parameter setting, algorithm $\lazyscheme (\varphi,k)$ achieves almost-optimal trade-offs even on adversarial edge insertion sequences, for every graph that has a bounded number of such small cycles (see \Cref{def:gammaok} for a formal definition of this class of graphs).

\subsection{A lower bound for adversarial sequences} \label{ssec:lowerbound}
The lower bound for the adversarial framework described above is formalized in the following result on the approximation ratio (see Def. \ref{def:coverage})


\begin{theorem}\label{thm:lower}
    For every $\varphi \in [0,1]$, and integer $k \geq 0$, if  \lazyscheme$(\varphi,k)$ has approximation ratio $\rho \ge 1$, then it must have an amortized update cost  of $\Omega(\Delta/\rho^3)$, where $\Delta$ is the maximum degree of the graph.
\end{theorem}\label{le:lb1}

\begin{figure}[ht]
    \centering
    \includegraphics[width=0.66\linewidth]{img/lower-bound.pdf}
    \caption{Black edges are present at $t = 0$, while red ones are inserted in the interval $\{1, 2,\ldots , \Delta \rho^2\}$. At time $t > 0$, an edge with one endpoint in $u_{1t\mod\Delta}$ and the other in a distinct 0-degree vertex in $S_2$ is added.}
    \label{fig:lb1}
\end{figure}

\begin{proof}
Fix $\rho \ge 1$, and assume that \lazyscheme$(\varphi,k)$ has an approximation ratio of at most $\rho$. We will show that there exist an initial graph $G^{(0)}$ with degree $\Delta$ and a sequence of edge insertions against which \lazyscheme$(\varphi,k)$ must incur an amortized update time of $\Omega(\Delta/\rho^3)$.

Note that for $k>0$, the algorithm is randomized. In order to address this, we prove our lower bound for every possible realization of the randomness used by the algorithm. Therefore, we assume the values of the random bits used by \lazyscheme$(\varphi,k)$ are fixed arbitrarily (and optimally) and we assume henceforth that the behavior of the algorithm is completely deterministic. 

The initial graph $G^{(0)}$ consists on $2 \Delta$ vertices forming a complete bipartite graph with sides $S_0=\{u_{01},\dots,u_{0\Delta}\}$ and $S_1=\{u_{11},\dots,u_{1\Delta}\}$, along with an additional set $S_2$ of $\Delta \rho^2$ isolated vertices (see \Cref{fig:lb1}).
The sequence of edge insertions is defined as follows: 
for each vertex in $S_1$, we insert $\rho^2$ new edges. Each of these $\Delta\rho^2$ edges connects a vertex in $S_1$ to a previously isolated vertex in $S_2$.
% We insert $\rho^2$ new edges incident to each vertex in $S_1$. Each of these $\Delta \rho^2$ new edges has an endpoint in $S_1$, while the other is a previously $0$-degree vertex in $S_2$. 

%These edges are inserted in a round-robin fashion: we first insert an edge incident to $u_{11}$, then one incident to $u_{12}$ up to one incident to $u_{1\Delta}$, then again another incident to $u_{11}$ and so on. We can view the sequence of insertions as organized in $\rho^2$ rounds of $\Delta$ steps each.
%In the $i$-th step of the $j$-th round one edge is added from $u_{1i}$ to a different vertex belonging to $S_2$.


Consider the time instant right after all edge insertions. 
Since we assumed that the algorithm guarantees an approximation ratio of $\rho$, it holds that for every vertex $u \in S_0$, $|\apxball_2(u)| \ge \frac{1}{\rho} |\ball_2(u)|=\frac{1}{\rho} (2\Delta+\Delta \rho^2) = \Delta \rho + 2\Delta/\rho$. This implies that after all edge insertions, $u$ must be aware of at least $\Delta \rho + 2\Delta/\rho - 2\Delta=\Delta\rho -2 \Delta(1-1/\rho)$ vertices from $S_2$. 

We say that there is a \emph{message} from $v$ to $u$ if vertex $v$ performs a union operation of the form $\apxball_2(u) \gets \apxball_2(u) \cup \apxball_1(v)$.

Since, at any time, every vertex $v \in S_1$ is adjacent to at most $\rho^2$ vertices in $S_2$, it must be that each $u \in S_0$ must have received at least 
$\frac{\Delta\rho -2 \Delta(1-1/\rho)}{\rho^2}=\Omega(\Delta/\rho)$
messages from vertices in $S_1$. As a consequence, the total number of messages are at least $\Omega(\Delta^2/\rho)$. As the number of insertions is $\Delta \rho^2$, the amortized update cost per insertion is at least $\frac{\Omega(\Delta^2/\rho)}{\Delta \rho^2}=\Omega(\Delta/\rho^3)$.
\end{proof}

We have special cases as corollaries. We need amortized update cost $\Omega(\Delta)$ if we want $\rho = O(1)$, $\Omega(\sqrt[4]{\Delta})$ if we want $\rho = O(\sqrt[4]{\Delta})$ and so on. 

\begin{remark}
    The lower bound in \Cref{le:lb1} in fact holds for a wider class of algorithms. Informally speaking, this class includes any \textit{local} algorithm that limits its online updates to the 2-hop neighbors of $u$ and $v$ only. Making this claim more formal requires addressing several technical issues that are outside the scope of the present work. 
\end{remark}

%\subsection{Adversarial edge sequences of large girth}
\subsection{Locally \texorpdfstring{$\gamma$}{gamma}-sparse graphs} \label{ssec:gammaok}

In this section, we provide the characterization of a class of graphs for which our lazy-update approach always guarantees good amortized cost/approximation trade-offs, even under the assumption of adversarial edge insertion sequences.

% In this section, we analyze the performance of our lazy-update approach over a class of graphs that satisfy a property of ``local-sparsity''.
Given a graph $G(V,E)$ and a subset $V' \subseteq V$, we denote by $G[V']$ the subgraph induced by $V'$. 
Informally, a graph is \gammaok\ if every node in $\ball_2(u) \setminus \{u\}$ has roughly at most $\gamma$ neighbors in $\lset_1(u)$. This notion can be formalized as follows. 

%We will prove that, for constant values of $\gamma$, it is possible to obtain a $(1-\varepsilon)$-covering with amortized update  cost   $O(\frac{1}{\varepsilon})$.

 

\begin{definition}[\gammaok\ graphs] \label{def:gammaok}
 Let  $\gamma \in 
 \{ 0,1\ldots , n-1\}$. A graph $G(V,E)$ is said \gammaok\ if for each vertex $u \in V$:
    (i) $\forall v \in \lset_1(u)$ the degree of $v$ in $G[\lset_1(u)]$ is at most $\gamma$, and (ii) $\forall w \in \lset_2(u)$ the degree of $w$ in $G[\lset_1(v) \cup \{ w \}]$ is at most $\gamma+1$.
\end{definition}

%In the following, we make an abuse of notation and say that a dynamic graph $G$ is \gammaok if any $G^{(t)}$ is at most \gammaok, for any $t$.

Observe that the class of \gammaok\ graphs grows monotonically with $\gamma$, including all possible graphs for $\gamma=n-1$, while the most restricted class is obtained for $\gamma=0$. It is interesting to note that \gammaok\ graphs are not necessarily sparse in absolute terms. For example, for $\gamma=0$, the class coincides with the well-known class of graphs with \emph{girth} at least $5$: these graphs can have up to $\Theta(n^{\frac{3}{2}})$ edges assuming Erd{\"o}s' Girth Conjecture \cite{erdos1965some} (the proof of such equivalence is given in \Cref{apx:gamma-ok-deterministic}).

A first, preliminary analysis of our lazy-update approach considers the deterministic version of \Cref{alg:det_thresh}, i.e., when $k = 0$. It turns out that this version achieves an approximation ratio of  $\frac{\gamma + 1}{1-\varepsilon}$ and amortized cost $O(1/\varepsilon)$ (see \Cref{apx:gamma-ok-deterministic}). So, the approximation accuracy decreases linearly in the parameter $\gamma$. 
Interestingly enough, we instead show that a suitable number of random light updates allows \Cref{alg:det_thresh} to perform much better than its deterministic version. This is the main result of this section and it is formalized in the next 

\begin{theorem}\label{thm:gamma-ok-main}
Let $\varepsilon \in (0,1)$, and let $G^{(0)}$ be an initial graph. Consider any sequence of edge insertions that yields a final graph $G$. If $G$ is \gammaok\, \lazyscheme$\left(\varphi =1,nk=\frac{4(\gamma+1)}{\varepsilon}\right)$ has approximation ratio of $\frac{1}{1-\varepsilon}$ and amortized cost $O\left(\frac{\gamma+1}{\varepsilon}\right)$.     
\end{theorem}

%We in fact prove that, by setting $k = \frac{4(\gamma + 1)}{\varepsilon}$ and $\varphi = 1$, \lazyscheme\ achieves an approximation ratio of $\frac{1}{1-\varepsilon}$ and amortized update cost of $O(\frac{\gamma+1}{\varepsilon})$ for \gammaok graphs. 

\subsubsection*{Proof of \Cref{thm:gamma-ok-main}}
% For the remainder of this proof, we define the notion of \emph{quasi-black} edge. Informally, a red edge $(v,w)$, with $v \in \lset_1(u)$ and $w \in \lset_2(u)$, is \textit{quasi-black} for $u$ if $u$ is selected in Line 14 of \Cref{alg:det_thresh} for the subsequent insertion of an edge $(v,w')$, ensuring that $w \in \apxball_2(u)$. More formally: \rem{forse la def formale si può togliere}

% \begin{definition}
% Let $u \in V$, and $v \in \lset_1(u)$. For $i=1,\dots,\rd_v$, let $e_i$ be the $i$-th red edge w.r.t. $v$ inserted in the graph. We say that $e_i$ is a \emph{quasi-black edge for $u$} if $u$ has been randomly selected at least once during the insertions of $e_i,\dots,e_{\rd_v}$ (\Cref{alg:det_thresh} lines 14-16). 
% \end{definition}

For the remainder of this proof, we define the notion of \emph{quasi-black} edge. A red edge $(v,w)$, with $v \in \lset_1(u)$ and $w \in \lset_2(u)$, is said to be \textit{quasi-black} for $u$ if $u$ has been randomly selected by $v$ at \Cref{line:random_selection} of \Cref{alg:det_thresh} \emph{at least once} during or after the insertion of $(v,w)$, ensuring that $w \in \apxball_2(u)$.

In this section, we use $\lrdr_v$ and $\lrd_v$ to denote the number of \emph{quasi-black} and \emph{red} edges, respectively, that connect $v$ to vertices in $\lset_2(u)$. Similarly, we use $\lbdd_v$ to represent the number of \emph{black} edges of $v$ having the other endpoint in $\lset_2(u)$. Notice that $\lrdr_v$ is a random variable that counts how many vertices out of $\lrd_v$ are in $\apxball_2(u)$. We now proceed by first stating a property whose proof can be found in Appendix~\ref{apx:proof_gamma_ok_expect_lowerbound}.

\begin{lemma}\label{le:gamma_ok_expect_lowerbound}
     For each $v \in \lset_1(u)$, we have $\Expec{}{\lrdr_v} \geq \lrd_v - \frac{2(\lbdd_v + \gamma + 1)}{k}$.
\end{lemma}

\iffalse
\begin{proof}
Let $e_1, \dots, e_{\ell_v}$ be the \emph{red edges} between $v$ and $\lset_2(u)$, and define the binary random variable $\lrdr_v(i)$ that is equal to $1$ if $e_i$ is a \emph{quasi-black edge} for $u$, $0$ otherwise, for $i = 1, \dots, \lrd_v$. Thus we can express $\lrdr_v = \sum_{i=1}^{\lrd_v} \lrdr_v(i)$, with expectation

\begin{equation}\label{eq:gamma_ok_lb_fact_eq_1}
\begin{aligned}
  \Expec{}{\lrdr_v} & = \sum_{i=1}^{\lrd_v}{\Prob{}{\lrdr_v(i)=1}} = \lrd_v - \sum_{i=1}^{\lrd_v} {\Prob{}{\lrdr_v(i)=0}}.
\end{aligned}
\end{equation}

Without loss of generality, assume that the edges $e_1, \dots, e_{\lrd_v}$ have been inserted at times $t_1 < \dots < t_{\lrd_v}$, respectively.
If $e_i$ is not a quasi-black edge for $u$, then it must be that $u$ is not selected by $v$ at \Cref{line:random_selection} of \Cref{alg:det_thresh}, at times $t_i, t_{i+1},\dots, t_{\lrd_v}$.
This holds with probability 
\begin{equation}\label{eq:gamma_ok_lb_fact_eq_2}
\begin{aligned} 
    &\Prob{}{\lrdr_v(i) = 0}
    \leq \prod_{j=i}^{\lrd_v} \left( 1-\frac{k}{\deg_v^{(t_j)}} \right)
    \leq \prod_{j=i}^{\lrd_v} \left( 1 - \frac{k}{\deg_{v}^{(t_{\lrd_v})}} \right) \\
    &\leq \left( 1-\frac{k}{\lbdd_v + \lrd_v + \gamma + 1}\right)^{\lrd_v - i + 1} 
    \leq \left(1-\frac{k}{2(\lbdd_v + \gamma + 1)}\right)^{\lrd_v - i}.
\end{aligned}
\end{equation}
The third inequality holds since the edges incident to $v$ having endpoints in $L_1(u)$ are at most $\gamma$, while those having endpoints in $L_2(u)$ are exactly $\lbdd_v+ \lrd_v$. Moreover, the last inequality holds because $\lrd_v \leq \rd_v \leq \bd_v \leq \lbdd_v + \gamma + 1$, given the assumption $\varphi = 1$.

By plugging in \eqref{eq:gamma_ok_lb_fact_eq_2} into \eqref{eq:gamma_ok_lb_fact_eq_1} and we obtain
\begin{align*}
    &\Expec{}{\lrdr_v} \geq \lrd_v - \sum_{i=1}^{\lrd_v}\left( 1-\frac{k}{2(\lbdd_v + \gamma + 1)}\right)^{\lrd_v - i} \\
    &= \lrd_v - \sum_{i=0}^{\lrd_v-1} \left(1-\frac{k}{2(\lbdd_v + \gamma + 1)}\right)^i 
    \leq \lrd_v - \frac{1-\left(1-\frac{k}{2(\lbdd_v+\gamma+1)}\right)^{\lrd_v}}{1-\left(1-\frac{k}{2(\lbdd_v + \gamma + 1)}\right)} \\
    &\geq \lrd_v - \frac{1}{1-\left(1-\frac{k}{2(\lbdd_v + \gamma + 1)}\right)}
    \geq \lrd_v - \frac{2(\lbdd_v + \gamma + 1)}{k}.
\end{align*}
\end{proof}
\fi
Now, let $\lbddt$ denote the number of vertices in $\lset_2(u)$ that have at least one black edge from $\lset_1(u)$. Consequently, these vertices are included in $\apxball_2(u)$. We have the following:

\begin{lemma}\label{lemma:gamma_ok_properties}
Let $G=(V,E)$ be \gammaok, and $u \in V$. Then
\begin{align*}
     \lbddt\geq \sum_{v \in \lset_1(u)} \frac{\lbdd_v}{\gamma + 1}.
\end{align*}
\end{lemma}
\begin{proof}
    The inequality  follows from the fact that every node in $\lset_2(u)$ can have at most $\gamma + 1$ neighbors in $\lset_1(u)$.  
\end{proof}
  
We are now ready to prove \Cref{thm:gamma-ok-main}. 

%%% BEGIN OF THE PROOF OF THE THEOREM %%%%%
%\begin{proof}

The amortized update cost follows directly from \Cref{lm:amortized_det_alg}.

For the approximation quality, let us consider any vertex $u \in V$. For technical convenience, we will define a subgraph $\widetilde{G}$ of $G$ by removing suitable edges from $G$, and we establish the following two properties: (i) if $k\ge \frac{2(\gamma+1)}{\varepsilon}$, the \lazyscheme\ guarantees a $(1-\varepsilon)$-covering of $\ball_2(u)$ when the sequence of edge insertion is restricted to edges in $\widetilde{G}$; (ii) property (i) implies that the \lazyscheme\ also guarantees a $(1-\varepsilon)$-covering of $\ball_2(u)$ for $G$, provided that $k \geq \frac{4(\gamma+1)}{\varepsilon}$.

% We start by defining $\widetilde{G}$ which is obtained from $G$ as follows.
% For each $w \in \lset_2(u)$, if there exists a vertex $v \in \lset_1(u)$ such that edge $(v,w)$ is black for $v$, then we remove all the red edges incident to $w$ that comes from $\lset_1(u)$. Otherwise, we have that all the edges coming from $\lset_1(u)$ are red. In this case we remove all such edges but one. See Figure-?? for an example.\rem{Dobbiamo fare una piccola figura esplicativa.}
The subgraph $\widetilde{G}$ is obtained from $G$ through the following process.
For each vertex $w \in \lset_2(u)$, if there exists a black edge $(v,w)$ with $v \in \lset_1(u)$, then we remove all the red edges incident to $w$ that originate from $\lset_1(u)$.
Otherwise, if all edges from $\lset_1(u)$ to $w$ are red, we retain only one and remove the rest (see \Cref{fig:pruned_graph}).

\begin{figure}[h]
    \centering
    \includegraphics[width=.8\linewidth]{img/pruned_graph.pdf}
    \caption{The $2$-hop neighborhood of a vertex $u$ (left), and its corresponding structure in the subgraph $\widetilde{G}$ (right).}
    \label{fig:pruned_graph}
\end{figure}

We now prove property (i). 
We analyze the process at a generic time $t>0$.
We want to prove that $\Expec{}{\vert \apxball_2(u) \vert} \ge (1-\varepsilon)\vert \ball_2(u) \vert$, for any vertex $u \in V$. 
% Since the set $\lset_1(u)$ is always contained in $\apxball_2(u)$, we focus on vertices belonging to $\lset_2(u)$.
Since $\lset_1(u)$ is always included in $\apxball_2(u)$, it is sufficient to prove that $\vert \apxball_2(u) \cap \lset_2(u) \vert \geq (1-\varepsilon) \vert \lset_2(u) \vert$ in expectation.

% Let $\lambda$ and $\hat{\lambda}$ be the number of vertices in $\lset_2(u)$ attached to $\lset_1(u)$ with a red and with a quasi-black edge, respectively.
By construction of $\widetilde{G}$, we have that $\vert \apxball_2(u) \cap \lset_2(u) \vert = \beta + \sum_{v \in \lset_1(v)} \lrdr_v$, while $\vert \lset_2(u) \vert = \beta + \sum_{v \in \lset_1(v)} \lrd_v$.
% Thus, we now want to show that $\lbddt+ \hat{\lambda} \geq (1-\varepsilon)(\lbddt+ \lambda)$ in expectation, i.e.
% \begin{align} \label{eq:errro_bound_2}
%     \hat{\lambda} \ge (1-\varepsilon)\lambda - \varepsilon \lbddt.
% \end{align}
Thus, we want to show that
\begin{align} \label{eq:errro_bound_2}
    \sum_{v \in \lset_1(v)} \lrdr_v \ge (1-\varepsilon)\sum_{v \in \lset_1(v)} \lrd_v - \varepsilon \beta.
\end{align}
%Then, by definition of $\lambda$ and $\hat{\lambda}$ and by \Cref{lemma:gamma_ok_properties}, we get that  \Cref{eq:errro_bound_2} in turn is implied by 
By \Cref{lemma:gamma_ok_properties}, \eqref{eq:errro_bound_2} is true when 
\begin{equation}
\begin{aligned} \label{eq:error_bound_3}
    \sum_{v \in \lset_1(u)}{\lrdr_v} &\ge (1-\varepsilon)\sum_{v \in \lset_1(u)}{\lrd_v} - \frac{\varepsilon}{\gamma + 1}\sum_{v \in \lset_1(u)}{\lbdd_v}\\
    &= \sum_{v \in \lset_1(u)}{\left((1-\varepsilon)\lrd_v - \frac{\varepsilon}{\gamma + 1}\lbdd_v \right)}.
\end{aligned}
\end{equation}
In turn, the inequality in \eqref{eq:error_bound_3} holds in expectation if it holds term-by-term, i.e., when
\begin{equation*}
    \Expec{}{\lrdr_v} \ge (1-\varepsilon)\lrd_v - \frac{\varepsilon}{1+\gamma}\lbdd_v, \;\; \forall v \in \lset_1(u).
\end{equation*}
Clearly, if $\lrd_v = 0$ then $\lrdr_v = 0$ and the inequality holds; thus we focus on the case $\lrd_v > 0$.
From \Cref{le:gamma_ok_expect_lowerbound} we know that $\Expec{}{\lrdr_v} \geq \lrd_v - \frac{2(\lbdd_v + \gamma + 1)}{k}$.
By setting $k \geq \frac{2(\gamma+1)}{\varepsilon}$ we have
\begin{align*}
    \lrd_v - \frac{2(\lbdd_v + \gamma + 1)}{k} \geq \lrd_v - \frac{\varepsilon}{\gamma + 1} \lbdd_v \geq (1-\varepsilon)\lrd_v - \frac{\varepsilon}{\gamma + 1} \lbdd_v,
\end{align*}
therefore proving property (i).

\iffalse
Therefore, we obtain
\begin{align*}
    &\lrd_v - \frac{2(\lbdd_v + \gamma + 1)}{k} \ge (1-\varepsilon)\lrd_v - \frac{\varepsilon}{1+\gamma}\lbdd_v\\
    &\iff \frac{2(\lbdd_v + \gamma + 1)}{k} \leq \varepsilon \lrd_v - \frac{\varepsilon}{1 + \gamma}\lbdd_v \\
    &\iff \frac{k}{2(\lbdd_v + \gamma + 1)} \geq \frac{1+\gamma}{(1+\gamma)\varepsilon \lrd_v + \varepsilon \lbdd_v} \\
    &\iff k \geq \frac{2(1+\gamma + \lbdd_v)(1 + \gamma)}{(1 + \gamma)\varepsilon \lrd_v +\varepsilon \lbdd_v}
\end{align*}
Lastly, observe that
\begin{align*}
    k \ge \frac{2(1+\gamma)}{\varepsilon} \implies k \ge \frac{2(1+\gamma + \lbdd_v)(1 + \gamma)}{(1 + \gamma)\varepsilon \lrd_v + \varepsilon \lbdd_v} =  \frac{2(1+\gamma)}{\varepsilon} \frac{1 + \gamma + \lbdd_v}{(1+\gamma)\lrd_v + \lbdd_v}.    
\end{align*}
Therefore, property (i) is proven.
\fi

% Finally, we prove (ii). Notice that we build $\widetilde{G}$ by removing \emph{only} red edges from $G$.
% On the one hand, this reduces (AUMENTA) the chance to select $u$ at Line 14 in \Cref{alg:det_thresh} as random neighbor of some vertex in $\lset_1(u)$. On the other hand, we reduce the degree of some vertex in $\lset_1(u)$. However, since $\varphi=1$, we have that the red degree of any vertex is at most its black degree. So, in $\widetilde{G}$ the degree of a vertex in $\lset_1(u)$ can at most be halved. In order to manage this we simply double the value of $k$, and the claim follows.
%\end{proof}

Finally, we prove (ii). Notice that we build $\widetilde{G}$ by removing \emph{only} red edges from $G$.
Since $\varphi = 1$, the degrees of the vertices $v \in \lset_1(u)$ in $\widetilde{G}$ are \emph{at most halved} compared to $G$. As a consequence, the probability that $v$ selects $u$ at \Cref{line:random_selection} of \Cref{alg:det_thresh} in $G$ is at most half of the probability we have in $\widetilde{G}$. Therefore, doubling the value of $k$ to $\frac{4(\gamma + 1)}{\varepsilon}$ ensures that the analysis we conducted on $\widetilde{G}$ also holds on $G$, guaranteeing a $(1-\varepsilon)$-covering of $\ball_2(u)$ in expectation on $G$ as well.







\section{Experiments}
\label{sec:experiments}
The experiments are designed to address two key research questions.
First, \textbf{RQ1} evaluates whether the average $L_2$-norm of the counterfactual perturbation vectors ($\overline{||\perturb||}$) decreases as the model overfits the data, thereby providing further empirical validation for our hypothesis.
Second, \textbf{RQ2} evaluates the ability of the proposed counterfactual regularized loss, as defined in (\ref{eq:regularized_loss2}), to mitigate overfitting when compared to existing regularization techniques.

% The experiments are designed to address three key research questions. First, \textbf{RQ1} investigates whether the mean perturbation vector norm decreases as the model overfits the data, aiming to further validate our intuition. Second, \textbf{RQ2} explores whether the mean perturbation vector norm can be effectively leveraged as a regularization term during training, offering insights into its potential role in mitigating overfitting. Finally, \textbf{RQ3} examines whether our counterfactual regularizer enables the model to achieve superior performance compared to existing regularization methods, thus highlighting its practical advantage.

\subsection{Experimental Setup}
\textbf{\textit{Datasets, Models, and Tasks.}}
The experiments are conducted on three datasets: \textit{Water Potability}~\cite{kadiwal2020waterpotability}, \textit{Phomene}~\cite{phomene}, and \textit{CIFAR-10}~\cite{krizhevsky2009learning}. For \textit{Water Potability} and \textit{Phomene}, we randomly select $80\%$ of the samples for the training set, and the remaining $20\%$ for the test set, \textit{CIFAR-10} comes already split. Furthermore, we consider the following models: Logistic Regression, Multi-Layer Perceptron (MLP) with 100 and 30 neurons on each hidden layer, and PreactResNet-18~\cite{he2016cvecvv} as a Convolutional Neural Network (CNN) architecture.
We focus on binary classification tasks and leave the extension to multiclass scenarios for future work. However, for datasets that are inherently multiclass, we transform the problem into a binary classification task by selecting two classes, aligning with our assumption.

\smallskip
\noindent\textbf{\textit{Evaluation Measures.}} To characterize the degree of overfitting, we use the test loss, as it serves as a reliable indicator of the model's generalization capability to unseen data. Additionally, we evaluate the predictive performance of each model using the test accuracy.

\smallskip
\noindent\textbf{\textit{Baselines.}} We compare CF-Reg with the following regularization techniques: L1 (``Lasso''), L2 (``Ridge''), and Dropout.

\smallskip
\noindent\textbf{\textit{Configurations.}}
For each model, we adopt specific configurations as follows.
\begin{itemize}
\item \textit{Logistic Regression:} To induce overfitting in the model, we artificially increase the dimensionality of the data beyond the number of training samples by applying a polynomial feature expansion. This approach ensures that the model has enough capacity to overfit the training data, allowing us to analyze the impact of our counterfactual regularizer. The degree of the polynomial is chosen as the smallest degree that makes the number of features greater than the number of data.
\item \textit{Neural Networks (MLP and CNN):} To take advantage of the closed-form solution for computing the optimal perturbation vector as defined in (\ref{eq:opt-delta}), we use a local linear approximation of the neural network models. Hence, given an instance $\inst_i$, we consider the (optimal) counterfactual not with respect to $\model$ but with respect to:
\begin{equation}
\label{eq:taylor}
    \model^{lin}(\inst) = \model(\inst_i) + \nabla_{\inst}\model(\inst_i)(\inst - \inst_i),
\end{equation}
where $\model^{lin}$ represents the first-order Taylor approximation of $\model$ at $\inst_i$.
Note that this step is unnecessary for Logistic Regression, as it is inherently a linear model.
\end{itemize}

\smallskip
\noindent \textbf{\textit{Implementation Details.}} We run all experiments on a machine equipped with an AMD Ryzen 9 7900 12-Core Processor and an NVIDIA GeForce RTX 4090 GPU. Our implementation is based on the PyTorch Lightning framework. We use stochastic gradient descent as the optimizer with a learning rate of $\eta = 0.001$ and no weight decay. We use a batch size of $128$. The training and test steps are conducted for $6000$ epochs on the \textit{Water Potability} and \textit{Phoneme} datasets, while for the \textit{CIFAR-10} dataset, they are performed for $200$ epochs.
Finally, the contribution $w_i^{\varepsilon}$ of each training point $\inst_i$ is uniformly set as $w_i^{\varepsilon} = 1~\forall i\in \{1,\ldots,m\}$.

The source code implementation for our experiments is available at the following GitHub repository: \url{https://anonymous.4open.science/r/COCE-80B4/README.md} 

\subsection{RQ1: Counterfactual Perturbation vs. Overfitting}
To address \textbf{RQ1}, we analyze the relationship between the test loss and the average $L_2$-norm of the counterfactual perturbation vectors ($\overline{||\perturb||}$) over training epochs.

In particular, Figure~\ref{fig:delta_loss_epochs} depicts the evolution of $\overline{||\perturb||}$ alongside the test loss for an MLP trained \textit{without} regularization on the \textit{Water Potability} dataset. 
\begin{figure}[ht]
    \centering
    \includegraphics[width=0.85\linewidth]{img/delta_loss_epochs.png}
    \caption{The average counterfactual perturbation vector $\overline{||\perturb||}$ (left $y$-axis) and the cross-entropy test loss (right $y$-axis) over training epochs ($x$-axis) for an MLP trained on the \textit{Water Potability} dataset \textit{without} regularization.}
    \label{fig:delta_loss_epochs}
\end{figure}

The plot shows a clear trend as the model starts to overfit the data (evidenced by an increase in test loss). 
Notably, $\overline{||\perturb||}$ begins to decrease, which aligns with the hypothesis that the average distance to the optimal counterfactual example gets smaller as the model's decision boundary becomes increasingly adherent to the training data.

It is worth noting that this trend is heavily influenced by the choice of the counterfactual generator model. In particular, the relationship between $\overline{||\perturb||}$ and the degree of overfitting may become even more pronounced when leveraging more accurate counterfactual generators. However, these models often come at the cost of higher computational complexity, and their exploration is left to future work.

Nonetheless, we expect that $\overline{||\perturb||}$ will eventually stabilize at a plateau, as the average $L_2$-norm of the optimal counterfactual perturbations cannot vanish to zero.

% Additionally, the choice of employing the score-based counterfactual explanation framework to generate counterfactuals was driven to promote computational efficiency.

% Future enhancements to the framework may involve adopting models capable of generating more precise counterfactuals. While such approaches may yield to performance improvements, they are likely to come at the cost of increased computational complexity.


\subsection{RQ2: Counterfactual Regularization Performance}
To answer \textbf{RQ2}, we evaluate the effectiveness of the proposed counterfactual regularization (CF-Reg) by comparing its performance against existing baselines: unregularized training loss (No-Reg), L1 regularization (L1-Reg), L2 regularization (L2-Reg), and Dropout.
Specifically, for each model and dataset combination, Table~\ref{tab:regularization_comparison} presents the mean value and standard deviation of test accuracy achieved by each method across 5 random initialization. 

The table illustrates that our regularization technique consistently delivers better results than existing methods across all evaluated scenarios, except for one case -- i.e., Logistic Regression on the \textit{Phomene} dataset. 
However, this setting exhibits an unusual pattern, as the highest model accuracy is achieved without any regularization. Even in this case, CF-Reg still surpasses other regularization baselines.

From the results above, we derive the following key insights. First, CF-Reg proves to be effective across various model types, ranging from simple linear models (Logistic Regression) to deep architectures like MLPs and CNNs, and across diverse datasets, including both tabular and image data. 
Second, CF-Reg's strong performance on the \textit{Water} dataset with Logistic Regression suggests that its benefits may be more pronounced when applied to simpler models. However, the unexpected outcome on the \textit{Phoneme} dataset calls for further investigation into this phenomenon.


\begin{table*}[h!]
    \centering
    \caption{Mean value and standard deviation of test accuracy across 5 random initializations for different model, dataset, and regularization method. The best results are highlighted in \textbf{bold}.}
    \label{tab:regularization_comparison}
    \begin{tabular}{|c|c|c|c|c|c|c|}
        \hline
        \textbf{Model} & \textbf{Dataset} & \textbf{No-Reg} & \textbf{L1-Reg} & \textbf{L2-Reg} & \textbf{Dropout} & \textbf{CF-Reg (ours)} \\ \hline
        Logistic Regression   & \textit{Water}   & $0.6595 \pm 0.0038$   & $0.6729 \pm 0.0056$   & $0.6756 \pm 0.0046$  & N/A    & $\mathbf{0.6918 \pm 0.0036}$                     \\ \hline
        MLP   & \textit{Water}   & $0.6756 \pm 0.0042$   & $0.6790 \pm 0.0058$   & $0.6790 \pm 0.0023$  & $0.6750 \pm 0.0036$    & $\mathbf{0.6802 \pm 0.0046}$                    \\ \hline
%        MLP   & \textit{Adult}   & $0.8404 \pm 0.0010$   & $\mathbf{0.8495 \pm 0.0007}$   & $0.8489 \pm 0.0014$  & $\mathbf{0.8495 \pm 0.0016}$     & $0.8449 \pm 0.0019$                    \\ \hline
        Logistic Regression   & \textit{Phomene}   & $\mathbf{0.8148 \pm 0.0020}$   & $0.8041 \pm 0.0028$   & $0.7835 \pm 0.0176$  & N/A    & $0.8098 \pm 0.0055$                     \\ \hline
        MLP   & \textit{Phomene}   & $0.8677 \pm 0.0033$   & $0.8374 \pm 0.0080$   & $0.8673 \pm 0.0045$  & $0.8672 \pm 0.0042$     & $\mathbf{0.8718 \pm 0.0040}$                    \\ \hline
        CNN   & \textit{CIFAR-10} & $0.6670 \pm 0.0233$   & $0.6229 \pm 0.0850$   & $0.7348 \pm 0.0365$   & N/A    & $\mathbf{0.7427 \pm 0.0571}$                     \\ \hline
    \end{tabular}
\end{table*}

\begin{table*}[htb!]
    \centering
    \caption{Hyperparameter configurations utilized for the generation of Table \ref{tab:regularization_comparison}. For our regularization the hyperparameters are reported as $\mathbf{\alpha/\beta}$.}
    \label{tab:performance_parameters}
    \begin{tabular}{|c|c|c|c|c|c|c|}
        \hline
        \textbf{Model} & \textbf{Dataset} & \textbf{No-Reg} & \textbf{L1-Reg} & \textbf{L2-Reg} & \textbf{Dropout} & \textbf{CF-Reg (ours)} \\ \hline
        Logistic Regression   & \textit{Water}   & N/A   & $0.0093$   & $0.6927$  & N/A    & $0.3791/1.0355$                     \\ \hline
        MLP   & \textit{Water}   & N/A   & $0.0007$   & $0.0022$  & $0.0002$    & $0.2567/1.9775$                    \\ \hline
        Logistic Regression   &
        \textit{Phomene}   & N/A   & $0.0097$   & $0.7979$  & N/A    & $0.0571/1.8516$                     \\ \hline
        MLP   & \textit{Phomene}   & N/A   & $0.0007$   & $4.24\cdot10^{-5}$  & $0.0015$    & $0.0516/2.2700$                    \\ \hline
       % MLP   & \textit{Adult}   & N/A   & $0.0018$   & $0.0018$  & $0.0601$     & $0.0764/2.2068$                    \\ \hline
        CNN   & \textit{CIFAR-10} & N/A   & $0.0050$   & $0.0864$ & N/A    & $0.3018/
        2.1502$                     \\ \hline
    \end{tabular}
\end{table*}

\begin{table*}[htb!]
    \centering
    \caption{Mean value and standard deviation of training time across 5 different runs. The reported time (in seconds) corresponds to the generation of each entry in Table \ref{tab:regularization_comparison}. Times are }
    \label{tab:times}
    \begin{tabular}{|c|c|c|c|c|c|c|}
        \hline
        \textbf{Model} & \textbf{Dataset} & \textbf{No-Reg} & \textbf{L1-Reg} & \textbf{L2-Reg} & \textbf{Dropout} & \textbf{CF-Reg (ours)} \\ \hline
        Logistic Regression   & \textit{Water}   & $222.98 \pm 1.07$   & $239.94 \pm 2.59$   & $241.60 \pm 1.88$  & N/A    & $251.50 \pm 1.93$                     \\ \hline
        MLP   & \textit{Water}   & $225.71 \pm 3.85$   & $250.13 \pm 4.44$   & $255.78 \pm 2.38$  & $237.83 \pm 3.45$    & $266.48 \pm 3.46$                    \\ \hline
        Logistic Regression   & \textit{Phomene}   & $266.39 \pm 0.82$ & $367.52 \pm 6.85$   & $361.69 \pm 4.04$  & N/A   & $310.48 \pm 0.76$                    \\ \hline
        MLP   &
        \textit{Phomene} & $335.62 \pm 1.77$   & $390.86 \pm 2.11$   & $393.96 \pm 1.95$ & $363.51 \pm 5.07$    & $403.14 \pm 1.92$                     \\ \hline
       % MLP   & \textit{Adult}   & N/A   & $0.0018$   & $0.0018$  & $0.0601$     & $0.0764/2.2068$                    \\ \hline
        CNN   & \textit{CIFAR-10} & $370.09 \pm 0.18$   & $395.71 \pm 0.55$   & $401.38 \pm 0.16$ & N/A    & $1287.8 \pm 0.26$                     \\ \hline
    \end{tabular}
\end{table*}

\subsection{Feasibility of our Method}
A crucial requirement for any regularization technique is that it should impose minimal impact on the overall training process.
In this respect, CF-Reg introduces an overhead that depends on the time required to find the optimal counterfactual example for each training instance. 
As such, the more sophisticated the counterfactual generator model probed during training the higher would be the time required. However, a more advanced counterfactual generator might provide a more effective regularization. We discuss this trade-off in more details in Section~\ref{sec:discussion}.

Table~\ref{tab:times} presents the average training time ($\pm$ standard deviation) for each model and dataset combination listed in Table~\ref{tab:regularization_comparison}.
We can observe that the higher accuracy achieved by CF-Reg using the score-based counterfactual generator comes with only minimal overhead. However, when applied to deep neural networks with many hidden layers, such as \textit{PreactResNet-18}, the forward derivative computation required for the linearization of the network introduces a more noticeable computational cost, explaining the longer training times in the table.

\subsection{Hyperparameter Sensitivity Analysis}
The proposed counterfactual regularization technique relies on two key hyperparameters: $\alpha$ and $\beta$. The former is intrinsic to the loss formulation defined in (\ref{eq:cf-train}), while the latter is closely tied to the choice of the score-based counterfactual explanation method used.

Figure~\ref{fig:test_alpha_beta} illustrates how the test accuracy of an MLP trained on the \textit{Water Potability} dataset changes for different combinations of $\alpha$ and $\beta$.

\begin{figure}[ht]
    \centering
    \includegraphics[width=0.85\linewidth]{img/test_acc_alpha_beta.png}
    \caption{The test accuracy of an MLP trained on the \textit{Water Potability} dataset, evaluated while varying the weight of our counterfactual regularizer ($\alpha$) for different values of $\beta$.}
    \label{fig:test_alpha_beta}
\end{figure}

We observe that, for a fixed $\beta$, increasing the weight of our counterfactual regularizer ($\alpha$) can slightly improve test accuracy until a sudden drop is noticed for $\alpha > 0.1$.
This behavior was expected, as the impact of our penalty, like any regularization term, can be disruptive if not properly controlled.

Moreover, this finding further demonstrates that our regularization method, CF-Reg, is inherently data-driven. Therefore, it requires specific fine-tuning based on the combination of the model and dataset at hand.
\vspace{-0.2cm}
\section{Impact: Why Free Scientific Knowledge?}
\vspace{-0.1cm}

Historically, making knowledge widely available has driven transformative progress. Gutenberg’s printing press broke medieval monopolies on information, increasing literacy and contributing to the Renaissance and Scientific Revolution. In today's world, open source projects such as GNU/Linux and Wikipedia show that freely accessible and modifiable knowledge fosters innovation while ensuring creators are credited through copyleft licenses. These examples highlight a key idea: \textit{access to essential knowledge supports overall advancement.} 

This aligns with the arguments made by Prabhakaran et al. \cite{humanrightsbasedapproachresponsible}, who specifically highlight the \textbf{ human right to participate in scientific advancement} as enshrined in the Universal Declaration of Human Rights. They emphasize that this right underscores the importance of \textit{ equal access to the benefits of scientific progress for all}, a principle directly supported by our proposal for Knowledge Units. The UN Special Rapporteur on Cultural Rights further reinforces this, advocating for the expansion of copyright exceptions to broaden access to scientific knowledge as a crucial component of the right to science and culture \cite{scienceright}. 

However, current intellectual property regimes often create ``patently unfair" barriers to this knowledge, preventing innovation and access, especially in areas critical to human rights, as Hale compellingly argues \cite{patentlyunfair}. Finding a solution requires carefully balancing the imperative of open access with the legitimate rights of authors. As Austin and Ginsburg remind us, authors' rights are also human rights, necessitating robust protection \cite{authorhumanrights}. Shareable knowledge entities like Knowledge Units offer a potential mechanism to achieve this delicate balance in the scientific domain, enabling wider dissemination of research findings while respecting authors' fundamental rights.

\vspace{-0.2cm}
\subsection{Impact Across Sectors}

\textbf{Researchers:} Collaboration across different fields becomes easier when knowledge is shared openly. For instance, combining machine learning with biology or applying quantum principles to cryptography can lead to important breakthroughs. Removing copyright restrictions allows researchers to freely use data and methods, speeding up discoveries while respecting original contributions.

\textbf{Practitioners:} Professionals, especially in healthcare, benefit from immediate access to the latest research. Quick access to newer insights on the effectiveness of drugs, and alternative treatments speeds up adoption and awareness, potentially saving lives. Additionally, open knowledge helps developing countries gain access to health innovations.

\textbf{Education:} Education becomes more accessible when teachers use the latest research to create up-to-date curricula without prohibitive costs. Students can access high-quality research materials and use LM assistance to better understand complex topics, enhancing their learning experience and making high-quality education more accessible.

\textbf{Public Trust:} When information is transparent and accessible, the public can better understand and trust decision-making processes. Open access to government policies and industry practices allows people to review and verify information, helping to reduce misinformation. This transparency encourages critical thinking and builds trust in scientific and governmental institutions.

Overall, making scientific knowledge accessible supports global fairness. By viewing knowledge as a common resource rather than a product to be sold, we can speed up innovation, encourage critical thinking, and empower communities to address important challenges.

\vspace{-0.2cm}
\section{Open Problems}
\vspace{-0.1cm}

Moving forward, we identify key research directions to further exploit the potential of converting original texts into shareable knowledge entities such as demonstrated by the conversion into Knowledge Units in this work:


\textbf{1. Enhancing Factual Accuracy and Reliability:}  Refining KUs through cross-referencing with source texts and incorporating community-driven correction mechanisms, similar to Wikipedia, can minimize hallucinations and ensure the long-term accuracy of knowledge-based datasets at scale.

\textbf{2. Developing Applications for Education and Research:}  Using KU-based conversion for datasets to be employed in practical tools, such as search interfaces and learning platforms, can ensure rapid dissemination of any new knowledge into shareable downstream resources, significantly improving the accessibility, spread, and impact of KUs.

\textbf{3. Establishing Standards for Knowledge Interoperability and Reuse:}  Future research should focus on defining standardized formats for entities like KU and knowledge graph layouts \citep{lenat1990cyc}. These standards are essential to unlock seamless interoperability, facilitate reuse across diverse platforms, and foster a vibrant ecosystem of open scientific knowledge. 

\textbf{4. Interconnecting Shareable Knowledge for Scientific Workflow Assistance and Automation:} There might be further potential in constructing a semantic web that interconnects publicly shared knowledge, together with mechanisms that continually update and validate all shareable knowledge units. This can be starting point for a platform that uses all collected knowledge to assist scientific workflows, for instance by feeding such a semantic web into recently developed reasoning models equipped with retrieval augmented generation. Such assistance could assemble knowledge across multiple scientific papers, guiding scientists more efficiently through vast research landscapes. Given further progress in model capabilities, validation, self-repair and evolving new knowledge from already existing vast collection in the semantic web can lead to automation of scientific discovery, assuming that knowledge data in the semantic web can be freely shared.

We open-source our code and encourage collaboration to improve extraction pipelines, enhance Knowledge Unit capabilities, and expand coverage to additional fields.

\vspace{-0.2cm}
\section{Conclusion}
\vspace{-0.1cm}

In this paper, we highlight the potential of systematically separating factual scientific knowledge from protected artistic or stylistic expression. By representing scientific insights as structured facts and relationships, prototypes like Knowledge Units (KUs) offer a pathway to broaden access to scientific knowledge without infringing copyright, aligning with legal principles like German \S 24(1) UrhG and U.S. fair use standards. Extensive testing across a range of domains and models shows evidence that Knowledge Units (KUs) can feasibly retain core information. These findings offer a promising way forward for openly disseminating scientific information while respecting copyright constraints.

\section*{Author Contributions}

Christoph conceived the project and led organization. Christoph and Gollam led all the experiments. Nick and Huu led the legal aspects. Tawsif led the data collection. Ameya and Andreas led the manuscript writing. Ludwig, Sören, Robert, Jenia and Matthias provided feedback. advice and scientific supervision throughout the project. 

\section*{Acknowledgements}

The authors would like to thank (in alphabetical order): Sebastian Dziadzio, Kristof Meding, Tea Mustać, Shantanu Prabhat for insightful feedback and suggestions. Special thanks to Andrej Radonjic for help in scaling up data collection. GR and SA acknowledge financial support by the German Research Foundation (DFG) for the NFDI4DataScience Initiative (project number 460234259). AP and MB acknowledge financial support by the Federal Ministry of Education and Research (BMBF), FKZ: 011524085B and Open Philanthropy Foundation funded by the Good Ventures Foundation. AH acknowledges financial support by the Federal Ministry of Education and Research (BMBF), FKZ: 01IS24079A and the Carl Zeiss Foundation through the project "Certification and Foundations of Safe ML Systems" as well as the support from the International Max Planck Research School for Intelligent Systems (IMPRS-IS). JJ acknowledges funding by the Federal Ministry of Education and Research of Germany (BMBF) under grant no. 01IS22094B (WestAI - AI Service Center West), under grant no. 01IS24085C (OPENHAFM) and under the grant DE002571 (MINERVA), as well as co-funding by EU from EuroHPC Joint Undertaking programm under grant no. 101182737 (MINERVA) and from Digital Europe Programme under grant no. 101195233 (openEuroLLM) 
%\subsection{Lloyd-Max Algorithm}
\label{subsec:Lloyd-Max}
For a given quantization bitwidth $B$ and an operand $\bm{X}$, the Lloyd-Max algorithm finds $2^B$ quantization levels $\{\hat{x}_i\}_{i=1}^{2^B}$ such that quantizing $\bm{X}$ by rounding each scalar in $\bm{X}$ to the nearest quantization level minimizes the quantization MSE. 

The algorithm starts with an initial guess of quantization levels and then iteratively computes quantization thresholds $\{\tau_i\}_{i=1}^{2^B-1}$ and updates quantization levels $\{\hat{x}_i\}_{i=1}^{2^B}$. Specifically, at iteration $n$, thresholds are set to the midpoints of the previous iteration's levels:
\begin{align*}
    \tau_i^{(n)}=\frac{\hat{x}_i^{(n-1)}+\hat{x}_{i+1}^{(n-1)}}2 \text{ for } i=1\ldots 2^B-1
\end{align*}
Subsequently, the quantization levels are re-computed as conditional means of the data regions defined by the new thresholds:
\begin{align*}
    \hat{x}_i^{(n)}=\mathbb{E}\left[ \bm{X} \big| \bm{X}\in [\tau_{i-1}^{(n)},\tau_i^{(n)}] \right] \text{ for } i=1\ldots 2^B
\end{align*}
where to satisfy boundary conditions we have $\tau_0=-\infty$ and $\tau_{2^B}=\infty$. The algorithm iterates the above steps until convergence.

Figure \ref{fig:lm_quant} compares the quantization levels of a $7$-bit floating point (E3M3) quantizer (left) to a $7$-bit Lloyd-Max quantizer (right) when quantizing a layer of weights from the GPT3-126M model at a per-tensor granularity. As shown, the Lloyd-Max quantizer achieves substantially lower quantization MSE. Further, Table \ref{tab:FP7_vs_LM7} shows the superior perplexity achieved by Lloyd-Max quantizers for bitwidths of $7$, $6$ and $5$. The difference between the quantizers is clear at 5 bits, where per-tensor FP quantization incurs a drastic and unacceptable increase in perplexity, while Lloyd-Max quantization incurs a much smaller increase. Nevertheless, we note that even the optimal Lloyd-Max quantizer incurs a notable ($\sim 1.5$) increase in perplexity due to the coarse granularity of quantization. 

\begin{figure}[h]
  \centering
  \includegraphics[width=0.7\linewidth]{sections/figures/LM7_FP7.pdf}
  \caption{\small Quantization levels and the corresponding quantization MSE of Floating Point (left) vs Lloyd-Max (right) Quantizers for a layer of weights in the GPT3-126M model.}
  \label{fig:lm_quant}
\end{figure}

\begin{table}[h]\scriptsize
\begin{center}
\caption{\label{tab:FP7_vs_LM7} \small Comparing perplexity (lower is better) achieved by floating point quantizers and Lloyd-Max quantizers on a GPT3-126M model for the Wikitext-103 dataset.}
\begin{tabular}{c|cc|c}
\hline
 \multirow{2}{*}{\textbf{Bitwidth}} & \multicolumn{2}{|c|}{\textbf{Floating-Point Quantizer}} & \textbf{Lloyd-Max Quantizer} \\
 & Best Format & Wikitext-103 Perplexity & Wikitext-103 Perplexity \\
\hline
7 & E3M3 & 18.32 & 18.27 \\
6 & E3M2 & 19.07 & 18.51 \\
5 & E4M0 & 43.89 & 19.71 \\
\hline
\end{tabular}
\end{center}
\end{table}

\subsection{Proof of Local Optimality of LO-BCQ}
\label{subsec:lobcq_opt_proof}
For a given block $\bm{b}_j$, the quantization MSE during LO-BCQ can be empirically evaluated as $\frac{1}{L_b}\lVert \bm{b}_j- \bm{\hat{b}}_j\rVert^2_2$ where $\bm{\hat{b}}_j$ is computed from equation (\ref{eq:clustered_quantization_definition}) as $C_{f(\bm{b}_j)}(\bm{b}_j)$. Further, for a given block cluster $\mathcal{B}_i$, we compute the quantization MSE as $\frac{1}{|\mathcal{B}_{i}|}\sum_{\bm{b} \in \mathcal{B}_{i}} \frac{1}{L_b}\lVert \bm{b}- C_i^{(n)}(\bm{b})\rVert^2_2$. Therefore, at the end of iteration $n$, we evaluate the overall quantization MSE $J^{(n)}$ for a given operand $\bm{X}$ composed of $N_c$ block clusters as:
\begin{align*}
    \label{eq:mse_iter_n}
    J^{(n)} = \frac{1}{N_c} \sum_{i=1}^{N_c} \frac{1}{|\mathcal{B}_{i}^{(n)}|}\sum_{\bm{v} \in \mathcal{B}_{i}^{(n)}} \frac{1}{L_b}\lVert \bm{b}- B_i^{(n)}(\bm{b})\rVert^2_2
\end{align*}

At the end of iteration $n$, the codebooks are updated from $\mathcal{C}^{(n-1)}$ to $\mathcal{C}^{(n)}$. However, the mapping of a given vector $\bm{b}_j$ to quantizers $\mathcal{C}^{(n)}$ remains as  $f^{(n)}(\bm{b}_j)$. At the next iteration, during the vector clustering step, $f^{(n+1)}(\bm{b}_j)$ finds new mapping of $\bm{b}_j$ to updated codebooks $\mathcal{C}^{(n)}$ such that the quantization MSE over the candidate codebooks is minimized. Therefore, we obtain the following result for $\bm{b}_j$:
\begin{align*}
\frac{1}{L_b}\lVert \bm{b}_j - C_{f^{(n+1)}(\bm{b}_j)}^{(n)}(\bm{b}_j)\rVert^2_2 \le \frac{1}{L_b}\lVert \bm{b}_j - C_{f^{(n)}(\bm{b}_j)}^{(n)}(\bm{b}_j)\rVert^2_2
\end{align*}

That is, quantizing $\bm{b}_j$ at the end of the block clustering step of iteration $n+1$ results in lower quantization MSE compared to quantizing at the end of iteration $n$. Since this is true for all $\bm{b} \in \bm{X}$, we assert the following:
\begin{equation}
\begin{split}
\label{eq:mse_ineq_1}
    \tilde{J}^{(n+1)} &= \frac{1}{N_c} \sum_{i=1}^{N_c} \frac{1}{|\mathcal{B}_{i}^{(n+1)}|}\sum_{\bm{b} \in \mathcal{B}_{i}^{(n+1)}} \frac{1}{L_b}\lVert \bm{b} - C_i^{(n)}(b)\rVert^2_2 \le J^{(n)}
\end{split}
\end{equation}
where $\tilde{J}^{(n+1)}$ is the the quantization MSE after the vector clustering step at iteration $n+1$.

Next, during the codebook update step (\ref{eq:quantizers_update}) at iteration $n+1$, the per-cluster codebooks $\mathcal{C}^{(n)}$ are updated to $\mathcal{C}^{(n+1)}$ by invoking the Lloyd-Max algorithm \citep{Lloyd}. We know that for any given value distribution, the Lloyd-Max algorithm minimizes the quantization MSE. Therefore, for a given vector cluster $\mathcal{B}_i$ we obtain the following result:

\begin{equation}
    \frac{1}{|\mathcal{B}_{i}^{(n+1)}|}\sum_{\bm{b} \in \mathcal{B}_{i}^{(n+1)}} \frac{1}{L_b}\lVert \bm{b}- C_i^{(n+1)}(\bm{b})\rVert^2_2 \le \frac{1}{|\mathcal{B}_{i}^{(n+1)}|}\sum_{\bm{b} \in \mathcal{B}_{i}^{(n+1)}} \frac{1}{L_b}\lVert \bm{b}- C_i^{(n)}(\bm{b})\rVert^2_2
\end{equation}

The above equation states that quantizing the given block cluster $\mathcal{B}_i$ after updating the associated codebook from $C_i^{(n)}$ to $C_i^{(n+1)}$ results in lower quantization MSE. Since this is true for all the block clusters, we derive the following result: 
\begin{equation}
\begin{split}
\label{eq:mse_ineq_2}
     J^{(n+1)} &= \frac{1}{N_c} \sum_{i=1}^{N_c} \frac{1}{|\mathcal{B}_{i}^{(n+1)}|}\sum_{\bm{b} \in \mathcal{B}_{i}^{(n+1)}} \frac{1}{L_b}\lVert \bm{b}- C_i^{(n+1)}(\bm{b})\rVert^2_2  \le \tilde{J}^{(n+1)}   
\end{split}
\end{equation}

Following (\ref{eq:mse_ineq_1}) and (\ref{eq:mse_ineq_2}), we find that the quantization MSE is non-increasing for each iteration, that is, $J^{(1)} \ge J^{(2)} \ge J^{(3)} \ge \ldots \ge J^{(M)}$ where $M$ is the maximum number of iterations. 
%Therefore, we can say that if the algorithm converges, then it must be that it has converged to a local minimum. 
\hfill $\blacksquare$


\begin{figure}
    \begin{center}
    \includegraphics[width=0.5\textwidth]{sections//figures/mse_vs_iter.pdf}
    \end{center}
    \caption{\small NMSE vs iterations during LO-BCQ compared to other block quantization proposals}
    \label{fig:nmse_vs_iter}
\end{figure}

Figure \ref{fig:nmse_vs_iter} shows the empirical convergence of LO-BCQ across several block lengths and number of codebooks. Also, the MSE achieved by LO-BCQ is compared to baselines such as MXFP and VSQ. As shown, LO-BCQ converges to a lower MSE than the baselines. Further, we achieve better convergence for larger number of codebooks ($N_c$) and for a smaller block length ($L_b$), both of which increase the bitwidth of BCQ (see Eq \ref{eq:bitwidth_bcq}).


\subsection{Additional Accuracy Results}
%Table \ref{tab:lobcq_config} lists the various LOBCQ configurations and their corresponding bitwidths.
\begin{table}
\setlength{\tabcolsep}{4.75pt}
\begin{center}
\caption{\label{tab:lobcq_config} Various LO-BCQ configurations and their bitwidths.}
\begin{tabular}{|c||c|c|c|c||c|c||c|} 
\hline
 & \multicolumn{4}{|c||}{$L_b=8$} & \multicolumn{2}{|c||}{$L_b=4$} & $L_b=2$ \\
 \hline
 \backslashbox{$L_A$\kern-1em}{\kern-1em$N_c$} & 2 & 4 & 8 & 16 & 2 & 4 & 2 \\
 \hline
 64 & 4.25 & 4.375 & 4.5 & 4.625 & 4.375 & 4.625 & 4.625\\
 \hline
 32 & 4.375 & 4.5 & 4.625& 4.75 & 4.5 & 4.75 & 4.75 \\
 \hline
 16 & 4.625 & 4.75& 4.875 & 5 & 4.75 & 5 & 5 \\
 \hline
\end{tabular}
\end{center}
\end{table}

%\subsection{Perplexity achieved by various LO-BCQ configurations on Wikitext-103 dataset}

\begin{table} \centering
\begin{tabular}{|c||c|c|c|c||c|c||c|} 
\hline
 $L_b \rightarrow$& \multicolumn{4}{c||}{8} & \multicolumn{2}{c||}{4} & 2\\
 \hline
 \backslashbox{$L_A$\kern-1em}{\kern-1em$N_c$} & 2 & 4 & 8 & 16 & 2 & 4 & 2  \\
 %$N_c \rightarrow$ & 2 & 4 & 8 & 16 & 2 & 4 & 2 \\
 \hline
 \hline
 \multicolumn{8}{c}{GPT3-1.3B (FP32 PPL = 9.98)} \\ 
 \hline
 \hline
 64 & 10.40 & 10.23 & 10.17 & 10.15 &  10.28 & 10.18 & 10.19 \\
 \hline
 32 & 10.25 & 10.20 & 10.15 & 10.12 &  10.23 & 10.17 & 10.17 \\
 \hline
 16 & 10.22 & 10.16 & 10.10 & 10.09 &  10.21 & 10.14 & 10.16 \\
 \hline
  \hline
 \multicolumn{8}{c}{GPT3-8B (FP32 PPL = 7.38)} \\ 
 \hline
 \hline
 64 & 7.61 & 7.52 & 7.48 &  7.47 &  7.55 &  7.49 & 7.50 \\
 \hline
 32 & 7.52 & 7.50 & 7.46 &  7.45 &  7.52 &  7.48 & 7.48  \\
 \hline
 16 & 7.51 & 7.48 & 7.44 &  7.44 &  7.51 &  7.49 & 7.47  \\
 \hline
\end{tabular}
\caption{\label{tab:ppl_gpt3_abalation} Wikitext-103 perplexity across GPT3-1.3B and 8B models.}
\end{table}

\begin{table} \centering
\begin{tabular}{|c||c|c|c|c||} 
\hline
 $L_b \rightarrow$& \multicolumn{4}{c||}{8}\\
 \hline
 \backslashbox{$L_A$\kern-1em}{\kern-1em$N_c$} & 2 & 4 & 8 & 16 \\
 %$N_c \rightarrow$ & 2 & 4 & 8 & 16 & 2 & 4 & 2 \\
 \hline
 \hline
 \multicolumn{5}{|c|}{Llama2-7B (FP32 PPL = 5.06)} \\ 
 \hline
 \hline
 64 & 5.31 & 5.26 & 5.19 & 5.18  \\
 \hline
 32 & 5.23 & 5.25 & 5.18 & 5.15  \\
 \hline
 16 & 5.23 & 5.19 & 5.16 & 5.14  \\
 \hline
 \multicolumn{5}{|c|}{Nemotron4-15B (FP32 PPL = 5.87)} \\ 
 \hline
 \hline
 64  & 6.3 & 6.20 & 6.13 & 6.08  \\
 \hline
 32  & 6.24 & 6.12 & 6.07 & 6.03  \\
 \hline
 16  & 6.12 & 6.14 & 6.04 & 6.02  \\
 \hline
 \multicolumn{5}{|c|}{Nemotron4-340B (FP32 PPL = 3.48)} \\ 
 \hline
 \hline
 64 & 3.67 & 3.62 & 3.60 & 3.59 \\
 \hline
 32 & 3.63 & 3.61 & 3.59 & 3.56 \\
 \hline
 16 & 3.61 & 3.58 & 3.57 & 3.55 \\
 \hline
\end{tabular}
\caption{\label{tab:ppl_llama7B_nemo15B} Wikitext-103 perplexity compared to FP32 baseline in Llama2-7B and Nemotron4-15B, 340B models}
\end{table}

%\subsection{Perplexity achieved by various LO-BCQ configurations on MMLU dataset}


\begin{table} \centering
\begin{tabular}{|c||c|c|c|c||c|c|c|c|} 
\hline
 $L_b \rightarrow$& \multicolumn{4}{c||}{8} & \multicolumn{4}{c||}{8}\\
 \hline
 \backslashbox{$L_A$\kern-1em}{\kern-1em$N_c$} & 2 & 4 & 8 & 16 & 2 & 4 & 8 & 16  \\
 %$N_c \rightarrow$ & 2 & 4 & 8 & 16 & 2 & 4 & 2 \\
 \hline
 \hline
 \multicolumn{5}{|c|}{Llama2-7B (FP32 Accuracy = 45.8\%)} & \multicolumn{4}{|c|}{Llama2-70B (FP32 Accuracy = 69.12\%)} \\ 
 \hline
 \hline
 64 & 43.9 & 43.4 & 43.9 & 44.9 & 68.07 & 68.27 & 68.17 & 68.75 \\
 \hline
 32 & 44.5 & 43.8 & 44.9 & 44.5 & 68.37 & 68.51 & 68.35 & 68.27  \\
 \hline
 16 & 43.9 & 42.7 & 44.9 & 45 & 68.12 & 68.77 & 68.31 & 68.59  \\
 \hline
 \hline
 \multicolumn{5}{|c|}{GPT3-22B (FP32 Accuracy = 38.75\%)} & \multicolumn{4}{|c|}{Nemotron4-15B (FP32 Accuracy = 64.3\%)} \\ 
 \hline
 \hline
 64 & 36.71 & 38.85 & 38.13 & 38.92 & 63.17 & 62.36 & 63.72 & 64.09 \\
 \hline
 32 & 37.95 & 38.69 & 39.45 & 38.34 & 64.05 & 62.30 & 63.8 & 64.33  \\
 \hline
 16 & 38.88 & 38.80 & 38.31 & 38.92 & 63.22 & 63.51 & 63.93 & 64.43  \\
 \hline
\end{tabular}
\caption{\label{tab:mmlu_abalation} Accuracy on MMLU dataset across GPT3-22B, Llama2-7B, 70B and Nemotron4-15B models.}
\end{table}


%\subsection{Perplexity achieved by various LO-BCQ configurations on LM evaluation harness}

\begin{table} \centering
\begin{tabular}{|c||c|c|c|c||c|c|c|c|} 
\hline
 $L_b \rightarrow$& \multicolumn{4}{c||}{8} & \multicolumn{4}{c||}{8}\\
 \hline
 \backslashbox{$L_A$\kern-1em}{\kern-1em$N_c$} & 2 & 4 & 8 & 16 & 2 & 4 & 8 & 16  \\
 %$N_c \rightarrow$ & 2 & 4 & 8 & 16 & 2 & 4 & 2 \\
 \hline
 \hline
 \multicolumn{5}{|c|}{Race (FP32 Accuracy = 37.51\%)} & \multicolumn{4}{|c|}{Boolq (FP32 Accuracy = 64.62\%)} \\ 
 \hline
 \hline
 64 & 36.94 & 37.13 & 36.27 & 37.13 & 63.73 & 62.26 & 63.49 & 63.36 \\
 \hline
 32 & 37.03 & 36.36 & 36.08 & 37.03 & 62.54 & 63.51 & 63.49 & 63.55  \\
 \hline
 16 & 37.03 & 37.03 & 36.46 & 37.03 & 61.1 & 63.79 & 63.58 & 63.33  \\
 \hline
 \hline
 \multicolumn{5}{|c|}{Winogrande (FP32 Accuracy = 58.01\%)} & \multicolumn{4}{|c|}{Piqa (FP32 Accuracy = 74.21\%)} \\ 
 \hline
 \hline
 64 & 58.17 & 57.22 & 57.85 & 58.33 & 73.01 & 73.07 & 73.07 & 72.80 \\
 \hline
 32 & 59.12 & 58.09 & 57.85 & 58.41 & 73.01 & 73.94 & 72.74 & 73.18  \\
 \hline
 16 & 57.93 & 58.88 & 57.93 & 58.56 & 73.94 & 72.80 & 73.01 & 73.94  \\
 \hline
\end{tabular}
\caption{\label{tab:mmlu_abalation} Accuracy on LM evaluation harness tasks on GPT3-1.3B model.}
\end{table}

\begin{table} \centering
\begin{tabular}{|c||c|c|c|c||c|c|c|c|} 
\hline
 $L_b \rightarrow$& \multicolumn{4}{c||}{8} & \multicolumn{4}{c||}{8}\\
 \hline
 \backslashbox{$L_A$\kern-1em}{\kern-1em$N_c$} & 2 & 4 & 8 & 16 & 2 & 4 & 8 & 16  \\
 %$N_c \rightarrow$ & 2 & 4 & 8 & 16 & 2 & 4 & 2 \\
 \hline
 \hline
 \multicolumn{5}{|c|}{Race (FP32 Accuracy = 41.34\%)} & \multicolumn{4}{|c|}{Boolq (FP32 Accuracy = 68.32\%)} \\ 
 \hline
 \hline
 64 & 40.48 & 40.10 & 39.43 & 39.90 & 69.20 & 68.41 & 69.45 & 68.56 \\
 \hline
 32 & 39.52 & 39.52 & 40.77 & 39.62 & 68.32 & 67.43 & 68.17 & 69.30  \\
 \hline
 16 & 39.81 & 39.71 & 39.90 & 40.38 & 68.10 & 66.33 & 69.51 & 69.42  \\
 \hline
 \hline
 \multicolumn{5}{|c|}{Winogrande (FP32 Accuracy = 67.88\%)} & \multicolumn{4}{|c|}{Piqa (FP32 Accuracy = 78.78\%)} \\ 
 \hline
 \hline
 64 & 66.85 & 66.61 & 67.72 & 67.88 & 77.31 & 77.42 & 77.75 & 77.64 \\
 \hline
 32 & 67.25 & 67.72 & 67.72 & 67.00 & 77.31 & 77.04 & 77.80 & 77.37  \\
 \hline
 16 & 68.11 & 68.90 & 67.88 & 67.48 & 77.37 & 78.13 & 78.13 & 77.69  \\
 \hline
\end{tabular}
\caption{\label{tab:mmlu_abalation} Accuracy on LM evaluation harness tasks on GPT3-8B model.}
\end{table}

\begin{table} \centering
\begin{tabular}{|c||c|c|c|c||c|c|c|c|} 
\hline
 $L_b \rightarrow$& \multicolumn{4}{c||}{8} & \multicolumn{4}{c||}{8}\\
 \hline
 \backslashbox{$L_A$\kern-1em}{\kern-1em$N_c$} & 2 & 4 & 8 & 16 & 2 & 4 & 8 & 16  \\
 %$N_c \rightarrow$ & 2 & 4 & 8 & 16 & 2 & 4 & 2 \\
 \hline
 \hline
 \multicolumn{5}{|c|}{Race (FP32 Accuracy = 40.67\%)} & \multicolumn{4}{|c|}{Boolq (FP32 Accuracy = 76.54\%)} \\ 
 \hline
 \hline
 64 & 40.48 & 40.10 & 39.43 & 39.90 & 75.41 & 75.11 & 77.09 & 75.66 \\
 \hline
 32 & 39.52 & 39.52 & 40.77 & 39.62 & 76.02 & 76.02 & 75.96 & 75.35  \\
 \hline
 16 & 39.81 & 39.71 & 39.90 & 40.38 & 75.05 & 73.82 & 75.72 & 76.09  \\
 \hline
 \hline
 \multicolumn{5}{|c|}{Winogrande (FP32 Accuracy = 70.64\%)} & \multicolumn{4}{|c|}{Piqa (FP32 Accuracy = 79.16\%)} \\ 
 \hline
 \hline
 64 & 69.14 & 70.17 & 70.17 & 70.56 & 78.24 & 79.00 & 78.62 & 78.73 \\
 \hline
 32 & 70.96 & 69.69 & 71.27 & 69.30 & 78.56 & 79.49 & 79.16 & 78.89  \\
 \hline
 16 & 71.03 & 69.53 & 69.69 & 70.40 & 78.13 & 79.16 & 79.00 & 79.00  \\
 \hline
\end{tabular}
\caption{\label{tab:mmlu_abalation} Accuracy on LM evaluation harness tasks on GPT3-22B model.}
\end{table}

\begin{table} \centering
\begin{tabular}{|c||c|c|c|c||c|c|c|c|} 
\hline
 $L_b \rightarrow$& \multicolumn{4}{c||}{8} & \multicolumn{4}{c||}{8}\\
 \hline
 \backslashbox{$L_A$\kern-1em}{\kern-1em$N_c$} & 2 & 4 & 8 & 16 & 2 & 4 & 8 & 16  \\
 %$N_c \rightarrow$ & 2 & 4 & 8 & 16 & 2 & 4 & 2 \\
 \hline
 \hline
 \multicolumn{5}{|c|}{Race (FP32 Accuracy = 44.4\%)} & \multicolumn{4}{|c|}{Boolq (FP32 Accuracy = 79.29\%)} \\ 
 \hline
 \hline
 64 & 42.49 & 42.51 & 42.58 & 43.45 & 77.58 & 77.37 & 77.43 & 78.1 \\
 \hline
 32 & 43.35 & 42.49 & 43.64 & 43.73 & 77.86 & 75.32 & 77.28 & 77.86  \\
 \hline
 16 & 44.21 & 44.21 & 43.64 & 42.97 & 78.65 & 77 & 76.94 & 77.98  \\
 \hline
 \hline
 \multicolumn{5}{|c|}{Winogrande (FP32 Accuracy = 69.38\%)} & \multicolumn{4}{|c|}{Piqa (FP32 Accuracy = 78.07\%)} \\ 
 \hline
 \hline
 64 & 68.9 & 68.43 & 69.77 & 68.19 & 77.09 & 76.82 & 77.09 & 77.86 \\
 \hline
 32 & 69.38 & 68.51 & 68.82 & 68.90 & 78.07 & 76.71 & 78.07 & 77.86  \\
 \hline
 16 & 69.53 & 67.09 & 69.38 & 68.90 & 77.37 & 77.8 & 77.91 & 77.69  \\
 \hline
\end{tabular}
\caption{\label{tab:mmlu_abalation} Accuracy on LM evaluation harness tasks on Llama2-7B model.}
\end{table}

\begin{table} \centering
\begin{tabular}{|c||c|c|c|c||c|c|c|c|} 
\hline
 $L_b \rightarrow$& \multicolumn{4}{c||}{8} & \multicolumn{4}{c||}{8}\\
 \hline
 \backslashbox{$L_A$\kern-1em}{\kern-1em$N_c$} & 2 & 4 & 8 & 16 & 2 & 4 & 8 & 16  \\
 %$N_c \rightarrow$ & 2 & 4 & 8 & 16 & 2 & 4 & 2 \\
 \hline
 \hline
 \multicolumn{5}{|c|}{Race (FP32 Accuracy = 48.8\%)} & \multicolumn{4}{|c|}{Boolq (FP32 Accuracy = 85.23\%)} \\ 
 \hline
 \hline
 64 & 49.00 & 49.00 & 49.28 & 48.71 & 82.82 & 84.28 & 84.03 & 84.25 \\
 \hline
 32 & 49.57 & 48.52 & 48.33 & 49.28 & 83.85 & 84.46 & 84.31 & 84.93  \\
 \hline
 16 & 49.85 & 49.09 & 49.28 & 48.99 & 85.11 & 84.46 & 84.61 & 83.94  \\
 \hline
 \hline
 \multicolumn{5}{|c|}{Winogrande (FP32 Accuracy = 79.95\%)} & \multicolumn{4}{|c|}{Piqa (FP32 Accuracy = 81.56\%)} \\ 
 \hline
 \hline
 64 & 78.77 & 78.45 & 78.37 & 79.16 & 81.45 & 80.69 & 81.45 & 81.5 \\
 \hline
 32 & 78.45 & 79.01 & 78.69 & 80.66 & 81.56 & 80.58 & 81.18 & 81.34  \\
 \hline
 16 & 79.95 & 79.56 & 79.79 & 79.72 & 81.28 & 81.66 & 81.28 & 80.96  \\
 \hline
\end{tabular}
\caption{\label{tab:mmlu_abalation} Accuracy on LM evaluation harness tasks on Llama2-70B model.}
\end{table}

%\section{MSE Studies}
%\textcolor{red}{TODO}


\subsection{Number Formats and Quantization Method}
\label{subsec:numFormats_quantMethod}
\subsubsection{Integer Format}
An $n$-bit signed integer (INT) is typically represented with a 2s-complement format \citep{yao2022zeroquant,xiao2023smoothquant,dai2021vsq}, where the most significant bit denotes the sign.

\subsubsection{Floating Point Format}
An $n$-bit signed floating point (FP) number $x$ comprises of a 1-bit sign ($x_{\mathrm{sign}}$), $B_m$-bit mantissa ($x_{\mathrm{mant}}$) and $B_e$-bit exponent ($x_{\mathrm{exp}}$) such that $B_m+B_e=n-1$. The associated constant exponent bias ($E_{\mathrm{bias}}$) is computed as $(2^{{B_e}-1}-1)$. We denote this format as $E_{B_e}M_{B_m}$.  

\subsubsection{Quantization Scheme}
\label{subsec:quant_method}
A quantization scheme dictates how a given unquantized tensor is converted to its quantized representation. We consider FP formats for the purpose of illustration. Given an unquantized tensor $\bm{X}$ and an FP format $E_{B_e}M_{B_m}$, we first, we compute the quantization scale factor $s_X$ that maps the maximum absolute value of $\bm{X}$ to the maximum quantization level of the $E_{B_e}M_{B_m}$ format as follows:
\begin{align}
\label{eq:sf}
    s_X = \frac{\mathrm{max}(|\bm{X}|)}{\mathrm{max}(E_{B_e}M_{B_m})}
\end{align}
In the above equation, $|\cdot|$ denotes the absolute value function.

Next, we scale $\bm{X}$ by $s_X$ and quantize it to $\hat{\bm{X}}$ by rounding it to the nearest quantization level of $E_{B_e}M_{B_m}$ as:

\begin{align}
\label{eq:tensor_quant}
    \hat{\bm{X}} = \text{round-to-nearest}\left(\frac{\bm{X}}{s_X}, E_{B_e}M_{B_m}\right)
\end{align}

We perform dynamic max-scaled quantization \citep{wu2020integer}, where the scale factor $s$ for activations is dynamically computed during runtime.

\subsection{Vector Scaled Quantization}
\begin{wrapfigure}{r}{0.35\linewidth}
  \centering
  \includegraphics[width=\linewidth]{sections/figures/vsquant.jpg}
  \caption{\small Vectorwise decomposition for per-vector scaled quantization (VSQ \citep{dai2021vsq}).}
  \label{fig:vsquant}
\end{wrapfigure}
During VSQ \citep{dai2021vsq}, the operand tensors are decomposed into 1D vectors in a hardware friendly manner as shown in Figure \ref{fig:vsquant}. Since the decomposed tensors are used as operands in matrix multiplications during inference, it is beneficial to perform this decomposition along the reduction dimension of the multiplication. The vectorwise quantization is performed similar to tensorwise quantization described in Equations \ref{eq:sf} and \ref{eq:tensor_quant}, where a scale factor $s_v$ is required for each vector $\bm{v}$ that maps the maximum absolute value of that vector to the maximum quantization level. While smaller vector lengths can lead to larger accuracy gains, the associated memory and computational overheads due to the per-vector scale factors increases. To alleviate these overheads, VSQ \citep{dai2021vsq} proposed a second level quantization of the per-vector scale factors to unsigned integers, while MX \citep{rouhani2023shared} quantizes them to integer powers of 2 (denoted as $2^{INT}$).

\subsubsection{MX Format}
The MX format proposed in \citep{rouhani2023microscaling} introduces the concept of sub-block shifting. For every two scalar elements of $b$-bits each, there is a shared exponent bit. The value of this exponent bit is determined through an empirical analysis that targets minimizing quantization MSE. We note that the FP format $E_{1}M_{b}$ is strictly better than MX from an accuracy perspective since it allocates a dedicated exponent bit to each scalar as opposed to sharing it across two scalars. Therefore, we conservatively bound the accuracy of a $b+2$-bit signed MX format with that of a $E_{1}M_{b}$ format in our comparisons. For instance, we use E1M2 format as a proxy for MX4.

\begin{figure}
    \centering
    \includegraphics[width=1\linewidth]{sections//figures/BlockFormats.pdf}
    \caption{\small Comparing LO-BCQ to MX format.}
    \label{fig:block_formats}
\end{figure}

Figure \ref{fig:block_formats} compares our $4$-bit LO-BCQ block format to MX \citep{rouhani2023microscaling}. As shown, both LO-BCQ and MX decompose a given operand tensor into block arrays and each block array into blocks. Similar to MX, we find that per-block quantization ($L_b < L_A$) leads to better accuracy due to increased flexibility. While MX achieves this through per-block $1$-bit micro-scales, we associate a dedicated codebook to each block through a per-block codebook selector. Further, MX quantizes the per-block array scale-factor to E8M0 format without per-tensor scaling. In contrast during LO-BCQ, we find that per-tensor scaling combined with quantization of per-block array scale-factor to E4M3 format results in superior inference accuracy across models. 

%\input{trunk/rand_worst}
%Logging:


UserID
ScenarioID
controlMode
requestID (Nummer der Request)
elapsed Time
distanceTravelledSinceLastLog
distanceToEndOfInstructedPath (Luftlinie zum Ende des gefolgten Pfades)
lengthOfCurrentInstructedPath (Gesamtlänge noch zu folgendem gezeichneter Pfad + generierte Verbindungsstrecke Auto <-> Pfad)
lengthOfCurrentInstructedInputPath (Gesamtlänge noch zu folgendem gezeichneter Pfad)
distanceToEnd (Auto <-> Ende bis wo hin operiert werden muss)
vehiclePosition (Verlauf der unity positions vektoren -> nun auch mit offset, also vergleichbar, wenn man damit was anfangen will)
vehicleSpeed (in kmh)
constructionSiteEntered (ob erstmalig unmittelbar vor der ConstructionSite angekommen)
endReached ("Request geschafft")
closest Lane (nummer lane 0-indiziert von links bis rechts)
currentLaneDeviation (Spurabweichung zur closest Lane Mitte  auch in metern)
timeOfCollisionAvoidanceTraffic (Zeit summe wenn Verkehr in der CollisionAvoidance Range ist (egal ob hinten oder vorne, noch ob es das auto stört))
timeOfCollisionAvoidanceObstacle (wie oben für Obstacles <- Baustellenfahrzeuge, Baustellenmarker, Metallgrenzen links und rechts von Straße)
timeOfCollisionAvoidancePedestrian (wie oben für Fußgänger)
amountOfAdditionInput (Summe wie oft der Pfad erweitert wurde +1 pro Aktion)
amountOfAdditionMarkers (Summe um wie viele "Stellen" der Pfad erweitert wurde +x pro Aktion <- relevant für Anteil an Snap to Mid in Trajectory, für PathPlanning und Waypoint immer +1)
amountOfSnapToMiddleInput (+1 wenn Addition Input und mindestens einmal Snap To Mid verwendet wurde, für path planning immer 1, weil im Grunde immer Mittig)
amountOfSnapToMiddleMarkers (wie amount Of AdditionMarkers, nur dass nur hochgezählt wird wenn die "Stelle" gesnappt wurde)
amountOfReadjustmentInput (wie obiges nur statt Addition: Readjustment, heißt: anfang und ende sind in alter Trajektorie oder nah parallel dazu, Waypoint wird verschoben, PathPlanning backwards, immer +1)
timeSinceLastInput (Zeit seit letztem input)
currentlyNeglectedTime (aktuelle Zeit wie lang vehikel schon still steht (colision avoidance) oder wenn Ende vom angegebenen Pfad)
blindTimeSum (wenn vehikel nicht neglected und schon eingaben gemacht wurden -> aufsummieren Zeit wenn nicht in Main oder secondary view)
isMainRequest
isSecondaryRequest
totalRequestAmount (insgesamte Anzahl an Requests in dem Szenario, hat nicht direkt was mit dieser request instanz zu tun, aber generelle info)
sideOfConstructionSite (ob construction site links oder rechts bei der Straße




Für ein Szenario/ case jeweils::
OverviewLog:
UserID
ScenarioID
controlMode
elapsedTime
activeRequests (wie viele grade angezeigt werden)
succeededRequests (wie viele bisher geschafft sind)
mousePositionX
mousePositionY
usingMultiView (also wenn zwei Requests gleichzeitig angezeigt = true sonst = false)
usingSingleViewMain (auch nur = true wenn nur ein vehikel in der Hauptanzeige ist)
EyeGazeArea (das was du noch meintest -> was aktuell angeschaut wird: string wert aus: RequestList, MainView, SecondaryView, Instructions, None)
mouseClickLeft (summe)
mouseClickRight (summe)
pressLeftCtrl (summe)
pressShift (summe)
TimestampLog:
UserID
ScenarioID
controlMode
elapsed time
timeStampEvent (String wert aus RequestStarted, RequestFinished, RequestOpenedMain, RequestOpenedSecondary, RequestRemovedMain, RequestRemovedSecondary)
additionalInfo (eigentlich immer nur die Request Nummer des vehikels)
-> Es kommen immer Logs rein sobald eines der genannten Events auftritt. Bei den Request Opened/removed events muss man aufpassen, da jede variablen änderung geloggt wird. Demnach sind für den selben elapsedTimeTimestamp nur die Anfangs und Endzustände der jeweiligen Slots relevant.



UnitEyeLog (-> Standard Implementation) + 
 4 areas of interest definierst: links das panel mit den requests, das video mit der aktuellen Szene (ego view),  die szene top down view, und noch das Panel mit den Gründen unten
ist im OverviewLog
%\nocite{*}
\bibliographystyle{plain}

%\clearpage
%\newpage
\bibliography{pool}

\clearpage
\newpage
\appendix

\section{Proofs from Section~\ref{sec:gammaok}} \label{app:gamma}

\subsection{On the girth of locally \texorpdfstring{$\gamma$}{gamma}-sparse graphs}
\begin{lemma}\label{lemma:girth_rev}
    Let $G = (V,E)$ be an undirected graph with girth $g(G)$.
    Then $G$ is \ok{0} if and only if $g(G) \geq 5$.
\end{lemma}
\begin{proof}
    We first prove that if $G$ is \ok{0} then $g(G)$ must be at least $5$.
    In order to prove that, we simply negate the statement and prove that if $G$ has girth $<5$ then $G$ can not be \ok{0}.
    Without loss of generality, assume that $g(G) = 4$ (the case $g(G) = 3$ is similar).
    Then there must exist a cycle $C = (u_1, u_2, u_3, u_4)$ of $4$ vertices.
    It is simple to see that $u_2,u_4 \in \lset_1(u_1)$ and $u_3 \in \lset_2(u_1)$.
    Since $u_3$ is a neighbor of both $u_2$ and $u_4$, the degree of $u_3$ in the subgraph $G\left[\lset_1(u_1) \cup \{u_4\} \right]$ is at least $2$, hence $G$ is not \ok{0} (see \Cref{subfig:girth1}).
    
    We now prove that if $g(G) \geq 5$ then $G$ must be \ok{0}.
    Again, we negate this statement and prove that if $G$ is not \ok{0} then the girth of $G$ must be less then $5$.
    Let us assume that $G$ is \gammaok, for any $\gamma > 0$, thus it is not \ok{0}.
    Since $G$ is not \ok{0} there exists a vertex $v \in V$ such that at least one of the following properties holds (see \Cref{subfig:girth2}):
    \begin{enumerate}
        \item $\exists u \in \lset_1(v)$ such that the degree of $u$ in $G\left[ \lset_1(v) \right]$ is greater then $0$, or;
        \item $\exists w \in \lset_2(v)$ such that the degree of $w$ in $G\left[ \lset_1(v) \cup \{ w \} \right]$ is greater then $1$.
    \end{enumerate}
    In the first case, we have a cycle of $3$ vertices, then $g(G) = 3$.
    In the second case, we have a cycle of $4$ vertices, then $g(G) = 4$.
    In both cases $g(G) < 5$.
\end{proof}
\begin{figure}[h]
    \centering
    \begin{subfigure}[b]{0.35\linewidth}
            \centering
            \includegraphics[width=\linewidth]{img/girth-1.pdf}
            \caption{}
            \label{subfig:girth1}
    \end{subfigure}
    \begin{subfigure}[b]{0.6\linewidth}
            \centering
            \includegraphics[width=\linewidth]{img/girth-2.pdf}
            \caption{}
            \label{subfig:girth2}
    \end{subfigure}%
    \caption{}
    \label{fig:example_girth}
\end{figure}

\subsection{Deterministic lazy-update on \texorpdfstring{$\gamma$}{gamma}-sparse graphs}\label{apx:gamma-ok-deterministic}

\begin{theorem}\label{lemma:gamma-ok-error-bound-balls}
    
Let $\varepsilon \in (0,1)$, and let $G^{(0)}$ be an initial graph. Consider any sequence of edge insertions that yields a final graph $G$. If $G$ is \gammaok, \lazyscheme$(\varphi = \frac{\varepsilon}{1 - \varepsilon},k=0)$ has an approximation ratio of  $\frac{\gamma + 1}{1-\varepsilon}$ and amortized update cost $O(1/\varepsilon)$. 
    
\end{theorem}
\begin{proof}
Recall that $\bd_u$ denotes the black degree of $u$, and that  \Cref{alg:det_thresh} guarantees that $\deg_u$ is at most $(1+\varphi)\bd_u$.
    Then, it is simple to give an upper bound to the size of $\ball_2(u)$, that is $\vert \ball_2(u) \vert \leq 1+ \sum_{v \in \lset_1(u)} (1 + \varphi)\bd_v$.Consider a vertex $v \in \lset_1(u)$. Since $G$ is \gammaok, the number of neighbors of $v$ belonging to $\lset_2(u)$ is at lest $\deg_v - (\gamma+1)$ of which $\bd_v - (\gamma+1)$ must belong to $\apxball_2(u)$. Moreover, a vertex in $\lset_2(u)$ has at most $\gamma+1$ neighbors in $\lset_1(u)$. Therefore: 
    \begin{align*}
    \vert \apxball_2(u) \vert
    &\geq  \bd_u + 1 + \frac{1}{\gamma + 1}\sum_{v \in \lset_1(u)}(\bd_v - (\gamma + 1))\\
    &= \bd_u + 1 + \frac{1}{\gamma + 1}\sum_{v \in \lset_1(u)}\bd_v - \underbrace{\frac{1}{\gamma + 1}\sum_{v \in \lset_1(u)}(\gamma + 1)}_{= \bd_u}\\
    &= 1+ \frac{1}{\gamma + 1}\sum_{v \in \lset_1(u)}\bd_v.
    \end{align*}
  
    As a consequence, $\vert \apxball_2(u) \vert/\vert \ball_2(u) \vert \ge \frac{1}{(1+\varphi)(\gamma+1)}$. By setting $\varphi = \frac{\varepsilon}{1 - \varepsilon}$, and by using \Cref{lm:amortized_det_alg},  the claim follows.
\end{proof}

\subsection{Proof of \Cref{le:gamma_ok_expect_lowerbound}}\label{apx:proof_gamma_ok_expect_lowerbound}
\begin{proof}
Let $e_1, \dots, e_{\ell_v}$ be the \emph{red edges} between $v$ and $\lset_2(u)$, and define the binary random variable $\lrdr_v(i)$ that is equal to $1$ if $e_i$ is a \emph{quasi-black edge} for $u$, $0$ otherwise, for $i = 1, \dots, \lrd_v$. Thus we can express $\lrdr_v = \sum_{i=1}^{\lrd_v} \lrdr_v(i)$, with expectation

\begin{equation}\label{eq:gamma_ok_lb_fact_eq_1}
\begin{aligned}
  \Expec{}{\lrdr_v} & = \sum_{i=1}^{\lrd_v}{\Prob{}{\lrdr_v(i)=1}} = \lrd_v - \sum_{i=1}^{\lrd_v} {\Prob{}{\lrdr_v(i)=0}}.
\end{aligned}
\end{equation}

Without loss of generality, assume that the edges $e_1, \dots, e_{\lrd_v}$ have been inserted at times $t_1 < \dots < t_{\lrd_v}$, respectively.
If $e_i$ is not a quasi-black edge for $u$, then it must be that $u$ is not selected by $v$ at \Cref{line:random_selection} of \Cref{alg:det_thresh}, at times $t_i, t_{i+1},\dots, t_{\lrd_v}$.
This holds with probability 
\begin{equation}\label{eq:gamma_ok_lb_fact_eq_2}
\begin{aligned} 
    &\Prob{}{\lrdr_v(i) = 0}
    \leq \prod_{j=i}^{\lrd_v} \left( 1-\frac{k}{\deg_v^{(t_j)}} \right)
    \leq \prod_{j=i}^{\lrd_v} \left( 1 - \frac{k}{\deg_{v}^{(t_{\lrd_v})}} \right) \\
    &\leq \left( 1-\frac{k}{\lbdd_v + \lrd_v + \gamma + 1}\right)^{\lrd_v - i + 1} 
    \leq \left(1-\frac{k}{2(\lbdd_v + \gamma + 1)}\right)^{\lrd_v - i}.
\end{aligned}
\end{equation}
The third inequality holds since the edges incident to $v$ having endpoints in $L_1(u)$ are at most $\gamma$, while those having endpoints in $L_2(u)$ are exactly $\lbdd_v+ \lrd_v$. Moreover, the last inequality holds because $\lrd_v \leq \rd_v \leq \bd_v \leq \lbdd_v + \gamma + 1$, given the assumption $\varphi = 1$.

By plugging in \eqref{eq:gamma_ok_lb_fact_eq_2} into   \eqref{eq:gamma_ok_lb_fact_eq_1} and we obtain
\begin{align*}
    &\Expec{}{\lrdr_v} \geq \lrd_v - \sum_{i=1}^{\lrd_v}\left( 1-\frac{k}{2(\lbdd_v + \gamma + 1)}\right)^{\lrd_v - i} \\
    &= \lrd_v - \sum_{i=0}^{\lrd_v-1} \left(1-\frac{k}{2(\lbdd_v + \gamma + 1)}\right)^i 
    \leq \lrd_v - \frac{1-\left(1-\frac{k}{2(\lbdd_v+\gamma+1)}\right)^{\lrd_v}}{1-\left(1-\frac{k}{2(\lbdd_v + \gamma + 1)}\right)} \\
    &\geq \lrd_v - \frac{1}{1-\left(1-\frac{k}{2(\lbdd_v + \gamma + 1)}\right)}
    \geq \lrd_v - \frac{2(\lbdd_v + \gamma + 1)}{k}.
\end{align*}
\end{proof}
\section{Appendix B: Scams} \label{scams}
Scams were, unfortunately, shared experiences that resonated with workers of all platforms. Although ``true'' scams occur rarely on Rover, Petsitter-4 described how they usually take the form of a ``classic check scheme'' where the scammer claims ``they're going to send you a check for \$500 and tell you to buy something and send back whatever is extra''. Manipulations of hours or number of pets involved are more commonplace, where clients would change hours to ``get charged less for a boarding or a house set, and they can manipulate the number of animals \dots the cost comes out to us [as sitters]'' (Petsitter-4). On Uber, Driver-7 described getting phone calls from fake numbers claiming to be Uber support who tries to assign him `` `a ride to a very important person. So we need to confirm your identity' ''. The scammer would then proceed to ask for their phone number to which send a 4-digit code, which they'll then use to access the drivers' account. Meanwhile, Freelancer-1 related how she enjoyed reading about others' \textit{Stories} of ``scams \dots cause there's quite a few of them on Upwork''.



\end{document}
