\section{Adversarial edge  sequences}\label{sec:gammaok}



We now study our lazy-update algorithm against an adversarial input framework. In \Cref{ssec:lowerbound}, we  show that when the adversary can \textit{both}: i) choose a worst-case  graph $G$ \textit{and}  ii)   submit $G$ according to an \textit{adaptive} sequence of edge insertions, then a strong lower bound on the achievable update-time/approximation trade-offs can be proved. Such a lower bound holds for the whole parameterized scheme $\lazyscheme (\varphi,k)$.

Then, in \Cref{ssec:gammaok}, we provide  a \textit{necessary} condition to get the adversarial, worst-case framework above: the \textit{girth} \cite{} of  $G$ must be smaller than 5: more precisely,  $G$ must contain an unbounded number of triangles and cycles of lenght 4.   We in fact prove that a suitable parameter setting of the randomized algorithm $\lazyscheme (\varphi,k)$ can obtain almost-optimal trade-offs even on adversarial edge insertion sequences for every graph that have a bounded number of such small cycles (see \Cref{def:gammaok} for a formal definition of this class of graphs).

\subsection{A lower bound for the lazy-update approach on adversarial edge-insertion sequences} \label{ssec:lowerbound}
The lower bound for the adversarial framework described above is formalized in the following 


\begin{theorem}\label{thm:lower}
    For every $\varphi \in [0,1]$, and integer $k \geq 0$, if  \lazyscheme$(\varphi,k)$ has approximation ratio $\rho \ge 1$, then it must have an amortized update time per edge insertion of $\Omega(\Delta/\rho^3)$, where $\Delta$ is the maximum degree of the graph.
\end{theorem}\label{le:lb1}

\begin{figure}[ht]
    \centering
    \includegraphics[width=0.66\linewidth]{img/lower-bound.pdf}
    \caption{Black edges are present at $t = 0$, while red ones are inserted in the interval $\{1, 2,\ldots , \Delta \rho^2\}$. At time $t > 0$, an edge with one endpoint in $u_{1t\mod\Delta}$ and the other in a distinct 0-degree vertex in $S_2$ is added.}
    \label{fig:lb1}
\end{figure}

\begin{proof}
Fix $\rho \ge 1$, and assume that \lazyscheme$(\varphi,k)$ has an approximation ratio of at most $\rho$. We will show that there exist an initial graph $G^{(0)}$ with degree $\Delta$ and a sequence of edge insertions against which \lazyscheme$(\varphi,k)$ must incur an amortized update time of $\Omega(\Delta/\rho^3)$.

Notice that for $k>0$ the algorithm is randomized. In order to manage this, we will prove the lower bound for every possible realization of the randomness used by the algorithm. Therefore, fix the value of the random bits used by \lazyscheme$(\varphi,k)$ arbitrarily. Hence, from now on we can then assume that the behavior of the algorithm is completely deterministic. 

The initial graph $G^{(0)}$ has $2 \Delta$ vertices forming a complete bipartite graph with sides $S_0=\{u_{01},\dots,u_{0\Delta}\}$ and $S_1=\{u_{11},\dots,u_{1\Delta}\}$, plus a set $S_2$ of $\Delta \rho^2$ vertices of degree 0 (see \Cref{fig:lb1}). The sequence of edge insertion is defined as follows. We insert $\rho^2$ new edges incident to each vertex in $S_1$. Each of these $\Delta \rho^2$ new edges has an endpoint in $S_1$ and the other is a previously $0$-degree vertex in $S_2$. These edges are inserted in a round-robin fashion: we first insert an edge incident to $u_{11}$, then one incident to $u_{12}$ up to one incident to $u_{1\Delta}$, and next another incident to $u_{11}$ and so on. We can view the sequence of insertions as organized in $\rho^2$ rounds of $\Delta$ steps each.
In the $i$-th step of the $j$-th round one edge is added from $u_{1i}$ to a different vertex belonging to $S_2$. 

Consider the time instant just after all the edge insertions. 
Since we assumed that the algorithm is a $\rho$-approximation, it holds that for every vertex $u \in S_0$, $|\apxball_2(u)| \ge \frac{1}{\rho} |\ball_2(u)|=\frac{1}{\rho} (2\Delta+\Delta \rho^2) = \Delta \rho + 2\Delta/\rho$. This implies that after all the edge insertions $u$ must be aware of at least $\Delta \rho + 2\Delta/\rho - 2\Delta=\Delta\rho -2 \Delta(1-1/\rho)$ vertices from $S_2$. 

We say that there is a \emph{message} from $v$ to $u$ if vertex $v$ performs a union operation of the form 
\[
\apxball_2(u) \gets \apxball_2(u) \cup \apxball_1(v).  
\]

Since at any time every vertex $v \in S_2$ is adjacent to at most $\rho^2$ vertices in $S_2$, it must be that each $u \in S_0$ must receive at least 
\[
\frac{\Delta\rho -2 \Delta(1-1/\rho)}{\rho^2}=\Omega(\Delta/\rho)
\]

\noindent messages from vertices in $S_1$. As a consequence, the total number of messages are at least $\Omega(\Delta^2/\rho)$. As the number of insertions is $\Delta \rho^2$, the amortized update cost per insertion is at least 
\[
\frac{\Omega(\Delta^2/\rho)}{\Delta \rho^2}=\Omega(\Delta/\rho^3).
\]
\end{proof}

We have special cases as corollaries. We need amortized update time $\Omega(\Delta)$ if we want $\rho = O(1)$, $\Omega(\sqrt[4]{\Delta})$ if we want $\rho = O(\sqrt[4]{\Delta})$ and so on. 

\begin{remark}
    The  lower bound in \Cref{le:lb1}   in fact holds for a wider class of algorithms. Informally speaking, this class includes any \textit{local} algorithm that limits its online update ``actions''  on the 2-hop neighbors   of $u$ and $v$ only. Making this claim more formal requires to cope with several technical issues which are quite far from the main (algorithmic) goals of this work and it is thus omitted.
    \end{remark}




%\subsection{Adversarial edge sequences of large girth}
\subsection{\texorpdfstring{\gammaok} \ graphs} \label{ssec:gammaok}
\subsection{\texorpdfstring{\gammaok} \ graphs} \label{ssec:gammaok}
In this section we introduce a special class of graphs (namely, \gammaok\ graphs) and we analyze the performances of \Cref{alg:det_thresh} as function of the parameter $\gamma$. In particular, we show that for constant values of $\gamma$ it is possible to obtain a $(1-\varepsilon)$-covering with update insertion time of $O(\frac{1}{\varepsilon})$.

Given a graph $G(V,E)$ and a subset $V' \subseteq V$, we denote by $G[V']$ the subgraph induced by $V'$. Informally speaking, a graph is \gammaok\ if every node in $\ball_2(u) \setminus \{u\}$ has roughly at most $\gamma$ neighbors in $\lset_1(u)$. More formally:

\begin{definition}[\gammaok\ graphs] \label{def:gammaok}
    A graph $G(V,E)$ is said \gammaok\ if for each vertex $u \in V$ we have:
    (i)   $\forall v \in \lset_1(u)$ the degree of $v$ in $G[\lset_1(u)]$ is at most $\gamma$, and (ii) 
          $\forall w \in \lset_2(u)$ the degree of $w$ in $G[\lset_1(v) \cup \{ w \}]$ is at most $\gamma+1$.
\end{definition}

%In the following, we make an abuse of notation and say that a dynamic graph $G$ is \gammaok if any $G^{(t)}$ is at most \gammaok, for any $t$.

Observe that the class of \gammaok\ graphs becomes larger as $\gamma$ increases. For $\gamma=n-1$ the class contains all possible graphs, while the most restricted class is obtained for $\gamma=0$. One might wonder whether the \gammaok\ graphs are always sparse graphs. It turns out that for $\gamma=0$ the class coincides with the well-known class of all the graphs with \emph{girth} at least $5$ whose can have up to $\Theta(n^{\frac{3}{2}})$ edges assuming Erdős' Girth Conjecture \cite{??}. The proof of such equivalence is given in Appendix-??.

\rem{Spostare in appendice il seguente lemma.}

\begin{lemma}\label{lemma:girth_rev}
    Let $G = (V,E)$ be an unweighted graph with girth $g(G)$.
    Then $G$ is \ok{0} if and only if $g(G) \geq 5$.
\end{lemma}
\begin{proof}
    ($\Rightarrow$) We first prove that if $G$ is \ok{0} then $g(G)$ must be at least $5$.
    In order to prove that, we simply negate the statement and prove that if $G$ has girth $<5$ then $G$ can not be \ok{0}.
    Without loss of generality, assume that $g(G) = 4$ (the case $g(G) = 3$ is similar).
    Then there must exist a cycle $C = (u_1, u_2, u_3, u_4)$ of $4$ vertices.
    It is simple to see that $u_2,u_4 \in \lset_1(u_1)$ and $u_3 \in \lset_2(u_1)$.
    Since $u_3$ is a neighbour of both $u_2$ and $u_4$, the degree of $u_3$ in the subgraph $G\left[\lset_1(u_1) \cup \{u_4\} \right]$ is at least $2$, hence $G$ is not \ok{0} (see \Cref{fig:girth-lemma}).

    \noindent ($\Leftarrow$) We now prove that if $g(G) \geq 5$ then $G$ must be \ok{0}.
    Again, we negate this statement and prove that if $G$ is not \ok{0} then the girth of $G$ must be less then $5$.
    So, let assume that $G$ is \gammaok, for any $\gamma > 0$, but it is not \ok{0}.
    Since $G$ is not \ok{0} there exists a vertex $v \in V$ such that at least one of the following properties holds:
    \begin{enumerate}
         \item $\exists u \in \lset_1(v)$ such that the degree of $u$ in $G\left[ \lset_1(v) \right]$ is $> 0$;
        \item $\exists w \in \lset_2(v)$ such that the degree of $w$ in $G\left[ \lset_1(v) \cup \{ w \} \right]$ is $> 1$.
    \end{enumerate}
    In the first case, we have a cycle of $3$ vertices, then $g(G) = 3$.
    In the second case, we have a cycle of $4$ vertices, then $g(G) = 4$.
    In both cases $g(G) < 5$.
\end{proof}

As a first result, we prove the following:

\begin{theorem}\label{lemma:gamma-ok-error-bound-balls}
    
Let $\varepsilon \in (0,1)$, and let $G^{(0)}$ be an initial graph. Consider any sequence of edge insertions that yields a final graph $G$. If $G$ is \gammaok, \lazyscheme$(\varphi = \frac{\varepsilon}{1 - \varepsilon},k=0)$ has an approximation ratio of  $\frac{\gamma + 1}{1-\varepsilon}$ and amortized update cost $O(1/\varepsilon)$. 
    
\end{theorem}
\begin{proof}
Recall that $\bd_u$ denotes the black degree of $u$, and that  \Cref{alg:det_thresh} guarantees that $\deg_u$ is at most $(1+\varphi)\bd_u$.
    Then, it is simple to give an upper bound to the size of $\ball_2(u)$, that is $\vert \ball_2(u) \vert \leq 1+ \sum_{v \in \lset_1(u)} (1 + \varphi)\bd_v$.Consider a vertex $v \in \lset_1(u)$. Since $G$ is \gammaok, the number of neighbors of $v$ belonging to $\lset_2(u)$ is at lest $\deg_v - (\gamma+1)$ of which $\bd_v - (\gamma+1)$ must belong to $\apxball_2(u)$. Moreover, a vertex in $\lset_2(u)$ has at most $\gamma+1$ neighbors in $\lset_1(u)$. Therefore: 
    \begin{align*}
    \vert \apxball_2(u) \vert
    &\geq  \bd_u + 1 + \frac{1}{\gamma + 1}\sum_{v \in \lset_1(u)}(\bd_v - (\gamma + 1))\\
    &= \bd_u + 1 + \frac{1}{\gamma + 1}\sum_{v \in \lset_1(u)}\bd_v - \underbrace{\frac{1}{\gamma + 1}\sum_{v \in \lset_1(u)}(\gamma + 1)}_{= \bd_u}\\
    &= 1+ \frac{1}{\gamma + 1}\sum_{v \in \lset_1(u)}\bd_v.
    \end{align*}
  
    As a consequence, $\vert \apxball_2(u) \vert/\vert \ball_2(u) \vert \ge \frac{1}{(1+\varphi)(\gamma+1)}$. By setting $\varphi = \frac{\varepsilon}{1 - \varepsilon}$, and by using \Cref{lm:amortized_det_alg},  the claim follows.
\end{proof}

\iffalse
\begin{algorithm}
\DontPrintSemicolon
\SetAlgoLined
\SetKwFunction{FMain}{Prune}
\SetKwProg{Fn}{Procedure}{:}{end}
\Fn{\FMain{$G, u$}}{
    $E_p \gets E$\;
    \For{$w \in \lset_2(u)$}{
        %\let $y \in \{u,v\} \setminus \{x\}$\;
        %$\apxball_1(x) \gets \apxball_1(x) \cup \{y\}$\;
        %\tcp{heavy update}
        \If{$\exists v\in \lset_1(u)$ s.t. $(v,w) \in E_p$ is \emph{black}}{
            let $D =\{ (\overline{v},w) \in E_p: \overline{v} \in \lset_1(u)$ and $ (\overline{v},w)$ is red$\}$\;
            $E_p \gets E_p \setminus D$\;
        } \Else{
            let $D =\{ (v,w) \in E_p: v \in \lset_1(u)$ and $ (v,w)$ is red$\}$\;
            let $e\in D$\;
            $D \gets D \setminus \{e\}$\;
            $E_p \gets E_p \setminus D$\;
        }   
    }
    \Return{$G_p = (V,E_p)$}
}

\caption{Pruning}
\label{alg:prune_procedure}
\end{algorithm}
\fi

Next, we prove that by setting $k = \frac{4(\gamma + 1)}{\varepsilon}$ and $\varphi = 1$, \lazyscheme\ achieves an approximation ratio of $\frac{1}{1-\varepsilon}$ and amortized update cost of $O(\frac{\gamma+1}{\varepsilon})$ for \gammaok graphs. This results strongly relies on the random selection performed in Line 14 of \Cref{alg:det_thresh} at each edge insertion. In order to keep track of the progress that such instruction guarantees, we define the concept of  \emph{quasi-black} edge. Informally, a red edge $(v,w)$ of $v$ with $v \in \lset_1(u)$ and $\lset_2(u)$ is said to be quasi-black for $u$ if $u$ is selected in Line 14 of \Cref{alg:det_thresh} for the subsequent insertion of an edge $(v,w')$, and hence $w \in \apxball_2(u)$. More formally: \rem{forse la def formale si può togliere}

\begin{definition}
Let $u \in V$, and $v \in \lset_1(u)$. For $i=1,\dots,\rd_v$, let $e_i$ be the $i$-th red edge w.r.t. $v$ inserted in the graph. We say that $e_i$ is a \emph{quasi-black edge for $u$} if $u$ has been randomly selected at least once during the insertions of $e_i,\dots,e_{\rd_v}$ (\Cref{alg:det_thresh} lines 14-16). 
\end{definition}

In the remainder of this section, we use $\lrdr_v$ and $\lrd_v$ to denote the number of \emph{quasi-black} and \emph{red} edges, respectively, that connect $v$ to nodes in $\lset_2(u)$. Similarly, we use $\lbdd_v$ to represent the number of \emph{black} edges of $v$ having the other endpoint in $\lset_2(u)$. Notice that  $\lrdr_v$ is a random variable that counts how many vertices out of $\lrd_v$ are in $\apxball_2(u)$. We can prove the following technical lemma.
\begin{lemma}\label{le:gamma_ok_expect_lowerbound}
     $\Expec{}{\lrdr_v} \geq \lrd_v - \frac{2(\lbdd_v + \gamma + 1)}{k}$
\end{lemma}
\begin{proof}
    Let $\lrdr_v(i)$ be a binary random variable defined as
    \begin{equation*}
    \lrdr_v(i)=
    \begin{cases}
      1, & \text{if}\ e_i \ \text{is \emph{quasi-black} for} \ u \\
      0, & \text{otherwise}
    \end{cases}
  \end{equation*}
    for $i=1,\dots,\lrd_v$ where $e_i,\dots, e_{\lrd_v}$ are the \emph{red egdes} between $v$ and $\lset_2(u)$. Therefore, we can rewrite $\lrdr_v$ as a sum of random variables and compute the expected value as follows
  \begin{align}\label{eq:gamma_ok_lb_fact_eq_1}
      \nonumber \Expec{}{\lrdr_v} &= \sum_{i=1}^{\lrd_v}{\Expec{}{\lrdr_v(i)}}  \notag \\
      \nonumber &= \sum_{i=1}^{\lrd_v}{\Prob{}{\lrdr_v(i)=1}} \notag \\
      %&= \sum_{i=1}^{\lrd_v}{1-\Prob{}{\lrdr_v(i)=0}}\\
      &= \lrd_v - \sum_{i=1}^{\lrd_v} {\Prob{}{\lrdr_v(i)=0}} 
  \end{align}
    Now we define $\rd_v^{(t)}$ the \emph{red degree} of $v$ at time $t$ and assume that edges $e_1,\dots,e_{\lrd_v}$ are inserted at times $t_1 < \dots < t_{\lrd_v}$, respectively. Thus
  \begin{align} \label{eq:gamma_ok_lb_fact_eq_2}
      \Prob{}{\lrdr_v(i) = 0} &\leq \prod_{j=i}^{\lrd_v}{(1-\frac{k}{\bd_v + \rd_v^{(t_j)}})} \notag\\
      &\leq \prod_{j=i}^{\lrd_v}{(1-\frac{k}{\lbdd_v + \gamma + 1 + j})} \notag \\
      &\leq (1-\frac{k}{\lbdd_v + \gamma + 1 + \lrd_v})^{\lrd_v - i + 1} \notag \\
      &\leq (1-\frac{k}{2(\lbdd_v + \gamma + 1)})^{\lrd_v - i}
  \end{align}

The second inequality holds because $\bd_v + \rd_v^{(t_j)} \leq \gamma + \lbdd_v + j +1$ since the number of edges incident to $v$ having the other endpoint in $\lset_1(u)$ is at most $\gamma$. Now we plug Equation \ref{eq:gamma_ok_lb_fact_eq_2} into Equation \ref{eq:gamma_ok_lb_fact_eq_1} and we get
  \begin{align}
      \Expec{}{\lrdr_v} &\geq \lrd_v - \sum_{i=1}^{\lrd_v}{(1-\frac{k}{2(\lbdd_v + \gamma + 1)})^{\lrd_v - i}} \notag \\
      &= \lrd_v - \sum_{i=0}^{\lrd_v-1}{(1-\frac{k}{2(\lbdd_v + \gamma + 1)})^{i}} \notag \\
      &= \lrd_v - \frac{1-(1-\frac{k}{2(\lbdd_v+\gamma+1)})^{\lrd_v}}{1-(1-\frac{k}{2(\lbdd_v + \gamma + 1)})} \notag \\
      &\geq \lrd_v - \frac{1}{1-(1-\frac{k}{2(\lbdd_v + \gamma + 1)})} \notag \\
      &\geq \lrd_v - \frac{2(\lbdd_v + \gamma + 1)}{k} \notag
  \end{align}
\end{proof}
Now, let $\lbddt$ denote the number of nodes in $\lset_2(u)$ that have at least one \emph{black} edge from $\lset_1(u)$. Consequently, these nodes are included in $\apxball_2(u)$. We have:

\begin{lemma}\label{lemma:gamma_ok_properties}
For any \gammaok graph $G=(V,E)$, and any $u \in V$, then
\begin{align*}
    \sum_{v \in \lset_1(u)}\lbdd_v \geq \lbddt\geq \sum_{v \in \lset_1(u)} \frac{\lbdd_v}{\gamma + 1}
\end{align*}
\end{lemma}
\begin{proof}
    %\item[Property 2.] 
    The first inequality is straightforward, while the latter follows from the fact that every node in $\lset_2(u)$ can have at most $\gamma + 1$ neighbors in $\lset_1(u)$.  
    %\item[Property 3.] \textcolor{red}{Here goes the proof of prop.3.}
\end{proof}
  
We are now ready to prove the main result of this section. 


\begin{theorem}\label{thm:gamma-ok-main}
    
Let $\varepsilon \in (0,1)$, and let $G^{(0)}$ be an initial graph. Consider any sequence of edge insertions that yields a final graph $G$. If $G$ is \gammaok, \lazyscheme$(\varphi =1,k=\frac{4(\gamma+1)}{\varepsilon})$ has an approximation ratio of  $\frac{1}{1-\varepsilon}$ and amortized update cost $O(\frac{\gamma+1}{\varepsilon})$.     
\end{theorem}

\begin{proof}
The amortized update cost immediately follows from \Cref{lm:amortized_det_alg}.Concerning the approximation ration, fix a vertex $u \in V$. For technical convenience, we will define a subgraph $\Tilde{G}$ of $G$ where we remove suitable edges from $G$, and we prove the following two properties: (i) if $k\ge \frac{2(\gamma+1)}{\varepsilon}$, the \lazyscheme\ computes a $(1-\varepsilon)$-covering of $\ball_2(u)$ when the sequence of the edge insertions is restricted to edge in $\Tilde{G}$; (ii) property (i) implies that the \lazyscheme\ computes a $(1-\varepsilon)$-covering of $\ball_2(u)$ for $G$ as well, as soon as $k\ge \frac{4(\gamma+1)}{\varepsilon}$.

We start by defining $\Tilde{G}$ which is obtained from $G$ as follows. For each $w \in \lset_2(u)$, if there exists a vertex $v \in \lset_1(u)$ such that edge $(v,w)$ is black for $v$, then we remove all the red edges incident to $w$ that comes from $\lset_1(u)$. Otherwise, we have that all the edges coming from $\lset_1(u)$ are red. In this case we remove all such edges but one. See Figure-?? for an example.\rem{Dobbiamo fare una piccola figura esplicativa.}

We now prove property (i). 
We analyze the process at a generic time $t>0$.
We want to prove that $\Expec{}{\vert \apxball_2(u) \vert} \ge (1-\varepsilon)\vert \ball_2(u) \vert$. Since the set $\lset_1(u)$ is always contained in  $\apxball_2(u)$, we focus on vertices belonging to $\lset_2(u)$.  

Let $\lambda$ and $\hat{\lambda}$ be the number of vertices in $\lset_2(u)$ attached to $\lset_1(u)$ with a red and with a quasi-black edge, respectively. We want to show that $\lbddt+ \hat{\lambda} \geq (1-\varepsilon)(\lbddt+ \lambda)$ in expectation. Which is equivalent to:
    \begin{align} \label{eq:errro_bound_2}
    \hat{\lambda} \ge (1-\varepsilon)\lambda - \varepsilon \lbddt   
    \end{align}
    Then, by definition of $\lambda$ and $\hat{\lambda}$ and by \Cref{lemma:gamma_ok_properties}, we get that Equation \ref{eq:errro_bound_2} in turn is implied by 
    \begin{align} \label{eq:error_bound_3}
    \sum_{v \in \lset_1(u)}{\lrdr_v} &\ge (1-\varepsilon)\sum_{v \in \lset_1(u)}{\lrd_v} - \frac{\varepsilon}{1 + \gamma}\sum_{v \in \lset_1(u)}{\lbdd_v}\\
    &= \sum_{v \in \lset_1(u)}{\left((1-\varepsilon)\lrd_v - \frac{\varepsilon}{1 + \gamma}\lbdd_v \right).} \notag
    \end{align}
    The inequality in \Cref{eq:error_bound_3} holds in expectation by showing that it holds term-by-term. That is:
    \begin{equation*}
        \Expec{}{\lrdr_v} \ge (1-\varepsilon)\lrd_v - \frac{\varepsilon}{1+\gamma}\lbdd_v
    \end{equation*}
    Clearly, if $\lrd_v = 0$ then $\lrdr_v = 0$ and the inequality holds; thus we focus on the case $\lrd_v > 0$.\\ From \Cref{le:gamma_ok_expect_lowerbound} we know that $\Expec{}{\lrdr_v} \geq \lrd_v - \frac{2(\lbdd_v + \gamma + 1)}{k}$. Therefore, we obtain
    \begin{align*}
        &\lrd_v - \frac{2(\lbdd_v + \gamma + 1)}{k} \ge (1-\varepsilon)\lrd_v - \frac{\varepsilon}{1+\gamma}\lbdd_v\\
        &\iff \frac{2(\lbdd_v + \gamma + 1)}{k} \leq \varepsilon \lrd_v - \frac{\varepsilon}{1 + \gamma}\lbdd_v \\
        &\iff \frac{k}{2(\lbdd_v + \gamma + 1)} \geq \frac{1+\gamma}{(1+\gamma)\varepsilon \lrd_v + \varepsilon \lbdd_v} \\
        &\iff k \geq \frac{2(1+\gamma + \lbdd_v)(1 + \gamma)}{(1 + \gamma)\varepsilon \lrd_v +\varepsilon \lbdd_v}
    \end{align*}
    Lastly, observe that
    \begin{align*}
    k \ge \frac{2(1+\gamma)}{\varepsilon} \implies k \ge \frac{2(1+\gamma + \lbdd_v)(1 + \gamma)}{(1 + \gamma)\varepsilon \lrd_v + \varepsilon \lbdd_v} =  \frac{2(1+\gamma)}{\varepsilon} \frac{1 + \gamma + \lbdd_v}{(1+\gamma)\lrd_v + \lbdd_v}.    
    \end{align*}
    Therefore, property (i) is proven.

Finally, we prove (ii). Notice that we obtain $\Tilde{G}$ by removing some red edges from $G$. On one hand, this reduces the chance to select $u$ in Line 14 in \Cref{alg:det_thresh} as random neighbor of some vertex in $\lset_1(u)$. On the other hand, we reduce the degree of some vertex in $\lset_1(u)$. However, since $\varphi=1$, we have that the red degree of any vertex is at most its black degree. So, in $\Tilde{G}$ the degree of a vertex in $\lset_1(u)$ can at most be halved. In order to manage this we simply double the value of $k$. And the claim follows.
\end{proof}



