\section{Discussion and outlook}\label{sec:concl}
In this work, we showed that relatively simple, lazy update algorithms can play a key role to efficiently support queries over multi-hop vertex neighborhoods on large, incremental graphs.
At the same time, this work leaves a number of open questions that might deserve further investigation. A first, obvious direction is extending our approach to handle edge deletions. While our approach is  potentially useful in the general case, the main problem here is handling deletions when compact, sketch-based data structures are used to represent $1$- and $2$-balls. Some recent contributions address similar issues in dynamic data streams \cite{BSS20,CGPS24}, but extending our analyses to this general case does not seem straightforward. Another interesting direction is investigating strategies to handle queries over $h$-balls when $h > 2$, for example maintaining their sizes under dynamic updates. In this case, each edge addition/deletion potentially has cascading effects over $h$-hops. Optimizing (amortized) update costs in this general setting does not seem trivial and we conjecture that a dependence on $h$ might be necessary. Finally, we remark that our algorithms are inherently local, i.e., whether or not to perform updates involving any vertex $v$ only depends on $v$'s immediate neighborhood. As a consequence, a potentially interesting avenue for further research is to investigate distributed variants of our approach, possibly with massive parallel architectures in mind \cite{lee2012parallel}.