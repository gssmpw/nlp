\newcommand{\Prob}[2]{\mathbb{P}_{#1} \left( #2 \right)}
\newcommand{\Expec}[2]{\mathbb{E}_{#1} \left[ #2 \right]}
\newcommand{\Var}[2]{\mathrm{Var}_{#1} \left( #2 \right)}
\newcommand{\Cov}[2]{\mathrm{Cov}_{#1} \left( #2 \right)}
%\newcommand{\proof}{\noindent\textit{Proof. }} 
%\newcommand{\qed}{\hspace{\stretch{1}$\square$}}
\newcommand{\skproof}{\noindent\textit{Sketch of Proof. }}
\newcommand{\ideaproof}{\noindent\textit{Idea of the proof. }}

\newcommand{\polylog}[1]{\mbox{polylog}\left(#1\right)}
\newcommand{\poly}{ {\mathrm{poly}}}
\newcommand{\real}{ {\mathbb{R}}}
\newcommand{\integer}{\mathbb{N}}
\newcommand{\bigO}{\mathbf{O}}



%%%%% Notation %%%%%
\newcommand{\bd}{\Delta} % black degree
\newcommand{\rd}{\delta} % red degree
\newcommand{\lbd}{N}
\newcommand{\lbdd}{b}
\newcommand{\lbddt}{\beta}
\newcommand{\lrd}{\ell}
\newcommand{\lrdr}{\hat{\ell}}

%%%%%% Overall degree is \deg and it is already a Latex command
\newcommand{\Normal}[1]{\mathcal{N}\left(#1\right)} % normal distribution
\newcommand{\neigh}{\mathcal{N}} % neighbourhood
\newcommand{\ball}{B} % ball
\newcommand{\lset}{L} % \lset_h(u) is distance h level set wrt u: set of vertices at distance exactly h from u
\newcommand{\apxball}{\hat{B}} % approximate ball
\newcommand{\ok}[1]{locally $#1$-sparse}% ok grpahs with paramater
\newcommand{\lazyscheme}{\textsc{Lazy-Alg}}
\newcommand{\gammaok}{locally $\gamma$-sparse} % gamma-ok
\newcommand{\jacc}{\mathbf J } % Jaccard similarity
\newcommand{\apxjacc}{\hat{J}} % approximate Jaccard similarity
\newcommand{\apxlset}{\hat{L}}
\newcommand{\cost}{T}




\newcommand{\ie}{i.e., }
\newcommand{\rem}[1]{\todo[inline,color=yellow]{#1}}

\def\bzero{{\bf 0}}
\def\bone{\mathbf{1}}
\def\bx{{\mathbf x}}
\def\by{{\mathbf y}}
\def\bs{{\mathbf s}}
\def\bz{{\mathbf z}}
\def\bw{{\mathbf w}}
\def\bu{{\mathbf u}}
\def\ba{{\mathbf a}}
\def\dynG{{\mathcal G}}

%%%%% Notation %%%%%

%%%%% Theorem style %%%%%

\newtheorem{theorem}{Theorem}
\newtheorem{corollary}[theorem]{Corollary}

\newtheorem{definition}{Definition}[section]
\newtheorem{observation}[definition]{Observation}
\newtheorem{example}[definition]{Example}
\newtheorem{conj}[definition]{Conjecture}

\newtheorem{fact}{Fact}
\newtheorem{remark}{Remark}
\newtheorem{claim}{Claim}
\newtheorem{lemma}{Lemma}

%%%%% Theorem style %%%%%

\newenvironment{myprocedure}[1]{%
    \let\oldalgorithmcfname\algorithmcfname % Save the current name ("Algorithm")
    \renewcommand{\algorithmcfname}{Procedure} % Change to "Procedure"
    \begin{algorithm}[#1] % Start the algorithm environment with optional argument
}{%
    \end{algorithm} % End the algorithm environment
    \renewcommand{\algorithmcfname}{\oldalgorithmcfname} % Restore the original name
}
