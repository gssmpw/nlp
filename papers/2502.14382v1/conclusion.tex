\section{Conclusion}
%\kurt{I would summarize quantitative improvements here as you did in the abstract.}
% We propose~\frameworkname, a simple and effective framework to scale test-time compute for code generation. It integrates iterative debugging methods into the existing repeated sampling based paradigm, together with \textit{adaptive input synthesis} that selects the final sample with grounded information. ~\frameworkname has been shown to consistently improve code generation capability on LiveCodeBench and CodeContests. With~\frameworkname, instruction-tuned model outperforms reasoning models: \fouromini with our method outperforms o1-preview without our method by 3.7\% on LiveCodeBench); open reasoning model can further match closed sourced state-of-the-art: DeepSeek-R1-Distill-Qwen-32B with our method achieves 86.7\% on LiveCodeBench, close to o1 (high reasoning efforts) without our method at 88.5\%.
We propose \frameworkname, the first hybrid test-time scaling framework for code generation that substantially improves both coverage and selection accuracy. \frameworkname extends the existing parallel scaling paradigm with sequential scaling through iterative debugging and incorporates \textit{adaptive input synthesis}, a novel mechanism that synthesizes distinguishing test inputs to differentiate candidates and identify correct solutions via execution results.

\frameworkname consistently improves code generation performance across benchmarks, including LiveCodeBench and CodeContests. Notably, \frameworkname enables a 3B model to outperform \fouromini, \fouromini to surpass o1-preview by 3.7\% on LiveCodeBench, and DeepSeek-R1-Distill-Qwen-32B to achieve 86.7\% on LiveCodeBench, approaching o1-high at 88.5\%.
% (1) instruction-tuned models to outperform reasoning models: \fouromini with \frameworkname surpasses o1-preview without \frameworkname by 3.7\% on LiveCodeBench; and (2) open reasoning models to match the performance of state-of-the-art closed-source models: DeepSeek-R1-Distill-Qwen-32B with \frameworkname achieves 86.7\% on LiveCodeBench, approaching o1-high without \frameworkname at 88.5\%.