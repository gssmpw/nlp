\pdfoutput=1
\documentclass{article}

\usepackage{PRIMEarxiv}
\usepackage{algorithm}
\usepackage{algpseudocode}
\usepackage[utf8]{inputenc} % allow utf-8 input
\usepackage[T1]{fontenc}    % use 8-bit T1 fonts
\usepackage{hyperref}       % hyperlinks
\usepackage{url}            % simple URL typesetting
\usepackage{booktabs}       % professional-quality tables
\usepackage{xcolor}
\usepackage{amsfonts}       % blackboard math symbols
\usepackage{nicefrac}       % compact symbols for 1/2, etc.
\usepackage{microtype}      % microtypography
\usepackage{lipsum}
\usepackage{fancyhdr}       % header
\usepackage{graphicx}       % graphics
\graphicspath{{media/}}     % organize your images and other figures under media/ folder
\usepackage[thicklines]{cancel}
\usepackage{amsfonts,amssymb}
\usepackage{dsfont}
\usepackage{enumitem}
\usepackage{pifont}
\usepackage{subfigure}
\usepackage{wrapfig}
%Header
\pagestyle{fancy}
\thispagestyle{empty}
\rhead{ \textit{ }} 
\usepackage{amsmath}
\usepackage{amsthm}

\newtheorem{assumption}{Assumption}[section]
\newtheorem{theorem}{Theorem}
\newtheorem{definition}{Definition}
\newtheorem{corollary}{Corollary}[theorem]
\newtheorem{lemma}[theorem]{Lemma}
\newtheorem{proposition}[theorem]{Proposition}
\newtheorem{remark}{Remark}[section]
\DeclareMathOperator*{\argminA}{arg\,min}
\DeclareMathOperator*{\argmaxA}{arg\,max}

% % Todonotes is useful during development; simply uncomment the next line
% %    and comment out the line below the next line to turn off comments
% % \usepackage[disable,textsize=tiny]{todonotes}
% \usepackage[textsize=tiny]{todonotes}


%Header
\pagestyle{fancy}
\thispagestyle{empty}
\rhead{ \textit{ }} 

% Update your Headers here
\fancyhead[LO]{Provably Efficient Multi-Objective Bandit Algorithms under Preference-Centric Customization}
% \fancyhead[RE]{Firstauthor and Secondauthor} % Firstauthor et al. if more than 2 - must use \documentclass[twoside]{article}



  
%% Title
\title{
Provably Efficient Multi-Objective Bandit Algorithms under Preference-Centric Customization
}

\author{
  Linfeng Cao \\
  Department of CSE \\
  The Ohio State University \\
  % City\\
  \texttt{cao.1378@osu.edu} \\
  %% examples of more authors
  \And
  Ming Shi \\
  Department of Electrical Engineering \\
  University at Buffalo \\
  % City\\
  \texttt{mshi24@buffalo.edu} \\
   \And
  Ness B. Shroff \\
  Department of ECE \\
  The Ohio State University \\
  % City\\
  \texttt{shroff.11@osu.edu} \\
}


\begin{document}
\maketitle


\begin{abstract}
Multi-objective multi-armed bandit (MO-MAB) problems traditionally aim to achieve Pareto optimality. However, real-world scenarios often involve users with varying preferences across objectives, resulting in a Pareto-optimal arm that may score high for one user but perform quite poorly for another. This highlights the need for \emph{customized learning}, a factor often overlooked in prior research.
To address this, we study a \emph{preference-aware} MO-MAB framework in the presence of explicit user preference.
% , where each user's overall-reward is modeled as the inner product of user preference and arm reward. 
It shifts the focus from achieving Pareto optimality to further optimizing within the Pareto front under preference-centric customization. To our knowledge, this is the first theoretical study of customized MO-MAB optimization with explicit user preferences.
Motivated by practical applications, we explore two scenarios: unknown preference and hidden preference, each presenting unique challenges for algorithm design and analysis. At the core of our algorithms are \emph{preference estimation} and \emph{preference-aware optimization} mechanisms to adapt to user preferences effectively. We further develop novel analytical techniques to establish near-optimal regret of the proposed algorithms. Strong empirical performance confirm the effectiveness of our approach.
\end{abstract}

\section{Introduction}
\label{sec:introduction}
The business processes of organizations are experiencing ever-increasing complexity due to the large amount of data, high number of users, and high-tech devices involved \cite{martin2021pmopportunitieschallenges, beerepoot2023biggestbpmproblems}. This complexity may cause business processes to deviate from normal control flow due to unforeseen and disruptive anomalies \cite{adams2023proceddsriftdetection}. These control-flow anomalies manifest as unknown, skipped, and wrongly-ordered activities in the traces of event logs monitored from the execution of business processes \cite{ko2023adsystematicreview}. For the sake of clarity, let us consider an illustrative example of such anomalies. Figure \ref{FP_ANOMALIES} shows a so-called event log footprint, which captures the control flow relations of four activities of a hypothetical event log. In particular, this footprint captures the control-flow relations between activities \texttt{a}, \texttt{b}, \texttt{c} and \texttt{d}. These are the causal ($\rightarrow$) relation, concurrent ($\parallel$) relation, and other ($\#$) relations such as exclusivity or non-local dependency \cite{aalst2022pmhandbook}. In addition, on the right are six traces, of which five exhibit skipped, wrongly-ordered and unknown control-flow anomalies. For example, $\langle$\texttt{a b d}$\rangle$ has a skipped activity, which is \texttt{c}. Because of this skipped activity, the control-flow relation \texttt{b}$\,\#\,$\texttt{d} is violated, since \texttt{d} directly follows \texttt{b} in the anomalous trace.
\begin{figure}[!t]
\centering
\includegraphics[width=0.9\columnwidth]{images/FP_ANOMALIES.png}
\caption{An example event log footprint with six traces, of which five exhibit control-flow anomalies.}
\label{FP_ANOMALIES}
\end{figure}

\subsection{Control-flow anomaly detection}
Control-flow anomaly detection techniques aim to characterize the normal control flow from event logs and verify whether these deviations occur in new event logs \cite{ko2023adsystematicreview}. To develop control-flow anomaly detection techniques, \revision{process mining} has seen widespread adoption owing to process discovery and \revision{conformance checking}. On the one hand, process discovery is a set of algorithms that encode control-flow relations as a set of model elements and constraints according to a given modeling formalism \cite{aalst2022pmhandbook}; hereafter, we refer to the Petri net, a widespread modeling formalism. On the other hand, \revision{conformance checking} is an explainable set of algorithms that allows linking any deviations with the reference Petri net and providing the fitness measure, namely a measure of how much the Petri net fits the new event log \cite{aalst2022pmhandbook}. Many control-flow anomaly detection techniques based on \revision{conformance checking} (hereafter, \revision{conformance checking}-based techniques) use the fitness measure to determine whether an event log is anomalous \cite{bezerra2009pmad, bezerra2013adlogspais, myers2018icsadpm, pecchia2020applicationfailuresanalysispm}. 

The scientific literature also includes many \revision{conformance checking}-independent techniques for control-flow anomaly detection that combine specific types of trace encodings with machine/deep learning \cite{ko2023adsystematicreview, tavares2023pmtraceencoding}. Whereas these techniques are very effective, their explainability is challenging due to both the type of trace encoding employed and the machine/deep learning model used \cite{rawal2022trustworthyaiadvances,li2023explainablead}. Hence, in the following, we focus on the shortcomings of \revision{conformance checking}-based techniques to investigate whether it is possible to support the development of competitive control-flow anomaly detection techniques while maintaining the explainable nature of \revision{conformance checking}.
\begin{figure}[!t]
\centering
\includegraphics[width=\columnwidth]{images/HIGH_LEVEL_VIEW.png}
\caption{A high-level view of the proposed framework for combining \revision{process mining}-based feature extraction with dimensionality reduction for control-flow anomaly detection.}
\label{HIGH_LEVEL_VIEW}
\end{figure}

\subsection{Shortcomings of \revision{conformance checking}-based techniques}
Unfortunately, the detection effectiveness of \revision{conformance checking}-based techniques is affected by noisy data and low-quality Petri nets, which may be due to human errors in the modeling process or representational bias of process discovery algorithms \cite{bezerra2013adlogspais, pecchia2020applicationfailuresanalysispm, aalst2016pm}. Specifically, on the one hand, noisy data may introduce infrequent and deceptive control-flow relations that may result in inconsistent fitness measures, whereas, on the other hand, checking event logs against a low-quality Petri net could lead to an unreliable distribution of fitness measures. Nonetheless, such Petri nets can still be used as references to obtain insightful information for \revision{process mining}-based feature extraction, supporting the development of competitive and explainable \revision{conformance checking}-based techniques for control-flow anomaly detection despite the problems above. For example, a few works outline that token-based \revision{conformance checking} can be used for \revision{process mining}-based feature extraction to build tabular data and develop effective \revision{conformance checking}-based techniques for control-flow anomaly detection \cite{singh2022lapmsh, debenedictis2023dtadiiot}. However, to the best of our knowledge, the scientific literature lacks a structured proposal for \revision{process mining}-based feature extraction using the state-of-the-art \revision{conformance checking} variant, namely alignment-based \revision{conformance checking}.

\subsection{Contributions}
We propose a novel \revision{process mining}-based feature extraction approach with alignment-based \revision{conformance checking}. This variant aligns the deviating control flow with a reference Petri net; the resulting alignment can be inspected to extract additional statistics such as the number of times a given activity caused mismatches \cite{aalst2022pmhandbook}. We integrate this approach into a flexible and explainable framework for developing techniques for control-flow anomaly detection. The framework combines \revision{process mining}-based feature extraction and dimensionality reduction to handle high-dimensional feature sets, achieve detection effectiveness, and support explainability. Notably, in addition to our proposed \revision{process mining}-based feature extraction approach, the framework allows employing other approaches, enabling a fair comparison of multiple \revision{conformance checking}-based and \revision{conformance checking}-independent techniques for control-flow anomaly detection. Figure \ref{HIGH_LEVEL_VIEW} shows a high-level view of the framework. Business processes are monitored, and event logs obtained from the database of information systems. Subsequently, \revision{process mining}-based feature extraction is applied to these event logs and tabular data input to dimensionality reduction to identify control-flow anomalies. We apply several \revision{conformance checking}-based and \revision{conformance checking}-independent framework techniques to publicly available datasets, simulated data of a case study from railways, and real-world data of a case study from healthcare. We show that the framework techniques implementing our approach outperform the baseline \revision{conformance checking}-based techniques while maintaining the explainable nature of \revision{conformance checking}.

In summary, the contributions of this paper are as follows.
\begin{itemize}
    \item{
        A novel \revision{process mining}-based feature extraction approach to support the development of competitive and explainable \revision{conformance checking}-based techniques for control-flow anomaly detection.
    }
    \item{
        A flexible and explainable framework for developing techniques for control-flow anomaly detection using \revision{process mining}-based feature extraction and dimensionality reduction.
    }
    \item{
        Application to synthetic and real-world datasets of several \revision{conformance checking}-based and \revision{conformance checking}-independent framework techniques, evaluating their detection effectiveness and explainability.
    }
\end{itemize}

The rest of the paper is organized as follows.
\begin{itemize}
    \item Section \ref{sec:related_work} reviews the existing techniques for control-flow anomaly detection, categorizing them into \revision{conformance checking}-based and \revision{conformance checking}-independent techniques.
    \item Section \ref{sec:abccfe} provides the preliminaries of \revision{process mining} to establish the notation used throughout the paper, and delves into the details of the proposed \revision{process mining}-based feature extraction approach with alignment-based \revision{conformance checking}.
    \item Section \ref{sec:framework} describes the framework for developing \revision{conformance checking}-based and \revision{conformance checking}-independent techniques for control-flow anomaly detection that combine \revision{process mining}-based feature extraction and dimensionality reduction.
    \item Section \ref{sec:evaluation} presents the experiments conducted with multiple framework and baseline techniques using data from publicly available datasets and case studies.
    \item Section \ref{sec:conclusions} draws the conclusions and presents future work.
\end{itemize}
\section{RELATED WORK}
\label{sec:relatedwork}
In this section, we describe the previous works related to our proposal, which are divided into two parts. In Section~\ref{sec:relatedwork_exoplanet}, we present a review of approaches based on machine learning techniques for the detection of planetary transit signals. Section~\ref{sec:relatedwork_attention} provides an account of the approaches based on attention mechanisms applied in Astronomy.\par

\subsection{Exoplanet detection}
\label{sec:relatedwork_exoplanet}
Machine learning methods have achieved great performance for the automatic selection of exoplanet transit signals. One of the earliest applications of machine learning is a model named Autovetter \citep{MCcauliff}, which is a random forest (RF) model based on characteristics derived from Kepler pipeline statistics to classify exoplanet and false positive signals. Then, other studies emerged that also used supervised learning. \cite{mislis2016sidra} also used a RF, but unlike the work by \citet{MCcauliff}, they used simulated light curves and a box least square \citep[BLS;][]{kovacs2002box}-based periodogram to search for transiting exoplanets. \citet{thompson2015machine} proposed a k-nearest neighbors model for Kepler data to determine if a given signal has similarity to known transits. Unsupervised learning techniques were also applied, such as self-organizing maps (SOM), proposed \citet{armstrong2016transit}; which implements an architecture to segment similar light curves. In the same way, \citet{armstrong2018automatic} developed a combination of supervised and unsupervised learning, including RF and SOM models. In general, these approaches require a previous phase of feature engineering for each light curve. \par

%DL is a modern data-driven technology that automatically extracts characteristics, and that has been successful in classification problems from a variety of application domains. The architecture relies on several layers of NNs of simple interconnected units and uses layers to build increasingly complex and useful features by means of linear and non-linear transformation. This family of models is capable of generating increasingly high-level representations \citep{lecun2015deep}.

The application of DL for exoplanetary signal detection has evolved rapidly in recent years and has become very popular in planetary science.  \citet{pearson2018} and \citet{zucker2018shallow} developed CNN-based algorithms that learn from synthetic data to search for exoplanets. Perhaps one of the most successful applications of the DL models in transit detection was that of \citet{Shallue_2018}; who, in collaboration with Google, proposed a CNN named AstroNet that recognizes exoplanet signals in real data from Kepler. AstroNet uses the training set of labelled TCEs from the Autovetter planet candidate catalog of Q1–Q17 data release 24 (DR24) of the Kepler mission \citep{catanzarite2015autovetter}. AstroNet analyses the data in two views: a ``global view'', and ``local view'' \citep{Shallue_2018}. \par


% The global view shows the characteristics of the light curve over an orbital period, and a local view shows the moment at occurring the transit in detail

%different = space-based

Based on AstroNet, researchers have modified the original AstroNet model to rank candidates from different surveys, specifically for Kepler and TESS missions. \citet{ansdell2018scientific} developed a CNN trained on Kepler data, and included for the first time the information on the centroids, showing that the model improves performance considerably. Then, \citet{osborn2020rapid} and \citet{yu2019identifying} also included the centroids information, but in addition, \citet{osborn2020rapid} included information of the stellar and transit parameters. Finally, \citet{rao2021nigraha} proposed a pipeline that includes a new ``half-phase'' view of the transit signal. This half-phase view represents a transit view with a different time and phase. The purpose of this view is to recover any possible secondary eclipse (the object hiding behind the disk of the primary star).


%last pipeline applies a procedure after the prediction of the model to obtain new candidates, this process is carried out through a series of steps that include the evaluation with Discovery and Validation of Exoplanets (DAVE) \citet{kostov2019discovery} that was adapted for the TESS telescope.\par
%



\subsection{Attention mechanisms in astronomy}
\label{sec:relatedwork_attention}
Despite the remarkable success of attention mechanisms in sequential data, few papers have exploited their advantages in astronomy. In particular, there are no models based on attention mechanisms for detecting planets. Below we present a summary of the main applications of this modeling approach to astronomy, based on two points of view; performance and interpretability of the model.\par
%Attention mechanisms have not yet been explored in all sub-areas of astronomy. However, recent works show a successful application of the mechanism.
%performance

The application of attention mechanisms has shown improvements in the performance of some regression and classification tasks compared to previous approaches. One of the first implementations of the attention mechanism was to find gravitational lenses proposed by \citet{thuruthipilly2021finding}. They designed 21 self-attention-based encoder models, where each model was trained separately with 18,000 simulated images, demonstrating that the model based on the Transformer has a better performance and uses fewer trainable parameters compared to CNN. A novel application was proposed by \citet{lin2021galaxy} for the morphological classification of galaxies, who used an architecture derived from the Transformer, named Vision Transformer (VIT) \citep{dosovitskiy2020image}. \citet{lin2021galaxy} demonstrated competitive results compared to CNNs. Another application with successful results was proposed by \citet{zerveas2021transformer}; which first proposed a transformer-based framework for learning unsupervised representations of multivariate time series. Their methodology takes advantage of unlabeled data to train an encoder and extract dense vector representations of time series. Subsequently, they evaluate the model for regression and classification tasks, demonstrating better performance than other state-of-the-art supervised methods, even with data sets with limited samples.

%interpretation
Regarding the interpretability of the model, a recent contribution that analyses the attention maps was presented by \citet{bowles20212}, which explored the use of group-equivariant self-attention for radio astronomy classification. Compared to other approaches, this model analysed the attention maps of the predictions and showed that the mechanism extracts the brightest spots and jets of the radio source more clearly. This indicates that attention maps for prediction interpretation could help experts see patterns that the human eye often misses. \par

In the field of variable stars, \citet{allam2021paying} employed the mechanism for classifying multivariate time series in variable stars. And additionally, \citet{allam2021paying} showed that the activation weights are accommodated according to the variation in brightness of the star, achieving a more interpretable model. And finally, related to the TESS telescope, \citet{morvan2022don} proposed a model that removes the noise from the light curves through the distribution of attention weights. \citet{morvan2022don} showed that the use of the attention mechanism is excellent for removing noise and outliers in time series datasets compared with other approaches. In addition, the use of attention maps allowed them to show the representations learned from the model. \par

Recent attention mechanism approaches in astronomy demonstrate comparable results with earlier approaches, such as CNNs. At the same time, they offer interpretability of their results, which allows a post-prediction analysis. \par


% \section{Preliminaries}
% \vspace{-12pt}
\section{Problem Formulation}
% \vspace{-6pt}

% \subsection{Problem Definition}

We consider MO-MAB with $K$ arms and $D$ objectives.
At each round $t \in [T]$, the learner chooses an arm $a_t$ to play and observes a stochastic $D$-dimensional \emph{reward vector} $\boldsymbol{r}_{a_t,t} \in \mathcal{R} \subseteq \mathbb{R}^D$ for action $a_t$, which we refer to as \emph{reward}.
For the reward, we make the following standard assumption:

\begin{assumption}
[Bounded stochastic reward]
\label{assmp: all_1}
For $i \in [K], t \in [T], d \in [D]$, each reward entry $\boldsymbol{r}_{i,t}(d)$ is independently drawn from a \textbf{fixed} but \textbf{unknown} distribution with mean $\boldsymbol{\mu}_{i}(d)$ and variance $\sigma_{r,i,d}^2$, satisfying
$\boldsymbol{r}_{i,t} (d) \in [0,1]$, and $\sigma_{r,i,d}^2 \in [\sigma^2_{r \downarrow}, \sigma^2_{r \uparrow}] $, where $\sigma^2_{r \downarrow}, \sigma^2_{r \uparrow} \in \mathbb{R}^{+}$.
\end{assumption}

\textbf{User preferences.}
At each round $t$, we consider the user to be associated with a stochastic $D$-dimensional \emph{preference vector} $\boldsymbol{c}_t \in \mathcal{C} \subseteq \mathbb{R}^D$, indicating the user preferences across the $D$ objectives. 
We refer to this vector as \emph{preference} for short. Specifically, we make the following assumptions:

\begin{assumption}[Bounded stochastic preference]
\label{assmp: all_2}
For $t \in [T], d \in [D]$, each preference entry $\boldsymbol{c}_t(d)$ is independently drawn from a fixed distribution \textbf{(either known or unknown)} with mean $\boldsymbol{\overline{c}}(d)$ and variance $\sigma_{c,d}^2$, satisfying
$\boldsymbol{c}(d) \geq 0$, $\Vert \boldsymbol{c}_t \Vert_1 \leq \delta$, $\sigma_{c,d}^2 \in [0, \sigma^2_{c}]$.
\end{assumption}

\begin{assumption}
[Independence]
\label{assmp: all_3}
% All entries of $\boldsymbol{c}_t$ and $\boldsymbol{r}_t$ are independent.
For any $t \!\in\! [T]$, $i \!\in\! [K]$, $d_1,d_2 \!\in\! [D]$, $\boldsymbol{r}_{i,t}(d_1)$, $\boldsymbol{c}_t(d_2)$ are independent.
\end{assumption}

Assumption \ref{assmp: all_3} is common in real applications since $\boldsymbol{c}_t$ and $\boldsymbol{r}_t$ are inherently determined by independent factors: user characteristics and arm properties. For example, an individual user's preferences do not influence a restaurant's location, environment, pricing level, etc., and vice versa.

% Optimizing $D$ objectives simultaneously is challenging, as optimizing one often leads to suboptimal outcomes for others. A common approach is to use Pareto optimality \cite{drugan2013designing} for arm selection, where the arms whose expected reward not dominated by that of others are deemed Pareto optimal. 
% However, as shown in Fig.~\ref{fig:intro}, achieving Pareto optimality may not ensure optimal user outcomes. We further illustrate this with a lower bound (Proposition \ref{prop: lower_bd}) in Section \ref{sec:lower_bd}. Thus, MO-MAB optimization should be tailored to user preferences for more favourable results.


% Since the optimal solution for one objective often leads to sub-optimal for others, optimizing $D$ objectives simultaneously is tricky.
% One common solution is using Pareto optimality in the reward vector space to define optimality~\cite{drugan2013designing}, which utilizes Pareto ordering to compare arms based on their reward vectors. Arms that are not outperformed by any others in all objectives are considered as Pareto optimal.
% However, as illustrated in Figure~\ref{fig:intro}, merely achieving Pareto Optimality would result in suboptimal feedback from users. At the end of this section, we'll also provide a lower bound (Proposition \ref{prop: lower_bd}) to further illustrate this point. Therefore, the optimization of MO-MAB should be customized according to user preferences to ensure more favourable outcomes.

\textbf{Preference-aware reward.}
We define an \emph{overall-reward} as the \emph{inner product} of arm's reward and user's preference, which models the user reward rating under their preferences.
Specifically, we refer to the inner product mapping $\Phi: \mathcal{C} \times \mathcal{R} \rightarrow \mathbb{R}$ as the \emph{aggregation function}.
In each round $t$, the overall-reward $g_{a_t,t}$ for the chosen arm $a_t$ is defined as: 
$\textstyle
g_{a_t,t} = \Phi (\boldsymbol{c}_t, \boldsymbol{r}_{a_t, t}) 
= \boldsymbol{c}_t^{\top} \boldsymbol{r}_{a_t, t}$.


To evaluate the learner’s performance, we define regret as the cumulative difference between the expected overall-reward by selecting the arm with the highest expected overall-reward at each time $t$ and the expected overall-reward under the learner’s policy:
% \[
\begin{equation}
% \label{eq: regret_def}
\textstyle
R(T) 
 =
\sum^{T}_{t=1} \mathbb{E}[g_{a^{*}, t} - g_{a_t, t}]  =
\sum^{T}_{t=1} \boldsymbol{\overline{c}}^{\top} (\boldsymbol{\mu}_{a^*} - \boldsymbol{\mu}_{a_t} )
\end{equation}
% \]
where $a^{*} \!=\! \argmaxA_{i \in [K]} \boldsymbol{\overline{c}}^{\top} \boldsymbol{\mu}_{i}$ is the optimal arm.
The goal is to minimize the regret $R(T)$.
We term this problem as \emph{Preference-Aware MO-MAB} (PAMO-MAB).


% \begin{remark}
% Despite the linear model of overall reward, PAMO-MAB differs fundamentally from linear (contextual) bandits \cite{abbasi2011improved, chu2011contextual} for the following reasons:
% \begin{itemize}[leftmargin=*]
% \vspace{-9pt}
% \item
% In linear bandits, the input features are observable before making decisions, whereas in PAMO-MAB, both the random reward and preference can be unknown and must be estimated.
% \vspace{-7pt}
% \item
% In linear bandits, the feedback is a scalar reward, whereas in PAMO-MAB, the feedback can take on various forms: a $D$-dimensional reward, a $D$-dimensional reward with a $D$-dimensional preference, or a $D$-dimensional reward with an overall-reward, depending on the interaction protocols.
% \end{itemize}
% \end{remark} 



% \vspace{-8pt}
\section{A Lower Bound}
\label{sec:lower_bd}
% \vspace{-6pt}
In the following, we develop a lower bound (Proposition \ref{prop: lower_bd}) on the defined regret for PAMO-MAB.
Such a lower bound will quantify how difficult it is to control regret without preference-adaptive policies under PAMO-MAB. 
Firstly, we present a definition characterizing a class of MO-MAB algorithms that are "preference-free".
% of which the sequential decision-making is independent of the preference information.


% \begin{definition}[Preference-Free Algorithm]
% \label{def: pref_free_alg}
% Let
% $\boldsymbol{\mathrm{c}}^{t} = \{\boldsymbol{c}_1, \boldsymbol{c}_2, ..., \boldsymbol{c}_{t}\} \in \mathbb{R}^{D \times t}$ be the preference sequence up to $t$ episode with mean $\boldsymbol{\overline{c}}$. 
% Let $\pi_t^{\mathcal{A}}$ be the policy of algorithm $\mathcal{A}$ at time $t$ for selecting arm $a_t$ in a PAMO-MAB problem. 
% Then $\mathcal{A}$ is defined as a preference-free algorithm if its policy 
% $\pi_t^{\mathcal{A}}$ is independent of $\boldsymbol{\mathrm{c}}^{t}$ and $\boldsymbol{\overline{c}}$, i.e., 
% $\mathbb{P}_{\pi_t^{\mathcal{A}}} (a_t = i | \boldsymbol{\mathrm{c}}^{t}, \boldsymbol{\overline{c}}^{t}) = \mathbb{P}_{\pi_t^{\mathcal{A}}} (a_t = i)$
% for all arms $i \in [K]$ and all episodes $t \in (0,T]$.
% \end{definition} 

\begin{definition}[Preference-Free Algorithm]
\label{def: pref_free_alg}
Let
$\boldsymbol{\mathrm{c}}^{t} = \{\boldsymbol{c}_1, \boldsymbol{c}_2, ..., \boldsymbol{c}_{t}\} \in \mathbb{R}^{D \times t}$ be the preference sequence up to $t$ episode with mean $\boldsymbol{\overline{c}}$. 
Let $\pi_t^{\mathcal{A}}$ be the policy of algorithm $\mathcal{A}$ at time $t$ for selecting arm $a_t$. 
Then $\mathcal{A}$ is defined as preference-free if its policy 
$\pi_t^{\mathcal{A}}$ is independent of $\boldsymbol{\mathrm{c}}^{t}$ and $\boldsymbol{\overline{c}}$, i.e., 
$\mathbb{P}_{\pi_t^{\mathcal{A}}} (a_t = i \mid \boldsymbol{\mathrm{c}}^{t}, \boldsymbol{\overline{c}}) = \mathbb{P}_{\pi_t^{\mathcal{A}}} (a_t = i)$
for all arms $i \in [K]$ and all episodes $t \in [T]$.
\end{definition} 

To our knowledge, most existing algorithms in theoretical MO-MAB studies \cite{drugan2013designing, busa2017multi, xu2023pareto, huyuk2021multi, cheng2024hierarchize} fall within the class of preference-free algorithms, which employ a global policy for arm selection, while neglecting users' preferences.


\begin{proposition}
\label{prop: lower_bd}
Assume an MO-MAB environment contains multiple objective-conflicting arms, i.e., $\vert \mathcal{O}^{*} \vert \geq 2$, where $\mathcal{O}^{*}$ is the Pareto Optimal front. 
Then, for any preference-free algorithm, there exists a subset of preference such that the regret $R(T) = \Omega(T)$. 
\end{proposition} 

Proposition \ref{prop: lower_bd} shows that for PAMO-MAB problem with $|\mathcal{O}^*| \geq 2$, sub-linear regret is unachievable for preference-free algorithms.
This is because, for any arm $i \in \mathcal{O}^*$ that is optimal in one preference subset $\mathcal{C}^{+}$, there exists another preference subset $\mathcal{C}^{-}$ where arm $i$ becomes suboptimal,
while preference-free algorithms cannot adapt their policies to varying preference across the entire space $\mathcal{C}$. 
Please see Appendix \ref{sce: app_pf_lower_bd} for the  detailed proof. 
We therefore ask the following question:
\textbf{\emph{Can we design preference-adaptive algorithms that achieve sub-linear regret for PAMO-MAB?}}
The answer is {\bf yes.}
In the following, we analyze PAMO-MAB under two scenarios: preference feedback provided and hidden preference, demonstrating that with preference adaptation, sub-linear regret can indeed be achieved.






% \newpage
\section{The Case when the Preference Is Provided}
\label{sec: knonw}

% \captionsetup[figure]{skip=5pt}

\begin{figure}[t]
\vskip -0.1in
\begin{center}
\centerline{\includegraphics[width=0.55\textwidth]{figures/s1_2.pdf}}
\caption{
An example where user explicitly provides her preference to LLM system before (\ding{182}) or after (\ding{183}) the response of movie recommendation (decision making).}
\label{fig: s1}
\end{center}
\end{figure}

% \begin{wrapfigure}{r}{0.47\textwidth} % Adjust the width as needed
%     \vspace{-20pt} % Adjust vertical space before the algorithm
%     \begin{minipage}{0.47\textwidth}
% \begin{algorithm}[H]
% \small
% \caption{Preference UCB (PRUCB)}
% \begin{algorithmic}[1]
% \STATE \textbf{Parameters:} $\alpha$.
% \STATE \textbf{Initialization:} 
% $N_{i, 1} \!\leftarrow\! 0$; $\boldsymbol{\hat{r}}_{i,1} \!\leftarrow\! [0]^{D}, \forall i \!\in\! [K]$.

% \For{$t=1,\cdots,$ T}
%     \STATE Obtain user expected preference $\boldsymbol{\overline{c}}_t$, $\boldsymbol{\hat{c}}_t \!\leftarrow\! \boldsymbol{\overline{c}}_t$.
%     % \STATE $a_t = \argmaxA_{i \in [K]} f ( \boldsymbol{\hat{c}}_t,  \hat{\boldsymbol{r}}_{i,t}+\sqrt{\frac{ \log(t/\alpha)}{\max \{1, N_{i, t } \} }} \boldsymbol{e} )$.
%     \STATE Draw $a_t$ by Eq. \ref{eq:prucb_at}, observe reward $\boldsymbol{r}_{a_t, t}$.
%     % \Comment{Computation of Eq.~\eqref{beta_update}}
%     % \STATE Update 
%     % \begin{small}
%     %     $N_{i, t+1} \!=\! N_{i, t} \!+\! \mathds{1}_{ \{a_t = i \}}$
%     % \end{small}, 
%     % reward estimation $\hat{\boldsymbol{r}}_{i,t+1} = \frac{\hat{\boldsymbol{r}}_{i,t} N_{i, t} + \boldsymbol{r}_{a_t, t} \cdot \mathds{1}_{ \{a_t = i \}} }{N_{i, t+1}}, \forall i \in [K]$.
%     \STATE Update $N_{i, t+1}$ and $\hat{\boldsymbol{r}}_{i,t+1}, \forall i \!\in\! [K]$ by Eq.\ref{eq:prucb_r_t}.
%     % \STATE Update estimated reward $\hat{\boldsymbol{r}}_{i,t+1} = \frac{\hat{\boldsymbol{r}}_{i,t} N_{i, t} + \boldsymbol{r}_{i, t}}{N_{i, t+1}}$ if $i==a_t$ else $\hat{\boldsymbol{r}}_{i,t+1} = \hat{\boldsymbol{r}}_{i,t}, \forall i \in [K]$.
% \EndFor
% \end{algorithmic}
% \label{alg:PRUCB}
% \end{algorithm}
%  \end{minipage}
%     \vspace{-10pt} % Adjust vertical space after the algorithm
% \end{wrapfigure}



% for understanding the structure of the problem and tackling the subsequent more challenging case where preferences are unknown
As a warm-up, we begin with a simpler case where the user's preferences are explicitly provided to the agent either before or after decision making. 
% Formally, at each round $t$, the learner obtains $\overline{\boldsymbol{c}}_t \in \mathbb{R}^D$ from user's input and selects an arm $a_t \in [K]$, then observes $\boldsymbol{r}_{a_t,t} \in \mathbb{R}^D$. 
This setup is prevalent in numerous real-world applications. 
% In personalized recommender, systems are typically informed of user preferences (e.g., quality, price, style) before or after recommendation.
Many online systems now allow users to express their preferences through interactive techniques, such as conversations and prompt design, either before or after taking action. For example, as shown in Fig. \ref{fig: s1}, a user can explicitly share her movie preferences with an LLM-based chat system either prior to receiving a recommendation (\textsf{\ding{182}} in the initial input) or after receiving one (\textsf{\ding{183}} in the follow-up conversation).

We observe that when the preference $\boldsymbol{c}_t$ is given before decision-making, by simply using the preference $\boldsymbol{c}_t$ as a weight vector on the reward estimate, the 
problem collapses into a single-objective MAB. 
Therefore, we focus on the case where the preference is unknown and only revealed after decision-making. 
Formally, at each round $t$, the learner first selects an arm $a_t$, and then observes the reward $\boldsymbol{r}_{a_t}$ and user's preference $\boldsymbol{c}_t$.
As we will show later, the known preference case can be handled as a special instance within our developed framework for unknown preference.

As discussed in Section \ref{sec:lower_bd}, policy adapting to user preference is crucial. 
% For the unknown preference case, the inaccessibility of the true preference expectation raises two fundamental questions for algorithm design:
% \emph{(1) How to estimate the unknown preference?
% (2) How to handle the uncertainty of preference estimation in decision making?}
To achieve this, we develop Preference-UCB for Unknown Preference (PRUCB-UP, presented in Algorithm \ref{alg:PRUCB_UP}). 
At a high level, PRUCB-UP is an extension of the UCB approach \cite{auer2002finite} with two new components.

\textbf{Preference estimation.}
Capturing user preferences is a fundamental step toward preference adaptation. Due to the unknown preference expectation, we leverage the empirical average of historical preference feedback as the preference estimate. For $t \geq 1$, preference estimate is updated as
\vspace{-3pt}
\begin{equation}
\label{eq:PRUCB_SPM_c_t}
\textstyle
\hat{\boldsymbol{c}}_{t+1} = \big( (t-1) \boldsymbol{\hat{c}}_{t} + \boldsymbol{c}_{t} \big)/t.
\vspace{-3pt}
\end{equation}
Similarly, the reward estimate $\hat{\boldsymbol{r}}_{i,t}$ is defined as empirical estimation. For $t \in [1,T]$, it is updated as:
\vspace{-3pt}
\begin{equation}
\begin{aligned}
\textstyle
\label{eq:prucb_r_t}
N_{i, t+1} &= N_{i, t} + \mathds{1}_{ \{a_{t} = i \}}, \\
\hat{\boldsymbol{r}}_{i,t+1} & = (\hat{\boldsymbol{r}}_{i,t} N_{i, t} + \boldsymbol{r}_{a_{t}, t} \cdot \mathds{1}_{ \{a_{t} = i \}} )/N_{i, t+1},
\vspace{-5pt}
\end{aligned}
\end{equation}
with $N_{i, 1} \leftarrow 0, \boldsymbol{\hat{r}}_{i, 1} \leftarrow [0]^D, \forall i \in [K]$. 
% $N_{i,t}$ denotes the number of pulls of arm $i$ within the first $t-1$ rounds.


\textbf{Preference-aware optimization.}
To adapt the policy to the estimated preference, one might consider following the "optimism in the face of uncertainty" principle \cite{auer2002finite} by constructing confidence intervals on both preference and reward estimates, and incorporating them into the aggregation function $\Phi$ for optimization.

% Due to the deviation of preference estimate from the true expectation, an intuitive approach could involve constructing a confidence region $\Theta_t$ for $\hat{\boldsymbol{c}}_t$, similar to how the UCB method handles reward estimates $\hat{\boldsymbol{r}}_t$. 
% The solution would then be to choose the pair $( \hat{\boldsymbol{c}}_t^{\prime}, a_t) \in \Theta_t \times [K]$ that jointly maximizes the upper-confidence bound of the overall-reward.

However, in this case, we claim that despite the deviation from the true expectation, \emph{a confidence term for the preference estimate $\hat{\boldsymbol{c}}_t$ is unnecessary} 
The fundamental reason is that the preference estimation does not involve \emph{sequential action decision-making} component, where the preference feedback ${\boldsymbol{c}}_t$ observed is independent of the chosen action $a_t$.
Consequently, a bonus term for exploration is not required.
In contrast, for reward estimation, the action $a_t$ determined by $\hat{\boldsymbol{r}}_t$ will also influence the future estimate $\hat{\boldsymbol{r}}_{t+1}$, making a confidence term on $\hat{\boldsymbol{r}}_t$ is necessary to encourage exploration.

\vspace{-2mm}
\SetAlgoLined
\DontPrintSemicolon
\begin{algorithm}[]
\caption{\method: API Generation}
\label{algo:api}
\SetKwData{Data}{Data}
\SetKw{Continue}{continue}
\SetKwInOut{KwInit}{Initialize}
\SetKwProg{Fn}{Function}{}{end}

\KwData{Questions $\mathcal{Q}$}
$\mathcal{S} \gets \{\} $\tcp*[r]{Signatures}
$\mathcal{A} \gets \{\text{Vision Models}\}$\tcp*[r]{API Methods}
\For{batch $B \subset \mathcal{Q}$} {
    $\mathcal{S} \gets \mathcal{S} \cup  \texttt{SignatureAgent}(B)$
}
\For{$S \in \mathcal{S}$} {
    $e_S \gets 0$ \tcp*[r]{Error count}
    $A \gets \texttt{ImplementationAgent}(S)$ \;
    $E \gets \texttt{TestAgent}(A)$\;
    \eIf{Python Exception $E$}{
        \lIf{$e_S = 5$}{\Continue}
        \uElseIf{$E$ is ``undefined method $U$''}{
            $e_S \gets e_S + 1$\;
            Recursively implement $U$
        }
        \Else{
            $e_S \gets e_S + 1$\;
            Re-implement $S$ using $E$
        }
    } {
    $\mathcal{A} \gets \mathcal{A} \cup A$
    }
}
\Return{$\mathcal{A}$}
\end{algorithm}
\vspace{-3mm}


Building upon this, the arm selection policy is designed as:
\vspace{-5pt}
\begin{equation}
\label{eq:prucb_spm_at}
\textstyle
a_t = \argmaxA_{i \in [K]} \Phi ( \boldsymbol{\hat{c}}_t,  \hat{\boldsymbol{r}}_{i,t} + \rho_{i,t}^{\alpha} \boldsymbol{e} ).
\vspace{-2pt}
\end{equation}
where $\Phi$ is the inner product aggregation function, $\rho_{i,t}^{\alpha} = \sqrt{ \log(t/\alpha) / \max \{1, N_{i, t } \} }$ is the standard Hoeffding bonus term.
We characterize the regret of PRUCB-UP below.

\begin{theorem}
\label{theorem:up_bd_stat}
Assume the preference follows unknown distribution with the value being revealed after each arm pull. 
Let $\eta_{i} = \boldsymbol{\overline{c}}^{T} \Delta_{i}$, 
$\Delta_{i} = \mu_{a^{*}} - \mu_{i} \in \mathbb{R}^D$,
Algorithm \ref{alg:PRUCB_UP} has
\[
R(T) 
=
O
\Bigg(
\sum_{i \neq a^{*}}
\Big(
\underbrace{
\textcolor{cyan}{
\frac{\delta^2 \log T}{\eta_{i}} 
+
D \pi^2 \alpha^2 \eta_{i}
}}_{R^{r}_T}
 +
\underbrace{
\textcolor{orange}{
\frac{ D^2 \Vert \Delta_{i} \Vert_2^2 \delta^2 }{ \eta_{i} }
}}_{R^{c}_T}
\Big)
\Bigg),
\]
where $R^{r}_T$ and $R^{c}_T$ refer to the regrets caused by reward estimate error and preference estimate error respectively.
\end{theorem}


\begin{remark}
Theorem \ref{theorem:up_bd_stat} shows that with preference feedback, PRUCB-UP achieves a regret of $O(K \delta \log T + K \delta D^2)$, demonstrating near-optimal performance. Notably, the regret caused by additional preference estimation error is bounded by a constant related to objective dimension $D$ and preference $\ell_1$-norm bound $\delta$. 
This implies that the impact of preference estimation error on the regret is small.

PRUCB-UP also admits an instance-independent regret bound. Due to space limit, we defer it to Appendix \ref{sec:app_unknown_inst_indep_bound}.
Additional, we show that the preference known case can be solved by PRUCB-UP as a special case, achieving a regret of $O(K \delta \log T)$, please see Appendix \ref{sec:app_up_bd_stat_known} for details. 

% The regret term by preference estimation errors is controlled by a constant related to objective dimension $D$ and $\ell_1$-norm bound $\delta$ of preferences.
% Theorem \ref{theorem:up_bd_stat} implies that the PRUCB-SPM without any prior knowledge of user preferences involves the upper-bounded regret of $\mathcal{O} \left( \delta \log T \right)$.
% Notably, the dominant term of regret achieves very close performance ($ (1 + 1/\sqrt{D})^2 $ times worse) with that in known-case. For regret term induced by preferences estimation error, it can be controlled by constant, which is related to the dimension $D$ and $L_1$-norm bound $\delta$ of preferences.
\end{remark}


To prove Theorem \ref{theorem:up_bd_stat}, the main difficulty lies in decoupling and capturing the effect of the joint error from both reward estimation and preference estimation on the final regret. 
To address this, we introduce a tunable parameter $\epsilon$ to quantify the accuracy of preference estimation $\boldsymbol{\hat{c}}_t$, and decompose suboptimal actions into two disjoint sets: 
(1) suboptimal pulls under sufficiently precise preference estimation and (2) suboptimal pulls under imprecise preference estimation.
For set (1), we demonstrate that it can be transferred to a preference-known instance with a diminished overall-reward gap to the optimal arm. For set (2), we transfer the original set with joint error to a preference estimation deviation event using Lemma \ref{lemma: error_distance} (Appendix \ref{sec:app_pr_up_bd_stat}), making it more tractable. 
% accounting for two regret terms of $R^r_T$ and $R^c_T$. 
% The derivation of $R^r_T$ relies on Proposition \ref{prop: N_known_changing} in Appendix \ref{sec:app_pr_up_bd_stat}, which characterizes the policy behavior under accurate preference estimation updates. The derivation of $R^c_T$ relies on Lemma \ref{lemma: error_distance} in Appendix \ref{sec:app_pr_up_bd_stat} to 
% transfer the original set with joint error to a preference estimation deviation event, making it more tractable. 
Please refer to Appendix
\ref{sec:app_up_bd_stat} for the full proof.
\section{The Case with Hidden Preference}
\label{sec: hidden}

\begin{figure}[t]
% \vskip -0.15in
\begin{center}
\centerline{\includegraphics[width=0.55\columnwidth]{figures/s3.pdf}}
\end{center}
\vskip -0.2in
\caption{A scenario of user's preference feedback is not explicitly provided.}
\label{fig: s3}
\end{figure}

Next, we consider another practical scenario where only feedback on the reward and overall reward is observable, while preference feedback is not provided. For instance, in hotel surveys, customers often provide ratings on specific objectives (e.g., price, location, environment, amenities) along with an overall rating (as depicted in Fig. \ref{fig: s3}). In such cases, user preferences can be inferred from the latent relationship between the overall rating and the individual objective ratings.
Formally, in each round $t$, the learner selects an arm $a_t \!\in\! [K]$, and observes the reward vector $\boldsymbol{r}_{a_t} \!\in\! \mathbb{R}^{D}$, and the corresponding overall-reward score:
\begin{equation}
\label{eq:g_at}
\textstyle
g_{a_t,t} = \Phi(\boldsymbol{c}_t, \boldsymbol{r}_{a_t, t}) = \boldsymbol{c}_t^{\top} \boldsymbol{r}_{a_t, t} \in \mathbb{R}
\end{equation}
Within this framework, we adhere to the original Assumption~\ref{assmp: all_1} regarding rewards. It is worth noting that, in many real-world applications like hotel rating systems, the overall rating often shares the same scale as individual objective ratings.  
Therefore, we assume in this problem that the bound on the overall reward is identical to that of the individual rewards. This introduces one additional assumption and one revised assumption, as outlined below:
\begin{assumption}
\label{assmp: hpm_2}
For $t \in [T]$, $a_t \in [K]$, the overall-reward score satisfies 
$g_{a_t,t} \in [0,1]$.
\end{assumption}

\begin{assumption}
\label{assmp: hpm_3}
For $t \in [T]$, the stochastic preference is bounded and satisfies $\Vert \boldsymbol{c}_t \Vert_1 \leq 1$. Without loss of generality, we assume $\boldsymbol{c}_t(d)$ is $R$-sub-Gaussian\footnote{By Hoeffding’s lemma, for any $X \in [a,b]$ almost surely, $X$ is a $R$-sub-Gaussian random variable with $R$ at most $(b-a)/2$.}, $\forall d \in [D]$.
\end{assumption}


% \begin{assumption}
% \label{assmp: hpm}
% The PAMO-MAB with hidden preference satisfies the following conditions:
% \begin{itemize}[leftmargin=*]
% \item 
% For $t \in [T]$, $a_t \in [K]$, the overall-reward score satisfies 
% $g_{a_t,t} \in [0,1]$.
% \item 
% For $t \in [T]$, the stochastic preference satisfies $\Vert \boldsymbol{c}_t \Vert_1 \leq 1$ with 
% $\boldsymbol{c}_t(d) \in [0,1], \forall d \!\in\! [D]$.
% \end{itemize}
% \end{assumption}



\subsection{Unique Challenges}
\label{sec:uniq_chllenge}
This problem introduces two unique challenges that distinguish us from previous MAB studies:

\textbf{Local Exploration \emph{vs} Global Exploration.}
Unlike traditional bandit algorithms that focus on a single goal (e.g., identifying the arm with the highest reward), the uncertainty in both preference and reward, combined with the need to infer latent preference, introduces a novel trade-off challenge: balancing \emph{global exploration} for better preference estimation and \emph{local exploration} of arm rewards:
\begin{itemize}[leftmargin=*]
\item \emph{Global exploration for preferences:}
% selecting arms that diversify $r_{a_t}$ to refine feature space for preference learning.
Selecting arms that reduce uncertainty in poorly explored direction of the feature space, refining the model for preference learning.
% Selecting diverse arms to gather information about the relationship between $\boldsymbol{r}_{a_t}$ and $g_{a_t}$, reducing uncertainty across less-explored directions in feature space to refine preference learning.
\item \emph{Local exploration for rewards:}
% Selecting arms that balance exploring their rewards and exploiting empirically good arms.
Selecting arms to reduce uncertainty for specific individual arm reward estimate,  while balancing exploiting empirically high reward arms.
\end{itemize}


\begin{figure}[t]
% \vskip 0.2in
\begin{center}
\centerline{\includegraphics[width=0.65\columnwidth]{figures/lr_exp_1.pdf}}
\end{center}
\vskip -0.1in
\caption{A 2-dimensional hidden preference PAMO-MAB toy example with mean preference $\overline{\boldsymbol{c}} = [0.5, 0.5]$, illustrating preference estimate $\hat{\boldsymbol{c}}$ via linear regression using reward data from (a) Arm-1 (dominated mean reward: $[0.2,0.2]$) and (b) Arm-2 (Pareto-optimal mean reward: $[0.8,0.8]$).
}
\label{fig: lr_1}
\end{figure}

Note that these two learning objectives may conflict, as arms with high rewards might lack sufficient information for latent preference learning and could even degrade estimation performance (further discussed in the second challenge). 
This can also be verified by Fig. \ref{fig: lr_1}, where 80 samples of $[\boldsymbol{r}_{t}, g_{t}]$ are collected by repeatedly pulling an arm, and preference $\boldsymbol{\hat{c}}$ is estimated using linear regression. Here, $\boldsymbol{c}_t$ at each step follows a Gaussian distribution with a mean of $[0.5, 0.5]$. The results demonstrate that samples from suboptimal Arm-1 (Fig. \ref{fig: lr_1}a) significantly outperform those from Pareto-optimal Arm-2 (Fig. \ref{fig: lr_1}b) in preference estimation.
This necessitates an exploration policy that effectively addresses both global and local learning objectives.

\textbf{Random Mapping from $\boldsymbol{r}_t$ to $g_t$.}
In the hidden preference case, the observed overall rewards are generated through a \emph{random mapping} of rewards.
Specifically,
$g_{a_t,t} = (\boldsymbol{\overline{c}} + \boldsymbol{\zeta}_t)^{\top} \boldsymbol{r}_{a_t,t} = \boldsymbol{\overline{c}}^{\top} \boldsymbol{r}_{a_t,t} + \boldsymbol{\zeta}_t^{\top} \boldsymbol{r}_{a_t,t}$, where $\boldsymbol{\zeta}_t = \boldsymbol{c}_t - \boldsymbol{\overline{c}} \in \mathbb{R}^D$ is an independent random noise vector. 
This formulation implies that the overall residual noise term $\zeta_{g,t} = \boldsymbol{\zeta}_t^{\top} \boldsymbol{r}_{a_t,t}$ is no longer independent of the input.
Consequently, standard regression models become infeasible for preference estimation, as they rely on the assumption that the residual noise in the output is independent of the input.

Additionally, the magnitude of overall residual noise is a monotonically non-decreasing function w.r.t each reward objective, i.e., 
$\Vert \zeta_{g,t}(\boldsymbol{r}_{i}) \Vert \leq \Vert \zeta_{g,t}(\boldsymbol{r}_{j}) \Vert$ iff $\boldsymbol{r}_{i} \preceq \boldsymbol{r}_{j}$.
This explains why suboptimal arms typically outperform Pareto-optimal arms for preference estimation in Fig. \ref{fig: lr_1}, as selecting profitable arms tends to amplify the residual error, thereby degrading preference learning.
Thus, a tailored latent preference estimator is essential to mitigate the expanding error w.r.t the reward and ensure effective preference learning.



\subsection{Our Algorithm}
To this end, we propose a novel PRUCB-HP method (Algorithm \ref{alg:PRUCB_HP}) involving two key designs as follows.

\textbf{Key design I: WLS-Preference Estimator.}
As we have seen before, the randomness of preference $\boldsymbol{c}_t$ leads to the overall residual noise $\zeta_{g,t}$ be a function w.r.t, input reward $\boldsymbol{r}_t$. Moreover, larger input rewards $\boldsymbol{r}_t$ result in greater corruption from the residual noise.
To resolve this, we employ a weighted least-squares (WLS) estimator for preference learning. Specifically, our algorithm assigns a weight $w_t$ to each observed sample and estimates the unknown preference using weighted ridge regression:
\[
\textstyle
\hat{\boldsymbol{c}}_t \xleftarrow{} \argminA_{\boldsymbol{c} \in \mathbb{R}^{D}} \lambda \Vert \boldsymbol{c} \Vert_2^2 
+
\sum_{\ell=1}^{t-1} w_{\ell} ( {\boldsymbol{c}}^{\top} \boldsymbol{r}_{a_{\ell},\ell} - g_{a_{\ell},\ell})^2,
\]
where $\lambda$ is the regularization parameter. Above optimization problem has a closed-form solution as:
\begin{equation}
\label{eq:hpm_c_t}
\textstyle
\boldsymbol{\hat{c}}_{t} = \boldsymbol{V}_{t-1}^{-1} \sum_{\ell=1}^{t-1} w_{\ell} g_{a_{\ell}, \ell} \boldsymbol{r}_{a_{\ell},\ell},
% , \text{and } \boldsymbol{V}_{1} = \lambda \boldsymbol{I}
\end{equation}
where the Gram matrix $\boldsymbol{V}_{t-1} = \lambda \boldsymbol{I} + \sum_{\ell=1}^{t-1} w_{\ell} \boldsymbol{r}_{a_{\ell},\ell} \boldsymbol{r}_{a_{\ell},\ell}^{\top}$.

\begin{algorithm}[t]
\caption{Preference UCB with Hidden Preference (PRUCB-HP)}
\label{alg:PRUCB_HP}
\begin{algorithmic}
\State \textbf{Parameters:}
$\alpha$, $\lambda$, $\beta_t$, $\omega$.
\State \textbf{Initialization:}
$\boldsymbol{\hat{r}}_{i,1} \!\leftarrow\! [0]^D, 
N_{i, 1} \!\leftarrow\! 0, \forall i \!\in\! [K]$, 
$\boldsymbol{\hat{c}}_1 \!\leftarrow\! [\frac{1}{D}]^D$, 
$\boldsymbol{V}_0 \!\leftarrow\! \lambda \boldsymbol{I}$.
\For{$t=1, \cdots, T$}
% \vspace{-5pt}
    \State Compute reward bonus term: $B_{i,t}^{r} = \Vert \hat{\boldsymbol{c}}_t \Vert_1 \sqrt{\frac{\log{t/\alpha}}{\max\{N_{i,t}, 1\}}}, \forall i \in [K]$.
    \State Compute pseudo information gain term: $B_{i,t}^{c} = \beta_t \left \Vert \hat{\boldsymbol{r}}_{i,t} \!+\! \sqrt{\frac{\log{t/\alpha}}{\max\{N_{i,t}, 1\}}} \boldsymbol{e} \right \Vert_{\boldsymbol{V}_{t-1}^{-1}}, \forall i \in [K]$.
    \State \textbf{Draw arm} $a_t \leftarrow \argmaxA_{i \in [K]} \hat{\boldsymbol{c}}_t^\top \hat{\boldsymbol{r}}_{i,t} + B_{i,t}^{r} + B_{i,t}^{c}$.
    \State \textbf{Observe} reward $\boldsymbol{r}_{a_t,t}$ and overall-reward $g_{a_t,t}$.
    \State \textbf{Updating phase:}
    % \Comment{(Preference-aware optimization)}
    \State \indent $N_{i, t+1} \!=\! N_{i, t} \!+\! \mathds{1}_{ \{a_{t} = i \}}$.
    \State \indent $\hat{\boldsymbol{r}}_{i,t+1} \!=\! (\hat{\boldsymbol{r}}_{i,t} N_{i, t} + \boldsymbol{r}_{a_{t}, t} \cdot \mathds{1}_{ \{a_{t} = i \}} )/N_{i, t+1}, \forall i \in [K]$.
    \State \indent $w_t = \frac{\omega}{\Vert \boldsymbol{r}_{a_t,t} \Vert_2^2}$, $\boldsymbol{V}_{t} = \boldsymbol{V}_{t-1} + w_t \boldsymbol{r}_{a_t,t}\boldsymbol{r}_{a_t,t}^{\top}$.
    \State \indent $\boldsymbol{\hat{c}}_{t+1} = \boldsymbol{V}_{t}^{-1} \sum_{\ell=1}^{t} w_{\ell} g_{a_{\ell}, \ell} \boldsymbol{r}_{a_{\ell},\ell}$.
    % \Comment{(Reward estimation)}    
\EndFor
\end{algorithmic}
\end{algorithm}


% \begin{algorithm}[t]
% \caption{Preference UCB with Unknown Preference (PRUCB-UP)}
% \label{alg:PRUCB_HP}
% \begin{algorithmic}
% \State \textbf{Parameters:}
% $\alpha$, $\lambda$, $\beta_t$, $\omega$.
% \State \textbf{Initialization:}
% $\boldsymbol{\hat{r}}_{i,1} \!\leftarrow\! [0]^D, 
% N_{i, 1} \!\leftarrow\! 0, \forall i \!\in\! [K]$, 
% $\boldsymbol{\hat{c}}_1 \!\leftarrow\! [\frac{1}{D}]^D$, 
% $\boldsymbol{V}_0 \!\leftarrow\! \lambda \boldsymbol{I}$.
% \FOR{$t=1, \cdots, T$}
% % \vspace{-5pt}
%     \State Compute reward bonus term:\\ $B_{i,t}^{r} = \Vert \hat{\boldsymbol{c}}_t \Vert_1 \sqrt{\frac{\log{t/\alpha}}{\max\{N_{i,t, 1\}}}}, \forall i \in [K]$.
%     \State Compute pseudo information gain term:\\ $B_{i,t}^{c} = \beta_t \left \Vert \hat{\boldsymbol{r}}_{i,t} \!+\! \sqrt{\frac{\log{t/\alpha}}{\max\{N_{i,t, 1\}}}} \boldsymbol{e} \right \Vert_{\boldsymbol{V}_{t-1}^{-1}}, \forall i \in [K]$.
%     \State \textbf{Draw arm} $a_t \leftarrow \argmaxA_{i \in [K]} \hat{\boldsymbol{c}}_t^\top \hat{\boldsymbol{r}}_{i,t} + B_{i,t}^{r} + B_{i,t}^{c}$.
%     \State \textbf{Observe} reward $\boldsymbol{r}_{a_t,t}$ and overall-reward $g_{a_t,t}$.
%     \State \textbf{Updating phase:}
%     % \Comment{(Preference-aware optimization)}
%     \State \indent $N_{i, t+1} \!=\! N_{i, t} \!+\! \mathds{1}_{ \{a_{t} = i \}}$.
%     \State \indent $\hat{\boldsymbol{r}}_{i,t+1} \!=\! (\hat{\boldsymbol{r}}_{i,t} N_{i, t} + \boldsymbol{r}_{a_{t}, t} \cdot \mathds{1}_{ \{a_{t} = i \}} )/N_{i, t+1}, \forall i \in [K]$.
%     \State \indent $w_t = \frac{\omega}{\Vert \boldsymbol{r}_{a_t,t} \Vert_2^2}$, $\boldsymbol{V}_{t} = \boldsymbol{V}_{t-1} + w_t \boldsymbol{r}_{a_t,t}\boldsymbol{r}_{a_t,t}^{\top}$.
%     \State \indent $\boldsymbol{\hat{c}}_{t+1} = \boldsymbol{V}_{t+1}^{-1} \sum_{\ell=1}^{t} w_{\ell} g_{a_{\ell}, \ell} \boldsymbol{r}_{a_{\ell},\ell}$.
%     % \Comment{(Reward estimation)}    
% \ENDFOR
% \end{algorithmic}
% \end{algorithm}



Inspired by \cite{zhou2021nearly} using the inverse of the noise variance as weight 
% to normalize the noise 
for tight variance-dependent regret guarantee, 
% To this end, 
we define the weight as the inverse of the squared $\ell_2$-norm of the reward: $w_t = \omega / \Vert \boldsymbol{r}_{a_t,t} \Vert_2^2$, where $\omega > 0$ is a threshold parameter guaranteeing $w_t \geq 1$. 
Intuitively, it ensures samples with high rewards will be assigned smaller weights to reduce the influence of potentially large residual noises, while samples with low rewards receive larger weights to ensure their contribution to the estimation.

To see how our choice of weight can tackle the \emph{random mapping} issue, we first define 
$\boldsymbol{r}_{a_t,t}^{\prime} = \sqrt{w_t} \boldsymbol{r}_{a_t,t}$, 
$g_{a_t,t}^{\prime} = \sqrt{w_t} g_{a_t,t}$, then the original formula Eq. \ref{eq:g_at} can be rewrite as
\[
\textstyle
\sqrt{w_t} \cdot g_{a_t,t} = \sqrt{w_t} \cdot \boldsymbol{c}_t^{\top} \boldsymbol{r}_{a_t, t} = \sqrt{w_t} \cdot (\boldsymbol{\overline{c}} + \boldsymbol{\zeta_t})^{\top} \boldsymbol{r}_{a_t, t}
\]
\begin{equation}
\label{eq:g_t_2}
\textstyle
\implies
g_{a_t,t}^{\prime} = \boldsymbol{\overline{c}}^{\top} \boldsymbol{r}_{a_t, t}^{\prime} + \sqrt{w_t} \cdot \boldsymbol{\zeta}_{t}^{\top} \boldsymbol{r}_{a_{t}, t}.
\end{equation}
For term $\sqrt{w_t} \boldsymbol{\zeta}_{t}^{\top} \boldsymbol{r}_{a_{t}, t}$, we have the following lemma:
\begin{lemma}
\label{lemma: R_normed}
Under Assumption \ref{assmp: hpm_3}, the random variable $\sqrt{w_t} \boldsymbol{\zeta}_{t}^{\top} \boldsymbol{r}_{a_{t}, t}$ is sub-Gaussian with constant $R^{\prime} = \sqrt{\omega} R$.
\end{lemma}
The proof is available in Appendix \ref{sec: app_pf_lemma_R_normed}. By above lemma, we observe that with the designed weight, the original random mapping regression problem is transferred into a new formula as (\ref{eq:g_t_2}). Specifically, the output $g_{a_t,t}^{\prime}$ is mapped from $\boldsymbol{r}_{a_t,t}^{\prime}$ via a \emph{fixed} vector $\boldsymbol{\overline{c}}$ with a normed $R^{\prime}$-sub-Gaussian residual noise, where $R^{\prime} = \sqrt{\omega} R$, independent of the input. 


% While weighted ridge regression is not new and has been used in prior work on bandits \cite{russac2019weighted, zhou2021nearly, he2022nearly}, our setting, motivation and weight design are fundamentally different.

% Assign weights to each sample inversely proportional to the noise variance, which is proportional to the 2-norm of samples $\Vert \boldsymbol{r} \Vert_2^2$.

\textbf{Key design II: Dual-Exploration Policy.}
As discussed earlier, there is a new global-local exploration dilemma in our setting. 
On the one hand, the algorithm must focus on local exploration by selecting optimistically profitable arms to discover better ones. Simultaneously, it must globally explore diverse arms to gather information about the relationship between $\boldsymbol{r}_t$ and $g_t$ for modeling the hidden preference.

To resolve this, we design an \emph{optimistic dual-exploration policy} by incorporating a \emph{preference-driven bonus} and \emph{reward-driven bonus} within the preference-aware optimization framework for trade-off.
The optimistic policy is defined as
\begin{equation}
\label{eq:a_t_hidden}
\textstyle
a_t \leftarrow \argmaxA_{i \in [K]} \Phi ( \boldsymbol{\hat{c}}_t,  \hat{\boldsymbol{r}}_{i,t}) + B_{i,t}^{r} + B_{i,t}^{c},
\end{equation}
where $\Phi$ is the inner product aggregation function, $B_{i,t}^{r}$ and $B_{i,t}^{c}$ are the dual-exploration bonus terms. We detail the design of these bonus terms below and will later theoretically demonstrate in Section \ref{sec:theoretical_result_hidden} how they establish a tight UCB for the expected overall reward, ensuring the effectiveness of the optimistic policy in (\ref{eq:a_t_hidden}).

\underline{\emph{Reward Bonus}} $B_{i,t}^{r}$.
The reward bonus term explicitly encourages local exploration of arms with potentially high rewards, for the principle of optimism in face of uncertainty. Specifically, the bonus $B_{i,t}^{r}$ is formulated as a reward uncertainty-aware regularization term:
\begin{equation}
B_{i,t}^{r} = \rho_{i,t}^{\alpha} \Vert \hat{\boldsymbol{c}}_t \Vert_1.
\end{equation}
$\rho_{i,t}^{\alpha} = \sqrt{ \log(t/\alpha) / \max \{1, N_{i, t } \} }$ represents the standard Hoeffding bonus that quantifies the uncertainty in the reward estimates for each arm, ensuring that arms with higher uncertainty or lower exploration counts will be prioritized.

\underline{\emph{Preference Bonus}} $B_{i,t}^{c}$.
The preference bonus term aims to encourage the exploration of arms that reduce uncertainty in preference estimation.
In previous bandit studies \cite{abbasi2011improved, zhao2020simple, he2022nearly} involving linear coefficient ($\theta^*$) learning, it has been shown that $\beta \Vert \boldsymbol{x}_i \Vert_{\boldsymbol{V}^{-1}}$ provides a tight confidence bonus for the payoff of arm $i$, where $\beta$ is the confidence set radius for coefficient estimation, and $\boldsymbol{x}_i$ is the observable arm feature. In information theory, $\Vert \boldsymbol{x}_i \Vert_{\boldsymbol{V}^{-1}}$ also reflects entropy reduction in the model posterior, and is used to measure the estimator uncertainty improvement contributed by the chosen action $\boldsymbol{x_i}$ \cite{li2010contextual}.

However, such design is not feasible in our setting, as the exact reward $\boldsymbol{r}_{a_t,t}$ is revealed only after pulling the arm $a_t$, making the actual information gain $\Vert \boldsymbol{r}_{a_t,t} \Vert_{\boldsymbol{V}^{-1}_{t-1}}$ from arm $a_t$ unpredictable beforehand.
To resolve this problem, we introduce a \emph{pseudo information gain} term, defined as $\Vert \boldsymbol{\hat{r}}_{i,t} + \rho_{i,t}^{\alpha} \boldsymbol{e} \Vert_{\boldsymbol{V}^{-1}_{t-1}}$, where $\rho_{i,t}^{\alpha}$ is the standard Hoeffding bonus. And then the preference bonus is set as 
\begin{equation}
B_{i,t}^{r} = \beta_t \cdot \Vert \boldsymbol{\hat{r}}_{i,t} + \rho_{i,t}^{\alpha} \boldsymbol{e} \Vert_{\boldsymbol{V}^{-1}_{t-1}},
\end{equation}
where the scalar $\beta_t$ is the confidence radius of the preference estimation we will give in Lemma \ref{lemma:g_estimator_upper_conf_bd}.
Intuitively, this pseudo information gain term captures the potential improvement in the preference estimator that could be achieved by \emph{optimistically} selecting arm $i$ based on its reward estimation.
In this way, it explicitly encourages global exploration of arms that reduce uncertainty in preference estimation while performing local exploration.

\subsection{Theoretical Results}
\label{sec:theoretical_result_hidden}

In this section, we provide theoretical guarantees for the PRUCB-HP algorithm.
We first characterize the estimation error of $\boldsymbol{\hat{c}}_t$ w.r.t $\overline{\boldsymbol{c}}$ by WLS preference estimator below.

\begin{lemma}
\label{lemma:c_estimator_conf_bd}
Under Assumption \ref{assmp: hpm_3}, for any $0 < \alpha <1$, $\omega>0$, with a probability at least $1-\vartheta$, the preference estimator $\hat{c}_t$ in Algorithm \ref{alg:PRUCB_HP} verifies for all $t \in [1,T]$:
\[
\Vert \hat{\boldsymbol{c}}_t - \boldsymbol{\overline{c}} \Vert_{\boldsymbol{V}_{t-1}} 
\leq
R \sqrt{\omega D \log\big((1 + \omega t/\lambda)/\vartheta\big)} + \sqrt{\lambda}.
\]
\end{lemma}
Please see Appendix \ref{sec:app_proof_lemma_c_estimator_conf_bd} for the proof. This estimate confidence bound essentially implies the effectiveness of WLS preference estimator with the designed weight to handle the random-mapping issue in our problem. With this in hand, we can then derive an upper confidence bound for the expected overall reward.

\begin{lemma}
% [Upper Confidence Bound of Expected Overall Reward]
\label{lemma:g_estimator_upper_conf_bd}
Set $\beta_t = \sqrt{\omega D \log\big((1 + \omega t/\lambda)/\vartheta\big)} + \sqrt{\lambda}$, then for any $i \in [K]$ and any $t > 0$, with probability at least equal to $1 - \vartheta - D\alpha^2/t^2$, we have
\begin{equation}
\label{eq:ucb_hidden}
\boldsymbol{\overline{c}}^{\top} \boldsymbol{\mu}_{i} 
\leq
\boldsymbol{\hat{c}}_{t}^{\top} \boldsymbol{\hat{r}}_{i,t} 
+
B_{i,t}^{r}
+
B_{i,t}^{c},
\end{equation}
with 
$B_{i,t}^{r} = \rho_{i,t}^{\alpha} \Vert \hat{\boldsymbol{c}}_t \Vert_1$
and 
$B_{i,t}^{c} = \beta_t \left \Vert \hat{\boldsymbol{r}}_{i,t} \!+\! \rho_{i,t}^{\alpha} \boldsymbol{e} \right \Vert_{\boldsymbol{V}_{t-1}^{-1}}$
as the bonus terms for dual-exploration.
\end{lemma}

Please see Appendix \ref{sec:app_pf_lemma_g_estimator_upper_conf_bd} for the proof. Lemma \ref{lemma:g_estimator_upper_conf_bd} essentially suggests an upper confidence bound of the expected reward, which is adopted in our dual-exploration policy (\ref{eq:a_t_hidden}).
Notably, the two bonus terms strike a balance between local and global explorations while guaranteeing optimization under the principle of optimism in face of uncertainty.

% \label{eq:ucb_hidden}

% \begin{figure}[t]
% \vspace{-8pt}
% \vskip 0.2in
% \begin{center}
% \centerline{\includegraphics[width=\columnwidth]{figures/lr_exp_2.pdf}}
% \end{center}
% \vspace{-12pt}
% \caption{A scenario of user's preference feedback is not explicitly provided.}
% \label{fig: s3}
% \vskip -0.1in
% \end{figure}


\begin{theorem}
\label{theorem:up_bd_hiden}
For PAMO-MAB with hidden preference, for any $\lambda > 0$, by setting
$\alpha = \sqrt{\frac{12 \vartheta}{KD(D+3) \pi^2}}$,
$\omega \geq D$,
$\beta_t = \sqrt{\omega D \log\big((1 + \omega T/\lambda)/\alpha\big)} + \sqrt{\lambda}$, 
with probability greater than $1- 2\vartheta$, Algorithm \ref{alg:PRUCB_HP} has
\[
\small
\begin{aligned}
\textstyle
R(T) 
=
& O\Big(
\textcolor{cyan}{
DR \sqrt{ \omega T \log^{2} \Big( \big(1 + \omega T/\lambda\big)/\vartheta \Big)}
}
+ 
\textcolor{teal}{
\frac{DR}{\sqrt{\lambda}} \sqrt{ \omega DK T \log^2 \left(\big( 1 + \omega T/\lambda \big)/\vartheta) \right)}
}
+ 
\textcolor{orange}{
\sqrt{ KT \log \left( T/\vartheta \right)}
}
+
M
\Big),
\end{aligned}
\]
\begin{small}
with $M = \left\lfloor \min \big \{ t^{\prime} \mid t  \sigma^2_{r \downarrow} + \lambda \geq 2D \omega \sqrt{Kt\log \frac{t}{\alpha} }, \forall t \geq t^{\prime} \big \} \right \rfloor$\footnote{Since $\sigma^2_{\boldsymbol{r} \downarrow} \!\in\! \mathbb{R}^{+}$, 
$\lim_{t \rightarrow \infty} 2D \omega \sqrt{Kt \log \frac{t}{\alpha} }/\big(\sigma^2_{r \downarrow}t\big) = \lim_{t \rightarrow \infty} C_1 \sqrt{ (\log t - C_2)/t} = 0$, as $\sqrt{\log t}$ grows much slowly compared to $\sqrt{t}$. Hence $M$ exists for sufficiently large $t^{\prime}$.}.
\end{small}
Here the first term represents regret from preference estimation error, the third from reward estimation error, and the second from the combined error of both.
\end{theorem} 

Please see Appendix \ref{sec: app_pf_thm_up_bd_hiden} for the proof. The key difficulty of the proof is to upper-bound the accumulative preference bonus $\sum_{t}B_{i,t}^{c}$. Specifically, we need to quantify the weighted $\ell_2$-norm of the empirical estimation $\boldsymbol{\hat{r}}_{a_t,t}$ with weighting matrix $\boldsymbol{V}_{t}$ constructed by the true reward $\boldsymbol{r}_{a_t,t}$ instead. 
This inconsistency renders the classical induction method \cite{abbasi2011improved} for deriving $\log (\frac{\det \boldsymbol{V}_{T}}{\det \boldsymbol{V}_{0}})$ infeasible for upper-bounding $\sum_{t} \Vert \boldsymbol{\hat{r}}_{a_t,t} \Vert_{\boldsymbol{V}_{t-1}^{-1}}^2$. 
To resolve this, we first transfer $\Vert \boldsymbol{\hat{r}}_{a_t,t} \Vert_{\boldsymbol{V}_{t-1}^{-1}}$ to $\Vert \boldsymbol{\mu}_{a_t} \Vert_{\boldsymbol{V}_{t-1}^{-1}}$. Then we show that for sufficiently large $t$, $a \Vert \boldsymbol{\mu}_{a_t} \Vert_{\mathbb{E}[\boldsymbol{V}_{t-1}]^{-1}}$ serves as an upper bound for $\Vert \boldsymbol{\mu}_{a_t} \Vert_{\boldsymbol{V}_{t-1}^{-1}}$ with constant $a > 1$ (see Lemma \ref{lemma: hidden_sum_reg_c_expectation_bd}).
This allows us to use $a \Vert \boldsymbol{\mu}_{a_t} \Vert_{\mathbb{E}[\boldsymbol{V}_{t-1}]^{-1}}$ as an upper-bound, where a new recursion relationship between $\mathbb{E}[\boldsymbol{V}_{t-1}]$ and $\boldsymbol{\mu}_{a_t}$ can be guaranteed, enabling us to bound $\log (\frac{\det \mathbb{E}[\boldsymbol{V}_{T}]}{\det \mathbb{E}[\boldsymbol{V}_{0}]})$ via induction, which can further be bounded by slightly modifying existing techniques in linear bandits.

\begin{remark}
Theorem \ref{theorem:up_bd_hiden} shows that, even without explicit preference feedback, PRUCB-HP achieves sub-linear regret through carefully designed mechanisms for preference adaptation.
In particular, for $t \geq M$, where $M$ is a constant independent of $T$, the regret asymptotically scales as $\tilde{O} ( D \sqrt{T} )$.
\emph{To the best of our knowledge, this is the first result characterizing the performance of PAMO-MAB with hidden preference in the literature.}
\end{remark}


% \begin{figure}[t]
% \vspace{-5pt}
% \vskip 0.2in
% \begin{center}
% \centerline{\includegraphics[width=1\columnwidth]{figures/expriments_main.pdf}}
% \end{center}
% \vspace{-12pt}
% \caption{(a) Regret comparison under unknown preference. (b) Regret comparison under hidden preference. 
% }
% \label{fig: exp1_sample}
% \vskip -0.1in
% \end{figure}


\begin{figure}[t]
\begin{center}
    \subfigure[Unknown preference case]{
        \includegraphics[width=0.35\columnwidth]{figures/expriments_main_1.pdf}
        \label{fig: exp1_sample_1}
    }
    % \hfill
    \subfigure[Hidden preference case]{
        \includegraphics[width=0.35\columnwidth]{figures/expriments_main_2.pdf}
        \label{fig: exp1_sample_1}
    }
    \caption{Regret comparison of our proposed PRUCB with other benchmarks under different preference environments, where our methods outperforms other methods significantly.}
\label{fig: exp1_sample}
\end{center}
\end{figure}


% \begin{figure}[t]
% \vspace{-5pt}
% \vskip 0.2in
% \begin{center}
% \centerline{\includegraphics[width=1\columnwidth]{figures/expriment_protocol.pdf}}
% \end{center}
% \vspace{-12pt}
% \caption{A 2-dimensional hidden preference PAMO-MAB toy example
% }
% \label{fig: exp1_sample}
% \vskip -0.1in
% \end{figure}

% \begin{figure*}[t]
%     \centering    
%     \includegraphics[width=1\textwidth]{figures/expriment_protocol.pdf}
%     \caption{(a) Users switching protocol for experimental evaluation of hidden preference and multi-objective reward modelings. (b) One real-world example of the experimental protocol.
% }
% \vspace{-10pt}
%     \label{fig:experiment3_protocol}
% \end{figure*}

\section{Numerical Analysis}
We evaluate the performance of PRUCB-UP and PRUCB-HP in unknown and hidden preference environments, respectively. The PAMO-MAB instance includes 
$K$ arms and $D$ objectives, with preferences and rewards following Gaussian distributions with randomly initialized means..
(Detailed settings refer to Appendix \ref{sec:app_exp_prucb_static}). 
For the hidden preference case, we introduce a user-switching protocol to simulate practical scenarios. The environment features multiple users, each exposed to a block of arms (5 in our setup) per round. Only arms within the current block can be selected for that user. In the next round, the arm block rotates to a another user. The objective is to maximize cumulative overall ratings across all users, and performance is measured by the averaged users' regrets. A more detailed illustration is provided in Appendix \ref{sec:app_exp_hidden}.

We compare our results with other baselines including S-UCB, Pareto-UCB \cite{drugan2013designing}, S-MOSS, Pareto-TS~\cite{yahyaa2015thompson}), UCB~\cite{auer2002finite}, MOSS~\cite{audibert2009minimax} and OFUL \cite{abbasi2011improved}.
The regret is averaged across 10 trials with round $T = 5000$.
Figure \ref{fig: exp1_sample} shows that our proposed algorithms significantly outperform other competitors under both environments.
It is worth noting that for all the preference-free competitors exhibit linear regret, aligning with Proposition \ref{prop: lower_bd}, demonstrating that approaches agnostic to user preferences cannot align their outputs with user preferences, even if they achieve Pareto optimality.
For more comprehensive experimental analyses, please refer to Appendix \ref{sec:app_exp}.

\section{Conclusion}
In this work, we propose a simple yet effective approach, called SMILE, for graph few-shot learning with fewer tasks. Specifically, we introduce a novel dual-level mixup strategy, including within-task and across-task mixup, for enriching the diversity of nodes within each task and the diversity of tasks. Also, we incorporate the degree-based prior information to learn expressive node embeddings. Theoretically, we prove that SMILE effectively enhances the model's generalization performance. Empirically, we conduct extensive experiments on multiple benchmarks and the results suggest that SMILE significantly outperforms other baselines, including both in-domain and cross-domain few-shot settings.



\bibliography{reference}
\bibliographystyle{plain}

\newpage
\subsection{Lloyd-Max Algorithm}
\label{subsec:Lloyd-Max}
For a given quantization bitwidth $B$ and an operand $\bm{X}$, the Lloyd-Max algorithm finds $2^B$ quantization levels $\{\hat{x}_i\}_{i=1}^{2^B}$ such that quantizing $\bm{X}$ by rounding each scalar in $\bm{X}$ to the nearest quantization level minimizes the quantization MSE. 

The algorithm starts with an initial guess of quantization levels and then iteratively computes quantization thresholds $\{\tau_i\}_{i=1}^{2^B-1}$ and updates quantization levels $\{\hat{x}_i\}_{i=1}^{2^B}$. Specifically, at iteration $n$, thresholds are set to the midpoints of the previous iteration's levels:
\begin{align*}
    \tau_i^{(n)}=\frac{\hat{x}_i^{(n-1)}+\hat{x}_{i+1}^{(n-1)}}2 \text{ for } i=1\ldots 2^B-1
\end{align*}
Subsequently, the quantization levels are re-computed as conditional means of the data regions defined by the new thresholds:
\begin{align*}
    \hat{x}_i^{(n)}=\mathbb{E}\left[ \bm{X} \big| \bm{X}\in [\tau_{i-1}^{(n)},\tau_i^{(n)}] \right] \text{ for } i=1\ldots 2^B
\end{align*}
where to satisfy boundary conditions we have $\tau_0=-\infty$ and $\tau_{2^B}=\infty$. The algorithm iterates the above steps until convergence.

Figure \ref{fig:lm_quant} compares the quantization levels of a $7$-bit floating point (E3M3) quantizer (left) to a $7$-bit Lloyd-Max quantizer (right) when quantizing a layer of weights from the GPT3-126M model at a per-tensor granularity. As shown, the Lloyd-Max quantizer achieves substantially lower quantization MSE. Further, Table \ref{tab:FP7_vs_LM7} shows the superior perplexity achieved by Lloyd-Max quantizers for bitwidths of $7$, $6$ and $5$. The difference between the quantizers is clear at 5 bits, where per-tensor FP quantization incurs a drastic and unacceptable increase in perplexity, while Lloyd-Max quantization incurs a much smaller increase. Nevertheless, we note that even the optimal Lloyd-Max quantizer incurs a notable ($\sim 1.5$) increase in perplexity due to the coarse granularity of quantization. 

\begin{figure}[h]
  \centering
  \includegraphics[width=0.7\linewidth]{sections/figures/LM7_FP7.pdf}
  \caption{\small Quantization levels and the corresponding quantization MSE of Floating Point (left) vs Lloyd-Max (right) Quantizers for a layer of weights in the GPT3-126M model.}
  \label{fig:lm_quant}
\end{figure}

\begin{table}[h]\scriptsize
\begin{center}
\caption{\label{tab:FP7_vs_LM7} \small Comparing perplexity (lower is better) achieved by floating point quantizers and Lloyd-Max quantizers on a GPT3-126M model for the Wikitext-103 dataset.}
\begin{tabular}{c|cc|c}
\hline
 \multirow{2}{*}{\textbf{Bitwidth}} & \multicolumn{2}{|c|}{\textbf{Floating-Point Quantizer}} & \textbf{Lloyd-Max Quantizer} \\
 & Best Format & Wikitext-103 Perplexity & Wikitext-103 Perplexity \\
\hline
7 & E3M3 & 18.32 & 18.27 \\
6 & E3M2 & 19.07 & 18.51 \\
5 & E4M0 & 43.89 & 19.71 \\
\hline
\end{tabular}
\end{center}
\end{table}

\subsection{Proof of Local Optimality of LO-BCQ}
\label{subsec:lobcq_opt_proof}
For a given block $\bm{b}_j$, the quantization MSE during LO-BCQ can be empirically evaluated as $\frac{1}{L_b}\lVert \bm{b}_j- \bm{\hat{b}}_j\rVert^2_2$ where $\bm{\hat{b}}_j$ is computed from equation (\ref{eq:clustered_quantization_definition}) as $C_{f(\bm{b}_j)}(\bm{b}_j)$. Further, for a given block cluster $\mathcal{B}_i$, we compute the quantization MSE as $\frac{1}{|\mathcal{B}_{i}|}\sum_{\bm{b} \in \mathcal{B}_{i}} \frac{1}{L_b}\lVert \bm{b}- C_i^{(n)}(\bm{b})\rVert^2_2$. Therefore, at the end of iteration $n$, we evaluate the overall quantization MSE $J^{(n)}$ for a given operand $\bm{X}$ composed of $N_c$ block clusters as:
\begin{align*}
    \label{eq:mse_iter_n}
    J^{(n)} = \frac{1}{N_c} \sum_{i=1}^{N_c} \frac{1}{|\mathcal{B}_{i}^{(n)}|}\sum_{\bm{v} \in \mathcal{B}_{i}^{(n)}} \frac{1}{L_b}\lVert \bm{b}- B_i^{(n)}(\bm{b})\rVert^2_2
\end{align*}

At the end of iteration $n$, the codebooks are updated from $\mathcal{C}^{(n-1)}$ to $\mathcal{C}^{(n)}$. However, the mapping of a given vector $\bm{b}_j$ to quantizers $\mathcal{C}^{(n)}$ remains as  $f^{(n)}(\bm{b}_j)$. At the next iteration, during the vector clustering step, $f^{(n+1)}(\bm{b}_j)$ finds new mapping of $\bm{b}_j$ to updated codebooks $\mathcal{C}^{(n)}$ such that the quantization MSE over the candidate codebooks is minimized. Therefore, we obtain the following result for $\bm{b}_j$:
\begin{align*}
\frac{1}{L_b}\lVert \bm{b}_j - C_{f^{(n+1)}(\bm{b}_j)}^{(n)}(\bm{b}_j)\rVert^2_2 \le \frac{1}{L_b}\lVert \bm{b}_j - C_{f^{(n)}(\bm{b}_j)}^{(n)}(\bm{b}_j)\rVert^2_2
\end{align*}

That is, quantizing $\bm{b}_j$ at the end of the block clustering step of iteration $n+1$ results in lower quantization MSE compared to quantizing at the end of iteration $n$. Since this is true for all $\bm{b} \in \bm{X}$, we assert the following:
\begin{equation}
\begin{split}
\label{eq:mse_ineq_1}
    \tilde{J}^{(n+1)} &= \frac{1}{N_c} \sum_{i=1}^{N_c} \frac{1}{|\mathcal{B}_{i}^{(n+1)}|}\sum_{\bm{b} \in \mathcal{B}_{i}^{(n+1)}} \frac{1}{L_b}\lVert \bm{b} - C_i^{(n)}(b)\rVert^2_2 \le J^{(n)}
\end{split}
\end{equation}
where $\tilde{J}^{(n+1)}$ is the the quantization MSE after the vector clustering step at iteration $n+1$.

Next, during the codebook update step (\ref{eq:quantizers_update}) at iteration $n+1$, the per-cluster codebooks $\mathcal{C}^{(n)}$ are updated to $\mathcal{C}^{(n+1)}$ by invoking the Lloyd-Max algorithm \citep{Lloyd}. We know that for any given value distribution, the Lloyd-Max algorithm minimizes the quantization MSE. Therefore, for a given vector cluster $\mathcal{B}_i$ we obtain the following result:

\begin{equation}
    \frac{1}{|\mathcal{B}_{i}^{(n+1)}|}\sum_{\bm{b} \in \mathcal{B}_{i}^{(n+1)}} \frac{1}{L_b}\lVert \bm{b}- C_i^{(n+1)}(\bm{b})\rVert^2_2 \le \frac{1}{|\mathcal{B}_{i}^{(n+1)}|}\sum_{\bm{b} \in \mathcal{B}_{i}^{(n+1)}} \frac{1}{L_b}\lVert \bm{b}- C_i^{(n)}(\bm{b})\rVert^2_2
\end{equation}

The above equation states that quantizing the given block cluster $\mathcal{B}_i$ after updating the associated codebook from $C_i^{(n)}$ to $C_i^{(n+1)}$ results in lower quantization MSE. Since this is true for all the block clusters, we derive the following result: 
\begin{equation}
\begin{split}
\label{eq:mse_ineq_2}
     J^{(n+1)} &= \frac{1}{N_c} \sum_{i=1}^{N_c} \frac{1}{|\mathcal{B}_{i}^{(n+1)}|}\sum_{\bm{b} \in \mathcal{B}_{i}^{(n+1)}} \frac{1}{L_b}\lVert \bm{b}- C_i^{(n+1)}(\bm{b})\rVert^2_2  \le \tilde{J}^{(n+1)}   
\end{split}
\end{equation}

Following (\ref{eq:mse_ineq_1}) and (\ref{eq:mse_ineq_2}), we find that the quantization MSE is non-increasing for each iteration, that is, $J^{(1)} \ge J^{(2)} \ge J^{(3)} \ge \ldots \ge J^{(M)}$ where $M$ is the maximum number of iterations. 
%Therefore, we can say that if the algorithm converges, then it must be that it has converged to a local minimum. 
\hfill $\blacksquare$


\begin{figure}
    \begin{center}
    \includegraphics[width=0.5\textwidth]{sections//figures/mse_vs_iter.pdf}
    \end{center}
    \caption{\small NMSE vs iterations during LO-BCQ compared to other block quantization proposals}
    \label{fig:nmse_vs_iter}
\end{figure}

Figure \ref{fig:nmse_vs_iter} shows the empirical convergence of LO-BCQ across several block lengths and number of codebooks. Also, the MSE achieved by LO-BCQ is compared to baselines such as MXFP and VSQ. As shown, LO-BCQ converges to a lower MSE than the baselines. Further, we achieve better convergence for larger number of codebooks ($N_c$) and for a smaller block length ($L_b$), both of which increase the bitwidth of BCQ (see Eq \ref{eq:bitwidth_bcq}).


\subsection{Additional Accuracy Results}
%Table \ref{tab:lobcq_config} lists the various LOBCQ configurations and their corresponding bitwidths.
\begin{table}
\setlength{\tabcolsep}{4.75pt}
\begin{center}
\caption{\label{tab:lobcq_config} Various LO-BCQ configurations and their bitwidths.}
\begin{tabular}{|c||c|c|c|c||c|c||c|} 
\hline
 & \multicolumn{4}{|c||}{$L_b=8$} & \multicolumn{2}{|c||}{$L_b=4$} & $L_b=2$ \\
 \hline
 \backslashbox{$L_A$\kern-1em}{\kern-1em$N_c$} & 2 & 4 & 8 & 16 & 2 & 4 & 2 \\
 \hline
 64 & 4.25 & 4.375 & 4.5 & 4.625 & 4.375 & 4.625 & 4.625\\
 \hline
 32 & 4.375 & 4.5 & 4.625& 4.75 & 4.5 & 4.75 & 4.75 \\
 \hline
 16 & 4.625 & 4.75& 4.875 & 5 & 4.75 & 5 & 5 \\
 \hline
\end{tabular}
\end{center}
\end{table}

%\subsection{Perplexity achieved by various LO-BCQ configurations on Wikitext-103 dataset}

\begin{table} \centering
\begin{tabular}{|c||c|c|c|c||c|c||c|} 
\hline
 $L_b \rightarrow$& \multicolumn{4}{c||}{8} & \multicolumn{2}{c||}{4} & 2\\
 \hline
 \backslashbox{$L_A$\kern-1em}{\kern-1em$N_c$} & 2 & 4 & 8 & 16 & 2 & 4 & 2  \\
 %$N_c \rightarrow$ & 2 & 4 & 8 & 16 & 2 & 4 & 2 \\
 \hline
 \hline
 \multicolumn{8}{c}{GPT3-1.3B (FP32 PPL = 9.98)} \\ 
 \hline
 \hline
 64 & 10.40 & 10.23 & 10.17 & 10.15 &  10.28 & 10.18 & 10.19 \\
 \hline
 32 & 10.25 & 10.20 & 10.15 & 10.12 &  10.23 & 10.17 & 10.17 \\
 \hline
 16 & 10.22 & 10.16 & 10.10 & 10.09 &  10.21 & 10.14 & 10.16 \\
 \hline
  \hline
 \multicolumn{8}{c}{GPT3-8B (FP32 PPL = 7.38)} \\ 
 \hline
 \hline
 64 & 7.61 & 7.52 & 7.48 &  7.47 &  7.55 &  7.49 & 7.50 \\
 \hline
 32 & 7.52 & 7.50 & 7.46 &  7.45 &  7.52 &  7.48 & 7.48  \\
 \hline
 16 & 7.51 & 7.48 & 7.44 &  7.44 &  7.51 &  7.49 & 7.47  \\
 \hline
\end{tabular}
\caption{\label{tab:ppl_gpt3_abalation} Wikitext-103 perplexity across GPT3-1.3B and 8B models.}
\end{table}

\begin{table} \centering
\begin{tabular}{|c||c|c|c|c||} 
\hline
 $L_b \rightarrow$& \multicolumn{4}{c||}{8}\\
 \hline
 \backslashbox{$L_A$\kern-1em}{\kern-1em$N_c$} & 2 & 4 & 8 & 16 \\
 %$N_c \rightarrow$ & 2 & 4 & 8 & 16 & 2 & 4 & 2 \\
 \hline
 \hline
 \multicolumn{5}{|c|}{Llama2-7B (FP32 PPL = 5.06)} \\ 
 \hline
 \hline
 64 & 5.31 & 5.26 & 5.19 & 5.18  \\
 \hline
 32 & 5.23 & 5.25 & 5.18 & 5.15  \\
 \hline
 16 & 5.23 & 5.19 & 5.16 & 5.14  \\
 \hline
 \multicolumn{5}{|c|}{Nemotron4-15B (FP32 PPL = 5.87)} \\ 
 \hline
 \hline
 64  & 6.3 & 6.20 & 6.13 & 6.08  \\
 \hline
 32  & 6.24 & 6.12 & 6.07 & 6.03  \\
 \hline
 16  & 6.12 & 6.14 & 6.04 & 6.02  \\
 \hline
 \multicolumn{5}{|c|}{Nemotron4-340B (FP32 PPL = 3.48)} \\ 
 \hline
 \hline
 64 & 3.67 & 3.62 & 3.60 & 3.59 \\
 \hline
 32 & 3.63 & 3.61 & 3.59 & 3.56 \\
 \hline
 16 & 3.61 & 3.58 & 3.57 & 3.55 \\
 \hline
\end{tabular}
\caption{\label{tab:ppl_llama7B_nemo15B} Wikitext-103 perplexity compared to FP32 baseline in Llama2-7B and Nemotron4-15B, 340B models}
\end{table}

%\subsection{Perplexity achieved by various LO-BCQ configurations on MMLU dataset}


\begin{table} \centering
\begin{tabular}{|c||c|c|c|c||c|c|c|c|} 
\hline
 $L_b \rightarrow$& \multicolumn{4}{c||}{8} & \multicolumn{4}{c||}{8}\\
 \hline
 \backslashbox{$L_A$\kern-1em}{\kern-1em$N_c$} & 2 & 4 & 8 & 16 & 2 & 4 & 8 & 16  \\
 %$N_c \rightarrow$ & 2 & 4 & 8 & 16 & 2 & 4 & 2 \\
 \hline
 \hline
 \multicolumn{5}{|c|}{Llama2-7B (FP32 Accuracy = 45.8\%)} & \multicolumn{4}{|c|}{Llama2-70B (FP32 Accuracy = 69.12\%)} \\ 
 \hline
 \hline
 64 & 43.9 & 43.4 & 43.9 & 44.9 & 68.07 & 68.27 & 68.17 & 68.75 \\
 \hline
 32 & 44.5 & 43.8 & 44.9 & 44.5 & 68.37 & 68.51 & 68.35 & 68.27  \\
 \hline
 16 & 43.9 & 42.7 & 44.9 & 45 & 68.12 & 68.77 & 68.31 & 68.59  \\
 \hline
 \hline
 \multicolumn{5}{|c|}{GPT3-22B (FP32 Accuracy = 38.75\%)} & \multicolumn{4}{|c|}{Nemotron4-15B (FP32 Accuracy = 64.3\%)} \\ 
 \hline
 \hline
 64 & 36.71 & 38.85 & 38.13 & 38.92 & 63.17 & 62.36 & 63.72 & 64.09 \\
 \hline
 32 & 37.95 & 38.69 & 39.45 & 38.34 & 64.05 & 62.30 & 63.8 & 64.33  \\
 \hline
 16 & 38.88 & 38.80 & 38.31 & 38.92 & 63.22 & 63.51 & 63.93 & 64.43  \\
 \hline
\end{tabular}
\caption{\label{tab:mmlu_abalation} Accuracy on MMLU dataset across GPT3-22B, Llama2-7B, 70B and Nemotron4-15B models.}
\end{table}


%\subsection{Perplexity achieved by various LO-BCQ configurations on LM evaluation harness}

\begin{table} \centering
\begin{tabular}{|c||c|c|c|c||c|c|c|c|} 
\hline
 $L_b \rightarrow$& \multicolumn{4}{c||}{8} & \multicolumn{4}{c||}{8}\\
 \hline
 \backslashbox{$L_A$\kern-1em}{\kern-1em$N_c$} & 2 & 4 & 8 & 16 & 2 & 4 & 8 & 16  \\
 %$N_c \rightarrow$ & 2 & 4 & 8 & 16 & 2 & 4 & 2 \\
 \hline
 \hline
 \multicolumn{5}{|c|}{Race (FP32 Accuracy = 37.51\%)} & \multicolumn{4}{|c|}{Boolq (FP32 Accuracy = 64.62\%)} \\ 
 \hline
 \hline
 64 & 36.94 & 37.13 & 36.27 & 37.13 & 63.73 & 62.26 & 63.49 & 63.36 \\
 \hline
 32 & 37.03 & 36.36 & 36.08 & 37.03 & 62.54 & 63.51 & 63.49 & 63.55  \\
 \hline
 16 & 37.03 & 37.03 & 36.46 & 37.03 & 61.1 & 63.79 & 63.58 & 63.33  \\
 \hline
 \hline
 \multicolumn{5}{|c|}{Winogrande (FP32 Accuracy = 58.01\%)} & \multicolumn{4}{|c|}{Piqa (FP32 Accuracy = 74.21\%)} \\ 
 \hline
 \hline
 64 & 58.17 & 57.22 & 57.85 & 58.33 & 73.01 & 73.07 & 73.07 & 72.80 \\
 \hline
 32 & 59.12 & 58.09 & 57.85 & 58.41 & 73.01 & 73.94 & 72.74 & 73.18  \\
 \hline
 16 & 57.93 & 58.88 & 57.93 & 58.56 & 73.94 & 72.80 & 73.01 & 73.94  \\
 \hline
\end{tabular}
\caption{\label{tab:mmlu_abalation} Accuracy on LM evaluation harness tasks on GPT3-1.3B model.}
\end{table}

\begin{table} \centering
\begin{tabular}{|c||c|c|c|c||c|c|c|c|} 
\hline
 $L_b \rightarrow$& \multicolumn{4}{c||}{8} & \multicolumn{4}{c||}{8}\\
 \hline
 \backslashbox{$L_A$\kern-1em}{\kern-1em$N_c$} & 2 & 4 & 8 & 16 & 2 & 4 & 8 & 16  \\
 %$N_c \rightarrow$ & 2 & 4 & 8 & 16 & 2 & 4 & 2 \\
 \hline
 \hline
 \multicolumn{5}{|c|}{Race (FP32 Accuracy = 41.34\%)} & \multicolumn{4}{|c|}{Boolq (FP32 Accuracy = 68.32\%)} \\ 
 \hline
 \hline
 64 & 40.48 & 40.10 & 39.43 & 39.90 & 69.20 & 68.41 & 69.45 & 68.56 \\
 \hline
 32 & 39.52 & 39.52 & 40.77 & 39.62 & 68.32 & 67.43 & 68.17 & 69.30  \\
 \hline
 16 & 39.81 & 39.71 & 39.90 & 40.38 & 68.10 & 66.33 & 69.51 & 69.42  \\
 \hline
 \hline
 \multicolumn{5}{|c|}{Winogrande (FP32 Accuracy = 67.88\%)} & \multicolumn{4}{|c|}{Piqa (FP32 Accuracy = 78.78\%)} \\ 
 \hline
 \hline
 64 & 66.85 & 66.61 & 67.72 & 67.88 & 77.31 & 77.42 & 77.75 & 77.64 \\
 \hline
 32 & 67.25 & 67.72 & 67.72 & 67.00 & 77.31 & 77.04 & 77.80 & 77.37  \\
 \hline
 16 & 68.11 & 68.90 & 67.88 & 67.48 & 77.37 & 78.13 & 78.13 & 77.69  \\
 \hline
\end{tabular}
\caption{\label{tab:mmlu_abalation} Accuracy on LM evaluation harness tasks on GPT3-8B model.}
\end{table}

\begin{table} \centering
\begin{tabular}{|c||c|c|c|c||c|c|c|c|} 
\hline
 $L_b \rightarrow$& \multicolumn{4}{c||}{8} & \multicolumn{4}{c||}{8}\\
 \hline
 \backslashbox{$L_A$\kern-1em}{\kern-1em$N_c$} & 2 & 4 & 8 & 16 & 2 & 4 & 8 & 16  \\
 %$N_c \rightarrow$ & 2 & 4 & 8 & 16 & 2 & 4 & 2 \\
 \hline
 \hline
 \multicolumn{5}{|c|}{Race (FP32 Accuracy = 40.67\%)} & \multicolumn{4}{|c|}{Boolq (FP32 Accuracy = 76.54\%)} \\ 
 \hline
 \hline
 64 & 40.48 & 40.10 & 39.43 & 39.90 & 75.41 & 75.11 & 77.09 & 75.66 \\
 \hline
 32 & 39.52 & 39.52 & 40.77 & 39.62 & 76.02 & 76.02 & 75.96 & 75.35  \\
 \hline
 16 & 39.81 & 39.71 & 39.90 & 40.38 & 75.05 & 73.82 & 75.72 & 76.09  \\
 \hline
 \hline
 \multicolumn{5}{|c|}{Winogrande (FP32 Accuracy = 70.64\%)} & \multicolumn{4}{|c|}{Piqa (FP32 Accuracy = 79.16\%)} \\ 
 \hline
 \hline
 64 & 69.14 & 70.17 & 70.17 & 70.56 & 78.24 & 79.00 & 78.62 & 78.73 \\
 \hline
 32 & 70.96 & 69.69 & 71.27 & 69.30 & 78.56 & 79.49 & 79.16 & 78.89  \\
 \hline
 16 & 71.03 & 69.53 & 69.69 & 70.40 & 78.13 & 79.16 & 79.00 & 79.00  \\
 \hline
\end{tabular}
\caption{\label{tab:mmlu_abalation} Accuracy on LM evaluation harness tasks on GPT3-22B model.}
\end{table}

\begin{table} \centering
\begin{tabular}{|c||c|c|c|c||c|c|c|c|} 
\hline
 $L_b \rightarrow$& \multicolumn{4}{c||}{8} & \multicolumn{4}{c||}{8}\\
 \hline
 \backslashbox{$L_A$\kern-1em}{\kern-1em$N_c$} & 2 & 4 & 8 & 16 & 2 & 4 & 8 & 16  \\
 %$N_c \rightarrow$ & 2 & 4 & 8 & 16 & 2 & 4 & 2 \\
 \hline
 \hline
 \multicolumn{5}{|c|}{Race (FP32 Accuracy = 44.4\%)} & \multicolumn{4}{|c|}{Boolq (FP32 Accuracy = 79.29\%)} \\ 
 \hline
 \hline
 64 & 42.49 & 42.51 & 42.58 & 43.45 & 77.58 & 77.37 & 77.43 & 78.1 \\
 \hline
 32 & 43.35 & 42.49 & 43.64 & 43.73 & 77.86 & 75.32 & 77.28 & 77.86  \\
 \hline
 16 & 44.21 & 44.21 & 43.64 & 42.97 & 78.65 & 77 & 76.94 & 77.98  \\
 \hline
 \hline
 \multicolumn{5}{|c|}{Winogrande (FP32 Accuracy = 69.38\%)} & \multicolumn{4}{|c|}{Piqa (FP32 Accuracy = 78.07\%)} \\ 
 \hline
 \hline
 64 & 68.9 & 68.43 & 69.77 & 68.19 & 77.09 & 76.82 & 77.09 & 77.86 \\
 \hline
 32 & 69.38 & 68.51 & 68.82 & 68.90 & 78.07 & 76.71 & 78.07 & 77.86  \\
 \hline
 16 & 69.53 & 67.09 & 69.38 & 68.90 & 77.37 & 77.8 & 77.91 & 77.69  \\
 \hline
\end{tabular}
\caption{\label{tab:mmlu_abalation} Accuracy on LM evaluation harness tasks on Llama2-7B model.}
\end{table}

\begin{table} \centering
\begin{tabular}{|c||c|c|c|c||c|c|c|c|} 
\hline
 $L_b \rightarrow$& \multicolumn{4}{c||}{8} & \multicolumn{4}{c||}{8}\\
 \hline
 \backslashbox{$L_A$\kern-1em}{\kern-1em$N_c$} & 2 & 4 & 8 & 16 & 2 & 4 & 8 & 16  \\
 %$N_c \rightarrow$ & 2 & 4 & 8 & 16 & 2 & 4 & 2 \\
 \hline
 \hline
 \multicolumn{5}{|c|}{Race (FP32 Accuracy = 48.8\%)} & \multicolumn{4}{|c|}{Boolq (FP32 Accuracy = 85.23\%)} \\ 
 \hline
 \hline
 64 & 49.00 & 49.00 & 49.28 & 48.71 & 82.82 & 84.28 & 84.03 & 84.25 \\
 \hline
 32 & 49.57 & 48.52 & 48.33 & 49.28 & 83.85 & 84.46 & 84.31 & 84.93  \\
 \hline
 16 & 49.85 & 49.09 & 49.28 & 48.99 & 85.11 & 84.46 & 84.61 & 83.94  \\
 \hline
 \hline
 \multicolumn{5}{|c|}{Winogrande (FP32 Accuracy = 79.95\%)} & \multicolumn{4}{|c|}{Piqa (FP32 Accuracy = 81.56\%)} \\ 
 \hline
 \hline
 64 & 78.77 & 78.45 & 78.37 & 79.16 & 81.45 & 80.69 & 81.45 & 81.5 \\
 \hline
 32 & 78.45 & 79.01 & 78.69 & 80.66 & 81.56 & 80.58 & 81.18 & 81.34  \\
 \hline
 16 & 79.95 & 79.56 & 79.79 & 79.72 & 81.28 & 81.66 & 81.28 & 80.96  \\
 \hline
\end{tabular}
\caption{\label{tab:mmlu_abalation} Accuracy on LM evaluation harness tasks on Llama2-70B model.}
\end{table}

%\section{MSE Studies}
%\textcolor{red}{TODO}


\subsection{Number Formats and Quantization Method}
\label{subsec:numFormats_quantMethod}
\subsubsection{Integer Format}
An $n$-bit signed integer (INT) is typically represented with a 2s-complement format \citep{yao2022zeroquant,xiao2023smoothquant,dai2021vsq}, where the most significant bit denotes the sign.

\subsubsection{Floating Point Format}
An $n$-bit signed floating point (FP) number $x$ comprises of a 1-bit sign ($x_{\mathrm{sign}}$), $B_m$-bit mantissa ($x_{\mathrm{mant}}$) and $B_e$-bit exponent ($x_{\mathrm{exp}}$) such that $B_m+B_e=n-1$. The associated constant exponent bias ($E_{\mathrm{bias}}$) is computed as $(2^{{B_e}-1}-1)$. We denote this format as $E_{B_e}M_{B_m}$.  

\subsubsection{Quantization Scheme}
\label{subsec:quant_method}
A quantization scheme dictates how a given unquantized tensor is converted to its quantized representation. We consider FP formats for the purpose of illustration. Given an unquantized tensor $\bm{X}$ and an FP format $E_{B_e}M_{B_m}$, we first, we compute the quantization scale factor $s_X$ that maps the maximum absolute value of $\bm{X}$ to the maximum quantization level of the $E_{B_e}M_{B_m}$ format as follows:
\begin{align}
\label{eq:sf}
    s_X = \frac{\mathrm{max}(|\bm{X}|)}{\mathrm{max}(E_{B_e}M_{B_m})}
\end{align}
In the above equation, $|\cdot|$ denotes the absolute value function.

Next, we scale $\bm{X}$ by $s_X$ and quantize it to $\hat{\bm{X}}$ by rounding it to the nearest quantization level of $E_{B_e}M_{B_m}$ as:

\begin{align}
\label{eq:tensor_quant}
    \hat{\bm{X}} = \text{round-to-nearest}\left(\frac{\bm{X}}{s_X}, E_{B_e}M_{B_m}\right)
\end{align}

We perform dynamic max-scaled quantization \citep{wu2020integer}, where the scale factor $s$ for activations is dynamically computed during runtime.

\subsection{Vector Scaled Quantization}
\begin{wrapfigure}{r}{0.35\linewidth}
  \centering
  \includegraphics[width=\linewidth]{sections/figures/vsquant.jpg}
  \caption{\small Vectorwise decomposition for per-vector scaled quantization (VSQ \citep{dai2021vsq}).}
  \label{fig:vsquant}
\end{wrapfigure}
During VSQ \citep{dai2021vsq}, the operand tensors are decomposed into 1D vectors in a hardware friendly manner as shown in Figure \ref{fig:vsquant}. Since the decomposed tensors are used as operands in matrix multiplications during inference, it is beneficial to perform this decomposition along the reduction dimension of the multiplication. The vectorwise quantization is performed similar to tensorwise quantization described in Equations \ref{eq:sf} and \ref{eq:tensor_quant}, where a scale factor $s_v$ is required for each vector $\bm{v}$ that maps the maximum absolute value of that vector to the maximum quantization level. While smaller vector lengths can lead to larger accuracy gains, the associated memory and computational overheads due to the per-vector scale factors increases. To alleviate these overheads, VSQ \citep{dai2021vsq} proposed a second level quantization of the per-vector scale factors to unsigned integers, while MX \citep{rouhani2023shared} quantizes them to integer powers of 2 (denoted as $2^{INT}$).

\subsubsection{MX Format}
The MX format proposed in \citep{rouhani2023microscaling} introduces the concept of sub-block shifting. For every two scalar elements of $b$-bits each, there is a shared exponent bit. The value of this exponent bit is determined through an empirical analysis that targets minimizing quantization MSE. We note that the FP format $E_{1}M_{b}$ is strictly better than MX from an accuracy perspective since it allocates a dedicated exponent bit to each scalar as opposed to sharing it across two scalars. Therefore, we conservatively bound the accuracy of a $b+2$-bit signed MX format with that of a $E_{1}M_{b}$ format in our comparisons. For instance, we use E1M2 format as a proxy for MX4.

\begin{figure}
    \centering
    \includegraphics[width=1\linewidth]{sections//figures/BlockFormats.pdf}
    \caption{\small Comparing LO-BCQ to MX format.}
    \label{fig:block_formats}
\end{figure}

Figure \ref{fig:block_formats} compares our $4$-bit LO-BCQ block format to MX \citep{rouhani2023microscaling}. As shown, both LO-BCQ and MX decompose a given operand tensor into block arrays and each block array into blocks. Similar to MX, we find that per-block quantization ($L_b < L_A$) leads to better accuracy due to increased flexibility. While MX achieves this through per-block $1$-bit micro-scales, we associate a dedicated codebook to each block through a per-block codebook selector. Further, MX quantizes the per-block array scale-factor to E8M0 format without per-tensor scaling. In contrast during LO-BCQ, we find that per-tensor scaling combined with quantization of per-block array scale-factor to E4M3 format results in superior inference accuracy across models. 


\end{document}