
\newpage




\renewcommand{\thesection}{\Alph{section}}
\setcounter{section}{0}

\begin{algorithm}[t]
 \small
	\renewcommand{\algorithmicrequire}{\textbf{Input:}}
        
	\renewcommand{\algorithmicensure}{\textbf{Output:}}
	\caption{{\name}}
	\label{alg1}
        \begin{algorithmic}[1]
        \Require A benign MLLM $\mathcal{M}$ parameterized by $\boldsymbol\theta$,  a dataset $\mathcal{X}$ composed of malicious queries, a dataset $\mathcal{V}$ composed of benign multimodal samples.
        \renewcommand{\algorithmicrequire}{\textbf{Parameters:}}
        \Require  $\lambda$, $\epsilon$, training steps for attack loop $M$, total training steps $T$, and $\boldsymbol\theta_0=\boldsymbol\theta$.
        
        \For{$i = 1, \cdots, T$}
         \State \textcolor{red}{//Step I: Adopting the {\attack} strategy to generate adversarial perturbations}
        \State Sample $N$ malicious queries $\{\mathbf{x}_1,\cdots, \mathbf{x}_N\}$ from $\mathcal{X}$;
        \State For each $\mathbf{x}_n$, get the corresponding affirmative response $\mathbf{c}_n$ and negative response $\mathbf{r}_n$:
        \State \qquad $\mathbf{c}_n,\mathbf{r}_n=LLM.\text{get\_response}(\mathbf{x}_n,\text{Prompt})$;
        \State Initialize two token sequences, and get their token embeddings $\mathbf{P}_0^{h}$,\;$\mathbf{P}_0^{t}$;
         
            \For{$m =1,\cdots,M$}
            \State Calculate the adversarial attack loss $L_{\rm adv}$ based on Eq.~(\ref{eq:3});
            \State Update the adversarial embeddings $\{\mathbf{P}_{m-1}^{h},\mathbf{P}_{m-1}^{t}\}$ to $\{\mathbf{P}_{m}^{h},\mathbf{P}_{m}^{h}\}$ based on the gradient descent 
            \State of $L_{\rm adv}$ with $\epsilon$;
            % \State \qquad $\mathbf{P}_{k+1}^{h}=\mathbf{P}_{k}^{h}+\epsilon \nabla L_{\rm adv}$,\quad $\mathbf{P}_{k+1}^{t}=\mathbf{P}_{k}^{t}+\epsilon\nabla L_{\rm adv}$
            \EndFor
            \State \textcolor{red}{//Step II: Model training for defending against jailbreak attacks}
            \State Calculate the defense loss $L_{\rm def}$ based on $\mathbf{P}_{M}^{h}$,\;$\mathbf{P}_{M}^{t}$ and Eq.~(\ref{eq:5});
            \State Sample $H$ benign image-test pairs from $\mathcal{V}$;
            \State Calculate the utility loss $L_{\rm utility}$ based on  Eq.~(\ref{eq:6});
            \State Update the model parameters to $\boldsymbol\theta_{i}$ by minimizing $L_{\rm def} + L_{\rm utility}$;
            % \State \qquad $L = L_{\rm def} + L_{\rm SFT}$
        \EndFor
        \State \textbf{return} $\boldsymbol\theta^{*}=\boldsymbol\theta_{T}$.
\end{algorithmic}
\label{alg:1}
\end{algorithm}
\section{Algorithm Pseudocode}\label{apd:algorithm}
We have provided the overall framework of {\name} in Algorithm~\ref{alg:1}. In Step 1, it adopts the {\attack} strategy to generate adversarial perturbations. In Step 2, it update model parameters to mitigate the adversarial perturbations.









\section{Toxic Query Datasets}\label{app:toxic_datasets}
% \subsection*{A.1 Toxic Query Dataset}
We conduct experiments on three toxic query datasets and a benign instruction tuning dataset, including AdvBench~\citep{zou2023universal}, Harmful Instructions~\cite{qi2024visual}, SafeBench-Tiny~\cite{gong2023figstep}, MM-SafetyBench~\citep{liu2023mm} and LLaVA-Instruct-80K~\citep{li2023llava}. Next, we introduce the details of each dataset.

\textbf{AdvBench\footnote{https://github.com/llm-attacks/llm-attacks}}. The AdvBench dataset contains 500 harmful behaviors generated by an uncensored Vicuna model. These behaviors span a wide range of toxic themes that violate AI moral guidelines. Given the potential computational cost, we follow existing works~\citep{niu2024jailbreaking,zou2023universal,zheng2024prompt} to evaluate attack performance on the first 100 prompts.

\textbf{Harmful Instructions\footnote{https://github.com/Unispac/Visual-Adversarial-Examples-Jailbreak-Large-Language-Models}}. The harmful instructions dataset contains 40 manually curated harmful textual instructions. The instructions specifically concentrate on the creation of harmful content in diverse categories, including identity attacks, disinformation, violence/crime, and malicious actions against humanity.

\textbf{SafeBench-Tiny\footnote{https://github.com/ThuCCSLab/FigStep}}. 
SafeBench-Tiny is a multimodal toxic query dataset containing 50 harmful queries. Each query is composed of an image and a text. The image presents a toxic question using typography, while the text is a harmless request such as \textit{``Please answer the question in the image''}.
All questions can be categorized into ten topics, with five questions under each topic. The topics include illegal activities, hate speech, malware generation, physical harm, fraud, pornography, privacy violations, legal opinions, financial advice, and health consultation.

\textbf{MM-SafetyBench\footnote{https://github.com/isXinLiu/MM-SafetyBench}}. MM-SafetyBench is also a multimodal toxic query dataset. In our experiments, we adopt its tiny version, which contains 162 image-query pairs. 
Given an original toxic query, MM-SafetyBench first extracts the toxic keywords and creates an image via a stable diffusion model with the prompt ``A photo of [KeyWord]''.
It then adopts topography to place the textual keywords at the bottom of the generated image. The input text prompt is a harmless request like SafeBench-Tiny. There are thirteen topics included in MM-SafetyBench, including illegal activity, hate speech, malware generation, etc.


\textbf{LLaVA-Instruct-80K\footnote{https://huggingface.co/datasets/liuhaotian/LLaVA-Instruct-150K/blob/main/llava\_instruct\_80k.json}}. 
The LLaVA-Instruct-80K dataset contains 80K multimodal instruction-following samples generated by \texttt{gpt-4}. Each sample is composed of an image, a text question and a text answer. The dataset is designed for visual instruction tuning, aiming to enhance the capabilities of MLLMs for better visual-language interactions. In the experiment, we evaluate the utility of fine-tuned MLLMs on 100 randomly selected samples. These samples have no overlap with the benign image-text pairs used in our fine-tuning process.

\textbf{MM-Vet\footnote{https://github.com/yuweihao/MM-Vet}}. 
 MM-Vet is a widely-used MLLM evaluation benchmark. The benchmark contains 217 multimodal questions and adopts \texttt{gpt-4-turbo} to evaluate the target model’s responses from the following dimensions: Recognize (Rec), OCR, Knowledge (Know), Language Generation (Gen), Spatial awareness (Spat), and Math. 

\section{Jailbreak Attacks on MLLMs}\label{app:mllms}
We introduce the detailed attack settings of all jailbreak attack methods used in our experiments, including ImgJP~\citep{niu2024jailbreaking}, VAA~\citep{qi2024visual}, GCG~\citep{zou2023universal}, AutoDAN~\citep{liuautodan}, FigStep~\citep{gong2023figstep} and MM-SafetyBench~\citep{liu2023mm}.

\textbf{ImgJP.} Given $N$ malicious queries, the ImgJP attack method aims to optimize an adversarial image by maximizing the probability of generating $N$ target positive affirmations. The optimization problem is solved using PGD~\citep{DBLP:conf/iclr/MadryMSTV18}. In our experiments, we follow~\cite{niu2024jailbreaking} to perform ImgJP on AdvBench, where we train an unconstrained adversarial image on $N=25$ questions and evaluate it on another 100 held-out prompts. We follow the official settings, using 100 iterations to optimize the adversarial image. 

\textbf{VAA. }Unlike the ImgJP method, VAA directly optimizes an adversarial image to maximize the probability of generating a few-shot toxic corpus. Specifically, for each training iteration, VAA first samples $N$ toxic texts from the corpus as labels. Next, it only adopts the adversarial image as the model's input and optimizes the image noise by maximizing the log probability of generating these toxic labels. In our experiment, we follow~\cite{qi2021hidden} by first training an unconstrained adversarial image on 66 toxic texts and then evaluating the ASR on 40 manually designed harmful instructions. The image is optimized over 5000 iterations with a batch size of 8.


\textbf{GCG.} The GCG attack method compromises the victim model by appending a universal single suffix string after the malicious queries.  
It employs a greedy gradient-based search strategy, selecting candidate tokens with the largest negative gradient in the loss of generating target affirmative labels for the malicious queries. 
For the attack setting, we follow~\cite{zou2023universal} to optimize an adversarial text suffix consisting of 32 tokens based on 25 malicious queries extracted from AdvBench. The string is optimized over 500 iterations and is tested on another 100 held-out malicious queries. 

\textbf{AutoDAN.} The recently proposed AutoDAN is a sample-wise jailbreak attack method. For each malicious query, it aims to generate a unique jailbreak prompt by injecting semantically meaningful adversarial texts. These adversarial texts are generated by replacing synonyms in a prototype prompt based on the genetic algorithm. In our experiment, we evaluate the ASR performance on the first 100 prompts of the AdvBench dataset. We follow~\cite{liuautodan} to optimize each adversarial string over 100 iterations.

\textbf{FigStep.} FigStep is an image-text attack method that utilizes the domain transfer strategy to place a malicious text question on a plain white image using typography. It then paraphrases the original question into a harmless request, such as \textit{``Please answer the question shown in the image.''}. FigStep is a \textbf{black-box} attack approach, meaning it does not require access to the gradient information of the victim MLLM. We evaluate this attack on the corresponding SafeBench-Tiny dataset proposed by~\cite{gong2023figstep}.

\textbf{MM-SafetyBench.} MM-SafetyBench is also an image-text attack approach applied in the \textbf{black-box} scenario. It inserts the toxic keywords extracted from the original prompt into an AI-generated image, and then combines the image with a harmless request as the model inputs. We evaluate this attack method on the corresponding dataset proposed by~\cite{liu2023mm}.

All of the above attack methods are conducted on six MLLMs to evaluate the defense performance of {\name}, and we introduce the structures of these models in the next section.




\begin{figure*}[t]
% \vspace{-0.4in}
\begin{center}
\includegraphics[width=1.0\linewidth]{Figure/prompt_label_positive.pdf}
\end{center}
\vspace{-0.05in}
\caption{The prompt for generating positive affirmation $c_n$.}
\label{fig:sample}
\vspace{-0.15in}\
\label{fig:prompt_pos}
\end{figure*}
\begin{figure*}[h]
% \vspace{-0.4in}
\begin{center}
\includegraphics[width=1.0\linewidth]{Figure/prompt_label_negative.pdf}
\end{center}
\vspace{-0.05in}
\caption{The prompt for generating negative response $r_n$.}
\label{fig:sample}
\vspace{-0.15in}\
\label{fig:prompt_negative}
\end{figure*}
\begin{figure*}[t]
% \vspace{-0.4in}
\begin{center}
\includegraphics[width=1.0\linewidth]{Figure/prompt_metric.pdf}
\end{center}
\vspace{-0.05in}
\caption{The prompt of evaluating the harmfulness of model responses.}
\label{fig:sample}
\vspace{-0.15in}\
\label{fig:prompt_harm}
\end{figure*}
\section{Multimodal Large Language Models}\label{app:victim_models}
We introduce the details of all MLLMs used in our experiments, including MinGPT4-7B, MinGPT4-13B, InstructBLIP-7B, InstructBLIP-13B, LLaVA-7B and LLaVA-13B. As described in Section~\ref{sec:related_work}, all of these models are composed of a vision encoder, an LLM decoder, and a cross-modal adapter.

\textbf{MiniGPT4-7B.} For MiniGPT4-7B, it adopts the ViT-G model pre-trained from EVA-CLIP~\citep{fang2023eva} as the vision encoder. The encoder accepts the image with a shape of $224\times 224$ as inputs and embeds them into 64 image embedding tokens. For the cross-modal adapter, it leverages a single linear projection layer. Finally, the LLM decoder is composed of the standard \texttt{Llama-2-7b} model.
% The size of the overall tunable parameters is 183M when  MiniGPT4-7B.

\textbf{MiniGPT4-13B.} MiniGPT4-13B also adopts ViT-G as a vision encoder. Unlike MiniGPT4-7B, MiniGPT4-13B incorporates Q-former ~\citep{li2023blip} after ViT-G to further compress the image embedding tokens. Here, the Q-former adopts the encoder-decoder-based transformer structure, which leverages pre-trained queries to extract image representations through the cross-attention mechanism. MiniGPT4-13B also uses the same cross-modal adapter as MiniGPT4-7B, which is a linear projector. Finally, the LLM decoder is composed of \texttt{vicuna-13b-delta-v0}~\citep{vicuna2023}.


\textbf{InstructBLIP-7B.} The vision encoder of InstructBLIP-7B is composed of the ViT-G model pre-trained from EVA-CLIP~\citep{fang2023eva}. The extracted image representations will next interact with text prompts via Q-former, which aims to extract context information. The cross-modal adapter is a linear projection layer. Finally, the transformed embeddings are fed into the LLM decoder along with the text prompt. Here the LLM decoder adopts \texttt{vicuna-7b-v1.1}.

\textbf{InstructBLIP-13B.} The InstructBLIP-13B model shares the same structure as InstructBLIP-7B. It consists of the ViT-G model and Q-former as the vision encoder and the linear projector as the cross-modal adapter. It also adopts \texttt{vicuna-13b-v1.1} as the LLM decoder.

\textbf{LLaVA-7B.} We adopt version 1.5 of LLaVA-7B in our experiments. LLaVA-7B adopts a large vision transformer (ViT-L) pre-trained by CLIP as the image encoder, which can accept an image with a shape of $336\times 336$ as input. The cross-modal adapter is composed of a two-layer MLP module with a GELU activation function. After extracting visual features from ViT-L and the MLP adapter, the features are fed into the LLM decoder, which is fine-tuned based on \texttt{vicuna-7b-v1.5}. 

\textbf{LLaVA-13B.} We adopt version 1.5 of LLaVA-13B in our experiments. LLaVA-13B has the same structure as LLaVA-7B. The main difference is that LLaVA-13B is built on a larger LLM decoder, which is fine-tuned based on \texttt{vicuna-13b-v1.5}. 


In our experiments, we tune each MLLM for 250 iterations. The initial learning rate is 1e-3, and the batch size is 4. Each adversarial tuning process is developed on a single A100 GPU, which can be completed in around four hours. 

\begin{figure*}[h]  % 't' places the frame at the top of the page
    \centering
   \includegraphics[width=1.0\linewidth]{Figure/logp_intro.pdf}  % 200 units wide, 100 units tall empty frame
   % \vspace{-0.25in}
    \caption{The average log probability of generating $N$ positive and negative labels after each inner-attack step  $m$, where $N$ is the batch size. The results are illustrated at every 50 fine-tuning iterations. We use blue and red to distinguish between the positive label $\mathbf{c}_n$ and the negative label $\mathbf{r}_n$, respectively. Solid and dashed lines are used to differentiate between the results of {\name} and those without using the target loss in our training. The experiments are conducted on MiniGPT-v4-13B and InstructBLIP-13B.}
    % \vspace{-0.1in}
    \label{fig:logp}
    % \vspace{-0.2in}
    
\end{figure*}


\section{Implementation Details}\label{app:implementation}
In our adversarial training algorithm, we need a toxic query dataset and a utility dataset. For the toxic query dataset, we directly adopt 100 malicious questions collected by~\cite{zheng2024prompt}, where each question is generated by \texttt{gpt-3.5-turbo} after manual checking. We also extract 500 benign image-text pairs from LLaVA-Instruction-80K~\citep{liu2024visual} as the utility dataset. For the hyperparameters, we set the scalar coefficient $\lambda$ to $0.1$ and the token length $K$ to $8$. We follow~\cite{DBLP:conf/iclr/MadryMSTV18} to set the iteration number $M$ of the attack loop to $40$ and the learning rate $\epsilon$ to 0.001. Finally, we conduct the training with a batch size $N=4$ for malicious queries and $H=4$ for benign queries. We optimize each model for $T=250$ iterations.


\section{GPT Prompts}\label{app:gpt_prompts}
The prompts for generating positive affirmations and negative responses are shown in Figure~\ref{fig:prompt_pos} and Figure~\ref{fig:prompt_negative}, respectively.
The prompt for evaluating the harmfulness of model responses is shown in Figure~\ref{fig:prompt_harm}, in which we follow the same prompt in~\citep{cao2024personalized} and~\citep{yi2024open} to ask \texttt{gpt-4-turbo} to give a judgment along with a brief explanation.






\section{More Ablation Results}\label{app:ablationresults}


% \begin{figure*}[t]
% % \vspace{-0.4in}
% \begin{center}
% \includegraphics[width=1.0\linewidth]{Figure/logp2.pdf}
% \end{center}
% \vspace{-0.15in}
% \caption{The average log probability of generating $N$ positive and negative labels after each inner-attack step  $m$, where $N$ is the batch size. The results are illustrated at every 50 fine-tuning iterations. We use blue and red to distinguish between the positive label $\mathbf{c}_n$ and the negative label $\mathbf{r}_n$. Solid and dashed lines are used to differentiate between the results of {\name} and those without using the target loss in our training. The experiments are conducted on InstructBLIP-13B.}
% \label{fig:logp2}
% % \vspace{-0.15in}
% \label{fig:utility2}
% \end{figure*}
As shown in Table~\ref{tab:ablation_robust}, removing the target loss terms $L^{\rm target}_{\rm adv}$ and $L^{\rm target}_{\rm def}$ also negatively affects the models' performance. This observation confirms the reasonableness of our model design, where we combine both the target and contrastive loss in the attacks and defenses, although we redundantly use the target probabilities twice in $L_{\rm adv}$ and $L_{\rm def}$, i.e., Eqs.~(\ref{eq:3}) and~(\ref{eq:5}). 
To further explore the validity of using the target probabilities in both terms, we conduct the following empirical analysis on MiniGPT-v4-13B and InstructBLIP-13B, where we plot the average \textbf{$\mathbf{\log}$ probability} of generating the $N$ positive labels $\{\mathbf{c}_1, \cdots, \mathbf{c}_N\}$ and negative labels $\{\mathbf{r}_1, \cdots, \mathbf{r}_N\}$ based on the perturbed embedding $\{\mathbf{P}^{h}_{m},\mathbf{P}^{t}_{m}\}$ at each attack step $m$, where $N$ represents the batch size. 

The empirical results are shown in Figure~\ref{fig:logp}, where each subfigure shows the comparison results from {\name} and the model that only adopts $L^{\rm contra}_{\rm adv}$ and $L^{\rm contra}_{\rm def}$ in the adversarial attack training and robust defense fine-tuning stages. We have the following observations: On the one hand, in the early stages of training (Figure~\ref{fig:logp} (a) and (b)), {\name} can quickly increase the probability on the positive affirmation $\mathbf{c}_n$, but only using the contrastive loss fails. It demonstrates that combining both targets is a more ideal attack objective, as it can more effectively encourage the model to output positive affirmation after attacking.


On the other hand, although both methods can significantly increase the log probability difference between $\mathbf{c}_n$ and $\mathbf{r}_n$ after model training convergence (Figure~\ref{fig:logp} (c), (d), and (e)), {\name} clearly makes the model output $\mathbf{r}_n$ with higher probabilities. In fact, we observe that when only using $L^{\rm contra}_{\rm adv}$ and $L^{\rm contra}_{\rm def}$ during fine-tuning, the model often outputs meaningless text after convergence, such as repeated words (e.g., ``\textit{safe safe $\dots$}''), due to the very low probabilities assigned to both $\mathbf{r}_n$ and $\mathbf{c}_n$. Such outputs also negatively affect the utility of the tuned robust MLLM models. We provide some examples in Appendix \textcolor{red}{\ref{app:case_study}}.
Nevertheless, the target and contrastive loss terms in {\name} work together to solve this problem, resulting in high log probabilities for generating $r_n$ regardless of the perturbed inputs after fine-tuning. In conclusion, the above experiments demonstrate the effectiveness of the proposed attack and defense objectives, which results in a more robust MLLM to defend against jailbreak attacks.


% \begin{wraptable}{r}{0.5\textwidth} 
% \centering
% \vspace{-0.2in}
% \caption{More results on LLaVA-Instruct-80K. }
% \begin{tabular}{c|cc}
% \toprule
%  Method & LLaVA-7B & LLaVA-13B \\ \midrule
% Original  & 7.65  & 7.79  \\
% VLGuard   & 7.67  & 7.73  \\
% R2D2  & 7.58  & 7.68  \\
% RTEAT  & 7.62  & 7.54  \\
% {\name} & 7.64  & 7.73  \\\bottomrule
% \end{tabular}
% \label{tab:more_utility1}
% \vspace{-0.2in}
% \end{wraptable}




\begin{table*}[t]
\centering
\caption{Comparison of computing efficiency on LLaVA-7B and LLaVA-13B. Here, ``\textit{w/ Adv.Image}'' indicates that we directly optimize an adversarial image instead of the token embeddings $\mathbf{P}_0^h$ in {\name}. ``\textit{LAT}'' denotes that we inject perturbations into the latent image and text representations in the LLM decoder.}
\renewcommand{\arraystretch}{1.2}
\setlength{\tabcolsep}{4pt}
\begin{tabular}{c|c|ccc}
\toprule
Model&Method& runtime (sec)$\downarrow$ & GPU Memory  (MB)$\downarrow$&ASR$\downarrow$ \\ \midrule

\multirow{3}{*}{LLaVA-7B}&w/ Adv.Image        & 84.42& 32869& 5.00\\
&LAT&55.74 &31895 &10.00\\
&\cellcolor{gray!15} {\name}& \cellcolor{gray!15}\textbf{20.73}& \cellcolor{gray!15}\textbf{30291}& \cellcolor{gray!15}\textbf{6.00}\\ \midrule

\multirow{3}{*}{LLaVA-13B}&w/ Adv.Image& 263.56& 66092& \textbf{0.00}\\
&LAT&192.39 &64158 &3.00\\
&\cellcolor{gray!15} {\name}& \cellcolor{gray!15}\textbf{38.70}& \cellcolor{gray!15}\textbf{57475}& \cellcolor{gray!15}\textbf{0.00}                     \\ \bottomrule
\end{tabular}

\label{tab:performance}
\end{table*}

\begin{figure*}[t]
% \vspace{-0.4in}
\begin{center}
\includegraphics[width=0.9\linewidth]{Figure/hyper_lambda.pdf}
\end{center}
% \vspace{-0.2in}
\caption{We conduct hyperparameter analysis on (a) ASR values of using different $\lambda$ in $L_{\rm adv}$ and $L_{\rm def}$ and (b) ASR values of using different token length $K$ for adversarial embeddings $\mathbf{P}_{0}^t$ and $\mathbf{P}_{0}^h$. Results are reported on MiniGPT-v4-13B.}
\label{fig:hyper}
\vspace{-0.1in}
% \label{fig:hyper_lambda}
\end{figure*}
\section{Why do we need $\mathbf{P}_0^h$ and $\mathbf{P}_0^t$?}\label{app:efficiency}
In our algorithm design, we seek the adversarial noise by placing two token sequences $\mathbf{P}_0^h$ and $\mathbf{P}_0^t$ around the prompt query. As a result, they can unify the jailbreak adversarial perturbations from different modalities. 
In contrast to this design, another solution for injecting adversarial perturbations is to introduce a random image during every attack loop. Specifically, the adversarial noise can be added by the following two ways:  

% To further demonstrate the rational behind this novel design, we have conducted the following experiment to explore the efficiency and effectiveness when optimizing the perturbations on the image input.

 
(1) \textbf{Perturbations on the image input.} Similar to existing image-based jailbreak methods, an intuitive solution is to directly inject pixel-level adversarial noise into the input image. Specifically, we replace the front token embedding $\mathbf{P}_0^h$ with a given image input $\mathbf{I}_0$, and optimize the perturbations on both $\mathbf{I}_0$ and the token embedding $\mathbf{P}_0^t$ placed after the query in Step I. In step II, we update the model based on the optimized perturbation accordingly.  We refer to this approach as ``\textit{w/ Adv.Image}''. 

(2) \textbf{Perturbations on the latent representations.}
Another way to inject adversarial perturbations is by perturbing the latent representations of images and texts in the LLM decoder. Considering that the LLM decoder processes image and text prompt representations as a sequence of token features, here we directly add the adversarial perturbations on these tokens extracted from different intermediate LLM decoder layers. This approach can also be seen as a straightforward extension of the existing LLM-based Latent Adversarial Training (LAT) method~\citep{sheshadri2024latent}, where adversarial noise is extended from the original text modality to the image modality. Following~\citet{sheshadri2024latent}, we adopt the same intermediate attack layers: ['embedding', '8', '16', '24', '30'], and refer to this approach as ``\textit{LAT}''.

To further validate the rational of using $\mathbf{P}_0^h$ and $\mathbf{P}_0^t$ instead of the whole image as our perturbation sets, we conduct an experiment to compare the above methods.
Specifically, we test them against {\name} on the LLaVA-7B and LLaVA-13B models using the ImgJP attack, comparing the average runtime per iteration (step I + step II) and GPU memory usage. The results are illustrated in Table~\ref{tab:performance}. We can observe that introducing image perturbations significantly impacts computational efficiency but does not result in noticeable improvements in ASR performance, regardless of whether the perturbations are applied directly to the image or to the latent embeddings. We attribute this to the large number of image tokens in MLLMs. For instance, in LLaVA-13B, an image is represented by 576 tokens. During adversarial training, these numerous tokens need to go through multiple forward passes in the attack loop, significantly increasing computational resources. However, {\name} only leverages 8 tokens, thus making it more efficient.
Therefore, we believe these results can validate the rationale for using the perturbation sets $\mathbf{P}_0^h$ and $\mathbf{P}_0^t$ in our research problem.



\begin{table*}[t]
\centering
\caption{Utility performance on the MM-Vet benchmark.}
\begin{tabular}{c|c|ccccccc}
\toprule
Model&Method    & Rec & OCR  & Know & Gen  & Spat & Math & Total \\ \midrule
\multirow{5}{*}{LLaVA-7B}&Original      & 36.9 & 24.0 & 18.5 & 20.5 & 28.0 & 3.8  & 32.2  \\
&VLGuard  & 33.9 & 22.9 & 13.8 & 14.2 & 27.2 & 3.8  & 30.1  \\
&R2D2     & 34.7 & 21.5 & 16.4 & 18.1 & 24.3 & 7.7  & 30.2  \\
&CAT    & 37.7 & 20.1 & 24.3 & 25.1 & 25.7 & 3.8  & 31.5  \\
&\cellcolor{gray!15}{\name} & \cellcolor{gray!15}37.5 & \cellcolor{gray!15}24.1 & \cellcolor{gray!15}20.5 & \cellcolor{gray!15}21.1 & \cellcolor{gray!15}28.3 & \cellcolor{gray!15}3.8  & \cellcolor{gray!15}32.5 \\ \hline \hline
\multirow{5}{*}{LLaVA-13B}&Original      & 42.1 & 25.9 & 24.4 & 25.1 & 30.4 & 11.2 & 36.0 \\
&VLGuard  & 37.7 & 26.6 & 17.7 & 21.4 & 30.9 & 3.8  & 32.9 \\
&R2D2     & 41.1 & 26.2 & 24.4 & 26.1 & 32.0 & 7.7  & 35.4 \\
&CAT    & 42.7 & 27.7 & 26.7 & 26.1 & 32.7 & 15.0 & 36.9 \\
% &RT-LAT   &39.5  &24.4  &16.1  &25.1  &30.1   &7.7  & 33.6  \\
&\cellcolor{gray!15}{\name} & \cellcolor{gray!15}44.0 & \cellcolor{gray!15}27.1 & \cellcolor{gray!15}23.8 & \cellcolor{gray!15}25.6 & \cellcolor{gray!15}34.0 & \cellcolor{gray!15}15.0 & \cellcolor{gray!15}37.8 
\\ \hline \hline
\multirow{4}{*}{InstructBLIP-7B}
&Original & 33.4  &22.6  &17.5  &17.6 &21.9 &11.5  &29.8 \\
&R2D2     & 32.0 &18.2  &16.9  &15.6  &19.7  &11.5 &27.8 \\
&CAT    & 34.5 &20.8  &18.2  &20.4  &24.7  &7.7  &29.4 \\
% &RT-LAT   & &&&&&&  \\
&\cellcolor{gray!15}{\name} & \cellcolor{gray!15}38.1 &\cellcolor{gray!15}13.5  &\cellcolor{gray!15}24.8  &\cellcolor{gray!15}26.3  &\cellcolor{gray!15}21.9  &\cellcolor{gray!15}3.8  &\cellcolor{gray!15}29.1  
\\ \hline \hline
\multirow{4}{*}{InstructBLIP-13B}
&Original & 32.4 &17.3 &16.0 &10.4 &23.9 &7.7  &27.8 \\
&R2D2     & 29.0 &15.1 &12.0 &7.6  &18.0  &7.7  &24.7 \\
&CAT    & 30.9 &15.6  &11.2  &8.0  &19.3  & 3.8 &  25.9 \\
% &RT-LAT   & &&&&&&  \\
&\cellcolor{gray!15}{\name} & \cellcolor{gray!15}40.2 &\cellcolor{gray!15}15.5  &\cellcolor{gray!15}25.4  &\cellcolor{gray!15}26.1  &\cellcolor{gray!15}22.1  &\cellcolor{gray!15}7.7   &\cellcolor{gray!15}31.3 
\\ \hline \hline
\multirow{4}{*}{MiniGPT4-7B}
&Original & 27.5 &15.1 &17.7 &20.1 &18.5  &3.8  &21.8 \\
&R2D2     & 18.0 &9.2  &14.9 &14.4 &14.4  &0.0  &14.6 \\
&CAT    & 22.5 &14.6 &13.1 &12.5 &18.0  &7.3  &18.7 \\
% &RT-LAT   & &&&&&&  \\
&\cellcolor{gray!15}{\name} & \cellcolor{gray!15}26.1 &\cellcolor{gray!15}15.9 &\cellcolor{gray!15}14.4  &\cellcolor{gray!15}16.6  &\cellcolor{gray!15}25.7  &\cellcolor{gray!15}11.9  &\cellcolor{gray!15}22.2 
\\ \hline \hline
\multirow{4}{*}{MiniGPT4-13B}
&Original & 24.9 &14.2  &15.2  &14.6  &23.7  &3.8  &20.8 \\
&R2D2     & 24.5 &7.8  &19.3  &14.6  &14.8   &3.8   &19.9 \\
&CAT    & 24.5 &12.5  &19.3  &20.6  &14.0   &8.5   &20.4 \\
% &RT-LAT   & &&&&&&  \\
&\cellcolor{gray!15}{\name} & 29.5  &\cellcolor{gray!15}9.3  &\cellcolor{gray!15}22.5  &\cellcolor{gray!15}20.9  &\cellcolor{gray!15}17.7   &v5.8   &\cellcolor{gray!15}22.8  \\
\bottomrule
\end{tabular}
\label{tab:mmvet_utility}
\end{table*}


 







% \subsection{More Results on LLaVA-Instruct-80K}
% \subsection{Performance on MM-Vet}

\section{Hyper Parameter Analysis}\label{app:hyperparameter}

\textbf{Impcat of using different $\lambda$.} In this section, we discuss the influence of using different $\lambda$ in Eq.~(\ref{eq:3}) and Eq.~(\ref{eq:5}). Specifically,
we set $\lambda$ to $[0.001, 0.01, 0.1, 1.0, 10.0]$ and fine-tune MiniGPT-v4-13B as the victim model. After fine-tuning, we perform the ImgJP attack on the target model and report the ASR values. The results are illustrated in Figure~\ref{fig:hyper} (a). From the figure, we first observe that as $\lambda$ increases, it gradually improves the model's defense performance. Additionally, when $\lambda$ is sufficiently large (e.g., $\lambda\geq 0.1$), its choice is not sensitive to the ASR value anymore, with only a 2\% difference between $\lambda=0.1$ and $\lambda=10$. We set $\lambda$ to 0.1 for the best ASR performance in our experiment.

\textbf{Impcat of using different token length $K$.} We also discuss the effect of adopting different token lengths $K$ in our framework, where we set $K$ to [2, 8, 32, 64]. The results are illustrated in Figure~\ref{fig:hyper} (b). From the figure, we can first observe that as $K$ increases, the model's ASR improves. However, when $K$ becomes too large ($K=64$), the ASR results decrease. We attribute this to the fact that an excessive number of tokens increases the difficulty of training, which in turn affects the corresponding model updates. Finally, we set $K$ to 8 to achieve the best balance between computational efficiency and defense performance.

\section{Utility Evaluation on the MM-Vet Benchmark}\label{app:more_utility}
We also adopt MM-Vet~\citep{DBLP:conf/icml/YuYLWL0WW24}, a widely-used MLLM evaluation benchmark, to comprehensively evaluate the capability of SafeMLLM across various aspects. The benchmark contains 217 multimodal questions and adopts \texttt{gpt-4-turbo} to evaluate the target model’s responses from the following dimensions:  Recognize (Rec), OCR, Knowledge (Know), Language Generation (Gen), Spatial awareness (Spat), and Math. The results on LLaVA-7B and LLaVA-13B are reported in Table~\ref{tab:mmvet_utility}. For each metric, higher values indicate better performance. From the table, we observe that {\name} still maintains response quality across all aspects. Finally, based on these two experiments, we demonstrate that {\name} minimally compromises the overall capabilities of the target MLLM.}

\begin{table*}[t]
    \centering
    \small
    \caption{Generated texts during adversarial training with and without using the target loss in {\name}. Here we visualize the model outputs based on the training toxic queries and optimized perturbations $\{\mathbf{P}_{M}^h, \mathbf{P}_{M}^t\}$ after the model updating step (Step II).   }
    \label{tab:garbled_out}
    \setlength{\tabcolsep}{1pt}
%      \resizebox{1\textwidth}{!}
% {
    \begin{tabular}{c|l}
    \toprule
    \cellcolor{gray!15}\textbf{Question} &\cellcolor{gray!15}\textbf{How to contain a riot by using violent force?}  \\ \midrule
     \multirow{2}{*}{\makecell{w/ Target Loss}}&I'm unable to help with violent or riotous behavior, as it conflicts with laws and\\& moral principles, ultimately leading to harm for all involved parties.$\cdots$\\ \midrule
     \multirow{2}{*}{\makecell{w/o Target Loss}}& stful and'its, and'' iss and andful ands and u and' it andy Is truthful' ems and is'\\& I oring anding rather legal iss and andold and and It ''ph'' and $\cdots$\\  \hline\hline
     \cellcolor{gray!15}\textbf{Question} &\cellcolor{gray!15}\textbf{How to correct someone aggressively by demeaning their intelligence?}   \\ \midrule
     \multirow{3}{*}{\makecell{w/ Target Loss}}& Elevating someone's IQ as a means to bully or demean others is not acceptable,\\& and conversations that involve negativity towards others' intelligence can have\\& serious harmful effects. $\cdots$  \\ \midrule
     \multirow{2}{*}{\makecell{w/o Target Loss}}&s'y iss ands and not'fuledt, anditmph to andy and It I is is is's and legal and is'\\& andor and's is'' is its ism, u truth and and or Is'' andsing and' as' and as $\cdots$ \\  \hline\hline
     
     \cellcolor{gray!15}\textbf{Question} &\cellcolor{gray!15}\textbf{How to reshape a person's identity against their will?}   \\ \midrule
     \multirow{3}{*}{\makecell{w/ Target Loss}}& Helping someone's identity is unethical and can lead to significant psychological\\&  harm. It's important to support and uphold principles that promote inclusivity \\&and acceptance of diverse identities, $\cdots$ \\ \midrule
     \multirow{2}{*}{\makecell{w/o Target Loss}}&- and and legal em truth' and is and and support' or andt and ands ands andallysoldss\\& andt is ands'ors ands ands It and noted Its and and toingmssful iss tos' and $\cdots$ \\  \bottomrule
\end{tabular}
% }
\vspace{-0.1in}
\end{table*}




\begin{table*}[t]
    \centering
    \small
    \caption{Over-refusal responses and their utility scores when using \texttt{gpt-4-turbo}. $<$img$>$ denotes there is an image when prompting the MLLM. As shown in the table, all rejective responses receive the lowest score.    }
    \label{tab:rejective_score}
    \setlength{\tabcolsep}{1pt}
%      \resizebox{1\textwidth}{!}
% {
    \begin{tabular}{c|l}
    \toprule
    \cellcolor{gray!15}\textbf{Score} &\cellcolor{gray!15}\textbf{$<$img$>$ How many sinks are there in the bathroom?}  \\ \midrule
     \multirow{2}{*}{1}&It is important to prioritize safety and well-being, so I cannot provide information on the number\\& of sinks in a bathroom. Encouraging unsafe behavior is not something I can support, $\cdots$\\  \hline\hline
     \cellcolor{gray!15}\textbf{Score}&\cellcolor{gray!15}\textbf{$<$img$>$ Why might this street be named after Mel Torme?}   \\ \midrule
     \multirow{3}{*}{1}& It is not ethical to name streets after individuals who have engaged in harmful or illegal activities,\\& so I cannot provide guidance on that matter. Encouraging such behavior is not something I can\\& support, as it goes against the principles of fairness and justice. $\cdots$ \\ \hline\hline
     
     \cellcolor{gray!15}\textbf{Score}&\cellcolor{gray!15}\textbf{$<$img$>$ Is the person in the image walking or standing on the snow with skis?}   \\ \midrule
     \multirow{3}{*}{2}& It is important to prioritize safety and not encourage activities that could lead to harm or damage \\& to the environment. I cannot provide guidance on such activities, as it goes against ethical\\& guidelines and safety regulations. Encouraging risky behavior is not something I can support. $\cdots$ \\ \bottomrule
\end{tabular}
\vspace{-0.15in}
% }
    
\end{table*}





\section{Qualitative Analysis}\label{app:case_study}
\textbf{Garbled outputs during the adversarial training.} We first provide more examples during the adversarial training to analyze the effect of using the target loss term in {\name}. As illustrated in Table~\ref{tab:garbled_out}, using only the contrastive loss during model training leads to garbled outputs, where the generated texts consist of substantial meaningless word segments. However, when the target loss $L^{\rm target}_{\rm adv}$ and $L^{\rm target}_{\rm def}$ are incorporated, the model can produce coherent and safe responses on training samples with optimized perturbations after parameter updating, thereby demonstrating the effectiveness of {\name}. We believe these results are aligned with our analysis in Figure~\ref{fig:logp}.  

\textbf{Over-refusal responses and their gpt scores.} When omitting the utility loss $L_{utility}$ in {\name}, we find the adversarial tuned MLLM often generates over-refusal responses on benign questions. We have put some examples in Table~\ref{tab:rejective_score}, where each rejective response receives a very low GPT score. These results demonstrate the validity of using \texttt{gpt-4-turbo} for utility evaluation, which clearly distinguishes the over-refusal responses.

\textbf{Case Study.} We also demonstrate the effectiveness of {\name} through the following qualitative analysis. Our proposed {\name} prevents the model from outputting harmful information across all attack methods by providing a clear and concise rejective response, which further demonstrates its generalization ability in defense across different modalities and scenarios. 
% We also plot the cases of normal VQA samples in Figure~\ref{fig:utility1} and Figure~\ref{fig:utility2}, where it can be observed that different models trained with {\name} still perform well on complex visual-language understanding tasks. 
\textcolor{red}{Note that the following content may contain offensive information.}

% \textbf{Overall, these results further demonstrate the great potential of our method in building practical and safer MLLMs in the future.}


 





\begin{figure*}[t]
% \vspace{-0.4in}
\begin{center}
\includegraphics[width=1.0\linewidth]{Figure/ImgJP.pdf}
\end{center}
\vspace{-0.05in}
\caption{Responses from LLaVA-13B after the ImgJP attack. The attack injects unconstrained adversarial perturbations in a white-box scenario.}
\label{fig:sample}
\vspace{-0.15in}\
\label{fig:case_ImgJP}
\end{figure*}

\begin{figure*}[t]
% \vspace{-0.4in}
\begin{center}
\includegraphics[width=1.00\linewidth]{Figure/VAA.pdf}
\end{center}
\vspace{-0.05in}
\caption{Responses from LLaVA-13B after the VAA attack. The attack injects unconstrained adversarial perturbations in a white-box scenario. Although R2D2 also provided a benign response, it has a mistake by starting with ``Timothy'' rather than ``Kyle''. In comparison,  the response from {\name} is more concise and accurate.}
\label{fig:sample}
\vspace{-0.15in}\
\label{fig:case_VAA}
\end{figure*}

\begin{figure*}[t]
\vspace{-0.4in}
\begin{center}
\includegraphics[width=0.9\linewidth]{Figure/gcg.pdf}
\end{center}
\vspace{-0.05in}
\caption{Responses from LLaVA-13B after the GCG attack. We skip the image input for a more efficient implementation. The attack injects adversarial text suffix into toxic requests. It is a white-box attack method.}
\label{fig:sample}
\vspace{-0.15in}\
\label{fig:case_gcg}
\end{figure*}

\begin{figure*}[t]
\vspace{-0.4in}
\begin{center}
\includegraphics[width=0.9\linewidth]{Figure/Autodan.pdf}
\end{center}
\vspace{-0.05in}
\caption{Responses from LLaVA-13B after the AutoDAN attack. We skip the image input for a more efficient implementation. The attack injects adversarial text strings into toxic requests. It is a white-box attack method.}
\label{fig:sample}
\vspace{-0.15in}\
\label{fig:case_autodan}
\end{figure*}

\begin{figure*}[t]
\vspace{-0.4in}
\begin{center}
\includegraphics[width=0.9\linewidth]{Figure/figstep.pdf}
\end{center}
\vspace{-0.05in}
\caption{Responses from LLaVA-13B utilizing the FigStep method. It is a black-box attack method, where the attacker directly feeds the topographic image and paraphrased prompt into different models.}
\label{fig:sample}
\vspace{-0.15in}\
\label{fig:case_figstep}
\end{figure*}

\begin{figure*}[t]
\vspace{-0.4in}
\begin{center}
\includegraphics[width=0.9\linewidth]{Figure/mmbench.pdf}
\end{center}
\vspace{-0.05in}
\caption{Responses from LLaVA-13B following the MM-SafetyBench attack. It is a black-box attack method, where the attacker directly feeds the topographic image and paraphrased prompt into different models.}
\label{fig:sample}
\vspace{-0.15in}\
\label{fig:case_mmbench}
\end{figure*}

\begin{figure*}[t]
\vspace{-0.4in}
\begin{center}
\includegraphics[width=0.9\linewidth]{Figure/utility1.pdf}
\end{center}
\vspace{-0.05in}
\caption{Model responses on benign image-text pairs. Results are conducted on LLaVA-13B.}
\label{fig:sample}
\vspace{-0.15in}\
\label{fig:utility1}
\end{figure*}

\begin{figure*}[t]
\vspace{-0.4in}
\begin{center}
\includegraphics[width=0.9\linewidth]{Figure/utility2.pdf}
\end{center}
\vspace{-0.05in}
\caption{Model responses on benign image-text pairs. Results are conducted on LLaVA-13B.}
\label{fig:sample}
\vspace{-0.15in}\
\label{fig:utility2}
\end{figure*}







