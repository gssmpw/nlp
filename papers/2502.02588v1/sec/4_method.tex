% \section{The general case: Proof of \texorpdfstring{\Cref{thm:main-decomp}}{Theorem 1.6}}\label{sec:algo}

First, we show that data structure of \Cref{l:max_min_query} can be used to compute distances witnessed by shortest paths that pass through a constant-size separator.

\begin{lemma}\label{l:single_adhesion}
Fix a constant $k \in \mathbb{N}$. There exists an algorithm which as the input receives an edge-weighted graph $G$ on $n$ vertices and $m$ edges together with a partition of its vertices into three sets $A, B, C$ such that $|B| \leq k$ and there are no edges between $A$ and $C$, and as the output computes $\max_{c \in C} \dist(a, c)$ for every $a \in A$. The running time is $\Oh(m \log n + n \log^{k - 1} n)$.
\end{lemma}

\begin{proof}
Let $B = \{b_1, \ldots, b_k\}$. For any $a \in A, c \in C$, we have $\dist(a, c) = \min_{i \in [k]} \dist(a, b_i) + \dist(c, b_i)$. First, we run Dijkstra's algorithm from every vertex in $B$ to find $\dist(v, b_i)$ for every $v \in V(G)$ and $i \in [k]$. Next, we use \Cref{l:max_min_query} to construct a data structure $\mathbb{D}$ for the point set $\{(\dist(c, b_1), \dots, \dist(c, b_k))\colon c\in C\}\subseteq \mathbb{R}^k$. Now, the value $\max_{c \in C} \dist(a, c)$ for any given $a$ is equal to the answer of $\mathbb{D}$ to the query with argument $(\dist(a, b_1), \dots, \dist(a, b_k))$.
\end{proof}

After computing the distances over a constant-size separator, we will use the following observation to simplify one of the sides of the separation.

\begin{lemma}\label{l:inserting_paths}
Let $G$ be a edge-weighted connected graph and let $A, B, C$ be a partition of its vertices such that there are no edges between $A$ and $C$. For every pair of vertices $u, v \in B$, let $P_{u, v}$ be any shortest path from $u$ to $v$ with all internal vertices in $C$ (assuming such a path exists).

Let $G'$ denote a graph obtained from $G[A \cup B]$ by adding an edge from $u$ to $v$ of weight equal to the length of $P_{u, v}$, for all $u, v \in B$ for which $P_{u, v}$ exists. Then,  $$\dist_G(s, t) = \dist_{G'}(s, t)\qquad\textrm{for all }s,t\in A\cup B.$$
\end{lemma}
\begin{proof}
Let $G''$ be the graph obtained by adding new edges of $G'$ to $G$.
Fix any $s, t \in A \cup B$ and let $P$ denote the shortest path from $s$ to $t$ in $G''$ which minimizes the number of vertices from $C$ visited. Naturally, the weight of $P$ is equal $\dist_G(s, t)$. Assume that such path visits at least one vertex of $C$. Then, the path $P$ is of the form $s \xrightarrow{P_1} x \xrightarrow{P_2} y \xrightarrow{P_3} t$, where $x, y \in B$ and all the internal vertices of $P_2$ are in $C$. By the construction of $G'$, $P_2$ can be replaced with a direct edge from $x$ to $y$ of the same weight. We obtain a same weight path with a smaller number of vertices of $C$ visited, which is a contradiction. Therefore, $P$ is entirely contained in $A \cup B$, hence it exists in $G'$. This shows that $\dist_G(s, t) = \dist_{G'}(s, t)$.
\end{proof}


The next lemma encapsulates the main algorithmic content of the proof of \Cref{thm:main-decomp}. The algorithm will split the tree decomposition provided on input into smaller parts for which the eccentricities are easier to calculate. We use the following lemma to handle a single such part.
\begin{lemma}\label{l:star}
Fix constants $k, g \in \mathbb{N}, 0 < \delta < \frac{1}{54}$. Assume we are given $n \in \mathbb{N}$, an edge-weighted graph $G$ on at most $n$ vertices with a weight function $w \colon E(G) \to \mathbb{N}$, a vertex subset $A$ and a collection of non-empty vertex subsets $V_0, V_1, \dots, V_\ell$ satisfying the following conditions:
\begin{itemize}[nosep]
	\item The sum of weights of all the edges in $G$ is bounded by $\Oh(n)$.
	\item $V(G) \setminus A = V_0 \cup V_1 \cup \dots \cup V_\ell$.
	\item $|A| \leq k$.
	\item For every $i \in [\ell]$, $G[V_i \setminus V_0]$ is connected, $N_G(V_i \setminus V_0) = V_i \cap V_0$, $|V_i| = \Oh(n^\delta)$, and $|V_0 \cap V_i| \leq 4$.
	\item For all $i, j \in [\ell], i \neq j$, $V_i \setminus V_0$ and $V_j \setminus V_0$ are disjoint and non-adjacent in $G$.
	\item Every edge $uv \in E(G)$ with $u, v \not\in A$ is contained in $G[V_i]$ for some $i\in \{0,1,\ldots,\ell\}$.
	\item The graph obtained by taking $G[V_0]$ and adding a clique on $V_0 \cap V_i$ for every $i \in [\ell]$ has Euler genus bounded by $g$.
\end{itemize}
Then, we can compute the eccentricity of every vertex of $G$ in time $\Oh \left( n^{1 + \frac{150 + 54 \delta}{151}} \log^k n \right)$.
\end{lemma}

\begin{proof}
Fix $\delta' = \frac{1 + 97 \delta}{151}$; we have $\delta' - \delta = \frac{1 - 54\delta}{151} > 0$.
Let $E_i$ denote the set of edges with one endpoint in $V_i$ and the other endpoint in $V_i \setminus V_0$. For $i \in [\ell]$, we shall say that $V_i$ is {\em{heavy}} if the sum of weights of $E_i$ is larger than $n^{\delta'}$. Since the sets $E_i$ are pairwise disjoint and the total sum of weights of all the edges is bounded by $\Oh(n)$, the number of heavy subsets is bounded by $\Oh(n^{1 - \delta'})$. Without loss of generality, we may assume that $V_{\ell' + 1}, \dots, V_\ell$ are heavy and $V_1, \dots, V_{\ell'}$ are not, for some $\ell'\in \{0,\ldots,\ell\}$.


For any source vertex $s$, we can calculate distances from $s$ to every vertex of $G$  using breadth first search in time $\Oh(\sum_{e \in E(G)} w(e)) = \Oh(n)$.
In particular, for every $\ell' < i \leq \ell$, we can compute the distances from every vertex of $V_i$ to every vertex of $G$ in total time $\Oh(n^{2 - \delta' + \delta})$, because $$|V_{\ell'+1}\cup \ldots\cup V_{\ell}|\leq n^{1-\delta'}\cdot \Oh(n^\delta)=\Oh(n^{1-\delta'+
\delta}).$$
Additionally, we calculate distances $\dist_G(a, v)$ for every $a \in A, v \in V(G)$ in time $O(n)$.

For every $i \in [\ell]$ and $u,v \in V_0 \cap V_i$, there exists a shortest path $P_{i,u,v}$ from $u$ to $v$ with all internal vertices belonging to $V_i - V_0$ due to the assumption that $G[V_i - V_0]$ is connected and $N_G(V_i - V_0) = V_i \cap V_0$. Therefore, the distance from $u$ to $v$ is bounded by the sum of weights of edges in $E_i$. In particular, for $i \in [\ell']$, $\dist_G(u, v) \leq n^{\delta'}$.

We define $\widetilde{G}$ to be the graph obtained by taking $G[A \cup V_0 \cup \dots \cup V_{\ell'}]$ and applying the following operation for every $i \in \{\ell' + 1, \dots, \ell\}$:
for each pair of vertices $u, v \in A \cup (V_0 \cap V_i)$, add an edge in $\widetilde{G}$ between $u$ and $v$ with weight equal to the total weight of $P_{i,u,v}$. For a fixed $i, u$, we can find $P_{i, u, v}$ for all $v$ using breadth first search in time $\Oh(n)$. Taking a sum over all $i, u$, we get that $\tilde{G}$ can be computed in total time $\Oh(n^{2 - \delta'})$.


\begin{claim}\label{cl:wG}
The sum of the edge weights in $\widetilde{G}$ is $\Oh(n)$. Moreover, for all $u, v \in V(\widetilde{G})$, we have $\dist_{\widetilde{G}}(u, v) = \dist_{G}(u, v)$.
\end{claim}

\begin{proof}
Consider $i \in \{\ell' + 1, \dots, \ell\}$ and any $u, v \in A \cup (V_0 \cap V_i)$ for which we added an edge. Its weight is bounded by the sum of weights of edges in $E_i$. Therefore, the total weight of all edges added is at most
$$
\sum_{i \in \{\ell' + 1, \dots, \ell\}} \left( |A \cup (V_0 \cap V_i)|^2 \sum_{e \in E_i} w(e) \right) \leq (4 + k)^2 \sum_{e \in E(G)} w(e) = \Oh(n).
$$
This proves the first part of the claim.

For the second part of the claim, consider any $i \in \{\ell' + 1, \dots, \ell \}$ and observe that by our assumptions, $A \cup (V_0 \cap V_i)$ separates $(V_0 \cup \dots \cup V_{\ell'} \cup V_{i + 1} \cup \dots \cup V_\ell) \setminus V_i$ from $V_i \setminus V_0$. Hence it suffices to repeatedly apply \Cref{l:inserting_paths}.
\end{proof}

For every $u \in V(\widetilde{G})$, we have $\ecc_G(u) = \max(\ecc_{\widetilde{G}}(v), \max_{v \in V(G) \setminus V(\widetilde{G})} \dist_G(u, v))$. Note, that we already know all the distances $\dist_G(u, v)$ for $v \in V(G) \setminus V(\widetilde{G})$. Similarly, we can already compute $\ecc_G(u)$ for every $u \in V(G) \setminus V(\widetilde{G})$. Therefore, it remains to compute $\ecc_{\widetilde{G}}(v)$ for each $v \in V(\widetilde{G})$. Our goal is to show that this can be done efficiently using \Cref{l:main_ecc}.

Now, let $G'$ be the graph obtained from $\tilde{G}$ by replacing every edge $e$ non-indicent to $A$ with $w(e)\geq 2$ with a path of length $w(e)$ consisting of unit-weight edges. This operation again preserves the distances. Since the sum of edge weights in $\tilde{G}$ is of $\Oh(n)$, the total number of vertices in $G'$ is of $\Oh(n)$. For $0 \leq i \leq \ell'$, we write $V'_i$ to denote the set $V_i$ together with all the vertices added as a part of a path between two endpoints in $V_i$.
As $V_i$ is not heavy for each $i\in [\ell']$, we have
$$
|V'_i \setminus V'_0| \leq |V_i| + \sum_{e \in E_i} w(e) = \Oh(n^{\delta'})\qquad \textrm{for all }i\in [\ell'].
$$

Let $G_0$ denote the graph $G'[V'_0]$ and let $G_0^*$ denote the graph $G'- A$ with $V'_i - V'_0$ contracted to a single vertex $v_i^*$, for each $i \in [\ell']$; note that, all edges of $G_0$ and $G_0^*$ have unit weight.

\begin{claim}
	The graph $G_0^*$ is does not contain $K_{t}$ as a minor, where $t = \Oh(\sqrt{g})$.
\end{claim}

\begin{proof}
Let $\bar{G}_0$ denote the graph obtained by taking $G_0$ and adding a clique on $V_0 \cap V_i$ for every $i \in [\ell']$.
By lemma assumptions and the fact that subdividing edges does not increase the Euler genus, $\bar{G}_0$ has Euler genus at most $g$. In particular, $\bar{G}_0$ is $K_{t'}$-minor-free for some $t' = \Oh(\sqrt{g})$, because the Euler genus of $K_{t'}$ is $\Omega({t'}^2)$.

Similarly, let $\bar{G}_0^*$ be the graph obtained by taking $G_0^*$ and adding a clique on each $V_0 \cap V_i$.
Note, that $\bar{G}_0^* - \{v_1^*, \dots, v_{\ell'}^*\}$ is precisely $\bar{G}_0$. Let $t = \max(t', 6)$.
Recall that a minor model of a clique $K_t$ consists of $t$ pairwise vertex-disjoint connected subgraphs, called
branch sets, such that there is at least one edge between each pair of the branch sets.
Consider a minor model $\varphi$ of $K_{t}$ in $\bar{G}^*_0$.
Note that $\varphi$ cannot contain any singleton branch set of the form $\{v^*_i\}$, for the degree of $v^*_i$ in $\bar{G}^*_0$ is at most $4 < t - 1$. Furthermore, since $N_{\bar{G}^*_0}(v^*_i) = V_0 \cap V_i$, any branch set containing $v^*_i$ and at least one other vertex contains some $u \in V_0 \cap V_i$, and $N_{\bar{G}^*_0}(v^*_i)\subseteq N_{\bar{G}^*_0}(u)$, hence removing $v^*_i$ from this branch set preserves the model. Therefore, we can assume without loss of generality that all branch sets of $\varphi$ are disjoint from $\{v^*_1, \dots, v^*_{\ell'}\}$, hence $\varphi$ is a minor model of $K_{t}$ in $\bar{G}_0$. This is a contradiction, as $t \geq t'$ and $\bar{G}_0$ is $K_{t'}$-minor-free. Therefore, $\bar{G}_0^*$ is $K_t$-minor-free, hence $G_0^*$ also.
\end{proof}

Let $\rho' = \frac{2 - 108 \delta}{151} > 0$. The graph $G^*_0$ is a unit-weight graph and is $K_{t}$-minor-free.
Hence, by applying \Cref{t:r_division} to $G^*_0$ (with $\varepsilon = \rho'/2$)
we obtain an $n^{\rho'}$-division $\mathcal{R}_0$ in time $\Oh(n^{1 + \rho'})$.
We extend it to $G' - A$ by mapping every contracted vertex $v^*_i$ to $N_{G' - A}[V'_i - V'_0] = (V'_i - V'_0) \cup (V_0 \cap V_i)$. Formally, we put $V''_i \coloneqq N_{G' - A}[V'_i - V'_0]$ and 
$$
\mathcal{R} \coloneqq \left\{ (R_0 \cap V'_0) \cup \bigcup_{i \colon v^*_i \in R_0} V''_i \colon R_0 \in \mathcal{R}_0 \right\}.
$$

Now, we argue that $\mathcal{R}$ is a reasonable division of $G' - A$. Clearly, all sets $R \in \mathcal{R}$ are connected in $G' - A$. Pick any $R \in \mathcal{R}$ and let $R_0$ be its corresponding set in $\mathcal{R}_0$.
Every vertex $v^*_i$ is mapped to a set of size $\Oh(n^{\delta'})$, therefore
$$|R| \leq |R_0| \cdot \Oh(n^{\delta'}) = \Oh(n^{\rho' + \delta'}).$$

By our construction, for every $i \in [\ell']$, $R$ is either disjoint from $V'_i - V'_0$ or contains whole $N_{G' - A}[V'_i - V'_0]$. This means that no vertex belonging to any $V'_i - V'_0$ can be in $\partial R$, hence $\partial R \subseteq V'_0$.

Pick any $u \in \partial R \cap R_0$. Assume that $u \not\in \partial R_0$. Then every vertex of $N_{G_0^*}(u)$ must be in $R_0$, hence $N_{G - A'}(u) \subseteq R$, which is a contradiction. This means that $\partial R \cap R_0 \subseteq \partial R_0$.

Pick any $u \in \partial R - R_0$. Then, $u \in V_0 \cap V_i$ for some $i \in [\ell']$ such that $v_i^* \in R_0$. Moreover, $v_i^* \in \partial R_0$ and is adjacent to $u$ in $G_0^*$. The number of such $u$ is bounded by $4 |\partial R_0 \cap \{ v_1^*, \dots, v_{\ell'}^* \}|$.

Putting two cases together, we obtain:
$$
\sum_{R \in \mathcal{R}} |\partial R| = \sum_{R \in \mathcal{R}} \left(|\partial R \cap R_0| + |\partial R - R_0|\right) \leq \sum_{R_0 \in \mathcal{R}_0} \left(|\partial R_0| + 4 |\partial R_0 \cap \{ v_1^*, \dots, v_{\ell'}^* \}|\right) = \Oh(n^{1 - \frac{1}{2}\rho'}).
$$

It remains to show the following claim.

\begin{claim}
Pick any $R \in \mathcal{R}, s_R \in R$. The number of different distance profiles on $R$ relative to $s_R$ in $G' - A$ is of $\Oh(n^{48\rho' + 54\delta'})$.
\end{claim}
\begin{proof}
We look at every vertex $v \in V(G') \setminus A$ and consider three cases: $v \in R$, $v \in V'_0$, and $v \in V'_i \setminus (V'_0 \cup R)$ for some $i \in [\ell']$. By our construction, $R \cap V'_0$ is non-empty, hence w.l.o.g. we can assume that $s_R \in V'_0$ as whether two vertices have the same profile on $R$ is independent of the choice of the pivot vertex.

In the first case, there are at most $|R| = \Oh(n^{\rho' + \delta'})$ such vertices, hence they realise at most that many profiles.

In the second case, we want to observe that profile of any vertex $v \in V'_0$ on $R$ depends only on its profile on $R \cap V'_0$ (relative to $s_R$). Pick any $t \in R - V'_0$. Then $t \in V'_i - V'_0$ for some $i \in [\ell']$, $V_i \cap V_0 \subseteq R \cap V'_0$, and every path from $v$ to $t$ intersects $V_i \cap V_0$. In particular, distances from $v$ to vertices of $V_i \cap V_0$ determine its distance to $t$, which proves the observation.

Let $\tilde{G}_0$ denote the graph obtained by taking $G'[V'_0]$ and for every $i \in [\ell'], u, v \in V_0 \cap V_i$ adding a disjoint path from $u$ to $v$ of length $\dist(u, v)$. Let $P_i$ denote the vertex set of paths added between $V_0 \cap V_i$. For every $t \in V'_0$ we have $\dist_{G' - A}(v, t) = \dist_{\tilde{G}_0}(v, t)$, so it suffices to bound the number of profiles on $R \cap V'_0$ in $\tilde{G}_0$. By our assumptions, $\tilde{G}_0$ has Euler genus bounded by $g$ and all $P_i$ are of size $\Oh(n^{\delta'})$.

Let $R_0$ be the set of $\mathcal{R}_0$ corresponding to $R$. Let $\tilde{R}_0$ denote the set $(R \cap V'_0) \cup \bigcup_{i : v^*_i \in R_0} P_i$. Such set is connected in $\tilde{G}_0$. Moreover, similarly to $R$, its size is $\Oh(n^{\rho' + \delta'})$. Applying \Cref{thm:distprofiles}, we get that the number of distance profiles on $\tilde{R}_0$ in $\tilde{G}_0$ is $\Oh(n^{12(\rho' + \delta')})$, which also bounds the number of profiles on $R$ in $G' - A$ realised by $V'_0$.

For the third case, assume $v \in V'_i \setminus (V'_0 \cup R)$ for some $i\in [\ell']$. Every path from $v$ to any vertex of $R$ in $G' - A$ intersects $V_i \cap V_0$. Let $v_1, \dots v_p$ be the vertices of $V_i \cap V_0$, where $p \leq 4$. The profile of $v$ on $R$ is then determined by the following:
\begin{itemize}[nosep]
 \item[(a)] the profile of each $v_j$ on $R$,
 \item[(b)] $\dist_{G' - A}(v, v_j) - \dist_{G' - A}(v, v_1)$ for each $2 \leq j \leq p$, and
 \item[(c)] $\dist_{G' - A}(s_R, v_j) - \dist_{G' - A}(s_R, v_1)$ for each $2 \leq j \leq p$ where $s_R$ is some pivot vertex of $R$.
\end{itemize}
By the previous case, the number of distance profiles of each $v_j$ is $\Oh(n^{12(\rho' + \delta')})$. The distances between $v$ and $v_j$ are bounded by $|V'_i|$, hence each quantity described in (b) can take $\Oh(n^{\delta'})$ different possible values. Similarly, since $v_1$ and $v_j$ are connected via $V'_i$, $|\dist_{G' - A}(s_R, v_j) - \dist_{G' - A}(s_R, v_1)| \leq \Oh(n^{\delta'})$. The number of different possible profiles of such $v$ is therefore bounded by $\Oh(n^{48(\rho' + \delta') + 6\delta'}) = \Oh(n^{48\rho' + 54\delta'})$. This finishes the proof of the claim.
\end{proof}

Now we can apply \Cref{l:main_ecc} to graph $G'$ with apex set $A$, $X = V(\widetilde{G})$, and the following constants: $$\rho = \rho' + \delta',\qquad \gamma = 1 - \frac{1}{2}\rho',\quad \textrm{and}\quad \alpha = 48\rho' + 54 \delta'.$$ This allows us to calculate all $V(\widetilde{G})$-eccentricities in $G'$ in time
$$
\Oh \left( \left(
	n^{ 2 - \frac{1}{2} \rho' } +
	n^{ 1 + 48\rho' + 54 \delta' }
\right) \log^k n \right) =
\Oh \left( n^{1 + \frac{150 + 54 \delta}{151}} \log^k n \right).
$$
Since for each $v\in V(\widetilde{G})$ we have $\ecc_{\widetilde{G}}(v) = \max_{u \in V(\widetilde{G})} \dist_{\widetilde{G}}(v, u) = \max_{u \in V(\widetilde{G})} \dist_{G'}(v, u)$, this means that we have successfully computed all the eccentricities in $\widetilde{G}$; and as we argued, this is enough to compute all the eccentricities in $G$ as well.

Finally, the total running time of the algorithm is
$$
\Oh \left( n^{1 + \frac{150 + 54 \delta}{151}} \log^k n + n^{2 - \delta' + \delta} \right) =
\Oh \left( n^{1 + \frac{150 + 54 \delta}{151}} \log^k n \right).
$$\qedhere
\end{proof}


\begin{lemma}\label{l:star2}
Fix constants $k, g \in \mathbb{N}, 0 < \delta < \frac{1}{54}$. Assume we are given $n \in \mathbb{N}$, an edge-weighted graph $G$ on at most $n$ vertices with a weight function $w \colon E(G) \to \mathbb{N}$, a vertex subset $A$ and a collection of non-empty vertex subsets $V_0, V_1, \dots, V_\ell$ satisfying the same conditions as in \Cref{l:star} with the following differences:
\begin{itemize}
	\item we don't require $G[V_i - V_0]$ to be connected and $V_i - V_0$ to be adjacent to whole $V_i \cap V_0$;
	\item instead of $|V_0 \cap V_i| \leq 4$, we require $|V_0 \cap V_i| \leq k$.
\end{itemize}
Then, we can compute the eccentricity of every vertex of $G$ in time $\Oh \left( n^{1 + \frac{150 + 54 \delta}{151}} \log^{k + 5g} n \right)$.
\end{lemma}

\begin{proof}
We will reduce our input to one which will satisfy the conditions of \Cref{l:star}. We start by addressing the adhesions $V_0 \cap V_i$ containing too many vertices.

Let $G_0$ denote the graph $G[V_0]$ with cliques placed at $V_0 \cap V_i$ for every $i \in [\ell]$.
For every $i \in [\ell]$ we repeat the following procedure: while $|V_0 \cap V_i| > 4$,
remove arbitrary $5$ vertices from $V_0 \cap V_i$. Since $|V_0 \cap V_i| \leq k$ for each $i\in [\ell]$,
this procedure can be implemented in total time $\Oh(n)$. As a result, at the end we have $|V_0 \cap V_i| \leq 4$ for all $i \in [\ell]$. Let $M$ be the set of all the removed vertices. By our assumptions, $G_0$ has Euler genus bounded by $g$, hence it cannot contain $g + 1$ pairwise disjoint copies of $K_5$
(as the Euler genus of a graph is the sum of the Euler genera of its 2-connected components~\cite{StahlB77} and $K_5$ is not planar). Each removed quintiple of vertices induces a $K_5$ in $G_0$, hence we have $|M| \leq 5g$. We set $A' = A \cup M$ and may thus assume that $V_i$ is disjoint from $A'$ for all $0 \leq i \leq \ell$.

Now, fix $i \in [\ell]$. Let $C^i_1, \dots, C^i_{r_i}$ denote the connected components of $V_i - V_0$ in $G - A'$. We define $W^i_j := N_{G - A'}[C^i_j]$ for every $j \in [r_i]$. Clearly, all $W^i_j$ induce a connected subgraph of $G$ and satisfy $N_{G - A'}(W^i_j - V_0) = W^i_j \cap V_0$. We put $V'_0 := V_0$ and enumerate
$$
\{V'_1, V'_2, \dots V'_{\ell'}\} := \{ W^i_j \colon i \in [\ell], j \in [r_i] \}.
$$
It is easy to verify that the sets $A'$ and $V'_0, V'_1, \dots, V'_{\ell'}$ satisfy the conditions of \Cref{l:star}. We apply said lemma to calculate the eccentricity of every vertex of $G$ in the desired time.
\end{proof}



The next statement is a reformulation of \Cref{thm:main-decomp}.

\begin{theorem}
Fix constants $k, g \in \mathbb{N}$. Assume we are given a graph $G$ on $n$ vertices together with its tree decomposition $(T, \beta)$ and a set of private apices $A_t \subseteq \beta(t)$ for each node $t\in V(T)$ such that the following conditions hold:
\begin{itemize}[nosep]
 \item For every node $t \in V(T)$, we have $|A_t| \leq k$.
 \item For every edge $st \in E(T)$,  we have $|\beta(v) \cap \beta(u)|\leq k$.
 \item For every node $t \in V(T)$, graph obtained by taking $G[\beta(t)] - A_t$ and turning  $(\beta(t) \cap \beta(s))\setminus A_t$ into a clique for every edge $st \in E(T)$ has Euler genus bounded by $g$.
\end{itemize}
Then, we can compute the eccentricity of every vertex of $G$ in time $\Oh \left( n^{1 + \frac{355}{356}} \log^{k + 5g} n \right)$.
\end{theorem}

\begin{proof}
We may assume that $|V(T)|\leq n$, for every tree decomposition with no two bags comparable by inclusion has this property; and adjacent comparable bags can be merged by contracting the edge between them.

For a node $t\in V(T)$, by the {\em{weight}} of $t$ we mean the size of the corresponding bag, that is, $|\beta(t)|$. For any subset of nodes $S \subseteq V(T)$, we define $\beta(S) \coloneqq \bigcup_{t \in S} \beta(t)$ By the {\em{weight}} of $S$, we mean the total weight of the elements of $S$, that is, $\sum_{t\in S} |\beta(t)|$. 

\begin{claim}\label{cl:weight-T}
The weight of $V(T)$ is of $\Oh(n)$.
\end{claim}

\begin{proof}
The sets $\beta'(t) := \beta(t) - \bigcup_{s \in N_T(t)} \beta(s)$ are pairwise disjoint. We have
$$
\sum_{t \in V(T)} |\beta(t)| =
\sum_{t \in V(T)} |\beta'(t)| + 2 \cdot \sum_{st \in E(T)} |\beta(s) \cap \beta(t)| \leq
|V(T)| + 2k|E(T)| = \Oh(n).
$$
\end{proof}

Since every bag induces a graph of bounded Euler genus, the number of edges contained in a bag is linear in its size. In particular, this implies that the total number of edges of $G$ is also bounded by $\Oh(n)$.

We set $$\delta \coloneqq \frac{1}{356}\qquad\textrm{and}\qquad \Delta \coloneqq \frac{355}{356}.$$ Root the tree $T$ in an arbitrarily chosen node; this naturally imposes an ancestor-descendant relation in $T$ (for convenience, every node is considered its own ancestor and descendant).

We start by partitioning $T$ into connected subtrees using the following procedure.
We proceed bottom-up over $T$, processing nodes in any order so that a node is processed after all its strict descendants have been processed. Along the way, we mark some nodes and split the edges of $T$ into heavy and light. Let $t \in V(T)$ be the currently processed non-root node of $T$ and let $e \in E(T)$ be the edge connecting $t$ with its parent. If the total weight of all the unmarked nodes that are descendants of $t$ is at least $n^\delta$ (recall that this includes $t$ itself as well), then we declare $e$ heavy and mark all the descendants of $t$ that were unmarked so far. Otherwise, the edge $e$ is declared light and the procedure proceeds to further nodes of $T$.

Observe that
removing all heavy edges splits $T$ into connected subtrees, say $T'_1, \cdots T'_m$. All of the subtrees, except for possibly the subtree containing the root node, are of weight at least $n^\delta$. In particular, the number of subtrees $m$, and therefore the number of heavy edges, is  bounded by $\Oh(n^{1 - \delta})$. Moreover, in every subtree $T'_i$, removing the node closest to the root splits $T'_i$ into smaller components, each of weight less than $n^\delta$.

Fix a heavy edge $e$ and let $T^e_1$ and $T^e_2$ be the two subtrees into which $T$ splits after removing~$e$. Let $X^e_i = \beta(T^e_i)$ for $i \in \{1, 2\}$. Put $A_e = X^e_1 \setminus X^e_2$, $C_e = X^e_2 \setminus X^e_1$, and $B_e = X^e_1 \cap X^e_2$. By the properties of tree decompositions, such choice of $A_e, B_e, C_e$ satisfies the conditions of \Cref{l:single_adhesion}, hence in time $\Oh(n \log^{k - 1} n)$ we can compute $\max_{v \in X^e_2} \dist_G(u,v)$ for every $u \in X^e_1$, and $\max_{u \in X^e_1} \dist_G(u,v)$ for every $v \in X^e_2$. Computing this for every heavy edge $e$ takes total time $\Oh(n^{2 - \delta} \log^{k - 1} n)$.

Fix any subtree $T'=T'_j$. Let $e_1 = t^{e_1}_1t^{e_1}_2, e_2 = t^{e_2}_1 t^{e_2}_2, \dots, e_\ell = t^{e_\ell}_1 t^{e_\ell}_2$ denote the heavy edges incident to $T'$, where $t^{e_i}_1 \in V(T')$ and $V(T') \subseteq V(T_1^{e_i})$ for every $i \in [\ell]$.
For a vertex $v \in \beta(T')$, let
$$d_0(v) = \max_{u \in \beta(T')} \dist_G(v, u)\qquad\textrm{and}\qquad d_i(v) = \max_{u \in X_2^{e_i}}\dist_G(v,u),\quad\textrm{for } i \in [\ell].$$ We have $\ecc(v) = \max \{ d_i(v)\colon i\in \{0,1,\ldots,\ell\}\}$.The values of $d_i(v)$ are already calculated for all $i\in [\ell]$, hence it remains to compute $d_0(v)$.

For every $i \in [\ell]$ and every pair of vertices $u, v \in \beta(t^{e_i}_1) \cap \beta(t^{e_i}_2)$ we find a shortest path between $u$ and $v$ with all internal vertices inside $X^{e_i}_2$ (or determine that it doesn't exist). For a fixed $u, v$ this can be done in time $\Oh(n)$. Since in total we perform this step at most $2k^2$ times per heavy edge, it takes $\Oh(n^{2 - \delta})$ time in total. Let $P_{i, u, v}$ denote such path, assuming it exists.

Let $G'$ denote the graph obtained from $G[\beta(T')]$ by taking every $i, u, v$ for which $P_{i, u, v}$ exists and adding an edge between $u$ and $v$ of weight equal to the total weight of $P_{i, u, v}$.
The weight of every edge inserted in $\beta(t^{e_i}_1) \cap \beta(t^{e_i}_2)$ is bounded by $|X^{e_i}_2|+1$. The total weight of all edges inserted is therefore at most
$$
\sum_{i \in [\ell]} |\beta(t^{e_i}_1) \cap \beta(t^{e_i}_2)|^2 \cdot (|X^{e_i}_2|+1) \leq
k^2 \sum_{i \in [\ell]} (|X^{e_i}_2|+1) = \Oh(n),
$$
where the last equality follows from the fact that all the trees $T^{e_i}_2$ are pairwise disjoint.
By \Cref{l:inserting_paths}, we have $\dist_{G'}(u, v) = \dist_G(u, v)$ for each $u, v \in \beta(T')$. Hence, computing $d_0(v)$ for every $v \in \beta(T')$ is equivalent to computing the eccentricity of every vertex in $G'$.

If the size of $\beta(T')$ is smaller than $n^\Delta$, we compute the eccentricities naively in time $\Oh(|\beta(T')|^2)$, 
noting that $G'$ has $\Oh(|\beta(T')|)$ edges (thanks to Claim~\ref{cl:weight-T} and bounded genus assumption 
of the last bullet of the theorem statement). Otherwise, we argue that we can use the algorithm in \Cref{l:star} as follows.

Let $t$ be the node of $T'$ closest to the root. Let $s_1, \dots, s_p$ be the children of $t$ in $T$ and let $T''_i$ denote the connected component of $T' - \{t\}$ containing $s_i$. Set $V_0 = \beta(t)$ and $V_i = \beta(T''_i)$ for $i \in [p]$.

It is now easy to verify that $G'$ and sets $A, \{V_i\colon 0\leq i\leq p\}$ selected this way satisfy the assumptions of \Cref{l:star2}. This allows us to use it to compute the eccentricities in $G'$ in time
$$
\Oh \left( n^{1 + \frac{150 + 54\delta}{151}} \log^{k + 5g} n \right) =
\Oh \left( n^{1 + \frac{354}{356}} \log^{k + 5g} n \right).
$$
As we argued, from these eccentricities, we may easily compute all the eccentricities in $G$.

Now, let us analyse the total running time of the whole algorithm. We invoke \Cref{l:star} $\Oh(n^{1 - \Delta})$ times, since we apply it only to subtrees $T'_i$ of size at least $n^\Delta$. The total running time of those applications is hence
$$
\Oh \left( n^{2 + \frac{354}{356} - \Delta} \log^{k + 5g} n \right) =
\Oh \left( n^{1 + \frac{355}{356}} \log^{k + 5g} n \right).
$$
We compute the eccentricities naively for subtrees smaller than $n^\Delta$, hence the total running time of this computation is
$$
\sum_{i \in [m] \colon |\beta(T'_i)| \leq n^\Delta} |\beta(T'_i)|^2 \leq
n^\Delta \cdot \sum_{i \in m} |\beta(T'_i)| = \Oh(n^{1 + \Delta})=\Oh\left(n^{1+\frac{355}{356}}\right).
$$
The rest of computation can be done in $\Oh(n^{2 - \delta} \log^k n)$. Therefore, the whole algorithm runs in time $\Oh \left( n^{1 + \frac{355}{356}} \log^{k + 5g} n \right)$.
\end{proof}


%HERE
\section{Proposed Method}
In this section, we introduce our method for calibrated preference optimization. We refer to Fig.~\ref{fig:overview} for overview.

\subsection{Motivation}\label{sec:problem}
The challenges in multi-reward optimization is in achieving the Pareto optimality among reward signals, especially even when they conflict. 
For example, when optimizing models for image aesthetics, it often results in reduced image-text alignment as aesthetic reward models do not consider textual information (\emph{e.g.}, see Tab.~\ref{tab:single}). 
One common practice is to use weighted sum of rewards as a proxy for the total reward function~\citep{clark2023directly}.
However, those rigid formulations cannot effectively consider all aspects of utilities, which might lead to suboptimal performance, \emph{e.g.}, biased towards certain reward signals.
Another approach is using the rewarded soups~\citep{rame2024rewarded}, which merges the independently reward fine-tuned model with model soup~\citep{wortsman2022model}.
Nevertheless, optimizing for a single reward is prone to reward over-optimization~\citep{gao2023scaling, rafailov2024scaling} and result in significant performance loss.

Our core assumption is that the difficulties in multi-reward optimization lie in the inconsistency between the black-box distribution of rewards. 
To address this challenge, we propose calibrated preference optimization to minimize inconsistencies by fine-tuning with general and unified metrics. 
In the following, we provide details of our method.

% \vspace{1mm}
% \noindent
% {\bf Overview.}
% We provide our general training pipeline in Fig.~\ref{fig:overview}.
% We first generate multiple samples from pretrained model and use multiple reward models to construct preference dataset for fine-tuning (\emph{e.g.}, Fig.~\ref{fig:overview}(a)). 
% Then to effectively leverage the information from reward model, we use reward calibration to approximate the general preference distribution induced by reward models (Fig.~\ref{fig:overview}(b)).
% Furthermore, we introduce a pair selection method, especially, frontier-based rejection sampling for multi-reward scenario to achieve Pareto optimality (Fig.~\ref{fig:overview}(c)).

%; in Sec.~\ref{sec:capo} we introduce calibrated rewards that accommodate various reward signals into an unified measure of goodness, and fine-tuning objective that distills this into diffusion models.
%Then, in Sec.~\ref{sec:frs}, we provide data selection method that equips joint optimization of multiple reward signals with efficiency.
%Lastly, we provide a loss weighting scheme to improve diffusion preference optimization (Sec.~\ref{sec:adv}).

% the inconsistency of reward signals by reward fine-tuning of T2I diffusion models introduces an unified and general metrics to handle the inconsistency between reward signals. effectively handles th
% One of the common practice in multi-reward optimization is to use the weighted sum of rewards as a proxy for the total reward function~\citep{clark2023directly} and control the trade-off by choosing coefficients.
% However, this resorts to heuristics that often results in suboptimal performance
% Alternative approach is using rewarded soups~\citep{rame2024rewarded}, which performs reward optimization for each model, and use model soup~\citep{wortsman2022model} (\emph{i.e.}, model weight interpolation) to merge fine-tuned models.

% While model soups have shown  
% multi-reward optimization necessity
% calibration necessity
% better options for multi-task learning
% \vspace{0.05in}
% \noindent
% {\bf Problem setup.}

% \vspace{0.05in}
% \noindent
% {\bf Method overview.}
% Given a prompt dataset $\mathcal{D}$, we generate $N$ images per prompt
% We generate $N$ images per prompt

% 1. general explanation
%  - our goal is to improve the performance of T2I diffusion model (image-text alignment, aesthetics)
%  - we assume no high-quality data / paired human preference data, but only reward signals as a proxy objective to align.
%  - rewards are from oracle, not differentiable
%  - we consider multi-reward
% 1. why do we need multi-reward?
%  - Using a single reward cannot fully achieve improvement, 
%  - single reward is more prone to reward over-optimization
% 2. Main challenges
% - reward signals are often not calibrated, they are good at predicting the better ones, but the values might not represent the general goodness of the data.
% - rewards form incoherent distributions, which disperses the effect at worst case, the model is biased towards certain reward signals. 
% Directly using Rewards are not uniformly scaled -> directly using the rewards can be misleading
% - Hard to obtain balanced optimization, is worse when objectives can conflict.
% - 
% usign the reward values from reward model can be misleading. 
% we use multi-reward to prevent over-optimization

% calibrated reward optimization
% - common objective e.g., RLHF trian reward model with bradley-terry model, and maximize the reward. however, using reward can be misleading b/c
% 1) reward does not represent the absolute goodness
% 2) different scales of reward prevents multi-reward optimization challenging

% \subsection{Problem setup}

% Our goal is to fine-tune a pretrained T2I diffusion models to improve its utility (\emph{e.g.}, image-text alignment, aesthetics) by using diverse reward signals that covers a wide-ranging aspect of text-to-image synthesis. 
% The reward optimization for T2I diffusion models has following challenges.
% First, 
% We remark some differences of our problem with previous works; in contrast to other works that utilizes high-quality data~\citep{dai2023emu} or human preference data~\citep{wallace2023diffusion} for fine-tuning, we narrow down the scope of data usage to be that generated from pretrained model. 
% Furthermore, we consider an oracle reward model, which supports a wider range of reward signals including large multimodal language models, than others that assumes differentiable rewards~\citep{clark2023directly}.

% Those problem setup gives us following challenges.
% First, it is unclear how to incorporate the reward information during fine-tuning. 
% Second, 

% Furthermore, 
% fine-tune a pre-trained T2I diffusion models to improve
% We consider fine-tuning T2I diffusion models to maximize the expected rewards of the output images from various prompts. 
% To this end, we collect dataset by sampling images from the base model, and obtain oracle rewards from reward model(s).
% $\max_\theta J(\theta)$
% collecting dataset by sampling images with various prompts, and updating the diffusion model with reward functions. 

% \paragraph{Comparison to previous works.}
% Previous works have also considered different setups in fine-tuning T2I diffusion models with reward models. Here, we elaborate the difference in the problem setup.
% While we only consider oracle feedback from the reward models, some works consider differentiable reward models and perform fine-tuning by using the reward gradients~\citep{clark2023directly, prabhudesai2023aligning}. 
% We remark that our method does not require differentiability of rewards, thus support more diverse form of rewards, and especially more beneficial when scaling to larger diffusion and reward models (\emph{e.g.}, using multimodal LLMs).
% Another line of works consider using pre-collected human-rated dataset such as Pick-a-pic dataset~\citep{kirstain2023pick} to perform DPO algorithm, but we assume there is no access to offline human preference dataset.

% and perform DPO While using DPO on pre-collected dataset showed promising results, collecting
% \citet{dong2023raft, prabhudesai2023aligning} considered using differentiable reward models to directly fine-tune T2I diffusion models with gradient signal from rewards, 


% \begin{itemize}
%     \item Difference between previous works
%     \item core challenge: how to mitigate reward-overoptimization while maximizing the gain?
%     \item how to achieve balanced gain over all axis
%     \item how to 
% \end{itemize}
% We assume the simplest setup

% \begin{itemize}
%     \item Our problem: how to fine-tune diffusion models when multiple rewards are given
%     \item We consider more general problem than previous methods, e.g., which requires gradient of reward or require the paired dataset collected.
%     \item Some approaches consider diffusion sampling as RL problem, yet those approach requires memory expensive sampling chain, which makes difficult to scale up to large diffusion models.
%     \item We consider generating images from current model, then score the images, and update the model. When doing this iteratively, this can simulate online learning. (see Fig. xx)
%     \item Two major questions: how to fine-tune (optimization method) and how to select training data (data)
%     \item Next section, we propose a novel approach in fine-tuning, which is not only effective for multi-reward problem, but also for a single reward optimization problem.
%     \item Then for multi-objective optimization, we propose an effective data sampling method for multi-objective problem.
% \end{itemize}


%HERE
%\subsection{Calibrated Preference Optimization}\label{sec:capo}
\subsection{CaPO}\label{sec:capo}
Although we consider reward models as a proxy to represent the utility of a sample, directly using the reward values can lead to unsatisfactory results if they are not properly calibrated. 
Specifically, when using Bradley-Terry model~\citep{bradley1952rank}, the reward value often does not measure the goodness of a sample, even though the model exhibits high prediction accuracy in classifying the human preference. 
% cannot most reward models are trained by performing ranking tasks (\emph{e.g.}, Bradley-Terry model~\citep{bradley1952rank}), where the value itself cannot meaningfully measure goodness of a sample. 
Furthermore, the varying range of reward becomes problematic when using multiple reward signals, making it difficult to obtain balanced updates. 
% the various range of different reward signals makes hard to obtain balanced optimization. distribution of reward values varies among different reward models, thus it requires heuristics to balance the trade-offs.
% and it requires heuristics to merge them.

\vspace{1mm}
\noindent
{\bf Calibrated rewards.}
To address these issues, we propose to use expected win-rate as a unified measure for maximization target. 
Formally, let $\mathbb{P}(\mathbf{x} \succ \mathbf{x}' | \mathbf{c})$ be a win-rate of data $\mathbf{x}$ over $\mathbf{x}'$ with prompt $\mathbf{c}$. 
We define the win-rate of data $\mathbf{x}$ over a distribution $p(\cdot | \mathbf{c})$:
\begin{equation}\label{eq:winrate}
    \mathbb{P}(\mathbf{x} \succ p | \mathbf{c}) \coloneqq \mathbb{E}_{\mathbf{x}' \sim p(\cdot | \mathbf{c})}\big[\mathbb{P}(\mathbf{x} \succ \mathbf{x}'|\mathbf{c})\big]\text{.}
\end{equation}
As our goal is to improve over reference model $p_{\textrm{ref}}$, we consider $\mathbb{P}(\mathbf{x} \succ p_{\textrm{ref}}|\mathbf{c})$ as our target of interest. 
By using expected win-rate over reference model, we directly seek for improvement over a pretrained model, which quantifies the general goodness of a data. Furthermore, the bounded range makes it more favorable for multi-reward optimization.
% favors to the application for multi-reward optimization.
Since the expected win-rate is not available in general,
we approximate it through averaging the pairwise win-rate computed by a reward model.
Suppose we generate $N$ batch of samples $\{\mathbf{x}_i\}_{i=1}^N$ from $p_{\textrm{ref}}(\cdot|\mathbf{c})$, then we define \emph{calibrated reward} $R_{\textrm{ca}}(\mathbf{x}_i, \mathbf{c})$ for each sample $i$:
\begin{equation}\label{eq:calre}
\vspace{-3mm}
    R_{\textrm{ca}}(\mathbf{x}_i, \mathbf{c}) = \frac{1}{N-1} \sum_{j\neq i} \sigma \big( r(\mathbf{x}_i, \mathbf{c}) - r(\mathbf{x}_j, \mathbf{c}) \big)\text{,}
\end{equation}
where we have $R_{\textrm{ca}}(\mathbf{x}, \mathbf{c}) \approx \mathbb{P}(\mathbf{x}\succ p_{\textrm{ref}}|\mathbf{c})$ for large $N$.

\vspace{1mm}
\noindent
{\bf CaPO loss.}
We replace $R(\mathbf{x}, \mathbf{c})$ in Eq.~\eqref{eq:obj} with $R_{\textrm{ca}}(\mathbf{x}, \mathbf{c})$, and introduce calibrated preference optimization objective that fine-tunes the model to maximize the calibrated reward. 
Similar to Eq.~\eqref{eq:dffpo}, we define CaPO loss by matching the difference of the calibrated rewards with regression loss~\citep{deng2024prdp, fisch2024robust}, which also guarantees the optimality condition. 
Specifically, given data pair $(\mathbf{x}^+, \mathbf{x}^-)$, we define CaPO objective:
% we define CRO loss by the regression objective between the difference between calibrated rewards and difference between implicit rewards:
% \begin{equation}
%     \bigg(R_{\textrm{ca}}(\mathbf{x}, \mathbf{c}) - R_{\textrm{ca}}(\mathbf{x}', \mathbf{c}) - \beta\log\frac{p_\theta(\mathbf{x}|\mathbf{c})p_{\textrm{ref}}(\mathbf{x}'|\mathbf{c}) }{p_{\textrm{ref}}(\mathbf{x}|\mathbf{c}) p_\theta(\mathbf{x}|\mathbf{c})} \bigg)^2\text{,}
% \end{equation}
% and corresponding CRO loss for diffusion model is given by
\vspace{-1mm}
\begin{align}\label{eq:capoloss}
\begin{split}
    \mathcal{L}_{\textrm{CaPO}}(\theta) &= \underset{t,\eps, \eps'}{\mathbb{E}}\bigg[
    \big( R_{\textrm{ca}}(\mathbf{x}^+, \mathbf{c}) - R_{\textrm{ca}}(\mathbf{x}^-, \mathbf{c}) \\
    &- \beta \big(R_\theta(\mathbf{x}_t^+,\mathbf{c},t) - R_\theta(\mathbf{x}_t^-, \mathbf{c}, t)\big)\big)_2^2\bigg]\text{,}
\end{split}
\end{align}
where $\mathbf{x}_t^+=\alpha_t\mathbf{x}^+ + \sigma_t\eps^+$, $\mathbf{x}_t^-=\alpha_t\mathbf{x}^- + \sigma_t\eps^-$, for $t\sim\mathcal{U}(0,1)$ and $(\eps^+, \eps^-)\sim \noise \times \noise$.
% \vspace{0.05in}
% \noindent
% {\bf Connection to IPO~\citep{azar2024general}.}
Note that CaPO is a special case of Eq.~\eqref{eq:dffpo} with $g(u)=(\Delta R - u)^2$, where $\Delta R = R^+ - R^- $ is a difference between calibrated rewards.
Thus, CaPO is a generalization of IPO~\citep{azar2024general}, which strictly assign $\Delta R = 1$ for all pairs.
Compared to IPO, CaPO assigns a dynamic target for the preference learning, which helps maximizing the gain without reward over-optimization. 
% Practically, this allows to explore smaller KL regularization without reward over-optimization (\emph{e.g.}, see xx).
% By using calibrated rewards, CaPO is beneficial at mitigating reward over-optimization compared to IPO. 
% In practice, this allows to use smaller KL regularization compared to IPO, which and helps 

% By using calibrated rewards, there are two advantages:
% First, CaPO helps mitigating reward over-optimization by incorporating the calibrated rewards.

% \lipsum[1]
% \noindent
% {\bf Comparison with IPO}
% By taking gradient of Eq.~\eqref{eq:cro}, we derive following equivalent form (see Appendix xx for derivation):
% \begin{equation}
%     \mathcal{L}_{\textrm{CRO}}(\theta) = \mathbb{E}_{t,\eps,\eps'}[(R_{\textrm{ca}}(\mathbf{x}, \mathbf{c}) - \beta Q_\theta(\mathbf{z}_t;\mathbf{c},t)) \|\eps_\theta(\mathbf{z}_t;\mathbf{c},t) - \eps\|_2^2 ]    
% \end{equation}

% and CRO loss for diffusion model is given as follows
% where we show that minimizer of CRO loss satisfies $p_\theta(\mathbf{x}|\mathbf{c}) = p_{\textrm{ref}}(\mathbf{x}|\mathbf{c})\exp (R_{\textrm{ca}}(\mathbf{x}, \mathbf{c}))$

% \noindent
% {\bf Comparison to DPO }


% We remark that calibrated reward is a sufficiently good estimator of $\mathbb{P}(\mathbf{x} \succ p_{\textrm{ref}}|\mathbf{c})$ compared to $r(\mathbf{x}, \mathbf{c})$. 
% Suppose $r^*(\mathbf{x},\mathbf{c})$ is a ground-truth reward model that follows Bradley-Terry model, then we have 
% \begin{equation}
%     \mathbb{P}(\mathbf{x} \succ p_{\textrm{ref}}|\mathbf{c}) = \mathbb{E}_{\mathbf{x}'\sim p_{\textrm{ref}}(\mathbf{x}|\mathbf{c})}[\sigma(r^*(\mathbf{x},\mathbf{c}) - r^*(\mathbf{x}', \mathbf{c}))]\text{.}
% \end{equation}
% Suppose the error of proxy reward $r(\mathbf{x}, \mathbf{c})$ in estimating $r^*(\mathbf{x}, \mathbf{c})$ can 

% Now our objective 
% \noindent
% {\bf Calibrated Reward Optimization}


% In practice, we approximate the win-rate by averaging the pairwise win-rate using reward models, \emph{i.e.}, given $N$ batch of samples $\mathbf{x}_1,\ldots, \mathbf{x}_N$, the calibrated reward $r_c(\mathbf{x},\mathbf{c})$  for  $\mathbf{x}$ we generate multiple samples $\{\mathbf{x}_i\}_{i=1}^N$

% Using Eq.~\eqref{eq:winrate} as reward function provides provides several advantages.
% Subsequently, by replacing $r(\mathbf{x}, \mathbf{c})$ in Eq.~\eqref{eq:obj} with $\mathbb{P}(\mathbf{x} \succ p_{\textrm{ref}} |\mathbf{c})$, our objective is given as follows:
% \begin{equation}\label{eq:cobj}
%     \max_\theta \mathbb{E}_{\mathbf{c},\mathbf{x}\sim p_\theta(\cdot|\mathbf{c})} \big[\mathbb{P}(\mathbf{x} \succ p_{\textrm{ref}}|\mathbf{c})\big] - \beta D_{\textrm{KL}}(p_\theta \| p_{\textrm{ref}})\text{.}
% \end{equation}
% In practice, we estimate the expected win-rate by averaging the pairwise win-rate 

% generate multiple samples from pretrained 
% to solve  
% $\mathbb{P}(\mathbf{x} \succ p | \mathbf{c})$
% Most of the reward models  
% The underlying misuse of reward models is due to the
% However, this often results in suboptimal performance when reward values are not properly calibrated,
% However, this can results in suboptimal performance when reward values are not properly calibrated, which requires heuristics such as reward standardization. 

% post-processing such as reward standardization. 
% However, most of the reward models are not properly calibrated, often results in misleading performance. 
% For example, most of the reward models such as human preference reward models are trained with ranking losses, which performs pairwise comparison to learning. Thus, 
% The common practice in reward optimization is directly use the reward valu use the reward as 


% Suppose we have access to the ground-truth reward model $r^*(\mathbf{x}, \mathbf{c})$, then it is reasonable choice to directly incorporate the reward values to optimize the model.
% To this end, we consider following reward-matching loss function for efficient offline optimization:
% \begin{equation}\label{eq:rewardmatching}
%     \mathbb{E}_{(\mathbf{x}, \mathbf{x}',\mathbf{c})}\left[ (r^*(\mathbf{x},\mathbf{c}) - r^*(\mathbf{x}', \mathbf{c}) - (r_\theta(\mathbf{x}, \mathbf{c}) - r_\theta(\mathbf{x}', \mathbf{c})))^2 \right]\text{,}
% \end{equation}
% where $\mathbf{x}, \mathbf{x}'\sim p_{\textrm{ref}}(\cdot | \mathbf{c})$ for $\mathbf{c}\sim\mathcal{D}$. 
% Then it is straightforward to check that the optimal $p_{\theta^*}$ that minimizes Eq.~\eqref{eq:rewardmatching} is equivalent to the optimal distribution of Eq.~\eqref{eq:rlhf}. 

% However, the ground-truth reward model $r^*$ is not known in practice, and we consider the proxy reward model to use for reward matching loss. From BT model, what we can assume is that $\|r^*(\mathbf{x}, \mathbf{c}) - r(\mathbf{x}, \mathbf{c}) - C(\mathbf{c}) \|_2^2 < \delta$ for some $\delta > 0$ and partition function $C(\mathbf{c})$. Then the difference between reward model $r(\mathbf{x}, \mathbf{c}) - r(\mathbf{x}', \mathbf{c})$ has at most error of $2\delta$. This makes problematic when $\delta$ is large. 

% \paragraph{Calibrated reward matching loss.}
% Instead of using $r(\mathbf{x}, \mathbf{c})$ as a target for our reward-matching loss, we consider win-rate $\mathbb{P}(\mathbf{x} \succ \mu | \mathbf{c})$ of $\mathbf{x}$ over a reference distribution $\mu$ as our target reward.
% While $\mathbb{P}(\mathbf{x} \succ \mu | \mathbf{c})$ is not achievable in general, we sample multiple images from $\mu$ to approximate $\mathbb{P}(\mathbf{x} \succ \mu | \mathbf{c})$. Suppose $N$ images $\mathbf{x}_1,\ldots, \mathbf{x}_N$ from $\mu(\cdot | \mathbf{c})$, then we compute the approximate win-rate $\hat{\mathbb{P}}(\mathbf{x} \succ \mu|\mathbf{c})$ as follows:
% \begin{equation}
%     \hat{\mathbb{P}}(\mathbf{x}_i \succ \mu | \mathbf{c}) = \frac{1}{N}\sum_{i=1}^N \sigma \big(r(\mathbf{x}, \mathbf{c}) - r(\mathbf{x}_i, \mathbf{c})\big)\text{.}
% \end{equation}
% Remark that this gives us multiple advantages over using $r(\mathbf{x}, \mathbf{c})$ as target. 
% First, for sufficiently large $N$, the error can be reduced with rate $O(\tfrac{1}{N})$. 
% Specifically, let us assume $|r^*(\mathbf{x}, \mathbf{c}) - r(\mathbf{x}, \mathbf{c}) -C(\mathbf{c}) | < \delta$. Then we have 
% \begin{align*}
% \begin{split}
%     &|\sigma (r(\mathbf{x}, \mathbf{c}) - r(\mathbf{x}', \mathbf{c})) - \sigma( r^*(\mathbf{x}, \mathbf{c}) - r^*(\mathbf{x}', \mathbf{c})| \\
%     <&\tfrac{1}{4} | r(\mathbf{x}, \mathbf{c}) - r(\mathbf{x}', \mathbf{c}) - (r^*(\mathbf{x}, \mathbf{c}) - r^*(\mathbf{x}', \mathbf{c})| < \frac{\delta}{2}\text{.}
% \end{split}
% \end{align*}
% Then for $N$ samples, we have 
% \begin{equation}
%     |\mathbb{P}(\mathbf{x} \succ \mu |\mathbf{c}) - \hat{\mathbb{P}}_N(\mathbf{x} \succ \mu | \mathbf{c}) | <\frac{\delta}{2N}\text{.}
% \end{equation}
% Furthermore, since the win-rate is in range $[0,1]$, this makes more amenable for multi-reward optimization. Otherwise, the vast difference in scale of $r(\mathbf{x}, \mathbf{c})$ makes multi-reward optimization non-harmonic, where large scaled $r$ is being more optimized.
% % Then, our final Win-rate matching Preference Optimization loss is given as
% % \begin{equation}
% %     \mathcal{L}_{WPO}(\theta) = \mathbb{E}
% % \end{equation}
% As a result, one can show the following:
% \begin{equation}
%     J(p^*) - J(p_{\theta}) < C_{\textrm{cov}}\frac{\delta}{N}\text{,}
% \end{equation}
% where $C_{\textrm{cov}} \coloneqq \max_{\mathbf{x},\mathbf{c}} \frac{p_\theta(\mathbf{x}|\mathbf{c})}{p_\textrm{ref}(\mathbf{x}|\mathbf{c})}$ is a global coverage.

% \paragraph{Relationship with IPO~\citep{azar2024general}.}

% we show that the error rate can go down to $O(\tfrac{\delta}{N})$

% our proxy reward model can only have a best estimate where we can assume $\|r^*(\mathbf{x}, \mathbf{c}) - r(\mathbf{x}, \mathbf{c}) - \C(\mathbf{c})\|_2\leq \delta$ for some normalization constant $C(\mathbf{c})$. That said, the scale of the reward $r$ can be largely differ for each prompt $\mathbf{c}$, and the scale of reward can be largely differ when considering multiple reward models. 
% This 

% reward model $r(\mathbf{x}, \mathbf{c})$ can be 

% Suppose we have access to the ground-truth reward function $r^*(\mathbf{x},\mathbf{c})$, its

% Suppose we have $N$ images $\mathbf{x}_1, \ldots, \mathbf{x}_N$ per prompt
% \begin{itemize}
%     \item Given $N$ generated images from a prompt, it is straightforward to consider winning image and losing image by comparing the rewards and apply DPO or IPO. 
%     \item While it's simple. It may prone to over-optimization, where all images have similar rewards. 
%     \item This occurs us to consider how to leverage the information of reward into direct alignment algorithms. 
%     \item Propose reward distillation loss
%     \item It is a common practice to choose R from the logits of BT model. However, this might be problematic, as the it is not calibrated.
%     \item That being said, high value of $r(x, c)$ does not guarantee the absolute goodness of a pair $(x,c)$. For example, $(x,c)$ can be a good pair, but the logit value can be in a lower value than a bad pair $(x',c')$.
%     \item Instead, we aim to compute the calibrated reward by computing the pairwise win-rate among $N$ samples. 
%     \item This approximates the goodness $P(x\succ \mu)$ where $\mu$ is a pretrained model. 
%     \item Good thing 1) better for multi-reward problem since it is scaled from [0,1], while the scale differs for each reward model
%     \item Comparison with pair-wise soft version of DPO / IPO. Those only consider the win-rate between pair. This is not idealistic as the both images can be bad or good, but the gap might be large.
%     \item (Theory) Theoretically, this aligns with the theory of IPO.
%     \item (Analysis) If we generalize to $k$ pairs, i.e, kC2 pairs, then one can show that it is equivalent to following, which can be considered as a reward weighted regression, while subtracting baseline by leave-one-out. Difference is that we subtract KL regularization term. 
% \end{itemize}

%HERE
\subsection{Preference Pair Selection}
\label{sec:frs}
The best-of-$N$ sampling~\citep{nakano2021webgpt, cobbe2021training} or rejection sampling~\citep{touvron2023llama} methods that 
select samples with highest reward from $N$ generation
are commonly used in RLHF.
% reward optimization. 
For a single reward, it is straightforward to choose the sample $\mathbf{x}^+$ with highest reward, and $\mathbf{x}^-$ that has lowest reward to maximize the margin between the pair.
For multi-reward optimization, the na\"ive approach is to use weighted sum as the total proxy reward model, and perform rejection sampling with it. 
However, choosing the weights often relies on heuristics, and the optimal weights might be dynamic depending on the input, which can lead to suboptimal performance.


% However, those strategy is not starightforward to apply when multi-reward optimization
% that has highest reward
% One of the common practice in reward fine-tuning is rejection sampling~\citep{touvron2023llama}, or as known as \emph{best-of-N} sampling, which takes a sample with highest reward from $N$ generation. For contrastive algorithm such as CaPO or DPO, it can be straightforward to consider the pair with highest reward and lowest reward, so that the model learns from the pair with maximal margin. When dealing with multiple rewards, formally let $r_i$, $i=1,\ldots, L$ be reward models, the common approach is to consider the weighted sum as the best proxy reward model, \emph{i.e.}, $r_{\textrm{sum}}(\mathbf{x}, \mathbf{c}) = \sum_{i=1}^L \lambda_i r_i(\mathbf{x}, \mathbf{c})$ for $\lambda_i \geq 0$ with $\sum_{i=1}^L \lambda_i = 1$. However, choosing $\lambda_i$ often relies on heuristics, and the optimal $\lambda_i$ might be dynamic depending on the input $(\mathbf{x}, \mathbf{c})$. Therefore, conducting rejection sampling using the sum of rewards might give suboptimal gain. 

In order to achieve the Pareto optimal solution, we propose frontier-based rejection sampling (FRS), which selects the set of positive samples $X^+(\mathbf{c})$ and negative samples $X^-(\mathbf{c})$ for each prompt $\mathbf{c}$ by finding Pareto optimal set. 
Specifically, we use a non-dominated sorting algorithm~\citep{deb2002fast} to find the upper and lower Pareto frontier. 
The goal of FRS is to push apart from the lower Pareto frontier and pull towards the upper Pareto frontier, which helps to achieve Pareto optimality.
Given $L$ reward models, let $R_{\textrm{ca}}^{(j)}$ be $j$-th calibrated rewards for $j=1,\ldots, L$, then we define $\mathbf{x}$ dominates $\mathbf{x}'$ if and only if $R_{\textrm{ca}}^{(j)}(\mathbf{x}, \mathbf{c}) \geq R_{\textrm{ca}}^{(j)}(\mathbf{x}', \mathbf{c})$ for all $j=1,\ldots, L$. 
Then finding a set of non-dominated data points is referred as finding Pareto set, which forms an upper frontier. Conversely, one can define a set of dominated data that forms a lower frontier. 
After removing the potential duplicates of non-dominated and dominated sets, we take $X^+(\mathbf{c})$ by filtered non-dominated sets and $X^-(\mathbf{c})$ by set of dominated set. 
Given positive set ${X}^+(\mathbf{c})$ and $X^-(\mathbf{c})$, we sample a positive sample $\mathbf{x}^+\sim X^+(\mathbf{c})$ and $\mathbf{x}^-\sim X^-(\mathbf{c})$ to construct a pair.
We use CaPO loss to update the model with ensemble of calibrated rewards for optimization target:
\vspace{-2mm}
\begin{equation*}
\vspace{-2mm}
    R_{\textrm{ca}}(\mathbf{x}, \mathbf{c}) = \frac{1}{L} \sum_{j=1}^L R_{\textrm{ca}}^{(j)}(\mathbf{x}, \mathbf{c})\text{,}
    % \Delta(\mathbf{x}^+, \mathbf{x}^-, \mathbf{c}) = \frac{1}{L}\sum_{j=1}^L \big[R_{\textrm{ca}}^{(j)}(\mathbf{x}^+, \mathbf{c}) - R_{\textrm{ca}}^{(j)}(\mathbf{x}^-, \mathbf{c})\big]  
\end{equation*}
and use CaPO loss in Eq.~\eqref{eq:capoloss} for the update.
% In Algorithm~\ref{alg:dm}, we provide a pseudocode for calibrated reward optimization with multiple rewards.
% multi-reward CRO in Algorithm~\ref{alg:dm}.

% and at each iteration we select positive sample $x^+$ from $X^+$ and negative sample $x^-$ from $X^-$ to construct a pair.

% joint optimization of multiple reward signals that learns pair with diverse directions, which helps achieving balanced optimization
% with  learning of multiple direction to ensure balanced learning of each reward and prevent the bias towards certain reward signal. 
% In Fig.~xx, we provide an illustration to our frontier-based rejection sampling method.

% $k$

% dominance of $\mathbf{x}$ over $\mathbf{x}'$ is defined

% we define $\mathbf{x}$ dominates $\mathbf{x}'$ 

% selects a pair $(\mathbf{x}^+, \mathbf{x}^-)$ of positive sample $\mathbf{x}^+$ and negative sample $\mathbf{x}^-$ by 

% selecting positive sample from upper Pareto frontier, and a negative sample from lower Pareto frontier by using non-dominated sorting algorithm~\cite{deb2002fast}. 
% Formally, we define $\mathbf{x}$ dominates $\mathbf{x}'$ if $\mathbf{x}$ is better or equal in each reward values than that of $\mathbf{x}'$, \emph{i.e.}, $r_i(\mathbf{x}, \mathbf{c}) \geq r_i(\mathbf{x}', \mathbf{c})$ for all $i=1,\ldots, L$. Then finding a set of non-dominated data is referred as finding Pareto set, which forms an upper frontier. Conversely, one can consider a set of dominated data that forms an lower frontier. Then, after removing the potential duplicates of non-dominated and dominated set, we choose a positive sample from non-dominated set and a negative sample from dominated set to feed into our preference learning objective. 

% \paragraph{Reward ensemble.}
% Given a pair of samples $(\mathbf{c}, \mathbf{x}^+, \mathbf{x}^-)$ and multiple reward models $\{r_i\}_{i=1}^L$, again we compute the approximate win-rate using batch of samples, \emph{i.e.}, $\mathbb{P}_i(\mathbf{x}\succ \mu)$. Then the simplest approach is to average the win-rate, \emph{i.e.}, $\mathbb{P}_{\textrm{avg}}(\mathbf{x}\succ \mu| \mathbf{c}) = \frac{1}{L}\sum_{j=1}^L \mathbb{P}_i(\mathbf{x} \succ \mu|\mathbf{c})$. While it is the most simplest, averaging the win-rate might be misleading if the overwhelming one has large error. 
% To circumvent this issue, we also consider uncertainty-weighted optimization (UWO)~\citep{coste2023reward, yu2020mopo, brantley2019disagreement, wu2021uncertainty}, where we penalize the estimate by variance of all estimates, \emph{i.e.}, 
% \begin{align}
% \begin{split}
%     &\mathbb{P}_{\textrm{UWO}}(\mathbf{x}\succ\mu |\mathbf{c}) \coloneqq \frac{1}{L}\sum_{j=1}^L \mathbb{P}_i(\mathbf{x}\succ \mu |\mathbf{c}) -\\
%     &\frac{\lambda}{L}\sum_{j=1}^L ( \mathbb{P}_i(\mathbf{x}\succ\mu|\mathbf{c}) - \mathbb{P}_{\textrm{avg}}(\mathbf{x}\succ\mu|\mathbf{c}))^2\text{,}
% \end{split}
% \end{align}
% where $\lambda > 0$ is a regularization hyperparameter.
%MH: add more sentences to the caption  to make the figure self-explanatory (without reading the text) 
% \begin{figure}[t]
    \centering
    \begin{subfigure}[t]{0.35\columnwidth}
        \includegraphics[width=\linewidth]{figure_files/eps_weight.png}
        \caption{Weighting for noise-prediction loss}
        \label{fig:eps_weight}
    \end{subfigure}
    % \hfill
    \begin{subfigure}[t]{0.35\columnwidth}
        \centering\small
        \includegraphics[width=\linewidth]{figure_files/v_weight.png}
        \caption{Weighting for flow matching loss}
        \label{fig:v_weight}
    \end{subfigure}
    \caption{\textbf{Loss weighting.} We plot the weighting function with bias $b\in \{-2, -1, 0, 1, 2\}$ for each noise prediction loss and flow matching loss.} 
    \label{fig:loss_weight}
    \vspace{-5pt}
\end{figure}

\subsection{Loss weighting}\label{sec:adv}
% \vspace{0.05in}
% \noindent
% {\bf Loss weighting.}
The choice of log-SNR $\lambda_t$ and weighting function $w_t$ has large impact on the generation quality and convergence of diffusion model pretraining.
Intuitively, when $\lambda_t$ is large, \emph{i.e.}, small amount of noise is added, the denoising task becomes easier, and conversely the task becomes harder as $\lambda_t$ becomes smaller, thus weighting function as a monotonically decreasing weighting function of $\lambda_t$ seems a reasonable choice. 
In \citep{kingma2023understanding}, those monotonic weighting are theoretically shown to be the weighted evidence lower bound (ELBO), and demonstrated better quality than the non-monotonic counterpart.
In this work, we also propose to use monotonic loss weighting to our CaPO loss, which is equivalent to regularizing with weighted ELBO instead of KL divergence in Eq.~\eqref{eq:obj}. 
Specifically, we apply sigmoid weighting with bias, \emph{i.e.}, $w_t = w(\lambda_t) = \sigma(-\lambda_t + b)$, where $b$ is a bias hyperparameter~\citep{kingma2023understanding, hoogeboom2024simpler}. See supplementary for details.
% Note that for flow matching objective~\citep{lipman2022flow} which predicts the velocity instead of noise, this weighting is equivalent to multiplying $(\exp(\lambda_t/2) + \exp(-\lambda_t/2))^{-1}$ to the flow matching objective (see supplementary for detail).
%MH: fill in xx


% Remark that original Diffusion-DPO~\citep{wallace2023diffusion} considers non-monotonic weighting (\emph{i.e.}, $-w_t \lambda_t' = 1$) for SDXL.
% We remark that this provides better performance compared to non-monotonic weighting approaches
% weighted flow matching objective becomes $(\exp(\lambda_t/2) +\exp(-\lambda_t/2))^{-1}\|v_\theta(\mathbf{z}_t;t) - (\eps-\mathbf{x})\|_2^2$

% using logit-normal sampling~\citep{esser2024scaling} (\emph{i.e.}, sampling $\lambda_t$ from normal distribution) is a popular choice. 



% When training diffusion models, it has been well-studied that the lower noise level, \emph{i.e.}, for small $t$, the model tends to learn the high-frequency detail of an image, while the model at higher $t$ entails the low-frequency detail of an image. Intuitively, it could be harmful when we change the model's capability in generating high-frequency detail, and modifying the low-frequency detail suffices. To this end, we provide dynamic $\beta_t$ for each timestep instead of constant $\beta$ by providing more weights on higher timesteps. For discrete-time diffusion model, \emph{e.g.}, DDPM~\citep{ho2020denoising, rombach2022high}, we use continuous adaptation of Min-SNR-$\gamma$ weighting, where we use $\beta_t = \frac{\beta\gamma}{\gamma + e^{\lambda_t}}$ and $\gamma>0$ is a hyperparameter. For continuous-time diffusion model (or flow model), we sample $\lambda_t \sim \mathcal{N}(m, \sigma^2)$, where $m, \sigma$ is a mean and standard deviation, following the practice in \citep{karras2022elucidating, esser2024scaling}. Intuitively, choosing higher $m$ leads to sampling from higher timestep region. We find that this techniques provide faster convergence, and better image quality for CaPO, yet the effect is marginal for Diffusion-DPO. We provide detailed analysis in Appendix~xx.
% \vspace{0.05in}
% \noindent
% {\bf Independent noise sampling.} 
% Note that the implicit reward $R_t(\mathbf{z}_t,\mathbf{c}, \theta)$ requires expectation over all $t\sim\mathcal{U}(0,1)$ and $\boldsymbol{\epsilon}\sim\mathcal{N}(\boldsymbol{0}, \mathbf{I})$. From Diffusion-DPO paper~\citep{wallace2023diffusion}, the author claimed that using sample noise for computing the implicit reward is beneficial as it reduces the variance of optimization. However, we empirically find that the effect of variance is mere, and providing independent sampled noise for each $\mathbf{x}^+$ and $\mathbf{x}^-$ shows improved performance for CaPO and even for DPO. Intuitively, using different noise enforces more stronger regularization that any trajectory that ends with $\mathbf{x}^+$ at $t=0$ should have higher likelihood than that ends with $\mathbf{x}^-$ at $t=0$. We provide detailed analysis in Appendix~xx.



% we find that since we fine-tune the model to keep the original capability while modify the 

% We find that providing more weights to the higher timestep region 

% only cleans up the fine-detail without changing the entire layout
% % can be regarded as improving the likelihood of $\mathbf{x}^+$ than $\mathbf{x}^-$ given same initial noise distribution, while 

% This can be thought as 
% it has been common practice that  original Diffusion-DPO~\citep{wallace2023diffusion}
% We remark some improved techniques to fine-tune diffusion models with CaPO. 
% First, in contrast to Diffusion-DPO~\citep{wallace2023diffusion}, we found that sampling 
% Given triplet $(\mathbf{c}, \mathbf{x}^+, \mathbf{x}^-)$ and the approximate win-rate $\mathbb{P}(\mathbf{x}\succ \mu |\mathbf{c})$ a

% % one overwhelms others. 

% % multiple reward si
% % We illustrate the 

% % , thus prevent the biased preference optimization

% % The illustration of  

% % Given a multiple reward signals $r_i$, $i=1,\ldots, N$, 

% % While this method is straightforward for a single reward, it is non-trivial to extend to 
% % The common practice in RLHF is using rejection sampling~\cite{touvron2023llama}
% % When considering a single reward signal for training, it is 
% % \begin{itemize}
% %     \item Given $N$ images per prompt, it is questionable if we have to use all samples for fine-tuning.
% %     \item Instead, we only consider the effective samples for efficiency. 
% %     \item For single reward, it is straightforward and empirically validated that choosing the maximum and minimum score pairs. 
% %     \item For multiple rewards, it is a common approach to set a sum (or weighted-sum) of reward as a proxy for ground-truth reward and solve as a single reward problem.
% %     \item However, this might give us spurious pair which wins big in axis1 but lose small at axis2.
% %     \item To this end, we propose a multi-objective rejection sampling method. 
% %     \item Define non-dominated sorting, dominated sorting. Explain sorting algorithm. Filter duplicates. explain with figure
% %     \item Confirm the necessity of sample complexity of $N$.
% %     \item Is it really efficient? show in ablation study.
% %     \item Then propose DPO-FRS, IPO-FRS, CAPO-FRS. explain with pseudocode.
% % \end{itemize}
