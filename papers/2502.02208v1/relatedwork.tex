\section{Related work}
While extensive research exists on quantum repeater chains (see~\cite{Azuma_2023} for a detailed review), finding the optimal strategies for quantum repeater protocols involving distillation remains largely unexplored.

Simple entanglement swapping (ES) strategies perform ES in a \emph{nested} fashion~\cite{Briegel_1998}, or as soon as two adjacent entangled links are available (\emph{SWAP-ASAP})~\cite{goodenough2024noise}.
In Ref.~\cite{Inesta_2023}, the authors consider homogeneous chains and conclude that a nested strategy generally reduces the waiting time needed to distribute entanglement.
Studies on optimal policies for ES have been proposed in Ref.~\cite{Inesta_2023, Haldar_2024_RL}, both based on a Markov Decision Process (MDP) for repeater chains, and a learning algorithm solving it. The optimal policies found lead to an improvement in the efficiency of the repeater protocols.
% 
The algorithm aims to choose the most efficient sequence of actions maximizing the quality of the end-to-end entangled link and minimizing the time needed to achieve it. 
However, they use a simplified model to describe the state of the chain. 
% 
They model decoherence using the \emph{age} of an entangled link, i.e. the count of time steps from when it is generated. Fidelity is derived from the age, and a link is generated again if its age exceeds a certain threshold. They assume Pauli noise, using the "addition rule" to compute the age for a link in output of a swap operation as the sum of the ages of the input links.  
We employ a more sophisticated model by building upon the framework developed in Ref.~\cite{Li_2021}, discussed in more detail in Section \ref{sec:model}, which, for instance, includes time-dependent exponential decay decoherence. It also treats both the waiting time and the end-to-end state's entanglement quality as random variables. 
For each instant of time, distributions of these variables are computed.
Thus, in contrast to~\cite{Haldar_2024_RL}, we get the complete view of the efficiency of all the possible sequences of actions performed to distribute entanglement.

In addition, Ref.~\cite{Li_2021} also examines the impact of cut-off conditions on the links of the chain. An example of such a condition is a limit on the maximum storage time of a link in an imperfect memory, after which it is discarded and regenerated.
Unlike~\cite{Inesta_2023, Haldar_2024_RL, Li_2021} we do not take cut-off conditions into account, thus considering a smaller set of possible actions to be performed on the repeater chain.

Finally, we refer to studies focusing on optimal strategies for entanglement distillation.
% 
Ref.~\cite{VanMeter_2008, Nagayama_2021} consistently find that performing entanglement distillation at link-level (immediately after entanglement generation) is essential for maintaining high-fidelity quantum states over long distances. In particular, Ref.~\cite{VanMeter_2008} finds that link-level distillation is particularly effective when the initial state fidelities are low, improving the success probability of the following swapping or distillation operations. Similarly, Ref.~\cite{Nagayama_2021} shows that, by addressing errors locally, link-level distillation prevents error accumulation, improving the overall efficiency of the repeater protocol.
% 
Ref.~\cite{Haldar_2024_Quasi} examines the efficiency of distillation-based policies and shows that performances depend heavily on hardware parameters, such as the quality of elementary links.
They find that distillation is often detrimental when elementary links have high fidelity, as it increases waiting times, but becomes beneficial for elementary links of low fidelity.

%%%%%%%%%%%%%%%%%%%%%%%%%%%%%%%%%