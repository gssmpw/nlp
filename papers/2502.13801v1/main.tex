%%%%%%%%%%%%%%%%%%%%%%%%%%%%%%%%%%%%%%%%%%%%%%%%%%%%%%%%%%%%%%%%%%%%%%%%

%%% LaTeX Template for AAMAS-2025 (based on sample-sigconf.tex)
%%% Prepared by the AAMAS-2025 Program Chairs based on the version from AAMAS-2025. 

%%%%%%%%%%%%%%%%%%%%%%%%%%%%%%%%%%%%%%%%%%%%%%%%%%%%%%%%%%%%%%%%%%%%%%%%

%%% Start your document with the \documentclass command.


%%% == IMPORTANT ==
%%% Use the first variant below for the final paper (including auithor information).
%%% Use the second variant below to anonymize your submission (no authoir information shown).
%%% For further information on anonymity and double-blind reviewing, 
%%% please consult the call for paper information
%%% https://aamas2025.org/index.php/conference/calls/submission-instructions-main-technical-track/

%%%% For anonymized submission, use this
%\documentclass[sigconf]{aamas} 

%%%% For camera-ready, use this
\documentclass[sigconf,nonacm]{aamas} 
\usepackage[utf8]{inputenc}

%%% Load required packages here (note that many are included already).

\usepackage{balance} % for balancing columns on the final page

% BEGIN custom packages
%\usepackage{amssymb,amsfonts} %amsmath
\usepackage{algorithm}
\usepackage{algorithmic}
\usepackage{graphicx}
\usepackage{textcomp}
\usepackage{siunitx}
\usepackage{xcolor}
\usepackage{subcaption}
\usepackage{wrapfig}

\usepackage[
%ragged
raggedright
%raggedleft
]{sidecap}   
\sidecaptionvpos{figure}{t} 
% END custom packages


%%%%%%%%%%%%%%%%%%%%%%%%%%%%%%%%%%%%%%%%%%%%%%%%%%%%%%%%%%%%%%%%%%%%%%%%

%%% AAMAS-2025 copyright block (do not change!)

\makeatletter
\gdef\@copyrightpermission{
  \begin{minipage}{0.2\columnwidth}
   \href{https://creativecommons.org/licenses/by/4.0/}{\includegraphics[width=0.90\textwidth]{by}}
  \end{minipage}\hfill
  \begin{minipage}{0.8\columnwidth}
   \href{https://creativecommons.org/licenses/by/4.0/}{This work is licensed under a Creative Commons Attribution International 4.0 License.}
  \end{minipage}
  \vspace{5pt}
}
%\makeatother

%\setcopyright{ifaamas}
%\acmConference[AAMAS '25]{Proc.\@ of the 24th International Conference on Autonomous Agents and Multiagent Systems (AAMAS 2025)}{May 19 -- 23, 2025}{Detroit, Michigan, USA}{Y.~Vorobeychik, S.~Das, A.~Nowé  (eds.)}
\acmConference[]{}{}{}{}
%\copyrightyear{2025}
\copyrightyear{}
%\acmYear{2025}
\acmYear{}
\acmDOI{}
\acmPrice{}
\acmISBN{}

%%%%%%%%%%%%%%%%%%%%%%%%%%%%%%%%%%%%%%%%%%%%%%%%%%%%%%%%%%%%%%%%%%%%%%%%

% Colors
\definecolor{valid}{rgb}{0., 0.8, 0.}

% Commands to ease readction
% Notations
\newcommand{\institutenfrancais}{\institution{Sorbonne Université, CNRS\\ Institut des Systèmes Intelligents et de Robotique, ISIR}}
\newcommand{\ifonlyf}{\textit{iff}}
\newcommand{\R}{\mathbb{R}}
\newcommand{\Distrib}{\mathcal{D}}
\newcommand{\proba}{\mathbb{P}}
\newcommand{\buffer}{\mathcal{B}}
\newcommand{\probSpace}{\mathcal{P}}
\newcommand{\actionSpace}{\mathcal{A}}
\newcommand{\stateSpace}{\mathcal{S}}
\newcommand{\goalSpace}{\mathcal{G}}
\newcommand{\rGC}{r_{\mathcal{G}}}
\newcommand{\pG}{p_{\mathcal{G}}}
\newcommand{\rS}{r_S}
\newcommand{\expect}{\mathbb{E}}
\newcommand{\aS}{a_S}
\newcommand{\aGC}{a_{GC}}
\newcommand{\piS}{\pi_{\phi_S}}
\newcommand{\piGC}{\pi_{\phi_{GC}}}
\newcommand{\thGCtoS}{\text{th}_{GC \rightarrow S}}
\newcommand{\thStoGC}{\text{th}_{S \rightarrow GC}}
\newcommand{\SafetyActivated}{\text{safety}\_\text{flag}}
\newcommand{\neighborhood}{N_0}
\newcommand{\vmin}{v_{min}}
\newcommand{\vmax}{v_{max}}

% Refs
\newcommand{\refAlg}[1]{Algorithm \ref{#1}}
\newcommand{\refFig}[1]{Figure \ref{#1}}
\newcommand{\refEq}[1]{(\ref{#1})}
\newcommand{\refSection}[1]{section \ref{#1}}
% END commands


%%%%%%%%%%%%%%%%%%%%%%%%%%%%%%%%%%%%%%%%%%%%%%%%%%%%%%%%%%%%%%%%%%%%%%%%

%%% == IMPORTANT ==
%%% Use this command to specify your submission number.
%%% In anonymous mode, it will be printed on the first page.

\acmSubmissionID{1042}

%%% Use this command to specify the title of your paper.

\title[AAMAS-2025 Formatting Instructions]{Learning to explore when mistakes are not allowed}

%%% Provide names, affiliations, and email addresses for all authors.



\author{Charly Pecqueux-Guézénec}
\affiliation{
  \institution{Sorbonne Université, CNRS, ISIR}%\institutenfrancais
  \city{F-75005 Paris}
  \country{France}}
\email{pecqueuxguezenec@isir.upmc.fr} % TODO: google or lab ? 
% TODO: ORCID

\author{Stéphane Doncieux}
\affiliation{
  \institution{Sorbonne Université, CNRS, ISIR}%\institutenfrancais
  \city{F-75005 Paris}
  \country{France}}
\email{doncieux@isir.upmc.fr}
% TODO: ORCID

\author{Nicolas Perrin-Gilbert}
\affiliation{
  \institution{Sorbonne Université, CNRS, ISIR}%\institutenfrancais
  \city{F-75005 Paris}
  \country{France}}
\email{perrin@isir.upmc.fr}
% TODO: ORCID

%%% Use this environment to specify a short abstract for your paper.

\begin{abstract}
Goal-Conditioned Reinforcement Learning (GCRL) provides a versatile framework for developing 
unified controllers capable of handling wide ranges of tasks, exploring environments, 
and adapting behaviors. However, its reliance on trial-and-error poses challenges for real-world 
applications, as errors can result in costly and potentially damaging consequences. To address the 
need for safer learning, we propose a method that enables agents to learn goal-conditioned behaviors 
that explore without the risk of making harmful mistakes. Exploration without risks can seem paradoxical, 
but environment dynamics are often uniform in space, therefore a policy trained for safety without 
exploration purposes can still be exploited globally. Our proposed approach involves two distinct 
phases. First, during a pretraining phase, we employ safe reinforcement learning and distributional
techniques to train a safety policy that actively tries to avoid failures in various situations. 
In the subsequent safe exploration phase, a goal-conditioned (GC) policy is learned while ensuring safety. 
To achieve this, we implement an action-selection mechanism leveraging the previously learned 
distributional safety critics to arbitrate between the safety policy and the GC policy, ensuring 
safe exploration by switching to the safety policy when needed.
We evaluate our method in simulated environments and demonstrate that it not only provides 
substantial coverage of the goal space but also reduces the occurrence of mistakes to a minimum, 
in stark contrast to traditional GCRL approaches. Additionally, we conduct an ablation study and 
analyze failure modes, offering insights for future research directions.

% Goal-Conditioned Reinforcement Learning (GCRL) offers a powerful framework to develop
% a single robotic controller that can achieve a wide variety of tasks, explore its environment,
% and eventually, adapt its behavior.
% Yet its trial-and-error approach often prevents real-world applications as mistakes, intended as action sequences leading to catastrophic consequences,  
% can lead to significant damage and high costs.
% Moving towards agents that learn and explore in the real world, we propose a method that 
% allows robotic agents to learn goal-oriented behaviors when mistakes are not allowed. 
% Our approach involves two phases. In the first phase, called pretraining, 
% we use safe reinforcement learning and distributional tools to train a safety policy
% in simulation that minimizes the risk of mistakes. In the second phase, called safe exploration, 
% a goal-conditioned (GC) policy is safely learned. To prevent mistakes, 
% we build an action selection mechanism, based on the previously learned distributional safety 
% critics, that arbitrates between the safety policy and the GC policy and chooses
% the action to execute at each step to maintain safety during exploration.
% We test our approach on simulated robotic environments and show that it not only achieves 
% satisfying goal space coverage but also results in only a few mistakes during exploration. In contrast, GCRL alone
% leads to hundreds of mistakes at least. We also perform an ablation study and investigate failure modes to 
% guide future research.

% Reinforcement learning (RL) offers a powerful framework to develop adaptable robotic behaviors.
% Yet its trial-and-error approach often prevent real-world applications due to safety concerns. 
% For example, a robot learning to walk may frequently fall during exploration before it performs a stable gait. 
% While safe reinforcement learning aims to design controllers that satisfy safety constraints after the training, 
% it does not prevent safety violations during the learning process.  
% Moving towards agents that learn and explore in the real world, we propose a method that enables robotic agents to 
% learn goal-oriented behaviors while avoiding dangerous states during exploration.
% Our approach involves two phases. The first is the pre-training in simulation of a policy optimizing a safety objective.
% The second is the safe exploration phase, where a goal-conditioned (GC) policy is safely learned. 
% In this phase, an action selection mechanism uses previously learned safety critics to choose between 
% actions proposed by the GC policy or the safety policy, thereby maintaining safety while exploring the environment.
% \textcolor{blue}{results and shows}

% Reinforcement learning (RL) provides a framework to design adaptable and hard-to-engineer behaviors for robots. 
% But to do so RL agents proceed by trial and error, which reduces the scope for real-world applications. 
% For example a robot that learns to walk often falls during the exploration phase before finding a stable gait. 
% Safe reinforcement learning aims to learn controllers that satisfy safety constraints after the training, 
% but do not prevent constraint violations during the learning process, which is why RL agents learning in the real world 
% are rare. 
% Moving towards agents that learn and explore in the real world, we propose a method that enables a robotic agent to learn 
% how to achieve goals without falling into dangerous states during exploration. In a first phase we pre-train a safety 
% policy in simulation. In a second phase, called \textit{safe exploration}, a goal-conditioned policy explores the 
% environment and learns to reach goals. To ensure safety while exploring it hands over to the safety policy when it 
% is about to violate the safety constraint. \textcolor{red}{Affiner et compléter}
\end{abstract}

%%% The code below was generated by the tool at http://dl.acm.org/ccs.cfm.
%%% Please replace this example with code appropriate for your own paper.


%%% Use this command to specify a few keywords describing your work.
%%% Keywords should be separated by commas.

\keywords{Safe Exploration, Safe Reinforcement Learning, Goal-Conditioned Reinforcement Learning}
%%%%%%%%%%%%%%%%%%%%%%%%%%%%%%%%%%%%%%%%%%%%%%%%%%%%%%%%%%%%%%%%%%%%%%%%

%%% Include any author-defined commands here.
         
\newcommand{\BibTeX}{\rm B\kern-.05em{\sc i\kern-.025em b}\kern-.08em\TeX}

%%%%%%%%%%%%%%%%%%%%%%%%%%%%%%%%%%%%%%%%%%%%%%%%%%%%%%%%%%%%%%%%%%%%%%%%

\begin{document}

%%% The following commands remove the headers in your paper. For final 
%%% papers, these will be inserted during the pagination process.

\pagestyle{fancy}
\fancyhead{}

%%% The next command prints the information defined in the preamble.

\maketitle 

%%%%%%%%%%%%%%%%%%%%%%%%%%%%%%%%%%%%%%%%%%%%%%%%%%%%%%%%%%%%%%%%%%%%%%%%

\section{Introduction}
\label{sec:intro}

\begin{figure*}[tb]
    \centering
    \includegraphics[width=0.848\linewidth]{figs/circuitnn.pdf} 
    \caption{Illustration of differentiable CircuitNN. CircuitNN is designed based on differentiable NAND gates. After DAS is guided by PI and PO pairs of the truth table, CircuitNN can get the precise circuit architecture logic equivalent to the truth table.}
    \label{fig:circuitnn}
\end{figure*}

% 1. Describe the importance of logic synthesis
% 2. Existing Problems
% (a) Neural Architecture Search: Unstable, Predefined Setting, etc.
% (b) Circuit Generation: Probabilistic Model, Logic Equivalence

With the rapid advancement of technology, the scale of integrated circuits (ICs) has expanded exponentially. 
This expansion has introduced significant challenges in chip manufacturing, particularly concerning power and area metrics.
A primary objective in IC design is achieving the same circuit function with fewer transistors, thereby reducing power usage and area occupancy.

Logic synthesis~\cite{hachtel2005logicsynth}, a critical step in electronic design automation (EDA), transforms behavioral-level circuit designs into optimized gate-level circuits, ultimately yielding the final IC layout. 
The primary goal of logic synthesis is to identify the physical implementation with the fewest gates for a given circuit function. 
This task constitutes a challenging NP-hard combinatorial optimization problem. 
Current logic synthesis tools~\cite{brayton2010abc, wolf2013yosys} rely on human-designed heuristics, often leading to sub-optimal outcomes.

Differentiable architecture search (DAS) techniques~\cite{liu2018darts, chu2020darts} offer novel perspectives on addressing challenges in this problem.
Circuit functions can be represented through truth tables, which map binary inputs to their corresponding outputs. 
Truth tables provide a precise representation of input-output relationships, ensuring the design of functionally equivalent circuits.
Inspired by this, researchers~\cite{deepmind2024ai4sys, wang2024tnet} have begun exploring the application of DAS to synthesize circuits directly from truth tables.
Specifically, \citet{deepmind2024ai4sys} proposed CircuitNN, a framework that learns differentiable connection structures with logic gates, enabling the automatic generation of logic circuits from truth tables.
This approach significantly reduces the complexity of traditional circuit generation. 
Building on this, \citet{wang2024tnet} introduced T-Net, a triangle-shaped variant of CircuitNN, incorporating regularization techniques to enhance the efficiency of DAS.

Despite these advancements, several challenges remain. 
The computational complexity of DAS grows quadratically with the number of gates, posing scalability issues.
Although triangle-shaped architecture~\cite{wang2024tnet} partially mitigates this problem, redundancy persists. 
%Additionally, DAS is susceptible to converging to local optima, limiting the ability to search architectures that satisfy the given truth tables~\cite{liu2018darts}. 
%Furthermore, hyperparameters (network depth and layer width) require extensive searches, introducing complexity and prolonging the synthesis process. 
Additionally, DAS is susceptible to converging to local optima~\cite{liu2018darts} and hyperparameters (network depth and layer width) require extensive searches. 
The challenges arise from the vast search space in DAS. 
% Even with predefined settings for CircuitNN, finding a configuration that meets the truth table requires extensive trial and error during the DAS process. 
Intuitively, limiting the search space through predefined parameters (network depth, gates per layer, and connection probabilities) can significantly reduce the complexity.

Recent advances~\cite{openai2023gpt4, abramson2024alphafold3, esser2024sd3, li2024mar} in conditional generative models have demonstrated remarkable performance across language, vision, and graph generation tasks. 
Motivated by these developments, we propose a novel approach to circuit generation that generates preliminary circuit structures to guide DAS in generating refined circuits matching specified truth tables. 
Firstly, we introduce CircuitVQ, a tokenizer with a discrete codebook for circuit tokenization. 
Built upon our Circuit AutoEncoder framework~\cite{hou2022graphmae,li2023maskgae,wu2025mgvga}, CircuitVQ is trained through a circuit reconstruction task. 
Specifically, the CircuitVQ encoder encodes input circuits into discrete tokens using a learnable codebook, while the decoder reconstructs the circuit adjacency matrix based on these tokens.
Subsequently, the CircuitVQ encoder serves as a circuit tokenizer for CircuitAR pretraining, which employs a masked autoregressive modeling paradigm~\cite{chang2022maskgit, li2023mage}. 
In this process, the discrete codes function as supervision signals. 
After training, CircuitAR can generate discrete tokens progressively, which can be decoded into initial circuit structures by the decoder of the CircuitVQ. 
These prior insights can guide DAS in producing refined circuits that match the target truth tables precisely.

Our key contributions can be summarized as follows:
\begin{itemize}
\item We introduce CircuitVQ, a circuit tokenizer that facilitates graph autoregressive modeling for circuit generation, based on our Circuit AutoEncoder framework;
\item Develop CircuitAR, a model trained using masked autoregressive modeling, which generates initial circuit structures conditioned on given truth tables;
\item Propose a refinement framework that integrates differentiable architecture search to produce functionally equivalent circuits guided by target truth tables;
\item Comprehensive experiments demonstrating the scalability and capability emergence of our CircuitAR and the superior performance of the proposed circuit generation approach.
\end{itemize}

% Motivation
% (a) Diffusion (Vision, Graph), Autoregressive (Language, Vision)
% (b) Circuit Generation for Predefined Setting
% (c) Neural Architecture Search for Strict Logic Equivalence

% Contribution
% (a) Circuit Tokenizer (new transformer arch, training strategy)
% (b) CircuitAR (train and gen strategies, post-ar strategy)
% (c) Extensive Evaluation including BitD (Bit Distance) for Scalability


\section{Related Work}
% \subsection{Vision Language Model}
% 시각장애인에서 상황을 설명할 DB가 없으니 만들었다. 그리고 이를 VLM에 튜닝했다.
\subsection{Technical approaches for assisting the visually-impaired}


\subsection{Datasets for visual instruction tuning}


\section{Basic Background: Supervised Learning and the PAC Model}
\label{sec:background}

At this point almost everyone has heard of machine learning (ML). Anyone likely to stumble upon this article will have also heard of its most influential special case, supervised learning, and those theoretically inclined will also be familiar with the PAC model. Nonetheless, I will set the stage by  recapping the basics.

\subsection{Basics of Supervised Learning}%Let's set the stage in any case

\emph{Supervised Learning} is the task of ``coming up'' with a function $f: \X \to \Y$ to ``explain'' or ``fit'' a sequence of input/output examples   $(x_1,y_1), \ldots, (x_n,y_n)$, with $x_i \in \X$ and $y_i \in \Y$.  Here $\X$ is a \emph{data domain} consisting of \emph{datapoints} $x \in \X$, $\Y$ is a \emph{label set} consisting of \emph{labels} $y \in \Y$, and the sequence $(x_1,y_1),\ldots,(x_n,y_n)$ is the \emph{training data} consisting of \emph{labeled examples (a.k.a. samples)}~$(x_i,y_i)$.  I~will refer to the chosen function $f$ as a \emph{predictor}, and to $n$ as the \emph{sample size}. A \emph{learning algorithm} takes as input training data, and outputs (some representation of) a predictor $f \in \Y^\X$.\footnote{Note that this describes the usual \emph{batch}, a.k.a.~\emph{offline}, setting of supervised learning. I do not discuss other paradigms such as online or active learning in this article.} 



Success in supervised learning is defined as \emph{generalization} to  future examples: For a typical \emph{test example}  $(x_{\tst},y_{\tst})$, the predicted label $y'_{\tst}=f(x_{\tst})$ should ``equal'' $y_{\tst}$, perhaps approximately. We usually assume the test example is drawn from the same  ``source'' as the training data  --- commonly, i.i.d.~from the same distribution. The quality of the prediction is quantified by $\ell(y'_{\tst},y_{\tst})$, where $\ell:~\Y~\times~\Y \to \RR_{\geq 0}$ is a \emph{loss function} chosen as part of the problem definition. Common loss functions include the 0-1 loss $\ell_{0-1}(y',y) = [y' \neq y]$ for \emph{classification} problems,\footnote{The notation $[P]$ denotes $1$ when predicate $P$ is true, and denotes $0$ when $P$ is false.} as well as the absolute loss $|y'-y|$ or squared loss $(y'-y)^2$ for \emph{regression problems} featuring $\Y  \sse \RR$.

Nontrivial generalization properties are typically only possible if one assumes something about the data.\footnote{The need for such an assumption is formalized by the  \emph{no free lunch theorems} of supervised learning \cite{wolpert_connection_1992,wolpert_lack_1996,schaffer_conservation_1994}.} The Bayesian approach to  machine learning, common in many applications, assumes some parametric form for the distribution generating the data, and postulates a prior on the parameters. This is not the approach I will take in this article. Instead, I will focus on the frequentist --- and some would say ``worst-case'' or ``adversarial'' ---  approach that is common in the computational learning theory community, embodied by the PAC model. Here we assume that the (training and test) data can be explained, perhaps approximately, by a function in some ``simple enough to learn'' class of functions $\H \sse \Y^\X$, often called the \emph{hypotheses}. Equivalently, we  seek a predictor which explains the unseen data roughly  as well as the best hypothesis $h^* \in \H$, whether or not we assume that $h^*$ itself provides a perfect explanation.



 \paragraph{Common Algorithmic Templates.} Perhaps the best known general-purpose supervised learning algorithm is \emph{empirical risk minimization (ERM)}, which chooses as its predictor a hypothesis $f \in \H$ minimizing $\frac{1}{n} \sum_{i=1}^n \ell(f(x_i),y_i)$ --- a quantity called the \emph{training error}, \emph{empirical error}, or \emph{empirical risk} of $f$. %\footnote{When multiple hypotheses minimize the empirical risk, we assume ERM breaks ties arbitrarily.}
A common template for generalizing ERM involves adding a \emph{regularization term} $\psi(f)$ to the  objective function, typically chosen to measure some notion of ``hypothesis complexity.'' An algorithm instantiating this template is known as a \emph{structural risk minimizer (SRM)}, and chooses as its predictor the hypothesis $f \in \H$ minimizing the \emph{structural risk} $\frac{1}{n} \sum_{i=1}^n \ell(f(x_i),y_i) + \psi(f)$. Other well-known algorithms, such as gradient descent and its variations,  can frequently be interpreted as approximate implementations of ERM or SRM.


\paragraph{Proper vs Improper Learning.} A learning algorithm is said to be \emph{proper} if its predictor $f$ is always chosen from the hypothesis class, i.e., $f \in \H$, otherwise it is said to be \emph{improper}. ERM  is an example of a proper learning algorithm, as are SRM algorithms of the form described above.  In the \emph{proper regime} of learning, algorithms are required to be proper. This article will be concerned with the more flexible \emph{improper regime} (a.k.a \emph{representation-independent learning}), where no such constraint is placed on the learner. In other words, all we care about is predictive power at test time, rather than any insights derived from the functional form or representation of the predictor~itself.


\subsection{The PAC Model}
A standard mathematical setup for evaluation of supervised learning algorithms, at least in the theoretical computer science community, is Valiant's \emph{Probably Approximately Correct (PAC) model} of learning (see e.g.~\cite{kearns_introduction_1994,mohri_foundations_2018}). Here, we assume there is an unknown distribution $\D$ on $\X \times \Y$ from which training and test data are  drawn.  Specifically, the labeled datapoints of the training set  $(x_1,y_1), \ldots, (x_n,y_n)$, as well as the test data  $(x_\tst,y_\tst)$, are i.i.d.~from $\D$. Often it is assumed that $\D$ lies in some class of distributions of interest. The \emph{true expected loss}, or simply \emph{loss}, of a predictor $f: \X \to \Y$ is the expected loss it incurs on draws from $\D$, written $L_\D(f) = \Ex_{(x,y) \sim \D} \ell(f(x),y)$.


There are two main ``settings'' in PAC learning. The  \emph{realizable setting} only requires that the data be perfectly explained by some hypothesis in $\H$. More generally, the \emph{agnostic setting} makes no assumption relating the data to the hypotheses, but shifts the goalposts as necessary to allow nontrivial guarantees: the expected loss at test time is evaluated only ``relative'' to that of the best hypothesis $h^* \in \H$. There are other settings which make more nuanced assumptions, such as $\D$ being of a particular parametric form or its support living in some (unknown) lower-dimensional space, etc. I will mostly discuss the realizable and agnostic settings in this article, those being the simplest and most studied from a theoretical perspective. %TODO:We will briefly discuss other settings in Section ??

The PAC model demands high probability guarantees of learners, in the worst case over distributions of interest. Consider first the realizable setting, where $\D$ is such that $\min_{h \in \H} L_{\D}(h) = 0$. A PAC learner has \emph{error} $\epsilon=\epsilon(n)$ and \emph{confidence} $\delta=\delta(n)$ if, when training data consists of $n$ i.i.d~samples from a realizable distribution $\D$, it produces a predictor $f$  satisfying $L_\D(f) \leq \epsilon$ with probability at least $1-\delta$. In the agnostic setting, where $\D$ can be arbitrary, we require $L_\D(f) - \min_{h \in \H} L_\D(h) \leq \epsilon$ with probability $1-\delta$.

In both the realizable and agnostic settings, we look for PAC learners with small $\epsilon$ and $\delta$ as a function of the sample size $n$. An equivalent perspective looks at the sample complexity $m(\epsilon,\delta)$, which is the minimum sample size which guarantees error  at most $\epsilon$ with probability at least $1-\delta$. We say a problem is \emph{PAC learnable} if its PAC sample complexity is finite whenever $\epsilon,\delta > 0$.

For most PAC learning problems, learnability and sample complexity are characterized in terms of a  ``dimension'' of the hypothesis class. Most prominently this is the \emph{VC dimension} for binary classification, the \emph{fat shattering dimension} for agnostic regression, and the \emph{DS dimension} for multiclass classification (see \cite{anthony_neural_1999,daniely_optimal_2014,brukhim_characterization_2022}). Treatment of these is beyond the scope of this article. The unfamiliar reader need not worry, however,  as dimensions will feature only tangentially in our~discussion.




%\paragraph{Learning settings: Realizable, Agnostic, etc.} In learning theory, evaluating a supervised learning algorithm requires specifying a data model and an objective. We will leave the details of the data model flexible for now, to allow for both the PAC model and the adversarial transductive model. Nonetheless we will describe two variations, which we call ``settings'', which cut across different models. The  \emph{realizable setting}  requires only that the data be perfectly explained by some hypothesis $h \in \H$ --- i.e., there exists a hypothesis which is guaranteed to suffer a loss of $0$ on training and test data. The performance of the learning algorithm is its expected loss at test time for some ``worst case'' realizable instance. More generally, the \emph{agnostic setting} makes no assumption relating the data to the hypotheses, but shifts the goalposts as necessary to allow nontrivial guarantees: the expected loss at test time is evaluated only ``relative'' to that of the best hypothesis $h^* \in \H$, again for some ``worst case'' instance. There are other settings which make more nuanced assumptions about the data, such as it is drawn from a distribution of a particular parametric form, or that it lives in some (unknown) lower-dimensional space, etc. We will mostly discuss the realizable and agnostic settings, those being the simplest and most studied from a theoretical perspective.




%%% Local Variables:
%%% mode: latex
%%% TeX-master: "learning_matching"
%%% End:


\section{Method}

In this section, we formalize the safe exploration problem and detail the method we propose to solve it.

\subsection{Defining the safe exploration problem}
\label{subsection:safe_exp_problem}

We aim for our agent to learn how to maximize coverage of a goal set, which corresponds to solving a multi-goal MDP (\refSection{subsection:GCRL}). However, the agent must also explore its environment safely, avoiding mistakes. In our framework, a mistake occurs when a terminal state is reached. Therefore, the agent must avoid terminal states during exploration. To do so, we provide our agent with a safety policy that we can activate when the GC policy is about to make a mistake and lead the agent out of danger. 

A way to build this safety policy is under the angle of classic RL. We consider a set $\neighborhood \subset \stateSpace$ of states that are desirable in terms of safety, for instance near some equilibrium point. The safety reward $\rS$ equals $1$ if $s \in \neighborhood$ and 0 otherwise. This reward setting is convenient as it relates the sum of rewards to the number of steps $T^{\piS}(s, a)$ necessary to reach $\neighborhood$ from a given state-action couple $(s, a)$. Then, to decide when to switch from one policy to the other we can compare the estimation given by the safety critic of the number of steps necessary to reach $ \neighborhood$ to a threshold.

However, critics are neural networks, which are continuous functions, while the number of steps is finite for states from which we can reach $\neighborhood$ and infinite for terminal states regarding an optimal safety policy. Thus, it pushes the critic to generalize safety to unsafe states and leads the agent to believe that it can still use the GC policy even though objectively, it is about to make a mistake. Therefore, in addition to the temporal distance, we added a notion of distance in the state space between the current state and the set of terminal states. To do so, we assume the agent receives a cost value $h(s)$ at each environment step, where $s$ is the current state and $h: \stateSpace \to \R$ a continuous constraint function that verifies $h(s) > 0$ if and only if $s$ is a terminal state. In our framework, we identify terminal states as mistakes the agent absolutely has to avoid. Like in RCRL, the function $h$ is related to a kind of distance between the current state $s$ and the set of terminal states \cite{RCRL2022}. Assuming to have access to such a function is not very restrictive as robots have sensors and often state estimation modules. The continuity of function $h$ is crucial as it allows for generalization from known states to unseen states. Indeed, an unvisited state near another visited and safe state which is far enough from terminal states is likely to be safe too. On the contrary, states near unsafe states are likely to be unsafe. As a result, we switch from the GC policy to the safety policy if the number of steps to reach $\neighborhood$ is too high or if $h(s)$ is close to $0$. 

All these considerations lead us to define the safe exploration problem as the combination of a CMDP $(\stateSpace, \actionSpace, p, p_0, \rS, h)$
and a multi-goal MDP $(\stateSpace \times \goalSpace, \actionSpace, p, p_0, \pG, \rGC)$:

\begin{equation}
    \mathcal{T} = (\stateSpace \times \goalSpace, \actionSpace, p, p_0, \pG, \rS, \rGC, h)
    \label{eq:SafeExpProblem}
\end{equation}

In the pretraining phase, we train a parametrized stochastic safety policy $\piS$ that solves the CMDP. 
The CMDP is independent of the goal space as the notion of safety is independent of the goals pursued by the agent.
We assume that we have access to simulation to perform the safety pretraining, allowing us to perform reset anywhere 
and then to train the policy on a wide variety of situations. The critics that have been trained are then used in the 
action selection mechanism.

During the safe exploration phase, the safety policy and its critics are fixed and we train a GC policy. 
The multi-goal MDP part follows the framework developed in \refSection{subsection:GCRL}. 
Transitions are collected and stored in an episodic replay buffer $\buffer$, regardless of the policy that generates them.
Thus transitions generated by both policies can be found in the same episode. The main idea behind is that both policies 
can learn from each other as their respective objectives may be complementary in some situations.

\subsection{Safety policy learning}
\label{subsection:safety_policy_learning}

To train the safety policy, we take inspiration from the distributional algorithm TQC, as it uses an ensemble critics for robustness and truncation in the critic update to prevent overestimation \cite{TQC}.
On the one hand, we train $M$ approximations $Z_{\psi_1}, ... Z_{\psi_M}$ of the safety policy return distribution $Z^{\piS}(s, a) = \Distrib_{\piS} \left[ \left. \sum_{t=0}^{+\infty} \gamma^t \rS(s_t, a_t) \right| s, a \right]$ using the TQC critic loss. The targets $Z_{\overline{\psi_1}}, ... Z_{\overline{\psi_M}}$ are initialized in the same way and follow $Z_{\psi_1}, ... Z_{\psi_M}$ via exponential moving average update. 
Similarly, we train an ensemble of $M$ reachability critics $R_{\xi_1}, ... R_{\xi_M}$ but we replace the usual target based on a sum with the RCRL \cite{RCRL2022} target based on a $\max$ operator, leading to equation \refEq{eq:RCRL_target}. 
Likewise, targets $R_{\overline{\xi_1}}, ... R_{\overline{\xi_M}}$ follow $R_{\xi_1}, ... R_{\xi_M}$ via exponential moving average. Unlike TQC, the reachability critics are updated separately using the quantile regression loss \cite{QR-DQN}.
Critic ensemble are trained on the same batched transitions but initialized with different seeds.

\begin{equation}
    \mathcal{T}\theta^{(j)} = (1-\gamma) h(s') + \gamma \max\left( h(s'), \theta_{\overline{\xi}}^{(j)}(s', a') \right)
    \label{eq:RCRL_target}
\end{equation}

Policy parameters $\phi_S$ are optimized to minimize the following loss, also inspired by TQC \cite{TQC}:

\begin{equation}
    \mathcal{L}_{\pi_S}(\phi_S) = \expect_{\substack{s, a \sim \mathcal{B}}}
    \left[ \alpha \log \piS(a|s) - \overline{Q}(s, a) + \lambda \overline{R}(s, a) \right]
    \label{eq:safety_actor_loss}
\end{equation}
where $\overline{Q}(s, a) = \frac{1}{M N} \sum_{i = 1}^{M} \sum_{j = 1}^{N} \theta_{\psi_i}^{(j)}(s, a)$

$\overline{R}(s, a) = \frac{1}{M N} \sum_{i = 1}^{M} \sum_{j = 1}^{N} \theta_{\xi_i}^{(j)}(s, a)$ and $\lambda \in \R_+$ is
a positive multiplier that we keep fixed. In our experiments, we performed an ablation study to identify the effect of 
reachability critics on safety during exploration. So we tested $\lambda = 0$ and $\lambda = 100$. The value of $100$
has been chosen so that $\overline{Q}(s, a)$ and $\overline{R}(s, a)$ have the same scale. We also tested a varying $\lambda$, 
like \citeauthor{ha2020SACLagLevine}, but it led to too much instability in the safety training \cite{ha2020SACLagLevine}.

\subsection{Action selection mechanism}
\label{subsection:Action selection mechanism}

\begin{algorithm}[H]
    \begin{algorithmic}[1]
    \STATE \textbf{Inputs:} $s, g, \SafetyActivated, \thGCtoS, \thStoGC$ 
    \STATE $\aGC \sim \piGC(.|s, g)$ ; $\aS \sim \piS(.|s)$
    \STATE $\text{lower} \leftarrow (\hat{\sigma}^{\piS}(s, \aGC) > \thStoGC)$
    \STATE $\text{raise} \leftarrow (\hat{\sigma}^{\piS}(s, \aS) > \thGCtoS)$
    \STATE $\SafetyActivated \leftarrow \SafetyActivated \, \text{\textbf{and}} \, \text{lower}$ \hfill // Lower the flag
    \STATE $\SafetyActivated \leftarrow \SafetyActivated \, \text{\textbf{or}} \, \text{raise}$  \hfill // Raise the flag
    \STATE $a \leftarrow \SafetyActivated . \aS + (1 - \SafetyActivated) . \aGC$
    \STATE \textbf{Return} $a$, $\SafetyActivated$
    \end{algorithmic}
    \caption{Action selection for safe exploration}
    \label{alg:Action selection}
\end{algorithm}

The action selection mechanism that must ensure safe exploration has been reproduced in \refAlg{alg:Action selection}.
It takes as input the current state $s$, current desired goal $g$, the current value of a boolean flag, called $\SafetyActivated$, and 
risk thresholds $\thGCtoS$ and $\thStoGC$. The flag indicates which action to choose, and the role of the action selection mechanism is to update this flag. If equal to $1$, action $\aS$ 
sampled by the safety policy is selected, else action $\aGC$ sampled by the GC policy is selected.
The flag is updated according to the level of risk associated  
with state $s$ and possible actions $\aGC$ and $\aS$. If the level of risk $\hat{\sigma}^{\piS}(s, \aS)$ 
associated with state $s$ and action $\aS$ exceeds threshold $\thGCtoS$, the safety flag is set to $1$ (\textit{raised})
so as to avoid a future potentially dangerous situation. On the contrary, if the level of risk $\hat{\sigma}^{\piS}(s, \aGC)$ 
associated with state $s$ and action $\aGC$ goes below threshold $\thStoGC$, the safety flag is set to $0$ (\textit{lowered}).
Note that the risk function $\hat{\sigma}^{\piS}$ is related to the safety policy, as its goal is to evaluate the capacity 
of the safety policy to put the agent out of danger. 
Also, because the safety policy is optimized on its critics, it tends to minimize the risk function undirectly. As a result, 
the thresholds should verify $\thStoGC \le \thGCtoS$. Otherwise, the safety policy would be the only one to act.
Depending on the environment, one could choose either equality or strict inequality between thresholds 
(See section \ref{subsec:ablation_dist}). 

The framework defined in section \refSection{subsection:safe_exp_problem} essentially leads to two 
definitions of the risk function, leading to three different strategies. 
The first is based on the time necessary to reach the set $\neighborhood$ from current state, using the safety policy. 
The second is based on the maximum of constraint function $h$ along future trajectories generated by the safety policy.
The third possibility is to combine both. We will use the labels \textbf{Time}, \textbf{Constraint} and \textbf{Time-constraint}
for the respective three strategies. 

\paragraph{\textbf{Time:}} We choose the set $\neighborhood$ so that, if the safety policy is well trained, 
then $\neighborhood$ is a forward set regarding the safety policy $\piS$. More precisely, we assume that, 
for any sequence of transitions $(s_0, a_0, ..., s_t, ...)$ generated by $\piS$, if 
the starting state $s_0$ lies in $\neighborhood$, then all future states $s_t$ also lie in $\neighborhood$.
Under this assumption the safety reward $\rS(s_t, a_t)$ equals $0$ until $s_t$ lies in $\neighborhood$, resulting in equation 
\refEq{eq:time_and_rewards}:

\begin{equation}
    T^{\piS}(s_0, a_0) = f_{\gamma}\left(\sum_{t = 0}^{+\infty} \gamma_t \rS(s_t, a_t)\right) 
    \label{eq:time_and_rewards}
\end{equation}
where $T^{\piS}(s_0, a_0)$ is the random variable corresponding to the number of steps necessary for $\piS$ to reach 
$\neighborhood$ starting from state-action couple $(s_0, a_0)$, and $f_{\gamma}(x) = \log \left( (1-\gamma)  x\right) / \log \gamma$,
which is a continuous bijection from $]0, 1/(1-\gamma)]$ to $\R^+$. By applying the bijective mapping $f_{\gamma}(x)$ to the atoms $\theta_{\psi_i}^{(j)}(s, a)$ 
of ensemble $Z_{\psi_1}, ... Z_{\psi_M}$, we obtain new atoms that approximate the distribution 
of the random variable $T^{\piS}(s, a)$.
We denote $\hat{T}^{\piS}(s, a ; \tau)$ the mean of the quantiles corresponding to cumulative probabilities 
greater than $\tau$. For example, $\tau = 0.9$ corresponds to the mean of worst $10\%$ of cases. $\tau$
is a hyperparameter of the algorithm. In the Time strategy : $\hat{\sigma}^{\piS}(s, a) = \hat{T}^{\piS}(s, a ; \tau)$.
So the thresholds $\thGCtoS$ and $\thStoGC$ are given in number of environment steps.

\paragraph{\textbf{Constraint:}}
In the same way, we denote $\hat{R}^{\piS}(s, a ; \tau)$ the mean of atoms from reachability critics $R_{\xi_1}, ... R_{\xi_M}$
corresponding to cumulative probabilities greater than $\tau$. 
In the constraint strategy : $\hat{\sigma}^{\piS}(s, a) = \hat{R}^{\piS}(s, a ; \tau)$.
So the thresholds $\thGCtoS$ and $\thStoGC$ are negative real numbers corresponding to safety margins.

\paragraph{\textbf{Time-constraint:}} This strategy combines the two previous ones. Let $\epsilon > 0$ be a positive real number 
representing a safety margin. If $\hat{R}^{\piS}(s, a ; \tau) > -\epsilon$, then: $\hat{\sigma}^{\piS}(s, a) = T_{max}$. 
Where $T_{max}$ is the maximum number of episode steps. 
Otherwise: $\hat{\sigma}^{\piS}(s, a) = \hat{T}^{\piS}(s, a ; \tau)$. 
The thresholds $\thGCtoS$ and $\thStoGC$ are given in number of environment steps.

\subsection{Algorithm}

\paragraph{\textbf{Pretraining phase:}} We use the same training loop as TQC 
\cite{TQC}. The only difference is the gradient step update where, in addition to the TQC update,
parameters $\xi_1, ... \xi_M$ are updated by performing Adam optimizer step on 
loss $\mathcal{L}_{Z}$ with target \refEq{eq:RCRL_target}, and the actor loss is replaced with loss $\mathcal{L}_{\pi_S}$.
\refAlg{alg:safety_update} describes how the update is done.
The safety policy interacts with the environment, generating a
transition that is stored in a replay buffer $\buffer_S$. Then a batch of transtions is sampled uniformly
from the buffer and a gradient step update is performed on all parameters, so that for each stored transition,
one step of gradient is performed. 
Also, we start the training after $5000$ steps for which we sample random actions
to favor exploration and learning at the start.

\begin{algorithm}[H]
\begin{algorithmic}[1]
\STATE Sample a batch from the replay buffer $\buffer_S$ uniformly
\STATE Perform TQC critic update on $\psi_1, ... \psi_M$ \cite{TQC}
\STATE Update $\xi_1, ... \xi_M$ using Adam on loss $\mathcal{L}_{Z}$ with target \refEq{eq:RCRL_target}
\STATE Update actor parameters $\phi_S$ using Adam on loss $\mathcal{L}_{\pi_S}$
\STATE Update temperature $\alpha_S$ according to SAC rule \cite{SAC}
\end{algorithmic}
\caption{Safety gradient step update}
\label{alg:safety_update}
\end{algorithm}

\paragraph{\textbf{Safe exploration phase:}} The procedure is summarized in \refAlg{alg:main_algorithm}, 
which is also off-policy. 
We first choose safety thresholds depending on the risk level we want. 
Then the parameters of the previously learned safety policy and its critics are loaded,
while the parameters related to the GC policy are randomly initialized. The episodic replay buffer is 
initially empty. For each environment step during training, the action selected according to 
\refAlg{alg:Action selection} to ensure safety during exploration. Each transition is stored in an 
episodic replay buffer $\buffer$ regardless of the policy that generated it. As a result, in a single stored episode,
there are probably samples generated by both policies. 

Initially, the GC policy performs poorly, as it has not yet been trained.
Thus the first stored episodes contain a large majority of transitions generated by the safety policy, as it had to
make up for the random behavior of the GC policy. Then, as the GC policy improves itself, the proportion of transitions 
it has generated increases in the buffer. The fact that many transitions have not been generated by the GC policy can 
lead off-policy algorithms like SAC to value over-estimation, due to the distributional drift between the dataset 
and the current learned policy \cite{levine2020offline}. This is why we used SAC-N instead of SAC, which is the 
same algorithm as SAC but with $N$ critics instead of $2$ \cite{sac_n_edac}. As the critics are initialized and updated
separately, disagreement between them regarding unvisited states is likely to lead the Bellman target to low values via the 
$\min$ operator, thus preventing over-estimation. 

\begin{algorithm}[H]
\begin{algorithmic}[1]
%\STATE \textbf{Initialize} Safety and GC parameters ;  %$\psi_1... \psi_M$, $\xi_1... \xi_M$, $\phi_S$, $\alpha_S$
%\STATE \textbf{Initialize} GC parameters %$\psi_{GC}... \psi_M$, $\xi_1... \xi_M$, $\phi_S$, $\alpha_S$
\STATE \textbf{Inputs:} $\thGCtoS, \thStoGC$ 
\STATE \textbf{Load} Safety parameters $\psi_i, \overline{\psi}_i$, $\xi_i, \overline{\xi}_i$, $i\in[\![1, M]\!]$, $\phi_S$, $\alpha_S$
\STATE \textbf{Initialize} GC parameters $\psi_{GC}, \overline{\psi}_{GC}$, $\phi_{GC}$, $\alpha_{GC}$
\STATE \textbf{Initialize} Episodic replay buffer $\buffer \leftarrow \emptyset$
\STATE \textbf{Sample} initial state $s_0 \sim p_0$, and goal $g \sim \pG$ 
\FOR{each iteration}
\FOR{each environment step, until done}
\STATE $a_t, \SafetyActivated \leftarrow \text{select\_action}(s, g, \SafetyActivated,$ \\$ \thGCtoS, \thStoGC)$
(\refAlg{alg:Action selection})
\STATE Collect transition $(s_t, a_t, s_{t+1}, r_{S, t}, r_{GC, t}, h_{t+1})$
\STATE $\buffer \leftarrow \buffer \cup \{ (s_t, a_t, s_{t+1}, r_{S, t}, r_{GC, t}, h_{t+1}) \}$
\ENDFOR 
\FOR{each gradient step}

\STATE Sample a batch $b_{GC} = (s_t, a_t, s_{t+1}, r_{GC, t})$ from $\buffer$ \\ using HER \cite{HER}
\STATE Update GC parameters on $b_{GC}$ using SAC-N rule \cite{SACN_edac}
% \IF{safety finetuning activated}
% \STATE Update safety policy using \refAlg{alg:safety_update}
% \ENDIF
\ENDFOR
\ENDFOR
\STATE \textbf{return} GC parameters $\psi_{GC}, \overline{\psi}_{GC}$, $\phi_{GC}$, $\alpha_{GC}$ %\\
%Safety parameters $\psi_i, \overline{\psi}_i$, $\xi_i, \overline{\xi}_i$, $i\in[\![1, M]\!]$, $\phi_S$, $\alpha_S$
%Replay buffer $\buffer$
\end{algorithmic}
\caption{Safe exploration algorithm}
\label{alg:main_algorithm}
\end{algorithm}


\section{Experiments: Planning outperforms Heuristics}
\label{sec:experiment}

We begin our empirical demonstrations by showcasing the effectiveness of our planning framework on both synthetic and real datasets. We focus on the simplest planning algorithm, 1-step lookaheads (Algorithm~\ref{alg:complete}), and show that even basic planning can hold great promise. 
We illustrate our framework using two uncertainty quantification modules---GPs and 
\ensembles/ \ensembleplus. 

Throughout this section, we focus on evaluating the mean squared error of 
a regression model $\model$,  and develop adaptive policies that minimize uncertainty on $g(f)$ defined in~\eqref{eqn:l2-g-f}.
When GPs provide a valid model of uncertainty, 
our experiments show that our planning framework significantly outperforms other baselines. 
We further demonstrate that our conceptual framework extends to deep learning-based uncertainty quantification methods such as  \ensembleplus while highlighting computational challenges that need to be resolved in order to scale our ideas. 
For simplicity, we assume a naive predictor, i.e., $\psi(\cdot) \equiv 0$. However, we emphasize that this problem is just as complex as if we were using a sophisticated model $\psi(.)$. The performance gap between the algorithms 
primarily depends
on the level  of uncertainty in our prior beliefs.

To evaluate the performance of our algorithm, we benchmark it against several baselines. 
%Active learning baselines use an acquisition function $\ac$ to select points that have the highest   function value: $X\opt_t \in \argmax_{X \in \xpoolj{t}} \ac({X})$ at every step $t$. These methods may also need an UQ module, which we simply use the same UQ module as in our algorithm, and it  outputs $V(X)$ that measures the the uncertainty of each point $X \in \xpoolj{t}$.
Our first set of baselines are from active learning~\citep{AggarwalKoGuHaPh14}:
\\ % \noindent\textbf{Active Learning Heuristics:} 
\textbf{(1)} 
\textsf{Uncertainty Sampling (Static):}  In this approach, we query the samples for which the model is least certain about. Specifically, we estimate the variance of the latent output $f(X)$ for each $X \in \xpool$ using the UQ module and select the top-$K$ points with the highest uncertainty. \\
\textbf{(2)} \textsf{Uncertainty Sampling (Sequential):} This is a greedy heuristic that sequentially selects the points with the highest uncertainty within a batch, while updating the posterior beliefs using pseudo labels from the current posterior state. Unlike \textsf{Uncertainty Sampling (Static)}, this method takes into account the information gained from each point within batch, and hence tries to diversify the selected points within a batch. 

 
We also compare our approach to the  \textbf{(3)} \textsf{Random Sampling}, which selects each batch uniformly at random from the pool. Additionally, we compare solving the planning problem using  \textsf{REINFORCE}-based policy gradients with   $\mathsf{Smoothed\text{-}Autodiff}$ policy gradients.\footnote{Our code repository is available at
  \url{https://github.com/namkoong-lab/adaptive-labeling}.}
%Detailed experimental setups are provided in Section \ref{sec:details-experiments}.

%We repeat all experiments with 10 random seeds.




\begin{figure}[t]
\centering
\begin{minipage}[b]{0.49\textwidth}
\centering
\includegraphics[width=\textwidth, height=5cm]{figures/original_scale/Var_of_l_2_loss.pdf}
\caption{(Synthetic data) Variance of mean squared loss evaluated through the posterior belief $\mu_t$ at each horizon $t$. This is the objective that policy gradient methods like \textsf{REINFORCE} and $\ouralgo$ optimizes. 1-step lookaheads are surprisingly effective even in long horizons.}
\label{fig:var-l2-sim}
\end{minipage}
\hfill
\begin{minipage}[b]{0.49\textwidth}
\centering \includegraphics[width=\textwidth, height=5cm]{figures/original_scale/Error_of_estimated_model_l_2_loss.pdf}
\caption{(Synthetic data) Error between MSE calculated based on collected data $\mc{D}^{0:T}$ vs. population oracle MSE over $\mc{D}_{\rm eval} \sim P_X$. Reducing uncertainty over posteriors directly leads to better OOD evaluations. 1-step lookaheads significantly outperform active learning heuristics in small horizons.}
\label{fig:mean-l2-sim}
\end{minipage}
%\caption{Simulated data for GPs}
%\label{fig:both_plots}
\end{figure}

\subsection{Planning with Gaussian processes}
\label{sec:experiment-plan-GP}
We now briefly describe the data generation process for the GP experiments,  deferring a more detailed discussion of the dataset generation to Section~\ref{sec:details-experiments}. 
We use both the synthetic data and the real data to test our methodology.
For the \emph{simulated data},  we construct a setting where the general population is distributed across \emph{51 non-overlapping clusters} while the initial labeled data $\dtrain$ just comes from one cluster. In contrast, both $\dpool \defeq (\xpool,\ypool),\deval \defeq (\xeval,\yeval)$ are generated   from all the clusters. 
We begin with a low-dimensional scenario, generating a one-dimensional regression setting using a GP. %Gaussian Process (GP).
Although the data-generating process is not known to the algorithms,  we assume that the GP hyperparameters are known to all the algorithms
to ensure fair comparisons. This can be viewed as a setting where our prior is well-specified, allowing us to isolate the effects
of different policy optimization approaches
 without any concerns about the misspecified priors. We select $10$ batches, each of size $K=5$ across $T = 10$ time horizons.

To examine the robustness of our method against the distributional assumptions made  in the simulated case, we then move to a real dataset where the correct prior is not known. We simulate selection bias from the eICU dataset~\citep{PollardJoRaCeMaBa18}, which contains real-world patient data with in-hospital mortality outcomes. 
We conduct a $k$-means clustering to generate 51 clusters and then select data from those clusters. We view this to be a credible replication of practice, as severe distribution shifts are common due to selection bias in clinical labels.  To convert the binary mortality labels into a regression setting, we train a  random forest classifier and fit a GP on predicted scores, which serves as the UQ module for all the algorithms. As before, the task is to select 10 batches, each consisting of 5 samples, across 10 time horizons.

 In Figures~\ref{fig:var-l2-sim} and~\ref{fig:mean-l2-sim}, we present results for the simulated data. 
Figure~\ref{fig:var-l2-sim} shows the variance of $\ell_2$ loss, and Figure~\ref{fig:mean-l2-sim} presents the error in the estimated $\ell_2$ loss using $\mu_t$ (relative to true $\ell_2$ loss, that is unknown to the algorithm). 
As we can see from these plots, our method one-step lookahead  gives substantial improvements  over active learning baselines and random sampling. In addition,
compared to the one-step lookahead planning approach using \textsf{REINFORCE}-based policy gradients, 
we observe that $\mathsf{Smoothed\text{-}Autodiff}$-based policy gradients provide significantly more robust performance over all horizons.

In Figures~\ref{fig:var-l2-real}~and~\ref{fig:mean-l2-real}, we observe similar findings on the eICU data. We see that planning policies (\textsf{REINFORCE} and $\mathsf{Smoothed\text{-}Autodiff}$) consistently outperform other heuristics by a large margin.  Active learning baselines perform poorly in these small-horizon batched problems and can sometimes be even worse than the random search baselines.  Overall, our results show the importance of careful planning in adaptive labeling for reliable model evaluation. 

We offer some intuition as to why one-step lookahead planning may outperform other heuristic algorithms. 
 First,  \textsf{Uncertainty sampling (Static)} while myopically selects the
 top-$K$ inputs with the highest uncertainty, it fails to consider 
the overlap in information content among the ``best” instances; see \citep{AggarwalKoGuHaPh14} for more details. 
In other words,  it might acquire points from the same region with high uncertainty while failing to induce diversity among the batch.
Although \textsf{Uncertainty Sampling (Sequential)} somewhat addresses the issue of information overlap, a significant drawback of 
this algorithm
is the disconnect between the objective we aim to optimize and the algorithm. For example, it might sample from a region with high uncertainty but very low density. 

\begin{figure}[t]
\centering
\begin{minipage}[b]{0.48\textwidth}
\centering
\includegraphics[width=\textwidth, height=5cm]{figures/original_scale/Var_of_l_2_loss_real.pdf}
\caption{(Real-world eICU data) Variance of mean squared loss evaluated through the posterior belief $\mu_t$ at each horizon $t$. Even 1-step lookaheads are extremely effective planners, and auto-differentiation-based pathwise policy gradients provide a reliable optimization algorithm based on low-variance gradient estimates.}
\label{fig:var-l2-real}
\end{minipage}
\hfill
\begin{minipage}[b]{0.48\textwidth}
\centering \includegraphics[width=\textwidth, height=5cm]{figures/original_scale/Error_of_estimated_model_l_2_loss_real.pdf}
\caption{(Real-world eICU data) Error between MSE calculated based on collected data $\mc{D}^{0:T}$ vs. population oracle MSE over $\mc{D}_{\rm eval} \sim P_X$. Reducing uncertainty over posteriors directly leads to better OOD evaluations. Our method significantly outperforms active learning-based heuristics, and random sampling.}
\label{fig:mean-l2-real}
\end{minipage}
%\caption{Real data for GPs}
\end{figure}
 
%\vspace{-1.5cm}
% \begin{wrapfigure}{r}{.32\columnwidth}
%   \vspace{-.5cm} 
%   \centering
% \includegraphics[scale=.29]{figures/Var of l2l_2 loss.pdf}
%   \vspace{-0.2cm}
%   \caption{Results of GP}
% \label{fig:var-l2-gp}
%   \vspace{-0.1cm}
% \end{wrapfigure}


% Attempts have been made  in the past to address these  drawbacks heuristically  (see \citep{AggarwalKoGuHaPh14}). We give a unified computational framework while approaching the problem in a more principled manner and solving it more optimally.




\subsection{Planning with  neural network-based uncertainty quantification methods ($\ensembleplus$)}


We now provide a proof-of-concept that shows the generalizability of our conceptual framework  to the deep learning-based UQ modules, specifically focusing on $\ensembleplus$ due to their previously observed superior performance~\citep{OsbandWenAsDwIbLuRo23}. Recall that implementing our framework with deep learning-based UQ modules  requires us to retrain the model across multiple possible random actions $\bm{a}(\theta)$ sampled from the current policy $\pi_\theta$.
This requires significant computational resources, in sharp contrast to the GPs where the posteriors are in closed form and can be readily updated and differentiated. 

Due to the computational constraints, we test $\ensembleplus$ on a toy setting to demonstrate the generalizability of our framework. We consider a setting where the general population consists of four clusters, while the initial labeled data only comes from one cluster. Again we generate data using GPs.  The task is to select a batch of 2 points in one horizon. We detail the $\ensembleplus$ architecture in Section \ref{sec:details-experiments}, and we assume prior uncertainty to be large (depends on the scaling of the prior generating functions). 
The results are summarized in the Table~\ref{tab:UQ_ensemble}.

% \begin{table}[H]
% \vspace{-10pt}
% \caption{Performance under \ensembleplus as UQ module}
%     \centering
%     \begin{tabular}{|m{3cm}|m{2.5cm}|m{2cm}|} 
%     \hline
%       Algorithm   & Variance of $\loss_2$ loss estimate & Error of $\loss_2$ loss estimate  \\ \hline Random Sampling 
%          & $1710.9 \pm 1352.1$ & $8.67\pm6.62$ 
%       \\ \hline \ouralgo & $1.30 \pm 0.68$ & $0.91\pm0.25$ \\ \hline
%     \end{tabular}
%     \label{tab:UQ_ensemble}
%     %\vspace{-10pt}
% \end{table}




\begin{table}[h]
\vspace{-10pt}
\caption{Performance under \ensembleplus as the UQ module}
\centering
\begin{tabular}{|l|l|l|}
\hline
Algorithm   & Variance of $\loss_2$ loss estimate & Error of $\loss_2$ loss estimate  \\
\hline
\textsf{Random sampling} & 7129.8 $\pm$ 1027.0 & 136.2 $\pm$ 8.28 \\ \hline
\textsf{Uncertainty sampling (Static)} & 10852 $\pm$ 0.0 & 162.156 $\pm$ 0.0 \\ \hline
\textsf{Uncertainty sampling (Sequential)} & 8585.5 $\pm$ 898.9 & 144 $\pm$ 6.93 \\ \hline
\textsf{REINFORCE} & 1697.1 $\pm$ 0.0 & 45.27 $\pm$ 0.0 \\ \hline
\ouralgo & 1697.1 $\pm$ 0.0 & 45.27 $\pm$ 0.0 \\ \hline
\end{tabular}
%\caption{Comparison of different algorithms based on variance   and   error in $\ell_2$ loss estimation with Ensemble $+$ as the UQ module. Our results demonstrate that {\ouralgo} and REINFORCE outperformthe other active learning based heuristics, confirming the benefits of our MDP formulation for the adaptive labeling problem, as also demonstrated in Section 4.\\
%\footnotesize{Experimental details: We use Gaussian Processes as our data generating process, GP parameters are the same as in Section D.3.  The task is to select a batch of 2 points along one horizon.The marginal distribution $p_X$ has 4 \textit{non-overlapping} clusters. Initial data comes from one cluster, while pool and evaluation points comes from all the clusters. We have $20$ initial labeled data points, $10$ pool points, and $252$ evaluation points.  Training procedures are similar to the one in Section D.3.} }
\label{tab:UQ_ensemble}
\end{table}



% We faced  issues in scaling up these experiments which will be our focus in the future. 





% \begin{itemize}
%     \item Posteriors should be consistent. Two dimensions: even with less training,  
%     \item the inference should be  fast enough
% \end{itemize}


% Potential research directions for uncertainty quantification

% In this section we consider a simple setting We consider a simpler setting and 


% For synthetic dataset generation, we use ...... For real datasets, we use ...... We compare our methodolgy to several baselines ()    This Section is structured as follows:
% \begin{itemize}
%     \item \textbf{GPs, square loss objective} (Section \ref{}): 
%     %the broad aim of the experiments  in this section is to isolate the performance of our methodology without any concerns for the inefficiencies induced due to a mis-specified prior or imperfect posterior inference. To accomplish this we generate synthetic datasets using GPs (detailed later). We use the well specified prior (GPs - with same hyperparameter setting) as our UQ module.   
%      As GPs provide differentaible posterior inference - any errors induced due to imperfect posterior updates are also isolated. We note that under this setting
%      \item In Section\ref{} we demonstrate why our methodology performs better than other baselines - by devising various synthetic experiments ()
%     \item  \textbf{UQ Benchmarking }(Section \ref{}): Before diving into the experiments using $\ensembleplus$ and ENNs,  we showcase our benchmarking experiments in Section \ref{}. We use real datasets We observe that ENNs perform better
%      \item \textbf{Ensemble $+$}, objective: recall, accuracy
%     \item \textbf{ENN}, objective: recall, accuracy
% \end{itemize}




% In Section {}, we test 
% \subsection{Experimental details}

% \begin{itemize}
%     \item UQ methodologies - GPs, ENNs
%     \item Objectives - Recall,  ATE
%     \item Datasets - ATE-synthetic datasets, Recall-synthetic, real datasets
%     \item Baselines - 
%     \begin{itemize}
%         \item Random sampling
%         \item Active learning - Uncertainty based sampling - In regression setting almost all of the 
%         \item Myopic greedy - Greedy Batch based sampling
%         \item Policy Gradient
%     \end{itemize}
    
% \end{itemize}

% \subsection{Experiments}
%     \begin{itemize}
%     \item GPs with square loss
%     \item Benchmarking ENN
%         \item ENNs with ATE
%         \item ENNs with Recall
%     \end{itemize}

% \subsection{Benefits over other algorithms - intuition and experiments}

%Active learning - Myopic greedy / Don't rely on the objective rather some entropy version.


%%% Local Variables:
%%% mode: latex
%%% TeX-master: "main"
%%% End:


\section*{Conclusion}
This paper aims to enhance our understanding of the computational complexity of computing various Shapley value variants. We found that for various ML models --- including decision trees, regression tree ensembles, weighted automata, and linear regression --- both local and global interventional and baseline SHAP can be computed in polynomial time under HMM modeled distributions. This extends popular algorithms, such as TreeSHAP, beyond their empirical distributional scope. We also establish strict complexity gaps between the various SHAP variants (baseline, interventional, and conditional) and prove the intractability of computing SHAP for tree ensembles and neural networks in simplified scenarios. Overall, we present SHAP as a versatile framework whose complexity depends on four key factors: \begin{inparaenum}[(i)] \item model type, \item SHAP variant, \item distribution modeling approach, \item and local vs. global explanations\end{inparaenum}. We believe this perspective provides deeper insight into the computational complexity of SHAP, paving the way for future work.




%We believe that our framework provides a more intricate understanding of SHAP computation complexity across different models, distributions, and variants, paving the way for further research.

Our work opens promising directions for future research. First, expanding our computational analysis to other SHAP-related metrics, such as asymmetric SHAP~\citep{frye20} and SAGE~\citep{covert2020understanding}, would be valuable. Additionally, we aim to explore more expressive distribution classes and relaxed assumptions beyond those in Section \ref{sec:tractable} while maintaining tractable SHAP computation. Finally, when exact computation is intractable (Section \ref{sec:intractable}), investigating the approximability of SHAP metrics through approximation and parameterized complexity theory~\citep{downey2012parameterized} is an important direction.

%Our work opens several promising avenues for future research on the computational properties of explainable AI methods, with a particular focus on SHAP. First, it would be interesting to broaden the computational analysis conducted in this work to include other popular SHAP-related metrics in the literature, such as asymmetric SHAP \cite{frye20} and SAGE \cite{covert2020understanding}. Also, in the future, we aim to explore more expressive distribution classes and relaxed distributional assumptions—extending beyond those examined in Section \ref{sec:tractable} —that still yield tractable SHAP computation. Finally, when exact computation proves intractable (Section \ref{sec:intractable}), it is worthwhile to theoretically investigate the question of the approximability of computing the SHAP metrics across various configurations, through the lens of approximation and parametrized complexity theory \cite{arora2009computational}.

%This paper aims to deepen our understanding of the computational complexity involved in obtaining different Shapley value variants. We found that for a variety of ML models, including decision trees, tree ensembles for regression, weighted automata, and linear regression models — computing both local and global interventional and baseline SHAP can be done in polynomial time when distributions are modeled by HMMs. This extends the distributional scope of popular algorithms like TreeSHAP, which is limited to empirical distributions. Additionally, we demonstrate a strict complexity gap between SHAP variants, showing that interventional and baseline SHAP can be strictly easier to compute than conditional SHAP. Despite these positive results, we uncovered intractability for various SHAP variants in neural networks and tree ensembles. Finally, we provided generalized complexity relations across SHAP variants. We believe that our framework offers a deeper understanding of the complexity involved in computing SHAP across various variants, models, distributions, as well as in both local and global computations, laying the groundwork for future research.

% \begin{acks}
% If you wish to include any acknowledgments in your paper (e.g., to 
% people or funding agencies), please do so using the `\texttt{acks}' 
% environment. Note that the text of your acknowledgments will be omitted
% if you compile your document with the `\texttt{anonymous}' option.
% \end{acks}

%%%%%%%%%%%%%%%%%%%%%%%%%%%%%%%%%%%%%%%%%%%%%%%%%%%%%%%%%%%%%%%%%%%%%%%%

%%% The next two lines define, first, the bibliography style to be 
%%% applied, and, second, the bibliography file to be used.

\bibliographystyle{ACM-Reference-Format} 
\bibliography{biblio}

%%%%%%%%%%%%%%%%%%%%%%%%%%%%%%%%%%%%%%%%%%%%%%%%%%%%%%%%%%%%%%%%%%%%%%%%

\newpage

\appendix

%\let\balance\relax
%\begin{multicols}{2}

\onecolumn

\section{Supplementary material}

\subsection{Videos}

\begin{itemize}
  \item \url{failure\_cartpole.mp4}: CartPoleGC failure mode described in the paper ($(\thGCtoS, \thStoGC) = (70, 30)$).
  \item \url{near\_bound\_cartpole.mp4}: CartPoleGC with a "difficult" goal near the environment bounds, thus near unsafe states ($(\thGCtoS, \thStoGC) = (70, 30)$).
  \item \url{simple\_cartpole.mp4}: CartPoleGC with a simple goal with $(\thGCtoS, \thStoGC) = (70, 30)$.
  \item \url{30\_30\_cartpole.mp4}: CartPoleGC with a simple goal with $(\thGCtoS, \thStoGC) = (30, 30)$ to show the impact of low thresholds on coverage.
  \item \url{skydiox2\_example.mp4}: Example of SkydioX2 reaching a goal ($(\thGCtoS, \thStoGC) = (10, 10)$)
\end{itemize}

Note that for CartPoleGC, the cart is blue when the safety policy is activated and green otherwise.
The goal is represented by a red box. 

\subsection{Further analysis of failure modes}

\begin{figure}[ht]
  \includegraphics[width=0.32\textwidth]{appendix_image/921_safety_value_estimation.pdf}
  \includegraphics[width=0.32\textwidth]{appendix_image/921_scritic_critic_disagreement.pdf}
  \includegraphics[width=0.32\textwidth]{appendix_image/921_scritic_reach_disagreement.pdf}
  \caption{From left to right: Value of the risk function $\hat{\sigma}^{\piS}(s, a)$ along the failed episode
  shown as an example in Figure 10, where the thresholds are represented in magenta and purple ; Critic disagreement (Figure 10) ; Reachability critic disagreement.
  Blue dots correspond to the safety policy and green dots to the GC policy. One can see the hysteretic behavior on the left plot. A video of the failure (\url{failure\_cartpole.mp4}) is also attached.}
  \label{fig:failures}
  \Description{Further analysis of failure modes}
\end{figure}

We can see on the left plot (Figure \ref{fig:failures}) that the agent switches from the GC policy to the safety policy before making a 
mistake. 
We also observe that the oscillations of the disagreement, for both ensemble of critics,
are synchronized with the change of policies. Indeed, as the GC policy has a different objective
than the safety policy, it goes towards states that have been less visited by the safety policy during 
its pretraining. This phenomenon motivates safety finetuning for future works. 

  

\newpage

\subsection{Goal-space coverage performance with safe exploration on CartPoleGC}

\begin{figure}[ht]
  \includegraphics[width=0.5\textwidth]{appendix_image/rbest1614_450000_coverage.png}
  \caption{Coverage map obtained with $L\&S$ safe exploration variant and $(\thGCtoS, \thStoGC) = (70, 30)$
  on CartPoleGC. Only for this experience, the cartpole is reset on different $x$ positions. 
  Each cell corresponds to the combination of a starting position and a desired goal and we measure the 
  success rate. We can see that the success rate is lower for starting positions and goals near the 
  environment bounds, than for positions around the center. 
  The safety policy tends to prevent the agent from reaching goals near the bounds.
  In the same way, if the initial state is too close to the bound, the safety policy prevents the 
  GC policy from acting most of the time.}
  \label{fig:cartpole_cov}
  \Description{CartPoleGC coverage}
\end{figure}

% \newpage

\subsection{Constraint strategy}

\begin{figure}[ht]
\includegraphics[width=0.3\textwidth]{appendix_image/constraint_only.pdf}
\caption{Occurrence of mistakes obtained with the time-constraint strategy and the constraint (only) strategy 
on the CartPoleGC environment. Performance of constraint strategy in terms of safety is catastrophic.}
\label{fig:constraint_only}
\Description{Occurrence of mistakes obtained with the time-constraint strategy and the constraint (only) strategy 
on the CartPoleGC environment.}
\end{figure}

\subsection{Agent hyperparameters}

\begin{table}[th]
	\caption{Safety pretraining: TQC's hyperparameters}
	\label{tab:safety_pretraining_TQC}
	\begin{tabular}{rll}
    \toprule
		\textit{Name} & \textit{Value} \\ \midrule
		Actor learning rate & $3\times 10^{-4}$  \\
		Critic learning rate & $3\times 10^{-4}$ \\
		Temperature learning rate & $3\times 10^{-4}$ \\
    Initial temperature & $1.0$ \\
    $\tau$ & $5\times 10^{-3}$ \\
		No entropy backup & -  \\
		Discount factor & 0.99  \\ 
    Hidden layers & (256, 256) \\
    Number of critics & 5 \\
    Number of atoms per critic & 25 \\
    Number of quantiles to drop & 2 for CartPoleGC ; 0 for SkydioX2GC\\
    \bottomrule
	\end{tabular}
\end{table}

\begin{table}[th]
	\caption{Safety pretraining: Reachability critics' hyperparameters}
	\label{tab:safety_pretraining_RCRL}
	\begin{tabular}{rll}
    \toprule
		\textit{Name} & \textit{Value} \\ \midrule
		Critic learning rate & $3\times 10^{-4}$ \\
    $\tau$ & $5\times 10^{-3}$ \\
		Discount factor & 0.99  \\ 
    Hidden layers & (256, 256) \\
    Number of critics & 5 \\
    Number of atoms per critic & 25 \\
    \bottomrule
	\end{tabular}
\end{table}

\begin{table}[th]
	\caption{SAC and SAC-N hyperparameters for safe exploration}
	\label{tab:sac_sac_n}
	\begin{tabular}{rll}
    \toprule
		\textit{Name} & \textit{Value} \\ \midrule
		Actor learning rate & $3\times 10^{-4}$  \\
		Critic learning rate & $3\times 10^{-4}$ \\
		Temperature learning rate & $3\times 10^{-4}$ \\
    Initial temperature & $1.0$ \\
    $\tau$ & $5\times 10^{-3}$ \\
		No entropy backup & -  \\
		Discount factor & 0.99  \\ 
    Hidden layers & (256, 256) \\
    Number of critics (Specific to SAC-N) & 50 for CartPoleGC ; 10 for SkydioX2GC \\
    \bottomrule
	\end{tabular}
\end{table}

As for the buffer we choose to keep all transitions. There is no forgetting. 
Thus, the buffer size is always larger than the number of training steps.

%\end{multicols}

\end{document}

%%%%%%%%%%%%%%%%%%%%%%%%%%%%%%%%%%%%%%%%%%%%%%%%%%%%%%%%%%%%%%%%%%%%%%%%

