%%%%%%%%%%%%%%%%%%%%%%%%%%%%%%%%%%%%%%%%%%%%%%%%%%%%%%%%%%%%%%%%%%%%%%%%

%%% LaTeX Template for AAMAS-2025 (based on sample-sigconf.tex)
%%% Prepared by the AAMAS-2025 Program Chairs based on the version from AAMAS-2025. 

%%%%%%%%%%%%%%%%%%%%%%%%%%%%%%%%%%%%%%%%%%%%%%%%%%%%%%%%%%%%%%%%%%%%%%%%

%%% Start your document with the \documentclass command.


%%% == IMPORTANT ==
%%% Use the first variant below for the final paper (including auithor information).
%%% Use the second variant below to anonymize your submission (no authoir information shown).
%%% For further information on anonymity and double-blind reviewing, 
%%% please consult the call for paper information
%%% https://aamas2025.org/index.php/conference/calls/submission-instructions-main-technical-track/

%%%% For anonymized submission, use this
%\documentclass[sigconf]{aamas} 

%%%% For camera-ready, use this
\documentclass[sigconf,nonacm]{aamas} 
\usepackage[utf8]{inputenc}

%%% Load required packages here (note that many are included already).

\usepackage{balance} % for balancing columns on the final page

% BEGIN custom packages
%\usepackage{amssymb,amsfonts} %amsmath
\usepackage{algorithm}
\usepackage{algorithmic}
\usepackage{graphicx}
\usepackage{textcomp}
\usepackage{siunitx}
\usepackage{xcolor}
\usepackage{subcaption}
\usepackage{wrapfig}

\usepackage[
%ragged
raggedright
%raggedleft
]{sidecap}   
\sidecaptionvpos{figure}{t} 
% END custom packages


%%%%%%%%%%%%%%%%%%%%%%%%%%%%%%%%%%%%%%%%%%%%%%%%%%%%%%%%%%%%%%%%%%%%%%%%

%%% AAMAS-2025 copyright block (do not change!)

\makeatletter
\gdef\@copyrightpermission{
  \begin{minipage}{0.2\columnwidth}
   \href{https://creativecommons.org/licenses/by/4.0/}{\includegraphics[width=0.90\textwidth]{by}}
  \end{minipage}\hfill
  \begin{minipage}{0.8\columnwidth}
   \href{https://creativecommons.org/licenses/by/4.0/}{This work is licensed under a Creative Commons Attribution International 4.0 License.}
  \end{minipage}
  \vspace{5pt}
}
%\makeatother

%\setcopyright{ifaamas}
%\acmConference[AAMAS '25]{Proc.\@ of the 24th International Conference on Autonomous Agents and Multiagent Systems (AAMAS 2025)}{May 19 -- 23, 2025}{Detroit, Michigan, USA}{Y.~Vorobeychik, S.~Das, A.~Nowé  (eds.)}
\acmConference[]{}{}{}{}
%\copyrightyear{2025}
\copyrightyear{}
%\acmYear{2025}
\acmYear{}
\acmDOI{}
\acmPrice{}
\acmISBN{}

%%%%%%%%%%%%%%%%%%%%%%%%%%%%%%%%%%%%%%%%%%%%%%%%%%%%%%%%%%%%%%%%%%%%%%%%

% Colors
\definecolor{valid}{rgb}{0., 0.8, 0.}

% Commands to ease readction
% Notations
\newcommand{\institutenfrancais}{\institution{Sorbonne Université, CNRS\\ Institut des Systèmes Intelligents et de Robotique, ISIR}}
\newcommand{\ifonlyf}{\textit{iff}}
\newcommand{\R}{\mathbb{R}}
\newcommand{\Distrib}{\mathcal{D}}
\newcommand{\proba}{\mathbb{P}}
\newcommand{\buffer}{\mathcal{B}}
\newcommand{\probSpace}{\mathcal{P}}
\newcommand{\actionSpace}{\mathcal{A}}
\newcommand{\stateSpace}{\mathcal{S}}
\newcommand{\goalSpace}{\mathcal{G}}
\newcommand{\rGC}{r_{\mathcal{G}}}
\newcommand{\pG}{p_{\mathcal{G}}}
\newcommand{\rS}{r_S}
\newcommand{\expect}{\mathbb{E}}
\newcommand{\aS}{a_S}
\newcommand{\aGC}{a_{GC}}
\newcommand{\piS}{\pi_{\phi_S}}
\newcommand{\piGC}{\pi_{\phi_{GC}}}
\newcommand{\thGCtoS}{\text{th}_{GC \rightarrow S}}
\newcommand{\thStoGC}{\text{th}_{S \rightarrow GC}}
\newcommand{\SafetyActivated}{\text{safety}\_\text{flag}}
\newcommand{\neighborhood}{N_0}
\newcommand{\vmin}{v_{min}}
\newcommand{\vmax}{v_{max}}

% Refs
\newcommand{\refAlg}[1]{Algorithm \ref{#1}}
\newcommand{\refFig}[1]{Figure \ref{#1}}
\newcommand{\refEq}[1]{(\ref{#1})}
\newcommand{\refSection}[1]{section \ref{#1}}
% END commands


%%%%%%%%%%%%%%%%%%%%%%%%%%%%%%%%%%%%%%%%%%%%%%%%%%%%%%%%%%%%%%%%%%%%%%%%

%%% == IMPORTANT ==
%%% Use this command to specify your submission number.
%%% In anonymous mode, it will be printed on the first page.

\acmSubmissionID{1042}

%%% Use this command to specify the title of your paper.

\title[AAMAS-2025 Formatting Instructions]{Learning to explore when mistakes are not allowed}

%%% Provide names, affiliations, and email addresses for all authors.



\author{Charly Pecqueux-Guézénec}
\affiliation{
  \institution{Sorbonne Université, CNRS, ISIR}%\institutenfrancais
  \city{F-75005 Paris}
  \country{France}}
\email{pecqueuxguezenec@isir.upmc.fr} % TODO: google or lab ? 
% TODO: ORCID

\author{Stéphane Doncieux}
\affiliation{
  \institution{Sorbonne Université, CNRS, ISIR}%\institutenfrancais
  \city{F-75005 Paris}
  \country{France}}
\email{doncieux@isir.upmc.fr}
% TODO: ORCID

\author{Nicolas Perrin-Gilbert}
\affiliation{
  \institution{Sorbonne Université, CNRS, ISIR}%\institutenfrancais
  \city{F-75005 Paris}
  \country{France}}
\email{perrin@isir.upmc.fr}
% TODO: ORCID

%%% Use this environment to specify a short abstract for your paper.

\begin{abstract}
Goal-Conditioned Reinforcement Learning (GCRL) provides a versatile framework for developing 
unified controllers capable of handling wide ranges of tasks, exploring environments, 
and adapting behaviors. However, its reliance on trial-and-error poses challenges for real-world 
applications, as errors can result in costly and potentially damaging consequences. To address the 
need for safer learning, we propose a method that enables agents to learn goal-conditioned behaviors 
that explore without the risk of making harmful mistakes. Exploration without risks can seem paradoxical, 
but environment dynamics are often uniform in space, therefore a policy trained for safety without 
exploration purposes can still be exploited globally. Our proposed approach involves two distinct 
phases. First, during a pretraining phase, we employ safe reinforcement learning and distributional
techniques to train a safety policy that actively tries to avoid failures in various situations. 
In the subsequent safe exploration phase, a goal-conditioned (GC) policy is learned while ensuring safety. 
To achieve this, we implement an action-selection mechanism leveraging the previously learned 
distributional safety critics to arbitrate between the safety policy and the GC policy, ensuring 
safe exploration by switching to the safety policy when needed.
We evaluate our method in simulated environments and demonstrate that it not only provides 
substantial coverage of the goal space but also reduces the occurrence of mistakes to a minimum, 
in stark contrast to traditional GCRL approaches. Additionally, we conduct an ablation study and 
analyze failure modes, offering insights for future research directions.

% Goal-Conditioned Reinforcement Learning (GCRL) offers a powerful framework to develop
% a single robotic controller that can achieve a wide variety of tasks, explore its environment,
% and eventually, adapt its behavior.
% Yet its trial-and-error approach often prevents real-world applications as mistakes, intended as action sequences leading to catastrophic consequences,  
% can lead to significant damage and high costs.
% Moving towards agents that learn and explore in the real world, we propose a method that 
% allows robotic agents to learn goal-oriented behaviors when mistakes are not allowed. 
% Our approach involves two phases. In the first phase, called pretraining, 
% we use safe reinforcement learning and distributional tools to train a safety policy
% in simulation that minimizes the risk of mistakes. In the second phase, called safe exploration, 
% a goal-conditioned (GC) policy is safely learned. To prevent mistakes, 
% we build an action selection mechanism, based on the previously learned distributional safety 
% critics, that arbitrates between the safety policy and the GC policy and chooses
% the action to execute at each step to maintain safety during exploration.
% We test our approach on simulated robotic environments and show that it not only achieves 
% satisfying goal space coverage but also results in only a few mistakes during exploration. In contrast, GCRL alone
% leads to hundreds of mistakes at least. We also perform an ablation study and investigate failure modes to 
% guide future research.

% Reinforcement learning (RL) offers a powerful framework to develop adaptable robotic behaviors.
% Yet its trial-and-error approach often prevent real-world applications due to safety concerns. 
% For example, a robot learning to walk may frequently fall during exploration before it performs a stable gait. 
% While safe reinforcement learning aims to design controllers that satisfy safety constraints after the training, 
% it does not prevent safety violations during the learning process.  
% Moving towards agents that learn and explore in the real world, we propose a method that enables robotic agents to 
% learn goal-oriented behaviors while avoiding dangerous states during exploration.
% Our approach involves two phases. The first is the pre-training in simulation of a policy optimizing a safety objective.
% The second is the safe exploration phase, where a goal-conditioned (GC) policy is safely learned. 
% In this phase, an action selection mechanism uses previously learned safety critics to choose between 
% actions proposed by the GC policy or the safety policy, thereby maintaining safety while exploring the environment.
% \textcolor{blue}{results and shows}

% Reinforcement learning (RL) provides a framework to design adaptable and hard-to-engineer behaviors for robots. 
% But to do so RL agents proceed by trial and error, which reduces the scope for real-world applications. 
% For example a robot that learns to walk often falls during the exploration phase before finding a stable gait. 
% Safe reinforcement learning aims to learn controllers that satisfy safety constraints after the training, 
% but do not prevent constraint violations during the learning process, which is why RL agents learning in the real world 
% are rare. 
% Moving towards agents that learn and explore in the real world, we propose a method that enables a robotic agent to learn 
% how to achieve goals without falling into dangerous states during exploration. In a first phase we pre-train a safety 
% policy in simulation. In a second phase, called \textit{safe exploration}, a goal-conditioned policy explores the 
% environment and learns to reach goals. To ensure safety while exploring it hands over to the safety policy when it 
% is about to violate the safety constraint. \textcolor{red}{Affiner et compléter}
\end{abstract}

%%% The code below was generated by the tool at http://dl.acm.org/ccs.cfm.
%%% Please replace this example with code appropriate for your own paper.


%%% Use this command to specify a few keywords describing your work.
%%% Keywords should be separated by commas.

\keywords{Safe Exploration, Safe Reinforcement Learning, Goal-Conditioned Reinforcement Learning}
%%%%%%%%%%%%%%%%%%%%%%%%%%%%%%%%%%%%%%%%%%%%%%%%%%%%%%%%%%%%%%%%%%%%%%%%

%%% Include any author-defined commands here.
         
\newcommand{\BibTeX}{\rm B\kern-.05em{\sc i\kern-.025em b}\kern-.08em\TeX}

%%%%%%%%%%%%%%%%%%%%%%%%%%%%%%%%%%%%%%%%%%%%%%%%%%%%%%%%%%%%%%%%%%%%%%%%

\begin{document}

%%% The following commands remove the headers in your paper. For final 
%%% papers, these will be inserted during the pagination process.

\pagestyle{fancy}
\fancyhead{}

%%% The next command prints the information defined in the preamble.

\maketitle 

%%%%%%%%%%%%%%%%%%%%%%%%%%%%%%%%%%%%%%%%%%%%%%%%%%%%%%%%%%%%%%%%%%%%%%%%

\section{Introduction}


\begin{figure}[t]
\centering
\includegraphics[width=0.6\columnwidth]{figures/evaluation_desiderata_V5.pdf}
\vspace{-0.5cm}
\caption{\systemName is a platform for conducting realistic evaluations of code LLMs, collecting human preferences of coding models with real users, real tasks, and in realistic environments, aimed at addressing the limitations of existing evaluations.
}
\label{fig:motivation}
\end{figure}

\begin{figure*}[t]
\centering
\includegraphics[width=\textwidth]{figures/system_design_v2.png}
\caption{We introduce \systemName, a VSCode extension to collect human preferences of code directly in a developer's IDE. \systemName enables developers to use code completions from various models. The system comprises a) the interface in the user's IDE which presents paired completions to users (left), b) a sampling strategy that picks model pairs to reduce latency (right, top), and c) a prompting scheme that allows diverse LLMs to perform code completions with high fidelity.
Users can select between the top completion (green box) using \texttt{tab} or the bottom completion (blue box) using \texttt{shift+tab}.}
\label{fig:overview}
\end{figure*}

As model capabilities improve, large language models (LLMs) are increasingly integrated into user environments and workflows.
For example, software developers code with AI in integrated developer environments (IDEs)~\citep{peng2023impact}, doctors rely on notes generated through ambient listening~\citep{oberst2024science}, and lawyers consider case evidence identified by electronic discovery systems~\citep{yang2024beyond}.
Increasing deployment of models in productivity tools demands evaluation that more closely reflects real-world circumstances~\citep{hutchinson2022evaluation, saxon2024benchmarks, kapoor2024ai}.
While newer benchmarks and live platforms incorporate human feedback to capture real-world usage, they almost exclusively focus on evaluating LLMs in chat conversations~\citep{zheng2023judging,dubois2023alpacafarm,chiang2024chatbot, kirk2024the}.
Model evaluation must move beyond chat-based interactions and into specialized user environments.



 

In this work, we focus on evaluating LLM-based coding assistants. 
Despite the popularity of these tools---millions of developers use Github Copilot~\citep{Copilot}---existing
evaluations of the coding capabilities of new models exhibit multiple limitations (Figure~\ref{fig:motivation}, bottom).
Traditional ML benchmarks evaluate LLM capabilities by measuring how well a model can complete static, interview-style coding tasks~\citep{chen2021evaluating,austin2021program,jain2024livecodebench, white2024livebench} and lack \emph{real users}. 
User studies recruit real users to evaluate the effectiveness of LLMs as coding assistants, but are often limited to simple programming tasks as opposed to \emph{real tasks}~\citep{vaithilingam2022expectation,ross2023programmer, mozannar2024realhumaneval}.
Recent efforts to collect human feedback such as Chatbot Arena~\citep{chiang2024chatbot} are still removed from a \emph{realistic environment}, resulting in users and data that deviate from typical software development processes.
We introduce \systemName to address these limitations (Figure~\ref{fig:motivation}, top), and we describe our three main contributions below.


\textbf{We deploy \systemName in-the-wild to collect human preferences on code.} 
\systemName is a Visual Studio Code extension, collecting preferences directly in a developer's IDE within their actual workflow (Figure~\ref{fig:overview}).
\systemName provides developers with code completions, akin to the type of support provided by Github Copilot~\citep{Copilot}. 
Over the past 3 months, \systemName has served over~\completions suggestions from 10 state-of-the-art LLMs, 
gathering \sampleCount~votes from \userCount~users.
To collect user preferences,
\systemName presents a novel interface that shows users paired code completions from two different LLMs, which are determined based on a sampling strategy that aims to 
mitigate latency while preserving coverage across model comparisons.
Additionally, we devise a prompting scheme that allows a diverse set of models to perform code completions with high fidelity.
See Section~\ref{sec:system} and Section~\ref{sec:deployment} for details about system design and deployment respectively.



\textbf{We construct a leaderboard of user preferences and find notable differences from existing static benchmarks and human preference leaderboards.}
In general, we observe that smaller models seem to overperform in static benchmarks compared to our leaderboard, while performance among larger models is mixed (Section~\ref{sec:leaderboard_calculation}).
We attribute these differences to the fact that \systemName is exposed to users and tasks that differ drastically from code evaluations in the past. 
Our data spans 103 programming languages and 24 natural languages as well as a variety of real-world applications and code structures, while static benchmarks tend to focus on a specific programming and natural language and task (e.g. coding competition problems).
Additionally, while all of \systemName interactions contain code contexts and the majority involve infilling tasks, a much smaller fraction of Chatbot Arena's coding tasks contain code context, with infilling tasks appearing even more rarely. 
We analyze our data in depth in Section~\ref{subsec:comparison}.



\textbf{We derive new insights into user preferences of code by analyzing \systemName's diverse and distinct data distribution.}
We compare user preferences across different stratifications of input data (e.g., common versus rare languages) and observe which affect observed preferences most (Section~\ref{sec:analysis}).
For example, while user preferences stay relatively consistent across various programming languages, they differ drastically between different task categories (e.g. frontend/backend versus algorithm design).
We also observe variations in user preference due to different features related to code structure 
(e.g., context length and completion patterns).
We open-source \systemName and release a curated subset of code contexts.
Altogether, our results highlight the necessity of model evaluation in realistic and domain-specific settings.






\section{RELATED WORK}
\label{sec:relatedwork}
In this section, we describe the previous works related to our proposal, which are divided into two parts. In Section~\ref{sec:relatedwork_exoplanet}, we present a review of approaches based on machine learning techniques for the detection of planetary transit signals. Section~\ref{sec:relatedwork_attention} provides an account of the approaches based on attention mechanisms applied in Astronomy.\par

\subsection{Exoplanet detection}
\label{sec:relatedwork_exoplanet}
Machine learning methods have achieved great performance for the automatic selection of exoplanet transit signals. One of the earliest applications of machine learning is a model named Autovetter \citep{MCcauliff}, which is a random forest (RF) model based on characteristics derived from Kepler pipeline statistics to classify exoplanet and false positive signals. Then, other studies emerged that also used supervised learning. \cite{mislis2016sidra} also used a RF, but unlike the work by \citet{MCcauliff}, they used simulated light curves and a box least square \citep[BLS;][]{kovacs2002box}-based periodogram to search for transiting exoplanets. \citet{thompson2015machine} proposed a k-nearest neighbors model for Kepler data to determine if a given signal has similarity to known transits. Unsupervised learning techniques were also applied, such as self-organizing maps (SOM), proposed \citet{armstrong2016transit}; which implements an architecture to segment similar light curves. In the same way, \citet{armstrong2018automatic} developed a combination of supervised and unsupervised learning, including RF and SOM models. In general, these approaches require a previous phase of feature engineering for each light curve. \par

%DL is a modern data-driven technology that automatically extracts characteristics, and that has been successful in classification problems from a variety of application domains. The architecture relies on several layers of NNs of simple interconnected units and uses layers to build increasingly complex and useful features by means of linear and non-linear transformation. This family of models is capable of generating increasingly high-level representations \citep{lecun2015deep}.

The application of DL for exoplanetary signal detection has evolved rapidly in recent years and has become very popular in planetary science.  \citet{pearson2018} and \citet{zucker2018shallow} developed CNN-based algorithms that learn from synthetic data to search for exoplanets. Perhaps one of the most successful applications of the DL models in transit detection was that of \citet{Shallue_2018}; who, in collaboration with Google, proposed a CNN named AstroNet that recognizes exoplanet signals in real data from Kepler. AstroNet uses the training set of labelled TCEs from the Autovetter planet candidate catalog of Q1–Q17 data release 24 (DR24) of the Kepler mission \citep{catanzarite2015autovetter}. AstroNet analyses the data in two views: a ``global view'', and ``local view'' \citep{Shallue_2018}. \par


% The global view shows the characteristics of the light curve over an orbital period, and a local view shows the moment at occurring the transit in detail

%different = space-based

Based on AstroNet, researchers have modified the original AstroNet model to rank candidates from different surveys, specifically for Kepler and TESS missions. \citet{ansdell2018scientific} developed a CNN trained on Kepler data, and included for the first time the information on the centroids, showing that the model improves performance considerably. Then, \citet{osborn2020rapid} and \citet{yu2019identifying} also included the centroids information, but in addition, \citet{osborn2020rapid} included information of the stellar and transit parameters. Finally, \citet{rao2021nigraha} proposed a pipeline that includes a new ``half-phase'' view of the transit signal. This half-phase view represents a transit view with a different time and phase. The purpose of this view is to recover any possible secondary eclipse (the object hiding behind the disk of the primary star).


%last pipeline applies a procedure after the prediction of the model to obtain new candidates, this process is carried out through a series of steps that include the evaluation with Discovery and Validation of Exoplanets (DAVE) \citet{kostov2019discovery} that was adapted for the TESS telescope.\par
%



\subsection{Attention mechanisms in astronomy}
\label{sec:relatedwork_attention}
Despite the remarkable success of attention mechanisms in sequential data, few papers have exploited their advantages in astronomy. In particular, there are no models based on attention mechanisms for detecting planets. Below we present a summary of the main applications of this modeling approach to astronomy, based on two points of view; performance and interpretability of the model.\par
%Attention mechanisms have not yet been explored in all sub-areas of astronomy. However, recent works show a successful application of the mechanism.
%performance

The application of attention mechanisms has shown improvements in the performance of some regression and classification tasks compared to previous approaches. One of the first implementations of the attention mechanism was to find gravitational lenses proposed by \citet{thuruthipilly2021finding}. They designed 21 self-attention-based encoder models, where each model was trained separately with 18,000 simulated images, demonstrating that the model based on the Transformer has a better performance and uses fewer trainable parameters compared to CNN. A novel application was proposed by \citet{lin2021galaxy} for the morphological classification of galaxies, who used an architecture derived from the Transformer, named Vision Transformer (VIT) \citep{dosovitskiy2020image}. \citet{lin2021galaxy} demonstrated competitive results compared to CNNs. Another application with successful results was proposed by \citet{zerveas2021transformer}; which first proposed a transformer-based framework for learning unsupervised representations of multivariate time series. Their methodology takes advantage of unlabeled data to train an encoder and extract dense vector representations of time series. Subsequently, they evaluate the model for regression and classification tasks, demonstrating better performance than other state-of-the-art supervised methods, even with data sets with limited samples.

%interpretation
Regarding the interpretability of the model, a recent contribution that analyses the attention maps was presented by \citet{bowles20212}, which explored the use of group-equivariant self-attention for radio astronomy classification. Compared to other approaches, this model analysed the attention maps of the predictions and showed that the mechanism extracts the brightest spots and jets of the radio source more clearly. This indicates that attention maps for prediction interpretation could help experts see patterns that the human eye often misses. \par

In the field of variable stars, \citet{allam2021paying} employed the mechanism for classifying multivariate time series in variable stars. And additionally, \citet{allam2021paying} showed that the activation weights are accommodated according to the variation in brightness of the star, achieving a more interpretable model. And finally, related to the TESS telescope, \citet{morvan2022don} proposed a model that removes the noise from the light curves through the distribution of attention weights. \citet{morvan2022don} showed that the use of the attention mechanism is excellent for removing noise and outliers in time series datasets compared with other approaches. In addition, the use of attention maps allowed them to show the representations learned from the model. \par

Recent attention mechanism approaches in astronomy demonstrate comparable results with earlier approaches, such as CNNs. At the same time, they offer interpretability of their results, which allows a post-prediction analysis. \par



\section{Background}\label{sec:backgrnd}

\subsection{Cold Start Latency and Mitigation Techniques}

Traditional FaaS platforms mitigate cold starts through snapshotting, lightweight virtualization, and warm-state management. Snapshot-based methods like \textbf{REAP} and \textbf{Catalyzer} reduce initialization time by preloading or restoring container states but require significant memory and I/O resources, limiting scalability~\cite{dong_catalyzer_2020, ustiugov_benchmarking_2021}. Lightweight virtualization solutions, such as \textbf{Firecracker} microVMs, achieve fast startup times with strong isolation but depend on robust infrastructure, making them less adaptable to fluctuating workloads~\cite{agache_firecracker_2020}. Warm-state management techniques like \textbf{Faa\$T}~\cite{romero_faa_2021} and \textbf{Kraken}~\cite{vivek_kraken_2021} keep frequently invoked containers ready, balancing readiness and cost efficiency under predictable workloads but incurring overhead when demand is erratic~\cite{romero_faa_2021, vivek_kraken_2021}. While these methods perform well in resource-rich cloud environments, their resource intensity challenges applicability in edge settings.

\subsubsection{Edge FaaS Perspective}

In edge environments, cold start mitigation emphasizes lightweight designs, resource sharing, and hybrid task distribution. Lightweight execution environments like unikernels~\cite{edward_sock_2018} and \textbf{Firecracker}~\cite{agache_firecracker_2020}, as used by \textbf{TinyFaaS}~\cite{pfandzelter_tinyfaas_2020}, minimize resource usage and initialization delays but require careful orchestration to avoid resource contention. Function co-location, demonstrated by \textbf{Photons}~\cite{v_dukic_photons_2020}, reduces redundant initializations by sharing runtime resources among related functions, though this complicates isolation in multi-tenant setups~\cite{v_dukic_photons_2020}. Hybrid offloading frameworks like \textbf{GeoFaaS}~\cite{malekabbasi_geofaas_2024} balance edge-cloud workloads by offloading latency-tolerant tasks to the cloud and reserving edge resources for real-time operations, requiring reliable connectivity and efficient task management. These edge-specific strategies address cold starts effectively but introduce challenges in scalability and orchestration.

\subsection{Predictive Scaling and Caching Techniques}

Efficient resource allocation is vital for maintaining low latency and high availability in serverless platforms. Predictive scaling and caching techniques dynamically provision resources and reduce cold start latency by leveraging workload prediction and state retention.
Traditional FaaS platforms use predictive scaling and caching to optimize resources, employing techniques (OFC, FaasCache) to reduce cold starts. However, these methods rely on centralized orchestration and workload predictability, limiting their effectiveness in dynamic, resource-constrained edge environments.



\subsubsection{Edge FaaS Perspective}

Edge FaaS platforms adapt predictive scaling and caching techniques to constrain resources and heterogeneous environments. \textbf{EDGE-Cache}~\cite{kim_delay-aware_2022} uses traffic profiling to selectively retain high-priority functions, reducing memory overhead while maintaining readiness for frequent requests. Hybrid frameworks like \textbf{GeoFaaS}~\cite{malekabbasi_geofaas_2024} implement distributed caching to balance resources between edge and cloud nodes, enabling low-latency processing for critical tasks while offloading less critical workloads. Machine learning methods, such as clustering-based workload predictors~\cite{gao_machine_2020} and GRU-based models~\cite{guo_applying_2018}, enhance resource provisioning in edge systems by efficiently forecasting workload spikes. These innovations effectively address cold start challenges in edge environments, though their dependency on accurate predictions and robust orchestration poses scalability challenges.

\subsection{Decentralized Orchestration, Function Placement, and Scheduling}

Efficient orchestration in serverless platforms involves workload distribution, resource optimization, and performance assurance. While traditional FaaS platforms rely on centralized control, edge environments require decentralized and adaptive strategies to address unique challenges such as resource constraints and heterogeneous hardware.



\subsubsection{Edge FaaS Perspective}

Edge FaaS platforms adopt decentralized and adaptive orchestration frameworks to meet the demands of resource-constrained environments. Systems like \textbf{Wukong} distribute scheduling across edge nodes, enhancing data locality and scalability while reducing network latency. Lightweight frameworks such as \textbf{OpenWhisk Lite}~\cite{kravchenko_kpavelopenwhisk-light_2024} optimize resource allocation by decentralizing scheduling policies, minimizing cold starts and latency in edge setups~\cite{benjamin_wukong_2020}. Hybrid solutions like \textbf{OpenFaaS}~\cite{noauthor_openfaasfaas_2024} and \textbf{EdgeMatrix}~\cite{shen_edgematrix_2023} combine edge-cloud orchestration to balance resource utilization, retaining latency-sensitive functions at the edge while offloading non-critical workloads to the cloud. While these approaches improve flexibility, they face challenges in maintaining coordination and ensuring consistent performance across distributed nodes.



\section{Method}

In this section, we formalize the safe exploration problem and detail the method we propose to solve it.

\subsection{Defining the safe exploration problem}
\label{subsection:safe_exp_problem}

We aim for our agent to learn how to maximize coverage of a goal set, which corresponds to solving a multi-goal MDP (\refSection{subsection:GCRL}). However, the agent must also explore its environment safely, avoiding mistakes. In our framework, a mistake occurs when a terminal state is reached. Therefore, the agent must avoid terminal states during exploration. To do so, we provide our agent with a safety policy that we can activate when the GC policy is about to make a mistake and lead the agent out of danger. 

A way to build this safety policy is under the angle of classic RL. We consider a set $\neighborhood \subset \stateSpace$ of states that are desirable in terms of safety, for instance near some equilibrium point. The safety reward $\rS$ equals $1$ if $s \in \neighborhood$ and 0 otherwise. This reward setting is convenient as it relates the sum of rewards to the number of steps $T^{\piS}(s, a)$ necessary to reach $\neighborhood$ from a given state-action couple $(s, a)$. Then, to decide when to switch from one policy to the other we can compare the estimation given by the safety critic of the number of steps necessary to reach $ \neighborhood$ to a threshold.

However, critics are neural networks, which are continuous functions, while the number of steps is finite for states from which we can reach $\neighborhood$ and infinite for terminal states regarding an optimal safety policy. Thus, it pushes the critic to generalize safety to unsafe states and leads the agent to believe that it can still use the GC policy even though objectively, it is about to make a mistake. Therefore, in addition to the temporal distance, we added a notion of distance in the state space between the current state and the set of terminal states. To do so, we assume the agent receives a cost value $h(s)$ at each environment step, where $s$ is the current state and $h: \stateSpace \to \R$ a continuous constraint function that verifies $h(s) > 0$ if and only if $s$ is a terminal state. In our framework, we identify terminal states as mistakes the agent absolutely has to avoid. Like in RCRL, the function $h$ is related to a kind of distance between the current state $s$ and the set of terminal states \cite{RCRL2022}. Assuming to have access to such a function is not very restrictive as robots have sensors and often state estimation modules. The continuity of function $h$ is crucial as it allows for generalization from known states to unseen states. Indeed, an unvisited state near another visited and safe state which is far enough from terminal states is likely to be safe too. On the contrary, states near unsafe states are likely to be unsafe. As a result, we switch from the GC policy to the safety policy if the number of steps to reach $\neighborhood$ is too high or if $h(s)$ is close to $0$. 

All these considerations lead us to define the safe exploration problem as the combination of a CMDP $(\stateSpace, \actionSpace, p, p_0, \rS, h)$
and a multi-goal MDP $(\stateSpace \times \goalSpace, \actionSpace, p, p_0, \pG, \rGC)$:

\begin{equation}
    \mathcal{T} = (\stateSpace \times \goalSpace, \actionSpace, p, p_0, \pG, \rS, \rGC, h)
    \label{eq:SafeExpProblem}
\end{equation}

In the pretraining phase, we train a parametrized stochastic safety policy $\piS$ that solves the CMDP. 
The CMDP is independent of the goal space as the notion of safety is independent of the goals pursued by the agent.
We assume that we have access to simulation to perform the safety pretraining, allowing us to perform reset anywhere 
and then to train the policy on a wide variety of situations. The critics that have been trained are then used in the 
action selection mechanism.

During the safe exploration phase, the safety policy and its critics are fixed and we train a GC policy. 
The multi-goal MDP part follows the framework developed in \refSection{subsection:GCRL}. 
Transitions are collected and stored in an episodic replay buffer $\buffer$, regardless of the policy that generates them.
Thus transitions generated by both policies can be found in the same episode. The main idea behind is that both policies 
can learn from each other as their respective objectives may be complementary in some situations.

\subsection{Safety policy learning}
\label{subsection:safety_policy_learning}

To train the safety policy, we take inspiration from the distributional algorithm TQC, as it uses an ensemble critics for robustness and truncation in the critic update to prevent overestimation \cite{TQC}.
On the one hand, we train $M$ approximations $Z_{\psi_1}, ... Z_{\psi_M}$ of the safety policy return distribution $Z^{\piS}(s, a) = \Distrib_{\piS} \left[ \left. \sum_{t=0}^{+\infty} \gamma^t \rS(s_t, a_t) \right| s, a \right]$ using the TQC critic loss. The targets $Z_{\overline{\psi_1}}, ... Z_{\overline{\psi_M}}$ are initialized in the same way and follow $Z_{\psi_1}, ... Z_{\psi_M}$ via exponential moving average update. 
Similarly, we train an ensemble of $M$ reachability critics $R_{\xi_1}, ... R_{\xi_M}$ but we replace the usual target based on a sum with the RCRL \cite{RCRL2022} target based on a $\max$ operator, leading to equation \refEq{eq:RCRL_target}. 
Likewise, targets $R_{\overline{\xi_1}}, ... R_{\overline{\xi_M}}$ follow $R_{\xi_1}, ... R_{\xi_M}$ via exponential moving average. Unlike TQC, the reachability critics are updated separately using the quantile regression loss \cite{QR-DQN}.
Critic ensemble are trained on the same batched transitions but initialized with different seeds.

\begin{equation}
    \mathcal{T}\theta^{(j)} = (1-\gamma) h(s') + \gamma \max\left( h(s'), \theta_{\overline{\xi}}^{(j)}(s', a') \right)
    \label{eq:RCRL_target}
\end{equation}

Policy parameters $\phi_S$ are optimized to minimize the following loss, also inspired by TQC \cite{TQC}:

\begin{equation}
    \mathcal{L}_{\pi_S}(\phi_S) = \expect_{\substack{s, a \sim \mathcal{B}}}
    \left[ \alpha \log \piS(a|s) - \overline{Q}(s, a) + \lambda \overline{R}(s, a) \right]
    \label{eq:safety_actor_loss}
\end{equation}
where $\overline{Q}(s, a) = \frac{1}{M N} \sum_{i = 1}^{M} \sum_{j = 1}^{N} \theta_{\psi_i}^{(j)}(s, a)$

$\overline{R}(s, a) = \frac{1}{M N} \sum_{i = 1}^{M} \sum_{j = 1}^{N} \theta_{\xi_i}^{(j)}(s, a)$ and $\lambda \in \R_+$ is
a positive multiplier that we keep fixed. In our experiments, we performed an ablation study to identify the effect of 
reachability critics on safety during exploration. So we tested $\lambda = 0$ and $\lambda = 100$. The value of $100$
has been chosen so that $\overline{Q}(s, a)$ and $\overline{R}(s, a)$ have the same scale. We also tested a varying $\lambda$, 
like \citeauthor{ha2020SACLagLevine}, but it led to too much instability in the safety training \cite{ha2020SACLagLevine}.

\subsection{Action selection mechanism}
\label{subsection:Action selection mechanism}

\begin{algorithm}[H]
    \begin{algorithmic}[1]
    \STATE \textbf{Inputs:} $s, g, \SafetyActivated, \thGCtoS, \thStoGC$ 
    \STATE $\aGC \sim \piGC(.|s, g)$ ; $\aS \sim \piS(.|s)$
    \STATE $\text{lower} \leftarrow (\hat{\sigma}^{\piS}(s, \aGC) > \thStoGC)$
    \STATE $\text{raise} \leftarrow (\hat{\sigma}^{\piS}(s, \aS) > \thGCtoS)$
    \STATE $\SafetyActivated \leftarrow \SafetyActivated \, \text{\textbf{and}} \, \text{lower}$ \hfill // Lower the flag
    \STATE $\SafetyActivated \leftarrow \SafetyActivated \, \text{\textbf{or}} \, \text{raise}$  \hfill // Raise the flag
    \STATE $a \leftarrow \SafetyActivated . \aS + (1 - \SafetyActivated) . \aGC$
    \STATE \textbf{Return} $a$, $\SafetyActivated$
    \end{algorithmic}
    \caption{Action selection for safe exploration}
    \label{alg:Action selection}
\end{algorithm}

The action selection mechanism that must ensure safe exploration has been reproduced in \refAlg{alg:Action selection}.
It takes as input the current state $s$, current desired goal $g$, the current value of a boolean flag, called $\SafetyActivated$, and 
risk thresholds $\thGCtoS$ and $\thStoGC$. The flag indicates which action to choose, and the role of the action selection mechanism is to update this flag. If equal to $1$, action $\aS$ 
sampled by the safety policy is selected, else action $\aGC$ sampled by the GC policy is selected.
The flag is updated according to the level of risk associated  
with state $s$ and possible actions $\aGC$ and $\aS$. If the level of risk $\hat{\sigma}^{\piS}(s, \aS)$ 
associated with state $s$ and action $\aS$ exceeds threshold $\thGCtoS$, the safety flag is set to $1$ (\textit{raised})
so as to avoid a future potentially dangerous situation. On the contrary, if the level of risk $\hat{\sigma}^{\piS}(s, \aGC)$ 
associated with state $s$ and action $\aGC$ goes below threshold $\thStoGC$, the safety flag is set to $0$ (\textit{lowered}).
Note that the risk function $\hat{\sigma}^{\piS}$ is related to the safety policy, as its goal is to evaluate the capacity 
of the safety policy to put the agent out of danger. 
Also, because the safety policy is optimized on its critics, it tends to minimize the risk function undirectly. As a result, 
the thresholds should verify $\thStoGC \le \thGCtoS$. Otherwise, the safety policy would be the only one to act.
Depending on the environment, one could choose either equality or strict inequality between thresholds 
(See section \ref{subsec:ablation_dist}). 

The framework defined in section \refSection{subsection:safe_exp_problem} essentially leads to two 
definitions of the risk function, leading to three different strategies. 
The first is based on the time necessary to reach the set $\neighborhood$ from current state, using the safety policy. 
The second is based on the maximum of constraint function $h$ along future trajectories generated by the safety policy.
The third possibility is to combine both. We will use the labels \textbf{Time}, \textbf{Constraint} and \textbf{Time-constraint}
for the respective three strategies. 

\paragraph{\textbf{Time:}} We choose the set $\neighborhood$ so that, if the safety policy is well trained, 
then $\neighborhood$ is a forward set regarding the safety policy $\piS$. More precisely, we assume that, 
for any sequence of transitions $(s_0, a_0, ..., s_t, ...)$ generated by $\piS$, if 
the starting state $s_0$ lies in $\neighborhood$, then all future states $s_t$ also lie in $\neighborhood$.
Under this assumption the safety reward $\rS(s_t, a_t)$ equals $0$ until $s_t$ lies in $\neighborhood$, resulting in equation 
\refEq{eq:time_and_rewards}:

\begin{equation}
    T^{\piS}(s_0, a_0) = f_{\gamma}\left(\sum_{t = 0}^{+\infty} \gamma_t \rS(s_t, a_t)\right) 
    \label{eq:time_and_rewards}
\end{equation}
where $T^{\piS}(s_0, a_0)$ is the random variable corresponding to the number of steps necessary for $\piS$ to reach 
$\neighborhood$ starting from state-action couple $(s_0, a_0)$, and $f_{\gamma}(x) = \log \left( (1-\gamma)  x\right) / \log \gamma$,
which is a continuous bijection from $]0, 1/(1-\gamma)]$ to $\R^+$. By applying the bijective mapping $f_{\gamma}(x)$ to the atoms $\theta_{\psi_i}^{(j)}(s, a)$ 
of ensemble $Z_{\psi_1}, ... Z_{\psi_M}$, we obtain new atoms that approximate the distribution 
of the random variable $T^{\piS}(s, a)$.
We denote $\hat{T}^{\piS}(s, a ; \tau)$ the mean of the quantiles corresponding to cumulative probabilities 
greater than $\tau$. For example, $\tau = 0.9$ corresponds to the mean of worst $10\%$ of cases. $\tau$
is a hyperparameter of the algorithm. In the Time strategy : $\hat{\sigma}^{\piS}(s, a) = \hat{T}^{\piS}(s, a ; \tau)$.
So the thresholds $\thGCtoS$ and $\thStoGC$ are given in number of environment steps.

\paragraph{\textbf{Constraint:}}
In the same way, we denote $\hat{R}^{\piS}(s, a ; \tau)$ the mean of atoms from reachability critics $R_{\xi_1}, ... R_{\xi_M}$
corresponding to cumulative probabilities greater than $\tau$. 
In the constraint strategy : $\hat{\sigma}^{\piS}(s, a) = \hat{R}^{\piS}(s, a ; \tau)$.
So the thresholds $\thGCtoS$ and $\thStoGC$ are negative real numbers corresponding to safety margins.

\paragraph{\textbf{Time-constraint:}} This strategy combines the two previous ones. Let $\epsilon > 0$ be a positive real number 
representing a safety margin. If $\hat{R}^{\piS}(s, a ; \tau) > -\epsilon$, then: $\hat{\sigma}^{\piS}(s, a) = T_{max}$. 
Where $T_{max}$ is the maximum number of episode steps. 
Otherwise: $\hat{\sigma}^{\piS}(s, a) = \hat{T}^{\piS}(s, a ; \tau)$. 
The thresholds $\thGCtoS$ and $\thStoGC$ are given in number of environment steps.

\subsection{Algorithm}

\paragraph{\textbf{Pretraining phase:}} We use the same training loop as TQC 
\cite{TQC}. The only difference is the gradient step update where, in addition to the TQC update,
parameters $\xi_1, ... \xi_M$ are updated by performing Adam optimizer step on 
loss $\mathcal{L}_{Z}$ with target \refEq{eq:RCRL_target}, and the actor loss is replaced with loss $\mathcal{L}_{\pi_S}$.
\refAlg{alg:safety_update} describes how the update is done.
The safety policy interacts with the environment, generating a
transition that is stored in a replay buffer $\buffer_S$. Then a batch of transtions is sampled uniformly
from the buffer and a gradient step update is performed on all parameters, so that for each stored transition,
one step of gradient is performed. 
Also, we start the training after $5000$ steps for which we sample random actions
to favor exploration and learning at the start.

\begin{algorithm}[H]
\begin{algorithmic}[1]
\STATE Sample a batch from the replay buffer $\buffer_S$ uniformly
\STATE Perform TQC critic update on $\psi_1, ... \psi_M$ \cite{TQC}
\STATE Update $\xi_1, ... \xi_M$ using Adam on loss $\mathcal{L}_{Z}$ with target \refEq{eq:RCRL_target}
\STATE Update actor parameters $\phi_S$ using Adam on loss $\mathcal{L}_{\pi_S}$
\STATE Update temperature $\alpha_S$ according to SAC rule \cite{SAC}
\end{algorithmic}
\caption{Safety gradient step update}
\label{alg:safety_update}
\end{algorithm}

\paragraph{\textbf{Safe exploration phase:}} The procedure is summarized in \refAlg{alg:main_algorithm}, 
which is also off-policy. 
We first choose safety thresholds depending on the risk level we want. 
Then the parameters of the previously learned safety policy and its critics are loaded,
while the parameters related to the GC policy are randomly initialized. The episodic replay buffer is 
initially empty. For each environment step during training, the action selected according to 
\refAlg{alg:Action selection} to ensure safety during exploration. Each transition is stored in an 
episodic replay buffer $\buffer$ regardless of the policy that generated it. As a result, in a single stored episode,
there are probably samples generated by both policies. 

Initially, the GC policy performs poorly, as it has not yet been trained.
Thus the first stored episodes contain a large majority of transitions generated by the safety policy, as it had to
make up for the random behavior of the GC policy. Then, as the GC policy improves itself, the proportion of transitions 
it has generated increases in the buffer. The fact that many transitions have not been generated by the GC policy can 
lead off-policy algorithms like SAC to value over-estimation, due to the distributional drift between the dataset 
and the current learned policy \cite{levine2020offline}. This is why we used SAC-N instead of SAC, which is the 
same algorithm as SAC but with $N$ critics instead of $2$ \cite{sac_n_edac}. As the critics are initialized and updated
separately, disagreement between them regarding unvisited states is likely to lead the Bellman target to low values via the 
$\min$ operator, thus preventing over-estimation. 

\begin{algorithm}[H]
\begin{algorithmic}[1]
%\STATE \textbf{Initialize} Safety and GC parameters ;  %$\psi_1... \psi_M$, $\xi_1... \xi_M$, $\phi_S$, $\alpha_S$
%\STATE \textbf{Initialize} GC parameters %$\psi_{GC}... \psi_M$, $\xi_1... \xi_M$, $\phi_S$, $\alpha_S$
\STATE \textbf{Inputs:} $\thGCtoS, \thStoGC$ 
\STATE \textbf{Load} Safety parameters $\psi_i, \overline{\psi}_i$, $\xi_i, \overline{\xi}_i$, $i\in[\![1, M]\!]$, $\phi_S$, $\alpha_S$
\STATE \textbf{Initialize} GC parameters $\psi_{GC}, \overline{\psi}_{GC}$, $\phi_{GC}$, $\alpha_{GC}$
\STATE \textbf{Initialize} Episodic replay buffer $\buffer \leftarrow \emptyset$
\STATE \textbf{Sample} initial state $s_0 \sim p_0$, and goal $g \sim \pG$ 
\FOR{each iteration}
\FOR{each environment step, until done}
\STATE $a_t, \SafetyActivated \leftarrow \text{select\_action}(s, g, \SafetyActivated,$ \\$ \thGCtoS, \thStoGC)$
(\refAlg{alg:Action selection})
\STATE Collect transition $(s_t, a_t, s_{t+1}, r_{S, t}, r_{GC, t}, h_{t+1})$
\STATE $\buffer \leftarrow \buffer \cup \{ (s_t, a_t, s_{t+1}, r_{S, t}, r_{GC, t}, h_{t+1}) \}$
\ENDFOR 
\FOR{each gradient step}

\STATE Sample a batch $b_{GC} = (s_t, a_t, s_{t+1}, r_{GC, t})$ from $\buffer$ \\ using HER \cite{HER}
\STATE Update GC parameters on $b_{GC}$ using SAC-N rule \cite{SACN_edac}
% \IF{safety finetuning activated}
% \STATE Update safety policy using \refAlg{alg:safety_update}
% \ENDIF
\ENDFOR
\ENDFOR
\STATE \textbf{return} GC parameters $\psi_{GC}, \overline{\psi}_{GC}$, $\phi_{GC}$, $\alpha_{GC}$ %\\
%Safety parameters $\psi_i, \overline{\psi}_i$, $\xi_i, \overline{\xi}_i$, $i\in[\![1, M]\!]$, $\phi_S$, $\alpha_S$
%Replay buffer $\buffer$
\end{algorithmic}
\caption{Safe exploration algorithm}
\label{alg:main_algorithm}
\end{algorithm}


\section{Experiments}
\label{sec:experiments}
The experiments are designed to address two key research questions.
First, \textbf{RQ1} evaluates whether the average $L_2$-norm of the counterfactual perturbation vectors ($\overline{||\perturb||}$) decreases as the model overfits the data, thereby providing further empirical validation for our hypothesis.
Second, \textbf{RQ2} evaluates the ability of the proposed counterfactual regularized loss, as defined in (\ref{eq:regularized_loss2}), to mitigate overfitting when compared to existing regularization techniques.

% The experiments are designed to address three key research questions. First, \textbf{RQ1} investigates whether the mean perturbation vector norm decreases as the model overfits the data, aiming to further validate our intuition. Second, \textbf{RQ2} explores whether the mean perturbation vector norm can be effectively leveraged as a regularization term during training, offering insights into its potential role in mitigating overfitting. Finally, \textbf{RQ3} examines whether our counterfactual regularizer enables the model to achieve superior performance compared to existing regularization methods, thus highlighting its practical advantage.

\subsection{Experimental Setup}
\textbf{\textit{Datasets, Models, and Tasks.}}
The experiments are conducted on three datasets: \textit{Water Potability}~\cite{kadiwal2020waterpotability}, \textit{Phomene}~\cite{phomene}, and \textit{CIFAR-10}~\cite{krizhevsky2009learning}. For \textit{Water Potability} and \textit{Phomene}, we randomly select $80\%$ of the samples for the training set, and the remaining $20\%$ for the test set, \textit{CIFAR-10} comes already split. Furthermore, we consider the following models: Logistic Regression, Multi-Layer Perceptron (MLP) with 100 and 30 neurons on each hidden layer, and PreactResNet-18~\cite{he2016cvecvv} as a Convolutional Neural Network (CNN) architecture.
We focus on binary classification tasks and leave the extension to multiclass scenarios for future work. However, for datasets that are inherently multiclass, we transform the problem into a binary classification task by selecting two classes, aligning with our assumption.

\smallskip
\noindent\textbf{\textit{Evaluation Measures.}} To characterize the degree of overfitting, we use the test loss, as it serves as a reliable indicator of the model's generalization capability to unseen data. Additionally, we evaluate the predictive performance of each model using the test accuracy.

\smallskip
\noindent\textbf{\textit{Baselines.}} We compare CF-Reg with the following regularization techniques: L1 (``Lasso''), L2 (``Ridge''), and Dropout.

\smallskip
\noindent\textbf{\textit{Configurations.}}
For each model, we adopt specific configurations as follows.
\begin{itemize}
\item \textit{Logistic Regression:} To induce overfitting in the model, we artificially increase the dimensionality of the data beyond the number of training samples by applying a polynomial feature expansion. This approach ensures that the model has enough capacity to overfit the training data, allowing us to analyze the impact of our counterfactual regularizer. The degree of the polynomial is chosen as the smallest degree that makes the number of features greater than the number of data.
\item \textit{Neural Networks (MLP and CNN):} To take advantage of the closed-form solution for computing the optimal perturbation vector as defined in (\ref{eq:opt-delta}), we use a local linear approximation of the neural network models. Hence, given an instance $\inst_i$, we consider the (optimal) counterfactual not with respect to $\model$ but with respect to:
\begin{equation}
\label{eq:taylor}
    \model^{lin}(\inst) = \model(\inst_i) + \nabla_{\inst}\model(\inst_i)(\inst - \inst_i),
\end{equation}
where $\model^{lin}$ represents the first-order Taylor approximation of $\model$ at $\inst_i$.
Note that this step is unnecessary for Logistic Regression, as it is inherently a linear model.
\end{itemize}

\smallskip
\noindent \textbf{\textit{Implementation Details.}} We run all experiments on a machine equipped with an AMD Ryzen 9 7900 12-Core Processor and an NVIDIA GeForce RTX 4090 GPU. Our implementation is based on the PyTorch Lightning framework. We use stochastic gradient descent as the optimizer with a learning rate of $\eta = 0.001$ and no weight decay. We use a batch size of $128$. The training and test steps are conducted for $6000$ epochs on the \textit{Water Potability} and \textit{Phoneme} datasets, while for the \textit{CIFAR-10} dataset, they are performed for $200$ epochs.
Finally, the contribution $w_i^{\varepsilon}$ of each training point $\inst_i$ is uniformly set as $w_i^{\varepsilon} = 1~\forall i\in \{1,\ldots,m\}$.

The source code implementation for our experiments is available at the following GitHub repository: \url{https://anonymous.4open.science/r/COCE-80B4/README.md} 

\subsection{RQ1: Counterfactual Perturbation vs. Overfitting}
To address \textbf{RQ1}, we analyze the relationship between the test loss and the average $L_2$-norm of the counterfactual perturbation vectors ($\overline{||\perturb||}$) over training epochs.

In particular, Figure~\ref{fig:delta_loss_epochs} depicts the evolution of $\overline{||\perturb||}$ alongside the test loss for an MLP trained \textit{without} regularization on the \textit{Water Potability} dataset. 
\begin{figure}[ht]
    \centering
    \includegraphics[width=0.85\linewidth]{img/delta_loss_epochs.png}
    \caption{The average counterfactual perturbation vector $\overline{||\perturb||}$ (left $y$-axis) and the cross-entropy test loss (right $y$-axis) over training epochs ($x$-axis) for an MLP trained on the \textit{Water Potability} dataset \textit{without} regularization.}
    \label{fig:delta_loss_epochs}
\end{figure}

The plot shows a clear trend as the model starts to overfit the data (evidenced by an increase in test loss). 
Notably, $\overline{||\perturb||}$ begins to decrease, which aligns with the hypothesis that the average distance to the optimal counterfactual example gets smaller as the model's decision boundary becomes increasingly adherent to the training data.

It is worth noting that this trend is heavily influenced by the choice of the counterfactual generator model. In particular, the relationship between $\overline{||\perturb||}$ and the degree of overfitting may become even more pronounced when leveraging more accurate counterfactual generators. However, these models often come at the cost of higher computational complexity, and their exploration is left to future work.

Nonetheless, we expect that $\overline{||\perturb||}$ will eventually stabilize at a plateau, as the average $L_2$-norm of the optimal counterfactual perturbations cannot vanish to zero.

% Additionally, the choice of employing the score-based counterfactual explanation framework to generate counterfactuals was driven to promote computational efficiency.

% Future enhancements to the framework may involve adopting models capable of generating more precise counterfactuals. While such approaches may yield to performance improvements, they are likely to come at the cost of increased computational complexity.


\subsection{RQ2: Counterfactual Regularization Performance}
To answer \textbf{RQ2}, we evaluate the effectiveness of the proposed counterfactual regularization (CF-Reg) by comparing its performance against existing baselines: unregularized training loss (No-Reg), L1 regularization (L1-Reg), L2 regularization (L2-Reg), and Dropout.
Specifically, for each model and dataset combination, Table~\ref{tab:regularization_comparison} presents the mean value and standard deviation of test accuracy achieved by each method across 5 random initialization. 

The table illustrates that our regularization technique consistently delivers better results than existing methods across all evaluated scenarios, except for one case -- i.e., Logistic Regression on the \textit{Phomene} dataset. 
However, this setting exhibits an unusual pattern, as the highest model accuracy is achieved without any regularization. Even in this case, CF-Reg still surpasses other regularization baselines.

From the results above, we derive the following key insights. First, CF-Reg proves to be effective across various model types, ranging from simple linear models (Logistic Regression) to deep architectures like MLPs and CNNs, and across diverse datasets, including both tabular and image data. 
Second, CF-Reg's strong performance on the \textit{Water} dataset with Logistic Regression suggests that its benefits may be more pronounced when applied to simpler models. However, the unexpected outcome on the \textit{Phoneme} dataset calls for further investigation into this phenomenon.


\begin{table*}[h!]
    \centering
    \caption{Mean value and standard deviation of test accuracy across 5 random initializations for different model, dataset, and regularization method. The best results are highlighted in \textbf{bold}.}
    \label{tab:regularization_comparison}
    \begin{tabular}{|c|c|c|c|c|c|c|}
        \hline
        \textbf{Model} & \textbf{Dataset} & \textbf{No-Reg} & \textbf{L1-Reg} & \textbf{L2-Reg} & \textbf{Dropout} & \textbf{CF-Reg (ours)} \\ \hline
        Logistic Regression   & \textit{Water}   & $0.6595 \pm 0.0038$   & $0.6729 \pm 0.0056$   & $0.6756 \pm 0.0046$  & N/A    & $\mathbf{0.6918 \pm 0.0036}$                     \\ \hline
        MLP   & \textit{Water}   & $0.6756 \pm 0.0042$   & $0.6790 \pm 0.0058$   & $0.6790 \pm 0.0023$  & $0.6750 \pm 0.0036$    & $\mathbf{0.6802 \pm 0.0046}$                    \\ \hline
%        MLP   & \textit{Adult}   & $0.8404 \pm 0.0010$   & $\mathbf{0.8495 \pm 0.0007}$   & $0.8489 \pm 0.0014$  & $\mathbf{0.8495 \pm 0.0016}$     & $0.8449 \pm 0.0019$                    \\ \hline
        Logistic Regression   & \textit{Phomene}   & $\mathbf{0.8148 \pm 0.0020}$   & $0.8041 \pm 0.0028$   & $0.7835 \pm 0.0176$  & N/A    & $0.8098 \pm 0.0055$                     \\ \hline
        MLP   & \textit{Phomene}   & $0.8677 \pm 0.0033$   & $0.8374 \pm 0.0080$   & $0.8673 \pm 0.0045$  & $0.8672 \pm 0.0042$     & $\mathbf{0.8718 \pm 0.0040}$                    \\ \hline
        CNN   & \textit{CIFAR-10} & $0.6670 \pm 0.0233$   & $0.6229 \pm 0.0850$   & $0.7348 \pm 0.0365$   & N/A    & $\mathbf{0.7427 \pm 0.0571}$                     \\ \hline
    \end{tabular}
\end{table*}

\begin{table*}[htb!]
    \centering
    \caption{Hyperparameter configurations utilized for the generation of Table \ref{tab:regularization_comparison}. For our regularization the hyperparameters are reported as $\mathbf{\alpha/\beta}$.}
    \label{tab:performance_parameters}
    \begin{tabular}{|c|c|c|c|c|c|c|}
        \hline
        \textbf{Model} & \textbf{Dataset} & \textbf{No-Reg} & \textbf{L1-Reg} & \textbf{L2-Reg} & \textbf{Dropout} & \textbf{CF-Reg (ours)} \\ \hline
        Logistic Regression   & \textit{Water}   & N/A   & $0.0093$   & $0.6927$  & N/A    & $0.3791/1.0355$                     \\ \hline
        MLP   & \textit{Water}   & N/A   & $0.0007$   & $0.0022$  & $0.0002$    & $0.2567/1.9775$                    \\ \hline
        Logistic Regression   &
        \textit{Phomene}   & N/A   & $0.0097$   & $0.7979$  & N/A    & $0.0571/1.8516$                     \\ \hline
        MLP   & \textit{Phomene}   & N/A   & $0.0007$   & $4.24\cdot10^{-5}$  & $0.0015$    & $0.0516/2.2700$                    \\ \hline
       % MLP   & \textit{Adult}   & N/A   & $0.0018$   & $0.0018$  & $0.0601$     & $0.0764/2.2068$                    \\ \hline
        CNN   & \textit{CIFAR-10} & N/A   & $0.0050$   & $0.0864$ & N/A    & $0.3018/
        2.1502$                     \\ \hline
    \end{tabular}
\end{table*}

\begin{table*}[htb!]
    \centering
    \caption{Mean value and standard deviation of training time across 5 different runs. The reported time (in seconds) corresponds to the generation of each entry in Table \ref{tab:regularization_comparison}. Times are }
    \label{tab:times}
    \begin{tabular}{|c|c|c|c|c|c|c|}
        \hline
        \textbf{Model} & \textbf{Dataset} & \textbf{No-Reg} & \textbf{L1-Reg} & \textbf{L2-Reg} & \textbf{Dropout} & \textbf{CF-Reg (ours)} \\ \hline
        Logistic Regression   & \textit{Water}   & $222.98 \pm 1.07$   & $239.94 \pm 2.59$   & $241.60 \pm 1.88$  & N/A    & $251.50 \pm 1.93$                     \\ \hline
        MLP   & \textit{Water}   & $225.71 \pm 3.85$   & $250.13 \pm 4.44$   & $255.78 \pm 2.38$  & $237.83 \pm 3.45$    & $266.48 \pm 3.46$                    \\ \hline
        Logistic Regression   & \textit{Phomene}   & $266.39 \pm 0.82$ & $367.52 \pm 6.85$   & $361.69 \pm 4.04$  & N/A   & $310.48 \pm 0.76$                    \\ \hline
        MLP   &
        \textit{Phomene} & $335.62 \pm 1.77$   & $390.86 \pm 2.11$   & $393.96 \pm 1.95$ & $363.51 \pm 5.07$    & $403.14 \pm 1.92$                     \\ \hline
       % MLP   & \textit{Adult}   & N/A   & $0.0018$   & $0.0018$  & $0.0601$     & $0.0764/2.2068$                    \\ \hline
        CNN   & \textit{CIFAR-10} & $370.09 \pm 0.18$   & $395.71 \pm 0.55$   & $401.38 \pm 0.16$ & N/A    & $1287.8 \pm 0.26$                     \\ \hline
    \end{tabular}
\end{table*}

\subsection{Feasibility of our Method}
A crucial requirement for any regularization technique is that it should impose minimal impact on the overall training process.
In this respect, CF-Reg introduces an overhead that depends on the time required to find the optimal counterfactual example for each training instance. 
As such, the more sophisticated the counterfactual generator model probed during training the higher would be the time required. However, a more advanced counterfactual generator might provide a more effective regularization. We discuss this trade-off in more details in Section~\ref{sec:discussion}.

Table~\ref{tab:times} presents the average training time ($\pm$ standard deviation) for each model and dataset combination listed in Table~\ref{tab:regularization_comparison}.
We can observe that the higher accuracy achieved by CF-Reg using the score-based counterfactual generator comes with only minimal overhead. However, when applied to deep neural networks with many hidden layers, such as \textit{PreactResNet-18}, the forward derivative computation required for the linearization of the network introduces a more noticeable computational cost, explaining the longer training times in the table.

\subsection{Hyperparameter Sensitivity Analysis}
The proposed counterfactual regularization technique relies on two key hyperparameters: $\alpha$ and $\beta$. The former is intrinsic to the loss formulation defined in (\ref{eq:cf-train}), while the latter is closely tied to the choice of the score-based counterfactual explanation method used.

Figure~\ref{fig:test_alpha_beta} illustrates how the test accuracy of an MLP trained on the \textit{Water Potability} dataset changes for different combinations of $\alpha$ and $\beta$.

\begin{figure}[ht]
    \centering
    \includegraphics[width=0.85\linewidth]{img/test_acc_alpha_beta.png}
    \caption{The test accuracy of an MLP trained on the \textit{Water Potability} dataset, evaluated while varying the weight of our counterfactual regularizer ($\alpha$) for different values of $\beta$.}
    \label{fig:test_alpha_beta}
\end{figure}

We observe that, for a fixed $\beta$, increasing the weight of our counterfactual regularizer ($\alpha$) can slightly improve test accuracy until a sudden drop is noticed for $\alpha > 0.1$.
This behavior was expected, as the impact of our penalty, like any regularization term, can be disruptive if not properly controlled.

Moreover, this finding further demonstrates that our regularization method, CF-Reg, is inherently data-driven. Therefore, it requires specific fine-tuning based on the combination of the model and dataset at hand.

\section{Conclusion}
In this work, we propose a simple yet effective approach, called SMILE, for graph few-shot learning with fewer tasks. Specifically, we introduce a novel dual-level mixup strategy, including within-task and across-task mixup, for enriching the diversity of nodes within each task and the diversity of tasks. Also, we incorporate the degree-based prior information to learn expressive node embeddings. Theoretically, we prove that SMILE effectively enhances the model's generalization performance. Empirically, we conduct extensive experiments on multiple benchmarks and the results suggest that SMILE significantly outperforms other baselines, including both in-domain and cross-domain few-shot settings.

% \begin{acks}
% If you wish to include any acknowledgments in your paper (e.g., to 
% people or funding agencies), please do so using the `\texttt{acks}' 
% environment. Note that the text of your acknowledgments will be omitted
% if you compile your document with the `\texttt{anonymous}' option.
% \end{acks}

%%%%%%%%%%%%%%%%%%%%%%%%%%%%%%%%%%%%%%%%%%%%%%%%%%%%%%%%%%%%%%%%%%%%%%%%

%%% The next two lines define, first, the bibliography style to be 
%%% applied, and, second, the bibliography file to be used.

\bibliographystyle{ACM-Reference-Format} 
\bibliography{biblio}

%%%%%%%%%%%%%%%%%%%%%%%%%%%%%%%%%%%%%%%%%%%%%%%%%%%%%%%%%%%%%%%%%%%%%%%%

\newpage

\appendix

%\let\balance\relax
%\begin{multicols}{2}

\onecolumn

\section{Supplementary material}

\subsection{Videos}

\begin{itemize}
  \item \url{failure\_cartpole.mp4}: CartPoleGC failure mode described in the paper ($(\thGCtoS, \thStoGC) = (70, 30)$).
  \item \url{near\_bound\_cartpole.mp4}: CartPoleGC with a "difficult" goal near the environment bounds, thus near unsafe states ($(\thGCtoS, \thStoGC) = (70, 30)$).
  \item \url{simple\_cartpole.mp4}: CartPoleGC with a simple goal with $(\thGCtoS, \thStoGC) = (70, 30)$.
  \item \url{30\_30\_cartpole.mp4}: CartPoleGC with a simple goal with $(\thGCtoS, \thStoGC) = (30, 30)$ to show the impact of low thresholds on coverage.
  \item \url{skydiox2\_example.mp4}: Example of SkydioX2 reaching a goal ($(\thGCtoS, \thStoGC) = (10, 10)$)
\end{itemize}

Note that for CartPoleGC, the cart is blue when the safety policy is activated and green otherwise.
The goal is represented by a red box. 

\subsection{Further analysis of failure modes}

\begin{figure}[ht]
  \includegraphics[width=0.32\textwidth]{appendix_image/921_safety_value_estimation.pdf}
  \includegraphics[width=0.32\textwidth]{appendix_image/921_scritic_critic_disagreement.pdf}
  \includegraphics[width=0.32\textwidth]{appendix_image/921_scritic_reach_disagreement.pdf}
  \caption{From left to right: Value of the risk function $\hat{\sigma}^{\piS}(s, a)$ along the failed episode
  shown as an example in Figure 10, where the thresholds are represented in magenta and purple ; Critic disagreement (Figure 10) ; Reachability critic disagreement.
  Blue dots correspond to the safety policy and green dots to the GC policy. One can see the hysteretic behavior on the left plot. A video of the failure (\url{failure\_cartpole.mp4}) is also attached.}
  \label{fig:failures}
  \Description{Further analysis of failure modes}
\end{figure}

We can see on the left plot (Figure \ref{fig:failures}) that the agent switches from the GC policy to the safety policy before making a 
mistake. 
We also observe that the oscillations of the disagreement, for both ensemble of critics,
are synchronized with the change of policies. Indeed, as the GC policy has a different objective
than the safety policy, it goes towards states that have been less visited by the safety policy during 
its pretraining. This phenomenon motivates safety finetuning for future works. 

  

\newpage

\subsection{Goal-space coverage performance with safe exploration on CartPoleGC}

\begin{figure}[ht]
  \includegraphics[width=0.5\textwidth]{appendix_image/rbest1614_450000_coverage.png}
  \caption{Coverage map obtained with $L\&S$ safe exploration variant and $(\thGCtoS, \thStoGC) = (70, 30)$
  on CartPoleGC. Only for this experience, the cartpole is reset on different $x$ positions. 
  Each cell corresponds to the combination of a starting position and a desired goal and we measure the 
  success rate. We can see that the success rate is lower for starting positions and goals near the 
  environment bounds, than for positions around the center. 
  The safety policy tends to prevent the agent from reaching goals near the bounds.
  In the same way, if the initial state is too close to the bound, the safety policy prevents the 
  GC policy from acting most of the time.}
  \label{fig:cartpole_cov}
  \Description{CartPoleGC coverage}
\end{figure}

% \newpage

\subsection{Constraint strategy}

\begin{figure}[ht]
\includegraphics[width=0.3\textwidth]{appendix_image/constraint_only.pdf}
\caption{Occurrence of mistakes obtained with the time-constraint strategy and the constraint (only) strategy 
on the CartPoleGC environment. Performance of constraint strategy in terms of safety is catastrophic.}
\label{fig:constraint_only}
\Description{Occurrence of mistakes obtained with the time-constraint strategy and the constraint (only) strategy 
on the CartPoleGC environment.}
\end{figure}

\subsection{Agent hyperparameters}

\begin{table}[th]
	\caption{Safety pretraining: TQC's hyperparameters}
	\label{tab:safety_pretraining_TQC}
	\begin{tabular}{rll}
    \toprule
		\textit{Name} & \textit{Value} \\ \midrule
		Actor learning rate & $3\times 10^{-4}$  \\
		Critic learning rate & $3\times 10^{-4}$ \\
		Temperature learning rate & $3\times 10^{-4}$ \\
    Initial temperature & $1.0$ \\
    $\tau$ & $5\times 10^{-3}$ \\
		No entropy backup & -  \\
		Discount factor & 0.99  \\ 
    Hidden layers & (256, 256) \\
    Number of critics & 5 \\
    Number of atoms per critic & 25 \\
    Number of quantiles to drop & 2 for CartPoleGC ; 0 for SkydioX2GC\\
    \bottomrule
	\end{tabular}
\end{table}

\begin{table}[th]
	\caption{Safety pretraining: Reachability critics' hyperparameters}
	\label{tab:safety_pretraining_RCRL}
	\begin{tabular}{rll}
    \toprule
		\textit{Name} & \textit{Value} \\ \midrule
		Critic learning rate & $3\times 10^{-4}$ \\
    $\tau$ & $5\times 10^{-3}$ \\
		Discount factor & 0.99  \\ 
    Hidden layers & (256, 256) \\
    Number of critics & 5 \\
    Number of atoms per critic & 25 \\
    \bottomrule
	\end{tabular}
\end{table}

\begin{table}[th]
	\caption{SAC and SAC-N hyperparameters for safe exploration}
	\label{tab:sac_sac_n}
	\begin{tabular}{rll}
    \toprule
		\textit{Name} & \textit{Value} \\ \midrule
		Actor learning rate & $3\times 10^{-4}$  \\
		Critic learning rate & $3\times 10^{-4}$ \\
		Temperature learning rate & $3\times 10^{-4}$ \\
    Initial temperature & $1.0$ \\
    $\tau$ & $5\times 10^{-3}$ \\
		No entropy backup & -  \\
		Discount factor & 0.99  \\ 
    Hidden layers & (256, 256) \\
    Number of critics (Specific to SAC-N) & 50 for CartPoleGC ; 10 for SkydioX2GC \\
    \bottomrule
	\end{tabular}
\end{table}

As for the buffer we choose to keep all transitions. There is no forgetting. 
Thus, the buffer size is always larger than the number of training steps.

%\end{multicols}

\end{document}

%%%%%%%%%%%%%%%%%%%%%%%%%%%%%%%%%%%%%%%%%%%%%%%%%%%%%%%%%%%%%%%%%%%%%%%%

