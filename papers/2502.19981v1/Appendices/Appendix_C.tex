\section{Example: H1 Failure on 4-Operand Addition}
\label{appendix:C}
Below is an example in which the heuristic \textbf{H1} fails in 4-operand addition, visualized in Figure \ref{fig:carry_4_op_fail}: 

\noindent\textbf{186 + 261 + 198 + 256.}
\[
\begin{split}
    t_{1} =8 + 6 + 9 + 5 = 28\\
    c_{2}^{h} \in \{ \left\lfloor \frac{c_{min} + 28}{10} \right\rfloor,\\
    \left\lfloor \frac{c_{max} + 28}{10} \right\rfloor \}
\end{split}
\]

with \(c_{max} = 3\)
\[c_{2}^{h} \in \{ \left\lfloor \frac{28}{10} \right\rfloor, \left\lfloor \frac{31}{10} \right\rfloor \} = \{2, 3\}\]
therefore $c_{2}^{h}$ is chosen uniformly at random between $2$ and $3$.
The heuristic thus fails in solving \textbf{186 + 261 + 198 + 256} with a chance of 50\%. 

\begin{figure}[ht]
    \centering
    \includegraphics[width=0.5\textwidth]{Images/figure9.png} 
    \caption{4-operand addition in which \textbf{H1} fails.} 
    \label{fig:carry_4_op_fail}
\end{figure}