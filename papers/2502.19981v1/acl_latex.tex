% This must be in the first 5 lines to tell arXiv to use pdfLaTeX, which is strongly recommended.
\pdfoutput=1
% In particular, the hyperref package requires pdfLaTeX in order to break URLs across lines.

\documentclass[11pt]{article}

% Change "review" to "final" to generate the final (sometimes called camera-ready) version.
% Change to "preprint" to generate a non-anonymous version with page numbers.
\usepackage[final]{acl}

% Standard package includes
\usepackage{times}
\usepackage{latexsym}

% For proper rendering and hyphenation of words containing Latin characters (including in bib files)
\usepackage[T1]{fontenc}
% For Vietnamese characters
% \usepackage[T5]{fontenc}
% See https://www.latex-project.org/help/documentation/encguide.pdf for other character sets

% This assumes your files are encoded as UTF8
\usepackage[utf8]{inputenc}

% This is not strictly necessary, and may be commented out,
% but it will improve the layout of the manuscript,
% and will typically save some space.
\usepackage{microtype}

% This is also not strictly necessary, and may be commented out.
% However, it will improve the aesthetics of text in
% the typewriter font.
\usepackage{inconsolata}

%Including images in your LaTeX document requires adding
%additional package(s)
\usepackage{graphicx}

% If the title and author information does not fit in the area allocated, uncomment the following
%
%\setlength\titlebox{<dim>}
%
% and set <dim> to something 5cm or larger.

\usepackage{subcaption}
\usepackage{mwe}

\usepackage{multirow}
\usepackage{microtype}
\usepackage{booktabs}

\usepackage{amsfonts}

\usepackage{float} % Add this in your preamble
\usepackage{amsmath}
\usepackage[colorinlistoftodos]{todonotes}
\usepackage{rotating}

\title{The Lookahead Limitation:\\ Why Multi-Operand Addition is Hard for LLMs}

\newcommand{\affilsup}[1]{\rlap{\textsuperscript{\normalfont#1}}}

\author{
    Tanja Baeumel\affilsup{1,2}
    \qquad
    Josef van Genabith\affilsup{1, 3}
    \qquad
    Simon Ostermann\affilsup{1,2}
    \\
    $^1$German Research Center for Artificial Intelligence (DFKI) \\
    $^2$Centre for European Research in Trusted AI (CERTAIN) \\
    $^3$Department of Language Science and Technology, Saarland University \\
    Saarland Informatics Campus, Saarbrücken, Germany\\
    \texttt{\{firstname.lastname\}@dfki.de} \\
}


\begin{document}
\maketitle
\begin{abstract}
Autoregressive large language models (LLMs) exhibit impressive performance across various tasks but struggle with simple arithmetic, such as additions of two or more operands. We show that this struggle arises from LLMs’ use of a \textbf{simple one-digit lookahead heuristic}, which works fairly well (but not perfect) for two-operand addition but fails in multi-operand cases, where the carry-over logic is more complex. Our probing experiments and digit-wise accuracy evaluation show that LLMs fail precisely where a one-digit lookahead is insufficient to account for cascading carries. We analyze the impact of tokenization strategies on arithmetic performance and show that all investigated models, regardless of tokenization, are inherently limited in the addition of multiple operands due to their reliance on a one-digit lookahead heuristic. Our findings reveal fundamental limitations that prevent LLMs from generalizing to more complex numerical reasoning.
\end{abstract}
\section{Introduction}

% \textcolor{red}{Still on working}

% \textcolor{red}{add label for each section}


Robot learning relies on diverse and high-quality data to learn complex behaviors \cite{aldaco2024aloha, wang2024dexcap}.
Recent studies highlight that models trained on datasets with greater complexity and variation in the domain tend to generalize more effectively across broader scenarios \cite{mann2020language, radford2021learning, gao2024efficient}.
% However, creating such diverse datasets in the real world presents significant challenges.
% Modifying physical environments and adjusting robot hardware settings require considerable time, effort, and financial resources.
% In contrast, simulation environments offer a flexible and efficient alternative.
% Simulations allow for the creation and modification of digital environments with a wide range of object shapes, weights, materials, lighting, textures, friction coefficients, and so on to incorporate domain randomization,
% which helps improve the robustness of models when deployed in real-world conditions.
% These environments can be easily adjusted and reset, enabling faster iterations and data collection.
% Additionally, simulations provide the ability to consistently reproduce scenarios, which is essential for benchmarking and model evaluation.
% Another advantage of simulations is their flexibility in sensor integration. Sensors such as cameras, LiDARs, and tactile sensors can be added or repositioned without the physical limitations present in real-world setups. Simulations also eliminate the risk of damaging expensive hardware during edge-case experiments, making them an ideal platform for testing rare or dangerous scenarios that are impractical to explore in real life.
By leveraging immersive perspectives and interactions, Extended Reality\footnote{Extended Reality is an umbrella term to refer to Augmented Reality, Mixed Reality, and Virtual Reality \cite{wikipediaExtendedReality}}
(XR)
is a promising candidate for efficient and intuitive large scale data collection \cite{jiang2024comprehensive, arcade}
% With the demand for collecting data, XR provides a promising approach for humans to teach robots by offering users an immersive experience.
in simulation \cite{jiang2024comprehensive, arcade, dexhub-park} and real-world scenarios \cite{openteach, opentelevision}.
However, reusing and reproducing current XR approaches for robot data collection for new settings and scenarios is complicated and requires significant effort.
% are difficult to reuse and reproduce system makes it hard to reuse and reproduce in another data collection pipeline.
This bottleneck arises from three main limitations of current XR data collection and interaction frameworks: \textit{asset limitation}, \textit{simulator limitation}, and \textit{device limitation}.
% \textcolor{red}{ASSIGN THESE CITATION PROPERLY:}
% \textcolor{red}{list them by time order???}
% of collecting data by using XR have three main limitations.
Current approaches suffering from \textit{asset limitation} \cite{arclfd, jiang2024comprehensive, arcade, george2025openvr, vicarios}
% Firstly, recent works \cite{jiang2024comprehensive, arcade, dexhub-park}
can only use predefined robot models and task scenes. Configuring new tasks requires significant effort, since each new object or model must be specifically integrated into the XR application.
% and it takes too much effort to configure new tasks in their systems since they cannot spawn arbitrary models in the XR application.
The vast majority of application are developed for specific simulators or real-world scenarios. This \textit{simulator limitation} \cite{mosbach2022accelerating, lipton2017baxter, dexhub-park, arcade}
% Secondly, existing systems are limited to a single simulation platform or real-world scenarios.
significantly reduces reusability and makes adaptation to new simulation platforms challenging.
Additionally, most current XR frameworks are designed for a specific version of a single XR headset, leading to a \textit{device limitation} 
\cite{lipton2017baxter, armada, openteach, meng2023virtual}.
% and there is no work working on the extendability of transferring to a new headsets as far as we know.
To the best of our knowledge, no existing work has explored the extensibility or transferability of their framework to different headsets.
These limitations hamper reproducibility and broader contributions of XR based data collection and interaction to the research community.
% as each research group typically has its own data collection pipeline.
% In addition to these main limitations, existing XR systems are not well suited for managing multiple robot systems,
% as they are often designed for single-operator use.

In addition to these main limitations, existing XR systems are often designed for single-operator use, prohibiting collaborative data collection.
At the same time, controlling multiple robots at once can be very difficult for a single operator,
making data collection in multi-robot scenarios particularly challenging \cite{orun2019effect}.
Although there are some works using collaborative data collection in the context of tele-operation \cite{tung2021learning, Qin2023AnyTeleopAG},
there is no XR-based data collection system supporting collaborative data collection.
This limitation highlights the need for more advanced XR solutions that can better support multi-robot and multi-user scenarios.
% \textcolor{red}{more papers about collaborative data collection}

To address all of these issues, we propose \textbf{IRIS},
an \textbf{I}mmersive \textbf{R}obot \textbf{I}nteraction \textbf{S}ystem.
This general system supports various simulators, benchmarks and real-world scenarios.
It is easily extensible to new simulators and XR headsets.
IRIS achieves generalization across six dimensions:
% \begin{itemize}
%     \item \textit{Cross-scene} : diverse object models;
%     \item \textit{Cross-embodiment}: diverse robot models;
%     \item \textit{Cross-simulator}: 
%     \item \textit{Cross-reality}: fd
%     \item \textit{Cross-platform}: fd
%     \item \textit{Cross-users}: fd
% \end{itemize}
\textbf{Cross-Scene}, \textbf{Cross-Embodiment}, \textbf{Cross-Simulator}, \textbf{Cross-Reality}, \textbf{Cross-Platform}, and \textbf{Cross-User}.

\textbf{Cross-Scene} and \textbf{Cross-Embodiment} allow the system to handle arbitrary objects and robots in the simulation,
eliminating restrictions about predefined models in XR applications.
IRIS achieves these generalizations by introducing a unified scene specification, representing all objects,
including robots, as data structures with meshes, materials, and textures.
The unified scene specification is transmitted to the XR application to create and visualize an identical scene.
By treating robots as standard objects, the system simplifies XR integration,
allowing researchers to work with various robots without special robot-specific configurations.
\textbf{Cross-Simulator} ensures compatibility with various simulation engines.
IRIS simplifies adaptation by parsing simulated scenes into the unified scene specification, eliminating the need for XR application modifications when switching simulators.
New simulators can be integrated by creating a parser to convert their scenes into the unified format.
This flexibility is demonstrated by IRIS’ support for Mujoco \cite{todorov2012mujoco}, IsaacSim \cite{mittal2023orbit}, CoppeliaSim \cite{coppeliaSim}, and even the recent Genesis \cite{Genesis} simulator.
\textbf{Cross-Reality} enables the system to function seamlessly in both virtual simulations and real-world applications.
IRIS enables real-world data collection through camera-based point cloud visualization.
\textbf{Cross-Platform} allows for compatibility across various XR devices.
Since XR device APIs differ significantly, making a single codebase impractical, IRIS XR application decouples its modules to maximize code reuse.
This application, developed by Unity \cite{unity3dUnityManual}, separates scene visualization and interaction, allowing developers to integrate new headsets by reusing the visualization code and only implementing input handling for hand, head, and motion controller tracking.
IRIS provides an implementation of the XR application in the Unity framework, allowing for a straightforward deployment to any device that supports Unity. 
So far, IRIS was successfully deployed to the Meta Quest 3 and HoloLens 2.
Finally, the \textbf{Cross-User} ability allows multiple users to interact within a shared scene.
IRIS achieves this ability by introducing a protocol to establish the communication between multiple XR headsets and the simulation or real-world scenarios.
Additionally, IRIS leverages spatial anchors to support the alignment of virtual scenes from all deployed XR headsets.
% To make an seamless user experience for robot learning data collection,
% IRIS also tested in three different robot control interface
% Furthermore, to demonstrate the extensibility of our approach, we have implemented a robot-world pipeline for real robot data collection, ensuring that the system can be used in both simulated and real-world environments.
The Immersive Robot Interaction System makes the following contributions\\
\textbf{(1) A unified scene specification} that is compatible with multiple robot simulators. It enables various XR headsets to visualize and interact with simulated objects and robots, providing an immersive experience while ensuring straightforward reusability and reproducibility.\\
\textbf{(2) A collaborative data collection framework} designed for XR environments. The framework facilitates enhanced robot data acquisition.\\
\textbf{(3) A user study} demonstrating that IRIS significantly improves data collection efficiency and intuitiveness compared to the LIBERO baseline.

% \begin{table*}[t]
%     \centering
%     \begin{tabular}{lccccccc}
%         \toprule
%         & \makecell{Physical\\Interaction}
%         & \makecell{XR\\Enabled}
%         & \makecell{Free\\View}
%         & \makecell{Multiple\\Robots}
%         & \makecell{Robot\\Control}
%         % Force Feedback???
%         & \makecell{Soft Object\\Supported}
%         & \makecell{Collaborative\\Data} \\
%         \midrule
%         ARC-LfD \cite{arclfd}                              & Real        & \cmark & \xmark & \xmark & Joint              & \xmark & \xmark \\
%         DART \cite{dexhub-park}                            & Sim         & \cmark & \cmark & \cmark & Cartesian          & \xmark & \xmark \\
%         \citet{jiang2024comprehensive}                     & Sim         & \cmark & \xmark & \xmark & Joint \& Cartesian & \xmark & \xmark \\
%         \citet{mosbach2022accelerating}                    & Sim         & \cmark & \cmark & \xmark & Cartesian          & \xmark & \xmark \\
%         ARCADE \cite{arcade}                               & Real        & \cmark & \cmark & \xmark & Cartesian          & \xmark & \xmark \\
%         Holo-Dex \cite{holodex}                            & Real        & \cmark & \xmark & \cmark & Cartesian          & \cmark & \xmark \\
%         ARMADA \cite{armada}                               & Real        & \cmark & \xmark & \cmark & Cartesian          & \cmark & \xmark \\
%         Open-TeleVision \cite{opentelevision}              & Real        & \cmark & \cmark & \cmark & Cartesian          & \cmark & \xmark \\
%         OPEN TEACH \cite{openteach}                        & Real        & \cmark & \xmark & \cmark & Cartesian          & \cmark & \cmark \\
%         GELLO \cite{wu2023gello}                           & Real        & \xmark & \cmark & \cmark & Joint              & \cmark & \xmark \\
%         DexCap \cite{wang2024dexcap}                       & Real        & \xmark & \cmark & \xmark & Cartesian          & \cmark & \xmark \\
%         AnyTeleop \cite{Qin2023AnyTeleopAG}                & Real        & \xmark & \xmark & \cmark & Cartesian          & \cmark & \cmark \\
%         Vicarios \cite{vicarios}                           & Real        & \cmark & \xmark & \xmark & Cartesian          & \cmark & \xmark \\     
%         Augmented Visual Cues \cite{augmentedvisualcues}   & Real        & \cmark & \cmark & \xmark & Cartesian          & \xmark & \xmark \\ 
%         \citet{wang2024robotic}                            & Real        & \cmark & \cmark & \xmark & Cartesian          & \cmark & \xmark \\
%         Bunny-VisionPro \cite{bunnyvisionpro}              & Real        & \cmark & \cmark & \cmark & Cartesian          & \cmark & \xmark \\
%         IMMERTWIN \cite{immertwin}                         & Real        & \cmark & \cmark & \cmark & Cartesian          & \xmark & \xmark \\
%         \citet{meng2023virtual}                            & Sim \& Real & \cmark & \cmark & \xmark & Cartesian          & \xmark & \xmark \\
%         Shared Control Framework \cite{sharedctlframework} & Real        & \cmark & \cmark & \cmark & Cartesian          & \xmark & \xmark \\
%         OpenVR \cite{openvr}                               & Real        & \cmark & \cmark & \xmark & Cartesian          & \xmark & \xmark \\
%         \citet{digitaltwinmr}                              & Real        & \cmark & \cmark & \xmark & Cartesian          & \cmark & \xmark \\
        
%         \midrule
%         \textbf{Ours} & Sim \& Real & \cmark & \cmark & \cmark & Joint \& Cartesian  & \cmark & \cmark \\
%         \bottomrule
%     \end{tabular}
%     \caption{This is a cross-column table with automatic line breaking.}
%     \label{tab:cross-column}
% \end{table*}

% \begin{table*}[t]
%     \centering
%     \begin{tabular}{lccccccc}
%         \toprule
%         & \makecell{Cross-Embodiment}
%         & \makecell{Cross-Scene}
%         & \makecell{Cross-Simulator}
%         & \makecell{Cross-Reality}
%         & \makecell{Cross-Platform}
%         & \makecell{Cross-User} \\
%         \midrule
%         ARC-LfD \cite{arclfd}                              & \xmark & \xmark & \xmark & \xmark & \xmark & \xmark \\
%         DART \cite{dexhub-park}                            & \cmark & \cmark & \xmark & \xmark & \xmark & \xmark \\
%         \citet{jiang2024comprehensive}                     & \xmark & \cmark & \xmark & \xmark & \xmark & \xmark \\
%         \citet{mosbach2022accelerating}                    & \xmark & \cmark & \xmark & \xmark & \xmark & \xmark \\
%         ARCADE \cite{arcade}                               & \xmark & \xmark & \xmark & \xmark & \xmark & \xmark \\
%         Holo-Dex \cite{holodex}                            & \cmark & \xmark & \xmark & \xmark & \xmark & \xmark \\
%         ARMADA \cite{armada}                               & \cmark & \xmark & \xmark & \xmark & \xmark & \xmark \\
%         Open-TeleVision \cite{opentelevision}              & \cmark & \xmark & \xmark & \xmark & \cmark & \xmark \\
%         OPEN TEACH \cite{openteach}                        & \cmark & \xmark & \xmark & \xmark & \xmark & \cmark \\
%         GELLO \cite{wu2023gello}                           & \cmark & \xmark & \xmark & \xmark & \xmark & \xmark \\
%         DexCap \cite{wang2024dexcap}                       & \xmark & \xmark & \xmark & \xmark & \xmark & \xmark \\
%         AnyTeleop \cite{Qin2023AnyTeleopAG}                & \cmark & \cmark & \cmark & \cmark & \xmark & \cmark \\
%         Vicarios \cite{vicarios}                           & \xmark & \xmark & \xmark & \xmark & \xmark & \xmark \\     
%         Augmented Visual Cues \cite{augmentedvisualcues}   & \xmark & \xmark & \xmark & \xmark & \xmark & \xmark \\ 
%         \citet{wang2024robotic}                            & \xmark & \xmark & \xmark & \xmark & \xmark & \xmark \\
%         Bunny-VisionPro \cite{bunnyvisionpro}              & \cmark & \xmark & \xmark & \xmark & \xmark & \xmark \\
%         IMMERTWIN \cite{immertwin}                         & \cmark & \xmark & \xmark & \xmark & \xmark & \xmark \\
%         \citet{meng2023virtual}                            & \xmark & \cmark & \xmark & \cmark & \xmark & \xmark \\
%         \citet{sharedctlframework}                         & \cmark & \xmark & \xmark & \xmark & \xmark & \xmark \\
%         OpenVR \cite{george2025openvr}                               & \xmark & \xmark & \xmark & \xmark & \xmark & \xmark \\
%         \citet{digitaltwinmr}                              & \xmark & \xmark & \xmark & \xmark & \xmark & \xmark \\
        
%         \midrule
%         \textbf{Ours} & \cmark & \cmark & \cmark & \cmark & \cmark & \cmark \\
%         \bottomrule
%     \end{tabular}
%     \caption{This is a cross-column table with automatic line breaking.}
% \end{table*}

% \begin{table*}[t]
%     \centering
%     \begin{tabular}{lccccccc}
%         \toprule
%         & \makecell{Cross-Scene}
%         & \makecell{Cross-Embodiment}
%         & \makecell{Cross-Simulator}
%         & \makecell{Cross-Reality}
%         & \makecell{Cross-Platform}
%         & \makecell{Cross-User}
%         & \makecell{Control Space} \\
%         \midrule
%         % Vicarios \cite{vicarios}                           & \xmark & \xmark & \xmark & \xmark & \xmark & \xmark \\     
%         % Augmented Visual Cues \cite{augmentedvisualcues}   & \xmark & \xmark & \xmark & \xmark & \xmark & \xmark \\ 
%         % OpenVR \cite{george2025openvr}                     & \xmark & \xmark & \xmark & \xmark & \xmark & \xmark \\
%         \citet{digitaltwinmr}                              & \xmark & \xmark & \xmark & \xmark & \xmark & \xmark &  \\
%         ARC-LfD \cite{arclfd}                              & \xmark & \xmark & \xmark & \xmark & \xmark & \xmark &  \\
%         \citet{sharedctlframework}                         & \cmark & \xmark & \xmark & \xmark & \xmark & \xmark &  \\
%         \citet{jiang2024comprehensive}                     & \cmark & \xmark & \xmark & \xmark & \xmark & \xmark &  \\
%         \citet{mosbach2022accelerating}                    & \cmark & \xmark & \xmark & \xmark & \xmark & \xmark & \\
%         Holo-Dex \cite{holodex}                            & \cmark & \xmark & \xmark & \xmark & \xmark & \xmark & \\
%         ARCADE \cite{arcade}                               & \cmark & \cmark & \xmark & \xmark & \xmark & \xmark & \\
%         DART \cite{dexhub-park}                            & Limited & Limited & Mujoco & Sim & Vision Pro & \xmark &  Cartesian\\
%         ARMADA \cite{armada}                               & \cmark & \cmark & \xmark & \xmark & \xmark & \xmark & \\
%         \citet{meng2023virtual}                            & \cmark & \cmark & \xmark & \cmark & \xmark & \xmark & \\
%         % GELLO \cite{wu2023gello}                           & \cmark & \xmark & \xmark & \xmark & \xmark & \xmark \\
%         % DexCap \cite{wang2024dexcap}                       & \xmark & \xmark & \xmark & \xmark & \xmark & \xmark \\
%         % AnyTeleop \cite{Qin2023AnyTeleopAG}                & \cmark & \cmark & \cmark & \cmark & \xmark & \cmark \\
%         % \citet{wang2024robotic}                            & \xmark & \xmark & \xmark & \xmark & \xmark & \xmark \\
%         Bunny-VisionPro \cite{bunnyvisionpro}              & \cmark & \cmark & \xmark & \xmark & \xmark & \xmark & \\
%         IMMERTWIN \cite{immertwin}                         & \cmark & \cmark & \xmark & \xmark & \xmark & \xmark & \\
%         Open-TeleVision \cite{opentelevision}              & \cmark & \cmark & \xmark & \xmark & \cmark & \xmark & \\
%         \citet{szczurek2023multimodal}                     & \xmark & \xmark & \xmark & Real & \xmark & \cmark & \\
%         OPEN TEACH \cite{openteach}                        & \cmark & \cmark & \xmark & \xmark & \xmark & \cmark & \\
%         \midrule
%         \textbf{Ours} & \cmark & \cmark & \cmark & \cmark & \cmark & \cmark \\
%         \bottomrule
%     \end{tabular}
%     \caption{TODO, Bruce: this table can be further optimized.}
% \end{table*}

\definecolor{goodgreen}{HTML}{228833}
\definecolor{goodred}{HTML}{EE6677}
\definecolor{goodgray}{HTML}{BBBBBB}

\begin{table*}[t]
    \centering
    \begin{adjustbox}{max width=\textwidth}
    \renewcommand{\arraystretch}{1.2}    
    \begin{tabular}{lccccccc}
        \toprule
        & \makecell{Cross-Scene}
        & \makecell{Cross-Embodiment}
        & \makecell{Cross-Simulator}
        & \makecell{Cross-Reality}
        & \makecell{Cross-Platform}
        & \makecell{Cross-User}
        & \makecell{Control Space} \\
        \midrule
        % Vicarios \cite{vicarios}                           & \xmark & \xmark & \xmark & \xmark & \xmark & \xmark \\     
        % Augmented Visual Cues \cite{augmentedvisualcues}   & \xmark & \xmark & \xmark & \xmark & \xmark & \xmark \\ 
        % OpenVR \cite{george2025openvr}                     & \xmark & \xmark & \xmark & \xmark & \xmark & \xmark \\
        \citet{digitaltwinmr}                              & \textcolor{goodred}{Limited}     & \textcolor{goodred}{Single Robot} & \textcolor{goodred}{Unity}    & \textcolor{goodred}{Real}          & \textcolor{goodred}{Meta Quest 2} & \textcolor{goodgray}{N/A} & \textcolor{goodred}{Cartesian} \\
        ARC-LfD \cite{arclfd}                              & \textcolor{goodgray}{N/A}        & \textcolor{goodred}{Single Robot} & \textcolor{goodgray}{N/A}     & \textcolor{goodred}{Real}          & \textcolor{goodred}{HoloLens}     & \textcolor{goodgray}{N/A} & \textcolor{goodred}{Cartesian} \\
        \citet{sharedctlframework}                         & \textcolor{goodred}{Limited}     & \textcolor{goodred}{Single Robot} & \textcolor{goodgray}{N/A}     & \textcolor{goodred}{Real}          & \textcolor{goodred}{HTC Vive Pro} & \textcolor{goodgray}{N/A} & \textcolor{goodred}{Cartesian} \\
        \citet{jiang2024comprehensive}                     & \textcolor{goodred}{Limited}     & \textcolor{goodred}{Single Robot} & \textcolor{goodgray}{N/A}     & \textcolor{goodred}{Real}          & \textcolor{goodred}{HoloLens 2}   & \textcolor{goodgray}{N/A} & \textcolor{goodgreen}{Joint \& Cartesian} \\
        \citet{mosbach2022accelerating}                    & \textcolor{goodgreen}{Available} & \textcolor{goodred}{Single Robot} & \textcolor{goodred}{IsaacGym} & \textcolor{goodred}{Sim}           & \textcolor{goodred}{Vive}         & \textcolor{goodgray}{N/A} & \textcolor{goodgreen}{Joint \& Cartesian} \\
        Holo-Dex \cite{holodex}                            & \textcolor{goodgray}{N/A}        & \textcolor{goodred}{Single Robot} & \textcolor{goodgray}{N/A}     & \textcolor{goodred}{Real}          & \textcolor{goodred}{Meta Quest 2} & \textcolor{goodgray}{N/A} & \textcolor{goodred}{Joint} \\
        ARCADE \cite{arcade}                               & \textcolor{goodgray}{N/A}        & \textcolor{goodred}{Single Robot} & \textcolor{goodgray}{N/A}     & \textcolor{goodred}{Real}          & \textcolor{goodred}{HoloLens 2}   & \textcolor{goodgray}{N/A} & \textcolor{goodred}{Cartesian} \\
        DART \cite{dexhub-park}                            & \textcolor{goodred}{Limited}     & \textcolor{goodred}{Limited}      & \textcolor{goodred}{Mujoco}   & \textcolor{goodred}{Sim}           & \textcolor{goodred}{Vision Pro}   & \textcolor{goodgray}{N/A} & \textcolor{goodred}{Cartesian} \\
        ARMADA \cite{armada}                               & \textcolor{goodgray}{N/A}        & \textcolor{goodred}{Limited}      & \textcolor{goodgray}{N/A}     & \textcolor{goodred}{Real}          & \textcolor{goodred}{Vision Pro}   & \textcolor{goodgray}{N/A} & \textcolor{goodred}{Cartesian} \\
        \citet{meng2023virtual}                            & \textcolor{goodred}{Limited}     & \textcolor{goodred}{Single Robot} & \textcolor{goodred}{PhysX}   & \textcolor{goodgreen}{Sim \& Real} & \textcolor{goodred}{HoloLens 2}   & \textcolor{goodgray}{N/A} & \textcolor{goodred}{Cartesian} \\
        % GELLO \cite{wu2023gello}                           & \cmark & \xmark & \xmark & \xmark & \xmark & \xmark \\
        % DexCap \cite{wang2024dexcap}                       & \xmark & \xmark & \xmark & \xmark & \xmark & \xmark \\
        % AnyTeleop \cite{Qin2023AnyTeleopAG}                & \cmark & \cmark & \cmark & \cmark & \xmark & \cmark \\
        % \citet{wang2024robotic}                            & \xmark & \xmark & \xmark & \xmark & \xmark & \xmark \\
        Bunny-VisionPro \cite{bunnyvisionpro}              & \textcolor{goodgray}{N/A}        & \textcolor{goodred}{Single Robot} & \textcolor{goodgray}{N/A}     & \textcolor{goodred}{Real}          & \textcolor{goodred}{Vision Pro}   & \textcolor{goodgray}{N/A} & \textcolor{goodred}{Cartesian} \\
        IMMERTWIN \cite{immertwin}                         & \textcolor{goodgray}{N/A}        & \textcolor{goodred}{Limited}      & \textcolor{goodgray}{N/A}     & \textcolor{goodred}{Real}          & \textcolor{goodred}{HTC Vive}     & \textcolor{goodgray}{N/A} & \textcolor{goodred}{Cartesian} \\
        Open-TeleVision \cite{opentelevision}              & \textcolor{goodgray}{N/A}        & \textcolor{goodred}{Limited}      & \textcolor{goodgray}{N/A}     & \textcolor{goodred}{Real}          & \textcolor{goodgreen}{Meta Quest, Vision Pro} & \textcolor{goodgray}{N/A} & \textcolor{goodred}{Cartesian} \\
        \citet{szczurek2023multimodal}                     & \textcolor{goodgray}{N/A}        & \textcolor{goodred}{Limited}      & \textcolor{goodgray}{N/A}     & \textcolor{goodred}{Real}          & \textcolor{goodred}{HoloLens 2}   & \textcolor{goodgreen}{Available} & \textcolor{goodred}{Joint \& Cartesian} \\
        OPEN TEACH \cite{openteach}                        & \textcolor{goodgray}{N/A}        & \textcolor{goodgreen}{Available}  & \textcolor{goodgray}{N/A}     & \textcolor{goodred}{Real}          & \textcolor{goodred}{Meta Quest 3} & \textcolor{goodred}{N/A} & \textcolor{goodgreen}{Joint \& Cartesian} \\
        \midrule
        \textbf{Ours}                                      & \textcolor{goodgreen}{Available} & \textcolor{goodgreen}{Available}  & \textcolor{goodgreen}{Mujoco, CoppeliaSim, IsaacSim} & \textcolor{goodgreen}{Sim \& Real} & \textcolor{goodgreen}{Meta Quest 3, HoloLens 2} & \textcolor{goodgreen}{Available} & \textcolor{goodgreen}{Joint \& Cartesian} \\
        \bottomrule
        \end{tabular}
    \end{adjustbox}
    \caption{Comparison of XR-based system for robots. IRIS is compared with related works in different dimensions.}
\end{table*}


\section{LLMs Struggle with Multi-Operand Addition}
\label{sec:acc_data}

In this section, we define the data and models used in this work and demonstrate that LLMs fail on multi-operand additions by looking at prediction accuracy.

\subsection{Models and Data}
\label{sec:models_data}
\paragraph{Models.}
We compare Mistral-7B \cite{jiang_mistral_2023}, Gemma-7B \cite{gemmateam2024gemmaopenmodelsbased} and Meta-Llama-3-8B \cite{grattafiori2024llama3herdmodels, llama3modelcard} as they employ different tokenization strategies for numerical outputs: 
While Mistral and Gemma exclusively employ a single-digit tokenization strategy for their numeric input and generated output (e.g., input = ['1', '4', '7', '+', '2', '5', '5', '='], output = ['4', '0', '2']), Llama-3 employs a multi-digit numeric tokenization strategy (e.g., input = [' 147', ' +', ' 255', ' ='], output = [' 402']), typically favoring numeric tokens of length 3. 
\paragraph{Data.}
For all experiments in this paper, we compile a range of datasets containing simple arithmetic task prompts of the form \textit{147 + 255 = }. We create a dataset for each addition task ranging from 2-operand to 11-operand addition, where each operand is a triple-digit number between 100 and 899. Each of the 10 datasets contains 5,000 unique arithmetic problems, both in a zero-shot and one-shot setting. In the zero-shot setting, an example for a 2-operand addition prompt is ``147 + 255 = ''. An example for a 4-operand addition prompt is ``251 + 613 + 392 + 137 = ''.  Our one-shot prompt template follows the scheme \textit{q1 r1; q2 }, e.g.~``359 + 276 = 635; 147 + 255 = '', where \textit{q1} is a sample query from the same dataset and \textit{r1} is the correct result of the addition task in \textit{q1}. \textit{q2} is the query containing the addition task to be solved.

In the remainder of the paper, we use $s_n$ (with $n\geq 0$) to denote the result digit generated at digit position \(10^n\).
For example, in ``147 + 255 ='', with expected output 402, $s_2 = 4$, $s_1 = 0$, and $s_0 = 2$. 

\subsection{LLM Accuracy on Addition Tasks}
Figure \ref{fig:multi_op_accuracy_overall} illustrates the significant decline in performance of Mistral-7B \cite{jiang_mistral_2023}, Gemma-7B \cite{gemmateam2024gemmaopenmodelsbased} and Meta-Llama-3-8B \cite{llama3modelcard} in multi-operand addition as the number of operands increases. This drastic decrease highlights the inability of these models to generalize effectively to addition tasks involving a higher number of operands, despite their strong overall capabilities. 

\begin{figure}[t]
    \centering
    \includegraphics[width=0.42\textwidth]{Images/figure2.png}
    \caption{Accuracy of Mistral, Gemma and Llama-3 on multi-operand addition of triple-digit numbers, in a zero- and one-shot setting.}
    \label{fig:multi_op_accuracy_overall}
\end{figure}
\section{Probing LLMs on Digits in Two-Operand Addition Tasks}
\label{sec:probing}

Solving arithmetic tasks presents a fundamental challenge for LLMs, as they generate text from left to right, while addition requires a right-to-left process due to carry propagation from the least significant to the most significant digit.
For instance, predicting the first result digit $s_2 = 4$ in ``147 + 255 = '' requires the model to anticipate that a carry originating from $s_0$ cascades through $s_1$ to $s_2$. Robust left-to-right addition thus requires a lookahead spanning all result digits, raising the question: Do LLMs internally represent future result digits when predicting $s_2$ - and if so, how far can they ``look into the future''?

To answer this question, we probe whether models accurately encode future result digits $s_1$ or $s_0$ while generating $s_2$. Building on \citet{levy2024language}, who show that, irrespective of a model's numeric tokenization strategy, LLMs internally represent numbers digit-by-digit in base 10, we analyze digit-wise probing accuracy on the two-operand addition dataset described in Section \ref{sec:models_data}.

\subsection{Methodology and Experiments} 
\paragraph{Data.}
We split the two-operand addition dataset (see Section \ref{sec:models_data}) into train (n=4500) and test (n=500) for the probing experiments. The two-operand addition dataset is designed such that correct results for the addition tasks are triple-digit numbers between 200 and 999. We use the zero-shot prompt setting for the probing experiment.

\paragraph{Probing Setup.}
Our goal is to determine which result digits are available at the prediction step of $s_2$. We thus train probes to predict the result digits $s_2$, $s_1$, and $s_0$ from hidden states of the model during the prediction step of $s_2$. 

Specifically, we train one-layer linear probes to predict individual digit values of the results from the hidden state of the last token at each model layer.  Probes are trained on the train split of the two-operand addition dataset and evaluated on the test split. 
We train separate probes to predict individual result digits $s_2$, $s_1$, and $s_0$, for all models at all layers.\footnote{We choose a low temperature of 0.1 during model inference to ensure deterministic and consistent outputs, reducing randomness in token generation and improving the reliability of numerical calculations.}

\begin{figure}[t]
    \centering
    \includegraphics[width=0.5\textwidth]{Images/figure3.png}
    \caption{Probing accuracy of individual result digits as predicted by the hidden states of Mistral, Gemma and Llama-3. For two-operand, zero-shot addition prompts.}
    \label{fig:probing_multi_op_accuracy_overall}
\end{figure}

\subsection{Results}
\label{subsec:results}
The probing accuracy of individual result digits is shown in Figure \ref{fig:probing_multi_op_accuracy_overall}. Gemma and Mistral with their digit-wise tokenization internally represent only $s_2$ with high accuracy. In contrast, there is a high probing accuracy across \textit{all} result digits in Llama-3. This is due to the fact that Llama-3 tokenizes numbers into 3-digit numeric tokens: It is forced by its tokenization to generate all result digits ($s_2$, $s_1$, and $s_0$) in one step as a single token.

The single-digit tokenization models Mistral and Gemma exhibit a low probing accuracy on $s_0$ ($< 0.24$) in all layers.
Recall that $s_0$ is probed from the models' hidden states while they autoregressively generate $s_2$. 
We interpret the lack of internal representation of $s_0$ as evidence that these models disregard the potential influence of $s_0$ (including any cascading carry) when generating $s_2$.

In line with this, Gemma and Mistral show notably higher probing accuracy on $s_1$ compared to $s_0$, when probing from the models' hidden states as they generate $s_2$. 
We thus conjecture that the single-digit-token models seem to recognize the potential influence of the carry resulting from the sum of the \(10^1\) operand digits. Simply put, generating the digit at $10^2$ might employ a lookahead of one digit to the \(10^1\) intermediate result. 
Based on this observation, we formulate a hypothesis for a heuristic used by LLMs:
\begin{center}
    \textbf{H1: \indent LLMs employ a look ahead of one digit to generate the current digit of an addition task.}
\end{center}

\textbf{H1} would explain why LLMs cannot effectively represent each necessary digit of the result during generation, making it difficult to anticipate later carry values correctly. We first formalize \textbf{H1}, which explains the patterns observed in Figure \ref{fig:probing_multi_op_accuracy_overall}, in the next Section, and then verify the fit of \textbf{H1} with empirical addition outcomes generated by the models in Sections \ref{sec:h1_2op}, \ref{sec:multi_fail}, and \ref{sec:llama}.

\section{The Carry Heuristic of LLMs}
\label{subsec:digit10}

Since LLMs generate numbers from left to right, they must anticipate whether a carry from later digits (with lower bases further on in the result) will impact the current digit they are generating. In this section, we evaluate the maximum accuracy LLMs can achieve in addition tasks, assuming they rely on \textbf{H1}, given the limited lookahead of one digit.

\subsection{Formalization of Left-to-Right Addition in Base 10}
We first formalize a recursive algorithm for solving addition of $k$ operands-where each operand is a base 10 integer- in a left-to-right manner. 

\noindent \textbf{We define:}
\begin{itemize}
    \setlength{\itemsep}{0.2pt}  
    \setlength{\parskip}{0.2pt} 
    \item \( k \): Number of operands.
    \item \( n_1, n_2, \dots, n_k \): Operands, each represented as digit sequences in base \( 10 \), with \(\quad 0 \leq i < d\), where $d$ is the number of digits in the operands: \( n_j = [n_{j, d-1}, \dots, n_{j, 0}], \quad n_{j, i} \in \{0, \dots, 9\}\)
    \item \(S\): The result of the addition. \( S = [s_d, s_{d-1}, \dots s_0] \), where 
    \(s_d = c_d\), i.e., the final carry.
\end{itemize}

\noindent We recursively define the calculation of individual result digits:
\begin{itemize}
    \item \textbf{Total Sum at Digit Position \( i \):}
    \[t_{i} =\sum_{j=1}^k n_{j, i}\]
    \[ T_i = t_i + c_i\]
    where \( t_i \) is the digit sum at the current position, \( c_i \) the carry from the previous digit position, and $k$ the number of operands. Base case: \(c_0 = 0\), no carry at the least significant digit.
    \item \textbf{Result Digit at Position \( i \):}
    \[
    s_i = T_i \mod 10
    \]
    \item \textbf{Carry to the Next Digit Position:}
    \[
    c_{i+1} = \left\lfloor \frac{T_i}{10} \right\rfloor
    \]
\end{itemize}

A worked example is provided in Appendix \ref{appendix:A}.

\subsection{A Naive Heuristic for Solving Addition Left-to-Right}
Due to the recursive nature of left-to-right addition, a lookahead of \(i-1\) digits is needed to determine any result digit $s_i$. 
There is however a simple, non-recursive heuristic for the estimation of $s_i$ with only a one-digit lookahead, to the digit sum of the next position, i.e. only considering $t_{i-1}$. 

We define $c_{min}$ and $c_{max}$ to be the minimal and maximal possible value for a carry, where trivially for all cases, $c_{min}=0$, and 
\[c_{max}(k) = \left\lfloor \frac{\sum_{j=1}^k 9}{10} \right\rfloor\] 
in base $10$ and for $k$ operands. 
We then define the carry heuristic $c_{i}^{h}$ as follows: 
\[c_{i}^{h} \in \{ \left\lfloor \frac{t_{i-1} + c_{min}}{10} \right\rfloor, \left\lfloor \frac{t_{i-1} + c_{max}}{10} \right\rfloor \} \]  
Where $c_{i}^{h}$ is chosen uniformly at random. 
We then accordingly define the predicted total sum at digit position i
\[T_i^h = t_i + c_i^h\] 

and the predicted result digit

\[s_i^h = T_i^h \mod 10\]

\paragraph{Examples.}
We show two examples of two-operand addition, one in which \textbf{H1} is successful, and one in which it fails.
For $k=2$, i.e., in two-operand addition:
\[c_{max}(2) = \left\lfloor \frac{\sum_{j=1}^2 9}{10} \right\rfloor = 1\] 

\paragraph{147 + 293.} See Figure \ref{fig:carry_2_op_success}. We need $T_2^h$ and thus $c_2^h$ to generate the first result digit $s_2^h$. 
\[c_{2}^{h} \in \{\left\lfloor \frac{4 + 9 + c_{min}}{10} \right\rfloor, \left\lfloor \frac{4 + 9 + c_{max}}{10} \right\rfloor \}\]
\[=\{ \left\lfloor \frac{13}{10} \right\rfloor, \left\lfloor \frac{14}{10} \right\rfloor \} = \{1, 1\}\]
therefore $c_{2}^{h} = 1$, $T_2^h = 4$, and $s_2^h = 4$. \textbf{H1} succeeds in predicting the first digit $s_2$ for \textbf{147 + 293}. 

\begin{figure}[t]
    \centering
    \includegraphics[width=0.372\textwidth]{Images/figure4.png} 
    \caption{Two-operand addition in which \textbf{H1} is successful.} 
    \label{fig:carry_2_op_success}
    \vspace{-0.2cm}
\end{figure}

\paragraph{147 + 255.} See Figure \ref{fig:carry_2_op_fail}. \\
\[c_{2}^{h} \in \{ \left\lfloor \frac{4 + 5 + c_{min}}{10} \right\rfloor, \left\lfloor \frac{4 + 5 + c_{max}}{10} \right\rfloor \}\]
\[= \{ \left\lfloor \frac{9}{10} \right\rfloor, \left\lfloor \frac{10}{10} \right\rfloor \} = \{0, 1\}\]
therefore $c_{2}^{h}$ is chosen uniformly at random between $0$ and $1$.
The heuristic fails in predicting the first digit $s_2$ for \textbf{147 + 255} with a 50\% chance. 
\section{H1 Predicts Difficulties of LLMs in Two-Operand Addition}
\label{sec:h1_2op}
In this section we show that single-digit token LLMs struggle exactly in those cases in which the heuristic \textbf{H1} is insufficient. 

\subsection{Predicted Accuracy} For two-operand addition, there are 19 possible values for each $t_i$ (ranging from 0 to 18, because this is the range of sums between two digits). In 18 out of these 19 cases, \textbf{H1} reliably determines the correct carry value. Only if $t_i = 9$, \textbf{H1} must randomly choose between two possible carry values, thus failing with a 50\% chance. This results in an overall predicted accuracy of
\[\frac{18\times1.0 + 1 \times 0.5}{19} = 0.974\]
for the first result digit $s_2$ in two-operand addition: \textbf{H1} achieves 97.4\% accuracy in correctly predicting the first result digit $s_2$. This corresponds almost exactly to Gemma's and Mistral's accuracies for generating $s_2$ during zero-shot and one-shot inference (Gemma: 0-shot: $97.12\%$, 1-shot: $98.04\%$; Mistral: 0-shot: $94.60\%$, 1-shot: $97.46\%$). Table \ref{tab:gen_accuracy_all} in Appendix \ref{sec:appendix_F} provides all generation accuracies for the data described in Section \ref{sec:models_data}.

\begin{figure}[t]
    \centering
    \includegraphics[width=0.37\textwidth]{Images/figure5.png} 
    \caption{Two-operand addition in which \textbf{H1} fails.} 
    \label{fig:carry_2_op_fail}
    \vspace{-0.2cm}
\end{figure}

\subsection{Finegrained Analysis} 
We further investigate whether it is true that especially cases with $t_i=9$ are challenging for LLMs. 

\paragraph{Data.} To this end, we evaluate prediction accuracy across five distinct newly introduced datasets, each containing 100 queries with distinct carry scenarios. The datasets follow the zero-shot template described in Section \ref{sec:models_data} and are designed to exhaustively capture all cases of carries affecting $s_2$ in two-operand addition of triple-digit numbers. 
\begin{itemize}
    \setlength{\itemsep}{0.1pt}
    \setlength{\parskip}{0pt}
    \setlength{\parsep}{0.5pt}
    \item \textbf{Dataset 1 (DS1): No carry.} The addition does not produce any carry (e.g., \(231 + 124 = 355\)).\footnote{We employ the additional constraint that the sum of the \(10^1\) operand digits $\neq 9$, i.e., ($s_1 \neq 9$)}. 
    \item \textbf{Dataset 2 (DS2): Carry in position \(10^0\), no cascading.} A carry is generated in the \(10^0\) ($s_0$) digit but does not cascade to the \(10^2\) ($s_2$) digit (e.g., \(236 + 125 = 361\)). 
    \item \textbf{Dataset 3 (DS3): Cascading carry from \(10^0\) to \(10^2\).} A carry originates in the \(10^0\) ($s_0$) digit and cascades to the \(10^2\) ($s_2$) digit (e.g., \(246 + 155 = 401\)).
    \item \textbf{Dataset 4 (DS4): Direct carry in position \(10^1\).} A carry is generated in the \(10^1\) ($s_1$) digit and directly affects the \(10^2\) ($s_2$) digit (e.g., \(252 + 163 = 415\)).
    \item \textbf{Dataset 5 (DS5): No carry, but position \(10^1\) digits sum to 9.} There is no carry in any digit, but the sum of the \(10^1\) operand digits is 9, i.e., ($s_1 = 9$) (e.g., \(256 + 142 = 398\)).
\end{itemize}
DS1 to DS5 can be neatly categorized according to whether the heuristic can accurately predict $s_2$: 

\begin{itemize}
    \item DS1 and 2: $t_1 = \sum_{j=1}^2 n_{j,1} < 9 \rightarrow c_{2}^{h} = 0$
    \item DS4: $t_1 = \sum_{j=1}^2 n_{j,i} > 9 \rightarrow c_{2}^{h} = 1$ 
    \item DS3 and 5: $t_1 = \sum_{j=1}^2 n_{j,1} = 9 \rightarrow c_{2}^{h} = ?$
\end{itemize}

\begin{figure}[t]
    \centering
    \begin{subfigure}[b]{0.33\textwidth} 
        \centering
        \includegraphics[width=\textwidth]{Images/figure6_a.png} 
    \end{subfigure} 
    \begin{subfigure}[b]{0.09\textwidth}
        \centering
        \includegraphics[width=\textwidth]{Images/figure6_b.png} 
        \vspace{0.8cm}
    \end{subfigure}
    \caption{Per-digit generation accuracy of Mistral and Gemma on datasets DS1-DS5. Each dataset represents a different carry scenario.}
    \label{fig:scenario_mistral_gemma}
\end{figure}

\paragraph{Results.} Figure \ref{fig:scenario_mistral_gemma} shows that  LLMs struggle with DS3 and DS5, which are precisely the cases where \textbf{H1} predicts issues. As \textbf{H1} suggests, predicting the first result digit $s_2$ at position $10^2$ is particularly error-prone in these scenarios. 
The difficult datasets are the ones where a lookahead of one digit position does not suffice to determine the value of the carry needed to generate $s_2$. Simply put:
Overall, addition results tend do be predicted correctly by LLMs, if and only if a lookahead of one digit is sufficient to determine the value of the carry bit affecting $s_2$. Prediction is often incorrect if a lookahead of two or more digits is needed to determine the value of the carry bit affecting $s_2$.

In cases where a lookahead of one digit is enough to accurately determine the value of $s_2$ (DS1, DS2, DS4), the models succeed.
However, when a lookahead of one digit is insufficient to determine the value of $s_2$ (DS3 and DS5), the model struggles with predicting $s_2$ correctly.
Table \ref{tab:accuracy_2-op} in Appendix \ref{appendix:B} provides the generation accuracy of $s_2$ for Gemma and Mistral, in addition to the plot. 
Additionally, Appendix \ref{sec:appendix_G} presents probing experiments that yield the same results.

\section{H1 Predicts the Deterioration of Accuracy in Multi-Operand Addition}
\label{sec:multi_fail}

As shown in the last section, \textbf{H1} is a good approximator for LLM behaviour on two-operand addition: In the majority of cases, a lookahead of one digit is sufficient to accurately determine the value of the carry bit affecting $s_2$. With a look-ahead of one digit, \textbf{H1} predicts a failure of the generation of $s_2$, if and only if the value of $s_1$ does not suffice to determine the value of the carry bit. 
In two-operand addition in base 10, this is the case if and only if $t_1 = 9$.
We now show that \textbf{H1} can also account for model performance on \textit{multi}-operand addition. 

\subsection{Multi-Operand Performance Predicted by H1}

The possible value of a carry increases with increasing numbers of operands. 
For instance in 4-operand addition ($k=4$) the maximal value of a carry is $3$:
\[c_{max}(4) = \left\lfloor \frac{\sum_{j=1}^4 9}{10} \right\rfloor = 3\] 

Therefore the carry heuristic \(c_{i}^{h}\) is unreliable in 4-operand addition whenever \(t_{i-1} =\sum_{j=1}^k n_{j,i-1} \in \{7, 8, 9, 17, 18, 19, 27, 28, 29\} \).

Put simply, because the value of the carry can be larger for more operands,  \textbf{the proportion of values of \(s_1\) for which the heuristic is insufficient (with its lookahead of one) increases with an increasing number of operands}. 

Consider an example in which the heuristic fails in 4-operand addition for clarification (see Figure \ref{fig:carry_4_op_fail} in Appendix \ref{appendix:C}): 

\noindent\textbf{186 + 261 + 198 + 256.}
\[
\begin{split}
    t_{1} =8 + 6 + 9 + 5 = 28\\
    c_{2}^{h} \in \{ \left\lfloor \frac{c_{min} + 28}{10} \right\rfloor,\\
    \left\lfloor \frac{c_{max} + 28}{10} \right\rfloor \}
\end{split}
\]

with \(c_{max} = 3\)
\[c_{2}^{h} \in \{ \left\lfloor \frac{28}{10} \right\rfloor, \left\lfloor \frac{31}{10} \right\rfloor \} = \{2, 3\}\]
therefore $c_{2}^{h}$ is chosen uniformly at random between $2$ and $3$.
The heuristic thus fails in solving \textbf{186 + 261 + 198 + 256} with a chance of 50\%. 


For 4-operand addition, there are 37 possible sums for the second digits (ranging from 0 to 36). In 28 out of these 37 cases, the heuristic reliably determines the correct carry bit. However, when \(t_1 \in \{7, 8, 9, 17, 18, 19, 27, 28, 29\}\), the heuristic must randomly choose between two possible carry values, leading to a 50\% chance of selecting the correct one. This results in an overall accuracy of:
\[\frac{28\times1.0 + 9 \times 0.5}{37} = 0.878\]
Thus, the heuristic only achieves 88\% accuracy in correctly predicting the first result digit $s_2$ in 4-operand addition, compared to the 97\% accuracy in two-operand addition. 
In Appendix \ref{appendix:D}, we provide exact values for $s_2$ accuracy as predicted by \textbf{H1}, for addition tasks between 2 and 11 operands. 


\begin{figure}[t]
    \centering
    \includegraphics[width=0.45\textwidth]{Images/figure7.png}
    \caption{Accuracy of first generated result digit $s_d$ in one-shot multi-operand addition for Mistral and Gemma, compared to the expected accuracy based on \textbf{H1}.}
    \label{fig:multi_op_accuracy}
    \vspace{-0.2cm}
\end{figure}


\subsection{Empricial Evidence on Multi-Operand Addition}
Intuitively, according to \textbf{H1}, Mistral and Gemma with their one-digit tokenization should fail at multi-operand addition at a certain rate: The amount of instances in which a lookahead of one digit is sufficient to accurately predict $s_i$ gets smaller and smaller because the carry bit value can get larger and larger for multiple operands. 
We test if \textbf{H1} holds in predicting the first generated digit $s_d$ in Mistral and Gemma for multiple operands. We evaluate prediction accuracy on the multi-operand datasets described in Section \ref{sec:models_data}.
\textbf{H1} should provide an upper bound for the performance of LLMs\footnote{Autoregressive LLMs with single-digit tokenization of numbers.} for predicting the first result digit $s_d$.
Figure \ref{fig:multi_op_accuracy} shows that \textbf{H1} is a good predictor for the accuracy of the one-shot\footnote{Results for the zero-shot setting are in Appendix \ref{sec:appendix_E}.} generation of the first result digit $s_d$ by Mistral and Gemma. We take this as further evidence that these LLMs make use of \textbf{H1}.
\section{Multi-Digit Tokenization Models Employ the Same Heuristic}
\label{sec:llama}
While \citet{levy2024language} demonstrate that all LLMs, regardless of the tokenization strategy, internally represent numbers as individual digits, it remained unclear whether models with multi-digit tokenization also rely on a one-digit lookahead when generating addition results. In this section, we show that perhaps surprisingly multi-digit tokenization models, such as Llama-3, also employ a lookahead of one \textbf{digit} when predicting carry bits. 
To show this, we design 3 controlled datasets that force the multi-digit tokenization model Llama-3 to generate results across multiple tokens. 

\paragraph{Experimental Setup.}
To examine whether Llama-3 employs a one-digit lookahead, we use six-digit numbers in two-operand addition (e.g., ``231234 + 124514 = ''), where each operand is tokenized into two three-digit tokens by the model's tokenizer, such as: [`` 231'',`` 234'', `` +'', `` 124'', `` 514'', `` =''] and the result is generated as two triple-digit tokens as well, in this example [`` 355'', `` 748'']. The first generated triple-digit token $s_5 s_4 s_3$ corresponds to digit base positions $10^5$, $10^4$, and $10^3$. If Llama-3 did employ \textbf{H1} it would look ahead to digit position $10^2$, but ignore digit positions $10^1$ and $10^0$, as they fall outside the lookahead window.

\paragraph{Carry Scenarios.}
We evaluate model behavior in three datasets with six-digit operands (ranging from 100,000 to 899,999) and results between 200,000 and 999,999. We use a zero-shot prompt template. Each dataset consist of 100 samples:
\begin{itemize}
    \setlength{\itemsep}{0.1pt}
    \setlength{\parskip}{0pt}
    \setlength{\parsep}{0.5pt}
    \item \textbf{DS6: No carry.} The addition does not produce any carry and no digits sum to 9.  (e.g., \(111,234 + 111,514 = 222,748\)).
    \item \textbf{DS7: Direct carry in position \(10^2\).} A carry is generated at \(10^2\) and directly affects \(10^3\) (e.g., \(111,721 + 111,435 = 223,156\)). 
    \item \textbf{DS8: Cascading carry from \(10^1\) to \(10^3\).} A carry originates at \(10^1\), cascades to \(10^2\) and then affects \(10^3\) (e.g., \(111,382 + 111,634 = 223,016\)).
\end{itemize}

\paragraph{Expected Outcomes.}
If Llama-3 employs \textbf{H1}, we expect that DS6 should be easy, as no carry propagation is required. DS7 should also be easy, since the carry affecting \(10^3\) is within the one-digit lookahead window. DS8 in contrast should be challenging, as the carry originates from \(10^1\), from beyond the model’s lookahead range. We expect a lower accuracy in generating \(10^3\), the result digit that is affected by the potentially inaccurate carry.


\begin{figure}[t]
    \centering
    \includegraphics[width=0.4\textwidth]{Images/figure8.png} 
    \caption{Per-digit generation accuracy of Llama on datasets DS6-DS8. Each dataset represents a different carry scenario.} 
    \label{fig:llama_carry_scenarios}
\end{figure}

\paragraph{Results.}
Figure \ref{fig:llama_carry_scenarios} shows that Llama-3 exhibits the expected pattern predicted by \textbf{H1}. The sharp drop in accuracy in dataset DS8 on digit \(10^3\) provides evidence that Llama-3, regardless of its multi-digit tokenization strategy, relies on the same one-digit lookahead for solving addition left to right. 
\section{Related Work}
\label{sec:related-work}
Recent work has benchmarked the arithmetic capabilities of LLMs using text-based evaluations and handcrafted tests \cite{yuan2023well,lightman2023lets,NEURIPS2023_58168e8a,zhuang2023efficiently}. Numerous studies consistently show that LLMs struggle with arithmetic tasks \cite{nogueira2021investigatinglimitationstransformerssimple, qian2022limitationslanguagemodelsarithmetic, dziri2023faithfatelimitstransformers, yu2024metamathbootstrapmathematicalquestions}. 

\citet{zhou2023algorithmstransformerslearnstudy} and \citet{zhou2024transformersachievelengthgeneralization} examine transformers' ability to learn algorithmic procedures and find challenges in length generalization \cite{anil2022exploringlengthgeneralizationlarge}. Similarly, \citet{xiao2024theorylengthgeneralizationlearning} propose a theoretical explanation for LLMs' difficulties with length generalization in arithmetic. \citet{gambardella2024language} find that LLMs can reliably predict the first digit in multiplication but struggle with subsequent digits.

The focus of research has recently shifted from mere benchmarking of LLMs to trying to understand \textit{why} LLMs struggle with arithmetic reasoning. Using circuit analysis, \citet{stolfo_mechanistic_2023} and \citet{hanna2023doesgpt2computegreaterthan} explore internal processing in arithmetic tasks, while \citet{nikankin2024arithmetic} reveal that LLMs use a variety of heuristics managed by identifiable circuits and neurons. In contrast, \citet{deng2024language} argue that LLMs rely on symbolic pattern recognition rather than true numerical computation. Recently, \citet{kantamneni2025languagemodelsusetrigonometry} showed that LLMs represent numbers as generalized helixes and perform addition using a “Clock” algorithm \cite{nanda_progress_2023}.

Related work has also examined how LLMs encode numbers. \citet{levy2024language} demonstrate that numbers are represented digit-by-digit, extending \citet{gould2023successor}, who find that LLMs encode numeric values modulo 10. \citet{zhu-etal-2025-language} suggest that numbers are encoded linearly, while \citet{marjieh2025number} indicate that number representations can blend string-like and numerical forms.

Another line of research explores how tokenization influences arithmetic capabilities. \citet{lee2024digitstodecisions} show that single-digit tokenization outperforms other methods in simple arithmetic tasks. \citet{singh2024tokenization} highlight that right-to-left (R2L) tokenization—where tokens are right-aligned—improves arithmetic performance. 
Additionally, the role of embeddings and positional encodings is emphasized by \citet{mcleish2024transformers}, who demonstrate that suitable embeddings enable transformers to learn arithmetic, and by \citet{shen2023positionaldescriptionmatterstransformers}, who show that positional encoding improves arithmetic performance.

\section{Conclusion}

Our study shows that LLMs, regardless of their numeric tokenization strategy, rely on a simple one-digit lookahead heuristic for anticipating carries when performing addition tasks. While this strategy is fairly effective for two-operand additions, it fails in the multi-operand additions due to the increasingly unpredictable value of cascading carry bits. Through probing experiments and targeted evaluations of digit-wise result accuracy, we demonstrate that model accuracy deteriorates precisely at the rate the heuristic predicts. 

These findings highlight an inherent weakness in current LLMs that prevents them from robustly generalizing to more complex arithmetic tasks. 

Our work contributes to a broader understanding of LLM limitations in arithmetic reasoning and highlights increasing LLMs' lookahead as a promising approach to enhancing their ability to handle complex numerical tasks.

\section*{Limitations}
Our work highlights limited lookahead as a key challenge for LLMs when adding multiple numbers. However, it remains unclear whether this limitation extends to other arithmetic operations, such as subtraction. Additionally, we cannot determine whether the limited lookahead is a heuristic explicitly learned for arithmetic tasks, or if it could also affect general language generation tasks as thus hinder performance of other tasks that require long-range dependencies. Future work should explore the depth of lookahead in tasks beyond arithmetic.

While the lookahead heuristic offers a straightforward explanation for the upper performance limit of LLMs on addition, it does not fully account for why LLMs still somewhat underperform relative to the heuristic in addition tasks with many operands (e.g., adding 8–11 numbers). We suspect this discrepancy may be related to limited training exposure to these many-operand addition tasks, but further investigation is needed to confirm this.

Our work also does not address whether larger models within the same family (e.g., 70B parameter models) exhibit a deeper lookahead. Future studies should examine whether scaling model size leads to improved performance by enabling a deeper lookahead.

Finally, we do not tackle methods to overcome the shallow lookahead. Future work should investigate whether targeted training on tasks requiring deeper lookahead can encourage models to deepen their lookahead.

%\section*{Ethics Statement}

\section*{Acknowledgements}
We thank Patrick Schramowski for his helpful feedback on the paper draft. This work has been supported by the German Ministry of Education and Research (BMBF) as part of the project TRAILS (01IW24005).


% Entries for the entire Anthology, followed by custom entries
\bibliography{custom, custom1}
\bibliographystyle{acl_natbib}

\appendix
\section*{Appendix A}
\label{AppendixA}

We used a series of data sets in our case studies and as examples in our paper. 

\subsection*{Bike Sharing}
The bike sharing data set is used to predict the number of bike rentals per hour. 

We trained a MLPC Regression model.

We used an 80:20 train:test split resulting in 13903 instances being used for training and in \textsc{Finch}.

In our example case, we used the following features:
\begin{itemize}
    \item count (target): the number of bike rentals that hour
    \item hour=3: the hour for which the bike rentals where recorded. Here: 3am.
    \item workingday=0: if the instance was recorded on a workingday or not. Categoric feature. 0=no, 1=yes.
    \item season=0: in which season the instance was recorded. Categoric feature. 0=winter, 1=spring, 2=summer, 3=autumn.
\end{itemize}



\begin{figure}[h!]
    \centering
    \begin{subfigure}[b]{0.5\textwidth}
        \centering
        \includegraphics[width=1\linewidth]{california1.png}
        \caption{house value per median income}
        \label{fig:california1}
    \end{subfigure}
    \hspace{0.001\textwidth} % Adds some space between the two images
    \begin{subfigure}[b]{0.5\textwidth}
        \centering
        \includegraphics[width=1\linewidth]{california2.png}
        \caption{house value per median income for areas with a population of 2000}
        \label{fig:california2}
    \end{subfigure}
    \hspace{0.001\textwidth} % Adds some space between the two images
    \begin{subfigure}[b]{0.5\textwidth}
        \centering
        \includegraphics[width=1\linewidth]{california3.png}
        \caption{house value per median income for areas with a population of 2000 and 1000 total rooms}
        \label{fig:california3}
    \end{subfigure}
    \hspace{0.001\textwidth} % Adds some space between the two images
    \begin{subfigure}[b]{0.5\textwidth}
        \centering
        \includegraphics[width=1\linewidth]{california4.png}
        \caption{The ground truth is even lower.}
        \label{fig:california4}
    \end{subfigure}
    \caption{The incremental visualization of the interaction of median income, population and total rooms in the california housing data set. Colored areas visualize the change in each step.}
\end{figure}

\subsection*{Titanic}
The titanic data set is used to predict the survival of people on board the titanic. 

We trained an MLPC classifier. The resulting accuracy was 70.23\%.

We used an 80:20 train:test split resulting in 1047 instances being used for training and in \textsc{Finch}.

In the described interaction, we used the following features:
\begin{itemize}
    \item survival(target): If the current person survived.
    \item pclass=1: Which passenger class the current person belonged to. Categoric feature. 1=first class, 2=second class, 3=third class. 
    \item sex=1: The sex of the person. Categoric feature. 0=male, 1=female.
    \item age:30: The age of the person. Here: 30yo.
\end{itemize}




\subsection*{California housing}
The california housing data set is used to predict housing values for block groups in California and was derived from the 190 US census. It contains only continuous variables. The mean predicted housing value is 200.000. 

We trained a GradientBoostingRegressor model on the data. 
It was trained with 100 boosting stages, a learning rate of 0.1 and squared error as the loss function.
The resulting R2 score was 0.77.

We used an 80:20 train:test split resulting in 16,512 instances being used for training and in \textsc{Finch}.

In our observed interaction, we used the following features:
\begin{itemize}
    \item housing value (target): Median house value in US Dollars.
    \item median income: The median income of that block group in 100,000 US Dollars.
    \item population: The number of people residing in the block group.
    \item total rooms: The total number of rooms in that block group.
\end{itemize}

\subsection*{Diabetes}
The diabetes risk factor data set. It is based on the BRFSS telephone study that is performed yearly in the united states.

We used a subset of 10,000 instances for model training. Using a 80:20 train/test split, this resulted in 8000 instnaces being used in \textsc{Finch}.

For better model training, half of the instances are diabetes positive, and half negative. Therefore, the probabilities generated by the model and \textsc{finch} cannot be directly used on a general public.

In our interaction, we considered the following features:
\begin{itemize}
    \item diabetes risk (target): The diabetes risk for the person, that the model predicted.
    \item sex=0: The sex of the person. 0=male, 1=female.
    \item exercise=1: If the person exercises. 1=yes, 0=no.
    \item high blood pressure=1: If the person has high blood pressure. Categoric feature. 1=yes, 0=no.
    \item weight category=3: The weight category of the person. Categoric feature. 0=underweight, 1= normal weight, 2=overweight, 3=obese.  
\end{itemize}

\begin{figure}[h]
    \centering
    \begin{subfigure}[b]{0.5\textwidth}

        \includegraphics[width=0.7\linewidth]{diabetes1.png}
        \caption{diabetes risk per sex}
        \label{fig:diabetes1}
    \end{subfigure}
    \hspace{0.001\textwidth} % Adds some space between the two images
    \begin{subfigure}[b]{0.5\textwidth}
        \centering
        \includegraphics[width=1\linewidth]{diabetes2.png}
        \caption{diabetes risk per sex for people who exercise}
        \label{fig:diabetes2}
    \end{subfigure}
    \hspace{0.001\textwidth} % Adds some space between the two images
    \begin{subfigure}[b]{0.5\textwidth}
        \centering
        \includegraphics[width=1\linewidth]{diabetes3.png}
        \caption{diabetes risk per sex for people who exercise but have high blood pressure}
        \label{fig:diabetes3}
    \end{subfigure}
    \hspace{0.001\textwidth} % Adds some space between the two images
    \begin{subfigure}[b]{0.5\textwidth}
        \centering
        \includegraphics[width=1\linewidth]{diabetes4.png}
        \caption{diabetes risk per sex for people who exercise, have high blood pressure and are obese}
        \label{fig:diabetes4}
    \end{subfigure}
    \caption{Diabetes risk per sex. Incremental interaction of sex, exercise, high blood pressure and weight. Based on the BRFSS data set.}
\end{figure}


\chapter{\textcolor{black}{Edge Network optimization}}\label{app: EN_ib}

In this section the mathematical solution of the optimization problem \eref{eq: EN_ib initial opt problem} in \sref{sec: EN_ib} reported below:

\begin{mini}|s|[0]
    {\mathbf{\Psi}(t)}{\lim_{T \to +\infty}\; \frac{1}{T} \sum_{t=1}^T  \mathbb{E}[P^{tot}(t)] }
    {}{}
    \addConstraint{\lim_{T \to +\infty}\; \frac{1}{T} \sum_{t=1}^T  \mathbb{E}[D_k^{tot}(t)] \leq D_k^{avg}\qquad \forall k }{}
    \addConstraint{ \lim_{T \to +\infty}\; \frac{1}{T} \sum_{t=1}^T  \mathbb{E}[G_k(t)] \leq G_k^{avg}\qquad \forall k }{}
    \addConstraint{0 \leq f_k(t) \leq f_k^{max} \qquad \forall k,t }{}
    \addConstraint{0 \leq R_k(t) \leq R_k^{max}(t) \qquad \forall k,t }{}
    \addConstraint{\beta_k(t) \in \mathcal{B}_k  \qquad \forall k,t}{}
    \addConstraint{0 \leq f^{es}(t) \leq f_{es}^{max} \qquad \forall t}{}
    \addConstraint{f_k^{es}(t) \geq 0 \quad \forall k,t}, \qquad {\sum_{k=1}^K f_k^{es}(t) \leq f_c(t)  \quad \forall t,}{}
\end{mini}

These virtual queues associated to the long-term delay and evaluation metric constraints, $T_k(t)$ and $U_k(t)$ respectively are introduced as follows \cite{Neely2010Lyapunov}:
\begin{align}
    T_k(t+1) &= \max [0,T_k(t) + \varepsilon_k(D_k^{tot}(t) - D_k^{avg})] \\
    U_k(t+1) &= \max [0,U_k(t) + \nu_k(G_k(t) - G_k^{avg})],  
\end{align}
where $\epsilon_k$ and $ \nu_k $ are the learning rate for the update of the virtual queues. 

Based on these virtual queues is possible to define the \textit{Lyapunov function} $L(\mathbf{\Theta}(t))$ as:
\begin{equation}
    L(\mathbf{\Theta}(t)) = \frac{1}{2} \sum_{k=1}^K T_k^2(t) + U_k^2(t),
    \tag{\ref{eq: EN_ib Lyapunov function}}
    \label{app: EN_ib Lyapunov function}
\end{equation}
where $\mathbf{\Theta}(t) = [\{T_k(t)\}_k, \{U_k(t)\}_k]$ is the vector composed by all the virtual queues at time $t$. The idea is to use this Lyapunov function to satisfy the constraints on $D_k^{avg}$ and $G_k^{avg}$ by enforcing the stability of $L(\mathbf{\Theta}(t))$. 

To this scope it is introduced the so called \textit{drift-plus-penalty function}:
\begin{align}
    \Delta(\Theta(t)) &= \mathbb{E}\left[L({\Theta}(t+1))-L({\Theta}(t))+V\cdot P^{tot}(t)  \;\Big|\; \Theta(t)\right] \\
    &=\mathbb{E}\left[\;\sum_{k=1}^K \frac{T_k^2(t+1)-T_k^2(t)}{2} +  \frac{U_k^2(t+1)-U_k^2(t)}{2} +V\cdot P^{tot}(t)\;\; \Big|\;\; \Theta(t)\right]\\
    &= \mathbb{E}\left[\;\sum_{k=1}^K \Delta_{T_k} +  \Delta_{U_k} +V\cdot P^{tot}(t) \;\; \Big|\;\; \Theta(t)\right],
    \label{app: EN_ib drift plus penalty}
\end{align}
where, starting from a generic virtual queue evolving as 
$H(t+1) = \max [0,H(t) +h(t) - \Bar{h}]$ the quantity $\Delta_H$ is defined as follows:
\begin{align*}
    \Delta_H &= \frac{H^2(t+1)-H^2(t)}{2} = \frac{\max [0,(H(t) +h(t) - \Bar{h})^2]-H^2(t)}{2} \\
   &\leq   \frac{(h(t) - \Bar{h})^2}{2} + H(t)[h(t)-\Bar{h}].
\end{align*} 

By applying the same upper bound to $\Delta_{T_k}$ it is possible to obtain:
\begin{align}
    \Delta_{T_k} &= \frac{T_k^2(t+1)-T_k^2(t)}{2} = \frac{\max [0,(T_k(t) + \nu_k(D_k^{tot}(t) - D_k^{avg}))^2]-T_k^2(t)}{2} \\
    &\leq   \nu_k^2\frac{(D_k^{tot}(t) - D_k^{avg})^2}{2} + \nu_k T_k(t)[D_k^{tot}(t) - D_k^{avg}] \\
    &\leq \nu_k^2\frac{(D_k^{max} - D_k^{avg})^2}{2}  + \nu_k T_k(t)[D_k(t) - D_k^{avg}],
    \label{app: EN_ib delta U_k}
\end{align}
where $D_k^{max}(t)$ is the maximum delay allowed for the $k$-th \gls{ed}.

By applying the same reasoning to $\Delta_{U_k}$ it is possible to obtain:
\begin{equation}
    \Delta_{U_k} \leq \nu_k^2\frac{(G_k^{max} - G_k^{avg})^2}{2}  + \nu_k U_k(t)[G_k(t) - G_k^{avg}],
    \label{app: EN_ib delta U_k}
\end{equation}
where $G_k^{max}(t)$ is the maximum value allowed for the evaluation metric for the $k$-th \gls{ed}.

Substituting now \eref{app: EN_ib delta U_k} and \eqref{app: EN_ib delta U_k} inside \eref{app: EN_ib drift plus penalty} and rearranging the terms it is possible to obtain:

\begin{align}
    \Delta_p(\Theta(t)) &\leq
    \sum_{k=1}^K \Bigg{[} \nu_k^2\frac{(D_k^{max} - D_k^{avg})^2}{2} + \nu_k^2\frac{(G_k^{max}(t) - G_k^{avg})^2}{2}  \Bigg{]}  \\ &\;\;\;
    + \mathbb{E} \Bigg{[}\;\sum_{k=1}^K \Big{[} - \varepsilon_k Z_k(t)Q_k^{avg} - \nu_k S_k(t)G_k^{avg}   + \Big|\;\; \Theta(t) \Bigg{]} \\ &\;\;\; + \mathbb{E} \Bigg{[}\;\sum_{k=1}^K \Big{[} \varepsilon_k Z_k(t)Q_k^{tot}(t)  +  \nu_k S_k(t)G_k(t) \Big{]} + V\cdot P^{tot}\;\; \Big|\;\; \Theta(t) \Bigg{]}, 
\end{align}
where some constants that have been taken out of the expected value (first line), while others even if within the expected value do not depend on the optimization parameters (second line).

Pivoting therefore on the Lyapunov optimization it is possible to neglect all these terms. Moreover, it is possible to remove the expected value to obtain the following per-slot optimization:

\begin{mini}|s|[0]
    {\mathbf{\Psi}(t)}{\sum_{k=1}^K \bigg[ \frac{\epsilon_kT_k(t)N_k(t)}{R_k(t)} + \frac{\epsilon_kT_k(t)W_k(t)}{f_k(t)\rho_k } + \frac{\epsilon_kT_k(t)W_{max}^{es}}{f_k^{es}(t) \rho_k^{es}}+}{}{} \breakObjective{\qquad +  \frac{B_k N_0}{h_k(t)} {\rm exp} \left(\frac{R_k(t) ln(2)}{B_k} \right) + V \Gamma_k \eta_k (f_k(t))^3 +}{}{} \breakObjective{\;+  V \eta (f_c(t))^3 + \nu_k U_k(t)G_k(t)\bigg]}{}{}
    \addConstraint{\mathbf{\Psi}(t) \in \mathcal{T}(t),}{}
    \label{eq: EN_ib per-slot opt problem structure}
\end{mini}
where $\mathcal{T}(t)$ indicates the space of possible solutions given by the constraints on the optimization variables. 

at this point it is possible to split the problem for the resource allocation at the \gls{ed} and at the \gls{es}.

\section{Edge Device optimization}\label{app: EN_ib ed opt}
The sub-problem for the \gls{ed} as defined in \eref{eq: EN_ib per-slot opt ed} can be split in two further sub-problems for the transmission rate $R_k(t)$ and the clock frequency $f_k(t)$.

\subsection*{Transmission rate optimal solution}
The sub-problem associated to the transmission rate $R_k(t)$ can be defined as follows:
\begin{mini}|s|[0]
    {R_k(t)}{\frac{\epsilon_kT_k(t)N_k(t)}{R_k(t)} +  V \frac{B_k N_0}{h_k(t)} {\rm exp} \left(\frac{R_k(t) ln(2)}{B_k} \right) }{}{}
    \addConstraint{0 \leq R_k(t) \leq R_k^{max}(t)}{} 
\end{mini}

To simplify the notation, define:
\[
A = \epsilon_k T_k(t) N_k(t), \quad B = V \dfrac{B_k N_0}{h_k(t)}, \quad C = \dfrac{\ln(2)}{B_k}.
\]

Computing the derivative of the objective function $J(R_k(t))$ with respect to $R_k(t)$and set it to zero it is possible to obtain:
\[
\frac{dJ}{dR_k(t)} = -\dfrac{A}{[R_k(t)]^2} + B C \exp\left( C R_k(t) \right) = 0.
\]

By defining Let $x = C R_k(t)$ and $d = \dfrac{A C}{B}$ the derivative can be rearranged as:
\[
x e^{\frac{x}{2}} = \sqrt{d}.
\]

Fortunately, there is an exact solution to this problem and it is based on the \textit{Lambert W function}. By applying the definition and substituting back all the terms it is possible to obtain the final solution:
\begin{equation}
    R_k^*(t) = \frac{2 B_k}{ln(2)}\; W\! \!\left(\sqrt{\frac{\epsilon_k T_k(t)\; ln(2)\; h_k(t)N_k(t)\; }{4 B_k^2\;V \;N_0}}\right)\; \Biggr|_0^{R_k^{max}(t)}
\end{equation}

\subsection*{Clock frequency optimal solution}
The sub-problem associated to the transmission rate $R_k(t)$ can be defined as follows:
\begin{mini}|s|[0]
    {f_k(t)}{\frac{\epsilon_k T_k(t)W_k(t)}{f_k(t)\rho_k } +  V \Gamma_k \eta_k (f_k(t))^3 }{}{}
    \addConstraint{0 \leq f_k(t) \leq f_k^{max}}{} 
\end{mini}

To simplify the notation define:
\[
A = \dfrac{\epsilon_k T_k(t) W_k(t)}{\rho_k}, \quad B = V \Gamma_k \eta_k
\]

Computing the derivative of the objective function $J(f_k(t))$ with respect to $f_k(t)$ and set it to zero it is possible to obtain:
\[
\frac{dJ}{df_k(t)} = -\dfrac{A}{[f_k(t)]^2} + 3B [f_k(t)]^2 = 0
\]

After multiply both sides by $[f_k(t)]^2$, rearranging the terms and substituting back  $A$ and $B$ the final solution is:
\[
f_k(t) = \left( \dfrac{A}{3B} \right)^{1/4} \; \Biggr|_0^{f_k^{max}} \implies f_k^* (t) = \sqrt[4]{\frac{\epsilon_k T_k(t) W_k(t)}{3 V \Gamma_k \eta_k \rho_k} }\; \Biggr|_0^{f_k^{max}},
\]


\section{Edge Server optimization}\label{app: EN_ib es opt}


\begin{mini}|s|[0]
    {\{f_f^{es}(t)\}_k, f_c(t)}{\sum_{k=1}^K \frac{\epsilon_k T_k(t)W_{max}^{es}}{f_k^{es}(t)\rho_k^{es}} + V \eta (f_c(t))^3 }{}{}
    \addConstraint{0 \leq f_c(t) \leq f_c^{max} }{}
    \addConstraint{f_k^{es}(t) \geq 0 \quad \forall k}, \qquad {\sum_{k=1}^K f_k^{es}(t) \leq f_c(t)}{}
\end{mini}

Define:
\[
A_k = \dfrac{ \epsilon_k T_k(t) W_{\text{max}}^{es} }{ \rho_k^{es} }, \quad B = V \eta, \quad S = \sum_{k=1}^K \sqrt{ A_k }
\]


The objective function becomes:
\[
J(\{f_k^{es}(t)\}_k,\ f_c(t)) = \sum_{k=1}^K \dfrac{A_k}{f_k^{es}(t)} + B [f_c(t)]^3
\]

As a first step it is possible to define the associated Lagrangian $L$ of the sub-problem with respect to  $f_k^{es}(t)$ given $f_c(t)$ as:
\[
L = \sum_{k=1}^K \dfrac{A_k}{f_k^{es}(t)} + \lambda \left( \sum_{k=1}^K f_k^{es}(t) - f_c(t) \right)
\]

By deriving it and isolating with respect to $f_k^{es}(t)$ it is possible to obtain:

Solve for $f_k^{es}(t)$:
\[
    \frac{\partial L}{\partial f_k^{es}(t)} = -\dfrac{A_k}{[f_k^{es}(t)]^2} + \lambda = 0  \implies [f_k^{es}(t)]^2 = \dfrac{A_k}{\lambda} \implies f_k^{es}(t) = \sqrt{ \dfrac{A_k}{\lambda} }
\]

Apply the coupling constraint on $f_c(t)$ and by solving for $\lambda$ it is possible to identify:
\[
\sum_{k=1}^K f_k^{es}(t) = \dfrac{1}{\sqrt{\lambda}} \sum_{k=1}^K \sqrt{ A_k } = f_c(t) \implies \sqrt{\lambda} = \dfrac{ S }{ f_c(t) } \implies \lambda = \left( \dfrac{ S }{ f_c(t) } \right)^2
\]

Therefore:
\[
f_k^{es}(t) = \dfrac{ \sqrt{ A_k } }{ S } f_c(t)
\]

This term can now be substituted back into the objective function that is then derived with respect to $f_c(t)$ and set to zero as:

\[
J(f_c(t)) = \dfrac{ S^2 }{ f_c(t) } + B [f_c(t)]^3 \implies \frac{dJ}{df_c(t)} = - \dfrac{ S^2 }{ [f_c(t)]^2 } + 3 B [f_c(t)]^2 = 0
\]

By solving for $f_c(t)$, substituting back the expressions of $A$, $B$ and $S$ and applying the constraints it is possible to obtain the final solution:
\[
    f_c^*(t) = \left[ \left( \dfrac{ S^2 }{ 3 B } \right)^{1/4} \right]_0^{f_c^{\text{max}}}  = \frac{\sqrt{\sum_{k=1}^K \sqrt{\frac{\epsilon_k T_k(t)W_{max}^{es}}{\rho_k^{es}}}}}{\sqrt[4]{3V\eta}} \; \Biggr|_0^{f_{c}^{max}}
\]


Therefore, for every $k$:
\[
f_k^{es}(t) = \dfrac{ \sqrt{ A_k } }{ S } f_c^*(t) = f_k^{es*}(t) = \frac{\sqrt{\frac{\epsilon_k T_k(t)W_{max}^{es}}{\rho_k^{es}}}}{\sqrt{\sum_{k=1}^K \sqrt{\frac{\epsilon_k T_k(t)W_{max}^{es}}{\rho_k^{es}}}}\sqrt[4]{3V\eta} }
\]

\section{Example: H1 Failure on 4-Operand Addition}
\label{appendix:C}
Below is an example in which the heuristic \textbf{H1} fails in 4-operand addition, visualized in Figure \ref{fig:carry_4_op_fail}: 

\noindent\textbf{186 + 261 + 198 + 256.}
\[
\begin{split}
    t_{1} =8 + 6 + 9 + 5 = 28\\
    c_{2}^{h} \in \{ \left\lfloor \frac{c_{min} + 28}{10} \right\rfloor,\\
    \left\lfloor \frac{c_{max} + 28}{10} \right\rfloor \}
\end{split}
\]

with \(c_{max} = 3\)
\[c_{2}^{h} \in \{ \left\lfloor \frac{28}{10} \right\rfloor, \left\lfloor \frac{31}{10} \right\rfloor \} = \{2, 3\}\]
therefore $c_{2}^{h}$ is chosen uniformly at random between $2$ and $3$.
The heuristic thus fails in solving \textbf{186 + 261 + 198 + 256} with a chance of 50\%. 

\begin{figure}[ht]
    \centering
    \includegraphics[width=0.5\textwidth]{Images/figure9.png} 
    \caption{4-operand addition in which \textbf{H1} fails.} 
    \label{fig:carry_4_op_fail}
\end{figure}
\section{Zero-shot Generation Accuracy}
\label{sec:appendix_E}

We test if \textbf{H1} holds up in predicting the generation accuracy on $s_d$ of Mistral and Gemma for multiple operands. Figure \ref{fig:s2_zero-shot} shows that \textbf{H1} provides an upper bound for the generation accuracy of $s_d$ in a zero-shot setting for Mistral and Gemma on $s_d$.

\begin{figure}[h]
    \centering
    \includegraphics[width=0.45\textwidth]{Images/figure10.png} 
    \caption{Accuracy of first generated result digit $s_d$ in zero-shot multi-operand addition tasks for Mistral and Gemma, compared to the expected accuracy on $s_d$ based on \textbf{H1}.} 
    \label{fig:s2_zero-shot}
\end{figure}
\section{Accuracy Prediction of Heuristic}
\label{appendix:D}

\renewcommand{\arraystretch}{1.2}
\setlength{\tabcolsep}{5pt}
\begin{table*}[ht]
\centering
\begin{tabular}{|c|c|c|c|c|}
\hline
Nr. Operands $k$ & \textbf{\(c_{max}(k)\)} & Values of \(t_i\) in which H1 fails & Expected acc. on \(s_d\)\\ \hline
2 & 1 & 1 fail:= 9 & $\frac{18\times1.0 + 1 \times 0.5}{19} = 0.974$ \\ \hline
3 & 2 & 4 fails:= 8, 9, 18, 19 & $\frac{24\times1.0 + 4 \times 0.5}{28} = 0.928$\\ \hline
4 & 3 & 9 fails:= 7, 8, 9, 17, 18, 19, 27, 28, 29 & $\frac{28\times1.0 + 9 \times 0.5}{37} = 0.878$ \\ \hline
5 & 4 & 16 fails:= 6, 7, 8, 9, 16, ..., 39 & $\frac{ 30 \times1.0 + 16 \times 0.5}{46} = 0.826$\\ \hline
6 & 5 & 25 fails:= 5, 6, 7, 8, 9, 15, ..., 49 & $\frac{ 30 \times1.0 + 25 \times 0.5}{55} = 0.773$\\ \hline
7 & 6 & 36 fails:= 4, 5, 6, ..., 59 & $\frac{ 28 \times1.0 + 36 \times 0.5}{64} = 0.719$\\ \hline
8 & 7 & 49 fails:= 3, 4, 5, ..., 69 & $\frac{ 24 \times1.0 + 49 \times 0.5}{73} = 0.664$\\ \hline
9 & 8 & 64 fails:= 2, 3, 4, ..., 79 & $\frac{ 18 \times1.0 + 64 \times 0.5}{82} = 0.610$\\ \hline
10 & 9 & 81 fails:= 1, 2, 3, ..., 89 & $\frac{ 10 \times1.0 + 81 \times 0.5}{91} = 0.555$\\ \hline
11 & 9 & 89 fails:= 1, 2, 3, ..., 99 & $\frac{ 10 \times1.0 + 90 \times 0.5}{100} = 0.540$\\ \hline
%12 & 108 & 10 & 109 fails:= 0, 1, 2, ..., 108 & /109 = 0.1 \\ \hline
\end{tabular}
\caption{Predicted accuracy on the first result digit $s_d$ in the addition of multiple numbers according to \textbf{H1}.}
\label{tab:heuristic}
\end{table*}

Table \ref{tab:heuristic} contains, for addition tasks with different numbers of operands $k$, the maximum value of the carry \(c_{max}(k)\). Based on \(c_max\) it list those values of \(t_i\) in which \textbf{H1} is insufficient to accurately predict \(s_2\). Based on the proportion of values of \(t_i\) for which \textbf{H1} is sufficient to the total number of possible values, it lists the predicted accuracy for \(s_2\).

\newpage
\section{Max attributes}\label{sec:max_attributes}


\begin{proof}
     If $\min \max P_e \leq 1/2$, then:
     \begin{equation}
         1 - \frac{1}{2\sqrt{d}}         
            \exp
        \left(\frac{m \epsilon^2}{2\sigma^2}
        \right)
        \leq \min \max P_2 
        \leq 1/2
     \end{equation}
     Or equivalently, if $\min \max P_e \leq 1/2$, then:
    \begin{equation}
        \frac{1}{\sqrt{d}}         
                \exp
            \left(\frac{m \epsilon^2}{2\sigma^2}
            \right)
            \geq 1
    \end{equation}
    
         Given the number of groups $d = 2^k$, the number of samples per group $m = n/d$, the total number of samples $n = 10^{4} $ and the threshold of $\epsilon = 0.01$, we get:
     \begin{equation}
         \phi(k) 
         = 1 - 
         \frac{1}{2^{k/2+1}}         
            \exp
        \left(
        \frac{ \frac{10^{4}} {2^k}\times 0.0001}{2\sigma^2}
        \right)
        = 1 - 
           \frac{1}{2^{k/2+1}}         
            \exp
        \left(
        \frac{1}{2^{k+1}\sigma^2}
        \right)
     \end{equation}

     We prove that this function is an increasing function in $k$. Indeed, consider the auxiliary function $f(x) = \frac{1}{2\sqrt{x}}\exp(\frac{a}{x})$. Its derivative is $f'(x) = -\frac{\exp(\frac{a}{x})(2a+x)}{4x^{5/2}}$. For $x, a > 0$, we have: $f'(x)< 0$, i.e., $f$ is a monotonically decreasing function. Consequently, $1 - f$ is a monotonically increasing function. Thus, the function $k \rightarrow 1 - f(2^k)$ with $a = \frac{1}{2\sigma^2}$ is a monotonically increasing function of $k > 0$.
\end{proof}
\section{Probing Accuracy on Carry Scenarios}
\label{sec:appendix_G}

We evaluate probing accuracy of the probes trained in Section \ref{sec:probing} across the five distinct carry scenarios, introduced in Section \ref{sec:h1_2op}. 

\paragraph{Results.} Figure \ref{fig:carry_plot} shows that  LLMs struggle with DS3 and DS5, which are exactly the cases where \textbf{H1} would predict problems. The difficult datasets are the ones where a lookahead of one digit position does not suffice to determine the value of the carry needed to generate $s_2$. Simply put: 
In cases where a lookahead of one digit is enough to accurately determine the value of $s_2$ (DS1, DS2, DS4), the models have a relatively good internal representation of the value of the second result digit $s_1$. This results in high performance on the currently generated digit $s_2$. However, when a lookahead of one digit is insufficient to determine the value of $s_2$ (DS3 and DS5), the model struggles with representing digits $s_1$ and $s_2$ correctly.

\begin{figure*}[t]
    \centering
    \begin{subfigure}[b]{0.31\textwidth}
        \centering
        \includegraphics[width=\textwidth]{Images/figure11_a.png}
        \caption{Mistral}
        \label{fig:plot1}
    \end{subfigure} 
    \hfill
    \begin{subfigure}[b]{0.31\textwidth}
        \centering
        \includegraphics[width=\textwidth]{Images/figure11_b.png}
        \caption{Gemma}
        \label{fig:plot2}
    \end{subfigure}
    \hfill
    \begin{subfigure}[b]{0.31\textwidth}
        \centering
        \includegraphics[width=\textwidth]{Images/figure11_c.png} 
        \caption{Llama-3}
        \label{fig:plot3}
    \end{subfigure}
    \hfill
    \begin{subfigure}[b]{0.038\textwidth}
        \centering
        \includegraphics[width=\textwidth]{Images/figure11_d.png} 
    \end{subfigure}
    
    \caption{Digit-wise probing accuracy of result digits of 2-operand addition tasks.
    Each subplot shows the probing accuracies of one model on Datasets DS1-DS5.}
    \label{fig:carry_plot}
\end{figure*}

\end{document}
