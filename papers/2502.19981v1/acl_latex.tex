% This must be in the first 5 lines to tell arXiv to use pdfLaTeX, which is strongly recommended.
\pdfoutput=1
% In particular, the hyperref package requires pdfLaTeX in order to break URLs across lines.

\documentclass[11pt]{article}

% Change "review" to "final" to generate the final (sometimes called camera-ready) version.
% Change to "preprint" to generate a non-anonymous version with page numbers.
\usepackage[final]{acl}

% Standard package includes
\usepackage{times}
\usepackage{latexsym}

% For proper rendering and hyphenation of words containing Latin characters (including in bib files)
\usepackage[T1]{fontenc}
% For Vietnamese characters
% \usepackage[T5]{fontenc}
% See https://www.latex-project.org/help/documentation/encguide.pdf for other character sets

% This assumes your files are encoded as UTF8
\usepackage[utf8]{inputenc}

% This is not strictly necessary, and may be commented out,
% but it will improve the layout of the manuscript,
% and will typically save some space.
\usepackage{microtype}

% This is also not strictly necessary, and may be commented out.
% However, it will improve the aesthetics of text in
% the typewriter font.
\usepackage{inconsolata}

%Including images in your LaTeX document requires adding
%additional package(s)
\usepackage{graphicx}

% If the title and author information does not fit in the area allocated, uncomment the following
%
%\setlength\titlebox{<dim>}
%
% and set <dim> to something 5cm or larger.

\usepackage{subcaption}
\usepackage{mwe}

\usepackage{multirow}
\usepackage{microtype}
\usepackage{booktabs}

\usepackage{amsfonts}

\usepackage{float} % Add this in your preamble
\usepackage{amsmath}
\usepackage[colorinlistoftodos]{todonotes}
\usepackage{rotating}

\title{The Lookahead Limitation:\\ Why Multi-Operand Addition is Hard for LLMs}

\newcommand{\affilsup}[1]{\rlap{\textsuperscript{\normalfont#1}}}

\author{
    Tanja Baeumel\affilsup{1,2}
    \qquad
    Josef van Genabith\affilsup{1, 3}
    \qquad
    Simon Ostermann\affilsup{1,2}
    \\
    $^1$German Research Center for Artificial Intelligence (DFKI) \\
    $^2$Centre for European Research in Trusted AI (CERTAIN) \\
    $^3$Department of Language Science and Technology, Saarland University \\
    Saarland Informatics Campus, Saarbrücken, Germany\\
    \texttt{\{firstname.lastname\}@dfki.de} \\
}


\begin{document}
\maketitle
\begin{abstract}
Autoregressive large language models (LLMs) exhibit impressive performance across various tasks but struggle with simple arithmetic, such as additions of two or more operands. We show that this struggle arises from LLMs’ use of a \textbf{simple one-digit lookahead heuristic}, which works fairly well (but not perfect) for two-operand addition but fails in multi-operand cases, where the carry-over logic is more complex. Our probing experiments and digit-wise accuracy evaluation show that LLMs fail precisely where a one-digit lookahead is insufficient to account for cascading carries. We analyze the impact of tokenization strategies on arithmetic performance and show that all investigated models, regardless of tokenization, are inherently limited in the addition of multiple operands due to their reliance on a one-digit lookahead heuristic. Our findings reveal fundamental limitations that prevent LLMs from generalizing to more complex numerical reasoning.
\end{abstract}

\section{Introduction}

\begin{figure*}
    \centering
    \includegraphics[width=\textwidth]{figures/Introduction.pdf}
    \caption{Showing the novel problem statement applied to traffic prediction use case. Multiple unstructured observations from the past are used to reconstruct a hidden traffic state from which a full traffic state is forecast with a set of query locations. }
    \label{fig:intro}
\end{figure*}

% Was sagen denn die anderen warum Traffic Prediction gut ist? 
Forecasting the traffic in the near future is an important task for city management.
Data from the near past is used to predict future traffic states with spatio-temporal Graph Neural Networks \cite{bui22}.
Accurate prediction provides the opportunity to optimize traffic flow, reduce traffic jams and increase air quality \cite{Po19}.

% Wieso ist Sparsity in allen Dimensionen wichtig.
While traffic prediction relies on the availability of data from traffic sensors, there exists a plethora of reasons why sensors may stop working temporarily, such as simple errors, energy saving, or overloaded communication systems.
Considering small- or medium-sized cities, the coverage of sensors may be low because the sensors are too expensive or not available.
Also, the sensors are typically static and do not adapt to changes in the traffic flow (e.g. caused by a construction site), which motivates moving sensors that for example could be mounted on cars. 
However, both missing and moving sensors introduce sparsity, since measurements may not be available for all locations at all times.
This sparsity must be explicitly addressed in traffic prediction for a realistic application scenario, which is illustrated in figure \ref{fig:intro}.
From one hour of data on Sunday morning, only few observations of the traffic state are available at each timestep.
The number of observations may differ throughout the observed time and the observation itself can be distributed arbitrarily in the city. 
We assume a relatively low number of sensors to account for resource saving and sensor failure in our proposed framework SUSTeR.
The task is to predict the dense traffic state one timestep after the observations at all possible sensor locations.
We study this problem on the traffic dataset Metr-LA and PEMS-BAY to test our assumption that only a fraction of the sensor values would be enough for good predictions.
By modifying an existing traffic dataset, we are able to compare our results from very sparse observations to the bottom line with all information available.
A successful study will provide insights in how sensors in new cities can be reduced before installing them and further mobile sensors would save more resources and are able to adapt to new traffic situations.
We argue that in order to be adaptable to other cities and changes in traffic flows, prior information like the road network should be neglected and just the sparse observations considered.
This comes with the added benefit of making our solution applicable in regions where no openly available road network is maintained or pathways change frequently (e.g. flood areas, animal observations). 


The aforementioned problem is novel and more challenging than the commonly considered traffic prediction problem, since there exist very few observations in each input sample.
Current works for the traffic prediction problem do not consider any missing values. \cite{Li2021, Shao22}
A common method among state of the art approaches is the usage of Graph Neural Networks on graphs that model the sensor network \cite{bui22}.
The values of a sensor are applied to the same graph node for each timestep which prohibits any non-stationary sensors . 
With fixed sensor locations, the resulting sensor network is highly correlated with the road network.
Streets connecting two intersections with sensors should be also an interesting point for correlations in the sensor network.
However, variable observations and high temporal sparsity rates can not be modeled adequately in a static network.
We show in our experiments that the road network has only a small influence on the traffic predictions.

Besides the traffic prediction for future timesteps, some works explore the field of traffic speed imputation \cite{Cini22, Cuza22} where missing sensor values are predicted.
But the amount of missing values is assumed to be at most 80\%, which on average are still over 40 given sensors in each timestep in the Metr-LA dataset with a total of 207 sensors.
We consider up to 99.9\% missing values which are on average 2.4 observations in each timestep that are used as input.
Such high sparsity rates drastically decrease the chance that multiple values are present in one input sample from the same sensor location, which makes it challenging to recognize and learn temporal correlations for each location on its own.

High sparsity rates (>95\%) result in few sensor values, but if a reconstruction of the traffic state would be possible, we question if spatio-temporal graphs require nodes for each sensor.
In SUSTeR we utilize only a small amount of graph nodes for the encoding of information and do not relate such nodes to the sensor network.
We call this the hidden graph (see figure \ref{fig:intro}), which is still able to reconstruct the complete traffic state.
Due to the reduced number of nodes SUSTeR achieves faster runtimes, as shown in the experiments.
This hidden graph is not embedded directly in the spatial domain, which is why the assignment of observations, as well as the querying of the future traffic, is done with an encoder and a decoder, implemented as neural networks.
The decoding from the hidden graph to future values depends on a set of query locations.
Figure \ref{fig:intro} shows the query locations as given from outside and in combination with the reconstructed traffic state the future values are predicted.

To construct the hidden graph we encode observations from each timestep into from multiple graphs, one for each timestep. 
The graphs are created in a residual style and information is added to the node embeddings from the previous timesteps.
We choose this method to incorporate all timesteps equally into the hidden state because the redundant information along the past is non-existing for high sparsity rates.
From the sequence of graphs where our framework inserted the observations step by step we apply STGCN \cite{Yu18}, an algorithm for traffic prediction to find and learn the spatio-temporal correlations on our small number of graph nodes.
The first future timestep of the STGCN is our hidden graph in which the traffic state is reconstructed. 

% Recent work has an implicit embedding of the graph nodes into the spatial domain as the assignment from the sensor to graph node is fixed one by one.
% Because the graph has the same structure as the road network spatio-temporal correlations can be learned between those sensors.
% We reduce the number of nodes and use a non-linear assignment learned data-driven from the observations.

We find in the experiments that SUSTeR outperforms the plain STGCN and modern traffic prediction frameworks like D2STGNN for high sparsity rates $(\geq 99\%)$.
This is equivalent to only $0.2$ to $2.4$ observation for each timestep on average.
SUSTeR uses fewer parameters than the baselines and can train faster and with less training data.
Our main contributions can be summarized as follows:
\begin{itemize}
    \item We introduce a sparse and unstructured variant of the traffic prediction problem with sparsity in all dimensions. The sensors report only a fraction of their values and are arbitrarily distributed in the spatial domain.
    \item We propose SUSTeR, a framework around the STGCN architecture, which maps sparse observations onto a dense hidden graph to reconstruct the complete traffic state.
    Our code is available at github.\footnote{https://github.com/ywoelker/SUSTeR}
    \item We conducts experiments that show that SUSTeR outperforms the baselines in very sparse situations ($\geq 95\%$) and has a competitive performance in low sparsity rates.
    % \item SUSTeR trains a third faster than the next competitor.
\end{itemize}

\section{LLMs Struggle with Multi-Operand Addition}
\label{sec:acc_data}

In this section, we define the data and models used in this work and demonstrate that LLMs fail on multi-operand additions by looking at prediction accuracy.

\subsection{Models and Data}
\label{sec:models_data}
\paragraph{Models.}
We compare Mistral-7B \cite{jiang_mistral_2023}, Gemma-7B \cite{gemmateam2024gemmaopenmodelsbased} and Meta-Llama-3-8B \cite{grattafiori2024llama3herdmodels, llama3modelcard} as they employ different tokenization strategies for numerical outputs: 
While Mistral and Gemma exclusively employ a single-digit tokenization strategy for their numeric input and generated output (e.g., input = ['1', '4', '7', '+', '2', '5', '5', '='], output = ['4', '0', '2']), Llama-3 employs a multi-digit numeric tokenization strategy (e.g., input = [' 147', ' +', ' 255', ' ='], output = [' 402']), typically favoring numeric tokens of length 3. 
\paragraph{Data.}
For all experiments in this paper, we compile a range of datasets containing simple arithmetic task prompts of the form \textit{147 + 255 = }. We create a dataset for each addition task ranging from 2-operand to 11-operand addition, where each operand is a triple-digit number between 100 and 899. Each of the 10 datasets contains 5,000 unique arithmetic problems, both in a zero-shot and one-shot setting. In the zero-shot setting, an example for a 2-operand addition prompt is ``147 + 255 = ''. An example for a 4-operand addition prompt is ``251 + 613 + 392 + 137 = ''.  Our one-shot prompt template follows the scheme \textit{q1 r1; q2 }, e.g.~``359 + 276 = 635; 147 + 255 = '', where \textit{q1} is a sample query from the same dataset and \textit{r1} is the correct result of the addition task in \textit{q1}. \textit{q2} is the query containing the addition task to be solved.

In the remainder of the paper, we use $s_n$ (with $n\geq 0$) to denote the result digit generated at digit position \(10^n\).
For example, in ``147 + 255 ='', with expected output 402, $s_2 = 4$, $s_1 = 0$, and $s_0 = 2$. 

\subsection{LLM Accuracy on Addition Tasks}
Figure \ref{fig:multi_op_accuracy_overall} illustrates the significant decline in performance of Mistral-7B \cite{jiang_mistral_2023}, Gemma-7B \cite{gemmateam2024gemmaopenmodelsbased} and Meta-Llama-3-8B \cite{llama3modelcard} in multi-operand addition as the number of operands increases. This drastic decrease highlights the inability of these models to generalize effectively to addition tasks involving a higher number of operands, despite their strong overall capabilities. 

\begin{figure}[t]
    \centering
    \includegraphics[width=0.42\textwidth]{Images/figure2.png}
    \caption{Accuracy of Mistral, Gemma and Llama-3 on multi-operand addition of triple-digit numbers, in a zero- and one-shot setting.}
    \label{fig:multi_op_accuracy_overall}
\end{figure}
\section{Probing LLMs on Digits in Two-Operand Addition Tasks}
\label{sec:probing}

Solving arithmetic tasks presents a fundamental challenge for LLMs, as they generate text from left to right, while addition requires a right-to-left process due to carry propagation from the least significant to the most significant digit.
For instance, predicting the first result digit $s_2 = 4$ in ``147 + 255 = '' requires the model to anticipate that a carry originating from $s_0$ cascades through $s_1$ to $s_2$. Robust left-to-right addition thus requires a lookahead spanning all result digits, raising the question: Do LLMs internally represent future result digits when predicting $s_2$ - and if so, how far can they ``look into the future''?

To answer this question, we probe whether models accurately encode future result digits $s_1$ or $s_0$ while generating $s_2$. Building on \citet{levy2024language}, who show that, irrespective of a model's numeric tokenization strategy, LLMs internally represent numbers digit-by-digit in base 10, we analyze digit-wise probing accuracy on the two-operand addition dataset described in Section \ref{sec:models_data}.

\subsection{Methodology and Experiments} 
\paragraph{Data.}
We split the two-operand addition dataset (see Section \ref{sec:models_data}) into train (n=4500) and test (n=500) for the probing experiments. The two-operand addition dataset is designed such that correct results for the addition tasks are triple-digit numbers between 200 and 999. We use the zero-shot prompt setting for the probing experiment.

\paragraph{Probing Setup.}
Our goal is to determine which result digits are available at the prediction step of $s_2$. We thus train probes to predict the result digits $s_2$, $s_1$, and $s_0$ from hidden states of the model during the prediction step of $s_2$. 

Specifically, we train one-layer linear probes to predict individual digit values of the results from the hidden state of the last token at each model layer.  Probes are trained on the train split of the two-operand addition dataset and evaluated on the test split. 
We train separate probes to predict individual result digits $s_2$, $s_1$, and $s_0$, for all models at all layers.\footnote{We choose a low temperature of 0.1 during model inference to ensure deterministic and consistent outputs, reducing randomness in token generation and improving the reliability of numerical calculations.}

\begin{figure}[t]
    \centering
    \includegraphics[width=0.5\textwidth]{Images/figure3.png}
    \caption{Probing accuracy of individual result digits as predicted by the hidden states of Mistral, Gemma and Llama-3. For two-operand, zero-shot addition prompts.}
    \label{fig:probing_multi_op_accuracy_overall}
\end{figure}

\subsection{Results}
\label{subsec:results}
The probing accuracy of individual result digits is shown in Figure \ref{fig:probing_multi_op_accuracy_overall}. Gemma and Mistral with their digit-wise tokenization internally represent only $s_2$ with high accuracy. In contrast, there is a high probing accuracy across \textit{all} result digits in Llama-3. This is due to the fact that Llama-3 tokenizes numbers into 3-digit numeric tokens: It is forced by its tokenization to generate all result digits ($s_2$, $s_1$, and $s_0$) in one step as a single token.

The single-digit tokenization models Mistral and Gemma exhibit a low probing accuracy on $s_0$ ($< 0.24$) in all layers.
Recall that $s_0$ is probed from the models' hidden states while they autoregressively generate $s_2$. 
We interpret the lack of internal representation of $s_0$ as evidence that these models disregard the potential influence of $s_0$ (including any cascading carry) when generating $s_2$.

In line with this, Gemma and Mistral show notably higher probing accuracy on $s_1$ compared to $s_0$, when probing from the models' hidden states as they generate $s_2$. 
We thus conjecture that the single-digit-token models seem to recognize the potential influence of the carry resulting from the sum of the \(10^1\) operand digits. Simply put, generating the digit at $10^2$ might employ a lookahead of one digit to the \(10^1\) intermediate result. 
Based on this observation, we formulate a hypothesis for a heuristic used by LLMs:
\begin{center}
    \textbf{H1: \indent LLMs employ a look ahead of one digit to generate the current digit of an addition task.}
\end{center}

\textbf{H1} would explain why LLMs cannot effectively represent each necessary digit of the result during generation, making it difficult to anticipate later carry values correctly. We first formalize \textbf{H1}, which explains the patterns observed in Figure \ref{fig:probing_multi_op_accuracy_overall}, in the next Section, and then verify the fit of \textbf{H1} with empirical addition outcomes generated by the models in Sections \ref{sec:h1_2op}, \ref{sec:multi_fail}, and \ref{sec:llama}.

\section{The Carry Heuristic of LLMs}
\label{subsec:digit10}

Since LLMs generate numbers from left to right, they must anticipate whether a carry from later digits (with lower bases further on in the result) will impact the current digit they are generating. In this section, we evaluate the maximum accuracy LLMs can achieve in addition tasks, assuming they rely on \textbf{H1}, given the limited lookahead of one digit.

\subsection{Formalization of Left-to-Right Addition in Base 10}
We first formalize a recursive algorithm for solving addition of $k$ operands-where each operand is a base 10 integer- in a left-to-right manner. 

\noindent \textbf{We define:}
\begin{itemize}
    \setlength{\itemsep}{0.2pt}  
    \setlength{\parskip}{0.2pt} 
    \item \( k \): Number of operands.
    \item \( n_1, n_2, \dots, n_k \): Operands, each represented as digit sequences in base \( 10 \), with \(\quad 0 \leq i < d\), where $d$ is the number of digits in the operands: \( n_j = [n_{j, d-1}, \dots, n_{j, 0}], \quad n_{j, i} \in \{0, \dots, 9\}\)
    \item \(S\): The result of the addition. \( S = [s_d, s_{d-1}, \dots s_0] \), where 
    \(s_d = c_d\), i.e., the final carry.
\end{itemize}

\noindent We recursively define the calculation of individual result digits:
\begin{itemize}
    \item \textbf{Total Sum at Digit Position \( i \):}
    \[t_{i} =\sum_{j=1}^k n_{j, i}\]
    \[ T_i = t_i + c_i\]
    where \( t_i \) is the digit sum at the current position, \( c_i \) the carry from the previous digit position, and $k$ the number of operands. Base case: \(c_0 = 0\), no carry at the least significant digit.
    \item \textbf{Result Digit at Position \( i \):}
    \[
    s_i = T_i \mod 10
    \]
    \item \textbf{Carry to the Next Digit Position:}
    \[
    c_{i+1} = \left\lfloor \frac{T_i}{10} \right\rfloor
    \]
\end{itemize}

A worked example is provided in Appendix \ref{appendix:A}.

\subsection{A Naive Heuristic for Solving Addition Left-to-Right}
Due to the recursive nature of left-to-right addition, a lookahead of \(i-1\) digits is needed to determine any result digit $s_i$. 
There is however a simple, non-recursive heuristic for the estimation of $s_i$ with only a one-digit lookahead, to the digit sum of the next position, i.e. only considering $t_{i-1}$. 

We define $c_{min}$ and $c_{max}$ to be the minimal and maximal possible value for a carry, where trivially for all cases, $c_{min}=0$, and 
\[c_{max}(k) = \left\lfloor \frac{\sum_{j=1}^k 9}{10} \right\rfloor\] 
in base $10$ and for $k$ operands. 
We then define the carry heuristic $c_{i}^{h}$ as follows: 
\[c_{i}^{h} \in \{ \left\lfloor \frac{t_{i-1} + c_{min}}{10} \right\rfloor, \left\lfloor \frac{t_{i-1} + c_{max}}{10} \right\rfloor \} \]  
Where $c_{i}^{h}$ is chosen uniformly at random. 
We then accordingly define the predicted total sum at digit position i
\[T_i^h = t_i + c_i^h\] 

and the predicted result digit

\[s_i^h = T_i^h \mod 10\]

\paragraph{Examples.}
We show two examples of two-operand addition, one in which \textbf{H1} is successful, and one in which it fails.
For $k=2$, i.e., in two-operand addition:
\[c_{max}(2) = \left\lfloor \frac{\sum_{j=1}^2 9}{10} \right\rfloor = 1\] 

\paragraph{147 + 293.} See Figure \ref{fig:carry_2_op_success}. We need $T_2^h$ and thus $c_2^h$ to generate the first result digit $s_2^h$. 
\[c_{2}^{h} \in \{\left\lfloor \frac{4 + 9 + c_{min}}{10} \right\rfloor, \left\lfloor \frac{4 + 9 + c_{max}}{10} \right\rfloor \}\]
\[=\{ \left\lfloor \frac{13}{10} \right\rfloor, \left\lfloor \frac{14}{10} \right\rfloor \} = \{1, 1\}\]
therefore $c_{2}^{h} = 1$, $T_2^h = 4$, and $s_2^h = 4$. \textbf{H1} succeeds in predicting the first digit $s_2$ for \textbf{147 + 293}. 

\begin{figure}[t]
    \centering
    \includegraphics[width=0.372\textwidth]{Images/figure4.png} 
    \caption{Two-operand addition in which \textbf{H1} is successful.} 
    \label{fig:carry_2_op_success}
    \vspace{-0.2cm}
\end{figure}

\paragraph{147 + 255.} See Figure \ref{fig:carry_2_op_fail}. \\
\[c_{2}^{h} \in \{ \left\lfloor \frac{4 + 5 + c_{min}}{10} \right\rfloor, \left\lfloor \frac{4 + 5 + c_{max}}{10} \right\rfloor \}\]
\[= \{ \left\lfloor \frac{9}{10} \right\rfloor, \left\lfloor \frac{10}{10} \right\rfloor \} = \{0, 1\}\]
therefore $c_{2}^{h}$ is chosen uniformly at random between $0$ and $1$.
The heuristic fails in predicting the first digit $s_2$ for \textbf{147 + 255} with a 50\% chance. 
\section{H1 Predicts Difficulties of LLMs in Two-Operand Addition}
\label{sec:h1_2op}
In this section we show that single-digit token LLMs struggle exactly in those cases in which the heuristic \textbf{H1} is insufficient. 

\subsection{Predicted Accuracy} For two-operand addition, there are 19 possible values for each $t_i$ (ranging from 0 to 18, because this is the range of sums between two digits). In 18 out of these 19 cases, \textbf{H1} reliably determines the correct carry value. Only if $t_i = 9$, \textbf{H1} must randomly choose between two possible carry values, thus failing with a 50\% chance. This results in an overall predicted accuracy of
\[\frac{18\times1.0 + 1 \times 0.5}{19} = 0.974\]
for the first result digit $s_2$ in two-operand addition: \textbf{H1} achieves 97.4\% accuracy in correctly predicting the first result digit $s_2$. This corresponds almost exactly to Gemma's and Mistral's accuracies for generating $s_2$ during zero-shot and one-shot inference (Gemma: 0-shot: $97.12\%$, 1-shot: $98.04\%$; Mistral: 0-shot: $94.60\%$, 1-shot: $97.46\%$). Table \ref{tab:gen_accuracy_all} in Appendix \ref{sec:appendix_F} provides all generation accuracies for the data described in Section \ref{sec:models_data}.

\begin{figure}[t]
    \centering
    \includegraphics[width=0.37\textwidth]{Images/figure5.png} 
    \caption{Two-operand addition in which \textbf{H1} fails.} 
    \label{fig:carry_2_op_fail}
    \vspace{-0.2cm}
\end{figure}

\subsection{Finegrained Analysis} 
We further investigate whether it is true that especially cases with $t_i=9$ are challenging for LLMs. 

\paragraph{Data.} To this end, we evaluate prediction accuracy across five distinct newly introduced datasets, each containing 100 queries with distinct carry scenarios. The datasets follow the zero-shot template described in Section \ref{sec:models_data} and are designed to exhaustively capture all cases of carries affecting $s_2$ in two-operand addition of triple-digit numbers. 
\begin{itemize}
    \setlength{\itemsep}{0.1pt}
    \setlength{\parskip}{0pt}
    \setlength{\parsep}{0.5pt}
    \item \textbf{Dataset 1 (DS1): No carry.} The addition does not produce any carry (e.g., \(231 + 124 = 355\)).\footnote{We employ the additional constraint that the sum of the \(10^1\) operand digits $\neq 9$, i.e., ($s_1 \neq 9$)}. 
    \item \textbf{Dataset 2 (DS2): Carry in position \(10^0\), no cascading.} A carry is generated in the \(10^0\) ($s_0$) digit but does not cascade to the \(10^2\) ($s_2$) digit (e.g., \(236 + 125 = 361\)). 
    \item \textbf{Dataset 3 (DS3): Cascading carry from \(10^0\) to \(10^2\).} A carry originates in the \(10^0\) ($s_0$) digit and cascades to the \(10^2\) ($s_2$) digit (e.g., \(246 + 155 = 401\)).
    \item \textbf{Dataset 4 (DS4): Direct carry in position \(10^1\).} A carry is generated in the \(10^1\) ($s_1$) digit and directly affects the \(10^2\) ($s_2$) digit (e.g., \(252 + 163 = 415\)).
    \item \textbf{Dataset 5 (DS5): No carry, but position \(10^1\) digits sum to 9.} There is no carry in any digit, but the sum of the \(10^1\) operand digits is 9, i.e., ($s_1 = 9$) (e.g., \(256 + 142 = 398\)).
\end{itemize}
DS1 to DS5 can be neatly categorized according to whether the heuristic can accurately predict $s_2$: 

\begin{itemize}
    \item DS1 and 2: $t_1 = \sum_{j=1}^2 n_{j,1} < 9 \rightarrow c_{2}^{h} = 0$
    \item DS4: $t_1 = \sum_{j=1}^2 n_{j,i} > 9 \rightarrow c_{2}^{h} = 1$ 
    \item DS3 and 5: $t_1 = \sum_{j=1}^2 n_{j,1} = 9 \rightarrow c_{2}^{h} = ?$
\end{itemize}

\begin{figure}[t]
    \centering
    \begin{subfigure}[b]{0.33\textwidth} 
        \centering
        \includegraphics[width=\textwidth]{Images/figure6_a.png} 
    \end{subfigure} 
    \begin{subfigure}[b]{0.09\textwidth}
        \centering
        \includegraphics[width=\textwidth]{Images/figure6_b.png} 
        \vspace{0.8cm}
    \end{subfigure}
    \caption{Per-digit generation accuracy of Mistral and Gemma on datasets DS1-DS5. Each dataset represents a different carry scenario.}
    \label{fig:scenario_mistral_gemma}
\end{figure}

\paragraph{Results.} Figure \ref{fig:scenario_mistral_gemma} shows that  LLMs struggle with DS3 and DS5, which are precisely the cases where \textbf{H1} predicts issues. As \textbf{H1} suggests, predicting the first result digit $s_2$ at position $10^2$ is particularly error-prone in these scenarios. 
The difficult datasets are the ones where a lookahead of one digit position does not suffice to determine the value of the carry needed to generate $s_2$. Simply put:
Overall, addition results tend do be predicted correctly by LLMs, if and only if a lookahead of one digit is sufficient to determine the value of the carry bit affecting $s_2$. Prediction is often incorrect if a lookahead of two or more digits is needed to determine the value of the carry bit affecting $s_2$.

In cases where a lookahead of one digit is enough to accurately determine the value of $s_2$ (DS1, DS2, DS4), the models succeed.
However, when a lookahead of one digit is insufficient to determine the value of $s_2$ (DS3 and DS5), the model struggles with predicting $s_2$ correctly.
Table \ref{tab:accuracy_2-op} in Appendix \ref{appendix:B} provides the generation accuracy of $s_2$ for Gemma and Mistral, in addition to the plot. 
Additionally, Appendix \ref{sec:appendix_G} presents probing experiments that yield the same results.

\section{H1 Predicts the Deterioration of Accuracy in Multi-Operand Addition}
\label{sec:multi_fail}

As shown in the last section, \textbf{H1} is a good approximator for LLM behaviour on two-operand addition: In the majority of cases, a lookahead of one digit is sufficient to accurately determine the value of the carry bit affecting $s_2$. With a look-ahead of one digit, \textbf{H1} predicts a failure of the generation of $s_2$, if and only if the value of $s_1$ does not suffice to determine the value of the carry bit. 
In two-operand addition in base 10, this is the case if and only if $t_1 = 9$.
We now show that \textbf{H1} can also account for model performance on \textit{multi}-operand addition. 

\subsection{Multi-Operand Performance Predicted by H1}

The possible value of a carry increases with increasing numbers of operands. 
For instance in 4-operand addition ($k=4$) the maximal value of a carry is $3$:
\[c_{max}(4) = \left\lfloor \frac{\sum_{j=1}^4 9}{10} \right\rfloor = 3\] 

Therefore the carry heuristic \(c_{i}^{h}\) is unreliable in 4-operand addition whenever \(t_{i-1} =\sum_{j=1}^k n_{j,i-1} \in \{7, 8, 9, 17, 18, 19, 27, 28, 29\} \).

Put simply, because the value of the carry can be larger for more operands,  \textbf{the proportion of values of \(s_1\) for which the heuristic is insufficient (with its lookahead of one) increases with an increasing number of operands}. 

Consider an example in which the heuristic fails in 4-operand addition for clarification (see Figure \ref{fig:carry_4_op_fail} in Appendix \ref{appendix:C}): 

\noindent\textbf{186 + 261 + 198 + 256.}
\[
\begin{split}
    t_{1} =8 + 6 + 9 + 5 = 28\\
    c_{2}^{h} \in \{ \left\lfloor \frac{c_{min} + 28}{10} \right\rfloor,\\
    \left\lfloor \frac{c_{max} + 28}{10} \right\rfloor \}
\end{split}
\]

with \(c_{max} = 3\)
\[c_{2}^{h} \in \{ \left\lfloor \frac{28}{10} \right\rfloor, \left\lfloor \frac{31}{10} \right\rfloor \} = \{2, 3\}\]
therefore $c_{2}^{h}$ is chosen uniformly at random between $2$ and $3$.
The heuristic thus fails in solving \textbf{186 + 261 + 198 + 256} with a chance of 50\%. 


For 4-operand addition, there are 37 possible sums for the second digits (ranging from 0 to 36). In 28 out of these 37 cases, the heuristic reliably determines the correct carry bit. However, when \(t_1 \in \{7, 8, 9, 17, 18, 19, 27, 28, 29\}\), the heuristic must randomly choose between two possible carry values, leading to a 50\% chance of selecting the correct one. This results in an overall accuracy of:
\[\frac{28\times1.0 + 9 \times 0.5}{37} = 0.878\]
Thus, the heuristic only achieves 88\% accuracy in correctly predicting the first result digit $s_2$ in 4-operand addition, compared to the 97\% accuracy in two-operand addition. 
In Appendix \ref{appendix:D}, we provide exact values for $s_2$ accuracy as predicted by \textbf{H1}, for addition tasks between 2 and 11 operands. 


\begin{figure}[t]
    \centering
    \includegraphics[width=0.45\textwidth]{Images/figure7.png}
    \caption{Accuracy of first generated result digit $s_d$ in one-shot multi-operand addition for Mistral and Gemma, compared to the expected accuracy based on \textbf{H1}.}
    \label{fig:multi_op_accuracy}
    \vspace{-0.2cm}
\end{figure}


\subsection{Empricial Evidence on Multi-Operand Addition}
Intuitively, according to \textbf{H1}, Mistral and Gemma with their one-digit tokenization should fail at multi-operand addition at a certain rate: The amount of instances in which a lookahead of one digit is sufficient to accurately predict $s_i$ gets smaller and smaller because the carry bit value can get larger and larger for multiple operands. 
We test if \textbf{H1} holds in predicting the first generated digit $s_d$ in Mistral and Gemma for multiple operands. We evaluate prediction accuracy on the multi-operand datasets described in Section \ref{sec:models_data}.
\textbf{H1} should provide an upper bound for the performance of LLMs\footnote{Autoregressive LLMs with single-digit tokenization of numbers.} for predicting the first result digit $s_d$.
Figure \ref{fig:multi_op_accuracy} shows that \textbf{H1} is a good predictor for the accuracy of the one-shot\footnote{Results for the zero-shot setting are in Appendix \ref{sec:appendix_E}.} generation of the first result digit $s_d$ by Mistral and Gemma. We take this as further evidence that these LLMs make use of \textbf{H1}.
\section{Multi-Digit Tokenization Models Employ the Same Heuristic}
\label{sec:llama}
While \citet{levy2024language} demonstrate that all LLMs, regardless of the tokenization strategy, internally represent numbers as individual digits, it remained unclear whether models with multi-digit tokenization also rely on a one-digit lookahead when generating addition results. In this section, we show that perhaps surprisingly multi-digit tokenization models, such as Llama-3, also employ a lookahead of one \textbf{digit} when predicting carry bits. 
To show this, we design 3 controlled datasets that force the multi-digit tokenization model Llama-3 to generate results across multiple tokens. 

\paragraph{Experimental Setup.}
To examine whether Llama-3 employs a one-digit lookahead, we use six-digit numbers in two-operand addition (e.g., ``231234 + 124514 = ''), where each operand is tokenized into two three-digit tokens by the model's tokenizer, such as: [`` 231'',`` 234'', `` +'', `` 124'', `` 514'', `` =''] and the result is generated as two triple-digit tokens as well, in this example [`` 355'', `` 748'']. The first generated triple-digit token $s_5 s_4 s_3$ corresponds to digit base positions $10^5$, $10^4$, and $10^3$. If Llama-3 did employ \textbf{H1} it would look ahead to digit position $10^2$, but ignore digit positions $10^1$ and $10^0$, as they fall outside the lookahead window.

\paragraph{Carry Scenarios.}
We evaluate model behavior in three datasets with six-digit operands (ranging from 100,000 to 899,999) and results between 200,000 and 999,999. We use a zero-shot prompt template. Each dataset consist of 100 samples:
\begin{itemize}
    \setlength{\itemsep}{0.1pt}
    \setlength{\parskip}{0pt}
    \setlength{\parsep}{0.5pt}
    \item \textbf{DS6: No carry.} The addition does not produce any carry and no digits sum to 9.  (e.g., \(111,234 + 111,514 = 222,748\)).
    \item \textbf{DS7: Direct carry in position \(10^2\).} A carry is generated at \(10^2\) and directly affects \(10^3\) (e.g., \(111,721 + 111,435 = 223,156\)). 
    \item \textbf{DS8: Cascading carry from \(10^1\) to \(10^3\).} A carry originates at \(10^1\), cascades to \(10^2\) and then affects \(10^3\) (e.g., \(111,382 + 111,634 = 223,016\)).
\end{itemize}

\paragraph{Expected Outcomes.}
If Llama-3 employs \textbf{H1}, we expect that DS6 should be easy, as no carry propagation is required. DS7 should also be easy, since the carry affecting \(10^3\) is within the one-digit lookahead window. DS8 in contrast should be challenging, as the carry originates from \(10^1\), from beyond the model’s lookahead range. We expect a lower accuracy in generating \(10^3\), the result digit that is affected by the potentially inaccurate carry.


\begin{figure}[t]
    \centering
    \includegraphics[width=0.4\textwidth]{Images/figure8.png} 
    \caption{Per-digit generation accuracy of Llama on datasets DS6-DS8. Each dataset represents a different carry scenario.} 
    \label{fig:llama_carry_scenarios}
\end{figure}

\paragraph{Results.}
Figure \ref{fig:llama_carry_scenarios} shows that Llama-3 exhibits the expected pattern predicted by \textbf{H1}. The sharp drop in accuracy in dataset DS8 on digit \(10^3\) provides evidence that Llama-3, regardless of its multi-digit tokenization strategy, relies on the same one-digit lookahead for solving addition left to right. 
\section{Related Work}
\label{sec:related-work}
Recent work has benchmarked the arithmetic capabilities of LLMs using text-based evaluations and handcrafted tests \cite{yuan2023well,lightman2023lets,NEURIPS2023_58168e8a,zhuang2023efficiently}. Numerous studies consistently show that LLMs struggle with arithmetic tasks \cite{nogueira2021investigatinglimitationstransformerssimple, qian2022limitationslanguagemodelsarithmetic, dziri2023faithfatelimitstransformers, yu2024metamathbootstrapmathematicalquestions}. 

\citet{zhou2023algorithmstransformerslearnstudy} and \citet{zhou2024transformersachievelengthgeneralization} examine transformers' ability to learn algorithmic procedures and find challenges in length generalization \cite{anil2022exploringlengthgeneralizationlarge}. Similarly, \citet{xiao2024theorylengthgeneralizationlearning} propose a theoretical explanation for LLMs' difficulties with length generalization in arithmetic. \citet{gambardella2024language} find that LLMs can reliably predict the first digit in multiplication but struggle with subsequent digits.

The focus of research has recently shifted from mere benchmarking of LLMs to trying to understand \textit{why} LLMs struggle with arithmetic reasoning. Using circuit analysis, \citet{stolfo_mechanistic_2023} and \citet{hanna2023doesgpt2computegreaterthan} explore internal processing in arithmetic tasks, while \citet{nikankin2024arithmetic} reveal that LLMs use a variety of heuristics managed by identifiable circuits and neurons. In contrast, \citet{deng2024language} argue that LLMs rely on symbolic pattern recognition rather than true numerical computation. Recently, \citet{kantamneni2025languagemodelsusetrigonometry} showed that LLMs represent numbers as generalized helixes and perform addition using a “Clock” algorithm \cite{nanda_progress_2023}.

Related work has also examined how LLMs encode numbers. \citet{levy2024language} demonstrate that numbers are represented digit-by-digit, extending \citet{gould2023successor}, who find that LLMs encode numeric values modulo 10. \citet{zhu-etal-2025-language} suggest that numbers are encoded linearly, while \citet{marjieh2025number} indicate that number representations can blend string-like and numerical forms.

Another line of research explores how tokenization influences arithmetic capabilities. \citet{lee2024digitstodecisions} show that single-digit tokenization outperforms other methods in simple arithmetic tasks. \citet{singh2024tokenization} highlight that right-to-left (R2L) tokenization—where tokens are right-aligned—improves arithmetic performance. 
Additionally, the role of embeddings and positional encodings is emphasized by \citet{mcleish2024transformers}, who demonstrate that suitable embeddings enable transformers to learn arithmetic, and by \citet{shen2023positionaldescriptionmatterstransformers}, who show that positional encoding improves arithmetic performance.

\section{Conclusion}

Our study shows that LLMs, regardless of their numeric tokenization strategy, rely on a simple one-digit lookahead heuristic for anticipating carries when performing addition tasks. While this strategy is fairly effective for two-operand additions, it fails in the multi-operand additions due to the increasingly unpredictable value of cascading carry bits. Through probing experiments and targeted evaluations of digit-wise result accuracy, we demonstrate that model accuracy deteriorates precisely at the rate the heuristic predicts. 

These findings highlight an inherent weakness in current LLMs that prevents them from robustly generalizing to more complex arithmetic tasks. 

Our work contributes to a broader understanding of LLM limitations in arithmetic reasoning and highlights increasing LLMs' lookahead as a promising approach to enhancing their ability to handle complex numerical tasks.

\section*{Limitations}
Our work highlights limited lookahead as a key challenge for LLMs when adding multiple numbers. However, it remains unclear whether this limitation extends to other arithmetic operations, such as subtraction. Additionally, we cannot determine whether the limited lookahead is a heuristic explicitly learned for arithmetic tasks, or if it could also affect general language generation tasks as thus hinder performance of other tasks that require long-range dependencies. Future work should explore the depth of lookahead in tasks beyond arithmetic.

While the lookahead heuristic offers a straightforward explanation for the upper performance limit of LLMs on addition, it does not fully account for why LLMs still somewhat underperform relative to the heuristic in addition tasks with many operands (e.g., adding 8–11 numbers). We suspect this discrepancy may be related to limited training exposure to these many-operand addition tasks, but further investigation is needed to confirm this.

Our work also does not address whether larger models within the same family (e.g., 70B parameter models) exhibit a deeper lookahead. Future studies should examine whether scaling model size leads to improved performance by enabling a deeper lookahead.

Finally, we do not tackle methods to overcome the shallow lookahead. Future work should investigate whether targeted training on tasks requiring deeper lookahead can encourage models to deepen their lookahead.

%\section*{Ethics Statement}

\section*{Acknowledgements}
We thank Patrick Schramowski for his helpful feedback on the paper draft. This work has been supported by the German Ministry of Education and Research (BMBF) as part of the project TRAILS (01IW24005).


% Entries for the entire Anthology, followed by custom entries
\bibliography{custom, custom1}
\bibliographystyle{acl_natbib}

\appendix
\section{Example Addition According to Formalization}
\label{appendix:A}
We show a concrete example for two-operand addition according to the formalization defined in Section \ref{subsec:digit10}. For \textbf{$147 + 255$}, we have:

$k=2, d=3, n1 = [1, 4, 7], n2 = [2, 5, 5]$. 

We then compute:

\[T_2 = c_2 + 1 + 2\] \[T_1 = c_1 + 4 + 5\] \[T_0 = c_0 + 7 + 5= 0 + 7 + 5 = 12\] \[s_0 = 12 \mod 10 = 2, \quad c_1 = \left\lfloor \frac{12}{10} \right\rfloor = 1\] \[T_1 = 1 + 4 + 5 = 10\]  \[s_1 = 10 \mod 10 = 0, \quad c_2 = \left\lfloor \frac{10}{10} \right\rfloor = 1\] \[T_2 = 1 + 1 + 2 = 4\]  \[s_2 = 4 \mod 10 = 4, \quad c_3 = \left\lfloor \frac{4}{10} \right\rfloor = 0\] \[S = [0, 4, 0, 2]\] 


The result of the addition is $402$. 
\chapter{\textcolor{black}{Edge Network optimization}}\label{app: EN_ib}

In this section the mathematical solution of the optimization problem \eref{eq: EN_ib initial opt problem} in \sref{sec: EN_ib} reported below:

\begin{mini}|s|[0]
    {\mathbf{\Psi}(t)}{\lim_{T \to +\infty}\; \frac{1}{T} \sum_{t=1}^T  \mathbb{E}[P^{tot}(t)] }
    {}{}
    \addConstraint{\lim_{T \to +\infty}\; \frac{1}{T} \sum_{t=1}^T  \mathbb{E}[D_k^{tot}(t)] \leq D_k^{avg}\qquad \forall k }{}
    \addConstraint{ \lim_{T \to +\infty}\; \frac{1}{T} \sum_{t=1}^T  \mathbb{E}[G_k(t)] \leq G_k^{avg}\qquad \forall k }{}
    \addConstraint{0 \leq f_k(t) \leq f_k^{max} \qquad \forall k,t }{}
    \addConstraint{0 \leq R_k(t) \leq R_k^{max}(t) \qquad \forall k,t }{}
    \addConstraint{\beta_k(t) \in \mathcal{B}_k  \qquad \forall k,t}{}
    \addConstraint{0 \leq f^{es}(t) \leq f_{es}^{max} \qquad \forall t}{}
    \addConstraint{f_k^{es}(t) \geq 0 \quad \forall k,t}, \qquad {\sum_{k=1}^K f_k^{es}(t) \leq f_c(t)  \quad \forall t,}{}
\end{mini}

These virtual queues associated to the long-term delay and evaluation metric constraints, $T_k(t)$ and $U_k(t)$ respectively are introduced as follows \cite{Neely2010Lyapunov}:
\begin{align}
    T_k(t+1) &= \max [0,T_k(t) + \varepsilon_k(D_k^{tot}(t) - D_k^{avg})] \\
    U_k(t+1) &= \max [0,U_k(t) + \nu_k(G_k(t) - G_k^{avg})],  
\end{align}
where $\epsilon_k$ and $ \nu_k $ are the learning rate for the update of the virtual queues. 

Based on these virtual queues is possible to define the \textit{Lyapunov function} $L(\mathbf{\Theta}(t))$ as:
\begin{equation}
    L(\mathbf{\Theta}(t)) = \frac{1}{2} \sum_{k=1}^K T_k^2(t) + U_k^2(t),
    \tag{\ref{eq: EN_ib Lyapunov function}}
    \label{app: EN_ib Lyapunov function}
\end{equation}
where $\mathbf{\Theta}(t) = [\{T_k(t)\}_k, \{U_k(t)\}_k]$ is the vector composed by all the virtual queues at time $t$. The idea is to use this Lyapunov function to satisfy the constraints on $D_k^{avg}$ and $G_k^{avg}$ by enforcing the stability of $L(\mathbf{\Theta}(t))$. 

To this scope it is introduced the so called \textit{drift-plus-penalty function}:
\begin{align}
    \Delta(\Theta(t)) &= \mathbb{E}\left[L({\Theta}(t+1))-L({\Theta}(t))+V\cdot P^{tot}(t)  \;\Big|\; \Theta(t)\right] \\
    &=\mathbb{E}\left[\;\sum_{k=1}^K \frac{T_k^2(t+1)-T_k^2(t)}{2} +  \frac{U_k^2(t+1)-U_k^2(t)}{2} +V\cdot P^{tot}(t)\;\; \Big|\;\; \Theta(t)\right]\\
    &= \mathbb{E}\left[\;\sum_{k=1}^K \Delta_{T_k} +  \Delta_{U_k} +V\cdot P^{tot}(t) \;\; \Big|\;\; \Theta(t)\right],
    \label{app: EN_ib drift plus penalty}
\end{align}
where, starting from a generic virtual queue evolving as 
$H(t+1) = \max [0,H(t) +h(t) - \Bar{h}]$ the quantity $\Delta_H$ is defined as follows:
\begin{align*}
    \Delta_H &= \frac{H^2(t+1)-H^2(t)}{2} = \frac{\max [0,(H(t) +h(t) - \Bar{h})^2]-H^2(t)}{2} \\
   &\leq   \frac{(h(t) - \Bar{h})^2}{2} + H(t)[h(t)-\Bar{h}].
\end{align*} 

By applying the same upper bound to $\Delta_{T_k}$ it is possible to obtain:
\begin{align}
    \Delta_{T_k} &= \frac{T_k^2(t+1)-T_k^2(t)}{2} = \frac{\max [0,(T_k(t) + \nu_k(D_k^{tot}(t) - D_k^{avg}))^2]-T_k^2(t)}{2} \\
    &\leq   \nu_k^2\frac{(D_k^{tot}(t) - D_k^{avg})^2}{2} + \nu_k T_k(t)[D_k^{tot}(t) - D_k^{avg}] \\
    &\leq \nu_k^2\frac{(D_k^{max} - D_k^{avg})^2}{2}  + \nu_k T_k(t)[D_k(t) - D_k^{avg}],
    \label{app: EN_ib delta U_k}
\end{align}
where $D_k^{max}(t)$ is the maximum delay allowed for the $k$-th \gls{ed}.

By applying the same reasoning to $\Delta_{U_k}$ it is possible to obtain:
\begin{equation}
    \Delta_{U_k} \leq \nu_k^2\frac{(G_k^{max} - G_k^{avg})^2}{2}  + \nu_k U_k(t)[G_k(t) - G_k^{avg}],
    \label{app: EN_ib delta U_k}
\end{equation}
where $G_k^{max}(t)$ is the maximum value allowed for the evaluation metric for the $k$-th \gls{ed}.

Substituting now \eref{app: EN_ib delta U_k} and \eqref{app: EN_ib delta U_k} inside \eref{app: EN_ib drift plus penalty} and rearranging the terms it is possible to obtain:

\begin{align}
    \Delta_p(\Theta(t)) &\leq
    \sum_{k=1}^K \Bigg{[} \nu_k^2\frac{(D_k^{max} - D_k^{avg})^2}{2} + \nu_k^2\frac{(G_k^{max}(t) - G_k^{avg})^2}{2}  \Bigg{]}  \\ &\;\;\;
    + \mathbb{E} \Bigg{[}\;\sum_{k=1}^K \Big{[} - \varepsilon_k Z_k(t)Q_k^{avg} - \nu_k S_k(t)G_k^{avg}   + \Big|\;\; \Theta(t) \Bigg{]} \\ &\;\;\; + \mathbb{E} \Bigg{[}\;\sum_{k=1}^K \Big{[} \varepsilon_k Z_k(t)Q_k^{tot}(t)  +  \nu_k S_k(t)G_k(t) \Big{]} + V\cdot P^{tot}\;\; \Big|\;\; \Theta(t) \Bigg{]}, 
\end{align}
where some constants that have been taken out of the expected value (first line), while others even if within the expected value do not depend on the optimization parameters (second line).

Pivoting therefore on the Lyapunov optimization it is possible to neglect all these terms. Moreover, it is possible to remove the expected value to obtain the following per-slot optimization:

\begin{mini}|s|[0]
    {\mathbf{\Psi}(t)}{\sum_{k=1}^K \bigg[ \frac{\epsilon_kT_k(t)N_k(t)}{R_k(t)} + \frac{\epsilon_kT_k(t)W_k(t)}{f_k(t)\rho_k } + \frac{\epsilon_kT_k(t)W_{max}^{es}}{f_k^{es}(t) \rho_k^{es}}+}{}{} \breakObjective{\qquad +  \frac{B_k N_0}{h_k(t)} {\rm exp} \left(\frac{R_k(t) ln(2)}{B_k} \right) + V \Gamma_k \eta_k (f_k(t))^3 +}{}{} \breakObjective{\;+  V \eta (f_c(t))^3 + \nu_k U_k(t)G_k(t)\bigg]}{}{}
    \addConstraint{\mathbf{\Psi}(t) \in \mathcal{T}(t),}{}
    \label{eq: EN_ib per-slot opt problem structure}
\end{mini}
where $\mathcal{T}(t)$ indicates the space of possible solutions given by the constraints on the optimization variables. 

at this point it is possible to split the problem for the resource allocation at the \gls{ed} and at the \gls{es}.

\section{Edge Device optimization}\label{app: EN_ib ed opt}
The sub-problem for the \gls{ed} as defined in \eref{eq: EN_ib per-slot opt ed} can be split in two further sub-problems for the transmission rate $R_k(t)$ and the clock frequency $f_k(t)$.

\subsection*{Transmission rate optimal solution}
The sub-problem associated to the transmission rate $R_k(t)$ can be defined as follows:
\begin{mini}|s|[0]
    {R_k(t)}{\frac{\epsilon_kT_k(t)N_k(t)}{R_k(t)} +  V \frac{B_k N_0}{h_k(t)} {\rm exp} \left(\frac{R_k(t) ln(2)}{B_k} \right) }{}{}
    \addConstraint{0 \leq R_k(t) \leq R_k^{max}(t)}{} 
\end{mini}

To simplify the notation, define:
\[
A = \epsilon_k T_k(t) N_k(t), \quad B = V \dfrac{B_k N_0}{h_k(t)}, \quad C = \dfrac{\ln(2)}{B_k}.
\]

Computing the derivative of the objective function $J(R_k(t))$ with respect to $R_k(t)$and set it to zero it is possible to obtain:
\[
\frac{dJ}{dR_k(t)} = -\dfrac{A}{[R_k(t)]^2} + B C \exp\left( C R_k(t) \right) = 0.
\]

By defining Let $x = C R_k(t)$ and $d = \dfrac{A C}{B}$ the derivative can be rearranged as:
\[
x e^{\frac{x}{2}} = \sqrt{d}.
\]

Fortunately, there is an exact solution to this problem and it is based on the \textit{Lambert W function}. By applying the definition and substituting back all the terms it is possible to obtain the final solution:
\begin{equation}
    R_k^*(t) = \frac{2 B_k}{ln(2)}\; W\! \!\left(\sqrt{\frac{\epsilon_k T_k(t)\; ln(2)\; h_k(t)N_k(t)\; }{4 B_k^2\;V \;N_0}}\right)\; \Biggr|_0^{R_k^{max}(t)}
\end{equation}

\subsection*{Clock frequency optimal solution}
The sub-problem associated to the transmission rate $R_k(t)$ can be defined as follows:
\begin{mini}|s|[0]
    {f_k(t)}{\frac{\epsilon_k T_k(t)W_k(t)}{f_k(t)\rho_k } +  V \Gamma_k \eta_k (f_k(t))^3 }{}{}
    \addConstraint{0 \leq f_k(t) \leq f_k^{max}}{} 
\end{mini}

To simplify the notation define:
\[
A = \dfrac{\epsilon_k T_k(t) W_k(t)}{\rho_k}, \quad B = V \Gamma_k \eta_k
\]

Computing the derivative of the objective function $J(f_k(t))$ with respect to $f_k(t)$ and set it to zero it is possible to obtain:
\[
\frac{dJ}{df_k(t)} = -\dfrac{A}{[f_k(t)]^2} + 3B [f_k(t)]^2 = 0
\]

After multiply both sides by $[f_k(t)]^2$, rearranging the terms and substituting back  $A$ and $B$ the final solution is:
\[
f_k(t) = \left( \dfrac{A}{3B} \right)^{1/4} \; \Biggr|_0^{f_k^{max}} \implies f_k^* (t) = \sqrt[4]{\frac{\epsilon_k T_k(t) W_k(t)}{3 V \Gamma_k \eta_k \rho_k} }\; \Biggr|_0^{f_k^{max}},
\]


\section{Edge Server optimization}\label{app: EN_ib es opt}


\begin{mini}|s|[0]
    {\{f_f^{es}(t)\}_k, f_c(t)}{\sum_{k=1}^K \frac{\epsilon_k T_k(t)W_{max}^{es}}{f_k^{es}(t)\rho_k^{es}} + V \eta (f_c(t))^3 }{}{}
    \addConstraint{0 \leq f_c(t) \leq f_c^{max} }{}
    \addConstraint{f_k^{es}(t) \geq 0 \quad \forall k}, \qquad {\sum_{k=1}^K f_k^{es}(t) \leq f_c(t)}{}
\end{mini}

Define:
\[
A_k = \dfrac{ \epsilon_k T_k(t) W_{\text{max}}^{es} }{ \rho_k^{es} }, \quad B = V \eta, \quad S = \sum_{k=1}^K \sqrt{ A_k }
\]


The objective function becomes:
\[
J(\{f_k^{es}(t)\}_k,\ f_c(t)) = \sum_{k=1}^K \dfrac{A_k}{f_k^{es}(t)} + B [f_c(t)]^3
\]

As a first step it is possible to define the associated Lagrangian $L$ of the sub-problem with respect to  $f_k^{es}(t)$ given $f_c(t)$ as:
\[
L = \sum_{k=1}^K \dfrac{A_k}{f_k^{es}(t)} + \lambda \left( \sum_{k=1}^K f_k^{es}(t) - f_c(t) \right)
\]

By deriving it and isolating with respect to $f_k^{es}(t)$ it is possible to obtain:

Solve for $f_k^{es}(t)$:
\[
    \frac{\partial L}{\partial f_k^{es}(t)} = -\dfrac{A_k}{[f_k^{es}(t)]^2} + \lambda = 0  \implies [f_k^{es}(t)]^2 = \dfrac{A_k}{\lambda} \implies f_k^{es}(t) = \sqrt{ \dfrac{A_k}{\lambda} }
\]

Apply the coupling constraint on $f_c(t)$ and by solving for $\lambda$ it is possible to identify:
\[
\sum_{k=1}^K f_k^{es}(t) = \dfrac{1}{\sqrt{\lambda}} \sum_{k=1}^K \sqrt{ A_k } = f_c(t) \implies \sqrt{\lambda} = \dfrac{ S }{ f_c(t) } \implies \lambda = \left( \dfrac{ S }{ f_c(t) } \right)^2
\]

Therefore:
\[
f_k^{es}(t) = \dfrac{ \sqrt{ A_k } }{ S } f_c(t)
\]

This term can now be substituted back into the objective function that is then derived with respect to $f_c(t)$ and set to zero as:

\[
J(f_c(t)) = \dfrac{ S^2 }{ f_c(t) } + B [f_c(t)]^3 \implies \frac{dJ}{df_c(t)} = - \dfrac{ S^2 }{ [f_c(t)]^2 } + 3 B [f_c(t)]^2 = 0
\]

By solving for $f_c(t)$, substituting back the expressions of $A$, $B$ and $S$ and applying the constraints it is possible to obtain the final solution:
\[
    f_c^*(t) = \left[ \left( \dfrac{ S^2 }{ 3 B } \right)^{1/4} \right]_0^{f_c^{\text{max}}}  = \frac{\sqrt{\sum_{k=1}^K \sqrt{\frac{\epsilon_k T_k(t)W_{max}^{es}}{\rho_k^{es}}}}}{\sqrt[4]{3V\eta}} \; \Biggr|_0^{f_{c}^{max}}
\]


Therefore, for every $k$:
\[
f_k^{es}(t) = \dfrac{ \sqrt{ A_k } }{ S } f_c^*(t) = f_k^{es*}(t) = \frac{\sqrt{\frac{\epsilon_k T_k(t)W_{max}^{es}}{\rho_k^{es}}}}{\sqrt{\sum_{k=1}^K \sqrt{\frac{\epsilon_k T_k(t)W_{max}^{es}}{\rho_k^{es}}}}\sqrt[4]{3V\eta} }
\]

\section{Example: H1 Failure on 4-Operand Addition}
\label{appendix:C}
Below is an example in which the heuristic \textbf{H1} fails in 4-operand addition, visualized in Figure \ref{fig:carry_4_op_fail}: 

\noindent\textbf{186 + 261 + 198 + 256.}
\[
\begin{split}
    t_{1} =8 + 6 + 9 + 5 = 28\\
    c_{2}^{h} \in \{ \left\lfloor \frac{c_{min} + 28}{10} \right\rfloor,\\
    \left\lfloor \frac{c_{max} + 28}{10} \right\rfloor \}
\end{split}
\]

with \(c_{max} = 3\)
\[c_{2}^{h} \in \{ \left\lfloor \frac{28}{10} \right\rfloor, \left\lfloor \frac{31}{10} \right\rfloor \} = \{2, 3\}\]
therefore $c_{2}^{h}$ is chosen uniformly at random between $2$ and $3$.
The heuristic thus fails in solving \textbf{186 + 261 + 198 + 256} with a chance of 50\%. 

\begin{figure}[ht]
    \centering
    \includegraphics[width=0.5\textwidth]{Images/figure9.png} 
    \caption{4-operand addition in which \textbf{H1} fails.} 
    \label{fig:carry_4_op_fail}
\end{figure}
\section{Zero-shot Generation Accuracy}
\label{sec:appendix_E}

We test if \textbf{H1} holds up in predicting the generation accuracy on $s_d$ of Mistral and Gemma for multiple operands. Figure \ref{fig:s2_zero-shot} shows that \textbf{H1} provides an upper bound for the generation accuracy of $s_d$ in a zero-shot setting for Mistral and Gemma on $s_d$.

\begin{figure}[h]
    \centering
    \includegraphics[width=0.45\textwidth]{Images/figure10.png} 
    \caption{Accuracy of first generated result digit $s_d$ in zero-shot multi-operand addition tasks for Mistral and Gemma, compared to the expected accuracy on $s_d$ based on \textbf{H1}.} 
    \label{fig:s2_zero-shot}
\end{figure}
\section{Accuracy Prediction of Heuristic}
\label{appendix:D}

\renewcommand{\arraystretch}{1.2}
\setlength{\tabcolsep}{5pt}
\begin{table*}[ht]
\centering
\begin{tabular}{|c|c|c|c|c|}
\hline
Nr. Operands $k$ & \textbf{\(c_{max}(k)\)} & Values of \(t_i\) in which H1 fails & Expected acc. on \(s_d\)\\ \hline
2 & 1 & 1 fail:= 9 & $\frac{18\times1.0 + 1 \times 0.5}{19} = 0.974$ \\ \hline
3 & 2 & 4 fails:= 8, 9, 18, 19 & $\frac{24\times1.0 + 4 \times 0.5}{28} = 0.928$\\ \hline
4 & 3 & 9 fails:= 7, 8, 9, 17, 18, 19, 27, 28, 29 & $\frac{28\times1.0 + 9 \times 0.5}{37} = 0.878$ \\ \hline
5 & 4 & 16 fails:= 6, 7, 8, 9, 16, ..., 39 & $\frac{ 30 \times1.0 + 16 \times 0.5}{46} = 0.826$\\ \hline
6 & 5 & 25 fails:= 5, 6, 7, 8, 9, 15, ..., 49 & $\frac{ 30 \times1.0 + 25 \times 0.5}{55} = 0.773$\\ \hline
7 & 6 & 36 fails:= 4, 5, 6, ..., 59 & $\frac{ 28 \times1.0 + 36 \times 0.5}{64} = 0.719$\\ \hline
8 & 7 & 49 fails:= 3, 4, 5, ..., 69 & $\frac{ 24 \times1.0 + 49 \times 0.5}{73} = 0.664$\\ \hline
9 & 8 & 64 fails:= 2, 3, 4, ..., 79 & $\frac{ 18 \times1.0 + 64 \times 0.5}{82} = 0.610$\\ \hline
10 & 9 & 81 fails:= 1, 2, 3, ..., 89 & $\frac{ 10 \times1.0 + 81 \times 0.5}{91} = 0.555$\\ \hline
11 & 9 & 89 fails:= 1, 2, 3, ..., 99 & $\frac{ 10 \times1.0 + 90 \times 0.5}{100} = 0.540$\\ \hline
%12 & 108 & 10 & 109 fails:= 0, 1, 2, ..., 108 & /109 = 0.1 \\ \hline
\end{tabular}
\caption{Predicted accuracy on the first result digit $s_d$ in the addition of multiple numbers according to \textbf{H1}.}
\label{tab:heuristic}
\end{table*}

Table \ref{tab:heuristic} contains, for addition tasks with different numbers of operands $k$, the maximum value of the carry \(c_{max}(k)\). Based on \(c_max\) it list those values of \(t_i\) in which \textbf{H1} is insufficient to accurately predict \(s_2\). Based on the proportion of values of \(t_i\) for which \textbf{H1} is sufficient to the total number of possible values, it lists the predicted accuracy for \(s_2\).

\newpage
\section{Generation Accuracy on All Datasets}
\label{sec:appendix_F}
See Table \ref{tab:gen_accuracy_all}.

\begin{sidewaystable*}[!ht]
\centering
\small
\renewcommand{\arraystretch}{1.0}
\setlength{\tabcolsep}{10pt}
\begin{tabular}{|c|c|c|c|c|c|c|c|c|c|c|c|c|}
\hline
Operands & \multicolumn{4}{c|}{Mistral} & \multicolumn{4}{c|}{Gemma} & \multicolumn{4}{c|}{Llama} \\ \hline
Setting & Overall & $s_2$ & $s_1$ & $s_0$ & Overall & $s_2$ & $s_1$ & $s_0$ & Overall & $s_2$ & $s_1$ & $s_0$ \\ \hline
\multicolumn{13}{|l|}{\textbf{Zero-shot}} \\ \hline
2 & 0.934 & 0.946 & 0.942 & 0.954 & 0.965 & 0.971 & 0.974 & 0.980 & 0.992 & 0.993 & 0.999 & 0.989 \\ \hline
3 & 0.649 & 0.878 & 0.776 & 0.789 & 0.664 & 0.851 & 0.766 & 0.753 & 0.955 & 0.990 & 0.962 & 0.992 \\ \hline
4 & 0.417 & 0.792 & 0.596 & 0.640 & 0.376 & 0.834 & 0.602 & 0.583 & 0.595 & 0.917 & 0.635 & 0.959 \\ \hline
5 & 0.204 & 0.723 & 0.406 & 0.494 & 0.151 & 0.804 & 0.437 & 0.399 & 0.259 & 0.824 & 0.316 & 0.878 \\ \hline
6 & 0.055 & 0.643 & 0.219 & 0.329 & 0.021 & 0.661 & 0.207 & 0.213 & 0.110 & 0.749 & 0.163 & 0.795 \\ \hline
7 & 0.007 & 0.507 & 0.101 & 0.228 & 0.002 & 0.357 & 0.100 & 0.110 & 0.058 & 0.712 & 0.120 & 0.691 \\ \hline
8 & 0.002 & 0.436 & 0.071 & 0.157 & 0.000 & 0.158 & 0.105 & 0.101 & 0.037 & 0.660 & 0.103 & 0.565 \\ \hline
9 & 0.001 & 0.296 & 0.088 & 0.137 & 0.000 & 0.116 & 0.103 & 0.103 & 0.023 & 0.606 & 0.104 & 0.426 \\ \hline
10 & 0.000 & 0.161 & 0.108 & 0.112 & 0.000 & 0.105 & 0.100 & 0.098 & 0.017 & 0.586 & 0.101 & 0.334  \\ \hline
11 & 0.000 & 0.054 & 0.105 & 0.106 & 0.000 & 0.122 & 0.097 & 0.099 & 0.009  & 0.570 & 0.113 & 0.265 \\ \hline
\multicolumn{13}{|l|}{\textbf{One-shot}} \\ \hline
2 & 0.965 & 0.975 & 0.969 & 0.987 & 0.980 & 0.981 & 0.984 & 0.992 & 0.999 & 0.999 & 1.000 & 1.000 \\ \hline
3 & 0.739 & 0.938 & 0.840 & 0.808 & 0.790 & 0.965 & 0.875 & 0.842 & 0.978 & 0.993 & 0.984 & 0.998 \\ \hline
4 & 0.437 & 0.814 & 0.615 & 0.657 & 0.545 & 0.907 & 0.700 & 0.728 & 0.599 & 0.893 & 0.637 & 0.981 \\ \hline
5 & 0.155 & 0.708 & 0.313 & 0.506 & 0.245 & 0.839 & 0.483 & 0.544 & 0.251 & 0.778 & 0.287 & 0.950 \\ \hline
6 & 0.053 & 0.689 & 0.207 & 0.348 & 0.048 & 0.779 & 0.244 & 0.313 & 0.106 & 0.664 & 0.151 & 0.859 \\ \hline
7 & 0.011 & 0.585 & 0.109 & 0.219 & 0.010 & 0.750 & 0.151 & 0.153 & 0.051 & 0.594 & 0.107 & 0.738 \\ \hline
8 & 0.004 & 0.523 & 0.085 & 0.137 & 0.002 & 0.639 & 0.106 & 0.107 & 0.033 & 0.500 & 0.103 & 0.581 \\ \hline
9 & 0.001 & 0.488 & 0.086 & 0.103 & 0.000 & 0.493 & 0.102 & 0.105 & 0.023 & 0.478 & 0.106 & 0.427 \\ \hline
10 & 0.001 & 0.401 & 0.086 & 0.103 & 0.000 & 0.324 & 0.099 & 0.095 & 0.011 & 0.451 & 0.099 & 0.263 \\ \hline
11 & 0.001 & 0.294 & 0.102 & 0.106 & 0.000 & 0.288 & 0.100 & 0.100 & 0.005 & 0.420 & 0.101 & 0.179 \\ \hline
\end{tabular}
\caption{Zero-shot and One-shot settings: Per-digit and overall generation accuracy for all multi-operand addition datasets and models described in Section \ref{sec:models_data}.}
\label{tab:gen_accuracy_all}
\end{sidewaystable*}


\section{MIMIC-III Experiment Results} 
\label{sec:MIMIC_Results} 

Below is all supplementary material for the MIMIC-III experiment. This includes BoP distribution plots and plots showing how incomprehensiveness and sufficiency change over the number of features removed. 




\subsection{Experiment Plots}\label{subsec:mimic_plots}
In the following section, we show supplementary plots for the regression task on the auditing dataset. We show the distribution of the BoP across participants for all three metrics we evaluate. We overlay Laplace and Gaussian distributions to see which fit the individual BoP distribution best, illustrating that prediction and incomprehensiveness are best fit by Laplace distributions and sufficiency by a Gaussian distribution. Additionally, we show how incomprehensiveness and sufficiency change for the number of important attributes $r$ that are kept are removed.

\begin{figure}[ht]
    \centering
    \includegraphics[width=0.8\textwidth]{Figures/mimic_regr_pred_histogram.png}
    \caption{Individual prediction cost for all groups using the square error loss function.}
    \label{fig:mimic-regr-pred}
\end{figure}

\begin{figure}[ht]
    \centering
    \includegraphics[ width=0.8\textwidth]{Figures/mimic_regr_incomp_histogram.png}
    \caption{Individual incomprehensiveness cost for all groups using the square error loss function.}
    \label{fig:mimic-regr-incomp}
\end{figure}

\begin{figure}[ht]
    \centering

    \includegraphics[width=0.84\textwidth]{Figures/mimic_regr_sufficiency_histogram.png}
    \caption{Individual sufficiency cost for all groups using the square error loss function.}
    \label{fig:mimic-regr-suff}
\end{figure}

\begin{figure}[ht]
    \centering
    \includegraphics[width=1\textwidth]{Figures/mimic_varying_r_values.pdf}
    \caption{Values of Sufficiency and Incomprehensiveness across varying $r$ top features selected using the square error loss function. Values are found for $h_0$ and $h_p$.}
    \label{fig:mimic-r-varying}
\end{figure}


\end{document}
