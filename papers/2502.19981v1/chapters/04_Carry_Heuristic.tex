\section{The Carry Heuristic of LLMs}
\label{subsec:digit10}

Since LLMs generate numbers from left to right, they must anticipate whether a carry from later digits (with lower bases further on in the result) will impact the current digit they are generating. In this section, we evaluate the maximum accuracy LLMs can achieve in addition tasks, assuming they rely on \textbf{H1}, given the limited lookahead of one digit.

\subsection{Formalization of Left-to-Right Addition in Base 10}
We first formalize a recursive algorithm for solving addition of $k$ operands-where each operand is a base 10 integer- in a left-to-right manner. 

\noindent \textbf{We define:}
\begin{itemize}
    \setlength{\itemsep}{0.2pt}  
    \setlength{\parskip}{0.2pt} 
    \item \( k \): Number of operands.
    \item \( n_1, n_2, \dots, n_k \): Operands, each represented as digit sequences in base \( 10 \), with \(\quad 0 \leq i < d\), where $d$ is the number of digits in the operands: \( n_j = [n_{j, d-1}, \dots, n_{j, 0}], \quad n_{j, i} \in \{0, \dots, 9\}\)
    \item \(S\): The result of the addition. \( S = [s_d, s_{d-1}, \dots s_0] \), where 
    \(s_d = c_d\), i.e., the final carry.
\end{itemize}

\noindent We recursively define the calculation of individual result digits:
\begin{itemize}
    \item \textbf{Total Sum at Digit Position \( i \):}
    \[t_{i} =\sum_{j=1}^k n_{j, i}\]
    \[ T_i = t_i + c_i\]
    where \( t_i \) is the digit sum at the current position, \( c_i \) the carry from the previous digit position, and $k$ the number of operands. Base case: \(c_0 = 0\), no carry at the least significant digit.
    \item \textbf{Result Digit at Position \( i \):}
    \[
    s_i = T_i \mod 10
    \]
    \item \textbf{Carry to the Next Digit Position:}
    \[
    c_{i+1} = \left\lfloor \frac{T_i}{10} \right\rfloor
    \]
\end{itemize}

A worked example is provided in Appendix \ref{appendix:A}.

\subsection{A Naive Heuristic for Solving Addition Left-to-Right}
Due to the recursive nature of left-to-right addition, a lookahead of \(i-1\) digits is needed to determine any result digit $s_i$. 
There is however a simple, non-recursive heuristic for the estimation of $s_i$ with only a one-digit lookahead, to the digit sum of the next position, i.e. only considering $t_{i-1}$. 

We define $c_{min}$ and $c_{max}$ to be the minimal and maximal possible value for a carry, where trivially for all cases, $c_{min}=0$, and 
\[c_{max}(k) = \left\lfloor \frac{\sum_{j=1}^k 9}{10} \right\rfloor\] 
in base $10$ and for $k$ operands. 
We then define the carry heuristic $c_{i}^{h}$ as follows: 
\[c_{i}^{h} \in \{ \left\lfloor \frac{t_{i-1} + c_{min}}{10} \right\rfloor, \left\lfloor \frac{t_{i-1} + c_{max}}{10} \right\rfloor \} \]  
Where $c_{i}^{h}$ is chosen uniformly at random. 
We then accordingly define the predicted total sum at digit position i
\[T_i^h = t_i + c_i^h\] 

and the predicted result digit

\[s_i^h = T_i^h \mod 10\]

\paragraph{Examples.}
We show two examples of two-operand addition, one in which \textbf{H1} is successful, and one in which it fails.
For $k=2$, i.e., in two-operand addition:
\[c_{max}(2) = \left\lfloor \frac{\sum_{j=1}^2 9}{10} \right\rfloor = 1\] 

\paragraph{147 + 293.} See Figure \ref{fig:carry_2_op_success}. We need $T_2^h$ and thus $c_2^h$ to generate the first result digit $s_2^h$. 
\[c_{2}^{h} \in \{\left\lfloor \frac{4 + 9 + c_{min}}{10} \right\rfloor, \left\lfloor \frac{4 + 9 + c_{max}}{10} \right\rfloor \}\]
\[=\{ \left\lfloor \frac{13}{10} \right\rfloor, \left\lfloor \frac{14}{10} \right\rfloor \} = \{1, 1\}\]
therefore $c_{2}^{h} = 1$, $T_2^h = 4$, and $s_2^h = 4$. \textbf{H1} succeeds in predicting the first digit $s_2$ for \textbf{147 + 293}. 

\begin{figure}[t]
    \centering
    \includegraphics[width=0.372\textwidth]{Images/figure4.png} 
    \caption{Two-operand addition in which \textbf{H1} is successful.} 
    \label{fig:carry_2_op_success}
    \vspace{-0.2cm}
\end{figure}

\paragraph{147 + 255.} See Figure \ref{fig:carry_2_op_fail}. \\
\[c_{2}^{h} \in \{ \left\lfloor \frac{4 + 5 + c_{min}}{10} \right\rfloor, \left\lfloor \frac{4 + 5 + c_{max}}{10} \right\rfloor \}\]
\[= \{ \left\lfloor \frac{9}{10} \right\rfloor, \left\lfloor \frac{10}{10} \right\rfloor \} = \{0, 1\}\]
therefore $c_{2}^{h}$ is chosen uniformly at random between $0$ and $1$.
The heuristic fails in predicting the first digit $s_2$ for \textbf{147 + 255} with a 50\% chance. 