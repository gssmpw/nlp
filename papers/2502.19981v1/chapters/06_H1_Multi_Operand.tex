\section{H1 Predicts the Deterioration of Accuracy in Multi-Operand Addition}
\label{sec:multi_fail}

As shown in the last section, \textbf{H1} is a good approximator for LLM behaviour on two-operand addition: In the majority of cases, a lookahead of one digit is sufficient to accurately determine the value of the carry bit affecting $s_2$. With a look-ahead of one digit, \textbf{H1} predicts a failure of the generation of $s_2$, if and only if the value of $s_1$ does not suffice to determine the value of the carry bit. 
In two-operand addition in base 10, this is the case if and only if $t_1 = 9$.
We now show that \textbf{H1} can also account for model performance on \textit{multi}-operand addition. 

\subsection{Multi-Operand Performance Predicted by H1}

The possible value of a carry increases with increasing numbers of operands. 
For instance in 4-operand addition ($k=4$) the maximal value of a carry is $3$:
\[c_{max}(4) = \left\lfloor \frac{\sum_{j=1}^4 9}{10} \right\rfloor = 3\] 

Therefore the carry heuristic \(c_{i}^{h}\) is unreliable in 4-operand addition whenever \(t_{i-1} =\sum_{j=1}^k n_{j,i-1} \in \{7, 8, 9, 17, 18, 19, 27, 28, 29\} \).

Put simply, because the value of the carry can be larger for more operands,  \textbf{the proportion of values of \(s_1\) for which the heuristic is insufficient (with its lookahead of one) increases with an increasing number of operands}. 

Consider an example in which the heuristic fails in 4-operand addition for clarification (see Figure \ref{fig:carry_4_op_fail} in Appendix \ref{appendix:C}): 

\noindent\textbf{186 + 261 + 198 + 256.}
\[
\begin{split}
    t_{1} =8 + 6 + 9 + 5 = 28\\
    c_{2}^{h} \in \{ \left\lfloor \frac{c_{min} + 28}{10} \right\rfloor,\\
    \left\lfloor \frac{c_{max} + 28}{10} \right\rfloor \}
\end{split}
\]

with \(c_{max} = 3\)
\[c_{2}^{h} \in \{ \left\lfloor \frac{28}{10} \right\rfloor, \left\lfloor \frac{31}{10} \right\rfloor \} = \{2, 3\}\]
therefore $c_{2}^{h}$ is chosen uniformly at random between $2$ and $3$.
The heuristic thus fails in solving \textbf{186 + 261 + 198 + 256} with a chance of 50\%. 


For 4-operand addition, there are 37 possible sums for the second digits (ranging from 0 to 36). In 28 out of these 37 cases, the heuristic reliably determines the correct carry bit. However, when \(t_1 \in \{7, 8, 9, 17, 18, 19, 27, 28, 29\}\), the heuristic must randomly choose between two possible carry values, leading to a 50\% chance of selecting the correct one. This results in an overall accuracy of:
\[\frac{28\times1.0 + 9 \times 0.5}{37} = 0.878\]
Thus, the heuristic only achieves 88\% accuracy in correctly predicting the first result digit $s_2$ in 4-operand addition, compared to the 97\% accuracy in two-operand addition. 
In Appendix \ref{appendix:D}, we provide exact values for $s_2$ accuracy as predicted by \textbf{H1}, for addition tasks between 2 and 11 operands. 


\begin{figure}[t]
    \centering
    \includegraphics[width=0.45\textwidth]{Images/figure7.png}
    \caption{Accuracy of first generated result digit $s_d$ in one-shot multi-operand addition for Mistral and Gemma, compared to the expected accuracy based on \textbf{H1}.}
    \label{fig:multi_op_accuracy}
    \vspace{-0.2cm}
\end{figure}


\subsection{Empricial Evidence on Multi-Operand Addition}
Intuitively, according to \textbf{H1}, Mistral and Gemma with their one-digit tokenization should fail at multi-operand addition at a certain rate: The amount of instances in which a lookahead of one digit is sufficient to accurately predict $s_i$ gets smaller and smaller because the carry bit value can get larger and larger for multiple operands. 
We test if \textbf{H1} holds in predicting the first generated digit $s_d$ in Mistral and Gemma for multiple operands. We evaluate prediction accuracy on the multi-operand datasets described in Section \ref{sec:models_data}.
\textbf{H1} should provide an upper bound for the performance of LLMs\footnote{Autoregressive LLMs with single-digit tokenization of numbers.} for predicting the first result digit $s_d$.
Figure \ref{fig:multi_op_accuracy} shows that \textbf{H1} is a good predictor for the accuracy of the one-shot\footnote{Results for the zero-shot setting are in Appendix \ref{sec:appendix_E}.} generation of the first result digit $s_d$ by Mistral and Gemma. We take this as further evidence that these LLMs make use of \textbf{H1}.