\section{H1 Predicts Difficulties of LLMs in Two-Operand Addition}
\label{sec:h1_2op}
In this section we show that single-digit token LLMs struggle exactly in those cases in which the heuristic \textbf{H1} is insufficient. 

\subsection{Predicted Accuracy} For two-operand addition, there are 19 possible values for each $t_i$ (ranging from 0 to 18, because this is the range of sums between two digits). In 18 out of these 19 cases, \textbf{H1} reliably determines the correct carry value. Only if $t_i = 9$, \textbf{H1} must randomly choose between two possible carry values, thus failing with a 50\% chance. This results in an overall predicted accuracy of
\[\frac{18\times1.0 + 1 \times 0.5}{19} = 0.974\]
for the first result digit $s_2$ in two-operand addition: \textbf{H1} achieves 97.4\% accuracy in correctly predicting the first result digit $s_2$. This corresponds almost exactly to Gemma's and Mistral's accuracies for generating $s_2$ during zero-shot and one-shot inference (Gemma: 0-shot: $97.12\%$, 1-shot: $98.04\%$; Mistral: 0-shot: $94.60\%$, 1-shot: $97.46\%$). Table \ref{tab:gen_accuracy_all} in Appendix \ref{sec:appendix_F} provides all generation accuracies for the data described in Section \ref{sec:models_data}.

\begin{figure}[t]
    \centering
    \includegraphics[width=0.37\textwidth]{Images/figure5.png} 
    \caption{Two-operand addition in which \textbf{H1} fails.} 
    \label{fig:carry_2_op_fail}
    \vspace{-0.2cm}
\end{figure}

\subsection{Finegrained Analysis} 
We further investigate whether it is true that especially cases with $t_i=9$ are challenging for LLMs. 

\paragraph{Data.} To this end, we evaluate prediction accuracy across five distinct newly introduced datasets, each containing 100 queries with distinct carry scenarios. The datasets follow the zero-shot template described in Section \ref{sec:models_data} and are designed to exhaustively capture all cases of carries affecting $s_2$ in two-operand addition of triple-digit numbers. 
\begin{itemize}
    \setlength{\itemsep}{0.1pt}
    \setlength{\parskip}{0pt}
    \setlength{\parsep}{0.5pt}
    \item \textbf{Dataset 1 (DS1): No carry.} The addition does not produce any carry (e.g., \(231 + 124 = 355\)).\footnote{We employ the additional constraint that the sum of the \(10^1\) operand digits $\neq 9$, i.e., ($s_1 \neq 9$)}. 
    \item \textbf{Dataset 2 (DS2): Carry in position \(10^0\), no cascading.} A carry is generated in the \(10^0\) ($s_0$) digit but does not cascade to the \(10^2\) ($s_2$) digit (e.g., \(236 + 125 = 361\)). 
    \item \textbf{Dataset 3 (DS3): Cascading carry from \(10^0\) to \(10^2\).} A carry originates in the \(10^0\) ($s_0$) digit and cascades to the \(10^2\) ($s_2$) digit (e.g., \(246 + 155 = 401\)).
    \item \textbf{Dataset 4 (DS4): Direct carry in position \(10^1\).} A carry is generated in the \(10^1\) ($s_1$) digit and directly affects the \(10^2\) ($s_2$) digit (e.g., \(252 + 163 = 415\)).
    \item \textbf{Dataset 5 (DS5): No carry, but position \(10^1\) digits sum to 9.} There is no carry in any digit, but the sum of the \(10^1\) operand digits is 9, i.e., ($s_1 = 9$) (e.g., \(256 + 142 = 398\)).
\end{itemize}
DS1 to DS5 can be neatly categorized according to whether the heuristic can accurately predict $s_2$: 

\begin{itemize}
    \item DS1 and 2: $t_1 = \sum_{j=1}^2 n_{j,1} < 9 \rightarrow c_{2}^{h} = 0$
    \item DS4: $t_1 = \sum_{j=1}^2 n_{j,i} > 9 \rightarrow c_{2}^{h} = 1$ 
    \item DS3 and 5: $t_1 = \sum_{j=1}^2 n_{j,1} = 9 \rightarrow c_{2}^{h} = ?$
\end{itemize}

\begin{figure}[t]
    \centering
    \begin{subfigure}[b]{0.33\textwidth} 
        \centering
        \includegraphics[width=\textwidth]{Images/figure6_a.png} 
    \end{subfigure} 
    \begin{subfigure}[b]{0.09\textwidth}
        \centering
        \includegraphics[width=\textwidth]{Images/figure6_b.png} 
        \vspace{0.8cm}
    \end{subfigure}
    \caption{Per-digit generation accuracy of Mistral and Gemma on datasets DS1-DS5. Each dataset represents a different carry scenario.}
    \label{fig:scenario_mistral_gemma}
\end{figure}

\paragraph{Results.} Figure \ref{fig:scenario_mistral_gemma} shows that  LLMs struggle with DS3 and DS5, which are precisely the cases where \textbf{H1} predicts issues. As \textbf{H1} suggests, predicting the first result digit $s_2$ at position $10^2$ is particularly error-prone in these scenarios. 
The difficult datasets are the ones where a lookahead of one digit position does not suffice to determine the value of the carry needed to generate $s_2$. Simply put:
Overall, addition results tend do be predicted correctly by LLMs, if and only if a lookahead of one digit is sufficient to determine the value of the carry bit affecting $s_2$. Prediction is often incorrect if a lookahead of two or more digits is needed to determine the value of the carry bit affecting $s_2$.

In cases where a lookahead of one digit is enough to accurately determine the value of $s_2$ (DS1, DS2, DS4), the models succeed.
However, when a lookahead of one digit is insufficient to determine the value of $s_2$ (DS3 and DS5), the model struggles with predicting $s_2$ correctly.
Table \ref{tab:accuracy_2-op} in Appendix \ref{appendix:B} provides the generation accuracy of $s_2$ for Gemma and Mistral, in addition to the plot. 
Additionally, Appendix \ref{sec:appendix_G} presents probing experiments that yield the same results.
