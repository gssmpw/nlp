\documentclass{article}

% Recommended, but optional, packages for figures and better typesetting:
\usepackage{microtype}
\usepackage{graphicx}
\usepackage{subfigure}
\usepackage{booktabs} % for professional tables

% hyperref makes hyperlinks in the resulting PDF.
% If your build breaks (sometimes temporarily if a hyperlink spans a page)
% please comment out the following usepackage line and replace
% \usepackage{icml2025} with \usepackage[nohyperref]{icml2025} above.
\usepackage{hyperref}


% Attempt to make hyperref and algorithmic work together better:
\newcommand{\theHalgorithm}{\arabic{algorithm}}

% Use the following line for the initial blind version submitted for review:
% \usepackage{icml2025}

% If accepted, instead use the following line for the camera-ready submission:
\usepackage[accepted]{icml2025}

% For theorems and such
\usepackage{amsmath}
\usepackage{amssymb}
\usepackage{mathtools}
\usepackage{amsthm}
\usepackage{dashbox}
\newcommand\dboxed[1]{\dbox{\ensuremath{#1}}}
%%%%%%%%%%%%
% \usepackage{hyperref}
\usepackage{url}
\usepackage[english]{babel}
\usepackage{ulem}
\usepackage{wrapfig}
\usepackage{booktabs}       % professional-quality tables
% \usepackage{amsfonts}       % blackboard math symbols
\usepackage{nicefrac}       % compact symbols for 1/2, etc.
\usepackage{microtype}      % microtypography
% \usepackage{xcolor}         % colors
% \definecolor{lightblue}{rgb}{0.68, 0.85, 0.9}
% \usepackage{amsmath}
% \usepackage[pdftex]{graphicx}
\usepackage{soul}
\usepackage{multirow}
\usepackage{colortbl}
\sethlcolor{LightBlue2}
\usepackage{mathrsfs}
\usepackage{amsmath,amsfonts,amssymb,amsthm}
\usepackage{resizegather}
\newcommand{\new}{\marginpar{NEW}}
\usepackage{placeins}
\usepackage{subcaption}
\usepackage{adjustbox}


% if you use cleveref..
\usepackage[capitalize,noabbrev]{cleveref}
% \pdfobjcompresslevel=2

\newcommand{\del}[1]{{\color{red}[#1]}}
%%%%%%%%%%%%%%%%%%%%%%%%%%%%%%%%
% THEOREMS
%%%%%%%%%%%%%%%%%%%%%%%%%%%%%%%%
\theoremstyle{plain}
\newtheorem{theorem}{Theorem}[section]
\newtheorem{proposition}[theorem]{Proposition}
\newtheorem{lemma}[theorem]{Lemma}
\newtheorem{corollary}[theorem]{Corollary}
\theoremstyle{definition}
\newtheorem{definition}[theorem]{Definition}
\newtheorem{assumption}[theorem]{Assumption}
\theoremstyle{remark}
\newtheorem{remark}[theorem]{Remark}
\newcommand{\numMale}{823}
\newcommand{\numParticipants}{1573}
\newcommand{\numTotalParticipants}{2,673}
\newcommand{\numFemale}{715}
\newcommand{\numOther}{35}
\newcommand{\numControl}{248}
\newcommand{\numBaseline}{240}
\newcommand{\numReorder}{269}
\newcommand{\numTreatment}{816}
\newcommand{\numExcluded}{6}

% Todonotes is useful during development; simply uncomment the next line
%    and comment out the line below the next line to turn off comments
\usepackage[disable,textsize=tiny]{todonotes}
% \usepackage[textsize=tiny]{todonotes}


% The \icmltitle you define below is probably too long as a header.
% Therefore, a short form for the running title is supplied here:
\icmltitlerunning{EigenLoRAx}

\begin{document}

\twocolumn[
% \icmltitle{EigenAug: Enhancing Resource-Efficient Adaptation with Augmented Principal Subspaces}
\icmltitle{EigenLoRAx: Recycling Adapters to Find Principal Subspaces for Resource-Efficient Adaptation and Inference}

% It is OKAY to include author information, even for blind
% submissions: the style file will automatically remove it for you
% unless you've provided the [accepted] option to the icml2025
% package.

% List of affiliations: The first argument should be a (short)
% identifier you will use later to specify author affiliations
% Academic affiliations should list Department, University, City, Region, Country
% Industry affiliations should list Company, City, Region, Country

% You can specify symbols, otherwise they are numbered in order.
% Ideally, you should not use this facility. Affiliations will be numbered
% in order of appearance and this is the preferred way.
\icmlsetsymbol{equal}{*}

\begin{icmlauthorlist}
\icmlauthor{Prakhar Kaushik*}{yyy}
\icmlauthor{Ankit Vaidya*}{yyy}
\icmlauthor{Shravan Chaudhari}{yyy}
\icmlauthor{Alan Yuille}{yyy}
% \icmlauthor{Firstname2 Lastname2}{equal,yyy,comp}
% \icmlauthor{Firstname3 Lastname3}{comp}
% \icmlauthor{Firstname4 Lastname4}{sch}
% \icmlauthor{Firstname5 Lastname5}{yyy}
% \icmlauthor{Firstname6 Lastname6}{sch,yyy,comp}
% \icmlauthor{Firstname7 Lastname7}{comp}
% %\icmlauthor{}{sch}
% \icmlauthor{Firstname8 Lastname8}{sch}
% \icmlauthor{Firstname8 Lastname8}{yyy,comp}
%\icmlauthor{}{sch}
%\icmlauthor{}{sch}
\end{icmlauthorlist}

\icmlaffiliation{yyy}{Department of Computer Science, Johns Hopkins University, Baltimore, USA}
% \icmlaffiliation{comp}{Company Name, Location, Country}
% \icmlaffiliation{sch}{School of ZZZ, Institute of WWW, Location, Country}

\icmlcorrespondingauthor{Prakhar Kaushik}{pkaushi1@jh.edu}
% \icmlcorrespondingauthor{Firstname2 Lastname2}{first2.last2@www.uk}

% You may provide any keywords that you
% find helpful for describing your paper; these are used to populate
% the "keywords" metadata in the PDF but will not be shown in the document
\icmlkeywords{Machine Learning, ICML}

\vskip 0.3in
]

% this must go after the closing bracket ] following \twocolumn[ ...

% This command actually creates the footnote in the first column
% listing the affiliations and the copyright notice.
% The command takes one argument, which is text to display at the start of the footnote.
% The \icmlEqualContribution command is standard text for equal contribution.
% Remove it (just {}) if you do not need this facility.

% \printAffiliationsAndNotice{}  % leave blank if no need to mention equal contribution
\printAffiliationsAndNotice{\icmlEqualContribution} % otherwise use the standard text.

\begin{abstract}
\begin{abstract}


The choice of representation for geographic location significantly impacts the accuracy of models for a broad range of geospatial tasks, including fine-grained species classification, population density estimation, and biome classification. Recent works like SatCLIP and GeoCLIP learn such representations by contrastively aligning geolocation with co-located images. While these methods work exceptionally well, in this paper, we posit that the current training strategies fail to fully capture the important visual features. We provide an information theoretic perspective on why the resulting embeddings from these methods discard crucial visual information that is important for many downstream tasks. To solve this problem, we propose a novel retrieval-augmented strategy called RANGE. We build our method on the intuition that the visual features of a location can be estimated by combining the visual features from multiple similar-looking locations. We evaluate our method across a wide variety of tasks. Our results show that RANGE outperforms the existing state-of-the-art models with significant margins in most tasks. We show gains of up to 13.1\% on classification tasks and 0.145 $R^2$ on regression tasks. All our code and models will be made available at: \href{https://github.com/mvrl/RANGE}{https://github.com/mvrl/RANGE}.

\end{abstract}


\end{abstract}

\section{Introduction}
Recent advancements in machine learning have driven the rise of large-scale models with billions of parameters. However, the size and complexity of these models not only make it impractical for most researchers to train or fine-tune them on downstream tasks but also contribute significantly to their carbon footprint, raising concerns about environmental sustainability.
To address these challenges, there has been growing interest in parameter-efficient finetuning (PEFT) methods, such as adapters~\citep{pmlr-v97-houlsby19a, Chen2022AdaptFormerAV, luo2023towards}, low rank adaptation (LoRA) methods~\citep{hu2021lora,kopiczko_vera_2023,dora}, prompt-based methods~\citep{tune1, tune2, tune3}.
LoRA and its follow-up works~\citep{meng_pissa_2024,dora} have gained significant attention for their simplicity. This has fueled the proliferation of thousands of low-rank adapters within the growing open-source community.
Given that these adapters are underutilized, an important question arises: Can we recycle the information contained in them to improve the efficiency of subsequent tasks?
Recent work has shown that weight updates in deep neural networks occur within low-dimensional invariant subspaces~\citep{kwon2024efficientcompressionoverparameterizeddeep}, aligning with the universality hypothesis that neural network behavior and learned representations often reside in shared, structured subspaces~\cite{chughtai2023toymodeluniversalityreverse, guth2024on}. This suggests that LoRA adapters may similarly share a \textit{principal subspace} that can be reused, eliminating the need to rediscover it during the training of new adapters.

We introduce \textbf{EigenLoRAx}, a parameter-efficient fine-tuning (PEFT) method that leverages this insight by decomposing the weights of a set of trained adapters into principal components, identifying a compact, information-dense subspace. EigenLoRAx reduces the number of learnable parameters by up to $\mathbf{100\times}$ compared to LoRA, accelerates optimization by up to $\mathbf{2\times}$ for new adapters, and enables more memory-efficient inference with multiple task adapters, particularly benefiting edge devices~\citep{edge}. Additionally, in low-resource domains, we demonstrate that EigenLoRAx can be further enhanced by augmenting the principal subspace with random components, orthogonalized with respect to the existing subspace, preserving its efficiency while retaining performance.

Furthermore, we provide an initial theoretical analysis of EigenLoRAx.
Our experiments across a wide range of vision and language tasks demonstrate its versatility and effectiveness, reinforcing the potential of shared subspaces in neural network adaptation.

\autoref{fig:fig1} provides an overview of our method. We introduce \textbf{EigenLoRAx} (ELoRAx), which recycles pretrained adapters by identifying a shared \textit{task-invariant} weight subspace. We hypothesize (and validate experimentally) that task-specific weights lie within this subspace, allowing for more efficient training with fewer parameters. This reduces memory footprint and enhances inference efficiency by enabling simultaneous serving of multiple adapters. EigenLoRAx is among the first to recycle pretrained adapters, replacing many while improving further training efficiency. Our key contributions are as follows:

\begin{itemize}
    \item \textbf{(Training)}: EigenLoRAx uses up to $\mathbf{100\times}$ \textbf{fewer parameters than LoRA} and converges up to $\mathbf{2\times}$ \textbf{faster} than comparable methods, achieving similar or better performance.
    \item \textbf{(Inference)}: EigenLoRAx enhances \textbf{memory efficiency during inference} by approximately $\mathbf{18\times}$ on multiple tasks, reducing the number of switchable parameters between tasks.
    \item \textbf{(Applicability)}: We empirically demonstrate the effectiveness of EigenLoRAx across a wide range 
    , including text and image data, validating the existence of shared principal subspaces across modalities. It also retains performance in \textbf{zero-shot} and \textbf{low resource} scenarios.
    \item \textbf{(Scaling)}: EigenLoRAx can be scaled up to recycle hundreds of underutilized pretrained adapters.
\end{itemize}
\begin{figure*}[!hbt]
\begin{center}
\includegraphics[width=\textwidth]{images/elorax.pdf}
\caption{\small{LoRA uses low-rank matrices for task-specific finetuning. We observe that LoRA adapters share a principal subspace across task domains. By recycling pretrained adapters, we extract \textit{task-invariant} principal components, enabling efficient representation of both existing and future LoRAs using compact \textit{task-specific} coefficients. This improves training speed, parameter efficiency, and memory usage. In low-resource settings, where pretrained adapters are scarce, we augment the subspace with randomly initialized components, ensuring orthogonality via the Gram-Schmidt process, ensuring they complement the extracted subspace without redundancy.}}
\label{fig:fig1}
\end{center}
\end{figure*}
\section{Related Works}
\section{Related Work}
\label{sec:relatedwork}

\subsection{Current AI Tools for Social Service}
\label{subsec:relatedtools}
% the title I feel is quite broad

Harnessing technology for social good has always been a grand challenge in social service \cite{berzin_practice_2015}. As early as the 90s, artificial neural networks and predictive models have been employed as tools for risk assessments, decision-making, and workload management in sectors like child protective services and mental health treatment \cite{fluke_artificial_1989, patterson_application_1999}. The recent rise of generative AI is poised to further advance social service practice, facilitating the automation of administrative tasks, streamlining of paperwork and documentation, optimisation of resource allocation, data analysis, and enhancing client support and interventions \cite{fernando_integration_2023, perron_generative_2023}.

Today, AI solutions are increasingly being deployed in both policy and practice \cite{goldkind_social_2021, hodgson_problematising_2022}. In clinical social work, AI has been used for risk assessments, crisis management, public health initiatives, and education and training for practitioners \cite{asakura_call_2020, gillingham2019can, jacobi_functions_2023, liedgren_use_2016, molala_social_2023, rice_piloting_2018, tambe_artificial_2018}. AI has also been employed for mental health support and therapeutic interventions, with conversational agents serving as on-demand virtual counsellors to provide clinical care and support \cite{lisetti_i_2013, reamer_artificial_2023}.
% commercial solutions include Woebot, which simulates therapeutic conversation, and Wysa, an “emotionally intelligent” AI coach, powered by evidenced-based clinical techniques \cite{reamer_artificial_2023}. 
% Non-clinical AI agents like Replika and companion robots can also provide social support and reduce loneliness amongst individuals \cite{ahmed_humanrobot_2024, chaturvedi_social_2023, pani_can_2024, ta_user_2020}.

Present research largely focuses on \textit{\textbf{AI-based decision support tools}} in social service \cite{james_algorithmic_2023, kawakami2022improving}, especially predictive risk models (PRMs) used to predict social service risks and outcomes \cite{gillingham2019can, van2017predicting}, like the Allegheny Family Screening Tool (AFST), which assesses child abuse risk using data from US public systems \cite{chouldechova_case_2018, vaithianathan2017developing}. Elsewhere, researchers have also piloted PRMs to predict social service needs for the homeless using Medicaid data\cite{erickson_automatic_2018, pourat_easy_2023}, and AI-powered algorithms to promote health interventions for at-risk populations, such as HIV testing among Californian homeless \cite{rice_piloting_2018, yadav_maximizing_2017}.

\subsection{Generative AI and Human-AI Collaboration}
\label{subsec:relatedworkhaicollaboration}
Beyond decision-making algorithms and PRMs, advancements in generative AI, such as large language models (LLMs), open new possibilities for human-AI (HAI) collaboration in social services. 
LLMs have been called "revolutionary" \cite{fui2023generative} and a "seismic shift" \cite{cooper2023examining}, offering "content support" \cite{memmert2023towards} by generating realistic and coherent responses to user inputs \cite{cascella2023evaluating}. Their vastly improved capabilities and ubiquity \cite{cooper2023examining} makes them poised to revolutionise work patterns \cite{fui2023generative}. Generative AI is already used in fields like design, writing, music, \cite{han2024teams, suh2021ai, verheijden2023collaborative, dhillon2024shaping, gero2023social} healthcare, and clinical settings \cite{zhang2023generative, yu2023leveraging, biswas2024intelligent}, with promising results. However, the social service sector has been slower in adopting AI \cite{diez2023artificial, kawakami2023training}.

% Yet, the social service sector is one that could perhaps stand to gain the most from AI technologies. As Goldkind \cite{goldkind_social_2021} writes, social service, as a "values-centred profession with a robust code of ethics" (p. 372), is uniquely placed to inform the development of thoughtful algorithmic policy and practice. 
Social service, however, stands to benefit immensely from generative AI. SSPs work in time-poor environments \cite{tiah_can_2024}, often overwhelmed with tedious administrative work \cite{meilvang_working_2023} and large amounts of paperwork and data processing \cite{singer_ai_2023, tiah_can_2024}. 
% As such, workers often work in time-poor environments and are burdened with information overload and administrative tasks \cite{tiah_can_2024, meilvang_working_2023}. 
Generative AI is well-placed to streamline and automate tasks like formatting case notes, formulating treatment plans and writing progress reports, which can free up valuable time for more meaningful work like client engagement and enhance service quality \cite{fernando_integration_2023, perron_generative_2023, tiah_can_2024, thesocialworkaimentor_ai_nodate}. 

Given the immense potential, there has been emerging research interest in HAI collaboration and teamwork in the Human-Computer Interaction and Computer Supported Cooperative Work space \cite{wang_human-human_2020}. HAI collaboration and interaction has been postulated by researchers to contribute to new forms of HAI symbiosis and augmented intelligence, where algorithmic and human agents work in tandem with one another to perform tasks better than they could accomplish alone by augmenting each other's strengths and capabilities  \cite{dave_augmented_2023, jarrahi_artificial_2018}.

However, compared to the focus on AI decision-making and PRM tools, there is scant research on generative AI and HAI collaboration in the social service sector \cite{wykman_artificial_2023}. This study therefore seeks to fill this critical gap by exploring how SSPs use and interact with a novel generative AI tool, helping to expand our understanding of the new opportunities that HAI collaboration can bring to the social service sector.

\subsection{Challenges in AI Use in Social Service}
\label{subsec:relatedworkaiuse}

% Despite the immense potential of AI systems to augment social work practice, there are multiple challenges with integrating such systems into real-life practice. 
Despite its evident benefits, multiple challenges plague the integration of AI and its vast potential into real-life social service practice.
% Numerous studies have investigated the use of PRMs to help practitioners decide on a course of action for their clients. 
When employing algorithmic decision-making systems, practitioners often experience tension in weighing AI suggestions against their own judgement \cite{kawakami2022improving, saxena2021framework}, being uncertain of how far they should rely on the machine. 
% Despite often being instructed to use the tool as part of evaluating a client, 
Workers are often reluctant to fully embrace AI assessments due to its inability to adequately account for the full context of a case \cite{kawakami2022improving, gambrill2001need}, and lack of clarity and transparency on AI systems and limitations \cite{kawakami2022improving}. Brown et al. \cite{brown2019toward} conducted workshops using hypothetical algorithmic tools 
% to understand service providers' comfort levels with using such tools in their work,
and found similar issues with mistrust and perceived unreliability. Furthermore, introducing AI tools can  create new problems of its own, causing confusion and distrust amongst workers \cite{kawakami2022improving}. Such factors are critical barriers to the acceptance and effective use of AI in the sector.

\citeauthor{meilvang_working_2023} (2023) cites the concept of \textit{boundary work}, which explores the delineation between "monotonous" administrative labour and "professional", "knowledge based" work drawing on core competencies of SSPs. While computers have long been used for bureaucratic tasks like client registration, the introduction of decision support systems like PRMs stirred debate over AI "threatening professional discretion and, as such, the profession itself" \cite{meilvang_working_2023}. Such latent concerns arguably drive the resistance to technology adoption described above. Generative AI is only set to further push this boundary, 
% these concerns are only set to grow in tandem with the vast capabilities of generative and other modern AI systems. Compared to the relatively primitive AI systems in past years, perceived as statistical algorithms \cite{brown2019toward} turning preset inputs like client age and behavioural symptoms \cite{vaithianathan2017developing} into simple numerical outputs indicating various risk scores, modern AI systems are vastly more capable: LLMs 
with its ability to formulate detailed reports and assessments that encroach upon the "core" work of SSPs.
% accept unrestricted and unstructured inputs and return a range of verbose and detailed evaluations according to the user's instructions. 
Introducing these systems exacerbate previously-raised issues such as understanding the limitations and possibilities of AI systems \cite{kawakami2022improving} and risk of overreliance on AI \cite{van2023chatgpt}, and requires a re-examination of where users fall on the algorithmic aversion-bias scale \cite{brown2019toward} and how they detect and react to algorithmic failings \cite{de2020case}. We address these critical issues through an empirical, on-the-ground study that to our knowledge is the first of its kind since the new wave of generative AI.

% W 

% Yet, to date, we have limited knowledge on the real-world impacts and implications of human-AI collaboration, and few studies have investigated practitioners’ experiences working with and using such AI systems in practice, especially within the social work context \cite{kawakami2022improving}. A small number of studies have explored practitioner perspectives on the use of AI in social work, including Kawakami et al. \cite{kawakami2022improving}, who interviewed social workers on their experiences using the AFST; Stapleton et al. \cite{stapleton_imagining_2022}, who conducted design workshops with caseworkers on the use of PRMs in child welfare; and Wassal et al. \cite{wassal_reimagining_2024}, who interviewed UK social work professionals on the use of AI. A common thread from all these studies was a general disregard for the context and users, with many practitioners criticising the failure of past AI tools arising from the lack of participation and involvement of social workers and actual users of such systems in the design and development of algorithmic systems \cite{wassal_reimagining_2024}. Similarly, in a scoping review done on decision-support algorithms in social work, Jacobi \& Christensen \cite{jacobi_functions_2023} reported that the majority of studies reveal limited bottom-up involvement and interaction between social workers, researchers and developers, and that algorithms were rarely developed with consideration of the perspective of social workers.
% so the \cite{yang_unremarkable_2019} and \cite{holten_moller_shifting_2020} are not real-world impacts? real-world means to hear practitioner's voice? I feel this is quite important but i didnt get this point in intro!

% why mentioning 'which have largely focused on existing ADS tools (e.g., AFST)'? i can see our strength is more localized, but without basic knowledge of social work i didnt get what's the 'departure' here orz
% the paragraph is great! do we need to also add one in line 20 21?

\subsection{Designing AI for Social Service through Participatory Design}
\label{subsec:relatedworkpd}
% i think it's important! but maybe not a whole subsection? but i feel the strong connection with practitioners is indeed one of our novelties and need to highlight it, also in intro maybe
% Participatory design (PD) has long been used extensively in HCI \cite{muller1993participatory}, to both design effective solutions for a specific community and gain a deep understanding of that community. Of particular interest here is the rich body of literature on PD in the field of healthcare \cite{donetto2015experience}, which in this regard shares many similarities and concerns with social work. PD has created effective health improvement apps \cite{ryu2017impact}, 

% PD offers researchers the chance to gather detailed user requirements \cite{ryu2017impact}...

Participatory design (PD) is a staple of HCI research \cite{muller1993participatory}, facilitating the design of effective solutions for a specific community while gaining a deep understanding of its stakeholders. The focus in PD of valuing the opinions and perspectives of users as experts \cite{schuler_participatory_1993} 
% In recent years, the tech and social work sectors have awakened to the importance of involving real users in designing and implementing digital technologies, developing human-centred design processes to iteratively design products or technologies through user feedback 
has gained importance in recent years \cite{storer2023reimagining}. Responding to criticisms and failures of past AI tools that have been implemented without adequate involvement and input from actual users, HCI scholars have adopted PD approaches to design predictive tools to better support human decision-making \cite{lehtiniemi_contextual_2023}.
% ; accordingly, in social service, a line of research has begun studying and designing for human-AI collaboration with real-world users (e.g. \cite{holten_moller_shifting_2020, kawakami2022improving, yang_unremarkable_2019}).
Section \ref{subsec:relatedworkaiuse} shows a clear need to better understand SSP perspectives when designing and implementing AI tools in the social sector. 
Yet, PD research in this area has been limited. \citeauthor{yang2019unremarkable} (2019), through field evaluation with clinicians, investigated reasons behind the failure of previous AI-powered decision support tools, allowing them to design a new-and-improved AI decision-support tool that was better aligned with healthcare workers’ workflows. Similarly, \citeauthor{holten_moller_shifting_2020} (2020) ran PD workshops with caseworkers, data scientists and developers in public service systems to identify the expectations and needs that different stakeholders had in using ADS tools.

% Indeed, it is as Wise \cite{wise_intelligent_1998} noted so many years ago on the rise of intelligent agents: “it is perhaps when technologies are new, when their (and our) movements, habits and attitudes seem most awkward and therefore still at the forefront of our thoughts that they are easiest to analyse” (p. 411). 
Building upon this existing body of work, we thus conduct a study to co-design an AI tool \textit{for} and \textit{with} SSPs through participatory workshops and focus group discussions. In the process, we revisit many of the issues mentioned in Section \ref{subsec:relatedworkaiuse}, but in the context of novel generative AI systems, which are fundamentally different from most historical examples of automation technologies \cite{noy2023experimental}. This valuable empirical inquiry occurs at an opportune time when varied expectations about this nascent technology abound \cite{lehtiniemi_contextual_2023}, allowing us to understand how SSPs incorporate AI into their practice, and what AI can (or cannot) do for them. In doing so, we aim to uncover new theoretical and practical insights on what AI can bring to the social service sector, and formulate design implications for developing AI technologies that SSPs find truly meaningful and useful.
% , and drive future technological innovations to transform the social service sector not just within [our country], but also on a global scale.

 % with an on-the-ground study using a real prototype system that reflects the state of AI in current society. With the presumption that AI will continue to be used in social work given the great benefits it brings, we address the pressing need to investigate these issues to ensure that any potential AI systems are designed and implemented in a responsible and effective manner.

% Building upon these works, this study therefore seeks to adopt a participatory design methodology to investigate social workers’ perspectives and attitudes on AI and human-AI collaboration in their social work practice, thus contributing to the nascent body of practitioner-centred HCI research on the use of AI in social work. Yet, in a departure from prior work, which have largely focused on existing ADS tools (e.g., AFST) and were situated in a Western context, our paper also aims to expand the scope by piloting a novel generative AI tool that was designed and developed by the researchers in partnership with a social service agency based in Singapore, with aims of generating more insights on wider use cases of AI beyond what has been previously studied.

% i may think 'While the current lacunae of research on applications of AI in social work may appear to be a limitation, it simultaneously presents an exciting opportunity for further research and exploration \cite{dey_unleashing_2023},' this point is already convincing enough, not sure if we need to quote here
% I like this end! it's a good transition to our study design, do we need to mention the localization in intro as well? like we target at singapore

% Given the increasing prominence and acceptance of AI in modern society, 

% These increased capabilities vastly exacerbate the issues already present with a simpler tool like the AFST: the boundaries and limitations of an LLM system are significantly more difficult to understand and its possible use cases are exponentially greater in scope. 

% Put this in discussion section instead?
% Kawakami et al's work "highlights the importance of studying how collaborative decision-making... impacts how people rely upon and make sense of AI models," They conclude by recommending designing tools that "support workers in understanding the boundaries of [an AI system's] capabilities", and implementing design procedures that "support open cultures for critical discussion around AI decision making". The authors outline critical challenges of implementing AI systems, elucidating factors that may hinder their effectiveness and even negatively affect operations within the organisation.


% Is this needed?:
% talk about the strengths of PD in eliciting user viewpoints and knowledge, in particular when it is a field that is novel or where a certain system has not been used or developed or tested before

\section{Method}
In this section, we present the theoretical foundation~\cref{sec:theory} and algorithmic details~\cref{sec:algo} of our method, followed by a discussion on hyperparameter selection and an assessment of its practical advantages. Note that we use the terms EigenLoRA, EigenLoRAx, ELoRA and ELoRAx interchangeably.

\subsection{Theoretical Motivation}
\label{sec:theory}

For a full rank weight matrix $W \in \mathbb{R}^{m \times n} $ that learns to map input space $X\in\mathbb{R}^{m}$ to output space $\mathbb{R}^{n}$, the rank is expressed as $\min(m,n)$. As the rank of $ W $ increases, modifying it to accommodate new tasks becomes computationally expensive and increasingly complex. This is a common challenge faced when finetuning pretrained large foundation models. LoRA is a parameter efficient finetuning approach used for large pretrained models with weights $W_0$ that mitigates this challenge by merely learning low-rank weight updates $W$ such that the risk between $Y$ and $W_0X + WX + b$ is minimized. Instead of directly learning $W$, LoRA proposes to learn a lower ranked decomposition of $W$ by learning two low-rank matrices, $ B \in \mathbb{R}^{m \times r} $ and $ A \in \mathbb{R}^{r \times n} $, both having ranks $r$.  This factorization ensures that the product $ BA $ retains the original dimensions of $ W_0 $ while having a significantly reduced rank.
As a result, although the transformation defined by $ BA $ maps from $ \mathbb{R}^m $ to $ \mathbb{R}^n $, it does not span the full space of such mappings due to its constrained rank. The low-rank weight matrices result in substantially smaller number of trainable parameters than the full rank parameter count of $m\cdot n$. Such parameter efficient finetuning makes LoRA a computationally viable alternative for fine-tuning large-scale models.    

Previous works such as \cite{meng_pissa_2024, dora} have proposed the existence of a common parameter subspace implying the idea of shared principal subspace. We highlight that LoRA adapters share such a lower dimensional shared principal subspace when finetuned for diverse tasks. Along with reduction in computational overhead, it reinforces the idea that task-relevant transformations reside within a compact, reusable subspace. To formalize this, we first define a space of tasks representable by linear transformation matrices, providing a foundation for analyzing the role of shared principal subspaces in model adaptation.

\begin{definition}[Task definition for LoRAs] 
\label{def:lineartasks}
We first define a LoRA task $t_i (X_i,Y_i): \mathbb{R}^m \rightarrow \mathbb{R}^n$ such that $Y_i = W^*_iX_i + b$ where $b$ is some constant. Then the LoRA task domain $\mathcal{T}_d$ is a set of $d$ such tasks, $\mathcal{T}_d = \{t_i\}^{d}_{i=1}$.
\end{definition}


For a given set of pretrained weights (such as those from a foundation model) $W_0\in\mathbb{R}^{m\times n}$, LoRA weights $BA$ at any layer modify the output as $W_0X + BAX + \epsilon_t$, allowing the model to adapt to the new task and converge toward the optimal solution $W^*_t$. The key here is that only $B$ and $A$ weights are updated during finetuning. 
Without loss of generality, 
% \prakhar{can we write the full form here}\aayush{its a common term in math}
assume $r\ll n$ and let the true transformation matrix $W^*_t \in \mathbb{R}^{r \times n}$ be interpreted as $r$ $n$-dimensional vectors: $\mathbf{w}^{*1}_t, ..., \mathbf{w}^{*r}_t \in \mathbb{R}^{n}$. Finding LoRA weights is equivalent to finding sets of these $r$ vectors in $\mathbb{R}^{n}$. 

\begin{definition}[Set of LoRA weights]
\label{def:loraset}
We define the weights of a LoRA adapted for task $t_i$ as $B_iA_i$. Both $B_i$ and $A_i$ will have their own individual subspaces. For the purpose of the analysis we will consider a generic task specific weight matrix $W_i \in \mathbb{R}^{m\times n}$ adapted to task $t_i$ such that $n<m$ and its rank $r<n$. The analysis, however, is valid for both $B_i$ and $A_i$. 
Now can define a set of LoRAs as stacked (along columns) weight matrices $\hat{W} = \{W_i \}^{d}_{i=1}$ where each $W_i$ is adapted for a task $t_i\in\mathcal{T}_d$ and a training set $\mathcal{S}_i=\{\{x,y\} \mid x\in X_t, y\in Y_t\}$ where the size of the training is $s_i = |\mathcal{S}_i|$. For theoretical analysis we assume that each training set $X_i \times Y_i$ is distributed according to some unknown Gaussian distribution with mean $\Bar{X_i}$ and $\lVert X_i\lVert_F\leq M$ for some constant $M>0$. Each weight matrix can have different ranks and the following method and analysis will still hold, however, for brevity we assume all weight matrices stacked in $\hat{W}$ to have the same rank $r$. 
% Next, we define a subspace in $\mathbb{R}^{n}$ spanned by $d$ LoRAs learnt for $d$ tasks from $\mathcal{T}_d$.
\end{definition}
\begin{definition}[Subspace spanned by LoRAs from a task domain $\mathcal{T}_{d}$]
\label{def:subspace}
We define the subspace of weights $\mathcal{Z}_{d} = \{C\hat{W} \mid C\in\mathbb{R}^{m\times m}\}$ spanned within $\mathbb{R}^{m\times n}$. 
\end{definition}
Using Singular Value Decomposition (SVD) or Principal Component Analysis (PCA for a zero-centered $\hat{W}$) , we can obtain $\hat{W}=\mathcal{U}\Sigma \mathcal{V}^T$. We then represent top $K$ right singular vectors of $\hat{W}$ (or top $K$ principal components if $\hat{W}$ is zero-centered) as $\Vk \in\mathbb{R}^{K\times n}=\{\Vk\in\mathbb{R}^{1\times n}\}^K_{k=1}$.
\begin{definition}[Shared principal subspace of LoRAs finetuned in domain $\mathcal{T}_d$]
\label{def:principalsubspace}
We define the shared principal subspace of weights for a task domain $\mathcal{T}_d$ as $\mathcal{Z}^{K}_d = \{\alpha \Vk \mid \alpha\in\mathbb{R}^{m\times K}\}$ spanned by top K principal components of the LoRAs within $\mathbb{R}^{m\times n}$. 
\end{definition}

Next, we introduce the idea of defining a new related task $\tnew$
\begin{definition}[New related task $\tnew$].
\label{def:newtask}
    A new linear task $\tnew$ with true solution $\Wtrue$ is said to be related if it is spanned by the basis of $\hat{W}$ i.e. $\Wtrue = C\hat{W}$ and it holds that $\lVert \Wtrue - \alphatrue\Vk\lVert^2_F \leq \Vert\Wtrue - \alphalearnt\Vk\lVert^2_F$ for all rank $K$ linear transformation matrices $\alphalearnt$ and $\lVert \Wtrue - \alphatrue\Vk\lVert^2_F \leq \singularsum$ where $\sigma_i$'s are singular values of $\hat{W}$. For such a task, we learn coefficients of $K$ principal components $\alphalearnt \in \mathbb{R}^{m\times K}$ resulting in EigenLoRAx weights $\Wegn = \alphalearnt\Vk$. \\
\end{definition}

Definition \ref{def:newtask} establishes a bound over the related-ness of a new task with those in the known task domain $\mathcal{T}_d$. If the true solution of the new task lies majorly in the principal subspace of $\mathcal{T}_d$ i.e. has major principal components (PCs) within the top $K$ principal components of $\hat{W}$ with some finite bound on the misalignment along the PCs orthogonal to the top $K$ PCs of $\hat{W}$, then we can ideally quantify the relation between a new task and a task domain. Any task that has its true solution within a subspace defined by the PCs orthogonal to the top $K$ PCs of $\hat{W}$ is not as closely related as a task with its solution completely or majorly within the principal subspace. A task that has its solution completely orthogonal to all the PCs of $\hat{W}$ is completely unrelated and is not the main focus of this study. 

Next, we present an algorithm to find the principal subspace of the trained adapters and our experiments in Section~\ref{sec:experiments}.


\begin{algorithm}[!htb]
\caption{\textbf{EigenLoRAx PCs} Calculation}
\label{algo:eigenlora}
\begin{algorithmic}
    \STATE {\bfseries Input:} 
    LoRA matrices 
    $\{W_t \in \mathbb{R}^{m \times n}\}_{t=1}^d$ 
    
     , number of PC ($K$), number of pseudo-PC ($P$)\\
    
    \STATE {\bfseries Output:} EigenLoRAx PCs $\Vk$

    \STATE $\hat{W} = \begin{bmatrix*} W_{1}\in \mathbb{R}^{m\times n} & \text{...} & W_{d}\in \mathbb{R}^{m\times n}\end{bmatrix*}$, \COMMENT{Stack LoRA matrices}\\
    
    \STATE Compute the mean of each feature: $ \Bar{W} = \frac{1}{n} \sum_{i=1}^{n} W_i $
    \STATE Subtract the mean: $ \hat{W}_c = \hat{W} - \Bar{W} $
    
   
    \STATE Perform SVD: $ \hat{W}_c = U \Sigma V^T $
    
    \STATE Extract the top $ K $ principal components
    \STATE Select the first $ K $ columns of $ \mathcal{V} $: $ \V = V[:, 1:K] $
    
    \STATE Optionally, augment the subspace with $P$ pseudo-PCs
    \FOR{$ p = 1 $ to $ P $}
        \STATE Sample a random vector $ v_p \sim \mathcal{N}(0, I_n) $  \COMMENT{Sample from a normal distribution}
        \STATE Orthogonalize $ v_p $ against all PCs in $ \V $ using Gram-Schmidt:
        \FOR{$ i = 1 $ to $ K+p-1 $}
            \STATE $ v_p = v_p - \frac{v_p^T V_K[:, i]}{\|\V[:, i]\|^2} \V[:, i] $
        \ENDFOR
        \STATE Normalize $ v_p $: $ v_p = \frac{v_p}{\|v_p\|} $
        \STATE Append $ v_p $ to $ \V $ if $ v_p $ is not a null vector
        \STATE $K = K + 1$
    \ENDFOR

    \STATE return $\V, \mu$
    
\end{algorithmic}
\end{algorithm}

\subsection{Algorithm}
\label{sec:algo}

Assume we have $N$ LoRA adapters, each consisting of a set of $A, B$ matrix pairs for every layer, trained on various tasks within a domain $\mathcal{T}_d$ for a given base pretrained model. Algorithm~\ref{algo:eigenlora} computes a list of top $K$ principal components—referred to as EigenLoRAx PCs—that define an initial principal subspace for this domain. 

To construct this subspace, the algorithm aggregates LoRA matrices across tasks for each layer, separately for $A$ and $B$ matrices (though it can also be applied to the product $BA$). Each LoRA matrix, having rank r, is treated as a list of vectors, and a decomposition is performed on this stacked set of vectors. The most significant components extracted from this process serve as a basis for the principal subspace, providing an efficient representation that can be linearly combined to approximate the original LoRA weight matrices. We showcase our algorithm using representative weight matrices $W_t$, where each $W_t$ represents a single $A$ or $B$ matrix from a single LoRA layer of the neural network. In practice, this procedure is applied to all relevant layers.

Since real-world scenarios often involve low-resource domains with limited availability of LoRA adapters, we extend our subspace by introducing additional pseudo-PCs. Specifically, we sample random vectors of the same dimension as each PC and orthogonalize them with respect to all existing PCs. This process can be iterated to generate more pseudo-PCs, thereby augmenting the principal subspace. As empirically shown in Table~\ref{tab:glue_low}, this augmentation strategy significantly outperforms naive random selection of PCs for subspace expansion.

\paragraph{Learning new tasks}\label{method:learningnewelora} Having extracted a set of PCs (including pseudo-PCs, if needed), $\mathcal{V}_K \in\mathbb{R}^{K\times n}=\{\V\in\mathbb{R}^{1\times n}\}^K_{k=1}$, we can approximate a given (LoRA) weight matrix by minimizing $\lVert W - \alpha\Vk\lVert_F$ where $\alpha$ are linear coefficients~\cref{sec:theory}. In fact, we can analytically compute of the given LoRA matrices by calculating the linear coefficients which minimizes the above objective. For new tasks however, for which we do not have a LoRA matrix, we freeze the EigenLoRAx PCs and randomly initialize the $\alpha$s. The forward pass in layer is calculated as 
\begin{align}
    h = W_0x + \dboxed{\alpha_B^T \mathcal{V}_B \alpha_A^T \mathcal{V}_A(x)}.
\end{align}
Here, $W_0$ are the pretrained weights of the base model and $\mathcal{V}_B, \mathcal{V}_A$ are EigenLoRAx components (which represent the shared subspace) that are frozen during training. The corresponding lightweight coefficients $\alpha_B$ and $\alpha_A$ are learned. This reduces the number of learnable parameters from $O(2rn)$ to $O(2K)$, by a factor of $\frac{rn}{K}$ (assuming $\alpha$ to be scalar). 

\phantomsection


Effective human-robot cooperation in CoNav-Maze hinges on efficient communication. Maximizing the human’s information gain enables more precise guidance, which in turn accelerates task completion. Yet for the robot, the challenge is not only \emph{what} to communicate but also \emph{when}, as it must balance gathering information for the human with pursuing immediate goals when confident in its navigation.

To achieve this, we introduce \emph{Information Gain Monte Carlo Tree Search} (IG-MCTS), which optimizes both task-relevant objectives and the transmission of the most informative communication. IG-MCTS comprises three key components:
\textbf{(1)} A data-driven human perception model that tracks how implicit (movement) and explicit (image) information updates the human’s understanding of the maze layout.
\textbf{(2)} Reward augmentation to integrate multiple objectives effectively leveraging on the learned perception model.
\textbf{(3)} An uncertainty-aware MCTS that accounts for unobserved maze regions and human perception stochasticity.
% \begin{enumerate}[leftmargin=*]
%     \item A data-driven human perception model that tracks how implicit (movement) and explicit (image transmission) information updates the human’s understanding of the maze layout.
%     \item Reward augmentation to integrate multiple objectives effectively leveraging on the learned perception model.
%     \item An uncertainty-aware MCTS that accounts for unobserved maze regions and human perception stochasticity.
% \end{enumerate}

\subsection{Human Perception Dynamics}
% IG-MCTS seeks to optimize the expected novel information gained by the human through the robot’s actions, including both movement and communication. Achieving this requires a model of how the human acquires task-relevant information from the robot.

% \subsubsection{Perception MDP}
\label{sec:perception_mdp}
As the robot navigates the maze and transmits images, humans update their understanding of the environment. Based on the robot's path, they may infer that previously assumed blocked locations are traversable or detect discrepancies between the transmitted image and their map.  

To formally capture this process, we model the evolution of human perception as another Markov Decision Process, referred to as the \emph{Perception MDP}. The state space $\mathcal{X}$ represents all possible maze maps. The action space $\mathcal{S}^+ \times \mathcal{O}$ consists of the robot's trajectory between two image transmissions $\tau \in \mathcal{S}^+$ and an image $o \in \mathcal{O}$. The unknown transition function $F: (x, (\tau, o)) \rightarrow x'$ defines the human perception dynamics, which we aim to learn.

\subsubsection{Crowd-Sourced Transition Dataset}
To collect data, we designed a mapping task in the CoNav-Maze environment. Participants were tasked to edit their maps to match the true environment. A button triggers the robot's autonomous movements, after which it captures an image from a random angle.
In this mapping task, the robot, aware of both the true environment and the human’s map, visits predefined target locations and prioritizes areas with mislabeled grid cells on the human’s map.
% We assume that the robot has full knowledge of both the actual environment and the human’s current map. Leveraging this knowledge, the robot autonomously navigates to all predefined target locations. It then randomly selects subsequent goals to reach, prioritizing grid locations that remain mislabeled on the human’s map. This ensures that the robot’s actions are strategically focused on providing useful information to improve map accuracy.

We then recruited over $50$ annotators through Prolific~\cite{palan2018prolific} for the mapping task. Each annotator labeled three randomly generated mazes. They were allowed to proceed to the next maze once the robot had reached all four goal locations. However, they could spend additional time refining their map before moving on. To incentivize accuracy, annotators receive a performance-based bonus based on the final accuracy of their annotated map.


\subsubsection{Fully-Convolutional Dynamics Model}
\label{sec:nhpm}

We propose a Neural Human Perception Model (NHPM), a fully convolutional neural network (FCNN), to predict the human perception transition probabilities modeled in \Cref{sec:perception_mdp}. We denote the model as $F_\theta$ where $\theta$ represents the trainable weights. Such design echoes recent studies of model-based reinforcement learning~\cite{hansen2022temporal}, where the agent first learns the environment dynamics, potentially from image observations~\cite{hafner2019learning,watter2015embed}.

\begin{figure}[t]
    \centering
    \includegraphics[width=0.9\linewidth]{figures/ICML_25_CNN.pdf}
    \caption{Neural Human Perception Model (NHPM). \textbf{Left:} The human's current perception, the robot's trajectory since the last transmission, and the captured environment grids are individually processed into 2D masks. \textbf{Right:} A fully convolutional neural network predicts two masks: one for the probability of the human adding a wall to their map and another for removing a wall.}
    \label{fig:nhpm}
    \vskip -0.1in
\end{figure}

As illustrated in \Cref{fig:nhpm}, our model takes as input the human’s current perception, the robot’s path, and the image captured by the robot, all of which are transformed into a unified 2D representation. These inputs are concatenated along the channel dimension and fed into the CNN, which outputs a two-channel image: one predicting the probability of human adding a new wall and the other predicting the probability of removing a wall.

% Our approach builds on world model learning, where neural networks predict state transitions or environmental updates based on agent actions and observations. By leveraging the local feature extraction capabilities of CNNs, our model effectively captures spatial relationships and interprets local changes within the grid maze environment. Similar to prior work in localization and mapping, the CNN architecture is well-suited for processing spatially structured data and aligning the robot’s observations with human map updates.

To enhance robustness and generalization, we apply data augmentation techniques, including random rotation and flipping of the 2D inputs during training. These transformations are particularly beneficial in the grid maze environment, which is invariant to orientation changes.

\subsection{Perception-Aware Reward Augmentation}
The robot optimizes its actions over a planning horizon \( H \) by solving the following optimization problem:
\begin{subequations}
    \begin{align}
        \max_{a_{0:H-1}} \;
        & \mathop{\mathbb{E}}_{T, F} \left[ \sum_{t=0}^{H-1} \gamma^t \left(\underbrace{R_{\mathrm{task}}(\tau_{t+1}, \zeta)}_{\text{(1) Task reward}} + \underbrace{\|x_{t+1}-x_t\|_1}_{\text{(2) Info reward}}\right)\right] \label{obj}\\ 
        \subjectto \quad
        &x_{t+1} = F(x_t, (\tau_t, a_t)), \quad a_t\in\Ocal \label{const:perception_update}\\ 
        &\tau_{t+1} = \tau_t \oplus T(s_t, a_t), \quad a_t\in \Ucal\label{const:history_update}
    \end{align}
\end{subequations} 

The objective in~\eqref{obj} maximizes the expected cumulative reward over \( T \) and \( F \), reflecting the uncertainty in both physical transitions and human perception dynamics. The reward function consists of two components: 
(1) The \emph{task reward} incentivizes efficient navigation. The specific formulation for the task in this work is outlined in \Cref{appendix:task_reward}.
(2) The \emph{information reward} quantifies the change in the human’s perception due to robot actions, computed as the \( L_1 \)-norm distance between consecutive perception states.  

The constraint in~\eqref{const:history_update} ensures that for movement actions, the trajectory history \( \tau_t \) expands with new states based on the robot’s chosen actions, where \( s_t \) is the most recent state in \( \tau_t \), and \( \oplus \) represents sequence concatenation. 
In constraint~\eqref{const:perception_update}, the robot leverages the learned human perception dynamics \( F \) to estimate the evolution of the human’s understanding of the environment from perception state $x_t$ to $x_{t+1}$ based on the observed trajectory \( \tau_t \) and transmitted image \( a_t\in\Ocal \). 
% justify from a cognitive science perspective
% Cognitive science research has shown that humans read in a way to maximize the information gained from each word, aligning with the efficient coding principle, which prioritizes minimizing perceptual errors and extracting relevant features under limited processing capacity~\cite{kangassalo2020information}. Drawing on this principle, we hypothesize that humans similarly prioritize task-relevant information in multimodal settings. To accommodate this cognitive pattern, our robot policy selects and communicates high information-gain observations to human operators, akin to summarizing key insights from a lengthy article.
% % While the brain naturally seeks to gain information, the brain employs various strategies to manage information overload, including filtering~\cite{quiroga2004reducing}, limiting/working memory, and prioritizing information~\cite{arnold2023dealing}.
% In this context of our setup, we optimize the selection of camera angles to maximize the human operator's information gain about the environment. 

\subsection{Information Gain Monte Carlo Tree Search (IG-MCTS)}
IG-MCTS follows the four stages of Monte Carlo tree search: \emph{selection}, \emph{expansion}, \emph{rollout}, and \emph{backpropagation}, but extends it by incorporating uncertainty in both environment dynamics and human perception. We introduce uncertainty-aware simulations in the \emph{expansion} and \emph{rollout} phases and adjust \emph{backpropagation} with a value update rule that accounts for transition feasibility.

\subsubsection{Uncertainty-Aware Simulation}
As detailed in \Cref{algo:IG_MCTS}, both the \emph{expansion} and \emph{rollout} phases involve forward simulation of robot actions. Each tree node $v$ contains the state $(\tau, x)$, representing the robot's state history and current human perception. We handle the two action types differently as follows:
\begin{itemize}
    \item A movement action $u$ follows the environment dynamics $T$ as defined in \Cref{sec:problem}. Notably, the maze layout is observable up to distance $r$ from the robot's visited grids, while unexplored areas assume a $50\%$ chance of walls. In \emph{expansion}, the resulting search node $v'$ of this uncertain transition is assigned a feasibility value $\delta = 0.5$. In \emph{rollout}, the transition could fail and the robot remains in the same grid.
    
    \item The state transition for a communication step $o$ is governed by the learned stochastic human perception model $F_\theta$ as defined in \Cref{sec:nhpm}. Since transition probabilities are known, we compute the expected information reward $\bar{R_\mathrm{info}}$ directly:
    \begin{align*}
        \bar{R_\mathrm{info}}(\tau_t, x_t, o_t) &= \mathbb{E}_{x_{t+1}}\|x_{t+1}-x_t\|_1 \\
        &= \|p_\mathrm{add}\|_1 + \|p_\mathrm{remove}\|_1,
    \end{align*}
    where $(p_\mathrm{add}, p_\mathrm{remove}) \gets F_\theta(\tau_t, x_t, o_t)$ are the estimated probabilities of adding or removing walls from the map. 
    Directly computing the expected return at a node avoids the high number of visitations required to obtain an accurate value estimate.
\end{itemize}

% We denote a node in the search tree as $v$, where $s(v)$, $r(v)$, and $\delta(v)$ represent the state, reward, and transition feasibility at $v$, respectively. The visit count of $v$ is denoted as $N(v)$, while $Q(v)$ represents its total accumulated return. The set of child nodes of $v$ is denoted by $\mathbb{C}(v)$.

% The goal of each search is to plan a sequence for the robot until it reaches a goal or transmits a new image to the human. We initialize the search tree with the current human guidance $\zeta$, and the robot's approximation of human perception $x_0$. Each search node consists consists of the state information required by our reward augmentation: $(\tau, x)$. A node is terminal if it is the resulting state of a communication step, or if the robot reaches a goal location. 

% A rollout from the expanded node simulates future transitions until reaching a terminal state or a predefined depth $H$. Actions are selected randomly from the available action set $\mathcal{A}(s)$. If an action's feasibility is uncertain due to the environment's unknown structure, the transition occurs with probability $\delta(s, a)$. When a random number draw deems the transition infeasible, the state remains unchanged. On the other hand, for communication steps, we don't resolve the uncertainty but instead compute the expected information gain reward: \philip{TODO: adjust notation}
% \begin{equation}
%     \mathbb{E}\left[R_\mathrm{info}(\tau, x')\right] = \sum \mathrm{NPM(\tau, o)}.
% \end{equation}

\subsubsection{Feasibility-Adjusted Backpropagation}
During backpropagation, the rewards obtained from the simulation phase are propagated back through the tree, updating the total value $Q(v)$ and the visitation count $N(v)$ for all nodes along the path to the root. Due to uncertainty in unexplored environment dynamics, the rollout return depends on the feasibility of the transition from the child node. Given a sample return \(q'_{\mathrm{sample}}\) at child node \(v'\), the parent node's return is:
\begin{equation}
    q_{\mathrm{sample}} = r + \gamma \left[ \delta' q'_{\mathrm{sample}} + (1 - \delta') \frac{Q(v)}{N(v)} \right],
\end{equation}
where $\delta'$ represents the probability of a successful transition. The term \((1 - \delta')\) accounts for failed transitions, relying instead on the current value estimate.

% By incorporating uncertainty-aware rollouts and backpropagation, our approach enables more robust decision-making in scenarios where the environment dynamics is unknown and avoids simulation of the stochastic human perception dynamics.


\section{Experiments and Results}\label{sec:experiments}
In this section, we demonstrate the efficacy and versatility of EigenLoRAx across diverse tasks, modalities, and model architectures, highlighting its individual advantages. EigenLoRAx requires significantly fewer parameters to match or surpass LoRA’s performance (Tables~\ref{tab:vision_models}, \ref{tab:glue_benchmark_results}) and achieves similar or faster loss convergence (Figure~\ref{fig:traininglosscola}), making it a cost-effective alternative to random initialization and other methods~\citep{meng_pissa_2024}. Additionally, we showcase its memory-efficient inference capabilities with a Stable Diffusion text-to-image generation model~\citep{diffusion} (Section~\ref{sec:diffusion}). Notably, EigenLoRAx retains its efficiency even in low-resource scenarios where a large number of LoRAs are unavailable.

\paragraph{Note on Baselines} Our focus is on recycling adapter knowledge and improving training and memory efficiency while maintaining performance, not solely on maximizing performance. We compare primarily with LoRA, as EigenLoRAx builds its principal subspace using pretrained LoRA adapters. Using better adapters and optimization could further enhance the subspace and performance. 

Due to lack of space, more experiments (3D Object pose estimation) and detailed ablation experiments are presented in \cref{sec:appendix}. 

\section{RESULTS}

\begin{figure*}[t!]
    %\vspace{-0.5cm}
    \centering
    \includegraphics[width=1\linewidth]{images/SystemArchitecture_2.png}
    \caption{From a single user demonstration, the system extracts the desired task goal state with the help of user interaction to solve ambiguities. Using the created environment variation, the system computes a task execution plan to bring new environments into the goal state. It sends the plan to agents in the environment to execute.} \label{fig:system_architecture}
\end{figure*}

Figure \ref{fig:system_architecture} shows our proposed framework to define a task goal, i.e. an environment goals state, and to turn a given environment into this goal state. The system visually observes a task execution by a user and segments this \underline{single} demonstration into \skills. \actions\ and \skills\ are defined in \ref{ssec:actions_skills}. The demonstration changed one or several properties of entities in the environment; environment which is now in the goal state. This information and the differences in entity properties from the start and end environment states are used to represent the task goal state. More on that in \ref{ssec:exp_model_def}. To turn a new environment into the defined goal state, a planning problem must be solved. This entails computing the differences between the environment's current state and the goal state, finding \actions\ that solve these differences, instantiating \skills\ that implement the \actions\ in the environment, selecting the \skills\ to execute by minimizing a given metric, and finally, sending the \skills\ to the agents in the environment to execute. This process is detailed in \ref{ssec:exp_model_use}.

% To prove the usability of our model, we present experiments to create a new goal state and turn the current environment into an (already-defined) goal state.

\subsection{Actions and Skills}\label{ssec:actions_skills}
A change in the environment is modeled using \actions, i.e. \textbf{what} has happened, and \skills, i.e. \textbf{how} did the change happen \cite{conceptHierarchyGeriatronicsSummit24}. Like in STRIPS \cite{strips} and PDDL \cite{pddl}, we represent \actions\ by their effects on entity properties and \skills\ by their preconditions and effects. \actions\ do not need preconditions because they only describe the \textbf{what} part of a change, not which conditions must be satisfied to perform the change. Besides preconditions and effects, \skills\ have a list of checks that tell our system if the \skill\ is executed in the environment. These checks allow the creation of a \skill\ recognition program, like the one presented in \cite{conceptHierarchyGeriatronicsSummit24}.
%\todo{citation of Geriatronics summit paper or the journal/unsubmitted paper?}
Using the \skill\ recognition output, we capture the changes from a task demonstration.

A \skill\ is thus the physical enactment of an abstract \action\ in an environment. Hence, \skills\ are correlated with \actions\ via their effects. A \skill\ can have more effects than a corresponding \action. For example, the \skill\ of scooping jam from a jar with a spoon implements the \action\ of \textit{TransferringContents}, but it also \textit{Dirties} the spoon.

\subsection{How To Parameterize The Model}\label{ssec:exp_model_def}
Creating a new goal state should be easier than manually specifying all variations wanted from the goal state. Doing so requires programming knowledge, which should not be needed to define goal states. One can let the system, which knows how to represent goal states, question the user about the desired state of the environment. However, this tedious process requires many questions from the system, also leading to decreased system usability.

Therefore, our approach is to let the user turn a given environment into a desired goal state and analyze the differences between the initial and final environment state to create the goal state representation. This single demonstration highlights the entity property values that were not in the desired goal state before being changed by the user.

We capture the demonstration via an Intel Realsense 3D camera \cite{realsense}, analyze the human skeleton via the OpenPose human pose estimation method \cite{openpose}, and determine the 3d pose of objects with AprilTag markers \cite{aprilTag}.

One demonstration contains the initial environment, not in the task goal state, and the final environment, in the goal state. The final environment state alone is not enough to create the environment variation. Thus, additional questions, guided by the differences between the two environment values, are posed by the system to the user to determine the desired variation in the environment state.

In a demonstration in which the user pours milk into a bowl, as shown in the top of Figure \ref{fig:system_architecture}, the initial question posed to the user is which entities that have changed properties are relevant for the goal state. If the goal state is to have more milk in the bowl, the milk carton is irrelevant; it is a means to achieve the goal state but not relevant to the goal itself. The bowl is thus selected as a relevant entity. 

Next, the list of relevant modified properties must also be determined for each relevant entity. It could have happened that during pouring of the milk into the bowl, the bowl's location also changed, e.g. touched accidentally by the user. Thus, not all modified properties could be relevant to the task. After selecting the relevant properties, the system knows from the knowledge base \cite{conceptHierarchyGeriatronicsSummit24} their \textit{ValueDomain} and the list of implemented \textbf{variations} for that \textit{ValueDomain}. Thus, the user parametrizes a selected \textbf{variation} from the list: choosing either a fixed value, a \textit{ValueDomain}-specific \textbf{RangeVariation} that must be parametrized, a conjunction or disjunction of \textbf{RangeVariations}, or the whole \textit{ValueDomain}.

In the example above, the user chooses the \textit{contentLevel} property as relevant. The system knows this property's defined set of values: a non-negative real number, and the possible range variation types: an open interval, a closed interval, an open-closed or closed-open interval, an intersection or union of intervals, etc. The user chooses a closed interval of $[0.28, 0.32]$ around the final \textit{contentLevel} value of $0.3L$. The user also specifies a variation for the entity's concept. It is generalized from that specific bowl instance to a \textit{LiquidContainer}.

After each modified property of each entity has a represented \textbf{variation}, the system automatically collects the entities into a variation of type $A$, see \ref{ssec:variations}, which is the assigned \textbf{variation} for the collection of entities in the environment.

Thus, the environment variation is determined in $\mathcal{O}\left(n\times m \times p\right)$ questions to the user, where $n$ is the number of entities in the environment, $m$ is the maximal number of properties that an entity can have, and $p$ is the maximal number of parameters that a \textbf{RangeVariation} needs to be represented. In the example above, $10$ questions were necessary to determine the task goal state shown in Figure \ref{fig:system_architecture} of a \textit{LiquidContainer} with \textit{contentLevel} between $0.28$ and $0.32L$. Figure \ref{fig:task_goal_state} shows the internal JSON-like representation of the goal state as the environment variation.
% 1 question which entities are relevant -> just bowl
% 1 question which properties are relevant -> contentLevel and concept
% 1 question about concept values being the same; should create variation?
% 1 question which ConceptValue-variation to select -> ConceptValue in Environment
% 1 question: which generalized concept?
% 1 question -> add other range-variation
% 1 question which Number-variation to select -> Interval
% 1 question: min-bound?
% 1 question: max-bound?
% 1 question -> add other range-variation

\begin{figure}[t!]
    %\vspace{-0.5cm}
    \centering
    \includegraphics[width=1\linewidth]{images/TaskDefinition_7.png}
    \caption{The goal state is a \textbf{RangeVariation} of the environment, of type EnvironmentDataRangeEntityVariation, which contains a \textbf{variation} of entities. This sub-variation is a \textbf{RangeVariation} of type MapRangeInstanceSubset (\textbf{variation} of type $A$, see \ref{ssec:variations}) and contains one instance \textbf{RangeVariation} of type InstanceRangePropertiesVariation. It defines the instance's concept \textbf{RangeVariation}, a \textit{LiquidContainer} to be found in the environment, and the \textit{contentLevel} property \textbf{RangeVariation}, the closed interval $\left[0.28, 0.32\right]$.} \label{fig:task_goal_state}
\end{figure}

\subsection{How To Use The Model}\label{ssec:exp_model_use}
Assuming the representation of a task's goal state is given, i.e. an environment variation, we detail our procedure (see Figure \ref{fig:experiment_description}) to turn the current environment into the goal state.

First, a Comparison between the environment and the goal variation is computed. This leads, as described in \ref{ssec:comparisons}, to a list of reasons why the environment is not in the variation. These reasons, i.e. differences $\delta$ of concept properties $p$, must be fixed to turn the environment into the goal state.

% Computing the differences between an EnvironmentData and an EnvironmentData-Variation, that has a Collection-Variation of type $A$, see \ref{ssec:variations}, is done via a maximal matching algorithm, where an edge between an entity $e$ an an entity variation $v_e$ means $e \in v_e$. 
For an EnvironmentData-Variation $v_{env}$ that defines a Collection-RangeVariation of type $A$, see \ref{ssec:variations}, computing the Comparison between an EnvironmentData $env$ and this target $v_{env}$ leads to a list of reasons for each entity $e_{env}$ in the entity collection of $env$, why $e_{env} \not\in v, \forall v \in A$. This can be seen in Figure \ref{fig:experiment_description}, where for each entity of \textit{LiquidContainer} concept in the environment, there is a list of differences, i.e. Comparisons, created for why the respective entity does not match the defined variation on the top-right.

\begin{figure}[t!]
    %\vspace{-0.5cm}
    \centering
    \includegraphics[width=1\linewidth]{images/Experiment_DescriptionUsingVariations_2.png}
    \caption{The procedure to turn an environment into its goal state is divided into 5 steps: computing differences, finding abstract solutions (i.e. \actions), computing practical solutions for the abstract ones (i.e. \actions\ $\rightarrow$ \skills), selecting the best practical solution, and executing the solution.} \label{fig:experiment_description}
\end{figure}
The second step of the procedure is to turn the list of differences into a list of \actions\ that can fix them. In notation, \action\ $A_x$ solves a difference in the concept property $p_x$. The system knows which properties \actions\ modify by analyzing the definition of their effects. Thus, \actions\ are created (parametrized) to fix the differences in entity properties.

% Because multiple instances can fit the instance variation, the third step is to match instances with the variations. Our matching optimization criterion is to minimize the amount of \textit{Actions} needed to fix the instances' property differences. \todo{continue!}

In the third step, each \action\ $A_x$ is converted into an execution plan $P_x$ that implements solving the difference $\delta_{p_x}$ in the environment. It is also possible that there is no possibility to implement the \action\ $A_x$ in the environment; this is represented as an execution plan $P_x = \emptyset$. An execution plan $P_x$ is otherwise, in its simplest form, a set of \skill\ alternatives $\left\{S_y\right\}$, where the \skill\ $S_y$ implements the \action\ $A_x$. There is the case to consider that the \skill\ $S_y$ has preconditions that are not met. And so, before executing the skill $S_y$, a different execution plan $P_{S_y}$ has to be computed and executed to allow the \skill\ $S_y$ to solve the property difference $\delta_{p_x}$. It is also possible that one single \skill\  $S_y$ is not enough to implement the \action\ $A_x$. Consider the case where the environment contains three cups with $0.1L$ of water, and the goal is to have one cup with $0.3L$ of content. One single \textit{Pouring} \skill\ is not enough to fulfill the goal; two \textit{Pouring} \skills\ must be executed. Thus, in the most general form, an execution plan $P_x = \left[\left\{ S_{iy}, P_{S_{iy}} \right\}_i\right]$ is a list of skill alternatives $\left\{ S_{iy}, P_{S_{iy}} \right\}_i$, that possibly contain other execution plans $P_{S_{iy}}$ to solve the skill's preconditions.

Our procedure to parameterize the \skills\ $S_y$ that implement the \action\ $A_x$ is a custom solution for each property $p_x$. One could backtrack through all possible parameter values of all possible skills to create a general solution that works for all properties. Another idea is to invert \skill\ effects and thus guide the \skill\ parameter search from the target variation to the value. However, both approaches would be computationally intense and would not create execution plans in a reasonable time. 
% reinforcement learning with policy for each property

The procedure to solve an entity $e$'s \underline{contentLevel} property difference searches for other \textit{Container} object instances in the environment, sorts them according to their content volume, and iterates through them in ascending order if $e.contentLevel \le target.contentLevel$; otherwise, in descending order. If a \skill\ $S$ can be executed with the two objects, that reduces the difference between $e.contentLevel$ and $target.contentLevel$, the \skill\ is added to the execution plan. If, after checking all objects, $e.contentLevel \not\in target.contentLevel$, there is no solution to solve this property difference.

Thus, the result of the third step is an execution plan $P_x$ for each entity property difference.

Fourth, after having the execution plans $P_x$ per entity-variation and entity, a \underline{solution selector} scores all solutions according to defined metrics and then, via a maximal matching algorithm, selects the solutions to execute to satisfy all variations of the Collection-RangeVariation of type $A$. The edges in the maximal matching have the cost of the solution score. For this paper, the scoring metric by the \underline{solution selector} is the number of steps of the execution plan.

The fifth and final step is to pass the execution plan to the agent(s) to execute in the environment. Figure \ref{fig:data_flow} presents the flow of data through the five steps.
We have used the Franka Emika Panda robot in CoppeliaSim \cite{coppeliaSim} to perform the computed execution plan.
% Note that the approach is independent of the used robot; only when instantiating \skills\ must the robot's abilities, manipulability region, and workspace be considered. How the \skills\ are executed in the environment is separated from the modeling of what must be done.

\begin{figure}[t!]
    % \vspace{-0.2cm}
    \centering
    \includegraphics[width=1\linewidth]{images/Experiment_DescriptionUsingVariations_DataFlow_2.png}
    \caption{Data flow when transforming an environment into a given goal state. $\Delta$ are differences of entity properties $p$, $A$ are \actions, $P$ is an execution plan and $S$ are \skills.} \label{fig:data_flow}
\end{figure}

The experiments aim to compute solution plans for solving the difference of the \textbf{contentLevel} property of \textit{Container} objects. For this, we consider the following criteria. $C1$: \textbf{variation} type = $\left\{\text{fixed},\text{interval},\text{interval union}\right\}$. $C2$: target relative to content = \{$\left\{t < cL \le cV \right\}$, $\left\{cL < t < cV \right\}$, $\left\{cL < t \ni cV \right\}$, $\left\{cL \le cV < t \right\}$\}, where $t$ is the \textbf{variation} value and $cL$ and $cV$ are the \textit{contentLevel} and \textit{contentVolume} properties respectively. $C3$: achievable in environment $ = \left\{\text{yes}, \text{no}\right\}$. Figure \ref{fig:experiment_table} presents planning results for different environments and the criteria described above. The lower table shows cases where the computed solution does not match the actual solution. This only happens when multiple instance variations are defined. The reason is that the implemented procedure to turn the list of differences into an execution plan treats each difference independently. Thus, dependencies between two variations are not accurately solved.

In the upper table of Figure \ref{fig:experiment_table}, there are two solutions for $C1.3$, $C2.3$, $C3.1$: one with the bowl $B$ as the instance in the \textbf{variation} $V1$, the other with $M$. The solution when $B$ is the matched instance has three steps: 1) pouring $0.1L$ from $M$ into $B$, 2) pouring $0.1L$ from $C1$ into $B$, and, finally, 3) pouring  $0.02L$ from $C2$ into $B$. This plan is sent to the robot in simulation and is executed as shown in Figure \ref{fig:robot_plan_execution}.

% \begin{figure}[t!]
%     % \vspace{-0.2cm}
%     \centering
%     \includegraphics[width=1\linewidth]{images/Experiment_Table_1Variation_compressed.png}
%     \caption{$B$ is a bowl with $0.5L$ \textit{contentVolume}, $M$ is a milk carton with $1.0L$ \textit{contentVolume}, $C1$ and $C2$ are cups with $0.3L$ \textit{contentVolume} each. Times, in seconds, averaged across 10 runs. Criteria $C2.4$ and $C3.1$ are mutually exclusive (a solution does not exist to let a container have more \textit{contentLevel} than its \textit{contentVolume}); thus, they are not included in the table.} \label{fig:experiment_table}
% \end{figure}
\begin{figure}[t!]
    % \vspace{-0.2cm}
    \centering
    \includegraphics[width=1\linewidth]{images/Experiment_Table_Results.png}
    \caption{$B$ is a bowl with $0.5L$ \textit{contentVolume}, $M$ is a milk carton with $1.0L$ \textit{contentVolume}, $C1$ and $C2$ are cups with $0.3L$ \textit{contentVolume} each. Times, in seconds, averaged across 10 runs. Criteria $C2.4$ and $C3.1$ are mutually exclusive (a solution does not exist to let a container have more \textit{contentLevel} than its \textit{contentVolume}); thus, they are not included in the upper table. The lower table presents results for open intervals and multiple variations in the environment.} \label{fig:experiment_table}
\end{figure}

\begin{figure}[t!]
    %\vspace{-0.1cm}
    \centering
    \includegraphics[width=1\linewidth]{images/Robot_PouringInBowl_M_PC1_PC2.png}
    \caption{Robot executing plan to bring $B$, the bowl, into the goal state. Because no liquids were simulated, the pouring amount was associated with the pouring time via: $t_{pour} = 10 * amount_{pour}$.} \label{fig:robot_plan_execution}
\end{figure}

\section{Conclusion}

We introduce EigenLoRAx, a significantly efficient model finetuning and inference method that recycles publicly available pretrained adapters by finding a shared principal subspace. This allows finetuning on new data by simply learning the lightweight coefficients of the shared subspace, and also requires less number of parameters to be saved for new tasks. Our comprehensive and diverse experiments show that EigenLoRAx is applicable to a large range of problems and model architectures. We believe that EigenLoRAx has the potential to mitigate the perpetually widening compute resource gap~\citep{Ahmed2020TheDO, Besiroglu2024TheCD} and reduce the environmental cost of training and using machine learning models~\citep{Wu2021SustainableAE, Ligozat2021UnravelingTH}. It also holds promise for training personalized models~\citep{tan2024democratizing} on low-resource devices, in privacy-critical use-cases. We have a large number of experiments ($6+$) on a diverse set of complex models, tasks and modalities. We have shown that EigenLoRAx excels in faster and efficient learning, memory savings and zero shot performance which differentiates it from conventional PeFT models. While our work is application-focused, we believe it is the first work to hypothesize and empirically prove the existence of shared weight subspaces of neural networks. This important insight has significant implications of model merging, efficiency, mechanistic interpretability, and neural learning theory.

{\small\paragraph{A Note on Prior Work} This paper works on some ideas initially introduced in the EigenLoRA work~\cite{kaushik2025eigenlora}, which was rejected from ICLR'25 and remains available on OpenReview. Due to irreconcilable differences regarding research ethics and authorship integrity, the primary authors of this work chose to independently develop and extend their contributions. These contributions in the previous draft included the initial algorithm, all experimental results (excluding Section 4.2.2), as well as all ablation studies and additional experiments. The code used for these experiments was also developed solely by the first authors of this work and is publicly available. Since the prior work was neither accepted nor published, there are no concerns regarding self-plagiarism. We also acknowledge discussions and exchanges with some of the earlier authors, which, while not directly contributing to the technical developments of this work, provided context and background that shaped our understanding of the broader problem. We have taken extensive measures to ensure that this paper exclusively reflects the contributions of the authors listed, and any similarities beyond our own contributions are purely coincidental.}

\nocite{sun2025transformersquaredselfadaptivellms, Gain_2020_WACV, Kaushik_2024_CVPR}
\bibliography{iclr2025_conference}
\bibliographystyle{icml2025}


%%%%%%%%%%%%%%%%%%%%%%%%%%%%%%%%%%%%%%%%%%%%%%%%%%%%%%%%%%%%%%%%%%%%%%%%%%%%%%%
%%%%%%%%%%%%%%%%%%%%%%%%%%%%%%%%%%%%%%%%%%%%%%%%%%%%%%%%%%%%%%%%%%%%%%%%%%%%%%%
% APPENDIX
%%%%%%%%%%%%%%%%%%%%%%%%%%%%%%%%%%%%%%%%%%%%%%%%%%%%%%%%%%%%%%%%%%%%%%%%%%%%%%%
%%%%%%%%%%%%%%%%%%%%%%%%%%%%%%%%%%%%%%%%%%%%%%%%%%%%%%%%%%%%%%%%%%%%%%%%%%%%%%%
\newpage
\appendix
\onecolumn
% E5数据集介绍,数据集处理过程
% 基线模型介绍

\definecolor{titlecolor}{rgb}{0.9, 0.5, 0.1}
\definecolor{anscolor}{rgb}{0.2, 0.5, 0.8}
\definecolor{labelcolor}{HTML}{48a07e}
\begin{table*}[h]
	\centering
	
 % \vspace{-0.2cm}
	
	\begin{center}
		\begin{tikzpicture}[
				chatbox_inner/.style={rectangle, rounded corners, opacity=0, text opacity=1, font=\sffamily\scriptsize, text width=5in, text height=9pt, inner xsep=6pt, inner ysep=6pt},
				chatbox_prompt_inner/.style={chatbox_inner, align=flush left, xshift=0pt, text height=11pt},
				chatbox_user_inner/.style={chatbox_inner, align=flush left, xshift=0pt},
				chatbox_gpt_inner/.style={chatbox_inner, align=flush left, xshift=0pt},
				chatbox/.style={chatbox_inner, draw=black!25, fill=gray!7, opacity=1, text opacity=0},
				chatbox_prompt/.style={chatbox, align=flush left, fill=gray!1.5, draw=black!30, text height=10pt},
				chatbox_user/.style={chatbox, align=flush left},
				chatbox_gpt/.style={chatbox, align=flush left},
				chatbox2/.style={chatbox_gpt, fill=green!25},
				chatbox3/.style={chatbox_gpt, fill=red!20, draw=black!20},
				chatbox4/.style={chatbox_gpt, fill=yellow!30},
				labelbox/.style={rectangle, rounded corners, draw=black!50, font=\sffamily\scriptsize\bfseries, fill=gray!5, inner sep=3pt},
			]
											
			\node[chatbox_user] (q1) {
				\textbf{System prompt}
				\newline
				\newline
				You are a helpful and precise assistant for segmenting and labeling sentences. We would like to request your help on curating a dataset for entity-level hallucination detection.
				\newline \newline
                We will give you a machine generated biography and a list of checked facts about the biography. Each fact consists of a sentence and a label (True/False). Please do the following process. First, breaking down the biography into words. Second, by referring to the provided list of facts, merging some broken down words in the previous step to form meaningful entities. For example, ``strategic thinking'' should be one entity instead of two. Third, according to the labels in the list of facts, labeling each entity as True or False. Specifically, for facts that share a similar sentence structure (\eg, \textit{``He was born on Mach 9, 1941.''} (\texttt{True}) and \textit{``He was born in Ramos Mejia.''} (\texttt{False})), please first assign labels to entities that differ across atomic facts. For example, first labeling ``Mach 9, 1941'' (\texttt{True}) and ``Ramos Mejia'' (\texttt{False}) in the above case. For those entities that are the same across atomic facts (\eg, ``was born'') or are neutral (\eg, ``he,'' ``in,'' and ``on''), please label them as \texttt{True}. For the cases that there is no atomic fact that shares the same sentence structure, please identify the most informative entities in the sentence and label them with the same label as the atomic fact while treating the rest of the entities as \texttt{True}. In the end, output the entities and labels in the following format:
                \begin{itemize}[nosep]
                    \item Entity 1 (Label 1)
                    \item Entity 2 (Label 2)
                    \item ...
                    \item Entity N (Label N)
                \end{itemize}
                % \newline \newline
                Here are two examples:
                \newline\newline
                \textbf{[Example 1]}
                \newline
                [The start of the biography]
                \newline
                \textcolor{titlecolor}{Marianne McAndrew is an American actress and singer, born on November 21, 1942, in Cleveland, Ohio. She began her acting career in the late 1960s, appearing in various television shows and films.}
                \newline
                [The end of the biography]
                \newline \newline
                [The start of the list of checked facts]
                \newline
                \textcolor{anscolor}{[Marianne McAndrew is an American. (False); Marianne McAndrew is an actress. (True); Marianne McAndrew is a singer. (False); Marianne McAndrew was born on November 21, 1942. (False); Marianne McAndrew was born in Cleveland, Ohio. (False); She began her acting career in the late 1960s. (True); She has appeared in various television shows. (True); She has appeared in various films. (True)]}
                \newline
                [The end of the list of checked facts]
                \newline \newline
                [The start of the ideal output]
                \newline
                \textcolor{labelcolor}{[Marianne McAndrew (True); is (True); an (True); American (False); actress (True); and (True); singer (False); , (True); born (True); on (True); November 21, 1942 (False); , (True); in (True); Cleveland, Ohio (False); . (True); She (True); began (True); her (True); acting career (True); in (True); the late 1960s (True); , (True); appearing (True); in (True); various (True); television shows (True); and (True); films (True); . (True)]}
                \newline
                [The end of the ideal output]
				\newline \newline
                \textbf{[Example 2]}
                \newline
                [The start of the biography]
                \newline
                \textcolor{titlecolor}{Doug Sheehan is an American actor who was born on April 27, 1949, in Santa Monica, California. He is best known for his roles in soap operas, including his portrayal of Joe Kelly on ``General Hospital'' and Ben Gibson on ``Knots Landing.''}
                \newline
                [The end of the biography]
                \newline \newline
                [The start of the list of checked facts]
                \newline
                \textcolor{anscolor}{[Doug Sheehan is an American. (True); Doug Sheehan is an actor. (True); Doug Sheehan was born on April 27, 1949. (True); Doug Sheehan was born in Santa Monica, California. (False); He is best known for his roles in soap operas. (True); He portrayed Joe Kelly. (True); Joe Kelly was in General Hospital. (True); General Hospital is a soap opera. (True); He portrayed Ben Gibson. (True); Ben Gibson was in Knots Landing. (True); Knots Landing is a soap opera. (True)]}
                \newline
                [The end of the list of checked facts]
                \newline \newline
                [The start of the ideal output]
                \newline
                \textcolor{labelcolor}{[Doug Sheehan (True); is (True); an (True); American (True); actor (True); who (True); was born (True); on (True); April 27, 1949 (True); in (True); Santa Monica, California (False); . (True); He (True); is (True); best known (True); for (True); his roles in soap operas (True); , (True); including (True); in (True); his portrayal (True); of (True); Joe Kelly (True); on (True); ``General Hospital'' (True); and (True); Ben Gibson (True); on (True); ``Knots Landing.'' (True)]}
                \newline
                [The end of the ideal output]
				\newline \newline
				\textbf{User prompt}
				\newline
				\newline
				[The start of the biography]
				\newline
				\textcolor{magenta}{\texttt{\{BIOGRAPHY\}}}
				\newline
				[The ebd of the biography]
				\newline \newline
				[The start of the list of checked facts]
				\newline
				\textcolor{magenta}{\texttt{\{LIST OF CHECKED FACTS\}}}
				\newline
				[The end of the list of checked facts]
			};
			\node[chatbox_user_inner] (q1_text) at (q1) {
				\textbf{System prompt}
				\newline
				\newline
				You are a helpful and precise assistant for segmenting and labeling sentences. We would like to request your help on curating a dataset for entity-level hallucination detection.
				\newline \newline
                We will give you a machine generated biography and a list of checked facts about the biography. Each fact consists of a sentence and a label (True/False). Please do the following process. First, breaking down the biography into words. Second, by referring to the provided list of facts, merging some broken down words in the previous step to form meaningful entities. For example, ``strategic thinking'' should be one entity instead of two. Third, according to the labels in the list of facts, labeling each entity as True or False. Specifically, for facts that share a similar sentence structure (\eg, \textit{``He was born on Mach 9, 1941.''} (\texttt{True}) and \textit{``He was born in Ramos Mejia.''} (\texttt{False})), please first assign labels to entities that differ across atomic facts. For example, first labeling ``Mach 9, 1941'' (\texttt{True}) and ``Ramos Mejia'' (\texttt{False}) in the above case. For those entities that are the same across atomic facts (\eg, ``was born'') or are neutral (\eg, ``he,'' ``in,'' and ``on''), please label them as \texttt{True}. For the cases that there is no atomic fact that shares the same sentence structure, please identify the most informative entities in the sentence and label them with the same label as the atomic fact while treating the rest of the entities as \texttt{True}. In the end, output the entities and labels in the following format:
                \begin{itemize}[nosep]
                    \item Entity 1 (Label 1)
                    \item Entity 2 (Label 2)
                    \item ...
                    \item Entity N (Label N)
                \end{itemize}
                % \newline \newline
                Here are two examples:
                \newline\newline
                \textbf{[Example 1]}
                \newline
                [The start of the biography]
                \newline
                \textcolor{titlecolor}{Marianne McAndrew is an American actress and singer, born on November 21, 1942, in Cleveland, Ohio. She began her acting career in the late 1960s, appearing in various television shows and films.}
                \newline
                [The end of the biography]
                \newline \newline
                [The start of the list of checked facts]
                \newline
                \textcolor{anscolor}{[Marianne McAndrew is an American. (False); Marianne McAndrew is an actress. (True); Marianne McAndrew is a singer. (False); Marianne McAndrew was born on November 21, 1942. (False); Marianne McAndrew was born in Cleveland, Ohio. (False); She began her acting career in the late 1960s. (True); She has appeared in various television shows. (True); She has appeared in various films. (True)]}
                \newline
                [The end of the list of checked facts]
                \newline \newline
                [The start of the ideal output]
                \newline
                \textcolor{labelcolor}{[Marianne McAndrew (True); is (True); an (True); American (False); actress (True); and (True); singer (False); , (True); born (True); on (True); November 21, 1942 (False); , (True); in (True); Cleveland, Ohio (False); . (True); She (True); began (True); her (True); acting career (True); in (True); the late 1960s (True); , (True); appearing (True); in (True); various (True); television shows (True); and (True); films (True); . (True)]}
                \newline
                [The end of the ideal output]
				\newline \newline
                \textbf{[Example 2]}
                \newline
                [The start of the biography]
                \newline
                \textcolor{titlecolor}{Doug Sheehan is an American actor who was born on April 27, 1949, in Santa Monica, California. He is best known for his roles in soap operas, including his portrayal of Joe Kelly on ``General Hospital'' and Ben Gibson on ``Knots Landing.''}
                \newline
                [The end of the biography]
                \newline \newline
                [The start of the list of checked facts]
                \newline
                \textcolor{anscolor}{[Doug Sheehan is an American. (True); Doug Sheehan is an actor. (True); Doug Sheehan was born on April 27, 1949. (True); Doug Sheehan was born in Santa Monica, California. (False); He is best known for his roles in soap operas. (True); He portrayed Joe Kelly. (True); Joe Kelly was in General Hospital. (True); General Hospital is a soap opera. (True); He portrayed Ben Gibson. (True); Ben Gibson was in Knots Landing. (True); Knots Landing is a soap opera. (True)]}
                \newline
                [The end of the list of checked facts]
                \newline \newline
                [The start of the ideal output]
                \newline
                \textcolor{labelcolor}{[Doug Sheehan (True); is (True); an (True); American (True); actor (True); who (True); was born (True); on (True); April 27, 1949 (True); in (True); Santa Monica, California (False); . (True); He (True); is (True); best known (True); for (True); his roles in soap operas (True); , (True); including (True); in (True); his portrayal (True); of (True); Joe Kelly (True); on (True); ``General Hospital'' (True); and (True); Ben Gibson (True); on (True); ``Knots Landing.'' (True)]}
                \newline
                [The end of the ideal output]
				\newline \newline
				\textbf{User prompt}
				\newline
				\newline
				[The start of the biography]
				\newline
				\textcolor{magenta}{\texttt{\{BIOGRAPHY\}}}
				\newline
				[The ebd of the biography]
				\newline \newline
				[The start of the list of checked facts]
				\newline
				\textcolor{magenta}{\texttt{\{LIST OF CHECKED FACTS\}}}
				\newline
				[The end of the list of checked facts]
			};
		\end{tikzpicture}
        \caption{GPT-4o prompt for labeling hallucinated entities.}\label{tb:gpt-4-prompt}
	\end{center}
\vspace{-0cm}
\end{table*}

% \begin{figure}[t]
%     \centering
%     \includegraphics[width=0.9\linewidth]{Image/abla2/doc7.png}
%     \caption{Improvement of generated documents over direct retrieval on different models.}
%     \label{fig:comparison}
% \end{figure}

\begin{figure}[t]
    \centering
    \subfigure[Unsupervised Dense Retriever.]{
        \label{fig:imp:unsupervised}
        \includegraphics[width=0.8\linewidth]{Image/A.3_fig/improvement_unsupervised.pdf}
    }
    \subfigure[Supervised Dense Retriever.]{
        \label{fig:imp:supervised}
        \includegraphics[width=0.8\linewidth]{Image/A.3_fig/improvement_supervised.pdf}
    }
    
    % \\
    % \subfigure[Comparison of Reasoning Quality With Different Method.]{
    %     \label{fig:reasoning} 
    %     \includegraphics[width=0.98\linewidth]{images/reasoning1.pdf}
    % }
    \caption{Improvements of LLM-QE in Both Unsupervised and Supervised Dense Retrievers. We plot the change of nDCG@10 scores before and after the query expansion using our LLM-QE model.}
    \label{fig:imp}
\end{figure}
\section{Appendix}
\subsection{License}
The authors of 4 out of the 15 datasets in the BEIR benchmark (NFCorpus, FiQA-2018, Quora, Climate-Fever) and the authors of ELI5 in the E5 dataset do not report the dataset license in the paper or a repository. We summarize the licenses of the remaining datasets as follows.

MS MARCO (MIT License); FEVER, NQ, and DBPedia (CC BY-SA 3.0 license); ArguAna and Touché-2020 (CC BY 4.0 license); CQADupStack and TriviaQA (Apache License 2.0); SciFact (CC BY-NC 2.0 license); SCIDOCS (GNU General Public License v3.0); HotpotQA and SQuAD (CC BY-SA 4.0 license); TREC-COVID (Dataset License Agreement).

All these licenses and agreements permit the use of their data for academic purposes.

\subsection{Additional Experimental Details}\label{app:experiment_detail}
This subsection outlines the components of the training data and presents the prompt templates used in the experiments.


\textbf{Training Datasets.} Following the setup of \citet{wang2024improving}, we use the following datasets: ELI5 (sample ratio 0.1)~\cite{fan2019eli5}, HotpotQA~\cite{yang2018hotpotqa}, FEVER~\cite{thorne2018fever}, MS MARCO passage ranking (sample ratio 0.5) and document ranking (sample ratio 0.2)~\cite{bajaj2016ms}, NQ~\cite{karpukhin2020dense}, SQuAD~\cite{karpukhin2020dense}, and TriviaQA~\cite{karpukhin2020dense}. In total, we use 808,740 training examples.

\textbf{Prompt Templates.} Table~\ref{tab:prompt_template} lists all the prompts used in this paper. In each prompt, ``query'' refers to the input query for which query expansions are generated, while ``Related Document'' denotes the ground truth document relevant to the original query. We observe that, in general, the model tends to generate introductory phrases such as ``Here is a passage to answer the question:'' or ``Here is a list of keywords related to the query:''. Before using the model outputs as query expansions, we first filter out these introductory phrases to ensure cleaner and more precise expansion results.



\subsection{Query Expansion Quality of LLM-QE}\label{app:analysis}
This section evaluates the quality of query expansion of LLM-QE. As shown in Figure~\ref{fig:imp}, we randomly select 100 samples from each dataset to assess the improvement in retrieval performance before and after applying LLM-QE.

Overall, the evaluation results demonstrate that LLM-QE consistently improves retrieval performance in both unsupervised (Figure~\ref{fig:imp:unsupervised}) and supervised (Figure~\ref{fig:imp:supervised}) settings. However, for the MS MARCO dataset, LLM-QE demonstrates limited effectiveness in the supervised setting. This can be attributed to the fact that MS MARCO provides higher-quality training signals, allowing the dense retriever to learn sufficient matching signals from relevance labels. In contrast, LLM-QE leads to more substantial performance improvements on the NQ and HotpotQA datasets. This indicates that LLM-QE provides essential matching signals for dense retrievers, particularly in retrieval scenarios where high-quality training signals are scarce.


\subsection{Case Study}\label{app:case_study}
\begin{figure}[htb]
\small
\begin{tcolorbox}[left=3pt,right=3pt,top=3pt,bottom=3pt,title=\textbf{Conversation History:}]
[human]: Craft an intriguing opening paragraph for a fictional short story. The story should involve a character who wakes up one morning to find that they can time travel.

...(Human-Bot Dialogue Turns)... \textcolor{blue}{(Topic: Time-Travel Fiction)}

[human]: Please describe the concept of machine learning. Could you elaborate on the differences between supervised, unsupervised, and reinforcement learning? Provide real-world examples of each.

...(Human-Bot Dialogue Turns)... \textcolor{blue}{(Topic: Machine learning Concepts and Types)}


[human]: Discuss antitrust laws and their impact on market competition. Compare the antitrust laws in US and China along with some case studies

...(Human-Bot Dialogue Turns)... \textcolor{blue}{(Topic: Antitrust Laws and Market Competition)}

[human]: The vertices of a triangle are at points (0, 0), (-1, 1), and (3, 3). What is the area of the triangle?

...(Human-Bot Dialogue Turns)... \textcolor{blue}{(Topic: Area and Circumcircle of Triangle)}

...(Human-Bot Dialogue Interaction on Other Topics)... 

\end{tcolorbox}
\begin{tcolorbox}[left=3pt,right=3pt,top=3pt,bottom=3pt,title=\textbf{Question:}]
What title reflects the thematic essence of the Doctor's experience with time travel as described in the story?
\end{tcolorbox}
\begin{tcolorbox}[left=3pt,right=3pt,top=3pt,bottom=3pt,title=\textbf{Ground-Truth Answer:}]
Awakening to Eternity: The Doctor's New Dawn
\end{tcolorbox}
\begin{tcolorbox}[left=3pt,right=3pt,top=3pt,bottom=3pt,title=\textbf{Retrieved Memory (Turn-Level):}]
[human]: Describe the process and write out the balanced chemical equation for the reaction that occurs when solid calcium carbonate reacts with hydrochloric acid to form aqueous calcium chloride, carbon dioxide, and water. What type of reaction is this, and what observations might indicate that the reaction is taking place?

[bot]: When solid calcium carbonate ($\text{CaCO}_{3}$) reacts with hydrochloric acid (HCl), the products formed are aqueous calcium chloride ($\text{CaCl}_{2}$), carbon dioxide ($\text{CO}_{2}$), and water ($\text{H}_{2}\text{O}$). The balanced chemical equation for this reaction is: ... \textcolor{blue}{(Topic: Calcium Carbonate and Hydrochloric Acid Reaction)}

\textcolor{brown}{
[human]: Write a title for your story.}

\textcolor{brown}{
[bot]: "Awakening to Eternity: The Doctor's New Dawn".
} \textcolor{blue}{(Topic: Time-Travel Fiction)}

[human]: Analyze the following customer reviews from different sources for three different smartphones - the latest iPhone, Samsung Galaxy, and Google Pixel - and provide an overall rating for each phone on a scale of 1 to 10. Consider the following complex and contradictory reviews: - TechRadar's review of the latest iPhone: The new iPhone is a stunning triumph of engineering that sets a new bar for smartphone performance and camera quality. However, the incremental design and high price mean it lacks the 'wow' factor of previous iPhones. Still, its power and intelligence are unrivaled. - CNET's review of the latest Samsung Galaxy: ... Return the answer as a JSON object with the overall ratings for each phone out of 10, to one decimal place.

[bot]: ```json { "iPhone": 8.5, "Samsung Galaxy": 8.0, "Google Pixel": 7.5 } ```

\textcolor{blue}{(Topic: Smartphone Review Ratings Analysis)}
\end{tcolorbox}
\begin{tcolorbox}[left=3pt,right=3pt,top=3pt,bottom=3pt,title=\textbf{Answer (Turn-Level Retrieval):}]
As an AI language model, I do not have access to the story you mentioned earlier. Please provide me with the story or the necessary details so that I can assist you better.

\textbf{\textcolor{red}{(The next part of this figure~\ref{fig: case_study_segment_vs_turn} will be continued on the next page.)}}
\end{tcolorbox}
\end{figure}

\begin{figure}[htb]
% \ContinuedFloat
\small
\begin{tcolorbox}[left=3pt,right=3pt,top=3pt,bottom=3pt,title=\textbf{Retrieved Memory (Ours):}]
[human]: Craft an intriguing opening paragraph for a fictional short story. The story should involve a character who wakes up one morning to find that they can time travel.

[bot]: In the hushed pre-dawn hours, when the world still slumbers under the cloak of night, Alex found themselves caught in the fragile boundary between sleep and wakefulness. ...

[human]: Summarize the story with three bullet points using only nouns and adjectives, without verbs.

[bot]: - Alex, pre-dawn awakening, time-travel discovery - Disorientation, vertigo, reality's malleability - Sunlight, new day, astonishing truth

[human]: Is there any famous fictional character that can time travel? If so, can you rewrite your story with him or her?

[bot]: - \textcolor{brown}{Doctor Who, a renowned time-traveling character from British television, is a perfect fit for this scenario.} In the silent embrace of early morning, where shadows merge with the fading night, the Doctor found themselves in the thin veil separating dreams from consciousness. It was here, amidst the soft whispers of time, that an unusual sensation enveloped them—a sense of time's fabric slipping through their fingers like grains of sand. As dawn's light broke through the darkness, painting the room in hues of gold and amber, the Doctor's eyes opened to what appeared to be an ordinary day. \textcolor{brown}{However, upon standing, a wave of dizziness overwhelmed them, the surroundings blurred, and in that moment of confusion, the Doctor realized a profound truth: they had not only awakened to a new dawn but had also stumbled upon the incredible ability to navigate the currents of time at will.}

\textcolor{brown}{[human]: Write a title for your story.}

\textcolor{brown}{[bot]: "Awakening to Eternity: The Doctor's New Dawn"} \textcolor{blue}{(Topic: Time-Travel Fiction)}
\end{tcolorbox}
\begin{tcolorbox}[left=3pt,right=3pt,top=3pt,bottom=3pt,title=\textbf{Answer (Ours):}]
The title "Awakening to Eternity: The Doctor's New Dawn" reflects the thematic essence of the Doctor's experience with time travel as described in the story. It captures the idea of a new beginning, a sense of eternity, and the awakening of a new power that will change the course of the Doctor's life forever.
\end{tcolorbox}
\caption{Case study comparing our \sysname\ with the \textit{turn-level} memory baseline. When constructing the memory bank at the \textit{turn-level}, the agent fails to provide a satisfactory response due to fragmentation: the retrieved dialogue turns can not capture a complete human-agent interaction related to the query, missing critical information highlighted in \textcolor{brown}{brown}. The dialogue topic is emphasized in \textcolor{blue}{blue}. Some context is omitted (replaced by ...) due to length constraints.}
\label{fig: case_study_segment_vs_turn}
\end{figure}

\newpage

\begin{figure}[htbp]
\small
\begin{tcolorbox}[left=3pt,right=3pt,top=3pt,bottom=3pt,title=\textbf{Conversation History:}]
[human]: Explain what's base rate fallacy and list five specific examples of ...

... (Human-Bot Dialogue Turns) ... \textcolor{blue}{(Topic: Base Rate Fallacy In Politics)}


[human]: How many integers are in the solution of the inequality $|x + 5| < 10$?

... (Human-Bot Dialogue Turns) ... \textcolor{blue}{(Topic: Number of Integers in Inequalities)}

[human]: How have the Alps and Rhine River influenced settlement and agriculture in Western Europe?

... (Human-Bot Dialogue Turns) ... \textcolor{blue}{(Topic: Alps and Rhine River Impact)}

[human]: In a survey conducted at a local high school, preferences for a new school color were measured: 58\% of students liked the color blue, 45\% preferred green, and 22\% liked both colors. If we randomly pick a student from the school, what's the probability that they would like neither blue nor green?

... (Human-Bot Dialogue Turns) ... \textcolor{blue}{(Topic: Probability of Liking Colors in Survey)}

[human]: Imagine you are participating in a race with a group of people. If you have just overtaken the second person, what's your current position? Where is the person you just overtook?

... (Human-Bot Dialogue Turns) ... \textcolor{blue}{(Topic: Overtaking Positions in Race)}
\end{tcolorbox}
\begin{tcolorbox}[left=3pt,right=3pt,top=3pt,bottom=3pt,title=\textbf{Question:}]
[human]: In the inequality question, how many positive integers are in the solution of the second inequality?
\end{tcolorbox}
\begin{tcolorbox}[left=3pt,right=3pt,top=3pt,bottom=3pt,title=\textbf{Ground-Truth Answer:}]
In the second inequality, $(|x + 10| < 5)$, the solution range is $(-15 < x < -5)$. Within this range, there are no positive integers, as all values of $(x)$ that satisfy the inequality are negative.
\end{tcolorbox}
\begin{tcolorbox}[left=3pt,right=3pt,top=3pt,bottom=3pt,title=\textbf{Retrieved Memory (Session-Level):}]
[human]: Explain what's base rate fallacy and list five specific examples of ...

... (Human-Bot Dialogue Turns) ... \textcolor{blue}{(Topic: Base Rate Fallacy In Politics)}

[human]: How many integers are in the solution of the inequality $|x + 5| < 10$?

[bot]: To solve the inequality $|x + 5| < 10$, ... 

\textcolor{brown}{
[human]: What about $|x + 10| < 5$?
}

\textcolor{brown}{
[bot]: To solve the inequality $|x + 10| < 5$, we ...} \textcolor{gray}{(``overwhelmed'' by other irrelevant, redundant content)}

... (Human-Bot Dialogue Turns) ...  \textcolor{blue}{(Topic: Number of Integers in Inequalities)}

% \textcolor{brown}{
% [human]: What about $|x + 10| < 5$
% }

% \textcolor{brown}{
% [bot]: To solve the inequality $|x + 10| < 5$, ...
% } \textcolor{blue}{(Topic: Number of Integers in Inequalities)}

[human]: How have the Alps and Rhine River influenced settlement and agriculture in Western Europe?

... (Human-Bot Dialogue Turns) ... \textcolor{blue}{(Topic: Alps and Rhine River Impact)}

[human]: In a survey conducted at a local high school, ...

... (Human-Bot Dialogue Turns) ... \textcolor{blue}{(Topic: Probability of Liking Colors in Survey)}
\end{tcolorbox}
\begin{tcolorbox}[left=3pt,right=3pt,top=3pt,bottom=3pt,title=\textbf{Answer (Session-Level Retrieval):}]
I apologize for the confusion earlier. To answer your question, the second inequality was not provided in our previous conversation. If you could provide the inequality, I would be happy to help you find the solution and the number of positive integers in the solution set.

\textbf{\textcolor{red}{(The next part of this figure~\ref{fig: case_study_segment_vs_session} will be continued on the next page.)}}
\end{tcolorbox}
\end{figure}
\begin{figure}[htbp]
% \ContinuedFloat
\small
\begin{tcolorbox}[left=3pt,right=3pt,top=3pt,bottom=3pt,title=\textbf{Retrieved Memory (Ours):}]
[human]: How many integers are in the solution of the inequality $|x + 5| < 10$?

[bot]: To solve the inequality $|x + 5| < 10$, we break it into two cases based on the definition of absolute value: 
1. $x + 5 < 10$ 
2. $-(x + 5) < 10$, which simplifies to $x + 5 > -10$ 
Solving each case: 
1. $x < 5$ 
2. $x > -15$ 
Combining these, we get $-15 < x < 5$. This inequality means that $x$ can be any number greater than $-15$ and less than $5$. To find how many integers satisfy this condition, we list them: $-14, -13, -12, -11, -10, -9, -8, -7, -6, -5, -4, -3, -2, -1, 0, 1, 2, 3, 4$. There are 19 integers that satisfy the inequality $|x + 5| < 10$.

\textcolor{brown}{
[human]: What about $|x + 10| < 5$?
}

\textcolor{brown}{
[bot]: To solve the inequality $|x + 10| < 5$, we again consider two cases: 
1. $x + 10 < 5$ 
2. $-(x + 10) < 5$, which simplifies to $x + 10 > -5$ 
Solving each case: 
1. $x < -5$ 
2. $x > -15$ 
Combining these, we get $-15 < x < -5$. This inequality means that $x$ can be any number greater than $-15$ and less than $-5$. To find how many integers satisfy this condition, we list them: $-14, -13, -12, -11, -10, -9, -8, -7, -6$. There are 9 integers that satisfy the inequality $|x + 10| < 5$.
}

[human]: What about $|x + 10| < 10$?

[bot]: To solve the inequality $|x + 10| < 10$, we consider two cases: 
1. $x + 10 < 10$ 
2. $-(x + 10) < 10$, which simplifies to $x + 10 > -10$ 
Solving each case: 
1. $x < 0$ 
2. $x > -20$ 
Combining these, we get $-20 < x < 0$. This inequality means that $x$ can be any number greater than $-20$ and less than $0$. To find how many integers satisfy this condition, we list them: $-19, -18, -17, -16, -15, -14, -13, -12, -11, -10, -9, -8, -7, -6, -5, -4, -3, -2, -1$. There are 19 integers that satisfy the inequality $|x + 10| < 10$. \textcolor{blue}{(Topic: Number of Integers in Inequalities)}
\end{tcolorbox}
\begin{tcolorbox}[left=3pt,right=3pt,top=3pt,bottom=3pt,title=\textbf{Answer (Ours):}]
To find the number of positive integers that satisfy the inequality $(|x + 10| < 5)$, we need to look for positive integers in the solution set $(-15 < x < -5)$. Since there are no positive integers in this interval, the answer is 0.

\end{tcolorbox}
\caption{Case study comparing our \sysname\ with the \textit{session-level} memory baseline. When constructing the memory bank at the \textit{session-level}, the agent fails to provide a satisfactory response due to the inclusion of excessive irrelevant information within a session. This distracts the generator from focusing on the truly useful information highlighted in \textcolor{brown}{brown}. The dialogue topic is emphasized in \textcolor{blue}{blue}. Some context is omitted (replaced by ...) due to length constraints.}
\label{fig: case_study_segment_vs_session}
\end{figure}



\begin{figure}[htb]
\small
\begin{tcolorbox}[left=3pt,right=3pt,top=3pt,bottom=3pt,title=\textbf{Conversation History:}]
[human]: Photosynthesis is a vital process for life on Earth. Could you outline the two main stages of photosynthesis, including where they take place within the chloroplast, and the primary inputs and outputs for each stage? ... (Human-Bot Dialogue Turns)... \textcolor{blue}{(Topic: Photosynthetic Energy Production)}

[human]: Please assume the role of an English translator, tasked with correcting and enhancing spelling and language. Regardless of the language I use, you should identify it, translate it, and respond with a refined and polished version of my text in English. 

... (Human-Bot Dialogue Turns)...  \textcolor{blue}{(Topic: Language Translation and Enhancement)}

[human]: Suggest five award-winning documentary films with brief background descriptions for aspiring filmmakers to study.

\textcolor{brown}{[bot]: ...
5. \"An Inconvenient Truth\" (2006) - Directed by Davis Guggenheim and featuring former United States Vice President Al Gore, this documentary aims to educate the public about global warming. It won two Academy Awards, including Best Documentary Feature. The film is notable for its straightforward yet impactful presentation of scientific data, making complex information accessible and engaging, a valuable lesson for filmmakers looking to tackle environmental or scientific subjects.}

... (Human-Bot Dialogue Turns)... 
\textcolor{blue}{(Topic: Documentary Films Recommendation)}

[human]: Given the following records of stock prices, extract the highest and lowest closing prices for each month in the year 2022. Return the results as a CSV string, with one line allocated for each month. Date,Open,High,Low,Close,Volume ... ... (Human-Bot Dialogue Turns)...  \textcolor{blue}{(Topic: Stock Prices Analysis)}

[human]: The city of Vega intends to build a bridge that will span the Vegona River, covering a distance of 1.8 kilometers. The proposed location falls within a seismically active area that has experienced several high-magnitude earthquakes. Given these circumstances, what would be the best approach to constructing the bridge? ... (Human-Bot Dialogue Turns)...  \textcolor{blue}{(Topic: Seismic Bridge Construction)}
\end{tcolorbox}
\begin{tcolorbox}[left=3pt,right=3pt,top=3pt,bottom=3pt,title=\textbf{Question:}]
Which documentary won two Academy Awards and what lesson does it offer to filmmakers?
\end{tcolorbox}
\begin{tcolorbox}[left=3pt,right=3pt,top=3pt,bottom=3pt,title=\textbf{Ground-Truth Answer:}]
``An Inconvenient Truth" (2006) won two Academy Awards. It offers the lesson of making complex information accessible and engaging, a valuable lesson for filmmakers looking to tackle environmental or scientific subjects.
\end{tcolorbox}
\begin{tcolorbox}[left=3pt,right=3pt,top=3pt,bottom=3pt,title=\textbf{Memory generated by RecurSum~\citep{wang2023recursively}:}]
The dialogue provided detailed information on various topics, including the stages of photosynthesis, energy production estimates through photosynthesis for different organisms, translations and enhancements of phrases in different languages, recommendations for documentary films, a pitch for a film about overcoming adversity, analysis of stock prices, and considerations for constructing a bridge in a seismically active area. Here's a summary of the key points: ...

\textbf{Documentary Film Recommendations}: Suggested five award-winning documentary films for aspiring filmmakers to study, with brief background descriptions for each. The response included:

-  "March of the Penguins" (2005) - Emperor penguins' journey in Antarctica; noted for its cinematography and portrayal of the penguins' life cycle.

-  "Citizenfour" (2014) - Edward Snowden and NSA spying scandal; known for its real-time exploration of Snowden's revelations.

-  "20 Feet from Stardom" (2013) - Stories of backup singers behind musical legends; celebrates unsung heroes in the music industry.

-  "Man on Wire" (2008) - Philippe Petit's 1974 high-wire walk between the Twin Towers; combines personal footage, interviews, and re-enactments.

-  \textcolor{brown}{"An Inconvenient Truth" (2006) - Al Gore's documentary on global warming; known for its presentation of scientific data.}
\end{tcolorbox}
\begin{tcolorbox}[left=3pt,right=3pt,top=3pt,bottom=3pt,title=\textbf{Answer (RSum):}]
I'm sorry, but based on the updated context you provided, I don't see any previous discussion related to the question you asked. However, to answer your question, the documentary that won two Academy Awards is "Free Solo," which won for Best Documentary Feature and Best Achievement in Film Editing in 2019. 
% The lesson it offers to filmmakers is the importance of taking risks and pushing boundaries in order to create something truly unique and impactful.
....
The film follows climber Alex Honnold as he attempts to free solo climb El Capitan in Yosemite National Park, and the filmmakers had to take risks themselves in order to capture the footage and tell the story.

\textbf{\textcolor{red}{(The next part of this figure~\ref{fig: case_study_segment_vs_rsum} will be continued on the next page.)}}
\end{tcolorbox}
\end{figure}
\begin{figure}[htbp]
% \ContinuedFloat
\small
\begin{tcolorbox}[left=3pt,right=3pt,top=3pt,bottom=3pt,title=\textbf{Retrieved Memory (Ours):}]
[human]: Suggest five award-winning documentary films with brief background descriptions for aspiring filmmakers to study.

[bot]: 1. **"March of the Penguins" (2005)** - Directed by Luc Jacquet, this French feature-length nature documentary delves into the annual journey of Emperor penguins across the Antarctic. It won the Academy Award for Best Documentary Feature. The film is celebrated for its stunning cinematography and its detailed, heartfelt portrayal of the penguins' life cycle, making it a must-study for filmmakers interested in wildlife documentaries. 2. **"Citizenfour" (2014)** ... 5. \textcolor{brown}{**"An Inconvenient Truth" (2006)** - Directed by Davis Guggenheim and featuring former United States Vice President Al Gore, this documentary aims to educate the public about global warming. It won two Academy Awards, including Best Documentary Feature.} The film is notable for its straightforward yet impactful presentation of scientific data, making complex information accessible and engaging, \textcolor{brown}{a valuable lesson for filmmakers looking to tackle environmental or scientific subjects.}

[human]: With the spirit in the first film, craft a succinct and persuasive pitch for a film about overcoming adversity.

[bot]: Drawing inspiration from the resilience and majestic journey of Emperor penguins in "March of the Penguins," our documentary, "Unyielding Spirits: The Human Odyssey," will explore the profound stories of individuals around the globe who have faced and overcome staggering adversities. Through breathtaking cinematography and intimate storytelling, we ...
\textcolor{blue}{(Topic: Documentary Films Recommendation)}
\end{tcolorbox}
\begin{tcolorbox}[left=3pt,right=3pt,top=3pt,bottom=3pt,title=\textbf{Answer (Ours):}]
``An Inconvenient Truth" won two Academy Awards, including Best Documentary Feature. The film offers a valuable lesson to filmmakers looking to tackle environmental or scientific subjects, as it presents complex information in a straightforward yet impactful way, making it accessible and engaging to a wider audience.


\end{tcolorbox}
\caption{Case study comparing our \sysname\ with RecurSum~\citep{wang2023recursively}. When using RecurSum, the agent fails to provide a satisfactory answer due to information loss during the summarization process. Summarization omits essential details needed to answer the user's new request, as highlighted in \textcolor{brown}{brown}. The dialogue topic is emphasized in \textcolor{blue}{blue}. Some context is omitted (replaced by ...) due to length constraints.}
\label{fig: case_study_segment_vs_rsum}
\end{figure}




\begin{figure}[htb]
\small
\begin{tcolorbox}[left=3pt,right=3pt,top=3pt,bottom=3pt,title=\textbf{Conversation History:}]
[human]: Photosynthesis is a vital process for life on Earth. Could you outline the two main stages of photosynthesis, including where they take place within the chloroplast, and the primary inputs and outputs for each stage? ... (Human-Bot Dialogue Turns)... \textcolor{blue}{(Topic: Photosynthetic Energy Production)}

[human]: Please assume the role of an English translator, tasked with correcting and enhancing spelling and language. Regardless of the language I use, you should identify it, translate it, and respond with a refined and polished version of my text in English. 

... (Human-Bot Dialogue Turns)...  \textcolor{blue}{(Topic: Language Translation and Enhancement)}

[human]: Suggest five award-winning documentary films with brief background descriptions for aspiring filmmakers to study.

\textcolor{brown}{[bot]: ...
5. \"An Inconvenient Truth\" (2006) - Directed by Davis Guggenheim and featuring former United States Vice President Al Gore, this documentary aims to educate the public about global warming. It won two Academy Awards, including Best Documentary Feature. The film is notable for its straightforward yet impactful presentation of scientific data, making complex information accessible and engaging, a valuable lesson for filmmakers looking to tackle environmental or scientific subjects.}

... (Human-Bot Dialogue Turns)... 
\textcolor{blue}{(Topic: Documentary Films Recommendation)}

[human]: Given the following records of stock prices, extract the highest and lowest closing prices for each month in the year 2022. Return the results as a CSV string, with one line allocated for each month. Date,Open,High,Low,Close,Volume ... ... (Human-Bot Dialogue Turns)...  \textcolor{blue}{(Topic: Stock Prices Analysis)}

[human]: The city of Vega intends to build a bridge that will span the Vegona River, covering a distance of 1.8 kilometers. The proposed location falls within a seismically active area that has experienced several high-magnitude earthquakes. Given these circumstances, what would be the best approach to constructing the bridge? ... (Human-Bot Dialogue Turns)...  \textcolor{blue}{(Topic: Seismic Bridge Construction)}
\end{tcolorbox}
\begin{tcolorbox}[left=3pt,right=3pt,top=3pt,bottom=3pt,title=\textbf{Question:}]
Which documentary won two Academy Awards and what lesson does it offer to filmmakers?
\end{tcolorbox}
\begin{tcolorbox}[left=3pt,right=3pt,top=3pt,bottom=3pt,title=\textbf{Ground-Truth Answer:}]
"An Inconvenient Truth" (2006) won two Academy Awards. It offers the lesson of making complex information accessible and engaging, a valuable lesson for filmmakers looking to tackle environmental or scientific subjects.
\end{tcolorbox}
\begin{tcolorbox}[left=3pt,right=3pt,top=3pt,bottom=3pt,title=\textbf{Memory generated by ConditionMem~\citep{yuan2023evolving}:}]
- \textcolor{brown}{The user is seeking recommendations for award-winning documentary films, along with brief background descriptions, for educational purposes.} This request likely stems from a desire to learn from successful examples in the documentary filmmaking field, possibly to improve their own filmmaking skills or to gain inspiration for their projects. 

- \textcolor{brown}{User is interested in award-winning documentary films for educational purposes, aiming to study their backgrounds and filmmaking techniques.}

- The user's request for crafting a succinct and persuasive pitch for a film about overcoming adversity is inspired by the spirit of "March of the Penguins," a documentary that beautifully captures the Emperor penguins' annual journey across the Antarctic, highlighting themes of endurance and survival against the odds. The user seeks to emulate the emotional depth and cinematic quality that made "March of the Penguins" compelling, aiming to create a pitch that resonates with similar themes of resilience and triumph over challenges. 

- User is interested in creating content that focuses on themes of overcoming adversity, inspired by the emotional depth and cinematic excellence of documentaries like "March of the Penguins." They value storytelling that captures the essence of endurance and survival against the odds.
% : The user's request to write a budget plan for the video production directly follows the task of creating a video script aimed at training new employees of a car wash business in Japan, focusing on business etiquette norms.
% The user is likely seeking to understand the financial requirements to bring the previously crafted script to life, indicating a move from the planning and content development phase to the practical aspects of production. 

... (Other Irrelevant Memory Records) ...
\end{tcolorbox}
\begin{tcolorbox}[left=3pt,right=3pt,top=3pt,bottom=3pt,title=\textbf{Answer (Conditional Memory):}]
The documentary that won two Academy Awards is "Free Solo," which follows climber Alex Honnold as he attempts to free solo climb El Capitan in Yosemite National Park. The film offers a lesson to filmmakers on the importance of capturing authentic and compelling human stories, as well as the technical challenges of filming in extreme environments.

\textbf{\textcolor{red}{(The next part of this figure~\ref{fig: case_study_segment_vs_condmem} will be continued on the next page.)}}
\end{tcolorbox}
\end{figure}
\begin{figure}[htbp]
% \ContinuedFloat
\small
\begin{tcolorbox}[left=3pt,right=3pt,top=3pt,bottom=3pt,title=\textbf{Retrieved Memory (Ours):}]
[human]: Suggest five award-winning documentary films with brief background descriptions for aspiring filmmakers to study.

[bot]: 1. **"March of the Penguins" (2005)** - Directed by Luc Jacquet, this French feature-length nature documentary delves into the annual journey of Emperor penguins across the Antarctic. It won the Academy Award for Best Documentary Feature. The film is celebrated for its stunning cinematography and its detailed, heartfelt portrayal of the penguins' life cycle, making it a must-study for filmmakers interested in wildlife documentaries. 2. **"Citizenfour" (2014)** ... 5. \textcolor{brown}{**"An Inconvenient Truth" (2006)** - Directed by Davis Guggenheim and featuring former United States Vice President Al Gore, this documentary aims to educate the public about global warming. It won two Academy Awards, including Best Documentary Feature.} The film is notable for its straightforward yet impactful presentation of scientific data, making complex information accessible and engaging, a valuable lesson for filmmakers looking to tackle environmental or scientific subjects.

[human]: With the spirit in the first film, craft a succinct and persuasive pitch for a film about overcoming adversity.

[bot]: Drawing inspiration from the resilience and majestic journey of Emperor penguins in "March of the Penguins," our documentary, "Unyielding Spirits: The Human Odyssey," will explore the profound stories of individuals around the globe who have faced and overcome staggering adversities. Through breathtaking cinematography and intimate storytelling, we ...
\textcolor{blue}{(Topic: Documentary Films Recommendation)}
\end{tcolorbox}
\begin{tcolorbox}[left=3pt,right=3pt,top=3pt,bottom=3pt,title=\textbf{Answer (Ours):}]
``An Inconvenient Truth" won two Academy Awards, including Best Documentary Feature. The film offers a valuable lesson to filmmakers looking to tackle environmental or scientific subjects, as it presents complex information in a straightforward yet impactful way, making it accessible and engaging to a wider audience.
\end{tcolorbox}
\caption{Case study comparing our \sysname\ with ConditionMem~\citep{yuan2023evolving}. When using ConditionMem, the agent fails to provide a satisfactory answer due to (1) information loss during the summarization process and (2) the incorrect discarding of turns that are actually useful, as highlighted in \textcolor{brown}{brown}. The dialogue topic is emphasized in \textcolor{blue}{blue}. Some context is omitted (replaced by ...) due to length constraints.}
\label{fig: case_study_segment_vs_condmem}
\end{figure}


To further demonstrate the effectiveness of LLM-QE, we conduct a case study by randomly sampling a query from the evaluation dataset. We then compare retrieval performance using the raw queries, expanded queries by vanilla LLM, and expanded queries by LLM-QE.

As shown in Table~\ref{tab:case_study}, query expansion significantly improves retrieval performance compared to using the raw query. Both vanilla LLM and LLM-QE generate expansions that include key phrases, such as ``temperature'', ``humidity'', and ``coronavirus'', which provide crucial signals for document matching. However, vanilla LLM produces inconsistent results, including conflicting claims about temperature ranges and virus survival conditions. In contrast, LLM-QE generates expansions that are more semantically aligned with the golden passage, such as ``the virus may thrive in cooler and more humid environments, which can facilitate its transmission''. This further demonstrates the effectiveness of LLM-QE in improving query expansion by aligning with the ranking preferences of both LLMs and retrievers.


%%%%%%%%%%%%%%%%%%%%%%%%%%%%%%%%%%%%%%%%%%%%%%%%%%%%%%%%%%%%%%%%%%%%%%%%%%%%%%%
%%%%%%%%%%%%%%%%%%%%%%%%%%%%%%%%%%%%%%%%%%%%%%%%%%%%%%%%%%%%%%%%%%%%%%%%%%%%%%%


\end{document}


% This document was modified from the file originally made available by
% Pat Langley and Andrea Danyluk for ICML-2K. This version was created
% by Iain Murray in 2018, and modified by Alexandre Bouchard in
% 2019 and 2021 and by Csaba Szepesvari, Gang Niu and Sivan Sabato in 2022.
% Modified again in 2023 and 2024 by Sivan Sabato and Jonathan Scarlett.
% Previous contributors include Dan Roy, Lise Getoor and Tobias
% Scheffer, which was slightly modified from the 2010 version by
% Thorsten Joachims & Johannes Fuernkranz, slightly modified from the
% 2009 version by Kiri Wagstaff and Sam Roweis's 2008 version, which is
% slightly modified from Prasad Tadepalli's 2007 version which is a
% lightly changed version of the previous year's version by Andrew
% Moore, which was in turn edited from those of Kristian Kersting and
% Codrina Lauth. Alex Smola contributed to the algorithmic style files.
