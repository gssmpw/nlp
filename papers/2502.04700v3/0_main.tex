\documentclass{article}

% Recommended, but optional, packages for figures and better typesetting:
\usepackage{microtype}
\usepackage{graphicx}
\usepackage{subfigure}
\usepackage{booktabs} % for professional tables

% hyperref makes hyperlinks in the resulting PDF.
% If your build breaks (sometimes temporarily if a hyperlink spans a page)
% please comment out the following usepackage line and replace
% \usepackage{icml2025} with \usepackage[nohyperref]{icml2025} above.
\usepackage{hyperref}


% Attempt to make hyperref and algorithmic work together better:
\newcommand{\theHalgorithm}{\arabic{algorithm}}

% Use the following line for the initial blind version submitted for review:
% \usepackage{icml2025}

% If accepted, instead use the following line for the camera-ready submission:
\usepackage[accepted]{icml2025}

% For theorems and such
\usepackage{amsmath}
\usepackage{amssymb}
\usepackage{mathtools}
\usepackage{amsthm}
\usepackage{dashbox}
\newcommand\dboxed[1]{\dbox{\ensuremath{#1}}}
%%%%%%%%%%%%
% \usepackage{hyperref}
\usepackage{url}
\usepackage[english]{babel}
\usepackage{ulem}
\usepackage{wrapfig}
\usepackage{booktabs}       % professional-quality tables
% \usepackage{amsfonts}       % blackboard math symbols
\usepackage{nicefrac}       % compact symbols for 1/2, etc.
\usepackage{microtype}      % microtypography
% \usepackage{xcolor}         % colors
% \definecolor{lightblue}{rgb}{0.68, 0.85, 0.9}
% \usepackage{amsmath}
% \usepackage[pdftex]{graphicx}
\usepackage{soul}
\usepackage{multirow}
\usepackage{colortbl}
\sethlcolor{LightBlue2}
\usepackage{mathrsfs}
\usepackage{amsmath,amsfonts,amssymb,amsthm}
\usepackage{resizegather}
\newcommand{\new}{\marginpar{NEW}}
\usepackage{placeins}
\usepackage{subcaption}
\usepackage{adjustbox}


% if you use cleveref..
\usepackage[capitalize,noabbrev]{cleveref}
% \pdfobjcompresslevel=2

\newcommand{\del}[1]{{\color{red}[#1]}}
%%%%%%%%%%%%%%%%%%%%%%%%%%%%%%%%
% THEOREMS
%%%%%%%%%%%%%%%%%%%%%%%%%%%%%%%%
\theoremstyle{plain}
\newtheorem{theorem}{Theorem}[section]
\newtheorem{proposition}[theorem]{Proposition}
\newtheorem{lemma}[theorem]{Lemma}
\newtheorem{corollary}[theorem]{Corollary}
\theoremstyle{definition}
\newtheorem{definition}[theorem]{Definition}
\newtheorem{assumption}[theorem]{Assumption}
\theoremstyle{remark}
\newtheorem{remark}[theorem]{Remark}
\newcommand{\numMale}{823}
\newcommand{\numParticipants}{1573}
\newcommand{\numTotalParticipants}{2,673}
\newcommand{\numFemale}{715}
\newcommand{\numOther}{35}
\newcommand{\numControl}{248}
\newcommand{\numBaseline}{240}
\newcommand{\numReorder}{269}
\newcommand{\numTreatment}{816}
\newcommand{\numExcluded}{6}

% Todonotes is useful during development; simply uncomment the next line
%    and comment out the line below the next line to turn off comments
\usepackage[disable,textsize=tiny]{todonotes}
% \usepackage[textsize=tiny]{todonotes}


% The \icmltitle you define below is probably too long as a header.
% Therefore, a short form for the running title is supplied here:
\icmltitlerunning{EigenLoRAx}

\begin{document}

\twocolumn[
% \icmltitle{EigenAug: Enhancing Resource-Efficient Adaptation with Augmented Principal Subspaces}
\icmltitle{EigenLoRAx: Recycling Adapters to Find Principal Subspaces for Resource-Efficient Adaptation and Inference}

% It is OKAY to include author information, even for blind
% submissions: the style file will automatically remove it for you
% unless you've provided the [accepted] option to the icml2025
% package.

% List of affiliations: The first argument should be a (short)
% identifier you will use later to specify author affiliations
% Academic affiliations should list Department, University, City, Region, Country
% Industry affiliations should list Company, City, Region, Country

% You can specify symbols, otherwise they are numbered in order.
% Ideally, you should not use this facility. Affiliations will be numbered
% in order of appearance and this is the preferred way.
\icmlsetsymbol{equal}{*}

\begin{icmlauthorlist}
\icmlauthor{Prakhar Kaushik*}{yyy}
\icmlauthor{Ankit Vaidya*}{yyy}
\icmlauthor{Shravan Chaudhari}{yyy}
\icmlauthor{Alan Yuille}{yyy}
% \icmlauthor{Firstname2 Lastname2}{equal,yyy,comp}
% \icmlauthor{Firstname3 Lastname3}{comp}
% \icmlauthor{Firstname4 Lastname4}{sch}
% \icmlauthor{Firstname5 Lastname5}{yyy}
% \icmlauthor{Firstname6 Lastname6}{sch,yyy,comp}
% \icmlauthor{Firstname7 Lastname7}{comp}
% %\icmlauthor{}{sch}
% \icmlauthor{Firstname8 Lastname8}{sch}
% \icmlauthor{Firstname8 Lastname8}{yyy,comp}
%\icmlauthor{}{sch}
%\icmlauthor{}{sch}
\end{icmlauthorlist}

\icmlaffiliation{yyy}{Department of Computer Science, Johns Hopkins University, Baltimore, USA}
% \icmlaffiliation{comp}{Company Name, Location, Country}
% \icmlaffiliation{sch}{School of ZZZ, Institute of WWW, Location, Country}

\icmlcorrespondingauthor{Prakhar Kaushik}{pkaushi1@jh.edu}
% \icmlcorrespondingauthor{Firstname2 Lastname2}{first2.last2@www.uk}

% You may provide any keywords that you
% find helpful for describing your paper; these are used to populate
% the "keywords" metadata in the PDF but will not be shown in the document
\icmlkeywords{Machine Learning, ICML}

\vskip 0.3in
]

% this must go after the closing bracket ] following \twocolumn[ ...

% This command actually creates the footnote in the first column
% listing the affiliations and the copyright notice.
% The command takes one argument, which is text to display at the start of the footnote.
% The \icmlEqualContribution command is standard text for equal contribution.
% Remove it (just {}) if you do not need this facility.

% \printAffiliationsAndNotice{}  % leave blank if no need to mention equal contribution
\printAffiliationsAndNotice{\icmlEqualContribution} % otherwise use the standard text.

\begin{abstract}
\begin{abstract}

% Recent works to jointly reconstruct 3D human and object from a single RGB image, are mostly model-based, that fail to capture the fine details of the clothed human body and object surface. In this paper, we introduce ReCHOR, a novel, model-free, first-method to produce realistic clothed human-object reconstructions from a monocular view. This is extremely challenging due to human-object occlusions, diverse interactions and depth ambiguity, as it needs to infer both 3D spatial awareness and high resolution details. Our core idea is based on estimating neural implicit representations for human and object respectively by an attention-based neural implicit model that attends to pixel-aligned features from both the global human-object image for spatial awareness and  the local separate view of human and object images for high quality details. Additionally, the network is conditioned on semantic features from an initial estimated human-object pose prior and a generative diffusion model that inpaints occluded regions, thus enabling the retrieval of details from them.
% We also propose a synthetic dataset with rendered scenes of diverse, inter-occluded 3D human and object scans, to train our network. We evaluate our method on the synthetic and real world BEHAVE dataset. Our experiments show that our method outperforms the SOTA in achieving realistic clothed human-object reconstructions.
Recent approaches to jointly reconstruct 3D humans and objects from a single RGB image represent 3D shapes with template-based or coarse models, which fail to capture details of loose clothing on human bodies. In this paper, we introduce a novel implicit approach for jointly reconstructing realistic 3D clothed humans and objects from a monocular view. For the first time, we model both the human and the object with an implicit representation, allowing to capture more realistic details such as clothing. This task is extremely challenging due to human-object occlusions and the lack of 3D information in 2D images, often leading to poor detail reconstruction and depth ambiguity. To address these problems, we propose a novel attention-based neural implicit model that leverages image pixel alignment from both the input human-object image for a global understanding of the human-object scene and from local separate views of the human and object images to improve realism with, for example, clothing details. Additionally, the network is conditioned on semantic features derived from an estimated human-object pose prior, which provides 3D spatial information about the shared space of humans and objects. To handle human occlusion caused by objects, we use a generative diffusion model that inpaints the occluded regions, recovering otherwise lost details. For training and evaluation, we introduce a synthetic dataset featuring rendered scenes of inter-occluded 3D human scans and diverse objects. Extensive evaluation on both synthetic and real-world datasets demonstrates the superior quality of the proposed human-object reconstructions over competitive methods.
\end{abstract}
\end{abstract}

\section{Introduction}
Recent advancements in machine learning have driven the rise of large-scale models with billions of parameters. However, the size and complexity of these models not only make it impractical for most researchers to train or fine-tune them on downstream tasks but also contribute significantly to their carbon footprint, raising concerns about environmental sustainability.
To address these challenges, there has been growing interest in parameter-efficient finetuning (PEFT) methods, such as adapters~\citep{pmlr-v97-houlsby19a, Chen2022AdaptFormerAV, luo2023towards}, low rank adaptation (LoRA) methods~\citep{hu2021lora,kopiczko_vera_2023,dora}, prompt-based methods~\citep{tune1, tune2, tune3}.
LoRA and its follow-up works~\citep{meng_pissa_2024,dora} have gained significant attention for their simplicity. This has fueled the proliferation of thousands of low-rank adapters within the growing open-source community.
Given that these adapters are underutilized, an important question arises: Can we recycle the information contained in them to improve the efficiency of subsequent tasks?
Recent work has shown that weight updates in deep neural networks occur within low-dimensional invariant subspaces~\citep{kwon2024efficientcompressionoverparameterizeddeep}, aligning with the universality hypothesis that neural network behavior and learned representations often reside in shared, structured subspaces~\cite{chughtai2023toymodeluniversalityreverse, guth2024on}. This suggests that LoRA adapters may similarly share a \textit{principal subspace} that can be reused, eliminating the need to rediscover it during the training of new adapters.

We introduce \textbf{EigenLoRAx}, a parameter-efficient fine-tuning (PEFT) method that leverages this insight by decomposing the weights of a set of trained adapters into principal components, identifying a compact, information-dense subspace. EigenLoRAx reduces the number of learnable parameters by up to $\mathbf{100\times}$ compared to LoRA, accelerates optimization by up to $\mathbf{2\times}$ for new adapters, and enables more memory-efficient inference with multiple task adapters, particularly benefiting edge devices~\citep{edge}. Additionally, in low-resource domains, we demonstrate that EigenLoRAx can be further enhanced by augmenting the principal subspace with random components, orthogonalized with respect to the existing subspace, preserving its efficiency while retaining performance.

Furthermore, we provide an initial theoretical analysis of EigenLoRAx.
Our experiments across a wide range of vision and language tasks demonstrate its versatility and effectiveness, reinforcing the potential of shared subspaces in neural network adaptation.

\autoref{fig:fig1} provides an overview of our method. We introduce \textbf{EigenLoRAx} (ELoRAx), which recycles pretrained adapters by identifying a shared \textit{task-invariant} weight subspace. We hypothesize (and validate experimentally) that task-specific weights lie within this subspace, allowing for more efficient training with fewer parameters. This reduces memory footprint and enhances inference efficiency by enabling simultaneous serving of multiple adapters. EigenLoRAx is among the first to recycle pretrained adapters, replacing many while improving further training efficiency. Our key contributions are as follows:

\begin{itemize}
    \item \textbf{(Training)}: EigenLoRAx uses up to $\mathbf{100\times}$ \textbf{fewer parameters than LoRA} and converges up to $\mathbf{2\times}$ \textbf{faster} than comparable methods, achieving similar or better performance.
    \item \textbf{(Inference)}: EigenLoRAx enhances \textbf{memory efficiency during inference} by approximately $\mathbf{18\times}$ on multiple tasks, reducing the number of switchable parameters between tasks.
    \item \textbf{(Applicability)}: We empirically demonstrate the effectiveness of EigenLoRAx across a wide range 
    , including text and image data, validating the existence of shared principal subspaces across modalities. It also retains performance in \textbf{zero-shot} and \textbf{low resource} scenarios.
    \item \textbf{(Scaling)}: EigenLoRAx can be scaled up to recycle hundreds of underutilized pretrained adapters.
\end{itemize}
\begin{figure*}[!hbt]
\begin{center}
\includegraphics[width=\textwidth]{images/elorax.pdf}
\caption{\small{LoRA uses low-rank matrices for task-specific finetuning. We observe that LoRA adapters share a principal subspace across task domains. By recycling pretrained adapters, we extract \textit{task-invariant} principal components, enabling efficient representation of both existing and future LoRAs using compact \textit{task-specific} coefficients. This improves training speed, parameter efficiency, and memory usage. In low-resource settings, where pretrained adapters are scarce, we augment the subspace with randomly initialized components, ensuring orthogonality via the Gram-Schmidt process, ensuring they complement the extracted subspace without redundancy.}}
\label{fig:fig1}
\end{center}
\end{figure*}
\section{Related Works}
\section{Related Work}
\label{sec:related work}
% In this section, we review the existing literature on point cloud denoising and unsupervised image denoising.
%-------------------------------------------------------------------------
\subsection{Point cloud denoising}

    \subsubsection{Traditional methods}
Traditional approaches to point cloud denoising include statistical methods \cite{computingpointset2003,definingpointset2004,wlop2009HH}, filtering techniques\cite{pointsetsurfaces2001,Robustmoving2005, zaman2017density}, and optimization-based methods \cite{l1sparse2010,clop2014PR,digne2017bilateral,multi-projection2018duan,hu2020featuregraph} . These techniques often rely on handcrafted features and heuristics to distinguish signal from noise. For example, statistical methods may use distribution models to identify and remove outliers. Filtering methods, such as mean or median filters, operate under the assumption that noise is statistically different from the signal. Optimization-based methods formulate denoising as an energy minimization problem, where regularization terms constrain the solution to ensure certain smoothness cirterion or adherence to prior knowledge.

%-------------------------------------------------------------------------
    \subsubsection{Supervised learning based methods}
In recent years, several deep learning-based methods \cite{rakotosaona2020PCN,hermosilla2019TotalDenoising,luo2020DMR,luo_score-based_2021} have been proposed for point cloud denoising. NPD \cite{NPD2019} is the first learning-based point cloud denoising method that directly processes noisy data without requiring noise characteristics or neighboring point definitions. This approach combines local and global information by projecting noisy points onto estimated reference planes, effectively removing noise and enhancing robustness against variations in noise intensity and curvature. PointCleanNet\cite{rakotosaona2020PCN} first removes outlier points and then combines them with residual connectivity to predict the inverse displacement \cite{Guerrero2017PCPNetLL}, and iteratively shifts noisy points to remove noise. Pistilli \etal proposed GPDNet \cite{gpdnet2020}, which is a graph convolutional network to improve denoising robustness at high noise levels. Luo \etal also proposed  DMRDenoise \cite{luo2020DMR}, which filter
points by first downsampling the noisy inputs and reconstructing the local subsurface to perform point upsampling. However, this resampling method is difficult to maintain a good local shape. ScoreDenoise \cite{luo_score-based_2021} is proposed to tackle the aforementioned issues by iteratively updating the point position in implicit gradient fields learned by neural networks. For inference, they follows an iterative procedure with a decaying step size, which stabilizes point movement and prevents over-correction, allowing points to converge gradually toward the underlying geometry. The authors of \cite{de_Silva_Edirimuni_2023_CVPR} proposed IterativePFN - an iterative method that use a novel loss function that utilizes an adaptive ground truth target at each iteration to capture the relationship between intermediate filtering results during training. Zheng \etal proposed a end-to-end network for joint normal filtering-based point cloud denoising \cite{10173632}. They introduce an auxiliary normal filtering task to enhance the network's ability to remove noise while preserving geometric features more accurately.

Supervised methods can achieve impressive results, but are limited by the availability and quality of the training data, as they typically require paired noisy and clean point clouds to train the neural network. The absence of clean data in real-world scenario pose a significant challenge on applicability of these algorithms.

%-------------------------------------------------------------------------
    \subsubsection{Unsupervised learning methods}
Unsupervised learning-based methods for point cloud denoising do not require ground-truth clean data. Instead, these methods leverage the inherent structure or distribution of the point cloud to guide the denoising process. Unsupervised methods show promise in scenarios where clean data is absent or hard to obtain.

Hermosilla \etal first introduced Total Denoising (TotalDn) \cite{hermosilla2019TotalDenoising} as an unsupervised learning approach for point cloud denoising, relying solely on noisy data without requiring clean ground truth. TotalDn approximates the underlying surfaces by regressing points from the distribution of unstructured total noise, utilizing a spatial prior term to refine the estimation of geometry. 

In DMRDenoise \cite{luo2020DMR}, an unsupervised version is proposed which utilizes a loss function that identify local neighborhoods using a probabilistic Gaussian mask on the k-nearest neighbors, which selectively retains points likely to represent the underlying surface. By leveraging an Earth Mover's Distance (EMD) assignment, it achieves a one-to-one correspondence between input and predicted points, aligning them to reduce noise within local neighborhoods.
This approach enhances robustness to noise and adapts well to varied surface geometries. However, the probabilistic masking and EMD calculation add computational complexity, which can slow down inference in dense or noisy point clouds. 

ScoreDenoise \cite{luo_score-based_2021} also introduced an unsupervised version that employs ensemble score function and an adaptive neighborhood-covering loss for model training.  
Score-u is probably the most relevant work to our method. However, the defined score in \cite{luo_score-based_2021} is only an displacement-alike version of the score function and there is no explicit formula relating the estimated score to the final denoising result. Also, the iterative process is computationally expensive, and can suffer from fluctuation, leading to perturbed and unstable solution.

Most recently, Noise4Denoise \cite{noise4Wang2024} method is proposed that use an additional doubly-noisy point cloud to learn noise displacement by comparing the two noise levels. This approach effectively leverages synthetic noise for training, allowing the model to estimate residuals without relying on clean data.
However, in practical applications, noise parameters are often unknown and variable, making it challenging to replicate the exact conditions assumed during training. This reliance on predefined noise characteristics can limit the model's applicability to real-world scenarios where noise distributions may differ significantly from synthetic settings. 
%-------------------------------------------------------------------------
\subsection{Unsupervised image denoising}
Recently unsupervised image denoising has made significant progress. Non-Bayesian methods include PURE \cite{luisier2010image}, SURE \cite{SURE2018} \textit{etc.}, which are based on various unbiased risk estimator under certain noise distribution. Other methods explore self-similarity in natural images \cite{xu2015patch, doi:10.1137/23M1614456} or exploits the statistical properties of noise to achieve denoising effect \cite{gravel2004method}.  

Noise2Noise \cite{2018Noise2NoiseLI} is a pioneering method that does not require clean data, but it requires multiple noisy versions of the same image for training. To address this limitation, methods such as Noise2Void \cite{2018Noise2VoidL}, Noise2Self \cite{2019Noise2SelfBD}, \textit{etc.}, have been developed, which use only a single noisy image. These methods are particularly important for practical applications where obtaining clean images or multiple noisy realizations of the same image is difficult or impossible. Neighbor2Neighbor \cite{huang2021neighbor2neighbor} proposed a two-step method with a a random neighbor sub-sampler that generates training  pairs and a denosing network. Kim \etal proposed Noise2Score\cite{kim_noise2score_2021}, a novel Bayesian framework for self-supervised image denoising without clean data. The core of Noise2Score is the usage of Tweedie's formula, which provides an explicit representation of the denoised image through a score function. Combined with score function estimation, Noise2Score can be applied to image denoising with any exponential family noise. Kim \etal also proposed the Noise Distribution Adaptive Self-Supervised Image Denoising method \cite{kim_noise_2022}, which further extends the application of Noise2Score by combining the Tweedie distribution with score matching. This enables adaptive handling of various noise distributions and dynamically adjusts the denoising process by estimating noise parameters. On the other hand, Xie \etal \cite{scoreXie2024} broadened the denoising scope of Noise2Score by allowing it to handle complex noise models, such as multiplicative and structurally correlated noise, demonstrating broad applicability to diverse noise models.

These development of unsupervised image denoising method motivate us to explore similar ideas in 3D point cloud denoising.





\section{Method}
In this section, we present the theoretical foundation~\cref{sec:theory} and algorithmic details~\cref{sec:algo} of our method, followed by a discussion on hyperparameter selection and an assessment of its practical advantages. Note that we use the terms EigenLoRA, EigenLoRAx, ELoRA and ELoRAx interchangeably.

\subsection{Theoretical Motivation}
\label{sec:theory}

For a full rank weight matrix $W \in \mathbb{R}^{m \times n} $ that learns to map input space $X\in\mathbb{R}^{m}$ to output space $\mathbb{R}^{n}$, the rank is expressed as $\min(m,n)$. As the rank of $ W $ increases, modifying it to accommodate new tasks becomes computationally expensive and increasingly complex. This is a common challenge faced when finetuning pretrained large foundation models. LoRA is a parameter efficient finetuning approach used for large pretrained models with weights $W_0$ that mitigates this challenge by merely learning low-rank weight updates $W$ such that the risk between $Y$ and $W_0X + WX + b$ is minimized. Instead of directly learning $W$, LoRA proposes to learn a lower ranked decomposition of $W$ by learning two low-rank matrices, $ B \in \mathbb{R}^{m \times r} $ and $ A \in \mathbb{R}^{r \times n} $, both having ranks $r$.  This factorization ensures that the product $ BA $ retains the original dimensions of $ W_0 $ while having a significantly reduced rank.
As a result, although the transformation defined by $ BA $ maps from $ \mathbb{R}^m $ to $ \mathbb{R}^n $, it does not span the full space of such mappings due to its constrained rank. The low-rank weight matrices result in substantially smaller number of trainable parameters than the full rank parameter count of $m\cdot n$. Such parameter efficient finetuning makes LoRA a computationally viable alternative for fine-tuning large-scale models.    

Previous works such as \cite{meng_pissa_2024, dora} have proposed the existence of a common parameter subspace implying the idea of shared principal subspace. We highlight that LoRA adapters share such a lower dimensional shared principal subspace when finetuned for diverse tasks. Along with reduction in computational overhead, it reinforces the idea that task-relevant transformations reside within a compact, reusable subspace. To formalize this, we first define a space of tasks representable by linear transformation matrices, providing a foundation for analyzing the role of shared principal subspaces in model adaptation.

\begin{definition}[Task definition for LoRAs] 
\label{def:lineartasks}
We first define a LoRA task $t_i (X_i,Y_i): \mathbb{R}^m \rightarrow \mathbb{R}^n$ such that $Y_i = W^*_iX_i + b$ where $b$ is some constant. Then the LoRA task domain $\mathcal{T}_d$ is a set of $d$ such tasks, $\mathcal{T}_d = \{t_i\}^{d}_{i=1}$.
\end{definition}


For a given set of pretrained weights (such as those from a foundation model) $W_0\in\mathbb{R}^{m\times n}$, LoRA weights $BA$ at any layer modify the output as $W_0X + BAX + \epsilon_t$, allowing the model to adapt to the new task and converge toward the optimal solution $W^*_t$. The key here is that only $B$ and $A$ weights are updated during finetuning. 
Without loss of generality, 
% \prakhar{can we write the full form here}\aayush{its a common term in math}
assume $r\ll n$ and let the true transformation matrix $W^*_t \in \mathbb{R}^{r \times n}$ be interpreted as $r$ $n$-dimensional vectors: $\mathbf{w}^{*1}_t, ..., \mathbf{w}^{*r}_t \in \mathbb{R}^{n}$. Finding LoRA weights is equivalent to finding sets of these $r$ vectors in $\mathbb{R}^{n}$. 

\begin{definition}[Set of LoRA weights]
\label{def:loraset}
We define the weights of a LoRA adapted for task $t_i$ as $B_iA_i$. Both $B_i$ and $A_i$ will have their own individual subspaces. For the purpose of the analysis we will consider a generic task specific weight matrix $W_i \in \mathbb{R}^{m\times n}$ adapted to task $t_i$ such that $n<m$ and its rank $r<n$. The analysis, however, is valid for both $B_i$ and $A_i$. 
Now can define a set of LoRAs as stacked (along columns) weight matrices $\hat{W} = \{W_i \}^{d}_{i=1}$ where each $W_i$ is adapted for a task $t_i\in\mathcal{T}_d$ and a training set $\mathcal{S}_i=\{\{x,y\} \mid x\in X_t, y\in Y_t\}$ where the size of the training is $s_i = |\mathcal{S}_i|$. For theoretical analysis we assume that each training set $X_i \times Y_i$ is distributed according to some unknown Gaussian distribution with mean $\Bar{X_i}$ and $\lVert X_i\lVert_F\leq M$ for some constant $M>0$. Each weight matrix can have different ranks and the following method and analysis will still hold, however, for brevity we assume all weight matrices stacked in $\hat{W}$ to have the same rank $r$. 
% Next, we define a subspace in $\mathbb{R}^{n}$ spanned by $d$ LoRAs learnt for $d$ tasks from $\mathcal{T}_d$.
\end{definition}
\begin{definition}[Subspace spanned by LoRAs from a task domain $\mathcal{T}_{d}$]
\label{def:subspace}
We define the subspace of weights $\mathcal{Z}_{d} = \{C\hat{W} \mid C\in\mathbb{R}^{m\times m}\}$ spanned within $\mathbb{R}^{m\times n}$. 
\end{definition}
Using Singular Value Decomposition (SVD) or Principal Component Analysis (PCA for a zero-centered $\hat{W}$) , we can obtain $\hat{W}=\mathcal{U}\Sigma \mathcal{V}^T$. We then represent top $K$ right singular vectors of $\hat{W}$ (or top $K$ principal components if $\hat{W}$ is zero-centered) as $\Vk \in\mathbb{R}^{K\times n}=\{\Vk\in\mathbb{R}^{1\times n}\}^K_{k=1}$.
\begin{definition}[Shared principal subspace of LoRAs finetuned in domain $\mathcal{T}_d$]
\label{def:principalsubspace}
We define the shared principal subspace of weights for a task domain $\mathcal{T}_d$ as $\mathcal{Z}^{K}_d = \{\alpha \Vk \mid \alpha\in\mathbb{R}^{m\times K}\}$ spanned by top K principal components of the LoRAs within $\mathbb{R}^{m\times n}$. 
\end{definition}

Next, we introduce the idea of defining a new related task $\tnew$
\begin{definition}[New related task $\tnew$].
\label{def:newtask}
    A new linear task $\tnew$ with true solution $\Wtrue$ is said to be related if it is spanned by the basis of $\hat{W}$ i.e. $\Wtrue = C\hat{W}$ and it holds that $\lVert \Wtrue - \alphatrue\Vk\lVert^2_F \leq \Vert\Wtrue - \alphalearnt\Vk\lVert^2_F$ for all rank $K$ linear transformation matrices $\alphalearnt$ and $\lVert \Wtrue - \alphatrue\Vk\lVert^2_F \leq \singularsum$ where $\sigma_i$'s are singular values of $\hat{W}$. For such a task, we learn coefficients of $K$ principal components $\alphalearnt \in \mathbb{R}^{m\times K}$ resulting in EigenLoRAx weights $\Wegn = \alphalearnt\Vk$. \\
\end{definition}

Definition \ref{def:newtask} establishes a bound over the related-ness of a new task with those in the known task domain $\mathcal{T}_d$. If the true solution of the new task lies majorly in the principal subspace of $\mathcal{T}_d$ i.e. has major principal components (PCs) within the top $K$ principal components of $\hat{W}$ with some finite bound on the misalignment along the PCs orthogonal to the top $K$ PCs of $\hat{W}$, then we can ideally quantify the relation between a new task and a task domain. Any task that has its true solution within a subspace defined by the PCs orthogonal to the top $K$ PCs of $\hat{W}$ is not as closely related as a task with its solution completely or majorly within the principal subspace. A task that has its solution completely orthogonal to all the PCs of $\hat{W}$ is completely unrelated and is not the main focus of this study. 

Next, we present an algorithm to find the principal subspace of the trained adapters and our experiments in Section~\ref{sec:experiments}.


\begin{algorithm}[!htb]
\caption{\textbf{EigenLoRAx PCs} Calculation}
\label{algo:eigenlora}
\begin{algorithmic}
    \STATE {\bfseries Input:} 
    LoRA matrices 
    $\{W_t \in \mathbb{R}^{m \times n}\}_{t=1}^d$ 
    
     , number of PC ($K$), number of pseudo-PC ($P$)\\
    
    \STATE {\bfseries Output:} EigenLoRAx PCs $\Vk$

    \STATE $\hat{W} = \begin{bmatrix*} W_{1}\in \mathbb{R}^{m\times n} & \text{...} & W_{d}\in \mathbb{R}^{m\times n}\end{bmatrix*}$, \COMMENT{Stack LoRA matrices}\\
    
    \STATE Compute the mean of each feature: $ \Bar{W} = \frac{1}{n} \sum_{i=1}^{n} W_i $
    \STATE Subtract the mean: $ \hat{W}_c = \hat{W} - \Bar{W} $
    
   
    \STATE Perform SVD: $ \hat{W}_c = U \Sigma V^T $
    
    \STATE Extract the top $ K $ principal components
    \STATE Select the first $ K $ columns of $ \mathcal{V} $: $ \V = V[:, 1:K] $
    
    \STATE Optionally, augment the subspace with $P$ pseudo-PCs
    \FOR{$ p = 1 $ to $ P $}
        \STATE Sample a random vector $ v_p \sim \mathcal{N}(0, I_n) $  \COMMENT{Sample from a normal distribution}
        \STATE Orthogonalize $ v_p $ against all PCs in $ \V $ using Gram-Schmidt:
        \FOR{$ i = 1 $ to $ K+p-1 $}
            \STATE $ v_p = v_p - \frac{v_p^T V_K[:, i]}{\|\V[:, i]\|^2} \V[:, i] $
        \ENDFOR
        \STATE Normalize $ v_p $: $ v_p = \frac{v_p}{\|v_p\|} $
        \STATE Append $ v_p $ to $ \V $ if $ v_p $ is not a null vector
        \STATE $K = K + 1$
    \ENDFOR

    \STATE return $\V, \mu$
    
\end{algorithmic}
\end{algorithm}

\subsection{Algorithm}
\label{sec:algo}

Assume we have $N$ LoRA adapters, each consisting of a set of $A, B$ matrix pairs for every layer, trained on various tasks within a domain $\mathcal{T}_d$ for a given base pretrained model. Algorithm~\ref{algo:eigenlora} computes a list of top $K$ principal components—referred to as EigenLoRAx PCs—that define an initial principal subspace for this domain. 

To construct this subspace, the algorithm aggregates LoRA matrices across tasks for each layer, separately for $A$ and $B$ matrices (though it can also be applied to the product $BA$). Each LoRA matrix, having rank r, is treated as a list of vectors, and a decomposition is performed on this stacked set of vectors. The most significant components extracted from this process serve as a basis for the principal subspace, providing an efficient representation that can be linearly combined to approximate the original LoRA weight matrices. We showcase our algorithm using representative weight matrices $W_t$, where each $W_t$ represents a single $A$ or $B$ matrix from a single LoRA layer of the neural network. In practice, this procedure is applied to all relevant layers.

Since real-world scenarios often involve low-resource domains with limited availability of LoRA adapters, we extend our subspace by introducing additional pseudo-PCs. Specifically, we sample random vectors of the same dimension as each PC and orthogonalize them with respect to all existing PCs. This process can be iterated to generate more pseudo-PCs, thereby augmenting the principal subspace. As empirically shown in Table~\ref{tab:glue_low}, this augmentation strategy significantly outperforms naive random selection of PCs for subspace expansion.

\paragraph{Learning new tasks}\label{method:learningnewelora} Having extracted a set of PCs (including pseudo-PCs, if needed), $\mathcal{V}_K \in\mathbb{R}^{K\times n}=\{\V\in\mathbb{R}^{1\times n}\}^K_{k=1}$, we can approximate a given (LoRA) weight matrix by minimizing $\lVert W - \alpha\Vk\lVert_F$ where $\alpha$ are linear coefficients~\cref{sec:theory}. In fact, we can analytically compute of the given LoRA matrices by calculating the linear coefficients which minimizes the above objective. For new tasks however, for which we do not have a LoRA matrix, we freeze the EigenLoRAx PCs and randomly initialize the $\alpha$s. The forward pass in layer is calculated as 
\begin{align}
    h = W_0x + \dboxed{\alpha_B^T \mathcal{V}_B \alpha_A^T \mathcal{V}_A(x)}.
\end{align}
Here, $W_0$ are the pretrained weights of the base model and $\mathcal{V}_B, \mathcal{V}_A$ are EigenLoRAx components (which represent the shared subspace) that are frozen during training. The corresponding lightweight coefficients $\alpha_B$ and $\alpha_A$ are learned. This reduces the number of learnable parameters from $O(2rn)$ to $O(2K)$, by a factor of $\frac{rn}{K}$ (assuming $\alpha$ to be scalar). 

\phantomsection


\section{Causal IL as CMRs}\label{sec:method}

In this section, we demonstrate that performing causal IL in our framework is possible using trajectory histories as instruments. In the next step, we show that the problem can be described as CMRs and propose an effective algorithm to solve it.

The typical target for IL would be the expert policy $\pi_E$ itself. However, since the expert has access to information, namely $u^o_t$, which the imitator does not, the best thing an imitator can do is to learn a history-dependent policy $\pi_h$ that is the closest to the expert. A natural choice is the conditional expectation of $\pi_E(s_t,u^o_t)$ on the history $h_t$:
\begin{align}
\pi_h(h_t)\coloneqq \expectE_{\probP(u^o_t\mid h_t)}[\pi_E(s_t,u^o_t)]=\expectE[\pi_E(s_t,u^o_t)\mid h_t],\nonumber
\end{align}
% where $p(u^o_t\mid h_t)$ is a distribution over expert-observable confounders and captures the information about $u^o_t$ can be inferred from the trajectory history. 
because the conditional expectation minimizes the least squares criterion~\citep{hastie01statisticallearning} and $\pi_h$ is the best predictor of $\pi_E$ given $h_t$. In $\pi_h$, the distribution $\probP(u^o_t\mid h_t)$ captures the information about $u^o_t$ that can be inferred from trajectory histories.
\begin{remark}
\emph{Learning $\pi_h$ is not trivial. Policies learnt naively using behaviour cloning (i.e., $\expectE[a_t\mid h_t]$) fail to match $\pi_E$. In view of~\cref{eq:action}, we have that
\begin{align} 
\expectE[a_t\mid h_t]&=\expectE[\pi_E(s_t,u^o_t) \mid h_{t}]+\expectE[u^\epsilon_t\mid h_{t}]\nonumber\\
&=\pi_h(h_t)+\expectE[u^\epsilon_t\mid h_{t}],\label{eq:history_policy}
\end{align}
where $\expectE[u^\epsilon_t\mid h_{t}]\neq 0$ due to the spurious correlation between $u^\epsilon_t$ and the trajectory history $h_t$. As a result, $\expectE[a_t\mid h_t]$ becomes biased, which can lead to arbitrarily worse performance compared to $\pi_E$.   }
\end{remark}

\vspace{-5pt}
\paragraph{Derivation of CMRs.} 
Leveraging the confounding horizon from Assumption~\ref{assump:horizon}, it becomes possible to break the spurious correlation using the independence of $u^\epsilon_t$ and $u^\epsilon_{t-k}$. We propose to use the $k$-step trajectory history $h_{t-k}=(s_{1},a_{1},...,s_{t-k})$ as an instrument for the current state $s_t$. Taking the expectation conditional on $h_{t-k}$ in~\cref{eq:history_policy} yields
\begin{align*}
    \expectE[a_t\mid h_{t-k}] & = \expectE\left[\expectE[a_t\mid h_{t}]\mid h_{t-k}\right] \\ & = \expectE[\pi_h(h_t)\mid h_{t-k}]+\expectE[\expectE[u^\epsilon_t\mid h_{t}]\mid h_{t-k}] \\
    & = \expectE[\pi_h(h_t) \mid h_{t-k}]+\expectE[u^\epsilon_t\mid h_{t-k}]
\end{align*}
where we use the fact that $h_{t-k}$ is $\sigma(h_t)$-measurable because $h_{t-k}\subseteq h_t$. Next, recall that $u^\epsilon_t\indep u^\epsilon_{t-k}$ by Assumption~\ref{assump:horizon}, which implies $u^\epsilon_t\indep h_{t-k}$, so that % Hence, since $\expectE[u^\epsilon_t] = 0$, we obtain
\begin{align}
    \expectE[a_t\mid h_{t-k}] &= \expectE[\pi_h(h_t) \mid h_{t-k}]+\expectE[u^\epsilon_t]\nonumber\\
    &=\expectE[\pi_h(h_t) \mid h_{t-k}].
\end{align}

As a result, the problem of learning $\pi_h$ reduces to solving for $\pi_h$ that satisfies the following identity
\begin{align}
    \expectE[a_t-\pi_h(h_t)\mid h_{t-k}]=0,\label{eq:CMR}
\end{align}
which is a CMR problem as defined in~\cref{sec:cmr}. In this case, both $a_t$ and $h_t$ are observed in the confounded expert demonstrations, and $h_{t-k}$ acts as the instrument. 

To make sure the instrument $h_{t-k}$ is valid, we check that it satisfies the conditions of~\cref{assump:iv}. Firstly, we have checked that $u^\epsilon_t\indep h_{t-k}$. Secondly, the environment and the expert policy are non-trivial, which means $\probP(h_t\mid h_{t-k})$ is not constant in $h_{t-k}$. Finally, $h_{t-k}$ indeed only affects $a_t$ through $s_t$ by the Markovian property. However, the strength of the instrument, which informally represents the correlation between the instrument $h_{t-k}$ and $h_t$, plays an important role in how well we can identify $\pi_h(h_t)$ by solving the CMRs in~\cref{eq:CMR}. In particular, we see that, as the confounding horizon $k$ increases, the correlation between $h_{t-k}$ and $h_t$ weakens and $h_{t-k}$ becomes a weaker instrument. This means that it is less able to identify $\pi_h$ via the CMR in~\cref{eq:CMR} and the final learnt imitator will have poorer performance. This is confirmed theoretically in Proposition~\ref{prop:ill-posed} and experimentally in~\cref{sec:exps}, and we will formalise this notion of instrument strength in~\cref{sec:theory}.


% Note this problem is equivalent to solving an IV regression on~\cref{eq:history_policy}, where $Y=\expectE[a_t\lvert h_t]$, $f(x)=\pi_h(h_t)$, $\epsilon=\expectE[u^\epsilon_t$ and the instrument $Z=h_{t-k}$.




\subsection{Practical Algorithms for Solving the CMRs}

\begin{algorithm}[tb]
   \caption{DML-IL}
   \label{alg:DML-IL}
\begin{algorithmic}[1]
   \STATE {\bfseries input} Dataset $\dataset_E$ of expert demonstrations, Confounding noise horizon $k$
   \STATE Initialize the roll-out model $\hat{M}$ as a Gaussian mixture model\label{algo:roll_out_1}
    \REPEAT
   \STATE Sample $(h_{t},a_t)$ from data $\dataset_E$
   \STATE Fit the roll-out model $(h_t,a_t)\sim\hat{M}(h_{t-k})$ to maximize the log likelihood 
\UNTIL{convergence}\label{algo:roll_out_2}
   \STATE Initialize the expert model $\hat \pi_h$ as a neural network
   \REPEAT
   % \FOR{$k=1$ {\bfseries to} $K$}
   \STATE Sample $h_{t-k}$ from $\dataset_E$
   \STATE Generate $\hat{h}_t$ and $\hat{a}_t$ using the roll-out model $\hat{M}$
   \STATE Update $\hat \pi_h$ to minimise the loss $\ell:= \norm{\hat{a}_t - \hat{\pi}_h (\hat h_t)}_2$
   % \ENDFOR
    \UNTIL{convergence}
    \STATE {\bfseries return} A history-dependent imitator policy $\hat{\pi}_h$
\end{algorithmic}
\end{algorithm}

There are various techniques~\citep{Shao2024,Bennett2019,Xu2020,Dikkala2020} for solving the CMRs $\expectE[a_t\lvert h_{t-k}]=\expectE[\pi_h(h_t) \lvert h_{t-k}]$. Here, the \textit{CMR error} that we aim to minimise is given by 
\begin{align*}
\sqrt{\expectE\big[\expectE[a_t-\hat{\pi}_h(h_t)\lvert h_{t-k}]^2\big]}=\norm{\expectE[a_t-\hat{\pi}_h(h_t)\lvert h_{t-k}]}_{2}.    
\end{align*}
In~\cref{alg:DML-IL}, we introduce DML-IL, an algorithm adapted from the IV regression algorithm DML-IV~\citep{Shao2024}\footnote{DML stands for double machine learning~\citep{Chernozhukov2018Double}, which is a statistical technique to ensure fast convergence rate for two-step regression, as is the case in~\cref{alg:DML-IL}.}, which solves our CMRs by minimising the CMR error. The first part of the algorithm (line 3-7) learns a roll-out model $\hat{M}$ that generates a trajectory $k$ steps ahead given $h_{t-k}$. Then, the roll-out model $\hat{M}$ is used to train the policy model $\hat{\pi}_h$ (line 8-13). $\hat{\pi}_h$ takes the generated trajectory $\hat{h}_t$ from $\hat{M}(h_{t-k})$ as inputs, and minimises the mean squared error to the next action. Using generated trajectories is crucial in breaking the spurious correlation caused by $u^\epsilon_t$ between past states and actions, and using the trajectory history before $h_{t-k}$ allows the imitator to infer information about $u^o_t$.

DML-IL can also be implemented with $K$-fold cross-fitting, where the dataset is partitioned into $K$ folds, with each fold alternately used to train $\hat{\pi}_h$ and the remaining folds to train $\hat{M}$. This ensures unbiased estimation and improves the stability of training. The base IV algorithm DML-IV with $K$-fold cross-fitting is theoretically shown to converge at the rate of $O(N^{-1/2})$~\citep{Shao2024}, where $N$ is the sample size, under regularity conditions. DML-IL with $K$-fold cross-fitting (see~\cref{appendix:dmlil} for details) will thus inherit this convergence rate guarantee. 

Note that~\cref{alg:DML-IL} requires the confounding noise horizon $k$ as input. While the exact value of $k$ can be difficult to obtain in reality, any upper bound $\bar{k}$ of $k$ is sufficient to guarantee the correctness of ~\cref{alg:DML-IL}, since $h_{t-\bar{k}}$ is also a valid instrument. Ideally, we would like a data-driven approach to determine $k$. Unfortunately, it is generally intractable to empirically verify whether $h_{t-k}$ is a valid instrument from a static dataset, especially the unconfounded instrument condition (i.e., $h_{t-k}\indep u^\epsilon_t$). Therefore, we rely on the user to provide a sensible choice of $\bar{k}$ based on the environment that does not substantially overestimate $k$.


\subsection{Theoretical Analysis}\label{sec:theory}

% \begin{align}
% p(u_t\lvert do(a_{t-k+1}),...,do(a_{t-1}),s_{t-k+1},...,s_{t-1})&\propto p(h_t)p_{\mu_0}(s_{t-k+1})\prod_{i=t-k+1}^{t-1} \transitions(s_{i+1}\lvert s_i,a_i,u_i)
% \end{align}

% since $$(u_t\indep a_{(t-k+1)...(t-1)} \lvert s_{(t-k+1)...(t_1)})_{\mathcal{G}_{\underline{a{(t-k+1)...(t-1)}}}}$$
% on the causal graph $\mathcal{G}_{\underline{a{(t-k+1)...(t-1)}}}$ where the arrows going into $a_{(t-k+1)...(t-1)}$ are removed.



In this section, we derive theoretical guarantees for our algorithm, focusing on the imitation gap and its relationship with existing work.


On a high level, in order to bound the imitation gap of the learnt policy $\hat{\pi}_h$, i.e., $J(\pi_E)-J(\hat{\pi}_h)$, we need to control:
\begin{enumerate}
    \item[($i$)] The amount of information about the hidden confounders that can be inferred from trajectory histories;
    \item[($ii$)] The ill-posedness (or identifiability) of the set of CMRs, which intuitively measures the strength of the instrument $h_{t-k}$;
    \item[($iii$)] The disturbance of the confounding noise to the states and actions at test time.
\end{enumerate}
These factors are all determined by the environment and the expert policy. To control ($i$), we measure how much information about $u^o_t$ is captured by the trajectory history $h_t$ by analysing the Total Variation (TV) distance between the distribution of $u^o_t$ and $\expectE[u^o_t\lvert h_t]$ along the trajectories of $\pi_E$. To control ($ii$) and ($iii$), we need to introduce the following two key concepts.

\begin{definition}[The ill-posedness of CMRs~\citep{Dikkala2020,Chen2012}]

Given the derived CMRs in~\cref{eq:CMR}, for a policy $\pi\in\Pi$, $\norm{\pi_E-\pi}_2$ is the root mean squared error to the expert and $\norm{\expectE[a_t-\pi(s_t)\lvert s_{t-k}]}_2$ is the CMR error we aim to minimise. Then, the \emph{ill-posedness} $\ill(\Pi,k)$ of the policy space with confounding noise horizon $k$ is given by
\begin{align*}
    \ill(\Pi,k)=\sup_{\pi\in\Pi} \frac{\norm{\pi_E-\pi}_{2}}{\norm{\expectE[a_t-\pi(h_t)\lvert h_{t-k}]}_{2}}.
\end{align*}
\end{definition}
The ill-posedness $\ill(\Pi,k)$ measures the strength of the instrument where a higher $\ill(\Pi,k)$ indicates a weaker instrument. It bounds the ratio between the learning error of the imitator following our CMR objective and its $L_2$ error to the expert policy. 

As discussed previously, intuitively, the strength of the instrument would decrease as the confounding horizon $k$ increases. This is in fact true and is confirmed by the following proposition. The proof is deferred to~\cref{appendix:prop}. 
\begin{proposition}\label{prop:ill-posed}
The ill-posedness $\ill(\Pi,k)$ is monotonically increasing as the confounded horizon $k$ increases.
\end{proposition}

Next, we introduce the notion of c-TV stability.
\begin{definition}[c-total variation stability~\citep{Bassily2021,Swamy2022_temporal}]
Let $P(X)$ be the distribution of a random variable $X:\Omega\rightarrow \mathcal{X}$. $P(X)$ is c-TV stable if for $a_1,a_2\in \mathcal{X}$ and $\Delta>0$,
\begin{align*}
\norm{a_1-a_2}\leq\Delta \implies \delta_{TV}(a_1+X,a_2+X)\leq c\Delta.
\end{align*}
where $\norm{\cdot}$ is some norm defined on $\mathcal{X}$ and $\delta_{TV}$ is the total variation distance.
\end{definition}
A wide range of distributions are c-TV stable. For example, standard normal distributions are $\frac{1}{2}$-TV stable. We apply this notion to the distribution over $u^\epsilon_t$ to bound the disturbance it induces in the trajectory and the expected return.

With the notion of ill-posedness and c-TV stability, we can now analyse and upper bound the imitation gap $J(\pi_E)-J(\hat{\pi}_h)$ by controlling the three components $(i)-(iii)$ discussed above. 
% We present the main result for this paper, where t
The full proof is deferred to~\cref{appendix:gap}.

\begin{theorem}[Imitation Gap Bound]\label{thm:gap}
Let $\hat{\pi}_h$ be the learnt policy with CMR error $\epsilon$ and let $\ill(\Pi,k)$ be the ill-posedness of the problem. Assume that $\delta_{TV}(u^o_t,\expectE_{\pi_E}[u^o_t\lvert h_t])\leq\delta$ for $\delta\in\realNumber^+$, $P(u^\epsilon_t)$ is c-TV stable and $\pi_E$ is deterministic. Then, the imitation gap is upper bounded by 
\begin{align*}
    J(\pi_E)-J(\hat{\pi}_h)\leq T^2\big(c\epsilon\ill(\Pi,k)+2\delta\big)=\mathcal{O}\big(T^2(\delta+\epsilon)\big).
\end{align*}
\end{theorem}
This upper bound scales at the rate of $T^2$, which aligns with the expected behaviour of imitation learning without an interactive expert~\citep{Ross2010}.
Next, we show that the upper bounds on the imitation gap from prior work~\citep{Swamy2022_temporal, Swamy2022} are special cases of
% of  subsumed by the unifying causal IL framework introduced in Section~\ref{sec:setting} are special cases of 
Theorem~\ref{thm:gap}. The proofs are deferred to~\cref{appendix:corollaries}.
\begin{corollary}\label{corollary:noUo}
In the special case that $u^o_t = 0$, i.e., there are no expert-observable confounders, or $u^o_t=\expectE_{\pi_E}[u^o_t\lvert h_t]$, i.e., $u^o_t$ is $\sigma(h_t)$ measurable (all information about $u^o_t$ is contained in the history), the imitation gap is upper bounded by
\begin{align*}
    J(\pi_E)-J(\hat{\pi}_h)\leq T^2\big(c\epsilon\ill(\Pi,k)\big)=\mathcal{O}\big(T^2\epsilon\big),
\end{align*}
which coincides with Theorem 5.1 of~\citet{Swamy2022_temporal}.
\end{corollary}

When there are no hidden confounders, i.e, $u^\epsilon_t=0$, our framework is reduced to that of~\citet{Swamy2022}. However, \citet{Swamy2022} provided an abstract bound that directly uses the supremum of key components in the imitation gap over all possible Q functions to bound the imitation gap. We further extend and concretise the bound using the learning error $\epsilon$ and the TV distance bound $\delta$ instead of relying on the suprema.


\begin{corollary}\label{corollary:unconfounded}
In the special case that $u^\epsilon_t=0$, if the learnt policy has optimisation error $\epsilon$,  the imitation gap is upper bounded by
\begin{align*}
    J(\pi_E)-J(\hat{\pi}_h)\leq T^2\left(\frac{2}{\sqrt{\dim(A)}}\epsilon+2\delta \right),
\end{align*}
which is a concrete bound that extends the abstract bound in Theorem 5.4 of~\cite{Swamy2022}.
\end{corollary}

\begin{remark}
\emph{If both $u^\epsilon_t$ and $u^o_t$ are zero, we then recover the classic setting of IL without confounders~\citep{Ross2010}, and the imitation gap bound is $T^2\epsilon$, where $\epsilon$ is the optimisation error of the algorithm.}
\end{remark}

\section{Experiments and Results}\label{sec:experiments}
In this section, we demonstrate the efficacy and versatility of EigenLoRAx across diverse tasks, modalities, and model architectures, highlighting its individual advantages. EigenLoRAx requires significantly fewer parameters to match or surpass LoRA’s performance (Tables~\ref{tab:vision_models}, \ref{tab:glue_benchmark_results}) and achieves similar or faster loss convergence (Figure~\ref{fig:traininglosscola}), making it a cost-effective alternative to random initialization and other methods~\citep{meng_pissa_2024}. Additionally, we showcase its memory-efficient inference capabilities with a Stable Diffusion text-to-image generation model~\citep{diffusion} (Section~\ref{sec:diffusion}). Notably, EigenLoRAx retains its efficiency even in low-resource scenarios where a large number of LoRAs are unavailable.

\paragraph{Note on Baselines} Our focus is on recycling adapter knowledge and improving training and memory efficiency while maintaining performance, not solely on maximizing performance. We compare primarily with LoRA, as EigenLoRAx builds its principal subspace using pretrained LoRA adapters. Using better adapters and optimization could further enhance the subspace and performance. 

Due to lack of space, more experiments (3D Object pose estimation) and detailed ablation experiments are presented in \cref{sec:appendix}. 

\section{Experimental Results}
We demonstrate the effectiveness of STAIR through extensive experiments on multiple benchmarks that reflect both the safety guardrails and general capabilities of LLMs. 

\subsection{Experimental Settings}

We hereby introduce the key experimental settings, with more details explained in~\cref{sec:appendix_data} and~\ref{sec:appendix_exp}.


\textbf{Models and Datasets.} We take two base LLMs for safety alignment, LLaMA-3.1-8B-Instruct~\cite{dubey2024llama} and Qwen-2-7B-Instruct~\cite{qwen2}. For test-time scaling and ablation studies, only LLaMA is utilized. All experiments use a seed dataset $\mathcal{D}$ comprising 50k samples from three sources. For safety-focused data, we use a modified version of 22k preference samples from PKU-SafeRLHF~\cite{ji2024pku} along with 3k jailbreak data from JailbreakV-28k~\cite{luo2024jailbreakv}. Additionally, 25k pairwise data are drawn from UltraFeedback~\cite{cui2024ultrafeedback} to maintain helpfulness, as done in prior works~\cite{qi2024safety,wu2024thinking}. Note that responses in $\mathcal{D}$ are in normal conversational style rather than reasoning-oriented. While we use the whole dataset with labels for training baselines, we only take 10k samples each from PKU-SafeRLHF and UltraFeedback to construct structured CoT data $\mathcal{D}_{\text{CoT}}$. During each self-improvement iteration, 5k safety and 5k helpfulness samples are utilized. Jailbreak prompts are used only in the final two iterations, with 1k and 2k samples, respectively.

\textbf{Baselines.} We first evaluate the performance of CoT prompting~\cite{wei2022chain} to assess the contribution of available reasoning capability to safety consolidation. We then include SFT and DPO~\cite{rafailov2024direct} on standard datasets as representative alignment techniques, both of which are employed in our framework. Besides, SACPO~\cite{wachi2024stepwise}, designed to mitigate the safety-performance trade-off with two-step DPO, and Self-Rewarding~\cite{yuanself}, which leverages self-generated and self-rewarded data in iterative DPO, are also used as baselines for comparison.


\textbf{Evaluation.} We use 10 popular benchmarks to evaluate harmlessness and general performance of the trained models. For harmlessness, models are required to provide clear refusals to harmful queries, following~\cite{guan2024deliberative}. We test the models on StrongReject~\cite{souly2024strongreject}, XsTest~\cite{rottger2023xstest}, highly toxic prompts from WildChat~\cite{zhaowildchat}, and the stereotype-related split from Do-Not-Answer~\cite{wang2023not}. We report the average goodness score on the top-2 jailbreak methods of PAIR~\cite{chaojailbreaking} and PAP~\cite{zeng2024johnny} for StrongReject, and refusal rates for the rest. For general performance, we use benchmarks reflecting diverse aspects of trustworthiness in addition to the popular ones for helpfulness like GSM8k~\cite{hendrycks2measuring}, AlpacaEval2.0~\cite{dubois2024length} and BIG-bench HHH~\cite{zhou2024beyond}. We take SimpleQA~\cite{wei2024measuring} for truthfulness, InfoFlow~\cite{mireshghallahcan} for privacy awareness, and AdvGLUE~\cite{wang2adversarial} for adversarial robustness. Official metrics are reported for all.

% We leave other details including hyperparameters and evaluation strategies in~\cref{sec:appendix_exp}.


\begin{table*}[ht]
    \centering
    \caption{Performance on diverse benchmarks reflecting both harmlessness and general performance. CoT Style represents whether the method adopt Chain-of-Thought reasoning, while Self Gen. denotes whether the method use self-generated data for training. For all reported metrics, the best results are marked in \textbf{bold} and the second best results are marked by \underline{underline}.}
    \renewcommand{\arraystretch}{1.1} % Increase row height
    
\resizebox{\textwidth}{!}{%
    \begin{tabular}{l@{\;\,}|@{\;\,}c@{\;\,}|@{\;\,}c@{\;\,}|c@{\;\,}c@{\;\,}c@{\;\,}c|c@{\;\,}c@{\;\,}c@{\;\,}c@{\;\,}c@{\;\,}c}
        \toprule[1.5pt]
       & \multirow{2}{*}{\makecell{CoT\\Style}} & \multirow{2}{*}{\makecell{Self\\Gen.}}  &  \multicolumn{4}{c|}{\textbf{Harmlessness}} & \multicolumn{6}{c}{\textbf{General}}  \\ \cmidrule(lr){4-7}\cmidrule(lr){8-13}
       & & & StrongReject  & XsTest  & WildChat  & Stereotype  &  SimpleQA 	&  InfoFlow  &  AdvGLUE  & GSM8k  & AlpacaEval  & HHH  \\\midrule
        \multicolumn{13}{c}{\sc Llama-3.1-8B-Instruct} \\ \midrule
        Base &  - & - & 0.4054 & 88.00\% & 47.94\% & 87.37\% & 2.52\% & 0.4229 & 58.33\% &85.60\% &  25.55\% & 82.50\%\\ 
        CoT & \cmark & - & 0.3790 & 87.00\% & 50.23\% & 65.26\% & 4.09\% &  0.7041 & 58.40\% & 87.11\% &22.04\% & 81.63\% \\
        SFT & \xmark & \xmark & 0.4698 & 94.50\% & 50.68\% & 94.74\% & 4.72\% &  0.7134 & 57.53\% &72.02\% & 9.21\% & 82.63\% \\
        DPO & \xmark & \xmark & 0.5054 & 86.00\% & 54.79\% & \bf 97.89\% & 4.46\% & 0.7081 & 66.27\% &84.15\% &  15.26\% & 83.84\% \\ 
        SACPO & \xmark & \xmark  & 0.7264 & 88.50\% & 58.45\% & 96.84\% & 0.74\% &  0.0503 & 65.60\% &86.50\% & 20.44\% & 85.21\%\\ 
        Self-Rewarding & \xmark & \cmark & 0.4633 & \bf 99.00\% & 49.77\% & 94.74\% & 2.70\%  & 0.6618 & 59.10\% & \bf 88.10\%& 26.41\% & 82.09\%\\\midrule
        STAIR-SFT & \cmark & \xmark & 0.6536 & 85.50\% & 50.68\% & 94.74\% & \underline{6.31\%} & \underline{0.7876} & \bf 70.57\% & 86.05\%  &  31.21\% & 83.13\%\\
        +DPO-1 & \cmark & \cmark & 0.6955 & 94.00\% & 57.99\% & \bf 97.89\% & 6.08\% & \bf 0.7998 & 65.93\% & 86.81\% & 34.48\% & 84.53\% \\
        +DPO-2 & \cmark & \cmark & \underline{0.7973} & 96.50\% & \underline{68.95\%} & 96.84\% & 6.00\% &  0.7700 & \underline{69.43\%} & 87.26\% &\underline{36.24\%} & \bf 87.09\% \\
        +DPO-3 & \cmark & \cmark & \bf  0.8798 &  \bf 99.00\% & \bf 69.86\% & 96.84\% & \bf 6.38\% &  0.7395 & 69.20\% &\underline{87.64\%} &\bf  38.66\% & \underline{85.66\%} \\ \midrule
        \multicolumn{13}{c}{\sc Qwen-2-7B-Instruct} \\ \midrule
        Base &  - & - & 0.3808 & 72.50\% & 47.49\% & 90.53\% & 3.79\% & 0.7221 & 66.50\%& \underline{87.49\%}  & 20.06\% & 87.87\%\\ 
        CoT & \cmark & -  & 0.3792 & 70.00\% & 42.92\% & 88.42\% & 3.03\%& 0.7628 & 65.60\% & \bf 88.10\%  & \underline{25.97\%} & 88.30\%\\
        SFT & \xmark & \xmark & 0.4952 & 84.00\% & 58.45\% & 91.58\% & 3.47\% & 0.6267 & 66.90\% &82.34\% &  8.94\% & 89.74\% \\
        DPO & \xmark & \xmark & 0.5026 & 69.00\% & 66.21\% & 88.42\% & 2.59\% &  0.6793 & 70.97\% & 81.43\% & 11.48\% & 88.08\% \\
        SACPO & \xmark & \xmark & 0.5577 & 75.00\% & 60.27\% & 95.79\% & 0.62\%  & 0.6213 & 64.10\% & 85.22\% & 17.04\% & 89.60\% \\ 
        Self-Rewarding & \xmark & \cmark & 0.5062 & 96.00\% & 52.51\% &  94.74\% & 3.37\% & 0.7140 & 66.13\% & 87.34\% & 14.69\% & 88.31\% \\\midrule
        STAIR-SFT & \cmark & \xmark & 0.7356 & 83.50\% & 62.56\% & 95.79\% & 3.81\% &  0.8215 & 70.57\% &84.61\% & 20.31\% & \underline{90.38\%} \\
        +DPO-1 & \cmark & \cmark & 0.7606 & 96.50\% & 65.19\% & 95.79\% & \underline{3.88\%} & \underline{0.8235} & \underline{73.10\%} & 84.76\% & 23.29\% & 90.21\% \\
        +DPO-2 & \cmark & \cmark & \underline{0.8137} & \underline{98.50\%} & \underline{67.90\%} & \underline{97.89\%} & 3.79\% & \bf 0.8646 & 72.83\% & 86.05\% & 24.86\% & 90.11\% \\
        +DPO-3 & \cmark & \cmark & \bf 0.8486 & \bf 99.00\% & \bf 80.56\% & \bf 98.95\% & \bf 4.07\% & 0.7644 & \bf 74.13\% & 85.75\% & \bf 26.31\% & \bf 90.71\% \\ \bottomrule[1.5pt]
    \end{tabular}}
    \label{tab:benchmarks}
    \vspace{-2ex}
\end{table*}



\subsection{Main Results}

We present the results on diverse benchmarks evaluating both the harmlessness and the general performance in~\cref{tab:benchmarks}, which shows the superiority of STAIR, attributed to the incorporation of introspective reasoning to safety alignment and the self-improvement on stepwise data generated with SI-MCTS. 
We use STAIR-SFT to represent the model trained on $\mathcal{D}_\text{CoT}$ with SFT and DPO-k to denote the model after the k-th iteration of self-improvement. Some qualitative examples are displayed in~\cref{sec:appendix_examples}.

First of all, though initially aligned with instruction tuning, the base LLMs remain vulnerable to harmful queries, especially jailbreak attacks. This is evidenced by the goodness scores below 0.40 on StrongReject. We then explore CoT prompting to stimulate the existing reasoning capability in LLMs. While it leads to improvements in reasoning-dependent tasks like GSM8k and InfoFlow, it shows no enhancement in safety. When applying SFT or DPO to the whole dataset $\mathcal{D}$, we observe significant safety-performance trade-offs due to the conflicting objectives. For instance, for both LLaMA-3.1 and Qwen-2 trained with SFT and DPO, their winning rates against GPT-4 on AlpacaEval decline sharply compared to base models. By employing safety-constrained optimization, the trade-off issue is mitigated to a large extent by SACPO, with better safety enhancements compared to previous methods. However, the performance on SimpleQA and InfoFlow degrades, reflecting losses in factual knowledge and over-refusals to benign privacy-related queries. For Self-Rewarding, their improvements on XsTest, which contains queries apparently harmful, are considerable due to the original behaviors of direct refusals in base LLMs. Nevertheless, the behaviors of refusals fail to generalize to jailbreak attacks, as they lack sufficient capabilities to analyze the underlying risks. 

In comparison, STAIR demonstrates more balanced and continuous improvements on diverse benchmarks. With CoT format alignment, the models acquire the basic ability of safety-aware reasoning, enhancing their resilience against harmful inputs. Further training with stepwise preference data generated by SI-MCTS leads to consistent safety enhancements while maintaining or even improving general performance. For example, LLaMA-3.1 achieves an increase of over 20\% in refusal rate on WildChat after three iterations of self-improvement, while its winning rate against GPT-4 on AlpacaEval reaches 38.66\%, a significant improvement compared to 25.55\% for the base model. Similar trends are observed on other benchmarks like SimpleQA and GSM8k. Besides, the accuracy on AdvGLUE is substantially higher than other baselines, highlighting the benefit to robustness from step-by-step reasoning. On StrongReject, both LLMs eventually reach goodness scores of 0.8798 and 0.8486 respectively, which firmly confirm the positive impact of integrating reasoning with safety alignment.

\subsection{Test-time Scaling}

Using the trained process reward model, we investigate the impact of test-time scaling. Since both stepwise and full-trajectory data are used for training, we employ Best-of-N (BoN) and Beam Search, with results presented in~\cref{fig:tts-safe} and~\ref{fig:tts-helpful} for StrongReject and AlpacaEval respectively. Extra computational costs are estimated based on the number of generated steps relative to one-time greedy decoding, expressed in logarithmic form. For example, Bo8 and beam search generating 4 successors with a beam width of 2 correspond to $\log_2(N)=3$. The results indicate that test-time scaling consistently improves both safety and helpfulness. Both searching methods bring improvements of 0.06 for goodness on StrongReject and more than 3.0\% for winning rates on Alpaca.
This supports that the effect of test-time scaling can generalize from math and coding~\cite{snell2024scaling,xie2024self} to more general scenarios like safety.


\begin{figure*}[t]
     \centering
     \begin{minipage}{0.3\textwidth}
         \centering
         \includegraphics[width=\textwidth, trim={1cm 1cm 1cm 1cm}]{images/draft/strongreject.png}
         \vspace{-4ex}
         \caption{Changes in goodness scores on StrongReject with test-time scaling.}
         \label{fig:tts-safe}
     \end{minipage}
     \hfill
     \begin{minipage}{0.3\textwidth}
         \centering
         \includegraphics[width=\textwidth, trim={1cm 1cm 1cm 1cm}]{images/draft/alpaca.png}
         \vspace{-4ex}
         \caption{Changes in winning rates on AlpacaEval when with test-time scaling.}
         \label{fig:tts-helpful}
     \end{minipage}
     \hfill
     \begin{minipage}{0.3\textwidth}
         \centering
         \includegraphics[width=\textwidth, trim={1cm 1cm 1cm 1cm}]{images/draft/balance.png}
         \vspace{-4ex}
         \caption{Results on StrongReject and AlpacaEval as the ratio of safety data varies.}
         \label{fig:data}
     \end{minipage}
        % \caption{Three simple graphs}
        % \label{fig:three graphs}
    \vspace{-1ex}
\end{figure*}


\subsection{Detailed Analysis}

We then conduct some ablation studies to confirm the effectiveness of our framework.

\textbf{Balance between Safety and Helpfulness Data.} To evaluate the impact of the ratio between safety and helpfulness data in the training dataset, we conduct a study during the CoT format alignment stage as a representative. We plot the performance in terms of safety and helpfulness to the varying ratios in~\cref{fig:data}. While a trade-off between safety and helpfulness is observed, consistent with prior findings~\cite{bai2022training}, the performance in both dimensions consistently exceeds that of the base model. This highlights the effectiveness of training with structured CoT data.

\textbf{Step-level Optimization.} To verify the effectiveness of stepwise preference data in the stage of self-improvement, we compare the performance of DPO-1, which is trained on stepwise data based on STAIR-SFT using DPO, with models trained on full trajectory data using either SFT or DPO. The full trajectory data is selected from the same search trees of SI-MCTS, with the total number of training samples kept equal to that of DPO-1. Results in~\cref{tab:iterative} support our strategy of step-level optimization, which brings more fine-grained supervision to safety-aware reasoning.

\textbf{Iterative Training.} We adopt iterative optimization for continuous improvement, motivated by the belief that data generated in later iterations is of higher quality. To validate this, we compare the results of DPO-3 with the model trained using data crafted from all prompts in a single iteration and the model trained on data from the first iteration for three times as many epochs. Results in~\cref{tab:iterative} demonstrate superior improvements on different benchmarks, confirming the improving data quality throughout iterations.




\begin{table}[ht]
\vspace{-1ex}
    \centering
    \caption{Ablation studies on iterative training on stepwise data}
    % \renewcommand{\arraystretch}{1.2} % Increase row height
\resizebox{\linewidth}{!}{%
    \begin{tabular}{l@{\;\,}|@{\;\,}c@{\;\,}c@{\;\,}c@{\;\,}c}
    \toprule[1.5pt]
         & StrongReject & XsTest & GSM8k & AlpacaEval  \\ \midrule
      \multicolumn{5}{c}{Stepwise Data}\\\midrule
      STAIR-SFT + Full (SFT) &  0.6222 & 87.00\% & 85.29\% & 28.10\% \\
      STAIR-SFT + Full (DPO) &  0.6663 & 92.50\% & 86.50\% & 32.87\%\\\midrule
      STAIR-SFT + Step (DPO) & \bf 0.6955 & \bf 94.00\% & \bf 86.81\% & \bf 34.48\% \\\midrule
      \multicolumn{5}{c}{Iterative Training}\\\midrule
      1st Split, 3$\times$ Epochs & 0.6745 & 97.50\%  & 85.75\% & 37.28\% \\
      Full Dataset, 1 Iteration   & 0.7342 & 90.00\%  & 86.58\% & 36.96\%\\\midrule
      STAIR-DPO-3 & \bf 0.8798 & \bf 99.00\% &  \bf 87.64\% & \bf 38.66\% \\\bottomrule[1.5pt]
    \end{tabular}}
    \label{tab:iterative}
    \vspace{-2ex}
\end{table}


\section{Conclusion}

We introduce EigenLoRAx, a significantly efficient model finetuning and inference method that recycles publicly available pretrained adapters by finding a shared principal subspace. This allows finetuning on new data by simply learning the lightweight coefficients of the shared subspace, and also requires less number of parameters to be saved for new tasks. Our comprehensive and diverse experiments show that EigenLoRAx is applicable to a large range of problems and model architectures. We believe that EigenLoRAx has the potential to mitigate the perpetually widening compute resource gap~\citep{Ahmed2020TheDO, Besiroglu2024TheCD} and reduce the environmental cost of training and using machine learning models~\citep{Wu2021SustainableAE, Ligozat2021UnravelingTH}. It also holds promise for training personalized models~\citep{tan2024democratizing} on low-resource devices, in privacy-critical use-cases. We have a large number of experiments ($6+$) on a diverse set of complex models, tasks and modalities. We have shown that EigenLoRAx excels in faster and efficient learning, memory savings and zero shot performance which differentiates it from conventional PeFT models. While our work is application-focused, we believe it is the first work to hypothesize and empirically prove the existence of shared weight subspaces of neural networks. This important insight has significant implications of model merging, efficiency, mechanistic interpretability, and neural learning theory.

{\small\paragraph{A Note on Prior Work} This paper works on some ideas initially introduced in the EigenLoRA work~\cite{kaushik2025eigenlora}, which was rejected from ICLR'25 and remains available on OpenReview. Due to irreconcilable differences regarding research ethics and authorship integrity, the primary authors of this work chose to independently develop and extend their contributions. These contributions in the previous draft included the initial algorithm, all experimental results (excluding Section 4.2.2), as well as all ablation studies and additional experiments. The code used for these experiments was also developed solely by the first authors of this work and is publicly available. Since the prior work was neither accepted nor published, there are no concerns regarding self-plagiarism. We also acknowledge discussions and exchanges with some of the earlier authors, which, while not directly contributing to the technical developments of this work, provided context and background that shaped our understanding of the broader problem. We have taken extensive measures to ensure that this paper exclusively reflects the contributions of the authors listed, and any similarities beyond our own contributions are purely coincidental.}

\nocite{sun2025transformersquaredselfadaptivellms, Gain_2020_WACV, Kaushik_2024_CVPR}
\bibliography{iclr2025_conference}
\bibliographystyle{icml2025}


%%%%%%%%%%%%%%%%%%%%%%%%%%%%%%%%%%%%%%%%%%%%%%%%%%%%%%%%%%%%%%%%%%%%%%%%%%%%%%%
%%%%%%%%%%%%%%%%%%%%%%%%%%%%%%%%%%%%%%%%%%%%%%%%%%%%%%%%%%%%%%%%%%%%%%%%%%%%%%%
% APPENDIX
%%%%%%%%%%%%%%%%%%%%%%%%%%%%%%%%%%%%%%%%%%%%%%%%%%%%%%%%%%%%%%%%%%%%%%%%%%%%%%%
%%%%%%%%%%%%%%%%%%%%%%%%%%%%%%%%%%%%%%%%%%%%%%%%%%%%%%%%%%%%%%%%%%%%%%%%%%%%%%%
\newpage
\appendix
\onecolumn
\appendix

\section{Appendix: Prompt}
\label{sec:appendix}
``Here is a sketch of an image. 
$\{input\_color\_mask\}$, while the rest of the white space is the background. 
I need you to infer details of the image based on the given sketch.
The details should include the possible background likely to be present with the $\{input\_color\_mask\}$, the attribute of each object (like wearing, texture, color etc.), the state (including action, posture, etc.) of each object, the direction of each object and the relationships between objects.

You should first analyze the mask carefully, considering the size, location, and relative position of each object mask. Ensure that specific actions are analyzed based on the mask, and infer each aspect with a reasoning process before providing the final output.
The final output format should be: $\{format\_example\}$, and you should refer to the example: $\{few\_shot\}$. You are going to complete the "" in each item, you need to complete them in multiple short phrases based on your above reasoning.

The state and relationship should be as detailed as possible while ensuring they align with the mask, formatted as: objectA action/spatial relation objectB, with both objectA and objectB included.
You should properly refer to some examples of attributes of object $\{attributes\}$ and relationships $\{relationships\}$.
Do not include words like `or', `possibly' in your final output, there should no ambiguity in your output.
Make sure all aspects of given mask is filled.''
%%%%%%%%%%%%%%%%%%%%%%%%%%%%%%%%%%%%%%%%%%%%%%%%%%%%%%%%%%%%%%%%%%%%%%%%%%%%%%%
%%%%%%%%%%%%%%%%%%%%%%%%%%%%%%%%%%%%%%%%%%%%%%%%%%%%%%%%%%%%%%%%%%%%%%%%%%%%%%%


\end{document}


% This document was modified from the file originally made available by
% Pat Langley and Andrea Danyluk for ICML-2K. This version was created
% by Iain Murray in 2018, and modified by Alexandre Bouchard in
% 2019 and 2021 and by Csaba Szepesvari, Gang Niu and Sivan Sabato in 2022.
% Modified again in 2023 and 2024 by Sivan Sabato and Jonathan Scarlett.
% Previous contributors include Dan Roy, Lise Getoor and Tobias
% Scheffer, which was slightly modified from the 2010 version by
% Thorsten Joachims & Johannes Fuernkranz, slightly modified from the
% 2009 version by Kiri Wagstaff and Sam Roweis's 2008 version, which is
% slightly modified from Prasad Tadepalli's 2007 version which is a
% lightly changed version of the previous year's version by Andrew
% Moore, which was in turn edited from those of Kristian Kersting and
% Codrina Lauth. Alex Smola contributed to the algorithmic style files.
