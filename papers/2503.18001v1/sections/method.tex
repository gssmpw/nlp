%! root=../main.tex
% \begin{figure*}[h!]
% 	\centering
% 	\includegraphics[width=0.7\linewidth]{figures/pipeline.pdf}
% 	\caption{\pnameexp overview.}
% 	\label{fig:overview}
% \end{figure*}

\section{\pnameexp Overview}\label{sec:design}

% \heading{Overview.} 
% \subsection{Explanation}

% \heading{Feature Pertubation.}

% \heading{Structural Pertubation.}

% \subsection{Explanation}

The workflow of \pnameexp is illustrated in \autoref{fig:overview} where \pnameexp first performs feature perturbation to find the important features and then performs graph structural perturbations to identify the important subgraph structures. We \textit{focus} on user-city due to the problem definition in \autoref{sec:problem-statement}, but this method works for any relationship.

\subsection{Feature Perturbation.}

To interpret the \gnn-based recommendation system, we first analyze the influence of target node features through a feature perturbation approach. Consider a user $u$ and their recommended city $c_t$ with their embeddings, $\mathbf{h}_u$ and $\mathbf{h}_{c_t}$. The embeddings are obtained from the trained \pname model and their similarity is computed using cosine similarity:

\begin{equation}
    \text{sim}(\mathbf{h}_u, \mathbf{h}_{c_t}) = \frac{\mathbf{h}_u \cdot \mathbf{h}_{c_t}}{\|\mathbf{h}_u\| \|\mathbf{h}_{c_t}\|}.
\end{equation}

The feature perturbation involves the following steps:
\begin{enumerate}    
    \item \textbf{Feature Perturbation:} 
    Perturb the target city's features $\mathbf{x}_{c_t}$ one by one by zeroing them out, and the indices of the features to be zeroed out are in $\mathcal{F}$. The perturbed features $\bar{\mathbf{x}}_{c_t}$ are defined as:
    \begin{equation}
        \bar{\mathbf{x}}_{c_t}[i] = 
        \begin{cases}
          0, & \text{if } i \in \mathcal{F}, \\
          \mathbf{x}_{c_t}[i], & \text{otherwise}.
        \end{cases}
    \end{equation}

    \item \textbf{Compare Performances:} 
    Compare the change in nDCG@K $\Delta \mathrm{nDCG}(\mathcal{F})$ due to the perturbed features $\bar{\mathbf{x}}_{c_t}$:
    \begin{equation}
        \Delta \mathrm{nDCG@K}(\mathcal{F}) 
        = \mathrm{nDCG@K}\bigl(\bar{\mathbf{x}}_{c_t}\bigr) 
        \;-\; \mathrm{nDCG@K}\bigl(\mathbf{x}_{c_t}\bigr).
    \end{equation}

\end{enumerate}

The \pname is re-evaluated using the perturbed features, and the resulting performance degradation is measured using the ranking metric of nDCG@K (Normalized Discounted Cumulative Gain). A significant drop in the metric indicates the importance of the perturbed feature for the recommendation.

\subsection{Structural Perturbation.}

While feature perturbation focuses on node-level characteristics, structural perturbation analyzes the impact of graph topology on the model's predictions. Thus, accounting for the whole graph context into each recommendation explanation. Please note that structural perturbation happens with only the subset of features identified in the previous step. By combining feature and structural perturbation techniques, we provide a comprehensive and interpretable explanation of the GNN-based product recommendation system. Feature perturbation identifies node attributes with significant influence on the recommendation outcomes, while structural perturbation uncovers graph edges and relationships critical to the model's predictions.

Specifically, we study how the presence or absence of edges in the graph influences the recommendation similarity. The procedure is as follows:

\begin{enumerate}
    \item \textbf{Graph Transformation}: From the heterogeneous graph, a user-city graph $\mathcal{G}_h$ is created by collapsing all user-city relationships and removing intermediate nodes (e.g., listings). A $k$-hop subgraph $\mathcal{G}_u^k$ centered around a target user $u$ is then extracted, focusing on both direct and indirect relationships with city nodes.
    
    \item \textbf{Identify Co-clicked Cities}: For the subgraph $\mathcal{G}_u^k$, we identify pairs of cities $(c_i, c_j)$ that share a common predecessor user $u_p$. These pairs are added as new edges, representing co-click relationships, and their contributions to the model predictions are evaluated.
    
    \item \textbf{Edge Removal and Similarity Change}: To assess the importance of structural connections, we iteratively remove identified edges and recompute the similarity between the user embedding $\mathbf{h}_u$ and the target city embedding $\mathbf{h}_{c'_t}$. The change in similarity $\Delta \text{sim}$ after edge removal is defined as:
    
    \begin{equation}
        \Delta \text{sim} = \text{sim}(\mathbf{h}_u, \mathbf{h}_{c'_t}) - \text{sim}(\mathbf{h}_u, \mathbf{h}_{c_t}).
    \end{equation}
    
\end{enumerate}

Edges with the highest absolute $\Delta \text{sim}$ values are identified as critical contributors to the recommendation. These edges represent strong graph relationships that drive user preferences for specific cities. The hyperparameters that influence the structural perturbations are: \nm{1} $k$ (hop distance) as it limits the number of edges perturbed to efficiently evaluate while focusing on the most impactful connections and increasing $k$ captures more indirect relationships but may introduce noise, and \nm{2} edge removal strategy as prioritizing edges based on shared predecessors ensures that only influential connections are analyzed. Hyperparameters such as $k$ (neighborhood size) and edge prioritization strategies enable a flexible and robust explanation framework, balancing depth and computational efficiency. 