\section{Related Work}
\label{sec:related_work}

QA systems are divided into extractive and abstractive types~\cite{fan-etal-2019-eli5}.
This work focuses on extractive QA, also known as reading comprehension, where text
passages are paired with questions, and answers are directly extracted from the text. A
well-known example of an extractive QA dataset is the Stanford Question Answering
Dataset (SQuAD), which includes over 100,000 QA pairs from Wikipedia
articles~\cite{rajpurkar2016squad}. In the case of Icelandic, a language closely related
to Faroese, several QA datasets have been developed.
\citet{snaebjarnarson-einarsson-2022-cross} introduced a cross-lingual open-domain QA
system using machine-translated data, and the Natural Questions in Icelandic, an
extractive QA dataset, which demonstrates approaches applicable to other low-resource
languages such as Faroese~\cite{snaebjarnarson-einarsson-2022-natural}. Similarly,
\citet{skarphedinsson-etal-2023-gameqa} developed a method to gamify QA dataset
creation. However, both approaches relied heavily on human question generation, which
bottlenecked the dataset creation process.

At the time of writing, few benchmark datasets exist for Faroese.
\citet{snaebjarnarson-etal-2023-transfer} introduced named entity
recognition\footnote{\url{https://huggingface.co/datasets/vesteinn/sosialurin-faroese-ner}}
and semantic text similarity
datasets\footnote{\url{https://huggingface.co/datasets/vesteinn/faroese-sts}}. The
FLORES-200 dataset~\cite{nllb2022} is another significant contribution to Faroese
benchmarks, being a multilingual parallel corpus covering over 200 languages, including
Faroese. Additionally, \citet{nielsen2023scandeval} introduced ScaLA-Fo, a linguistic
acceptability dataset for Faroese. Despite these resources, a dedicated Faroese QA
dataset is still lacking, which this work aims to address.