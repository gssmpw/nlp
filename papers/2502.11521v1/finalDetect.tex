\begin{table*}[!t]
    \small
    \centering
    \caption{Price change information-directed attack patterns, systematically mined from all types of DeFi applications (see \mysec\ref{sec:backg_defiProtocol}).}
    \vspace{-2ex}
    \begin{tabular}{ccl}
         \toprule
         Type&\multicolumn{1}{c}{Pattern}&\multicolumn{1}{c}{Details}\\
         \midrule
         %------------------------------------------------------------
         \multirow{2}{5em}[-0.8em]{\makecell[c]{Buy\\\&\\Sell\\~\cite{elephantMoney}}}&I& \makecell[l]{1) Swap $Token_x$ to $Token_y$ through $Pool_{buy}$\\2) The price of $Token_y$ in  $Pool_{sell}$ increases / The price of $Token_z$ in $Pool_{sell}$ decreases\\3) Swap $Token_y$ to $Token_z$ through $Pool_{sell}$}\\
         \cline{2-3}
         &II& \makecell[l]{1) The price of $Token_x$ in $Pool_{buy}$ increases / The price of $Token_y$ in $Pool_{buy}$ decreases/\\ \ \ \ \ The price of $Token_y$ in $Pool_{sell}$ increases / The price of $Token_z$ in $Pool_{sell}$ decreases\\2) Swap $Token_x$ to $Token_y$ through $Pool_{buy}$\\3) Swap $Token_y$ to $Token_z$ through $Pool_{sell}$}\\
         \hline
         %------------------------------------------------------------
         \multirow{2}{5em}[-0.8em]{\makecell[c]{Deposit\\\&\\Borrow\\~\cite{CreamFinance}}}&III& \makecell[l]{1) Deposit $Token_x$ into $Contract_{deposit}$ and get $Token_y$ as credential\\2) The price of $Token_x$ in $Contract_{borrow}$ increases / The price of $Token_z$ in $Contract_{borrow}$ decreases\\3) Borrow $Token_z$ using $Token_x$ as collateral from $Contract_{borrow}$}\\
         \cline{2-3}
         &IV& \makecell[l]{1) The price of $Token_x$ in $Contract_{borrow}$ increases / The price of $Token_z$ in $Contract_{borrow}$ decreases\\2) Deposit $Token_x$ into $Contract_{deposit}$ and get $Token_y$ as credential\\3) Borrow $Token_z$ using $Token_x$ as collateral from $Contract_{borrow}$}\\
         \hline
         %------------------------------------------------------------
         \multirow{2}{5em}[-0.8em]{\makecell[c]{Stake\\\&\\Claim\\~\cite{ATK}}}&V& \makecell[l]{1) Stake $Token_x$ into $Contract_{stake}$\\2) The price of $Token_y$ in $Contract_{claim}$ decreases\\3) Claim $Token_y$ from $Contract_{claim}$}\\
         \cline{2-3}
         &VI& \makecell[l]{1) The price of $Token_x$ in $Contract_{stake}$ increases / The price of $Token_y$ in $Contract_{claim}$ decreases \\2) Stake $Token_x$ into $Contract_{stake}$\\3) Claim $Token_y$ from $Contract_{claim}$}\\
         \hline
         %------------------------------------------------------------
         \multirow{2}{5em}[-0.8em]{\makecell[c]{Deposit\\\&\\Withdraw\\~\cite{Harvest}}}&VII& \makecell[l]{1) Deposit $Token_x$ into $Contract_{deposit}$ and get $Token_y$ as credential\\2) The price of $Token_y$ in $Contract_{withdrawal}$ increases / The price of $Token_z$ in $Contract_{withdrawal}$ decreases \\3) Withdraw $Token_z$ from $Contract_{withdrawal}$ by burning $Token_y$}\\
         \cline{2-3}
         &VIII& \makecell[l]{1) The price of $Token_x$ in $Contract_{deposit}$ increases / The price of $Token_y$ in $Contract_{deposit}$ decreases /\\ \ \ \ \ The price of $Token_y$ in $Contract_{withdrawal}$ increases / The price of $Token_z$ in $Contract_{withdrawal}$ decreases\\2) Deposit $Token_x$ and get $Token_y$ as credential\\3) Withdraw $Token_z$ from $Contract_{withdrawal}$ by burning $Token_y$}\\
         \bottomrule
    \end{tabular}
    \label{tab:attack_pattern}
\end{table*}

\section{Price Manipulation Detection}
\label{sec:finalDetect}

The price change information and high-level DeFi operations recovered from \mysec\ref{sec:priceInfer} and \mysec\ref{sec:defiOperation}, respectively, are finally checked against the detection rules (c.f. Table~\ref{tab:attack_pattern}).
We analyzed all four specific types of DeFi protocols (c.f.~\mysec\ref{sec:backg_defiProtocol}) and their associated attack instances.
Specifically, we examined ElephantMoney~\cite{elephantMoney} for DEX, Cream Finance~\cite{CreamFinance} for lending protocols, ATK~\cite{ATK} for staking-based yield farming protocols, and Harvest~\cite{Harvest} for deposit-based yield farming protocols, which collectively incurred a loss of \$163.3M.

Based on this in-depth analysis, we identify four attack types targeting different DeFi protocols and their eight generalized attack patterns, as depicted in Table~\ref{tab:attack_pattern}.
These attack types are \textit{Buy \& Sell}, \textit{Deposit \& Borrow}, \textit{Stake \& Claim}, and \textit{Deposit \& Withdraw}, with each type corresponding to two specific attack patterns.
We will discuss these attack types and patterns in more detail.
%and generalized eight price change information-directed attack patterns (as shown in \mytab\ref{tab:attack_pattern}) and their corresponding detection rules composed of price change inference (from steps \ding{178} \ding{179}) and high-level DeFi operations (from step \ding{175}).

\noindent\textbf{Buy \& Sell.}
In this type of attack strategy, the attacker primarily profits by first buying $Token_y$ with $Token_x$ through a swap in $Pool_{buy}$ and then selling $Token_y$ for $Token_z$ through a swap in $Pool_{sell}$.
$Token_x$ and $Token_z$ can be the same or different tokens.
In the attack against ElephantMoney, the attacker first conducted a swap in $Pool_{buy}$ to exchange WBNB for ELEPHANT and then invoked the \texttt{mint} function, which triggered a swap in $Pool_{sell}$ to exchange ELEPHANT for WBNB, resulting in a price increase of ELEPHANT in $Pool_{sell}$.
Ultimately, the attacker utilized a reverse swap in $Pool_{sell}$ to obtain WBNB by selling ELEPHANT at the manipulated price.
We design Pattern I based on this attack and subsequently generalize it to Pattern II.
The major difference between these two patterns is that the token price is manipulated before the first swap in Pattern II, allowing the price of tokens in either $Pool_{buy}$ or $Pool_{sell}$ to be manipulated.

\noindent\textbf{Deposit \& Borrow.}
In this type, the attacker inflates the price of the deposited tokens or deflates the price of the borrowed assets as calculated by $Contract_{borrow}$, bypassing the protective mechanism of over-collateralization, thereby borrowing more assets than the actual value of the collateral.
In the Cream Finance incident, the attacker first deposited yUSD as collateral and obtained an equivalent amount of crYUSD as proof of deposit, then inflated the price of yUSD calculated by $Contract_{borrow}$ by transferring a large quantity of yCrv to a specific contract account.
Finally, using yUSD as collateral, the attacker borrowed a large amount of assets, which far exceeded the actual value of the deposited yUSD, from $Contract_{borrow}$.
We design Pattern III based on this attack and then generalize it to Pattern IV.
Pattern IV differs from Pattern III in that the attacker can preemptively increase the price of tokens designated for deposit or decrease the price of assets intended for borrowing as calculated by $Contract_{borrow}$ before the deposit operation.
%Specifically, 
In particular, the motivating example in \mysec\ref{fig:motivationExample} conforms to Pattern IV.
%Particularly, in step 2, the attacker orchestrated a substantial swap of USDe across various liquidity pools, to deflate the price of sUSDe as calculated by the UwULend contract.
%Then, in step 3, the attacker deposited WETH as collateral.
%Finally, in step 4, the attacker borrowed a large amount of sUSDe, which had been manipulated to a low price.

\noindent\textbf{Stake \& Claim.}
This attack type primarily targets yield-farming protocols that offer staking services.
Typically, an attacker first stakes $Token_x$ into the application in one transaction.
The share ratio of the user is calculated based on the value and quantity of the staked asset in real-time and is stored in the state variables.
Then, the attacker decreases the calculated price of $Token_y$ in $Contract_{claim}$ and subsequently claims $Token_y$ from the contract.
$Token_x$ and $Token_y$ can be the same or different tokens.
We derive Pattern V based on the analysis of the attack against ATK.
Specifically, in the first transaction, the attacker initially staked ATK into $Contract_{stake}$.
Since the staking service required that the ATK be held for 24 hours by the contract account before claiming, the attacker waited for a period and executed the second transaction, exploiting a flash loan to deflate the price of ATK in $Contract_{claim}$, subsequently claiming back an amount of ATK significantly higher than the appropriate quantity.
Considering that the attacker can inflate the price of tokens intended for staking beforehand to get an incorrectly calculated share ratio, we further derive Pattern VI from Pattern V.

\noindent\textbf{Deposit \& Withdraw.}
In this attack type, the attacker exploits vulnerabilities in the token pricing mechanism within the deposit or withdrawal contract to conduct price manipulation attacks.
We design and generalize Pattern VII based on the Harvest attack.
In this hack, the attacker first deposited USDC ($Token_x$) in $Contract_{deposit}$ and received fUSDC ($Token_y$) as proof.
Then, by exchanging USDC for USDT, the price of USDC calculated by $Contract_{withdrawal}$ decreased, and the attacker withdrew an excessive amount of USDC ($Token_z$) from $Contract_{withdrawal}$ by burning fUSDC.
In this case, $Token_x$ and $Token_z$ are the same; however, some protocols, such as~\cite{LUSD}, allow different tokens for deposit and withdrawal.
Besides deflating the price of tokens to be withdrawn, the attacker can also inflate the price of tokens used for calculating the withdrawal amount, i.e., $Token_y$.
If the attacker manipulates the token price before depositing, the price of tokens involved in the deposit can also be affected.
Based on this assumption, we generalize Pattern VII to create Pattern VIII.

% \begin{table*}
%     \centering
%     \begin{tabular}{ll}
%          \Xhline{3\arrayrulewidth}
%          \multicolumn{1}{c}{Attack Pattern}& \multicolumn{1}{c}{Detect Rules}\\
%          \Xhline{2\arrayrulewidth}
%          Pattern I& \makecell[l]{\juantao{TODO}DR I}\\
%          \hline
%          Pattern II& \makecell[l]{\juantao{TODO}DR II}\\
%          \hline
%          Pattern III& \makecell[l]{\juantao{TODO}DR III}\\
%          \hline
%          Pattern IV& \makecell[l]{\juantao{TODO}DR IV}\\
%          \hline
%          Pattern V& \makecell[l]{\juantao{TODO}DR V}\\
%          \hline
%          Pattern VI& \makecell[l]{\juantao{TODO}DR VI}\\
%          \hline
%          Pattern VII& \makecell[l]{\juantao{TODO}DR VII}\\
%          \hline
%          Pattern VIII& \makecell[l]{\juantao{TODO}DR VIII}\\
%          \Xhline{3\arrayrulewidth}
%     \end{tabular}
%     \caption{Caption.}
%     \label{tab:my_label}
% \end{table*}

% \juantao{TODO}Write expressions of detection rules