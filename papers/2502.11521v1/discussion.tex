\vspace{-0.5ex}
\section{Discussion}
\label{sec:discussion}

While the precision of \tool is quite satisfactory, 
%\fixme{its recall rate is still affected by cross-transaction attacks (8 cases), closed-sourced contracts (3 cases), slither compilation errors (5 cases) and non-standard ERC20 token (1 case).
its recall rate primarily affected by the following two threats. 
Future work can further optimize the following two aspects:
%enhance the performance by targeting these two issues for optimization.

\noindent\textbf{Cross-transaction price manipulation attacks.} 
\tool currently detects only single-transaction attacks.
However, some price manipulation attacks are designed across multiple transactions to circumvent time restrictions coded in DeFi protocols, as studied in~\cite{chen2024demystifying}.
Detecting cross-transaction attacks is often challenging.
For example, attacks like INUKO~\cite{Inuko} span 48 hours (crossing around 57,000 blocks), making it extremely difficult to precisely identify all related transactions.
Fortunately, our study observed that cross-transaction attacks are less common than single-transaction ones.

\noindent\textbf{Closed-source price calculation functions.} 
The availability of code for price models could affect the detection accuracy of \tool.
According to our study, most DeFi applications are open-source to gain user trust.
To mitigate issues with closed-source price models, we design the Type-II prompt to cover price models in those closed-source liquidity pools.