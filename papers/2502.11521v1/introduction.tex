\section{Introduction}
\label{sec:introduction}

DeFi represents a form of finance that eliminates traditional financial intermediaries by utilizing smart contracts on a blockchain.
A smart contract~\cite{zou2019smart} is a self-executing program with the terms of the agreement between the buyer and the seller directly written into lines of code.
The code and the agreements contained therein exist across a distributed, decentralized blockchain network~\cite{dinh2018untangling}.
Smart contracts enable trusted transactions and agreements to be carried out among disparate, anonymous parties without the need for a central authority, legal system, or external enforcement mechanism.

While price manipulation is a pervasive risk in traditional financial markets, it poses even more significant damage in the context of DeFi.
In the cryptocurrency space, attackers strategically add or remove large amounts of tokens in specific contracts, or impact the total supply of target tokens by burning or minting, to exploit vulnerabilities in the pricing mechanisms of DeFi applications, ultimately inflating or deflating the price of cryptocurrencies~\cite{cointelegraph}.
To detect price manipulation attacks in DeFi applications, researchers have proposed several state-of-the-art (SOTA) approaches, primarily focusing on transaction monitoring-based methods that offer real-time protection, e.g., DeFiRanger~\cite{DeFiRanger23} and DeFort~\cite{DeFort24}.

%Attackers can implement price manipulation attacks through various methods, including flash loans~\cite{qin2021attacking}, pump and dump~\cite{xu2019anatomy}, and sandwich attacks~\cite{zhou2021high}.
%Flash loans in DeFi allow borrowing without collateral, often abused to manipulate cryptocurrency prices and profit from the resultant differentials. Similarly, pump and dump schemes involve inflating cryptocurrency values through deceit, attracting investors before culprits sell off for profit. Additionally, sandwich attacks exploit transaction sequences to manipulate prices and gain financially. The latter two methods are not in the scope of discussion in this paper, \fixme{since pump-and-dump schemes involve social engineering, and sandwich attacks exploit the order of multiple transactions (at least three transactions) in a sequence. Moreover, as an increasing number of applications implement slippage protection, executing sandwich attacks has become increasingly challenging.}\dao{Need to mitigate this\juantao{Done.Mark as red}}

%\dao{Need to summarize this paragraph into just a few sentences and put into the above paragraph.\juantao{Done.}}
% Flash loans in DeFi, serving an innovative financial mechanism which enables users to borrow and repay large amounts of assets without collateral in a single transaction, is commonly utilized by attackers to significantly influence the instantaneous price of cryptocurrencies, subsequently allowing the attacks to profit from the price differentials. This is the most prevalent approach of price manipulation, fundamentally attributable to vulnerabilities within the price calculation mechanism.
% Pump and dump approach involves coordinated efforts of malicious users to artificially inflate the price of a cryptocurrency through spreading misleading information. After attracting investors to buy in high price, the malicious users sell off their holdings, resulting in a drastic price collapse and significant losses for investors. Sandwich attack is a trading strategy where an attacker exploits the price impact of a pending transaction by inserting their own transactions before and after it, which artificially inflates and then deflates asset prices, allowing the attacker to make profit.

However, our analysis indicates that these SOTA approaches are less effective to non-standard price models in custom DeFi protocols.
This is because they typically require token exchange rates to identify abnormal price changes, which are only suitable to calculate under standard price models such as CPMM (Constant Product Market Makers) and Stableswap Invariant (detailed in \mysec\ref{sec:backg_priceModel}).
Unfortunately, our subsequent evaluation shows that 44.2\% of the 95 DeFi price manipulation attacks reported in the past three years used non-standard price models.
% 42 / 95 = 44.2%
Hence, \textit{instead of explicitly calculating the exchange rate for a pair of tokens, we aim to capture only the abnormal price fluctuations of tokens}, which can be directly derived from the high-level price model and changes in token balances.

To achieve this objective, we sought help from large language models (LLMs), considering that their trained intelligence might aid in inferring price changes associated with DeFi operations during the transaction process.
As such, we introduce \tool, the first LLM-based approach to detecting DeFi price manipulation attacks in both standard and custom price models.
\tool features several novel designs, such as constructing a transfer graph to recover high-level DeFi operations (\mysec\ref{sec:defiOperation}) and systematically mining price manipulation patterns across all types of DeFi applications (\mysec\ref{sec:finalDetect}).
Among them, the key design is to fine-tune a DeFi price-specific LLM (\mysec\ref{sec:priceInfer}), where we propose (i) simulating transactions using Foundry~\cite{Foundry} to generate our own fine-tuning data, and (ii) conducting a Chain-of-Thought (CoT)-style fine-tuning that integrates both on-chain data and the price context.
The ablation study shows that fine-tuning increases the overall detection effectiveness by up to 31\% compared to the baseline LLM (under \tool) without fine-tuning.
% 31 =(0.8 - 0.61)/0.61
%GPT-3.5-turbo: overall: 97.4% vs 74.4%, for custom price model: 93.3% vs 60%, GPT-40: overall: 96.2% vs 81.8%
%with the raw price change prediction accuracy reaching \fixme{up to 97.4\%}.

To thoroughly evaluate \tool's effectiveness and practicality, we conduct both a benchmark and a large-scale experiment.
In the benchmark experiment, we collected 95 real-world price manipulation attacks from multiple sources and compared \tool with three SOTA tools.
The results show that \tool achieves a superior detection accuracy (recall) of 80\%, significantly higher than 51.6\% with DeFiRanger, 52.6\% with DeFort, and 35.8\% with DeFiTainter~\cite{DeFiTainter23} (a static source code analysis tool).
% 80%=76/95
In a large-scale experiment involving 968 suspicious transactions (a mix of various DeFi attacks, not just price manipulations) from our industry partner and 96,800 benign transactions from DeFort~\cite{DeFort24}, we helped confirm 147 out of 153 price manipulation attacks reported by \tool, yielding a precision of 96\%.
% 96%=147/153
In particular, 81 of these attacks are previously unknown historical incidents.
Moreover, \name achieves zero false alarms on benign transactions, with an average of 2.5-second time overhead across suspicious and benign transactions and merely \$0.0107 per-request LLM inference cost.
% Table 5
% Moreover, \tool successfully detected three latest price manipulation attacks in August 2024.
%Both experiments demonstrate the practicality of applying \tool in real-world transaction monitoring.
%we evaluated \tool's precision. \tool flagged 151 price manipulation attacks in total. To evaluate the false positives, we corroborated these detected attacks against security companies' attack reports or alerts, confirming 38 of them in total, and employed a combination of manual analysis as well as ancillary evidence to judge the remaining transactions without ground truth (107), yielding a precision of 96\%.

In sum, this paper makes the following contributions:
\begin{itemize}[leftmargin=*,noitemsep,topsep=0pt]
%\begin{compactitem}
%\item We propose a novel framework named \tool for automatic on-chain price manipulation attack detection with LLM reasoning. To the best of our knowledge, this is the first study that employs LLMs specifically designed for detecting price manipulation attacks.
\item We introduce the first LLM-based approach, \tool, for automatic on-chain price manipulation attack detection.
In particular, LLMs, when properly fine-tuned, have the capability to abstract price models and infer price changes. 

%\item We design a new approach based on directed graphs to model transfer actions within transactions, aiming to encompass more complex DeFi operations and facilitate the recovery of high-level DeFi operations from low-level transfer actions.
%\item We present eight price change information-directed attack patterns tailored for our framework integrating the trend of token price changes and high-level DeFi operations.
\item To support \tool's detection, we propose a graph-based method to recover high-level DeFi operations and systematically mine eight price manipulation patterns.

\item We extensively evaluate \tool with three real-world transaction datasets, showing \tool's superior performance over SOTA tools in terms of precision and recall. 
%\end{compactitem}
\end{itemize}


\noindent
\textbf{Open Science and Ethics.}
To facilitate future research, all source code and experimental results are available in an (anonymous) GitHub repository~\cite{anon_github_repo}.
%We also strictly adhere to the API service policy by OpenAI and considerably initiate essential but minimized requests that will not undermine the API service and cause problems to other regular API users.
For ethical research, we ensure that all attacks elaborated in this work have been published or investigated by authoritative security companies through their established channels, and did not involve any undisclosed zero-day vulnerabilities.

% Our research rigorously addressed various ethical considerations to ensure responsible and ethical conduct throughout the research process. 
% We collaborate with our industry partners to responsibly disclose attack detection results where project developers will be informed and suggested to perform emergency operations and regular users will be alerted to protect their own asset. 
% Particularly, we ensure that all attacks elaborated in this work have been officially published and thoroughly investigated by authoritative security companies through their established channels, and did not involve any undisclosed zero-day vulnerabilities.
% We respect the API service policy by OpenAI and considerably initiate essential but minimized requests that will not undermine the API service and cause problems to other regular API users.
% Despite the advantage of price manipulation attack detection, \tool may be misused for front-run attacks where attackers could deploy \tool to analyze mempool transactions and then imitate the price manipulation-related transactions to make profit via transaction reordering.
% However, such scenarios can be largely avoided as many high-profile transactions are submitted through MEV bots that can defend front-run attacks. 