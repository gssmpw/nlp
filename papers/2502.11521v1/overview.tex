\begin{figure*}[t]
    \centering
    \includegraphics[width=.9\linewidth]{Figure/Workflow_v8.pdf}
    \vspace{-1ex}
    \caption{A high-level overview of \tool.}
    \label{fig:overview}
\end{figure*}

\vspace{-1.5ex}
\section{Overview}
\label{sec:overview}
\vspace{-0.5ex}

Based on the analysis of the motivating example illustrated in \mysec\ref{sec:motivatingexample}, we establish an intuition of inferring price change associated with DeFi operations during the transaction process to detect DeFi price manipulations in various scenarios.
While inferring through standard price models (e.g., CPMM and Stableswap as introduced in \mysec\ref{sec:backg_priceModel}) is straightforward, it is challenging to (i) interpret low-level price calculation Solidity code into high-level price calculation formulas, and (ii) infer the price change from the (custom) price calculation formulas and the information on the token balance changes in related accounts.
% \dao{@Juantao, Is this same as "amount changes in token balances"?\juantao{Unified the expression.}}
%\dao{What is this? Balance change map?\juantao{Yes, change "the change in the values of related variables" to "the change of tokens' balance in related accounts" may be more accurate}}.
Both tasks require certain intelligence.

To address this key challenge, we introduce the first LLM-based approach, \name, for effective DeFi price manipulation detection.
As depicted in \myfig~\ref{fig:overview}, \name consists of ten steps.
In step \ding{172}, \name first decodes and slices raw transaction data.
Then in steps \ding{173} \ding{176}
% \juantao{step 5 should belong to here}, 
\name extracts the code of possible price calculation functions from smart contracts based on their signature, and token balance changes in relevant accounts.
% \ye{not clear of what relevant variables}\juantao{Unified to token balance change}
%\dao{This information is from step 2?\juantao{Yes, "the change in the relevant variables" here means the balance change / total supply change of tokens in relevant accounts, including the contract invoked, the account queried by balanceOf(), the token queried by totalSupply()}}.
Subsequently, in steps \ding{177}, \name embeds these two information into a prompt template, which will be used by the LLM to extract the price model and infer the price change in steps \ding{178} \ding{179}.
%\textit{Third}, we require LLMs to infer the trend of changes in the relative prices of tokens involved in the price model based on the numerical changes of given variables.
In the meantime, in steps \ding{174} \ding{175}, \name constructs the transfer graph (\mysec\ref{sec:transferGraph}) and uses it to recover the high-level DeFi operations (\mysec\ref{sec:searchDeFiOperation}) associated with those price change.
This is because using the trend of token price changes alone is insufficient for detecting price manipulation.
Finally, based on the recovered DeFi operations and their price change information, \name maps them into eight attack patterns listed in \mysec\ref{sec:finalDetect} and detects DeFi manipulation attacks in step \ding{180}.

% Steps \ding{172}-\ding{180} are performed in real time for each transaction and its multiple UserCall.
% Before all these steps, 
We conduct an off-line fine-tuning step (step \textcircled{0})
%\dao{@Juantao, fins the symbol for this one\juantao{No circled-zero provided in "ding", could only use "textcircled" or replace the symbol later}})
to enhance LLMs' capabilities in extracting price calculation models and token price change reasoning.
This is because while the off-the-shelf LLMs may exhibit certain capability in reasoning and code understanding, they are limited for predicting the trend of token price changes given unlabeled code snippet and numerical changes of tokens balance, which will be illustrated in \mysec\ref{sec:evaluation}.
We will detail our fine-tuning technique in \mysec\ref{sec:priceInfer}. 
% and then discuss how to detect DeFi price manipulation attack with high-level operations in \mysec\ref{sec:defiOperation} and effective detection patterns in \mysec\ref{sec:finalDetect}.
%To make LLMs more suited for the tasks they are required to complete, we establish a fine-tuning dataset by simulating transactions of on-chain Constant Product Market Maker(CPMM), and subsequently fine-tune the GPT-3.5 model. More details of this process will be further explained in \mysec\ref{sec:priceInfer}.