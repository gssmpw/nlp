\section{Background and Motivation}
\label{sec:background}


%\dao{Need to talk about transaction, userCall, balance change map, etc. For example, one transaction can contain multiple userCall, something like this \juantao{Done. Have changed userCall to "user invocation", "balance change map" to "Tokens' Balance Change in Relevant Accounts". The explanation of "A transaction could contain multiple userCall" is expressed in "Smart Contract and Invocation", marked as red color}}


% \subsection{Ethereum Accounts and Transactions}
%\label{sec:backg_transactions}

As~\myfig\ref{fig:transaction} shows,
on blockchains, two primary account types exist, i.e., externally owned accounts (EOAs), controlled by individuals using private keys, and contract accounts (CAs), governed by their contract code. 
% Transactions on Ethereum are categorized into external and internal types. 
External transactions are initiated externally by EOAs, while internal transactions occurs when a smart contract calls another contract internally. 
%\subsection{Smart Contract and Invocation}
%There are two types of invocations, i.e., the external and internal invocations.
Smart contracts exchange information with other smart contracts through internal transactions, where
% A smart contract triggers functions in other contracts through external invocation and calls its own functions via internal invocation.
a transaction involving communication between contracts can be regarded as a sequence of function calls, which has been greatly utilized in DeFi protocols for interoperability.
For simplicity, we denote as \emph{user invocation} an internal transaction between the user-controlled smart contracts and other smart contracts, which plays a key role in price manipulation attacks.
% In this paper, we do not highlight the difference between the external and internal transactions when there is no need. \ye{check later}





\subsection{DeFi Protocols}
\label{sec:backg_defiProtocol}

A DeFi protocol typically consists of multiple smart contracts and offers various services including \textit{trading} on decentralized exchange, \textit{lending}, and \textit{yield farming}. % with profits.
%\dao{@Juantao, Corresponding to four types in Table 2.\juantao{Done.Mark as red}}

\noindent\textbf{Decentralized EXchange (DEX).}
A DEX is a peer-to-peer marketplace where users can trade cryptocurrencies. The two predominant forms of DEXs are order book DEXs and automated market makers (AMMs). In the order book DEXs, the management and matching processes of open buy and sell orders take place off-chain while the settlement of trades occurs on-chain. Alternatively, the AMM is fully on-chain, and maintains liquidity through liquidity pools without replying on real-time buyers and sellers. A liquidity pool maintains two or more tokens, and the prices of these tokens are determined by a constant mathematical formula. In this paper, we only target fully on-chain DEXs using AMM~\cite{DEX_Intro_1,DEX_Intro_2}.

\noindent\textbf{Lending.}
To borrow assets from a lending protocol, borrowers are required to deposit other cryptocurrencies as collateral first. To ensure that borrowers can repay their loans, the collateral factor of supplied assets is utilized to calculate the maximum value of assets that users can borrow. For instance, consider a scenario where the collateral factor of ETH is set to 0.75, borrowers can only obtain assets up to 75\% of the value of ETH they have deposited as collateral. If the value of the collateral decreases and falls below a certain level (the health factor of a borrower is less than 1), the liquidation of the collateral (the lending protocol sells the collateral at a discount rate) occurs to repay the borrowed amount~\cite{Lending_Intro_2}.

\noindent\textbf{Yield Farming.}
It provides a way to earn rewards in the form of Annual Percentage Yields (APYs). A yield farmer could deposit cryptocurrencies into yield-farming applications, which invest these assets according to various strategies. Based on the returns from these investments, yield farmers can obtain rewards.
There are two types of yield farming. The first involves users depositing cryptocurrencies into an application to obtain proof tokens, which can be traded, allowing holders to withdraw the corresponding deposited assets. The second, a stake-and-claim model, has users lock their assets into the contract and later claim them back. This process does not generate proof tokens, meaning the staked shares are non-tradable, and only the staking account can make the claim.
%\dao{This corresponds to Stake \& Claim?\juantao{Yield farming comprises two types: deposit and withdraw, stake and claim, explain in red}}
%\dao{@Juantao: What protocol is for Deposit \& Withdraw? Airdrop? Add one more paragraph. \juantao{Explain respectively in yield farming}}


\begin{figure}[!t]
     \centering
     \includegraphics[width=.9\linewidth]{Figure/Transaction_v3.pdf}
     \vspace{-1ex}
     \caption{The blockchain transactions.}
     \label{fig:transaction}
\end{figure}



\subsection{Price Models}
\label{sec:backg_priceModel}

A price model represents the pricing mechanism within the DeFi application, which are typically expressed as equations correlating the price of a certain token with the balances of and the total supply of various tokens as well as other constant. 

% \noindent\textbf{Constant Sum Market Maker (CSMM).}
% CSMM is the simplest form of an AMM. The reserves of two tokens in the liquidity pool adhere to the following relationship:
% \begin{equation}
% \label{CSMM}
% \begin{split}
%     R_x+R_y=k
% \end{split}
% \end{equation}
% where $R_x$, $R_y$ denote to the reserve of $Token_x$ and $Token_y$ respectively. The swap exchange rate of tokens is always equal to 1, thus the prices of two tokens are equal. This price model is hardly employed as an independent mechanism for DEXs since it allows the reserve of one token reduce to zero, thereby potentially draining the entire pool~\cite{gogol2024sok}.
% \ye{From our evaluation in Sect.7, there seems no case for CSMM. @juantao: please have a check. If it is true, we can remove description about CSMM. \juantao{Removed. Yes, no cases use CSMM}}

\noindent\textbf{Constant Product Market Maker (CPMM).}
CPMM is one of the most prevalent DeFi AMMs and used in many well-known DEXs~\cite{Uniswap,PancakeSwap}. It maintains liquidity through a constant function $R_x*R_y=k$,
% as follows:
% \begin{equation}
% \label{CPMM}
% \begin{split}
%     R_x*R_y=k
% \end{split}
% \end{equation}
where $k$ is a constant, and $R_x$ and $R_y$ represent the reserves of two distinct assets $token_x$ and $token_y$ in a liquidity pool. In a swap operation, let $\Delta x$ amount of $token_x$ can exchange for $\Delta y$ amount of $token_y$, resulting in $(x+\Delta x)*(y-\Delta y)=k$ where the instantaneous price of $token_x$ denominated in $token_y$ is $P_{x,y} = \frac{y}{x}$~\cite{DeFiRanger23,gogol2024sok}.


\noindent\textbf{Stableswap Invariant.}
The Stableswap Invariant~\cite{StableSwap}, widely used in Curve AMMs~\cite{Curve}, and is defined as follows:
\begin{align}
\label{Stableswap}
% \begin{split}
    \frac{a\prod_{i=1}^{n} x_i}{(D/n)^{n}} \cdot D^{n-1} \cdot \sum_{i=1}^{n} x_i + \prod_{i=1}^{n} x_i &= \frac{a\prod_{i=1}^{n} x_i}{(D/n)^n} \cdot D^n + \frac{D^n}{n^n} \nonumber
    % \chi &= \frac{A\prod R_i}{(D/n)^n} 
% \end{split}
\end{align}
where $a$ indicates a constant amplification coefficient and n is the number of tokens in the liquidity pool, while $D$ represents the total underlying value, i.e., the total amount of tokens in the pool when token prices are equal, and $x_i$ denotes the current reserve of $token_i$. 
As $a$ approaches zero, the function behaves similarly to a CPMM.

\noindent\textbf{Custom Price Model.}
Besides the aforementioned common price models, DeFi applications can customize their own pricing mechanisms, which are often more diverse and complicated. 
For example, 
% LUSD protocol~\cite{LUSD} offers yield-farming services, enabling users to mint LUSD by depositing other assets, and the price of LUSD denominated in the deposited token is calculated by the following formula:
% \begin{equation}
% \label{LUSD}
% \begin{split}
%     P_{LUSD} = \frac{R_{USDT} * SR_{Token_{deposit}}}{R_{Token_{deposit}}*10^4}
% \end{split}
% \end{equation}
% where $P_LUSD$ refers to the calculated price of LUSD, $R_{token}$ denotes the reserve of token in the liquidity pool, and $SR_{token}$ denotes the supply ratio of the token defined by the app. The price of LUSD is decided by the real-time exchange rate between the deposited asset and USDT. 
%
% Besides of depending on real-time reserves of other assets in a liquidity pool, a DeFi protocol might price a token based on the balance of a specific token held by other contract accounts. For instance, serving as a lending application, Cheese Bank~\cite{Cheesebank} calculates the price of Liquidity Provider (LP) tokens, which is deposited by users as collateral, as follows:
% \begin{equation}
% \label{Cheesebank}
% \begin{split}
%     P_{LP} = \frac{B_{WETH}*TWAP_{WETH}*2}{TS_{LP}}
% \end{split}
% \end{equation}
% where $P_{LP}$ denotes the price of LP calculated by the app, $B_{WETH}$ refers to the balance of WETH in the relevant contract account, $TWAP_{WETH}$ denotes the Time-Weighted Average Price (TWAP)~\cite{UniswapCore} of WETH returned by Uniswap V2, and $TS_{LP}$ refers to the total supply of the LP token. Due to the relative stability of WETH's TWAP over a short time intervals, the price of LP in a transaction is primarily inversely proportional to its total supply and propertional to the WETH holdings of the related contract account.
UwULend~\cite{uwulend_official} offers lending service, permitting users to borrow sUSDe by depositing cryptocurrencies as collateral. 
To ensure that the user can repay the debt, it is necessary to examine whether the total value of sUSDe are below that of the collateral when lending it.
In this process, UwULend calculate the price of sUSDe by the following custom price model, distinct from the aforementioned CPMM and Stableswap Invariant models:
\begin{equation}
\label{PriceOfsUSDe}
\begin{split}
    P_{sUSDe} = median(\{IP_{USDe,Pool_1},\cdots, IP_{USDe,Pool_5}, \\EMAP_{USDe,Pool_1},\cdots, EMAP_{USDe,Pool_5}\})
\end{split}
\end{equation}
where $P_{sUSDe}$ denotes the price of sUSDe, calculated by determining the median of a set of prices relevant to USDe in five liquidity pools\footnote{$Pool_{FRAXUSDe}$~\cite{FRAXUSDe}, $Pool_{USDeUSDC}$~\cite{USDeUSDC}, $Pool_{USDeDAI}$~\cite{USDeDAI}, $Pool_{USDecrvUSD}$~\cite{USDecrvUSD}, and $Pool_{GHOUSDe}$~\cite{GHOUSDe}.}.
$EMAP_{USDe,Pool_i}$ represents the Exponential Moving Average (EMA)~\cite{EMA} price of USDe in the i-th liquidity pool but remains a constant value within a transaction block.
% \fixme{$Pool_i$ is one of the five liquidity pools\footnote{$Pool_{FRAXUSDe}$~\cite{FRAXUSDe}, $Pool_{USDeUSDC}$~\cite{USDeUSDC}, $Pool_{USDeDAI}$~\cite{USDeDAI}, $Pool_{USDecrvUSD}$~\cite{USDecrvUSD} and $Pool_{GHOUSDe}$~\cite{GHOUSDe}} possesses USDe}.
% \ye{@juantao: what $Pool_i$ means\juantao{Update definition, mark as red}}
%Since the influence of price changes resulting from variations in tokens' balance in one block will be mitigated by the EMA, manipulating the EMA price requires a prolonged attack over several blocks, which makes it resistant to manipulation from malicious activity within a transaction.
In contrast, $IP_{USDe,Pool_i}$ refers to the instantaneous price of USDe in the i-th liquidity pool. 
Unlike the EMA, the latter price is more volatile and vulnerable to fluctuations in tokens' balance. % within a transaction.
%\ye{what's the difference between $IP_{USDe,Pool_i}$ and $AP_{USDe,Pool_i}$ \juantao{Updated. Marked as red}}




\subsection{A Motivating Example}
\label{sec:motivatingexample}

\begin{figure}[!t]
    \centering
    \includegraphics[trim=1cm 0cm 1cm 0cm, width=0.65\linewidth]{Figure/UwULend_Attack_Flow_V5.pdf}
    \vspace{-1ex}
    \caption{A simplified attack flow against UwULend.}
    \label{fig:motivationExample}
\end{figure}

% Our approach is motivated by a real-world price manipulation attack on the DeFi protocol named UwULend~\cite{MotivationAnalysis},~\cite{MotivationTrax}.
Our approach is motivated by a real-world price manipulation attack on UwULend~\cite{MotivationAnalysis,MotivationTrax} in 2024. \myfig~\ref{fig:motivationExample} shows a attack flow of the price manipulation towards UwULend. The attack includes six key steps, and the root cause is the flawed price dependency in \cref{PriceOfsUSDe} related to two functions \texttt{borrow} and \texttt{liquidationCall}.
% UwULend has a customized formula \ref{PriceOfsUSDe}, distinct from the Constant Product Market Maker (CPMM) model, to calculate the price of sUSDe.
% \begin{equation}
% \label{PriceOfsUSDe}
% \begin{split}
%     P_{sUSDe} = Median(\{IP_{USDe,Pool_i}\} \cup \{EMAP_{USDe,Pool_i}\})
% \end{split}
% \end{equation}
% $P_{sUSDe}$ indicates the price of sUSDe calculated by determining the median of a set of price relevant to USDe. \fixme{$EMAP_{USDe,Pool_i}$ represents the Exponential Moving Average (EMA)~\cite{EMA} price of USDe calculated by $Pool_i$, and updates once per block. Since the influence of price changes resulting from variations in tokens' balance in one block will be mitigated by the EMA, manipulating the EMA price requires a prolonged attack over several blocks, which makes it resistant to manipulation from malicious activity within a transaction. $IP_{USDe,Pool_i}$ refers to the instant price of USDe, which is the instantaneous exchange rate of USDe and the other token in $Pool_i$. Compared to EMA price, instant price is more volatile and changes immediately with the variation in tokens' balance within a transaction.} $Pool_i$ is one of the five liquidity pools\footnote{$Pool_{FRAXUSDe}$~\cite{FRAXUSDe}, $Pool_{USDeUSDC}$~\cite{USDeUSDC}, $Pool_{USDeDAI}$~\cite{USDeDAI}, $Pool_{USDecrvUSD}$~\cite{USDecrvUSD} and $Pool_{GHOUSDe}$~\cite{GHOUSDe}} possesses USDe.
% \ye{what's the difference between $IP_{USDe,Pool_i}$ and $AP_{USDe,Pool_i}$ \juantao{Updated. Marked as red}}

By swapping a large amount of USDe into five liquidity pools, the attacker could deflate the instant price of USDe while the EMA prices remain unchanged, leading to a lower median price for sUSDe.
Hence, the attack was able to borrow an exceedingly large amount of sUSDe. 
Subsequently, swapping back could increase the calculated price of sUSDe, {resulting in the attacker's collateral being unable to repay the debt}
% the \texttt{healthFactor} of the attacker falling below 1
% \ye{what \texttt{healthFactor} means? no explanation \juantao{Explain in \mysec\ref{sec:backg_defiProtocol} Lending, change to understandable express}}
, thereby allowing liquidation and acquiring deposited collateral with a bonus, leading to a \$19M loss.

While detecting a price manipulation attack, existing tools generally define the exchange rate between two different tokens as their price.
For example, the price of sUSDe could be calculated by dividing the amount of WETH deposited into the protocol at step 3 by the amount of sUSDe borrowed from it at step 4.
Yet, defining abnormal price change range or even precisely capturing token prices is not a trivial task.
% \fixme{To confirm an attack, existing studies primarily rely on identifying abnormal changes in the exchange rate between two tokens:}
% Identifying abnormal changes in the exchange rate between two tokens is crucial for confirming a price manipulation attack, and existing studies primarily achieve this through the following two transaction-based approaches:
% \ye{@juantao: unclear. what do you want to highlight?\juantao{mainly want to highlight "detecting abnormal changes in exchange rate is important", and then introduce the practices of existing methods and their shortage in detecting abnormal changes in exchange rate. Revised, mark as red}}
%\dao{Also DeFiTainter?}

\begin{compactitem}
\item DeFort~\cite{DeFort24} uses historical exchange rates, i.e., prices, between two tokens for computing a so-called normal fluctuation range. 
However, it may be error-prone as historical prices, especially for low-liquidity or newly created pools, could vary significantly, thus being insensitive to the detection of subtle price manipulations. 
For instance, in the UwULend attack, slight manipulations observed in sUSDe prices --- decreasing by 4.2\,\%  or increasing by 4.43\,\%, did not exceed the predefined bounds that evades the detection by DeFort illustrated in our evaluation in \mysec\ref{sec:evaluation}.

\item DeFiRanger~\cite{DeFiRanger23} detects abnormal price changes by tracing token exchange sequences within a transaction and comparing token exchange rates at different steps. 
However, completely tracking these sequences can be challenging for complex transactions. 
In the UwULend attack, the attacker crafted complicated deposit and withdrawal operations concerning WETH, making it difficult for DeFiRanger to detect.
\end{compactitem}

%\begin{itemize}
%\item DeFort~\cite{DeFort24} utilizes historical exchange rates between two tokens and defines the bound of normal exchange rate fluctuation by calculating the variance of historical exchange rates. However, this method is error-prone due to the fact that \fixme{the historical exchange rates (obtained without attack)} of token pairs can exhibit substantial fluctuations especially for tokens in low-liquidity or new-created pools, which makes the calculated bound is not sensitive to attacks with small price fluctuations. For instance, the price of sUSDe is only slightly manipulated (decreases and increases by 4.28\% and 4.43\% in step 2 and step 5 respectively) in the attack towards UwULend, thereby does not exceed the calculated bound, and consequently, it could not be detected by DeFort.
%
%\item DeFiRanger~\cite{DeFiRanger23} identifies abnormal changes in token exchange rates by tracing the chain of token exchanges. It compares the exchange rates of the same pair of tokens at different execution steps within a transaction, thereby detecting any abnormal fluctuations in the exchange rates. However, tracing such an exchange chain is not trivial in many complicated transactions. For example, in the actual transaction of the attack against UwULend, the WETH deposited in step 3 will be firstly exchanged to uWETH serving as credentials. Then the amount of sUSDe allowed to borrow is calculated based on the amount of uWETH held by users, as well as the respective prices of uWETH and sUSDe. During the attack, the attacker executed several deposit and withdrawal operations to adjust the amount of WETH deposited in the protocol. This strategy complicated the process of accurately tracing the total amount of WETH deposited, thus DeFiRanger fails to detect the attack.
%\end{itemize}

To address the aforementioned issues about price identification, 
% Existing methods, which require at least two exchange rates for the same token pair and detect anomalies by comparison, are error-prone and do not cover all scenarios. 
our observation is that abnormal price fluctuations of tokens could be directly derived from the high-level price model and amount changes in token balances, without needing to explicitly calculate the exchange rate for a pair of tokens. 
Taking the UwULend attack as an example, swapping USDe into liquidity pools in step 2 decreases the value of $IP_{USDe,Pool_i}$, causing an abnormal drop in the median price, namely the price of sUSDe. 
% This allows for real-time tracking of sUSDe price changes during the transaction process.
Based on this observation, in this paper, we propose a novel price change reasoning approach powered by LLM and integrate with predefined rules to enhance the capability of price manipulation attack detection.  
%Existing methods require at least two exchange rates for the same pair of tokens, and detect anomalous fluctuations through comparison, which makes them more prone to errors and unable to cover all cases.
%Instead, \textit{the abnormal fluctuations in the price of tokens, we believe, can be directly obtained based on the price calculation model, e.g. formula~\ref{PriceOfsUSDe}, and \fixme{the information on tokens' balance changes}, without calculating the exchange rate of a pair of tokens}.
%For example, swapping USDe into liquidity pools in step 2 will result in an decrease in the value of $IP_{USDe,Pool_i}$ within the price calculation model, ultimately leading to an abnormal decrease in the median of the price set and the price of sUSDe. Following this analysis, we can obtain real-time change in the price of sUSDe during the transaction process.



%\juantao{TODO}
%put an example (an exmaple that doesn't comply with using instant exchange rate to define the price of tokens)
%(1) to illustrate the weakness of previous detection works to identify (2) research challenges, and
%(3) illuminate the novelty point of our approach