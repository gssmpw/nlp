\section{DeFi Operations}
\label{sec:defiOperation} 

The standalone fluctuations in token prices are meaningless; they need to be considered within the DeFi context to serve as evidence for detecting price manipulation.
However, the raw transactions obtained from the blockchain consist solely of low-level information, such as token transfer actions and smart contract invocations.
There exists a gap between raw transactions and high-level DeFi semantics.

To bridge this gap, we first model token transfer actions using a directed graph (\mysec\ref{sec:transferGraph}), and then recover high-level DeFi operations from it (\mysec\ref{sec:searchDeFiOperation}).
Since our detection is based on a single transaction, it should be noted that all described operations are derived from one raw transaction, and we do not consider the DeFi operations expressed by the combination of multiple raw transactions.
Based on our study of the top-10 high-value DeFi applications across all three categories mentioned in \mysec\ref{sec:backg_defiProtocol} (the full list is available in Appendix~\ref{sec:high_value_apps}) with active transactions in each category, due to the susceptibility to front-running across multiple transactions and the atomicity of transactions ensuring complete execution of operations, only a very few DeFi operations span multiple transactions.



\subsection{Transfer Graph Construction}
\label{sec:transferGraph}

We define the Transfer Graph (TG) (\mydef~\ref{def:TG}), a directed graph where the edges represent transfer actions (\mydef~\ref{def:transfer}) and the vertices represent related accounts, to model transfer actions within each user invocation.
\begin{definition}[Transfer]
\label{def:transfer}
    A transfer $T := \langle s, r, t, v \rangle$, if performed successfully, deducts amount $v \in \mathbb{N}$ of token $t \in Addr$ from the sender's account $s \in Addr$ and the balance of token $t$ in the receiver's account $r \in Addr$ increases by $v$.
\end{definition}
\begin{definition}[Transfer Graph]
\label{def:TG}
    A \textit{Transfer Graph} (\textit{TG}) is a tuple ($\mathcal{A}$,$\mathcal{E}$), 
    where $\mathcal{A}$ is the set of all accounts (including EOAs, CAs and $\emptyset$) involved in a user invocation, 
    $\mathcal{E}$ is the set of directed edges, i.e., $\mathcal{E} = \{E_1,...,E_m\}\subseteq\mathcal{A}\times\mathcal{A}$, where each $E_i := \langle j, T_k \rangle$, j is the time index of $T_k$, $T_k \in \mathcal{T}$, 
    $\mathcal{T}$ is the set of all transfer actions involved in the user invocation, i.e., $\mathcal{T} = \{T_1,...,T_n\}$, where $T_i.s, T_i.r \in \mathcal{A}$ for each $T_i$.
\end{definition}
% \begin{definition}[Transfer Graph]
% \label{def:TG}
%     Let $\mathcal{A}$ be the set of all accounts (including EOAs, CAs and $\emptyset$) involved in a user invocation, $\mathcal{T}$ be the set of all transfer actions involved in the user invocation, i.e., $\mathcal{T} = \{T_1,...,T_n\}$, where $T_i.s, T_i.r \in \mathcal{A}$ for each $T_i$, $\mathcal{E}$ be the set of directed edges, i.e., $\mathcal{E} = \{E_1,...,E_m\}\subseteq\mathcal{A}\times\mathcal{A}$, where each $E_i := \langle j, T_k \rangle$, j is the time index of $T_k$, $T_k \in \mathcal{T}$.
    
%     A \textit{Transfer Graph} (\textit{TG}) is the tuple ($\mathcal{A}$,$\mathcal{E}$).
% \end{definition}
% \noindent
% where an edge (denoted as $E$) includes two attributes: the \textit{transfer action} and the \textit{time index} of this transfer action in the graph to represent the sequential order of occurrences.
% Meanwhile, a vertex (denoted as $V$) in TG contains the attribute of the \textit{account address}.

According to our categorization, a transfer action can be one of three types: \textit{transferring token}, \textit{burning token}, and \textit{minting token}.
All transfer actions can be expressed as ``\textit{Sender} transfers \textit{amount} of \textit{token} to \textit{Receiver}.''
In a \textit{transferring token} action, all accounts involved must be either EOAs or CAs, and must not be a zero address\footnote{0x0000000000000000000000000000000000000000} or a dead address\footnote{0x000000000000000000000000000000000000dEaD} (we uniformly denote these two special addresses by $\emptyset$).
Meanwhile, the \textit{receiver} in a \textit{burning token} action and the \textit{sender} in a \textit{minting token} action must be $\emptyset$.

% A transfer action can be one of three types: \textit{transferring token}, \textit{burning token}, and \textit{minting token}.
% All transfer actions can be expressed as ``\textit{Sender} transfers \textit{amount} of \textit{token} to \textit{Receiver}.''
% In a \textit{transferring token} action, all accounts involved must be either EOAs or CAs, and must not be a zero address or a dead address.
% Meanwhile, the \textit{receiver} in a \textit{burning token} action and the \textit{sender} in a \textit{minting token} action must be either a zero or a dead address.

\myfig~\ref{fig:TG}, as demonstrated in step \ding{172}, illustrates the construction of a TG from the raw transaction of a user invocation. This user invocation includes six contract accounts and a collection of user-controlled accounts $UC$, which includes EOAs and CAs, along with seven \textit{transferring token} actions.
The \textit{Sender} and \textit{Receiver} of a transfer action are connected by a directed edge, from the \textit{Sender} to the \textit{Receiver}, with a time index to indicate the order of occurrence.
$T_1$, from one of the user-controlled accounts in $UC$ to $CA_1$, is the first transfer action in this user invocation.
Before $CA_1$ transfers tokens to $CA_2$ through $T_4$, user-controlled accounts initiate two transfers to $CA_4$ and $CA_5$ respectively, resulting in $T_4$ having a larger time index compared to $T_2$ and $T_3$.
Similarly, since $T_6$ occurs between $T_5$ (with a time index of 5) and $T_7$ (with a time index of 7), its time index is set to 6.
%\dao{@Juantao, this paragraph not finished? Explain a little bit more.\juantao{I think there are not too many details we could talk in constructing the TG, since it is quite straight-forward, I add some new details about time index, mark in red}}

Compared to existing methods that model transfer actions in raw transactions utilized {for price manipulation detection}, such as the Cash Flow Tree (CFT) constructed in DeFiRanger~\cite{DeFiRanger23}, the novelty of the Transfer Graph (TG) lies in its unique modeling approach.
The CFT models invocation and transfer actions within raw transactions, where the edges represent invocation relationships and the vertices represent a called function or a transfer action.
In contrast, the TG models transfer actions in each user invocation within raw transactions, thereby offering greater granularity.
Furthermore, the TG employs a directed graph rather than a tree structure for modeling and retains temporal relationships between different transfer actions, which enables it to cover more complicated DeFi operations and accelerate the speed. % of recovery.
{The proposed transfer graph are different from those fund flow graphs in prior works, such as ~\cite{mclaughlin2023large}.
TG differs from the existing works in terms of granularity and design purpose.
First, their directed graphs are built on transactions, while TG is defined on user invocations where a transaction can have multiple user invocations. 
Second, their directed graphs are to analyze transaction-related fund flow behaviors like arbitrage. 
In contrast, TG focuses on extracting high-level DeFi operations with the fund flow among user invocations.}
% \juantao{Compared with the work mentioned by review C}

%\dao{@Juantao, is "Balance Change Map" a common term? "Map"?\juantao{Changed to "Transfer Actions"}}

%\begin{table}
%    \centering
%    \caption{The definition of Transfer Graph}
%    \begin{tabular}{ccc}
%    \toprule
%        Name & Symbol & Attributes \\
%        \midrule
%        Edge & $E$ & \textit{time}, \textit{transfer action} \\
%        Vertex & $V$ & \textit{account address} \\
%    \bottomrule
%    \end{tabular}
%    \label{tab:transfer_graph}
%\end{table}

\begin{figure}[!t]
    \centering
    \includegraphics[width=1\linewidth]{Figure/TG_v6.pdf}
    %\vspace{-1ex}
    \caption{An illustrative example showing the workflow of recovering DeFi operations. $UC$: a collection of user-controlled accounts, including EOAs and CAs; $CA_i$: the $i$-th contract account; $T_i$: the $i$-th \textit{transferring token} action.}
    \label{fig:TG}
\end{figure}



\subsection{DeFi Operation Inference}
\label{sec:searchDeFiOperation}

Based on our in-depth analysis of the top-30 DeFi apps shown in Table~\ref{tab:high_value} (see Appendix~\ref{sec:high_value_apps}), we identify six DeFi operations that need to be recovered:
1) \textit{Swap}, primarily from DEXs such as Uniswap; 
2) \textit{Deposit}, defined by yield-farming and lending apps such as AAVE and Pendle; 
3) \textit{Withdraw}, mainly from yield-farming apps; 
4) \textit{Borrow}, based on lending protocols like Compound; 
5) \textit{Stake}; and 6) \textit{Claim}, both predominantly from yield-farming apps like Convex Finance.
%\dao{Explain Table 7 a little bit more, such as top-XXX across three types of DeFi applications, and briefly state how do you identify these six operations? For example, XXX category like UniSwap is mainly YYY operation...}


\textbf{Swap} involves a user exchanging one token ($Token_{in}$) for another ($Token_{out}$) from a liquidity pool. Relevant contract accounts, excluding user-controlled ones, are noted as liquidity pools for price trend analysis when source code is unavailable.
In a \textbf{Deposit}, a user transfers a token ($Token_{deposit}$) to a yield-farming or lending protocol and receives a proof token ($Token_{proof}$) through minting. %, as shown in \myfig\ref{fig:motivationExample}.
\textbf{Withdraw} occurs when a user retrieves tokens ($Token_{withdraw}$) from a protocol by burning a proof token ($Token_{proof}$).
\textbf{Borrow} refers to a DeFi operation where a borrower receives a token ($Token_{borrow}$) and incurs a debt evidenced by another token ($Token_{debt}$), issued through minting.
\textbf{Stake} happens in yield-farming protocols like Convex Finance~\cite{convex_fi}, where users deposit tokens and can earn rewards without receiving a minted proof token.
\textbf{Claim} enables users to retrieve staked tokens ($Token_{claim}$) and bonuses without burning any tokens, unlike the \textit{Withdraw} operation which requires burning tokens to retrieve assets.


%\noindent\textbf{Swap.} \textit{Swap} refers to an operation that a user transfers one kind of token ($Token_{in}$) to and then gets another token ($Token_{out}$) back from a liquidity pool. The contract accounts (except the user-controlled accounts) involved in the \textit{Swap} operations will be recorded as liquidity pools, which will be used in inferring trend of tokens' price change if the source code is not available.
%
%\noindent\textbf{Deposit.} A user can transfer a type of token ($Token_{deposit}$) into the yield-farming protocol or a lending protocol, and then acquire a difference type of token ($Token_{proof}$) as a proof of deposit through minting. We use \textit{Deposit} to denote this kind of operation. For instance, in the motivating example (as shown in \myfig\ref{fig:motivationExample}), the attacker deposited WETH to UwULend in step 3, and obtained minted uWETH as a proof.
%
%\noindent\textbf{Withdraw.} We use \textit{Withdraw} to define the operation that a user obtains tokens ($Token_{withdraw}$) from a yield-farming protocol after burning another type of token ($Token_{proof}$).
%
%\noindent\textbf{Borrow.} \textit{Borrow} represents an DeFi operation that a borrower get one kind of token ($Token_{borrow}$) as borrowed asset from a lending protocol while also obtaining another type of token ($Token_{debt}$) as debt through minting. For example, as demonstrated in step 4 of \myfig\ref{fig:motivationExample}, the attacker borrowed sUSDe from UwULend protocol, and received minted variableDebtSUSDE as debt.
%
%\noindent\textbf{Stake.} In some yield-farming protocol, such as Convex Finance~\cite{convex_fi}, allows user to stake tokens and record the amount of tokens in state variables, and user could get reward from the application later. The main difference between \textit{Stake} and \textit{Deposit} is that there is no proof of staking mint to the user. We use \textit{Stake} to refer to the operation that users only transfer tokens ($Token_{stake}$) to a protocol without getting back any tokens.
%
%\noindent\textbf{Claim.} For DeFi protocols that support stake operations, they also allow users to get back their staked tokens ($Token_{claim}$) along with bonuses. We define this operation as \textit{Claim}. Compared to \textit{Withdraw} operation, \textit{Claim} does not require the burning of a specific token to retrieve tokens previously transferred to the application; instead, it directly obtains the tokens from the app.


\noindent\textbf{An Operation Recovering Example.}
To recover DeFi operations from the TG, we design and employ a search algorithm based on directed graphs for each operation. Since the recovery algorithm of DeFi operations is not the central focus of this work, we use the recovery of the Swap operation as an illustrative example here.

\myfig~\ref{fig:TG} illustrates that, to recover a \textit{Swap} operation, we use the depth-first search algorithm to identify cycles that both start and end at user-controlled accounts.
These cycles must satisfy three constraints:
(i) the transfer action of each edge must be \textit{transferring token};
(ii) the time index of each edge must be monotonically increasing;
and (iii) the token involved in the first transfer action must differ from the token in the last transfer action.
Therefore, from the well-constructed TG, we can recover the \textit{Swap} operation expressed as $UC \rightarrow T_1 \rightarrow CA_1 \rightarrow T_4 \rightarrow CA_2 \rightarrow T_5 \rightarrow CA_3 \rightarrow T_7 \rightarrow UC$, and we label $CA_1$, $CA_2$, and $CA_3$ as liquidity pools.