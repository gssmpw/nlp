\section*{Appendix}


\definecolor{verylightgray}{rgb}{.97,.97,.97}

\lstdefinelanguage{Solidity}{
	keywords=[1]{anonymous, assembly, assert, balance, break, call, callcode, case, catch, class, constant, continue, constructor, contract, debugger, default, delegatecall, delete, do, else, emit, event, experimental, export, external, false, finally, for, function, gas, if, implements, import, in, indexed, instanceof, interface, internal, is, length, library, log0, log1, log2, log3, log4, memory, modifier, new, payable, pragma, private, protected, public, pure, push, require, return, returns, revert, selfdestruct, send, solidity, storage, struct, suicide, super, switch, then, this, throw, transfer, true, try, typeof, using, value, view, while, with, addmod, ecrecover, keccak256, mulmod, ripemd160, sha256, sha3}, %
	keywordstyle=[1]\color{blue}\bfseries,
	keywords=[2]{address, bool, byte, bytes, bytes1, bytes2, bytes3, bytes4, bytes5, bytes6, bytes7, bytes8, bytes9, bytes10, bytes11, bytes12, bytes13, bytes14, bytes15, bytes16, bytes17, bytes18, bytes19, bytes20, bytes21, bytes22, bytes23, bytes24, bytes25, bytes26, bytes27, bytes28, bytes29, bytes30, bytes31, bytes32, enum, int, int8, int16, int24, int32, int40, int48, int56, int64, int72, int80, int88, int96, int104, int112, int120, int128, int136, int144, int152, int160, int168, int176, int184, int192, int200, int208, int216, int224, int232, int240, int248, int256, mapping, string, uint, uint8, uint16, uint24, uint32, uint40, uint48, uint56, uint64, uint72, uint80, uint88, uint96, uint104, uint112, uint120, uint128, uint136, uint144, uint152, uint160, uint168, uint176, uint184, uint192, uint200, uint208, uint216, uint224, uint232, uint240, uint248, uint256, var, void, ether, finney, szabo, wei, days, hours, minutes, seconds, weeks, years},	%
	keywordstyle=[2]\color{teal}\bfseries,
	keywords=[3]{block, blockhash, coinbase, difficulty, gaslimit, number, timestamp, msg, data, gas, sender, sig, value, now, tx, gasprice, origin},	%
	keywordstyle=[3]\color{violet}\bfseries,
	identifierstyle=\color{black},
	sensitive=true,
	comment=[l]{//},
	morecomment=[s]{/*}{*/},
	commentstyle=\color{gray}\ttfamily,
	stringstyle=\color{red}\ttfamily,
	morestring=[b]',
	morestring=[b]"
}

\lstset{
	language=Solidity,
	backgroundcolor=\color{verylightgray},
	extendedchars=true,
	basicstyle=\footnotesize\ttfamily,
	showstringspaces=false,
	showspaces=false,
	numbers=none,
	numberstyle=\footnotesize,
	numbersep=9pt,
	tabsize=2,
	breaklines=true,
	showtabs=false,
	captionpos=b
}

% \section{Complete Code Used in Fine-Tuning}\label{complete_code_in_ft}
% \mylist\ref{lst:swapExactTokensForTokens} shows the complete code we use in fine-tuning process. The entry function is \texttt{swapExactTokensForTokens}, and its three sub-calls are \texttt{getAmountsOut}, \texttt{getAmountOut}, \texttt{\_swap}.

% \begin{lstlisting}[language=Solidity, caption= Function \texttt{swapExactTokensForTokens} and its sub-calls in UniswapV2Router02 contract, label=lst:swapExactTokensForTokens]
% function swapExactTokensForTokens(
%     uint amountIn,
%     uint amountOutMin,
%     address[] calldata path,
%     address to,
%     uint deadline
% ) external virtual override ensure(deadline) returns (uint[] memory amounts) {
%     amounts = UniswapV2Library.getAmountsOut(factory, amountIn, path);
%     require(amounts[amounts.length - 1] >= amountOutMin, 'UniswapV2Router: INSUFFICIENT_OUTPUT_AMOUNT');
%     TransferHelper.safeTransferFrom(
%         path[0], msg.sender, UniswapV2Library.pairFor(factory, path[0], path[1]), amounts[0]
%     );
%     _swap(amounts, path, to);
% }
% function getAmountsOut(address factory, uint amountIn, address[] memory path) internal view returns (uint[] memory amounts) {
%     require(path.length >= 2, 'UniswapV2Library: INVALID_PATH');
%     amounts = new uint[](path.length);
%     amounts[0] = amountIn;
%     for (uint i; i < path.length - 1; i++) {
%         (uint reserveIn, uint reserveOut) = getReserves(factory, path[i], path[i + 1]);
%         amounts[i + 1] = getAmountOut(amounts[i], reserveIn, reserveOut);
%     }
% }
% function getAmountOut(uint amountIn, uint reserveIn, uint reserveOut) internal pure returns (uint amountOut) {
%     require(amountIn > 0, 'UniswapV2Library: INSUFFICIENT_INPUT_AMOUNT');
%     require(reserveIn > 0 && reserveOut > 0, 'UniswapV2Library: INSUFFICIENT_LIQUIDITY');
%     uint amountInWithFee = amountIn.mul(997);
%     uint numerator = amountInWithFee.mul(reserveOut);
%     uint denominator = reserveIn.mul(1000).add(amountInWithFee);
%     amountOut = numerator / denominator;
% }
% function _swap(uint[] memory amounts, address[] memory path, address _to) internal virtual {
%     for (uint i; i < path.length - 1; i++) {
%         (address input, address output) = (path[i], path[i + 1]);
%         (address token0,) = UniswapV2Library.sortTokens(input, output);
%         uint amountOut = amounts[i + 1];
%         (uint amount0Out, uint amount1Out) = input == token0 ? (uint(0), amountOut) : (amountOut, uint(0));
%         address to = i < path.length - 2 ? UniswapV2Library.pairFor(factory, output, path[i + 2]) : _to;
%         IUniswapV2Pair(UniswapV2Library.pairFor(factory, input, output)).swap(
%             amount0Out, amount1Out, to, new bytes(0)
%         );
%     }
% }
% \end{lstlisting}
% \section{PancakeSwap version of \texttt{getAmountOut}}
% \mylist\ref{lst:pancakeswap_getAmountOut} demonstrates the \texttt{getAmountOut} function implemented by PancakeSwap~\cite{PancakeSwap} in PancakeRouter contract~\cite{PancakeRouter}.

% \begin{lstlisting}[language=Solidity, caption= Function \texttt{getAmountOut} in PancakeRouter contract, label=lst:pancakeswap_getAmountOut]
% function getAmountOut(uint amountIn, uint reserveIn, uint reserveOut) internal pure returns (uint amountOut) {
%         require(amountIn > 0, 'PancakeLibrary: INSUFFICIENT_INPUT_AMOUNT');
%         require(reserveIn > 0 && reserveOut > 0, 'PancakeLibrary: INSUFFICIENT_LIQUIDITY');
%         uint amountInWithFee = amountIn.mul(9975);
%         uint numerator = amountInWithFee.mul(reserveOut);
%         uint denominator = reserveIn.mul(10000).add(amountInWithFee);
%         amountOut = numerator / denominator;
% }
% \end{lstlisting}
\begin{table}[t]
    \centering
    \small
    \caption{Details of six false positives mentioned in \mysec\ref{sec:RQ3}.}
    % \vspace{-2ex}
    \begin{tabular}{c|c|c|c|c}
    \toprule
        \makecell[c]{Transaction\\hash} & Chain & Block & Type & \makecell[c]{Root\\Cause} \\
        \midrule
        0x130c6370 & BSC & 38218540 & Benign & -\\
        0x4b59af93 & BSC & 38218538 & Benign & -\\
        0xe158a2b9 & BSC & 38218537 & Benign & -\\
        0x59942848 & BSC & 38218536 & Benign & -\\
        0x2e9ceb16 & BSC & 38218539 & Benign & -\\
        0x640ce34c & BSC & 11403670 & Malicious & \makecell[c]{Logic\\issue}\\
        \bottomrule
    \end{tabular}
    \label{tab:false_positives}
\end{table}

\section{A Case Study of Type-II Prompt}
\label{sec:case_study_type_II_prompt}

\myfig\ref{fig:type-II_case_study} illustrates an use case of Type-II prompt. 
The contract (0x2120...3379) is closed source DEX contract that allows users to trade SVT and USDT. 
While recovering the DeFi operations, this contract was identified as a two-token liquidity pool. So, we applied the Type-II prompt to inferring the token price trend in its transactions. 
As shown by the LLM response in \myfig\ref{fig:type-II_case_study}, the primary difference from the response to a Type-I prompt lies in the absence of the analysis on the price model; instead, the scoring is directly yielded. 
The is because the prompt itself assumes that the CPMM is employed in this contract. 
Regarding the CPMM, the relationship between token price and token balance is quite standardized, i.e., the direction of balance change is opposite to that of price change.

\begin{figure}[!t]
    \centering
    \includegraphics[width=1\linewidth]{Figure/type-II_case_study.pdf}
    \caption{An example of Type-II prompt and its response generated while detecting the attack against SVT~\cite{SVT}.}
    \label{fig:type-II_case_study}
\end{figure}



%\section{Complete Prompt for Detecting UwULend Incident}\label{complete_prom_resp}
%\myfig\ref{fig:example_prompt} shows the prompt used to query the fine-tuned LLM for inferring trend of sUSDe's price change in contract account named ``LendingPool'' after manipulating the balance of USDe in 5 liquidity pools through swapping in USDe and swapping out other assets.
%\begin{figure*}[!t]
%    \centering
%    \includegraphics[width=0.75\linewidth]{Figure/UwULend prompt(full page).pdf}
%    \caption{An example prompt for querying the fine-tuned LLM to infer trend of sUSDe's price change after manipulating tokens' balance in relevant contract accounts}
%    \label{fig:example_prompt}
%\end{figure*}
% \begin{figure*}[!t]
%     \centering
%     \includegraphics[width=0.75\linewidth]{Figure/UwULend response.pdf}
%     \caption{An example response of inferring trend of sUSDe's price change after manipulating its price}
%     \label{fig:example_response}
% \end{figure*}


\begin{table}[t]
    \centering
    \small
    \caption{High-value DeFi applications (from DefiLlama~\cite{defillama}) with active transactions of each category we have done a comprehensive study on. TVL refers to the Total Value Locked in the app. 
    % \dao{@Juantao, make this table smaller.}\juantao{Adjusted.}
    }
    \resizebox{\columnwidth}{!}{
    \begin{tabular}{cc|cc|cc}
    \toprule
         DEX App&  TVL & Lending App &  TVL & Yield farming App &  TVL \\
         \midrule
         Uniswap &  \$4.8B & Compound &  \$2.0B& Pendle &  \$2.8B\\
         Curve &  \$1.9B & AAVE &  \$12.3B & Convex Finance &  \$1.1B\\
         Balancer &  \$777.2M & Morpho &  \$1.5B & Aura &  \$361.9M\\
         Sushi &  \$250.2M & Fraxlend &  \$134.4M & Magpie &  \$192.5M\\
         PancakeSwap &  \$1.7B & Venus &  \$1.5B & StakeDAO &  \$78.8M\\
         1inch &  \$4.58M & Strike &  \$9.7M & Equilibria Finance &  \$80.0M\\
         ParaSwap &  \$6.34M & Planet &  \$1.2M & Kine Finance &  \$8.0M\\
         ShibaSwap &  \$18.19M & Kinza Finance &  \$42.8M & Dot Dot Finance &  \$2.1M\\
         BiSwap &  \$27.1M & Radiant &  \$7.8M & Solo Top &  \$1.8M\\
         MDEX &  \$16.0M & Ambit &  \$6M & Jetfuel Finance &  \$1.5M\\
    \bottomrule
    \end{tabular}
    }
    \label{tab:high_value}
\end{table}

\section{High-value DeFi Applications}
\label{sec:high_value_apps}

The design of DeFi operations and the category of transfer actions in \mysec\ref{sec:defiOperation} is based on an in-depth study of the high-value DeFi applications listed in \mytab\ref{tab:high_value} as of August 2024.

\section{Details of False Positives}
\label{sec:fp_details}
\mytab\ref{tab:false_positives} presents the details of six false positives discovered in our experiments described in \mysec\ref{sec:RQ3}.
% \dao{Move to appendix}
The reason for these false positives is that their transactions involve two contract accounts that were created three months ago by the transaction initiator. 
Any fund transfers among these accounts and the initiator should be considered benign operations rather than price manipulation operations. 
{Yet, these accounts were incorrectly marked as closed-source DEXes in the detection, leading to false inferences.} 
Such false positives could be mitigated by conducting a historical analysis of account ownership relationships and clustering user accounts that are controlled by the same owners.


% \dao{Add some description.} \juantao{Done. Mark as red}

% \begin{lstlisting}[language=Solidity, caption= Vulnerable \texttt{convertInternalSwap()} exploited in the attack against NovaX., label=lst:convertInternalSwap]
% function convertInternalSwap (uint256 _value, bool toToken) public view returns (uint256) {
%     uint256 usdtAmount = InternalSwap(swapAddress).getUsdtAmount();
%     uint256 tokenAmount = InternalSwap(swapAddress).getTokenAmount();
%     if (tokenAmount > 0 && usdtAmount > 0) {
%         uint256 amountTokenDecimal;
%         if (toToken) {
%             amountTokenDecimal = (_value * tokenAmount) / usdtAmount;
%         } else {
%             amountTokenDecimal = (_value * usdtAmount) / tokenAmount;
%         }

%         return amountTokenDecimal;
%     }
%     return 0;
% }
% \end{lstlisting}

% \begin{lstlisting}[language=Solidity, caption= Vulnerable \texttt{\_transfer()} exploited in the attack against iVest., label=lst:transfer]
% function _transfer  (address from, address to, uint256 amount) internal override  {
%     ...//Omitted
%     //Transfers directly to the burn address will be considered a Burn donation
%     if (to == address(0)){
%         __MakeDonation(from,amount,3);
%     }
%     ...//Omitted
%     _transferFromLP(from,to,amount,isWhale);
% }
% function __MakeDonation(address donor, uint256 tAmount,uint8 mode)private{ 
%     ...//Omitted
%     if (mode==3){_transferBURNDonation(donor, tAmount);donationType="Burn Donation";}
%     ...//Omitted
% }
% \end{lstlisting}

% \begin{lstlisting}[language=Solidity, caption= Vulnerable \texttt{getPrice()} exploited in the attack against HFLH., label=lst:getPrice]
% function getPrice() public view returns (uint256) {
%     uint256 usdtBalance = IERC20(usdtAddress).balanceOf(lpAddress);
%     uint256 HFLHBalance = IERC20(HFLHAddress).balanceOf(lpAddress);
    
%     require(usdtBalance > 0, "USDT balance is zero");

%     uint256 price = 1e18*HFLHBalance/usdtBalance ;
%     return price;
% }
% \end{lstlisting}

% \section{Real-World Attacks}
% \label{sec:case_studies}
% % \dao{Why "two" now?}\juantao{add ``Other''}
% In \mysec\ref{sec:RQ3}, we have conducted a detailed analysis of the attack on NovaX. Below, we provide in-depth analyses of the other two attacks, i.e., iVest and HFLH.
% % \dao{Refer to the main content where mention this appendix.}\juantao{Done. Mark as red}

% \subsection{The iVest Attack}\label{case_study_ivest}
% As shown in \mylist\ref{lst:transfer}, \texttt{\_transfer} is invoked while the liquidity transferring tokens. By setting the receiver to zero address, the condition on the fourth line held, resulting in the invocation of \texttt{\_\_MakeDonation} to burn iVest from the pool. Subsequently, in \texttt{\_transferFromLP}, the pool burned an identical quantity of iVest repeatedly, leading to a double burning. Consequently, the balance of iVest in the pool was substantially lower than the expected value, leading a great price push. To make profit, the attacker initially exchanged WBNB to iVest, and then repeated the operation of transferring iVest to the pool followed by skimming to the zero address for four times. 
% Finally, through transferring a small quantity of iVest to the pool, the attacker drained almost all WBNB, and gained \$172K. \tool detected this attack with \textit{Pattern I}.

% \subsection{The HFLH Attack}\label{case_study_hflh}
% In this attack, attackers transferred a large amount of HFLH to the liquidity pool to cause a liquidity imbalance, which led to a dramatically inflated price of WBNB calculated through the function \texttt{getPrice} shown in \mylist\ref{lst:getPrice}. 
% To exploit the manipulated price, the attacker invoked the function \texttt{borrow} to borrow HFLH with a small amount of WBNB, and then made profits by returning WBNB. 
% This attack is detectable using \textit{Pattern II}. 

% \section{Vulnerable Code Exploited in Attacks}
% \label{sec:vulerable_code}

% % \dao{Add a few text description, such as pointer.}
% The vulnerable code of three detected latest attacks (NovaX, iVest, HFLH) mentioned in \mysec\ref{sec:RQ3} are listed as the following.
