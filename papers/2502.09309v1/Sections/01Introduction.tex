\section{Introduction} \label{Introduction}
% Reset Introduction
    \IEEEPARstart{R}{eset} elements are nonlinear filters used to overcome the fundamental performance limitations of linear time-invariant (LTI) control systems \cite{zhao2019overcoming}.
 The increasing demand for extremely fast and accurate performance in fields such as precision motion control is pushing linear controllers to their limits \cite{boyd1991linear}.
The concept of reset control was initially introduced in \cite{clegg1958nonlinear} as a nonlinear integrator, later known as the Clegg integrator (CI). It demonstrated promising behavior in overcoming the limitations inherent in linear feedback control caused by Bode’s gain-phase relationship \cite{CgLp}.
Over time, more advanced reset components were created, including the first-order reset element (FORE) \cite{horowitz1975non}, the generalized first-order reset element (GFORE) \cite{guo2009frequency}, and the second-order reset element (SORE) \cite{hazeleger2016second}.

% Frequency response analysis of  reset control system
By using the sinusoidal-input describing function (SIDF) method \cite{vander1968SIDF}, a reset element can be represented in the frequency domain. This allows us to perform frequency domain analysis when a reset element is incorporated with LTI elements in a control loop. Frequency response analysis is a technique that is used to assess the magnitude and phase properties of a control system. Thus, a designer can shape and tune the performance of closed-loop systems based on their open-loop analysis, a process known as loop shaping \cite{van2017frequency}. This design approach enables us to evaluate the closed-loop performance of the control system without developing a parametric model of the plant. Instead, we can use a frequency response function (FRF), which can be derived solely from measurement data \cite{franklin2002feedback}.

%Stability of a reset control system
The stability analysis of a reset control system (RCS) is as important as the performance analysis to ensure reliable and predictable operation. To assess the stability of a closed-loop RCS, several methods have been proposed \cite{banos2010reset,paesa2011design,banos2012reset,guo2015analysis,vettori2014geometric,griggs2009interconnections,hollot1997stability}. These include the reset instants dependent method \cite{banos2010reset,paesa2011design}, the quadratic Lyapunov functions method \cite{banos2012reset,guo2015analysis,vettori2014geometric}, the small gain, passivity, and IQC approaches \cite{griggs2009interconnections,hollot1997stability}. Utilizing these approaches typically requires solving LMIs and deriving a parametric approximation of the plant, both of which are time-consuming processes and introduce uncertainties. Therefore, similar to frequency domain-based performance analysis, a frequency domain-based stability analysis is desired for RCSs.

% Frequency-domain based method
To this respect, some frequency domain-based stability methods are introduced \cite{beker2004fundamental,beker1999stability,van2017frequency,van2024scaled,dastjerdi2023frequency,guo2015analysis}.
In \cite{van2017frequency}, a frequency domain-based stability method was developed for reset control systems, focusing on their input/output behavior. However, this method can be applied when the output is confined to a certain sector bound, as well as when the reset action is only triggered by the zero crossing of the reset element input.
In \cite{van2024scaled}, they approximate the scale graph \cite{huang2020tight} of reset controllers, resulting in a graphical tool to assess the stability of an RCS. However, it is not applicable to the zero-crossing triggered reset elements, which are the most common reset controllers.
In \cite{beker1999stability, beker2004fundamental}, the $H_\beta$ method was introduced, which includes a strictly positive real condition to guarantee closed-loop stability. It has been
proven that an RCS can be quadratically
stable if and only if the $H_\beta$ condition is satisfied. In fact,
the $H_\beta$ condition is a necessary and sufficient condition for the existence of a quadratic Lyapunov function \cite{beker2004fundamental,banos2010reset}. However, in general, this approach still relies on a parametric model to identify both a positive definite matrix and a vector that define the output of a transfer matrix, which must be strictly positive real.\\ Additionally, the method discussed in \cite{beker2004fundamental} is limited to RCSs where the error is forced to be the reset-triggered signal, and no pre-filtering of the reset element is possible. The same limitations apply to \cite{guo2015analysis}, as the quadratic stability presented there relies on the stability theorem from \cite{beker2004fundamental}. Although \cite[Remark 3.4]{guo2015analysis} introduces an additional condition in the $H_\beta$ method for non-zero after-reset values ($\gamma$), the exact proof and the effect of the shaping filter are not investigated there. In this study, we extend the $H_\beta$ method to accommodate any arbitrary reset surface (condition) and non-zero after-reset values ($\gamma$). While this extension is not the primary contribution of our work, it serves to validate the other theorems and lemmas presented. The proof is straightforward, drawing from existing studies such as \cite{banos2012reset}, \cite{guo2015analysis}, and \cite{beker2004fundamental}.

In \cite{dastjerdi2023frequency}, novel graphical stability conditions in the frequency domain are proposed for control systems incorporating first- and second-order reset elements, along with a shaping filter in the reset line. This approach facilitates the assessment of uniform bounded-input bounded-state (UBIBS) stability of reset control systems using the $H_\beta$ method without the need to solve LMI-based conditions. The matrix-based $H_\beta$ transfer function was first converted to an FRF-based transfer function in \cite[Corollary 3.12]{beker2001analysis} for a simple reset control system, where the reset element functions as the sole controller in the loop. However, in \cite{dastjerdi2023frequency}, a broader class of reset control systems is considered. Although an FRF-based transfer function is derived, the derivation is intuitive, lacking a formal mathematical connection between the matrix-based $H_\beta$ transfer function and the FRF-based transfer function for the proposed structure. Furthermore, the introduced method applies only to the series structure of a RCS without an LTI element in parallel with the reset element. Consequently, the analysis is not valid for reset systems with a feedthrough term. Therefore, since satisfying the $H_\beta$ conditions is a necessary requirement in the uniformly exponential convergence lemma presented in \cite{dastjerdi2022closed}, a significant gap remains in guaranteeing the convergence of non-series reset control systems using FRF-based methods.   

In \cite{banos2010reset} and \cite{guo2015analysis}, the effect of delay on the $H_\beta$ condition has been studied. However, the dynamics of the delay must still be known to calculate the $H_\beta$ transfer function. Based on the graphical method in \cite{dastjerdi2023frequency}, time delay could be part of the plant's FRF; thus, it directly affects the $H_\beta$ transfer function. In this study, we aim to investigate the effect of the $H_\beta$ transfer function with time delay on the quadratic stability conditions of a RCS when using different reset elements. This helps avoid wasting time assessing the stability of certain types of reset elements (GFORE, CI, PCI) using the FRF-based $H_\beta$ method in the presence of time delay.\\
%In addition, it is important to study the effect of time delay on the $H_\beta$ stability method while different reset elements are used. In \cite{banos2010reset}, the stability of time-delayed RCSs is investigated. However, in the end, the dynamics of the delay must still be known to calculate the $H_\beta$ transfer function. Here, the challenge is to determine which reset element would always lead to infeasible stability in the presence of delay. This helps avoid wasting time in the design process and ensures that an infeasible stability solution is not pursued.

Regarding the mentioned gaps in the existing literature, the contributions of this paper are as follows:
\begin{itemize}
    \item Provide the extended $H_\beta$ method for cases involving non-zero after-reset values, including the shaping filter.
    \item Mathematically calculate the FRF-based $H_\beta$ transfer function (in \cite{dastjerdi2023frequency} it is intuitively presented only for a limited class of RCSs):
    \begin{itemize}
        \item For general reset control systems, including parallel branch.
        \item Using matrix equality tools to prove the equivalence between a state-space-based transfer function and an FRF-based transfer function.
    \end{itemize}
    \item{Develop the conditions in the $H_\beta$ theorem for the new $H_\beta$ transfer function, while still requiring only the frequency response functions of the loop components.}
    \item Showing that the FRF-based $H_\beta$ method always leads to an infeasible stability assessment in the presence of delay when using CI or PCI.
\end{itemize}

The remainder of this paper is structured as follows. Section \ref{sec: preliminaries} provides descriptions of the reset element, LTI components, and the closed-loop system while covering some key theorems and lemmas that serve as the foundation for the rest of this study.
Section \ref{sec: main results} presents the main results of this paper. It first introduces the new FRF-based transfer function for the matrix-based $H_\beta$ transfer function and then develops the frequency domain-based conditions for the quadratic stability of general RCSs.
In Section \ref{sec: delay}, the effect of delay, which is an inseparable part of industrial settings, is examined in terms of its influence on the feasibility of this approach.
The utility and validation of the findings of this study are showcased through simulated examples in Section \ref{sec: example}. Finally, conclusions and suggestions for future studies are given in Section \ref{sec: conclusion}.