\section{Conclusion} \label{sec: conclusion}
In this paper, we analytically prove the mapping between the matrix-based $H_\beta$ transfer function and an FRF-based one for general (parallel) RCSs. We then use this mapping to relax the LMI-based conditions and assess quadratic stability using only some graphical, frequency domain-based conditions.
Using the results of this study, we can now assess the stability of a parallel RCS or an RCS with a feed-through term, as well as the UBIBS and convergence properties of the system, achievements that were not possible with existing FRF-based methods. Additionally, we analyzed the effect of delay on this method. The results showed that in cases where the plant suffers from delay, only the FORE/GFORE element results in feasible decisions regarding stability. We validated this finding by considering a mass-spring-damper system in the presence and absence of delay for a controller using GFORE. Furthermore, we used another illustrative example to show the model-free characteristics of this method by considering an FRF of the industrial setup as a plant. This means that no matter which structure of reset control we use or which system we want to control (with or without delay), using this method, we are able to assess its stability. Further research will focus on examining the robustness of the current approach and defining specific stability margins, providing valuable insights into its performance under various operating conditions and uncertainties. Moreover, a comprehensive conservativeness analysis of the existing stability methods for RCSs is required.



