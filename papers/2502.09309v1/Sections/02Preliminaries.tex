\section{Preliminaries}\label{sec: preliminaries}
\subsection{System description}
In the following, we provide a formal introduction to the reset control system and demonstrate its parallel interconnection with an LTI system, resulting in a SISO closed-loop system, as depicted in Fig. \ref{fig:Block diagram cl}. The linear part contains $G$ as the plant; $C_{1}$, $C_{2}$, and $C_{3}$ as the linear controllers; and a shaping filter $C_\text{s}$ with a proper stable transfer
function. Additionally, $C_\text{R}$ serves as the reset element, with its state-space realization presented as:
\begin{equation} 
		C_\text{R} : \begin{cases}
            \dot{x}_r(t)=A_rx_r(t)+B_ru_1(t),  \qquad \qquad e_r(t)\neq0,\\
            x_r(t^+)=\gamma x_r(t), \qquad e_r=0\wedge(1-\gamma)x_r(t)\neq0,\\
            u_r(t)=C_rx_r(t)+D_ru_1(t),
		\end{cases}
  \label{eq.SS reset}
	\end{equation}
where $A_r\in\mathbb{R}$, $B_r\in\mathbb{R}$, $C_r\in\mathbb{R}$, and $D_r\in\mathbb{R}$ represent the state-space matrices of the reset element, and the reset value is denoted by $\gamma \in\mathbb{R} $. $x_r(t)\in\mathbb{R}$ is the only state of the reset element, $x_r(t^{+})\in\mathbb{R}$ is the after reset state, $u_1(t)\in\mathbb{R}$ and $u_r(t)\in\mathbb{R}$ represent the input and output of the reset element, respectively. Also, $e_r(t)$ is the output of the shaping filter $C_s$.

\begin{figure}[!t]
\centerline{\includegraphics[scale=0.45,,trim=0 0 0 0,clip]{Figures/Block_diagram.pdf}}
\caption{The closed-loop architecture of a general reset control system.}
\label{fig:Block diagram cl}
\end{figure}

In this study, $C_R$ is considered to be a first-order reset element (CI or GFORE). Due to the utilization of a parallel filter with the reset element, PCI is not considered, as it can be constructed using a CI in parallel with the desired gain. By considering $C_rB_r=\omega_k>0$ and $A_r=-\omega_r$ ($\omega_r\geq0$), the general form of the reset element used in this paper is defined, and its base linear transfer function is as follows
\begin{equation}
\begin{split}
    \label{RCS expression}
    R(s)&=C_r(s-A_r)^{-1}B_r+D_r \\
    &=\frac{\omega_k}{s+\omega_r}+D_r,
    \end{split}
\end{equation}
where $s \in\mathbb{C}$ is the Laplace variable. The reset element represents a CI or PCI when $\omega_r=0$ and represents a GFORE element when $\omega_r \neq0$.

For the LTI part of the closed-loop system, which is denoted by $\mathcal{L}$, we have
\begin{equation} 
		\mathcal{L} : \begin{cases}
			\label{eq.SS linear}
			\dot{x}_l(t)=Ax_l(t)+B_uu_r(t)+Bw(t),\\
			y(t)=Cx_l(t),\\
			u_1(t)=C_ux_l(t)+D_uw(t), \\
                e_r(t)=C_ex_l(t)+D_e w(t),
		\end{cases}
	\end{equation}
where $x_l(t)\in\mathbb{R}^{n_l}$ is the state of the LTI part of the system, and $w(t)=\begin{bmatrix}
    r(t) & d(t)
\end{bmatrix}^T\in\mathbb{R}^{2}$, in which $r(t)$ and $d(t)$ are the reference and disturbance signal of the system, respectively. Also, $A\in\mathbb{R}^{n_l \times n_l}$, $B\in\mathbb{R}^{n_l \times 2}$, $C\in\mathbb{R}^{1 \times n_l}$, $B_u\in\mathbb{R}^{n_l \times 1}$, $C_u\in\mathbb{R}^{1 \times n_l}$, $D_u\in\mathbb{R}^{1 \times 2}$, $C_e\in\mathbb{R}^{1 \times n_l}$ and $D_e\in\mathbb{R}^{1 \times 2}$ are the corresponding dynamic matrices.\\

\begin{assumption}
    \label{ass: assumption no direct feed through}
    In this study, it is assumed that there is no direct feedthrough from input $w(t)$ and $u_r(t)$ to plant output $y(t)$ and from $u_r(t)$ to $u_1(t)$ and $e_r(t)$.\\
\end{assumption}

Assumption \ref{ass: assumption no direct feed through} is reasonable, as it pertains to any causal LTI element $C_1$, $C_2$, $C_3$, and any plant \( G \) with a relative degree greater than zero, which includes the vast majority of motion and mass-based systems. Hence, the closed-loop state-space representation of the overall reset control system (RCS) can be expressed as follows
\begin{equation} 
		\text{RCS} : \begin{cases}
			\label{eq.SS closed loop}
            \dot{x}(t)=\bar{A}x(t)+\bar{B}w(t), \qquad   x(t)\notin \mathcal{F},\\
            x(t^+)=A_\rho x(t), \: \: \: \, \qquad   \qquad x(t)\in \mathcal{F},\\
            e_r(t)=\bar{C}_ex(t)+\bar{D}_e w(t), \\
            y(t)=\bar{C}x(t),
		\end{cases}
	\end{equation}
in which $x(t)=\begin{bmatrix}
    x_r(t)^T & x_l(t)^T
\end{bmatrix}^T \in\mathbb{R}^{1+n_l}$, $\bar{C}=\begin{bmatrix}
    0 & C
\end{bmatrix}$, $\bar{B}=\begin{bmatrix}  0_{1\times 2}\\ B \end{bmatrix} +$ $\begin{bmatrix}  B_rD_u & 0_{1\times 1}\\ B_uD_rD_u & 0_{n_l\times 1} \end{bmatrix}$, $\bar{A}=\begin{bmatrix}
    A_r & B_rC_u\\
    B_uC_r & A+B_uD_rC_u
\end{bmatrix}$, $A_\rho = \begin{bmatrix}
    \gamma & 0_{1\times n_l}\\
    
    0_{n_l\times 1} & I_{n_l\times n_l}
\end{bmatrix}$, and $\bar{C}_e=\begin{bmatrix}
    0 & C_e
\end{bmatrix}$. With reset surface $\mathcal{F}=\{x(t) \in \mathbb{R}^{n_l+1}:\bar{C}_ex(t)+\bar{D}_e w(t)=0 \wedge (I-A_\rho)x(t)\neq0\}$.

\subsection{Foundational theorems and lemmas}
This part covers the fundamental theorems and lemmas that form the foundation for the rest of this study.
As mentioned earlier, this study focuses on the $H_\beta$ stability approach. The $H_\beta$ condition is based on a quadratic Lyapunov function that must be decreasing over the entire state space along the system trajectories and non-increasing at the reset jumps. This method was first introduced in \cite{beker2004fundamental} for the case where the reset surface is defined as $\{x(t) \in \mathbb{R}^{n_l+1}:\bar{C}x(t)=0 \wedge (I-A_\rho)x(t)\neq0\}$ and $\gamma=0$, as in \cite[Chapter 4.4]{banos2012reset}. However, in this study, the reset surface is considered as $\mathcal{F}$, with non-zero cases for $\gamma$ also examined. Consequently, the modified $H_\beta$ method for assessing the stability of the reset control system in \eqref{eq.SS closed loop} is described in the following theorem.\\  

\begin{theorem}
 The zero equilibrium of the reset control system \eqref{eq.SS closed loop} with $w = 0$ is globally uniformly asymptotically stable if there exist $\varrho=\varrho^T > 0$ and $\beta\in\mathbb{R}$ such that the transfer function
\begin{equation}
    \label{eq.H beta}
    H_{\beta}(s)=C_0(sI-\bar{A})^{-1}B_0,
\end{equation}
with
\begin{equation}
 C_0=\begin{bmatrix}
\varrho & \beta C_e
\end{bmatrix}, \quad
B_0= \begin{bmatrix}
    1 \\
    0_{n_l \times 1}
\end{bmatrix},
\label{eq. C0B0}
\end{equation}
is Strictly Positive Real (SPR), $(\bar{A},B_0)$ and $(\bar{A},C_0)$ are controllable and observable respectively, and $-1\leq\gamma\leq1$.
\label{Th. Theorem1}
\end{theorem}
As previously mentioned, the only differences between Theorem \ref{Th. Theorem1} and the $H_\beta$ theorem presented in \cite[Proposition 4.5]{banos2012reset} are the reset surface and the value of $\gamma$. Therefore, the proof of Theorem \ref{Th. Theorem1} is provided in Appendix \ref{App: H_b}, following the approach used in \cite[Proposition 4.5]{banos2012reset}.\\
%As mentioned earlier, the only difference between Theorem \ref{Th. Theorem1} and the $H_\beta$ theorem presented in \cite{beker2004fundamental} lies in the reset surface. Since the reset signal here is $e_r(t)$, the matrix $C_e$ is incorporated into the $C_0$ matrix. Consequently, the proof follows the same steps as in \cite[Proof of Theorem 5]{beker2004fundamental}.\\

%\begin{remark}
%    \label{reset independency}
%Based on \cite{beker2004fundamental} and \cite[Section 3.2.1]{banos2012reset}, the $H_\beta$ method is a reset-time independent stability method. This means that the shaping filter $C_s$ does not influence the stability conditions. Therefore, in Theorem \ref{Th. Theorem1}, for simplicity, we can consider $C_e = C_u$ and $D_e = D_u$ (i.e., $C_s = 1$).\\
%\end{remark}

\begin{lemma}
\cite[Lemma 6.1]{khalil2002nonlinear}, Let $H(s)$ be a proper rational $p\times p$ transfer function and assume det$[H(s) + H^{T}(-s)]$ is not identically zero. Then, $H(s)$ is SPR if and only if :
\begin{itemize}
    \item $H(s)$ is Hurwitz,
    \item $H(j\omega) + H^{T}(-j\omega)$ is positive definite for all $\omega \in \mathbb{R}$,
    \item either $H(\infty) + H^{T}(\infty)$ is positive definite or
     if $H(\infty) + H^{T}(\infty)$ is positive semi-definite, \\ $\lim_{\omega \to \infty} \omega^2 M^{T}[H(j\omega) + H^{T}(-j\omega)]M>0$, for any $p (p - q)$ full rank matrix $M$ such that $M^{T}[H(\infty) + H^{T}(\infty)]M=0$, and $q = \text{rank}[H(\infty) + H^{T}(\infty)]$.\\
\end{itemize}
\label{lem: SPR}
\end{lemma}

\begin{definition}
    \label{well posedness}
\cite{dastjerdi2023frequency}, A time $\bar{T} > 0$ is called a reset instant for the reset control system \eqref{eq.SS closed loop} if $e_r(\bar{T})=0\,\wedge\,(I-A_\rho)x(\bar{T})\neq0$. For any given initial condition and input $w$, the resulting set of all reset instants defines the reset sequence $\{t_k\}$, with $t_k \leq t_{k+1}$ for all $k \in \mathbb{N}$. The reset instants $t_k$ have the well-posedness property if for any initial condition $x_0$ and any input $w$, all the reset instants are distinct, and there exists $\lambda > 0$ such that, for all $k \in \mathbb{N}$, $\lambda \leq t_{k+1} - t_k$.\\
\end{definition}
Note that the second condition in $\mathcal{F}$, $(I- A_\rho)x(t)\neq0$, is imposed for well-posedness, to avoid the so-called re-resetting, that is, the fact that immediately after a reset, the state vector could satisfy the reset condition again, implying formally an infinite sequence of re-settings to be performed
instantaneously, also referred to as beating. Thus, to avoid this, the term $(I- A_\rho)x(t)\neq0$ is added in the definition of $\mathcal{F}$ (see \cite[Section 1.4.1]{banos2012reset}). Simply it means
\begin{equation}
    \nonumber
    x(t)\in \mathcal{F} \Rightarrow  x(t^{+})\notin \mathcal{F}.\\
\end{equation}

\begin{assumption}
    \label{assumption}
    There are infinitely reset instants and $\lim_{k \to \infty} t_k=\infty$.\\
\end{assumption}

Assumption \ref{assumption} is necessary to guarantee the existence of a rest instant $t_k$ as $k\rightarrow \infty$ for the proof of Lemma \ref{lemma UBIBS} and Lemma \ref{convrgence} (will be presented further). However, if it does not hold, it means there is no reset after a time $t_{k'}$, which implies that the RCS behaves like its base linear system ($A_\rho=I$ in \eqref{eq.SS closed loop}). In this case, the stability and convergence properties of the RCS are equivalent to those of its base linear system. This means that if the base linear system is stable and has a convergent solution, then the RCS is stable and has a convergent solution as well.\\
In the following, the definition of the Bohl function and its characteristics are presented, as UBIBS stability and convergence of the RCSs have been established for this class of input signals in Lemma \ref{lemma UBIBS} and \ref{convrgence}.
\begin{definition}
    \label{def: bohl function}
    \cite[Definition 2.5]{Bohl_trentelman2002control}, A function that is a linear combination of functions of the form $t^k e^{\lambda t}$, where the $k$'s are nonnegative integers and $\lambda \in \mathbb{C}$, is called a Bohl function. The numbers $\lambda$ that appear in this linear combination (and cannot be canceled) are called the characteristic exponents of the Bohl function.\\
\end{definition}

\begin{remark}
    In \cite[Theorem 2.7]{Bohl_trentelman2002control} it is shown that if $p$ and $q$ are Bohl functions then $p+q$, $pq$ and $\dot{p}$ are also Bohl functions. Additionally, it can easily be shown that step functions, ramp functions, and any sinusoidal functions are Bohl functions.\\
\end{remark}

\begin{lemma}
    \label{lemma UBIBS}
    \cite[Lemma 2]{dastjerdi2023frequency}, Consider the reset control system \eqref{eq.SS closed loop}. Suppose that
    \begin{itemize}
        \item Assumption \ref{assumption} holds;
        \item $-1<\gamma<1$;
    \item All conditions in Theorem \ref{Th. Theorem1} hold;
        %\item $-1<\gamma<1$;
        \item at least one of the following conditions holds:
        \begin{itemize}
            \item $C_s=1$;
            \item the reset instants have the well-posedness property.
        \end{itemize}
    \end{itemize}
    Then, the reset control system \eqref{eq.SS closed loop} has a well-defined unique left-continuous response for any initial condition $x_0$ and any input $w$, which is a Bohl function. In addition, the reset control system \eqref{eq.SS closed loop} has the UBIBS property.\\
%\textbf{Proof.} \cite[Appendix A]{dastjerdi2023frequency}.
\end{lemma}

Please note that the well-posedness property (Definition \ref{well posedness}) is a reasonable assumption, as it is ensured by the second condition ($(I-A_\rho)x(t) \neq 0$) on the reset surface $\mathcal{F}$. This condition effectively prevents what is known as 're-resetting,' a situation where, immediately after a reset, the state vector may once again satisfy the reset condition, potentially resulting in an infinite sequence of resets (see \cite[Sections 1.41 and 2.2.1]{banos2012reset}).\\

\begin{definition}
    \label{def: convergence}
    \cite[Definition 2]{pavlov2007frequency}, The system $\dot{x}(t) \in F(x(t),w(t))$, with a given continuous on $\mathbb{R}$ input $w(t)$ is said to be (uniformly, exponentially) convergent if:
    \begin{itemize}
        \item all solutions $x_w(t,t_0,x_0)$ are defined for all $t \in [t_0,+\infty)$ and all initial conditions $t_0 \in \mathbb{R}$, $x_0\in\mathbb{R}^n$;
        \item there is a solution $\bar{x}_w(t)$ defined and bounded on $\mathbb{R}$;
        \item the solution $\bar{x}_w(t)$ is (uniformly, exponentially) globally asymptotically stable. \\
    \end{itemize}
\end{definition}

\begin{lemma}
\label{convrgence}
\cite[Lemma 2]{dastjerdi2022closed}, Consider the reset control system in \eqref{eq.SS closed loop} with $C_u=C_e$ and $D_u=D_e$ ($C_s=1$), the system is uniformly exponentially convergent \cite[Definition 2]{pavlov2007frequency} for any input $w$ that qualifies as a Bohl function, if
\begin{itemize}
 \item Assumption \ref{assumption} holds;
        \item $-1<\gamma<1$;
    \item All conditions in Theorem \ref{Th. Theorem1} hold;
\item the initial condition of the reset element is zero.\\
\end{itemize}
\end{lemma}

\begin{remark}
    \label{proof explanation}
    In Lemma \ref{convrgence}, it is assumed $C_u=C_e$ and $D_u=D_e$ since \cite{dastjerdi2022closed} only considers systems with $C_s=1$.\\
\end{remark}

The main stability condition presented in Theorem \ref{Th. Theorem1} is in the frequency domain; however, it still requires the system model parameters to calculate $H_\beta$ transfer function and the use of an LMI-based method to find $\varrho$ and $\beta$. In \cite[Theorem 2]{dastjerdi2023frequency}, the results from Theorem \ref{Th. Theorem1} and Lemma \ref{lemma UBIBS} are combined into FRF-based conditions for a class of RCSs where $C_3=0$ and $D_r=0$ (the result presented in \cite{dastjerdi2023frequency} is not valid for $D_r\neq0$) using an intuitive derivation of an FRF-based $H_\beta$ transfer function.\\

The following two lemmas are presented as the foundation for calculating the FRF-based $H_\beta$ transfer function for the RCS \eqref{eq.SS closed loop} in the next section.
\begin{comment}
\begin{theorem}
    \label{th: Ali's theorem}
    \cite[Theorem 2]{dastjerdi2023frequency}, The zero equilibrium of the reset control system \eqref{eq.SS closed loop} with $C_3=0$ and $D_r=0$ is globally uniformly asymptotically
stable when $w = 0$, and the system has the UBIBS property for
any input $w$, which is a Bohl function if all of the following conditions are satisfied.
\begin{itemize}
        \item The base linear system is stable, and its open loop transfer function does not have any unstable pole-zero cancellation.
        \item $C_s(s) = 1$ and/or the reset instants have the well-posedness property.
        \item $-1<\gamma<1$.
        \item The transfer function $\frac{\beta_x L(s)C_s(s)R(s)+\varrho_x R(s)}{1+L(s)R(s)}$ must be SPR, where $\beta_x=-\beta$, and $\varrho_x=\frac{\varrho}{C_r}$, and $L(s)=C_1(s)C_2(s)G(s)$. \\
        \end{itemize}
\end{theorem}
\end{comment}
\begin{lemma}[Partitioned matrix inversion] 
    \label{lem: Mat inv}
     \quad \\
    \cite[Section 9.2.14]{mahmoud2021cyberphysical}, Let's consider an invertible matrix $M$, composed of blocks $Q_1$, $Q_2$, $Q_3$, and $Q_4$, as follows
    \begin{equation}
    \label{eq: M}
        M = \begin{bmatrix}
    Q_1 & Q_2\\
    Q_3 & Q_4
\end{bmatrix},
    \end{equation}
    the inverse of matrix $M$ can be expressed using a similarly structured block representation
    \begin{equation}
        M^{-1} =
\begin{bmatrix}
W & X \\
Y & Z \\
\end{bmatrix},
    \end{equation}
    where the blocks $W$, $X$, $Y$, $Z$, are as follows
\begin{equation}
\begin{cases}
W = \left( Q_1 - Q_2 Q_4^{-1} Q_3 \right)^{-1}, \\
Z = \left( Q_4 - Q_3 Q_1^{-1} Q_2 \right)^{-1},
\end{cases}
\end{equation}
and,
\begin{equation}
\begin{cases}
Y = -Q_4^{-1} Q_3 W, \\
X = -Q_1^{-1} Q_2 Z.
\end{cases}
\end{equation}\\
\end{lemma}

\begin{lemma}[Woodbury matrix identity]
    \label{lem: Woodbury}
    \quad \\
      \cite[Appendix. A, Solution to Problem 13.9]{higham2002accuracy}, For matrices $K$, $U$, $J$, and $V$, where $K$ is a $n\times n$ matrix, $J$ is a $k\times k$ matrix, $U$ is a $n\times k$ matrix, and $V$ is a $k\times n$ matrix, the following equation holds
    \begin{equation}
    \begin{split}
        (K + &UJV)^{-1} \\ &= K^{-1} - K^{-1}U(J^{-1} + VK^{-1}U)^{-1}VK^{-1}.
        \end{split}
    \end{equation}
\end{lemma}

%\subsection{Problem definition}
%In Fig. \ref{fig:Block diagram cl}, the structure of a general reset control system is depicted. Based on the results from \cite{caporale2024practical} and [Luke Ifac], the parallel implementation of reset control systems can reduce the negative effects of limit cycles and discretization on performance.
