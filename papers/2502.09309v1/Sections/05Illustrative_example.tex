\section{Illustrative examples} \label{sec: example}
In this section, we test the validity of the proposed method for assessing the stability of a general RCS using the frequency response of its components. First, we evaluate the stability of a mass-spring-damper system, both with and without delay, to validate the results presented in Section \ref{sec: delay}. Additionally, we analyze the stability of an industrial precision positioning system, where only the FRF of the plant is available.

The closed-loop structure for both cases is the same as Fig. \ref{fig:Block diagram cl}, where
\begin{equation}
\begin{split}
    \label{eq:controllers}
    &C_R=\text{GFORE}\,(A_r=-\omega_r, B_r=1, C_r=\omega_k, D_r=0), \\
   &C_2(s)=k_p\;\omega_{i}\bigg(k_{g}+\frac{1}{s}\bigg)\bigg(\frac{\frac{s}{\omega_d}+1}{\frac{s}{\omega_t}+1}\bigg),\, C_1(s)=1,  \\
   &C_3(s)=\frac{s}{\Big(k_{g} s+1\Big)\omega_{i}},\quad C_s(s)=1,
\end{split}
\end{equation}
and
\begin{equation}
    k_{g}=\frac{1}{\omega_r|1+\frac{4j}{\pi}\frac{1-\gamma}{1+\gamma}|}.
\end{equation}
More details about the controller design and parameters tuning can be found in \cite{Yixuan}.

\subsection{Mass-spring-damper system}
This example demonstrates the effectiveness and repeatability of the stability method in a general form. The transfer function of a mass-spring-damper (MSD) system with transport delay is given by

\begin{equation}
\label{eq: M_S_D}
G(s) = \frac{\omega_n^2}{s^2 + 2\zeta\omega_n s + \omega_n^2} D(s, T)
\end{equation}
where $\zeta = 0.2$, $\omega_n = 30$, and $D(s, T)$ represents a time delay of $T$. Two scenarios are examined: one with a delay of $T = 0.0015\,$sec and one without any delay ($T = 0$). The controller structure remains consistent with the one presented in \eqref{eq:controllers}, and the corresponding parameters are listed in Table \ref{tab: parameters}.
%In this example, the base linear system is stable and $-1<\gamma<1$. Moreover, $\theta_{\mathcal{N}}(\omega)$ is depicted in Fig. \ref{fig: theta_N MSD} for both cases with and without delay. It's evident that $-\frac{\pi}{2}<\theta_{\mathcal{N}}(\omega)<\pi$ and $(\theta_2-\theta_1)<\pi$.
%As a result, according to Theorem \ref{lem:Hb frequency} and Lemma \ref{lemma UBIBS}, the closed-loop system in this instance qualifies as a UBIBS stable reset control system, regardless of the presence or absence of delay. Moreover, regarding the Lemma \ref{convrgence}, and considering the zero initial condition for the designed reset controller, the RCS in this example has a uniformly exponentially convergent solution.

In this example, the base linear system is stable, with $-1 < \gamma < 1$. Fig. \ref{fig: theta_N MSD} illustrates $\theta_{\mathcal{N}}(\omega)$ for both scenarios: with and without delay. It is clear that $-\frac{\pi}{2} < \theta_{\mathcal{N}}(\omega) < \pi$ and $(\theta_2 - \theta_1) < \pi$. Also, it can be seen that the delay only adds oscillation around $\theta_{\mathcal{N}}(\omega)=\frac{\pi}{2}$ ($\mathcal{N}_x (\omega)=0$) which means only $\mathcal{N}_x (\omega)$ becomes $\mathcal{Z}_L$ function and still we can assess the stability for this GFORE-based RCS like the result in Corollary \ref{CI,GFORE}. 

Consequently, based on Theorem \ref{lem:Hb frequency} and Lemma \ref{lemma UBIBS}, the closed-loop system in this case qualifies as a UBIBS stable reset control system, irrespective of the presence or absence of delay. Additionally, according to Lemma \ref{convrgence} and considering the zero initial condition for the designed reset controller, the RCS in this example has a uniformly exponentially convergent solution.

Note that here, for both cases (with and without delay), the same controller parameters are considered to be able to observe only the effect of delay. The difference in the mid-frequency range (the peak of $\theta_{\mathcal{N}}(\omega)$) could be related to the stability of the base linear system since the system with delay has less phase margin. However, despite observing this behavior (a more stable base linear system leads to a more robust $\theta_{\mathcal{N}}(\omega)$) in every example, a mathematical proof is still needed, which could be part of future studies.


%\begin{figure*}[!t]
%\footnotesize

\begin{table*}[]
\centering
\caption{Controller parameters.}
\label{tab: parameters}
\resizebox{\textwidth}{!}{%
\begin{tabular}{|c|c|c|c|c|c|c|c|c|}
\hline
          & $\gamma$ &$D_r$& $\omega_r$           &$\omega_k$& $\omega_i$ & $\omega_d$ & $\omega_t$ & $k_p$    \\ \hline
MSD       & 0        &0& 42.66                &42.66& 38.71      & 50         & 450       & 6.5     \\ \hline
Y$\Theta$ & 0        &0& 67.5$\times 10^{-4}$ &67.5$\times 10^{-4}$& 61.25$\times 10^{-4}$      & 79.167$\times 10^{-4}$     & 356.25$\times 10^{-4}$     & 3518300 \\ \hline
\end{tabular}
}
\end{table*}
%\vspace*{4pt}
%\hrulefill
%\end{figure*}
\begin{figure}[!t]
\centering
\includegraphics[width=0.9\columnwidth]{Figures/Fig3_Stability_MSD_v5.pdf}	\caption{$\theta_{\mathcal{N}}(\omega)$ for the MSD system.}
	\label{fig: theta_N MSD}
\end{figure}
\subsection{Precision positioning wire-bonding machine}
Here, the stability of a general reset control system is assessed, where the plant under control is the precision positioning stage of ASMPT's wire-bonding machine. It is assumed that the only information about the plant is its measured FRF, which is depicted in Fig. \ref{fig: FRF}. To maintain confidentiality, adjustments have been made to the frequency axis using an arbitrary constant $\eta$, while excluding both magnitude and phase information.

The control parameters for this case are presented in Table \ref{tab: parameters}. These parameters are obtained from \cite{Yixuan} based on an automated tuning algorithm that aims to maximize the open-loop bandwidth while considering the closed-loop performance.

\begin{figure}
	\centering
 	\includegraphics[width=0.9\columnwidth]{Figures/FRF_v5.pdf}
	\caption{Frequency response data from the x-axis motion platform of the ASMPT's wire-bonding machine.}
	\label{fig: FRF}
\end{figure}

Considering the plant in Fig. \ref{fig: FRF}, controller elements in \eqref{eq:controllers}, and parameters in Table. \ref{tab: parameters}, the Theorem \ref{lem:Hb frequency} is applied to the closed-loop system. The controller is designed to have a stable base linear system and also $-1<\gamma<1$. The phase of the NSV ($\theta_{\mathcal{N}}$), shown in Fig. \ref{fig: theta_N}, reveals that $-\frac{\pi}{2} < \theta_{\mathcal{N}}(\omega) < \pi$ and $(\theta_2 - \theta_1) < \pi$. As a result, conditions three and four in Theorem \ref{lem:Hb frequency} for the GFORE case ($\omega_r \neq 0$) are satisfied. This indicates that the closed-loop system is globally uniformly asymptotically stable for the zero-input case.

Also, since the reset control system has been designed for the zero initial condition and $C_s=1$, by using Lemma \ref{lemma UBIBS} and Lemma \ref{convrgence}, the closed-loop system has the UBIBS property and a uniformly exponentially convergent solution for any input $w$ which is a Bohl function.

This example shows that even without a parametric model of the system or any transfer function, it is still possible to assess the stability of the most general form of a reset control system (including pre-, post-, and parallel filters along reset element in the loop) by using only the measured FRF of the plant.

\begin{figure}
	\centering
 	\includegraphics[width=0.9\columnwidth]{Figures/WireBonder_Stability_v5.pdf}
	\caption{ $\theta_{\mathcal{N}}(\omega)$ for the precision positioning control system.}
	\label{fig: theta_N}
\end{figure}