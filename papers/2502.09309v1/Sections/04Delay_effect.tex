\section{Effect of delay} \label{sec: delay}
Regarding the Definition \ref{def:NSV}, the system's plant ($G$) is a component of both $\mathcal{N}_x (\omega)$ and $\mathcal{N}_y (\omega)$ in the NSV. Considering a plant in the presence of a delay can bring about changes in $\theta_{\mathcal{N}} (\omega)$ and consequently in the stability status. Thus, in this section, it is investigated in which circumstances the delay could lead to a situation where demonstrating the stability of the closed-loop reset control system in \eqref{eq.SS closed loop} becomes unfeasible. The findings of this analysis are summarized in Corollary \ref{CI,GFORE}. To support the proof of this corollary, we introduce Definition \ref{Zl}, Lemma \ref{lemma: sum_mult Zl}, and Lemma \ref{NxNyZl} as supplementary tools.\\

\begin{definition}
    \label{Zl}
    The function $K(x)$ is called a $\mathcal{Z}_L$ function if it can be defined in the form $K(x)=g(x)h(x)$, where
    \begin{equation}
        \label{ZL function}
         \lim_{x\to\infty} |g(x)|=0,
    \end{equation}
    and $h(x)$ be an oscillating function with zero mean as follows
    \begin{equation}
        \label{ZL function h}
         h(x)=A\sin{(x+\phi)},
    \end{equation}
    with $A\in \mathbb{R}^{+}$ and $\phi \in [0,2\pi).$\\
\end{definition}

\begin{lemma}
    \label{lemma: sum_mult Zl}
    If $K_1(x_1)$ and $K_2(x_2)$ ($K_1(x_1)\neq K_2(x_2)$) are $\mathcal{Z}_L$ functions then $K_1(x_1)+K_2(x_2)$ and $K_1(x_1)\times K_2(x_2)$ are also $\mathcal{Z}_L$ functions.\\
    \textit{\textbf{Proof.}} Appendix \ref{App: III}\\
\end{lemma}

\begin{lemma}
    \label{NxNyZl}
    In the presence of the delay, if both $\mathcal{N}_x (\omega)$ and $\mathcal{N}_y (\omega)$ in the NSV are $\mathcal{Z}_L$ functions, it is not possible to demonstrate the stability of the reset control system \eqref{eq.SS closed loop} using Theorem \ref{lem:Hb frequency}. 
\end{lemma}
\begin{proof}
    \label{P.NxNyZl}
    When both $\mathcal{N}_x (\omega)$ and $\mathcal{N}_y (\omega)$ are $\mathcal{Z}_L$ functions, it means they take both positive and negative values as $\omega \rightarrow \infty$, which implies that $\stackrel{\rightarrow}{\mathcal{N}}(\omega)=\begin{bmatrix}
        \mathcal{N}_x (\omega) & \mathcal{N}_y (\omega)
    \end{bmatrix}^T$ spans all four quadrants. Therefore, $\theta_{\mathcal{N}} (\omega)$ takes all values in $(0,2\pi]$, and non of the constraints on $\theta_{\mathcal{N}} (\omega)$ in Theorem \ref{lem:Hb frequency} cannot be satisfied.\\
\end{proof}

\begin{corollary}
    \label{CI,GFORE}
    For a system with delay, Theorem \ref{lem:Hb frequency} cannot be employed to establish the stability of the RCS in \eqref{eq.SS closed loop} with CI ($\omega_r=0$). However, it can be utilized to demonstrate the stability of a reset control system with a GFORE element ($\omega_r\neq0$).
\end{corollary}
\begin{proof}
    We will demonstrate that in the case of CI, both $\mathcal{N}_x (\omega)$ and $\mathcal{N}_y (\omega)$ are $\mathcal{Z}_L$ functions but in the case of GFORE, only $\mathcal{N}_x (\omega)$ is a $\mathcal{Z}_L$ function. Therefore, using Lemma \ref{NxNyZl}, the conditions in Theorem \ref{lem:Hb frequency} cannot be satisfied if the system contains a CI element.\\
\begin{comment}
Here we consider the delay function as 
\begin{equation}
\label{Euler}
    D(s,T)=e^{-Ts}=\cos{(\omega T)}-j\sin{(\omega T)},
\end{equation}
with a delay time $T$. By using the Pade approximation (\cite[chapter 9]{golub2013matrix}, \cite{gragg1972pade}) we have
\begin{equation}
    D(s,T)\simeq D(s,T)
\end{equation}
where
\begin{equation}
D(s,T)=\frac{a_0+a_1 Ts+a_2 T^2 s^2+...+a_m T^m s^m}{b_0+b_1 Ts+b_2 T^2 s^2+...+b_n T^n s^n},
\end{equation}
with stable poles. According to \eqref{Euler} we have
\begin{equation}
    D(\omega,T)\simeq\cos{(\omega T)}-j\sin{(\omega T)}.
\end{equation}
\end{comment}
Here we consider the delay function as 
\begin{equation}
\label{Euler}
    D(s,T)=e^{-Ts},
\end{equation}
with a delay time of $T\in  \mathbb{R}_{>0}$. Also, we can write 
\begin{equation}
    D(\omega,T)=\cos{(\omega T)}-j\sin{(\omega T)}.
\end{equation}
Thus, the plant in the presence of the delay can be expressed as follows:
\begin{equation}
\label{Gd}
G_d(j\omega)=G(j\omega)D(\omega,T),
\end{equation}
and, subsequently,
\begin{equation}
\label{Ld}
    L_d(j\omega)=L(j\omega)D(\omega,T).
\end{equation}
Therefore, equations in (\ref{eq.M}) can be rewritten as below
\begin{equation}
    \begin{split}
        M_1(j\omega)&=1+\bigg(L(j\omega) D(\omega,T)\Bigl(R(j\omega)+C_3(j\omega)\Bigl)\bigg), \\
        M_2(j\omega)&=L(j\omega) D(\omega,T)C_s(j\omega)\Big(R(j\omega)-D_r\Big), \\
        M_3(j\omega)&=\bigg(1+L(j\omega) D(\omega,T)\Big(C_3(j\omega)+D_r\Big)\bigg)...\\
        &\Bigl(R(j\omega)-D_r\Bigl),
    \end{split}
    \label{eq.M new}
\end{equation}
where based on the Definition \ref{def:NSV}, in order to derive $\mathcal{N}_x (\omega)$ and $\mathcal{N}_y (\omega)$, it is needed to calculate $M_1^*(j\omega)$, $M_2(j\omega)$, and $M_3(j\omega)$.\\
To calculate $M_1^*(j\omega)$, consider
\begin{equation}
\label{eq: a1b1}
    L(j\omega)\Bigl(R(j\omega)+C_3(j\omega)\Bigl)=a_1(\omega)+jb_1(\omega),
\end{equation}
where because $L(j\omega)$ is a strictly proper transfer function, $\lim_{\omega\to\infty} |a_1(\omega)|=0$, and $\lim_{\omega\to\infty} |b_1(\omega)|=0$. Hence,
\begin{equation}
\begin{split}
\label{eq: E1E2}
    &L(j\omega)D(\omega,T)\Bigl(R(j\omega)+C_3(j\omega)\Bigl)\\
    &=\Big(a_1(\omega)+jb_1(\omega)\Big)\Big(\cos{(\omega T)}-j\sin{(\omega T)}\Big) \\
    &=\Big(a_1(\omega)\cos{(\omega T)}+b_1(\omega)\sin{(\omega T)}\Big)\\
    &+j\Big(-a_1(\omega)\sin{(\omega T)}+b_1(\omega)\cos{(\omega T)}\Big)\\
    &=E_1(\omega)+jF_1(\omega),
    \end{split}
\end{equation}
thus,
\begin{equation}
    \label{eq: M1*}
    M_1^{*}(j\omega)=1+E_1(\omega)+jF_1(\omega),
\end{equation}
where regarding the Definition \ref{Zl}, $E_1(\omega)$ and $F_1(\omega)$ are $\mathcal{Z}_L$ functions.\\
%Delay effects in high frequencies and also $L(j\omega)$ is always a strictly proper transfer function. Therefore, in (\ref{eq: a1b1}), $a_1(\omega)<<1$, and $b_1(\omega)<<1$ when delay effects. Consequently, regarding (\ref{eq: E1E2}), $E_1(\omega)$ and $F_1(\omega)$ are two oscillating values around zero line where also $|E_1|<<1$, and $|F_1|<<1$. From now on, every value named $E_n(\omega)$ and $F_n(\omega)$ will have the same properties as $E_1(\omega)$ and $F_1(\omega)$.
To calculate $M_2(j\omega)$ it is considered
\begin{equation}
    \label{eq: M2}
     L(j\omega)C_s(j\omega)=a_2(\omega)+jb_2(\omega),
\end{equation}
and from \eqref{RCS expression},
\begin{equation}
    \label{eq: M2'}
    \begin{split}
     R(j\omega)-D_r&=\frac{\omega_k}{j\omega+\omega_r} \\
     &=\frac{\omega_k \omega_r}{\omega^2+\omega_r ^{2}}-j\frac{\omega_k\omega}{\omega^2+\omega_r ^{2}}=a_3(\omega)-jb_3(\omega).
     \end{split}
\end{equation}
Therefore,
 \begin{equation}
    \label{eq: M2''}
    \begin{split}
    &L(j\omega)C_s(j\omega)\Big(R(j\omega)-D_r\Big)\\
     &=\Big(a_2(\omega)a_3(\omega)+b_2(\omega)b_3(\omega)\Big) \\
     &+j\Big(b_2(\omega)a_3(\omega)-b_3(\omega)a_2(\omega)\Big)=E_2(\omega)+jF_2(\omega),
     \end{split}
\end{equation}
where $\lim_{\omega\to\infty} |E_2(\omega)|=0$, and $\lim_{\omega\to\infty} |F_2(\omega)|=0$. Thus, for $M_2(j\omega)$ it gives,
\begin{equation}
    \label{eq: M2'''}
    \begin{split}
      M_2(j\omega)&=L(j\omega)C_s(j\omega)\Big(R(j\omega)-D_r\Big)D(\omega,T)\\
     &=\Big(E_2(\omega)+jF_2(\omega)\Big)\Big(\cos{(\omega T)}-j\sin{(\omega T)}\Big)\\
     &=\Big(E_2(\omega)\cos{(\omega T)}+F_2(\omega)\sin{(\omega T)}\Big)\\
    &+j\Big(F_2(\omega)\cos{(\omega T)}-E_2(\omega)\sin{(\omega T)}\Big)\\
    &=E_3(\omega)+jF_3(\omega),
     \end{split}
\end{equation}
where $E_3(\omega)$ and $F_3(\omega)$ are $\mathcal{Z}_L$ functions.

For $M_3(\omega)$, first similar to what has been calculated in (\ref{eq: a1b1}) and (\ref{eq: E1E2}), we have
 \begin{equation}
 \label{M3'}
     \begin{split}
         &1+L(j\omega) D(\omega,T)\Big(C_3(j\omega)+D_r\Big)\\
         &=1+E_4(\omega)+jF_4(\omega),
     \end{split}
 \end{equation}
 where $E_4(\omega)$ and $F_4(\omega)$ are also $\mathcal{Z}_L$ functions. Then, considering
 \begin{equation}
     \begin{split}
     \label{eq: M''''}
         M_3(j\omega)&=\bigg(1+L(j\omega) D(\omega,T)\Big(C_3(j\omega)+D_r\Big)\bigg)...\\
         &\Bigl(R(j\omega)-D_r\Bigl),
              \end{split}
 \end{equation}
 and replacing \eqref{eq: M2'} and \eqref{M3'} in \eqref{eq: M''''}, results in
 \begin{equation}
     \begin{split}
     \label{eq: M'''''}
         M_3(j\omega)&=\Big(1+E_4(\omega)+jF_4(\omega)\Big)\Big(a_3(\omega)-jb_3(\omega)\Big)\\
         &=\bigg(a_3(\omega)+E_4(\omega)a_3(\omega)+F_4(\omega
         )b_3(\omega)\bigg)\\
         &+j\bigg(-b_3(\omega)-E_4(\omega)b_3(\omega)+F_4(\omega) a_3(\omega)\bigg)\\
         &=\Big(a_3(\omega)+E_5(\omega)\Big)-j\Big(b_3(\omega)+F_5(\omega)\Big),
     \end{split}
 \end{equation}
where $E_5(\omega)$ and $F_5(\omega)$ are $\mathcal{Z}_L$ functions.

%, $M_3(j\omega)$ in (\ref{eq: M'''''}) can be rewritten as
%\begin{equation}
%    M_3(j\omega)\simeq a_3(\omega)-jb_3(\omega).
%\end{equation}
Now we are able to calculate $\mathcal{N}_x (\omega)$ and $\mathcal{N}_y (\omega)$. From (\ref{eq.NSV}), we have
\begin{equation}
    \mathcal{N}_x(\omega)=\mathfrak{R}(M_1^{*}(j\omega)M_2(j\omega)),
\end{equation}
where, 
\begin{equation}
\begin{split}
    &M_1^{*}(j\omega)M_2(j\omega)\\
    &=\Big(1+E_1(\omega)+jF_1(\omega)\Big)\Big(E_3(\omega)+jF_3(\omega)\Big)\\
    &=\bigg(E_3(\omega)+E_1(\omega)E_3(\omega)-F_1(\omega)F_3(\omega)\bigg)\\
         &+j\bigg(F_1(\omega)E_3(\omega)+F_3(\omega)+E_1(\omega)F_3(\omega)\bigg)\\
         &= E_6(\omega)+jF_6(\omega),
    \end{split}
    \end{equation}
which $E_6(\omega)$ and $F_6(\omega)$ are $\mathcal{Z}_L$ functions. Thus,
\begin{equation}
\begin{split}
    \mathcal{N}_x (\omega)=\mathfrak{R}\Big(E_6(\omega)+jF_6(\omega)\Big)=E_6(\omega),
    \end{split}
\end{equation}
then, it concludes that $\mathcal{N}_x (\omega)$ is always a $\mathcal{Z}_L$ function.

%regarding that $E_1(\omega)$, $F_1(\omega)$, $E_3(\omega)$, and $F_3(\omega)$ are $\mathcal{Z}_L$ functions, $ \mathcal{N}_x (\omega)$ is also a $\mathcal{Z}_L$ function and oscillating around $\mathcal{N}_x=0$ line for $\omega\gg0$. \\
For $\mathcal{N}_y (\omega)$, we have
\begin{equation}
    \mathcal{N}_y (\omega)=\mathfrak{R}(M_1^{*}(j\omega)M_3(j\omega)),
\end{equation}
where, 
\begin{equation}
\begin{split}
    &M_1^{*}(j\omega)M_3(j\omega)\\
    &=\bigg(1+E_1(\omega)+jF_1(\omega)\bigg)\bigg(\Big(a_3(\omega)+E_5(\omega)\Big)\\
    &-j\Big(b_3(\omega)+F_5(\omega)\Big)\bigg)\\
    &=\Big(a_3(\omega)+E_5(\omega)+E_1(\omega)a_3(\omega)+E_1(\omega)E_5(\omega)\\
   &+F_1(\omega)b_3(\omega)+F_1(\omega)F_5(\omega)\Big)+j\Big(F_1(\omega)a_3(\omega)\\
   &+F_1(\omega)E_5(\omega)-b_3(\omega)-F_5(\omega)-E_1(\omega)b_3(\omega)\\
   &-E_1(\omega)F_5(\omega)\Big)\\
   &=\Big(a_3(\omega)+E_7(\omega)\Big)+j\Big(-b_3(\omega)+F_7(\omega)\Big),
    \end{split}
\end{equation}
where $E_7(\omega)$ and $F_7(\omega)$ are $\mathcal{Z}_L$ functions. Therefore,
\begin{equation}
\begin{split}
    \mathcal{N}_y (\omega)=\mathfrak{R}\Big(M_1^{*}(j\omega)M_3(j\omega)\Big)=a_3(\omega)+E_7(\omega),
    \end{split}
\end{equation}
by replacing $a_3(\omega)$ from (\ref{eq: M2'}), we have
\begin{equation}
\begin{split}
    \mathcal{N}_y (\omega)=\frac{\omega_k \omega_r}{\omega^2+\omega_r ^{2}}+E_7(\omega).
    \end{split}
\end{equation}
Therefore, for the case $C_R=\text{CI}$ where $\omega_r= 0$, $\mathcal{N}_y (\omega)=E_7(\omega)$, which is a $\mathcal{Z}_L$ function. Thus, based on Lemma \ref{NxNyZl}, it is not possible to determine the stability of the reset control system \eqref{eq.SS closed loop} in the presence of delay by using Theorem \ref{lem:Hb frequency}.
However, in the case where $\omega_r\neq 0$ ($C_R=\text{GFORE}$), $\mathcal{N}_y$ is not a $\mathcal{Z}_L$ function and it is possible to apply Theorem \ref{lem:Hb frequency} to demonstrate the stability of a reset control system with GFORE element. To enhance comprehension, Fig. \ref{fig:NSV delay} depicts an example where both $\mathcal{N}_x$ and $\mathcal{N}_y$ are $\mathcal{Z}_L$ functions ($\omega_r= 0$), as well as a scenario where only $\mathcal{N}_x$ is a $\mathcal{Z}_L$ function ($\omega_r\neq 0$).
\end{proof}

\begin{comment}
\begin{figure}
		\centering
		\begin{subfigure}[b]{0.5\textwidth}
			\centering
			\includegraphics[width=0.65\columnwidth]{Figures/wr_0_v5.eps}
			\caption{$\omega_r\neq0$}
			\label{fig:Nx=0}
		\end{subfigure}%
		
		~ %add desired spacing between images, e. g. ~, \quad, \qquad etc.
		%(or a blank line to force the subfigure onto a new line)
\begin{subfigure}[b]{0.5\textwidth}
			\centering
			\includegraphics[width=0.65\columnwidth]{Figures/wr_neq0_v5.eps}
   \centering
			\caption{$\omega_r=0$}
			\label{fig:nxny=0}
		\end{subfigure}
		%\setlength{\belowcaptionskip}{-15pt}
		%\setlength{\abovecaptionskip}{-10pt}
		\caption{NSV plot ($\stackrel{\rightarrow}{\mathcal{N}}(\omega)$) in the presence of delay at high frequencies.}
		\label{fig:NSV delay}
	\end{figure}
\end{comment}


    \begin{figure}[!t]
\centering
\subfloat[]{\includegraphics[width=0.5\columnwidth]{Figures/wr_0_v6.pdf}%
\label{fig:Nx=0}}
%\hfil
\subfloat[]{\includegraphics[width=0.5\columnwidth]{Figures/wr_neq0_v6.pdf}}%
\label{fig:nxny=0}
\caption{NSV plot ($\stackrel{\rightarrow}{\mathcal{N}}(\omega)$) in the presence of delay at high frequencies ($\omega \rightarrow \infty$), for (a) $\omega_r\neq 0$, and (b) $\omega_r$= 0.}
\label{fig:NSV delay}
\end{figure}