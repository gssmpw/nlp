\section{FRF based $H_\beta$ method for parallel reset control systems } \label{sec: main results}
In this section, we present our main result in the form of a theorem. Our objective is to employ the $H_\beta$ method in order to demonstrate the stability of the system described in (\ref{eq.SS closed loop}) in the frequency domain. We aim to transform the conditions specified by the $H_\beta$ method in such a manner that allows us to fulfill them by utilizing the measured FRF instead of the state-space model of the plant.
%\subsection{From state-space to transfer function}
Regarding the transfer function in (\ref{eq.H beta}), it can be observed that it is not possible to directly utilize the measured FRF to assess the stability mentioned in Theorem \ref{Th. Theorem1} and Lemma \ref{lemma UBIBS}, because matrix $\bar{A}$ should be known. To this respect, in Lemma \ref{lem: Hb transfer}, the $H_\beta$ transfer function is rewritten.\\

\begin{lemma}
\label{lem: Hb transfer}
    Under Assumption \ref{ass: assumption no direct feed through}, the transfer function in (\ref{eq.H beta}), can be rewritten as follows
    \begin{equation}
    \label{eq: H_b FRF}
    \resizebox{1\hsize}{!}{
    $H_\beta(s)=\frac{\beta^{'}L(s)C_s(s)\Big(R(s)-D_r\Big)+\varrho^{'}\bigg(1+L(s)\Big(C_3(s)+D_r\Big)\bigg)\Bigl(R(s)-D_r\Bigl)}{1+L(s)\Bigl(R(s)+C_3(s)\Bigl)},$
    }
    \end{equation}
    where $L(s)=C_1(s)C_2(s)G(s)$, $\beta^{'}=\frac{-\beta}{B_r}$, and $\varrho^{'}=\frac{\varrho}{C_r B_r}$, which $C_r \in \mathbb{R}$ and $B_r \in \mathbb{R}$ are the one-dimensional input and output matrix of the reset element $C_R$ with $B_r C_r >0$.\\
    \textbf{ Proof.} Appendix \ref{App: I}\\
\end{lemma}

Regarding the Theorem \ref{Th. Theorem1} it is crucial to determine the sign of the real part of $H_\beta(s)$ to prove SPRness. Thus, the Nyquist Stability Vector (NSV) is given by the following definition.\\
%based on step 1 in \hyperref[App: th2]{Appendix.~\ref*{App: th2}}.  

\begin{definition}
\label{def:NSV}
    For the transfer function in (\ref{eq: H_b FRF}), the NSV is defined as below ($\forall \,\omega \in \mathbb{R}$)
    \begin{equation}
    \begin{split}
    \stackrel{\rightarrow}{\mathcal{N}}(\omega)=&\begin{bmatrix}
        \mathcal{N}_x (\omega) & \mathcal{N}_y (\omega)
    \end{bmatrix}^T \\
    = &\begin{bmatrix}
        \mathfrak{R}(M_1^{*}(j\omega)M_2(j\omega)) & \mathfrak{R}(M_1^{*}(j\omega)M_3(j\omega))
    \end{bmatrix}^T,
    \label{eq.NSV}
    \end{split}
\end{equation}
where $ \mathfrak{R}(.)$ means the real part, $(.)^*$ means the complex conjugate, $j=\sqrt{-1}$, and
\begin{equation}
    \begin{split}
        M_1(j\omega)&=1+L(j\omega)\Bigl(R(j\omega)+C_3(j\omega)\Bigl), \\
        M_2(j\omega)&=L(j\omega)C_s(j\omega)\Big(R(j\omega)-D_r\Big), \\
        M_3(j\omega)&=\bigg(1+L(j\omega)\Big(C_3(j\omega)+D_r\Big)\bigg)\Bigl(R(j\omega)-D_r\Bigl).
    \end{split}
    \label{eq.M}
\end{equation}\\
\end{definition}

As previously stated, we aim to employ FRF-based conditions instead of matrix-based LMIs. Therefore, the following definitions are introduced as a foundation for the main theorem.
\begin{definition}
    \label{def:Types}
     We define $\theta_{\mathcal{N}}(\omega)=\phase{\overrightarrow{\mathcal{N}}(\omega)}$ where $\phase{\overrightarrow{\mathcal{N}}(\omega)}=\mathrm{tan}^{-1}\big({\frac{\mathcal{N}_y (\omega)}{\mathcal{N}_x (\omega)}}\big) \, \forall \,\omega \in \mathbb{R}$, and also
    \begin{equation}
\theta_1=\smash{\displaystyle\min_{\forall\omega \in\mathbb{R}^{}}}\theta_{\mathcal{N}}(\omega),  \quad \theta_2=\smash{\displaystyle\max_{\forall\omega \in\mathbb{R}^{}}}\theta_{\mathcal{N}}(\omega).
\label{eq: teta}
    \end{equation}
    In this paper for simplicity it is considered $\theta_{\mathcal{N}}(\omega)\in [-\frac{\pi}{2},  \frac{3\pi}{2})$.\\
\end{definition}

\begin{definition}
    \label{eq: LCs}
    The transfer functions $L(s)C_s(s)$, is defined as:
\begin{equation}
\label{eq: L(s)Cs}
     L(s)C_s(s)=\frac{K_{m} s^{m} + K_{m-1} s^{m-1} + \ldots + K_{m_0}}{K_{n} s^{n} + K_{n-1} s^{n-1} + \ldots + K_{n_0}}.
\end{equation}\\
\end{definition}
\begin{comment}
\begin{definition}
    \label{eq: LCs}
    Transfer functions $C_s(s)$ and $L(s)C_s(s)$, are defined as:
    \begin{equation}
\label{eq:C_s}
    C_s(s) = \frac{K_{m'} s^{m'} + K_{m'-1} s^{m'-1} + \ldots + K_{m'_0}}{K_{n'} s^{n'} + K_{n'-1} s^{n'-1} + \ldots + 1},
\end{equation}
and,
\begin{equation}
     L(s)C_s(s)=\frac{K_{m} s^{m} + K_{m-1} s^{m-1} + \ldots + K_{m_0}}{K_{n} s^{n} + K_{n-1} s^{n-1} + \ldots + 1}.
\end{equation}\\
\end{definition}
\end{comment}
To be able to use Lemma \ref{lemma UBIBS} and Lemma \ref{convrgence} in the frequency domain, it is necessary to convert the conditions stated in Theorem \ref{Th. Theorem1} into frequency domain-based conditions. Therefore, with the presented lemmas and definitions, Theorem \ref{lem:Hb frequency} is introduced to establish the same property as stated in Theorem \ref{Th. Theorem1}, with conditions exclusively in the frequency domain.\\

\begin{theorem}
\label{lem:Hb frequency}
   The zero equilibrium of the reset control system (\ref{eq.SS closed loop}) satisfying the Assumption \ref{ass: assumption no direct feed through}, with $w(t) = 0$ is globally uniformly asymptotically stable if all the conditions listed below are satisfied
   \begin{itemize}
    \item The base linear system is stable, and its open loop transfer function does not have any unstable pole-zero cancellation.
    \item The shaping filter $C_s(s)$ is proper and stable.
    \item $-1<\gamma<1$.
    \item $B_r C_r >0$.
    \item $(\theta_2-\theta_1)<\pi$.
     \end{itemize}
     In the case of $\omega_r \neq 0$ (GFORE)
    \begin{itemize}
        \item $-\frac{\pi}{2}<\theta_{\mathcal{N}} (\omega)<\pi$ \text{and/or} $0<\theta_{\mathcal{N}} (\omega)<\frac{3\pi}{2}$, \quad for all $\omega \in [0,\infty)$.
    \end{itemize}
    In the case of $\omega_r=0$ (CI)
    \begin{itemize}
        \item The relative degree of the transfer function $L(s)$ must be 1.
         \item If $\lim_{s \to \infty} \operatorname{phase}\left(L(s) C_s(s)\right) = -90$ ($\frac{K_{n}}{K_{m}} > 0$), then $0 < \theta_{\mathcal{N}}(\omega) < \frac{3\pi}{2}$ for all $\omega \in [0,\infty)$.
    \item If $\lim_{s \to \infty} \operatorname{phase}\left(L(s) C_s(s)\right) = -270$ ($\frac{K_{n}}{K_{m}} < 0$), then $-\frac{\pi}{2} < \theta_{\mathcal{N}}(\omega) < \pi$ for all $\omega \in [0,\infty)$.
   \end{itemize}
   \textit{\textbf{Proof.}} Appendix \ref{App: II}\\
\end{theorem}

\begin{corollary}
\label{col: new convergence}
Based on Lemma \ref{convrgence}, suppose that $w$ is a Bohl function and Theorem \ref{lem:Hb frequency} holds. Then, the reset control system \eqref{eq.SS closed loop} with zero initial condition and $C_u=C_e$ and $D_u=D_e$ is uniformly exponentially convergent.\\
\end{corollary}

Suppose that the base linear system is stable and $-1 < \gamma < 1$. Then, according to Theorem \ref{lem:Hb frequency}, to investigate the stability of a reset control system, it is only necessary to plot $\theta_{\mathcal{N}}(\omega)$ and check the conditions associated with it. This process only requires the measured FRF of the plant. In this regard, with respect to Lemma \ref{lemma UBIBS} and Lemma \ref{convrgence}, the UBIBS and uniformly exponentially convergent property of RCS in \eqref{eq.SS closed loop} can also be achieved in the frequency domain by satisfying the conditions in Theorem \ref{lem:Hb frequency} rather than those in Theorem \ref{Th. Theorem1}.

Furthermore, it can be observed that the final condition in the case of $\omega_r=0$ is overly conservative, requiring the sum of the relative degrees of the plant, pre-filter, and post-filter to be 1. This conservatism makes the use of this method for reset controllers with CI nearly impossible. In other words, designing the controller based on the GFORE element is the preferred choice, as it offers advantages in terms of both performance \cite{LukeIFAC2024} and stability.\\

\begin{comment}
\begin{remark}
    \label{delay_exp}
According to Lemma \ref{lem: SPR}, demonstrating the SPRness of the $H_\beta$ transfer function for a system with delay is not feasible. Hence, in this study, we approximate the system delay using the Pade approximation with a sufficient number of stable poles. This approximation is then integrated into the dynamics of the system's linear components, as described in \eqref{eq.SS linear}.
\end{remark}
\end{comment}
\begin{remark}
    \label{delay_exp}
To analytically analyze the effect of delay on the stability conditions in Theorem \ref{lem:Hb frequency} while using different reset elements, we approximate the system delay using the Pade approximation. This is done with a sufficient number of stable poles to ensure a Hurwitz $H_\beta$ transfer function based on Lemma \ref{lem: SPR}. This approximation is then integrated into the dynamics of the linear components of the system, as described in \eqref{eq.SS linear}.
\end{remark}