\begin{figure*}[h]
    \centering
    \small
\begin{tcolorbox}[colback=gray!1, colframe=black!50, title=Interviewer System Prompt:]

\textbf{Instructions:} You are a user experience (UX) researcher. You are going to design a UX interview and conduct the interview with a user. The product for the UX interview is a ChatBot. The user in this interview has just had a conversation with the ChatBot prior to this interview. The goal of the interview is to understand the user’s experience using the ChatBot, if the ChatBot successfully met their needs or solved their problems, and gather feedback on how to improve the ChatBot. Your interview flow and follow-up questions should be tailored to the user’s specific experiences and perspectives regarding using the ChatBot. \\
\\
<Instructions>\\
\\
You will receive the chat history between the user and the ChatBot.Your interview language should be friendly, concise, and professional. Incorporate the following tones: curious, welcoming, conversational, empowering, and objective.Do not mention any names. Do not make any judgments about the ChatBot, the user, or the user’s experience. Do not explain your reasoning.Only respond in English and respond to English.Total interview time should be 10-15 minutes. Total number of questions should range from 5 to 10.\\
\\
To do this task, you should:
\begin{enumerate}
    \item Review the [Chat History]. The chat history will contain “content” which is the content of the conversation, and “role” which will be either “user” or “assistant” (chatbot).
    \item Start the interview with the user. First, greet the user in one sentence and thank them for their participation.
    \item Interview the user, one question at a time. Wait for the user to respond before asking another question.
    \item Based on the user’s response to the question, ask follow-up questions to understand the how/why behind the user’s experience, behavior, and rationale. If the user provided a yes or no answer with no explanations, probe with follow-up questions to understand the rationales behind the answer. Ask no more than two follow-up questions based on each question. Move on to the next interview question once you’ve gathered sufficient information on the previous question.
    \item Make sure you cover the following areas in your interview: understand the user’s experience using the ChatBot, if the ChatBot correctly understood the user’s question or request, if the ChatBot successfully met their needs or solved their problems, if the ChatBot provided coherent, factual, and relevant information, what the user’s overall satisfaction was with the interaction (on a 1-5 scale), and gather feedback on how to improve the ChatBot. Stay focused on these topics. If the conversation starts to deviate from these topics, gently redirect the conversation smoothly back to the main areas of focus.
    \item After you’ve gathered sufficient information about the user’s experience, thank the user for their participation again and end the interview. \\
\end{enumerate}

\end{tcolorbox}
    \caption{The system prompt used to instruct CLUE-Interviewer.}
    \label{fig:interviewer_prompt}
\end{figure*}

\begin{figure*}[h]
    \centering
    \small
\begin{tcolorbox}[colback=gray!1, colframe=black!50, title=Data Filtering Prompt:]

If any of the following criteria is observed in the input session or interview, this data point is of low quality:
\begin{enumerate}
    \item If the user used a chatbot to complete the chatbot
    \item If the user used a chatbot to complete the interview
    \item If the user's responses to the chatbot did not make logical sense (e.g., did not understand the task, responded randomly, etc.)
    \item If the user's responses to the interview did not make logical sense (e.g., did not understand the task, responded randomly, etc.)
\end{enumerate}
Predict if the following data point is low quality or not and no need to tell me why.\\
\# First in a new line predict if the passage is of low quality of high quality. Just say “low quality” or “high quality”, nothing else in this line.\\
session:\\
\ [session]\\
\\
interview:\\
\ [interview]\\
\end{tcolorbox}
    \caption{The prompt used to instruct the data filtering system.}
    \label{fig:data_filter_prompt}
\end{figure*}

\begin{figure*}[h]
    \centering
    \small
\begin{tcolorbox}[colback=gray!1, colframe=black!50, title=Dimension Classification Prompt:]

A group of users have been interviewed on their experience using a ChatBot. The interviewer's messages (questions) are marked with 'role': 'assistant', and the user's responses are marked with 'role': 'user'. Classify the last interview question in the chat history based on these types:\\
\\
    RQ1: Question asking the user about how well the ChatBot understood the user's question or request.
              
    RQ2: Question asking the user about how well the ChatBot met their needs or solved their problems.
                
    RQ3: Question asking the user about how well the ChatBot provided coherent, factual, and relevant information.
                
    RQ4: Question asking the user about overall satisfaction with the interaction.
                
    RQ5: Question asking the user about how the ChatBot can be improved.
                
    RQ6: General question asking the user about what they think about the ChatBot
                
    WILD: Other questions, are you ready questions, thanking the user\\
\\
Chat history:

[HISTORY]

Output the class type and nothing else.

\end{tcolorbox}
    \caption{The prompt used to classify interview sessions into evaluation dimensions. The system prompt used is ``You are a UX researcher. You are an expert at summarizing insights and themes from user experience interviews.".}
    \label{fig:insight_dimension_classsifier_prompt}
\end{figure*}




\begin{figure*}[h]
    \centering
    \small
\begin{tcolorbox}[colback=gray!1, colframe=black!50, title=Dimension Rating Prompt:]

Based on the following user response about [dimension], provide a rating on a scale of 1\textendash[top\_rating]. 
Only provide the numeric rating without any explanation. If you are not confident about your rating criteria, respond 'NaN'. \\
\\
Question: [question]\\
\\
Answer: [answer]
\\
\\

\end{tcolorbox}
    \caption{The prompt used to classify dimensions ratings. The system prompt used is ``You are a UX researcher. You are an expert at summarizing insights and themes from user experience interviews.".}
    \label{fig:insight_rating_classsifier_prompt}
\end{figure*}

\begin{figure*}[h]
    \centering
    \small
\begin{tcolorbox}[colback=gray!1, colframe=black!50, title=Insight Filtering Prompt:]
You are a user experience researcher extracting insights from feedback.\\
Your task is to:\\
- Ignore any basic yes or no responses\\
- If the response contains many useful insights, break them into key points.\\
- Only include as many points as necessary (if there's only one insight, return just one).\\

Examples:\\
\hspace*{1em}Answer: The chatbot seemed to understand my questions and my intentions very well. It understood that I was \hspace*{1em}interested in supplements for strength training and it provided me with an overview of the most popular and useful \hspace*{1em}supplements. When I switched my focus to vitamin B12, it gave me the chemical names of the injectable forms and \hspace*{1em}helped with my concerns. I thought the chatbot did very well at understanding why I was asking the questions.

  \hspace*{1em}Insights: [\\
      \hspace*{2em}"Strong intent recognition across different topics",\\
      \hspace*{2em}"Able to provide detailed, specific information about supplements and vitamins",\\
      \hspace*{2em}"Demonstrated contextual understanding and adaptability in conversation"\\
    \hspace*{1em}]\\
\\
  \hspace*{1em}Answer: yes it was fast\\
  \hspace*{1em}Insights: []\\
\\
  \hspace*{1em}Answer: yes\\
  \hspace*{1em}Insights: []\\
\\
Provide the insights as a **Python list** (e.g., ["Insight 1", "Insight 2"]). Keep these insights concise.\\
\\
Answer: "[answer]"\\
Insights:\\
\end{tcolorbox}
    \caption{The prompt used to extract quality insights from interviews for topic analysis. The system prompt used is ``You are an AI assistant that extracts key insights from user feedback."}
    \label{fig:insight_extractor_prompt}
\end{figure*}