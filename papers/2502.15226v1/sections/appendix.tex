\newpage

\begin{figure*}
\centering
    \includegraphics[width=0.24\textwidth]{images/explicit_understanding.png}
    \includegraphics[width=0.24\textwidth]{images/explicit_meetneed.png}
    \includegraphics[width=0.24\textwidth]{images/explicit_credibility.png}
    \includegraphics[width=0.24\textwidth]{images/explicit_general.png}
    \caption{Scatter plots between 5-run mean dimension scores automatically extracted from user responses and explicit ratings given by users in the interview.\label{fig:scatter_coor}}
\end{figure*}

\begin{figure*}[t!]
    \centering
    \begin{subfigure}[]{\linewidth}
         \includegraphics[width=1\linewidth]{images/chatbot-ui.png}
    \caption{Chat with target LLMs.}
    \label{fig:target_llm_ui}
    \end{subfigure}
        \begin{subfigure}[]{\linewidth}
         \includegraphics[width=1\linewidth]{images/interviewer-ui.png}
    \caption{\method-Interviews.}
    \label{fig:clue_interviewer_ui}
    \end{subfigure}
    \caption{An example of the UIs used for the hosted LLM chat and \method-Interviewer.}\label{fig:clue_ui}
\end{figure*}



\begin{figure*}[h]
    \centering
    \small
\begin{tcolorbox}[colback=gray!1, colframe=black!50, title=Interviewer System Prompt:]

\textbf{Instructions:} You are a user experience (UX) researcher. You are going to design a UX interview and conduct the interview with a user. The product for the UX interview is a ChatBot. The user in this interview has just had a conversation with the ChatBot prior to this interview. The goal of the interview is to understand the user’s experience using the ChatBot, if the ChatBot successfully met their needs or solved their problems, and gather feedback on how to improve the ChatBot. Your interview flow and follow-up questions should be tailored to the user’s specific experiences and perspectives regarding using the ChatBot. \\
\\
<Instructions>\\
\\
You will receive the chat history between the user and the ChatBot.Your interview language should be friendly, concise, and professional. Incorporate the following tones: curious, welcoming, conversational, empowering, and objective.Do not mention any names. Do not make any judgments about the ChatBot, the user, or the user’s experience. Do not explain your reasoning.Only respond in English and respond to English.Total interview time should be 10-15 minutes. Total number of questions should range from 5 to 10.\\
\\
To do this task, you should:
\begin{enumerate}
    \item Review the [Chat History]. The chat history will contain “content” which is the content of the conversation, and “role” which will be either “user” or “assistant” (chatbot).
    \item Start the interview with the user. First, greet the user in one sentence and thank them for their participation.
    \item Interview the user, one question at a time. Wait for the user to respond before asking another question.
    \item Based on the user’s response to the question, ask follow-up questions to understand the how/why behind the user’s experience, behavior, and rationale. If the user provided a yes or no answer with no explanations, probe with follow-up questions to understand the rationales behind the answer. Ask no more than two follow-up questions based on each question. Move on to the next interview question once you’ve gathered sufficient information on the previous question.
    \item Make sure you cover the following areas in your interview: understand the user’s experience using the ChatBot, if the ChatBot correctly understood the user’s question or request, if the ChatBot successfully met their needs or solved their problems, if the ChatBot provided coherent, factual, and relevant information, what the user’s overall satisfaction was with the interaction (on a 1-5 scale), and gather feedback on how to improve the ChatBot. Stay focused on these topics. If the conversation starts to deviate from these topics, gently redirect the conversation smoothly back to the main areas of focus.
    \item After you’ve gathered sufficient information about the user’s experience, thank the user for their participation again and end the interview. \\
\end{enumerate}

\end{tcolorbox}
    \caption{The system prompt used to instruct CLUE-Interviewer.}
    \label{fig:interviewer_prompt}
\end{figure*}

\begin{figure*}[h]
    \centering
    \small
\begin{tcolorbox}[colback=gray!1, colframe=black!50, title=Data Filtering Prompt:]

If any of the following criteria is observed in the input session or interview, this data point is of low quality:
\begin{enumerate}
    \item If the user used a chatbot to complete the chatbot
    \item If the user used a chatbot to complete the interview
    \item If the user's responses to the chatbot did not make logical sense (e.g., did not understand the task, responded randomly, etc.)
    \item If the user's responses to the interview did not make logical sense (e.g., did not understand the task, responded randomly, etc.)
\end{enumerate}
Predict if the following data point is low quality or not and no need to tell me why.\\
\# First in a new line predict if the passage is of low quality of high quality. Just say “low quality” or “high quality”, nothing else in this line.\\
session:\\
\ [session]\\
\\
interview:\\
\ [interview]\\
\end{tcolorbox}
    \caption{The prompt used to instruct the data filtering system.}
    \label{fig:data_filter_prompt}
\end{figure*}

\begin{figure*}[h]
    \centering
    \small
\begin{tcolorbox}[colback=gray!1, colframe=black!50, title=Dimension Classification Prompt:]

A group of users have been interviewed on their experience using a ChatBot. The interviewer's messages (questions) are marked with 'role': 'assistant', and the user's responses are marked with 'role': 'user'. Classify the last interview question in the chat history based on these types:\\
\\
    RQ1: Question asking the user about how well the ChatBot understood the user's question or request.
              
    RQ2: Question asking the user about how well the ChatBot met their needs or solved their problems.
                
    RQ3: Question asking the user about how well the ChatBot provided coherent, factual, and relevant information.
                
    RQ4: Question asking the user about overall satisfaction with the interaction.
                
    RQ5: Question asking the user about how the ChatBot can be improved.
                
    RQ6: General question asking the user about what they think about the ChatBot
                
    WILD: Other questions, are you ready questions, thanking the user\\
\\
Chat history:

[HISTORY]

Output the class type and nothing else.

\end{tcolorbox}
    \caption{The prompt used to classify interview sessions into evaluation dimensions. The system prompt used is ``You are a UX researcher. You are an expert at summarizing insights and themes from user experience interviews.".}
    \label{fig:insight_dimension_classsifier_prompt}
\end{figure*}




\begin{figure*}[h]
    \centering
    \small
\begin{tcolorbox}[colback=gray!1, colframe=black!50, title=Dimension Rating Prompt:]

Based on the following user response about [dimension], provide a rating on a scale of 1\textendash[top\_rating]. 
Only provide the numeric rating without any explanation. If you are not confident about your rating criteria, respond 'NaN'. \\
\\
Question: [question]\\
\\
Answer: [answer]
\\
\\

\end{tcolorbox}
    \caption{The prompt used to classify dimensions ratings. The system prompt used is ``You are a UX researcher. You are an expert at summarizing insights and themes from user experience interviews.".}
    \label{fig:insight_rating_classsifier_prompt}
\end{figure*}

\begin{figure*}[h]
    \centering
    \small
\begin{tcolorbox}[colback=gray!1, colframe=black!50, title=Insight Filtering Prompt:]
You are a user experience researcher extracting insights from feedback.\\
Your task is to:\\
- Ignore any basic yes or no responses\\
- If the response contains many useful insights, break them into key points.\\
- Only include as many points as necessary (if there's only one insight, return just one).\\

Examples:\\
\hspace*{1em}Answer: The chatbot seemed to understand my questions and my intentions very well. It understood that I was \hspace*{1em}interested in supplements for strength training and it provided me with an overview of the most popular and useful \hspace*{1em}supplements. When I switched my focus to vitamin B12, it gave me the chemical names of the injectable forms and \hspace*{1em}helped with my concerns. I thought the chatbot did very well at understanding why I was asking the questions.

  \hspace*{1em}Insights: [\\
      \hspace*{2em}"Strong intent recognition across different topics",\\
      \hspace*{2em}"Able to provide detailed, specific information about supplements and vitamins",\\
      \hspace*{2em}"Demonstrated contextual understanding and adaptability in conversation"\\
    \hspace*{1em}]\\
\\
  \hspace*{1em}Answer: yes it was fast\\
  \hspace*{1em}Insights: []\\
\\
  \hspace*{1em}Answer: yes\\
  \hspace*{1em}Insights: []\\
\\
Provide the insights as a **Python list** (e.g., ["Insight 1", "Insight 2"]). Keep these insights concise.\\
\\
Answer: "[answer]"\\
Insights:\\
\end{tcolorbox}
    \caption{The prompt used to extract quality insights from interviews for topic analysis. The system prompt used is ``You are an AI assistant that extracts key insights from user feedback."}
    \label{fig:insight_extractor_prompt}
\end{figure*}

\section{Appendix}
We provide more details about the implementation of \method{}, our user study, and additional results.

\subsection{More Implementation Details}
\label{app.implementation}

\textbf{CLUE System UI.} \method{} system UI examples can be found in Figures \ref{fig:clue_ui}. Figure \ref{fig:target_llm_ui} shows the chat interface with the target LLM. Figure \ref{fig:clue_interviewer_ui} shows the user interface of \method-Interviewer.
We design the UI to resemble mainstream LLM interfaces.


\textbf{Interviewer Prompts.} The full prompt used in the CLUE-Interviewer system can be seen in Figure~\ref{fig:interviewer_prompt}. In the prompt, we instruct the LLM to serve as a UX researcher and conduct an interview with a user who had just chatted with a chatbot. Via step-by-step instructions, we provided the specific evaluation dimensions to cover and encouraged follow-up questions to be asked. 

In an earlier version of the interviewer development, we encouraged the interviewer to probe users for multiple rounds. Sometimes the interviewers were probed too much. We limit the interviewer to ask no more than two follow-ups per question, a conservative choice as the first step towards LLM-based UX interviewers. Future research can explore a better balance of thoroughness and user experience.



\textbf{Insighter Data Filtering Prompt} is in Figure \ref{fig:data_filter_prompt}. It is a simple prompt that leverages the LLM (Claude-3.5-Sonnet) to filter out obvious noisy data. As discussed in Sec.~\ref{sec.userstudy}, the automatic filter has very high precision but is lenient in recall.

\textbf{Insighter Dimension Classification Prompt.} The prompt used to classify the interview sessions into evaluation dimensions can be seen in Figure~\ref{fig:insight_dimension_classsifier_prompt}. We simply describe each dimension to the LLM (Claude-3.5-Sonnet) and utilize its zero-shot ability for the classification.

\textbf{Insighter Rating Prompt }is in Figure~\ref{fig:insight_rating_classsifier_prompt}. 

We acknowledge that there is room to further improve the implementation of \method-Interviewer and \method-Insighter. A better prompt engineering, finetuning dedicated LLMs for our tasks, or using next generation of LLMs (e.g., GPT-5 or Claude-4), will certainly improve the performance of \method-Interviewer and \method-Insighter. Our simple design already illustrated many interesting user opinions of LLMs, and it is only the beginning.


\begin{table}[t]
\small
\centering
\caption{\label{fig:demographic_mturkers} Demographic statistics of the MTurk participants in the user study.}
\resizebox{\linewidth}{!}{
\begin{tabular}{l|rr}
\toprule
Demographic Variable & \textbf{Frequency} & \textbf{Percentage}
\\
\hline
\textbf{Gender} \\
\texttt{Male} & 688 & 58.35      \\
\texttt{Female}  & 446 & 37.83 \\
\texttt{Nonbinary / Third Gender}& 2 & 0.17 \\
\texttt{Dual Gender} & 1 & 0.08 \\
\texttt{Prefer Not to Say} & 42 & 3.56 \\ 

\hline
\textbf{Race/Ethnicity} \\
\texttt{White or Caucasian} & 1036 & 87.87 \\
\texttt{Black or African American} & 42 & 3.56 \\
\texttt{Asian} & 71 & 6.02 \\
\texttt{Native American or Alaska Native} & 8 & 0.68 \\
\texttt{Native Hawaiian or other Pacific Islander} & 4 & 0.34 \\
\texttt{Other} & 7 & 0.59 \\
\texttt{Prefer Not to Say} & 41 & 3.48 \\

\hline
\textbf{Age} \\
\texttt{18-20} & 5 & 0.42 \\
\texttt{21-29} & 326 & 27.65 \\
\texttt{30-39} & 522 & 44.27 \\
\texttt{40-49} & 154 & 13.06 \\
\texttt{50-59} & 76 & 6.45 \\
\texttt{60+} & 53 & 4.5 \\
\texttt{Prefer Not to Say} & 43 & 3.65 \\ 

\hline
\textbf{Education} \\
\texttt{High School Diploma} & 95 & 8.06 \\
\texttt{Associate's Degree} & 7 & 0.59 \\
\texttt{Bachelor's Degree} & 774 & 65.65 \\
\texttt{Master's Degree or Higher} & 246 & 20.87 \\
\texttt{Other} & 11 & 0.93 \\
\texttt{Prefer Not to Say} & 46 & 3.9 \\ 

\hline
\textbf{Marital Status} \\
\texttt{Single} & 237 & 20.10 \\
\texttt{Married} & 858 & 72.77 \\
\texttt{Divorced} & 29 & 2.46 \\
\texttt{Prefer Not to Say} & 55 & 4.66 \\
\bottomrule
\end{tabular}}
\end{table}


\textbf{Insighter Topic Analysis Filtering.} We filter some common generic responses by checking if the response provided by the user is a substring of any of the generic responses, which include responses like ``nothing much" or ``i don't know". In addition, this filter removes all responses that have less than 10 characters, as it is unlikely for an user to describe an quality insight in such brevity. For misspelled responses and other generic responses not caught by this rule-based filtering, a three-shot prompt, which is in Figure \ref{fig:insight_extractor_prompt}, filters out any remaining basic yes or no responses. This prompt also extracts a list of insights from the answer. Using this filtering process, roughly 27.06\% of the answers are kept and used for insight extraction.


\subsection{More Details of User Study}
\label{app.user_study}


\textbf{Informed Consent and Instructions.} Figure~\ref{fig:mturk_consent} shows the informed consent and instructions for the MTurk user study. Participants were informed of the study procedures, potential risks, compensation, future use of information, confidentiality, and voluntary participation. It specifically calls out that participants should not provide any personally identifiable information during this study.

\begin{figure*}[h]
    \centering
    \small
    \resizebox{0.95\textwidth}{!}{
\begin{tcolorbox}[colback=gray!1, colframe=black!50, title=MTurk User Study Informed Consent:]
This AI Chatbot Evaluation task is part of a research study conducted by XXX at XXX University. You will chat with a chatbot for 10-15 minutes and participate in a survey study about your chatbot experience.\\
\\
Purpose\\
The purpose of the research is to evaluate large language model powered chatbots.\\
\\
Procedures \\
First, chat with a chatbot for 10-15 minutes. You are encouraged to engage with the chatbot for up to 15 minutes but can choose to stop engaging with the chatbot at any point of time. Second, immediately after this chatbot interaction, you will complete a survey including questions about the chatbot experience and demographics. This study is voluntary and you can choose to stop or not answer any questions.\\
\\
Participant Requirements \\
Participation in this study is limited to individuals age 18 and older located in the United States. They must have completed at least 1000 HITs and have an approval rate of 99\% or above. Participants must have had prior chatbot experience to participate in this study.\\
\\
Risks\\
The risks and discomfort associated with participation in this study are no greater than those ordinarily encountered in daily life or during other online activities. To minimize the potential risk of a breach of confidentiality, please do not share any personally identifiable information during this study.\\
If you use your personal device for research purpose, there is a risk of incurring data charges. There might be data charges on your device if you are not connected to Wi-Fi.\\
\\
Benefits\\
There may be no personal benefit from your participation in the study. However, we will gain knowledge about the potential pros and cons of different large language models; such knowledge will be valuable to the scientific community and to the public who use these large language models for various tasks.\\
\\
Compensation \& Costs\\
You will be compensated \$5 for your participation in this 30-min study.
There will be no cost to you if you participate in this study.\\
\\
Future Use of Information\\
In the future, once we have removed all identifiable information from your data, we may use the data for our future research studies, or we may distribute the data to other researchers for their research studies.  We would do this without getting additional informed consent from you (or your legally authorized representative). Sharing of data with other researchers will only be done in such a manner that you will not be identified.\\
\\
Confidentiality\\
The data captured for the research does not include any personally identifiable information about you.  Your IP address will not be captured. The study will collect your research data through your use of Amazon MTurk and Qualtrics. These companies are not owned by XXX University. These companies will have access to the research data that you produce and any identifiable information that you share with them while using their product. Please note that XXX University does not control the Terms and Conditions of the companies or how they will use or protect any information that they collect.\\
\\
Payment Confidentiality: Payment methods, especially those facilitated by third-party vendors (such as Visa, Venmo, MTurk, Prolific, Amazon, PayPal), may require that the researchers and/or the vendor collect and use personal information (such as your first and last name, email addresses, phone numbers, banking information) provided by you in order for your payment to be processed. As with any payment transaction, there is the risk of a breach of confidentiality from the third-party vendor. All personal information collected by the researcher will be held as strictly confidential and stored in a password-protected digital file, or in a locked file cabinet, until payments are processed and reconciled. This information will be destroyed at the earliest acceptable time. Personal information held by the third-party vendor will be held according to their terms of use policy.\\
\\
Right to Ask Questions \& Contact Information\\
If you have any questions about this study, you should feel free to ask them by contacting XXX, XXX University, XXX@XXX .  If you have questions later, desire additional information, or wish to withdraw your participation please contact the Principal Investigator by e-mail. If you have questions pertaining to your rights as a research participant; or to report concerns to this study, you should contact the Office of XXX at XXX University.  Email: XXX@XXX . Phone: XXX-XXX -XXXX.\\
\\
Voluntary Participation\\
Your participation in this research is voluntary.  You may discontinue participation at any time during the research activity.  You may print a copy of this consent form for your records.\\

\end{tcolorbox}}
    \caption{Informed consent and instructions provided to MTurk user study participants}
    \label{fig:mturk_consent}
\end{figure*}

\textbf{Demographics.} Table \ref{fig:demographic_mturkers} shows the demographic statistics of the MTurk participants in our user study. Demographic variables reported included gender, race/ethnicity, age, education, and marital status. 
Our participant population is biased towards certain demographic groups, perhaps due to the population distributions of MTurkers during our study period. As a result, all the opinions reflected in this paper are from this specific population, which might be different from current user bases of mainstream LLMs.

\textbf{Payment.} All participants were paid between \$4 and \$6 for their time and participation in this study. This payment rate was determined to be above the US federal minimum wage of \$7.25 per hour. We started with \$4 but then increased to \$6 to facilitate more participation. On average, users spent a total of 25 minutes on our study, corresponding to an average hourly rate of \$12, which is significantly more than the federal minimum wage.

\subsection{Additional Results}
\label{app.add_results}

\textbf{Correlations with User Explicit Rating.} Figure~\ref{fig:scatter_coor} plots the detailed correlations between explicit rating provided by user when asked by \method-Interviewer, and the ratings automatically generated based on user interview responses. Users are significantly more lenient when asked for an explicit rating, giving a lot of perfect 5s. Their responses are more scattered, with a significant fraction of 1 (bad) and 2 (mediocre). It aligns with the findings in UX research that interview is a more effective tool to discover users' true opinions.

\textbf{Interview Examples with Human Annotations.} Figure \ref{fig:chatbot_interview} shows an example of an interview session with human annotations on evaluation dimension, probing occurrences and depth, and previous chat mentioned.

\begin{figure*}[h]
    \centering
    \small
    \resizebox{0.95\textwidth}{!}{
    \begin{tcolorbox}[colback=gray!1, colframe=black!50, title=Example User Experience Interview Session with Human Annotations]
    \textbf{Assistant:} Hello! I'm here to interview you about your experience with the chatbot you just spoke to. Are you ready?\\
    \textbf{User:} yes\\
    
    \textbf{Assistant:} \hl{Great! To start, could you share your overall experience using the ChatBot? How did you feel about the interaction?} \\
    \textit{Dimension: General}\\

    \textbf{User:} My experience with the chatbot was rather informative and engaging.\\
    \textit{Human Dimension Rating: 3 - good}\\

    \textbf{Assistant:} \hl{I'm glad to hear that! Could you tell me more about what made the interaction informative and engaging for you?}\\
    \textit{Dimension: Improvement; Probing Occurrence 1 (Depth 1)}\\

    \textbf{User:} Well it provided me with factual and historical information about the conversations we were having.\\
    \textit{Human Dimension Rating: 3 - good}\\

    \textbf{Assistant:} That's good to know. \hl{Did you feel that the ChatBot understood your questions or requests accurately throughout the conversation?}\\
    \textit{Dimension: Understand}\\

    \textbf{User:} Yes I do,\\
    \textit{Human Dimension Rating for Understand: 3 - good}\\

    \textbf{Assistant:} That's great to hear. \hl{Did the ChatBot successfully meet your needs or solve any problems you had during the conversation?}\\
    \textit{Dimension: Meet Needs}\\

    \textbf{User:} I didn't really have any problems to be solved, {We had a really good conversation about the recent and }\colorbox{teal!30!white}{rare winter storm affecting the south where I live}\\
    \textit{Dimension: Improvement; Previous chat mentioned by user}\\
    \textit{Human Dimension Rating for Meet Needs: 3 - good}\\

    \textbf{Assistant:} Thank you for sharing that. \hl{Did the ChatBot provide coherent, factual, and relevant information} about \colorbox{teal!30!white}{the winter storm and its impact on your area?}\\
    \textit{Dimension: Credibility \& Improvement; Previous chat mentioned by interviewer}\\

    \textbf{User:} Absolutely the chatbot was incredibly relevant and facutal. I was very impressed.\\
    \textit{Human Dimension Rating for Credibility: 3 - good}\\

    \textbf{Assistant:} I'm glad to hear that the information was helpful and accurate. \hl{On a scale of 1 to 5, how satisfied are you with your overall interaction with the ChatBot?}\\
    \textit{Dimension: User Explicit Rating}\\

    \textbf{User:} I would give it a 4.5\\
    \textit{Human Dimension Rating for User Explicit Rating: 4.5}\\

    \textbf{Assistant:} Thank you for your rating. \hl{Is there anything specific that could be improved to make your experience a perfect 5?}\\
    \textit{Dimension: Improvement; Probing Occurrence 2 (Depth 1)}\\

    \textbf{User:} I knew I was chatting with AI, in order for it to be a perfect 5, train AI responses to be a bit less robotic and more humanlike.\\

    \textbf{Assistant:} Thank you for that valuable feedback. \hl{It's important for interactions to feel natural and engaging. Is there anything else you'd like to add or suggest for improving the ChatBot?}\\
    \textit{Dimension: Improvement}\\

    \textbf{User:} None that I can think of at the moment.\\
    
    \textbf{Assistant:} Thank you for your insights and feedback. I appreciate your time and participation in this interview. Have a great day!\\
    
    \textbf{User:} Thank You\\
    
    \textbf{Assistant:} You're welcome! Take care!\\
        
    \end{tcolorbox}}
    \caption{Example interview session with human annotations.}
    \label{fig:chatbot_interview}
\end{figure*}