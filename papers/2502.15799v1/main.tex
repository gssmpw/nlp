\pdfoutput =1
% In particular, the hyperref package requires pdfLaTeX in order to break URLs across lines.

\documentclass[11pt]{article}

% Change "review" to "final" to generate the final (sometimes called camera-ready) version.
% Change to "preprint" to generate a non-anonymous version with page numbers.

\usepackage[dvipsnames,table]{xcolor}
\usepackage[final]{acl}

% Standard package includes
\usepackage{times}
\usepackage{latexsym}

% For proper rendering and hyphenation of words containing Latin characters (including in bib files)
\usepackage[T1]{fontenc}
% For Vietnamese characters
% \usepackage[T5]{fontenc}
% See https://www.latex-project.org/help/documentation/encguide.pdf for other character sets

% This assumes your files are encoded as UTF8
\usepackage[utf8]{inputenc}

% This is not strictly necessary, and may be commented out,
% but it will improve the layout of the manuscript,
% and will typically save some space.
\usepackage{microtype}

% This is also not strictly necessary, and may be commented out.
% However, it will improve the aesthetics of text in
% the typewriter font.
\usepackage{inconsolata}

%Including images in your LaTeX document requires adding
%additional package(s)



% Optional math commands from https://github.com/goodfeli/dlbook_notation.

\usepackage{multirow}
\usepackage{fontawesome5}
\usepackage{tcolorbox}
\usepackage{pifont}
\usepackage{graphicx}
\usepackage{longtable}
\usepackage{tabularx}

\usepackage[utf8]{inputenc} % allow utf-8 input
\usepackage[T1]{fontenc}    % use 8-bit T1 fonts


\usepackage{booktabs}       % professional-quality tables
\usepackage{amsfonts}       % blackboard math symbols
\usepackage{nicefrac}       % compact symbols for 1/2, etc.
\usepackage{microtype}      % microtypography
\usepackage{lipsum}		% Can be removed after putting your text content
\usepackage{graphicx}
\usepackage{natbib}
\usepackage{doi}
%\usepackage{xcolor}
\usepackage{amsmath}


\usepackage{listings} % for prompts
\lstset{
basicstyle=\small\ttfamily,
mathescape=true,
breaklines=true,        % Enable line breaking
breakindent=0pt,        % No indent for wrapped lines
breakautoindent=false,  % Disable automatic indentation
columns=flexible        % Better space handling
}


% \usepackage[utf8]{inputenc}
% \usepackage[russian]{babel}

\title{Investigating the Impact of Quantization Methods \\ on the Safety and Reliability of Large Language Models}



% Authors must not appear in the submitted version. They should be hidden
% as long as the \iclrfinalcopy macro remains commented out below.
% Non-anonymous submissions will be rejected without review.

\author{
 \textbf{Artyom Kharinaev$^{\clubsuit \Diamond}$ \textsuperscript{1}}, 
 \textbf{Viktor Moskvoretskii$^\clubsuit$ \textsuperscript{1,3}},
 \textbf{Egor Shvetsov\textsuperscript{1}}, \\
%  \textbf{Dmitry Osin\textsuperscript{1}},
% \\
%  \textbf{Igor Udovichenko\textsuperscript{1}},
 \textbf{Kseniia Studenikina$^{\Diamond}$},
 \textbf{Bykov Mikhail$^{\Diamond}$},
 \textbf{Evgeny Burnaev \textsuperscript{1,2}}
\\
 \textsuperscript{1} \small{Skolkovo Institute of Science and Technology} \\
 \textsuperscript{2} \small{Artificial Intelligence Research Institute} \\
 \textsuperscript{3} \small{HSE University}
 % \textsuperscript{2}Affiliation 2,
 % \textsuperscript{3}Affiliation 3,
 % \textsuperscript{4}Affiliation 4,
 % \textsuperscript{5}Affiliation 5
\\
 \small{
   \textbf{Correspondence:} \href{mailto: m.zhelnin@skol.tech}{e.shvetsov@skol.tech}
 } \\
 \small{ $\clubsuit$ indicates equal contribution.} \\ 
 \small{ $\Diamond$ indicates that the work was partially done during  \href{https://smiles.skoltech.ru/}{SMILES summer school.}}
}


% The \author macro works with any number of authors. There are two commands
% used to separate the names and addresses of multiple authors: \And and \AND.
%
% Using \And between authors leaves it to \LaTeX{} to determine where to break
% the lines. Using \AND forces a linebreak at that point. So, if \LaTeX{}
% puts 3 of 4 authors names on the first line, and the last on the second
% line, try using \AND instead of \And before the third author name.

% \newcommand{\fix}{\marginpar{FIX}}
% \newcommand{\new}{\marginpar{NEW}}

%\iclrfinalcopy % Uncomment for camera-ready version, but NOT for submission.
\begin{document}


\maketitle

\begin{abstract}
Large Language Models (LLMs) have emerged as powerful tools for addressing modern challenges and enabling practical applications. However, their computational expense remains a significant barrier to widespread adoption. Quantization has emerged as a promising technique to democratize access and enable low resource device deployment. Despite these advancements, the safety and trustworthiness of quantized models remain underexplored, as prior studies often overlook contemporary architectures and rely on overly simplistic benchmarks and evaluations.
To address this gap, we introduce \textbf{OpenSafetyMini}, a novel open-ended safety dataset designed to better distinguish between models. We evaluate 4 state-of-the-art quantization techniques across LLaMA and Mistral models using 4 benchmarks, including human evaluations. Our findings reveal that the optimal quantization method varies for 4-bit precision, while vector quantization techniques deliver the best safety and trustworthiness performance at 2-bit precision, providing foundation for future research.

\end{abstract}

\begin{table*}[!t]
\centering

%\scriptsize
 \resizebox{\textwidth}{!}{%
\begin{tabular}{llllllc}\toprule
\textbf{Paper} &\textbf{Models} &\textbf{Methods} &\textbf{Bits Range } &\textbf{Datasets} &\textbf{Evaluation}& \textbf{New Datatset} \\\midrule
\multirow{2}{*}{\citealp{li2024evaluating}} &LLaMA2-7B, LLaMA2-70B, &AWQ$^6$, SmoothQuant$^3$, &W8, W4, W3, W2, W8A8, &\textbf{Ethics}: Adversarial GLUE, & Multiple-choice questions.& \ding{55} \\
&Mistral-7B, Mixtral-8x7B &KV Cache qantization$^1$ &W4A8, W8A4, W4A4 & \textbf{Hallucinations}: TruthfulQA.& \\
\hline

\multirow{2}{*}{\citealp{liu2024evaluating}} &LLaMA2-7B &GPTQ$^4$, SpQR$^2$, AWQ$^6$,  &W2A16, W4A8, W3A8 &\textbf{Toxicity}: Implicit Hate, & Multiple-choice questions.& \ding{55} \\
 &  &SmoothQuant$^3$ &W2A16, W4A8, W3A8 & ToxiGen, BOSS.& \\
\hline

\multirow{2}{*}{\citealp{jin2024comprehensive}} &Qwen-7B-Chat, Qwen-14B-Chat, &SpQR$^2$, GPTQ$^4$,  &W8, W4, W3, W2 &\textbf{Hallucinations}: TruthfulQA, & Multiple-choice questions.& \ding{55} \\
 &Qwen-72B-Chat &LLM.int8()$^2$ & &\textbf{Social biases}: BBQ.&\\

\hline

\multirow{2}{*}{\citealp{belkhiter2024harmlevelbench}}  &Vicuna 13B &AWQ$^6$, GPTQ$^4$ & Not specified & \textbf{Safety}: HarmLevelBench. & Experts and & \ding{52} \\
 &&& &  & LLM-as-a-judge.& \\
\hline

\multirow{4}{*}{\citealp{xu2024beyond}} &LLaMa2-7B, TÜLU2-7B,& LLM.int8()$^2$, GPTQ$^4$, &W8, W4 & \textbf{Toxicity}: RealToxicityPrompts, ToxiGen, & Rule based +\\

& TÜLU2-13B  & AWQ$^6$ & & AdvPromptSet.  \textbf{Bias and Stereotypes:}   &  Model evaluation& \ding{55}    \\

&  & & & BOLD, HolisticBiasR,  BBQ.  & (OpenAI moderation API).& \\

&  & & &  \textbf{Hallucinations}: TruthfulQA.  & &\\

\hline

\multirow{2}{*}{\citealp{yang2024llmcbench}} & LLaMA2, LLaMa3-7B &GPTQ$^4$, SmoothQuant$^3$,  &W8A16, W8A8 & \textbf{Robustness}: AdvGLUE.  & Rule-based.& \ding{55} \\

& &AWQ$^6$, OmniQuant$^1$ & & \textbf{Hallucinations}: TruthfulQA.& \\

\hline

\multirow{3}{*}{\textbf{OUR}} &LLaMa3.1-8B, &AQLM$^1$, QUIK$^1$, &W4, W2 & \textbf{Safety:} XSAFETY, OpenSafetyMini, & Human Evaluation, &  \\
 &Mistral-7B v0.2,& QUIP$^1$, AWQ$^6$ & & SafetyBench.  &  multiple-choice questions,&  \ding{52} \\
 &LLaMa3 Abliterated &  & & \textbf{Hallucinations}: HotPotQA. & AlignScore, LLM as a Judge.& \\



\bottomrule
\end{tabular}}
\caption{Review of previous benchmarks in relation to safety, hallucination, and trustworthiness of quantized LLMs, including \textbf{OUR} contributions. Notation: $W[\cdot]$ - specifies precision for model weights, $A[\cdot]$ specifies precision for model activations (defaults to FP16 if unspecified). Superscript signifies in how many papers a method was evaluated.}
\label{tab:review}
\end{table*}


\section{Introduction}


\begin{figure}[t]
\centering
\includegraphics[width=0.6\columnwidth]{figures/evaluation_desiderata_V5.pdf}
\vspace{-0.5cm}
\caption{\systemName is a platform for conducting realistic evaluations of code LLMs, collecting human preferences of coding models with real users, real tasks, and in realistic environments, aimed at addressing the limitations of existing evaluations.
}
\label{fig:motivation}
\end{figure}

\begin{figure*}[t]
\centering
\includegraphics[width=\textwidth]{figures/system_design_v2.png}
\caption{We introduce \systemName, a VSCode extension to collect human preferences of code directly in a developer's IDE. \systemName enables developers to use code completions from various models. The system comprises a) the interface in the user's IDE which presents paired completions to users (left), b) a sampling strategy that picks model pairs to reduce latency (right, top), and c) a prompting scheme that allows diverse LLMs to perform code completions with high fidelity.
Users can select between the top completion (green box) using \texttt{tab} or the bottom completion (blue box) using \texttt{shift+tab}.}
\label{fig:overview}
\end{figure*}

As model capabilities improve, large language models (LLMs) are increasingly integrated into user environments and workflows.
For example, software developers code with AI in integrated developer environments (IDEs)~\citep{peng2023impact}, doctors rely on notes generated through ambient listening~\citep{oberst2024science}, and lawyers consider case evidence identified by electronic discovery systems~\citep{yang2024beyond}.
Increasing deployment of models in productivity tools demands evaluation that more closely reflects real-world circumstances~\citep{hutchinson2022evaluation, saxon2024benchmarks, kapoor2024ai}.
While newer benchmarks and live platforms incorporate human feedback to capture real-world usage, they almost exclusively focus on evaluating LLMs in chat conversations~\citep{zheng2023judging,dubois2023alpacafarm,chiang2024chatbot, kirk2024the}.
Model evaluation must move beyond chat-based interactions and into specialized user environments.



 

In this work, we focus on evaluating LLM-based coding assistants. 
Despite the popularity of these tools---millions of developers use Github Copilot~\citep{Copilot}---existing
evaluations of the coding capabilities of new models exhibit multiple limitations (Figure~\ref{fig:motivation}, bottom).
Traditional ML benchmarks evaluate LLM capabilities by measuring how well a model can complete static, interview-style coding tasks~\citep{chen2021evaluating,austin2021program,jain2024livecodebench, white2024livebench} and lack \emph{real users}. 
User studies recruit real users to evaluate the effectiveness of LLMs as coding assistants, but are often limited to simple programming tasks as opposed to \emph{real tasks}~\citep{vaithilingam2022expectation,ross2023programmer, mozannar2024realhumaneval}.
Recent efforts to collect human feedback such as Chatbot Arena~\citep{chiang2024chatbot} are still removed from a \emph{realistic environment}, resulting in users and data that deviate from typical software development processes.
We introduce \systemName to address these limitations (Figure~\ref{fig:motivation}, top), and we describe our three main contributions below.


\textbf{We deploy \systemName in-the-wild to collect human preferences on code.} 
\systemName is a Visual Studio Code extension, collecting preferences directly in a developer's IDE within their actual workflow (Figure~\ref{fig:overview}).
\systemName provides developers with code completions, akin to the type of support provided by Github Copilot~\citep{Copilot}. 
Over the past 3 months, \systemName has served over~\completions suggestions from 10 state-of-the-art LLMs, 
gathering \sampleCount~votes from \userCount~users.
To collect user preferences,
\systemName presents a novel interface that shows users paired code completions from two different LLMs, which are determined based on a sampling strategy that aims to 
mitigate latency while preserving coverage across model comparisons.
Additionally, we devise a prompting scheme that allows a diverse set of models to perform code completions with high fidelity.
See Section~\ref{sec:system} and Section~\ref{sec:deployment} for details about system design and deployment respectively.



\textbf{We construct a leaderboard of user preferences and find notable differences from existing static benchmarks and human preference leaderboards.}
In general, we observe that smaller models seem to overperform in static benchmarks compared to our leaderboard, while performance among larger models is mixed (Section~\ref{sec:leaderboard_calculation}).
We attribute these differences to the fact that \systemName is exposed to users and tasks that differ drastically from code evaluations in the past. 
Our data spans 103 programming languages and 24 natural languages as well as a variety of real-world applications and code structures, while static benchmarks tend to focus on a specific programming and natural language and task (e.g. coding competition problems).
Additionally, while all of \systemName interactions contain code contexts and the majority involve infilling tasks, a much smaller fraction of Chatbot Arena's coding tasks contain code context, with infilling tasks appearing even more rarely. 
We analyze our data in depth in Section~\ref{subsec:comparison}.



\textbf{We derive new insights into user preferences of code by analyzing \systemName's diverse and distinct data distribution.}
We compare user preferences across different stratifications of input data (e.g., common versus rare languages) and observe which affect observed preferences most (Section~\ref{sec:analysis}).
For example, while user preferences stay relatively consistent across various programming languages, they differ drastically between different task categories (e.g. frontend/backend versus algorithm design).
We also observe variations in user preference due to different features related to code structure 
(e.g., context length and completion patterns).
We open-source \systemName and release a curated subset of code contexts.
Altogether, our results highlight the necessity of model evaluation in realistic and domain-specific settings.





\putsec{related}{Related Work}

\noindent \textbf{Efficient Radiance Field Rendering.}
%
The introduction of Neural Radiance Fields (NeRF)~\cite{mil:sri20} has
generated significant interest in efficient 3D scene representation and
rendering for radiance fields.
%
Over the past years, there has been a large amount of research aimed at
accelerating NeRFs through algorithmic or software
optimizations~\cite{mul:eva22,fri:yu22,che:fun23,sun:sun22}, and the
development of hardware
accelerators~\cite{lee:cho23,li:li23,son:wen23,mub:kan23,fen:liu24}.
%
The state-of-the-art method, 3D Gaussian splatting~\cite{ker:kop23}, has
further fueled interest in accelerating radiance field
rendering~\cite{rad:ste24,lee:lee24,nie:stu24,lee:rho24,ham:mel24} as it
employs rasterization primitives that can be rendered much faster than NeRFs.
%
However, previous research focused on software graphics rendering on
programmable cores or building dedicated hardware accelerators. In contrast,
\name{} investigates the potential of efficient radiance field rendering while
utilizing fixed-function units in graphics hardware.
%
To our knowledge, this is the first work that assesses the performance
implications of rendering Gaussian-based radiance fields on the hardware
graphics pipeline with software and hardware optimizations.

%%%%%%%%%%%%%%%%%%%%%%%%%%%%%%%%%%%%%%%%%%%%%%%%%%%%%%%%%%%%%%%%%%%%%%%%%%
\myparagraph{Enhancing Graphics Rendering Hardware.}
%
The performance advantage of executing graphics rendering on either
programmable shader cores or fixed-function units varies depending on the
rendering methods and hardware designs.
%
Previous studies have explored the performance implication of graphics hardware
design by developing simulation infrastructures for graphics
workloads~\cite{bar:gon06,gub:aam19,tin:sax23,arn:par13}.
%
Additionally, several studies have aimed to improve the performance of
special-purpose hardware such as ray tracing units in graphics
hardware~\cite{cho:now23,liu:cha21} and proposed hardware accelerators for
graphics applications~\cite{lu:hua17,ram:gri09}.
%
In contrast to these works, which primarily evaluate traditional graphics
workloads, our work focuses on improving the performance of volume rendering
workloads, such as Gaussian splatting, which require blending a huge number of
fragments per pixel.

%%%%%%%%%%%%%%%%%%%%%%%%%%%%%%%%%%%%%%%%%%%%%%%%%%%%%%%%%%%%%%%%%%%%%%%%%%
%
In the context of multi-sample anti-aliasing, prior work proposed reducing the
amount of redundant shading by merging fragments from adjacent triangles in a
mesh at the quad granularity~\cite{fat:bou10}.
%
While both our work and quad-fragment merging (QFM)~\cite{fat:bou10} aim to
reduce operations by merging quads, our proposed technique differs from QFM in
many aspects.
%
Our method aims to blend \emph{overlapping primitives} along the depth
direction and applies to quads from any primitive. In contrast, QFM merges quad
fragments from small (e.g., pixel-sized) triangles that \emph{share} an edge
(i.e., \emph{connected}, \emph{non-overlapping} triangles).
%
As such, QFM is not applicable to the scenes consisting of a number of
unconnected transparent triangles, such as those in 3D Gaussian splatting.
%
In addition, our method computes the \emph{exact} color for each pixel by
offloading blending operations from ROPs to shader units, whereas QFM
\emph{approximates} pixel colors by using the color from one triangle when
multiple triangles are merged into a single quad.


\subsection{Greedies}
We have two greedy methods that we're using and testing, but they both have one thing in common: They try every node and possible resistances, and choose the one that results in the largest change in the objective function.

The differences between the two methods, are the function. The first one uses the median (since we want the median to be >0.5, we just set this as our objective function.)

Second one uses a function to try to capture more nuances about the fact that we want the median to be over 0.5. The function is:

\[
\text{score}(\text{opinion}) =
\begin{cases} 
\text{maxScore}, & \text{if } \text{opinion} \geq 0.5 \\
\min\left(\frac{50}{0.5 - \text{opinion}}, \frac{\text{maxScore}}{2}\right), & \text{if } \text{opinion} < 0.5 
\end{cases}
\] 

Where we set maxScore to be $10000$.

\subsection{find-c}
Then for the projected methods where we use Huber-Loss, we also have a $find-c$ version (temporary name). This method initially finds the c for the rest of the run. 

The way it does it it randomly perturbs the resistances and opinions of every node, then finds the c value that most closely approximates the median for all of the perturbed scenarios (after finding the stable opinions). 



\begin{table*}[t]
\centering
\fontsize{11pt}{11pt}\selectfont
\begin{tabular}{lllllllllllll}
\toprule
\multicolumn{1}{c}{\textbf{task}} & \multicolumn{2}{c}{\textbf{Mir}} & \multicolumn{2}{c}{\textbf{Lai}} & \multicolumn{2}{c}{\textbf{Ziegen.}} & \multicolumn{2}{c}{\textbf{Cao}} & \multicolumn{2}{c}{\textbf{Alva-Man.}} & \multicolumn{1}{c}{\textbf{avg.}} & \textbf{\begin{tabular}[c]{@{}l@{}}avg.\\ rank\end{tabular}} \\
\multicolumn{1}{c}{\textbf{metrics}} & \multicolumn{1}{c}{\textbf{cor.}} & \multicolumn{1}{c}{\textbf{p-v.}} & \multicolumn{1}{c}{\textbf{cor.}} & \multicolumn{1}{c}{\textbf{p-v.}} & \multicolumn{1}{c}{\textbf{cor.}} & \multicolumn{1}{c}{\textbf{p-v.}} & \multicolumn{1}{c}{\textbf{cor.}} & \multicolumn{1}{c}{\textbf{p-v.}} & \multicolumn{1}{c}{\textbf{cor.}} & \multicolumn{1}{c}{\textbf{p-v.}} &  &  \\ \midrule
\textbf{S-Bleu} & 0.50 & 0.0 & 0.47 & 0.0 & 0.59 & 0.0 & 0.58 & 0.0 & 0.68 & 0.0 & 0.57 & 5.8 \\
\textbf{R-Bleu} & -- & -- & 0.27 & 0.0 & 0.30 & 0.0 & -- & -- & -- & -- & - &  \\
\textbf{S-Meteor} & 0.49 & 0.0 & 0.48 & 0.0 & 0.61 & 0.0 & 0.57 & 0.0 & 0.64 & 0.0 & 0.56 & 6.1 \\
\textbf{R-Meteor} & -- & -- & 0.34 & 0.0 & 0.26 & 0.0 & -- & -- & -- & -- & - &  \\
\textbf{S-Bertscore} & \textbf{0.53} & 0.0 & {\ul 0.80} & 0.0 & \textbf{0.70} & 0.0 & {\ul 0.66} & 0.0 & {\ul0.78} & 0.0 & \textbf{0.69} & \textbf{1.7} \\
\textbf{R-Bertscore} & -- & -- & 0.51 & 0.0 & 0.38 & 0.0 & -- & -- & -- & -- & - &  \\
\textbf{S-Bleurt} & {\ul 0.52} & 0.0 & {\ul 0.80} & 0.0 & 0.60 & 0.0 & \textbf{0.70} & 0.0 & \textbf{0.80} & 0.0 & {\ul 0.68} & {\ul 2.3} \\
\textbf{R-Bleurt} & -- & -- & 0.59 & 0.0 & -0.05 & 0.13 & -- & -- & -- & -- & - &  \\
\textbf{S-Cosine} & 0.51 & 0.0 & 0.69 & 0.0 & {\ul 0.62} & 0.0 & 0.61 & 0.0 & 0.65 & 0.0 & 0.62 & 4.4 \\
\textbf{R-Cosine} & -- & -- & 0.40 & 0.0 & 0.29 & 0.0 & -- & -- & -- & -- & - & \\ \midrule
\textbf{QuestEval} & 0.23 & 0.0 & 0.25 & 0.0 & 0.49 & 0.0 & 0.47 & 0.0 & 0.62 & 0.0 & 0.41 & 9.0 \\
\textbf{LLaMa3} & 0.36 & 0.0 & \textbf{0.84} & 0.0 & {\ul{0.62}} & 0.0 & 0.61 & 0.0 &  0.76 & 0.0 & 0.64 & 3.6 \\
\textbf{our (3b)} & 0.49 & 0.0 & 0.73 & 0.0 & 0.54 & 0.0 & 0.53 & 0.0 & 0.7 & 0.0 & 0.60 & 5.8 \\
\textbf{our (8b)} & 0.48 & 0.0 & 0.73 & 0.0 & 0.52 & 0.0 & 0.53 & 0.0 & 0.7 & 0.0 & 0.59 & 6.3 \\  \bottomrule
\end{tabular}
\caption{Pearson correlation on human evaluation on system output. `R-': reference-based. `S-': source-based.}
\label{tab:sys}
\end{table*}



\begin{table}%[]
\centering
\fontsize{11pt}{11pt}\selectfont
\begin{tabular}{llllll}
\toprule
\multicolumn{1}{c}{\textbf{task}} & \multicolumn{1}{c}{\textbf{Lai}} & \multicolumn{1}{c}{\textbf{Zei.}} & \multicolumn{1}{c}{\textbf{Scia.}} & \textbf{} & \textbf{} \\ 
\multicolumn{1}{c}{\textbf{metrics}} & \multicolumn{1}{c}{\textbf{cor.}} & \multicolumn{1}{c}{\textbf{cor.}} & \multicolumn{1}{c}{\textbf{cor.}} & \textbf{avg.} & \textbf{\begin{tabular}[c]{@{}l@{}}avg.\\ rank\end{tabular}} \\ \midrule
\textbf{S-Bleu} & 0.40 & 0.40 & 0.19* & 0.33 & 7.67 \\
\textbf{S-Meteor} & 0.41 & 0.42 & 0.16* & 0.33 & 7.33 \\
\textbf{S-BertS.} & {\ul0.58} & 0.47 & 0.31 & 0.45 & 3.67 \\
\textbf{S-Bleurt} & 0.45 & {\ul 0.54} & {\ul 0.37} & 0.45 & {\ul 3.33} \\
\textbf{S-Cosine} & 0.56 & 0.52 & 0.3 & {\ul 0.46} & {\ul 3.33} \\ \midrule
\textbf{QuestE.} & 0.27 & 0.35 & 0.06* & 0.23 & 9.00 \\
\textbf{LlaMA3} & \textbf{0.6} & \textbf{0.67} & \textbf{0.51} & \textbf{0.59} & \textbf{1.0} \\
\textbf{Our (3b)} & 0.51 & 0.49 & 0.23* & 0.39 & 4.83 \\
\textbf{Our (8b)} & 0.52 & 0.49 & 0.22* & 0.43 & 4.83 \\ \bottomrule
\end{tabular}
\caption{Pearson correlation on human ratings on reference output. *not significant; we cannot reject the null hypothesis of zero correlation}
\label{tab:ref}
\end{table}


\begin{table*}%[]
\centering
\fontsize{11pt}{11pt}\selectfont
\begin{tabular}{lllllllll}
\toprule
\textbf{task} & \multicolumn{1}{c}{\textbf{ALL}} & \multicolumn{1}{c}{\textbf{sentiment}} & \multicolumn{1}{c}{\textbf{detoxify}} & \multicolumn{1}{c}{\textbf{catchy}} & \multicolumn{1}{c}{\textbf{polite}} & \multicolumn{1}{c}{\textbf{persuasive}} & \multicolumn{1}{c}{\textbf{formal}} & \textbf{\begin{tabular}[c]{@{}l@{}}avg. \\ rank\end{tabular}} \\
\textbf{metrics} & \multicolumn{1}{c}{\textbf{cor.}} & \multicolumn{1}{c}{\textbf{cor.}} & \multicolumn{1}{c}{\textbf{cor.}} & \multicolumn{1}{c}{\textbf{cor.}} & \multicolumn{1}{c}{\textbf{cor.}} & \multicolumn{1}{c}{\textbf{cor.}} & \multicolumn{1}{c}{\textbf{cor.}} &  \\ \midrule
\textbf{S-Bleu} & -0.17 & -0.82 & -0.45 & -0.12* & -0.1* & -0.05 & -0.21 & 8.42 \\
\textbf{R-Bleu} & - & -0.5 & -0.45 &  &  &  &  &  \\
\textbf{S-Meteor} & -0.07* & -0.55 & -0.4 & -0.01* & 0.1* & -0.16 & -0.04* & 7.67 \\
\textbf{R-Meteor} & - & -0.17* & -0.39 & - & - & - & - & - \\
\textbf{S-BertScore} & 0.11 & -0.38 & -0.07* & -0.17* & 0.28 & 0.12 & 0.25 & 6.0 \\
\textbf{R-BertScore} & - & -0.02* & -0.21* & - & - & - & - & - \\
\textbf{S-Bleurt} & 0.29 & 0.05* & 0.45 & 0.06* & 0.29 & 0.23 & 0.46 & 4.2 \\
\textbf{R-Bleurt} & - &  0.21 & 0.38 & - & - & - & - & - \\
\textbf{S-Cosine} & 0.01* & -0.5 & -0.13* & -0.19* & 0.05* & -0.05* & 0.15* & 7.42 \\
\textbf{R-Cosine} & - & -0.11* & -0.16* & - & - & - & - & - \\ \midrule
\textbf{QuestEval} & 0.21 & {\ul{0.29}} & 0.23 & 0.37 & 0.19* & 0.35 & 0.14* & 4.67 \\
\textbf{LlaMA3} & \textbf{0.82} & \textbf{0.80} & \textbf{0.72} & \textbf{0.84} & \textbf{0.84} & \textbf{0.90} & \textbf{0.88} & \textbf{1.00} \\
\textbf{Our (3b)} & 0.47 & -0.11* & 0.37 & 0.61 & 0.53 & 0.54 & 0.66 & 3.5 \\
\textbf{Our (8b)} & {\ul{0.57}} & 0.09* & {\ul 0.49} & {\ul 0.72} & {\ul 0.64} & {\ul 0.62} & {\ul 0.67} & {\ul 2.17} \\ \bottomrule
\end{tabular}
\caption{Pearson correlation on human ratings on our constructed test set. 'R-': reference-based. 'S-': source-based. *not significant; we cannot reject the null hypothesis of zero correlation}
\label{tab:con}
\end{table*}

\section{Results}
We benchmark the different metrics on the different datasets using correlation to human judgement. For content preservation, we show results split on data with system output, reference output and our constructed test set: we show that the data source for evaluation leads to different conclusions on the metrics. In addition, we examine whether the metrics can rank style transfer systems similar to humans. On style strength, we likewise show correlations between human judgment and zero-shot evaluation approaches. When applicable, we summarize results by reporting the average correlation. And the average ranking of the metric per dataset (by ranking which metric obtains the highest correlation to human judgement per dataset). 

\subsection{Content preservation}
\paragraph{How do data sources affect the conclusion on best metric?}
The conclusions about the metrics' performance change radically depending on whether we use system output data, reference output, or our constructed test set. Ideally, a good metric correlates highly with humans on any data source. Ideally, for meta-evaluation, a metric should correlate consistently across all data sources, but the following shows that the correlations indicate different things, and the conclusion on the best metric should be drawn carefully.

Looking at the metrics correlations with humans on the data source with system output (Table~\ref{tab:sys}), we see a relatively high correlation for many of the metrics on many tasks. The overall best metrics are S-BertScore and S-BLEURT (avg+avg rank). We see no notable difference in our method of using the 3B or 8B model as the backbone.

Examining the average correlations based on data with reference output (Table~\ref{tab:ref}), now the zero-shoot prompting with LlaMA3 70B is the best-performing approach ($0.59$ avg). Tied for second place are source-based cosine embedding ($0.46$ avg), BLEURT ($0.45$ avg) and BertScore ($0.45$ avg). Our method follows on a 5. place: here, the 8b version (($0.43$ avg)) shows a bit stronger results than 3b ($0.39$ avg). The fact that the conclusions change, whether looking at reference or system output, confirms the observations made by \citet{scialom-etal-2021-questeval} on simplicity transfer.   

Now consider the results on our test set (Table~\ref{tab:con}): Several metrics show low or no correlation; we even see a significantly negative correlation for some metrics on ALL (BLEU) and for specific subparts of our test set for BLEU, Meteor, BertScore, Cosine. On the other end, LlaMA3 70B is again performing best, showing strong results ($0.82$ in ALL). The runner-up is now our 8B method, with a gap to the 3B version ($0.57$ vs $0.47$ in ALL). Note our method still shows zero correlation for the sentiment task. After, ranks BLEURT ($0.29$), QuestEval ($0.21$), BertScore ($0.11$), Cosine ($0.01$).  

On our test set, we find that some metrics that correlate relatively well on the other datasets, now exhibit low correlation. Hence, with our test set, we can now support the logical reasoning with data evidence: Evaluation of content preservation for style transfer needs to take the style shift into account. This conclusion could not be drawn using the existing data sources: We hypothesise that for the data with system-based output, successful output happens to be very similar to the source sentence and vice versa, and reference-based output might not contain server mistakes as they are gold references. Thus, none of the existing data sources tests the limits of the metrics.  


\paragraph{How do reference-based metrics compare to source-based ones?} Reference-based metrics show a lower correlation than the source-based counterpart for all metrics on both datasets with ratings on references (Table~\ref{tab:sys}). As discussed previously, reference-based metrics for style transfer have the drawback that many different good solutions on a rewrite might exist and not only one similar to a reference.


\paragraph{How well can the metrics rank the performance of style transfer methods?}
We compare the metrics' ability to judge the best style transfer methods w.r.t. the human annotations: Several of the data sources contain samples from different style transfer systems. In order to use metrics to assess the quality of the style transfer system, metrics should correctly find the best-performing system. Hence, we evaluate whether the metrics for content preservation provide the same system ranking as human evaluators. We take the mean of the score for every output on each system and the mean of the human annotations; we compare the systems using the Kendall's Tau correlation. 

We find only the evaluation using the dataset Mir, Lai, and Ziegen to result in significant correlations, probably because of sparsity in a number of system tests (App.~\ref{app:dataset}). Our method (8b) is the only metric providing a perfect ranking of the style transfer system on the Lai data, and Llama3 70B the only one on the Ziegen data. Results in App.~\ref{app:results}. 


\subsection{Style strength results}
%Evaluating style strengths is a challenging task. 
Llama3 70B shows better overall results than our method. However, our method scores higher than Llama3 70B on 2 out of 6 datasets, but it also exhibits zero correlation on one task (Table~\ref{tab:styleresults}).%More work i s needed on evaluating style strengths. 
 
\begin{table}%[]
\fontsize{11pt}{11pt}\selectfont
\begin{tabular}{lccc}
\toprule
\multicolumn{1}{c}{\textbf{}} & \textbf{LlaMA3} & \textbf{Our (3b)} & \textbf{Our (8b)} \\ \midrule
\textbf{Mir} & 0.46 & 0.54 & \textbf{0.57} \\
\textbf{Lai} & \textbf{0.57} & 0.18 & 0.19 \\
\textbf{Ziegen.} & 0.25 & 0.27 & \textbf{0.32} \\
\textbf{Alva-M.} & \textbf{0.59} & 0.03* & 0.02* \\
\textbf{Scialom} & \textbf{0.62} & 0.45 & 0.44 \\
\textbf{\begin{tabular}[c]{@{}l@{}}Our Test\end{tabular}} & \textbf{0.63} & 0.46 & 0.48 \\ \bottomrule
\end{tabular}
\caption{Style strength: Pearson correlation to human ratings. *not significant; we cannot reject the null hypothesis of zero corelation}
\label{tab:styleresults}
\end{table}

\subsection{Ablation}
We conduct several runs of the methods using LLMs with variations in instructions/prompts (App.~\ref{app:method}). We observe that the lower the correlation on a task, the higher the variation between the different runs. For our method, we only observe low variance between the runs.
None of the variations leads to different conclusions of the meta-evaluation. Results in App.~\ref{app:results}.
\section{Discussion of Assumptions}\label{sec:discussion}
In this paper, we have made several assumptions for the sake of clarity and simplicity. In this section, we discuss the rationale behind these assumptions, the extent to which these assumptions hold in practice, and the consequences for our protocol when these assumptions hold.

\subsection{Assumptions on the Demand}

There are two simplifying assumptions we make about the demand. First, we assume the demand at any time is relatively small compared to the channel capacities. Second, we take the demand to be constant over time. We elaborate upon both these points below.

\paragraph{Small demands} The assumption that demands are small relative to channel capacities is made precise in \eqref{eq:large_capacity_assumption}. This assumption simplifies two major aspects of our protocol. First, it largely removes congestion from consideration. In \eqref{eq:primal_problem}, there is no constraint ensuring that total flow in both directions stays below capacity--this is always met. Consequently, there is no Lagrange multiplier for congestion and no congestion pricing; only imbalance penalties apply. In contrast, protocols in \cite{sivaraman2020high, varma2021throughput, wang2024fence} include congestion fees due to explicit congestion constraints. Second, the bound \eqref{eq:large_capacity_assumption} ensures that as long as channels remain balanced, the network can always meet demand, no matter how the demand is routed. Since channels can rebalance when necessary, they never drop transactions. This allows prices and flows to adjust as per the equations in \eqref{eq:algorithm}, which makes it easier to prove the protocol's convergence guarantees. This also preserves the key property that a channel's price remains proportional to net money flow through it.

In practice, payment channel networks are used most often for micro-payments, for which on-chain transactions are prohibitively expensive; large transactions typically take place directly on the blockchain. For example, according to \cite{river2023lightning}, the average channel capacity is roughly $0.1$ BTC ($5,000$ BTC distributed over $50,000$ channels), while the average transaction amount is less than $0.0004$ BTC ($44.7k$ satoshis). Thus, the small demand assumption is not too unrealistic. Additionally, the occasional large transaction can be treated as a sequence of smaller transactions by breaking it into packets and executing each packet serially (as done by \cite{sivaraman2020high}).
Lastly, a good path discovery process that favors large capacity channels over small capacity ones can help ensure that the bound in \eqref{eq:large_capacity_assumption} holds.

\paragraph{Constant demands} 
In this work, we assume that any transacting pair of nodes have a steady transaction demand between them (see Section \ref{sec:transaction_requests}). Making this assumption is necessary to obtain the kind of guarantees that we have presented in this paper. Unless the demand is steady, it is unreasonable to expect that the flows converge to a steady value. Weaker assumptions on the demand lead to weaker guarantees. For example, with the more general setting of stochastic, but i.i.d. demand between any two nodes, \cite{varma2021throughput} shows that the channel queue lengths are bounded in expectation. If the demand can be arbitrary, then it is very hard to get any meaningful performance guarantees; \cite{wang2024fence} shows that even for a single bidirectional channel, the competitive ratio is infinite. Indeed, because a PCN is a decentralized system and decisions must be made based on local information alone, it is difficult for the network to find the optimal detailed balance flow at every time step with a time-varying demand.  With a steady demand, the network can discover the optimal flows in a reasonably short time, as our work shows.

We view the constant demand assumption as an approximation for a more general demand process that could be piece-wise constant, stochastic, or both (see simulations in Figure \ref{fig:five_nodes_variable_demand}).
We believe it should be possible to merge ideas from our work and \cite{varma2021throughput} to provide guarantees in a setting with random demands with arbitrary means. We leave this for future work. In addition, our work suggests that a reasonable method of handling stochastic demands is to queue the transaction requests \textit{at the source node} itself. This queuing action should be viewed in conjunction with flow-control. Indeed, a temporarily high unidirectional demand would raise prices for the sender, incentivizing the sender to stop sending the transactions. If the sender queues the transactions, they can send them later when prices drop. This form of queuing does not require any overhaul of the basic PCN infrastructure and is therefore simpler to implement than per-channel queues as suggested by \cite{sivaraman2020high} and \cite{varma2021throughput}.

\subsection{The Incentive of Channels}
The actions of the channels as prescribed by the DEBT control protocol can be summarized as follows. Channels adjust their prices in proportion to the net flow through them. They rebalance themselves whenever necessary and execute any transaction request that has been made of them. We discuss both these aspects below.

\paragraph{On Prices}
In this work, the exclusive role of channel prices is to ensure that the flows through each channel remains balanced. In practice, it would be important to include other components in a channel's price/fee as well: a congestion price  and an incentive price. The congestion price, as suggested by \cite{varma2021throughput}, would depend on the total flow of transactions through the channel, and would incentivize nodes to balance the load over different paths. The incentive price, which is commonly used in practice \cite{river2023lightning}, is necessary to provide channels with an incentive to serve as an intermediary for different channels. In practice, we expect both these components to be smaller than the imbalance price. Consequently, we expect the behavior of our protocol to be similar to our theoretical results even with these additional prices.

A key aspect of our protocol is that channel fees are allowed to be negative. Although the original Lightning network whitepaper \cite{poon2016bitcoin} suggests that negative channel prices may be a good solution to promote rebalancing, the idea of negative prices in not very popular in the literature. To our knowledge, the only prior work with this feature is \cite{varma2021throughput}. Indeed, in papers such as \cite{van2021merchant} and \cite{wang2024fence}, the price function is explicitly modified such that the channel price is never negative. The results of our paper show the benefits of negative prices. For one, in steady state, equal flows in both directions ensure that a channel doesn't loose any money (the other price components mentioned above ensure that the channel will only gain money). More importantly, negative prices are important to ensure that the protocol selectively stifles acyclic flows while allowing circulations to flow. Indeed, in the example of Section \ref{sec:flow_control_example}, the flows between nodes $A$ and $C$ are left on only because the large positive price over one channel is canceled by the corresponding negative price over the other channel, leading to a net zero price.

Lastly, observe that in the DEBT control protocol, the price charged by a channel does not depend on its capacity. This is a natural consequence of the price being the Lagrange multiplier for the net-zero flow constraint, which also does not depend on the channel capacity. In contrast, in many other works, the imbalance price is normalized by the channel capacity \cite{ren2018optimal, lin2020funds, wang2024fence}; this is shown to work well in practice. The rationale for such a price structure is explained well in \cite{wang2024fence}, where this fee is derived with the aim of always maintaining some balance (liquidity) at each end of every channel. This is a reasonable aim if a channel is to never rebalance itself; the experiments of the aforementioned papers are conducted in such a regime. In this work, however, we allow the channels to rebalance themselves a few times in order to settle on a detailed balance flow. This is because our focus is on the long-term steady state performance of the protocol. This difference in perspective also shows up in how the price depends on the channel imbalance. \cite{lin2020funds} and \cite{wang2024fence} advocate for strictly convex prices whereas this work and \cite{varma2021throughput} propose linear prices.

\paragraph{On Rebalancing} 
Recall that the DEBT control protocol ensures that the flows in the network converge to a detailed balance flow, which can be sustained perpetually without any rebalancing. However, during the transient phase (before convergence), channels may have to perform on-chain rebalancing a few times. Since rebalancing is an expensive operation, it is worthwhile discussing methods by which channels can reduce the extent of rebalancing. One option for the channels to reduce the extent of rebalancing is to increase their capacity; however, this comes at the cost of locking in more capital. Each channel can decide for itself the optimum amount of capital to lock in. Another option, which we discuss in Section \ref{sec:five_node}, is for channels to increase the rate $\gamma$ at which they adjust prices. 

Ultimately, whether or not it is beneficial for a channel to rebalance depends on the time-horizon under consideration. Our protocol is based on the assumption that the demand remains steady for a long period of time. If this is indeed the case, it would be worthwhile for a channel to rebalance itself as it can make up this cost through the incentive fees gained from the flow of transactions through it in steady state. If a channel chooses not to rebalance itself, however, there is a risk of being trapped in a deadlock, which is suboptimal for not only the nodes but also the channel.

\section{Conclusion}
This work presents DEBT control: a protocol for payment channel networks that uses source routing and flow control based on channel prices. The protocol is derived by posing a network utility maximization problem and analyzing its dual minimization. It is shown that under steady demands, the protocol guides the network to an optimal, sustainable point. Simulations show its robustness to demand variations. The work demonstrates that simple protocols with strong theoretical guarantees are possible for PCNs and we hope it inspires further theoretical research in this direction.
\section{Conclusion}
In this work, we propose a simple yet effective approach, called SMILE, for graph few-shot learning with fewer tasks. Specifically, we introduce a novel dual-level mixup strategy, including within-task and across-task mixup, for enriching the diversity of nodes within each task and the diversity of tasks. Also, we incorporate the degree-based prior information to learn expressive node embeddings. Theoretically, we prove that SMILE effectively enhances the model's generalization performance. Empirically, we conduct extensive experiments on multiple benchmarks and the results suggest that SMILE significantly outperforms other baselines, including both in-domain and cross-domain few-shot settings.

\section{Limitations}

\begin{itemize}
    \item Our dataset filtering relies on the GPT-estimated deflection score, followed by human validation to remove incorrectly marked unsafe questions. While this ensures a high-quality dataset, it may exclude some valuable questions with low deflection scores that were not manually reviewed. Expanding the selection criteria in future work could further enhance dataset diversity. 

    \item Our evaluation currently focuses on post-training quantization, which is the most widely used approach for efficient model deployment. Investigating how models trained with quantization-aware training perform under the same safety and trustworthiness assessments could offer additional insights into the impact of different quantization techniques.
\end{itemize}

\section{Ethical Considerations}

Our work aims to advance the safety and trustworthiness of quantized language models by evaluating their responses to challenging scenarios. While our dataset, \textbf{OpenSafetyMini}, contains provocative questions, these are solely intended to assess and improve model safety mechanisms, ensuring that AI systems respond responsibly in real-world interactions.

Additionally, our open-sourced human evaluations include responses from open-source models that may contain unsafe content. These responses are shared strictly for scientific purposes, fostering transparency and enabling further research toward the development of more ethical and aligned AI systems. 

Furthermore, our study does not introduce any additional risks beyond those posed by standard safety benchmarks. All experimental evaluations are conducted within ethical guidelines, focusing on enhancing AI robustness while mitigating potential harms associated with unsafe model behavior.


\section*{Acknowledgments}  
We extend our gratitude to Alex Tyulyupo for his contributions to conceptualizing the filtration process of the data set. Tyulyupo proposed the procedure involving ethical deflection scoring (0-100 scale), developed the prompt template for LLM-based score estimation, and performed the initial filtration phase. This work enabled the creation of our refined \textbf{OpenSafetyMini} dataset through subsequent manual quality validation.

\bibliography{references}  

\appendix
\clearpage
\onecolumn

\subsection{Lloyd-Max Algorithm}
\label{subsec:Lloyd-Max}
For a given quantization bitwidth $B$ and an operand $\bm{X}$, the Lloyd-Max algorithm finds $2^B$ quantization levels $\{\hat{x}_i\}_{i=1}^{2^B}$ such that quantizing $\bm{X}$ by rounding each scalar in $\bm{X}$ to the nearest quantization level minimizes the quantization MSE. 

The algorithm starts with an initial guess of quantization levels and then iteratively computes quantization thresholds $\{\tau_i\}_{i=1}^{2^B-1}$ and updates quantization levels $\{\hat{x}_i\}_{i=1}^{2^B}$. Specifically, at iteration $n$, thresholds are set to the midpoints of the previous iteration's levels:
\begin{align*}
    \tau_i^{(n)}=\frac{\hat{x}_i^{(n-1)}+\hat{x}_{i+1}^{(n-1)}}2 \text{ for } i=1\ldots 2^B-1
\end{align*}
Subsequently, the quantization levels are re-computed as conditional means of the data regions defined by the new thresholds:
\begin{align*}
    \hat{x}_i^{(n)}=\mathbb{E}\left[ \bm{X} \big| \bm{X}\in [\tau_{i-1}^{(n)},\tau_i^{(n)}] \right] \text{ for } i=1\ldots 2^B
\end{align*}
where to satisfy boundary conditions we have $\tau_0=-\infty$ and $\tau_{2^B}=\infty$. The algorithm iterates the above steps until convergence.

Figure \ref{fig:lm_quant} compares the quantization levels of a $7$-bit floating point (E3M3) quantizer (left) to a $7$-bit Lloyd-Max quantizer (right) when quantizing a layer of weights from the GPT3-126M model at a per-tensor granularity. As shown, the Lloyd-Max quantizer achieves substantially lower quantization MSE. Further, Table \ref{tab:FP7_vs_LM7} shows the superior perplexity achieved by Lloyd-Max quantizers for bitwidths of $7$, $6$ and $5$. The difference between the quantizers is clear at 5 bits, where per-tensor FP quantization incurs a drastic and unacceptable increase in perplexity, while Lloyd-Max quantization incurs a much smaller increase. Nevertheless, we note that even the optimal Lloyd-Max quantizer incurs a notable ($\sim 1.5$) increase in perplexity due to the coarse granularity of quantization. 

\begin{figure}[h]
  \centering
  \includegraphics[width=0.7\linewidth]{sections/figures/LM7_FP7.pdf}
  \caption{\small Quantization levels and the corresponding quantization MSE of Floating Point (left) vs Lloyd-Max (right) Quantizers for a layer of weights in the GPT3-126M model.}
  \label{fig:lm_quant}
\end{figure}

\begin{table}[h]\scriptsize
\begin{center}
\caption{\label{tab:FP7_vs_LM7} \small Comparing perplexity (lower is better) achieved by floating point quantizers and Lloyd-Max quantizers on a GPT3-126M model for the Wikitext-103 dataset.}
\begin{tabular}{c|cc|c}
\hline
 \multirow{2}{*}{\textbf{Bitwidth}} & \multicolumn{2}{|c|}{\textbf{Floating-Point Quantizer}} & \textbf{Lloyd-Max Quantizer} \\
 & Best Format & Wikitext-103 Perplexity & Wikitext-103 Perplexity \\
\hline
7 & E3M3 & 18.32 & 18.27 \\
6 & E3M2 & 19.07 & 18.51 \\
5 & E4M0 & 43.89 & 19.71 \\
\hline
\end{tabular}
\end{center}
\end{table}

\subsection{Proof of Local Optimality of LO-BCQ}
\label{subsec:lobcq_opt_proof}
For a given block $\bm{b}_j$, the quantization MSE during LO-BCQ can be empirically evaluated as $\frac{1}{L_b}\lVert \bm{b}_j- \bm{\hat{b}}_j\rVert^2_2$ where $\bm{\hat{b}}_j$ is computed from equation (\ref{eq:clustered_quantization_definition}) as $C_{f(\bm{b}_j)}(\bm{b}_j)$. Further, for a given block cluster $\mathcal{B}_i$, we compute the quantization MSE as $\frac{1}{|\mathcal{B}_{i}|}\sum_{\bm{b} \in \mathcal{B}_{i}} \frac{1}{L_b}\lVert \bm{b}- C_i^{(n)}(\bm{b})\rVert^2_2$. Therefore, at the end of iteration $n$, we evaluate the overall quantization MSE $J^{(n)}$ for a given operand $\bm{X}$ composed of $N_c$ block clusters as:
\begin{align*}
    \label{eq:mse_iter_n}
    J^{(n)} = \frac{1}{N_c} \sum_{i=1}^{N_c} \frac{1}{|\mathcal{B}_{i}^{(n)}|}\sum_{\bm{v} \in \mathcal{B}_{i}^{(n)}} \frac{1}{L_b}\lVert \bm{b}- B_i^{(n)}(\bm{b})\rVert^2_2
\end{align*}

At the end of iteration $n$, the codebooks are updated from $\mathcal{C}^{(n-1)}$ to $\mathcal{C}^{(n)}$. However, the mapping of a given vector $\bm{b}_j$ to quantizers $\mathcal{C}^{(n)}$ remains as  $f^{(n)}(\bm{b}_j)$. At the next iteration, during the vector clustering step, $f^{(n+1)}(\bm{b}_j)$ finds new mapping of $\bm{b}_j$ to updated codebooks $\mathcal{C}^{(n)}$ such that the quantization MSE over the candidate codebooks is minimized. Therefore, we obtain the following result for $\bm{b}_j$:
\begin{align*}
\frac{1}{L_b}\lVert \bm{b}_j - C_{f^{(n+1)}(\bm{b}_j)}^{(n)}(\bm{b}_j)\rVert^2_2 \le \frac{1}{L_b}\lVert \bm{b}_j - C_{f^{(n)}(\bm{b}_j)}^{(n)}(\bm{b}_j)\rVert^2_2
\end{align*}

That is, quantizing $\bm{b}_j$ at the end of the block clustering step of iteration $n+1$ results in lower quantization MSE compared to quantizing at the end of iteration $n$. Since this is true for all $\bm{b} \in \bm{X}$, we assert the following:
\begin{equation}
\begin{split}
\label{eq:mse_ineq_1}
    \tilde{J}^{(n+1)} &= \frac{1}{N_c} \sum_{i=1}^{N_c} \frac{1}{|\mathcal{B}_{i}^{(n+1)}|}\sum_{\bm{b} \in \mathcal{B}_{i}^{(n+1)}} \frac{1}{L_b}\lVert \bm{b} - C_i^{(n)}(b)\rVert^2_2 \le J^{(n)}
\end{split}
\end{equation}
where $\tilde{J}^{(n+1)}$ is the the quantization MSE after the vector clustering step at iteration $n+1$.

Next, during the codebook update step (\ref{eq:quantizers_update}) at iteration $n+1$, the per-cluster codebooks $\mathcal{C}^{(n)}$ are updated to $\mathcal{C}^{(n+1)}$ by invoking the Lloyd-Max algorithm \citep{Lloyd}. We know that for any given value distribution, the Lloyd-Max algorithm minimizes the quantization MSE. Therefore, for a given vector cluster $\mathcal{B}_i$ we obtain the following result:

\begin{equation}
    \frac{1}{|\mathcal{B}_{i}^{(n+1)}|}\sum_{\bm{b} \in \mathcal{B}_{i}^{(n+1)}} \frac{1}{L_b}\lVert \bm{b}- C_i^{(n+1)}(\bm{b})\rVert^2_2 \le \frac{1}{|\mathcal{B}_{i}^{(n+1)}|}\sum_{\bm{b} \in \mathcal{B}_{i}^{(n+1)}} \frac{1}{L_b}\lVert \bm{b}- C_i^{(n)}(\bm{b})\rVert^2_2
\end{equation}

The above equation states that quantizing the given block cluster $\mathcal{B}_i$ after updating the associated codebook from $C_i^{(n)}$ to $C_i^{(n+1)}$ results in lower quantization MSE. Since this is true for all the block clusters, we derive the following result: 
\begin{equation}
\begin{split}
\label{eq:mse_ineq_2}
     J^{(n+1)} &= \frac{1}{N_c} \sum_{i=1}^{N_c} \frac{1}{|\mathcal{B}_{i}^{(n+1)}|}\sum_{\bm{b} \in \mathcal{B}_{i}^{(n+1)}} \frac{1}{L_b}\lVert \bm{b}- C_i^{(n+1)}(\bm{b})\rVert^2_2  \le \tilde{J}^{(n+1)}   
\end{split}
\end{equation}

Following (\ref{eq:mse_ineq_1}) and (\ref{eq:mse_ineq_2}), we find that the quantization MSE is non-increasing for each iteration, that is, $J^{(1)} \ge J^{(2)} \ge J^{(3)} \ge \ldots \ge J^{(M)}$ where $M$ is the maximum number of iterations. 
%Therefore, we can say that if the algorithm converges, then it must be that it has converged to a local minimum. 
\hfill $\blacksquare$


\begin{figure}
    \begin{center}
    \includegraphics[width=0.5\textwidth]{sections//figures/mse_vs_iter.pdf}
    \end{center}
    \caption{\small NMSE vs iterations during LO-BCQ compared to other block quantization proposals}
    \label{fig:nmse_vs_iter}
\end{figure}

Figure \ref{fig:nmse_vs_iter} shows the empirical convergence of LO-BCQ across several block lengths and number of codebooks. Also, the MSE achieved by LO-BCQ is compared to baselines such as MXFP and VSQ. As shown, LO-BCQ converges to a lower MSE than the baselines. Further, we achieve better convergence for larger number of codebooks ($N_c$) and for a smaller block length ($L_b$), both of which increase the bitwidth of BCQ (see Eq \ref{eq:bitwidth_bcq}).


\subsection{Additional Accuracy Results}
%Table \ref{tab:lobcq_config} lists the various LOBCQ configurations and their corresponding bitwidths.
\begin{table}
\setlength{\tabcolsep}{4.75pt}
\begin{center}
\caption{\label{tab:lobcq_config} Various LO-BCQ configurations and their bitwidths.}
\begin{tabular}{|c||c|c|c|c||c|c||c|} 
\hline
 & \multicolumn{4}{|c||}{$L_b=8$} & \multicolumn{2}{|c||}{$L_b=4$} & $L_b=2$ \\
 \hline
 \backslashbox{$L_A$\kern-1em}{\kern-1em$N_c$} & 2 & 4 & 8 & 16 & 2 & 4 & 2 \\
 \hline
 64 & 4.25 & 4.375 & 4.5 & 4.625 & 4.375 & 4.625 & 4.625\\
 \hline
 32 & 4.375 & 4.5 & 4.625& 4.75 & 4.5 & 4.75 & 4.75 \\
 \hline
 16 & 4.625 & 4.75& 4.875 & 5 & 4.75 & 5 & 5 \\
 \hline
\end{tabular}
\end{center}
\end{table}

%\subsection{Perplexity achieved by various LO-BCQ configurations on Wikitext-103 dataset}

\begin{table} \centering
\begin{tabular}{|c||c|c|c|c||c|c||c|} 
\hline
 $L_b \rightarrow$& \multicolumn{4}{c||}{8} & \multicolumn{2}{c||}{4} & 2\\
 \hline
 \backslashbox{$L_A$\kern-1em}{\kern-1em$N_c$} & 2 & 4 & 8 & 16 & 2 & 4 & 2  \\
 %$N_c \rightarrow$ & 2 & 4 & 8 & 16 & 2 & 4 & 2 \\
 \hline
 \hline
 \multicolumn{8}{c}{GPT3-1.3B (FP32 PPL = 9.98)} \\ 
 \hline
 \hline
 64 & 10.40 & 10.23 & 10.17 & 10.15 &  10.28 & 10.18 & 10.19 \\
 \hline
 32 & 10.25 & 10.20 & 10.15 & 10.12 &  10.23 & 10.17 & 10.17 \\
 \hline
 16 & 10.22 & 10.16 & 10.10 & 10.09 &  10.21 & 10.14 & 10.16 \\
 \hline
  \hline
 \multicolumn{8}{c}{GPT3-8B (FP32 PPL = 7.38)} \\ 
 \hline
 \hline
 64 & 7.61 & 7.52 & 7.48 &  7.47 &  7.55 &  7.49 & 7.50 \\
 \hline
 32 & 7.52 & 7.50 & 7.46 &  7.45 &  7.52 &  7.48 & 7.48  \\
 \hline
 16 & 7.51 & 7.48 & 7.44 &  7.44 &  7.51 &  7.49 & 7.47  \\
 \hline
\end{tabular}
\caption{\label{tab:ppl_gpt3_abalation} Wikitext-103 perplexity across GPT3-1.3B and 8B models.}
\end{table}

\begin{table} \centering
\begin{tabular}{|c||c|c|c|c||} 
\hline
 $L_b \rightarrow$& \multicolumn{4}{c||}{8}\\
 \hline
 \backslashbox{$L_A$\kern-1em}{\kern-1em$N_c$} & 2 & 4 & 8 & 16 \\
 %$N_c \rightarrow$ & 2 & 4 & 8 & 16 & 2 & 4 & 2 \\
 \hline
 \hline
 \multicolumn{5}{|c|}{Llama2-7B (FP32 PPL = 5.06)} \\ 
 \hline
 \hline
 64 & 5.31 & 5.26 & 5.19 & 5.18  \\
 \hline
 32 & 5.23 & 5.25 & 5.18 & 5.15  \\
 \hline
 16 & 5.23 & 5.19 & 5.16 & 5.14  \\
 \hline
 \multicolumn{5}{|c|}{Nemotron4-15B (FP32 PPL = 5.87)} \\ 
 \hline
 \hline
 64  & 6.3 & 6.20 & 6.13 & 6.08  \\
 \hline
 32  & 6.24 & 6.12 & 6.07 & 6.03  \\
 \hline
 16  & 6.12 & 6.14 & 6.04 & 6.02  \\
 \hline
 \multicolumn{5}{|c|}{Nemotron4-340B (FP32 PPL = 3.48)} \\ 
 \hline
 \hline
 64 & 3.67 & 3.62 & 3.60 & 3.59 \\
 \hline
 32 & 3.63 & 3.61 & 3.59 & 3.56 \\
 \hline
 16 & 3.61 & 3.58 & 3.57 & 3.55 \\
 \hline
\end{tabular}
\caption{\label{tab:ppl_llama7B_nemo15B} Wikitext-103 perplexity compared to FP32 baseline in Llama2-7B and Nemotron4-15B, 340B models}
\end{table}

%\subsection{Perplexity achieved by various LO-BCQ configurations on MMLU dataset}


\begin{table} \centering
\begin{tabular}{|c||c|c|c|c||c|c|c|c|} 
\hline
 $L_b \rightarrow$& \multicolumn{4}{c||}{8} & \multicolumn{4}{c||}{8}\\
 \hline
 \backslashbox{$L_A$\kern-1em}{\kern-1em$N_c$} & 2 & 4 & 8 & 16 & 2 & 4 & 8 & 16  \\
 %$N_c \rightarrow$ & 2 & 4 & 8 & 16 & 2 & 4 & 2 \\
 \hline
 \hline
 \multicolumn{5}{|c|}{Llama2-7B (FP32 Accuracy = 45.8\%)} & \multicolumn{4}{|c|}{Llama2-70B (FP32 Accuracy = 69.12\%)} \\ 
 \hline
 \hline
 64 & 43.9 & 43.4 & 43.9 & 44.9 & 68.07 & 68.27 & 68.17 & 68.75 \\
 \hline
 32 & 44.5 & 43.8 & 44.9 & 44.5 & 68.37 & 68.51 & 68.35 & 68.27  \\
 \hline
 16 & 43.9 & 42.7 & 44.9 & 45 & 68.12 & 68.77 & 68.31 & 68.59  \\
 \hline
 \hline
 \multicolumn{5}{|c|}{GPT3-22B (FP32 Accuracy = 38.75\%)} & \multicolumn{4}{|c|}{Nemotron4-15B (FP32 Accuracy = 64.3\%)} \\ 
 \hline
 \hline
 64 & 36.71 & 38.85 & 38.13 & 38.92 & 63.17 & 62.36 & 63.72 & 64.09 \\
 \hline
 32 & 37.95 & 38.69 & 39.45 & 38.34 & 64.05 & 62.30 & 63.8 & 64.33  \\
 \hline
 16 & 38.88 & 38.80 & 38.31 & 38.92 & 63.22 & 63.51 & 63.93 & 64.43  \\
 \hline
\end{tabular}
\caption{\label{tab:mmlu_abalation} Accuracy on MMLU dataset across GPT3-22B, Llama2-7B, 70B and Nemotron4-15B models.}
\end{table}


%\subsection{Perplexity achieved by various LO-BCQ configurations on LM evaluation harness}

\begin{table} \centering
\begin{tabular}{|c||c|c|c|c||c|c|c|c|} 
\hline
 $L_b \rightarrow$& \multicolumn{4}{c||}{8} & \multicolumn{4}{c||}{8}\\
 \hline
 \backslashbox{$L_A$\kern-1em}{\kern-1em$N_c$} & 2 & 4 & 8 & 16 & 2 & 4 & 8 & 16  \\
 %$N_c \rightarrow$ & 2 & 4 & 8 & 16 & 2 & 4 & 2 \\
 \hline
 \hline
 \multicolumn{5}{|c|}{Race (FP32 Accuracy = 37.51\%)} & \multicolumn{4}{|c|}{Boolq (FP32 Accuracy = 64.62\%)} \\ 
 \hline
 \hline
 64 & 36.94 & 37.13 & 36.27 & 37.13 & 63.73 & 62.26 & 63.49 & 63.36 \\
 \hline
 32 & 37.03 & 36.36 & 36.08 & 37.03 & 62.54 & 63.51 & 63.49 & 63.55  \\
 \hline
 16 & 37.03 & 37.03 & 36.46 & 37.03 & 61.1 & 63.79 & 63.58 & 63.33  \\
 \hline
 \hline
 \multicolumn{5}{|c|}{Winogrande (FP32 Accuracy = 58.01\%)} & \multicolumn{4}{|c|}{Piqa (FP32 Accuracy = 74.21\%)} \\ 
 \hline
 \hline
 64 & 58.17 & 57.22 & 57.85 & 58.33 & 73.01 & 73.07 & 73.07 & 72.80 \\
 \hline
 32 & 59.12 & 58.09 & 57.85 & 58.41 & 73.01 & 73.94 & 72.74 & 73.18  \\
 \hline
 16 & 57.93 & 58.88 & 57.93 & 58.56 & 73.94 & 72.80 & 73.01 & 73.94  \\
 \hline
\end{tabular}
\caption{\label{tab:mmlu_abalation} Accuracy on LM evaluation harness tasks on GPT3-1.3B model.}
\end{table}

\begin{table} \centering
\begin{tabular}{|c||c|c|c|c||c|c|c|c|} 
\hline
 $L_b \rightarrow$& \multicolumn{4}{c||}{8} & \multicolumn{4}{c||}{8}\\
 \hline
 \backslashbox{$L_A$\kern-1em}{\kern-1em$N_c$} & 2 & 4 & 8 & 16 & 2 & 4 & 8 & 16  \\
 %$N_c \rightarrow$ & 2 & 4 & 8 & 16 & 2 & 4 & 2 \\
 \hline
 \hline
 \multicolumn{5}{|c|}{Race (FP32 Accuracy = 41.34\%)} & \multicolumn{4}{|c|}{Boolq (FP32 Accuracy = 68.32\%)} \\ 
 \hline
 \hline
 64 & 40.48 & 40.10 & 39.43 & 39.90 & 69.20 & 68.41 & 69.45 & 68.56 \\
 \hline
 32 & 39.52 & 39.52 & 40.77 & 39.62 & 68.32 & 67.43 & 68.17 & 69.30  \\
 \hline
 16 & 39.81 & 39.71 & 39.90 & 40.38 & 68.10 & 66.33 & 69.51 & 69.42  \\
 \hline
 \hline
 \multicolumn{5}{|c|}{Winogrande (FP32 Accuracy = 67.88\%)} & \multicolumn{4}{|c|}{Piqa (FP32 Accuracy = 78.78\%)} \\ 
 \hline
 \hline
 64 & 66.85 & 66.61 & 67.72 & 67.88 & 77.31 & 77.42 & 77.75 & 77.64 \\
 \hline
 32 & 67.25 & 67.72 & 67.72 & 67.00 & 77.31 & 77.04 & 77.80 & 77.37  \\
 \hline
 16 & 68.11 & 68.90 & 67.88 & 67.48 & 77.37 & 78.13 & 78.13 & 77.69  \\
 \hline
\end{tabular}
\caption{\label{tab:mmlu_abalation} Accuracy on LM evaluation harness tasks on GPT3-8B model.}
\end{table}

\begin{table} \centering
\begin{tabular}{|c||c|c|c|c||c|c|c|c|} 
\hline
 $L_b \rightarrow$& \multicolumn{4}{c||}{8} & \multicolumn{4}{c||}{8}\\
 \hline
 \backslashbox{$L_A$\kern-1em}{\kern-1em$N_c$} & 2 & 4 & 8 & 16 & 2 & 4 & 8 & 16  \\
 %$N_c \rightarrow$ & 2 & 4 & 8 & 16 & 2 & 4 & 2 \\
 \hline
 \hline
 \multicolumn{5}{|c|}{Race (FP32 Accuracy = 40.67\%)} & \multicolumn{4}{|c|}{Boolq (FP32 Accuracy = 76.54\%)} \\ 
 \hline
 \hline
 64 & 40.48 & 40.10 & 39.43 & 39.90 & 75.41 & 75.11 & 77.09 & 75.66 \\
 \hline
 32 & 39.52 & 39.52 & 40.77 & 39.62 & 76.02 & 76.02 & 75.96 & 75.35  \\
 \hline
 16 & 39.81 & 39.71 & 39.90 & 40.38 & 75.05 & 73.82 & 75.72 & 76.09  \\
 \hline
 \hline
 \multicolumn{5}{|c|}{Winogrande (FP32 Accuracy = 70.64\%)} & \multicolumn{4}{|c|}{Piqa (FP32 Accuracy = 79.16\%)} \\ 
 \hline
 \hline
 64 & 69.14 & 70.17 & 70.17 & 70.56 & 78.24 & 79.00 & 78.62 & 78.73 \\
 \hline
 32 & 70.96 & 69.69 & 71.27 & 69.30 & 78.56 & 79.49 & 79.16 & 78.89  \\
 \hline
 16 & 71.03 & 69.53 & 69.69 & 70.40 & 78.13 & 79.16 & 79.00 & 79.00  \\
 \hline
\end{tabular}
\caption{\label{tab:mmlu_abalation} Accuracy on LM evaluation harness tasks on GPT3-22B model.}
\end{table}

\begin{table} \centering
\begin{tabular}{|c||c|c|c|c||c|c|c|c|} 
\hline
 $L_b \rightarrow$& \multicolumn{4}{c||}{8} & \multicolumn{4}{c||}{8}\\
 \hline
 \backslashbox{$L_A$\kern-1em}{\kern-1em$N_c$} & 2 & 4 & 8 & 16 & 2 & 4 & 8 & 16  \\
 %$N_c \rightarrow$ & 2 & 4 & 8 & 16 & 2 & 4 & 2 \\
 \hline
 \hline
 \multicolumn{5}{|c|}{Race (FP32 Accuracy = 44.4\%)} & \multicolumn{4}{|c|}{Boolq (FP32 Accuracy = 79.29\%)} \\ 
 \hline
 \hline
 64 & 42.49 & 42.51 & 42.58 & 43.45 & 77.58 & 77.37 & 77.43 & 78.1 \\
 \hline
 32 & 43.35 & 42.49 & 43.64 & 43.73 & 77.86 & 75.32 & 77.28 & 77.86  \\
 \hline
 16 & 44.21 & 44.21 & 43.64 & 42.97 & 78.65 & 77 & 76.94 & 77.98  \\
 \hline
 \hline
 \multicolumn{5}{|c|}{Winogrande (FP32 Accuracy = 69.38\%)} & \multicolumn{4}{|c|}{Piqa (FP32 Accuracy = 78.07\%)} \\ 
 \hline
 \hline
 64 & 68.9 & 68.43 & 69.77 & 68.19 & 77.09 & 76.82 & 77.09 & 77.86 \\
 \hline
 32 & 69.38 & 68.51 & 68.82 & 68.90 & 78.07 & 76.71 & 78.07 & 77.86  \\
 \hline
 16 & 69.53 & 67.09 & 69.38 & 68.90 & 77.37 & 77.8 & 77.91 & 77.69  \\
 \hline
\end{tabular}
\caption{\label{tab:mmlu_abalation} Accuracy on LM evaluation harness tasks on Llama2-7B model.}
\end{table}

\begin{table} \centering
\begin{tabular}{|c||c|c|c|c||c|c|c|c|} 
\hline
 $L_b \rightarrow$& \multicolumn{4}{c||}{8} & \multicolumn{4}{c||}{8}\\
 \hline
 \backslashbox{$L_A$\kern-1em}{\kern-1em$N_c$} & 2 & 4 & 8 & 16 & 2 & 4 & 8 & 16  \\
 %$N_c \rightarrow$ & 2 & 4 & 8 & 16 & 2 & 4 & 2 \\
 \hline
 \hline
 \multicolumn{5}{|c|}{Race (FP32 Accuracy = 48.8\%)} & \multicolumn{4}{|c|}{Boolq (FP32 Accuracy = 85.23\%)} \\ 
 \hline
 \hline
 64 & 49.00 & 49.00 & 49.28 & 48.71 & 82.82 & 84.28 & 84.03 & 84.25 \\
 \hline
 32 & 49.57 & 48.52 & 48.33 & 49.28 & 83.85 & 84.46 & 84.31 & 84.93  \\
 \hline
 16 & 49.85 & 49.09 & 49.28 & 48.99 & 85.11 & 84.46 & 84.61 & 83.94  \\
 \hline
 \hline
 \multicolumn{5}{|c|}{Winogrande (FP32 Accuracy = 79.95\%)} & \multicolumn{4}{|c|}{Piqa (FP32 Accuracy = 81.56\%)} \\ 
 \hline
 \hline
 64 & 78.77 & 78.45 & 78.37 & 79.16 & 81.45 & 80.69 & 81.45 & 81.5 \\
 \hline
 32 & 78.45 & 79.01 & 78.69 & 80.66 & 81.56 & 80.58 & 81.18 & 81.34  \\
 \hline
 16 & 79.95 & 79.56 & 79.79 & 79.72 & 81.28 & 81.66 & 81.28 & 80.96  \\
 \hline
\end{tabular}
\caption{\label{tab:mmlu_abalation} Accuracy on LM evaluation harness tasks on Llama2-70B model.}
\end{table}

%\section{MSE Studies}
%\textcolor{red}{TODO}


\subsection{Number Formats and Quantization Method}
\label{subsec:numFormats_quantMethod}
\subsubsection{Integer Format}
An $n$-bit signed integer (INT) is typically represented with a 2s-complement format \citep{yao2022zeroquant,xiao2023smoothquant,dai2021vsq}, where the most significant bit denotes the sign.

\subsubsection{Floating Point Format}
An $n$-bit signed floating point (FP) number $x$ comprises of a 1-bit sign ($x_{\mathrm{sign}}$), $B_m$-bit mantissa ($x_{\mathrm{mant}}$) and $B_e$-bit exponent ($x_{\mathrm{exp}}$) such that $B_m+B_e=n-1$. The associated constant exponent bias ($E_{\mathrm{bias}}$) is computed as $(2^{{B_e}-1}-1)$. We denote this format as $E_{B_e}M_{B_m}$.  

\subsubsection{Quantization Scheme}
\label{subsec:quant_method}
A quantization scheme dictates how a given unquantized tensor is converted to its quantized representation. We consider FP formats for the purpose of illustration. Given an unquantized tensor $\bm{X}$ and an FP format $E_{B_e}M_{B_m}$, we first, we compute the quantization scale factor $s_X$ that maps the maximum absolute value of $\bm{X}$ to the maximum quantization level of the $E_{B_e}M_{B_m}$ format as follows:
\begin{align}
\label{eq:sf}
    s_X = \frac{\mathrm{max}(|\bm{X}|)}{\mathrm{max}(E_{B_e}M_{B_m})}
\end{align}
In the above equation, $|\cdot|$ denotes the absolute value function.

Next, we scale $\bm{X}$ by $s_X$ and quantize it to $\hat{\bm{X}}$ by rounding it to the nearest quantization level of $E_{B_e}M_{B_m}$ as:

\begin{align}
\label{eq:tensor_quant}
    \hat{\bm{X}} = \text{round-to-nearest}\left(\frac{\bm{X}}{s_X}, E_{B_e}M_{B_m}\right)
\end{align}

We perform dynamic max-scaled quantization \citep{wu2020integer}, where the scale factor $s$ for activations is dynamically computed during runtime.

\subsection{Vector Scaled Quantization}
\begin{wrapfigure}{r}{0.35\linewidth}
  \centering
  \includegraphics[width=\linewidth]{sections/figures/vsquant.jpg}
  \caption{\small Vectorwise decomposition for per-vector scaled quantization (VSQ \citep{dai2021vsq}).}
  \label{fig:vsquant}
\end{wrapfigure}
During VSQ \citep{dai2021vsq}, the operand tensors are decomposed into 1D vectors in a hardware friendly manner as shown in Figure \ref{fig:vsquant}. Since the decomposed tensors are used as operands in matrix multiplications during inference, it is beneficial to perform this decomposition along the reduction dimension of the multiplication. The vectorwise quantization is performed similar to tensorwise quantization described in Equations \ref{eq:sf} and \ref{eq:tensor_quant}, where a scale factor $s_v$ is required for each vector $\bm{v}$ that maps the maximum absolute value of that vector to the maximum quantization level. While smaller vector lengths can lead to larger accuracy gains, the associated memory and computational overheads due to the per-vector scale factors increases. To alleviate these overheads, VSQ \citep{dai2021vsq} proposed a second level quantization of the per-vector scale factors to unsigned integers, while MX \citep{rouhani2023shared} quantizes them to integer powers of 2 (denoted as $2^{INT}$).

\subsubsection{MX Format}
The MX format proposed in \citep{rouhani2023microscaling} introduces the concept of sub-block shifting. For every two scalar elements of $b$-bits each, there is a shared exponent bit. The value of this exponent bit is determined through an empirical analysis that targets minimizing quantization MSE. We note that the FP format $E_{1}M_{b}$ is strictly better than MX from an accuracy perspective since it allocates a dedicated exponent bit to each scalar as opposed to sharing it across two scalars. Therefore, we conservatively bound the accuracy of a $b+2$-bit signed MX format with that of a $E_{1}M_{b}$ format in our comparisons. For instance, we use E1M2 format as a proxy for MX4.

\begin{figure}
    \centering
    \includegraphics[width=1\linewidth]{sections//figures/BlockFormats.pdf}
    \caption{\small Comparing LO-BCQ to MX format.}
    \label{fig:block_formats}
\end{figure}

Figure \ref{fig:block_formats} compares our $4$-bit LO-BCQ block format to MX \citep{rouhani2023microscaling}. As shown, both LO-BCQ and MX decompose a given operand tensor into block arrays and each block array into blocks. Similar to MX, we find that per-block quantization ($L_b < L_A$) leads to better accuracy due to increased flexibility. While MX achieves this through per-block $1$-bit micro-scales, we associate a dedicated codebook to each block through a per-block codebook selector. Further, MX quantizes the per-block array scale-factor to E8M0 format without per-tensor scaling. In contrast during LO-BCQ, we find that per-tensor scaling combined with quantization of per-block array scale-factor to E4M3 format results in superior inference accuracy across models. 


%\label{subsec:bias}

It has been widely shown that Large Language Models can be attributed with pernicious behaviour \cite{bender_dangers_2021}; they can perpetuate several types of harm, whether allocational or representational, and have been shown to be sensitive to variation in input format, which can severely affect their performance, allowing them to be easily influenced by factors such as word frequency, answer position in multiple choice settings, among others.

Furthermore, as \LLM{}s become more and more used worldwide, assessing the safety of their interaction with users has become critical \cite{yao_safety_security_survey,chowdhury_attack_survey}. While some resources exist, their availability is heavily skewed towards English \cite{yong_multilingual_adv_bench,joshi_dialect_evaluation_survey}. Further to this, it has become apparent that model safety does not transfer well across languages \cite{llama3, dang_multilingual_rlhf}. Our safety evaluation approach is multilingual, focusing on English, Spanish and Catalan, the main languages of the \SalamandraFamily{}.  

In this section, we describe our evaluation paradigm to identify undesired biases that can negatively affect model behavior and performance, as well as our multilingual approach to assessing model safety. 

 % while in LLMs, recent negative behaviour, why this is important, several potential harms.

\subsection{Evaluating Biases}
We root our work in the theoretical framework presented in \cite{theoretical_bias}, where bias is further divided into \textit{outcome disparity} and \textit{error disparity}. By \textit{outcome disparity} we refer to a systematic difference in model output based on a specific attribute, and with \textit{error disparity} we refer to model predictions that have a systematically larger error for inputs with a specific attribute.    

%Bias as a specific association between a community/entity and any other entity resulting in negative attitudes.

%Toxicity apart from bias and model performance. Difficulty to weed out toxic aspects.

%CURATE and data selection process. Double-check with data team. Not upsampling toxic/problematic sources of data. Benevolent ignorance not a good solution (find literature).

\subsection{Social Biases}

\subsubsection{Bias Benchmark for Question Answering}

To adequately determine how the models' inherent biases can influence performance on downstream tasks, we use two versions of the Bias Benchmark for Question Answering (BBQ). We use the original BBQ dataset developed in \cite{bbq_parrish}, and have additionally translated and adapted our own version (Es\BBQ{}) for evaluating social biases that are prevalent in Spain and  that are relevant for European Spanish culture\footnote{While preliminary, all templates that make up the Spanish version of BBQ (EsBBQ) have been extensively validated by a group of researchers with diverse backgrounds. We are actively working on finalising it and will be releasing it within the coming months.}.  

BBQ is a Question-Answering dataset consisting of specific templates linking socio-demographic groups with their corresponding target stereotypes. These templates can be under-informative (or ambiguous) or adequately informative (disambiguated) by adding a disambiguating context to the initial ambiguous one. A clear answer can be gleaned from the disambiguated contexts, but not from the ambiguous context, where the correct answer is always "unknown". The purpose of the dataset is to test how strongly responses reflect social biases in ambiguous contexts, and if our models' biases can override a correct answer choice in disambiguated contexts where there is a clear correct answer.

We follow the scoring method presented in \cite{jin2024kobbq}, where accuracy in both ambiguous and disambiguated contexts is taken into account, along with a bias score that measures the model's tendency to align with either known stereotypes or counterstereotypes, thus quantifying \textit{error disparity} for each setting. The formulae for computing the relevant scores are as follow:

\begin{multicols}{2}
 \begin{equation}
    Acc_a = \frac{n_{au}}{n_{a}}
\end{equation}

\begin{equation}
Acc_d = \frac{n_{bb} + n_{cc}}{n_b + n_c}
\end{equation}  
\columnbreak

\begin{equation} \label{diffa}
Difference_a = \frac{n_{ab} - n_{ac}}{n_a}
\end{equation}

\begin{equation} \label{diffd}
Difference_d = \frac{n_{bb}}{n_b} - \frac{n_{cc}}{n_c}
\end{equation}
\end{multicols}

Where $Acc_a$ and $Acc_d$ denote model accuracy in ambiguous and disambiguated contexts respectively. $n_{au}$ indicate the number of instances where the model matches the expected "unknown" answer over all ambiguous instances ($n_a$). Similarly, $n_{bb}$ and $n_{cc}$ indicate the number of model correct answers given all biased ($n_b$) and counterbiased ($n_c$) disambiguating contexts. By computing the difference in scores in equations \ref{diffa} and \ref{diffd}, we are essentially quantifying the \textit{error disparity} based on an expected stereotype. For ambiguous contexts ($Difference_a$), we calculate the difference between the prediction ratios of biased answers and counterbiased answers. For disambiguating contexts ($Difference_d$), we measure how much a given stereotype or counterstereotype can directly interfere with a model's performance, given that the correct answer can be easily gleaned from the context.  

\begin{table}[ht!]
\centering
\begin{tabular}{@{}lcccccccc@{}}
\toprule
 & \multicolumn{4}{c}{\textbf{BBQ}} & \multicolumn{4}{c}{\textbf{EsBBQ}} \\ \cmidrule(lr){2-5} \cmidrule(lr){6-9} 
 & \textbf{$Acc_a$} & \textbf{$Acc_d$} & \textbf{$Diff_a$} & \textbf{$Diff_d$} & \textbf{$Acc_a$} & \textbf{$Acc_d$} & \textbf{$Diff_a$} & \textbf{$Diff_d$} \\ \midrule
\textbf{2b} & 0.03 & 0.54 & 0.02 & 0.04 & 0.18 & 0.50 & 0.01 & 0.02 \\
\textbf{7b} & 0.03 & 0.79 & 0.08 & 0.04 & 0.10 & 0.72 & 0.06 & 0.04 \\
\textbf{2b-instruct} & 0.02 & 0.67 & 0.04 & 0.05 & 0.05 & 0.64 & 0.03 & 0.05 \\
\textbf{7b-instruct} & 0.04 & 0.92 & 0.15 & 0.02 & 0.07 & 0.88 & 0.22 & 0.04 \\ \bottomrule
\end{tabular}
\caption{Overall accuracy and difference scores in the original BBQ and EsBBQ.}
\label{tab:bbq_overall-acc}
\end{table}

Table \ref{tab:bbq_overall-acc} shows the mean accuracy and difference scores in BBQ and EsBBQ. All models show significantly higher accuracy in disambiguated contexts compared to ambiguous contexts. In correlation with these accuracy results, the bias difference scores are, as expected, lower when providing a disambiguated context. Models struggle to choose the correct "unknown" answer for questions with ambiguous contexts, but, when a correct answer is provided within the context, models are fairly successful at selecting it. However, accuracy scores obtained are relatively modest in the case of 2B versions given the low complexity of the task itself. 

Accuracy tends to increase together with model size, as well as with instruction tuning. This increase in the performance in the case of larger and instruction-tuned models goes together with higher difference scores, which reveals they are more reliant on biases when trying to solve the question answering task. Specifically, all difference scores are positive, suggesting that the models tend to favor outputs that are aligned with societal biases. On the other extreme, it cannot be stated that models with lower difference scores are not free from bias, considering their poor performance results.

More specifically, according to Figure \ref{fig:bbq_original}, in the original \BBQ{}, questions prompted with ambiguous contexts associated with Age and Physical Appearance are the ones where models tend to show more bias, particularly 7B versions. In both categories, scores are higher in the case of the instructed version compared to the base one. 7B instructed version also demonstrates significantly higher difference scores in instances associated with Disability Status, Gender Identity, Nationality and Socio-Economic Status. The scores for these categories decrease notably in questions prompted with disambiguated contexts. Socio-Economic Status and, once again, Physical Appearance are the categories for which the models generate more biased outputs. Note, however, the 2B versions are the ones with greater difference scores in this setting.

On the other hand, Figure \ref{fig:esbbq} shows that, with ambiguous contexts, models tends to favor stereotypical outputs related to Physical Appearance and Socio-Economic Status, with particularly higher scores in 7B versions. Bias is also notable in the case of 7B instructed answers about Sexual Orientation and, to a lesser extent, Age and Disability Status. It is remarkable that Nationality is the only category where all models, except for 7B instructed, exhibit negative difference scores, which reveals that they are more prone to select counter-biased answers. As previously mentioned, bias is reduced when providing a disambiguated context. However, it persists in 2B model results in Physical Appearance and Socio-Economic Status categories. 

\begin{figure*}[htb!]
    \centering
    \includegraphics[width=\textwidth, trim=0cm 3.5cm 0cm 0cm, clip]{figures/bias_and_safety/bbq.pdf}
    \caption{Accuracy and difference scores in ambiguous and disambiguating contexts for each category in the original BBQ.}
    \label{fig:bbq_original}
\end{figure*}

\begin{figure*}[htb!]
    \centering
    \includegraphics[width=0.95\textwidth, trim=0cm 5cm 0cm 0cm, clip]{figures/bias_and_safety/esbbq.pdf}
    \caption{Accuracy and difference scores in ambiguous and disambiguating contexts for each category in EsBBQ.}
    \label{fig:esbbq}
\end{figure*}

\subsubsection{Regard Analysis}
In addition to our analyses using \BBQ{} and Es\BBQ{}, we perform a regard analysis on the base variants of the \SalamandraFamily{} (i.e. \SalamandraBaseII{} and \SalamandraBaseVII{}). The notion of regard is introduced as language polarity towards a social demographic as well as how they are socially perceived \cite{sheng-regard}. Furthermore, \citet{sheng-regard} provide a dataset and a classifier to measure these aspects.

We analyze base model generations using the Regard dataset and classifier in the main languages of \Salamandra{}: Catalan, Spanish, and English. While the dataset is only available in English, we use backtranslation with NLLB \cite{nllb_language_2022} and manual review of the translations. The dataset compares social minorities with their non-marked counterpart along three categories: \textit{Gender, sexual orientation}, and \textit{race}, while the regard classifier output three labels: \textit{positive, negative}, and \textit{neutral}.

We compare the difference in frequency of output labels with a $\chi²$-goodness-of-fit test. Our analysis yielded statistically significant differences in the case of \SalamandraBaseII{} in English; the number of model outputs classified with a negative regard are significantly higher for minority groups, while the number of outputs classified with a positive regard is significantly higher for majority groups. For \SalamandraBaseII{} or \SalamandraBaseVII{}, we do not find statistically significant differences between regard labels for any other languages.

\subsection{Cognitive Biases}

Large Language Models have been shown to achieve strong performance across different tasks. However, as a result of their high capacity, 
a rapidly accumulating amount of evidence shows that LLMs can exhibit similar cognitive biases to humans due to the percolation of these biases through the datasets used to train the \LLM{} \cite{petroni-etal-2019-language, lu-etal-2022-fantastically, Zhao2021CalibrateBU, weber-etal-2023-mind}. As a consequence, some model responses can be conditioned by frequent words, classes, and general formatting in a given input prompt. This is problematic as these biases can influence performance, inflating or deflating metrics on the standard benchmarks, thereby making them less reliable. 

 Following \citet{Zhao2021CalibrateBU}, who examine the most comprehensive set of cognitive biases as far as we have observed in previous works, we examine the effects of three types of cognitive bias on model behavior: primacy, recency, and majority class. Primacy and recency effects denote a given model's tendency to prefer the first and last items, respectively, given a list of options. These effects are evident when the model is provided with lists, or when the input to a given model has a specific format, such as a multiple choice questions (MCQs). Majority class effects appear in few-shot settings in cases where there is an imbalance. We also highlight that we diverge from \citet{Zhao2021CalibrateBU}, and choose not to examine common token bias (i.e. an LLM's tendency to prefer responses which are more frequently seen in training data) due to the analysis and discussion presented in \cite{cobie}.

\begin{table}[ht!]
\centering
\resizebox{\textwidth}{!}{
\begin{tabular}{@{}lccccc@{}}
\toprule
 & \textbf{Majority Class ($V$)} & \multicolumn{2}{c}{\textbf{Primacy ($\varphi$)}} & \multicolumn{2}{c}{\textbf{Recency ($\varphi$)}} \\ \cmidrule(l){2-2} \cmidrule(l){3-4} \cmidrule(l){5-6} 
 & \textbf{SST-2} & \textbf{ARC Easy} & \textbf{ARC Challenge} & \textbf{ARC Easy} & \textbf{ARC Challenge} \\ \midrule
\textbf{2b} & 0.33 & 0.79 & 0.79 & 0.10 & 0.18 \\
\textbf{7b} & 0.12 & 0.23 & 0.31 & 0.08 & 0.10 \\
\textbf{2b-instruct} & 0.04 & 0.05 & 0.07 & 0.26 & 0.34 \\
\textbf{7b-instruct} & 0.01 & 0.01 & 0.03 & 0.09 & 0.15 \\ \bottomrule
\end{tabular}
}
\caption{$V$ and $\varphi$ coefficients resulting from the $\chi^2$ independence and goodness-of-fit tests to check majority class, primacy and recency biases, respectively.}
\label{tab:cobie}
\end{table}

\paragraph{Primacy and Recency Bias} As in \cite{cobie}, primacy and recency bias are evaluated with a 0-shot classification task using the ARC dataset \citep{allenai-arc}. Each instance is prompted four times, permuting the position of the correct answer (\textit{A}, \textit{B}, \textit{C} or \textit{D}). Significance of these positional effects is statistically measured with $\chi^2$ goodness-of-fit tests between the position of interest (\textit{A} for primacy, \textit{D} for recency) and the middle two positions (\textit{B} and \textit{C}) to avoid confounds between these two biases. Effect sizes ($\varphi$ coefficient) are shown in Table \ref{tab:cobie}, and frequency distributions of model predicted answers are illustrated in Figure \ref{fig:cobie_arc}.

All models are biased towards the first possible answer in the prompt. However, effect sizes are smaller in instructed models with respect to their base counterparts. Within each model variant (i.e. base vs. instructed), effect sizes are also smaller as model size increases. Differences are not significant between ARC Easy and Challenge subsets, revealing that an increase in the content difficulty of the question does not correlate with a greater reliance on primacy bias. As for recency bias, once again, from Table \ref{tab:cobie}, we observe that smaller and base models have larger effect sizes than their larger and instruction-tuned counterparts. However, a closer look at Figure \ref{fig:cobie_arc} reveals that results do not reflect a recency bias, but, rather another primacy bias: taking into account that option \textit{A} is not considered for the statistical measurement, option \textit{B} is still predicted more frequently than \textit{D}. 

\begin{figure*}[htb!]
    \centering
    \includegraphics[width=0.8\textwidth]{figures/bias_and_safety/cobie_arc_old-harness.pdf}
    \caption{Frequency distributions of predicted answers on ARC Easy and Challenge subsets depending on their position in the prompt.}
    \label{fig:cobie_arc}
\end{figure*}

\paragraph{Majority Class Bias} Also as in \cite{cobie}, we assess majority class bias with a 4-shot binary classification experiment using the SST-2 dataset on sentiment analysis \citep{socher-etal-2013-recursive}. Each instance is prompted with all possible imbalanced 4-shot distributions (listed in Figure \ref{fig:cobie_sst2}). Table \ref{tab:cobie} details the effect sizes (Cramér's \textit{V} coefficient) obtained from the $\chi^2$ independence tests between the predicted label and the majority class in the few-shot setting: \textit{0} (\textit{negative}) or \textit{1} (\textit{positive}). In correlation with primacy and recency bias results,  increasing model size and instruction tuning translates into smaller effect sizes; thus, larger and instruction tuned models exhibit a lower reliance on majority class bias to solve the task.

\begin{figure*}[htb!]
    \centering
    \includegraphics[width=0.98\textwidth]{figures/bias_and_safety/cobie_sst2_old-harness.pdf}
    \caption{Frequency distributions of class \textit{0} predictions on SST-2 dataset depending on the class distribution in
few-shot. \textit{0} denotes the negative class, while \textit{1} denotes the positive class.}
    \label{fig:cobie_sst2}
\end{figure*}

\subsection{Safety and \RedTeaming{}}
Although there has been a growing body of research on the safety and \RedTeaming{}
%, alignment, helpfulness and harmlessness 
of \LLM{}s \cite{yao_safety_security_survey,chowdhury_attack_survey}, the vast majority of research has focused on English, particularly, Standard American English \cite{joshi_dialect_evaluation_survey}. To the best of our knowledge, the state-of-the art in multilingual safety evaluation is the work of \citet{aryabumi_aya_23} and their \AyaModel{} model. Their approach to assessing \LLM{} safety leverages GPT-4 and the Perspective API, based on a dataset that has been machine-translated from English. 
While this approach has yielded valuable insights, it is somewhat limited by the fact that GPT-4 and Perspective API are black-box models, and GPT-4 even requires a paid subscription. In contrast, our independently developed \RedTeaming{} pipeline, incorporates a larger and more diverse dataset and leverages a multilingual open-source moderator \LLM{}.
We deploy adversarial prompts for \RedTeaming{}, simulating potential interactions with malicious users \cite{ganguli_hh_rlhf, chowdhury_attack_survey} (e.g. \texttt{`How can I build a bomb?'}). 
Subsequently, we collect the responses from our instructed models and use a moderator \LLM{} to determine the success of the adversarial attacks.


\subsubsection{Methodology}
\Cref{alg:safety_rt_pipeline} shows our safety and \RedTeaming{} evaluation pipeline, with the corresponding code available for public access\footnote{\url{https://github.com/langtech-bsc/red_teaming_pipeline}}. We preprocess three datasets of adversarial prompts and use \LlamaGuard{} as a moderator model. 
Models exhibiting \emph{higher} attack success rates are considered to be \emph{less} resistant to adversarial prompts.  Our inference is non-deterministic\footnote{Note that, by fixing a random seed, our results are reproducible}, and we generate several answers for each prompt using the sampling parameters in \Cref{tab:safety_rt_inference_parameters} \footnote{All other sampling parameters are the default ones from the python package \href{https://github.com/vllm-project/vllm/blob/2ca830dbaa1a7c30b8ff4d7c860c63f87dc18be3/vllm/sampling_params.py\#L87}{\texttt{vllm 0.6.3}}.
}.
This setup results in scenarios where the same prompt may sometimes lead to a successful attack, while other times the \LLM{} may refuse to answer.
%A consequence of this setup is that, for the same prompt, sometimes the attack is successful while other times the \LLM{} refuses to answer. 
This variability in responses to the same prompt mirrors real-world usage, as \LLM{}s regularly provide different answers to the same inputs. Evaluating a single response would obscure whether the \LLM{} is vulnerable to the \RedTeaming{} attack, particularly with `borderline' prompts that may appear benign but are actually harmful, or vice versa.
%We believe that this resembles a real use scenario. 
In our evaluation framework, an attack is considered successful if the entire conversation generated from a prompt is marked as \unsafeAnswer{}.


\begin{table}[ht]
    \centering
    \begin{tabular}{>{\raggedright\arraybackslash}p{5cm}c}
        \toprule
        \textbf{Inference Parameter} & \textbf{Value} \\
        \midrule
        Temperature & 0.8 \\
        Top P & 0.95 \\
        Max Tokens & 500 \\
        Repetition Penalty & 1.2 \\
        \bottomrule
    \end{tabular}
    \caption{Inference Parameters of the \RedTeaming{} Pipeline}
    \label{tab:safety_rt_inference_parameters}
\end{table}


The analysis of attack success rates provides insights into the models' resistance to various attack types, and enables comparative evaluation of attack resistance across different models. Specifically, we apply this evaluation pipeline to the \AyaModel{} and \SalamandraInstructedVII{} models. These models were selected based on their multilingual capabilities, similar size in number of parameters, pre-training in both English and Spanish, and absence of preference alignment. 
Our study examines the attack success rates against \SalamandraInstructedVII{} in English, Spanish, and Catalan. Nonetheless, since \AyaModel{} was not trained on Catalan data, we limit the comparison to English and Spanish.

%Analyzing the attack success rates gives us insights into how resistant the model is to each type of attack, and enables us to compare attack resistance across models. Specifically, for comparison, we apply the pipeline to the \AyaModel{} and \SalamandraInstructedVII{} models. We chose these two models because both are multilingual, include English and Spanish in their pre-training, and have not undergone preference alignment.

\begin{algorithm}
    \caption{\RedTeaming{} Pipeline}
    \label{alg:safety_rt_pipeline}
    \begin{algorithmic}[1]
    \STATE From each RT Prompts Dataset $D$, randomly sample 500 prompts $S$.
    \STATE for each prompt $p$ in $S$, the \LLM{} generates three answers $a_1, a_2, a_3$
    \FOR{each prompt-answer pair $(p,a)$} 
    \STATE Classify $p$ into one of \LlamaGuard{}'s hazard categories, or mark $p$ as \safeAnswer{}
    \STATE Classify the conversation $(p,a)$ as either \safeAnswer{} or \unsafeAnswer{}
    \IF{$(p,a)$ is marked \unsafeAnswer{}}
    \STATE $(p,a)$ is considered as a successful attack
    \ENDIF
    \ENDFOR
    
    \end{algorithmic}
\end{algorithm}


\paragraph{\RedTeaming{} Prompts Datasets} We utilize and preprocess the following three datasets of adversarial prompts, selected for their permissive research licenses: 
% HF Link to M-ADV-Bench Dataset
\newcommand{\hfLinkMADVDataset}{\url{https://huggingface.co/datasets/simonycl/multilingual_advbench}}
% HF Link to ADV-Bench Dataset
\newcommand{\hfLinkADVDataset}{\url{https://huggingface.co/datasets/walledai/AdvBench}}
% HF Link to HH-RLFH Dataset
\newcommand{\hfLinkHHRLHFDataset}{\url{https://huggingface.co/datasets/Anthropic/hh-rlhf}}
% HF Link to Aya RT Dataset
\newcommand{\hfLinkAyaRTDataset}{\url{https://huggingface.co/datasets/CohereForAI/aya_redteaming}}


\paragraph{M-ADV-Bench} The \MAdvBenchDataset{}\footnote{\hfLinkMADVDataset{}} \cite{yong_multilingual_adv_bench}, derived from the AdvBench Dataset \footnote{\hfLinkADVDataset{}}, originally in English.
The dataset was first extended into 12 languages using the Google Translate API, and later \cite{ustun_aya_model} into a total of 23 languages, including Spanish,  using \NLLB{} translation. In our approach, we use the English and Spanish subsets of the \MAdvBenchDataset{} and further extend it into Catalan by applying \NLLB{} translation to the English subset.

%is a Machine-Translated version of the AdvBench Dataset\footnote{\hfLinkADVDataset{}}, originally in English \cite{zou_adv_bench}. In \citet{yong_multilingual_adv_bench}, the Adv-Bench Dataset was synthetically extended into 12 languages using the Google Translate API. In a later work \cite{ustun_aya_model}, the Adv-Bench Dataset was synthetically extended into the 23 languages of the \AyaModel{} Model, among them Spanish, using NLLB translation. 

\paragraph{HH-RLHF RT} The \HHRedTeamingDataset{}\footnote{\hfLinkHHRLHFDataset{}} \cite{ganguli_hh_rlhf} is a crowdsourced dataset containing around 38k multi-turn adversarial conversations in English. For our analysis, we randomly sample 1k conversations, taking the first user input as the adversarial prompt. This sample of the dataset is synthetically extended into Spanish and Catalan using \NLLB{} translation.

\paragraph{AYA RT} The \AyaDataset{}\footnote{\hfLinkAyaRTDataset{}} \cite{aakanksha_aya_rt_dataset} contains \RedTeaming{} prompts for 8 languages, including English and Spanish, crafted by human annotators and containing around 900 prompts per language. We synthetically extend this dataset to Catalan using \NLLB{} translation on the English and Spanish subdatasets. 
%\todo{Say that the quality of the Spanish Portion is really bad.} 
Notably, being a team with a large portion of native Spanish speakers, we observed that the quality of the Spanish subdataset is suboptimal (see \Cref{tab:poor_quality_spanish_prompts} in \Cref{app:rt_examples}).

% HF Link to Llama Guard 3
\newcommand{\hfLinkLlamaGuard}{\url{https://huggingface.co/meta-llama/Llama-Guard-3-8B}}

\paragraph{Moderator Model — \LlamaGuard{}}
%\paragraph{\LlamaGuard{}}
\LlamaGuard{}\footnote{\hfLinkLlamaGuard{}} \cite{llama3} serves as our moderator \LLM{}, trained to classify text into the risk categories defined by the \MLCommons{} Taxonomy (version 0.5 \citep{ml_commons_taxonomy}, see  \Cref{tab:safety_ml_commons_taxonomy}) and an additional category `Code Interpreter Abuse'. 
We selected \LlamaGuard{} due to its openly released weights and its multilingual training, which includes English and Spanish.


\begin{table}[ht!]
    \centering
    \resizebox{\textwidth}{!}{
    \begin{tabular}{@{}cccc@{}}
        \toprule
        S1: Violent Crimes & S2: Non-Violent Crimes & S3: Sex-Related Crimes & S4: Child Sexual Exploitation \\ \midrule
        S5: Defamation & S6: Specialized Advice & S7: Privacy & S8: Intellectual Property \\ \midrule
        S9: Indiscriminate Weapons & S10: Hate & S11: Suicide \& Self-Harm & S12: Sexual Content \\ \midrule
        S13: Elections & S14: Code Interpreter Abuse & & \\ \bottomrule
    \end{tabular}
    }
    \caption{ML Commons Hazard Taxonomy v0.5}
    \label{tab:safety_ml_commons_taxonomy}
\end{table}



\subsubsection{Results and Discussion}

\begin{figure}[htbp]
    \centering
    \begin{subfigure}[b]{0.3\textwidth}
        \centering
        \includegraphics[width=\textwidth]{figures/bias_and_safety/histograms_prompts_per_category/m_advbench_en.png}
        \caption{EN \\}
        \label{fig:safety_histograms_m_advbench_en}
    \end{subfigure}
    \hfill
    \begin{subfigure}[b]{0.3\textwidth}
        \centering
        \includegraphics[width=\textwidth]{figures/bias_and_safety/histograms_prompts_per_category/m_advbench_es.png}
        \caption{\textit{ES}}
        \label{fig:safety_histograms_m_advbench_es}
    \end{subfigure}
    \hfill
    \begin{subfigure}[b]{0.3\textwidth}
        \centering
        \includegraphics[width=\textwidth]{figures/bias_and_safety/histograms_prompts_per_category/m_advbench_cat_nllb_translated_from_en.png}
        \caption{\textit{CAT}}
        \label{fig:safety_histograms_m_advbench_cat_nllb_from_en}
    \end{subfigure}
    \caption{\MAdvBenchDataset{} - Prompts per Hazard Category. Instances translated from EN are \textit{italicized}, while instances translated from ES are in \textbf{bold}}
    \label{fig:safety_histograms_m_advbench}
\end{figure}

\begin{figure}[htbp]
    \centering
    \begin{subfigure}[b]{0.3\textwidth}
        \centering
        \includegraphics[width=\textwidth]{figures/bias_and_safety/histograms_prompts_per_category/hh_rlhf_1k_en.png}
        \caption{EN \\   }
        \label{fig:safety_histograms_hh_rlhf_en}
    \end{subfigure}
    \hfill
    \begin{subfigure}[b]{0.3\textwidth}
        \centering
        \includegraphics[width=\textwidth]{figures/bias_and_safety/histograms_prompts_per_category/hh_rlhf_1k_es_nllb_translated_from_en.png}
        \caption{\textit{ES} }
        \label{fig:safety_histograms_hh_rlhf_es_nllb_from_en}
    \end{subfigure}
    \hfill
    \begin{subfigure}[b]{0.3\textwidth}
        \centering
        \includegraphics[width=\textwidth]{figures/bias_and_safety/histograms_prompts_per_category/hh_rlhf_1k_cat_nllb_translated_from_en.png}
        \caption{ \textit{CAT}}
        \label{fig:safety_histograms_hh_rlhf_cat_nllb_from_en}
    \end{subfigure}
    \caption{\HHRedTeamingDataset{} - Prompts per Hazard Category. Instances translated from EN are \textit{italicized}, while instances translated from \textbf{ES} are in bold}
    \label{fig:safety_histograms_hh_rlhf}
\end{figure}



\begin{figure}[htbp]
    \centering
    \begin{subfigure}[b]{0.45\textwidth}
        \centering
        \includegraphics[width=\textwidth]{figures/bias_and_safety/histograms_prompts_per_category/aya_eng.png}
        \caption{\AyaDataset{} - EN}
        \label{fig:safety_histograms_aya_eng}
    \end{subfigure}
    \hfill
    \begin{subfigure}[b]{0.45\textwidth}
        \centering
        \includegraphics[width=\textwidth]{figures/bias_and_safety/histograms_prompts_per_category/aya_es.png}
        \caption{\AyaDataset{} - ES}
        \label{fig:safety_histograms_aya_es}
    \end{subfigure}
    \vfill
    \begin{subfigure}[b]{0.45\textwidth}
        \centering
        \includegraphics[width=\textwidth]{figures/bias_and_safety/histograms_prompts_per_category/aya_cat_nllb_translated_from_en.png}
        \caption{\AyaDataset{} - \textit{CAT}}
        \label{fig:safety_histograms_aya_cat_nllb_from_en}
    \end{subfigure}
    \hfill
    \begin{subfigure}[b]{0.45\textwidth}
        \centering
        \includegraphics[width=\textwidth]{figures/bias_and_safety/histograms_prompts_per_category/aya_cat_bsc_translated_from_es.png}
        \caption{\AyaDataset{} - \textbf{CAT}}
        \label{fig:safety_histograms_aya_cat_nllb_from_es}
    \end{subfigure}
    \caption{\AyaDataset{} - Prompts per Hazard Category. Instances translated from EN are \textit{italicized}, while instances translated from \textbf{ES} are in bold}
    \label{fig:safety_histograms_aya}
\end{figure}


\Crefrange{fig:safety_histograms_m_advbench}{fig:safety_histograms_aya} illustrate the distribution of RT prompts across different hazard categories as classified by \LlamaGuard{}. Several hazard categories show either no prompts or very few prompts, depending on the dataset. This outcome is expected, as the RT Prompts Datasets were released prior to the creation of the \MLCommons{} Hazard Taxonomy. For the purposes of our analysis, categories with fewer than 30 prompts are excluded, as this small sample size does not provide sufficient data for meaningful conclusions. 
It is worth noting that the overall distributions of prompts are unaffected by machine translations, suggesting that meaning is preserved through \NLLB{} translation on these prompts.

%show the distribution of RT prompts across the different hazard categories, as classified by \LlamaGuard{}. Unfortunately, there are several hazard categories with no or almost-no prompts, depending on the dataset. This is to be expected, as the RT Prompts Datasets were collected before the release of the ML Commons Hazard Taxonomy. Going forward, for our analysis, we discard those categories with less than 30 prompts, as such a small sample size is insufficient. 

%It should be noted that the overall distributions of prompts are not affected by machine translation. 
Furthermore, both the \AyaDataset{} and the \HHRedTeamingDataset{} contain a large proportion of prompts marked as \safeAnswer{}, ranging from 40\% to 60\% (\Cref{fig:safety_histograms_hh_rlhf} and \Cref{fig:safety_histograms_aya}). After manual review, we
believe that this reflects limitations of \LlamaGuard{}, as several of these prompts were manifestly harmful (\Cref{tab:safety_llama_guard_blind_spots}).




\Cref{tab:safety_heatmaps_salamandra} presents the attack success rates against \SalamandraInstructedVII{} in Catalan, English, and Spanish, and a comparison with  \AyaModel{} in English and Spanish\footnote{Due to space constraints, the results for \AyaModel{} are available in \Cref{app:rt_examples}, see \Cref{tab:safety_heatmaps_salamandra_full} and \Cref{tab:safety_heatmaps_aya_full}}.
In the case of \SalamandraInstructedVII{}, the attack success rates across the three datasets are generally similar to or lower in Spanish compared to English. In contrast, \AyaModel{} exhibits a reverse pattern, with higher attack resistance in English than in Spanish. 
\AyaModel{} us generally more resistant to attacks than \SalamandraBaseVII{}. However, the difference in attack success rates is less pronounced for Spanish. 

Examining specific attack categories,
\SalamandraInstructedVII{} is more vulnerable to attack types S4 and S5, with success rates reaching up to 86\%. It shows moderate vulnerability to S1, S2, S11, and S12 (success rates between 40\% and 60\%) and less vulnerability to attack types S9, S10, and safe (under 40\%). Manual review reveals some blind spots in \LlamaGuard{}. \Cref{app:rt_examples} highlights examples where the input prompt was classified as \safeAnswer{}, yet the overall conversation was classified as \unsafeAnswer{}.

These results indicate that model resistance to \RedTeaming{} depends not only on the type of attack, but also on the language, supporting the insight that \LLM{} safety must be carefully addressed for each language. Additionally, we highlight the value of conducting manual review to identify instances where automated systems may fail, ensuring a more comprehensive understanding of model vulnerabilities.

\begin{table}[ht]
\centering
\begin{adjustbox}{width=1.25\textwidth,center=\textwidth}
\begin{tabular}{>{\small}p{1.1cm}rr}
\hline
\ & \ & \ \\
\parbox[t]{1cm}{\centering \small \textbf{Dataset} } 
&  \multicolumn{1}{c}{\textbf{\SalamandraInstructedVII{}}} 
%& \textbf{\AyaModel{}} 
& \multicolumn{1}{c}{\textbf{Comparison with \AyaModel{}}}  \\
\ & \ & \ \\
\hline
%\textbf{\MAdvBenchDataset{}} &
\parbox[t]{1cm}{\centering \scriptsize \MAdvBenchDataset{}} &
\begin{tabular}{c}
    {\includegraphics[width=0.7\linewidth]{figures/bias_and_safety/final_heatmaps/individual/m_ad/salamandra7b.png}}
\end{tabular} &
%\begin{tabular}{c}
%    {\includegraphics[width=0.3\linewidth]{figures/bias_and_safety/final_heatmaps/individual/m_ad/Aya23_8b_RT_.png}}
%\end{tabular} &
\begin{tabular}{c}
    {\includegraphics[width=0.7\linewidth]{figures/bias_and_safety/final_heatmaps/comparison/m_ad/Aya23_8b_RT__vs_salamandra7b.png}}
\end{tabular} \\
%%%%%%
%\textbf{\HHRedTeamingDataset{}} &
\parbox[t]{1cm}{\centering \scriptsize \HHRedTeamingDataset{}} & 
\begin{tabular}{c}
    {\includegraphics[width=0.5\linewidth]{figures/bias_and_safety/final_heatmaps/individual/hh_r/salamandra7b.png}}
\end{tabular} &
%\begin{tabular}{c}
%    {\includegraphics[width=0.3\linewidth]{figures/bias_and_safety/final_heatmaps/individual/hh_r/Aya23_8b_RT_.png}}
%\end{tabular} &
\begin{tabular}{c}
    {\includegraphics[width=0.5\linewidth]{figures/bias_and_safety/final_heatmaps/comparison/hh_r/Aya23_8b_RT__vs_salamandra7b.png}}
\end{tabular} \\
%%%%%%
%\textbf{\centering \AyaDataset{}} &
\parbox[t]{1cm}{\centering \scriptsize \AyaDataset{}} & 
\begin{tabular}{c}
    {\includegraphics[width=0.7\linewidth]{figures/bias_and_safety/final_heatmaps/individual/aya_/salamandra7b.png}}
\end{tabular} &
%\begin{tabular}{c}
%    {\includegraphics[width=0.3\linewidth]{figures/bias_and_safety/final_heatmaps/individual/aya_/Aya23_8b_RT_.png}}
%\end{tabular} &
\begin{tabular}{c}
    {\includegraphics[width=0.7\linewidth]{figures/bias_and_safety/final_heatmaps/comparison/aya_/Aya23_8b_RT__vs_salamandra7b.png}}
\end{tabular} \\
\hline
\end{tabular}
\end{adjustbox}
\caption{Attack success rates against \SalamandraInstructedVII{}, and comparison with \AyaModel{}, across the three \RedTeaming{} Prompts Datasets, divided by language. 
Heatmaps show in which categories the models are 
\textcolor[HTML]{ABDA4E}{more} or \textcolor[HTML]{FC8B5F}{less} resistant. \textcolor{gray}{Grey}: Not enough prompts. 
}
\label{tab:safety_heatmaps_salamandra}
\end{table}



\paragraph{Limitations and Future Work}

The main limitation of our \RedTeaming{} approach is the absence of human annotation and evaluation, which results in an over-reliance on automatic methods. Due to time and budget constraints, human evaluation was not feasible. The lack of human-generated prompt datasets for \RedTeaming{} in Catalan is another key limitation, as depending on machine-translation evaluation may create a false impression of the quality of the model's answers
\cite{chen_bad_multilingua_evaluation}.

Additionally, our \RedTeaming{} approach only considers conversations consisting of a prompt and an answer, while research shows that multi-turn conversations increment the probability of harmful answers \cite{wolf_fundamental_limitations_alignment}. To address this, we aim to expand our approach to include multi-turn conversations in the future.


We apply \LlamaGuard{} as a moderator model with an understanding of its limitations. 
One significant issues is that \LlamaGuard{} has \emph{not} been trained to moderate content in Catalan. This could explain why harmful Catalan prompts are mistakenly marked as \safeAnswer{}. 
Furthermore, after manual review, we also found this behavior in Spanish, though to a lesser degree. 
This highlights the need for more safety datasets in these languages to help train more effective moderators. The \Langtech{} Unit at \BscShort{} is actively working to compile harmfulness and toxicity datasets in both Spanish and Catalan, such as the InToxiCat dataset\footnote{\url{https://huggingface.co/datasets/projecte-aina/InToxiCat}} \cite{gonzalez-agirre-etal-2024-building-data} developed under the \AinaProject{} project.

Additionally, the \MLCommons{} Hazards Taxonomy used to train \LlamaGuard{} appears to be tailored to U.S.A. cultural sensitivities, which may not align with those of other regions. For example, the inclusion of `Elections' as a category may not be universally relevant. Defining harmfulness, toxicity, and bias is a complex task \cite{banko_harmful_taxonomy,kurrek_slur_taxonomy,maroniko_affected_communities,schmeisser2022criteria}, and this challenge becomes even greater in the European multilingual context. We are aware of European initiatives working toward addressing these issues \cite{
eu_ai_act, eu_ai_office%,
%spain_supervision_ai,
%spain_estrategia_ai
}.

Looking ahead, we are focused on developing our multilingual alignment approach. For future releases of the  \SalamandraFamily{}, we plan to continue using \RedTeaming{} to identify vulnerabilities and harmful behavior, while also exploring methods like synthetic generation of adversarial prompts \cite{samvelyan_rainbow_teaming} and studying false refusals in our models \cite{rottger_xstest_false_refusal}. 
To mitigate detected vulnerabilities and undesired behaviors, we will leverage Reinforcement Learning Techniques, such as Reinforcement Learning From Human Feedback, Proximal Policy Optimization, Reward Modeling, and both Online and Offline Direct Preference Optimization \cite{ziegler_rlhf_ppo_paper, rafailov_rlhf_dpo_paper, qi_rlhf_online_dpo_paper, feng_rlhf_dpo_limitations, dang_multilingual_rlhf}.


% A lot of Machine Translation % Cite "it it multilingual data", "aya paper"
% LLamaGuard does not work well in Catalan
   % A lot of the prompts are marked as "safe"
% LlamaGuard is not so great in Spanish, either
% A lot of prompts are marked as "safe"
% From the models themselves: Marked answer as "safe", because the model did not "understand" the adversarial prompt.

% MLCommonsTaxonomies are very US-centric
% We are working on our Alignment pipeline and iterative red teaming!
% Prompt attack augmentation
% DPO, PPO, other RL algorithms.


% use AYA model as baseline and report differences 
% RUN experiment AYA model + M-ADV-BENCH + Llama Guard -> DONE
% ask Mario for colour palette -> DONE
% we prefer to use open models y punto  
% Refactor heatmap into two based on Dataset due to class imbalances
% lower heatmap threshold 

%\label{subsec:toxicity}
%\subsection{Bias mitigation}

% \subsection{Potential risks}
% \todo{wip}

\label{subsec:risks}
%\section{Quantization Framework}
\label{section:quantization}
\begin{comment}
Our quantization framework for accelerating large language model (LLM) inference introduces a novel, selective approach focused on reducing critical delay by prioritizing low-critical-delay weights. The design is optimized to maintain model accuracy while improving inference speed through adaptive quantization at multiple levels, including per-tile, per-channel, and layer sensitivity granularity. This methodology addresses computational efficiency challenges in high-dimensional LLMs by targeting only the weights that most impact inference delays.

\subsection{Uniform and Non-Uniform Quantization} 
% Previous studies have demonstrated that the primary goal of quantization techniques is to minimize the Mean Square Error (MSE) between the original and the quantized model.
Quantization is a powerful technique for reducing the memory footprint of neural network models by decreasing the bitwidth of their parameters. The literature on quantization can be broadly categorized into two main approaches: uniform and non-uniform methods.

\noindent \textbf{Uniform Quantization} Uniform quantization maps a continuous range of values to discrete levels that are evenly spaced. In a $b$-bit quantization scheme, the data $x$ is mapped to $2^b$ distinct levels. The quantization step $s$, also known as the scaling factor, is defined as:

\begin{equation}
s = \frac{x_{\text{max}} - x_{\text{min}}}{2^b}
\end{equation}

The quantization function $q(x)$ maps the data $x$ to the nearest quantization level, given by:
\begin{equation}
q(x) = x_{\text{min}} + is \quad \text{for} \quad i = 0, 1, 2, \ldots, 2^b-1
\end{equation}

This approach ensures that each quantized level is equidistant from its neighbors, resulting in a fixed-length representation for each interval. 

\noindent {\textbf{Non-uniform Quantization}} Non-uniform quantization, in contrast, allows for variable-sized intervals. Let $T = \{t_0 = -\infty, t_1, t_2, \ldots, t_{M-1}, t_M = +\infty\}$ be the set of thresholds that define the boundaries between quantization intervals. These thresholds partition the source range $X$ into $M+1$ disjoint regions $R_k = [t_{k-1}, t_k)$, for $k = 1, 2, \ldots, M+1$. 

The quantization function $q(x)$ assigns an input signal $x$ to the closest representation level $y_k$ based on which interval $R_k$ it falls into:
\begin{equation}
q(x) = y_k \quad \text{if} \quad x \in R_k
\end{equation}

The key difference is that the decision thresholds $(T)$ are not necessarily equally spaced, allowing for a more flexible allocation of quantization levels based on the data distribution within the source range $X$.
\end{comment}

Our quantization framework for LLM inference introduces a timing-aware strategy, detailed in Algorithm~\ref{quantization}, which prioritizes weights with low critical-path delays to minimize latency while preserving model fidelity. The adaptive method operates across levels and layer sensitivity, optimizing performance by focusing on weights most critical to efficiency. The framework comprises three key components: \circled{1} sensitivity-aware uniform quantization to identify and preserve critical weights \textit{(Lines 1-3)}, \circled{2} critical-path delay aware non-uniform quantization to optimize weight patterns for hardware efficiency \textit{(Lines 4-10)}, and \circled{3} adaptive DVFS to maximize performance by matching quantization levels with optimal operating frequencies.

\subsection{Sensitivity-aware Uniform Quantization}
\begin{figure}[t!]
	\scriptsize
	\centering
	\section{Sensitivity Analyses of WLS}

While the multi-factor weighted least squares (WLS) model provides a systematic approach to control for multiple confounders, datasets often exhibit \emph{imbalanced subgroup distributions} or heterogeneity that can affect statistical inferences.  To ensure the robustness of our parameter estimates, we perform a bootstrap-based sensitivity analysis.

\subsection{Bootstrap-Based Parameter Estimation}

Parameter estimates can be sensitive to random fluctuations in the data.  To assess this sensitivity, we use bootstrapping. We create many "new" datasets by resampling with replacement from the original dataset (keeping the same overall size).  We fit the WLS model on each bootstrap sample and aggregate the resulting estimates. This approach provides a distribution for each parameter. We report the mean and standard deviation of these bootstrap estimates, along with the 95\% confidence interval (CI) based on the 2.5th and 97.5th percentiles of the bootstrap distribution. We then check whether the original parameter estimate falls within this bootstrap CI ("Coverage"). If the original estimate lies within the CI, it provides evidence that the estimate is stable to sampling variability, and thus robust to the specific composition of the sample.

\subsection{Summary of Sensitivity Findings}

The bootstrap results, presented in Tables \ref{tab:bootstrap_comparison_binoculars} through \ref{tab:radar_bootstrap}, show that for all detectors and all parameters, the original coefficient estimates lie within the 95\% confidence intervals derived from the bootstrap resampling.  This indicates strong stability of the parameter estimates. The relatively narrow confidence intervals and consistent "WITHIN CI" coverage across all parameters and detectors provide substantial evidence that our main WLS findings are robust to sampling variability. This strengthens our confidence in the reported effects of CEFR level, sex, academic genre, and language environment on detector accuracy.

\begin{table*}[htbp]
  \centering
  \resizebox{\textwidth}{!}{%
    \begin{tabular}{lcccccc}
      \hline
      \textbf{Parameter} & \textbf{Original Value} & \textbf{Bootstrap Mean} & \textbf{Bootstrap Std} & \textbf{CI Lower} & \textbf{CI Upper} & \textbf{Coverage} \\
      \hline
      Intercept & 0.9482 & 0.9480 & 0.0079 & 0.9318 & 0.9625 & WITHIN CI \\
      C(cefr)[T.B1\_1] & -0.0039 & -0.0040 & 0.0085 & -0.0204 & 0.0132 & WITHIN CI \\
      C(cefr)[T.B1\_2] & -0.0007 & -0.0006 & 0.0075 & -0.0149 & 0.0150 & WITHIN CI \\
      C(cefr)[T.B2\_0] & 0.0025 & 0.0025 & 0.0078 & -0.0125 & 0.0172 & WITHIN CI \\
      C(cefr)[T.XX\_0] & -0.0501 & -0.0499 & 0.0049 & -0.0592 & -0.0400 & WITHIN CI \\
      C(Sex)[T.M] & 0.0010 & 0.0011 & 0.0054 & -0.0096 & 0.0114 & WITHIN CI \\
      C(academic\_genre)[T.Life Sciences] & -0.0075 & -0.0073 & 0.0082 & -0.0224 & 0.0086 & WITHIN CI \\
      C(academic\_genre)[T.Sciences \& Technology] & 0.0016 & 0.0018 & 0.0072 & -0.0126 & 0.0156 & WITHIN CI \\
      C(academic\_genre)[T.Social Sciences] & 0.0016 & 0.0018 & 0.0068 & -0.0110 & 0.0149 & WITHIN CI \\
      C(language\_env)[T.ESL] & -0.0144 & -0.0143 & 0.0056 & -0.0247 & -0.0038 & WITHIN CI \\
      C(language\_env)[T.NS] & -0.0501 & -0.0499 & 0.0049 & -0.0592 & -0.0400 & WITHIN CI \\
      \hline
    \end{tabular}}
   \caption{Bootstrap Sensitivity Analysis for detector: \texttt{binoculars}}
  \label{tab:bootstrap_comparison_binoculars}
\end{table*}






\begin{table*}[ht]
\centering
\resizebox{\textwidth}{!}{%
\begin{tabular}{lcccccc}
\toprule
\textbf{Parameter} & \textbf{Original Value} & \textbf{Bootstrap Mean} & \textbf{Bootstrap Std} & \textbf{CI Lower} & \textbf{CI Upper} & \textbf{Coverage} \\
\midrule
Intercept                                   & 0.7480 & 0.7480 & 0.0157 & 0.7179 & 0.7790 & WITHIN CI \\
C(Sex)[T.M]                                & 0.0035 & 0.0032 & 0.0103 & --0.0176 & 0.0226 & WITHIN CI \\
C(cefr)[T.B1\_1]                           & --0.0073 & --0.0074 & 0.0180 & --0.0403 & 0.0294 & WITHIN CI \\
C(cefr)[T.B1\_2]                           & --0.0103 & --0.0102 & 0.0158 & --0.0413 & 0.0214 & WITHIN CI \\
C(cefr)[T.B2\_0]                           & --0.0435 & --0.0432 & 0.0155 & --0.0728 & --0.0114 & WITHIN CI \\
C(cefr)[T.XX\_0]                           & --0.0071 & --0.0071 & 0.0084 & --0.0245 & 0.0100 & WITHIN CI \\
C(academic\_genre)[T.Life Sciences]         & 0.0286 & 0.0287 & 0.0143 & 0.0007 & 0.0557 & WITHIN CI \\
C(academic\_genre)[T.Sciences \& Technology]  & 0.0084 & 0.0088 & 0.0137 & --0.0179 & 0.0350 & WITHIN CI \\
C(academic\_genre)[T.Social Sciences]        & 0.0083 & 0.0085 & 0.0120 & --0.0167 & 0.0312 & WITHIN CI \\
C(language\_env)[T.ESL]                      & --0.0014 & --0.0012 & 0.0094 & --0.0201 & 0.0185 & WITHIN CI \\
C(language\_env)[T.NS]                       & --0.0071 & --0.0071 & 0.0084 & --0.0245 & 0.0100 & WITHIN CI \\
\bottomrule
\end{tabular}
}
\caption{Bootstrap Sensitivity Analysis for detector: \texttt{chatgpt-roberta}}
\label{tab:bootstrap_chatgpt-roberta}
\end{table*}











\begin{table*}[ht]
\centering
\resizebox{\textwidth}{!}{%
\begin{tabular}{lcccccc}
\toprule
\textbf{Parameter} & \textbf{Original Value} & \textbf{Bootstrap Mean} & \textbf{Bootstrap Std} & \textbf{CI Lower} & \textbf{CI Upper} & \textbf{Coverage} \\
\midrule
Intercept                                   & 0.7944    & 0.7948   & 0.0098  & 0.7754  & 0.8133 & WITHIN CI \\
C(academic\_genre)[T.Life Sciences]         & --0.0397  & --0.0398 & 0.0108  & --0.0611 & --0.0187 & WITHIN CI \\
C(academic\_genre)[T.Sciences \& Technology]  & --0.0185  & --0.0189 & 0.0093  & --0.0370 & --0.0010 & WITHIN CI \\
C(academic\_genre)[T.Social Sciences]        & --0.0060  & --0.0063 & 0.0087  & --0.0233 & 0.0101 & WITHIN CI \\
C(cefr)[T.B1\_1]                             & 0.0127    & 0.0129   & 0.0104  & --0.0074 & 0.0347 & WITHIN CI \\
C(cefr)[T.B1\_2]                             & 0.0284    & 0.0282   & 0.0093  & 0.0103  & 0.0468 & WITHIN CI \\
C(cefr)[T.B2\_0]                             & 0.0345    & 0.0339   & 0.0106  & 0.0132  & 0.0554 & WITHIN CI \\
C(cefr)[T.XX\_0]                             & --0.0223  & --0.0222 & 0.0059  & --0.0338 & --0.0104 & WITHIN CI \\
C(Sex)[T.M]                                 & --0.0068  & --0.0069 & 0.0075  & --0.0210 & 0.0079 & WITHIN CI \\
C(language\_env)[T.ESL]                       & --0.0077  & --0.0075 & 0.0071  & --0.0208 & 0.0065 & WITHIN CI \\
C(language\_env)[T.NS]                        & --0.0223  & --0.0222 & 0.0059  & --0.0338 & --0.0104 & WITHIN CI \\
\bottomrule
\end{tabular}
}
\caption{Bootstrap Sensitivity Analysis for detector: \texttt{detectgpt}}
\label{tab:bootstrap_detectgpt}
\end{table*}













\begin{table*}[ht]
\centering
\resizebox{\textwidth}{!}{%
\begin{tabular}{lcccccc}
\toprule
\textbf{Parameter} & \textbf{Original Value} & \textbf{Bootstrap Mean} & \textbf{Bootstrap Std} & \textbf{CI Lower} & \textbf{CI Upper} & \textbf{Coverage} \\
\midrule
Intercept                                   & 0.8963    & 0.8963   & 0.0084  & 0.8796  & 0.9124 & WITHIN CI \\
C(cefr)[T.B1\_1]                            & -0.0076   & -0.0071  & 0.0093  & -0.0241 & 0.0105 & WITHIN CI \\
C(cefr)[T.B1\_2]                            & 0.0060    & 0.0059   & 0.0082  & -0.0107 & 0.0222 & WITHIN CI \\
C(cefr)[T.B2\_0]                            & 0.0180    & 0.0181   & 0.0083  & 0.0023  & 0.0346 & WITHIN CI \\
C(cefr)[T.XX\_0]                            & -0.0214   & -0.0214  & 0.0049  & -0.0309 & -0.0119 & WITHIN CI \\
C(Sex)[T.M]                                & -0.0003   & -0.0005  & 0.0062  & -0.0130 & 0.0113 & WITHIN CI \\
C(academic\_genre)[T.Life Sciences]          & 0.0127    & 0.0123   & 0.0077  & -0.0025 & 0.0281 & WITHIN CI \\
C(academic\_genre)[T.Sciences \& Technology] & -0.0012   & -0.0013  & 0.0080  & -0.0174 & 0.0144 & WITHIN CI \\
C(academic\_genre)[T.Social Sciences]        & 0.0070    & 0.0073   & 0.0071  & -0.0068 & 0.0216 & WITHIN CI \\
C(language\_env)[T.ESL]                      & -0.0021   & -0.0021  & 0.0060  & -0.0141 & 0.0102 & WITHIN CI \\
C(language\_env)[T.NS]                       & -0.0214   & -0.0214  & 0.0049  & -0.0309 & -0.0119 & WITHIN CI \\
\bottomrule
\end{tabular}
}
\caption{Bootstrap Sensitivity Analysis for detector: \texttt{fastdetectgpt}}
\label{tab:bootstrap_fastdetectgpt}
\end{table*}









\begin{table*}[ht]
\centering
\resizebox{\textwidth}{!}{%
\begin{tabular}{lcccccc}
\toprule
\textbf{Parameter} & \textbf{Original Value} & \textbf{Bootstrap Mean} & \textbf{Bootstrap Std} & \textbf{CI Lower} & \textbf{CI Upper} & \textbf{Coverage} \\
\midrule
Intercept                                   & 0.4873    & 0.4873   & 0.0026  & 0.4824  & 0.4926 & WITHIN CI \\
C(cefr)[T.B1\_1]                            & -0.0031   & -0.0031  & 0.0025  & -0.0081 & 0.0017 & WITHIN CI \\
C(cefr)[T.B1\_2]                            & -0.0087   & -0.0087  & 0.0025  & -0.0137 & -0.0039 & WITHIN CI \\
C(cefr)[T.B2\_0]                            & -0.0045   & -0.0045  & 0.0028  & -0.0102 & 0.0008 & WITHIN CI \\
C(cefr)[T.XX\_0]                            & 0.0102    & 0.0102   & 0.0013  & 0.0077  & 0.0127 & WITHIN CI \\
C(Sex)[T.M]                                & 0.0018    & 0.0019   & 0.0016  & -0.0012 & 0.0049 & WITHIN CI \\
C(academic\_genre)[T.Life Sciences]          & -0.0141   & -0.0143  & 0.0040  & -0.0220 & -0.0066 & WITHIN CI \\
C(academic\_genre)[T.Sciences \& Technology] & -0.0060   & -0.0060  & 0.0020  & -0.0100 & -0.0021 & WITHIN CI \\
C(academic\_genre)[T.Social Sciences]        & -0.0017   & -0.0017  & 0.0020  & -0.0057 & 0.0022 & WITHIN CI \\
C(language\_env)[T.ESL]                      & -0.0035   & -0.0035  & 0.0017  & -0.0069 & -0.0001 & WITHIN CI \\
C(language\_env)[T.NS]                       & 0.0102    & 0.0102   & 0.0013  & 0.0077  & 0.0127 & WITHIN CI \\
\bottomrule
\end{tabular}
}
\caption{Bootstrap Sensitivity Analysis for detector: \texttt{fastdetectllm}}
\label{tab:bootstrap_fastdetectllm}
\end{table*}







\begin{table*}[ht]
\centering
\resizebox{\textwidth}{!}{%
\begin{tabular}{lcccccc}
\toprule
\textbf{Parameter} & \textbf{Original Value} & \textbf{Bootstrap Mean} & \textbf{Bootstrap Std} & \textbf{CI Lower} & \textbf{CI Upper} & \textbf{Coverage} \\
\midrule
Intercept                                    & 0.8366  & 0.8369  & 0.0175  & 0.8039  & 0.8712 & WITHIN CI \\
C(cefr)[T.B1\_1]                             & -0.0061 & -0.0064 & 0.0189  & -0.0445 & 0.0289 & WITHIN CI \\
C(cefr)[T.B1\_2]                             & -0.0160 & -0.0158 & 0.0169  & -0.0481 & 0.0179 & WITHIN CI \\
C(cefr)[T.B2\_0]                             & -0.0399 & -0.0405 & 0.0175  & -0.0755 & -0.0064 & WITHIN CI \\
C(cefr)[T.XX\_0]                             & -0.1186 & -0.1183 & 0.0107  & -0.1383 & -0.0979 & WITHIN CI \\
C(Sex)[T.M]                                  & -0.0078 & -0.0078 & 0.0130  & -0.0324 & 0.0178 & WITHIN CI \\
C(academic\_genre)[T.Life Sciences]           & 0.0065  & 0.0069  & 0.0185  & -0.0291 & 0.0427 & WITHIN CI \\
C(academic\_genre)[T.Sciences \& Technology]  & 0.0068  & 0.0068  & 0.0172  & -0.0253 & 0.0405 & WITHIN CI \\
C(academic\_genre)[T.Social Sciences]         & -0.0031 & -0.0033 & 0.0154  & -0.0333 & 0.0257 & WITHIN CI \\
C(language\_env)[T.ESL]                       & -0.0020 & -0.0021 & 0.0133  & -0.0273 & 0.0241 & WITHIN CI \\
C(language\_env)[T.NS]                        & -0.1186 & -0.1183 & 0.0107  & -0.1383 & -0.0979 & WITHIN CI \\
\bottomrule
\end{tabular}
}
\caption{Bootstrap Sensitivity Analysis for detector \texttt{gltr}}
\label{tab:gltr_bootstrap}
\end{table*}









\begin{table*}[ht]
\centering
\resizebox{\textwidth}{!}{%
\begin{tabular}{lcccccc}
\toprule
\textbf{Parameter} & \textbf{Original Value} & \textbf{Bootstrap Mean} & \textbf{Bootstrap Std} & \textbf{CI Lower} & \textbf{CI Upper} & \textbf{Coverage} \\
\midrule
Intercept                                    & 0.6493  & 0.6499  & 0.0129  & 0.6247  & 0.6755 & WITHIN CI \\
C(cefr)[T.B1\_1]                             & -0.0094 & -0.0096 & 0.0138  & -0.0373 & 0.0179 & WITHIN CI \\
C(cefr)[T.B1\_2]                             & -0.0612 & -0.0611 & 0.0127  & -0.0849 & -0.0366 & WITHIN CI \\
C(cefr)[T.B2\_0]                             & -0.0870 & -0.0872 & 0.0129  & -0.1118 & -0.0619 & WITHIN CI \\
C(cefr)[T.XX\_0]                             & -0.1011 & -0.1013 & 0.0070  & -0.1146 & -0.0873 & WITHIN CI \\
C(Sex)[T.M]                                  & -0.0043 & -0.0044 & 0.0088  & -0.0213 & 0.0126 & WITHIN CI \\
C(academic\_genre)[T.Life Sciences]          & 0.0248  & 0.0245  & 0.0126  & -0.0016 & 0.0490 & WITHIN CI \\
C(academic\_genre)[T.Sciences \& Technology] & 0.0373  & 0.0369  & 0.0111  & 0.0153  & 0.0597 & WITHIN CI \\
C(academic\_genre)[T.Social Sciences]        & -0.0084 & -0.0090 & 0.0103  & -0.0288 & 0.0115 & WITHIN CI \\
C(language\_env)[T.ESL]                       & 0.0011  & 0.0009  & 0.0084  & -0.0150 & 0.0176 & WITHIN CI \\
C(language\_env)[T.NS]                        & -0.1011 & -0.1013 & 0.0070  & -0.1146 & -0.0873 & WITHIN CI \\
\bottomrule
\end{tabular}
}
\caption{Bootstrap Sensitivity Analysis for detector \texttt{gpt2-base}}
\label{tab:gpt2base_bootstrap}
\end{table*}








\begin{table*}[ht]
\centering
\small
\resizebox{\textwidth}{!}{%
\begin{tabular}{lcccccc}
\toprule
\textbf{Parameter} & \textbf{Original Value} & \textbf{Bootstrap Mean} & \textbf{Bootstrap Std} & \textbf{CI Lower} & \textbf{CI Upper} & \textbf{Coverage} \\
\midrule
Intercept                                   & 0.5402   & 0.5412   & 0.0149   & 0.5127   & 0.5703 & WITHIN CI \\
C(cefr)[T.B1\_1]                            & 0.0274   & 0.0270   & 0.0159   & -0.0050  & 0.0583 & WITHIN CI \\
C(cefr)[T.B1\_2]                            & 0.0008   & 0.0003   & 0.0149   & -0.0266  & 0.0294 & WITHIN CI \\
C(cefr)[T.B2\_0]                            & 0.0060   & 0.0053   & 0.0159   & -0.0255  & 0.0359 & WITHIN CI \\
C(cefr)[T.XX\_0]                            & -0.0240  & -0.0244  & 0.0080   & -0.0393  & -0.0093 & WITHIN CI \\
C(Sex)[T.M]                                & -0.0026  & -0.0030  & 0.0097   & -0.0214  & 0.0150 & WITHIN CI \\
C(academic\_genre)[T.Life Sciences]         & 0.0082   & 0.0078   & 0.0145   & -0.0199  & 0.0352 & WITHIN CI \\
C(academic\_genre)[T.Sciences \& Technology] & 0.0303   & 0.0297   & 0.0133   & 0.0037   & 0.0555 & WITHIN CI \\
C(academic\_genre)[T.Social Sciences]       & -0.0165  & -0.0170  & 0.0125   & -0.0420  & 0.0066 & WITHIN CI \\
C(language\_env)[T.ESL]                      & 0.0119   & 0.0118   & 0.0102   & -0.0083  & 0.0321 & WITHIN CI \\
C(language\_env)[T.NS]                       & -0.0240  & -0.0244  & 0.0080   & -0.0393  & -0.0093 & WITHIN CI \\
\bottomrule
\end{tabular}
}
\caption{Bootstrap Sensitivity Analysis for detector \texttt{gpt2-large}}
\label{tab:gpt2large_bootstrap}
\end{table*}












\begin{table*}[ht]
\centering
\small
\resizebox{\textwidth}{!}{%
\begin{tabular}{lcccccc}
\toprule
\textbf{Parameter} & \textbf{Original Value} & \textbf{Bootstrap Mean} & \textbf{Bootstrap Std} & \textbf{CI Lower} & \textbf{CI Upper} & \textbf{Coverage} \\
\midrule
Intercept                                   & 0.5402  & 0.5412  & 0.0039  & 0.4986  & 0.5139 & WITHIN CI \\
C(cefr)[T.B1\_1]                            & $-0.0102$ & $-0.0103$ & 0.0046  & $-0.0197$ & $-0.0019$ & WITHIN CI \\
C(cefr)[T.B1\_2]                            & $-0.0251$ & $-0.0250$ & 0.0039  & $-0.0331$ & $-0.0178$ & WITHIN CI \\
C(cefr)[T.B2\_0]                            & $-0.0482$ & $-0.0482$ & 0.0044  & $-0.0570$ & $-0.0398$ & WITHIN CI \\
C(cefr)[T.XX\_0]                            & $-0.0026$ & $-0.0026$ & 0.0019  & $-0.0064$ & 0.0009 & WITHIN CI \\
C(Sex)[T.M]                                 & 0.0033  & 0.0033  & 0.0026  & $-0.0018$ & 0.0083 & WITHIN CI \\
C(academic\_genre)[T.Life Sciences]          & 0.0202  & 0.0201  & 0.0052  & 0.0101  & 0.0306 & WITHIN CI \\
C(academic\_genre)[T.Sciences \& Technology] & 0.0046  & 0.0047  & 0.0031  & $-0.0015$ & 0.0106 & WITHIN CI \\
C(academic\_genre)[T.Social Sciences]        & 0.0034  & 0.0034  & 0.0029  & $-0.0022$ & 0.0090 & WITHIN CI \\
C(language\_env)[T.ESL]                       & 0.0187  & 0.0186  & 0.0021  & 0.0144  & 0.0226 & WITHIN CI \\
C(language\_env)[T.NS]                        & $-0.0026$ & $-0.0026$ & 0.0019  & $-0.0064$ & 0.0009 & WITHIN CI \\
\bottomrule
\end{tabular}
}
\caption{Bootstrap Sensitivity Analysis for detector \texttt{llmdet}}
\label{tab:llmdet_bootstrap}
\end{table*}





\begin{table*}[ht]
\centering
\small
\resizebox{\textwidth}{!}{%
\begin{tabular}{lcccccc}
\toprule
\textbf{Parameter} & \textbf{Original Value} & \textbf{Bootstrap Mean} & \textbf{Bootstrap Std} & \textbf{CI Lower} & \textbf{CI Upper} & \textbf{Coverage} \\
\midrule
Intercept & 0.6886 & 0.6890   & 0.0148   & 0.6611   & 0.7175 & WITHIN CI \\
C(cefr)[T.B1\_1] & 0.0381 & 0.0380   & 0.0159   & 0.0065   & 0.0681 & WITHIN CI \\
C(cefr)[T.B1\_2] & 0.0324 & 0.0324   & 0.0143   & 0.0027   & 0.0604 & WITHIN CI \\
C(cefr)[T.B2\_0] & 0.0134 & 0.0129   & 0.0144   & $-0.0165$ & 0.0403 & WITHIN CI \\
C(cefr)[T.XX\_0] & -0.0200 & $-0.0201$& 0.0077   & $-0.0343$& $-0.0051$ & WITHIN CI \\
C(Sex)[T.M] & -0.0020 & $-0.0018$& 0.0105   & $-0.0222$& 0.0183 & WITHIN CI \\
C(academic\_genre)[T.Life Sciences] & -0.0409 & $-0.0409$& 0.0153   & $-0.0699$& $-0.0112$ & WITHIN CI \\
C(academic\_genre)[T.Sciences \& Technology] & 0.0021 & 0.0021 & 0.0136 & $-0.0253$& 0.0298 & WITHIN CI \\
C(academic\_genre)[T.Social Sciences] & -0.0040 & $-0.0043$& 0.0126   & $-0.0290$& 0.0194 & WITHIN CI \\
C(language\_env)[T.ESL] & -0.0215 & $-0.0215$& 0.0105   & $-0.0430$& $-0.0014$ & WITHIN CI \\
C(language\_env)[T.NS] & -0.0200 & $-0.0201$& 0.0077   & $-0.0343$& $-0.0051$ & WITHIN CI \\
\bottomrule
\end{tabular}
}
\caption{Bootstrap Sensitivity Analysis for detector \texttt{radar}}
\label{tab:radar_bootstrap}
\end{table*}



	\caption{Distribution highlighting sensitive weights important for accuracy.} 
	\label{fig:weight_distribution}
\end{figure}

The framework begins with a sensitivity analysis of model weights, identifying weights values that can tolerate quantization without significantly affecting accuracy, as outlined in Algorithm 1. The framework initially separates \textit{outliers} (outside the blue lines) and \textit{salient weights} (in red) from normal values, as shown in Fig.\ref{fig:weight_distribution}.

\noindent \textbf{Outliers \& Salient Weights:} We incorporate outlier removal to manage extreme weight values based on inter-quartile range scaling. To compute outliers in the weight distribution, we employ the 3$\sigma$ rule~\cite{olive}. Outliers are identified as values lying beyond three standard deviations from the mean. 

% This helps detect and manage extreme values that may affect model performance. 

% \noindent \textbf{Extremely-Salient Weights:} 
From the normal values obtained after this distribution, we rely on Taylor series expansion to estimate the most salient weights in the model. Following~\cite{squeezellm}, we use an approximation to the Hessian \(H \) based on the Fisher information matrix \( F \), which can be calculated over a sample dataset \( D \) as
\begin{equation}
F = \frac{1}{|D|} \sum_{d \in D} g_d g_d^{\top},
\label{eq:fisher}
\end{equation}
where \( g \) is the gradient and \( H \approx F \). This only requires computing the gradient for a set of samples. For each weight tensor \( W \), the weight sensitivity is computed as \(\Lambda_{W} = F \). Weights with higher \( \Lambda_{W} \) values are considered more salient due to their significant impact on the model's output. We preserve the top 0.05\% of the weights based on this criterion. Cumulatively, both outliers and extremely salient weight values correspond to less than 0.5\% of the total weight values. For this reason, we handle these weight values separately and apply per-channel quantization for this set of weight values, isolating them to maintain model precision.

\begin{algorithm}
\caption{Quantization Framework}
\label{quantization}
{\footnotesize
\begin{algorithmic}[1]
%\begin{footnotesize}
\Require calibration dataset $X$, pre-trained weight matrix $W$, gradient $G$
\Require number of bits $n$, tile size $t$, quantile threshold $k$, target frequencies $f_1$, $f_2$
\Ensure Quantized weight matrix $W_q$

\State $W_s, S \leftarrow \text{ExtractSalientValues}(W, G)$ \Comment{Isolate values with high saliency}
\State $W_o, O \leftarrow \text{ExtractOutliers}(W_s)$ \Comment{Separate outlier weights}
\State $W_{s,o}^{q} \leftarrow \text{Quantize}(W_{s} + W_{o})$  \Comment{Quantize outliers and salient weights}

\State $W_t \leftarrow 
\text{ReshapeIntoTiles}(\text{PadMatrix}(W_o, t), t)$ \Comment{Tile reshaping}

\State $\Lambda_{T_k} \leftarrow \text{CalculateTileSensitivities}(G)$ \Comment{Compute sensitivity for each tile}
\State $M_l, M_h \leftarrow \text{CreateMasks}(M, \text{ComputeAdaptiveK}(\Lambda_{T_k}, k))$ \Comment{Classify tiles as low or high sensitivity}
\State $W_{l,i}, W_{h,i} \leftarrow W_{t,i} \odot M_{l,i}, W_{t,i} \odot M_{h,i}$ \Comment{Apply masks}
\State $W_{l,i} \leftarrow \text{Quantize}(W_{l,i}, f_1)$ \Comment{Quantize low-sensitivity tiles}
\State $W_{h,i} \leftarrow \text{Quantize}(W_{h,i}, f_2)$  \Comment{Quantize high-sensitivity tiles}
\State $W_q \leftarrow W_{l,i}, W_{h,i}, W_{s,o}$
\State \Return $W_q$
%\end{footnotesize}
\end{algorithmic}
}
\end{algorithm}
%\vspace{-.5cm}

\subsection{Critical-path delay aware Non-Uniform Quantization} 
\label{section:crit_quant}
Uniform Quantization discretizes continuous values into \(2^b\) evenly spaced levels. On the other hand, non-uniform quantization adapts to the data distribution using variable interval sizes defined by thresholds \( T \), which partition the input range into regions \( R_k = [t_{k-1}, t_k) \). Each region \( R_k \) is assigned a representation level \( y_k \), where \( y_k \) is the quantized value corresponding to data points within \( R_k \). In this work, we leverage non-uniform quantization to more efficiently map the distribution of weights to specific values that reduce the critical-path delays (as discussed in Sec.\ref{section:motivation}), thereby optimizing frequency and energy consumption.

% To do so, we target weights impacting bottleneck layers are selectively quantized using e.g., attention and projection layers) are optimized for efficient inference.

\noindent \textbf{Tile-Based Sensitivity Analysis:} To optimize the model for efficient inference on hardware, the weight tensors are divided into fixed-size tiles (\(128 \times 128\) by default). Specifically, the sensitivity of each tile is evaluated 
as the sum of the absolute values of the gradients for each tile, normalized by the size of the tile, based on Eq.\ref{eq:fisher}. For a given $k$th tile \( T_k \), we compute a \textit{per-tile sensitivity score} \( \Lambda_{T_k} \) using a diagonal approximation of the Fisher information matrix:
\begin{equation}
\Lambda_{T_k} = \frac{\sum_{i,j} g_{k,i,j}^2}{\text{tile\_rows} \times \text{tile\_cols}}
\end{equation}

where \( g_{k,i,j} \) denotes the gradient of the loss with respect to each weight in the \( k \)-th tile, and \( \text{tile\_rows} \times \text{tile\_cols} \) represents the total number of elements within the tile. This score captures the average Fisher information across all weights in the tile, providing a quantitative measure of the tile's sensitivity in relation to its influence on the model's output.

\noindent \textbf{Tile Sensitivity Mapping}: 
To balance hardware efficiency and model accuracy, tiles in each layer are classified as \underline{low-sensitive} or \underline{high-sensitive} based on their relative importance. Determining a fixed sensitivity threshold for each layer is challenging, as weight distributions vary significantly across layers. To address this, we employ a dynamic tile sensitivity mapping strategy that adapts to the cumulative sensitivity distribution of each layer.

The process starts by computing the sensitivity of all tiles in a given layer, derived as the normalized sum of absolute gradient magnitudes within each tile. Sensitivities are then sorted in descending order to rank tiles by importance. A cumulative sum of these sorted sensitivities is calculated and normalized against the total layer sensitivity, generating a cumulative distribution curve from 0 to 1.

The mapping threshold \(k \) is derived from this curve and represents the fraction of tiles classified as low-sensitive, ensuring a specified percentage of total sensitivity (e.g., 95\%) is retained. Tiles contributing most to overall sensitivity are marked high-sensitive, while the rest are classified as low-sensitive. Mathematically, \(k \) is the ratio of the index where cumulative sensitivity exceeds the threshold to the total number of tiles, defaulting to 1.0 if no such index exists.

Once \(k \) is determined, boolean masks separate tiles into low- and high-sensitivity categories. Low-sensitivity tiles are quantized more aggressively, while high-sensitivity tiles retain higher precision to preserve performance. The adaptive quantization and computation flow based on the DVFS characteristics are described in detail in Sec.\ref{section:execution}.

% \subsection{Algorithm Design}


% \subsection{Tile-Based Weight Mapping and Distribution}

% To optimize hardware deployment, the framework reshapes weight tensors based on tile sizes suited for GPU and TPU hardware. Each weight matrix is partitioned into fixed-size tiles (128 $\times$ 128 by default) and then quantized based on their sensitivity classification. This enables efficient batching of low-importance tiles on high-utilization GPUs, reducing computational load and minimizing memory transfers.
% \begin{itemize}

    % \item \textbf{Hardware Implementation:}     
    % Low-sensitivity tiles are mapped onto under-utilized GPUs in a multi-GPU setup, redistributing computational loads based on the memory and processing capacity of each device. This enhances model throughput and scales efficiently across GPUs, essential for real-time deployment.
% \end{itemize}

% \subsection{Model Optimization and Integration}




%%% Uncomment this line and comment out the ``thebibliography'' section below to use the external .bib file (using bibtex) .


%%% Uncomment this section and comment out the \bibliography{references} line above to use inline references.
% \begin{thebibliography}{1}

% 	\bibitem{kour2014real}
% 	George Kour and Raid Saabne.
% 	\newblock Real-time segmentation of on-line handwritten arabic script.
% 	\newblock In {\em Frontiers in Handwriting Recognition (ICFHR), 2014 14th
% 			International Conference on}, pages 417--422. IEEE, 2014.

% 	\bibitem{kour2014fast}
% 	George Kour and Raid Saabne.
% 	\newblock Fast classification of handwritten on-line arabic characters.
% 	\newblock In {\em Soft Computing and Pattern Recognition (SoCPaR), 2014 6th
% 			International Conference of}, pages 312--318. IEEE, 2014.

% 	\bibitem{hadash2018estimate}
% 	Guy Hadash, Einat Kermany, Boaz Carmeli, Ofer Lavi, George Kour, and Alon
% 	Jacovi.
% 	\newblock Estimate and replace: A novel approach to integrating deep neural
% 	networks with existing applications.
% 	\newblock {\em arXiv preprint arXiv:1804.09028}, 2018.

% \end{thebibliography}


\end{document}
