\section{Related Work}
\textbf{Intra-Query Elasticity}. Currently, the database and big data area mainly use ``dynamic query optimization'' to change resource usage during query execution. It can be categorized into three types: adaptive query processing, adaptive query execution, and query re-planning. Adaptive query processing \cite{judicious-reOp-adaptive-proecessing, re-opt2} is primarily applied in traditional standalone relational databases. These methods break down a query into multiple sub-queries, re-optimizing subsequent queries based on the results of earlier ones. Adaptive query execution \cite{runtime-dynamic-op, presto-cbo-hbo-adaptive, sparkAdaptiveExecutionWeb, redshift-scaling-adaptive} is more common in distributed environments, such as big data and cloud-native databases, and involves running queries in stages, using intermediate results to re-optimize the remaining query. Query re-planning focuses on adapting queries to new computing environments \cite{QoS-deadline-replan, QOOP} or execution configurations \cite{byteDance-stream-replan}, allowing re-planned queries to continue from a checkpoint. However, these methods typically require materializing intermediate results and halting data processing, making them unsuitable for frequent and efficient parallelism tuning.


\noindent\textbf{Inter-Query (Workload) Elasticity}. Current research in the field of cloud databases predominantly emphasizes the runtime elasticity of query workloads. These studies leverage the auto-scaling capabilities provided by cloud vendors to implement elastic computing. Prominent cloud databases, including Redshift \cite{redshift-scaling-adaptive,redshift2024}, Snowflake \cite{snowflakeWeb}, BigQuery \cite{bigQueryWeb}, and Azure SQL Database \cite{AzureWeb}, are well-equipped to efficiently support workload elasticity. In addition, serverless computing technologies \cite{cloudfunction1,cloudfunction2,cloudfunction3} enable users to execute computational tasks using cloud functions, offering a scalable and cost-effective alternative to traditional architectures. In this paper, we extend runtime elasticity research from inter-query to intra-query.


\noindent\textbf{Query optimization and scheduling of cloud databases}. Cloud databases primarily rely on rule-based and cost-based optimizers \cite{presto-cbo-hbo-adaptive, polardb-cbo, microsoft-rbo-workload, steering-ML-rbo-workload, basic-rbo, presto-rbo, scope-rbo, spark-rbo, bytegraph-cbo-rbo}. Various machine learning-based query optimization methods have been proposed \cite{ML-autowlm, ML-planOptimize-cuttlefish, ML-optimize, autosteer-bao-workload, bao2}. \cite{DOPML} uses machine learning to determine a near-optimal DOP for query execution. Most query schedulers \cite{ML-workload-umbra,schedule-quickstep} aim to optimize workloads. Additionally, numerous machine learning-based query schedulers have been developed \cite{workload-ML-bufferpool, ML-planOptimize-workload-skinnerdb, ML-workload-schedule-lsched, ML-workload-schedule-decima, ML-autowlm}.  However, these methods typically lack the capability for intra-query runtime optimization and scheduling.