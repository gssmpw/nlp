\section{Related Work}
In this section, we review previous research on the transferability of skills between physical and virtual environments across different tasks and applications, and the use of virtual building blocks to enhance learning experiences and foster creativity in virtual and mixed reality environments.

\textit{Transferability of Skills between Physical and Virtual Environments.}
The transferability of skills between physical and virtual environments has been a subject of great interest in prior research. 
This encompasses a diverse range of tasks, including surgical and dental procedures~\cite{Ayoub2019application}, driving~\cite{Risto2014}, climbing~\cite{Kosmalla2017}, throwing~\cite{tirp2015virtual, Zindulka2020, Kelling2023}, pipetting~\cite{petersen2022pipetting}, table tennis~\cite{Oagaz2022}, pump maintenance~\cite{Winther2020}, and training of paramedics~\cite{schild2018epicsave}.
For a comprehensive review of past research on learning and transfer of complex motor skills in VR, please refer to~\cite{Levac2019}.
In the context of Human-Computer Interaction (HCI) research, evaluating and interacting with physical interfaces in VR~\cite{Paneva2020}, and conducting virtual field studies~\cite{Maekelae2020} have also been investigated. 
However, despite this extensive research, there are no studies focused specifically on the transferability of building with building blocks between physical and virtual environments.

\textit{Virtual Building Blocks for Learning and Creativity.}
In the realm of educational and training applications, building blocks have been integrated into virtual and mixed reality settings to enhance learning experiences and foster creativity.
%Some studies explored the use of building blocks in virtual and mixed reality settings, however, mostly in the context of designing applications for educational and training purposes.
For example, Zhao et al.~\cite{Zhao2019} developed a simulation game aimed at educating engineering students in manufacturing tasks, by designing and building LEGO cars in VR. 
The VR design tool Dreamscape Bricks VR~\cite{Doma2022dreamscape} utilises LEGO bricks as base components for creating architectural designs.
Another educational application, ScalebridgeVR~\cite{Pietroszek2019}, employs LEGO bricks to teach children proportional reasoning skills by challenging them to use to bricks to balance structures like bridges in VR.
%Scalebridge VR educational game that teaches children the mathematical skill of proportional reasoning - use LEGO bricks to balance a bridge.
BlocklyVR~\cite{Hedlund2023} focuses on teaching programming skills through the use of virtual building blocks in a VR setting.
%The effectiveness of a VR serious game compared to a learning activity based on the LEGO Serious Play methodology for learning Scrum and related agile practices has also been explored~\cite{Gordillo2024}.
%In the context of AR, researchers have explored interaction techniques for assembly tasks using LEGO blocks, examining AR-based interaction methods such as 3D controllers, video see-through, and head-mounted displays (HMD)~\cite{Marques2020}.
%assembly task using Lego blocks in augmented reality testing AR-based interaction methods such as  3D controllers and Video see-through HMD.
In the context of Mixed Reality (MR), MagiBricks~\cite{Stefanidi2023} explored the use of smart toy bricks for distributed intergenerational play.
However, these studies primarily focused on the evaluation of the VR/MR building applications, and not on a comparison between the physical and virtual experience.

\textit{Summary.} Previous research has explored the transferability of skills across various tasks between physical and virtual environments, as well as the use of virtual building blocks to enhance learning and creativity. 
However, a gap remains in understanding how the experience of building with blocks translates between physical and virtual realms, which this study addresses by providing a structured comparison of LEGO building in both physical and VR settings.