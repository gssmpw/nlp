%%%%%%%% ICML 2025 EXAMPLE LATEX SUBMISSION FILE %%%%%%%%%%%%%%%%%

\documentclass{article}

\usepackage[utf8]{inputenc} % allow utf-8 input
\usepackage[T1]{fontenc}    % use 8-bit T1 fonts
\usepackage{microtype,inconsolata}
\usepackage{times,latexsym}
\usepackage{graphicx} \graphicspath{{figures/}}
\usepackage{amsmath,amssymb,mathabx,mathtools,amsthm,nicefrac}
\usepackage[linesnumbered,ruled,vlined]{algorithm2e}
\usepackage{acronym}
\usepackage{enumitem}
\usepackage[pagebackref,breaklinks,colorlinks]{hyperref}
\usepackage{balance}
\usepackage{xspace}
\usepackage{setspace}
\usepackage[skip=3pt,font=small]{subcaption}
\usepackage[skip=3pt,font=small]{caption}
\usepackage[capitalise,noabbrev,nameinlink]{cleveref}
\usepackage{booktabs,tabularx,colortbl,multirow,multicol,array,makecell,tabularray}
\usepackage{overpic,wrapfig}
\usepackage{dblfloatfix}
\usepackage[misc]{ifsym}
\usepackage{pifont}
\usepackage{fancyvrb}

% Add a period to the end of an abbreviation unless there's one
% already, then \xspace.
\makeatletter
\DeclareRobustCommand\onedot{\futurelet\@let@token\@onedot}
\def\@onedot{\ifx\@let@token.\else.\null\fi\xspace}

\def\eg{\emph{e.g}\onedot} \def\Eg{\emph{E.g}\onedot}
\def\ie{\emph{i.e}\onedot} \def\Ie{\emph{I.e}\onedot}
\def\cf{\emph{c.f}\onedot} \def\Cf{\emph{C.f}\onedot}
\def\etc{\emph{etc}\onedot} \def\vs{\emph{vs}\onedot}
\def\wrt{w.r.t\onedot} \def\dof{d.o.f\onedot}
\def\etal{\emph{et al}\onedot}

\makeatother

\acrodef{sota}[SOTA]{State-of-the-Art}
\acrodef{method}[\textsc{PRA}]{Preference-based Robot Assistant}
\acrodef{pbp}[\textsc{PbP}]{Preference-based Planning}
\acrodef{vln}[VLN]{Vision-and-Language Navigation}
\acrodef{llm}[LLM]{Large Language Model}
\acrodef{EILEV}[EILEV]{Efficient In-context Learning on Egocentric Videos}
\acrodef{vlm}[VLM]{Vision-Language Model}
\acrodef{vivit}[ViViT]{Video Vision Transformer}
\acrodef{llava}[LLaVA]{Large Language and Vision Assistant}
\acrodef{ai}[AI]{Artificial Intelligence}
\acrodef{ik}[IK]{Inverse Kinematics}
\acrodef{ompl}[OMPL]{Open Motion Planning Library}
\acrodef{sem}[SEM]{Structural Equation Model}

% Spacing
% \medmuskip=2mu   % reduce spacing around binary operators
% \thickmuskip=3mu % reduce spacing around relational operators
\setlength{\abovedisplayskip}{3pt}
\setlength{\belowdisplayskip}{3pt}
\setlength{\abovecaptionskip}{3pt}
\setlength{\belowcaptionskip}{3pt}
% \setlength\floatsep{1\baselineskip plus 3pt minus 2pt}
% \setlength\textfloatsep{1\baselineskip plus 3pt minus 2pt}
% \setlength\dbltextfloatsep{1\baselineskip plus 3pt minus 2pt}
% \setlength\intextsep{1\baselineskip plus 3pt minus 2pt}

\newcolumntype{x}{>{\columncolor{LightCyan1}}c}
\newcolumntype{y}{>{\columncolor{MistyRose}}c}

% Use the following line for the initial blind version submitted for review:
\usepackage{bigai2025}

% If accepted, instead use the following line for the camera-ready submission:
% \usepackage[accepted]{icml2025}

% if you use cleveref..
\usepackage[capitalize,noabbrev]{cleveref}
\usepackage{xurl}

%%%%%%%%%%%%%%%%%%%%%%%%%%%%%%%%
% THEOREMS
\theoremstyle{plain}
\newtheorem{theorem}{Theorem}[section]
\newtheorem{proposition}[theorem]{Proposition}
\newtheorem{lemma}[theorem]{Lemma}
\newtheorem{corollary}[theorem]{Corollary}
\theoremstyle{definition}
\newtheorem{definition}[theorem]{Definition}
\newtheorem{assumption}[theorem]{Assumption}
\theoremstyle{remark}
\newtheorem{remark}[theorem]{Remark}
%%%%%%%%%%%%%%%%%%%%%%%%%%%%%%%%

%%%%%%% Macros for coloring tabulars
\definecolor{bigaired}{RGB}{156, 0, 0}
\definecolor{uclablue}{RGB}{39, 116, 174}

\definecolor{darkred}{RGB}{200, 0, 0}
\definecolor{darkblue}{RGB}{0, 0, 200}
\definecolor{blue}{RGB}{0, 0, 250}

\definecolor{light}{RGB}{225, 250, 250}
\definecolor{lightgray}{RGB}{0.9, 0.9, 0.9}
\definecolor{lightred}{RGB}{250, 200, 200}
\definecolor{lightblue}{RGB}{210, 220, 250}

\definecolor{doderblue}{RGB}{30, 144, 255}
\definecolor{select}{RGB}{222, 235, 247}
\definecolor{unselect}{RGB}{247, 207, 206}

\hypersetup{colorlinks=true, citecolor=uclablue, linkcolor=blue, urlcolor=darkblue}

\newcommand{\blue}{\cellcolor{lightblue}}
\newcommand{\red}{\cellcolor{lightred}}

\newcommand{\hl}[1]{\textcolor{purple}{#1}}
\newcommand{\hlb}[1]{\textcolor{doderblue}{#1}}

%%%%%%%

\newcommand{\ourslong}{}
\newcommand{\ours}{\textsc{TokenSwift}\xspace}

\newcommand{\yarnllama}{\texttt{YaRN-LLaMA2-7b-128k}\xspace}
\newcommand{\llama}{\texttt{LLaMA3.1-8b}\xspace}
\newcommand{\llamasmall}{\texttt{LLaMA3.2-1b}\xspace}
\newcommand{\smallqwen}{\texttt{Qwen2.5-1.5b}\xspace}
\newcommand{\qwen}{\texttt{Qwen2.5-7b}\xspace}
\newcommand{\bigqwen}{\texttt{Qwen2.5-14b}\xspace}

\acrodef{llm}[LLM]{large language model}
\acrodef{ar}[AR]{autoregressive}
\acrodef{sd}[SD]{speculative decoding}

% The \icmltitle you define below is probably too long as a header.
% Therefore, a short form for the running title is supplied here:
\bigaititlerunning{Lossless Acceleration of Ultra Long Sequence Generation up to 100K Tokens}

\usepackage{minitoc}
\noptcrule
\renewcommand{\partname}{}
\renewcommand{\thepart}{}



% \everypar{\looseness=-1}

% \setlength{\parskip}{0.4em}

\begin{document}


\bigaidate{\today}


% \doparttoc
% \faketableofcontents
% \twocolumn[
\bigaititle{From Hours to Minutes: Lossless Acceleration of Ultra Long Sequence Generation up to 100K Tokens}

% It is OKAY to include author information, even for blind
% submissions: the style file will automatically remove it for you
% unless you've provided the [accepted] option to the icml2025
% package.

% List of affiliations: The first argument should be a (short)
% identifier you will use later to specify author affiliations
% Academic affiliations should list Department, University, City, Region, Country
% Industry affiliations should list Company, City, Region, Country

% You can specify symbols, otherwise they are numbered in order.
% Ideally, you should not use this facility. Affiliations will be numbered
% in order of appearance and this is the preferred way.
% \icmlsetsymbol{equal}{*}

\begin{bigaiauthorlist}
\bigaiauthor{Tong Wu$^{\,*\spadesuit}$,}{}
\bigaiauthor{Junzhe Shen$^{\,*\spadesuit\heartsuit}$,}{}
\bigaiauthor{Zixia Jia$^{\,\spadesuit}$,}{}
\bigaiauthor{Yuxuan Wang$^{\,\spadesuit}$}{} and\,
\bigaiauthor{Zilong Zheng$^{\,\spadesuit}$\textsuperscript{\Letter}}{} \\
 $^\spadesuit$ NLCo Lab, BIGAI \quad $^\heartsuit$ LUMIA Lab, Shanghai Jiao Tong University \\
\vskip .02in $^*$ Equal contribution.
\end{bigaiauthorlist}

% \icmlaffiliation{bigai}{State Key Laboratory of General Artificial Intelligence, BIGAI, Beijing, China}
% \icmlaffiliation{sjtu}{LUMIA Lab, Shanghai Jiao Tong University}

\bigaicorrespondingauthor{Zilong Zheng}{zlzheng@bigai.ai}

% You may provide any keywords that you
% find helpful for describing your paper; these are used to populate
% the "keywords" metadata in the PDF but will not be shown in the document
% \icmlkeywords{Machine Learning, ICML}

\begin{abstract}
Generating ultra-long sequences with \acp{llm} has become increasingly crucial but remains a highly time-intensive task, particularly for sequences \textbf{up to 100K tokens}. While traditional speculative decoding methods exist, simply extending their generation limits fails to accelerate the process and can be detrimental. 
Through an in-depth analysis, we identify three major challenges hindering efficient generation: frequent model reloading, dynamic key-value (KV) management and repetitive generation. To address these issues, we introduce \textbf{\ours}, a novel framework designed to substantially accelerate the generation process of ultra-long sequences while maintaining the target model's inherent quality.
Experimental results demonstrate that \ours achieves over $\mathbf{3\times}$ speedup across models of varying scales (1.5B, 7B, 8B, 14B) and architectures (MHA, GQA). This acceleration translates to hours of time savings for ultra-long sequence generation, establishing \ours as a scalable and effective solution at unprecedented lengths. Code can be found at \url{github.com/bigai-nlco/TokenSwift}.
\end{abstract}

\vskip 0.3in

\begin{figure}[h!]
    % {\centering
    \centering
    \includegraphics[width=.7\linewidth]{Figure/speed.pdf}
    \vskip -0.1in
    \captionof{figure}{Comparison of the time taken to generate 100K tokens using autoregressive (AR) and \ours with prefix length of 4096 on \llama. As seen, \ours accelerates the AR process from nearly 5 hours to just 90 minutes.}
    \label{fig:speed_up}
    % }
    % \vskip -0.25in
\end{figure}

% \vskip 0.1in

% ]

% this must go after the closing bracket ] following \twocolumn[ ...

% This command actually creates the footnote in the first column
% listing the affiliations and the copyright notice.
% The command takes one argument, which is text to display at the start of the footnote.
% The \icmlEqualContribution command is standard text for equal contribution.
% Remove it (just {}) if you do not need this facility.

\printAffiliationsAndNotice{}  % leave blank if no need to mention equal contribution
% \printAffiliationsAndNotice{\icmlEqualContribution} % otherwise use the standard text.



% TODO paragraph space
\section{Introduction}

Chain-of-Thought (CoT) prompting~\cite{Nye:2021, cot, Kojima:2022cotzero} has emerged as a cornerstone strategy for enhancing Large Language Models (LLMs) in complex reasoning tasks. By eliciting step-by-step inference, CoT enables LLMs to decompose intricate problems into manageable subtasks, thereby improving their problem-solving performance~\cite{Yao:2023tot, Wang:2023self-consistency, Zhou:2023least, Shinn:2023Reflexion}. Recent advancements, such as OpenAI's o1~\cite{o1} and DeepSeek-R1~\cite{deepseekr1}, further demonstrate that scaling up CoT lengths from hundreds to thousands of reasoning steps could continuously improve LLM reasoning. These breakthroughs have underscored CoT’s potential to advance LLM capabilities, expanding the boundaries of AI-driven problem-solving.

\begin{figure}[t]
\centering
    \includegraphics[width=0.95\columnwidth]{fig/intro.pdf}
    \caption{In contrast to vanilla CoT that generates all reasoning tokens sequentially, \method enables LLMs to \textit{skip} tokens with less semantic importance (\textit{e.g.,} \includegraphics[width=7pt]{fig/token.pdf}~) and learn shortcuts between critical reasoning tokens, facilitating controllable CoT compression.}
    \label{fig:intro}
\end{figure}

Despite its effectiveness, the increased length of CoT sequences introduces substantial computational overhead. Due to the autoregressive nature of LLM decoding, longer CoT outputs lead to proportional increases in both inference latency and memory footprints of key-value cache. Additionally, the quadratic computational cost of attention layers further exacerbates this burden. These issues become particularly pronounced when CoT sequences extend into thousands of reasoning steps, resulting in significant computational costs and prolonged response times. While prior research has explored methods for selectively skipping reasoning steps~\cite{Ding:2024cotshortcut, liu2024skipstep}, recent findings~\cite{jin:2024cotlength, Merrill:2024cotlength} suggest that such reductions may conflict with test-time scaling~\cite{o1-blog, snell2025scaling}, ultimately impairing LLM reasoning performance. Therefore, striking an optimal balance between CoT efficiency and reasoning accuracy remains a critical open challenge.

In this work, we delve into CoT efficiency and seek the answer to an important question: \textit{``Does every token in the CoT output contribute equally to deriving the answer?''} We empirically analyze the semantic importance of tokens within CoT outputs and reveal that their contributions to the reasoning performance vary, as depicted in Figure 2. Building on this insight, we introduce \method, a simple yet effective approach that enables LLMs to \textit{skip} less important tokens within CoT sequences and learn shortcuts between critical reasoning tokens, thereby allowing for controllable CoT compression with adjustable ratios. Specifically, as shown in Figure~\ref{fig:intro}, \method constructs compressed CoT training data with various compression ratios, by pruning unimportance tokens from original LLM CoT trajectories. Then, it conducts a general supervised fine-tuning process on target LLMs with this training data, facilitating LLMs to automatically trim redundant tokens during reasoning.

We conduct extensive experiments across various models, including LLaMA-3.1-8B-Instruct and the Qwen2.5-Instruct series, using two widely recognized math reasoning benchmarks: GSM8K and MATH-500. The results validate the effectiveness of \method in compressing CoT outputs while maintaining robust reasoning performance. Notably, Qwen2.5-14B-Instruct exhibits almost \textbf{NO} performance drop (less than $0.4\%$) with a $\bm{40\%}$ reduction in token usage on GSM8K. On the challenging MATH-500 dataset, LLaMA-3.1-8B-Instruct effectively reduces CoT token usage by $\bm{30}\%$ with a performance decline of less than $4\%$, resulting in a $\bm{1.4}\times$ inference speedup. Further analysis underscores the coherence of \method in specified compression ratios and its potential scalability with stronger compression techniques.

\method is distinguished by its low training cost. For Qwen2.5-14B-Instruct, \method fine-tunes only 0.2\% of the model's parameters using LoRA. The size of the compressed CoT training data is no larger than that of the original training set, with 7,473 examples in GSM8K and 7,500 in MATH. The training is completed in approximately 2 hours for the 7B model and 2.5 hours for the 14B model on two 3090 GPUs. These characteristics make \method an efficient and reproducible approach, suitable for use in efficient and cost-effective LLM deployment.

To sum up, our key contributions are:
\begin{enumerate}
    \item To the best of our knowledge, this work is the \textit{first} to investigate the potential of enhancing CoT efficiency through \textit{token skipping}, inspired by the varying semantic importance of tokens in CoT trajectories of LLMs.
    \item We introduce \method, a simple yet effective approach that enables LLMs to skip redundant tokens within CoTs and learn shortcuts between critical tokens, facilitating CoT compression with adjustable ratios.
    \item Our experiments validate the effectiveness of \method. When applied to Qwen2.5-14B-Instruct, \method reduces reasoning tokens by $40\%$ (from 313 to 181) on GSM8K, with less than a $0.4\%$ performance drop.
\end{enumerate}

\section{Challenges}
\label{sec:challenge}
Accelerating long sequence generation is nevertheless a non-trivial task, even built upon prior success in \acf{sd}.  In this section, we identify critical challenges encountered in accelerating ultra-long sequence generation.
% , along with an exploration of the underlying causes. 

\paragraph{Challenge I: Frequent Model Reloading}
\label{sec:reload}
% It is well-known that \ac{ar} generate text in a token-by-token manner. 
One fundamental speed obstacle lies in the \ac{ar} generation scheme of \ac{llm}.
For each token, the entire model must be loaded from GPU's storage unit to the computing unit~\citep{llm_viewer}, which takes significantly more time than the relatively small amount of computation performed (as shown in \cref{tab:time}). Consequently, the primary bottleneck in generation stems from I/O memory access rather than computation.
%, rendering the process highly memory-intensive.

\begin{table}[ht]
    \centering
    \small
    % \vskip -0.2 in
    % \captionsetup{width=.4\linewidth}
    % \parbox{10cm}{
    \caption{Experimental results of TriForce~\citep{triforce} and MagicDec~\citep{magicdec} with default parameters on \llama. The Batch Size of MagicDec is set to 1.\label{tab:short}}
    % }
    \vskip 0.15 in
% \resizebox{\linewidth}{!}{
    \begin{tabular}{l|ccc}
    \toprule
    \textbf{Method} & \textbf{Gen. Len.} & \textbf{Draft Form} & \textbf{Speed Up} \\ \midrule
    \textbf{TriForce} & 256 & Standalone Draft & 1.02 \\ \midrule
    \multirow{2}{*}{\textbf{MagicDec}} & \multirow{2}{*}{64} & Self-Speculation & 1.20 \\
     &  & Standalone Draft & 1.06 \\
     \bottomrule
    \end{tabular}
% }
\vskip -0.1 in
\end{table}

\begin{table}[ht!]
    \centering
    \small
    \vskip -0.15 in
    \caption{Taking NVIDIA A100 80G and \llama as example, \textit{MAX} refers to the scenario with a maximum context window 128K. The calculation method is from \citet{llm_viewer}.}
    \label{tab:time}
    \vskip 0.15 in
% \resizebox{\linewidth}{!}{
    \begin{tabular}{l|l}
    \toprule
    \textsc{Memory} & \textsc{Computation} \\ \midrule
    \textit{Bandwidth}: 2.04e12 B/s & \textit{BF16}: 312e12 FLOPS \\ 
    \textit{Model Weights}: 15.0 GB & \textit{MAX Operations}: 83.9 GB\\ \midrule
    \textit{Loading Time}: \textbf{7.4} ms & \textit{MAX Computing Time}: \textbf{0.3} ms\\
    \bottomrule
    \end{tabular}
% }
% \vspace{-.2in}
\end{table}


% \begin{table}[ht!]
    \centering
    \small
    \vskip -0.15 in
    \caption{Taking NVIDIA A100 80G and \llama as example, \textit{MAX} refers to the scenario with a maximum context window 128K. The calculation method is from \citet{llm_viewer}.}
    \label{tab:time}
    \vskip 0.15 in
% \resizebox{\linewidth}{!}{
    \begin{tabular}{l|l}
    \toprule
    \textsc{Memory} & \textsc{Computation} \\ \midrule
    \textit{Bandwidth}: 2.04e12 B/s & \textit{BF16}: 312e12 FLOPS \\ 
    \textit{Model Weights}: 15.0 GB & \textit{MAX Operations}: 83.9 GB\\ \midrule
    \textit{Loading Time}: \textbf{7.4} ms & \textit{MAX Computing Time}: \textbf{0.3} ms\\
    \bottomrule
    \end{tabular}
% }
% \vspace{-.2in}
\end{table}
% As shown in \cref{tab:time}, even at the highest computational demand, memory access time is approximately \textbf{25} times longer than the computation time.

% \textit{$\rhd$ When generating  ultra-long sequence, such as 100K tokens, the GPU must load the model weights over 100,000 times. This repetitive loading exacerbates the issue, significantly impacting overall efficiency.}

\textit{$\rhd$ When generating  ultra-long sequence, such as 100K tokens, the GPU must reload the model weights over 100,000 times. This repetitive process poses the challenge: How can we reduce the frequency of model reloading?}

% \vspace{-0.05 in}
% \subsection{Challenge 2: Dynamic KV Growing}
\paragraph{Challenge II: Prolonged Growing of KV Cache}
\label{sec:load_kv}
% Prior works, such as TriForce~\citep{triforce} and MagicDec~\citep{magicdec}, have highlighted the challenge posed by the growth of KV cache size as sequence length increases, potentially surpassing the size of model weights. This growth renders KV cache loading time a critical bottleneck in text generation. 
% Prior works like TriForce~\citep{triforce} and MagicDec~\citep{magicdec} have demonstrated that growing KV cache size significantly increases loading time, urging us to use partial KV cache when drafting. However, they all fail to appropriately update partial KV cache dynamically to support ultra-long sequences generation.
Previous studies, such as TriForce~\citep{triforce} and MagicDec~\citep{magicdec} have demonstrated that, a small KV cache budget can be used during the drafting phase to reduce the time increase caused by the loading enormous KV cache. 
While their one-time compression strategy at the prefill stage can handle scenarios with long prefixes and short outputs, it fails to address cases involving ultra-long outputs, as the growing size of KV cache would far exceed the allocated length budget.
% However, their focus is on long prefixes with short outputs, allowing for a one-time compression at the prefill stage. This static strategy cannot be directly applied to the generation of ultra-long sequences, as the growing size of KV cache would far exceed budgeted length.

% However, these works address scenarios involving extremely long prefixes and short outputs, whereas we focus on the challenges associated with ultra-long outputs.
% However, these works address scenarios involving extremely long prefixes and short outputs, where compressing the prefix once results in almost no growth of KV cache. In contrast, we focus on the challenges associated with ultra-long outputs, which require dynamic compression to manage the continuously growing KV cache. 

% In ultra-long text generation, a static KV cache strategy—updating the cache only once during the prefill phase—fails to support prolonged generation. As generation progresses, cached entries lose relevance to the evolving context, resulting in degraded quality. To tackle this, dynamic updates to KV cache are essential, enabling the retention of only the most relevant KV pairs. 

% \textit{$\rhd$ While prior works, as well as ours, aim to reduce KV loading time, the key distinction lies in the fact that generating ultra-long sequences necessitates determining when and how to dynamically update the KV cache.}
\textit{$\rhd$ To dynamically manage partial KV cache within limited budget during ultra-long sequence generation, the challenge lies in determining when and how to dynamically update the KV cache.}
% \vspace{-0.05 in}
\paragraph{Challenge III: Repetitive Content Generation}
\label{sec:repeat}
The degeneration of \ac{ar} in text generation tasks — characterized by output text that is bland, incoherent, or gets stuck in repetitive loops — is a widely studied challenge~\citep{topp,minp,eta}. 
% \todo{The repetition is severe when generating long sequeunce. May some some thoerticial support here? When generating long sequence, it is easier to lost in a local optima? } 
When generating sequences of considerable length, \eg, 100K, the model tends to produce repetitive sentences (\cref{fig:case}).

% In fact, our objective is \textbf{lossless acceleration}, meaning that the upper bound on the quality of generated text is align with target model. Therefore, our primary emphasis is more on achieving efficient acceleration, rather than addressing degradation issue.

% Nevertheless, when generating sequences of considerable length, \eg 100K, the model tends to produce repetitive sentences, resulting in meaningless content (\cref{fig:case}). This phenomenon arises because the probability of generating erroneous tokens increases with sequence length, and the propagation of these errors becomes pronounced over time.

% \textit{$\rhd$ Therefore, while prioritizing accelerated generation without sacrificing performance, it is essential to mitigate repetition patterns in ultra-long sequences. Ensuring that the generated content remains meaningful is necessary, even if completely eliminating this challenge may not be feasible.}
\textit{$\rhd$ Since our objective is lossless acceleration and repetition is an inherent problem in \acp{llm}, eliminating this issue is not our focus. However, it is still essential and challenging to mitigate repetition patterns in ultra-long sequences.}
\begin{figure*}[t!]
    \centering
    \includegraphics[width=\linewidth]{Figure/TokenSwift.pdf}
    % \vskip -0.1 in
    \caption{\textbf{Illustration of \ours Framework.} First, target model (LLM) with partial KV cache and three linear layers outputs 4 logits in a single forward pass. Tree-based attention is then applied to construct candidate tokens. Secondly, top-$k$ candidate $4$-grams are retrieved accordingly. These candidates compose draft tokens, which are fed into the LLM with full KV cache to generate target tokens. The verification is performed by checking if draft tokens match exactly with target tokens (\cref{alg:algorithm}). Finally, we randomly select one of the longest valid draft tokens, and update $n$-gram table and KV cache accordingly.}
    \label{fig:frame}
    % \vskip -0.15 in
\end{figure*}

\section{\ours}
\label{sec:method}
To achieve \textbf{lossless acceleration in generating ultra-long sequences}, we propose tailored solutions for each challenge inherent to this process. These solutions are seamlessly integrated into a unified framework, \ie \ours.

\subsection{Overview}
\label{sec:overall}
The overall framework is depicted in \cref{fig:frame}. \ours 
% is highly lightweight and conceptually similar to \ac{sd}. It 
generate a sequence of draft tokens with self-drafting, which are then passed to the target (full) model for validation using a tree-based attention mechanism (See \cref{app:tree_attn} for more tree-based attention details). This process ensures that the final generated output aligns with the target model’s predictions, effectively achieving lossless acceleration.

\ours is lightweight because the draft model is the target model itself with a partial KV cache. This eliminates the need to train a separate draft \ac{llm}; instead, only $\gamma$ linear layers need to be trained, where $\gamma + 1$\footnote{The target model itself can also predict one logit, making the total number of logits $\gamma+1$. We take $\gamma=3$.} represents the number of logits predicted in a single forward pass. In addition, during the verification process, once we obtain the target tokens from the target model with full KV cache, we directly compare draft tokens with target tokens sequentially to ensure that the process is lossless~\citep{rest}.

\subsection{Multi-token Generation and Token Reutilization}
\label{sec:multi_token}
\paragraph{Multi-token Self-Drafting} 
% Inspired by Medusa~\citep{medusa}, we propose a modification where the final output of \ac{llm} is used as input to train $3$ linear layers, enabling the model to generate multiple draft tokens in a single forward pass. However, we argue that the generated draft tokens should not be independent of each other. Unlike Medusa, where the linear layers operate entirely independently, we introduce a simple adjustment to this structure. 
Inspired by Medusa~\citep{medusa}, we enable the \ac{llm} to generate multiple draft tokens in a single forward pass by incorporating $\gamma$ additional linear layers. However, we empirically note that \textbf{the additional linear layers should not be independent of each other}. Specifically, we propose the following structure:
\begin{equation}
\label{equ:ours}
% \small
% \resizebox{.9\hsize}{!}{
% $
    \begin{aligned}
    h_1=f_1(h_0) + h_0,\quad{}h_2=&f_2(h_1) + h_1,\quad{}h_3=f_3(h_2) + h_2,\\
l_0,~l_{1},~l_{2},~l_{3}=&~g(h_0),~g(h_1),~g(h_2),~g(h_3),
    \end{aligned}
% $
% }
\end{equation}
where $h_0$ denotes the last hidden state of \ac{llm}, $f_i(\cdot)$ represents the $i$-th linear layer, $h_i$ refers to the $i$-th hidden representation, $g(\cdot)$ represents the LM Head of target model, and $l_i$ denotes output logits.
% By comparing \cref{equ:medsua} (Medusa) and \cref{equ:ours} (\ours), it is evident that in \ours, the generation of each token depends on the previously generated token, which aligns more closely with the \ac{ar} nature of the model. Moreover, this adjustment incurs no additional computational cost.
This structure aligns more closely with the \ac{ar} nature of the model. Moreover, this adjustment incurs no additional computational cost.
\vspace{-0.05 in}
\paragraph{Token Reutilization} 
% Given the relatively low acceptance rate of using linear to approximate the entire \ac{llm} for generating draft tokens, we propose a method named \textbf{token reutilization}  to further reduce the frequency of model reloads. 
Given the relatively low acceptance rate of using linear layers to generate draft tokens, we propose a method named \textbf{token reutilization} to further reduce the frequency of model reloads. The idea behind token reutilization is that some phrases could appear frequently, and they are likely to reappear in subsequent generations.

% Specifically, we define $(\mathcal{G}, \mathcal{F})$, where $\mathcal{G}=\{x_{i+1}, ..., x_{i+n}\}$ represents an $n$-gram and $\mathcal{F}$ denotes its corresponding frequency $\mathcal{F}$ within the generated token sequence $S=\{x_0, x_1, ..., x_{t-1}\}$ by time step $t$ ($t \geq n$). At subsequent time steps, we use the token generated by target model as the first token to select top-$k$ most frequent $n$-grams $\{\mathcal{G}_1, \mathcal{G}_2,...,\mathcal{G}_k\}$ and incorporate them as additional draft tokens. These selected draft tokens, along with the newly generated ones, are then fed to the \ac{llm} for parallel validation. 
Specifically, we maintain a set of tuples $\{(\mathcal{G}, \mathcal{F})\}$, where $\mathcal{G}=\{x_{i+1}, ..., x_{i+n}\}$ represents an $n$-gram and $\mathcal{F}$ denotes its corresponding frequency $\mathcal{F}$ within the generated token sequence $S=\{x_0, x_1, ..., x_{t-1}\}$ by time step $t$ ($t \geq n$). After obtaining $\{p_0,\ldots, p_3\}$ as described in \S \ref{sec:penalty}, we retrieve the top-$k$ most frequent $n$-grams beginning with token $\arg\max p_0$ to serve as additional draft tokens.

Although this method can be applied to tasks with long prefixes, its efficacy is constrained by the limited decoding steps, which reduces the opportunities for accepting $n$-gram candidates. Additionally, since the long prefix text is not generated by the \ac{llm} itself, a distributional discrepancy exists between the generated text and the authentic text~\citep{detectgpt}. As a result, this method is particularly suitable for generating ultra-long sequences.
 
% \subsection{Dynamic and Memory-Saving KV Pruning}
\subsection{Dynamic KV Cache Management}
\label{sec:kv_update}
\paragraph{Dynamic KV Cache Updates}
Building upon the findings of~\citet{stram_llm}, we preserve the initial $|S|$ KV pairs within the cache during the drafting process, while progressively evicting less important KV pairs. Specifically, we enforce a fixed budget size $|B|$, ensuring that the KV cache at any given time can be represented as:
\begin{equation}
    \nonumber
    % \resizebox{\hsize}{!}{$
    \mathbf{KV}=\{(\mathbf{K}_0,\mathbf{V}_0), ..., (\mathbf{K}_{|S|},\mathbf{V}_{|S|}), (\mathbf{K}_{|S|+1},\mathbf{V}_{|S|+1}),..., (\mathbf{K}_{|B|-1},\mathbf{V}_{|B|-1})\},
   % $},
\end{equation}
where the first $|S|$ pairs remain fixed, and the pairs from position $|S|$ to $|B|-1$ are ordered by decreasing importance. 
As new tokens are generated, less important KV pairs are gradually replaced, starting from the least important ones at position $|B|-1$ and moving towards position $|S|$. Once replacements extend beyond the $|S|$ position, we recalculate the \textit{importance scores} of all preceding tokens and select the most relevant $|B|-|S|$ pairs to reconstruct the cache. 
This process consistently preserves the critical information required for ultra-long sequence generation. 
\vspace{-0.05 in}
% \paragraph{Memory-Saving Top-K Pruning} 
% To implement dynamic updates efficiently, we employ a simple yet effective Top-K pruning strategy. Specifically, we rank the KV pairs based on the importance scores derived from the dot product between the query ($\mathbf{Q}$) and key ($\mathbf{K}$), \ie $\mathbf{Q}\mathbf{K}^T$. 
\paragraph{Importance Score of KV pairs} 
We rank the KV pairs based on the \textit{importance scores} derived from the dot product between the query ($\mathbf{Q}$) and key ($\mathbf{K}$), \ie $\mathbf{Q}\mathbf{K}^T$. 

In the case of Group Query Attention (GQA), since each $\mathbf{K}$ corresponds to a group of $\mathcal{Q}=\{\mathbf{Q}_0, ..., \mathbf{Q}_{g-1}\}$, direct dot-product computation is not feasible. Unlike methods such as SnapKV~\citep{snapkv}, we do not replicate the $\mathbf{K}$. Instead, we partition the $\mathcal{Q}$, as shown in \cref{equ:gqa}:
\begin{equation}
    \label{equ:gqa}
    \vspace{-2mm}
    \text{importance score}_i = \sum_{j=i\cdot g}^{((i+1)\cdot g)-1}\mathbf{Q}_j \cdot \mathbf{K}_i,
        % \vspace{-2mm}
\end{equation}
where for position $i$, $\mathbf{Q}_j$ in the group $\mathcal{Q}_i$ are dot-product with the same $\mathbf{K}_i$, and their results are aggregated to obtain the final \textit{importance score}. This approach enhances memory saving while preserving the quality of the attention mechanism, ensuring that each query is effectively utilized without introducing unnecessary redundancy.

\subsection{Contextual Penalty and Random N-gram Selection}
\label{sec:penalty}
% \paragraph{Contextual Length Penalty} 
\paragraph{Contextual Penalty} 
% To mitigate repetition in generated text, we have explored various sampling strategies. However, with the significantly larger sequence length, the likelihood of repetition increases compared to generating shorter texts (\cref{sec:repeat}). As a result, we decided to apply an additional penalty to the generated tokens to further mitigate repetition.
To mitigate repetition in generated text, we have explored various sampling strategies. However, with the significantly larger sequence length, the likelihood of repetition increases significantly (\S \ref{sec:repeat}). As a result, we decided to apply an additional penalty to the generated tokens to further mitigate repetition.

The penalized sampling approach proposed in \citep{penalty} suggests applying a penalty to all generated tokens. However, when generating ultra-long sequences, the set of generated tokens may cover nearly all common words, which limits the ability to sample appropriate tokens. Therefore, we propose an improvement to this method. 

Specifically, we introduce a fixed \emph{penalty window} $W$ and apply \emph{penalty value} $\theta$ to the most recent $W$ tokens, denoted as $\mathbb{W}$, generated up to the current position, as illustrated in \cref{equ:repeat}: 
\begin{equation}
% \small
    \label{equ:repeat}
% \vspace{-3mm}
    \begin{aligned}
        p_i &= \frac{\exp \big(l_i/(t\cdot I(l_i))\big)}{\sum_j \exp \big(l_j/(t\cdot I(l_j))\big)},\\
    I(l)=\theta\,\,&\text{if}\,\,l \in \mathbb{W}\,\text{else}\,\,1.0,\quad \theta \in (1, \infty),
    \end{aligned}
    % \vspace{-1mm}
\end{equation}
where $t$ denotes temperature, $l_i$ and $p_i$ represent the logit and probability of $i$-th token. This adjustment aims to maintain diversity while still mitigating repetitive generation.

\section{\thename}
\subsection{End-to-End Driving Policy}
The overall framework of \thename{} is depicted in Fig.~\ref{fig:framework}. 
\thename{} takes multi-view image sequences as input, transforms the sensor data into scene token embeddings, outputs the probabilistic distribution of actions, and samples an action to control the vehicle. 

\boldparagraph{BEV Encoder.} 
We first employ a BEV encoder~\cite{li2022bevformer} to transform multi-view image features from the perspective view to the Bird's Eye View (BEV), obtaining a feature map in the BEV space. This feature map is then used to learn instance-level map features and agent features.

\boldparagraph{Map Head.} 
Then we utilize a group of map tokens~\cite{maptrv2, liao2022maptr, lanegap} to learn the vectorized map elements of the driving scene from the BEV feature map, including lane centerlines, lane dividers, road boundaries, arrows, traffic signals, \etc.

\boldparagraph{Agent Head.} 
Besides, a group of agent tokens~\cite{jiang2022pip} is adopted to predict the motion information of other traffic participants, including location, orientation, size, speed, and multi-mode future trajectories.

\boldparagraph{Image Encoder.} 
Apart from the above instance-level map and agent tokens, we also use an individual image encoder~\cite{vit,he2016resnet} to transform the original images into image tokens. These image tokens provide dense and rich scene information for planning, complementary to the instance-level tokens.

\begin{figure}[t]
\centering
\includegraphics[width=0.98\linewidth]{fig/post-training-2.pdf} 
\caption{\textbf{Post-training.}  $N$  workers parallelly run. The generated rollout data $(s_t,a_t, r_{t+1},s_{t+1},...)$ are recorded in a rollout buffer. Rollout data and human driving demonstrations are used in RL- and IL-training steps to fine-tune the AD policy synergistically.
}
\label{fig:post-training}
\end{figure}

\boldparagraph{Action Space.} 
To accelerate the convergence of RL training, we design a decoupled discrete action representation. 
We divide the action into two independent components: lateral action and longitudinal action. 
The action space is constructed over a short $0.5$-second time horizon, during which the vehicle's motion is approximated by assuming constant linear and angular velocities. 
Under this assumption, the lateral action $a^x$ and longitudinal action $a^y$ can be directly computed based on the current linear and angular velocities.
By combining decoupling with a limited temporal scope and simplified motion model, our approach effectively reduces the dimensionality of the action space, accelerating training convergence.


\boldparagraph{Planning Head.} 
We use $E_\text{scene}$ to denote the scene representation, which consists of map tokens, agent tokens, and image tokens. We initialize a planning embedding denoted as $E_\text{plan}$. A cascaded Transformer decoder $\phi$ takes the planning embedding $E_\text{plan}$ as the query and the scene representation $E_\text{scene}$ as both key and value.

The output of the decoder $\phi$ is then combined with navigation information $E_\text{navi}$ and ego state $E_\text{state}$ to output the probabilistic distributions of the lateral action $a^x$ and the longitudinal action $a^y$:
\begin{equation}
\begin{aligned}
     \pi(a^x\mid s) = & \text{softmax}(\text{MLP}(\phi(E_\text{plan}, E_\text{scene}) \\
    & + E_\text{navi} + E_\text{state})), \\
     \pi(a^y\mid s) = & \text{softmax}(\text{MLP}(\phi(E_\text{plan}, E_\text{scene}) \\
     & + E_\text{navi} + E_\text{state})),
\label{eq:action distribution}
\end{aligned}
\end{equation}
where $E_\text{plan}$, $E_\text{navi}$, $E_\text{state}$, and the output of $\text{MLP}$ are all of the same dimension ($1 \times D$).

The planning head also outputs the value functions $V_x(s)$ and $V_y(s)$, which estimate the expected cumulative rewards for the lateral and longitudinal actions, respectively: 
\begin{equation}
\begin{aligned}
    & V_x(s) = \text{MLP}(\phi(E_\text{plan}, E_\text{scene}) + E_\text{navi} + E_\text{state}), \\
    & V_y(s) = \text{MLP}(\phi(E_\text{plan}, E_\text{scene}) + E_\text{navi} + E_\text{state}).
\end{aligned}
\end{equation}
The value functions are used in RL training (Sec.~\ref{sec:optimization}).

\subsection{Training Paradigm}
We adopt a three-stage training paradigm: perception pre-training, planning pre-training, and reinforced post-training, as shown in Fig.~\ref{fig:framework}.

\boldparagraph{Perception Pre-Training.} 
Information in the image is sparse and low-level. In the first stage,  
the map head and the agent head explicitly output map elements and agent motion information, which are supervised with ground-truth labels. Consequently,  
map tokens and agent tokens implicitly encode the corresponding high-level information.  
In this stage, we only update the parameters of the BEV encoder, the map head, and the agent head.



\boldparagraph{Planning Pre-Training.} 
In the second stage, to prevent the unstable cold start of RL training, IL is first performed to initialize the probabilistic distribution of actions based on large-scale real-world driving demonstrations from expert drivers. In this stage, we only update the parameters of the image encoder and the planning head, while the parameters of the BEV encoder, map head, and agent head are frozen. The optimization objectives of perception tasks and planning tasks may conflict with each other. However, with the training stage and parameters decoupled, such conflicts are mostly avoided.

\boldparagraph{Reinforced Post-Training.} 
In the reinforced post-training, RL and IL synergistically fine-tune the distribution. RL aims to guide the policy to be sensitive to critical risky events and adaptive to out-of-distribution situations. IL serves as the regularization term to keep the policy's behavior similar to that of humans.

We select a large amount of risky dense-traffic clips from collected driving demonstrations. For each clip, we train an independent 3DGS model that reconstructs the clip and serves as a digital driving environment.  
As shown in Fig.~\ref{fig:post-training}, we set $N$ parallel workers.  
Each worker randomly samples a 3DGS environment and begins rollout, i.e., the AD policy controls the ego vehicle to move and iteratively interacts with the 3DGS environment. After the rollout process of this 3DGS environment ends, the generated rollout data $(s_t,a_t, r_{t+1},s_{t+1},...)$ are recorded in a rollout buffer, and the worker will sample a new 3DGS environment for another round of rollout.

As for policy optimization, we iteratively perform RL-training steps and IL-training steps. For RL-training steps, we sample data from the rollout buffer and follow the Proximal Policy Optimization (PPO) framework~\cite{PPO} to update the AD policy. For IL-training steps, we use real-world driving demonstrations to update the policy. After a fixed number of training steps, the updated AD policy is sent to every worker to replace the old one, to avoid a distribution shift between data collection and optimization.
We only update the parameters of the image encoder and the planning head. The parameters of the BEV encoder, the map head, and the agent head are frozen.  
The detailed RL design is presented below.

\subsection{Interaction Mechanism between AD Policy and 3DGS Environment}
In the 3DGS environment, the ego vehicle acts according to the AD policy. Other traffic participants act according to real-world data in a log-replay manner.  
A simplified kinematic bicycle model is employed to iteratively update the ego vehicle's pose at every $\Delta t$ seconds as follows:  
\begin{equation}
\begin{aligned}
x_{t+1}^{w} & = x_{t}^w + v_t \cos \left(\psi_{t}^w\right) \Delta t, \\
y_{t+1}^{w} & = y_{t}^w + v_t \sin \left(\psi_{t}^w\right) \Delta t, \\
\psi_{t+1}^{w} & = \psi_{t}^w + \frac{v_t}{L} \tan \left(\delta_t\right) \Delta t,
\label{equation:kinematic_model}
\end{aligned}
\end{equation}  
where $x_t^{w}$ and $y_t^{w}$ denote the position of the ego vehicle relative to the world coordinate; $\psi_t^w$ is the heading angle that defines the vehicle's orientation with respect to the world $x$-coordinate; $v_t$ is the linear velocity of the ego vehicle; $\delta_t$ is the steering angle of the front wheels; and $L$ is the wheelbase, i.e., the distance between the front and rear axles.

During the rollout process, the AD policy outputs actions $(a_t^x, a_t^y)$ for a $0.5$-second time horizon at time step $t$. We derive the linear velocity $v_t$ and steering angle $\delta_t$ based on $(a_t^x, a_t^y)$.  
Based on the kinematic model in Eq.~\ref{equation:kinematic_model},  
the pose of the ego vehicle in the world coordinate system is updated from ${p}_t = (x_{t}^w, y_{t}^w, \psi_{t}^w)$ to ${p}_{t+1} = (x_{t+1}^{w}, y_{t+1}^{w}, \psi_{t+1}^{w})$.  

Based on the updated ${p}_{t+1}$, the 3DGS environment computes the new ego vehicle's state $s_{t+1}$. The updated pose ${p}_{t+1}$ and state $s_{t+1}$ serve as the input for the next iteration of the inference process.

The 3DGS environment also generates rewards $\mathcal{R}$ (Sec.~\ref{sec:reward}) according to multi-source information (including trajectories of other agents, map information, the expert trajectory of the ego vehicle, and the parameters of Gaussians), which are used to optimize the AD policy (Sec.~\ref{sec:optimization}).

\begin{figure}[t]
\centering
\includegraphics[width=1.0\linewidth]{fig/reward.pdf} 
\caption{\textbf{Example diagram of four types of reward sources.}  (1): Collision with a dynamic obstacle ahead triggers a reward $r_{\text{dc}}$. (2): Hitting a static roadside obstacle incurs a reward $r_{\text{sc}}$. (3): Moving onto the curb exceeds the positional deviation threshold $d_{\text{max}}$, triggering a reward $r_{\text{pd}}$. (4): Drifting toward the adjacent lane exceeds the heading deviation threshold $\psi_{\text{max}}$, triggering a reward $r_{\text{hd}}$.
}
\label{fig: reward source}
\end{figure}
\subsection{Reward Modeling}
\label{sec:reward}
The reward is the source of the training signal, which determines the optimization direction of RL. The reward function is designed to guide the ego vehicle's behavior by penalizing unsafe actions and encouraging alignment with the expert trajectory. It is composed of four reward components: (1) collision with dynamic obstacles, (2) collision with static obstacles, (3) positional deviation from the expert trajectory, and (4) heading deviation from the expert trajectory:
\begin{equation}
\begin{aligned}
\mathcal{R} = \{r_{\text{dc}}, r_{\text{sc}}, r_{\text{pd}}, r_{\text{hd}}  \}. 
\end{aligned}
\end{equation}

As illustrated in Fig.~\ref{fig: reward source}, these reward components are triggered under specific conditions.  
In the 3DGS environment, dynamic collision is detected if the ego vehicle's bounding box overlaps with the annotated bounding boxes of dynamic obstacles, triggering a negative reward $r_{\text{dc}}$. Similarly, static collision is identified when the ego vehicle's bounding box overlaps with the Gaussians of static obstacles, resulting in a negative reward $r_{\text{sc}}$.  
Positional deviation is measured as the Euclidean distance between the ego vehicle's current position and the closest point on the expert trajectory. A deviation beyond a predefined threshold $d_{\text{max}}$ incurs a negative reward $r_{\text{pd}}$.  
Heading deviation is calculated as the angular difference between the ego vehicle's current heading angle $ \psi_t $ and the expert trajectory's matched heading angle $\psi_{\text{expert}}$. A deviation beyond a threshold $ \psi_{\text{max}}$ results in a negative reward $r_{\text{hd}}$.

Any of these events, including dynamic collision, static collision, excessive positional deviation, or excessive heading deviation, triggers immediate episode termination. Because after such events occur, the 3DGS environment typically generates noisy sensor data, which is detrimental to RL training.

\subsection{Policy Optimization}
\label{sec:optimization}
In the closed-loop environment, the error in each single step accumulates over time. The aforementioned rewards are not only caused by the current action but also by the actions of the preceding steps.  
The rewards are propagated forward with Generalized Advantage Estimation (GAE)~\cite{gae} to optimize the action distribution of the preceding steps.

Specifically, for each time step $t$, we store the current state $s_t$, action $a_t$, reward $r_t$, and the estimate of the value $V(s_t)$.  
Based on the decoupled action space, and considering that different rewards have different correlations to lateral and longitudinal actions, the reward $r_t$ is divided into lateral reward $r_t^x$ and longitudinal reward $r_t^y$:
\begin{equation}
\begin{aligned}
r_t^x &= r_t^{\text{sc}} + r_t^{\text{pd}} + r_t^{\text{hd}}, \\
r_t^y &= r_t^{\text{dc}}.
\label{eq:reward-decouple}
\end{aligned}
\end{equation}
Similarly, the value function $V(s_t)$ is decoupled into two components: $V_x(s_t)$ for the lateral dimension and $V_y(s_t)$ for the longitudinal dimension. These value functions estimate the expected cumulative rewards for the lateral and longitudinal actions, respectively. The advantage estimates $\hat{A}_t^x$ and $\hat{A}_t^y$ are then computed as follows:
\begin{equation}
\begin{aligned}
\delta_t^x &= r_t^x + \gamma V_x(s_{t+1}) - V_x(s_t), \\
\delta_t^y &= r_t^y + \gamma V_y(s_{t+1}) - V_y(s_t), \\
\hat{A}_t^x &= \sum_{l=0}^{\infty}(\gamma \lambda)^l \delta_{t+l}^x, \\
\hat{A}_t^y &= \sum_{l=0}^{\infty}(\gamma \lambda)^l \delta_{t+l}^y,
\label{eq:advantage}
\end{aligned}
\end{equation}
where $\delta_t^x$ and $\delta_t^y$ are the temporal difference errors for the lateral and longitudinal dimensions, $\gamma$ is the discount factor, and $\lambda$ is the GAE parameter that controls the trade-off between bias and variance.

To further clarify the relationship between the advantage estimates and the reward components, we decompose $\hat{A}_t^x$ and $\hat{A}_t^y$ based on the reward decomposition in Eq.~\ref{eq:reward-decouple} and the advantage estimation in Eq.~\ref{eq:advantage}. Specifically, we derive the following decomposition:
\begin{equation}
\begin{aligned}
\hat{A}_t^x &= \hat{A}_t^{\text{sc}} + \hat{A}_t^{\text{pd}} + \hat{A}_t^{\text{hd}}, \\
\hat{A}_t^y &= \hat{A}_t^{\text{dc}},
\end{aligned}
\end{equation}
where $\hat{A}_t^{\text{sc}}$ is the advantage estimate for avoiding static collisions, $\hat{A}_t^{\text{pd}}$ is the advantage estimate for minimizing positional deviations, $\hat{A}_t^{\text{hd}}$ is the advantage estimate for minimizing heading deviations, and $\hat{A}_t^{\text{dc}}$ is the advantage estimate for avoiding dynamic collisions.

These advantage estimates are used to guide the update of the AD policy $\pi_{\theta}$, following the PPO framework~\cite{PPO}. By leveraging the decomposed advantage estimates $\hat{A}_t^x$ and $\hat{A}_t^y$, we can independently optimize the lateral and longitudinal dimensions of the policy. This is achieved by defining separate objective functions $\mathcal{L}_x^{\text{CLIP}}(\theta)$ and $\mathcal{L}_y^{\text{CLIP}}(\theta)$ for each dimension,  as follows:
\begin{equation}
\begin{aligned}
\mathcal{L}_x^{\text{PPO}}(\theta) &= \mathbb{E}_t \left[ \min \left( \rho_t^x \hat{A}_t^x, \ \text{clip}(\rho_t^x, 1-\epsilon_x, 1+\epsilon_x) \hat{A}_t^x \right) \right], \\
\mathcal{L}_y^{\text{PPO}}(\theta) &= \mathbb{E}_t \left[ \min \left( \rho_t^y \hat{A}_t^y, \ \text{clip}(\rho_t^y, 1-\epsilon_y, 1+\epsilon_y) \hat{A}_t^y \right) \right], \\
\mathcal{L}^{\text{PPO}}(\theta) &= \mathcal{L}_x^{\text{PPO}}(\theta) + \mathcal{L}_y^{\text{PPO}}(\theta),
\end{aligned}
\end{equation}
where $\rho_t^x = \frac{\pi_{\theta}(a_t^x \mid s_t)}{\pi_{\theta_{\text{old}}}(a_t^x \mid s_t)}$ is the importance sampling ratio for the lateral dimension, $\rho_t^y = \frac{\pi_{\theta}(a_t^y \mid s_t)}{\pi_{\theta_{\text{old}}}(a_t^y \mid s_t)}$ is the importance sampling ratio for the longitudinal dimension, $\epsilon_x$ and $\epsilon_y$ are small constants that control the clipping range for the lateral and longitudinal dimensions, ensuring stable policy updates.

The clipped objective function $\mathcal{L}^{\text{PPO}}(\theta)$ prevents excessively large updates to the policy parameters $\theta$, thereby maintaining training stability.

\begin{table*}[ht]
    \centering
{
\begin{tabular}{lccccccccc}
    \toprule
    RL:IL & CR$\downarrow$ & DCR$\downarrow$ & SCR$\downarrow$ & DR$\downarrow$ & PDR$\downarrow$ & HDR$\downarrow$ &ADD$\downarrow$ & Long. Jerk$\downarrow$ & Lat. Jerk$\downarrow$ \\
    \midrule
     0:1  & 0.229 & 0.211 & 0.018 & 0.066 & 0.039 & 0.027  & 0.238 & 3.928 & 0.103\\
     1:0  & 0.143 & 0.128 & 0.015 &0.080 &0.065 &0.015 &0.345 &4.204 &0.085\\
     2:1 & 0.137 & 0.125 & 0.012 & 0.059 & 0.050 & 0.009  & 0.274 & 4.538 & 0.092\\
     4:1 & 0.089 & 0.080 & 0.009 & 0.063 & 0.042 & 0.021  & 0.257 & 4.495 & 0.082 \\
     8:1 & 0.125 & 0.116 & 0.009 & 0.084 & 0.045 & 0.039  & 0.323 & 5.285 & 0.115\\
    \bottomrule
\end{tabular}
}
    \caption{\textbf{Ablation on RL-to-IL step mixing ratios in the reinforced post-training stage.}}
    \label{tab:ratio}
\end{table*}

\subsection{Auxiliary Objective}
RL usually faces the challenge of sparse rewards, which makes the convergence process unstable and slow. To speed up convergence, we introduce auxiliary objectives that provide dense guidance to the entire action distribution.

The auxiliary objectives are designed to penalize undesirable behaviors by incorporating specific reward sources, including dynamic collisions, static collisions, positional deviations, and heading deviations. These objectives are computed based on the actions \( a_t^{x, \text{old}} \) and \( a_t^{y, \text{old}} \) selected by the old AD policy \( \pi_{\theta_{\text{old}}} \) at time step \( t \). To facilitate the evaluation of these actions, we separate the probability distribution of the action into four parts:
\begin{equation}
\begin{aligned}
\Delta \pi_y^{\text{dec}} &= \sum_{a_t^y < a_t^{y, \text{old}}} \pi_\theta(a_t^y \mid s_t), \\
\Delta \pi_y^{\text{acc}} &= \sum_{a_t^y > a_t^{y, \text{old}}} \pi_\theta(a_t^y \mid s_t), \\
\Delta \pi_x^{\text{left}} &= \sum_{a_t^x < a_t^{x, \text{old}}} \pi_\theta(a_t^x \mid s_t), \\
\Delta \pi_x^{\text{right}} &= \sum_{a_t^x > a_t^{x, \text{old}}} \pi_\theta(a_t^x \mid s_t).
\end{aligned}
\end{equation}
Here, \( \Delta \pi_y^{\text{dec}} \) represents the total probability of deceleration actions, \( \Delta \pi_y^{\text{acc}} \) represents the total probability of acceleration actions, \( \Delta \pi_x^{\text{left}} \) represents the total probability of leftward steering actions, and \( \Delta \pi_x^{\text{right}} \) represents the total probability of rightward steering actions.

\boldparagraph{Dynamic Collision Auxiliary Objective.}  
The dynamic collision auxiliary objective adjusts the longitudinal control action \(a_t^y\) based on the location of potential collisions relative to the ego vehicle. If a collision is detected ahead, the policy prioritizes deceleration actions (\(a_t^y < a_t^{y, \text{old}}\)); if a collision is detected behind, it encourages acceleration actions (\(a_t^y > a_t^{y, \text{old}}\)). To formalize this behavior, we define a directional factor \(f_\text{dc}\):
\begin{equation}
\begin{aligned}
f_\text{dc} = \begin{cases} 
1 & \text{if the collision is ahead}, \\
-1 & \text{if the collision is behind}.
\end{cases} 
\end{aligned}
\end{equation}

The auxiliary objective for dynamic collision avoidance is defined as:
\begin{equation}
\begin{aligned}
\mathcal{L}_\text{dc}(\theta_y) = \mathbb{E}_t \left[ 
    \hat{A}_t^\text{dc} \cdot f_\text{dc} \cdot (\Delta \pi_y^{\text{dec}} - \Delta \pi_y^{\text{acc}})
\right],
\end{aligned}
\end{equation}
where \(\hat{A}_t^\text{dc}\) is the advantage estimate for dynamic collision avoidance.

\boldparagraph{Static Collision Auxiliary Objective.}  
The static collision auxiliary objective adjusts the steering control action $a_t^x$ based on the proximity to static obstacles. If the static obstacle is detected on the left side, the policy promotes rightward steering actions ($a_t^x > a_t^{x,\text{old}}$); if the static obstacle is detected on the right side, it promotes leftward steering actions ($a_t^x < a_t^{x,\text{old}}$). To formalize this behavior, we define a directional factor $f_\text{sc}$:  
\begin{equation}
\begin{aligned}
f_\text{sc} = \begin{cases} 
1 & \text{if static obstacle is on the left}, \\
-1 & \text{if static obstacle is on the right}.
\end{cases} 
\end{aligned}
\end{equation}

The auxiliary objective for static collision avoidance is defined as:  
\begin{equation}
\begin{aligned}
\mathcal{L}_\text{sc}(\theta_x) = \mathbb{E}_t \left[ 
    \hat{A}_t^\text{sc} \cdot f_\text{sc} \cdot (\Delta \pi_x^{\text{right}} - \Delta \pi_x^{\text{left}})
\right],
\end{aligned}
\end{equation}  
where $\hat{A}_t^\text{sc}$ is the advantage estimate for static collision avoidance.  

\boldparagraph{Positional Deviation Auxiliary Objective.}  
The positional deviation auxiliary objective adjusts the steering control action $a_t^x$ based on the ego vehicle's lateral deviation from the expert trajectory. If the ego vehicle deviates leftward, the policy promotes rightward corrections ($a_t^x > a_t^{x,\text{old}}$); if it deviates rightward, it promotes leftward corrections ($a_t^x < a_t^{x,\text{old}}$). We formalize this with a directional factor $f_\text{pd}$:  
\begin{equation}
\begin{aligned}
f_\text{pd} = \begin{cases} 
1 & \text{if ego vehicle deviates leftward}, \\
-1 & \text{if ego vehicle deviates rightward}.
\end{cases} 
\end{aligned}
\end{equation}

The auxiliary objective for positional deviation correction is:
\begin{equation}
\begin{aligned}
\mathcal{L}_\text{pd}(\theta_x) = \mathbb{E}_t \left[ 
    \hat{A}_t^\text{pd} \cdot f_\text{pd} \cdot (\Delta \pi_x^{\text{right}} - \Delta \pi_x^{\text{left}})
\right],
\end{aligned}
\end{equation}  
where $\hat{A}_t^\text{pd}$ estimates the advantage of trajectory alignment.

\boldparagraph{Heading Deviation Auxiliary Objective.}  
The heading deviation auxiliary objective adjusts the steering control action $a_t^x$ based on the angular difference between the ego vehicle’s current heading and the expert’s reference heading. If the ego vehicle deviates counterclockwise, the policy promotes clockwise corrections ($a_t^x > a_t^{x,\text{old}}$); if it deviates clockwise, it promotes counterclockwise corrections ($a_t^x < a_t^{x,\text{old}}$). To formalize this behavior, we define a directional factor $f_\text{hd}$:  
\begin{equation}
\begin{aligned}
f_\text{hd} = \begin{cases} 
1 & \text{if ego vehicle deviates clockwise}, \\
-1 & \text{if ego vehicle deviates counterclockwise}.
\end{cases} 
\end{aligned}
\end{equation}

The auxiliary objective for heading deviation correction is then defined as:  
\begin{equation}
\begin{aligned}
\mathcal{L}_\text{hd}(\theta_x) = \mathbb{E}_t \left[ 
    \hat{A}_t^\text{hd} \cdot f_\text{hd} \cdot (\Delta \pi_x^{\text{right}} - \Delta \pi_x^{\text{left}})
\right],
\end{aligned}
\end{equation}  
where $\hat{A}_t^\text{hd}$ is the advantage estimate for heading alignment.  

\begin{table*}[ht]
\begin{center}
\centering
\resizebox{0.98\textwidth}{!}{
\begin{tabular}{cccccccccccccc}
\toprule
\multirow{2}{*}{ID} & Dynamic & Static & Position & Heading & \multirow{2}{*}{CR$\downarrow$} &\multirow{2}{*}{DCR$\downarrow$} &\multirow{2}{*}{SCR$\downarrow$} &\multirow{2}{*}{DR$\downarrow$} &\multirow{2}{*}{PDR$\downarrow$} &\multirow{2}{*}{HDR$\downarrow$} &\multirow{2}{*}{ADD$\downarrow$} &\multirow{2}{*}{Long. Jerk$\downarrow$} &\multirow{2}{*}{Lat. Jerk$\downarrow$}\\
& Collision & Collision & Deviation & Deviation & & & & & & & & & \\
\midrule
1 & \cmark  &  &  &  & 0.172 & 0.154 & 0.018 & 0.092 & 0.033 & 0.059  & 0.259 & 4.211 & 0.095 \\
2 &  & \cmark & \cmark & \cmark & 0.238 & 0.217 & 0.021 & 0.090 & 0.045 & 0.045  & 0.241 & 3.937 & 0.098 \\
3 & \cmark &  & \cmark & \cmark & 0.146 & 0.128 & 0.018 & 0.060 & 0.030 & 0.030  & 0.263 & 3.729 & 0.083\\
4 & \cmark & \cmark &  & \cmark & 0.151 & 0.142 & 0.009 & 0.069 & 0.042 & 0.027 & 0.303 & 3.938 & 0.079\\
5 & \cmark & \cmark & \cmark &  & 0.166 & 0.157 & 0.009 & 0.048 & 0.036 & 0.012 & 0.243 & 3.334 & 0.067\\
6 & \cmark & \cmark & \cmark & \cmark & 0.089 & 0.080 & 0.009 & 0.063 & 0.042 & 0.021 & 0.257 & 4.495 & 0.082 \\
\bottomrule
\end{tabular}
}
\end{center}
\vspace{-2mm}
\caption{\textbf{Ablation on reward sources.} The table shows the impact of different reward components on performance.}
\label{tab:reward_ablation}
\end{table*}

\begin{table*}[ht]
\begin{center}
\centering
\resizebox{0.98\textwidth}{!}{
\begin{tabular}{ccccccccccccccc}
\toprule
\multirow{2}{*}{ID} & \multirow{2}{*}{PPO Obj.}  & Dynamic Col. & Static Col. & Position Dev. & Heading Dev. & \multirow{2}{*}{CR$\downarrow$} & \multirow{2}{*}{DCR$\downarrow$}  & \multirow{2}{*}{SCR$\downarrow$} & \multirow{2}{*}{DR$\downarrow$} & \multirow{2}{*}{PDR$\downarrow$} & \multirow{2}{*}{HDR$\downarrow$} & \multirow{2}{*}{ADD$\downarrow$} & \multirow{2}{*}{Long. Jerk$\downarrow$} & \multirow{2}{*}{Lat. Jerk$\downarrow$} \\
& & Auxiliary Obj. & Auxiliary Obj. & Auxiliary Obj. & Auxiliary Obj. & & & & & & & & & \\
\midrule
1 &\cmark&  &  &  &  & 0.249 & 0.223 & 0.026 & 0.077 & 0.047 & 0.030  & 0.266 & 4.209 & 0.104 \\
2 &\cmark& \cmark &  &  &  & 0.178 & 0.163 & 0.015 & 0.151 & 0.101 & 0.050 & 0.301 & 3.906 & 0.085 \\
3 &\cmark&  & \cmark & \cmark & \cmark & 0.137 & 0.125 & 0.012 & 0.157 & 0.145 & 0.012 & 0.296 & 3.419 & 0.071 \\
4 &\cmark& \cmark &  & \cmark & \cmark & 0.169 & 0.151 & 0.018 & 0.075 & 0.042 & 0.033 & 0.254 & 4.450 & 0.098 \\
5 &\cmark& \cmark & \cmark &  & \cmark & 0.149 & 0.134 & 0.015 & 0.063 & 0.057 & 0.006 & 0.324 & 3.980 & 0.086 \\
6 &\cmark& \cmark & \cmark & \cmark & & 0.128 & 0.119  & 0.009 & 0.066 & 0.030 & 0.036  & 0.254 & 4.102 & 0.092 \\
7 &&\cmark  &\cmark  &\cmark  &\cmark  & 0.187 &0.175  &0.012 &0.077 &0.056  &0.021  &0.309  &5.014  &0.112  \\
8 &\cmark& \cmark & \cmark & \cmark & \cmark & 0.089 & 0.080 & 0.009 & 0.063 & 0.042 & 0.021  & 0.257 & 4.495 & 0.082 \\
\bottomrule
\end{tabular}
}
\end{center}
\vspace{-2mm}
\caption{\textbf{Ablation on auxiliary objectives.} The table shows the impact of different auxiliary objectives on performance.}
\label{tab:auxiliary_ablation}
\end{table*}

\boldparagraph{Overall Auxiliary Objectives.}  
The overall auxiliary objectives are a weighted sum of the individual objectives:
\begin{equation}
\begin{aligned}
\mathcal{L}_\text{aux}(\theta) = &\lambda_1 \mathcal{L}_\text{dc}(\theta_y) + \lambda_2 \mathcal{L}_\text{sc}(\theta_x)  + \\ 
&\lambda_3 \mathcal{L}_\text{pd}(\theta_x) +\lambda_4 \mathcal{L}_\text{hd}(\theta_x),
\end{aligned}
\end{equation}
where $\lambda_1$, $\lambda_2$, $\lambda_3$, and $\lambda_4$ are weighting coefficients that balance the contributions of each auxiliary objective.

\boldparagraph{Optimization Objective.}  
The final optimization objective combines the clipped PPO objective with the auxiliary objective:
\begin{equation}
\mathcal{L}(\theta) = \mathcal{L}^{\text{PPO}}(\theta) + \mathcal{L}_\text{aux}(\theta).
\end{equation}

\vspace{-0.05 in}
\paragraph{Random $n$-gram Selection}
% In the process of reutilizing generated $n$-grams as draft tokens and applying repetition penalty, there exists an inherent trade-off. Meanwhile, 

In our experiments, we observe that the draft tokens provided to the target model for parallel validation often yield multiple valid groups. Building on this observation, we randomly select one valid $n$-gram to serve as the final output. By leveraging the fact that multiple valid $n$-grams emerge during verification, we ensure that the final output is both diverse and accurate.

% we observe that the draft tokens provided to the target model for parallel validation can yield multiple valid groups.

In summary, the overall flow of our framework is presented in \cref{alg:algorithm}. 

\section{Experiment}\label{sec: exp}
In this section, we assess the efficacy of our algorithm by addressing the following key questions. 
(1) Can offline RL algorithms achieve stronger performance on the reduced datasets selected by~\name?
(2) How does \name~perform compare to other offline data selection methods? 
(3) What are the factors that contribute to \name's effectiveness?

\begin{figure}[t]
    \centering
    \subfigure{\includegraphics[scale=0.24]{d4rl-hard/walker2d-medium-v0-hard.pdf}}
    \hspace{0.2cm}
    \subfigure{\includegraphics[scale=0.24]{d4rl-hard/walker2d-expert-v0-hard.pdf}}
    \hspace{0.2cm}
    \subfigure{\includegraphics[scale=0.24]{d4rl-hard/walker2d-medium-replay-v0-hard.pdf}}
    % \subfigure{\includegraphics[scale=0.20]{d4rl-hard/walker2d-medium-expert-v0-hard.pdf}}
    \subfigure{\includegraphics[scale=0.24]{d4rl-hard/hopper-medium-v0-hard.pdf}}
    \hspace{0.2cm}
    \subfigure{\includegraphics[scale=0.24]{d4rl-hard/hopper-expert-v0-hard.pdf}}
    \hspace{0.2cm}
    \subfigure{\includegraphics[scale=0.24]{d4rl-hard/hopper-medium-replay-v0-hard.pdf}}
    % \subfigure{\includegraphics[scale=0.20]{d4rl-hard/hopper-medium-expert-v0-hard.pdf}}
    \subfigure{\includegraphics[scale=0.24]{d4rl-hard/halfcheetah-medium-expert-v0-hard.pdf}}
    \hspace{0.2cm}
    \subfigure{\includegraphics[scale=0.24]{d4rl-hard/halfcheetah-expert-v0-hard.pdf}}
    \hspace{0.2cm}
    \subfigure{\includegraphics[scale=0.24]{d4rl-hard/halfcheetah-medium-replay-v0-hard.pdf}}
    % \subfigure{\includegraphics[scale=0.20]{d4rl-hard/halfcheetah-medium-v0-hard.pdf}}
    \caption{Experimental results on the D4RL (Hard) offline datasets. All experiment results were averaged over five random seeds. Our method achieves better or
    comparable results than the baselines with lower computational complexity.}
    \label{fig: d4rl hard}
    \vspace{-0.5cm}
\end{figure}

% \begin{figure*}[t]
%     \centering
%     \includegraphics[width=\linewidth]{mujoco/fig1.pdf}
%     \vspace{-2em}
%     \caption{Sample-based selection performance of several baselines and \name~with different selected subset sizes~($x\%$).
%     The horizontal line is the performance of TD3+BC trained with the original dataset.}
%     \label{fig: d4rl minimal ratio}
%     \vspace{-1em}
% \end{figure*}

% \begin{figure}[t]
%     \centering
%     \includegraphics[width=\linewidth]{mujoco/traj.pdf}
%     \caption{In trajectory-based selection, \name~outperforms behavior cloning (\nameh) using trajectories with the highest accumulative returns, presenting a robust method for selecting the most useful data from training sets of compromised quality.}
%     \label{fig: d4rl topbc}
%     \vspace{-1em}
% \end{figure}

\subsection{Setup}
We evaluate algorithms on the offline RL benchmark D4RL~\citep{fu2020d4rl} to answer the aforementioned questions.
In addition, we consider a more challenging scenario where we add additional low-quality data to the dataset to simulate noise in real-world tasks, named D4RL~(hard).
The evaluation process commences with the selection of offline data, followed by the training of a widely recognized offline RL algorithm, TD3+BC~\citep{fujimoto2021minimalist}, on this reduced dataset for 1 million time steps.
To ensure a fair comparison, we apply the same offline RL algorithm to data subsets obtained by different algorithms. 
Evaluation points are set at every 5,000 training time steps and involve calculating the return of 10 episodes per point.
The results, comprising averages and standard deviations, are computed with five independent random seeds.
On the other hand, we can also incorporate our method into offline model-based approaches, such as MOPO~\citep{yu2020mopo} and MoERL~\citep{kidambi2020morel}.
Similarly, we only need to replace the current offline loss with the corresponding policy and model loss.

\textbf{Baselines}. 
We compare \name~with data selection methods in RL.
Specifically, previous work on prioritized experience replay for online RL~\citep{schaul2015prioritized} aligns closely with our objective. 
We make this a baseline \namep~where samples with the highest TD losses form the reduced dataset. 
Baseline \nameo~presents the performance by training TD3+BC with the original, complete dataset. 
Baseline \namer~randomly selects subsets from the D4RL dataset that are of the same size as \name.
We also compare our method with general dataset reduction techniques from supervised learning.
Specifically, we adopt the coherence criterion from Kernel recursive least squares~($\mathtt{KRLS}$)~\citep{engel2004kernel}, the log det criterion by forward selection in informative vector machines~($\mathtt{LogDet}$)~\citep{seeger2004greedy} and the adapting kernel representation~($\mathtt{BlockGreedy}$)~\citep{schlegel2017adapting} as our baselines.

%Specifically, we consider randomly selecting offline coreset as our baseline algorithms.
% In addition, we consider separately selecting high-reward offline datasets and low-reward offline datasets as our baseline algorithms.

\subsection{Experimental Results}
\label{sec:exp_perf}
% To compare the performance of different algorithms, we adopt two data selection schemes: sample-based selection and trajectory-based selection. They differ in the smallest unit of selection: the first selects samples in each iteration, while the second selects trajectories.

% As for the trajectory-based selection, prioritized sampling is no loner applicable. As an alternative, we compare with \nameh, which selects trajectories with the highest accumulative reward from the complete dataset. We again compare with the \nameo~as the reference to an upper limit of performance.

\begin{table*}[t]
    \centering
    \begin{tabular}{c|cccc}
    \toprule
    & KRLS & Log-Det & BlockGreedy & \name \\
    \midrule
    Hopper-medium-v0 & 69.4$\pm$2.5 & 58.4$\pm$3.6 & 83.7$\pm$2.2 & \textbf{94.3$\pm$4.6}\\
    Hopper-expert-v0 & 91.0$\pm$1.1 & 90.7$\pm$1.3 & 98.7$\pm$0.5 & \textbf{110.0$\pm$0.5}\\
    Hopper-medium-replay-v0 & 28.5$\pm$3.2 & 29.4$\pm$1.2 & 30.5$\pm$2.4 & \textbf{35.3$\pm$3.2}\\
    Walker2d-medium-v0 & 49.1$\pm$2.8 & 47.5$\pm$3.4 & 53.3$\pm$3.6 & \textbf{80.5$\pm$2.9}\\
    Walker2d-expert-v0 & 68.4$\pm$3.2 & 67.5$\pm$5.6 & 74.8$\pm$3.4 & \textbf{104.6$\pm$2.5}\\
    Walker2d-medium-replay-v0 & 14.3$\pm$1.2 & 15.2$\pm$2.2 & 16.7$\pm$1.3 & \textbf{21.1$\pm$1.8}\\
    Halfcheetah-medium-v0 & 23.4$\pm$0.5 & 21.9$\pm$0.9 & 27.5$\pm$0.7 & \textbf{41.0$\pm$0.2}\\
    Halfcheetah-expert-v0 & 73.9$\pm$1.4 & 72.1$\pm$2.2 & 79.2$\pm$1.8 & \textbf{88.5$\pm$2.4}\\
    Halfcheetah-medium-replay-v0 & 39.5$\pm$0.3 &39.9$\pm$0.5 & 40.5$\pm$1.0 & \textbf{41.1$\pm$0.4}\\
    \bottomrule
    \end{tabular}
    \caption{Experimental results on the D4RL~(Hard) offline datasets. All experiment results were averaged over five random seeds. Our method performs better than the dataset reduction baselines.}
    \label{tab: varied performance}
\end{table*}

\begin{figure}[t]
    \centering
    \subfigure{\includegraphics[scale=0.20]{d4rl/halfcheetah-medium-expert-v0.pdf}}
    \subfigure{\includegraphics[scale=0.20]{d4rl/hopper-medium-v0.pdf}}
    \subfigure{\includegraphics[scale=0.20]{d4rl/hopper-medium-expert-v0.pdf}}
    \subfigure{\includegraphics[scale=0.20]{d4rl/walker2d-medium-expert-v0.pdf}}
    \caption{Experimental results on the D4RL offline datasets. All experiment results were averaged over five random seeds. Our method achieves better or comparable results than the baselines consistently.}
    \label{fig: d4rl original}
\end{figure}

\paragraph{Answer of Question 1:}
To show that \name~can improve offline RL algorithms, we compare \name~with Complete Dataset, Prioritized, and Random in the Mujoco domain.
The experimental results in Figure~\ref{fig: d4rl hard} show that our method achieves superior performance than baselines.
By leveraging the reduced dataset generated from \name, the agent can learn much faster than learning from the complete dataset.
Further, the results in Figure~\ref{fig: d4rl original} show that \name~also performs better than the complete dataset and data selection RL baselines in the standard D4RL datasets. 
This is because prior methods select data in a random or loss-priority manner, which lacks guidance for subset selection and leads to degraded performance for downstream tasks.

In addition, to test \name's generality across various offline RL algorithms on various domains, we also conduct experiments on Antmaze tasks.
We use IQL~\citep{kostrikov2021offline} as the backbone of offline RL algorithms.
The experimental results in Table~\ref{tab: other domain2} show that our method achieves stronger performance than baselines.
In the antmaze tasks, the agent is required to stitch together various trajectories to reach the target location.
In this scenario, randomly removing data could result in the loss of critical data, thereby preventing complete the task.
Differently, \name~extracts valuable subset by balancing data quantity with performance, achieving a stronger performance than the complete dataset.

% In Figure~\ref{fig: d4rl minimal ratio}, we show the performance of different algorithms with the sample-based selection scheme. The experimental results show that \name~can achieve performance close to \nameo~with a small amount of data. For example, we use only $3\%$ of the original dataset in the Hopper tasks. \namer~and \namep, on the other hand, present a stark contrast, even not showing a stable learning trend with the same amount of training data. 
% In addition, we also evaluate the performance on the trajectory-based selection setting. Please refer to Appendix~\ref{appendix: trajectory} for the detailed experimental results.
% For the trajectory-based selection, experimental results in Figure~\ref{fig: d4rl topbc} show that \name~maintains its superiority in this setting with suboptimal (e.g., \texttt{medium}) datasets. This evidence suggests that \name~provides a valuable strategy for selecting data conducive to effective training under conditions of compromised data quality.

\paragraph{Answer of Question 2:}
To test whether \name~can select more valuable data than the data selection algorithms in supervised learning, we compare our method with KRLS~\citep{engel2004kernel}, Log-Det~\citep{seeger2004greedy} and BlockGreedy~\citep{schlegel2017adapting} in the D4RL~(Hard) datasets.
The experimental results in Table~\ref{tab: varied performance} show that our method generally outperforms baselines.
We hypothesize that supervised learning is static with fixed learning objectives, while offline RL's dynamic nature makes the target values evolve with policy updates, complicating the data selection process.
Therefore, the data selection methods in supervised learning cannot be directly applied to offline RL scenarios.

% Additionally, we observe that  $\texttt{Random}$ performs better than $\texttt{Q-diff}$.
% We attribute this phenomenon to the broader data coverage of $\texttt{Random}$, while the data coverage of $\texttt{Q-diff}$ is narrow.
% However, we also note that in some tasks, such as $\texttt{Hopper-medium-expert-v0}$, $\texttt{Hopper-expert-v0}$ and $\texttt{Walker2d-expert-v0}$, $\texttt{Random}$ initially performs well, but as training progresses, its performance starts to decline.
% We find that this coincides with unstable Q-values, which can be attributed to the increased extrapolation error caused by the reduced training dataset.
% In contrast, \name~performs better since it closely approximates the original gradients, thus preventing Q-values from diverging.


% For this reason, when the dataset quality is high~(e.g., \texttt{medium-expert} dataset), TopBC performs comparably to \name.

% \begin{table*}[t]
%     \centering
%     \caption{\name~with varying dataset sizes~($x\%$). Highlighted is the performance comparable to training TD3+BC with the complete dataset. \name~typically achieves good results with 5\%-15\% data, indicating that existing offline RL datasets contain a high degree of redundancy.
%     We adopt the normalized score metric proposed by the D4RL benchmark. Scores roughly range from 0 to 100, where 0 corresponds to the performance of a random policy and 100 indicates the performance of an expert.} 
%     \label{tab: varied performance}
%     \begin{tabular}{c|cccc}
%     \toprule
%         & 5\% & 10\% & 15\% & 20\% \\
%         \midrule
%         Hopper-medium-v0 & 91.8$\pm$3.6 & 92.6$\pm$3.0 & 94.0$\pm$4.8 & 95.2$\pm$1.6\\
%         Walker2d-medium-v0 & 14.8$\pm$7.3 & 57.9$\pm$3.6 & 69.3$\pm$4.0 & 71.7$\pm$1.9 \\
%         Halfcheetah-medium-v0 & 40.5$\pm$0.0 & 40.9$\pm$0.1 & 41.3$\pm$0.1 & 41.2$\pm$0.5 \\
%         Hopper-expert-v0 & 111.6$\pm$0.9 & 110.6$\pm$1.9 & 112.7$\pm$0.1 & 112.4$\pm$0.1 \\
%         Walker2d-expert-v0 & 74.5$\pm$6.4 & 84.4$\pm$5.0 & 97.6$\pm$3.1 & 100.2$\pm$1.0 \\
%         Halfcheetah-expert-v0 & 57.5$\pm$6.4 & 84.3$\pm$2.7 & 97.8$\pm$0.8 & 100.1$\pm$3.0 \\
%         Hopper-medium-expert-v0 & 108.1$\pm$1.1 & 112.4$\pm$0.3 & 112.3$\pm$0.05 & 112.8$\pm$0.1\\
%         Walker2d-medium-expert-v0 & 79.3$\pm$2.1 & 85.4$\pm$5.3 & 96.2$\pm$6.7 & 101.4$\pm$3.6 \\
%         Halfcheetah-medium-expert-v0 & 67.5$\pm$0.5 & 86.2$\pm$5.0 & 85.8$\pm$1.5 & 92.4$\pm$1.3\\
%     \bottomrule
%     \end{tabular}
% \end{table*}


% \subsection{Ablation Study}\label{sec:exp_ab}
% \textbf{Varying dataset size}.\ \ In Table~\ref{tab: varied performance}, we show the performance of \name~with varying dataset sizes ranging from $5\%$ to $20\%$.
% The results demonstrate that \name~requires only $5\%$ or $10\%$ of the original dataset to obtain good performance.
% Further, \name~can achieve similar performance with \nameo~with $20\%$ data of the original dataset.
% This indicates that existing offline RL datasets are characterized by a high degree of redundancy.

\begin{figure}[t]
    \centering
    \includegraphics[width=0.97\linewidth]{visual.jpg}
    \caption{Visualization of the \textcolor{blue}{complete dataset} and the \textcolor{orange}{reduced dataset} in \texttt{halfcheetah} task. The higher opacity of a point represents a large time step towards the end of an episode. The dataset embedding is characterized by its division into different components. 
    % In \texttt{walker2d} (upper), components vary with time steps.
     Samples selected by \name~connect different components by focusing on the data related to the task.}
    \label{fig: t-sne}
\end{figure}

\begin{table}[t]
    \centering 
    \begin{tabular}{c|cccc}
    \toprule
        Env & Random & Prioritized & Complete Dataset & \name\\
        \midrule
        Antmaze-umaze-v0 & 75.1$\pm$2.5 & 70.2$\pm$3.6 & 87.5$\pm$1.3 & \textbf{90.7$\pm$3.3}\\
        Antmaze-umaze-diverse-v0 & 46.3$\pm$1.9 & 44.7$\pm$2.7 & 62.2$\pm$2.0 & \textbf{76.7$\pm$2.2} \\
        Antmaze-medium-play-v0 & 59.3$\pm$1.6 & 60.3$\pm$2.9 & 71.2$\pm$2.2 & \textbf{80.3$\pm$2.9}\\
        Antmaze-medium-diverse-v0 & 43.6$\pm$2.7 & 46.9$\pm$3.8 & 70.0$\pm$1.6 & \textbf{84.9$\pm$3.8}\\
        Antmaze-large-play-v0 &	3.7$\pm$0.7 & 15.0$\pm$3.5 & 39.6$\pm$3.6 & \textbf{46.0$\pm$3.5}\\
        Antmaze-large-diverse-v0 & 16.0$\pm$3.6 & 20.5$\pm$3.7 & 47.5$\pm$1.1 & \textbf{52.0$\pm$3.7}\\
    \bottomrule
    \end{tabular}
    \caption{Experimental results on the Antmaze offline datasets. All experiment results were averaged over five random seeds. Our method performs better than baselines. }
    \label{tab: other domain2}
\end{table}

% \begin{figure*}[t]
%     \centering
%     \subfigure{\includegraphics[scale=0.27]{ablation_moduler1.pdf}}
%     \hspace{0.3cm}\subfigure{\includegraphics[scale=0.27]{ablation_moduler2.pdf}}
%     \caption{Ablation results on D4RL~(Hard) tasks with the normalized score metric.}
%     \label{fig: modular ablation}
% \end{figure*}

% In this subsection, we conduct ablation studies to study the effect of different modules and import hyper-parameters.


\paragraph{Answer of Question 3:}
To study the contribution of each component in our learning framework, we conduct the following ablation study. 
\nameq: We replace the empirical returns used to update Q functions with the standard target Q function in the TD loss function. 
\namei: We set the number of data selection rounds to 1 and study the function of multi-round data selection.
The experimental results in Figure~\ref{fig: modular ablation} in Appendix~\ref{sec: ablation} show that removing any of these two modules will worsen the performance of \name. In case like $\texttt{walker2d-medium}$, ablation \namei~even decrease the performance by over 80\%, and ablation \nameq~results in a 95\% performance drop in $\texttt{walker2d-expert}$. Furthermore, we also find that in the $\texttt{halfcheetah}$ tasks, the impact of removing the two modules is relatively small. This result can be attributable to the fact that this task has a limited state space, and we can directly apply OMP to the entire dataset and identify important and diverse data.

We visualize the selected data by \name~to better understand how it works. 
Figure~\ref{fig: t-sne} displays the t-SNE low-dimensional embeddings, with the complete dataset in blue and the selected data in orange. 
The higher opacity of a point indicates a larger time step. The dataset's structure is revealed by its segmentation into diverse components: 
In \texttt{halfcheetah}, each component reflects a distinct skill of the agent.
For example, from 1 to 7, they represent falling, leg lifting, jumping, landing, leg swapping, stepping, and starting, respectively.
We can observe that the selected samples by \name~ not only cover each component of the dataset but also effectively bridge the gaps between them, enhancing the dataset's versatility and coherence. 
Moreover, we find that \name~is less concerned with the falling data and instead focuses on the data related to the task.
This observation can explain the improved performance of \name. For additional visualizations, please refer to Appendix~\ref{appendix: visual}.

% \textbf{Generalizability of \name}. \ \
% We evaluate the generalizability of \name~from two perspectives.
% First, we add IQL~\cite{kostrikov2021offline} as a baseline and apply \name~to IQL by using the gradient of the training loss of the V-function in IQL as the criterion.
% On the other hand, we evaluate \name~on the other domains, such as robotic manipulation (Adroit) and sparse reward (Antmaze) tasks.
% The experiments in Appendix~\ref{appendix: other domain} and Appendix~\ref{appendix: other algorithm} show that \name~is not only applicable to other algorithms, such as IQL~\cite{kostrikov2021offline}, but also to other domains.

% \textbf{Generalizability of subset}. \ \
% To test the generalizability of the dataset selected by~\name, we select subset by applying~\name~to TD3+BC.
% Then we evaluate the performance of IQL on the selected subset. 
% The experimental results in Table~\ref{tab: td3bc2iql} in Appendix~\ref{appendix: tb3bc2iql} demonstrate that the selected subset based on TD3+BC is effectively applicable to IQL.

% \textbf{Sensitivity for hyperparameter}. \ \
% We evaluate the performance of \name~with various cluster numbers~(from 1 to 50) and approximation bounds~(from 0.0001 to 0.05).
% The experimental results in Appendix~\ref{appendix: cluster number} and Appendix~\ref{appendix: approx bound} show that the suitable cluster number is between 25 and 50.
% Too few clusters (e.g., less than 5) are detrimental to the algorithm.
% In addition, a smaller approximation bound represents a larger reduced dataset.
% Similar to the ablation of the size of the reduced dataset in Table~\ref{tab: varied performance}, \name~requires only a 0.01 approximation bound to obtain good performance.

\subsection{Computational complexity}
We report the computational overhead of \name~on various datasets. 
All experiments are conducted on the same computational device (GeForce RTX 3090 GPU). 
The results in Appendix~\ref{appendix: computation complexity} indicate that even on datasets containing millions of data points, the computational overhead of our method remains low~(e.g., several minutes).
This low computational complexity can be attributed to the trajectory-based selection technique in Sec.~\ref{sec: offline omp}~(II) and the regularized constraint technique in Sec.~\ref{sec:method:outer}, making our method easily scalable to large-scale datasets. 

% This low computational complexity can be attributed to the batch mechanism designed in section 3.2 (IV), which reduces the computational complexity from $O(MN)$ to $O(|\mathcal{B}|N)$, making our method easily scalable to large-scale datasets. $M, N, |\mathcal{B}|$ are the size of the full dataset, reduced dataset, and batch respectively.

% We conduct t-SNE based dimensionality reduction to the cluster centroids and these five trajectories.
% The experimental results are shown in the , where darker colors indicate moving towards the end of the trajectory.

% From the experimental results, we find that in the walker2d task, \name~ tends to select more low-reward but more diverse data points ~(upper right) while selecting a few high-reward data points~(left and bottom).
% We attribute this phenomenon to the narrow distribution of the high-reward points, allowing us to approximate the original gradients with only a few points. 
% In the halfcheetah task, \name~ connects useful information while ignoring low-quality data~(e.g., data point \texttt{1}).

\section{Related Works}
\subsection{Speculative Decoding}
\label{app:sd}
% The acceleration of large language model (LLM) inference has gained significant attention in recent years. Speculative Decoding \citep{sd1, sd2} has been introduced as an innovative sampling algorithm to accelerate inference. Subsequent advancements \citep{layerskip, kangaroo, eagle, eagle2, draft_verify, medusa} eliminated the need for a draft model, enabling Self-Speculative Decoding. While these methods successfully improve LLM inference efficiency, they overlook the challenge posed by the large size of the KV cache. Approaches such as Triforce~\citep{triforce} and MagicDec~\citep{magicdec} address this limitation by incorporating KV cache compression during the drafting phase. However, their applicability is limited to scenarios characterized by long prefixes and short outputs, making them unsuitable for ultra-long sequence generation tasks. In such tasks, which are the focus of our work, the need for efficient inference spans both extended input contexts and lengthy outputs, presenting challenges that existing methods fail to address.

Recent advancements in speculative decoding have significantly accelerated large language model (LLM) inference through diverse methodologies. Speculative decoding~\citep{sd1,sd2} traditionally leverages smaller draft models to propose candidate tokens for verification by the target model. Early works like SpecTr~\citep{spectr} introduced optimal transport for multi-candidate selection, while SpecInfer~\citep{specinfer} and Medusa~\citep{medusa} pioneered tree-based structures with tree-aware attention and multi-head decoding to enable parallel verification of multiple candidates. Subsequent innovations, such as Sequoia~\citep{sequoia} and EAGLE-2~\citep{eagle2}, optimized tree construction using dynamic programming and reordering strategies, while Hydra~\citep{hydra} and ReDrafter~\citep{redrafter} enhanced tree dependencies through sequential or recurrent heads. Hardware-aware optimizations, exemplified by SpecExec~\citep{specexec} and Triforce~\citep{triforce}, further improved efficiency by leveraging hierarchical KV caching and quantized inference.  

Self-speculative approaches eliminate the need for external draft models by exploiting internal model dynamics. Draft\&Verify~\citep{draft_verify} and LayerSkip~\citep{layerskip} utilized early-exit mechanisms and Bayesian optimization to skip layers adaptively, whereas Kangaroo~\citep{kangaroo} integrated dual early exits with lightweight adapters. \citet{optimal} and SpecDec++~\citep{specdec++} introduced theoretical frameworks for block-level token acceptance and adaptive candidate lengths. Parallel decoding paradigms, such as PASS~\citep{pass} and MTJD~\citep{mtjd}, employed look-ahead embeddings or joint probability modeling to generate multiple candidates in a single pass, while CLLMs~\citep{cllms} and Lookahead~\citep{Lookahead2} reimagined autoregressive consistency through Jacobi decoding and n-gram candidate pools.  

Retrieval-augmented methods like REST~\citep{rest}, and NEST~\citep{nearest} integrated vector or phrase retrieval to draft context-aware tokens, often combining copy mechanisms with confidence-based attribution. Training-centric strategies, including TR-Jacobi~\citep{TR-Jacobi}, enhanced parallel decoding capability via noisy training or self-distilled multi-head architectures. System-level optimizations such as PipeInfer~\citep{PipeInfer} and \citet{faster} addressed scalability through asynchronous pipelines and latency-aware scheduling, while Goodput~\citep{throughput_sd} focused on dynamic resource allocation and nested model deployment. 

Approaches such as Triforce~\citep{triforce} and MagicDec~\citep{magicdec} incorporate KV cache compression during the drafting phase. However, their applicability is limited to scenarios characterized by long prefixes and short outputs, making them unsuitable for ultra-long sequence generation tasks. In such tasks, which are the focus of our work, the need for efficient inference spans both extended input contexts and lengthy outputs, presenting challenges that existing methods fail to address.


\subsection{Long Sequence Generation}
% Generating high-quality long sequences with LLMs presents another critical challenge. Existing sampling algorithms, including top-$p$~\citep{topp}, min-$p$~\citep{minp}, and $\eta$-sampling~\citep{eta}, aim to improve generation quality. Additionally, frameworks like LongWriter~\citep{longwriter} optimize supervised fine-tuning (SFT) datasets to enhance generation quality on long-sequence generation tasks. Despite these advancements, the issue of poor quality in long-sequence generation persists. However, it is worth emphasizing that addressing quality concerns is not the primary focus of our research.

Recent advances in long sequence generation have focused on addressing the challenges of coherence, efficiency, and scalability in producing extended outputs. A pivotal contribution is the LongWriter~\citep{longwriter} framework, which introduces a task decomposition strategy to generate texts exceeding 20,000 words. Complementing this, Temp-Lora~\citep{temp_lora} proposes inference-time training with temporary Lora modules to dynamically adapt model parameters during generation, offering a scalable alternative to traditional KV caching. Similarly, PLANET~\citep{planet} leverages dynamic content planning with sentence-level bag-of-words objectives to improve logical coherence in opinion articles and argumentative essays, demonstrating the effectiveness of structured planning in autoregressive transformers.

In addition, lightweight decoding-side sampling strategies have emerged for repetition mitigation. The foundational work on Nucleus Sampling~\citep{topp} first demonstrated that dynamically truncating low-probability token sets could reduce repetitive outputs while maintaining tractable decoding latency. Building on this, \citet{eta} introduced $\eta$-sampling explicitly linking candidate set reduction to repetition mitigation by entropy-guided token pruning. Recent variants like Min-p~\citep{minp} optimize truncation rules in real-time—scaling thresholds to the maximum token probability. 
And Mirostat Sampling~\citep{mirostat} further integrate lightweight Bayesian controllers to adjust $\eta$ parameters on-the-fly. Our work systematically analyzing how parameterized sampling (\eg, Top-p Min-p, $\eta$-sampling) balances computational overhead and repetition suppression in ultra-long sequence generation pipelines.

\section{Conclusion}

This work analysed the results of evolutionary wrapper approaches using decision tree based models as proxies and compared them with common \gls{FE} techniques on a \gls{HL} detection problem. Three experiments were conducted using the proposed framework, each employing different proxy models.

When comparing the three experiments, an interesting behaviour of the framework was discovered, when changing the proxy model. The \gls{DT} experiment drastically reduced the number of features, while the other models did not. To further reduce the number of features, one could bias the grammar or apply some penalty in the fitness function for the individuals that use a large number of features. This might not change the behaviour when using different models other than a \gls{DT}, but it forcefully reduces the number of features.  

The results confirm that FEDORA can reduce the dimensionality of the data while statistically maintaining baseline performance, in every experiment. The framework consistently outperforms the remaining \gls{FE} methods, with statistical significance and large effect sizes, proving itself as a viable alternative.

The best result obtained is 76.2\% balanced accuracy using an individual from the \gls{RF} experiment, and a \gls{XGB} algorithm as the testing model, using 57 total features (45 Original, 6 Engineered and 6 Complex) out of the 60 original ones. When using the least amount of features, the best result is 72,8\% balanced accuracy using an individual from the \gls{DT} experiment and a \gls{RF} algorithm as the testing model, using a single complex feature.

In future work, exploring the above-mentioned behaviours might be relevant to better understanding them, namely when biasing the grammar or penalizing the use of many features in the fitness function. Concerning the explainability of the FEDORA transformations, researching meaningful grammar operators might prove useful in addressing problem-specific needs. In this case, having logical operators for the boolean features, which have values of "yes" or "no", and the choice of a simple decision algorithm as the proxy, may increase explainability. Additionally, the previous study has identified several areas for future research, yet to be addressed. For instance, comparing the framework with other common and more complex methods and completing the full \gls{ML} pipeline through the use of a method that addresses the \gls{CASH}, such as \cite{assunccao2020evolution}, and comparing it to other full pipeline frameworks, could be beneficial for contextualizing and evaluating the framework within the \gls{AutoML} and \gls{EC} domains. The framework still needs to be analysed with different datasets to properly assess its generalization capabilities.

\section*{Acknowledgements}
We thank Haoyi Wu from ShanghaiTech University, Xuekai Zhu from Shanghai Jiaotong University, Hengli Li from Peking University for helpful discussions on speculative decoding and language modeling. This work presented herein is supported by the National Natural Science Foundation of China (62376031).

% \newpage
% \section{Impact Statements}
% This paper presents work aimed at advancing the field of Machine Learning, specifically in the context of improving efficiency and scalability in generating ultra-long sequences. There are many potential societal consequences of our work, none which we feel must be specifically highlighted here. None of the ethical concerns we foresee require specific actions or warnings in the context of this work.

{
\bibliography{ref}
\bibliographystyle{icml2025}
}


\section{Appendix}
\label{appendix}

\subsection{Survey Questions}
\label{app:survey}

\subsubsection{Scenarios}

Participants were asked about three classes of hiring scenarios: technical coding assessments, resume review, and behavioral interviews (the scenarios are listed by class below). For each scenario, they answered two questions, both on 5-point Likert scales:
\begin{itemize}
    \item How fair does this hiring process seem to you? (``This hiring process seems fair'', 1: Strongly disagree to 5: Strongly agree)
    \item If you were applying for a technology job, would you want to be evaluated this way? (``I want to be evaluated this way'', 1: Strongly disagree to 5: Strongly agree)
\end{itemize}

[Technical Coding Assessments]
\begin{enumerate}
\item An applicant submits a sample of code, which is reviewed by a recruitment team, who determines whether the applicant advances to the next phase.
\item An applicant is given an online coding assessment, which is evaluated by an algorithm. If the applicant reaches a certain score on the autograder, the applicant advances to the next phase. All algorithmic decisions are reviewed by a recruitment team.
\item An applicant is given an online coding assessment, which is evaluated by an algorithm. If the algorithm rejects the applicant, the decision is reviewed by a recruitment team. 
\item An applicant is given an online coding assessment, which is evaluated by an algorithm. If the algorithm advances the applicant to the next phase, the decision is reviewed by a recruitment team. 
\item An applicant is given an online coding assessment, which is evaluated by an algorithm that determines whether an applicant advances to the next phase. 
% \item Why did you select the answers above for the different scenarios related to coding assessments?
\end{enumerate}

[Resume Review]
\begin{enumerate}
\item An applicant submits a resume, which is reviewed by a recruitment team, who determines whether the applicant advances to the next phase.
\item An applicant submits a resume, which is evaluated by an algorithm. The algorithm determines whether the applicant advances to the next phase. All algorithmic decisions are reviewed by a recruitment team. 
\item An applicant submits a resume, which is evaluated by an algorithm. If the algorithm rejects your application, the decision is reviewed by a recruitment team. 
\item An applicant submits a resume, which is evaluated by an algorithm. If the algorithm advances the applicant to the next phase, the decision is reviewed by a recruitment team. 
\item An applicant submits a resume, which is evaluated by an algorithm that determines whether an applicant advances to the next phase. 
% \item Why did you select the answers above for the different scenarios related to resumes?
\end{enumerate}

[Behavioral Interviews]
\begin{enumerate}
\item An applicant has an interview with a member of the recruitment team. The recruitment team determines whether the applicant advances to the next phase.
\item An applicant participates in an automated video interview, where the applicant receives interview questions and records video responses. The video, including the applicant’s speech (fluency, prosody, pronunciation, language usage) and nonverbal behaviors (facial expressions, posture, and eye movements), is evaluated by an algorithm. Whether you advance to the next phase is determined by the algorithm. All algorithmic decisions are reviewed by a recruitment team.
\item An applicant participates in an automated video interview, where the applicant receives interview questions and records video responses. The video, including the applicant’s speech (fluency, prosody, pronunciation, language usage) and nonverbal behaviors (facial expressions, posture, and eye movements), is evaluated by an algorithm. If the algorithm rejects the applicant,  the decision is reviewed by a recruitment team. 
\item An applicant participates in an automated video interview, where the applicant receives interview questions and records video responses. The video, including the applicant’s speech (fluency, prosody, pronunciation, language usage) and nonverbal behaviors (facial expressions, posture, and eye movements), is evaluated by an algorithm. If the algorithm advances the applicant to the next phase, the decision is reviewed by a recruitment team. 
\item An applicant participates in an automated video interview, where the applicant receives interview questions and records video responses. The video, including the applicant’s speech (fluency, prosody, pronunciation, language usage) and nonverbal behaviors (facial expressions, posture, and eye movements), is evaluated by an algorithm that determines whether an applicant advances to the next phase.
% \item Why did you select the answers above for the different scenarios related to interviews?
\end{enumerate}

At the end of each set of Likert questions, participants were also asked an open response question (``Why did you select the answers above for the different scenarios related to [coding assessments/resumes/interviews]?'').

\subsubsection{Awareness of AEDTs}

In this section, participants were asked for each hiring process (online coding assessment, automated resume readers, and automated interviews) to check the box to indicate whether they have experience or knowledge of it:
\begin{itemize}
    \item[$\square$] Yes, I have experienced it
    \item[$\square$] No, but I have heard of it
    \item[$\square$] I'm not sure, but have heard of it
    \item[$\square$] No, I have not heard of or experienced it
\end{itemize}

Participants also responded to ``I know how my data was used in the hiring process'' and ``I received feedback from automated hiring algorithms'' from 1: Strongly disagree to 5: Strongly agree.

\subsubsection{Strategy Use}

Participants were asked the following questions about strategy use:
\begin{itemize}
\item Have you modified your resume specifically for automated resume readers? (Yes/No)
\item Have you added keywords from your job description? (Yes/No)
\item Have you changed the layout? (Yes/No)
\item Have you put it through a resume scanner? (Yes/No)
\item Have you modified your resume in some other way for automated hiring? (please specify)
\item Did you use a tool (LeetCode, HackerRank, etc.) to practice for coding assessments? (Yes/No)
\item Have you used anything else to prepare for automated assessments? (please specify)
\item Have you ever received a job referral? (Yes/No)
\item What proportion of your job applications did you have a referral for? (approximate percentage)
\item Approximately how many companies did you apply to? 
\item How did you learn about the application process? (check all that apply)
    \begin{itemize}
        \item[$\square$] Application materials and descriptions
        \item[$\square$] Online resources
        \item[$\square$] Career services through university 
        \item[$\square$] People who had gone through the application process
        \item[$\square$] Recruiter outside of company
        \item[$\square$] Recruiter through company
        \item[$\square$] Family members who worked at companies 
        \item[$\square$] Friends who worked at companies 
        \item[$\square$] Other people who worked at companies
    \end{itemize}
There was also an option to include additional strategies and an attention check in this stage.
\end{itemize}

\subsubsection{Hiring Outcome}
Participants were also asked about their hiring process and its outcome.
\begin{itemize}
\item Have you completed your hiring process? (Yes/No/Not applying to jobs)
\item I am satisfied with my hiring process so far. (1: Strongly disagree to 5: Strongly agree)
\item What is the outcome of your hiring process so far? 
    \begin{itemize}
        \item[$\square$] Multiple job offers
        \item[$\square$] One job offer
        \item[$\square$] No job offers
        \item[$\square$] Not applying to jobs
    \end{itemize}
\end{itemize}

\subsubsection{Demographic Information}
All questions in this section were optional and asked participants to disclose demographic information.

\begin{itemize}
    \item How would you describe your gender identity? (Select all that apply)
        \begin{itemize}
            \item[$\square$] Woman
            \item[$\square$] Man
            \item[$\square$] Non-binary
            \item[$\square$] Genderqueer
            \item[$\square$] Agender
            \item[$\square$] A gender not listed
        \end{itemize}
    \item What best describes you? (Select all that apply)
        \begin{itemize}
            \item[$\square$] Black or African-American
            \item[$\square$] American Indian or Alaskan Native
            \item[$\square$] Asian American or Asian
            \item[$\square$] Hispanic or Latinx
            \item[$\square$] Middle Eastern or North African
            \item[$\square$] Pacific Islander
            \item[$\square$] White or Caucasian
            \item[$\square$] Some other race, ethnicity, or origin 
        \end{itemize}
    \item What is your family’s approximate household income? 
\end{itemize}

\clearpage 

\subsection{Complete Statistical Results}
\label{app:stats}

\begin{table}[ht]
\begin{tabular}{lrrrrl}
\hline
\textbf{}                                            & \textbf{Estimate} & \textbf{Std. Error} & \textbf{t value} & \textbf{Pr(\textgreater{}|t|)} & \textbf{} \\ \hline
(Intercept)                                       & 2.786  & 0.266 & 10.493 & \textless{}0.01 &   \\
Added job description keywords to resume & 0.139  & -1.468    & 0.144 & 0.121            &   \\
Modified resume layout for resume readers & 0.150         & 0.133           & 1.119            & 0.265                         &           \\
Put resume through a resume scanner               & 0.001  & 0.136 & 0.007  & 0.995           &   \\
Practiced for online coding assessment            & 0.249  & 0.140 & 1.787  & 0.075           &   \\
Used referrals                                    & -0.336 & 0.136 & -2.478 & 0.014           & * \\
Percent of companies applied to with referral   & 0.002         & 0.003           & 0.817            & 0.415                         &           \\
Number of companies applied to                    & 0.001  & -0.516    & 0.606 & 0.405            &   \\
Awareness of online coding assessments            & -0.551 & 0.235 & -2.349 & 0.020           & * \\
Awareness of resume scanners                      & 0.014  & 0.183 & 0.076  & 0.940           &   \\
Awareness of automated video interviews           & 0.354  & 0.170 & 2.113  & 0.036           & * \\
Knowledge of data use                             & 0.055  & 0.047 & 1.162  & 0.247           &   \\
Received feedback in the hiring process           & 0.058  & 0.046 & 1.257  & 0.210           &   \\
Used application materials and descriptions       & -0.176 & 0.114 & -1.539 & 0.125           &   \\
Used online resources                             & 0.288  & 0.133 & 2.160  & 0.032           & * \\
Used career services through university           & 0.063  & 0.108 & 0.588  & 0.557           &   \\
Talked with people who had recently applied       & 0.129  & 0.127 & 1.012  & 0.313           &   \\
Connected with recruiter outside of company       & 0.053  & 0.159 & 0.336  & 0.737           &   \\
Connected with recruiter through company          & 0.124  & -1.346    & 0.180 & 0.191            &   \\
Had family who worked at companies        & 0.044  & 0.144 & 0.306  & 0.760           &   \\
Had friends who worked at companies               & 0.140  & 0.112 & 1.247  & 0.214           &   \\
Connected with other company contacts             & -0.022 & 0.126 & -0.179 & 0.858           &   \\
Race                                              & 0.005  & 0.109 & 0.425  & 0.671           &   \\
Gender                                            & -0.003 & 0.142 & -0.024 & 0.981           &   \\
Income                                            & 0.0000002  & 0.0000003 & 0.569  & 0.570           &   \\ \hline
\end{tabular}
\caption{\label{tab:fairStats} Linear regression model of procedural fairness perceptions for automated processes based on strategy use, awareness of AEDTs, gender, race, and income.}
\end{table}

\begin{table}[ht]
\begin{tabular}{lrrrrl}
\hline
\textbf{}                                            & \textbf{Estimate} & \textbf{Std. Error} & \textbf{t value} & \textbf{Pr(\textgreater{}|t|)} & \textbf{} \\ \hline
(Intercept)                                 & 2.479  & 0.268 & 9.267  & \textless{}0.01 &    \\
Added job description keywords to resume    & 0.140         & -1.374              & 0.171           & 0.210                         &           \\
Modified resume layout for resume readers & 0.169         & 0.135          & 1.257            & 0.210                         &           \\
Put resume through a resume scanner         & 0.038  & 0.137 & 0.273  & 0.785           &    \\
Practiced for online coding assessment      & 0.201  & 0.141 & 1.427  & 0.155           &    \\
Used referrals                              & -0.316 & 0.137 & -2.312 & 0.022           & *  \\
Percent of companies applied to with referral   & 0.002         & 0.003           & 0.670            & 0.504                         &           \\
Number of companies applied to              & 0.0004  & 0.001 & 0.544  & 0.589           &    \\
Awareness of online coding assessments      & -0.557 & 0.237 & -2.356 & 0.019           & *  \\
Awareness of resume scanners                & -0.046 & 0.184 & -0.248 & 0.805           &    \\
Awareness of automated video interviews     & 0.440  & 0.169 & 2.608  & {0.010}           & * \\
Knowledge of data use                       & 0.106  & 0.047 & 2.240  & 0.026           & *  \\
Received feedback in the hiring process     & 0.027  & 0.046 & 0.588  & 0.558           &    \\
Used application materials and descriptions & -0.220 & 0.012 & -1.911 & 0.057           &    \\
Used online resources                       & 0.261  & 0.134 & 1.942  & 0.054           &    \\
Used career services through university     & 0.152  & 0.108 & 1.399  & 0.163           &    \\
Talked with people who had recently applied & 0.172  & 0.128 & 1.344  & 0.181           &    \\
Connected with recruiter outside of company & 0.160  & -0.005    & 0.996 & 0.180          &    \\
Connected with recruiter through company    & 0.125  & -1.392    & 0.165 & 0.968           &    \\
Had family who worked at companies  & -0.006 & 0.145 & -0.040 & 0.968           &    \\
Had friends who worked at companies         & 0.134  & 0.113 & 1.188  & 0.236           &    \\
Connected with other company contacts       & 0.049  & 0.127 & 0.385  & 0.700           &    \\
Race                                        & 0.013  & 0.110 & 0.122  & 0.903           &    \\
Gender                                      & -0.116 & 0.143 & -0.815 & 0.416           &    \\
Income                                      & 0.0000002  & 0.0000003 & 0.623  & 0.534           &    \\ \hline
\end{tabular}
\caption{\label{tab:evalStats} Linear regression model of willingness to be evaluated by automated processes based on strategy use, awareness of AEDTs, gender, race, and income.}
\end{table}

\clearpage

\begin{table}[ht]
\begin{tabular}{lrrrrrl}
\toprule
& \textbf{Estimate}  & \textbf{Std. Error} & \textbf{t value} & \textbf{Pr(\textgreater{}|t|)} &   \\
\hline
(Intercept)                                          & 0.329     & 0.237      & 1.386   & 0.168                &   \\
Added job description keywords to resume    & 0.168     & 0.107      & 1.563   & 0.121                 &   \\
Modified resume layout for resume readers & 0.103     & -0.724     & 0.471   & 0.515                 &   \\
Put resume through a resume scanner                  & 0.020     & 0.101      & 0.201   & 0.841                 &   \\
Practiced for online coding assessment               & -0.201    & 0.133      & -1.513  & 0.133                 &   \\
Used referrals                                       & 0.122     & 0.100      & 1.213   & 0.227                 &   \\
Percent of companies applied to with referral   & 0.004     & 0.002      & 2.063   & 0.041                 & * \\
Number of companies applied to                       & 0.0004    & 0.001     & 0.835   & 0.405                 &   \\
Awareness of online coding assessments               & 0.050     & 0.199      & 0.252   & 0.801                 &   \\
Awareness of resume scanners                         & -0.019    & 0.173      & -0.109  & 0.913                 &   \\
Awareness of automated video interviews              & -0.036    & 0.157      & -0.228  & 0.820                 &   \\
Knowledge of data use                                & 0.039     & 0.004      & 0.984   & 0.327                 &   \\
Received feedback in the hiring process              & 0.011     & 0.004      & 0.302   & 0.763                 &   \\
Used application materials and descriptions          & 0.025     & 0.009      & 0.279   & 0.781                 &   \\
Used online resources                                & -0.174    & 0.115      & -1.518  & 0.132                 &   \\
Used career services through university              & 0.055     & 0.085      & 0.644   & 0.521                 &   \\
Talked with people who had recently applied          & 0.024     & 0.107      & 0.225   & 0.823                 &   \\
Connected with recruiter outside of company          & 0.009     & 0.112      & 0.080   & 0.937                 &   \\
Connected with recruiter through company             & 0.115     & 0.088      & 1.314   & 0.191                 &   \\
Had family who worked at companies           & -0.140    & 0.109      & -1.287  & 0.200                 &   \\
Had friends who worked at companies                  & 0.160     & 0.087      & 1.841   & 0.068                 &   \\
Connected with other company contacts                & -0.101    & 0.093     & -1.089  & 0.278                &   \\
Race                                                 & -0.008    & 0.119      & -0.070  & 0.945                 &   \\
Gender                                               & 0.081     & 0.081     & 0.991   & 0.324                 &   \\
Income                                               & 0.000001 & 0.0000002  & 2.530   & 0.013                 & * \\
\bottomrule
\end{tabular}
\caption{\label{tab:jobStats} Linear regression model of job success based on strategy use, awareness of AEDTs, gender, race, and income.}
\end{table}



\end{document}


