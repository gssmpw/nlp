%%
%% This is file `sample-second-authordraft.tex',
%% generated with the docstrip utility.
%%
%% The source files were:
%%
%% samples.dtx  (with options: `all,proceedings,bibtex,authordraft')
%% 
%% IMPORTANT NOTICE:
%% 
%% For the copyright see the source file.
%% 
%% Any modified versions of this file must be renamed
%% with new filenames distinct from sample-sigconf-authordraft.tex.
%% 
%% For distribution of the original source see the terms
%% for copying and modification in the file samples.dtx.
%% 
%% This generated file may be distributed as long as the
%% original source files, as listed above, are part of the
%% same distribution. (The sources need not necessarily be
%% in the same archive or directory.)
%%
%%
%% Commands for TeXCount
%TC:macro \cite [option:text,text]
%TC:macro \citep [option:text,text]
%TC:macro \citet [option:text,text]
%TC:envir table 0 1
%TC:envir table* 0 1
%TC:envir tabular [ignore] word
%TC:envir displaymath 0 word
%TC:envir math 0 word
%TC:envir comment 0 0
%%
%%
%% The first command in your LaTeX source must be the \documentclass
%% command.
%%
%% For submission and review of your manuscript please change the
%% command to \documentclass[manuscript, screen, review]{acmart}.
%%
%% When submitting camera ready or to TAPS, please change the command
%% to \documentclass[sigconf]{acmart} or whichever template is required
%% for your publication.
%%
%%
%\documentclass[sigconf,authordraft]{acmart}
%\documentclass[manuscript,review,anonymous]{acmart}
%\documentclass[manuscript,anonymous]{acmart}
%\documentclass[sigconf]{acmart}

\documentclass[manuscript]{acmart}
\usepackage{xspace}   
\usepackage{textcomp} 
\usepackage{graphicx} 
\usepackage{xcolor} % For text colors
\usepackage{multirow}
\usepackage{booktabs}
\usepackage{array}

\DeclareRobustCommand{\projectname}{\textsc{CareMap}\xspace}

%%
%% \BibTeX command to typeset BibTeX logo in the docs
\AtBeginDocument{%
  \providecommand\BibTeX{{%
    Bib\TeX}}}

%% Rights management information.  This information is sent to you
%% when you complete the rights form.  These commands have SAMPLE
%% values in them; it is your responsibility as an author to replace
%% the commands and values with those provided to you when you
%% complete the rights form.
\setcopyright{acmlicensed}
\copyrightyear{2018}
\acmYear{2018}
\acmDOI{XXXXXXX.XXXXXXX}

%% These commands are for a PROCEEDINGS abstract or paper.
\acmConference[Conference acronym 'XX]{Make sure to enter the correct
  conference title from your rights confirmation emai}{June 03--05,
  2018}{Woodstock, NY}
%%
%%  Uncomment \acmBooktitle if the title of the proceedings is different
%%  from ``Proceedings of ...''!
%%
%%\acmBooktitle{Woodstock '18: ACM Symposium on Neural Gaze Detection,
%%  June 03--05, 2018, Woodstock, NY}
\acmISBN{978-1-4503-XXXX-X/18/06}


%%
%% Submission ID.
%% Use this when submitting an article to a sponsored event. You'll
%% receive a unique submission ID from the organizers
%% of the event, and this ID should be used as the parameter to this command.
%%\acmSubmissionID{123-A56-BU3}

%%
%% For managing citations, it is recommended to use bibliography
%% files in BibTeX format.
%%
%% You can then either use BibTeX with the ACM-Reference-Format style,
%% or BibLaTeX with the acmnumeric or acmauthoryear sytles, that include
%% support for advanced citation of software artefact from the
%% biblatex-software package, also separately available on CTAN.
%%
%% Look at the sample-*-biblatex.tex files for templates showcasing
%% the biblatex styles.
%%
\newcommand{\arthur}[1]{{\bf \color{red} [arthur: #1]}}
%%
%% The majority of ACM publications use numbered citations and
%% references.  The command \citestyle{authoryear} switches to the
%% "author year" style.
%%
%% If you are preparing content for an event
%% sponsored by ACM SIGGRAPH, you must use the "author year" style of
%% citations and references.
%% Uncommenting
%% the next command will enable that style.
%%\citestyle{acmauthoryear}


%%
%% end of the preamble, start of the body of the document source.
\begin{document}

%%
%% The "title" command has an optional parameter,
%% allowing the author to define a "short title" to be used in page headers.
\title[Information Needs and Design Opportunities for Family Caregivers of Older Adult Patients in Critical Care]{``It Felt Like I Was Left in the Dark'': Exploring Information Needs and Design Opportunities for Family Caregivers of Older Adult Patients in Critical Care Settings}

%%
%% The "author" command and its associated commands are used to define
%% the authors and their affiliations.
%% Of note is the shared affiliation of the first two authors, and the
%% "authornote" and "authornotemark" commands
%% used to denote shared contribution to the research.


\author{Shihan Fu}
\affiliation{%
  \institution{Northeastern University}
  \city{Boston}
  \state{Massachusetts}
  \country{United States}}
\email{fu.shiha@northeastern.edu}

\author{Bingsheng Yao}
\affiliation{%
  \institution{Northeastern University}
  \city{Boston}
  \state{Massachusetts}
  \country{United States}}
\email{b.yao@northeastern.edu}



\author{Smit Desai}
\affiliation{%
  \institution{Northeastern University}
  \city{Boston}
  \state{Massachusetts}
  \country{United States}}
\email{sm.desai@northeastern.edu}

\author{Yuqi Hu}
\affiliation{%
  \institution{Northeastern University}
  \city{Boston}
  \state{Massachusetts}
  \country{United States}}
\email{hu.yuqi@northeastern.edu}

\author{Yuling Sun}
\affiliation{%
  \institution{University of Michigan}
  \city{Ann Arbor}
  \state{Michigan}
  \country{United States}}
\email{}

\author{Samantha Stonbraker}
\affiliation{%
  \institution{University of Colorado College of Nursing}
  \city{Aurora}
  \state{Colorado}
  \country{United States}}
\email{Samantha.Stonbraker@CUAnschutz.edu}


\author{Yanjun Gao}
\affiliation{%
  \institution{University of Colorado Anschutz Medical Campus}
  \city{Aurora}
  \state{Colorado}
  \country{United States}}
\email{}


\author{Elizabeth M. Goldberg}
\affiliation{%
  \institution{University of Colorado Anschutz Medical Campus}
  \city{Aurora}
  \state{Colorado}
  \country{United States}}
\email{elizabeth.goldberg@cuanschutz.edu}




\author{Dakuo Wang}
\authornote{Corresponding author}
\affiliation{%
  \institution{Northeastern University}
  \city{Boston}
  \state{Massachusetts}
  \country{United States}}
\email{d.wang@northeastern.edu}

%%
%% By default, the full list of authors will be used in the page
%% headers. Often, this list is too long, and will overlap
%% other information printed in the page headers. This command allows
%% the author to define a more concise list
%% of authors' names for this purpose.
\renewcommand{\shortauthors}{Fu et al.}

%%
%% The abstract is a short summary of the work to be presented in the
%% article.
\begin{abstract}
Older adult patients constitute a rapidly growing subgroup of Intensive Care Unit (ICU) patients. 
In these situations, their family caregivers are expected to represent the unconscious patients to access and interpret patients' medical information. 
However, caregivers currently have to rely on overloaded clinicians for information updates and typically lack the health literacy to understand complex medical information.
Our project aims to explore the information needs of caregivers of ICU older adult patients, from which we can propose design opportunities to guide future AI systems. 
The project begins with formative interviews with 15 caregivers to identify their challenges in accessing and understanding medical information; 
From these findings, we then synthesize design goals and propose an AI system prototype to cope with caregivers' challenges. 
The system prototype has two key features: a timeline visualization to show the AI extracted and summarized older adult patients' key medical events; and an LLM-based chatbot to provide context-aware informational support.
We conclude our paper by reporting on the follow-up user evaluation of the system and discussing future AI-based systems for ICU caregivers of older adults.
\end{abstract}

%%
%% The code below is generated by the tool at http://dl.acm.org/ccs.cfm.
%% Please copy and paste the code instead of the example below.
%%
\begin{CCSXML}
<ccs2012>
   <concept>
       <concept_id>10003120.10003121</concept_id>
       <concept_desc>Human-centered computing~Human computer interaction (HCI)</concept_desc>
       <concept_significance>500</concept_significance>
       </concept>
   <concept>
       <concept_id>10003120.10003121.10003122</concept_id>
       <concept_desc>Human-centered computing~HCI design and evaluation methods</concept_desc>
       <concept_significance>500</concept_significance>
       </concept>
   <concept>
       <concept_id>10003120.10003130.10003131</concept_id>
       <concept_desc>Human-centered computing~Collaborative and social computing theory, concepts and paradigms</concept_desc>
       <concept_significance>300</concept_significance>
       </concept>
   <concept>
       <concept_id>10010405.10010444</concept_id>
       <concept_desc>Applied computing~Life and medical sciences</concept_desc>
       <concept_significance>500</concept_significance>
       </concept>
 </ccs2012>
\end{CCSXML}

\ccsdesc[500]{Human-centered computing~Human computer interaction (HCI)}
\ccsdesc[500]{Human-centered computing~HCI design and evaluation methods}
\ccsdesc[300]{Human-centered computing~Collaborative and social computing theory, concepts and paradigms}
\ccsdesc[500]{Applied computing~Life and medical sciences}

%%
%% Keywords. The author(s) should pick words that accurately describe
%% the work being presented. Separate the keywords with commas.
\keywords{Human-AI collaborations, family caregivers, critical care, large language models, information need}
%% A "teaser" image appears between the author and affiliation
%% information and the body of the document, and typically spans the
%% page.


\received{20 February 2007}
\received[revised]{12 March 2009}
\received[accepted]{5 June 2009}

%%
%% This command processes the author and affiliation and title
%% information and builds the first part of the formatted document.
\begin{teaserfigure}
    \centering
    \includegraphics[width=1\linewidth]{PICS/Frame54.png}
    \caption{The structure of the CareMap system. CareMap takes raw EHR data, such as patient information and clinical notes, as input and transforms it into caregiver-facing outputs. These outputs are presented through two key features: a visual timeline of major medical events and a chatbot that supports real-time, personalized inquiries. The system is powered by three LLM-driven modules: summarization, information extraction, and conversational support. \projectname is designed to enhance caregivers’ access to and understanding of a patient’s clinical course in critical care.}
    \label{teaser}
\end{teaserfigure}

\maketitle

\section{Introduction}

The Intensive Care Unit (ICU) is a specialized critical care ward in hospitals, which is dedicated to providing acute, life-saving treatment for patients with life-threatening conditions caused by severe illnesses~\cite{hanberger2005intensive, valentin2011recommendations}. 
In the United States, older adults (aged 65 and above)
account for more than half of ICU admissions, with the amount continuing to grow~\cite{angus2006critical, flaatten2017status, jarvis2023physical}.
Such a vulnerable population of ICU patients has been the focus of care provision~\cite{goldfarb2017outcomes, grieve2019analysis}, however, caregivers of older adult patients are often overlooked in such critical care settings~\cite{gonccalves2023surviving}.
Caregivers of older adult patients are often their family members and are assigned powers of attorney to unconscious patients for making critical medical decisions~\cite{au2017family, hines2011effectiveness, johnson1995perceived}. 
Prior research suggests that if the family caregivers can actively engage with the medical decision-making process, it can significantly enhance the ICU patients' final outcomes~\cite{choi2019exploring}.


However, caregivers often find it challenging to access and understand the necessary clinical information to make informed decisions on behalf of their care recipients in the ICU~\cite{gaeeni2014informational, al2013family, paul2004meeting}. 
Firstly, caregivers can only access limited information about the ICU patients, as they cannot enter the ICU ward frequently, their older adult patients are often unable to talk to them due to unconsciousness, and they primarily rely on the ICU clinical team, who are already overwhelmingly busy, to obtain critical information.~\cite{schnock2017identifying, au2017family, hines2011effectiveness}. 
Secondly, caregivers often find it difficult to fully understand the highly specialized medical language or results due to their low level of health  literacy~\cite{young2017family}.
To address these challenges, various solutions have been proposed, yet mostly centered on the ICU clinical team side to innovate their best practices. 
For example, ICU clinicians are recommended to communicate more frequently with caregivers~\cite{pronovost2003improving}, and they
may use a checklist~\cite{caligtan2012bedside, nelson2006improving} or a daily-goal form~\cite{scheunemann2011randomized, azoulay2002impact}to present more complete clinical information and in a more understandable way to caregivers.
However, these best-practice recommendations often fail to meet the caregivers' information needs~\cite{davidson2017guidelines, goldfarb2017outcomes, meert2013family}.
To truly address this gap, we need novel, caregiver-centered perspectives and technical solutions that directly support their access to and understanding of ICU-related information.
% often have limited health literacy, which makes it difficult for them to interpret the highly specialized and complex medical jargon used in the ICU~\cite{young2017family}.


Researchers in Human-Computer Interaction (HCI) and Computer-Supported Cooperative Work (CSCW) have proposed various technological solutions to support caregivers' information needs~\cite{Sette2023, Yuexing2024, Nikkhah2021, Eunkyung2024}.
One group of works designed new tools to facilitate clinicians sharing information with caregivers~\cite{ziqi2024, Yuexing2024} and caregivers seeking clinical information of older adult patients~\cite{montagna2023data, wei2024leveraging}.
Another group of works focused on the development of visualization tools to decipher complex clinical information to support caregivers' understanding (e.g., a visualization dashboard for caregivers to understand autism children's behaviors~\cite{kong2017comparative} and a dashboard for caregivers to support patients' upper-limb stroke rehabilitation therapy process~\cite{ploderer2016armsleeve}).
However, these designs remain limited in the ICU context: as these tools are not tailored to support caregivers' information needs in high-stakes and time-sensitive ICU settings; and the complexity and density of information generated by the ICU environment can easily overwhelm caregivers in comparison to the other contexts. As such, existing visualization dashboards may be insufficient to support caregivers in effectively understanding patient information in critical care environments.

Recent advancements in artificial intelligence (AI) have opened new possibilities to help family caregivers better access and easily understand clinical information. 
For instance, AI models demonstrate great potential to extract relevant information from massive and unrelated raw clinical data from electronic health records (EHRs)~\cite{hayrinen2008definition}, and AI can also help translate complicated medical jargon into simplified language for the general audience to understand ~\cite{yim2024preliminary, wachter2024will, wong2018using}. 
In addition, chatbot systems powered by large language models (LLMs) can support caregivers' individualized information-seeking requests in real time~\cite{afshar2024prompt}. 
However, it remains underexplored whether and how these AI-based technologies may effectively support caregivers' information needs while their older adult patients are in the ICU.
To this end, we propose this study to focus on exploring the information needs of caregivers of ICU older adult patients and designing novel AI functionalities to support their needs within the critical care environment. 

Specifically, we conducted a two-stage study.
First, we conducted a formative study consisting of semi-structured interviews with 15 caregivers to understand the challenges they faced in how they access and understand clinical information in critical care settings.
Caregivers reported that the primary difficulties come from fragmented clinical information updates from healthcare providers, complicated clinical reports with limited or no explanations available, and frustration caused by repetitive but ineffective communication with the clinical team.

Building on these findings, we propose a set of design guidelines and implement an AI-based prototype system, \projectname, to support caregivers’ access to and understanding of ICU patient information. As illustrated in Figure~\ref{teaser}, the system processes complex clinical information from the ICU and leverages three LLM-driven modules—summarization, extraction, and conversation—to generate caregiver-facing outputs, including daily summaries, treatment plans, and context-aware responses. 
These outputs are delivered through two core features: (1) a visual timeline of key medical events that streamlines access to systematic status updates, and (2) an LLM-based chatbot that enables personalized information inquiries through natural language conversation. We then conducted usability evaluations with 10 family caregivers to assess the prototype. Qualitative analysis revealed the potential of LLM-based systems to reduce caregivers’ cognitive burdens, improve their access to complex clinical information, and enhance engagement with care teams.
 
% Building on the findings, we propose a set of design guidelines and implement an AI-based prototype, \projectname, which offers two core functionalities  (as shown in Fig~\ref{teaser}):
% a visual timeline of patients' medical events to support streamlined access with systematic patient status updates and an LLM-based chatbot to facilitate personalized information inquiries through versatile text-based conversations. We then conducted usability evaluations with 10 family caregivers to assess the prototypes. Qualitative analysis reveals the promising opportunity of leveraging AI to support caregivers' access and understanding of complex medical information, reduce cognitive burdens, and enhance engagement with care provider teams.

This work makes the following contributions. 
First, we systematically investigated the challenges caregivers of older adults encountered in accessing and understanding clinical information in critical care settings.
Then, we presented a suite of design guidelines to support the design of caregiver-facing interfaces with advanced AI technologies and evaluated two AI-based functionality designs.
Finally, our findings contribute to the broader discourse on technology-mediated caregiver support and the role of AI in supporting caregiving experiences for family caregivers in high-stakes, time-sensitive critical care scenarios.


\section{Related Work}~\label{relatedwork}

We conduct a comprehensive review of the experiences of family caregivers for older adult patients in critical care environments, various technologies developed for caregivers in clinical settings, and the role of AI-based clinical information technology in this context.
\subsection{Family Caregivers of Older Adults in Critical Care Settings}

% Patients admitted to the Intensive Care Unit (ICU) often face life-threatening conditions caused by illnesses, injuries, or complications~\cite{hanberger2005intensive, valentin2011recommendations}. 
In the United States, over half of ICU admissions involve older adults aged 65 and above~\cite{flaatten2017status}. 
This population is receiving growing attention due to their complex health needs, which are often compounded by chronic diseases and acute health crises~\cite{jarvis2023physical}. 
When older adult patients are unable to talk or are unconscious due to severe injuries, family caregivers of older adults have to actively engage in the care process~\cite{au2017family, johnson1995perceived}.
For instance, family caregivers could be assigned powers of attorney to make critical decisions on behalf of their loved ones~\cite{hines2011effectiveness}.
On the other hand, they need to prepare for the patient’s post-discharge care by coordinating the necessary support with healthcare providers, as caregiving demands often continue or escalate following patients' ICU discharge~\cite{choi2014fatigue}. 
The demanding responsibilities of family caregivers often place an immense burden on them, which is often compounded by the difficulties of accessing and understanding medical information in the ICU.


% These factors contribute to significant challenges both during their ICU stay and before discharge.
% Meanwhile, caregivers of older adult patients face immense challenges~\cite{au2017family, hines2011effectiveness}. 
However, caregivers are often not familiar with the critical care environment and only possess limited health literacy, which causes significant difficulty for them in understanding the overwhelming and complicated critical care situation.
ICU wards impose restrictive visitation policies to avoid disruptions caused by the presence of unnecessary non-professional personnel~\cite{dragoi2022visitation, tabah2022variation}. 
Such restrictions leave caregivers with difficulties in comprehensively accessing and understanding the patient's health condition and treatment progression in a timely manner.
Moreover, caregivers experience deep anxiety and helplessness about their loved ones because of the sudden deterioration of older adults' health and are unable to anticipate what could happen to the older adults~\cite{jennerich2020unplanned}.
% Many are unfamiliar with the ICU environment and deeply anxious about the unplanned admission of their loved ones ~\cite{jennerich2020unplanned}.
Studies report that up to 54\% of family caregivers experience psychological trauma or severe stress after caregiving experiences in critical care situations~\cite{alfheim2019post}.

% During older adult patients' ICU stays and before discharge, family caregivers bear compounded responsibilities, serving a dual role that is both emotionally and practically demanding.
% On the one hand, they act as crucial advocates, representing the values and preferences of critically ill older adults who are unable to express themselves~\cite{au2017family, hines2011effectiveness}. 
% Family caregivers are frequently called upon to make critical decisions on behalf of their loved ones~\cite{hines2011effectiveness}.
% On the other hand, they must prepare for the patient’s post-discharge care by coordinating the necessary support with healthcare providers, as caregiving demands often continue or escalate following ICU discharge~\cite{choi2014fatigue}. 

% This dual role—speaking on behalf of the patient and preparing for their care—places an immense burden on caregivers, particularly as they often have limited direct access to patients due to ICU restrictions. 
% ICU wards impose restrictive visitation policies to avoid disruptions caused by the presence of unnecessary non-professional personnel~\cite{dragoi2022visitation, tabah2022variation}. 
% These restrictions leave caregivers with an incomplete understanding of the patient’s condition and limit the opportunities for in-person updates.
Consequently, caregivers have to rely heavily on direct communication with healthcare providers to stay informed about the patient’s status~\cite{yoo2020critical}. 
However, the provider teams are always overloaded with caring for critically ill patients and have very limited opportunities and time to talk with caregivers.
Such constrained caregiver-provider communications can hardly meet caregivers' information needs.
% However, current communication practices, often limited to sporadic in-person updates or brief phone calls, are typically time-constrained and fail to meet caregivers’ informational needs. 
Research highlights persistent dissatisfaction among caregivers despite institutional efforts to improve clinical communication through tools like daily goals forms~\cite{pronovost2003improving}, critical care family satisfaction surveys~\cite{steel2008impact}, and allocation of additional resources to ICU wards~\cite{breslow2005technology}. 
% Caregivers consistently report unclear information as significant barriers to their ability to effectively advocate for and support their loved ones~\cite{azoulay2016communication}.
Further, the lack of clear, timely, and actionable information heightens caregivers' stress and disrupts their ability to prepare for the patient’s post-discharge care~\cite{azoulay2016communication}.
% As a result, family caregivers of older adult ICU patients often overwhelmed with multi-faceted feelings, including anxiety, confusion, and responsibility.
% navigate a complex and critical medical environment, 
% grappling with feelings of overwhelm, confusion, and a profound sense of responsibility, further intensifying the psychological burden of caregiving in such high-stakes situations.

%Family caregivers of older adult ICU patients are key stakeholders in this context, as they often act as crucial patient advocates, representing the values and preferences of critically ill patients who are unable to express themselves~\cite{au2017family, hines2011effectiveness}.

% Futhermore, to know more about the patients condition relays heaily on the communication with healthcare providers. 
% However, current communication practices, often limited to sporadic in-person updates or phone calls with them, are time-constrained and insufficient. 
% Research indicates that despite organizational calls for improved clinical communication practices, such as daily goals form~\cite{pronovost2003improving}, Critical care family satisfaction survey~\cite{steel2008impact}, and allocate the appropriate resources to the ICU ward~\cite{breslow2005technology}and etc, caregiver satisfaction with ICU communication remains low. 
% They consistently report inadequate information and unclear guidance as persistent barriers to effective advocacy and care. 

% As a result, family members of an ICU older adult patient experience xxxx complicated, overwhelming, ... 

% addressing the information challenges faced by caregivers remains a pressing issue within the critical and high-pressure environment of the ICU.


% The Intensive Care Unit (ICU) is a specialized ward providing acute care for patients experiencing life-threatening events caused by illnesses, injuries, or complications~\cite{hanberger2005intensive, valentin2011recommendations}. 
% In the United States, older adults aged 65 and above account for more than half of ICU admissions, a trend driven by the global growth of aging populations~\cite{flaatten2017status}. 
% This demographic faces a higher likelihood of experiencing critical health events~\cite{jarvis2023physical}, positioning them as an expanding and significant subgroup of ICU patients~\cite{angus2006critical, flaatten2017status}.
% During critical transition moments, \textbf{family caregivers} of older adults often act as crucial patient advocates, representing the values and preferences of critically ill patients who are unable to express themselves.~\cite{au2017family, hines2011effectiveness}.   
% This trend, driven by the growing aging population, imposes substantial challenges on healthcare systems~\cite{flaatten2017status}.
% caregivers are frequently called upon to make critical decisions on behalf of their loved ones~\cite{hines2011effectiveness}. 
% The combination of emotional distress, constrained communication, and the weight of decision-making responsibilities often leads to feelings of anxiety and powerlessness~\cite{}. 


% Patients are admitted to intensive care units (ICUs) for acute, life-threatening illnesses that entail substantial emotional distress, high treatment burdens, uncertain prospects for survival, and the potential for new disabilities~\cite{needham2012improving, elliott2014exploring, netzer2014recognizing}. 
% Notably, the number of older adults requiring ICU admission is on the rise~\cite{flaatten2017status}, driven by the aging global population~\cite{united_nations_2019}. This increase is associated with significant costs and resource utilization~\cite{bagshaw2009very, lerolle2010increased}, placing additional strain on healthcare systems. Older adults not only face a higher likelihood of experiencing unexpected critical health events but also tend to recover more slowly after acute illnesses~\cite{jarvis2023physical}. Upon admission to the ICU, the intensity of treatments—such as the use of vasopressors, mechanical ventilation, and renal replacement therapy—often varies between younger and older patients. Studies have shown that older patients typically receive less intensive treatment compared to their younger counterparts~\cite{nguyen2011challenge}.

% While improving survival rates remains a primary goal of ICU care for all patients, another crucial objective, especially for the elderly, is to minimize inherent risks and enhance recovery~\cite{boyd2008recovery}. However, older patients are more susceptible to a range of complications, including nosocomial infections, iatrogenic injuries from invasive monitoring, prolonged bed or chair rest, sleep deprivation, delirium, extended hospital stays, and more restrictive visiting hours for families~\cite{mercier2010iatrogenic, pisani2010factors}. These risks contribute to increased morbidity, cognitive impairment, and functional disability~\cite{herridge2009legacy, girard2010delirium}. Given these challenges, it is imperative to pay special attention to older adult patients hospitalized in ICUs. Age itself is a significant independent risk factor for in-ICU mortality~\cite{halpern2004critical, fuchs2012icu}. Additionally, advanced age is associated with longer ICU stays and a higher likelihood of unsuccessful discharge~\cite{flaatten2017status, wang2019predictors}. Most older adult patients admitted to the ICU present with critical illnesses accompanied by severe physical limitations that complicate their care needs~\cite{coombs2017implementing}. Furthermore, both the hospitalization process and the use of invasive technologies in the ICU are highly stressful for patients and their families~\cite{carlson2015care}.

% In response to the growing number of older adults requiring intensive care, there is an urgent need for healthcare systems to prepare for effective care delivery~\cite{lasiter2011older}. Implementing suitable tools to support the transition of older adults into the ICU can help alleviate the associated pressures and improve outcomes for this vulnerable population.




%\subsection{Family Caregivers' Information Needs and Challenges in the ICU}

%The importance of recognizing information needs for family caregivers during critical ICU moments has been well-documented in health literature~\cite{gaeeni2014informational}.
%Caregivers’ information needs are dynamic and evolve throughout the patient’s ICU stay. Because the ICU environment can change rapidly, even from hour to hour, family members need continuous up-to-date information to stay informed about the patient's condition, treatment, prognosis and diagnostic tests \cite{davidson2010facilitated, bueno2018principales, engstrom2004experiences}.  They also seek to understand the rationale behind specific treatments and daily care decisions, as well as the ICU setting itself (including equipment and protocols) and the various disciplines involved in patient care. Furthermore, caregivers want to know what they can do to support the patient after discharge, emphasizing the need for clear guidance on transitional care and rehabilitation \cite{verhaeghe2005needs}.Obtaining such information about older adult patients is consistently a top priority for caregivers~\cite{naderi2013family}, as it helps reduce stress levels~\cite{bijttebier2001needs}, allows better adaptation to stressful conditions~\cite{azoulay2002impact, yaman2010evaluation}, and improves post-ICU care outcomes~\cite{verhaeghe2005needs}.However, caregivers frequently report that their most critical needs — access to information and reassurance — often go unmet during ICU hospitalizations~\cite{al2013family, paul2004meeting}. 

%The primary issue lies in the lack of a clear and efficient information pathway (e.g., channels or systems for accessing patient-related information) for caregivers, which can be analyzed across several key dimensions. First, information resources for family caregivers are often clinician-driven and passive~\cite{schnock2017identifying}, leaving caregivers dependent on updates from healthcare providers. With limited access to healthcare providers due to the busy ICU schedule, caregivers often remain underinformed. For example, studies report that only a small proportion of ICU families at a prominent academic medical center had the opportunity to meet with an attending ICU physician as part of routine care~\cite{lilly2000intensive}.  Given that the average ICU in the U.S. cares for approximately 10 patients daily~\cite{halpern2006changes}, it is unrealistic to expect the ICU care team to engage in comprehensive discussions with every family. Further compounding this issue is the fragmentation of information across multiple stakeholders~\cite{grant2015resolving, pham2008alterations}.  ICU patients are typically cared for by interdisciplinary teams of specialists and nurses, with frequent handoffs occurring across shifts, nights, and weekends.  This fragmented communication often leaves caregivers uncertain about whom to approach for updates or clarification~\cite{gay2009intensive}.  Additionally, wide variations in health literacy of different caregivers — combined with the prevalence of specialized medical jargon — pose significant barriers~\cite{young2017family}. These factors make it harder for caregivers to interpret critical updates and advocate effectively. 
%Although some research has focused on improving caregiver experiences, efforts remain fragmented. 
%Existing interventions, such as pamphlets, checklists, and bedside electronic medical viewers~\cite{caligtan2012bedside, nelson2006improving, scheunemann2011randomized, azoulay2002impact}, provide limited support for creating a clear and caregiver-centric information pathway. 
%Addressing the information challenges faced by caregivers remains a critical and pressing need within the high-stakes, high-pressure environment of the ICU.


% Several interventions have been implemented to improve family access to important information in the form of pamphlets, checklists, and bedside electronic medical viewers~\cite{caligtan2012bedside, nelson2006improving, scheunemann2011randomized, azoulay2002impact}.

%Need for Interventions Supporting ICU Care Transitions

% A lack of clear information pathways for informal caregivers leads to poor communication with care professionals and fragmented or scattered information sources, and so increases the burden for informal caregivers~\cite{bueno2018main}. 

% There is a clear need to support people with dementia and their caregivers through the journey from diagnosis to care [63] through personalized information and support interventions that increase confidence in making decisions about formal residential care placements [64]. 

% Most interventions to support the care transition focus on providing individual or group-based counseling to informal caregivers provided via phone, email, or in person [53]. Aside from management systems for digital waiting lists, e.g., [46], few digital interventions are developed to address such transitions to higher levels of care in dementia.

% Recent studies show that intentional support for caregivers is lacking, especially in the post-ICU period~\cite{major2019survivors, aagaard2015spouse}

% Prior studies on caregivers of ICU survivors have focused on quantifying the psychosocial burden of caregiving~\cite{van2015reported}, but there is little research to inform improvements in care delivery~\cite{sevin2021optimizing}. 

% Caregivers suggest these needed supports could be supplied in a variety of infrastructural forms—phone calls, home health, post-ICU clinic, peer support, online forums—and individualized to maximize impact at necessary time points~\cite{sevin2021optimizing}.

%  The ICU can change quickly, even from hour to hour, so the family members need continual, progressive information~\cite{davidson2010facilitated}.

\subsection{Technologies Supporting Family Caregivers in Clinical Settings}

%Technology Solutions for Non-expert in Clinical Information
%EMR system/ EHR data 
%refine structured EHR data

In recent years, the fields of Computer-Supported Cooperative Work (CSCW) and Human-Computer Interaction (HCI) have emphasized the critical role caregivers play in patient care within clinical settings~\cite{Don2024, Siddiqui2023, Yunan2013} and emphasized the challenges caregivers face associated with clinical information~\cite{Stefanidi2023}. In response, prior research has explored various technological solutions to address these challenges~\cite{Pine2018}.

On the one hand, researchers have explored various technology designs to enhance clinical information accessibility, including facilitating the sharing of information between clinicians and caregivers~\cite{ziqi2024, Yuexing2024, Bowers2024} and assisting caregivers to seek clinical information of older adult patients~\cite{montagna2023data, wei2024leveraging}.
For example, ~\citet{Bowers2024} developed a mobile application that offers template support for caregivers to document and assess behavioral changes in patients with congenital heart disease, facilitating effective sharing of clinical information with clinicians. 
SaludConectaMX is a mobile application designed for caregivers to access clinical information, including patient medical history and oncology treatments, to assist in tracking patients' evolving health trajectories~\cite{Schnur2024}.
~\citet{Barbarossa2023} developed a dashboard to present the most relevant clinical information of older adult patients with dementia, including fall events, sleep data, and wellbeing data, for caregiver access to patient status. 
While effective, most of these tools are intended for caregivers of patients with chronic conditions, such as dementia.
Consequently, their accessibility is not suitable for the time-sensitive and high-stakes clinical environment of the ICU.

Another area of research explores using visualization tools to decipher complex clinical information to support caregivers' understanding of clinical information~\cite{kong2017comparative, Hong2017, Nyapathy2019, liu2011, Shea2019, Hwang2014}.
For instance,~\citet{kong2017comparative} introduced EnGaze, a web-based tool designed to visualize the communication behaviors of children with autism during clinical visits, aiming at improving caregivers' understanding of the condition and promoting their active involvement in care planning.
An interactive system designed by~\citet{Hong2017} assists families and clinicians in reviewing radiology imaging results by offering simplified definitions and diagrams of medical term during consultations.
~\citet{Nyapathy2019} presented a visualization mobile application designed to assist caregivers in recording and visualizing the long-term condition of asthma patients, thereby enhancing the shared understanding of the condition with clinicians.
While these technologies provide valuable assistance for caregivers in routine clinical settings, the complexity and density of information produced in the ICU can significantly overwhelm caregivers compared to other contexts.
The proposed visualization technology solutions are insufficient to support caregivers in understanding the clinical information of older adult patients in the ICU. 

In conclusion, existing tools and technologies remain inadequate in supporting caregivers of older adult ICU patients in accessing and understanding information.
Existing solutions are fragmented in scope, indicating a significant need to leverage novel technologies to systematically address caregivers' difficulties in accessing and understanding clinical information in critical care settings.



%In recent years, the fields of Computer-Supported Cooperative Work (CSCW) and Human-Computer Interaction (HCI) have advocated for caregivers' critical role in clinical settings, information needs in clinical settings~\cite{Pine2018}. 

%Prior work identified a significant barrier that impedes effective information sharing for family caregivers is the absence of a centralized data storage framework~\cite{adams2021perspectives, amir2015care}.
% Despite the critical need for timely and accessible health information for family caregivers in ICU settings, the absence of a centralized, user-friendly data storage system creates significant barriers~\cite{adams2021perspectives, amir2015care}.
%Today's Electronic Health Records (EHR) systems, which are widely adopted by hospitals to support clinical information organization and storage, still suffer from critical limitations~\cite{Sepehri2023}.
% A primary barrier lies within today’s Electronic Health Records (EHR) systems, which, despite being widespread, suffer from critical limitations.
% While these systems are widely adopted, their potential is constrained by critical limitations. 
% Chief among these is the lack of standardized frameworks, which severely impedes efficient information sharing~\cite{Sepehri2023}. 
%For instance, data in EHR often remain fragmented, making it difficult to be clearly organized and not possible to be understandable by non-experts ~\cite{o2010electronic}.
% and pose interpretability challenges, especially for non-specialists~\cite{o2010electronic}. 
%Furthermore, EHR was primarily designed to document the data generated or recorded from the patients, but not necessarily capture the interactions between providers and the caregivers. 
% Furthermore, EHR documentation often lacks critical care milestones, capturing what was recorded rather than what was actually delivered or experienced by patients and their families.
%For instance, clinicians may engage in important goals-of-care discussions with patients but fail to adequately document these interactions, which leads to difficulties when caregivers want to retrieve such information at a later time~\cite{moody2004electronic, mack2012end}.



% for understanding the comprehensive approaches that address the full range of information management and emotional support challenges.


\subsection{AI-based Clinical Information Technologies}
 
Recent advancements in artificial intelligence (AI) have shown significant potential to improve caregivers' access to and understanding of clinical information. 

In order to improve caregivers' access to complex clinical information, researchers have developed AI models to extract complex clinical information from electronic health records (EHRs) through summarization~\cite{Sette2023, Nikkhah2021, gopinath2020fast}, identify high-risk factors in medical records~\cite{corey2018development}, and predict risks from clinical data, such as sepsis risk~\cite{suresh2017clinical, yin2024sepsiscalc}. 
AI can also enhance the understanding of clinical information.
For instance, some models can translate complicated medical jargon into accessible language for the general audience to understand ~\cite{yim2024preliminary, wachter2024will, wong2018using}, and simplify lengthy medical texts into shorter versions~\cite{basu2023med, artsi2024large}. In addition, chatbot systems based on large language models (LLMs) show potential in supporting caregivers' personalized information-seeking needs~\cite{afshar2024prompt}. 
Such solutions have proven effective in delivering health advice regarding screening, diagnosis, treatment, and disease prevention~\cite{huo2025large}. 
For example, \citet{ramjee2024cataractbot} developed an LLM-based chatbot named CataractBot that addresses patient inquiries regarding cataract surgery.

Overall, AI have demonstrated the potential to address the aforementioned challenges faced by caregivers of older adult patients in the ICU. 
However, there has been a lack of systematic investigation from the caregivers' perspective regarding their needs in accessing and understanding clinical information.
Moreover, how to design effective AI-based technologies to address caregivers' needs in accessing and understanding information remains underexplored.

% caregivers reflects a broader oversight in human-centered AI design: systems often fail to account for indirect end-users who lack formal medical training but bear responsibility for care decisions. 
%For example, AI tools designed to improve clinician-patient communication~\cite{ziqi2024} rarely consider how caregivers might misinterpret or operationalize AI-generated insights. 

% A growing body of work has also explored patient-centered AI technologies, such as symptom-tracking chatbots~\cite{Eunkyung2024}, personalized treatment recommendation systems, and tools to improve patient-provider communication~\cite{ziqi2024}.


% This disconnect underscores an urgent need to reconceptualize AI systems as caregiver-facing tools, capable of distilling complex data into contextually relevant, emotionally sensitive, and ethically transparent guidance tailored to caregivers’ literacy levels, cultural contexts, and decision-making autonomy.

% While these efforts emphasize clinical stakeholders and patients, they largely overlook the critical role of family caregivers, particularly in the ICU. 
% In ICU contexts, caregivers serve as intermediaries between clinicians and patients, advocating for care preferences, interpreting complex medical information, and making time-sensitive decisions under significant emotional duress. 
% Despite the critical role, their experiences and needs remain understudied in AI-based healthcare research.
% Few research address the sociotechnical challenges caregivers face, such as reconciling conflicting clinician advice, navigating fragmented health records, or translating jargon-heavy prognoses into actionable choices. 
% This gap is particularly problematic in ICUs, where caregivers  high-stakes information (e.g., prognostic uncertainty, treatment trade-offs) while coordinating with care provider teams.




% Researchers have successfully developed AI-based clinical pre-screening systems (e.g., for diagnostic triage)~\cite{wu2024clinical}, risk prediction algorithms (e.g., for sepsis risk prediction)~\cite{suresh2017clinical, yin2024sepsiscalc}, and automated information processing (e.g., EHR summarization)~\cite{Sette2023, Nikkhah2021, gopinath2020fast}. These innovations have primarily targeted healthcare providers to optimize clinical workflow efficiency and reduce clinicians' cognitive burdens in high-pressure environments~\cite{Yuexing2024}.
% Existing AI solutions in critical care settings prioritize clinical workflow optimization (e.g., predictive analytics for resource allocation)~\cite{Yuexing2024} and direct patient monitoring (e.g., AI-based vital sign analysis)~\cite{Zhang2024}.

\begin{figure}
    \centering
    \includegraphics[width=1\linewidth]{PICS/studyprocess.png}
    \caption{Study Procedure}
    \label{fig: study design}
\end{figure}
\section{Formative Study: Method}\label{study1}
As discussed in Section~\ref{relatedwork}, there is a lack of understanding regarding whether and how AI-based technologies should be designed to meet caregivers' information needs while their older adult patients are in the ICU. To address this gap and identify key design requirements, we conducted a two-stage study, as illustrated in Fig~\ref{fig: study design}.
At the first stage, we carried out a formative study with 15 caregivers using semi-structured interviews~\cite{longhurst2003semi}.
The study aimed to gather insights on (1) the challenges caregivers encountered in accessing and understanding clinical information in the ICU and (2) their user experience with existing technologies.
%As discussed above, there is a lack of understanding of the challenges caregivers face in accessing and understanding clinical information for older adult patients in ICU settings.
%We conducted a formative study that aims to explore the challenges faced by family caregivers of older adult patients in critical care settings, particularly with respect to accessing and understanding clinical information, their coping strategies, and limitations of existing supportive technologies.

%We recruited 11 caregivers who had prior caregiving experience with older adult patients in the ICU and conducted semi-structured interviews with the participants to gain deeper insights into their interactions with information needs and technology uses in this high-stakes and high-risk context.
% We conducted semi-structured interviews with caregivers who had prior caregiving experience with older adults in the ICU, which allowed us to prompt participants to share their unique caregiving experience and adapt follow-up questions to gain deeper insights into their interactions with information needs and technology uses in this high-stakes and high-risk context.


\subsection{Participant Recruitment}
To recruit and screen qualified participants, we used convenience sampling~\cite{etikan2016comparison} and designed a survey to collect demographic information and caregiving background details. The survey was distributed via social media platforms, including Twitter, Instagram, and Facebook. Exclusion criteria included individuals who were under 18, not U.S. residents, non-English speakers, or not close family members of older adult patients.
In the end, we recruited 15 participants with experience as family caregivers involved in the hospitalization of older adult ICU patients. 
Details regarding participant demographics and caregiving backgrounds are presented in Table~\ref{tab:participant_demographics}.
This study received approval from the Institutional Review Board (IRB) at the first author’s institution.

\subsection{Interview Procedure}

The interviews were conducted remotely via Zoom and lasted approximately 40–60 minutes.
At the beginning of the interview, we inquired about participants' verbal consent for audio recording and transcription.
During the interviews, we asked participants to recall their experiences as family caregivers when their older family member was admitted to the ICU. We also asked them not to share any identifiable personal information.
The interviews focused on the clinical information caregivers sought to access and understand, as well as the challenges they faced.
We also asked about any technologies they used for accessing and understanding clinical information.
Building on the experiences participants shared, we also asked participants to discuss their usage of any technologies that helped them access and understand clinical information. 
At the end of the interview, we thanked the participants for their time, and each participant received a \$20 Amazon digital gift card as compensation.
The detailed semi-structured interview protocol can be found in Appendix~\ref{Study1script}.
 
After each interview, the audio was transcribed.
Two researchers then independently coded the transcripts using an open coding approach~\cite{corbin2014basics}. They then discussed and reconciled their initial coding schemas, iteratively refining them for clarity and consistency.
The finalized codebook (Appendix ~\ref{Appendix:codebookofformative}) was applied across all transcripts by both researchers, with any discrepancies resolved through team discussions until a consensus was reached.




\begin{table}[h]
\caption{Demographics of Participants and Their ICU Experiences}
\label{tab:participant_demographics}
\begin{tabular}{c|c|c|c|c}
\hline
\textbf{P\#} & \textbf{Gender} & \textbf{Age} & \textbf{Relationship to Patient} & \textbf{Patient Age} \\ \hline
P1  & M  & 26–35 & Father        & 65+  \\ \hline
P2  & M  & 26–35  & Mother         & 65+  \\ \hline
P3  & M  & 26–35  & Grandmother    & 65+  \\ \hline
P4  & F  & 18–25 & Relative       & 80+  \\ \hline
P5  & F  & 26–35 & Mother        & 65+  \\ \hline
P6  & M  & 26–35  & Father         & 67   \\ \hline
P7  & F  & 18–25 & Step-Grandmother & 73  \\ \hline
P8  & M  & 26–35 & Father         & 65   \\ \hline
P9  & M  & 36–45  & Uncle          & 75   \\ \hline
P10 & M  & 18–25  & Father         & 65   \\ \hline
P11 & F  & 26–35 & Uncle          & 65   \\ \hline
P12 & M  & 26–35  & Grandmother    & 65   \\ \hline
P13 & M  & 18-25  & Grandmother    & 70   \\ \hline
P14 & M  & 26–35  & Grandmother    & 68   \\ \hline
P15 & M  & 18-25 & Grandfather    & 72   \\ \hline
\end{tabular}
\end{table}




\section{Formative Study: Results}
In the following section, we detail the challenges caregivers face in accessing and understanding clinical information within the ICU setting.

\subsection{Caregivers' Challenges in Accessing Clinical Information}

\subsubsection{Staying informed about patients' changing status}\label{c1}
While older adult patients are admitted to the ICU, caregivers prioritize staying informed about their daily status. Some even juggle work responsibilities while making trips to the hospital for updates, as P4 noted~\textit{`` it's hard to get to know who's on the shift that day, and sometimes you have to go there physically [to know his situation] ''}.
Beyond the current status, caregivers are also concerned with tracking changes in the patient’s condition from the previous day. Since older adult ICU patients' health can fluctuate drastically, family caregivers seek consistent clinical updates. 
%As P1 explained:~\textit{``I always ask, how is the condition [compared to his previous condition]? Is it (the condition) getting any better?''}

Many interviewees expressed frustration about their limited access to patients' information, which relied solely on verbal updates from the ICU healthcare team. 
This lack of direct access often left caregivers feeling~\textbf{\textit{``left in the dark''}}:
\begin{quote}
\textit{``There were times when I had to seek out doctors or nurses for updates,... I sometimes felt that my patients were seen as an interruption to the busy ICU schedule... In cases where I didn't get information, I didn't get to know what was going on,... I just felt I was ~\textbf{left in the dark}''}(P1).
\end{quote}




% Some participants expressed a strong desire for consistent and timely updates from the healthcare team regarding the patient’s condition. 

% Most caregivers have little to no prior experience in an ICU environment, making it challenging to grasp the patient’s health condition in real time. 
% P3, whose grandmother suddenly collapsed and was admitted to the ICU, described the shock and uncertainty of the experience:~\textit{``I would say the most challenging part is not knowing... what's really wrong with her. Like, before the doctor actually came out to tell me [her condition], okay, some of the seizure and cardiac arrest [happened to her] ..., I was really perplexed... I'm really anxious about what could be happening''}. 



\subsubsection{Struggling to align treatment plans and goals with healthcare team}\label{c2}
An extreme yet common issue reported by caregivers is their lack of access to clinical information about the treatments the patients were receiving in the ICU. 
This gap in information access leaves many caregivers feeling excluded from critical decisions about their loved one’s care, creating a sense of helplessness and frustration. 
While most participants expressed trust in the clinicians' expertise and judgment, some had experienced situations where they were only informed after treatments had been administered, leaving no room for questions or understanding. 
P11 articulated this frustration: ~\textit{``I was kind of curious to know the name of the drugs that [were] taken at the moment, but then I think I wasn't given the opportunity to know''}. This lack of access to treatment procedures and plans, even for minor details, can undermine caregivers' confidence in patients' recovery and intensify their anxiety. In life-threatening situations, the lack of clear information on ongoing treatment plans can heighten caregivers' concerns, especially when their loved ones undergo critical treatments (e.g., being connected to life-sustaining machines).

\begin{quote}
    \textit{``[Clinicians did] all the tests, the procedure - [but] what they tested and what they saw, what they got, that kind of deep parts - [I don't know]. I wanted to know the whole [situation] about it, but I think they didn't open it all up, and they were just telling us the diagnosis.''}(P7).
\end{quote}
% \begin{quote}
% \textit{``Because I felt when I see him connected with so many machines, I was like, what's happening? So at some point, I want to know, why are they connecting? Why did they [put] such machines on him? I didn't get to get that information from them (clinicians). So basically they just told me, the vital signs are okay, he's responding to treatment...''} (P1)
% \end{quote}

Another challenge in information access is the treatment goals set by the healthcare team. Caregivers sometimes receive only a general overview of treatment options but lack insight into major care interventions. As a result, they have to piece together scattered information to access the overall treatment goal.
\begin{quote}
~\textit{``so they told me that,... We should probably be more than a week at the ICU and just informing me about... like they might have to maybe perform another surgery if he... doesn't get responsive, and just telling me other treatment options, like ultrafiltration that the end add transplants that might be considered if it isn't responsive.''}(P6)
\end{quote}





%~\textit{``so they told me that, ...it might be a short stay depends on [his condition, or] a longer stay, but considering that is another severe heart failure,[so] there's a possibility that it would be a longer stay. We should probably be more than a week at the ICU and just informing me about... like they might have to maybe perform another surgery if he... doesn't get responsive, and just telling me other treatment options, like ultrafiltration that the end add transplants that might be considered if it isn't responsive''}(P6).

%These moments of uncertainty highlight a critical disconnect between caregivers and the healthcare team regarding key treatment goals, as caregivers' need for clarity is often overshadowed by the clinical priorities of medical staff. This lack of information leaves caregivers sidelined and unable to fully align the ongoing treatment with the healthcare team.

\subsection{Caregivers' Challenges in Understanding Clinical Information}
\subsubsection{Interpreting medical terms in clinical report and conversations}\label{c3}

The concept of being ``lost in translation'' is often discussed in relation to the medical term caregivers encounter. From our interviews, we identified two primary sources contributing to this challenge, which frequently leads to information overload: verbal communication with clinicians and the interpretation of clinical reports.

While conversations with healthcare providers often involve complex and unfamiliar medical term, caregivers typically try to seek clarification from doctors. As P4 remarked,~\textit{``some doctors tend to use ambiguous terms''}. In many cases, caregivers feel compelled to ask clinicians for further clarification to make sense of the information shared with them. P5 described her experience:~\textit{``We couldn't understand some medical terms. So we asked him (the doctor) to use a clear word where he explained everything that is happening around...''}.
However, some participants did not have the opportunity to seek clarification from clinicians, making it even more challenging to grasp critical information.
% \begin{quote}
% \textit{``because most of the information they give they speak in their big medical terms that I find it very difficult to grasp... I didn't get hold of the information they were trying to pass to me and there were not enough time for me to ask questions for that clarification.''}(P1) 
% \end{quote}


A comparable yet even greater challenge lies in medical reports, where technical content often leaves caregivers confused, prompting them to seek external resources for clarification. Lab test results, in particular, pose a significant challenge, as they are one of the most frequently provided pieces of medical information to caregivers. However, their complex nature makes them difficult to interpret without proper context or guidance.
\begin{quote}
\textit{``when it comes to having to pick reports,...the report is not actually something I could compliment, because there are times there that I do have to... use Google to actually go through and know what it actually means.''}(P8) 
\end{quote}

Caregivers often turn to search engines for quick understanding of medical information, such as definitions of terms or procedural details.
However, without contextual knowledge of the patient’s status, the relevance and quality of the information received varied significantly, often leaving caregivers frustrated and confused.

\begin{quote}
    \textit{ ``When I search online, sometimes the information shows] conflict... It really [just] shows minor information... the equipment used, visiting hours, the policies, it's just like minor information.''}(P2)
\end{quote}


% Caregivers frequently struggle to navigate the medical terms they encounter when trying to understand lab test results or their loved one’s condition. The heavy use of technical jargon compounds their confusion, leaving them feeling overwhelmed and disconnected. 




\subsubsection{Asking meaningful questions with healthcare team}\label{c4}

Caregivers also struggle to engage in meaningful conversations with the ICU healthcare team when asking questions. These interactions can often feel unproductive, as caregivers may repeatedly ask broad or unclear questions and receive vague or insufficient responses from healthcare providers in return.
For instance, P1 shared the experience as feeling like ``running in circles'':~\textit{"At some point,...When I asked, [they just said] it is okay, we are doing our work, and his vital signs are okay, and they just tell me I should relax, not to bother myself about it..."}. 
In some cases, caregivers even refrain from asking questions due to negative past experiences, and such interactions discourage open conversation and can prevent caregivers from understanding clinical information about the patient’s care.

% \begin{quote}
% \textit{``Because there was this time I did ask the question, and I was told, even if I did explain to you, you wouldn't understand. Just believe that we are doing this for the good of your relative. You see, so for someone who's afraid to know what they should know, [their choice] is not going to ask questions, which is very wrong..."}(P4) 
% \end{quote}

To navigate these challenges, some caregivers adopt proactive strategies, such as preparing their questions in advance. This approach helps them feel more confident and ensures they make the most of their limited opportunities to have direct conversations with healthcare providers. For example, as P7 illustrated: \textit{``[I'll prepare] my question, depending on what I see the day I visit her... I'll write what I would like to ask the doctor. So if I go the next day, I can just walk straight to the doctor's office and ask the question."}


% \begin{quote}
% \textit{``[I'll prepare] my question, depending on what I see the day I visit her... let's say if I go today, and then I notice a change when I get home, I'll prepare the question. I'll write what I would like to ask the doctor. So if I go the next day, I can just walk straight to the doctor's office and ask the question."}(P7)
% \end{quote}


%P7 described:~\textit{"[I'll prepare] my question so depend on what I see the day I visit her... let's say if I go like today, and then I notice a change when I get home, I'll prepare the question. I'll write what I would like to ask the doctor. So if I go the next day, I can just walk straight to the doctor's office and ask the question. The question said... depends on how mode the changes in our body, in our system and our behavior"}. 
%Also, they prepare questions in a dedicated manner that is closely connected with the patient's condition, as P7: ~\textit{``my question so depends on what I see the day I visit her. So if I, let's say if I go like today, and then I notice a change when I get home, I'll prepare the question. I'll write what I would like to ask the doctor. So if I go the next day, I can just walk straight to the doctor's office and ask the question. The question said, I like accent depends on how mode the changes in our body, in our system and our behavior''}

% \paragraph{\textbf{Existing Digital Tools Offer Support, but Remain Incomplete}}

% In our interviews, caregivers identified three primary digital tools they relied on during the period when their older adult patients were in the ICU: search engines (e.g., Google), conversational agents (e.g., ChatGPT), and patient portals, which are commonly provided apps for accessing medical information such as lab results, appointment summaries, and doctor’s notes.
% While these tools offered varying levels of support, caregivers noted significant limitations that prevent them from fully addressing their needs.

% Caregivers often turned to search engines and conversational agents for quick access to information, such as definitions of medical terms or procedural details. However, the quality and relevance of the information obtained varied widely, leaving many feeling dissatisfied. For instance, P2 described encountering conflicting and overly simplified information when using search engines:
% ~\textit{ `` when I search online, sometimes the the information shows] conflict, yes, the conflict...it's good possibility. It's really show just minor information... the visiting hours, everything, the equipment used, visiting hours, the policies, like, it's just like minor information''}.
%Similarly, P8 noted that they used search engines and chatbots only for very specific tasks, such as looking up unfamiliar terms. They avoided using these tools for more complex questions, such as understanding complications or test results, citing concerns about exaggerated or inaccurate information:~\textit{I will say I won't put it as searching online. I use both search engines [and chatbot] for maybe when I need to know a particular term, that's just it...I don't like using it for telling me the complications, the possible results from a particular test. I don't like using it because I feel like is actually... well exaggerated''}.

% Patient portals emerged as another widely used tool, offering caregivers direct access to clinical information such as lab results, medical summaries, and doctors’ notes. P6 highlighted the utility of patient portals in tracking and reviewing medical updates:~\textit{``[I use the patient portal to access] like his lab results, all the results get from the testing that we've done and the summaries that the doctors have provided the doctor's notes and all that from previous doctor's appointment''}.
% However, the utility of these portals often depended on the caregiver’s ability to interpret the information. For P8, the portal was only helpful for making basic assessments, such as identifying whether a symptom was mild or serious:~\textit{``Then I didn't use it for all...[unless] maybe when he feels somehow, and I would perform a particular test, and I actually use that information to actually see if it is something alarming or maybe just a mild thing, maybe due to the fact that he sleeps on one side, maybe that's the cause''}.

%Despite the usefulness of patient portals, many caregivers ultimately turned to doctors for clarification when they struggled to understand the information provided, as P8 explained~\textit{``... any time that I felt like I was not able to understand [the lab test result], I do meet the doctor,...Then I asked him to actually help me out because I don't actually understand what's going on, then he helps me in understanding it''}. 




%\subsubsection{Technology Needs and Design Space}~\label{design space}

\subsection{Design Guidelines: Translating Caregiver Needs into Action}~\label{design space}
Based on the challenges shared by caregivers, several key areas emerge where technology can be improved to better support their needs in accessing and understanding clinical information. From these insights, we derived four design guidelines to shape our prototype: structuring daily patient updates with a timeline, providing accessible clinical information across three key aspects of care, enhancing medical term understanding with context-aware support, and supporting caregivers in structured and informed clinical discussions.
%Based on the challenges and strategies shared by caregivers, we derived three design guidelines to shape our prototype:

\begin{itemize}
\item \textbf{DG1: Structuring daily patient updates with a timeline:} 
As our formative study results suggest, caregivers need more than just daily patient updates—this aligns with findings from previous research~\cite{mcgonigal2020providing}. They also require comparisons to the previous day (Section~\ref{c1}) to reduce reliance on fragmented verbal updates for accessing clinical information. To support this need, a structured timeline of the patient’s daily updates is essential for helping caregivers better understand the patient’s status throughout their ICU stay.

\item \textbf{DG2: Providing accessible clinical information across three key aspects of care:}
Our interview study also highlighted caregivers' focus on three key aspects of clinical information: treatment plans, treatment goals (Section~\ref{c2}), and lab test results (Section~\ref{c3}). Additionally, this information should be presented in a unified and easy-to-understand manner.

\item \textbf{DG3: Enhancing medical term understanding with context-aware support:} 
Moreover, medical term encountered in different situations can confuse caregivers, whether in conversations with healthcare providers or in test reports, regardless of their ability to seek clarification from the professional healthcare team (Section~\ref{c3}). Additionally, searching for information online without the context of the patient’s condition can lead to even greater confusion. Thus, a context-aware chatbot that serves as an always-available resource can facilitate caregivers' personalized information inquiries and improve their understanding.

\item \textbf{DG4: Supporting caregivers in structured and informed clinical discussions:}
Caregivers' proactive engagement emerged as a key theme in our findings. Despite limited opportunities, they actively seek conversations with the healthcare team (Section~\ref{c1}). One effective practice is preparing a meaningful list of questions in advance. Thus, a feature designed to suggest context-relevant questions can help caregivers better understand the patient’s condition and provide a structured framework for discussions (Section~\ref{c4}).
\end{itemize}

% From the challenges and strategies shared by caregivers, several key areas emerge where technology can be enhanced to better support their needs. Three key areas emerge as critical for improving the caregiving experience:~\textbf{centralizing patients’ status updates}, ~\textbf{simplifying medical term with tailored granularity}, and~\textbf{fostering more meaningful communication with ICU teams}.

% \paragraph{\textbf{Centralizing Sequential Patients' Status}}

% A recurring issue highlighted by caregivers is their limited access to updates about the patient’s changing status. Current methods, such as verbal updates from clinicians or scattered details across patient portals, often leave caregivers feeling anxious and uninformed. Tools that provide a centralized and continuous record of patient status changes could empower caregivers with timely and consistent information. One potential solution is a personalized patient status dashboard, accessible via mobile or web applications, that consolidates key updates including lab test result, treatment progress and goal, and upcoming procedures. 

% To further alleviate caregivers’ anxiety, such a system could include predictive explanations (e.g., potential treatment trajectories for future ICU stays) to help caregivers anticipate and prepare for the next steps. As P6 noted:~\textit{[they just told me] it might be a short stay depends on, or a longer stay, but considering that is a rather severe heart failure and just informing me about like that surgery, like they might have to, maybe perform another surgery if he doesn't get responsive, and just telling me other treatment options, like ultrafiltration... that might be considered if he isn't responsive..''}. Centralizing these details would allow caregivers to stay informed remotely, reducing the need for physical presence at the hospital and minimizing reliance on fragmented verbal updates from busy healthcare staff.

% \paragraph{\textbf{Building Explainable Medical Terms with Dedicated Granularity}}

% Medical term presents a significant barrier for caregivers, often leading to information overload and confusion, particularly when interpreting lab results or medical reports. To address this, there is a clear need for tools that provide explainable, context-aware medical information tailored to caregivers’ varying levels of expertise. The pervasive use of medical jargon in both verbal communication and written documents exacerbates this challenge. AI-powered conversational agents, integrated with domain-specific, fact-based knowledge and access to patient's electronic health records (EHR), could play a transformative role. These agents could provide real-time, tailored explanations of complex medical terms and offer follow-up clarifications, sparing caregivers the effort of cross-validating information with clinicians or relying on external search engines.

% By delivering accurate, explainable information aligned with the caregiver’s level of understanding, such tools would empower them to comprehend their loved one’s medical status more effectively, reducing the feeling of being "lost in translation." This would not only enhance caregivers' confidence but also alleviate their dependence on clinicians for routine clarifications, allowing healthcare professionals to focus on critical tasks.

% \paragraph{\textbf{Supporting Meaningful Clinical Conversation}}

% Caregivers often struggle to engage in meaningful conversations with ICU teams, leading to frustration and a sense of exclusion. Technology could bridge this gap by facilitating better-prepared and more structured interactions.

% An AI-powered conversational agent, integrated with patient portals or mobile apps, could assist caregivers by providing context-aware responses and clarifying procedures, test results, or treatments. The tool could suggest relevant questions based on changes in the patient’s condition or new updates, helping caregivers navigate conversations more confidently. This aligns with caregivers' existing practices of noting down questions in preparation for discussions, as highlighted in Section~\ref{c: conversation}. If the agent cannot provide satisfactory answers, it could prompt caregivers to reframe their inquiries for clearer communication or escalate the question to clinicians. This approach empowers caregivers to engage in productive discussions, reduces repetitive questions, and enhances their confidence in navigating the complexities of ICU care.




\section{\projectname: an AI-Based Prototype for Supporting Caregivers of Older Adult ICU Patients }
% \subsection{System Design}
% \subsection{User Interface}
% \subsection{Synthesizing into Design Prototype}
Building on the design space in Section~\ref{design space}, we developed an AI-based prototype, \projectname, to help caregivers access and understand clinical information through two interconnected components: a visual timeline of patients' medical events and an LLM-based chatbot to facilitate personalized information inquiries through conversations. 
\begin{figure}
    \centering
    \includegraphics[width=1\linewidth]{PICS/Frame52.png}
    \caption{\projectname Prototype Designs: (A) A visual timeline that maps a patient's medical events, featuring AI-generated summaries of lab test results, key treatment plans, and daily goals to provide a clear overview of the patient’s status. (B) An example of a context-aware LLM-based chatbot that provides insights into medical reports, helping users interpret test results and identify trends.}
    \label{fig:Overview}
\end{figure}



\subsection{Visual Timeline of Patients' Medical Events}
To help caregivers better \textbf{\textit{access}} patients' conditions, this interface transforms fragmented clinical data into a structured report, organized in a timeline format (as shown in Fig.~\ref{fig:Overview} (A)). 
The top section features a sequential timeline highlighting key treatment milestones (e.g., ventilator placement). By selecting a specific day, caregivers can track and compare clinical reports to monitor the patient's ICU progress (\textbf{DG1}). Below, a daily overview organizes information into three key modules that caregivers primarily focus on regarding the patient (\textbf{DG2}), including: 

~\textbf{The AI-generated summary} module takes structured lab test results as input and generates plain-language explanations to help caregivers make sense of medical values and trends. The underlying LLM-based summarization model continuously updates the summary when new lab results are available, helping caregivers stay informed about the patient’s physiological changes.

~\textbf{The Treatment for the Day} and ~\textbf{Goals for the Day modules} take unstructured clinical notes as input and extract relevant content using an information extraction module. Specifically, the Treatment module identifies key interventions and medications and generates brief explanations of their purpose. The Goals module highlights actionable priorities in the patient’s care plan, outlining next steps that caregivers can track. Together, these modules help caregivers understand not only what is being done but also why, enabling them to stay aligned with ongoing care decisions.

\subsection{Context-Aware LLM-Based Chatbot}

To help caregivers better \textbf{\textit{understand}} clinical information, we designed an LLM-based chatbot that leverages EHR data as part of its information source (as shown in Fig.~\ref{fig:chatbot}). 
Instead of providing generic medical explanations, it tailors responses based on the patient’s specific clinical data. This chatbot enables personalized inquiries through flexible text-based conversations and supports caregivers in three key ways while also suggesting relevant follow-up questions (\textbf{DG3}). The chatbot operates through three core interaction modes:

\textbf{Fact Finder} (Fig.\ref{fig:chatbot} (A)) responds to direct caregiver questions by retrieving relevant clinical context and offering lay explanations—for example, clarifying the need for a ventilator as ``supporting oxygen exchange during lung recovery.'' To encourage proactive advocacy, it also suggests follow-up questions such as ``What are the signs indicating ventilator removal?'', helping caregivers navigate unfamiliar clinical territory.

\textbf{Data Interpreter} (Fig. \ref{fig:chatbot} (B)) takes lab test results and other structured clinical data as input and translates them into caregiver-friendly responses and generates caregiver-friendly explanations that clarify both the meaning of individual values and their implications for the patient’s condition.

\textbf{Intention Prompter} chatbot (Fig.~\ref{fig:chatbot} (C)) is designed to support caregivers when the chatbot lacks access to specific clinical information or when caregiver questions are too vague to generate a precise response. In these situations, the chatbot offers proactive guidance by helping caregivers prepare questions for upcoming discussions with clinicians (\textbf{DG4}). It presents categorized follow-up prompts across three thematic areas: ``Current Treatment,'' which encourages caregivers to ask about the patient's ongoing treatments and medical status, ``Prepare for the Future,'' which helps them anticipate potential care decisions and long-term planning, and ``Learn What You Can Do,'' which guides them in inquiring about actionable steps they can take to provide better support. 
\begin{figure}
    \centering
    \includegraphics[width=1\linewidth]{PICS/Frame53.png}
    \caption{Context-Aware LLM-Based Chatbot Prototype Design: (A) Fact Finder – Responds to user queries about ICU treatments and suggests relevant follow-up questions. (B) Data Interpreter – Explains lab test results and clinical data in an accessible manner. (C) Intention Prompter – Assists users in formulating meaningful questions for healthcare providers.}
    \label{fig:chatbot}
\end{figure}



\section{Prototype Evaluation}
% Our prototype design aims to address the challenges caregivers face in accessing and understanding clinical information in high-stakes, time-sensitive ICU settings.
% To achieve this, we designed a prototype featuring a visual timeline of patients’ medical events to improve access to clinical information. 
% Additionally, we integrated a context-aware, LLM-based chatbot to support caregivers' personal inquiries and enhance their overall understanding of clinical information.
\subsection{Participants and Recruitment}

For the second stage of our study, we recruited the same ten caregivers (P6-P15) from Section~\ref{study1} to conduct a usability evaluation of our prototype.
Each prototype evaluation session was conducted remotely via Zoom and lasted approximately 30–45 minutes.
To comply with usage regulations requiring approval for accessing MIMIC-III~\cite{johnson2016mimic} data, we integrated de-identified, synthesized patient data into our prototype. In this paper, we present only mock data to illustrate the user interface.

All sessions were recorded and transcribed with participant consent. For each design prototype, we allocated approximately 10–15 minutes for participants to explore its features. After a brief introduction to the interface, participants were encouraged to interact with the system, share their impressions, and evaluate its perceived usefulness in accessing and understanding a patient’s status. They also reflected on how they might have used the tool to obtain relevant medical information, identified the most helpful aspects, and suggested potential improvements to enhance overall usability. The detailed semi-structured interview protocol is provided in Appendix~\ref{Study2script}. Following the interviews, two researchers independently analyzed the transcripts using an inductive approach~\cite{thomas2006general}, identifying key insights related to usability and potential design improvements for each prototype, the final codebook is provided in  Appendix~\ref{Appendix:codebookofforevaluation}.

\subsection{Result}
In the following section, we present the results of the usability evaluation, which captures valuable feedback from family caregivers on our prototype.
Overall, participants greatly appreciated the prototype’s ability to enhance their access to clinical information by providing consistent, timely, and easy-to-understand updates.
Meanwhile, they found the prototype’s capability to integrate with personalized patient EHRs particularly helpful for them to understand clinical information, as it could effectively respond to their specific inquiries, uncover additional relevant details, and offer deeper insights into the patient’s condition.

\subsubsection{Visual Timeline of ICU Stay: Supporting Caregivers’ Access to Consistent Information}

\paragraph{Synthesizing Fragmented Information into a Cohesive Timeline}
Overall, our participants found the visual timeline view to be a valuable tool for better supporting their access to dynamic patient status changes during ICU stays.
Building on findings from the formative study (See Section~\ref{c1}), which highlighted caregivers' difficulties with inconsistent information structuring and accessibility in ICU workflows, the timeline interface is designed to address this challenge. Caregivers often rely on fragmented communication with healthcare providers, making it difficult to piece together a comprehensive understanding of the patient’s condition and treatment process.
A unified interface that uses a timeline to visualize patient progress in the ICU over time could address this gap by enabling them to track trends and identify critical events of the patient's ICU stay.
%For example, P6 highlighted the timeline’s importance in accessing the patient’s progress:~\textit{``[The medical timeline] should be helpful to monitor the progress, and if [the patient's condition is] deteriorating or [if] the treatment is being progressive''}.
% Additionally, P12 emphasized its role in ensuring continuity of care:~\textit{`` It should help me in doing follow-ups of the treatments he has been receiving''}.
The timeline was useful not only for individual caregivers but also as a shared communication tool for the patient’s family.
Participants noted that it helped bridge information gaps between family caregivers and other family members, making it easier to share updates on the patient’s condition.

\begin{quote}
~\textit{``It's like every time I want to check out the progress or every time I want to send some results to my siblings, I would try to use the tab [on the switch of different days] and just saying, this is this progress chart [shows the patient's condition.''}(P6)
\end{quote}

Additionally, the timeline helped reduce information overload — a common challenge in ICU environments — by organizing complex clinical data into a clear, time-based summary. This design minimized caregivers' cognitive effort, allowing them to focus on understanding the patient's overall status rather than manually piecing together fragmented updates. The interface supported on-demand access to critical information, enabling caregivers to check in as needed rather than constantly monitoring data:

\begin{quote}
\textit{``[I] might not be daily [to check on that information], but anytime you want to really know about what is really wrong, to have more understanding [on the patient's status], you just look at it.''}(P7)
\end{quote}

\paragraph{Balancing Simplicity and Depth in AI-Generated Summaries} 
%From Data to Insight
AI-generated summaries of lab test results effectively provided family caregivers with a quick overview of the patient’s status.
However, participants emphasized the need to balance simplicity with sufficient medical context. While they found these summaries valuable for gaining an initial understanding—especially in high-stress ICU environments where cognitive load is a concern—they also expressed concerns about oversimplification. 
Reducing complex medical information to brief summaries risks omitting critical details, making it harder for caregivers to fully grasp the care process and provide effective support.

% \begin{quote}
% \textit{``[I want to know more details about] after the test is conducted, either a blood test or any test, what they found out after tests.''}(P7)
% \end{quote}

Since older adult patients often have multiple chronic conditions before being admitted to the ICU, caregivers prefer more than just AI-generated summaries of daily lab test results. Incorporating a patient’s prior medical history into these summaries is crucial, as it would help caregivers better understand acute conditions and how they relate to the patient’s overall health.

\begin{quote}
\textit{``Any [other] information I would like to see is the previous illness of the patient... if the patient is having any other illness [that leads to this result].''}(P10)
\end{quote}

\paragraph{Involving Caregivers in Treatment Plans and Goals}

For caregivers who cannot always be physically present in the ICU, access to detailed information about their loved one’s daily treatment plans and goals is crucial for maintaining a sense of connection and involvement in care.
While our prototype outlines treatment plans—including key milestones, medications, dosages, and their purposes—to help caregivers stay engaged, many participants emphasized the need for even greater transparency to further enhance their involvement in the care process. 
% For instance, caregivers emphasize the importance of precise timing for drug administration.

% \begin{quote}
% \textit{``Because even if I'm not around, maybe due to work purposes or any situations, [if] I could actually know the medication that is planned for the day [and] the dosage. I do feel like the time for this to administer the drugs was actually there (on the interface). It would have been a little bit better, maybe in case of times where I feel like I want to be there when this particular drug is being administered.''}(P7)
% \end{quote}

Beyond clarity, caregivers also seek insight into the broader arc of care, including how treatments correlate with patient progress. P9 highlighted this need, noting that understanding expected outcomes helps contextualize daily interventions:~\textit{``Yeah. It[the goals] like the kind of results, [and improvements] that we should be expecting, that's the only thing I noticed''}. Similarly, P10 stressed the value of linking treatments to patient responses:~\textit{``[besides providing the treatment, I want to know] how he is responding to treatment, [based on] the kind of treatment has been given that day, and the activities of the day''}.


\subsubsection{Context-Aware LLM-Based Chatbot: Enhancing Caregivers’ Understanding of ICU Information}
% Participants widely recognized the LLM-based chatbot as an effective tool for providing timely, context-aware knowledge support. Its ability to synthesize complex medical information into accessible responses resonated with caregivers, who sought an immediate and reliable understanding of clinical information.

% \begin{quote}
% \textit{``I think it (the chatbot) helps to understand the system's information, and it's something that [is] readily available, so it's very helpful...And I think that everyone would use such a feature, because right now AI is like a big part of everyone's daily life, so it's something that everyone would be keen to using and trusting.''}(P6)
% \end{quote}

% While participants generally appreciated the chatbot, some raised concerns about verifying its responses and ensuring accuracy. For instance, P15 highlighted the importance of cross-checking AI-generated information with medical professionals, stating:~\textit{``I wouldn't like one hundred percent trust the chatbot that answers my question. But I think with time, I might verify [the] answers [by] speaking to a doctor.''}

% In the following section, we summarize participants' detailed feedback on different chatbots.

\paragraph{Providing Reliable, On-Demand Responses}
Caregivers prioritized the need for accurate, instant explanations of ICU equipment and treatments—a demand met by the chatbot’s Fact Finder feature. 

%As P13 expressed a desire to understand new medications given to their grandmother, stating,~\textit{``if [my grandma] have like a new drug,...I would be curious to know what that drug does [better] among all other types of drug, so I could ask [this chatbot] what specific effect does it do for the patient''}.
By providing immediate explanations of medical treatment plans and goals, the chatbot helps caregivers understand care decisions without solely depending on clinician availability. For example, when asked about ventilator use, the chatbot explains how the device stabilizes oxygen saturation and supports lung healing, making complex medical concepts more accessible and reducing delays in understanding essential information.

\begin{quote}
\textit{`` It helps because there are times ... instead of waiting to see a doctor to actually explain [the treatment plan], you could actually just use the [chatbot] in this portal to know the reason why a particular equipment or a particular thing is actually administered.''}(P8)
\end{quote}

While participants generally appreciated the chatbot, some raised concerns about verifying its responses and ensuring accuracy. For instance, P15 highlighted the importance of cross-checking AI-generated information with medical professionals, stating:~\textit{``I wouldn't like one hundred percent trust the chatbot that answers my question. But I think with time, I might verify [the] answers [by] speaking to a doctor.''}

\paragraph{Translating Complex Data into Clear Insights}
Medical data in professional reports, often dense with jargon and numerical thresholds, can be overwhelming for caregivers unfamiliar with clinical term. Caregivers in our study expressed a need for clearer explanations of medical references to better understand their significance. The Data Interpreter helps address this challenge by providing definitions and contextual insights.
For example, when faced with an elevated MCHC level, the chatbot can explain what the acronym stands for, describe its role in assessing blood health, and clarify what an abnormal result might indicate for the patient. 
%As P13 highlighted their need for such explanations:~\textit{``[I would ask] what some references mean and what they could mean for the patient''}.

Caregivers also valued the chatbot’s ability to cross-check and validate information by referencing the patient’s medical history, rather than merely summarizing or explaining test results in isolation. This feature helps them interpret medical data more accurately, reducing unnecessary anxiety over abnormal test results by providing context and a clearer understanding of their significance.

\begin{quote}
    \textit{``I think having past diagnosis [connected to this chatbot] would be helpful... [then] you know [the] diagnose [is] proper, [and] without having any side effects [to the patients]''}(P15)
\end{quote}

%P7 emphasized the importance of such a capability, stating:~\textit{``if there is kind of mistake somewhere with the digits the number, but the chatbot indicates that there was a mistake, or to just give it answer based on the digits or the numbers that are provided, That's good [if it will identify the mistake in the report and tell you]''}.

\paragraph{Structuring Conversations and Improving Health Literacy}
Beyond providing reactive support, the Intention Prompter helps caregivers anticipate and navigate conversations with the clinical team more effectively.
% P13 shared their experience, stating:~\textit{sometimes I feel that my questions are really just basic questions that I asked over again in different ways and so they(clinicians) don't really pay attention or they just reply back with the same answer.''}
By suggesting questions based on the patient's status, the system helps move conversations beyond repetitive exchanges, making them more meaningful and informative. 

\begin{quote}
\textit{``Instead of racking my brain on what to ask the doctor,... I'll just like go straight to the point instead of trying to find things out, I'll just ask doctors the possible questions that I got from [the chatbot].''}(P12)
\end{quote}


% illustrated this shift, explaining how prompts about crutches or wheelchair readiness shaped their preparedness:~\textit{``Yeah, [with chatbot] Then you can actually know that what to prepare for in the future, maybe is going to use crutches or is going to use a wheelchair. Then it's actually helped me to know what you actually do to support [the patient] at that point. So it's actually very good.''}

This feature also helps caregivers improve their health literacy by guiding them to identify key topics for discussion with clinicians in advance. By preparing them with relevant information beforehand, it ensures they retain critical details during consultations and feel more confident in their conversations.
P6 emphasized its role in reducing cognitive overload, stating:~\textit{``Just the questions [listed] here would help me prepare questions [for the conversations], because the doctor mentioned so many things that I wouldn't even recall [and] I wouldn't even understand. But if I had prepared my question before and after [that] the doctor answered, then it helped me understand the situation''}. 
% Similarly, P10 highlighted its role in caregiver education, noting:~\textit{`` it will help me if we have the doctor to give me there certain answer that I needed. Although there might be changes in this. There might be an improvement on my own [medical knowledge].''}

\section{Discussion}
Drawing from these findings, we begin by exploring how our design can reposition and empower family caregivers as active participants in ICU care, rather than passive or ancillary stakeholders  (Section \ref{D1}).
Next, we outline key design considerations for future systems that support family caregivers' information needs in clinical settings for older adult patients(Section \ref{D2}).
Finally, we highlight risks and ethical concerns associated with the use of technology and AI in real-world implementations (Section \ref{D3}).
Finally, we present the limitations of our work and suggest directions for future research (Section \ref{D4}).

\subsection{Beyond Ancillary Roles: Empowering Caregivers as Informal Healthcare Partners}\label{D1}
Traditionally, in clinical settings, efforts to improve the quality of healthcare provision have centered around patient-centered care, which emphasizes increasing patient engagement in accessing clinical information and participating in communication throughout the care process~\cite{baker2001crossing, davis20052020}.
Within this framework, family caregivers have often been characterized as proxies or supplementary stakeholders, positioned primarily as assistants to clinicians or patients rather than active contributors to care~\cite{Foong2020, Bhat2023, essen2004proxy}.
Our formative study (see Section~\ref{study1}) further highlights how family caregivers often feel excluded—or even~\textbf{\textit{``left in the dark''}}—in ICU settings when trying to access and understand clinical information. 
In the ICU, caregivers shift from peripheral roles to active participants, who need to advocate for patients and make critical decisions on their behalf~\cite{au2017family}.
% In such situations, 
Recognizing the critical role of caregivers' involvement, recent healthcare research has increasingly acknowledged the importance of integrating them into clinical settings, proposing frameworks such as family-centered care (FCC)~\cite{jolley2009evolution, care2012patient, dunst1996empowerment} and practices to institutionalize their participation. 
% For instance, initiatives that encourage healthcare providers to involve caregivers in chronic and newborn care have demonstrated that structured family engagement can enhance patient outcomes and ensure continuity of care~\cite{kuo2012family, wolff2015look}. 

In parallel, research in Computer-Supported Cooperative Work (CSCW) and Human-Computer Interaction (HCI) has adopted an ecological perspective to investigate the socio-technical ecosystems surrounding family caregivers in clinical settings. 
Empirical studies have systematically mapped caregivers’ multifaceted roles in clinical settings - from providing emotional support to coordinating appointments, transportation, and daily care during hospital stays~\cite{Miller2016, Pina2017, Barbarin2015}. 
% Similarly, \citet{Kaziunas2015} explored the transitional experiences of caregivers of pediatric bone marrow transplant (BMT) patients. Their work illustrates how caregivers must quickly adjust to the unfamiliar routines and communication styles of the clinical environment, while simultaneously managing the stressful emotions and overwhelming information demands involved in caring for a loved one. 
% Such studies underscore the invisible labor and cognitive-emotional juggling required of caregivers in high-stakes clinical environments.
These findings collectively emphasize the importance of designing systems and practices that can better support caregivers’ evolving roles within complex clinical environments.
% This sense of exclusion is driven by inconsistent updates on patient status, misaligned treatment plans, and ambiguous communication from healthcare providers. In addition, the complexity of medical reports further complicates their ability to participate in care-related decisions.
% To address these challenges, we sought to center caregivers’ experiences by designing and evaluating an AI-based prototype tailored to their informational needs. 
%— while others have designed technologies to scaffold their caregiving labor~\cite{Kaziunas2015, Nikkhah2022, Sepehri2023, Schnur2024}. 
%For example,~\citet{Miller2016} describes how family caregivers navigate fragmented hospital workflows, often stepping in to fill gaps in care coordination.~\citet{kong2020addressing} developed a real-time communication platform that bridges family caregivers and clinicians, enabling collaborative interpretation of patient data through shared visualizations. 
Our research seeks to complement this body of work by further unpacking the complexities of caregivers’ informational needs as primary stakeholders, particularly within the high-stakes environment of the Intensive Care Unit (ICU).
Specifically, we highlight the importance of two-way information flow and adaptive support systems that accommodate caregivers' varying levels of health literacy.
Our findings show that technology can empower caregivers to take on a more active role, not just as recipients of information, but as engaged participants in care coordination. 
%For instance, our prototype features a context-aware, LLM-based chatbot, Intention Prompter, which helps caregivers formulate relevant questions before speaking with healthcare providers. By guiding caregivers in structuring their inquiries, this tool bridges communication gaps between medical teams and family caregivers, facilitating more meaningful conversations. It ensures that caregivers enter discussions better prepared, boosting their confidence and enabling them to advocate more effectively for the patient’s needs.
%Second, integrating the older adult’s medical history into the chatbot’s responses allows caregivers to receive more contextually relevant explanations of patients' conditions, supporting their understanding and participation in care coordination.
%By offering personalized context and supporting free-form, caregiver-driven inquiries, the chatbot helps caregivers better understand patient information in ways that align with their health literacy levels, while also creating more opportunities to seek clarification when healthcare providers are unavailable.

While our work demonstrates the potential of technology to mitigate caregiver exclusion, it also surfaces critical tensions in scaling these interventions. 
For example, while AI-based continuous patient-status tracking is essential in critical care, it may unintentionally increase caregiver mental strain in chronic care settings, where long-term stability is the primary concern rather than constant changes~\cite{Bhat2023}.
These technologies should not be applied uniformly across various clinical settings, and we argue against one-size-fits-all solutions. 
Instead, technologies should adapt to the context of caregiving: urgent situations require real-time updates, while long-term care may benefit more from tracking clinical trends over time.
% We support the development of flexible, customizable systems that adapt to the real needs of caregivers.
In a word, technology should be designed with input from caregivers and a focus on their values, especially in situations where collaboration with clinicians is crucial, such as dementia care \cite{smriti2024emotion} and post-operative recovery \cite{Kaziunas2015}.
%This also requires balancing transparency with ethical considerations, like allowing selective data sharing to protect caregiver and patient privacy or adding ``pause'' features to prevent feeling constantly monitored. 
%We advocate for designs that support caregivers while respecting clinical expertise, ensuring technology helps manage the challenges of caregiving without creating new burdens.

\subsection{Design Implications for Future Systems}\label{D2}

% Our research demonstrates that AI-based systems can support family caregivers in accessing and understanding information in high-pressure clinical environments (e.g., ICU) through two key interventions: 1) a visual timeline of a patient's critical medical events and 2) LLM-based chatbot guidance grounded in clinical evidence. 
% Findings from the prototype evaluation study indicated that this approach helps alleviate caregivers' cognitive burden and anxiety when managing rapidly changing patient conditions, particularly in situations where healthcare providers are not immediately available.

Based on insights from the formative study, we established a set of design goals to guide the development of an AI-based, caregiver-centered clinical information prototype (See Section~\ref{design space}). 
% Findings from the prototype evaluation study suggest that our prototype can help alleviate caregivers' cognitive burden when managing rapidly changing patient conditions, particularly in situations where healthcare providers are not immediately available.
In this section, we propose design implications for future systems that support caregivers in clinical settings, drawing on insights from Human-AI Collaboration and Computer-Supported Cooperative Work (CSCW). 
% These perspectives offer valuable guidance for enhancing caregivers’ access to and understanding of medical information in real-world healthcare environments.

\subsubsection{From the perspective of Human-AI Collaboration} Accessing and understanding clinical information is one of the most cognitively demanding tasks for family caregivers, especially in the high-pressure, fast-paced environment of the ICU. 
%This process heavily relies on caregivers’ ability to proactively seek help from healthcare providers. Unfortunately, such support is not always readily available.
Based on findings from our two-stage study, we explored with participants how AI could help streamline their processes of accessing and making sense of information. A key direction for future design is to go beyond basic summarization or extraction. 
Instead, AI systems should engage in adaptive, ongoing interactions that learn from caregivers’ responses and progressively personalize information delivery.

In addition, AI systems deployed in clinical settings should be designed to foster appropriate user trust and expectations. Given the complexity of these systems and the high-stakes nature of ICU environments, it is critical to clearly communicate the system’s capabilities and limitations to caregivers. 
Caregivers in our study expressed concerns about the ambiguity of AI-generated clinical summaries, particularly when numerical data were presented without adequate explanation. 
%To address this, our prototype integrates references to original data sources, allowing caregivers to verify and contextualize the information.
We suggest that designers offer explanations tailored to the underlying technologies used in the system, such as clarifying how outputs were generated and presenting insights in concise, easy-to-understand formats.
% Following human-AI collaboration principles, future systems should integrate adaptive AI interactions that empower caregivers while preserving the essential role of clinicians in the caregiving process. AI should act as a supportive partner, helping caregivers access and understand informed care decisions for the older adult patients without fostering over-reliance or replacing human expertise.


% Family caregivers face significant cognitive demands with limited health literacy in high-stakes clinical environments to access and understand clinical information.
% They must interpret lab results, navigate complex medical term, and synthesize fragmented verbal updates.
% With our participants, we explored the various AI-based features on how it can help caregivers....
% By applying a set of AI-based features, we believe that future design can offload the cognitive burden futhuer for caregivers , so they can further customize their information needs by ...
% While AI can help caregivers navigate complex medical data, its role should extend beyond simple automation. Participants emphasized that AI should not just generate summaries but also facilitate deeper understanding, guide caregivers in decision-making, and provide tailored explanations based on the patient’s history. Additionally, human oversight—whether from clinicians, caregivers, or medical professionals—remains essential for ensuring that AI-generated insights align with individual care needs and real-world medical decisions. Aligning with human-AI collaboration principles, future systems should integrate adaptive AI interactions that empower caregivers without replacing the critical role of clinicians.

% The effectiveness of AI-driven prototypes, such as our fact-based chatbot, underscores the need for balanced automation. Participants reported that just-in-time explanations of medical changes helped reduce anxiety, reinforcing the importance of AI systems that provide timely, relevant information while allowing caregivers to maintain a sense of control. However, some caregivers found the response length overwhelming, highlighting the need for concise, adaptable communication.
% Future systems must deepen this collaboration by integrating EHR data and adapting to individual needs, ensuring support aligns with both caregivers' health literacy levels and the patient's changing condition. For example, AI could deliver critical alerts during ICU emergencies while minimizing information overload in more stable recovery phases. 
% By designing AI as a bridge between complex clinical information and caregivers, these tools can simplify clinical data, offer clearer insights, and incorporate stress-aware features—such as alerts that prioritize the most relevant updates. This approach aligns with HCI principles, ensuring that AI serves as a supportive tool that enhances, rather than overwhelms, caregivers in high-stakes healthcare settings.

\subsubsection{From the perspective of CSCW}
ICU caregiving is both collaborative and isolating—while families play a crucial role, they often feel excluded from clinical workflows structured around the ICU’s fast-paced environment.  
CSCW research explores ways to improve teamwork in healthcare by developing systems that help caregivers and healthcare providers share and interpret important information across different roles and data sources~\cite{ziqi2024, Yuexing2024, Bowers2024}. 
In our study, caregivers expressed a desire to supplement clinical data with their firsthand insights and their need for greater transparency and inclusion in coordinated care. 
To address this, future systems could allow caregivers to contribute their observations (e.g., concerns about specific treatments), with both clinicians and AI validating the information. 
This approach would facilitate structured information exchange and strengthen collaboration with healthcare providers.
Additionally, if caregivers have the opportunity to provide input to the system and receive feedback from AI or healthcare providers, they can develop a better understanding of medical information as well as improve their health literacy. 
%A system designed in this way would present relevant information more clearly and help define role boundaries within the care team. 

%While AI cannot replace the detailed explanations and personal interactions provided by healthcare providers,
%Future systems could leverage AI to better support family-clinician communication by delivering relevant, context-aware information. Instead of enforcing rigid roles, AI-driven tools—such as dynamic patient timelines—could help caregivers interpret medical updates, track key changes over time, and stay informed about their loved one’s condition. Participants in our study highlighted the need for clearer guidance and more transparent information-sharing. AI-powered timelines could address this need by visually integrating care milestones, family observations, and clinical updates into a single, accessible interface, reducing ambiguity and improving coordination among all stakeholders.  

\subsection{Risks and Ethical Concerns of AI-based Clinical Information Technologies}\label{D3}

While AI-based technologies hold promise in supporting caregivers' access and understanding of clinical information, certain emerging risks and ethical concerns should also be carefully considered.

A central tension emerges from our prototype’s dual role as both a support tool and a potential stressor. 
Caregivers acknowledged that our prototype could facilitate their access to information, such as tracking changes in a patient’s condition, yet they also expressed concerns about information overload, particularly when frequent, non-essential updates increased their cognitive burden. 
This aligns with previous research on AI-supported decision tools, which found that frequent notifications and complex data presentations can inadvertently increase user stress~\cite{wang2023}. While AI can automate data collection and generate insights, human oversight remains crucial to interpret alerts, prioritize key information, and prevent caregivers from becoming overwhelmed.

Beyond cognitive burden, trust in AI-generated insights emerged as a major concern among caregivers. Participants found AI-generated explanations helpful in understanding lab results, but still felt the need to cross-validate responses with healthcare providers before making care decisions. 
This reflects a broader challenge in AI adoption—while AI can enhance information accessibility, caregivers do not perceive it as an authoritative source but rather as a supplementary tool. Prior research on clinical decision support systems has noted similar patterns, where users rely on AI-generated insights but defer final decision-making to human experts~\cite{zhang2024}. This underscores the need for explainable AI (XAI) approaches that offer not only summaries but also data sources, helping caregivers understand why certain answers are made and when they should seek human validation.  


% This aligns with broader HCI discussions on trust in automated systems, highlighting how the lack of mechanisms for questioning or contextualizing AI outputs can undermine trust and confidence in the system.
% To address this, AI design should prioritize clarity and contextual support—such as offering plain-language explanations for alerts (e.g., ``This oxygen saturation drop may be linked to recent medication changes'')—while ensuring caregivers can better interpret and engage with critical medical information.

% Privacy concerns and trust were equally significant, particularly given caregivers’ intermediary role in managing sensitive patient data.
% Participants reflected on their prior ICU experiences, describing them as information-dense and high-pressure, which underscored the need for privacy-preserving data practices. To address this, future systems should integrate granular consent mechanisms (e.g., allowing caregivers to opt out of specific data-sharing features), aligning with HCI principles of privacy by design.
% Regarding the potential risks of over-trust and under-trust in AI-generated responses, it is crucial to ensure transparency and accuracy in the system’s outputs. Although our data is built on patients’ EHR records, additional safeguards—such as explainable AI techniques, source attribution, and user verification prompts—can help caregivers critically assess AI-generated information and mitigate misplaced trust.

% We advocate for AI systems that go beyond simple task automation to actively support caregivers in managing complex medical situations. To achieve this, AI design must incorporate caregivers' real-world experiences, ensuring that these technologies enhance—rather than limit—their ability to understand and engage with clinical care.


\subsection{Limitations and Future Work}\label{D4}

Our study has several limitations that should be acknowledged. 
First, the scope of our participant pool was limited, with 15 participants in the formative study and 10 participants in the evaluation phase. While these participants provided valuable insights, the small sample size may limit the generalizability of our findings. Future research should aim to include a larger and more diverse group of caregivers to ensure broader applicability.
Nevertheless, our analysis reached a point of saturation, ensuring that the results comprehensively reflect the experiences and perspectives of caregivers for older adult patients in critical care settings.
Second, the prototype developed in this study is a proof-of-concept and has not been integrated with Electronic Health Record (EHR) systems or deployed in real-world clinical settings. While the prototype demonstrates the potential of AI-based tools to support caregivers, its effectiveness and usability in practical, high-stakes environments remain untested. Future work should focus on integrating such tools with existing healthcare infrastructures and evaluating their performance in real-world scenarios.

Additionally, our study specifically focused on designing a prototype to address caregiver-facing challenges. While this approach provides valuable insights into the needs of family caregivers, it does not fully account for the perspectives of clinicians or other stakeholders in the care process. Future studies should involve clinicians to explore how such tools can facilitate collaboration between caregivers and healthcare providers, ensuring that the design meets the needs of all parties involved.
Lastly, while our research primarily addresses the ICU setting, there is a broader need to explore how similar tools can be adapted for other clinical environments. The challenges faced by caregivers in ICUs may differ from those in other settings, such as long-term care facilities or home-based care. Future work should delve deeper into these contexts to develop more versatile solutions that can support caregivers across a wide range of clinical scenarios.

\section{Conclusion}

In this study, we investigated the information needs of family caregivers of ICU older adult patients and designed an AI-based prototype, \projectname,  to address their challenges in accessing and understanding clinical information.
Through a formative study with 15 caregivers, we first uncovered the multifaceted challenges caregivers face in the ICU. 
Building on these insights, we designed an AI-based prototype and evaluated it with 10 caregivers.
Results from the evaluation study suggested that, despite participants identifying potential trust risks in using an AI-based system, they emphasized that the plain-language explanations and unified visual timeline empowered them to more effectively access and understand information in critical care settings.
This paper contributes to HCI research in healthcare by highlighting the need for AI systems that actively engage caregivers and ensure they feel informed and supported throughout the care process, rather than sidelined.
%These findings contribute to HCI research in healthcare, providing the foundation for designing inclusive, human-centered AI systems that amplify—rather than displace—caregivers’ critical role in patient advocacy and care continuity.



\bibliographystyle{ACM-Reference-Format}
\bibliography{sample-base}


%%
%% If your work has an appendix, this is the place to put it.
\appendix



\section{Formative Study - Interview Script}\label{Study1script}

\begin{itemize}
    \item \textbf{Question 1 - Experience:} 
    Can you describe your experience as a caregiver when your loved one was transferred to the ICU?
    %remove the following to be more clear/concise? 
    % During the ICU stay, how did you receive updates on your loved one’s condition? How often did you talk with the clinical team?

    \item \textbf{Question 2 - Information:}
    What kind of information did you seek as a caregiver and what was your experience accessing and understanding the information?
    %remove the following to be more clear/concise?
    % What did you expect to learn from the clinical team and what did they share with you? Did you find the information clear and well organized? How did you interpret and use this information?
    
    \item \textbf{Question 3 - Technology Use:} How did you utilize digital tools, such as the patient portal or search engines, to gather information?
    %remove the following to be more clear/concise?
    % What information were you looking for, and how did you validate the information?

    \item \textbf{Question 4 - Technology Expectation:} What do you expect of an AI-based digital tool that is designed to help you in such a situation? Would you trust the AI?

    %not sure add or not
    % \item \textbf{Question 5 - Closing:} Is there anything else you would like to share or do you have any questions for us?

    
\end{itemize}

\section{Evaluation Study - Interview Script}\label{Study2script}
The following set of questions was asked for each interface presented to the participants.
\begin{itemize}
    \item \textbf{Question 1 - Feedback on the Design:} (After showing a screenshot of the UI) How do you like this interface? 

    \item \textbf{Question 2 - Utility:} Do you think this function would be useful to you? Why or why not? 

    \item \textbf{Question 3 - Use Cases:} 
    How would you use this interface/function? [What kind of questions would you ask this chatbot?]

    \item \textbf{Question 4 - Improvement:} What improvements can be made?
    
\end{itemize}

\section{Qualitative Codebook}
\onecolumn


\begin{table}[h!]
\centering

% \renewcommand{\arraystretch}{1}
\caption{Qualitative Codebook of Formative Study Findings (Stage 1)}
\label{Appendix:codebookofformative}
\resizebox{\textwidth}{!}{%
\begin{tabular}{p{0.19\textwidth}|p{0.19\textwidth}|p{0.6\textwidth}}
\toprule
% \hline
\textbf{Theme} & \textbf{Sub-Theme} & \textbf{Example} \\
\midrule
% \hline

\textbf{Challenges in Accessing Clinical Information}
% \multirow{9}{*}{\textbf{Challenges in Cardiotoxicity Decision-making}}  
% \multirow{9}{=}{\centering \textbf{Challenges in Cardiotoxicity Decision-making}}  
& \multirow{3}{=}{Staying Informed about Patients' Changing Status}  
& "It's hard to get to know who's on the shift that day, and sometimes you have to go there physically [to know his situation]." (P4) \\
\cmidrule{3-3}
& & "There were times when I had to seek out doctors or nurses for updates,... I sometimes felt that my patients were seen as an interruption to the busy ICU schedule... In cases where I didn't get information, I didn't get to know what was going on,... I just felt I was left in the dark" (P1) \\
\cmidrule{3-3}
& & "They are also doing the other stuff, [like] attending to other patients, I wouldn't want to like bother them that much... they are too busy you weren't able to ask all the questions you want to, [and] they weren't able to like explain in details within the short time." (P12) \\
\cmidrule{2-3}
% \hline

& \multirow{3}{=}{Struggling to Align Treatment Plans and Goals with Healthcare Team}
& "So they told me that,... We should probably be more than a week at the ICU and just informing me about... like they might have to maybe perform another surgery if he... doesn't get responsive, and just telling me other treatment options, like ultrafiltration that the end add transplants that might be considered if it isn't responsive." (P6) \\
\cmidrule{3-3}
% \cline{2-3}
& & "[Clinicians did] all the tests, the procedure - [but] what they tested and what they saw, what they got, that kind of deep parts - [I don't know]. I wanted to know the whole [situation] about it, but I think they didn't open it all up, and they were just telling us the diagnosis." (P7) \\
\cmidrule{3-3}
& & "I don't feel like [I'm totally involved in the treatment plan]...Most of the time, the treatment plan is actually well explained, but I do feel like the side effects of certain medications [are] not actually being spelled out." (P8) \\
%\cmidrule{2-3}

\midrule


\textbf{Challenges in Understanding Clinical Information}
% \multirow{6}{=}{\textbf{Opportunities}} 
& \multirow{3}{=} {Interpreting Medical terms in Clinical Report and Conversations} 
& "Some doctors tend to use ambiguous terms..." (P4) \\
\cmidrule{3-3}
% \cline{2-3}
& & "We couldn't understand some medical terms. So we asked him (the doctor) to use a clear word where he explained everything that is happening around..." (P5) \\
%\cmidrule{2-3}
\cmidrule{3-3}
& & "when it comes to having to pick reports,...the report is not actually something I could compliment, because there are times there that I do have to... use Google to actually go through and know what it actually means." (P8) \\
%\hline
%\midrule
\cmidrule{2-3}
& \multirow{2}{=}{Asking meaningful questions with healthcare team}
& "Because there was this time I did ask the question, and I was told, even if I did explain to you, you wouldn't understand. Just believe that we are doing this for the good of your relative. You see, so for someone who's afraid to know what they should know, [their choice] is not going to ask questions, which is very wrong." (P4) \\
\cmidrule{3-3}

& & "I was on myself, so I do ask random questions [like] will it be fine? How much is it going to cost us to be here? How long are we going to be here?" (P11) \\

\bottomrule
% \hline
\end{tabular}
}
\end{table}



\begin{table}[h!]
\centering

% \renewcommand{\arraystretch}{1}
\caption{Qualitative Codebook of Prototype Evaluation Study Findings (Stage 2)}
\label{Appendix:codebookofforevaluation}
\resizebox{\textwidth}{!}{%
\begin{tabular}{p{0.19\textwidth}|p{0.19\textwidth}|p{0.6\textwidth}}
\toprule
% \hline
\textbf{Theme} & \textbf{Sub-Theme} & \textbf{Example} \\
\midrule
% \hline

\textbf{Visual Timeline of ICU Stay: Supporting Caregivers’ Access to Consistent Information}
% \multirow{9}{*}{\textbf{Challenges in Cardiotoxicity Decision-making}}  
% \multirow{9}{=}{\centering \textbf{Challenges in Cardiotoxicity Decision-making}}  
& \multirow{3}{=}{Synthesizing Fragmented Information into a Cohesive Timeline}  
& "[The medical timeline] should be helpful to monitor the progress, and if [the patient's condition is] deteriorating or [if] the treatment is being progressive" (P6) \\
\cmidrule{3-3}
& & "[I] might not be daily [to check on that information], but anytime you want to really know about what is really wrong, to have more understanding [on the patient's status], you just look at it." (P7) \\
\cmidrule{2-3}
% \hline

& \multirow{3}{=}{Balancing Simplicity and Depth in AI-Generated Summaries}
& "Any [other] information I would like to see is the previous illness of the patient... if the patient is having any other illness [that leads to this result]." (P10) \\
\cmidrule{3-3}
% \cline{2-3}
& & "[I want to know more details about] after the test is conducted, either a blood test or any test, what they found out after tests." (P7) \\
%\cmidrule{2-3}

\cmidrule{2-3}
% \hline

& \multirow{3}{=}{Involving Caregivers in Treatment Plans and Goals}
& "`Yeah. It[the goals] like the kind of results, [and improvements] that we should be expecting, that's the only thing I noticed" (P9) \\
\cmidrule{3-3}
% \cline{2-3}
& & "[besides providing the treatment, I want to know] how he is responding to treatment, [based on] the kind of treatment has been given that day, and the activities of the day." (P10) \\
%\cmidrule{2-3}

\midrule


\textbf{Context-Aware LLM-Based Chatbot: Enhancing Caregivers’ Understanding of ICU Information}
% \multirow{6}{=}{\textbf{Opportunities}} 
& \multirow{3}{=} {Providing Reliable, On-Demand Responses} 
& "It helps because there are times ... instead of waiting to see a doctor to actually explain [the treatment plan], you could actually just use the [chatbot] in this portal to know the reason why a particular equipment or a particular thing is actually administered." (P8) \\
\cmidrule{3-3}
% \cline{2-3}
& & "I wouldn't like one hundred percent trust the chatbot that answers my question. But I think with time, I might verify [the] answers [by] speaking to a doctor." (P15) \\
%\cmidrule{2-3}
%\hline
%\midrule
\cmidrule{2-3}
& \multirow{2}{=}{Translating Complex Data into Clear Insights}
& "[I would ask] what some references mean and what they could mean for the patient." (P13) \\
\cmidrule{3-3}

& & "I think having past diagnosis [connected to this chatbot] would be helpful... [then] you know [the] diagnose [is] proper, [and] without having any side effects [to the patients]." (P15) \\


\cmidrule{2-3}
& \multirow{2}{=}{Structuring Conversations and Improving Health Literacy}
& "``Instead of racking my brain on what to ask the doctor,... I'll just like go straight to the point instead of trying to find things out, I'll just ask doctors the possible questions that I got from [the chatbot]." (P12) \\
\cmidrule{3-3}

& & "Just the questions [listed] here would help me prepare questions [for the conversations], because the doctor mentioned so many things that I wouldn't even recall [and] I wouldn't even understand. But if I had prepared my question before and after [that] the doctor answered, then it helped me understand the situation" (P6) \\


\bottomrule
% \hline
\end{tabular}
}
\end{table}


\end{document}
\endinput
%%
%% End of file `sample-sigconf-authordraft.tex'.
