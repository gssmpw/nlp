\section{Related Work}
~\label{relatedwork}

We conduct a comprehensive review of the experiences of family caregivers for older adult patients in critical care environments, various technologies developed for caregivers in clinical settings, and the role of AI-based clinical information technology in this context.
\subsection{Family Caregivers of Older Adults in Critical Care Settings}

% Patients admitted to the Intensive Care Unit (ICU) often face life-threatening conditions caused by illnesses, injuries, or complications~\cite{hanberger2005intensive, valentin2011recommendations}. 
In the United States, over half of ICU admissions involve older adults aged 65 and above~\cite{flaatten2017status}. 
This population is receiving growing attention due to their complex health needs, which are often compounded by chronic diseases and acute health crises~\cite{jarvis2023physical}. 
When older adult patients are unable to talk or are unconscious due to severe injuries, family caregivers of older adults have to actively engage in the care process~\cite{au2017family, johnson1995perceived}.
For instance, family caregivers could be assigned powers of attorney to make critical decisions on behalf of their loved ones~\cite{hines2011effectiveness}.
On the other hand, they need to prepare for the patient’s post-discharge care by coordinating the necessary support with healthcare providers, as caregiving demands often continue or escalate following patients' ICU discharge~\cite{choi2014fatigue}. 
The demanding responsibilities of family caregivers often place an immense burden on them, which is often compounded by the difficulties of accessing and understanding medical information in the ICU.


% These factors contribute to significant challenges both during their ICU stay and before discharge.
% Meanwhile, caregivers of older adult patients face immense challenges~\cite{au2017family, hines2011effectiveness}. 
However, caregivers are often not familiar with the critical care environment and only possess limited health literacy, which causes significant difficulty for them in understanding the overwhelming and complicated critical care situation.
ICU wards impose restrictive visitation policies to avoid disruptions caused by the presence of unnecessary non-professional personnel~\cite{dragoi2022visitation, tabah2022variation}. 
Such restrictions leave caregivers with difficulties in comprehensively accessing and understanding the patient's health condition and treatment progression in a timely manner.
Moreover, caregivers experience deep anxiety and helplessness about their loved ones because of the sudden deterioration of older adults' health and are unable to anticipate what could happen to the older adults~\cite{jennerich2020unplanned}.
% Many are unfamiliar with the ICU environment and deeply anxious about the unplanned admission of their loved ones ~\cite{jennerich2020unplanned}.
Studies report that up to 54\% of family caregivers experience psychological trauma or severe stress after caregiving experiences in critical care situations~\cite{alfheim2019post}.

% During older adult patients' ICU stays and before discharge, family caregivers bear compounded responsibilities, serving a dual role that is both emotionally and practically demanding.
% On the one hand, they act as crucial advocates, representing the values and preferences of critically ill older adults who are unable to express themselves~\cite{au2017family, hines2011effectiveness}. 
% Family caregivers are frequently called upon to make critical decisions on behalf of their loved ones~\cite{hines2011effectiveness}.
% On the other hand, they must prepare for the patient’s post-discharge care by coordinating the necessary support with healthcare providers, as caregiving demands often continue or escalate following ICU discharge~\cite{choi2014fatigue}. 

% This dual role—speaking on behalf of the patient and preparing for their care—places an immense burden on caregivers, particularly as they often have limited direct access to patients due to ICU restrictions. 
% ICU wards impose restrictive visitation policies to avoid disruptions caused by the presence of unnecessary non-professional personnel~\cite{dragoi2022visitation, tabah2022variation}. 
% These restrictions leave caregivers with an incomplete understanding of the patient’s condition and limit the opportunities for in-person updates.
Consequently, caregivers have to rely heavily on direct communication with healthcare providers to stay informed about the patient’s status~\cite{yoo2020critical}. 
However, the provider teams are always overloaded with caring for critically ill patients and have very limited opportunities and time to talk with caregivers.
Such constrained caregiver-provider communications can hardly meet caregivers' information needs.
% However, current communication practices, often limited to sporadic in-person updates or brief phone calls, are typically time-constrained and fail to meet caregivers’ informational needs. 
Research highlights persistent dissatisfaction among caregivers despite institutional efforts to improve clinical communication through tools like daily goals forms~\cite{pronovost2003improving}, critical care family satisfaction surveys~\cite{steel2008impact}, and allocation of additional resources to ICU wards~\cite{breslow2005technology}. 
% Caregivers consistently report unclear information as significant barriers to their ability to effectively advocate for and support their loved ones~\cite{azoulay2016communication}.
Further, the lack of clear, timely, and actionable information heightens caregivers' stress and disrupts their ability to prepare for the patient’s post-discharge care~\cite{azoulay2016communication}.
% As a result, family caregivers of older adult ICU patients often overwhelmed with multi-faceted feelings, including anxiety, confusion, and responsibility.
% navigate a complex and critical medical environment, 
% grappling with feelings of overwhelm, confusion, and a profound sense of responsibility, further intensifying the psychological burden of caregiving in such high-stakes situations.

%Family caregivers of older adult ICU patients are key stakeholders in this context, as they often act as crucial patient advocates, representing the values and preferences of critically ill patients who are unable to express themselves~\cite{au2017family, hines2011effectiveness}.

% Futhermore, to know more about the patients condition relays heaily on the communication with healthcare providers. 
% However, current communication practices, often limited to sporadic in-person updates or phone calls with them, are time-constrained and insufficient. 
% Research indicates that despite organizational calls for improved clinical communication practices, such as daily goals form~\cite{pronovost2003improving}, Critical care family satisfaction survey~\cite{steel2008impact}, and allocate the appropriate resources to the ICU ward~\cite{breslow2005technology}and etc, caregiver satisfaction with ICU communication remains low. 
% They consistently report inadequate information and unclear guidance as persistent barriers to effective advocacy and care. 

% As a result, family members of an ICU older adult patient experience xxxx complicated, overwhelming, ... 

% addressing the information challenges faced by caregivers remains a pressing issue within the critical and high-pressure environment of the ICU.


% The Intensive Care Unit (ICU) is a specialized ward providing acute care for patients experiencing life-threatening events caused by illnesses, injuries, or complications~\cite{hanberger2005intensive, valentin2011recommendations}. 
% In the United States, older adults aged 65 and above account for more than half of ICU admissions, a trend driven by the global growth of aging populations~\cite{flaatten2017status}. 
% This demographic faces a higher likelihood of experiencing critical health events~\cite{jarvis2023physical}, positioning them as an expanding and significant subgroup of ICU patients~\cite{angus2006critical, flaatten2017status}.
% During critical transition moments, \textbf{family caregivers} of older adults often act as crucial patient advocates, representing the values and preferences of critically ill patients who are unable to express themselves.~\cite{au2017family, hines2011effectiveness}.   
% This trend, driven by the growing aging population, imposes substantial challenges on healthcare systems~\cite{flaatten2017status}.
% caregivers are frequently called upon to make critical decisions on behalf of their loved ones~\cite{hines2011effectiveness}. 
% The combination of emotional distress, constrained communication, and the weight of decision-making responsibilities often leads to feelings of anxiety and powerlessness~\cite{}. 


% Patients are admitted to intensive care units (ICUs) for acute, life-threatening illnesses that entail substantial emotional distress, high treatment burdens, uncertain prospects for survival, and the potential for new disabilities~\cite{needham2012improving, elliott2014exploring, netzer2014recognizing}. 
% Notably, the number of older adults requiring ICU admission is on the rise~\cite{flaatten2017status}, driven by the aging global population~\cite{united_nations_2019}. This increase is associated with significant costs and resource utilization~\cite{bagshaw2009very, lerolle2010increased}, placing additional strain on healthcare systems. Older adults not only face a higher likelihood of experiencing unexpected critical health events but also tend to recover more slowly after acute illnesses~\cite{jarvis2023physical}. Upon admission to the ICU, the intensity of treatments—such as the use of vasopressors, mechanical ventilation, and renal replacement therapy—often varies between younger and older patients. Studies have shown that older patients typically receive less intensive treatment compared to their younger counterparts~\cite{nguyen2011challenge}.

% While improving survival rates remains a primary goal of ICU care for all patients, another crucial objective, especially for the elderly, is to minimize inherent risks and enhance recovery~\cite{boyd2008recovery}. However, older patients are more susceptible to a range of complications, including nosocomial infections, iatrogenic injuries from invasive monitoring, prolonged bed or chair rest, sleep deprivation, delirium, extended hospital stays, and more restrictive visiting hours for families~\cite{mercier2010iatrogenic, pisani2010factors}. These risks contribute to increased morbidity, cognitive impairment, and functional disability~\cite{herridge2009legacy, girard2010delirium}. Given these challenges, it is imperative to pay special attention to older adult patients hospitalized in ICUs. Age itself is a significant independent risk factor for in-ICU mortality~\cite{halpern2004critical, fuchs2012icu}. Additionally, advanced age is associated with longer ICU stays and a higher likelihood of unsuccessful discharge~\cite{flaatten2017status, wang2019predictors}. Most older adult patients admitted to the ICU present with critical illnesses accompanied by severe physical limitations that complicate their care needs~\cite{coombs2017implementing}. Furthermore, both the hospitalization process and the use of invasive technologies in the ICU are highly stressful for patients and their families~\cite{carlson2015care}.

% In response to the growing number of older adults requiring intensive care, there is an urgent need for healthcare systems to prepare for effective care delivery~\cite{lasiter2011older}. Implementing suitable tools to support the transition of older adults into the ICU can help alleviate the associated pressures and improve outcomes for this vulnerable population.




%\subsection{Family Caregivers' Information Needs and Challenges in the ICU}

%The importance of recognizing information needs for family caregivers during critical ICU moments has been well-documented in health literature~\cite{gaeeni2014informational}.
%Caregivers’ information needs are dynamic and evolve throughout the patient’s ICU stay. Because the ICU environment can change rapidly, even from hour to hour, family members need continuous up-to-date information to stay informed about the patient's condition, treatment, prognosis and diagnostic tests \cite{davidson2010facilitated, bueno2018principales, engstrom2004experiences}.  They also seek to understand the rationale behind specific treatments and daily care decisions, as well as the ICU setting itself (including equipment and protocols) and the various disciplines involved in patient care. Furthermore, caregivers want to know what they can do to support the patient after discharge, emphasizing the need for clear guidance on transitional care and rehabilitation \cite{verhaeghe2005needs}.Obtaining such information about older adult patients is consistently a top priority for caregivers~\cite{naderi2013family}, as it helps reduce stress levels~\cite{bijttebier2001needs}, allows better adaptation to stressful conditions~\cite{azoulay2002impact, yaman2010evaluation}, and improves post-ICU care outcomes~\cite{verhaeghe2005needs}.However, caregivers frequently report that their most critical needs — access to information and reassurance — often go unmet during ICU hospitalizations~\cite{al2013family, paul2004meeting}. 

%The primary issue lies in the lack of a clear and efficient information pathway (e.g., channels or systems for accessing patient-related information) for caregivers, which can be analyzed across several key dimensions. First, information resources for family caregivers are often clinician-driven and passive~\cite{schnock2017identifying}, leaving caregivers dependent on updates from healthcare providers. With limited access to healthcare providers due to the busy ICU schedule, caregivers often remain underinformed. For example, studies report that only a small proportion of ICU families at a prominent academic medical center had the opportunity to meet with an attending ICU physician as part of routine care~\cite{lilly2000intensive}.  Given that the average ICU in the U.S. cares for approximately 10 patients daily~\cite{halpern2006changes}, it is unrealistic to expect the ICU care team to engage in comprehensive discussions with every family. Further compounding this issue is the fragmentation of information across multiple stakeholders~\cite{grant2015resolving, pham2008alterations}.  ICU patients are typically cared for by interdisciplinary teams of specialists and nurses, with frequent handoffs occurring across shifts, nights, and weekends.  This fragmented communication often leaves caregivers uncertain about whom to approach for updates or clarification~\cite{gay2009intensive}.  Additionally, wide variations in health literacy of different caregivers — combined with the prevalence of specialized medical jargon — pose significant barriers~\cite{young2017family}. These factors make it harder for caregivers to interpret critical updates and advocate effectively. 
%Although some research has focused on improving caregiver experiences, efforts remain fragmented. 
%Existing interventions, such as pamphlets, checklists, and bedside electronic medical viewers~\cite{caligtan2012bedside, nelson2006improving, scheunemann2011randomized, azoulay2002impact}, provide limited support for creating a clear and caregiver-centric information pathway. 
%Addressing the information challenges faced by caregivers remains a critical and pressing need within the high-stakes, high-pressure environment of the ICU.


% Several interventions have been implemented to improve family access to important information in the form of pamphlets, checklists, and bedside electronic medical viewers~\cite{caligtan2012bedside, nelson2006improving, scheunemann2011randomized, azoulay2002impact}.

%Need for Interventions Supporting ICU Care Transitions

% A lack of clear information pathways for informal caregivers leads to poor communication with care professionals and fragmented or scattered information sources, and so increases the burden for informal caregivers~\cite{bueno2018main}. 

% There is a clear need to support people with dementia and their caregivers through the journey from diagnosis to care [63] through personalized information and support interventions that increase confidence in making decisions about formal residential care placements [64]. 

% Most interventions to support the care transition focus on providing individual or group-based counseling to informal caregivers provided via phone, email, or in person [53]. Aside from management systems for digital waiting lists, e.g., [46], few digital interventions are developed to address such transitions to higher levels of care in dementia.

% Recent studies show that intentional support for caregivers is lacking, especially in the post-ICU period~\cite{major2019survivors, aagaard2015spouse}

% Prior studies on caregivers of ICU survivors have focused on quantifying the psychosocial burden of caregiving~\cite{van2015reported}, but there is little research to inform improvements in care delivery~\cite{sevin2021optimizing}. 

% Caregivers suggest these needed supports could be supplied in a variety of infrastructural forms—phone calls, home health, post-ICU clinic, peer support, online forums—and individualized to maximize impact at necessary time points~\cite{sevin2021optimizing}.

%  The ICU can change quickly, even from hour to hour, so the family members need continual, progressive information~\cite{davidson2010facilitated}.

\subsection{Technologies Supporting Family Caregivers in Clinical Settings}

%Technology Solutions for Non-expert in Clinical Information
%EMR system/ EHR data 
%refine structured EHR data

In recent years, the fields of Computer-Supported Cooperative Work (CSCW) and Human-Computer Interaction (HCI) have emphasized the critical role caregivers play in patient care within clinical settings~\cite{Don2024, Siddiqui2023, Yunan2013} and emphasized the challenges caregivers face associated with clinical information~\cite{Stefanidi2023}. In response, prior research has explored various technological solutions to address these challenges~\cite{Pine2018}.

On the one hand, researchers have explored various technology designs to enhance clinical information accessibility, including facilitating the sharing of information between clinicians and caregivers~\cite{ziqi2024, Yuexing2024, Bowers2024} and assisting caregivers to seek clinical information of older adult patients~\cite{montagna2023data, wei2024leveraging}.
For example, ~\citet{Bowers2024} developed a mobile application that offers template support for caregivers to document and assess behavioral changes in patients with congenital heart disease, facilitating effective sharing of clinical information with clinicians. 
SaludConectaMX is a mobile application designed for caregivers to access clinical information, including patient medical history and oncology treatments, to assist in tracking patients' evolving health trajectories~\cite{Schnur2024}.
~\citet{Barbarossa2023} developed a dashboard to present the most relevant clinical information of older adult patients with dementia, including fall events, sleep data, and wellbeing data, for caregiver access to patient status. 
While effective, most of these tools are intended for caregivers of patients with chronic conditions, such as dementia.
Consequently, their accessibility is not suitable for the time-sensitive and high-stakes clinical environment of the ICU.

Another area of research explores using visualization tools to decipher complex clinical information to support caregivers' understanding of clinical information~\cite{kong2017comparative, Hong2017, Nyapathy2019, liu2011, Shea2019, Hwang2014}.
For instance,~\citet{kong2017comparative} introduced EnGaze, a web-based tool designed to visualize the communication behaviors of children with autism during clinical visits, aiming at improving caregivers' understanding of the condition and promoting their active involvement in care planning.
An interactive system designed by~\citet{Hong2017} assists families and clinicians in reviewing radiology imaging results by offering simplified definitions and diagrams of medical term during consultations.
~\citet{Nyapathy2019} presented a visualization mobile application designed to assist caregivers in recording and visualizing the long-term condition of asthma patients, thereby enhancing the shared understanding of the condition with clinicians.
While these technologies provide valuable assistance for caregivers in routine clinical settings, the complexity and density of information produced in the ICU can significantly overwhelm caregivers compared to other contexts.
The proposed visualization technology solutions are insufficient to support caregivers in understanding the clinical information of older adult patients in the ICU. 

In conclusion, existing tools and technologies remain inadequate in supporting caregivers of older adult ICU patients in accessing and understanding information.
Existing solutions are fragmented in scope, indicating a significant need to leverage novel technologies to systematically address caregivers' difficulties in accessing and understanding clinical information in critical care settings.



%In recent years, the fields of Computer-Supported Cooperative Work (CSCW) and Human-Computer Interaction (HCI) have advocated for caregivers' critical role in clinical settings, information needs in clinical settings~\cite{Pine2018}. 

%Prior work identified a significant barrier that impedes effective information sharing for family caregivers is the absence of a centralized data storage framework~\cite{adams2021perspectives, amir2015care}.
% Despite the critical need for timely and accessible health information for family caregivers in ICU settings, the absence of a centralized, user-friendly data storage system creates significant barriers~\cite{adams2021perspectives, amir2015care}.
%Today's Electronic Health Records (EHR) systems, which are widely adopted by hospitals to support clinical information organization and storage, still suffer from critical limitations~\cite{Sepehri2023}.
% A primary barrier lies within today’s Electronic Health Records (EHR) systems, which, despite being widespread, suffer from critical limitations.
% While these systems are widely adopted, their potential is constrained by critical limitations. 
% Chief among these is the lack of standardized frameworks, which severely impedes efficient information sharing~\cite{Sepehri2023}. 
%For instance, data in EHR often remain fragmented, making it difficult to be clearly organized and not possible to be understandable by non-experts ~\cite{o2010electronic}.
% and pose interpretability challenges, especially for non-specialists~\cite{o2010electronic}. 
%Furthermore, EHR was primarily designed to document the data generated or recorded from the patients, but not necessarily capture the interactions between providers and the caregivers. 
% Furthermore, EHR documentation often lacks critical care milestones, capturing what was recorded rather than what was actually delivered or experienced by patients and their families.
%For instance, clinicians may engage in important goals-of-care discussions with patients but fail to adequately document these interactions, which leads to difficulties when caregivers want to retrieve such information at a later time~\cite{moody2004electronic, mack2012end}.



% for understanding the comprehensive approaches that address the full range of information management and emotional support challenges.


\subsection{AI-based Clinical Information Technologies}
 
Recent advancements in artificial intelligence (AI) have shown significant potential to improve caregivers' access to and understanding of clinical information. 

In order to improve caregivers' access to complex clinical information, researchers have developed AI models to extract complex clinical information from electronic health records (EHRs) through summarization~\cite{Sette2023, Nikkhah2021, gopinath2020fast}, identify high-risk factors in medical records~\cite{corey2018development}, and predict risks from clinical data, such as sepsis risk~\cite{suresh2017clinical, yin2024sepsiscalc}. 
AI can also enhance the understanding of clinical information.
For instance, some models can translate complicated medical jargon into accessible language for the general audience to understand ~\cite{yim2024preliminary, wachter2024will, wong2018using}, and simplify lengthy medical texts into shorter versions~\cite{basu2023med, artsi2024large}. In addition, chatbot systems based on large language models (LLMs) show potential in supporting caregivers' personalized information-seeking needs~\cite{afshar2024prompt}. 
Such solutions have proven effective in delivering health advice regarding screening, diagnosis, treatment, and disease prevention~\cite{huo2025large}. 
For example, \citet{ramjee2024cataractbot} developed an LLM-based chatbot named CataractBot that addresses patient inquiries regarding cataract surgery.

Overall, AI have demonstrated the potential to address the aforementioned challenges faced by caregivers of older adult patients in the ICU. 
However, there has been a lack of systematic investigation from the caregivers' perspective regarding their needs in accessing and understanding clinical information.
Moreover, how to design effective AI-based technologies to address caregivers' needs in accessing and understanding information remains underexplored.

% caregivers reflects a broader oversight in human-centered AI design: systems often fail to account for indirect end-users who lack formal medical training but bear responsibility for care decisions. 
%For example, AI tools designed to improve clinician-patient communication~\cite{ziqi2024} rarely consider how caregivers might misinterpret or operationalize AI-generated insights. 

% A growing body of work has also explored patient-centered AI technologies, such as symptom-tracking chatbots~\cite{Eunkyung2024}, personalized treatment recommendation systems, and tools to improve patient-provider communication~\cite{ziqi2024}.


% This disconnect underscores an urgent need to reconceptualize AI systems as caregiver-facing tools, capable of distilling complex data into contextually relevant, emotionally sensitive, and ethically transparent guidance tailored to caregivers’ literacy levels, cultural contexts, and decision-making autonomy.

% While these efforts emphasize clinical stakeholders and patients, they largely overlook the critical role of family caregivers, particularly in the ICU. 
% In ICU contexts, caregivers serve as intermediaries between clinicians and patients, advocating for care preferences, interpreting complex medical information, and making time-sensitive decisions under significant emotional duress. 
% Despite the critical role, their experiences and needs remain understudied in AI-based healthcare research.
% Few research address the sociotechnical challenges caregivers face, such as reconciling conflicting clinician advice, navigating fragmented health records, or translating jargon-heavy prognoses into actionable choices. 
% This gap is particularly problematic in ICUs, where caregivers  high-stakes information (e.g., prognostic uncertainty, treatment trade-offs) while coordinating with care provider teams.




% Researchers have successfully developed AI-based clinical pre-screening systems (e.g., for diagnostic triage)~\cite{wu2024clinical}, risk prediction algorithms (e.g., for sepsis risk prediction)~\cite{suresh2017clinical, yin2024sepsiscalc}, and automated information processing (e.g., EHR summarization)~\cite{Sette2023, Nikkhah2021, gopinath2020fast}. These innovations have primarily targeted healthcare providers to optimize clinical workflow efficiency and reduce clinicians' cognitive burdens in high-pressure environments~\cite{Yuexing2024}.
% Existing AI solutions in critical care settings prioritize clinical workflow optimization (e.g., predictive analytics for resource allocation)~\cite{Yuexing2024} and direct patient monitoring (e.g., AI-based vital sign analysis)~\cite{Zhang2024}.

\begin{figure}
    \centering
    \includegraphics[width=1\linewidth]{PICS/studyprocess.png}
    \caption{Study Procedure}
    \label{fig: study design}
\end{figure}