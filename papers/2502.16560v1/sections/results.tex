\section{Results and Discussion}
In this section, we apply our framework to the collected data, and present experiment results and discuss our findings.

% emotion distribution
\begin{figure}[t!]
    \centering
    \includegraphics[width=1\columnwidth]{figures/emotion_dist/pheme_emotion_cmt.pdf}
    \caption{PHEME Comment Emotion Distribution}
	\label{fig:pheme_emotion}
\end{figure}

\begin{figure}[t!]
    \centering
    \includegraphics[width=1\columnwidth]{figures/emotion_dist/twitter15_emotion_cmt.pdf}
    \caption{Twitter15 Comment Emotion Distribution}
	\label{fig:twitter15_emotion}
\end{figure}

\begin{figure}[t!]
    \centering
    \includegraphics[width=1\columnwidth]{figures/emotion_dist/twitter16_emotion_cmt.pdf}
    \caption{Twitter16 Comment Emotion Distribution}
	\label{fig:twitter16_emotion}
\end{figure}

% \begin{figure}[t!]
%     \centering
%     \includegraphics[width=1\columnwidth]{figures/emotion_dist/coaid_emotion_cmt.pdf}
%     \caption{CoAID Comment Emotion Distribution}
% 	\label{fig:coaid_emotion}
% \end{figure}

\paragraph{Emotion Polarity: Sentiment Valence}
We present the sentiment valence ordinal regression results in \Cref{tab:voc_results}. The numbers are balanced by random down-sampling, i.e.\ rumour and non-rumour, true rumour and false rumour both have equal numbers of posts. As shown in the table, sentiment in rumour root posts and comments is significantly more negative than that in non-rumours across all datasets and settings ($p<0.05$). This means both publishers and commenters engaged in the thread exhibit a more negative mindset towards rumour content. Compared with rumour posts at the root level, comment posts exhibit more negative sentiment for all datasets. Additionally, we break down the rumour data into true, false, and unverified rumours according to their original labels in the dataset. Interestingly, we found that unverified content exhibits more negative sentiment compared to both true and false rumours in the PHEME dataset, as well as in the root posts of Twitter15 and the U vs.\ T setting in Twitter16. Given that sentiment is more negative in comments and they form the main part of the conversation, we conduct the following experiments using only comments. 


\begin{table}[t]
\centering
\small
\begin{tabular}{ccccc}
\toprule
\textbf{Dataset} & \textbf{Type} & \textbf{Neg Emo} & \textbf{Neutral} & \textbf{Pos Emo} \\ 
\toprule
\multirow{2}{*}{PHEME}  
        & Ru  & \textbf{5894} & 1842 & 2731 \\
        & Non & 5650 & \textbf{1864} & \textbf{3728} \\ 
\multirow{2}{*}{Twitter15} 
        & Ru  & \textbf{785}  & 313  & 636 \\
        & Non & 674  & \textbf{349}  & \textbf{776} \\ 

\multirow{2}{*}{Twitter16} 
        & Ru  & \textbf{438}  & 176  & 300 \\
        & Non & 388  & \textbf{191}  & \textbf{406} \\ 

% \multirow{2}{*}{CoAID} 
%         & Ru & \textbf{294}  & 130  & \textbf{259} \\
%         & Non & 240  & \textbf{149}  & 200 \\ 
\bottomrule
\end{tabular}
\caption{Statistics of Negative (Neg Emo), Neutral, and Positive Emotions (Pos Emo) across the different datasets for rumour (Ru) and Non-rumour (Non) threads.}
\label{tab:emo_stats}
\end{table}


\paragraph{Emotion Distribution}
In order to further understand emotions expressed in rumour and non-rumour content, we present the emotion distribution results in \Cref{fig:pheme_emotion,fig:twitter15_emotion,fig:twitter16_emotion}. Overall, we observe a sharper distribution in emotions like anger, disgust, neutral, optimism, and joy. In the PHEME, Twitter15, and Twitter16 datasets, rumour posts tend to show more negative emotions such as anger, disgust, fear, and sadness, while non-rumour posts display more positive emotions like trust, optimism, joy and love. We present emotions statistics in \Cref{tab:emo_stats}.


% Besides these similarities, CoAID exhibits a slightly different pattern, with rumours showing more optimism and joy and non-rumours present more fear and sadness. 
% TJB: present a table of neg vs. neutral vs. positive emotions
% RX: add a table of neg vs. neutral vs. postitive emotion statistics, please see Table 3.


% emotion transition delta matrix
\begin{figure}[t!]
    \centering
    % \includegraphics[width=1\columnwidth]{figures/emotion_transition_diff/pheme_diff_transition.pdf}
    \includegraphics[width=1\columnwidth]{figures/emotion_transition_diff/pheme_diff_transition_noroot.pdf}
    \caption{PHEME rumour emotion transition matrix}
	\label{fig:pheme_transition_diff}
\end{figure}

\begin{figure}[t!]
    \centering
    % \includegraphics[width=1\columnwidth]{figures/emotion_transition_diff/twitter15_diff_transition.pdf}
    \includegraphics[width=1\columnwidth]{figures/emotion_transition_diff/twitter15_diff_transition_noroot.pdf}
    \caption{Twitter15 rumour emotion transition matrix}
	\label{fig:twitter15_transition_diff}
\end{figure}

\begin{figure}[t!]
    \centering
    % \includegraphics[width=1\columnwidth]{figures/emotion_transition_diff/twitter16_diff_transition.pdf}
    \includegraphics[width=1\columnwidth]{figures/emotion_transition_diff/twitter16_diff_transition_noroot.pdf}
    \caption{Twitter16 rumour emotion transition matrix}
	\label{fig:twitter16_transition_diff}
\end{figure}

% \begin{figure}[t!]
%     \centering
%     % \includegraphics[width=1\columnwidth]{figures/emotion_transition_diff/coaid_diff_transition.pdf}
%     \includegraphics[width=1\columnwidth]{figures/emotion_transition_diff/coaid_diff_transition_noroot.pdf}
%     \caption{CoAID rumour emotion transition matrix}
% 	\label{fig:coaid_transition_diff}
% \end{figure}

\begin{table}[h]
\centering
\begin{tabular}{cccc}
\toprule
\textbf{Emo Transit} & \textbf{PHEME} & \textbf{Twitter15} & \textbf{Twitter16} \\ 
\midrule
Neg $\rightarrow$ Neg & 0.05 & 0.05 & 0.07 \\ 
Neu $\rightarrow$ Neg & 0.02 & 0.06 & 0.01 \\ 
Pos $\rightarrow$ Neg & 0.07 & 0.06 & 0.24 \\ 
\midrule
Neg $\rightarrow$ Pos & -0.18 & 0.02 & -0.14 \\ 
Neu $\rightarrow$ Pos & -0.13 & 0.01 & -0.08 \\ 
Pos $\rightarrow$ Pos & -0.19 & 0.02 & -0.38 \\ 
\midrule
Neg $\rightarrow$ Neu & 0.03 & -0.20 & 0.03 \\ 
Neu $\rightarrow$ Neu & 0.08 & -0.16 & 0.07 \\ 
Pos $\rightarrow$ Neu & 0.13 & -0.19 & -0.05 \\ 
\bottomrule
\end{tabular}
\caption{Emotion Transition Delta values across datasets. Emo Transit represents transitions between emotional states, and shows the corresponding delta values for each dataset. Positive values indicate that the pattern occurs more frequently in rumour comments, while negative values mean they are more common in non-rumour comments.}
\label{tab:emo_transit_delta}
\end{table}


\begin{table}[h]
\centering
\small
\begin{tabular}{c cc cc cc}
\toprule
\textbf{Emo} & \multicolumn{2}{c}{\textbf{PHEME}} & \multicolumn{2}{c}{\textbf{Twitter15}} & \multicolumn{2}{c}{\textbf{Twitter16}} \\ 
 & \textbf{Ru} & \textbf{Non} &  \textbf{Ru} & \textbf{Non} &  \textbf{Ru} & \textbf{Non} \\ 
\midrule
Ang          & 320.50 & \textbf{342.31} & 41.18 & 41.75 & 32.92 & 25.33  \\
Disg        & 401.39 & \textbf{420.20} & \textbf{58.67} & 53.42  & \textbf{44.05} & 33.77  \\
Fear           & \textbf{154.48} & 128.42  & \textbf{13.95} & 7.01 & \textbf{6.56} & 4.82  \\
Sad        & \textbf{227.86} & 195.43 & \textbf{45.02} & 31.97  & \textbf{18.15} & 18.01  \\
Pess     & \textbf{49.04} & 37.88 & 5.26 & 5.28 & 3.06 & \textbf{4.32}  \\
Neu        & \textbf{408.02} & 378.33 & 63.69 & \textbf{75.41}  & 50.41 & 49.62 \\
Surp       & \textbf{38.79} & 30.30 & \textbf{14.94} & 9.52 & 5.48 & 5.84 \\
Antic   & 77.45 & \textbf{87.14} & \textbf{14.42} & 13.34 & 8.55 & 7.98 \\
Trust          & 26.85 & \textbf{33.94} &  4.16 & \textbf{5.45}  & 2.89 & 2.59 \\
Opti       & 124.37 & \textbf{178.74} & 24.80 & \textbf{32.19} & 15.52 & 20.10 \\
Joy            & 145.83 & \textbf{195.16} & 47.65 & \textbf{51.07}  & 24.75 & \textbf{33.98}  \\
Love           & 29.69 & \textbf{55.08}  & 10.97 & \textbf{12.88}  & 4.19 & \textbf{10.16}  \\
\bottomrule
\end{tabular}
\caption{Cumulative emotion regression coefficient across different datasets for rumour and non-rumour comments. Ang = Anger, Disg = Disgutst, Sad = Sadness, Pess = Pessimism, Neu =  Neutral, Surp = Surprise, Antic = Anticipation, Opti = Optimism. }
\label{tab:emotion_slope}
\end{table}


\paragraph{Emotion Transitions}
We present emotion transition results for each dataset in~\Cref{fig:pheme_transition_diff,fig:twitter15_transition_diff,fig:twitter16_transition_diff}. The computation was conducted as follows: for each emotion transition pair, we compute the probability based on pair frequency. In order to better reveal the gap between rumours and non-rumours, we define the difference of Emotion Transition (ET) probability as follows: 

\textbf{Emotion Transition (ET)}: Let's assume there are $N$ emotions ($N=12$ in our case), let $ET(i, j)$ represent the probability of transitioning from emotion $i$ (i.e.\ joy) to emotion $j$ (i.e.\ anger), where $0\leq i< N$ and $0\leq j< N$. This probability is calculated based on the frequency of all pairs that starts with emotion $i$.
\begin{equation}
    ET(i, j) = \frac{Freq(i, j)}{\Sigma_{k}^{N}Freq(i, k)}
\end{equation}

\textbf{Emotion Transition Delta ($\Delta ET$)} Define $\Delta ET(i, j)$ as the difference in emotion transition probabilities between rumours and non-rumours for the pair $(i, j)$:
\begin{equation}
    \Delta ET(i, j) = \frac{ET_\text{rumour}(i, j) - ET_\text{non-rumour}(i, j)}{ET_\text{rumour}(i, j)}
\end{equation}

Then we visualize it using a heatmap, e.g.\ in~\Cref{fig:twitter15_transition_diff}, the cell 0.55 in the last row of the third column has is dark red in color, indicating that the emotion transition pair (love, fear) appears more frequently in rumour than non-rumour comments in Twitter15. Overall, we observe larger emotion transition probability mass in positive--positive and negative--negative emotion transitions. This indicates that emotions are contagious, aligning with psychological findings~\citep{Goldenberg2019DigitalEC, Herrando2021EmotionalCA}. Contrasting rumour and non-rumour comments, we observe common patterns, namely that fear--fear and love--sadness are more common in rumour comments, and love--joy and love--optimism appear more frequently in non-rumour comments. We also see differences among datasets: Twitter15 has more anger response to almost all emotions more in non-rumour posts; Twitter16 has a lot of anger and disgust in response to positive emotions in rumours. We aggregate emotions into Negative, Neutral and Positive emotions in \Cref{tab:emo_transit_delta}. We observe positive values in emotion pairs where the transition ends with a negative emotion, indicating that discussions in rumours often trigger negative responses. On the contrary, negative delta values are observed in PHEME, Twitter16, suggesting non-rumours tend to prompt more positive responses.
% TJB: aggregate the emotions into negative vs. neutral vs. positive and perform higher-level analysis
% RX: I aggregated emotions into negative, neutral and postive and provided emotion transition delta in Table 5.


\begin{figure}[t!]
    \centering
    \includegraphics[width=1\columnwidth,scale=1]{figures/emotion_accum/pheme_rumour_accumulative.pdf}
    \includegraphics[width=1\columnwidth,scale=1]{figures/emotion_accum/pheme_non_accumulative.pdf}
    \caption{Cumulative Emotion Trajectory of PHEME.}
\label{fig:pheme_accum}
\end{figure}

\begin{figure}[t!]
    \centering
    \includegraphics[width=1\columnwidth,scale=1]{figures/emotion_accum/twitter15_rumour_accumulative.pdf}
    \includegraphics[width=1\columnwidth,scale=1]{figures/emotion_accum/twitter15_non_accumulative.pdf}
    \caption{Cumulative Emotion Trajectory of Twitter15.}
\label{fig:twitter15_accum}
\end{figure}

\begin{figure}[t!]
    \centering
    \includegraphics[width=1\columnwidth,scale=1]{figures/emotion_accum/twitter16_rumour_accumulative.pdf}
    \includegraphics[width=1\columnwidth,scale=1]{figures/emotion_accum/twitter16_non_accumulative.pdf}
    \caption{Cumulative Emotion Trajectory of Twitter16.}
\label{fig:twitter16_accum}
\end{figure}

\paragraph{Emotion Trajectory}
\Cref{fig:pheme_accum,fig:twitter15_accum,fig:twitter16_accum} illustrate the cumulative emotion across each dataset over time. At each chronological step, the counts represent the total number of observed emotions. Generally, we see a strong linear trend across datasets for all emotions. To better capture the rate of growth for each emotion, we apply linear regression and present the slopes in \Cref{tab:emotion_slope}. From the table, it is apparent that negative emotions tend to grow faster in rumour posts than in non-rumour posts across all datasets, while positive emotions grow faster in non-rumour posts. 
% CoAID shows a slightly different pattern that all emotions grow faster in rumour posts except for neutral. A possible explanation is that COVID-related rumours are often highly controversial, which can easily trigger emotional responses.


\paragraph{Causal Analysis}
The causal relationships revealed in \Cref{fig:causal graph} demonstrate several key patterns. First, we find out that the fact that a given thread is a rumour is not directly connected with other emotions. Specifically, the rumour has to rely on the emotion of surprise as a bridge to interact with other emotions, namely, $rumour \not \indep Fear | Surprise$ and $rumour \not \indep Anticipation | Surprise$. The change in the distribution of the rumour does not influence other emotions except surprise. This aligns with cognitive basis of rumour transmission, where surprising or counterintuitive information tends to capture attention and facilitate rumour spreading~\citep{Knapp1944APO,llport1947ThePO}.
% TJB: need ref
% RX: added
Second, pessimism is primarily influenced by negative emotions (sadness and fear), while optimism is causally influenced by positive emotions (joy, love, and trust). Notably, there's undirected edge between anger and disgust, this relationship aligns with our previous findings that both rumour and non-rumour posts exhibit intense expressions of these emotions.


\begin{figure}[t!]
    \centering
    \includegraphics[width=1\columnwidth,scale=1]{figures/causal_analysis/causal_graph.pdf}
    \caption{Causal graph of is\_rumour and emotions. Arrows represents the causal relationships. Orange emotions represent positive emotions, green emotions are negative emotions. Surprise serves as a bridge between is\_rumour and other emotions in this context and is depicted in light blue.}
\label{fig:causal graph}
\end{figure}

% what is \ex here?
There are also a few counterintuitive findings, including sadness leading to joy, joy causing pessimism, and the causal relationship between disgust and love. We conducted a qualitative analysis of the 50 samples and found there are several possible reasons for this: (1) there is a complex interplay of emotions in social media interactions, where emotional responses are shaped by context and individual perspectives. For example, we had one response where joy was detected, ``this tweet gives me hope that she may write an eighth'' to the post ``once again, jk rowling is not working on an eighth harry potter book.'' where sadness is detected. Sadness can sometimes lead to joy when people use humor or shared memories to find solace in sadness, which serves as a coping mechanism for processing uncomfortable or shocking topics. Similarly, expressions of joy can paradoxically evoke pessimism in certain contexts, as the same post can be interpreted in vastly different ways depending on the readers' emotional state, cultural background, or personal experiences. The dual causal relationship between disgust and love further emphasizes the complexity of emotions expressed in text. A post that initially provokes disgust might also elicit admiration or affection when audiences recognize an underlying message of authenticity, vulnerability, or humor. (2) Moreover, the analysis reveals challenges in accurately interpreting emotions through automatic labeling methods. Sarcasm and humor are frequently misclassified, with sarcasm often mistaken for joy due to its seemingly optimistic wording. The lack of contextual information leads to noise and inaccuracies in emotional categorizations. (3) The complexity of social media participants also contributes to this. Some users engage with posts for self-serving purposes, such as promoting their brand or gaining visibility, rather than genuinely responding to the content. These interactions together add a layer of noise to the results, making it even more challenging.


%%% Local Variables:
%%% mode: latex
%%% TeX-master: "../main_anonymous"
%%% End:
