\section{Causal Analysis of Emotion in Rumour}

To gain deeper insights into the relationship between rumors and the emotions underlying them, we extend our analysis beyond statistical correlations by conducting a causal analysis.
Specifically, we conduct the PC algorithm \cite{Spirtes2000}, a classical constraint-based causal discovery algorithm on the merged datasets including all three datasets we evaluated in this paper. 
Figure~\ref{fig:causal graph} presents the resulting causal graph.
The following section first provides a general introduction to the PC algorithm, followed by a detailed analysis of the relationships between different emotions and rumors based on the causal graph in Figure~\ref{fig:causal graph}.


Uncovering causal relations of interested variables is never an easy problem. 
The whole causal discovery field is developed for this task, aiming at looking for the true causal structure by utilizing observed data. 
Under the fundamental assumption \textit{causal Markov condition} that a variable is conditional independent of all its noneffects given its direct cause, the \textit{faithfulness} ensures that the casual graph exactly encodes the independence and conditional independence relations among variables. 
In one sentence, those two assumptions allow us to infer causal relationships from observed statistical independencies, forming the cornerstone of constraint-based causal discovery methods, e.g., PC algorithm used in this paper. The next section provides a brief introduction to this algorithm.


The PC algorithm identifies causal relationships among the variables of interested, represented as a Directed Acyclic Graph (DAG) by numerating the independence and conditional independence relationship. The algorithm consists of two main steps: 
\begin{enumerate}
    \item \textbf{Skeleton Identification}: It starts with a complete undirected graph where all variables are connected. Then, edges are iteratively removed based on conditional independence and independence relationships among variables, inferred by conditional independence test. This step will return a undirected graph, we call it skeleton. 
    \item \textbf{Edge Orientation}: After constructing the skeleton, it orients edges by a set of predefined rules (Meek's Rule \cite{meek1997graphical}) by avoiding cycles and orienting collider structures.
\end{enumerate}

The complete procedure of PC algorithm is provided in algorithm \ref{pc}. It will return a  Completed Partially Directed Acyclic Graph (CPDAG), which represents an equivalence class of causal graphs that are consistent with the observed data’s independence and conditional independence relations. In our implementation, we adpot  Fisher-z test \cite{fisher_probable_1921} to infer the conditional independence relations. 


\begin{algorithm}[ht!] 
\caption{PC Algorithm}
\label{pc}
\begin{algorithmic}[1] 
\State \textbf{Input:} Data $\mathbf{X}$, significance level $\alpha$
\State \textbf{Output:} Completed Partially Directed Acyclic Graph (CPDAG)

\State Initialize a complete undirected graph $G$ with all variables as nodes.

\State \textbf{Step 1: Skeleton Identification}
\For{each pair of variables $(X, Y)$ in $G$}
    \State Find the subset $S \subseteq \text{Adj}(X, G) \setminus \{Y\}$ such that 
    $X \indep Y \mid S$ with significance $\alpha$.
    \If{such a subset $S$ exists}
        \State Remove the edge $X - Y$ from $G$.
    \EndIf
\EndFor

\State \textbf{Step 2: Edge Orientation}
\For{each triple of variables $(X, Y, Z)$ in $G$ where $X - Z - Y$ and $X, Y$ are not adjacent}
    \If{$Z \notin S$ for all separating sets $S$ for $X$ and $Y$}
        \State Orient as $X \to Z \leftarrow Y$ (identify a collider).
    \EndIf
\EndFor

\While{possible}
    \For{each edge $(X - Y)$ in $G$}
        \If{there exists a directed path $X \to \dots \to Z$ such that $Z - Y$}
            \State Orient as $X \to Y$ (acyclicity rule).
        \ElsIf{orienting $X - Y$ as $X \to Y$ creates a new v-structure}
            \State Orient as $X \to Y$ (v-structure rule).
        \EndIf
    \EndFor
\EndWhile

\State \textbf{return} The CPDAG representing the equivalence class of causal graphs.

\end{algorithmic}
\end{algorithm}


\begin{figure}[t!]
    \centering
    \includegraphics[width=1\columnwidth,scale=1]{figures/causal_analysis/sample_graph.png}
    \caption{This is just a sample graph}
\label{fig:causal graph}
\end{figure}

From the resulting causal graph illustrated in Figure~\ref{fig:causal graph}, interestingly, we find out that the rumour is not directly connected with other emotions. Specifically, the "rumour" has to rely on the emotion "surprise" as a bridge to interact with other emotions. Formally, $Rumour \not \indep Fear | Surprise$ and $Rumour \not \indep Anticipation | Surprise$. 
One should note that the arrow in Figure~\ref{fig:causal graph} represents the causal relationship, i.e., the distribution change of rumour will not influence other emotions except surprise. 
This result is ... ( Some arugments here to support this findings). 
