\section{Related Work}
% TODO can add more automatic emotion detection methods in rumour and say we are using EmoLLM which is trained on tweets dataset and generally perform better in multiple emotion-detection related tasks
The definition of rumour is generally complicated and varies from one publication to another. Some early work treated rumour as information that is false~\citep{rumour_chinese2014}. Most recent definitions of rumours are ``unverified and instrumentally relevant information statements in circulation''~\citep{DiFonzo_Bordia_2007} and ``unverified information at the time of the posting''. This definition also aligns with the concept in recent work~\citep{rumour_survey2018, pheme2015, tian-etal-2022-duck} and the Oxford English Dictionary, which defines the rumour as ``an unverified or unconfirmed statement or report circulating in a community''.\footnote{https://www.oed.com/dictionary/rumour\_n?tab=meaning\_and\_use}

Existing research highlights the significant role of emotions in understanding general misinformation, mostly fake news. Research has found relationships exist between negative sentiment and fake news, and between positive sentiment and genuine news~\citep{Zaeem2020OnSO}. Fake news also expresses a higher level of overall emotion, negative emotion, and anger than real news~\citep{Zhou2022DoesFN}. Negative emotions like sadness and anger can serve as indicators of misinformation~\citep{Prabhala2019DoED}. The role of emotions in rumours has been recognized since the Second World War, reflecting the interactive and community-driven nature of rumour spreading. Knapp’s taxonomy~\citep{Knapp1944APO} of rumours categorizes them into three types, each deeply embedded with emotions: (1) ‘pipedream’ rumours, which evoke wishful thinking; (2) ‘bogy’ rumours, which heighten anxiety or fear; and (3) ‘wedge-driving’ rumours, which incite hatred. This taxonomy underscores how rumours are inherently embedded with emotional undercurrents.”

Recent research on emotion in rumours largely focuses on their role in spreading behaviour, some studies have used questionnaires to gather participants’ reactions to specific rumours~\citep{Zhang2022EMOTIONALCI,RIJO2023107619, ALI2022107307}, while others have employed cascade size and lifespan as indicators~\cite{Prllochs2021EmotionsIO,Prllochs2021EmotionsED}. Key findings of such work include: rumours conveying anticipation, anger, trust, or offensiveness tend to generate more shares, have longer lifespans, and exhibit higher virality~\citep{Prllochs2021EmotionsIO}. Additionally, false rumours containing a high proportion of terms reflecting positive sentiment, trust, anticipation, anger, or condemnation are more likely to go viral~\citep{solovev2022moralemotionsshapevirality,Prllochs2021EmotionsED}. However, existing research has notable gaps: it often focuses on isolated aspects of emotions in rumours, primarily identifies correlations rather than causality, and tends to examine rumour data in isolation. To address these gaps, we aim to propose a comprehensive emotional analytical framework that integrates multiple emotion-related tasks, contrasting rumour and non-rumour content with the aim to enhance our understanding of emotions and, ultimately, to improve rumour detection in online social media.



