\section{Introduction}
% Intro framework
% rumour definitions, influence and impacts rumours bring to human society
% some other factors that used to investigate rumours
% important cognitive, social aspects of rumours that relates to emotion (why emotion is important, especially in online rumour)
% existing work: virality, cascades, also represent user's belief in rumours; emotion as an attribute in detecting rumours
% gap: lack of emotion changes and contrast rumours and non-rumours, also lack of causal understanding of emotions in the rumours
% we propose analytical work and automatic detection work that treat emotion

In part due to the ubiquity of edge devices like mobile phones, the major of the world's population now has access to the internet.\footnote{https://datareportal.com/reports/digital-2024-october-global-statshot} This increasing ease of access and interaction through online social media has brought both opportunities and challenges. One significant challenge is the rapid spread of rumours. Rumours on online social media have become a major threat to society~\citep{tian-etal-2022-duck,pheme2015,kochkina-etal-2018-one,ma-etal-2017-detect}. The circulation of unsubstantiated rumours has impacted a large group of people, with consequences ranging from seeding skepticism and discrediting science, to endangering public health and safety. For example, during COVID-19, an Arizona man died, and his wife was hospitalized after ingesting a form of chloroquine in an attempt to prevent the disease. Additionally, 77 cell phone towers were set on fire due to conspiracy theories linking 5G networks to the spread of COVID-19~\citep{coaid}. Recent advancements in Large Language Models (LLM) and generative AI have exacerbated this phenomenon, creating an urgent need to understand and better deal with rumours on social media~\citep{chen2024combatingmisinformation}. 

Previous research has highlighted several factors driving the spread of rumours on social media~\citep{emotion_dynamics}. These factors often relate to the characteristics of publishers; for instance, users with more followers can reach wider audiences, and the number of reshares and likes reflects users’ beliefs and attitudes toward a post~\citep{Zaman2013ABA, Vosoughi2018TheSO}. Other studies have focused on the online diffusion of specific topics, such as elections or disasters~\citep{Starbird2017ExaminingTA,DeDomenico2013TheAO}, and other harmful online social contents~\citep{Aleksandric2024UsersBA}.
% TJB: add ref
% RX: added

Emotions have a strong influence on human behavior in both offline and online settings~\citep{emotion_dynamics,Herrando2021EmotionalCA,Ekman1992AnAF}. They shape the type of information users seek, how they process and remember it, and the judgments and decisions they make. Misinformation is often associated with high-arousal emotions such as anger, sadness, anxiety, surprise, and fear~\citep{liu2024emosurvey}. Rumours conveying these emotions are more likely to generate higher numbers of shares and exhibit long-lived, viral patterns~\citep{Prllochs2021EmotionsIO}. However, existing research on emotions in misinformation analysis is fragmented, focusing primarily on the emotions within the original rumour posts themselves and often overlooking the comparative differences between rumours and non-rumours~\citep{Ferrara_2015,emotion_dynamics}. In this work, we address this gap by introducing a systematic analysis of various aspects of emotion, contrasting the emotional patterns in rumour and non-rumour content on social media. Using popular NLP rumour detection datasets, we provide new insights into the emotional dynamics of rumour and non-rumour data.


%%% Local Variables:
%%% mode: latex
%%% TeX-master: "../main_anonymous"
%%% End:
