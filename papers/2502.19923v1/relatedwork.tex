\section{Related Work and Discussion}
\label{relatedwork}
In our work, we have outlined a series of phenomena inherent to the iterative application of Bellman operators. 
Crucial was the fact that the update coefficients were non-negative.

Notice that any PAM on the domain~$[0,1]^d$ can be represented as a 
nested min-max of its affine components~\cite{Ovchinnikov2002maxmin,Gorokhovik1994maxmin}.
This supports the relevance of PAMs with updates defined by the maximum of affine pieces, 
Bellman operators of MDPs being among them.
It is worth pointing out that we capture a large class of PAMs 
for which the assumptions of the known techniques do not hold.
In proving decidability for~$d=2$,
we do not impose the restrictions used in works on the one-dimensional version of the problem.
Bellman operators are, in general, neither injective as in~\cite{ghahremani2023injective} nor complete~\cite{bournez2018onedpam}, nor even surjective.

Our techniques and results can be applied to other problems about PAMs.
Consider, in particular, the universally quantified version of~BOR:
given a vector $\bm{t} \in [0,1]^d$ and a Bellman operator~$\Phi: [0,1]^d \rightarrow [0,1]^d$,
does there exist~$n \in \NN$ \emph{for every}~$\bm{s} \in [0,1]^d$ such that $\Phi^n(\bm{s}) = \bm{t}$?
This is the mortality problem, known to be undecidable for general PAMs in dimension~2~\cite{BenAmram2015mortal,blondel2001note}.
For Bellman operators, however, it is equivalent to solving the BOR problem for $\bm{s} = \bm{0}$ and for $\bm{s} = \bm{1}$.
Indeed, both BOR instances are ``yes''-instances if and only if every instance with $\bm{0} \leq \bm{s} \leq \bm{1}$ is a~``yes''.
We can answer this using our novel algorithm (\cref{compar-decid}), hence mortality for Bellman operators is decidable in all dimensions.

Without restrictions on the dimension~$d$,
the decidability of the BOR problem remains open.
Notably, if there exist states with multiple tight actions,
then the dynamics of the sequence $\langle F^n(\eps)\rangle_{n\in \NN}$ as defined early in~\cref{sec:2d} is intricate for~$\eps \bowtie \bm{0}$.
For more works discussing the iterative dynamics of map~$F$, see~\cite{Sladky1980max,Zijm1984max}.
We highlight that
$F^n(\eps)$ does not have a closed form, in contrast to $M^n\eps$ for a fixed matrix~$M$.
While the entries of the vector~$M^n\eps$ are terms of linear recurrence sequences, 
the class of such sequences is not closed under max~\cite{Heerdt2018maxlrs}.
The behaviour of entries in~$F^n(\eps)$ is even subtler than that, as 
witnessed by the non-positionality discussion and~\cref{ex:3d}.


We emphasise that due to the undecidability of the vector reachability problem
for matrix semigroups~\cite{Bell2008undec},
we need to argue about ultimate non-positivity of~$F^n(\eps)$, see~\cref{not-just-reach}.
However, even for one~$M$, deciding whether $(M^n\eps)_1 \leq 0$ for some~$n$ is equivalent to the positivity problem~\cite{Ouaknine2014pos5}, open in dimension $d > 5$.
This hardness carries on to the stochastic matrices~\cite{Mihir2024ergodic}.
Moreover, the question whether there exists~$n$ 
such that $M^n\eps \leq \bm{0}$
corresponds to the polyhedron-hitting problem, and is Diophantine-hard~\cite{ChonevOW15} for general matrices~$M$.