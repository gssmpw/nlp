\section{Introduction}
Solving reachability problems is central to formal verification, but it is challenging, as evidenced by a body of work that has been pursued for decades.
Specifically, the reachability problem that we are interested in asks the following question: 
Given two vectors $\bm{s}, \bm{t}$ and a map $f$, is there $n\in  \NN$ such that $f^n(\bm{s}) = \bm{t}$? 
One of the seminal results is given by Kannan and Lipton~\cite{kannan1980decid,kannan1986poly}.
They answer the decidability question in the affirmative---by presenting a novel polynomial-time algorithm---for \emph{affine maps} $f$ and vectors $\bm{s}, \bm{t}$ with rational coefficients.

Unfortunately, the reachability problem becomes undecidable by slightly extending the class of maps beyond the affine maps---\emph{piecewise affine maps (PAMs)}. 
Koiran et al.~\cite{koiran1994pam} show that the reachability problem for PAMs is undecidable even in the two-dimensional space, 
which witnesses the significant difficulty of the problem, compared to that of affine maps. 
Specifically, a PAM~$f$ on the domain~$\mathcal{D}$ is a function with the property that
for some family~$\{P_1, \dots, P_k\}$ of sets 
such that $P_1 \cup \dots \cup P_k = \mathcal{D}$, 
the restriction of~$f$ to each~$P_i$ is an affine function.
We consider piecewise affine maps over the domain~$\mathcal{D} = [0,1]^d$, where $[0,1]$ is the unit interval.
Each of the finitely many \emph{pieces}~$P_1, \dots, P_k$ is defined by a conjunction of finitely many linear inequalities.

\begin{example}\label{ex:intro}
	Consider an example of a PAM in the dimension~$d=2$. Let
\(f : [0,1]\times[0,1] \rightarrow  [0,1]\times[0,1]\) be defined by $f(x_1, x_2) = (x_1', x_2')$ with
\begin{equation*}
	x_1' = \frac{1}{2}x_2 + \frac{1}{3}, \hspace{4cm}
	x_2' =	\begin{cases}
		\frac{1}{2}x_1 + \frac{1}{2},\quad &\text{ if $x_1 \geq x_2$},\\
		\frac{1}{4}x_1 + \frac{1}{4}x_2 + \frac{1}{2}, \quad &\text{ if $x_1 < x_2$.}
	\end{cases}
\end{equation*}
Here, $P_1$ is defined as $\{\bm{x} = (x_1, x_2) : x_1 \geq x_2\}$ and $P_2$ as $\{\bm{x} = (x_1, x_2) : x_1 < x_2\}$,
where according to the definition $P_1 \cup P_2 = \mathcal{D}$.
\end{example}

In fact, piecewise affine maps characterise a very rich mathematical object~\cite{Bell2008undec,blondel2000survey}.
The reachability problem for PAMs is thus one of a series of challenging problems
whose crux lies in the unpredictable behaviour of iterative maps 
and corresponding discrete-time dynamical systems, see also~\cite{Bell2008undec,ChonevOW15,Karimov2022dynsys,Tiwari2005termination}.

A natural question might be: \emph{When does the reachability problem for PAMs become decidable?}	
Indeed, the reachability problem for PAMs on a unit interval (dimension one) has received attention in the recent research~\cite{bournez2018onedpam,ghahremani2023injective,kuijpers2021skew},
where classes of PAMs with decidable reachability problems have been found
while fostering new elaborate techniques.
Yet, the decidability of the general one-dimensional reachability problem for PAMs remains open,
even for maps defined with two pieces.

\subparagraph{Our Approach.}
In this work, we propose an orthogonal approach to investigate the challenges behind the reachability problem for PAMs, focusing not on the restriction of the \emph{dimension} or the number of \emph{pieces},
but on the restriction of the \emph{structure} of PAMs.
Specifically, we consider the reachability problem for the \emph{Bellman operators} on \emph{Markov decision processes (MDPs)}---Bellman operators are, in fact, PAMs that have been studied mostly in the context of software verification or reinforcement learning~\cite{bk,bellman1957mdp,Puterman94}.
For instance, a PAM in~\cref{ex:intro} is a Bellman operator.

MDPs are a standard probabilistic model for systems with uncertainties, and the least fixed points of the Bellman operators $\Phi\colon [0, 1]^{d}\rightarrow [0, 1]^{d}$ represent the ``optimal'' reachability probability to the specific target state. 
Here, optimal means that the maximum reachability probability is induced by a \emph{scheduler} that resolves the non-deterministic behaviour on MDPs. 
We formulate our target problem as follows: Given two vectors $\bm{s}, \bm{t}\in [0, 1]^{d}$ and a Bellman operator $\Phi\colon [0, 1]^{d}\rightarrow [0, 1]^{d}$, is there $n\in \NN$ such that $\Phi^{n}(\bm{s}) = \bm{t}$?
We refer to this problem as the \emph{reachability problem for Bellman operators} (or BOR, for Bellman Operator Reachability).

Under a reasonable assumption, the Bellman operator $\Phi$ is \emph{contracting}~\cite{HaddadM18}, which means that from any vector $\bm{s}$, the sequence $\langle\Phi^{n}(\bm{s})\rangle_{n\in \NN}$ converges to the unique fixed point $\mu\Phi$. 
Here, the unique fixed point is precisely the vector of optimal reachability probabilities from each state to the target state.
The iterative procedure where~$\mu\Phi$ is approximated by applying~$\Phi$ is referred to as~\emph{value iteration} and is widely studied~\cite{bk,Baier0L0W17,chatterjee2008vi}.
Note that the unique fixed point is computable in polynomial-time by linear programming~\cite{bk}, so for our problem we can assume that we know the unique fixed point $\mu\Phi$ a priori. 
Nevertheless, even a convergent sequence does not generally reach~$\mu\Phi$ at any~$n$.
In fact, the question of the reachability to the fixed point (in finite time) is known in both theoretical computer science and software verification communities.
A case in point is the discussion in~\cite{koiran1994pam}, where this question is explicitly asked for one-dimensional PAMs,
see also~\cite{blondel2001note}.
Furthermore, in a value iteration survey~\cite{chatterjee2008vi} the same property is listed.
While the authors observe that fixed points are not reachable in general, they do not discuss the decidability aspect.
In the present paper, we investigate the decidability of reaching~$\bm{t} = \mu\Phi$ as part of our problem.

\subparagraph{Contributions.}
We present some decidability results for our target problem under the standard assumption that ensures the existence of the unique fixed point $\mu\Phi$ of the Bellman operators~\cite{HaddadM18}. 
First, we show that the reachability problem for Bellman operators is decidable for any dimension if the target vector $\bm{t}$ does not coincide with $\mu\Phi$ ($\bm{t}\not = \mu\Phi$). 
This is true due to the contraction property of the Bellman operator. 
It becomes rather non-trivial when $\bm{t} = \mu\Phi$, that is, for the reachability problem to the unique fixed point $\mu\Phi$. 
We show that the reachability problem for Bellman operators when $\bm{t} = \mu\Phi$ is decidable for any dimension if $\bm{s}$ is comparable to $\mu\Phi$, that is, either $\bm{s}\leq \mu\Phi$ or $\mu\Phi \leq \bm{s}$ holds for the pointwise order.  
The crux is to show that eventually only ``optimal'' actions are chosen, and we reduce the reachability problem to a simple qualitative reachability problem that can be shown decidable. 

Finally, we address the remaining case: $\bm{t} = \mu\Phi$ and $\bm{s}$ is incomparable to $\mu\Phi$. 
In dimension~$2$, we show an algorithmic procedure also for this case---
finding the last piece of the puzzle---the reachability problem for two-dimensional Bellman operators is thus decidable. 
Our argument is based on the equivalent problem for matrix semigroups, and 
our proof exploits a natural total order on the actions, which exists for the two-dimensional case. 
To the best of our knowledge, this is the first result to give a reasonably large class of PAMs for which the reachability problem is decidable in the two-dimensional case.

\subparagraph{Organization.}
We outline our paper as follows. 
\begin{itemize}
	\item In~\cref{sec:prelim}, we formally define the main objects of our study, MDPs and Bellman operators, and recall some known properties. 
	\item In~\cref{sec:efs}, we show that the reachability problem for Bellman operators is decidable when either $\bm{t} \not = \mu\Phi$ (\cref{t-not-fp}) or $\bm{s}$ is comparable to $\bm{t} = \mu\Phi$ (\cref{compar-decid}). 
	Importantly, the result holds for arbitrary dimension. 
	\item In~\cref{sec:2d}, we prove that the reachability problem for Bellman operators is decidable in the two-dimensional case (\cref{2d-efs}), by presenting an algorithm that solves the remaining case ($\bm{s}$ is incomparable to $\bm{t} = \mu\Phi$). 
	\item In~\cref{relatedwork}, we discuss our result with related work, and list some future directions. 
\end{itemize}