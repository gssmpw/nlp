\section{Related work}
\subsection{Spoken human-robot interaction}

% Where are they used?
When interacting with robots, their embodiment and the situated nature of the interaction add additional requirements to the dialogue system \cite{jokinen2018dialogue}. Various approaches for dialogue management have been used for human-robot spoken interaction in recent years, ranging from hand-crafted to data-driven approaches, or a hybrid combination of both \cite{reimann2024survey}. Although many of the robots used for spoken HRI are humanoids, other morphologies are also used for speech-based interactions \cite{reimann2024survey}. %These can range from functional robots \cite{gervits2021classification,amiri2019augmenting}, over wheelchairs \cite{hemachandra2015information,doshi2007efficient} to robotic heads \cite{campos2018challenges}, arms \cite{she2014back}, and android robots \cite{glas2016erica}. 
The morphology of the robot plays a role in how people perceive and evaluate it \cite{kunold2023not}.

For spoken human-robot interaction, speech recognition is still a challenge, especially if the robot is deployed in noisy environments or is making noise by moving. Strategies that are used to mitigate that effect, like the use of wakewords, are influencing the user behavior \cite{wen2023fresh}.
Moore \cite{moore2015talking} argues that just hearing and producing speech is not enough to create intelligent communicative robots, but that understanding is crucial.
As robots become more capable and are used in more complex scenarios, the requirements for spoken interaction increase as well \cite{marge2022spoken}. Especially with Large Language Models (LLMs) becoming more common, this can influence the user's expectations of the robot's capabilities.


\subsection{Transparency in human-robot interaction}

Tulli and colleagues \cite{Tulli_Correia_Mascarenhas_Gomes_Melo_Paiva_2019} define transparency as ``an appropriate mutual understanding and trust that leads to effective collaboration between humans and agents.'' 
Transparent systems often focus on making the robot's behavior understandable to the user \cite{Schött_Amin_Butz_2023} and are used in various domains, such as teaching robots \cite{Hirschmanner_Gross_Zafari_Krenn_Neubarth_Vincze_2021} and healthcare \cite{Nesset_Robb_Lopes_Hastie_2021}. Explainability is used to make systems more understandable and communicate their capabilities, by providing explanations of the shown behavior. Capability communication is a form of self-explanation, which aims at enhancing the transparency of a system.
The types of explanations people prefer vary between situations, with why-explanations being highly rated for most scenarios \cite{Wachowiak_Fenn_Kamran_Coles_Celiktutan_Canal_2024}.
Kulesza et al. \cite{Kulesza_Stumpf_Burnett_Yang_Kwan_Wong_2013} found that the completeness of the explanations helps users form more accurate mental models.
The use of different (un)solicited pre-/post-corrective communication strategies does impact the level of trust users have if failures happen \cite{ye2019human}.

Although the efficiency of the interactions depends on the transparency \cite{lyons2013being}, it is still unclear which level of transparency is the best \cite{Bhaskara_Skinner_Loft_2020}, and which one is needed to not hinder the interaction through too much transparency \cite{fischer2018transparent}.
For the evaluation of transparent systems, trust, and performance are commonly used metrics \cite{Schött_Amin_Butz_2023,hoffman2023measures}. For the perceived usability of transparent systems, the literature shows inconsistent conclusions, with some studies indicating an increase in usability, while others do not \cite{Bhaskara_Skinner_Loft_2020}.

Explanation generation is especially important in the context of failures, where it can be used to explain the failure to the user \cite{diehl2022did, das2021explainable, khanna2023user, lemasurier2024reactive}. Those failure explanations should be understandable also by non-experts who are using the robot \cite{das2021explainable}. The explanations can include various levels of detail, by including a possible failure resolution, the cause of the failure, and examples of previous successful actions \cite{khanna2023user}. Different strategies have been used to deal with failures also in non-conversational ways \cite{reig2021flailing}. In contrast to the work on failures, we are not looking at failure resolution strategies, but at communication of capabilities in general.