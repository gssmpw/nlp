\section{Related work}
\subsection{Spoken human-robot interaction}

% Where are they used?
When interacting with robots, their embodiment and the situated nature of the interaction add additional requirements to the dialogue system ____. Various approaches for dialogue management have been used for human-robot spoken interaction in recent years, ranging from hand-crafted to data-driven approaches, or a hybrid combination of both ____. Although many of the robots used for spoken HRI are humanoids, other morphologies are also used for speech-based interactions ____. %These can range from functional robots ____, over wheelchairs ____ to robotic heads ____, arms ____, and android robots ____. 
The morphology of the robot plays a role in how people perceive and evaluate it ____.

For spoken human-robot interaction, speech recognition is still a challenge, especially if the robot is deployed in noisy environments or is making noise by moving. Strategies that are used to mitigate that effect, like the use of wakewords, are influencing the user behavior ____.
Moore ____ argues that just hearing and producing speech is not enough to create intelligent communicative robots, but that understanding is crucial.
As robots become more capable and are used in more complex scenarios, the requirements for spoken interaction increase as well ____. Especially with Large Language Models (LLMs) becoming more common, this can influence the user's expectations of the robot's capabilities.


\subsection{Transparency in human-robot interaction}

Tulli and colleagues ____ define transparency as ``an appropriate mutual understanding and trust that leads to effective collaboration between humans and agents.'' 
Transparent systems often focus on making the robot's behavior understandable to the user ____ and are used in various domains, such as teaching robots ____ and healthcare ____. Explainability is used to make systems more understandable and communicate their capabilities, by providing explanations of the shown behavior. Capability communication is a form of self-explanation, which aims at enhancing the transparency of a system.
The types of explanations people prefer vary between situations, with why-explanations being highly rated for most scenarios ____.
Kulesza et al. ____ found that the completeness of the explanations helps users form more accurate mental models.
The use of different (un)solicited pre-/post-corrective communication strategies does impact the level of trust users have if failures happen ____.

Although the efficiency of the interactions depends on the transparency ____, it is still unclear which level of transparency is the best ____, and which one is needed to not hinder the interaction through too much transparency ____.
For the evaluation of transparent systems, trust, and performance are commonly used metrics ____. For the perceived usability of transparent systems, the literature shows inconsistent conclusions, with some studies indicating an increase in usability, while others do not ____.

Explanation generation is especially important in the context of failures, where it can be used to explain the failure to the user ____. Those failure explanations should be understandable also by non-experts who are using the robot ____. The explanations can include various levels of detail, by including a possible failure resolution, the cause of the failure, and examples of previous successful actions ____. Different strategies have been used to deal with failures also in non-conversational ways ____. In contrast to the work on failures, we are not looking at failure resolution strategies, but at communication of capabilities in general.