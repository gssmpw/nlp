\documentclass[journal]{IEEEtran}

\ifCLASSINFOpdf
\else
   \usepackage[dvips]{graphicx}
\fi
\usepackage{url}
\usepackage{amsmath}
\usepackage{amssymb}
\usepackage{booktabs}
\usepackage{multirow}
\usepackage{color}
\hyphenation{op-tical net-works semi-conduc-tor}

\newcommand{\figref}[1]{Fig.\ref{#1}}
\newcommand{\tabref}[1]{Tab.\ref{#1}}
\newcommand{\equref}[1]{Eqn.(\ref{#1})}
\newcommand{\secref}[1]{Sec.\ref{#1}}
\newcommand{\eg}{\textit{e}.\textit{g}.}
\newcommand{\ie}{\textit{i}.\textit{e}.}
\newcommand{\etal}{\textit{et al}.}
\newcommand{\best}[1]{\textbf{#1}}   % Bold text for best
\newcommand{\sbest}[1]{\underline{#1}} % Underline for second best
\newcommand{\tbest}[1]{\textit{#1}}   % Italics for third best

\newcommand{\xcy}[1]{{\color{black} {#1}}}
\newcommand{\hbz}[1]{{\color{black} {#1}}}
\usepackage{graphicx}


\begin{document}

\title{Adapting Human Mesh Recovery with Vision-Language Feedback}

\author{Chongyang Xu, Buzhen Huang, Chengfang Zhang, Ziliang Feng, Yangang Wang
% \thanks{Manuscript submitted December 15, 2024;}
\thanks{This work was supported by the Sichuan Science and Technology Program (No. 2024NSFSC2046).}
\thanks{Chongyang Xu and Ziliang Feng are with the College of Computer Science, Sichuan University, Chengdu 610065, China.}
\thanks{Buzhen Huang and Yangang Wang are with the Key Laboratory of Measurement and Control of Complex Systems of Engineering, Ministry of Education, and the
School of Automation, Southeast University, Nanjing 210096, China.
}
% \thanks{Wentao Tang is with the College of Computer Science, Nankai University, Tianjin, 300071, China.}
\thanks{Chengfang Zhang is with the Intelligent Policing Key Laboratory of Sichuan Province, Sichuan Police College, Luzhou, 646000, China.}
\thanks{Corresponding author: Yangang Wang. E-mail: yangangwang@seu.edu.cn.}
}

\maketitle

% \section{Introduction}

% \IEEEPARstart{T}{his} document is a template for \LaTeX. If you are reading a paper or PDF version of this document, please download the electronic file, trans\_jour.tex, from the IEEE Web site at \url{http://www.ieee.org/authortools/trans_jour.tex} so you can use it to prepare your manuscript. If you would prefer to use LaTeX, download IEEE's LaTeX style and sample files from the same Web page. You can also explore using the Overleaf editor at {https://www.overleaf.com/blog/278-how-to-use-overleaf-with-ieee-collabratec-your-quick-guide-to-getting-started\#.xsVp6tpPkrKM9}

\begin{abstract}

Human mesh recovery can be approached using either regression-based or optimization-based methods. Regression models achieve high pose accuracy but struggle with model-to-image alignment due to the lack of explicit 2D-3D correspondences. In contrast, optimization-based methods align 3D models to 2D observations but are prone to local minima and depth ambiguity. In this work, we leverage large vision-language models (VLMs) to generate interactive body part descriptions, which serve as implicit constraints to enhance 3D perception and limit the optimization space. Specifically, we formulate monocular human mesh recovery as a distribution adaptation task by integrating both 2D observations and language descriptions. To bridge the gap between text and 3D pose signals, we first train a text encoder and a pose VQ-VAE, aligning texts to body poses in a shared latent space using contrastive learning. Subsequently, we employ a diffusion-based framework to refine the initial parameters guided by gradients derived from both 2D observations and text descriptions. Finally, the model can produce poses with accurate 3D perception and image consistency. Experimental results on multiple benchmarks validate its effectiveness. The code will be made publicly available.

\begin{IEEEkeywords}
human mesh recovery, multi-modal signal, diffusion for optimization
\end{IEEEkeywords}

\end{abstract}
\vspace{-8mm}

\section{Introduction}
\hbz{
\IEEEPARstart{M}{ONOCULAR} human mesh recovery aims to reconstruct 3D human meshes from a single image, which can be applied to various downstream tasks, such as 3D pose estimation~\cite{splposeestimation, splposeestimation1}, person re-identification~\cite{splperson-re1, splperson-re2, splperson-re3}, and crowd analysis~\cite{splcrowd}. This task is typically addressed using either regression-based~\cite{HMR, HMR2.0} or optimization-based~\cite{SMPLify, Refit} methods. Recent regression models~(\figref{fig:head_fig}(a)) leverage extensive human data to learn pose priors, enabling the prediction of accurate joint positions and body meshes. However, they often face challenges in aligning 3D models with 2D images due to the absence of explicit 2D-3D correspondences. In contrast, optimization-based methods(\figref{fig:head_fig}(b)) provide better model-to-image alignment but are sensitive to local minima and depth ambiguity, resulting in suboptimal joint accuracy. Additionally, off-the-shelf detectors\cite{fang2022alphapose} may introduce noises, which can degrade 3D reconstruction performance.

\begin{figure}[t] % 't' 选项确保图片在页面顶部
    \centering
    \includegraphics[width=.5\textwidth]{figs/head.pdf}
    \caption{\hbz{(a) Regression-based methods struggle with model-image alignment for challenging poses. (b) Optimization-based methods are prone to overfitting noisy 2D inputs and suffer from severe depth ambiguity. (c) Our method leverages prior knowledge from large vision-language models to improve both 2D and 3D performance.}}
    \label{fig:head_fig}
\vspace{-8mm}
\end{figure}

Several works~\cite{kolotouros2019learning, stathopoulos2024score} have attempted to integrate regression and optimization methods into a unified framework. These approaches first train a regression model to generate initial parameters and then refine the results using additional observations, such as 2D keypoints~\cite{kolotouros2019learning} and physical laws~\cite{huang2022neural, CloseInteraction}. However, 2D keypoints are often unreliable in complex environments (e.g., occlusions). Physics-based optimization also suffers from a knowledge gap between simulation and the real world, which may result in suboptimal simulated outcomes under the given constraints. Therefore, existing approaches have yet to fully address the trade-off between image observations and model-based assumptions.

Recently, human motion generation works~\cite{Guo_2022_CVPR, petrovich23tmr, jiang2024motiongpt} reveal that texts can provide rich 3D pose information. Therefore, our key idea is to leverage textual descriptions from large vision-language models (VLMs)~\cite{ChatPose} to compensate for insufficient 2D image observations. Benefiting from the 3D reasoning capabilities of VLMs (e.g., a person sitting with one leg crossed over the other), text-image inputs can enhance 3D perception and 2D-3D consistency for human pose estimation, thereby reducing the trade-off between image observations and model-based assumptions.
}

\hbz{To this end, we propose a framework that combines regression and optimization approaches, leveraging both image observations and vision-language models (VLMs) to facilitate human mesh recovery. The initial pose is first predicted using a Vision Transformer (ViT)~\cite{ViT}, which may be inaccurate due to depth ambiguity. To refine the pose, part-aware interactive descriptions are further extracted from the image using a Vision-Language Model (VLM)~\cite{ChatPose} with carefully designed prompts. Since text cannot directly provide detailed pose information, we define the alignment between pose and text in the latent space as a guiding signal. Consequently, we train a shared space based on VQ-VAE to bridge the gap between these two modalities. In the reverse diffusion process, we evaluate the reconstructed pose using re-projection error and text-pose similarity, and then use the derived gradients as conditions in each timestep. With the text-image conditions, the initial pose is iteratively updated and will ultimately converge to the real pose. In summary, our key contributions are: (1) We propose a framework that integrates multi-modal feedback to achieve both accurate 3D pose estimation and precise model-image alignment. (2) We demonstrate that fine-grained textual interactive descriptions can enhance human mesh recovery. (3) We introduce a novel conditioning mechanism that combines vision and language observations to guide the diffusion process.}
% \section{Related works}

\subsection{Monocular human mesh recovery}

SMPL model~\cite{SMPL} accelerates the development of human mesh recovery in recent years. SMPLify~\cite{SMPLify} is the first method to fit 3D body models to 2D keypoints detected by  keypoint detectors. HMR~\cite{HMR} then constructs the first end-to-end regression-based framework to directly predict the pose, betas and camera translation from the cropped person image. Since then, many methods have been developed based on HMR framework, aiming to improve reconstruction quality by feature pyramid alignment~\cite{PyMAF}, body part attention guide~\cite{PARE} and hierarchically design~\cite{HKHMR}. Recently, the first end-to-end transformer-based architecture HMR2.0~\cite{HMR2.0} outperforms the CNN-based architecture and achieves incredible results. Nonetheless, directly predicting the parameters by a neural network is very challenging due to the depth ambiguity and highly non-linear mapping~\cite{zhang2020object,huang2022object,huang2022pose2uv}. Hence, some methods are raised to optimize the regression results. SPIN~\cite{SPIN} uses the optimized results from SMPLify~\cite{SMPLify} to supervise the regression model. HoloPose~\cite{HoloPose} aligns the 3D estimates from regression model with the dense pose and 2D/3D joint positions, which is predicted by separate decoders. Some methods~\cite{HuMoR,TokenHMR} learn meaningful priors from AMASS~\cite{AMASS} dataset and then impose a constraint to the estimated distribution. Therefore, in this work we also use a regression optimization strategy to regress a initial SMPL parameters and then refine it to be consistent with other observations. 


\subsection{Diffusion-based human pose optimization}

% diffusion optimization 
Diffusion models~\cite{DDPM} have shown promising performance in various areas~(\eg, text-to-image generation~\cite{stable_diffusion}, text-to-motion generation~\cite{MDM}), which typically diffuse clean data towards standard Gaussian distribution and train a model to recover the clean value from the constructed noise. In the inference phase, the model samples the corresponding noise and iteratively optimizes the values at each timestep, which is usually accelerated by the DDIM method. In the field of pose estimation, ScoreHMR~\cite{ScoreHMR} uses a diffusion model to predict the noise in each step guided by the 2D keypoints information from an off-the-shelf detector. This method requires explicitly defining the correction strength, which might lead to extreme adjustments. BUDDI~\cite{BUDDI} learns the 3D proxemics prior with a diffusion model, and uses it to guide the close interaction human mesh reconstruction. CloseInteraction~\cite{CloseInteraction}
leverages knowledge from proxemic behavior and physics to compensate the lack of visual information to estimate accurate human interactions. However, all these methods only consider additional observations from a single modality or ignore the relationship between information from different modalities, which may lead to overfitting to additional information.

\subsection{Data driven human multi-modal priors}

As deep learning explores the depth of neural network architectures, many large models have gained strong generalization capabilities by increasing the network parameters and the amount of training data. Trained on hundreds of millions of images and captions, CLIP~\cite{CLIP} acquires powerful multi-modal semantic alignment capabilities, which can be used for several downstream tasks and achieves promising results. In the 3D domain, MotionCLIP~\cite{MotionCLIP} aligns the latent space of motion and text by contrastive pretraining, implicitly infusing the rich semantic knowledge of CLIP into the manifold. Based on pretrained large language models, MotionGPT~\cite{MotionGPT} employs discrete vector quantization of human motion and trains on both motion and text in a unified manner, which achieves remarkable performance in downstream tasks. ChatPose~\cite{ChatPose} directly regards the pose of a person as a token \texttt {\textless POSE\textgreater}, and the hidden states will be projected to regress the pose parameters once the token is predicted. With the aid of GPT-4 \cite{GPT4}, a huge VQA dataset can be constructed to finetune the above models. However, although these models can learn rich semantic information and have good generalization with higher-order information reasoning ability, they are still insensitive to pixel-level cues due to the discrete nature of tokens. Therefore, we utilize the higher-order semantic information from the large model for fine-grained feature extraction, and then transfer the discrete textual information into latent space for continuous guidance for human mesh recovery adaption.
\section{Projected Safe Set Algorithm for Infeasible Constraint Sets}
\label{sec:method}

In this section, we propose several approaches to handling infeasible safe control problems.
To facilitate further derivation, we first expand the safe control constraints in \eqref{prob:multi_safe_control} by plugging in the system dynamics as follows.
\begin{align}\label{eq:expand_safe_constraint}
\dot{\bphi}(x,u) &= \underbrace{\frac{\partial\bphi}{\partial x}f(x)}_{L_f\bphi(x)} + \underbrace{\frac{\partial\bphi}{\partial x}g(x)}_{L_g\bphi(x)} u \\\nonumber
&= L_f\bphi(x) + L_g\bphi(x) u \leq -\bETA
\end{align}
While our approaches are compatible with arbitrary task objective $\cJ$, we instantiate an example in this section to aid discussions.
We assume that \eqref{prob:multi_safe_control} operates as a safety filter at the downstream of some nominal controller.
With nominal control signal $u_\mathrm{ref}$, we can set $\cJ(x,u) = \|u-u_\mathrm{ref}\|_{2,Q}^2$ to compute a minimally invasive control $u_\mathrm{safe}$ that satisfies safety constraints.
Incorporating the control limits as well, \eqref{prob:multi_safe_control} can be written as
\begin{subequations}\label{prob:naive_ssa}
\begin{align}
\minimizewrt{{u}}~~ & \|u-u_\mathrm{ref}\|_{2,Q}^2   \\
\st~~ & L_f\bphi(x) + L_g\bphi(x) u \leq -\bETA~\mathrm{if}~\bphi(x) \geq 0  \label{eq:infeas_ssa_contr_lie} \\ 
& u^- \leq u \leq u^+ \label{eq:control_limit}
\end{align}
\end{subequations}

\subsection{Relaxed Safe Set Algorithm}\label{sec:rssa}

When \eqref{prob:naive_ssa} becomes infeasible due to complex humanoid-environment interactions, the most straight-forward remedy is to incorporate slack variables that relaxes the constraints to allow feasible solutions.
Specifically, we introduce positive slack variables only for each of the safety constraints in \eqref{eq:infeas_ssa_contr_lie} since the control limits cannot be relaxed.
Hence, we have
\begin{subequations}\label{prob:rssa}
\begin{align}
\minimizewrt{{u, \bs}}~~ & \|u-u_\mathrm{ref}\|_{2,Q}^2+ \frac{1}{p}(\|\bs\|_{p,Q^{rssa}_s})^p  \\
\st~~ & L_f\bphi(x) + L_g\bphi(x) u \leq -\bETA + \bs ~\mathrm{if}~\bphi(x) \geq 0 \label{eq:safe_contr_rssa}\\ 
& u^- \leq u \leq u^+ \\
&  \bs \geq 0 
\end{align}
\end{subequations}
where $\bs\in\RR^{M}$ is the slack variable.
We regularize $\bs$ measured in $Q^{rssa}_s$-weighted $p-$norm, given by
\begin{equation}
\|\bs\|_{p,Q^{rssa}_s} = \left(\sum_{i=1}^{M}Q^{rssa}_{s,i}|\bs_i|^p\right)^{1/p}
\end{equation}
where $Q^{rssa}_s$ is a diagonal matrix with positive coefficients.
We refer to \eqref{prob:rssa} as Relaxed Safe Set Algorithm (r-SSA).
The solved $\bs$ indicates the cost of safety violations, and should be as close to zero as possible to try to respect the safety constraints.
As will be shown later, r-SSA can effectively produce safe control when the naive SSA \eqref{prob:naive_ssa} becomes infeasible, and enhances humanoid safety in cluttered environments.

Importantly, r-SSA optimizes a combination of both performance objective (i.e., the first term) and safety objectives (i.e., the second term), which may be conflicting in general.
It also balances multiple safety objectives represented by each $\phi_i$.
r-SSA may prioritize the performance (i.e., reference tracking) and accept large slack variables (i.e., significant safety violations) if partial safety objectives are dominated, especially when the weighting parameters $Q$ and $Q^{rssa}_s$ are not properly tuned.
As a result, r-SSA may still lead to critical safety failures in practice unless specifically tuned for each task.
This challenge motivates us to take another step by removing potential racing conditions between the two objectives, which will be covered in the next section.

\subsection{Projected Safe Set Algorithm}\label{sec:pssa}

% \ruic{state problem of parameter tuning for r-SSA}

% \ruic{propose p-SSA that finds the closest feasible constraint set with minimal safety violation measured by p-norm}

In this section, we propose the Projected Safe Set Algorithm (p-SSA) that improves over r-SSA by always respecting the safety constraints to the maximal extend.
The core idea behind p-SSA is to first project the current safe control constraint set, which can be infeasible, to the nearest feasible set.
The $u_\mathrm{safe}$ is only solved with the projected constraint set which is guaranteed to be feasible by the projection operation.
It can be shown that with such decoupling, p-SSA always operates within the maximal feasible region indicated by the given constraint set, while being totally tuning-free.

In phase I, the p-SSA first resolves infeasible safe control constraints in \eqref{prob:naive_ssa} by projecting the constraint set \eqref{eq:infeas_ssa_contr_lie}
 and \eqref{eq:control_limit} on to the nearest feasible region, measured by the $p$-norm of  total relaxation.
Specifically, we solve the following optimization
\begin{subequations}\label{prob:pssa_phase_1}
\begin{align}
\minimizewrt{\bs}~~ & \frac{1}{p}(\|\bs\|_{p,Q^{pssa}_s})^p  \\
\st~~ & L_f\bphi(x) + L_g\bphi(x) u \leq -\bETA + \bs ~\mathrm{if}~\bphi(x) \geq 0 \label{eq:safe_contr_pssa_phase_1}\\ 
& u^- \leq u \leq u^+ \label{eq:control_limit_pssa_phase_1}\\
&  \bs \geq 0
\end{align}
\end{subequations}
to compute an optimal slack variable $\bs^*$.
Then, in phase II, we solve $u_\mathrm{safe}$ with the solved relaxation $\bs^*$
\begin{subequations}\label{prob:pssa_phase_2}
\begin{align}
\minimizewrt{u}~~ & \|u-u_\mathrm{ref}\|_{2,Q}^2  \\
\st~~ & L_f\bphi(x) + L_g\bphi(x) u \leq -\bETA + \bs^* ~\mathrm{if}~\bphi(x) \geq 0 \label{eq:safe_contr_pssa_phase_2}\\ 
& u^- \leq u \leq u^+ \label{eq:control_limit_pssa_phase_2}
\end{align}
\end{subequations}
Since $\bs^*$ is feasible for \eqref{prob:pssa_phase_1}, we know that the constraints \eqref{eq:safe_contr_pssa_phase_1} with \eqref{eq:control_limit_pssa_phase_1} will be made feasible if relaxed by $\bs^*$ (i.e.,  \eqref{eq:safe_contr_pssa_phase_2} and \eqref{eq:control_limit_pssa_phase_2} ).
Hence, phase II is guaranteed to be feasible without additional relaxation. 
Notably, p-SSA does not involve any direct trade-off between performance and safety since they are optimized independently via \eqref{prob:pssa_phase_1} and \eqref{prob:pssa_phase_2}.
% It does not need any parameter tuning for balancing those objectives.
Meanwhile, p-SSA guarantees to operate with minimal safety violations thanks to Phase I.
As will be shown in the next section, p-SSA, in fact, achieves the top performance across various task settings without parameter tuning, while r-SSA only matches p-SSA performance with careful tuning.

\textbf{Remark.} Unlike $Q^{rssa}_s$, $Q^{pssa}_s$ only influences the balance among multiple constraints.
In practice, $Q^{pssa}_s$ should be chosen to reflect the relative importance of different constraints.
For example, more weight can be assigned to a constraint if it covers a critical aspect such as the safety of a high-torque link, or if it is closer to being violated (e.g., $\phi_i$ closer to zero).
In this paper, we set $Q^{pssa}_s$ to identity and leave the investigation of smart ways to balance multiple safety constraints for future work.

When implementing the above approaches, one needs to derive the two Lie derivatives, $L_f\bphi(x)$ and $L_g\bphi(x)$, based on the designed energy functions $\phi_i$ and plug into \eqref{eq:safe_contr_rssa} for r-SSA or \eqref{eq:safe_contr_pssa_phase_1} and \eqref{eq:safe_contr_pssa_phase_2} for p-SSA to complete the control constraints.
For readability, we provide an example derivation of control constraints based on first-order energy functions designed for both robot-obstacle collision and self-collision in Appendix \ref{append:safe_control_constraint}.

\section{Proof of Concept Experiments}
\label{sec:experiments}

%\begin{itemize}
%    \item joint exploration non e' spesso un opzione
%    \item specificare che le policy sono decentralizzate a differenza di tutti i casi precedenti
%    \item decentralizzata con feedback decentralizzato non si coordina e il problema e' abbastanza semplice da portare a policy quasi deterministiche
%\end{itemize}



%\mirco{questo primo paragrafo è un po' convoluto. Prova a ristruttura la sezione in questo modo: quali sono le domande a cui cerchiamo risposta? Quali sono i domini sperimentali? Quali sono gli algoritmi che compariamo? Quali sono i take away? Per l'ultimo potresti anche evidenziare qualche frase in grassetto o emph con le principali conclusioni empiriche}

In this section, we provide some empirical validations of the findings discussed so far. Especially, we aim to answer the following questions: (\textbf{a}) Is Algorithm~\ref{alg:trpe} actually capable of optimizing finite-trials objectives? (\textbf{b}) Do different objectives enforce different behaviors, as expected from Section~\ref{sec:problem_formulation}? (\textbf{c}) Does the \emph{clustering} behavior of mixture objectives play a crucial role? If yes, when and why?\\
Throughout the experiments, we will compare the result of optimizing finite-trial objectives, either joint, disjoint, mixture ones, through Algorithm~\ref{alg:trpe} via fully decentralized policies. The experiments will be performed with different values of the exploration horizon $T$, so as to test their capabilities in different exploration efficiency regimes.\footnote{The exploration horizon $T$, rather than being a given trajectory length, has to be seen as a parameter of the exploration phase which allows to tradeoff exploration quality with exploration efficiency.} The full implementation details are reported in Appendix~\ref{apx:exp}.
\vspace{-6pt}
\paragraph*{Experimental Domains.}~The experiments were performed on two domains. The first is a notoriously difficult multi-agent exploration task called \emph{secret room}~\citep[MPE,][]{pmlr-v139-liu21j},\footnote{We highlight that all previous efforts in this task employed centralized policies. We are interested on the role of the entropic feedback in fostering coordination rather than full-state conditioning, then maintaining fully decentralized policies instead.} referred to as  Env.~(\textbf{i}). In such task, two agents are required to reach a target while navigating over two rooms divided by a door. In order to keep the door open, at least one agent have to remain on a switch. Two switches are located at the corners of the two rooms. The hardness of the task then comes from the need of coordinated exploration, where one agent allows for the exploration of the other. The second is a simpler exploration task yet over a high dimensional state-space, namely a 2-agent instantiation of \emph{Reacher}~\citep[MaMuJoCo,][]{peng2021facmac}, referred to as Env.~(\textbf{ii}). Each agent corresponds to one joint and equipped with decentralized policies conditioned on her own states. In order to allow for the use of plug-in estimator of the entropy~\citep{paninski2003}, each state dimension was discretized over 10 bins.


\begin{figure*}[!]
    \centering
    \begin{tikzpicture}
    % Draw rounded box for the legend
    \node[draw=black, rounded corners, inner sep=2pt, fill=white] (legend) at (0,0) {
        \begin{tikzpicture}[scale=0.8]
            % Mixture
            \draw[thick, color={rgb,255:red,230; green,159; blue,0}, opacity=0.8] (0,0) -- (1,0);
            \fill[color={rgb,255:red,230; green,159; blue,0}, opacity=0.2] (0,-0.1) rectangle (1,0.1);
            \node[anchor=west, font=\scriptsize] at (1.2,0) {Mixture};
            
            % Joint
            \draw[thick, dashed, color={rgb,255:red,86; green,180; blue,233}, opacity=0.8] (2.5,0) -- (3.5,0);
            \fill[color={rgb,255:red,86; green,180; blue,233}, opacity=0.2] (2.5,-0.1) rectangle (3.5,0.1);
            \node[anchor=west, font=\scriptsize] at (3.7,0) {Joint};
            
            
            % Disjoint
            \draw[thick, dotted, color={rgb,255:red,204; green,121; blue,167}, opacity=0.8] (4.7,0) -- (5.7,0);
            \fill[color={rgb,255:red,204; green,121; blue,167}, opacity=0.2] (4.7,-0.1) rectangle (5.7,0.1);
            \node[anchor=west, font=\scriptsize] at (5.9,0) {Disjoint};
            
            % Uniform
            \draw[thick, color={rgb,255:red,153; green,153; blue,153}, opacity=0.8] (7.2,0) -- (8.2,0);
            \fill[color={rgb,255:red,153; green,153; blue,153}, opacity=0.2] (7.2,-0.1) rectangle (8.2,0.1);
            \node[anchor=west, font=\scriptsize] at (8.4,0) {Random Initialization};
        \end{tikzpicture}
    };
\end{tikzpicture}

    %\hfill
    \vfill
    %vspace{-0.2cm}
    \begin{subfigure}[b]{0.3\textwidth}
        \includegraphics[width=\textwidth]{figures/room_150_AverageReturnnokl.pdf}
        %\vspace{-0.8cm}
        \caption{\centering MA-TRPO with TRPE Pre-Training (Env.~(\textbf{i}), $T=150$).}
        \label{subfig:image9}
    \end{subfigure}
    \hfill
    \begin{subfigure}[b]{0.3\textwidth}
        \includegraphics[width=\textwidth]{figures/room_50_AverageReturnnokl.pdf}
        %\vspace{-0.8cm}
        \caption{\centering MA-TRPO with TRPE Pre-Training (Env.~(\textbf{i}), $T=50$).}
        \label{subfig:image10}
    \end{subfigure}
    \hfill
    \begin{subfigure}[b]{0.3\textwidth}
        \centering
        \includegraphics[width=0.8\textwidth]{figures/hand_100_AverageReturn.pdf}
        %\vspace{-0.8cm}
        \caption{\centering MA-TRPO with TRPE Pre-Training (Env.~(\textbf{ii}), $T=100$).}
        \label{subfig:image11}
    \end{subfigure}
\caption{\centering Effect of pre-training in sparse-reward settings.(\emph{left}) Policies initialized with either Uniform or TRPE pre-trained policies over 4 runs over a worst-case goal. (\emph{rigth}) Policies initialized with either Zero-Mean or TRPE pre-trained policies over 4 runs over 3 possible goal state. We report the average and 95\% c.i.}
\label{fig:pretraining}
\end{figure*}
\vspace{-10pt}
\paragraph*{Task-Agnostic Exploration.}~Algorithm~\ref{alg:trpe} was first tested in her ability to address task-agnostic exploration \emph{per se}. This was done by considering the well-know hard-exploration task of Env.~(\textbf{i}). The results are reported in Figure~\ref{fig:room} for a short exploration horizon $(T=50)$. Interestingly, at this efficiency regime, when looking at the joint entropy in Figure~\ref{subfig:image2}, joint and disjoint objectives perform rather well compared to mixture ones in terms of induced joint entropy, while they fail to address mixture entropy explicitly, as seen in Figure~\ref{subfig:image3}. On the other hand mixture-based objectives result in optimizing both mixture \emph{and} joint entropy effectively, as one would expect by the bounds in Th.~\ref{lem:entropymismatch}. By looking at the actual state visitation induced by the trained policies, the difference between the objectives is apparent. While optimizing joint objectives, agents exploit the high-dimensionality of the joint space to induce highly entropic distributions even without exploring the space uniformly via coordination (Fig.~\ref{subfig:image5}); the same outcome happens in disjoint objectives, with which agents focus on over-optimizing over a restricted space loosing any incentive for coordinated exploration (Fig.\ref{subfig:image6}). On the other hand, mixture objectives enforce a clustering behavior (Fig.\ref{subfig:image6}) and result in a better efficient exploration. 

\paragraph*{Policy Pre-Training via Task-Agnostic Exploration.}~More interestingly, we tested the effect of pre-training policies via different objectives as a way to alleviate the well-known hardness of sparse-reward settings, either throught faster learning or zero-short generalization. In order to do so, we employed a multi-agent counterpart of the TRPO algorithm~\cite{schulman2017trustregionpolicyoptimization} with different pre-trained policies. First, we investigated the effect on the learning curve in the hard-exploration task of Env.~(\textbf{i}) under long horizons ($T=150$), with a worst-case goal set on the the opposite corner of the closed room. Pre-training via mixture objectives still lead to a faster learning compared to initializing the policy with a uniform distribution. On the other hand, joint objective pre-training did not lead to substantial improvements over standard initializations. More interestingly, when extremely short horizons were taken into account ($T=50$) the difference became appalling, as shown in Fig.~\ref{subfig:image9}: pre-training via mixture-based objectives leaded to faster learning and higher performances, while pre-training via disjoint objectives turned out to be even \emph{harmful} (Fig.~\ref{subfig:image10}). This was motivated by the fact that the disjoint objective overfitted the task over the states reachable without coordinated exploration, resulting in almost deterministic policies, as shown in Fig~\ref{fig:333} in Appendix~\ref{apx:exp}. Finally, we tested the zero-shot capabilities of policy pre-training on the simpler but high dimensional exploration task of Env.~(\textbf{ii}), where the goal was sampled randomly between worst-case positions at the boundaries of the region reachable by the arm. As shown in Fig.~\ref{subfig:image11}, both joint and mixture were able to guarantee zero-shot performances via pre-training compatible with MA-TRPO after learning over $2$e$4$ samples, while disjoint objectives were not. On the other hand, pre-training with joint objectives showed an extremely high-variance, leading to worst-case performances not better than the ones of random initialization. Mixture objectives on the other hand showed higher stability in guaranteeing compelling zero-shot performance.
\vspace{-6pt}
\paragraph*{Take-Aways.}~Overall, the proposed proof of concepts experiments managed to answer to all of the experimental questions: (\textbf{a}) Algorithm~\ref{alg:trpe} is indeed able to explicitly optimize for finite-trial entropic objectives. Additionally, (\textbf{b}) \textbf{mixture distributions enforce diverse yet coordinated exploration}, that helps when high efficiency is required. Joint or disjoint objectives on the other hand may fail to lead to relevant solutions because of under or over optimization. Finally, (\textbf{c}) \textbf{efficient exploration} enforced by mixture distributions was shown to be a \textbf{crucial factor} not only for the sake of task-agnostic exploration per se, but also for the ability of \textbf{pre-training via task-agnostic exploration} to lead to \textbf{faster and better training} and even \textbf{zero-shot generalization}.
\section{Conclusion}
We present live monitoring and mid-run interventions for multi-agent systems. We demonstrate that monitors based on simple statistical measures can effectively predict future agent failures, and these failures can be prevented by restarting the communication channel. Experiments across multiple environments and models show consistent gains of up to 17.4\%-20\% in system performance, with an addition in inference-time compute.
Our work also introduces \ourenv{}, a new environment for studying multi-agent cooperation.

\bibliographystyle{IEEEbib}
\bibliography{references}

\newpage
\appendix

\renewcommand{\figurename}{Supplementary Figure}
\renewcommand{\tablename}{Supplementary Table}
\setcounter{figure}{0}
\setcounter{table}{0}

    



\section{Details of datasets}
This section provides additional details about the dataset used to evaluate the downstream tasks. \Cref{tab:disease_definition} lists the ICD-10 codes and medications used to identify the diagnoses for each disease. \Cref{tab:characteristic} presents the distribution of patient characteristics for each disease. \Cref{fig:nyu_langone_prevalence,fig:nyu_longisland_prevalence} illustrates the prevalence of each disease in the downstream tasks for the NYU Langone and NYU Long Island datasets, highlighting the imbalances present in these tasks.

\begin{table}[!htpb]
    \centering
    \caption{The definition of diseases in EHR by diagnosis codes and medications.}
    \begin{tabular}{lr}
    \toprule
         Disease &  Definition in EHR \\
    \midrule
       IPH  &  I61.0, I61.1, I61.2, I61.3, I61.4, I61.8, I61.9 \\
       IVH  &  I61.5, P52.1, P52.2, P52.3  \\
       ICH  &  IPH + IVH + I61.6, I62.9, P10.9, P52.4, P52.9 \\
       SDH  &  S06.5, I62.0 \\
       EDH  &  S06.4, I62.1 \\
       SAH  &  I60.*, S06.6, P52.5, P10.3  \\
       Tumor  &  C71.*, C79.3, D33.0, D33.1, D33.2, D33.3, D33.7, D33.9  \\
       Hydrocephalus  &  G91.* \\
       Edema  &  G93.1, G93.5, G93.6, G93.82, S06.1 \\
       \multirow{2}{*}{ADRD}  &  G23.1, G30.*, G31.01, G31.09, G31.83, G31.85, G31.9, F01.*, F02.*, F03.*, G31.84, G31.1, \\ 
       & \textbf{Medication:} DONEPEZIL, RIVASTIGMINE, GALANTAMINE, MEMANTINE, TACRINE \\ 
    \bottomrule
    \end{tabular}
    \label{tab:disease_definition}
\end{table}

\begin{table}[!htbp]
\centering
\caption{Demographic characteristics of patients associated with scans from the NYU Langone dataset, matched with electronic health records (EHR) and utilized in downstream tasks.}
\label{tab:characteristic}

 The characteristic table on NYU Langone dataset matched with EHR.
\begin{tabular}{ll|rr|r}
\toprule
                       \textbf{Cohort} &  &           \textbf{Male (\%)} &          \textbf{Female (\%)} &     \textbf{Age (std)} \\
\midrule
 --- & All (n=270,205) & 128,113 (47.41\%) & 142,092 (52.59\%) & 63.64 (19.68) \\
\midrule
       Tumor & Neg (n=260,704) & 123,338 (47.31\%) & 137,366 (52.69\%) & 63.85 (19.72) \\
             & Pos (n=9,501) &   4,775 (50.26\%) &   4,726 (49.74\%) & 57.80 (17.67) \\
\midrule
HCP & Neg (n=253,000) & 118,881 (46.99\%) & 134,119 (53.01\%) & 63.67 (19.72) \\
              & Pos (n=17,205) &   9,232 (53.66\%) &   7,973 (46.34\%) & 63.18 (19.11) \\
\midrule
Edema & Neg (n=242,576) & 112,987 (46.58\%) & 129,589 (53.42\%) & 63.96 (19.84) \\
      & Pos (n=27,629) &  15,126 (54.75\%) &  12,503 (45.25\%) & 60.81 (17.97) \\
\midrule
ADRD  & Neg (n=232,667) & 111,159 (47.78\%) & 121,508 (52.22\%) & 61.31 (19.55) \\
      & Pos (n=37,538) &  16,954 (45.16\%) &  20,584 (54.84\%) & 78.09 (13.30) \\
\midrule
          IPH & Neg (n=251,308) & 117,692 (46.83\%) & 133,616 (53.17\%) & 63.58 (19.82) \\
              & Pos (n=18,897) &  10,421 (55.15\%) &   8,476 (44.85\%) & 64.39 (17.69) \\
\midrule
          IVH & Neg (n=258,232) & 121,686 (47.12\%) & 136,546 (52.88\%) & 63.65 (19.79) \\
              & Pos (n=11,973) &   6,427 (53.68\%) &   5,546 (46.32\%) & 63.45 (17.19) \\
\midrule
          SDH & Neg (n=248,468) & 114,869 (46.23\%) & 133,599 (53.77\%) & 63.44 (19.78) \\
              & Pos (n=21,737) &  13,244 (60.93\%) &   8,493 (39.07\%) & 65.95 (18.33) \\
\midrule
          EDH & Neg (n=265,431) & 125,113 (47.14\%) & 140,318 (52.86\%) & 63.77 (19.64) \\
              & Pos (n=4,774) &   3,000 (62.84\%) &   1,774 (37.16\%) & 56.53 (20.75) \\
\midrule
          SAH & Neg (n=251,594) & 118,424 (47.07\%) & 133,170 (52.93\%) & 63.79 (19.76) \\
              & Pos (n=18,611) &   9,689 (52.06\%) &   8,922 (47.94\%) & 61.59 (18.49) \\
\midrule
          ICH & Neg (n=229,851) & 105,498 (45.90\%) & 124,353 (54.10\%) & 63.41 (19.93) \\
              & Pos (n=40,354) &  22,615 (56.04\%) &  17,739 (43.96\%) & 64.93 (18.14) \\
\bottomrule
\end{tabular}
\end{table}


\begin{table}[!h]
    \centering
    \caption*{\textbf{Supplementary \Cref{tab:characteristic} Continued.} Demographic characteristics of patients associated with scans from the NYU Long Island dataset, matched with electronic health records (EHR) and utilized in downstream tasks.}
\begin{tabular}{ll|rr|r}
\toprule
                       \textbf{Cohort} &  &           \textbf{Male (\%)} &          \textbf{Female (\%)} &     \textbf{Age (std)} \\
\midrule
--- & All (n=22,158) & 9,580 (43.23\%) & 12,578 (56.77\%) & 68.33 (18.14) \\
\midrule
Tumor & Neg (n=21,578) & 9,275 (42.98\%) & 12,303 (57.02\%) & 68.59 (18.08) \\
      & Pos (n=580) &   305 (52.59\%) &    275 (47.41\%) & 58.78 (17.79) \\
\midrule
HCP   & Neg (n=20,653) & 8,718 (42.21\%) & 11,935 (57.79\%) & 69.05 (17.90) \\
      & Pos (n=1,505) &   862 (57.28\%) &    643 (42.72\%) & 58.52 (18.48) \\
\midrule
Edema & Neg (n=19,402) & 8,068 (41.58\%) & 11,334 (58.42\%) & 68.89 (18.27) \\
      & Pos (n=2,756) & 1,512 (54.86\%) &  1,244 (45.14\%) & 64.36 (16.66) \\
\midrule
ADRD  & Neg (n=19,537) & 8,391 (42.95\%) & 11,146 (57.05\%) & 66.78 (18.28) \\
      & Pos (n=2,621) & 1,189 (45.36\%) &  1,432 (54.64\%) & 79.90 (11.77) \\
\midrule
IPH   & Neg (n=19,357) & 7,974 (41.19\%) & 11,383 (58.81\%) & 68.97 (18.27) \\
      & Pos (n=2,801) & 1,606 (57.34\%) &  1,195 (42.66\%) & 63.89 (16.48) \\
\midrule
IVH   & Neg (n=19,636) & 8,164 (41.58\%) & 11,472 (58.42\%) & 68.96 (18.22) \\
      & Pos (n=2,522) & 1,416 (56.15\%) &  1,106 (43.85\%) & 63.43 (16.66) \\
\midrule
SDH   & Neg (n=20,885) & 8,870 (42.47\%) & 12,015 (57.53\%) & 68.33 (18.21) \\
      & Pos (n=1,273) &   710 (55.77\%) &    563 (44.23\%) & 68.37 (16.83) \\
\midrule
EDH   & Neg (n=21,912) & 9,443 (43.10\%) & 12,469 (56.90\%) & 68.33 (18.16) \\
      & Pos (n=246) &   137 (55.69\%) &    109 (44.31\%) & 68.19 (15.59) \\
\midrule
SAH   & Neg (n=20,652) & 8,824 (42.73\%) & 11,828 (57.27\%) & 68.68 (18.12) \\
      & Pos (n=1,506) &   756 (50.20\%) &    750 (49.80\%) & 63.58 (17.65) \\
\midrule
ICH   & Neg (n=18,388) & 7,456 (40.55\%) & 10,932 (59.45\%) & 68.92 (18.35) \\
      & Pos (n=3,770) & 2,124 (56.34\%) &  1,646 (43.66\%) & 65.48 (16.77) \\
\bottomrule
\end{tabular}
\end{table}

\begin{figure}[!ht]
    \centering
    \includegraphics[width=0.8\textwidth]{images/NYU_Langone_prevalence.pdf}
    \caption{Disease prevalence of NYU Langone }
    \label{fig:nyu_langone_prevalence}
\end{figure}

\begin{figure}[!h]
    \centering
    \includegraphics[width=0.8\textwidth]{images/NYU_Longisland_prevalence.pdf}
    \caption{Disease prevalence of NYU Longisland dataset}
    \label{fig:nyu_longisland_prevalence}
\end{figure}



\section{Data augmentation details}
\label{sec:dataaug_details}
We applied Random Flipping across all three dimensions, Random Shift Intensity with offset $0.1$ for both pre-training and fine-tuning. For DINO training. random Gaussian Smoothing with sigma=$(0.5-1.0)$ is applied across all dimensions, Random Gamma Adjust is applied with gamma=$(0.2-1.0)$.


\section{Additional experiment results}
This section provides additional experimental results with more details.
Supplementary \Cref{fig:channels-ablation,fig:patches-ablation} compares the performance of the foundation model using different numbers of channels and patch sizes, demonstrating that the architecture design of our foundation model is optimal. 

Supplementary \Cref{fig:radar-comparison-merlin} compares our foundation model with a foundation CT model from previous studies, Merlin\cite{blankemeier2024merlinvisionlanguagefoundation}, which was trained on abdomen CT scans with corresponding radiology report pairs. Our model demonstrates superior performance on head CT scans.

Supplementary \Cref{fig:probing-comparison-gemini} compares our foundation model with Google CT Foundation model~\cite{yang2024advancingmultimodalmedicalcapabilities}, which was trained on large scale and diverse CT scans from different anatomy with corresponding radiology report pairs. Our model consistently shows improved performance across the board even though our model was pre-trained with less samples.

Supplementary \Cref{fig:probing_comparison} compares the performance on downstream tasks with various supervised tuning methods applied to foundation models pretrained with the MAE and DINO frameworks. Per-pathology comparisons are shown in Supplementary \Cref{fig:probing-comparison-perpath,fig:probing-comparison-perpath-dino}. Meanwhile, supplementary \Cref{fig:boxplot_scaling} complements \Cref{fig:scaling_law}, illustrating the per-pathology performances of foundation models pretrained with different scales of training data.

Supplementary \Cref{fig:batch_effect,fig:thickness-ablation} studies the impact of batch effect caused by different CT scan protocols of slice thickness and machine manufacturer. Detailed per-pathology performances are shown in Supplementary \Cref{fig:slice_thickness_per_pathology,fig:manufacturer_per_pathology}.

\begin{figure}[!htpb]
    \centering
    \makebox[\textwidth][l]{%
        \hspace{0.3\textwidth}\textbf{NYU Langone}
    } \\[0.2cm]
    \includegraphics[trim={0 0 0 0},clip,height=0.3\textwidth, width=0.3\textwidth]{figures/abla_chans/AUC_chans_NYU.pdf}
    \includegraphics[trim={0 0 0 0},clip,height=0.3\textwidth, width=0.55\textwidth]{figures/abla_chans/AP_chans_NYU.pdf}\\
    \makebox[\textwidth][l]{
        \hspace{0.34\textwidth}\textbf{RSNA}
    } \\[0.2cm]
    \includegraphics[trim={0 0 0 0},clip,height=0.3\textwidth, width=0.3\textwidth]{figures/abla_chans/AUC_chans_RSNA.pdf}
    \includegraphics[height=0.3\textwidth, width=0.55\textwidth]{figures/abla_chans/AP_chans_RSNA.pdf} 
    \caption{\textbf{Comparison of Different Channels Performance.} This plot compares the performance of models trained using different numbers of channels (channels from multiple HU intervals vs. a single HU interval). Across two datasets, models using three channels from different HU intervals consistently outperform those using a single channel with a fixed HU interval. All models were pre-trained on $100\%$ of the pretraining data with MAE.}
    \label{fig:channels-ablation}
\end{figure}


\begin{figure}[!htpb]
    \centering
    \makebox[\textwidth][l]{%
        \hspace{0.3\textwidth}\textbf{NYU Langone}
    } \\[0.2cm]
    \includegraphics[trim={0 0 0 0},clip,height=0.3\textwidth, width=0.3\textwidth]{figures/abla_patches/AUC_patches_NYU.pdf}
    \includegraphics[trim={0 0 0 0},clip,height=0.3\textwidth, width=0.55\textwidth]{figures/abla_patches/AP_patches_NYU.pdf}\\
    \makebox[\textwidth][l]{
        \hspace{0.34\textwidth}\textbf{RSNA}
    } \\[0.2cm]
    \includegraphics[trim={0 0 0 0},clip,height=0.3\textwidth, width=0.3\textwidth]{figures/abla_patches/AUC_patches_RSNA.pdf}
    \includegraphics[height=0.3\textwidth, width=0.55\textwidth]{figures/abla_patches/AP_patches_RSNA.pdf} 
    \caption{\textbf{Comparison of Different Patches Performance.} This plot compares the performance of models trained with different patch sizes (12 vs. 16). The results demonstrate that smaller patch sizes consistently achieve better performance. All models were pre-trained on $100\%$ of the pretraining data with MAE.}
    \label{fig:patches-ablation}
\end{figure}


\begin{figure*}
    \centering
    \makebox[\textwidth][l]{%
        \hspace{0.06\textwidth}
        \textbf{NYU Langone} \hspace{0.06\textwidth} \textbf{NYU Long Island} \hspace{0.11\textwidth} \textbf{RSNA} \hspace{0.18\textwidth} \textbf{CQ500}
    } \\[0.2cm]
    \includegraphics[trim={0 0 0 0},clip,height=0.21\textwidth, width=0.21\textwidth]{figures/abla_radarplot_merlin/AUC_NYU.pdf}
    \includegraphics[trim={0 0 0 0},clip,height=0.21\textwidth, width=0.21\textwidth]{figures/abla_radarplot_merlin/AUC_Longisland.pdf}
    \includegraphics[trim={0 0 0 0},clip,height=0.21\textwidth, width=0.21\textwidth]{figures/abla_radarplot_merlin/AUC_RSNA.pdf}
    \includegraphics[trim={0 0 0 0},clip,height=0.21\textwidth, width=0.35\textwidth]{figures/abla_radarplot_merlin/AUC_CQ500.pdf}\\[0.2cm]
    \includegraphics[height=0.21\textwidth, width=0.21\textwidth]{figures/abla_radarplot_merlin/AP_NYU.pdf} 
    \includegraphics[height=0.21\textwidth, width=0.21\textwidth]{figures/abla_radarplot_merlin/AP_Longisland.pdf} 
    \includegraphics[height=0.21\textwidth, width=0.21\textwidth]{figures/abla_radarplot_merlin/AP_RSNA.pdf}
    \includegraphics[height=0.21\textwidth, width=0.35\textwidth]{figures/abla_radarplot_merlin/AP_CQ500.pdf}
    \caption{\textbf{Comparison to previous 3D Foundation Model.} This plot compares the performance of our model with Merlin~\cite{blankemeier2024merlinvisionlanguagefoundation} and models trained from scratch across four datasets for our model and ResNet50-3D. Our DINO trained model is used in this comparison. Our model demonstrates consistently superior performance across majority of diseases, with the exception of epidural hemorrhage (EDH) in the CQ500 dataset.}
    \label{fig:radar-comparison-merlin}
\end{figure*}



\begin{figure*}
    \centering
    \makebox[\textwidth][l]{%
        \hspace{0.10\textwidth}
        \textbf{NYU Langone} \hspace{0.08\textwidth} \textbf{NYU Long Island} \hspace{0.1\textwidth} \textbf{RSNA} \hspace{0.15\textwidth} \textbf{CQ500}
    } \\[0.2cm]
    \includegraphics[trim={0 0 0 0},clip, width=0.22\textwidth]{figures/abla_probing/AUC_NYU.pdf}
    \includegraphics[trim={0 0 0 0},clip, width=0.22\textwidth]{figures/abla_probing/AUC_Longisland.pdf}
    \includegraphics[trim={0 0 0 0},clip, width=0.22\textwidth]{figures/abla_probing/AUC_RSNA.pdf}
    \includegraphics[trim={0 0 0 0},clip, width=0.28\textwidth]{figures/abla_probing/AUC_CQ500.pdf}
    \\[0.2cm]
    \includegraphics[width=0.22\textwidth]{figures/abla_probing/AP_NYU.pdf} 
    \includegraphics[width=0.22\textwidth]{figures/abla_probing/AP_Longisland.pdf} 
    \includegraphics[width=0.22\textwidth]{figures/abla_probing/AP_RSNA.pdf}
    \includegraphics[width=0.28\textwidth]{figures/abla_probing/AP_CQ500.pdf}
    \caption{\textbf{Comparison of Different Downstream Training Methods.} This plot illustrates the downstream performance of models evaluated using fine-tuning and various probing methods across four datasets. In most cases, the DINO pre-trained model outperforms the MAE pre-trained model. All models were pre-trained on $100\%$ of the available pretraining data.}
    \label{fig:probing_comparison}
\end{figure*}


\begin{figure}
\centering
\makebox[\textwidth][l]{%
    \hspace{0.39\textwidth}\textbf{RSNA}
} \\[0.2cm]
\includegraphics[trim={0 0 0mm 0},clip,height=0.27\textwidth]{figures/abla_gemini/AUC_RSNA_Gemini.pdf}
\includegraphics[trim={0 0 5mm 0},clip,height=0.27\textwidth]{figures/abla_gemini/AP_RSNA_Gemini.pdf}

\makebox[\textwidth][l]{%
    \hspace{0.38\textwidth}\textbf{CQ500}
} \\[0.2cm]
\includegraphics[trim={0 0 10mm 0},clip,height=0.345\textwidth]{figures/abla_gemini/AUC_CQ500_Gemini.pdf}
\includegraphics[trim={0 0 5mm 0},clip,height=0.345\textwidth]{figures/abla_gemini/AP_CQ500_Gemini.pdf}

\caption{\textbf{Performance comparison of linear probing for Our Model vs. Google CT Foundation model} This plot compares our model performance vs. Google CT Foundation model\cite{yang2024advancing} and Merlin \cite{blankemeier2024merlinvisionlanguagefoundation} across all diseases on RSNA and CQ500. Since Google CT Foundation moudel requires uploading data to Google Cloud (not allowed on our private data) for requesting model embeddings with model weights inaccessible, only public dataset comparison is provided in this study. Similar to other evaluations, we observed that our model outperforms Google CT Foundation model across the board with the only exception on Midline Shift for Google CT Foundation model and EDH for Merlin.}
\label{fig:probing-comparison-gemini}
\end{figure}



\begin{figure}
    \centering
    \makebox[\textwidth][l]{%
        \hspace{0.35\textwidth}\textbf{NYU Langone}
    } \\[0.2cm]
    \includegraphics[trim={0 0 120mm 0},clip,height=0.255\textwidth]{figures/abla_probing_perpath/DINO_AUC_NYU_Langone.pdf}
    \includegraphics[trim={0 0 0 0},clip,height=0.255\textwidth]{figures/abla_probing_perpath/DINO_AP_NYU_Langone.pdf} \\
    \makebox[\textwidth][l]{
        \hspace{0.35\textwidth}\textbf{NYU Long Island}
    } \\[0.2cm]
    \includegraphics[trim={0 0 120mm 0},clip,height=0.255\textwidth]{figures/abla_probing_perpath/DINO_AUC_NYU_Long_Island.pdf}
    \includegraphics[trim={0 0 0 0},clip,height=0.255\textwidth]{figures/abla_probing_perpath/DINO_AP_NYU_Long_Island.pdf} 
    \makebox[\textwidth][l]{
        \hspace{0.4\textwidth}\textbf{RSNA}
    } \\[0.2cm]
    \includegraphics[trim={0 0 120mm 0},clip,height=0.24\textwidth]{figures/abla_probing_perpath/DINO_AUC_RSNA.pdf}
    \hspace{5mm}
    \includegraphics[trim={0 0 0 0},clip,height=0.24\textwidth]{figures/abla_probing_perpath/DINO_AP_RSNA.pdf} 
    \makebox[\textwidth][l]{
        \hspace{0.4\textwidth}\textbf{CQ500}
    } \\[0.2cm]
    \includegraphics[trim={0 0 120mm 0},clip,height=0.30\textwidth]{figures/abla_probing_perpath/DINO_AUC_CQ500.pdf} \hspace{5mm}
    \includegraphics[trim={0 0 0 0},clip,height=0.30\textwidth]{figures/abla_probing_perpath/DINO_AP_CQ500.pdf} 
    \caption{\textbf{Performance comparison of supervised finetuning methods per pathology on the foundation model trained with DINO.} This plot breaks down the average performance across all diseases shown in Supplementary \Cref{fig:probing_comparison}. The results show that fine-tuning the entire network achieves the best performance in most scenarios. However, linear probing closely approaches finetuning performance for many diseases especially on small or imbalanced dataset, underscoring the capability of our pre-trained models to generate representations that adapt effectively to diverse disease detection tasks.}
    \label{fig:probing-comparison-perpath-dino}
\end{figure}

\begin{figure}
    \centering
    \makebox[\textwidth][l]{%
        \hspace{0.35\textwidth}\textbf{NYU Langone}
    } \\[0.2cm]
    \includegraphics[trim={0 0 0 0},clip,height=0.24\textwidth, width=0.3\textwidth]{figures/abla_probing_perpath/AUC_NYU.pdf}
    \includegraphics[trim={0 0 0 0},clip,height=0.24\textwidth, width=0.45\textwidth]{figures/abla_probing_perpath/AP_NYU.pdf}\\
    \makebox[\textwidth][l]{
        \hspace{0.35\textwidth}\textbf{NYU Long Island}
    } \\[0.2cm]
    \includegraphics[trim={0 0 0 0},clip,height=0.24\textwidth, width=0.3\textwidth]{figures/abla_probing_perpath/AUC_Longisland.pdf}
    \includegraphics[trim={0 0 0 0},clip,height=0.24\textwidth, width=0.45\textwidth]{figures/abla_probing_perpath/AP_Longisland.pdf} 
    \makebox[\textwidth][l]{
        \hspace{0.4\textwidth}\textbf{RSNA}
    } \\[0.2cm]
    \includegraphics[trim={0 0 0 0},clip,height=0.24\textwidth, width=0.3\textwidth]{figures/abla_probing_perpath/AUC_RSNA.pdf}
    \includegraphics[height=0.24\textwidth, width=0.45\textwidth]{figures/abla_probing_perpath/AP_RSNA.pdf} 
    \makebox[\textwidth][l]{
        \hspace{0.4\textwidth}\textbf{CQ500}
    } \\[0.2cm]
    \includegraphics[trim={0 0 120mm 0},clip,height=0.24\textwidth]{figures/abla_probing_perpath/AUC_CQ500.pdf}
    \includegraphics[trim={0 0 0 0},clip,height=0.24\textwidth]{figures/abla_probing_perpath/AP_CQ500.pdf} 
    \caption{\textbf{Performance comparison of supervised finetuning methods per pathology on the foundation model trained with MAE.} The results reveal that attentive probing is significantly more effective than linear probing, consistent with findings from~\cite{Chen2024}. Furthermore, for many diseases, the performance of probing models approaches that of fine-tuning, demonstrating that our pre-trained models produce adaptable representations capable of detecting diverse diseases.}
    \label{fig:probing-comparison-perpath}
\end{figure}









\begin{figure}
    \centering
    \textbf{NYU Langone} \\
    \includegraphics[trim={0 0 0 0},clip,height=0.24\textwidth, width=0.38\textwidth]{figures/abla_perpath_perf/AUC_NYU.pdf}
    \includegraphics[height=0.24\textwidth, width=0.45\textwidth]{figures/abla_perpath_perf/AP_NYU.pdf} \\
    \textbf{NYU Long Island} \\
    \includegraphics[trim={0 0 0 0},clip,height=0.24\textwidth, width=0.38\textwidth]{figures/abla_perpath_perf/AUC_Longisland.pdf}
    \includegraphics[height=0.24\textwidth, width=0.45\textwidth]{figures/abla_perpath_perf/AP_Longisland.pdf} \\
    \textbf{RSNA} \\
    \includegraphics[trim={0 0 0 0},clip,height=0.24\textwidth, width=0.38\textwidth]{figures/abla_perpath_perf/AUC_RSNA.pdf}
    \includegraphics[height=0.24\textwidth, width=0.45\textwidth]{figures/abla_perpath_perf/AP_RSNA.pdf}\\
    \textbf{CQ500} \\
    \includegraphics[trim={0 0 0 0},clip,height=0.24\textwidth, width=0.38\textwidth]{figures/abla_perpath_perf/AUC_CQ500.pdf}
    \includegraphics[height=0.24\textwidth, width=0.45\textwidth]{figures/abla_perpath_perf/AP_CQ500.pdf}
    \caption{\textbf{Performance for Different Percentage of Pre-training Samples (Per-Pathology).} This plot illustrates label efficiency for individual pathologies using Tukey plots, alongside the average performance across all diseases shown in \Cref{fig:scaling_law}. The results indicate that the majority of pathologies show improved downstream performance as the amount of pretraining data increases.}
    \label{fig:boxplot_scaling}
\end{figure}


\newpage

\section{Time complexity increase with reduced patch size}
\label{apd:self_attention_rate}
Assume we have 3D image input of shape $H\times W\times D$, patch size $P$ and reducing factor $s$. By time complexity of self-attention $O(n^2 d)$ for sequence length $n$ and embedding dimension $d$, the new time complexity after reducing patch size can be derived as
\begin{align*}
    O(n^2d)&=O((\frac{H\times W\times D}{(\frac{P}{s})^3})^2d) \\
           &=O((\frac{H\times W\times D}{P^3})^2 s^6d)  \\
           &=O(s^6)O(n_{ori}^2d)
\end{align*}
where $n_{ori}=\frac{H\times W\times D}{P^3}$ is the original sequence length before reducing patch size.



















\newpage
\begin{figure}[ht]
    \centering
    \includegraphics[width=\textwidth]{images/tsne_embedding_visualization_per_pathology.png}
    \caption{The 2D projection with t-SNE of CT volume representation extracted from the foundation model. Interestingly, certain subgroups still exhibited slightly better AUCs. For instance, scans with slice thicknesses between 1–4 mm (represented by light green points in the upper panel of \Cref{fig:batch_effect}) align with a specialized protocol for CT angiography (CTA), which uses contrast dye to improve diagnosis on particular diseases.}
    \label{fig:batch_effect}
\end{figure}


\begin{figure*}[ht]
    \centering
    \begin{subfigure}[b]{0.33\textwidth}
        \centering
        \includegraphics[width=\textwidth]{images/AUROC_vs_Slice_thickness_binned.png}
        \caption{AUROC Performance}
    \end{subfigure}
    \hfill
    \begin{subfigure}[b]{0.33\textwidth}
        \centering
        \includegraphics[width=\textwidth]{images/AUPRC_vs_Slice_thickness_binned.png}
        \caption{AUPRC Performance}
    \end{subfigure}
    \hfill
    \begin{subfigure}[b]{0.33\textwidth}
        \centering
        \includegraphics[width=\textwidth]{images/Histogram_of_slice_thickness_distribution_across_scans.png}
        \caption{Histogram of slice thickness distribution}
    \end{subfigure}
    \caption{The downstream task performances on various ranges of slice thickness.}
    \label{fig:thickness-ablation}
\end{figure*}


\begin{figure*}[ht]
    \centering
    \begin{subfigure}[b]{\textwidth}
        \centering
        \includegraphics[width=\textwidth]{images/AUROC_vs_slice_thickness_for_each_disease_category.png}
        \caption{AUROC Performance}
    \end{subfigure}
    \hfill
    \begin{subfigure}[b]{\textwidth}
        \centering
        \includegraphics[width=\textwidth]{images/AUPRC_vs_slice_thickness_for_eachdisease_category.png}
        \caption{AUPRC Performance}
    \end{subfigure}
    \hfill
    \begin{subfigure}[b]{\textwidth}
        \centering
        \includegraphics[width=\textwidth]{images/Ratio_of_positive_labels_vs_slice_thickness_for_each_disease_category.png}
        \caption{Ratio of Positive Labels}
    \end{subfigure}
    \caption{Performance for Each Slice Thickness Bin (Per Pathology).}
    \label{fig:slice_thickness_per_pathology}
\end{figure*}


\begin{figure*}[ht]
    \centering
    \begin{subfigure}[b]{0.3\textwidth}
        \centering
        \includegraphics[width=\textwidth]{images/AUROC_by_Disease_and_Manufacturer.png}
        \caption{AUROC Performance}
    \end{subfigure}
    \hfill
    \begin{subfigure}[b]{0.3\textwidth}
        \centering
        \includegraphics[width=\textwidth]{images/AUPRC_by_Disease_and_Manufacturer.png}
        \caption{AUPRC Performance}
    \end{subfigure}
    \hfill
    \begin{subfigure}[b]{0.39\textwidth}
        \centering
        \includegraphics[width=\textwidth]{images/Positive_Label_Ratio_by_Disease_and_Manufacturer.png}
        \caption{Distribution of Scans from Each Manufacturer}
    \end{subfigure}
    \caption{Performance for Each Manufacturer (Per Pathology).}
    \label{fig:manufacturer_per_pathology}
\end{figure*}







% If your paper is intended for a conference, please contact your conference editor concerning acceptable word processor formats for your particular conference.  


% \section{Guidelines For Manuscript Preparation}


% The IEEEtran\_HOWTO.pdf is the complete guide of \LaTeX\ for manuscript preparation included with this stuff. 


% \subsection{Information for Authors}

% {\em IEEE Signal Processing Letters} allows only four-page articles. A fifth page is allowed for ``References'' only, though ``References'' may begin before the fifth page. Author biographies or photographs are not allowed in Signal Processing Letters. Please review the Information for Authors at for {\em IEEE Signal Processing Letters:} https://signalprocessingsociety.org/publications-resources/ieee-signal-processing-letters/information-authors-spl



% \section{Guidelines for Graphics Preparation and Submission}
% \label{sec:guidelines}

% \subsection{Types of Graphics}
% The following list outlines the different types of graphics published in 
% {\it IEEE Signal Processing Letters}. They are categorized based on their construction, and use of 
% color/shades of gray:

% \subsubsection{Color/Grayscale figures}
% {Figures that are meant to appear in color, or shades of black/gray. Such 
% figures may include photographs, illustrations, multicolor graphs, and 
% flowcharts.}

% \subsubsection{Line Art figures}
% {Figures that are composed of only black lines and shapes. These figures 
% should have no shades or half-tones of gray, only black and white.}

% \subsubsection{Tables}
% {Data charts which are typically black and white, but sometimes include 
% color.}



% \subsection{Multipart figures}
% Figures compiled of more than one sub-figure presented side-by-side, or 
% stacked. If a multipart figure is made up of multiple figure
% types (one part is lineart, and another is grayscale or color) the figure 
% should meet the stricter guidelines.

% \subsection{File Formats For Graphics}\label{formats}
% Format and save your graphics using a suitable graphics processing program 
% that will allow you to create the images as PostScript (PS), Encapsulated 
% PostScript (.EPS), Tagged Image File Format (.TIFF), Portable Document 
% Format (.PDF), Portable Network Graphics (.PNG), or Metapost (.MPS), sizes them, and adjusts 
% the resolution settings. When 
% submitting your final paper, your graphics should all be submitted 
% individually in one of these formats along with the manuscript.

% \subsection{Sizing of Graphics}
% Most charts, graphs, and tables are one column wide (3.5 inches/88 
% millimeters/21 picas) or page wide (7.16 inches/181 millimeters/43 
% picas). The maximum depth a graphic can be is 8.5 inches (216 millimeters/54
% picas). When choosing the depth of a graphic, please allow space for a 
% caption. Figures can be sized between column and page widths if the author 
% chooses, however it is recommended that figures are not sized less than 
% column width unless when necessary. 

% \begin{figure}
% \centerline{\includegraphics[width=\columnwidth]{fig1.png}}
% \caption{Magnetization as a function of applied field. Note that ``Fig.'' is abbreviated. There is a period after the figure number, followed by two spaces. It is good practice to explain the significance of the figure in the caption.}
% \end{figure}

% \begin{table}
% \caption{Units for Magnetic Properties}
% \label{table}
% \small
% \setlength{\tabcolsep}{3pt}
% \begin{tabular}{|p{25pt}|p{75pt}|p{110pt}|}
% \hline
% Symbol& 
% Quantity& 
% Conversion from Gaussian and \par CGS EMU to SI$^{\mathrm{a}}$ \\
% \hline
% $\Phi $& 
% Magnetic flux& 
% 1 Mx $\to  10^{-8}$ Wb $= 10^{-8}$ V $\cdot$ s \\
% $B$& 
% Magnetic flux density, \par magnetic induction& 
% 1 G $\to  10^{-4}$ T $= 10^{-4}$ Wb/m$^{2}$ \\
% $H$& 
% Magnetic field strength& 
% 1 Oe $\to  10^{-3}/(4\pi )$ A/m \\
% $m$& 
% Magnetic moment& 
% 1 erg/G $=$ 1 emu \par $\to 10^{-3}$ A $\cdot$ m$^{2} = 10^{-3}$ J/T \\
% $M$& 
% Magnetization& 
% 1 erg/(G $\cdot$ cm$^{3}) =$ 1 emu/cm$^{3}$ \par $\to 10^{-3}$ A/m \\
% 4$\pi M$& 
% Magnetization& 
% 1 G $\to  10^{-3}/(4\pi )$ A/m \\
% $\sigma $& 
% Specific magnetization& 
% 1 erg/(G $\cdot$ g) $=$ 1 emu/g $\to $ 1 A $\cdot$ m$^{2}$/kg \\
% $j$& 
% Magnetic dipole \par moment& 
% 1 erg/G $=$ 1 emu \par $\to 4\pi \times  10^{-10}$ Wb $\cdot$ m \\
% $J$& 
% Magnetic polarization& 
% 1 erg/(G $\cdot$ cm$^{3}) =$ 1 emu/cm$^{3}$ \par $\to 4\pi \times  10^{-4}$ T \\
% $\chi , \kappa $& 
% Susceptibility& 
% 1 $\to  4\pi $ \\
% $\chi_{\rho }$& 
% Mass susceptibility& 
% 1 cm$^{3}$/g $\to  4\pi \times  10^{-3}$ m$^{3}$/kg \\
% $\mu $& 
% Permeability& 
% 1 $\to  4\pi \times  10^{-7}$ H/m \par $= 4\pi \times  10^{-7}$ Wb/(A $\cdot$ m) \\
% $\mu_{r}$& 
% Relative permeability& 
% $\mu \to \mu_{r}$ \\
% $w, W$& 
% Energy density& 
% 1 erg/cm$^{3} \to  10^{-1}$ J/m$^{3}$ \\
% $N, D$& 
% Demagnetizing factor& 
% 1 $\to  1/(4\pi )$ \\
% \hline
% \multicolumn{3}{p{251pt}}{Vertical lines are optional in tables. Statements that serve as captions for 
% the entire table do not need footnote letters. }\\
% \multicolumn{3}{p{251pt}}{$^{\mathrm{a}}$Gaussian units are the same as cg emu for magnetostatics; Mx 
% $=$ maxwell, G $=$ gauss, Oe $=$ oersted; Wb $=$ weber, V $=$ volt, s $=$ 
% second, T $=$ tesla, m $=$ meter, A $=$ ampere, J $=$ joule, kg $=$ 
% kilogram, H $=$ henry.}
% \end{tabular}
% \label{tab1}
% \end{table}


% \subsection{Resolution }
% The proper resolution of your figures will depend on the type of figure it 
% is as defined in the ``Types of Figures'' section. Author photographs, 
% color, and grayscale figures should be at least 300dpi. Line art, including 
% tables should be a minimum of 600dpi.

% \subsection{Vector Art}
% In order to preserve the figures' integrity across multiple computer 
% platforms, we accept files in the following formats: .EPS/.PDF/.PS. All 
% fonts must be embedded or text converted to outlines in order to achieve the 
% best-quality results.


% \subsection{Accepted Fonts Within Figures}
% When preparing your graphics IEEE suggests that you use of one of the 
% following Open Type fonts: Times New Roman, Helvetica, Arial, Cambria, and 
% Symbol. If you are supplying EPS, PS, or PDF files all fonts must be 
% embedded. Some fonts may only be native to your operating system; without 
% the fonts embedded, parts of the graphic may be distorted or missing.

% A safe option when finalizing your figures is to strip out the fonts before 
% you save the files, creating ``outline'' type. This converts fonts to 
% artwork what will appear uniformly on any screen.

% \subsection{Using Labels Within Figures}

% \subsubsection{Figure Axis labels }
% Figure axis labels are often a source of confusion. Use words rather than 
% symbols. As an example, write the quantity ``Magnetization,'' or 
% ``Magnetization M,'' not just ``M.'' Put units in parentheses. Do not label 
% axes only with units. As in Fig. 1, for example, write ``Magnetization 
% (A/m)'' or ``Magnetization (A$\cdot$m$^{-1}$),'' not just ``A/m.'' Do not label axes with a ratio of quantities and 
% units. For example, write ``Temperature (K),'' not ``Temperature/K.'' 

% Multipliers can be especially confusing. Write ``Magnetization (kA/m)'' or 
% ``Magnetization (10$^{3}$ A/m).'' Do not write ``Magnetization 
% (A/m)$\,\times\,$1000'' because the reader would not know whether the top 
% axis label in Fig. 1 meant 16000 A/m or 0.016 A/m. Figure labels should be 
% legible, approximately 8 to 10 point type.

% \subsubsection{Subfigure Labels in Multipart Figures and Tables}
% Multipart figures should be combined and labeled before final submission. 
% Labels should appear centered below each subfigure in 8 point Times New 
% Roman font in the format of (a) (b) (c). 

% \subsection{File Naming}
% Figures (line artwork or photographs) should be named starting with the 
% first 5 letters of the author's last name. The next characters in the 
% filename should be the number that represents the sequential 
% location of this image in your article. For example, in author 
% ``Anderson's'' paper, the first three figures would be named ander1.tif, 
% ander2.tif, and ander3.ps.

% Tables should contain only the body of the table (not the caption) and 
% should be named similarly to figures, except that `.t' is inserted 
% in-between the author's name and the table number. For example, author 
% Anderson's first three tables would be named ander.t1.tif, ander.t2.ps, 
% ander.t3.eps.

% \subsection{Referencing a Figure or Table Within Your Paper}
% When referencing your figures and tables within your paper, use the 
% abbreviation ``Fig.'' even at the beginning of a sentence. Do not abbreviate 
% ``Table.'' Tables should be numbered with Roman Numerals.

% \subsection{Checking Your Figures: The IEEE Graphics Analyzer}
% The IEEE Graphics Analyzer enables authors to pre-screen their graphics for 
% compliance with IEEE Transactions and Journals standards before submission. 
% The online tool, located at
% \underline{http://graphicsqc.ieee.org/}, allows authors to 
% upload their graphics in order to check that each file is the correct file 
% format, resolution, size and colorspace; that no fonts are missing or 
% corrupt; that figures are not compiled in layers or have transparency, and 
% that they are named according to the IEEE Transactions and Journals naming 
% convention. At the end of this automated process, authors are provided with 
% a detailed report on each graphic within the web applet, as well as by 
% email.

% For more information on using the Graphics Analyzer or any other graphics 
% related topic, contact the IEEE Graphics Help Desk by e-mail at 
% graphics@ieee.org.

% \subsection{Submitting Your Graphics}
% Because IEEE will do the final formatting of your paper,
% you do not need to position figures and tables at the top and bottom of each 
% column. In fact, all figures, figure captions, and tables can be placed at 
% the end of your paper. In addition to, or even in lieu of submitting figures 
% within your final manuscript, figures should be submitted individually, 
% separate from the manuscript in one of the file formats listed above in 
% Section \ref{formats}. Place figure captions below the figures; place table titles 
% above the tables. Please do not include captions as part of the figures, or 
% put them in ``text boxes'' linked to the figures. Also, do not place borders 
% around the outside of your figures.

% \subsection{Color Processing/Printing in IEEE Journals}
% All IEEE Transactions, Journals, and Letters allow an author to publish 
% color figures on IEEE Xplore\textregistered\ at no charge, and automatically 
% convert them to grayscale for print versions. In most journals, figures and 
% tables may alternatively be printed in color if an author chooses to do so. 
% Please note that this service comes at an extra expense to the author. If 
% you intend to have print color graphics, include a note with your final 
% paper indicating which figures or tables you would like to be handled that 
% way, and stating that you are willing to pay the additional fee.


% \section{Conclusion}

% A conclusion section is not required. Although a conclusion may review the main points of the paper, do not replicate the abstract as the conclusion. A conclusion might elaborate on the importance of the work or suggest applications and extensions. 

% \section*{Acknowledgment}

% The preferred spelling of the word ``acknowledgment'' in American English is without an ``e'' after the ``g.'' Use the singular heading even if you have many acknowledgments. Avoid expressions such as ``One of us (S.B.A.) would like to thank . . . .'' Instead, write “F. A. Author thanks ... .” In most cases, sponsor and financial support acknowledgments are placed in the unnumbered footnote on the first page, not here.

% \section*{References and Footnotes}

% \subsection{References}

% References need not be cited in text. When they are, they appear on the line, in square brackets, inside the punctuation.  Multiple references are each numbered with separate brackets. When citing a section in a book, please give the relevant page numbers. In text, refer simply to the reference number. Do not use ``Ref.'' or ``reference'' except at the beginning of a sentence: ``Reference [3] shows . . . .'' Please do not use automatic endnotes in {\em Word}, rather, type the reference list at the end of the paper using the ``References'' style.

% Reference numbers are set flush left and form a column of their own, hanging out beyond the body of the reference. The reference numbers are on the line, enclosed in square brackets. In all references, the given name of the author or editor is abbreviated to the initial only and precedes the last name. Use them all; use {\em et al.} only if names are not given. Use commas around Jr., Sr., and III in names. Abbreviate conference titles.  When citing IEEE transactions, provide the issue number, page range, volume number, year, and/or month if available. When referencing a patent, provide the day and the month of issue, or application. References may not include all information; please obtain and include relevant information. Do not combine references. There must be only one reference with each number. If there is a URL included with the print reference, it can be included at the end of the reference.

% Other than books, capitalize only the first word in a paper title, except for proper nouns and element symbols. For papers published in translation journals, please give the English citation first, followed by the original foreign-language citation. See the end of this document for formats and examples of common references. For a complete discussion of references and their formats, see the IEEE style manual at www.ieee.org/authortools.

% \subsection{Footnotes}

% Number footnotes separately in superscripts (Insert $\mid$ Footnote).\footnote{It is recommended that footnotes be avoided (except for the unnumbered footnote with the receipt date on the first page). Instead, try to integrate the footnote information into the text.}  Place the actual footnote at the bottom of the column in which it is cited; do not put footnotes in the reference list (endnotes). Use letters for table footnotes (see Table I). 


% \section*{References}

% \subsection*{Basic format for books:}

% J. K. Author, ``Title of chapter in the book,'' in {\em Title of His Published Book}, xth ed. City of Publisher, (only U.S. State), Country: Abbrev. of Publisher, year, ch. x, sec. x, pp. xxx--xxx.

% \subsection*{Examples:}
% \def\refname{}
% \begin{thebibliography}{34}

% \bibitem{}G. O. Young, ``Synthetic structure of industrial plastics,'' in {\em Plastics}, 2nd ed., vol. 3, J. Peters, Ed. New York, NY, USA: McGraw-Hill, 1964, pp. 15--64.

% \bibitem{}W.-K. Chen, {\it Linear Networks and Systems}. Belmont, CA, USA: Wadsworth, 1993, pp. 123--135.

% \end{thebibliography}

% \subsection*{Basic format for periodicals:}

% J. K. Author, ``Name of paper,'' Abbrev. Title of Periodical, vol. x,   no. x, pp. xxx--xxx, Abbrev. Month, year, DOI. 10.1109.XXX.123--456.

% \subsection*{Examples:}

% \begin{thebibliography}{34}
% \setcounter{enumiv}{2}

% \bibitem{}J. U. Duncombe, ``Infrared navigation Part I: An assessment of feasibility,'' {\em IEEE Trans. Electron Devices}, vol. ED-11, no. 1, pp. 34--39, Jan. 1959,10.1109/TED.2016.2628402.

% \bibitem{}E. P. Wigner, ``Theory of traveling-wave optical laser,''
% {\em Phys. Rev.},  vol. 134, pp. A635--A646, Dec. 1965.

% \bibitem{}E. H. Miller, ``A note on reflector arrays,'' {\em IEEE Trans. Antennas Propagat.}, to be published.
% \end{thebibliography}


% \subsection*{Basic format for reports:}

% J. K. Author, ``Title of report,'' Abbrev. Name of Co., City of Co., Abbrev. State, Country, Rep. xxx, year.

% \subsection*{Examples:}
% \begin{thebibliography}{34}
% \setcounter{enumiv}{5}

% \bibitem{} E. E. Reber, R. L. Michell, and C. J. Carter, ``Oxygen absorption in the earth’s atmosphere,'' Aerospace Corp., Los Angeles, CA, USA, Tech. Rep. TR-0200 (4230-46)-3, Nov. 1988.

% \bibitem{} J. H. Davis and J. R. Cogdell, ``Calibration program for the 16-foot antenna,'' Elect. Eng. Res. Lab., Univ. Texas, Austin, TX, USA, Tech. Memo. NGL-006-69-3, Nov. 15, 1987.
% \end{thebibliography}

% \subsection*{Basic format for handbooks:}

% {\em Name of Manual/Handbook}, x ed., Abbrev. Name of Co., City of Co., Abbrev. State, Country, year, pp. xxx--xxx.

% \subsection*{Examples:}

% \begin{thebibliography}{34}
% \setcounter{enumiv}{7}

% \bibitem{} {\em Transmission Systems for Communications}, 3rd ed., Western Electric Co., Winston-Salem, NC, USA, 1985, pp. 44--60.

% \bibitem{} {\em Motorola Semiconductor Data Manual}, Motorola Semiconductor Products Inc., Phoenix, AZ, USA, 1989.
% \end{thebibliography}

% \subsection*{Basic format for books (when available online):}

% J. K. Author, ``Title of chapter in the book,'' in {\em Title of Published Book}, xth ed. City of Publisher, State, Country: Abbrev. of Publisher, year, ch. x, sec. x, pp. xxx xxx. [Online]. Available: http://www.web.com 

% \subsection*{Examples:}

% \begin{thebibliography}{34}
% \setcounter{enumiv}{9}

% \bibitem{}G. O. Young, ``Synthetic structure of industrial plastics,'' in Plastics, vol. 3, Polymers of Hexadromicon, J. Peters, Ed., 2nd ed. New York, NY, USA: McGraw-Hill, 1964, pp. 15--64. [Online]. Available: http://www.bookref.com. 

% \bibitem{} {\em The Founders Constitution}, Philip B. Kurland and Ralph Lerner, eds., Chicago, IL, USA: Univ. Chicago Press, 1987. [Online]. Available: http://press-pubs.uchicago.edu/founders/

% \bibitem{} The Terahertz Wave eBook. ZOmega Terahertz Corp., 2014. [Online]. Available: http://dl.z-thz.com/eBook/zomega\_ebook\_pdf\_1206\_sr.pdf. Accessed on: May 19, 2014. 

% \bibitem{} Philip B. Kurland and Ralph Lerner, eds., {\em The Founders Constitution}. Chicago, IL, USA: Univ. of Chicago Press, 1987, Accessed on: Feb. 28, 2010, [Online] Available: http://press-pubs.uchicago.edu/founders/ 
% \end{thebibliography}

% \subsection*{Basic format for journals (when available online):}

% J. K. Author, ``Name of paper,'' {\em Abbrev. Title of Periodical}, vol. x, no. x, pp. xxx--xxx, Abbrev. Month, year. Accessed on: Month, Day, year, doi: 10.1109.XXX.123456, [Online].

% \subsection*{Examples:}

% \begin{thebibliography}{34}
% \setcounter{enumiv}{13}

% \bibitem{}J. S. Turner, ``New directions in communications,'' {\em IEEE J. Sel. Areas Commun.}, vol. 13, no. 1, pp. 11--23, Jan. 1995. 

% \bibitem{} W. P. Risk, G. S. Kino, and H. J. Shaw, ``Fiber-optic frequency shifter using a surface acoustic wave incident at an oblique angle,'' {\em Opt. Lett.}, vol. 11, no. 2, pp. 115--117, Feb. 1986.

% \bibitem{} P. Kopyt {\em et al.}, ``Electric properties of graphene-based conductive layers from DC up to terahertz range,'' {\em IEEE THz Sci. Technol.}, to be published. doi: 10.1109/TTHZ.2016.2544142.
% \end{thebibliography}

% \subsection*{Basic format for papers presented at conferences (when available online):}

% J.K. Author. (year, month). Title. presented at abbrev. conference title. [Type of Medium]. Available: site/path/file

% \subsection*{Example:}

% \begin{thebibliography}{34}
% \setcounter{enumiv}{16}

% \bibitem{}PROCESS Corporation, Boston, MA, USA. Intranets: Internet technologies deployed behind the firewall for corporate productivity. Presented at INET96 Annual Meeting. [Online]. Available: http://home.process.com/Intranets/wp2.htp
% \end{thebibliography}

% \subsection*{Basic format for reports  and  handbooks (when available online):}
  
% J. K. Author. ``Title of report,'' Company. City, State, Country. Rep. no., (optional: vol./issue), Date. [Online] Available: site/path/file 

% \subsection*{Examples:}

% \begin{thebibliography}{34}
% \setcounter{enumiv}{17}

% \bibitem{}R. J. Hijmans and J. van Etten, ``Raster: Geographic analysis and modeling with raster data,'' R Package Version 2.0-12, Jan. 12, 2012. [Online]. Available: http://CRAN.R-project.org/package=raster 

% \bibitem{}Teralyzer. Lytera UG, Kirchhain, Germany [Online]. Available: http://www.lytera.de/Terahertz\_THz\_Spectroscopy.php?id=home, Accessed on: Jun. 5, 2014.
% \end{thebibliography}

% \subsection*{Basic format for computer programs and electronic documents (when available online):}

% Legislative body. Number of Congress, Session. (year, month day). {\em Number of bill or resolution, Title}. [Type of medium]. Available: site/path/file
% {\em NOTE:} ISO recommends that capitalization follow the accepted practice for the language or script in which the information is given.

% \subsection*{Example:}

% \begin{thebibliography}{34}
% \setcounter{enumiv}{19}

% \bibitem{}U. S. House. 102nd Congress, 1st Session. (1991, Jan. 11). {\em H. Con. Res. 1, Sense of the Congress on Approval of Military Action}. [Online]. Available: LEXIS Library: GENFED File: BILLS 
% \end{thebibliography}

% \subsection*{Basic format for patents (when available online):}

% Name of the invention, by inventor’s name. (year, month day). Patent Number [Type of medium]. Available:site/path/file

% \subsection*{Example:}

% \begin{thebibliography}{34}
% \setcounter{enumiv}{20}

% \bibitem{}Musical tooth brush with mirror, by L. M. R. Brooks. (1992, May 19). Patent D 326 189
% [Online]. Available: NEXIS Library: LEXPAT File:   DES 

% \end{thebibliography}

% \subsection*{Basic format for conference proceedings (published):}

% J. K. Author, ``Title of paper,'' in {\em Abbreviated Name of Conf.}, City of Conf., Abbrev. State (if given), Country, year, pp. xxx--xxx.

% \subsection*{Example:}

% \begin{thebibliography}{34}
% \setcounter{enumiv}{21}

% \bibitem{}D. B. Payne and J. R. Stern, ``Wavelength-switched passively coupled single-mode optical network,'' in {\em Proc. IOOC-ECOC}, Boston, MA, USA, 1985,
% pp. 585--590.

% \end{thebibliography}

% \subsection*{Example for papers presented at conferences (unpublished):}

% \begin{thebibliography}{34}
% \setcounter{enumiv}{22}

% \bibitem{}D. E behard and E. Voges, ``Digital single sideband detection for inter ferometric sensors,'' presented at the {\em 2nd Int. Conf. Optical Fiber Sensors}, Stuttgart, Germany, Jan. 2--5, 1984.
% \end{thebibliography}

% \subsection*{Basic formatfor patents:}

% J. K. Author, ``Title of patent,'' U. S. Patent x xxx xxx, Abbrev. Month, day, year.

% \subsection*{Example:}

% \begin{thebibliography}{34}
% \setcounter{enumiv}{23}

% \bibitem{}G. Brandli and M. Dick, ``Alternating current fed power supply,'' U. S. Patent 4 084 217, Nov. 4, 1978.
% \end{thebibliography}

% \subsection*{Basic format for theses (M.S.) and dissertations (Ph.D.):}

% a) J. K. Author, ``Title of thesis,'' M. S. thesis, Abbrev. Dept., Abbrev. Univ., City of Univ., Abbrev. State, year.

% b) J. K. Author, ``Title of dissertation,'' Ph.D. dissertation, Abbrev. Dept., Abbrev. Univ., City of Univ., Abbrev. State, year.

% \subsection*{Examples:}

% \begin{thebibliography}{34}
% \setcounter{enumiv}{24}

% \bibitem{}J. O. Williams, ``Narrow-band analyzer,'' Ph.D. dissertation, Dept. Elect. Eng., Harvard Univ., Cambridge, MA, USA, 1993.

% \bibitem{}N. Kawasaki, ``Parametric study of thermal and chemical nonequilibrium nozzle flow,'' M.S. thesis, Dept. Electron. Eng., Osaka Univ., Osaka, Japan, 1993.
% \end{thebibliography}

% \subsection*{Basic format for the most common types of unpublished references:}

% a) J. K. Author, private communication, Abbrev. Month, year.

% b) J. K. Author, ``Title of paper,'' unpublished.

% c) J. K. Author, ``Title of paper,'' to be published.

% \subsection*{Examples:}

% \begin{thebibliography}{34}
% \setcounter{enumiv}{26}

% \bibitem{}A. Harrison, private communication, May 1995.

% \bibitem{}B. Smith, ``An approach to graphs of linear forms,'' unpublished.

% \bibitem{}A. Brahms, ``Representation error for real numbers in binary computer arithmetic,'' IEEE Computer Group Repository, Paper R-67-85.
% \end{thebibliography}

% \subsection*{Basic formats for standards:}

% a) {\em Title of Standard}, Standard number, date.

% b) {\em Title of Standard}, Standard number, Corporate author, location, date.

% \subsection*{Examples:}

% \begin{thebibliography}{34}
% \setcounter{enumiv}{29}


% \bibitem{}IEEE Criteria for Class IE Electric Systems, IEEE Standard 308, 1969.

% \bibitem{} Letter Symbols for Quantities, ANSI Standard Y10.5-1968.
% \end{thebibliography}

% \subsection*{Article number in reference examples:}

% \begin{thebibliography}{34}
% \setcounter{enumiv}{31}

% \bibitem{}R. Fardel, M. Nagel, F. Nuesch, T. Lippert, and A. Wokaun, ``Fabrication of organic light emitting diode pixels by laser-assisted forward transfer,'' {\em Appl. Phys. Lett.}, vol. 91, no. 6, Aug. 2007, Art. no. 061103. 

% \bibitem{} J. Zhang and N. Tansu, ``Optical gain and laser characteristics of InGaN quantum wells on ternary InGaN substrates,'' {\em IEEE Photon.} J., vol. 5, no. 2, Apr. 2013, Art. no. 2600111
% \end{thebibliography}

% \subsection*{Example when using et al.:}

% \begin{thebibliography}{34}
% \setcounter{enumiv}{33}

% \bibitem{}S. Azodolmolky {\em et al.}, Experimental demonstration of an impairment aware network planning and operation tool for transparent/translucent optical networks,'' {\em J. Lightw. Technol.}, vol. 29, no. 4, pp. 439--448, Sep. 2011.
% \end{thebibliography}

\end{document}


