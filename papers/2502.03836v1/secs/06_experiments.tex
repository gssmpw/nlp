\section{Experiments}
\xcy{
\subsection{Experimental setup}

\subsubsection{Datasets and Metrics.} In line with previous studies, we utilize the Human3.6M~\cite{Human3.6m}, COCO~\cite{COCO}, MPII~\cite{MPII}, and MPI-INF-3DHP~\cite{3DHP} datasets for training. These image and video datasets are employed to train both the regressor network and the diffusion model. We evaluate our method on the 3DPW test split~\cite{3DPW} and the Human3.6M validation split~\cite{Human3.6m}. For 3D pose accuracy, we report the Mean Per Joint Position Error (MPJPE), as well as the MPJPE after rigid alignment of the predicted poses with the ground truth (PA-MPJPE).

\subsubsection{Implementation Details.} First, we train the initial prediction regressor. We adopt the ViT-H/16~\cite{ViT} and the standard transformer decoder~\cite{transformer}, as proposed in~\cite{HMR2.0}. We use ChatPose~\cite{ChatPose} to extract descriptive information, and AlphaPose~\cite{fang2022alphapose} as an additional keypoint detector to provide 2D keypoint data. Next, we align the pose and text features in the latent space. Finally, we train the diffusion optimization module while keeping the other modules frozen. For training the regressor, we use 20 epochs with a batch size of 128 and a learning rate of $1e{-5}$. The pose-text alignment is performed across multiple datasets for 100 epochs with a batch size of 256. During diffusion training, we run 30 epochs with a batch size of 128 on four RTX 3090 GPUs.

\vspace{-4mm}

\subsection{Comparisons with the state-of-art methods}

\begin{figure}[t] % 使用 figure* 环境
\centering
\includegraphics[width=0.5\textwidth]{figs/quantative.pdf}
% \vspace{-3mm}
\caption{\textbf{Qualitative results.} From left to right: RGB image, ProHMR~\cite{kolotouros2021probabilistic}, HMR2.0~\cite{HMR2.0}, and our method. Our approach ensures accurate 3D joint positions with minimal depth ambiguity while achieving robust front-facing alignment.}
\label{fig:quantative}
% \vspace{-3mm}
\end{figure}

We compare our method with state-of-the-art human mesh recovery approaches on the Human3.6M and 3DPW datasets, reporting MPJPE and PA-MPJPE metrics in \tabref{tab:3d_metrics}. Since our method incorporates multiple conditional constraints, it outperforms most existing methods. Specifically, the PA-MPJPE improves by 0.4 on the 3DPW dataset and by 1.2 on the Human3.6M dataset. We also present a qualitative comparison in \figref{fig:quantative}. While HMR2.0 exhibits deviations in mesh alignment, ProHMR faces challenges, particularly in cases of depth ambiguity, leading to poorer optimization results. In contrast, our method demonstrates enhanced robustness, as the textual information provides supplementary context that helps mitigate the limitations of unreliable 2D observations.

\begin{table}[t]
  \centering
   \caption{\textbf{Reconstruction evaluation on 3D joint accuracy.} We report reconstruction errors on the 3DPW and Human3.6M datasets. 
    }
  \footnotesize
    \begin{tabular}{l|cc|cc}
    \toprule[1pt]
    \multicolumn{1}{c|}{\multirow{2}{*}{Method}} & \multicolumn{2}{c|}{3DPW} & \multicolumn{2}{c}{Human3.6M} \\
         & MPJPE & PA-MPJPE & MPJPE & PA-MPJPE \\
         \midrule[1pt]
        HMR~\cite{kanazawa2018end} & 130.0 & 76.7 & 88.0 & 56.8 \\
        SPIN~\cite{kolotouros2019learning} & 96.9  & 59.2 & 62.5  & 41.1 \\
        DaNet~\cite{zhang2019danet} &  -  &  56.9   & 61.5  & 48.6 \\
        PyMAF~\cite{zhang2021pymaf}               & 92.8        & 58.9         & 57.7         & 40.5         \\
        ProHMR~\cite{kolotouros2021probabilistic} & - & 55.1 & - & 39.3 \\
        PARE~\cite{kocabas2021pare}                & 82.0& 50.9 & 76.8         & 50.6         \\
        PyMAF-X~\cite{zhang2023pymaf}             & 78.0& 47.1 & 54.2 & 37.2 \\
        HMR 2.0~\cite{HMR2.0}      & 70.0         & 44.5  & \best{44.8}  & 33.6  \\
        \textbf{Ours} & \best{69.3} & \best{43.9} & 47.7 & \best{32.4} \\
    \bottomrule[1pt]
    \end{tabular}%
  \label{tab:3d_metrics}%
  
\end{table}%

\vspace{-3mm}

\subsection{Ablation study}

\begin{table}[t]
\centering
\footnotesize
\caption{\textbf{Ablation study.} The initial parameters are regressed by the regressor. We report the results under different conditions in the diffusion process. All numbers are in millimeters (mm).}
\begin{tabular}{l|cc|cc}
    \toprule[1pt]
    \multirow{2}{*}{Method} & \multicolumn{2}{c|}{3DPW} & \multicolumn{2}{c}{Human3.6M} \\
                  & MPJPE & PA-MPJPE & MPJPE & PA-MPJPE \\
    \midrule[1pt]
    Standard Gaussian  & 87.8 & 54.9 & 62.8 & 44.6  \\
    Initial Prediction & 73.4 & 47.5 & 56.4 & 34.0  \\
    \ w/ image      &  70.6   & 45.6  &  53.5  &  33.4   \\
    \ w/ keypoints  &  72.3   & 46.0  &  54.4  &  33.7\\
    \ w/ text       &  72.8   & 47.0  &  56.2  &  33.9 \\
    \ w/o keypoints &  70.3   & 45.1  &  52.7  &  33.1 \\
    \ w/o text      &  69.8   & 44.5  &  48.3  &  32.8 \\
    \ w/ all conditions &  \best{69.3}   & \best{43.9}  &  \best{47.7}  &  \best{32.4} \\
    
    \bottomrule[1pt]
\end{tabular}
\vspace{-3mm}
\label{table2}
% \vspace{-5mm}
\end{table}

\subsubsection{Initial Prediction.}
We investigated the importance of the initial regressor and found that, compared to a standard Gaussian distribution, using one with prior knowledge of human pose leads to better optimization results.
\subsubsection{Multi-modal Conditions}
We further investigate the impact of different conditions during the optimization process. We report the results for three scenarios: using a single modality condition (denoted as "w/ modality"), using all conditions except one modality (denoted as "w/o modality"), and using all conditions for optimization. Our findings show that the diffusion adaptation process effectively enhances the accuracy of initial predictions, achieving the best results when all three modalities are used. The most significant improvement comes from the image features and keypoint information, while the inclusion of text information further refines pose optimization. Text information provides additional constraints, helping to guide the optimization process and preventing it from getting stuck in local minima caused by noisy 2D keypoints.

\vspace{-1mm}

}