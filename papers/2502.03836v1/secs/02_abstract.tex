\begin{abstract}

Human mesh recovery can be approached using either regression-based or optimization-based methods. Regression models achieve high pose accuracy but struggle with model-to-image alignment due to the lack of explicit 2D-3D correspondences. In contrast, optimization-based methods align 3D models to 2D observations but are prone to local minima and depth ambiguity. In this work, we leverage large vision-language models (VLMs) to generate interactive body part descriptions, which serve as implicit constraints to enhance 3D perception and limit the optimization space. Specifically, we formulate monocular human mesh recovery as a distribution adaptation task by integrating both 2D observations and language descriptions. To bridge the gap between text and 3D pose signals, we first train a text encoder and a pose VQ-VAE, aligning texts to body poses in a shared latent space using contrastive learning. Subsequently, we employ a diffusion-based framework to refine the initial parameters guided by gradients derived from both 2D observations and text descriptions. Finally, the model can produce poses with accurate 3D perception and image consistency. Experimental results on multiple benchmarks validate its effectiveness. The code will be made publicly available.

\begin{IEEEkeywords}
human mesh recovery, multi-modal signal, diffusion for optimization
\end{IEEEkeywords}

\end{abstract}