\newpage
\section*{\centering Appendices}




\section{Data Summary}
\label{appndx:data_summary}
Table~\ref{tab:data_summary} summarizes the dataset statistics used in this study. The numbers of official test samples are reported. 
Due to the budget, we evaluated Llama-3-8B on the full test set, GPT-4o-mini and Llama-3-70B on a random sample of up to 2,000 test set instances, and GPT-4o on 1,000 test set instances.

Additionally, the number of temporal relations considered in each dataset is included in Table~\ref{tab:data_summary}. TempEvalQA-Bi and TRACIE focus mainly on the before-after relation. MCTACO includes diverse temporal relations, and the number of annotated candidates is reported.
The questions in MCTACO are categorized into 5 question types, and examples for each type are provided in Figure~\ref{fig:mctaco_examples}.

\begin{table}[h]
\centering
\scalebox{0.95}{
\begin{tabular}{c|cc}
\toprule
                             & \#Test      & \#Temporal relations   \\ \midrule
TempEvalQA-Bi                & 448      &  2           \\ 
 TRACIE                      & 4248     &  2          \\
 MCTACO                      & 9442     & 1-19 \\ \bottomrule
\end{tabular}
}
\caption{Dataset Statistics.
For TempEvalQA-Bi, the numbers represent the total number of questions. For TRACIE, the numbers refer to the number of story-hypothesis pairs. For MCTACO, the numbers reflect question-and-answer candidate pairs.}
\label{tab:data_summary}
\end{table}
\begin{table*}[h]
\centering
\scalebox{0.85}{
\begin{tabular}{|l|l|l|l|}
\hline
\textbf{Dataset} & \textbf{Relation} & \textbf{Revision Type} & \textbf{Example} \\ \hline
TempEvalBi-QA & $r_1(e_1,e_2)$ & Original & Did they get married \textbf{after} they moved to Maine? \\ 
 & $r_2(e_1,e_2)$ & $r_1 \rightarrow r_2$ & Did they get married \textbf{before} they moved to Maine? \\ \hline
TRACIE & $r_1(e_1,e_2)$ & Original & Chad found the cap \textbf{before} he got off of the ride. \\ 
 & $r_2(e_1,e_2)$ & $r_1 \rightarrow r_2$ & Chad found the cap \textbf{after} he got off of the ride. \\ \hline
MCTACO: Duration & $r_1(e_1)$ & Original & It has existed for \textbf{1 year}. \\ 
 & $r_2(e_1)$ & $r_1 \rightarrow r_2$ & It has existed for \textbf{centuries}. \\ \hline
MCTACO: Frequency & $r_1(e_1)$ & Original & It rained \textbf{a couple of times} in the summer. \\ 
 & $r_2(e_1)$ & $r_1 \rightarrow r_2$ & It rained \textbf{every month} in the summer. \\ \hline
MCTACO: Stationarity & $r_1(e_1)$ & Original & She is \textbf{still} in Ranchipur. \\  
 & $r_2(e_1)$ & $r_1 \rightarrow r_2$ & She is \textbf{not} in Ranchipur. \\ \hline
MCTACO: Ordering & $r_1(e_1,e_2)$ & Original & They \textbf{went to the store} after they were put in jail. \\  
 & $r_1(e_1,e_3)$ & $e_2 \rightarrow e_3$ & They \textbf{repented} after they were put in jail. \\ \hline
MCTACO: Typical Time & $r_1(e_1)$ & Original & Durer died \textbf{40 years} later. \\  
 & $r_2(e_1)$ & $r_1 \rightarrow r_2$ & Durer died \textbf{360 years} later. \\ \hline
\end{tabular}
}
\vspace{-3mm}
\caption{The full range of temporal aspects targeted by each dataset and types of counterfactuals addressed by our framework.
The examples illustrate how counterfactual questions modify the semantics regarding temporal relations ($r_1$,$r_2$) for events ($e_1$,$e_2$,$e_3$).
}
\vspace{-5mm}
\label{tab:full_contrast_examples}
\end{table*}

\begin{figure}[h]
    \centering
    \includegraphics[width=1\linewidth]{figures/mctaco_examples.pdf}
    \caption{Examples of MCTACO Question Types. MCTACO covers various temporal aspects including event duration, frequency, stationarity, ordering, and typical time.}
    \label{fig:mctaco_examples}
\end{figure}

Table~\ref{tab:full_contrast_examples} demonstrates counterfactual types and examples addressed across the targeted temporal aspects in our datasets.
Among the five temporal aspects in the MCTACO dataset, we generate counterfactuals for duration, frequency, and typical time in the same way by intervening in the temporal property, modifying $r_1(e1)$ to $r_2(e1)$.

\section{Details of Evaluation Settings}
\label{appndx:hyperparam}

This section outlines the detailed evaluation settings, including hyperparameters, resources, efficiency, and parsing methods.
We use greedy decoding for \textbf{SP}, \textbf{CoT}, and \textbf{CCP}. For \textbf{Consistency}, \textbf{Reflect}, and \textbf{Debate}, we adopt the approach from~\citet{wang2023self}, employing top-k sampling with 
$k=40$ and a temperature of 0.5 for the LLaMA model. For GPT-based models, we set the temperature to 0.7.
\textbf{Consistency} samples 40 outputs from the decoder. \textbf{Reflect} refines the output iteratively for two iterations, including the initial output. In \textbf{Debate}, three agents engage in a debate over two rounds\cite{du2024improving}.
The implementations of the latter two baselines (\textbf{Reflect}, \textbf{Debate}) are based on the GitHub repository~\footnote{\url{https://github.com/composable-models/llm_multiagent_debate}} from~\citet{du2024improving}. Single-run performances are reported.

We note that our method prompts 3 times: for counterfactual question generation, counterfactual answer generation, and original question's answer generation, whose efficiency is compatible with or even more efficient than the three baselines. We also note that the Consistency baseline of Llama-3-70B cannot be reported due to its computation inefficiency.


For resources, we used the Transformers library~\cite{wolf-etal-2020-transformers} and vLLM~\cite{kwon2023efficient} with 4 RTX A6000 GPUs for Llama-3 models. We used Openai API~\footnote{\url{platform.openai.com}} for GPT models.
For output parsing, the models generate the final answer after the phrase \texttt{``Final answer:''}. Counterfactual exemplars are generated by modifying each dataset's questions, hypotheses, and candidate answers.

% Please add the following required packages to your document preamble:

% Beamer presentation requires \usepackage{colortbl} instead of \usepackage[table,xcdraw]{xcolor}
\begin{table*}[t]
\centering
\caption{Main Results. Eurus-2-7B-PRIME demonstrates the best reasoning ability.}
\label{tab:main_results}
\resizebox{\textwidth}{!}{
\begin{tabular}{lcccccc}
\toprule
\textbf{Model}                     & \textbf{AIME 2024}                           & \textbf{MATH-500} & \textbf{AMC}          & \textbf{Minerva Math} & \textbf{OlympiadBench} & \textbf{Avg.}          \\ \midrule
\textbf{GPT-4o}                    & 9.3                                          & 76.4              & 45.8                  & 36.8                  & \textbf{43.3}          & 43.3                   \\
\textbf{Llama-3.1-70B-Instruct}    & 16.7                                         & 64.6              & 30.1                  & 35.3                  & 31.9                   & 35.7                   \\
\textbf{Qwen-2.5-Math-7B-Instruct} & 13.3                                         & \textbf{79.8}     & 50.6                  & 34.6                  & 40.7                   & 43.8                   \\
\textbf{Eurus-2-7B-SFT}            & 3.3                                          & 65.1              & 30.1                  & 32.7                  & 29.8                   & 32.2                   \\
\textbf{Eurus-2-7B-PRIME}          & \textbf{26.7 {\color[HTML]{009901} (+23.3)}} & 79.2 {\color[HTML]{009901}(+14.1)}      & \textbf{57.8 {\color[HTML]{009901}(+27.7)}} & \textbf{38.6 {\color[HTML]{009901}(+5.9)}}  & 42.1 {\color[HTML]{009901}(+12.3) }          & \textbf{48.9 {\color[HTML]{009901}(+ 16.7)}} \\ \bottomrule
\end{tabular}
}
\end{table*}
\begin{table}[h]
\centering
\scalebox{1.0}{
\begin{tabular}{c|c|c|ccc}
\toprule
Models & Methods   & \multicolumn{2}{c}{MCTACO}    \\ \midrule
       &           &   EM         & F1              \\ \midrule
  % & CoT       & 41.1        & 56.7       \\ 
GPT-4o  & MCQA-CoT  & 51.7        & 67.7        \\ 
-mini   &  \ours   & \textbf{58.9} & \textbf{78.6}  \\ \midrule
 % & CoT       & 50.3        & 67.2          \\ 
GPT-4o       & MCQA-CoT  & 65.6        & 73.3           \\ 
       &  \ours    & \textbf{66.2} & \textbf{80.2}    \\
 \bottomrule
\end{tabular}
}
\caption{Performance comparison on MCTACO with multiple-choice question answering setting.}
\label{tab:mcqa}
\end{table}




\section{Details of Main Results}

\subsection{Full Main Results}
\label{appndx:full_main_result}
Table~\ref{tab:main} shows the performance of our method compared with baseline methods on relative event understanding tasks. The results show that our method outperforms the baselines across the board.




\subsection{Task generalizability}
To demonstrate that our solution extends beyond binary question answering to multiple-choice question answering (MCQA), we evaluated the performance of GPT models using the original MCTACO evaluation setting~\cite{zhou2019going}. While our primary evaluation decomposed the multiple-choice format into binary questions to measure inconsistency, it can be reconstructed for multiple-choice evaluation. We additionally introduced a baseline for MCQA (MCQA-CoT) that provides the context, question, and all candidate answers, generating one or more correct answers step-by-step. 
% We excluded the Llama-3-8B model due to its inability to match the required output format. 
The results in Table~\ref{tab:mcqa} indicate that our method (CCP) outperforms the MCQA-CoT baseline on multiple-choice tasks, demonstrating its effectiveness in the MCQA setting.


% \section{K-shot}
\section{Further Analysis}

\subsection{Generated vs Retrieved Questions}

% \input{tables/mctaco_each_type}
\begin{figure}[h]
    \centering
    \includegraphics[width=1\linewidth]{figures/mctaco_retq_genq.pdf}
    \vspace{-8mm}
    \caption{Comparison between counterfactual example collection methods on MCTACO with different models.}
    \label{fig:mctaco_retq_genq}
    \vspace{-3mm}
\end{figure}

% Table~\ref{tab:mctaco_each_type}
We tested whether our claim in Figure~\ref{fig:mctaco_each_type} can be generalized to other models. Figure~\ref{fig:mctaco_retq_genq} consistently confirms that creating counterfactual questions by generation handles diverse temporal relations better than retrieving questions across different models.



\begin{figure}[h]
    \centering
    \includegraphics[width=1\linewidth]{figures/num_cq.pdf}
    \vspace{-8mm}
    \caption{Inconsistency changes with the different number of counterfactual questions. The Llama-3-8B model is used.}
    \label{fig:num_cq}
    \vspace{-3mm}
\end{figure}

\begin{table}[h]
\centering
\scalebox{1.0}{
\begin{tabular}{c|cc|ccc}
\toprule
             & \multicolumn{5}{c}{MCTACO}    \\ \midrule
       & \#P-Q  & \#C     &  ACC          & F1           & INC            \\ \midrule
 CoT  &  3     & 3           & 77.6        & 69.8      & 63.4          \\
 CoT  &  12     & 12         & 78.9        &   72.2    &   60.0      \\ \midrule
 \ours &  3    & 11          & 78.7  & 81.7 & 57.7 \\ 
 \ours &   12   & 26    & \textbf{85.0} & \textbf{82.2} & \textbf{49.8} \\
 \bottomrule
\end{tabular}
}
\caption{Performance comparison of Llama-3-8B on MCTACO with the different number of ICL examples.}
\label{tab:moreshot}
\end{table}

\subsection{Number of In-context Learning Examples}

Our approach inevitably introduces additional counterfactual examples during in-context learning (ICL), leading to a higher total number of shots compared to the baseline. To ensure a more competitive baseline, we increased the total number of shots in the baseline. In the MCTACO dataset and with the Llama model, we additionally experimented with the 12-shot CoT, which includes 12 passage (P)-question (Q)-candidate (C) pairs, and compared them with our 3-shot. We note that our 3-shot examples include 3 passage-question pairs and 11 candidates.

The results in Table~\ref{tab:moreshot} demonstrate that our method significantly outperforms the CoT, even with the increased number of examples in the baseline (INC score: 60.0 for CoT vs. 57.7 for Ours). This indicates that the performance gains are not simply due to the inclusion of more examples but are primarily driven by leveraging temporal constraints through counterfactual questions to enhance reasoning.

Additionally, we tested whether our approach benefits from additional ICL examples. The results in the last row of Table~\ref{tab:moreshot} confirm this, showing an improvement in INC score from 57.7 to 49.8, further validating the potential performance gain of our method.

\subsection{Number of Counterfactual Questions}
We analyze the impact of varying the number of counterfactual questions on performance by testing with 1, 3, 5, and 7 questions. As shown in Figure~\ref{fig:num_cq}, performance is highest with a single counterfactual question, with degradation observed as the number increases. The result aligns with findings from prior studies, where incorporating excessive counterfactual or contrastive information in prompts often results in diminished performance~\cite{yao2024largelanguagemodelscontrastive,fang2024counterfactualdebatingpresetstances,storaï2024harphesitationawarereframingtransformer}. For instance, multiple information degrade the performance in arithmetic and symbolic reasoning~\cite{yao2024largelanguagemodelscontrastive}, multi-hop and commonsense question answering~\cite{fang2024counterfactualdebatingpresetstances}, and speculative decoding~\cite{storaï2024harphesitationawarereframingtransformer}. These results indicate that such a phenomenon is not specific to our method but highlights a broader limitation in LLMs' ability to process multiple counterfactual information effectively.

\begin{table}[ht]
\centering
\scalebox{0.85}{
\begin{tabular}{cc|cc|cc}
\toprule
            &             & \multicolumn{2}{c}{TimeQA} & \multicolumn{2}{c}{TimexNLI}        \\
\midrule
            &             & ACC         & F1           & ACC & F1  \\
\midrule
Llama & 3 shot       & \textbf{34.3} & 40.8          & 68.0                    & 65.3            \\
 3-8B      & CoT 3 shot   & 32.3          & 38.4          & \textbf{74.0}                    & \textbf{73.3}          \\
            & \ours 3 shot   & 34            & \textbf{41.5} &            67.3       &   62.2 \\
\midrule
GPT-4o & 3 shot       & 40                                & 52.36                             & 86.4                    & 85.3             \\
 -mini     & CoT 3 shot   & \textbf{43.3}                     & \textbf{56.75}                    & \textbf{90.4}           & \textbf{90.3}        \\
            & \ours 3 shot  & 41                                & 53.59                             & 90.3                    & 90.0          \\ 
\bottomrule
\end{tabular}
}
\caption{Performance comparison on TimeQA and TimexNLI.}
\label{tab:bad}
\end{table}



\section{Details of Limitations}

\subsection{Tasks with temporal indicators}
\label{appndx:detail_limitation_temporal_indicator}
Table~\ref{tab:bad} shows the experimental results for the tasks requiring the understanding of temporal indicators. We evaluated our method on TimeQA~\cite{chen2dataset}, the event-time ordering task, and TimexNLI-T1~\cite{thukral2021probing}, the time-time ordering task, where \ours showed limited performance gains.

\subsection{Prompt Sensitivity}
\label{appndx:detail_limitation_prompt_sensitivity}
\begin{table*}[h]
    \centering
    \renewcommand{\arraystretch}{1.2}
    \setlength{\tabcolsep}{5pt}
    \begin{tabular}{l |l |c c c c c c c c c}
        \toprule
        \multirow{2}{*}{Models} & \multirow{2}{*}{Prompts} & \multicolumn{3}{c}{TempEvalQA-Bi} & \multicolumn{3}{c}{TRACIE} & \multicolumn{3}{c}{MCTACO} \\
        \cmidrule(lr){3-5} \cmidrule(lr){6-8} \cmidrule(lr){9-11}
        & & ACC & F1 & INC & ACC & F1 & INC & ACC & F1 & INC \\
        \midrule
        Llama-3-8B & CoT & 69.6 & 70.6 & 50.0 & 63.0 & 64.9 & 56.0 & 77.6 & 69.8 & 63.4 \\
                   & CCP (e2e) & \underline{75.4} & \underline{73.6} & \underline{33.9} & \underline{66.8} & \underline{65.4} & \underline{46.7} & \textbf{83.1} & \textbf{81.0} & \textbf{53.7} \\
                   & CCP        & \textbf{75.9} & \textbf{75.2} & \textbf{32.7} & \textbf{68.8} & \textbf{70.4} & \textbf{39.8} & \underline{78.7} & \underline{81.7} & \underline{57.7} \\ \hline
        GPT-4o-mini & CoT & \underline{81.3} & 79.9 & 29.0 & 73.2 & 68.5 & 42.7 & 80.9 & 73.7 & 58.9 \\
                    & CCP (e2e) & 80.8 & \underline{80.2} & \underline{27.7} & \underline{79.0} & \underline{78.3} & \underline{28.0} & \underline{87.1} & \underline{84.4} & \underline{52.1} \\
                    & CCP      & \textbf{88.8} & \textbf{88.7} & \textbf{19.6} & \textbf{82.5} & \textbf{81.2} & \textbf{20.2} & \textbf{87.8} & \textbf{85.8} & \textbf{42.1} \\
        \bottomrule
    \end{tabular}
    \caption{Performance comparison with different prompts. The best and second-best performances are in boldface and underlined, respectively.}
    \label{tab:prompt_sensitivity}
\end{table*}
Our key decision in the prompt design was to separate the counterfactual question generation prompt (CCP), described in Appendix~\ref{appndx:prompt_examples_gen_counter_q}, from the prompt that answers counterfactual and original questions, detailed in Appendix~\ref{appndx:prompt_examples_gen_ans}. If we generate counterfactual questions and answers in an end-to-end manner using only the prompt in Appendix~\ref{appndx:prompt_examples_gen_ans} (CCP(e2e)), while it still outperforms CoT, the performance slightly decreases, as shown in Table~\ref{tab:prompt_sensitivity}.


\section{Usage of AI Assistants}
ChatGPT was employed to generate answers in the prompt examples.


\newpage

\section{Prompt Templates}
We list the prompts that we used.

\subsection{Prompt Templates for Generating Counterfactual Questions}
\label{appndx:prompt_examples_gen_counter_q}

To generate the counterfactual questions, we use the prompts provided below.
We control question aspects by tailoring ICL examples and prompting the model to follow few-shot examples with the prompt: (\texttt{``following previous examples''}). 



\begin{myColorbox}[]{TempEvalQA-Bi}

\textbf{[System]}

You are an expert in temporal knowledge. Generate one related question to help determine the correctness of the original question, following the previous examples.

\textbf{[User]}

Question: Is Farkas sent into space on board the Soyuz before McBride on board the Orbiter Challenger?

\textbf{[Assistant]}

Is Farkas sent into space on board the Soyuz after McBride on board the Orbiter Challenger?

\textbf{[User]}

Question: Is McBride on board the Orbiter Challenger after Farkas was made a brigadier general?

\textbf{[Assistant]}

Is McBride on board the Orbiter Challenger before Farkas was made a brigadier general?

\textbf{[User]}

Question: Is Farkas was appointed military attache at the Hungarian embassy before he was made a brigadier?

\textbf{[Assistant]}

Is Farkas was appointed military attache at the Hungarian embassy after he was made a brigadier?
\end{myColorbox}

\begin{myColorbox}[]{TRACIE}

\textbf{[System]}

You are an expert in temporal knowledge. Generate one related hypothesis to help determine the plausibility of the original hypothesis, following previous examples.

\textbf{[User]}

Hypothesis: Chad looked for his baseball cap starts after he got off the ride.

\textbf{[Assistant]}

Chad looked for his baseball cap starts before he got off the ride.

\textbf{[User]}

Hypothesis: Tom's friend says he is not gay starts before One day he noticed someone else asking

\textbf{[Assistant]}

Tom's friend says he is not gay starts after One day he noticed someone else asking

\textbf{[User]}

Hypothesis: Liza read a book on economics in the library starts before she got to the library

\textbf{[Assistant]}

Liza read a book on economics in the library starts after she got to the library 

\end{myColorbox}



\begin{myColorbox}[]{MCTACO}

\textbf{[System]}

You are an expert in temporal commonsense knowledge. Generate related candidate answers that are either plausible or implausible that help determine the correctness of the original candidate answer following the previous examples.

\textbf{[User]}

Passage: The majority religion during the centuries of Ottoman rule, though a significant Christian minority remained. Question: What happened before Islam was the majority religion? Candidate answer: christianity was the majority religion.

\textbf{[Assistant]}

The Ottoman Empire had just begun /\ The spread of Byzantine influence in the region /\ The emergence of Islam as a minor religion 

\textbf{[User]}

Passage: It's hail crackled across the comm, and Tara spun to retake her seat at the helm. Question: How long was the storm? Candidate answer: 6 years.

\textbf{[Assistant]}

an hour /\ a week /\ a month


\textbf{[User]}

Passage: Passage: About 30\% of Ratners's profit already is derived from the U.S. Question: Is Ratners's profit derived from the U.S. today? Candidate answer: yes.

\textbf{[Assistant]}

no

\end{myColorbox}


\subsection{Prompt Templates for CCP and Baselines}
\label{appndx:prompt_examples_gen_ans}
We only show the SP, CoT, and CCP since Consistency is the method that leverages CoT multiple times, and the Reflect and Debate follow the implementation of~\citet{du2024improving} after CoT. 

Given the system and user message, the model generates an assistant's message. In our approach, the model is prompted twice: first to create an assistant's answer to the counterfactual question, and then to generate an assistant's answer to the original question. Though we use 3-shot examples, we provide 1 example for simplicity.



Prompt templates used to generate answers for the TempevalQA-Bi dataset are provided below.

\begin{myColorbox}[]{TempevalQA-Bi: SP}

\textbf{[System]}

You are an expert in temporal knowledge. Answer the question according to the article. Answer the question yes or no (Format: final answer: yes/no)

\textbf{[User]}

Article: Farkas, an air force captain, was sent into space on board the Soyuz 36 on May 26, 1980. He spent six days aboard the Salyut 6 spacecraft with three Soviet astronauts, Valery Kubasov, Leonid Popov and Valery Riumin. McBride, 54, of Lewisburg, West Virginia, was part of a seven-member crew aboard the Orbiter Challenger [...] 

Question: Is Farkas sent into space on board the Soyuz before McBride on board the Orbiter Challenger? 
\newline 
\textbf{[Assistant]}
\newline
Final answer: yes
\end{myColorbox}

\begin{myColorbox}[]{TempevalQA-Bi: CoT}

\textbf{[System]}

You are an expert in temporal knowledge. Answer the question according to the article. Let's think step by step to answer the question. The answer should be "yes" or "no" followed by "Final answer: ".\newline
\textbf{[User]}

Article: [...] \newline
Question: Is Farkas sent into space on board the Soyuz before McBride on board the Orbiter Challenger? \newline 
\textbf{[Assistant]}
\newline
Let's think step by step.
Farkas was sent into space on May 26, 1980 and McBride was on board the Orbiter Challenger in October 1984. May 26, 1980 is before October 1984. So the answer is: yes.

Final answer: yes

\end{myColorbox}

\begin{myColorbox}[]{TempevalQA-Bi: CCP}

\textbf{[System]}

You are an expert in temporal knowledge. Answer the question according to the article. To improve understanding, follow these steps: Answer the related question to better understand the timeline (Format: answer: [yes/no]) Using the answers to the related questions as the additional information, answer the original question yes or no (Format: final answer: [yes/no]). \newline 
\textbf{[User]}

Article: [...]\newline
Original question: Is Farkas sent into space on board the Soyuz before McBride on board the Orbiter Challenger?\newline
Related question: Is Farkas sent into space on board the Soyuz after McBride on board the Orbiter Challenger?\newline 
Answer the related question\newline
\textbf{[Assistant]} \textit{\textcolor{blue}{\# Prediction for the counterfactual question}}

Farkas was sent into space on board the Soyuz 36 on May 26, 1980. McBride was part of the crew on board the Orbiter Challenger in October 1984. 
Since 1980 is before 1984, no, Farkas was not sent into space after McBride.\newline
answer: no\newline 
\textbf{[User]}
\newline
Now, answer the original question. Question: Is Farkas sent into space on board the Soyuz before McBride on board the Orbiter Challenger?\newline
\textbf{[Assistant]} \textit{\textcolor{blue}{\# Prediction for the original question}}

Farkas was sent into space on May 26, 1980 and McBride was on board the Orbiter Challenger in October 1984. May 26, 1980 is before October 1984. So the answer is: yes. \newline
Final answer: yes

\end{myColorbox}

Prompt templates used to generate answers for
the TRACIE dataset are provided below.

\begin{myColorbox}[]{TRACIE: SP}

\textbf{[System]}

You are an expert in temporal knowledge. You can understand the implicit temporal relationships between events to make judgments. Your task is to determine the relationship between the story and hypothesis. A `positive' label means the hypothesis can be inferred from the story or logically sound. A `negative' label means the hypothesis cannot be inferred from the story. You must answer only with the labels `positive' or `negative'. (Format: final answer: positive/negative)

\textbf{[User]}

Story: Chad looked for his baseball cap starts after he got off the ride story: Chad had gone to an amusement park. He was riding on the roller coaster. \newline
Hypothesis: Chad looked for his baseball cap starts after he got off the ride.
\newline 
\textbf{[Assistant]}
\newline
Final answer: positive
\end{myColorbox}

\begin{myColorbox}[]{TRACIE: CoT}

\textbf{[System]}

You are an expert in temporal knowledge. You can understand the implicit temporal relationships between events to make judgments. Your task is to determine the relationship between the story and hypothesis. A `positive' label means the hypothesis can be inferred from the story or logically sound. A `negative' label means the hypothesis cannot be inferred from the story. You must give final answer only with the labels `positive' or `negative' after the ``Final answer:'' prompt.
Let's think step by step to evaluate the relationship.

\textbf{[User]}

Story: [...]	\newline
Hypothesis: Chad looked for his baseball cap starts after he got off the ride.

\textbf{[Assistant]}

Let's think step by step: \newline
The story states that Chad's baseball cap fell off during the roller coaster ride.\newline
It also mentions that Chad found the cap after he got off the ride.
To find something, one typically needs to look for it first.
Since Chad found the cap after getting off the ride, he must have started looking for it after getting off the ride.
The hypothesis directly aligns with this sequence of events.
\newline
Final answer: positive

\end{myColorbox}

\begin{myColorbox}[]{TRACIE: CCP}

\textbf{[System]}

You are an expert in temporal knowledge. You can understand the implicit temporal relationships between events to make judgments. Your task is to determine the relationship between the story and hypothesis. A `positive' label means the hypothesis can be inferred from the story or logically sound. A `negative' label means the hypothesis cannot be inferred from the story. You must give final answer only with the labels `positive' or `negative'. 
To improve understanding, follow these steps: Evaluate the related hypothesis to help clarify the story's timeline (Format: answer: [positive/negative]). Using answers to the related hypothesis as the additional information, evaluate the original hypothesis (Format: final answer: [positive/negative]).

\textbf{[User]}

Story: [...] \newline
Hypothesis: Chad looked for his baseball cap starts after he got off the ride. \newline
Related hypothesis: Chad looked for his baseball cap starts before he got off the ride. \newline
Answer the related hypothesis

\textbf{[Assistant]}  \textit{\textcolor{blue}{\# Prediction for the counterfactual question}}

The hypothesis conflicts with the story, as Chad finds the cap after the ride.
\newline
Answer: negative
\newline 
\textbf{[User]}
\newline
Now, evaluate the original hypothesis: Chad looked for his baseball cap starts after he got off the ride.

\textbf{[Assistant]}  \textit{\textcolor{blue}{\# Prediction for the original question}}

Chad rides the roller coaster while wearing a baseball cap, which falls off during the ride. After getting off the ride, Chad finds the cap. The hypothesis suggests that Chad started looking for the cap after getting off the ride, which aligns with the story's sequence of events.
\newline
Final answer: positive

\end{myColorbox}


Prompt templates used to generate answers for the MCTACO dataset are provided below.

\begin{myColorbox}[]{MCTACO: SP}

\textbf{[System]}

You are expert in temporal commonsense knowledge. Your goal is to determine if the candidate answer to a given question is plausible based on the information in the passage. Answer the related candidate answers then answer the candidate answer. Provide a `yes' or `no' response to indicate whether the original candidate answer is correct. (Format: final answer: yes/no)
If the passage does not provide explicit details, provide an educated guess based on what is common or plausible in similar situations

\textbf{[User]}

Passage: the majority religion during the centuries of Ottoman rule, though a significant Christian minority remained. 

Question: What happened before Islam was the majority religion? 

Candidate answer: the end of white-minority rule.
\newline 
\textbf{[Assistant]}
\newline
Final answer: no
\end{myColorbox}

\begin{myColorbox}[]{MCTACO: CoT}

\textbf{[System]}

You are an expert in temporal commonsense knowledge. Your goal is to determine if the candidate answer to a given question is plausible based on the information in the passage. Let's think step by step and provide a final `yes' or `no' response to indicate whether the original candidate answer is correct. If the passage does not provide explicit details, provide an educated guess based on what is common or plausible in similar situations. The final answer should be in the format: ``Final answer: yes'' or ``Final answer: no''.

\textbf{[User]}

Passage: the majority religion during the centuries of Ottoman rule, though a significant Christian minority remained. 

Question: What happened before Islam was the majority religion? 

Candidate answer: christianity was the majority religion

\textbf{[Assistant]}

Let's think step by step.
\newline
Step 1: Analyze the passage.
The passage states that during the centuries of Ottoman rule, Islam was the majority religion, while a significant Christian minority remained. This implies that before the Ottoman Empire (which was Islamic), there could have been a different majority religion.
\newline
Step 2: Historical context.
Before the rise of the Ottoman Empire, large parts of the Middle East, North Africa, and parts of Europe were under the rule of the Byzantine Empire, which was a Christian empire. Additionally, many regions that later came under Ottoman rule were predominantly Christian.
\newline
Step 3: Plausibility of the candidate answer.
Given that Christianity was the dominant religion in many regions before the Ottoman Empire and its Islamic rule, it is plausible that Christianity was the majority religion before Islam in those areas.
\newline
Final answer: yes

\end{myColorbox}

\begin{myColorbox}[]{MCTACO: CCP}

\textbf{[System]}

You are an expert in temporal commonsense knowledge. Your goal is to determine if the candidate answer to a given question is plausible based on the information in the passage. To improve understanding, follow these steps: generate related candidate answers that are either plausible or implausible (Format: answer: yes/no) to help in determining the correctness of the original candidate answer. Provide a 'yes' or 'no' response to indicate whether the original candidate answer is correct (Format: final answer: yes/no). If the passage does not provide explicit details, provide an educated guess based on what is common or plausible in similar situations

\textbf{[User]}

Passage: the majority religion during the centuries of Ottoman rule, though a significant Christian minority remained. 

Question: What happened before Islam was the majority religion? 

original candidate answer: christianity was the majority religion
\newline
Related candidate: the end of white-minority rule
\newline 
Can the candidate answer the given question? yes or no

\textbf{[Assistant]}  \textit{\textcolor{blue}{\# Prediction for the counterfactual question}}

This is incorrect because the passage discusses events during Ottoman rule, not before it.
\newline
answer: no

\textbf{[User]}

Now, can the candidate answer the given question? candidate answer: christianity was the majority religion

\textbf{[Assistant]}  \textit{\textcolor{blue}{\# Prediction for the original question}}

This is plausible because Islam would have started to emerge before becoming the majority.

Final answer: yes

\end{myColorbox}


\section{Scientific Artifacts}

We used existing scientific artifacts for research purposes, and the use of existing artifacts was consistent with their intended applications.

TempEvalQA-Bi: \href{https://github.com/yfqiu-nlp/temporal-llms?tab=MIT-1-ov-file#readme}{MIT license}

TRACIE: \href{https://www.apache.org/licenses/LICENSE-2.0}{Apache-2.0 license}

Llama-3: \href{https://www.llama.com/llama3/license/}{custom commercial license}

OpenAI API: \href{https://www.apache.org/licenses/LICENSE-2.0}{Apache-2.0 license}
