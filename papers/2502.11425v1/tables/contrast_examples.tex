

\begin{table*}[h]
\centering
\scalebox{0.82}{
\begin{tabular}{|l|l|l|l|}
\hline
 \textbf{Dataset} & \textbf{Relation} & \textbf{Revision Type} & \textbf{Example} \\ \hline
 TempEvalBi-QA \& TRACIE & $r_1(e_1,e_2)$ & Original & They got married \textbf{after} they moved to Maine. \\ 
 & $r_2(e_1,e_2)$ & $r_1 \rightarrow r_2$ & They got married \textbf{before} they moved to Maine. \\ \hline
 MCTACO: Ordering & $r_1(e_1,e_2)$ & Original & They \textbf{went to the store} after they were put in jail. \\ 
& $r_1(e_1,e_3)$ & $e_2 \rightarrow e_3$ & They \textbf{repented} after they were put in jail. \\ \hline
 MCTACO: Duration, & $r_1(e_1)$ & Original & It has existed for \textbf{1 year}. \\ 
 Frequency, Typical Time  & $r_2(e_1)$ & $r_1 \rightarrow r_2$ & It has existed for \textbf{centuries}. \\ \hline
 MCTACO: Stationarity & $r_1(e_1)$ & Original & She is \textbf{still} in Ranchipur. \\  
& $\neg r_1(e_1)$ & $r_1 \rightarrow \neg r_2$ & She is \textbf{not} in Ranchipur. \\ \hline

\end{tabular}
}
\vspace{-3mm}
\caption{Types of counterfactuals targeted for generation.
The examples illustrate how counterfactual questions modify the semantics regarding temporal aspects ($r_1$,$r_2$) including relations and properties, for events ($e_1$,$e_2$,$e_3$).
}
\vspace{-5mm}
\label{tab:contrast_examples}
\end{table*}
