
Prompt templates used to generate answers for
the TRACIE dataset are provided below.

\begin{myColorbox}[]{TRACIE: SP}

\textbf{[System]}

You are an expert in temporal knowledge. You can understand the implicit temporal relationships between events to make judgments. Your task is to determine the relationship between the story and hypothesis. A `positive' label means the hypothesis can be inferred from the story or logically sound. A `negative' label means the hypothesis cannot be inferred from the story. You must answer only with the labels `positive' or `negative'. (Format: final answer: positive/negative)

\textbf{[User]}

Story: Chad looked for his baseball cap starts after he got off the ride story: Chad had gone to an amusement park. He was riding on the roller coaster. \newline
Hypothesis: Chad looked for his baseball cap starts after he got off the ride.
\newline 
\textbf{[Assistant]}
\newline
Final answer: positive
\end{myColorbox}

\begin{myColorbox}[]{TRACIE: CoT}

\textbf{[System]}

You are an expert in temporal knowledge. You can understand the implicit temporal relationships between events to make judgments. Your task is to determine the relationship between the story and hypothesis. A `positive' label means the hypothesis can be inferred from the story or logically sound. A `negative' label means the hypothesis cannot be inferred from the story. You must give final answer only with the labels `positive' or `negative' after the ``Final answer:'' prompt.
Let's think step by step to evaluate the relationship.

\textbf{[User]}

Story: [...]	\newline
Hypothesis: Chad looked for his baseball cap starts after he got off the ride.

\textbf{[Assistant]}

Let's think step by step: \newline
The story states that Chad's baseball cap fell off during the roller coaster ride.\newline
It also mentions that Chad found the cap after he got off the ride.
To find something, one typically needs to look for it first.
Since Chad found the cap after getting off the ride, he must have started looking for it after getting off the ride.
The hypothesis directly aligns with this sequence of events.
\newline
Final answer: positive

\end{myColorbox}

\begin{myColorbox}[]{TRACIE: CCP}

\textbf{[System]}

You are an expert in temporal knowledge. You can understand the implicit temporal relationships between events to make judgments. Your task is to determine the relationship between the story and hypothesis. A `positive' label means the hypothesis can be inferred from the story or logically sound. A `negative' label means the hypothesis cannot be inferred from the story. You must give final answer only with the labels `positive' or `negative'. 
To improve understanding, follow these steps: Evaluate the related hypothesis to help clarify the story's timeline (Format: answer: [positive/negative]). Using answers to the related hypothesis as the additional information, evaluate the original hypothesis (Format: final answer: [positive/negative]).

\textbf{[User]}

Story: [...] \newline
Hypothesis: Chad looked for his baseball cap starts after he got off the ride. \newline
Related hypothesis: Chad looked for his baseball cap starts before he got off the ride. \newline
Answer the related hypothesis

\textbf{[Assistant]}  \textit{\textcolor{blue}{\# Prediction for the counterfactual question}}

The hypothesis conflicts with the story, as Chad finds the cap after the ride.
\newline
Answer: negative
\newline 
\textbf{[User]}
\newline
Now, evaluate the original hypothesis: Chad looked for his baseball cap starts after he got off the ride.

\textbf{[Assistant]}  \textit{\textcolor{blue}{\# Prediction for the original question}}

Chad rides the roller coaster while wearing a baseball cap, which falls off during the ride. After getting off the ride, Chad finds the cap. The hypothesis suggests that Chad started looking for the cap after getting off the ride, which aligns with the story's sequence of events.
\newline
Final answer: positive

\end{myColorbox}