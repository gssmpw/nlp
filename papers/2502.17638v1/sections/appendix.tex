\appendix

\onecolumn

\section{Appendix / supplemental material}

\subsection{Simplified Chubb Hospital Cash Benefit Policy}
\label{app:simplified_chubb}
\noindent Between:\\
CODEX INSURANCE LIMITED (\textquotedblleft us\textquotedblright)\\
and\\
\_\_\_\_\_\_\_\_\_\_\_\_\_\_\_\_ (\textquotedblleft You\textquotedblright)

\vspace{1em}
This policy is provided on the following terms and conditions:

POLICY IN EFFECT AND CONDITIONS

1.1 The payment of any benefit under this policy is conditioned on the policy being in effect at the time of the hospitalization for sickness or accidental injury on which the claim for such benefit is premised. The policy will be in effect if:
\begin{enumerate}%[label=(\alph*)]
    \item This agreement is signed, 
    \item The applicable premium for the policy period has been paid, and
    \item The condition set out in Section 1.3 is still pending or has been satisfied in a timely fashion, and
    \item The policy has not been canceled.
\end{enumerate}

1.2 Cancelation will be deemed to have occurred if there is fraud, or any misrepresentation or material withholding of any information provided by you to the Company in connection with any communication or information relating to this policy, or if the condition set out in Section 1.3 has not been satisfied in a timely fashion. It will also be automatically canceled at midnight, US Eastern time then in effect, on the last day of the policy term described in Section 5 below.

1.3 No later than the 7th month anniversary of the effective date of this policy, you will supply us with written confirmation from the medical provider in question of a wellness visit for yourself with a qualified medical provider occurring no later than the 6th month anniversary of the effective date of this policy.

GENERAL EXCLUSIONS

2.1 Your policy will not apply to, and no benefit will be paid with respect to, any event causing sickness or accidental injury arising directly or indirectly out of:
\begin{enumerate}
    \item Skydiving; or
    \item Service in the military; or
    \item Service as a fire fighter; or
    \item Service in the police; or
    \item If your age at the time of the hospitalization is equal to or greater than 80 years of age.
\end{enumerate}

GENERAL CONDITIONS

3.1 Where does Your Policy apply?

3.1.1 Your Policy insures You twenty-four (24) hours a day anywhere in the world.

3.2 Arbitration

3.2.1 If any dispute or disagreement arises regarding any matter pertaining to or concerning this Policy, the dispute or disagreement must be referred to arbitration in accordance with the provisions of the Arbitration Act (Cap. 10) and any statutory modification or re-enactment thereof then in force, such arbitration to be commenced within three (3) months from the day such parties are unable to settle the dispute or difference. If You fail to commence arbitration in accordance with this clause, it is agreed that any cause of action and any right to make a claim that You have or may have against Us shall be extinguished completely. Where there is a dispute or disagreement, the issuance of a valid arbitration award shall also be a condition precedent to our liability under this Policy. In no case shall You seek to recover on this Policy before the expiration of sixty (60) days after written proof of claim has been submitted to Us in accordance with the provisions of this Policy.

3.3 Laws of New York

3.3.1 Your Policy is governed by the laws of New York.

3.4 US Currency

3.4.1 All payments by You to Us and by Us to You or someone else under your policy must be in United States currency.

3.5 Premium

3.5.1 The premium described in Section 5 below shall be paid in one lump sum at the signing of the policy.

3.6 Policy Term
The term of this policy will begin on the date accepted by Us as signified by our signature of the policy (the effective date) and will last for a period of one year from that date, unless previously canceled pursuant to Section 1 above.

\subsection{Queries and Correct Answers for Empirical Evaluation on Chubb Contract}
\label{app:queries_and_answers}

All queries are preceded by the disclaimer: \say{Assuming all other conditions are met and no other exclusions apply (where by 'other,' I mean anything not referenced in the query that follows),\ldots} \\

\textbf{Query 1: } \say{will my policy apply if I was hospitalized by burns suffered while doing my duty as a firefighter?}
\textbf{Answer: } \say{No.}
\vspace{0.1in}

\textbf{Query 2: } \say{will my policy apply if I am 78 years old at the time of hospitalization?}
\textbf{Answer: } \say{Yes.}
\vspace{0.1in}

\textbf{Query 3: } \say{will my policy apply if I was hospitalized for pneumonia 5 months after the policy's effective date, and my age at the time of hospitalization is 65?}
\textbf{Answer: } \say{Yes.}
\vspace{0.1in}

\textbf{Query 4: } \say{will my policy apply if I was hospitalized due to a fall while traveling abroad and I had given confirmation of my wellness visit 8 months after the policy's effective date?}
\textbf{Answer: } \say{No.}
\vspace{0.1in}

\textbf{Query 5: } \say{will my policy apply if I was hospitalized for punching my own face to show off for my friends and I did not commit fraud or misrepresentation?}
\textbf{Answer: } \say{No.}
\vspace{0.1in}

\textbf{Query 6: } \say{will my policy apply if I was hospitalized due to an injury sustained while skydiving, my age at the time of hospitalization was 79, and proof of my wellness visit was provided 6.5 months after the policy's effective date?}
\textbf{Answer: } \say{No.}
\vspace{0.1in}

\textbf{Query 7: } \say{will my policy apply if I was hospitalized for a heart attack, proof of the wellness visit was submitted 2 months after the policy's effective date, and my age at the time of hospitalization was 75?}
\textbf{Answer: } \say{Yes.}
\vspace{0.1in}

\textbf{Query 8: } \say{will my policy apply if I was hospitalized after being injured in a military training exercise, the hospitalization occurred within the policy term, and I did not commit fraud?}
\textbf{Answer: } \say{No.}
\vspace{0.1in}

\textbf{Query 9: } \say{will my policy apply if I was hospitalized due to my son biting me in the ankle, proof of my wellness visit was provided 6 months after the effective date, and I was serving as a police officer at the time of hospitalization?}
\textbf{Answer: } \say{Yes.}
\vspace{0.1in}


\subsection{Prompts Provided to LLMs}
\label{app:prompts_vanilla_unguided}

\subsubsection{Prompt for Vanilla LLM Approach}
\label{app:prompt_vanilla}
The following is the prompt used in the Vanilla LLM approach described in \S\ref{sec:vanilla_llm}. 

\begin{itemize}
\item [--] Below, you are provided
    \begin{enumerate}
        \item {The full text of an insurance contract} 
        \item {A specific question about whether a claim in the given scenario is covered under the terms of this insurance contract}
    \end{enumerate}
    
\item [--] Assume that the policy agreement has been signed, and the premium has been paid on time.

\item [--] Assume that all other conditions are satisfied, and no exclusions apply unless explicitly referenced in the query.

\item [--] Your task:
    \begin{enumerate}
        \item Evaluate whether the claim described in the question is covered under the insurance contract.
        \item Respond with **only** one of the following: ``Yes'', ``No'', or ``I do not know''.
        \item Do not provide any explanations or reasoning.
    \end{enumerate}

\item [--] Insurance contract: \{text\_content\}
\item [--] Question: \{query\}

\end{itemize}

\subsubsection{Prompt for Unguided Prolog Generation}
\label{app:prompt_unguided}

The following is the prompt used in \S\ref{sec:exp_unguided} to generate Chubb insurance policy encoding. 

\begin{itemize}
    \item [--] Given the insurance contract below, translate the document into valid Prolog rules so that I can run a Prolog query on the code regarding whether or not some claim is covered under the policy and receive the correct answer to the question.

    \item [--] Please fully define all predicates and DO NOT define any facts, only rules that can be used to answer queries on this insurance contract.

    \item [--] Assume that all dates/times in any query to this code (apart from the claimant's age) will be given RELATIVE to the effective date of the policy (i.e. there will never be a need to calculate the time elapsed between two dates). Take dates RELATIVE TO the effective date into account when writing this encoding.

    \item [--] Assume that the agreement has been signed and the premium has been paid (on time). There is no need to encode rules or facts for these conditions.

    \item [--] Return only Prolog code in your reply. No explanation is necessary.

    \item [--] Ensure that:
        \begin{enumerate}
            \item The legal text is appropriately translated into correct Prolog rules.
            \item The output does not redefine, misuse, or conflict with any built-in Prolog predicates.
            \item If dynamic predicates are necessary, they are declared and managed correctly.
            \item All predicates used in the generated Prolog code, including those referenced in the query, are fully defined and error-free to prevent issues like ``procedure does not exist.''
            \item Logical relationships, conditions, and dependencies in the text are faithfully represented in the Prolog rules to ensure accurate query results.
        \end{enumerate}
    \item [--] Insurance contract: \{text\_content\}
\end{itemize}

The following is the prompt used in \S\ref{sec:exp_unguided} to generate claim encodings.

\begin{itemize}
    \item [--] I have given below:
        \begin{enumerate}
            \item A question about whether or not the policy defined in a given insurance contract applies in a particular situation
            \item The text of the insurance contract
            \item A Prolog encoding of the insurance contract
        \end{enumerate}
    \item [--] Encode the question into a Prolog query such that it can be run on the given Prolog encoding of the insurance contract, returning the correct answer to the question.
    \item [--] Assume that the agreement has been signed and the premium has been paid (on time). There is no need to encode rules or facts for these conditions.
    \item [--] Return only Prolog query in your reply. No explanation is necessary.
    \item [--] Ensure that:
        \begin{enumerate}
            \item The output does not redefine, misuse, or conflict with any built-in Prolog predicates.
            \item If dynamic predicates are necessary, they are declared and managed correctly.
            \item All predicates used in the generated Prolog code, including those referenced in the query, are fully defined and error-free to prevent issues like "procedure does not exist."
            \item Logical relationships, conditions, and dependencies in the text are faithfully represented in the Prolog rules to ensure accurate query results.
            \item No absolute dates/times (apart from the claimant's age) are encoded in your query. Only include dates/times RELATIVE to the effective date of the policy (again, except for age).
            \item Set any facts/rules/parameters in the code such that ALL conditions (for the policy to apply) which are UNRELATED to the above query are satisfied.
            \item Set any facts/rules/parameters in the code such that NO exclusions (which would prevent the policy from applying) which are UNRELATED to the above query are satisfied.
        \end{enumerate}
    \item [--] Question:\{query\}
    \item [--] Insurance contract: \{text\_content\}
    \item [--] Insurance contract Prolog encoding: \{policy\_encoding\}
\end{itemize}


% \subsection{Raw Data for Preliminary Experiments in Sections \S\ref{sec:vanilla_llm} and \S\ref{sec:exp_unguided}}
% \label{app:raw_data_preliminary_exp}

% \begin{table*}[ht]
% \centering
% \begin{minipage}{0.48\textwidth}
% \centering
% \begin{tabular}{|c|c|}
% \hline
% \textbf{Model} & \textbf{Accuracy} \\ \hline
% Mistral-large-latest & 7 ± 0 \\ \hline
% Gemini-1.5-pro & 7 ± 0 \\ \hline
% Claude-3.5-sonnet & 7 ± 0 \\ \hline
% Llama-3.1-405B-instruct & 7 ± 0 \\ \hline
% GPT-4o-2024-08-06 & 7 ± 0 \\ \hline
% DeepSeek-R1 & 7.3 ± 0.15 \\ \hline
% O1-preview & 7.9 ± 0.18 \\ \hline
% \end{tabular}
% \end{minipage}
% \hfill
% \begin{minipage}{0.48\textwidth}
% \centering
% \begin{tabular}{|c|c|}
% \hline
% \textbf{Model} & \textbf{Accuracy} \\ \hline
% Mistral-large-latest & 4.5 ± 0.56 \\ \hline
% Gemini-1.5-pro & 5.0 ± 0.45 \\ \hline
% Claude-3.5-sonnet & 6.7 ± 0.26 \\ \hline
% Llama-3.1-405B-instruct & 3.7 ± 0.40 \\ \hline
% DeepSeek-R1 & 5.7 ± 0.30 \\ \hline
% GPT-4o-2024-08-06 & 5.4 ± 0.62 \\ \hline
% O1-preview & 8.0 ± 0.15 \\ \hline
% \end{tabular}
% \end{minipage}
% \caption{Average accuracy of LLMs on the Chubb insurance claim coverage dataset. The $\pm$ values are the standard error of the mean across 10 trials. Left: Vanilla LLM Approach; Right: Unguided LLM-Generated Prolog Approach.}
% \label{tab:vanilla_unguided_chubb}
% \end{table*}

\subsubsection{Prompt for Generating LLM Encodings of Insurance Analyst Coverages}
\label{app:art-prompt}
\begin{itemize}
  \item I have provided below all of the text that pertains to a coverage (or section) 
        of a health insurance policy.
    \begin{itemize}
      \item The text defines all conditions and exclusions that determine 
            whether a patient's claim is covered under this coverage of the policy.
    \end{itemize}

  \item Please encode a Prolog rule, \verb|'covered(C)'|, which is true 
        exactly when the patient's claim, \verb|'C'|, is covered.
    \begin{itemize}
      \item I have also provided below a set of claim facts that will be 
            defined in any Prolog query to the \verb|'covered(C)'| rule.
        \begin{itemize}
          \item These are of the form 
                \verb|'claim_Fact(C, Value)'|, meaning they will be defined for 
                the claim \verb|'C'| by filling in a value for \verb|Value|.
          \item Under each of these facts, I have listed the valid inputs which 
                can be filled in for \verb|Value| (or in some cases, the type of 
                valid input, such as number or date) in the fact.
          \item Use ALL OF (AND ONLY) these claim facts to write your 
                Prolog encoding for the policy coverage.
        \end{itemize}

      \item Finally, I have provided below a set of supporting (pre-defined) 
            Prolog predicates.
        \begin{itemize}
          \item Under each supporting predicate, I have listed the arguments to 
                the predicate with their category (e.g.\ atom, number) and a 
                description, as well as a description of what the predicate does.
          \item Use ALL OF these supporting predicates to write your 
                Prolog encoding for the policy coverage.
        \end{itemize}
    \end{itemize}

  \item The Prolog rule that you encode, \verb|'covered(C)'|, must incorporate 
        all relevant criteria from the policy coverage.

  \item Return only Prolog code in your reply. 
        NO EXPLANATION is necessary.

\end{itemize}

\subsection{LLM Encodings of FSH Criteria in ART Coverage}
\subsubsection{GPT-4o Encoding of FSH Criteria in ART Coverage}
\begin{verbatim}
validate_day_3_fsh(C) :-
    % Get patient age at time of service
    patient_age_at_time_of_service(C, Age),
    (Age < 40 -> % For patients under 40
        most_recent_day_3_fsh_test(C, fsh_test(FSHLevel, _, _)),
        FSHLevel < 19
    ; Age >= 40 -> % For patients 40 and above
        day_3_fsh_tests_since_age_40_in_claim(C, FSHTests),
        \+ exists_test_with_level_at_least_19(FSHTests)).
\end{verbatim}
\label{app:4o-fsh}

\subsubsection{OpenAI o1 Encoding of FSH Criteria in ART Coverage}
\begin{verbatim}
meets_fsh_criteria(C) :-
    patient_age_at_time_of_service(C, Age),
    (
        ( Age < 39 ) ->
            most_recent_day_3_fsh_test(C, fsh_test(Level, _, _)),
            Level < 19
        ;
        ( Age = 39 ) ->
            claim_date_service_received(C, ServiceDate),
            most_recent_day_3_fsh_test(C, fsh_test(Level, TestDate, _)),
            leq_n_months_diff(TestDate, ServiceDate, 6),
            Level < 19
        ;
        ( Age >= 40 ) ->
            claim_date_service_received(C, ServiceDate),
            (
                claim_patient_has_premature_ovarian_failure(C, yes) ->
                    most_recent_day_3_fsh_test(C, fsh_test(Level, TestDate, _)),
                    leq_n_months_diff(TestDate, ServiceDate, 6),
                    Level < 19
                ;
                    day_3_fsh_tests_since_age_40_in_claim(C, Tests),
                    \+ exists_test_with_level_at_least_19(Tests),
                    most_recent_day_3_fsh_test(C, fsh_test(_, TestDate, _)),
                    leq_n_months_diff(TestDate, ServiceDate, 6)
            )
    ).
\end{verbatim}
\label{app:o1-fsh}