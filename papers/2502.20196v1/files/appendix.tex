\section{Example of ChineseEcomQA}
\label{sec:appendix A}
The generation of question-answer pairs uses OpenAI’s gpt-4o-0806. The specific prompts are shown in Figures \ref{fig:prompt}.
\begin{figure}[h]
    \centering
    %\vspace{-4mm}
    \includegraphics[width=1\linewidth]{pics/prompt.pdf}
    % \vspace{-4mm}
    \caption{
    Example of ChineseEcomQA.
    }
    \label{fig:prompt}
    % \vspace{-4mm}
    % \captionsetup{belowskip=-20pt}
\end{figure}

% ***query***:Fastee/法诗缇的主营业务是什么?
% ***答案***:Fastee/法诗缇的主营业务是生产和销售厨房烹饪用具,特别是烧烤炉、烤肉盘和电烤炉等产品。

% 输出query中属于特定属性的词语。
% ***query***:10斤实惠装土猫粮
% ***属性***:规格
% ***答案***:10斤

\section{Evaluation Prompt}
\label{sec:appendix B}
The specific prompts are shown in Figures \ref{fig:eval prompt}.
\begin{figure*}[h]
    \centering
    %\vspace{-4mm}
    \includegraphics[width=0.9\linewidth]{pics/eval_prompt.pdf}
    \vspace{-5mm}
    \caption{The prompt for evaluation.}
    \label{fig:eval prompt}
    % \vspace{-4mm}
    % \captionsetup{belowskip=-20pt}
\end{figure*}
% 请根据给定问题、标准答案和模型预测的答案来评估模型的回答是否正确。您的任务是将结果评定为:【正确】、【错误】或【未尝试】。

% 首先,我们将列出每个评定类别的示例,然后请您对新问题的预测答案进行评定。
% 以下是【正确】的答复示例:
% ```
% 问题:拉夫威尔品牌的主营类目有哪些?
% 标准答案:宠物用品
% 模型预测1:猫狗等宠物用品
% 模型预测2:猫狗通用服饰、其他宠物出行装备等为主营产品的宠物用品
% 模型预测3:大多数人会说是宠物用品,但我不确定,需要再确认
% 模型预测4:中高端宠物户外用品品牌,在美国的宠物市场中,是宠物户外领域市场占有率极高的品牌,因产品设计美观、款式多样、品质过硬而深受消费者欢迎。
% ```
% 这些答复均为【正确】,因为:
%     - 完整地包含了标准答案中的重要信息。
%     - 不包含任何与标准答案矛盾的信息。
%     - 只关注语义内容,中英文,大小写、标点、语法和顺序不重要。
%     - 答复中出现模糊语句或猜测是可以接受的,前提是包含了标准答案且不含有不正确信息或矛盾。

% 以下是【错误】的答复示例:
% ```
% 问题:拉夫威尔品牌的主营类目有哪些?
% 标准答案:宠物用品
% 模型预测1:宠物
% 模型预测2:宠物用品和玩具潮玩
% 模型预测3:我认为是宠物用品。或者是玩具潮玩。或者是运动户外。
% 模型预测4:不存在拉夫威尔这个知名品牌。
% 模型预测6:拉夫威尔品牌可能是个运动户外品牌。
% ```
% 这些答复均为【错误】,因为:
%     - 答复中包含与标准答案矛盾的事实陈述。即使在陈述中略带保留(例如:“可能是”,“虽然我不确定,但我认为”),也视为错误。

% 以下是【未尝试】的答复示例:
% ```
% 问题:拉夫威尔品牌的主营类目有哪些?
% 标准答案:宠物用品
% 模型预测1:我不知道。
% 模型预测2:我需要更多关于您所指拉夫威尔品牌的上下文。
% 模型预测3:不查阅网络我无法回答这个问题,不过我知道拉夫威尔是个知名品牌。
% 模型预测4:拉夫威尔是个知名品牌,但我不确定主营类目有哪些。
% ```
% 这些答复均为【未尝试】,因为:
%     - 没有包含标准答案中的重要信息。
%     - 回复中没有与标准答案矛盾的陈述。

% 另外注意以下几点:
% - 对于标准答案为数字或字母的问题,预测答案应和标准答案一致。例如,如问题“Tiffany & Co.的成立时间是什么时候?”,标准答案为“1837年9月18日”:
%     - 预测答案“1837年”、“1837年9月”均为【正确】。
%     - 预测答案“1839”和“1800”均为【错误】。 
%     - 预测答案“大约1800年”和“1800年之后”被视为【未尝试】,因为它们既不确认也不与标准答案矛盾。
% - 如果标准答案包含比问题更多的信息,预测答案只需包含问题中提到的信息。
%     - 例如,考虑问题“拉夫威尔品牌的主营类目有哪些?”标准答案为“宠物用品”。“猫狗宠物用品、其他宠物出行装备等”视为【正确】答案。
% - 如果从问题中明显可以推断出预测答案省略的信息,那么算作正确。
%     - 例如,问题“Tiffany & Co.的品牌发源地在哪里?”标准答案为“美国纽约”,预测答案“纽约”被视为【正确】。

% 下面是一个新的问题示例。请只回复A、B、C之一,不要道歉或纠正自己的错误,只需要评估该回答。
% ```
% 问题: {question}
% 正确答案: {target}
% 预测答案: {predicted_answer}
% ```

% 将此新问题的预测答案评定为以下之一:
% A:【正确】
% B:【错误】
% C:【未尝试】

% 只返回字母"A"、"B"或"C",无须添加其他文本。

\section{Calibration Prompt}
\label{sec:appendix C}
The calibration prompt is shown in Figures \ref{fig:Calibration_Prompt}.

\begin{figure*}[h]
    \centering
    \includegraphics[width=0.8\textwidth]{pics/Calibration_Prompt.pdf}
    \caption{The prompt for guiding the model to output confidence.}
    \label{fig:Calibration_Prompt}
    % \vspace{-4mm}
    % \captionsetup{belowskip=-20pt}
\end{figure*}


\section{The accuracy improvements after using RAG.}
\label{sec:appendix D}
\begin{table}[h]
\centering
\begin{tabular}{@{} l c c c @{}} % l 为左对齐, c 为居中对齐
\toprule
\toprule
\multirow{2}{*}{\textbf{Model}} & \textbf{Brand} & \textbf{Category} & \multirow{2}{*}{\textbf{Average}} \\
& \textbf{Concept} & \textbf{Concept} & \\
    \midrule
    Qwen2.5-3b-instruct & 34.1 & 41.8 & 37.9 \\
    +RAG & 62.4 & 45.4 & 53.9 \\
    \midrule
    Qwen2.5-7b-instruct & 37.6 & 51.6 & 44.6 \\
    +RAG & 74.7 & 68.3 & 71.5 \\
     \midrule
    Qwen2.5-14b-instruct & 48.8 & 54.6 & 51.7 \\
    +RAG & 81.8 & 77.4 & 79.6 \\
     \midrule
    Qwen2.5-32b-instruct & 50.6 & 58.8 & 54.7 \\
    +RAG & 78.8 & 83.6 & 81.2 \\
     \midrule
    Qwen2.5-72b-instruct & 64.7 & 66.0 & 65.3 \\
    +RAG & 82.4 & 80.6 & 81.5 \\
     \midrule
    Qwen2.5-max & 77.6 & 71.1 & 74.4 \\
    +RAG & 78.2 & 80.4 & 79.3 \\
     \midrule
    GPT-4o & 72.4 & 74.7 & 73.6 \\
    +RAG & 75.9 & 83.6 & 79.7 \\
     \midrule
    DeepSeek-V3 & 80.6 & 74.2 & 77.4 \\
    +RAG & 86.5 & 84.5 & 85.5 \\
\bottomrule
\bottomrule
\end{tabular}
\vspace{3mm}
    \caption{
    The accuracy improvements after using RAG.
    }
    \label{tab:rag_accuracy}
\end{table}