\section{RELATED WORK}
\label{sec:related_work}

\subsection{Trajectory Planning for Aerial Swarms}
\label{sec:RW_trajectory_planning}

Bio-inspired flocking algorithms \cite{reynolds1987flocks, vasarhelyi2018optimized} plan trajectories for groups of robots based on simple rules, but they face challenges in dense environments and precise individual robot control. 
Velocity obstacle (VO)-based methods \cite{fiorini1998motion, van2008reciprocal, van2011reciprocal} compute feasible velocities for robots to avoid collisions, but their first-order integrator dynamics models and constant velocity assumptions are not suitable for highly agile quadrotors.
In order to achieve rapid swarm motions in dense real-world environments, some methods \cite{augugliaro2012generation, mellinger2012mixed, kushleyev2013towards, honig2018trajectory, park2020efficient} adopt a centralized approach to jointly plan the global trajectories of all robots. Augugliaro and Mellinger \cite{augugliaro2012generation,mellinger2012mixed, kushleyev2013towards} formulate joint planning as a Sequential Convex Program (SCP) and a Mixed Integer Quadratic Program (MIQP), respectively. H{\"o}nig and Park \cite{honig2018trajectory, park2020efficient} use $\rm{B\acute{e}zier}$ curves and Bernstein polynomials to generate safe and conservative trajectories, respectively. However, these methods rely on known environments, global localization devices, and entail high computational costs, thereby limiting the full-autonomy and scalability of swarm systems.

In contrast, some methods \cite{chen2015decoupled, luis2019trajectory, liu2018towards} adopt a decentralized approach to improve computational efficiency. However, the assumption of synchronous replanning limits their applicability in real-world scenarios. Tordesillas et al. \cite{tordesillas2021mader} propose a decentralized and asynchronous algorithm that handles multi-robot and dynamic obstacles in convex obstacle environments.
Zhou et al. \cite{zhou2021ego} establish a fully autonomous decentralized quadrotor swarm system suitable for non-convex unknown environments. However, the challenge of temporal optimization leads to twisted trajectories when multiple robots encounter each other. To address this issue, Zhou et al. \cite{zhou2022swarm} deploy a spatial-temporal trajectory optimization algorithm \cite{wang2022geometrically}.

\begin{figure}[t]
	\centering
	\includegraphics[width=1.0\linewidth]{swarm_comparison}
	\caption{Qualitative comparison with RBP\cite{park2020efficient}, MADER\cite{tordesillas2021mader}, EGO-Swarm\cite{zhou2021ego}, MINCO-Swarm\cite{zhou2022swarm}, and AMSwarmX\cite{adajania2023amswarmx}. Axis ticks from inside to outside represent: Computation efficiency - heavyweight, moderate, lightweight, ultra lightweight; Scalability - low, medium, high; Trajectory quality - low, medium, high, very high; System Autonomy - none, partially autonomous, fully autonomous. Please refer to Section \ref{sec:RW_trajectory_planning} for the detailed discussion of the qualitative comparison.}
	\label{pic:swarm_comparison}
	\vspace{-0.5cm}
\end{figure}

In our previous work \cite{hou2022enhanced}, we proposed a group planning strategy aimed at enhancing the scalability of swarm systems.
However, these methods \cite{tordesillas2021mader,zhou2021ego,zhou2022swarm, hou2022enhanced} require complex environment representation, such as inflated grid maps or convex obstacles, and demand good initial trajectories for nonlinear trajectory optimization, leading to substantial onboard computation.
Furthermore, the inclusion of non-convex collision avoidance terms for robot-obstacle and robot-robot interactions makes the trajectory susceptible to infeasible local minima, resulting in replanning failure. Additionally, the generated trajectory is searched only within a limited portion of the solution space around the initial values.

The proposed method effectively mitigates these challenges and introduces an ultra-lightweight algorithm suitable for large-scale autonomous swarms. Comprehensive benchmark comparisons with representative state-of-the-art methods are conducted in Section \ref{sec:evaluation}, which are then summarized in Fig. \ref{pic:swarm_comparison} to give an intuitive knowledge of some main characteristics of the proposed method. In Fig. \ref{pic:swarm_comparison}, \textit{Computation Efficiency} is evaluated in Table \ref{tab:comparison_empty}, Fig. \ref{pic:obs_collision_time}, \ref{pic:agent_collision_time} and \ref{pic:30_radnom_comp}c. \textit{Trajectory Quality} is compared in Fig. \ref{pic:comparison_single} to \ref{pic:30_radnom_comp} where the proposed method generates the smoothest trajectory while maintaining the maximum speed along the whole flight. Flight time and flight distance in Table \ref{tab:comparison_empty} are also metrics of trajectory quality. High \textit{Scalability} is the main characteristic of this work as featured in Fig. \ref{pic:E1000} and \ref{pic:1000_circle} and analysed in Section \ref{sec:large_scale}. RBP receives a relatively low grade in this metric for its high computation time (Table \ref{tab:comparison_empty}). \textit{System Autonomy} refers to the ability of navigating in the real world relying on only onboard sensing and computation. RBP is a centralized trajectory planner that requires a high-performance ground computer and a pre-built map, thus it lacks autonomy. MADER and AMSwarmX are decentralized swarm planners validated with pre-built maps in simulation only. Although both of them have the potential of using online generated maps, a lot of future work is required. EGO-Swarm, MINCO-Swarm and the proposed method are all fully autonomous methods validated in real-world experiments.


\subsection{Motion Primitive-based Trajectory Planning}


% Please add the following required packages to your document preamble:
% \usepackage{multirow}
% \usepackage{graphicx}
\begin{table}[]
	\centering
	\caption{Evaluation of Motion Primitive-based Methods}
    \label{tab:primitive}
    \renewcommand{\arraystretch}{1.3}
    \setlength\tabcolsep{2pt}
	\resizebox{\columnwidth}{!}{%
		\begin{tabular}{|cc|c|c|c|c|c|}
			\hline
			\multicolumn{2}{|c|}{\textbf{Method}} &
			\textbf{\begin{tabular}[c]{@{}c@{}}Primitive \\ Generation\end{tabular}} &
			\textbf{\begin{tabular}[c]{@{}c@{}}Time \\ Parameterized\end{tabular}} &
			\textbf{\begin{tabular}[c]{@{}c@{}}Dynamical\\ Feasiblity\end{tabular}} &
			\textbf{\begin{tabular}[c]{@{}c@{}}Time\\ Optimality\end{tabular}} &
			\textbf{\begin{tabular}[c]{@{}c@{}}Collision\\ Checking\end{tabular}} \\ \hline
			\multicolumn{1}{|c|}{\multirow{3}{*}{\begin{tabular}[c]{@{}c@{}}State\\  Space\end{tabular}}} &
			\cite{zhang2020falco} &
			Offline &
			No &
			\textbackslash{} &
			\multirow{4}{*}{No} &
			Medium \\ \cline{2-5} \cline{7-7} 
			\multicolumn{1}{|c|}{} &
			\cite{ryll2019efficient} &
			\multirow{3}{*}{Online} &
			\multirow{4}{*}{\textbf{Yes}} &
			\multirow{2}{*}{Post-check} &
			&
			\multirow{3}{*}{Slow} \\ \cline{2-2}
			\multicolumn{1}{|c|}{} &
			\cite{yang2021intention,collins2020efficient} &
			&
			&
			&
			&
			\\ \cline{1-2} \cline{5-5}
			\multicolumn{2}{|c|}{Control Space\cite{florence2020integrated}} &
			&
			&
			\multirow{2}{*}{\textbf{\begin{tabular}[c]{@{}c@{}}Satisfied in \\ Generation\end{tabular}}} &
			&
			\\ \cline{1-3} \cline{6-7} 
			\multicolumn{2}{|c|}{\textbf{Proposed}} &
			Offline &
			&
			&
			\textbf{Yes} &
			\textbf{Fast} \\ \hline
		\end{tabular}%
	}
	\vspace{-0.4cm}
\end{table}


\begin{figure*}[t]
	\centering
	\includegraphics[width=1.0\linewidth]{system_overview}
	\caption{The overview of our decentralized and asynchronous autonomous aerial swarm system, which includes state estimation (yellow), planning (blue), control (green), and communication (red) modules. $\mathcal{W}$ refers to the world coordinate system and $\mathcal{V}$ is called the velocity-aligned frame defined by current drone position and velocity in Sec. \ref{sec:vcs}.}
	\label{pic:system_overview}
	\vspace{-0.2cm}
\end{figure*}

Motion primitive-based methods are commonly employed for generating multiple paths or trajectories for single-robot autonomous navigation. They effectively reduce the planning problem's complexity and can cover a wide solution space simultaneously. 
A summary of typical methods is presented in TABLE \ref{tab:primitive}. The index ``Time Parameterized" means whether the primitive library consists of time-parameterized continuous trajectories, whose higher derivatives (i.e., velocity, acceleration, jerk, and more) can be easily obtained. The index ``Dynamical Feasibility'' records how dynamical feasibility are guaranteed. The index ``Collision Checking'' contains 
the theoretical computation time for both robot-obstacle and robot-robot collision avoidance. If a method being compared is initially designed for robot-obstacle avoidance only, we assume the space swept by nearby drones as a normal static obstacle, thus compatible with the compared method.

Zhang et al. \cite{zhang2020falco} propose an offline motion primitive library that samples fixed-length paths with only position information, lacking dynamical details and not fully exploiting the robot's maneuverability. Collision checking is realized by building an adjacency list that records the occupancy status between primitives and the surrounding space offline during primitive library generation. We draw inspiration in collision checking from \cite{zhang2020falco}.
Ryll et al. \cite{ryll2019efficient} generate multiple fixed-duration min-jerk trajectories by sampling different local end positions along with a start state (position, velocity, and acceleration).
Yang et al. \cite{yang2021intention} perform online sampling of different local end velocities in an action space $a={v_x, v_z, \omega}$, combining them with a start state to generate multiple fixed-duration 8-th order polynomial trajectories, which are subsequently integrated to obtain position information.
These methods \cite{ryll2019efficient, yang2021intention} generate online primitives that include a significant number of dynamically infeasible trajectories, necessitating a rechecking mechanism \cite{mueller2015computationally} to filter out feasible ones.
Collins et al. \cite{collins2020efficient} employ adaptive end velocity sampling based on a robot's reference velocity, enhancing the dynamical reliability of motion primitive libraries.
Florence et al. \cite{florence2020integrated} perform online sampling of a set of constant control variables considering a quadrotor dynamics model. These variables are then integrated forward for a fixed duration to generate dynamically feasible primitives.
The aforementioned methods \cite{ryll2019efficient, yang2021intention, collins2020efficient, florence2020integrated} suffer from the drawback of checking collisions for each primitive individually on a fusion map or kd-tree. As the number of primitives increases, these methods become progressively time-consuming.
In contrast, Bucki et al. \cite{bucki2020rectangular} accelerate collision checking using a pyramid partitioning method, but it results in more conservative trajectories and performs poorly in dense environments.

In summary, the aforementioned methods fail to simultaneously meet three critical requirements: (1) ultra-lightweight online computational cost; (2) time-optimal and dynamically feasible primitives; (3) fast collision checking methods capable of handling both robot-obstacle and robot-robot conflicts in batches. These limitations have hindered the advancement of motion primitive libraries in autonomous navigation. The proposed method is the first to successfully integrate all three aspects and deploy motion primitives on autonomous swarms, to the best of our knowledge.


\subsection{Time-optimal Trajectory Generation }

Time-optimal trajectory generation address the problem of finding the fastest way to traverse a region while adhering to dynamical constraints.
Mellinger et al. \cite{mellinger2011minimum} find the time-optimal trajectory by optimizing time allocation through backtracking gradient descent. Then dynamical feasibility is achieved by temporal scaling. However, both strategies consume significant computation as the number of trajectory pieces grows.
Liu et al. \cite{liu2017planning} managed to achieve comparable performance with only a single scaling operation, but it is only valid for rest-to-rest trajectories.
Sun et al. \cite{sun2021fast} adopt a dual-level optimization scheme by analytically estimating the projected gradient by leveraging the dual solution of the low-level Quadratic Programming (QP), resulting in enhanced precision. However, computing the projected gradient remains less effective.
Richter et al. \cite{bry2015aggressive} address this challenge by treating each duration as an independent variable and employing total duration as a regularization term. They optimize time allocation through gradient descent and ensuring that actuator constraints are met through scaling. However, the optimal time allocation is sensitive to constraints, resulting in the potential to disrupt a trajectory.
Burri et al. \cite{burri2015real} relieve the sensitivity by converting constraints as weighted objectives and optimizing through nonlinear programming, but at the cost of being dependent of initial values and the risk of constraints violation.