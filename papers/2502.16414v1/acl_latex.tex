% This must be in the first 5 lines to tell arXiv to use pdfLaTeX, which is strongly recommended.
\pdfoutput=1
% In particular, the hyperref package requires pdfLaTeX in order to break URLs across lines.

\documentclass[11pt]{article}

% Change "review" to "final" to generate the final (sometimes called camera-ready) version.
% Change to "preprint" to generate a non-anonymous version with page numbers.
\usepackage[preprint]{acl}

% Standard package includes
\usepackage{times}
\usepackage{latexsym}

% For proper rendering and hyphenation of words containing Latin characters (including in bib files)
\usepackage[T1]{fontenc}
% For Vietnamese characters
% \usepackage[T5]{fontenc}
% See https://www.latex-project.org/help/documentation/encguide.pdfac for other character sets

% This assumes your files are encoded as UTF8
\usepackage[utf8]{inputenc}

% This is not strictly necessary, and may be commented out,
% but it will improve the layout of the manuscript,
% and will typically save some space.
\usepackage{microtype}

% This is also not strictly necessary, and may be commented out.
% However, it will improve the aesthetics of text in
% the typewriter font.
\usepackage{inconsolata}

%Including images in your LaTeX document requires adding
%additional package(s)
\usepackage{graphicx}

% If the title and author information does not fit in the area allocated, uncomment the following
%
%\setlength\titlebox{<dim>}
%
% and set <dim> to something 5cm or larger.
\usepackage{hyperref}
% \hypersetup{
%     colorlinks=true,
%     urlcolor=teal  
% }
\usepackage{url}            % simple URL typesetting
\usepackage{booktabs}       % professional-quality tables
\usepackage{amsfonts}       % blackboard math symbols
\usepackage{nicefrac}       % compact symbols for 1/2, etc.
\usepackage{microtype}      % microtypography
\usepackage{amsmath}
\usepackage{threeparttable}
\usepackage{graphicx}
\usepackage{multirow}
\usepackage{subfigure}
\usepackage{listings}
\usepackage{tikz}
\usepackage{colortbl}
\usepackage{xspace}
\usepackage{algorithm}
\usepackage{algpseudocode}
\usepackage{enumitem}
\usepackage{xcolor}         % colors
\usepackage{tabularx} % for better table width handling
\usepackage{booktabs} % for enhanced table styling

\usepackage{amsthm}
\usepackage[english]{babel}
\theoremstyle{definition}
\newtheorem{definition}{Definition}
\newtheorem{remark}{Remark}

\lstdefinestyle{mystyle}{
    backgroundcolor=\color{backcolour},   
    commentstyle=\color{codegreen},
    keywordstyle=\color{magenta},
    numberstyle=\tiny\color{codegray},
    stringstyle=\color{codepurple},
    basicstyle=\ttfamily\footnotesize,
    breakatwhitespace=false,         
    breaklines=true,                 
    captionpos=b,                    
    keepspaces=true,                 
    numbers=left,                    
    numbersep=5pt,                  
    showspaces=false,                
    showstringspaces=false,
    showtabs=false,                  
    tabsize=2
}

\lstset{style=mystyle}

% Define colors
\definecolor{codegreen}{rgb}{0,0.6,0}
\definecolor{codegray}{rgb}{0.5,0.5,0.5}
\definecolor{codepurple}{rgb}{0.58,0,0.82}
\definecolor{backcolour}{rgb}{0.95,0.95,0.92}
\definecolor{brickred}{rgb}{0.8, 0.25, 0.1}
\definecolor{midnightblue}{rgb}{0.1, 0.1, 0.44}
\definecolor{oceanboatblue}{rgb}{0.0, 0.47, 0.75}
\newcommand{\redbf}[1]{\bf{\textcolor{brickred}{#1}}}
\newcommand{\bluebf}[1]{\bf{\textcolor{oceanboatblue}{#1}}}
\newcommand{\grey}{\cellcolor{gray!20}}
\definecolor{lightgray1}{gray}{0.95}  

\newcommand{\circlednumber}[1]{%
    \tikz[baseline=(char.base)]{%
        \node[shape=circle,draw,inner sep=1pt] (char) {\scriptsize#1};%
    }%
}

\newcommand{\modelname}{{\sc TabGen-ICL}\xspace}
\title{\modelname: Residual-Aware In-Context Example Selection for Tabular Data Generation}

\author{
 \textbf{Liancheng Fang \textsuperscript{1}},
 \textbf{Aiwei Liu \textsuperscript{2}},
 \textbf{Hengrui Zhang \textsuperscript{1}},
 \textbf{Henry Peng Zou \textsuperscript{1}},
\\
 \textbf{Weizhi Zhang\textsuperscript{1}},
 \textbf{Philip S. Yu\textsuperscript{1}},
\\
 \textsuperscript{1}University of Illinois Chicago,
 \textsuperscript{2}Tsinghua University,
\\
\href{mailto:zhonghaoli@hkust-gz.edu.cn}{lfang87@uic.edu},
\href{mailto:xuminghu@hkust-gz.edu.cn}{liuaw20@mails.tsinghua.edu.cn},
\href{mailto:psyu@uic.edu}{psyu@uic.edu}
}

\begin{document}
\maketitle
\begin{abstract}
    Large Language models (LLMs) have achieved encouraging results in tabular data generation. However, existing approaches require fine-tuning, which is computationally expensive. This paper explores an alternative: prompting a fixed LLM with in-context examples. We observe that using randomly selected in-context examples hampers the LLM's performance, resulting in sub-optimal generation quality.
    To address this, we propose a novel in-context learning framework: \modelname, to enhance the in-context learning ability of LLMs for tabular data generation. 
    \modelname operates iteratively, retrieving a subset of real samples that represent the \textit{residual} between currently generated samples and true data distributions. This approach serves two purposes: locally, it provides more effective in-context learning examples for the LLM in each iteration; globally, it progressively narrows the gap between generated and real data.
    Extensive experiments on five real-world tabular datasets demonstrate that \modelname significantly outperforms the random selection strategy. Specifically, it reduces the error rate by a margin of $3.5\%-42.2\%$ on fidelity metrics. We demonstrate for the first time that prompting a fixed LLM can yield high-quality synthetic tabular data. 
    The code is provided in the \href{https://github.com/fangliancheng/TabGEN-ICL}{link}.
\end{abstract}

\section{Introduction}\label{sec:intro}

In computational finance, Monte Carlo simulations are used extensively to estimate the expected value of financial payoffs based on the solution of stochastic differential equations (SDEs) which model the evolution of stock prices, interest rates, exchange rates and other quantities \cite{glasserman04}.  Monte Carlo methods are very general and flexible, but for high accuracy it requires generating a large number of costly SDE path approximations, which has motivated research into a number of variance reduction or, equivalently, cost reduction techniques. One such method is
Multilevel Monte Carlo (MLMC), which was proposed in \cite{GILES2008} and was adapted for various applications that are summarised in \cite{Giles_overview17} and successfully combined with other methods such as quasi-Monte Carlo methods. The main idea of MLMC is to approximate the payoff using different time stepping resolutions when numerically solving the underlying SDE and to generate an optimal number of samples on each level, such that the overall computational cost is minimised subject to the desired bound on the variance. %, such that the total computational cost is minimised. 
The computational savings come from the fact that most samples are computed on the coarser levels and hence are less expensive while only a few samples from the finest levels are required \cite{GILES2008}.


Among the directions in which the computational cost 
of MLMC methods could further be reduced, an important avenue is the use of lower precision calculations, especially for the first Monte Carlo levels where the targeted accuracy is relatively low. 
 An overview of the research on mixed precision for the standard Monte Carlo (MC) framework is provided in \cite{ChowMixedPrecisionStandardMC} but only a few references study the potential of low precision computation in the MLMC framework \cite{Rounding_error_oliver}. To the best of our knowledge, the only MLMC framework with customised precision in the literature is \cite{brugger2014mixed}, but they use a uniform precision for all operations on each Monte Carlo level instead of optimising 
 the precision of each intermediary variable to reduce as much as possible the cost of path generation.
 
An important motivation for an MLMC framework with variable precision would be performing the low precision computations on reconfigurable hardware devices such as Field Programmable Gate Arrays (FPGAs). FPGAs contain customizable logic blocks and connectors that make it easy to adapt the digital circuit architecture for a specific application, leading to a highly parallel and optimised implementation. Therefore they are successfully exploited in applications that require high speed and have high computational workload, such as signal processing \cite{woods2008fpga}, and real time applications like high frequency trading \cite{HFT1,HFT2}. That is why a number of previous works in hardware architecture design implemented the MLMC algorithm to price financial options using FPGAs as accelerators, which resulted in improved speed and power efficiency compared to full CPU architectures \cite{Schryver2013AMM}. The paper \cite{lindsey2016domain} also proposed 
a Domain Specific Language to automate the configuration of FPGAs for this specific application. However, only \cite{brugger2014mixed} proposed a heuristic to reduce the precision in calculations.

In addition, all aforementioned works considered that the random number generation (RNG) is performed in single or double precision. Yet in most cases an important portion of the workload in the overall MLMC simulation comes from the RNG and in \cite{brugger2014mixed} this limited the total computational savings.
To reduce the cost of MLMC simulations in particular those based on the Geometric Brownian Motion (GBM), \cite{approximateICDF_Oliver, NestedOliver} have proposed to use approximate random numbers that are generated by applying an approximation of the inverse CDF to uniform random numbers. In \cite{NestedOliver}, the authors proposed a way to integrate these lower precision random variables into a \textit{nested} MLMC framework and completed a numerical analysis to bound the resulting error at each MC level by a product of the time step and the error in the random number approximation. The same authors show in \cite{approximateICDF_Oliver} that using approximate random variables reduces the cost of path generation by a factor 7.


In this paper we propose a nested MLMC framework that combines the use of approximate random normal variables and lower precision calculations to reduce the computational cost of MLMC even further than \cite{brugger2014mixed,NestedOliver}. We illustrate the efficiency of our framework in Matlab, after making several assumptions on the cost of operations and size of the errors that we carefully justify. We focus on the case of GBM and use the approximate RNG methods presented in \cite{approximateICDF_Oliver} as well as a new slightly modified method that combines CDF inversion and the central limit theorem. To choose the precision of the variables in the low precision path generation, we introduce a novel method to optimise the bit-widths. This optimisation is performed before the main path generation loop is executed and is based on a linear model of the payoff error  
due to rounding when computing in low precision. The error model relies on algorithmic differentiation in a similar manner to \cite{unifying-bwoptim,bitwidth-AD,ADAPT}. The bit-width optimisation procedure can be performed off-line, so this stage can be excluded from the on-line time complexity of our framework. The user specified desired accuracy is then enforced by calculating on-line the number of samples that need to be generated.

In terms of hardware design, we suggest implementing the low precision path generation on FPGAs and the full-precision ones on a CPU or GPU. 
The FPGA offers enough flexibility to define a separate bit-width for every variable in the low precision path generation, and can be reconfigured periodically to update the bit-widths when the market parameters have changed considerably. 


The paper is organized as follows : \Cref{sec:MLMC} introduces MLMC and nested MLMC to make clear the estimator that is implemented in our framework. Then in \Cref{sec:RNG} we detail the methods that could be used to obtain approximate random normally distributed numbers very cheaply for the low precision path generation. In \Cref{sec:error_model} and \Cref{sec:costModel} we propose an error model and a cost model (resp.) that we then use to formulate the optimisation problem that is solved to obtain the optimal bit-widths of fixed point variables in \Cref{sec:optimisation}. Finally we summarise our results and future directions in \Cref{sec:conclusion}.



\section{Related Work}

\subsection{First-order logic for natural entailment}

Since the start of the RTE challenge \citep{rte}, multiple works have attempted using FOL representations to solve natural language entailment. These methods first obtain the syntactic/semantic parse tree and apply a rule-based transformation to get the FOL representation \citep{bos-markert-2005-recognising, bos-nli}. However, it was repeatedly shown that these FOL representations are not empirically effective in solving natural language entailment. For instance, \citet{bos-nli} reported that FOL representations translated from the discourse representation structure (DRS) yield only 1.9\% recall in detecting the entailment in the single-premise RTE benchmark \citep{rte}.

Independently from these works, multi-premise logical entailment benchmarks \citep{tafjord-etal-2021-proofwriter, logicnli, folio} were developed to evaluate the reasoning ability of generative models. These benchmarks adopt the classic 3-way entailment label classification format (\textit{entailment, contradiction, neutral}) of single-premise RTE tasks, in which both the NL sentences and their gold FOL representations point to the same entailment label. 

Recent works have applied LLMs to obtain FOL representations for these multi-premise logical entailment tasks \citep{logiclm, linc, divide-and-translate}, fueled by the code generation ability of LLMs. While they achieve significant performance in synthetic, controlled logical reasoning benchmarks, whether they can generalize to natural entailment has remained unanswered. Furthermore, \citet{linc} observed that LLMs are highly susceptible to \textit{arbitrariness}, as they fail to produce coherent predicate names or numbers of arguments even when generating FOL representations of premises and hypotheses in a single inference.

\subsection{Executable semantic representations}

Apart from FOL, a stream of research focuses on the \textit{executability} of semantic representations. From this perspective, semantic representations are \textit{program codes} that can be executed to solve downstream tasks, such as query intent analysis \citep{spider, dligach-etal-2022-exploring} and question answering \citep{semparse-qa}. The performance of the semantic parser is directly assessed by the accuracy of execution results for the downstream tasks, rather than the similarity between the prediction and the reference parse.

To improve the execution accuracy that is often non-differentiable, reinforcement learning (RL) and its variants have been applied to train neural semantic parsers \citep{cheng-etal-2019-learning, cheng-lapata-2018-weakly}. Using only the input sentence and the desired execution result, these methods learn to maximize the probability of the representations that lead to the correct execution result. However, these approaches are not directly applicable to EPF, as EPF requires taking account of \textit{interactions between premises and hypotheses} during execution (\textit{i.e.} theorem proving) while these methods assume that sentences are isolated.




\section{Methodology}
\paragraph{Preliminaries.}
We primarily focus on the homologous model merging, in which $\boldsymbol{\theta}_i$ all come from the same base model $\boldsymbol{\theta}_{\rm{base}}$. Given $K$ tasks $\{T_1,T_2,\cdots,T_K\}$ and $K$ corresponding fine-tuned models with parameters $\{\boldsymbol{\theta}_1,\boldsymbol{\theta}_2,\cdots,\boldsymbol{\theta}_K\}$, model merging aims to combine $K$ fine-tuned models into one single model simultaneously performing on $\{T_1,T_2,\cdots,T_K\}$ without post-training~\cite{method_p1_1,method_p1_2}.
Task vector~\cite{ilharco2023editing,yang2024adamerging} is a key element in merging method which could enhances the base model‘s ability or enable the model to handle other tasks. Specifically, for task $T_i$, the task vector $\boldsymbol\tau_i\in \mathbb{R}^D$ is defined as the vector obtained by subtracting the SFT weights $\boldsymbol{\theta}_i$ from the base model weight
$\boldsymbol{\theta}_{\rm{base}}$, \emph{i.e.}, $\boldsymbol\tau_i=\boldsymbol{\theta}_i-\boldsymbol{\theta}_{\rm{base}}$. The merged model could be denoted as $\boldsymbol{\theta}_m=\boldsymbol{\theta}_{\rm{base}}+\sum_i \lambda_i\boldsymbol{\tau}_i$, which $\lambda_i$ is the scaling factor measuring the importance of task vector. For clarification, we also denote the neuron set in $\boldsymbol{\theta}_i$ as $\mathcal{N}_i$, the neuron set in $\boldsymbol{\tau}_i$ as $\mathcal{T}_i$.



\begin{algorithm}[!ht]
    \caption{LED-Merging}
    \label{alg1}
    \begin{algorithmic}[1]
        \REQUIRE  base model $\boldsymbol{\theta}_{\rm{base}}$, SFT models $\{\boldsymbol{\theta}_{i}\mid i\in [K]\}$, mask ratios \{$r_{i} \mid i\in [K]\}$, scaling factors $\{\lambda_i\mid i\in[K]\}$, location datasets $\{\mathcal{X}_{i}\mid i\in[K]\}$
        \ENSURE merged parameter $\boldsymbol{\theta}_{m}$
        \STATE $\mathcal{M}\leftarrow\phi$
        \STATE $\boldsymbol{\theta}_{m}\leftarrow \boldsymbol{\theta}_{\rm{base}}$
        \FOR{$i\in [K]$}
        \STATE $I(\boldsymbol{\theta}_i)=\mathbb{E}_{x\sim \mathcal{X}_i}|\boldsymbol{\theta}_{i}\odot \nabla_{\boldsymbol{\theta}_i}\mathcal{L}(x)|$
        \STATE $I(\boldsymbol{\theta}_{\rm{base}})=\mathbb{E}_{x\sim \mathcal{X}_i}|\boldsymbol{\theta}_{\rm{base}}\odot \nabla_{\boldsymbol{\theta}_{\rm{base}}}\mathcal{L}(x)|$
        
        \STATE calculate $\mathcal{T}^{r_i}_{i}$ following Equation \ref{vote}
        \STATE  $\mathcal{M}\leftarrow \mathcal{M}\cup\{\mathcal{T}^{r_i}_i\}$
       
        
   
        
        
        \ENDFOR  
        \FOR{$i\in [K]$}
        
        \STATE calculate $\text{Disjoint}(\mathcal{T}_i^{r_i})$ use Equation~\ref{disjoint_safety}
        \STATE $\boldsymbol{m}_i \leftarrow \boldsymbol{0}$
        \FOR{$d\in \mathcal{T}_i^{r_i}$}
        \STATE $\boldsymbol{m}_{i,d}=1$
        \ENDFOR
        \STATE $\boldsymbol{\theta}_{m}\leftarrow \boldsymbol{\theta}_{m}+\lambda_i \boldsymbol{\tau}_i\odot \boldsymbol{m}_{i}$
        \ENDFOR
    \end{algorithmic}
\end{algorithm}
    %\vspace{-5pt}
\begin{figure*}[h!]
    \centering
    \includegraphics[width=\linewidth]{figs/pipeline_v2.pdf}
    \vspace{-40mm}
    \caption{Overview of our two-stage training pipeline {\ours}.}
    \label{fig:pipeline}
\end{figure*}


\paragraph{LED-Merging: Location, Election, and Disjoint Merging}
To address the neuron misidentification and interference issues in existing model merging methods, we propose LED-Merging (Location, Election, and Disjoint Merging). Specifically, previous studies \cite{modelstock, ilharco2023editing, tiesmerging} fail to accurately identify safety-related neurons in task vectors with a single magnitude score, namely \textit{neuron misidentification}. Meanwhile, there exists an interference between safety-related and utility-related task vector neurons during the merging process, namely \textit{neuron interference}. To address neuron misidentification, we first locate important neurons both in the base and fine-tuned models and then elect neurons from the task vector considering these two scores together. Subsequently, to mitigate the interference, we introduce a disjoint step, isolating these important neurons so that they influence different base neurons. The whole process is illustrated in Figure~\ref{fig:method}. 




In the location and election step, we consider the importance score from base and fine-tuned models simultaneously to locate task-specific neurons. In this way, it is more accurate than relying on the magnitude score alone because task-specific neurons with high importance score in the fine-tuned model may not necessarily score high in the base model, and vice versa.

{\textbf{Location}}.  We first calculate importance scores for each neuron in a base/fine-tuned model. Given a location dataset $\mathcal{X}_i=\{(x,y)_k\}$, where $x$ is the question and $y$ is the answer, we calculate the importance scores for the weight $\boldsymbol{\theta}_i\in\mathbb{R}^D$ in any  layer as follows~\cite{snip,spareseGPT,sun2024a}:
\begin{equation}
    I(\boldsymbol{\theta}_i)=\mathbb{E}_{x\sim \mathcal{X}_i}[\boldsymbol{\theta}_i\odot \nabla _{\boldsymbol{\theta}_i}\mathcal{L}(x)],
    \label{location}
\end{equation}
which $\mathcal{L}(x)=-\log p(y\mid x)$ is the conditional negative log-likelihood loss. We choose the SNIP score~\cite{snip} because it balances computational efficiency and performance~\cite{cq}. Please refer to Sec.~\ref{sec:ablation} for the comparison between different location methods. After computing importance scores, we choose top-$r_i$ neurons as the important neuron subset $\mathcal{N}_{i}^{r_i}$ from $I(\boldsymbol{\theta}_i)$.
 
 % After computing locating scores, we select the neurons scoring both high in base and fine-tuned models as important neurons in task vectors. Then in the disjoint step,  with preventing  polysemantic neurons  from receiving gradient updates towards different directions,
 % we use set difference to isolate the safety   and utility-related neurons  and construct corresponding masks for merging process,

{\textbf{Election}}. A natural question is how to select important neurons in the task vector $\boldsymbol{\tau}_i$ based on $I(\boldsymbol{\theta}_{\rm{base}})$ and $I(\boldsymbol{\theta}_{i})$. The important neurons in the base model may be different from neurons in the fine-tuned model. Therefore, we introduce the following election strategy to select neurons with high scores in both base and fine-tuned models:
\begin{equation}
    \mathcal{T}_i^{r_i}=\mathcal{N}_i^{r_i}\cap \mathcal{N}_{\rm{base}}^{r_i}.
    \label{vote}
\end{equation}
\emph{Remark}. We compare different choosing methods, including scoring low or high in base or fine-tuned model in Section~\ref{sec:ablation} and find that Equation \ref{vote} achieves the best performance.





{\textbf{Disjoint}}. As important neurons from different task vectors may conflict with each other at the same position, we use the set difference to disjoint the neurons from others to prevent interference:
\begin{equation}
    \text{Disjoint}(\mathcal{T}^{r_i}_{i})=\mathcal{T}^{r_i}_{i}-\mathop{\cup}\limits_{{J}\subsetneqq [K],|J|\geq 2}\mathop{\cap}\limits_{j\in {J}}\mathcal{T}^{r_j}_{j}.
    \label{disjoint_safety}
\end{equation}

Next, we construct a mask $\boldsymbol{m}_i\in\mathbb{R}^D$ to implement disjoint in the merging process. Specifically, this mask $\boldsymbol{m}_i$ is used to select neurons from $\mathcal{T}_i$. The mask ratio is $r_i$, where $r\in(0,1]$. The mask $\boldsymbol{m}_i$ can be derived from:
\begin{equation}
    \boldsymbol{m}_{i,d}=\begin{aligned} &\left\{ \begin{array}{ll} 1, & \text{if } d\in \text{Disjoint}(\mathcal{T}_{i}^{r_i}), \\ 0, & \text{otherwise}. \end{array} \right. \end{aligned}
    \label{mask_safety}
\end{equation}


% \subsection{Merging Models with Masks}
{\textbf{Merging}}. The final
merged task vector $\boldsymbol{\tau}_m$ is as follows:
\begin{equation}
    \boldsymbol{\tau}_m= \sum_i \lambda_i\boldsymbol{\tau}_{i}\odot\boldsymbol{m}_i.
    \label{merged_task_vector}
\end{equation}
We summarize the workflow in Algorithm \ref{alg1}.



\section{Experiments}
\label{sec:experiment}

Experiments are carried out on NVIDIA RTX4090 GPUs using PyTorch 2.2.0 \cite{paszke2019pytorch} and the rotation detection tool kits: MMRotate 1.0.0 \cite{zhou2022mmrotate}. All the experiments follow the same hyper-parameters (learning rate, batch size, optimizer, etc.).

Average precision (AP) is adopted as the primary metric. All the models are configured upon ResNet50 \cite{he2016deep} and trained with AdamW \cite{loshchilov2018decoupled}.
\textbf{1) Learning rate.} Initialized at 5e-5, warm-up for 500 iterations, and divided by ten at each decay step. 
\textbf{2) Epochs.} 72 for HRSC; 12 for the others.
\textbf{3) Augmentation.} Random rotation/flip for HRSC; random flip for the others.
\textbf{4) Image size.} Split into 1,024 $\times$ 1,024 with an overlap of 200 for DOTA/FAIR1M/STAR; scaled to 800 $\times$ 800 for others.
\textbf{5) Multi-scale.} All experiments evaluated without multi-scale technique \cite{zhou2022mmrotate}. 
\textbf{6) Datasets.} Six remote sensing and one retail scene datasets, covering all datasets used by the main counterparts \cite{yu2024point2rbox, luo2024pointobb, cao2023p2rbox}:

\begin{table*}[!tb]
\fontsize{8.5pt}{10pt}\selectfont
\setlength{\tabcolsep}{0.65mm}
\setlength{\aboverulesep}{0.4ex}
\setlength{\belowrulesep}{0.4ex}
\setlength{\abovecaptionskip}{1.5mm}
\centering
\begin{tabular}{l|c|c|c|c|c|c|c|c|c|c}
\toprule
{\textbf{Methods}} & {*} & {\textbf{\,DOTA-v1.0\,}} & {\textbf{\,DOTA-v1.5\,}} & {\textbf{\,DOTA-v2.0\,}} & {\textbf{~~DIOR~~}} & {\textbf{~~HRSC~~}} & {\textbf{\,FAIR1M\,}} & {\textbf{~~STAR~~}} & {\textbf{\,SKU110K\,}} & {\textbf{~~RSAR~~}} \\
\hline
\rowcolor{gray!20} \multicolumn{11}{l}{$\blacktriangledown$ \textit{RBox-supervised OOD}} \\ \hline
RetinaNet (2017) \cite{lin2017focal} & \checkmark & 68.69 & 60.57        & 47.00 & 54.96 & 84.49   & 37.67   & 21.80 & 78.50 & 57.67  \\
GWD (2021) \cite{yang2021rethinking} & \checkmark & 71.66 & 63.27        & 48.87 & 57.60 & 86.67   & 39.11   & 25.30 & 79.16 & 57.80 \\
FCOS (2019) \cite{tian2019fcos} & \checkmark & 72.44 & 64.53        & 51.77    &  59.83  & 88.99  & 41.25   & \textbf{28.10} & 80.09 & \textbf{66.66} \\
S$^2$A-Net (2022) \cite{han2022align} & \checkmark & \textbf{75.81} & \textbf{66.53} & \textbf{52.39} & \textbf{61.41} & \textbf{90.10} & \textbf{42.44}   & 27.30 & \textbf{80.36} & 66.47 \\
\hline
\rowcolor{gray!20} \multicolumn{11}{l}{$\blacktriangledown$ \textit{HBox-supervised OOD}} \\ \hline
Sun et al. (2021) \cite{sun2021oriented} & $\times$ & 38.60 & - & - & - & - & - & - & - & - \\
KCR (2023) \cite{zhu2023knowledge} & \checkmark & - & - & - & - &  79.10  & -  & - & - & -  \\
H2RBox (2023) \cite{yang2023h2rbox} & \checkmark & 70.05 & 61.70        & 48.68    & 57.80 &  7.03  & 35.94  & 17.20 & 57.15 & 49.92    \\
H2RBox-v2 (2023) \cite{yu2023h2rboxv2} & \checkmark & 72.31 & 64.76 & 50.33 & 57.64 & \textbf{89.66} & \textbf{42.27} & \textbf{27.30} & \textbf{70.70} & \textbf{65.16} \\
AFWS (2024) \cite{lu2024afws} & \checkmark & \textbf{72.55} & \textbf{65.92} & \textbf{51.73} & \textbf{59.07} & - & 41.80 & - & - & - \\
\hline
\rowcolor{gray!20} \multicolumn{11}{l}{$\blacktriangledown$ \textit{Point-supervised OOD}} \\ \hline
P2RBox (2024) \cite{cao2023p2rbox}$^\dagger$ & $\times$ & \underline{59.04} & -        & - & - & -   & -  & -  & - & -  \\
PointSAM (2024) \cite{liu2024pointsam}$^\dagger$ & $\times$ & - & - & - & \textbf{46.20} & -   & -  & -  & - & - \\
PointOBB (2024) \cite{luo2024pointobb} & $\times$ & 30.08 & 10.66        & 5.53     &  37.31  & -   & 11.19 & 9.19  & - & 13.80    \\
Point2RBox+SK (2024) \cite{yu2024point2rbox}$^\dagger$ & \checkmark & 40.27 & 30.51        & 23.43    & 27.34 & 79.40   & 20.03 & 7.86  & 3.41 & 27.81    \\
PointOBB-v2 (2025) \cite{ren2024pointobbv2} & $\times$ & 41.68 & 30.59        & 20.64    &  39.56  & -   & 13.36 & 9.00  & 56.63 & 18.99   \\
PointOBB-v3 (2025) \cite{zhang2025pointobbv3} & $\checkmark$ & 41.20 & 31.25 & 22.82 & 37.60 & - & 11.42  & 11.31 & - & 15.84 \\
PointOBB-v3 (2025) \cite{zhang2025pointobbv3} & $\times$ & 49.24 & 33.79 & 23.52 & 40.18 & - & 18.35 & \underline{12.85} & - & 22.60 \\
\rowcolor{gray!20} Point2RBox-v2 (ours) & \checkmark & 51.00 & \underline{39.45} & \underline{27.11} & 34.70 & \underline{82.67} & \underline{25.72} & 7.80 & \underline{64.00} & \underline{28.60}
 \\
\rowcolor{gray!20} Point2RBox-v2 (ours) & $\times$ & \textbf{62.61} & \textbf{54.06}        & \textbf{38.79}   & \underline{44.45}  & \textbf{86.15}   & \textbf{34.71}  & \textbf{14.20} & \textbf{65.64} & \textbf{30.90}    \\
\bottomrule
\specialrule{0pt}{2pt}{0pt}
\multicolumn{11}{l}{$^*$Comparison tracks: \checkmark = End-to-end training and testing; $\times$ = Generating pseudo labels to train the FCOS detector (two-stage training).} \\
\multicolumn{11}{l}{$^\dagger$Using additional priors. P2RBox/PointSAM: Pre-trained SAM model; Point2RBox+SK: One-shot sketches for each class.} \\
\bottomrule
\end{tabular}
\caption{Accuracy (AP$_{50}$) comparisons on the DOTA-v1.0/1.5/2.0, DIOR, HRSC, FAIR1M, STAR, SKU110K, and RSAR datasets.}
\label{tab:exp_other}
\vspace{-4pt}
\end{table*}

\begin{itemize}
    \item \textbf{DOTA \cite{xia2018dota}.} DOTA-v1.0 has 2,806 aerial images annotated with 15 categories, while DOTA-v1.5/2.0 are the extended versions with more small objects and categories.
    
    \item \textbf{DIOR \cite{cheng2022anchor}.} It is an aerial image dataset re-annotated with RBoxes based on its original HBox version \cite{li2020object}, with a high variation in object size and high intra‐class diversity. 

    \item \textbf{HRSC \cite{liu2017hrsc}.} It contains ship instances on the sea and inshore. The train/val/test set includes 436/181/444 images.

    \item \textbf{FAIR1M \cite{sun2022fair1m}.} It has more than 1 million instances and more than 40,000 images for fine-grained object recognition in remote sensing imagery, annotated with 37 categories. The results are evaluated on FAIR1M-1.0.

    \item \textbf{STAR \cite{li2024star}.} It is extensive for scene graph generation, covering more than 210,000 objects with diverse spatial resolutions, classified into 48 fine-grained categories and precisely annotated with oriented bounding boxes. 

    \item \textbf{SKU110K \cite{pan2020dynamic}.} It focuses on the detection of densely packed retail scenes with 110,712 objects in 11,762 images. The density reaches 86 instances per image. 

    \item \textbf{RSAR \cite{zhang2025rsar}.} It is a remote sensing dataset based on Synthetic Aperture Radar (SAR) imagery with 6 categories.

\end{itemize}

\begin{table*}[!tb]
\fontsize{8.5pt}{10pt}\selectfont
\setlength{\tabcolsep}{2.08mm}
\setlength{\aboverulesep}{0.4ex}
\setlength{\belowrulesep}{0.4ex}
\setlength{\abovecaptionskip}{1.5mm}
\hspace{1pt}
\begin{minipage}[t]{0.315\linewidth}
\centering
\begin{tabular}{c|cc|cc}
\toprule
\multirow{2}{*}{$w_\text{O}$} & \multicolumn{2}{c|}{\textbf{DOTA}} & \multicolumn{2}{c}{\textbf{HRSC}} \\
                  & {E2E} & {FCOS} & {E2E} & {FCOS} \\ \midrule
3  & 48.76 & 61.62 & 81.85 & 84.36 \\
5  & 49.81 & 62.44 & 82.46 & 85.76 \\
\rowcolor{gray!20} 10 & \textbf{51.00} & \textbf{62.61} & \textbf{82.67} & \textbf{86.15} \\
30 & 45.88 & 57.83 & 81.56 & 85.61 \\
\bottomrule
\end{tabular}
\caption{Ablation with the weight of $\mathcal{L}_\text{O}$.}
\label{tab:abl_lo}
\end{minipage}
\quad
\begin{minipage}[t]{0.315\linewidth}
\centering
\begin{tabular}{c|cc|cc}
\toprule
\multirow{2}{*}{$w_\text{W}$} & \multicolumn{2}{c|}{\textbf{DOTA}} & \multicolumn{2}{c}{\textbf{HRSC}} \\
                  & {E2E} & {FCOS} & {E2E} & {FCOS} \\ \midrule
3  & 50.85 & 56.78 & 78.42 & 83.49 \\
\rowcolor{gray!20} 5  & \textbf{51.00} & \textbf{62.61} & \textbf{82.67} & \textbf{86.15} \\
10 & 49.15 & 60.54 & 30.37 & 35.13 \\
30 & 42.84 & 52.53 & 23.89 & 25.91 \\
\bottomrule
\end{tabular}
\caption{Ablation with the weight of $\mathcal{L}_\text{W}$.}
\label{tab:abl_lw}
\end{minipage}
\quad
\begin{minipage}[t]{0.315\linewidth}
\setlength{\tabcolsep}{2.04mm}
\centering
\begin{tabular}{c|cc|cc}
\toprule
\multirow{2}{*}{$w_\text{E}$} & \multicolumn{2}{c|}{\textbf{DOTA}} & \multicolumn{2}{c}{\textbf{HRSC}} \\
                  & {E2E} & {FCOS} & {E2E} & {FCOS} \\ \midrule
0.1 & 48.75 & 57.62 & 34.71 & 39.45 \\
\rowcolor{gray!20} 0.3 & 51.00 & 62.61 & \textbf{82.67} & \textbf{86.15} \\
0.5 & \textbf{51.36} & \textbf{62.63} & 76.85 & 85.22 \\
1.0 & 49.05 & 60.63 & 56.59 & 59.59 \\
\bottomrule
\end{tabular}
\caption{Ablation with the weight of $\mathcal{L}_\text{E}$.}
\label{tab:abl_le}
\end{minipage}
\vspace{-4pt}
\end{table*}

\begin{table*}[!tb]
\fontsize{8.5pt}{10pt}\selectfont
\setlength{\tabcolsep}{2.04mm}
\setlength{\aboverulesep}{0.4ex}
\setlength{\belowrulesep}{0.4ex}
\setlength{\abovecaptionskip}{1.5mm}
\hspace{1pt}
\begin{minipage}[t]{0.315\linewidth}
\centering
\begin{tabular}{c|cc|cc}
\toprule
\multirow{2}{*}{$w_\text{ss}$} & \multicolumn{2}{c|}{\textbf{DOTA}} & \multicolumn{2}{c}{\textbf{HRSC}} \\
                  & {E2E} & {FCOS} & {E2E} & {FCOS} \\ \midrule
0.1 & 49.28 & 59.66 & 73.66 & 78.92 \\
\rowcolor{gray!20} 1.0 & \textbf{51.00} & \textbf{62.61} & \textbf{82.67} & \textbf{86.15} \\
3.0 & 49.15 & 59.20 & 1.30  & 1.65 \\
\bottomrule
\end{tabular}
\caption{Ablation with the weight of $\mathcal{L}_\text{ss}$.}
\label{tab:abl_lss}
\end{minipage}
\quad
\begin{minipage}[t]{0.647\linewidth}
\setlength{\tabcolsep}{3.05mm}
\centering
\begin{tabular}{c|c|c||c|c|c}
\toprule
{R / F / S} & {\textbf{DOTA}} & {\textbf{HRSC}} & {R / F / S} & {\textbf{DOTA}} & {\textbf{HRSC}} \\
 \midrule
90\% / 10\% / 0\% & 60.42 & 85.46 & 80\% / 20\% / 0\%  & 59.46 & 84.73 \\
75\% / 0\% / 25\% & 60.79 & 86.22 & 60\% / 15\% / 25\% & 62.38 & 84.21 \\
\cellcolor{gray!20}68\% / 7\% / 25\% & \cellcolor{gray!20}\textbf{62.61} & \cellcolor{gray!20}\textbf{86.15} & 38\% / 37\% / 25\% & 45.87 & 8.56  \\
45\% / 5\% / 50\% & 60.55 & 85.34 & 40\% / 10\% / 50\% & 60.49 & 10.74 \\
\bottomrule
\end{tabular}
\caption{Ablation with the proportion of augmented views in self-supervision.}
\label{tab:abl_pro}
\end{minipage}
\vspace{-10pt}
\end{table*}

\subsection{Main Results on DOTA-v1.0}
\label{sec:experiment-main}

Table \ref{tab:exp_dota} compares Point2RBox-v2 with the state-of-the-art methods, which can be categorized into two tracks: 

\textbf{1) End-to-end training.} These methods apply the trained weakly-supervised detector directly to the test set. Without relying on priors, our approach demonstrates an improvement of 16.93\% (51.00\% vs. 34.07\%) compared to Point2RBox. Even when compared to Point2RBox+SK, which incorporates additional data-side priors (i.e. one-shot examples for each class), our method still outperforms it by 10.73\% (51.00\% vs. 40.27\%).

\textbf{2) Two-stage training.} These methods generate RBox labels on train/val sets, with which the FCOS detector is trained. In this two-stage mode, Point2RBox-v2 achieves an accuracy of 62.61\%, considerably surpassing PointOBB series. Remarkably, it even outperforms the SAM-powered method P2RBox by 3.57\% (62.61\% vs. 59.04\%).

\textbf{Class-wise analysis.} The FCOS detector trained with labels generated by Point2RBox-v2 achieves accuracy nearly equivalent to RBox-supervised FCOS across six high-density categories: SH (86.9\% vs. 87.1\%), SV (79.6\% vs. 79.8\%), LV (76.3\% vs. 79.8\%), PL (88.0\% vs. 89.1\%), ST (82.9\% vs. 84.6\%), and TC (89.1\% vs. 90.4\%). Interestingly, these six high-density categories account for 88\% of DOTA instances. By annotating these categories with points and generating RBoxes using Point2RBox-v2 while labeling the other sparse categories with RBoxes, we can significantly reduce annotation labor without sacrificing much accuracy, highlighting the valuable role our method can play.

\begin{figure*}[t!]
\setlength{\abovecaptionskip}{1.2mm}
\centering
\includegraphics[width=0.96\linewidth]{figs/case.pdf}
\caption{Qualitative analysis on failed cases and overlap cases.}
\label{fig:case}
\vspace{-6pt}
\end{figure*}

\subsection{Results on More Datasets}

The results are displayed in Table \ref{tab:exp_other}.
On more challenging DOTA-v1.5/2.0, Point2RBox-v2 presents a similar trend, 23.47\%/18.15\% higher than PointOBB-v2 in the pseudo-generation track. 
On the ship detection dataset HRSC, the gap between Point2RBox-v2 and RBox-supervised FCOS is only 2.84\% (86.15\% vs. 88.99\%).
DIOR is relatively sparse, leading to less improvement with our methods---lower than PointSAM (44.45\% vs. 46.20\%) but still higher than methods that do not use SAM. 
Our method also provides competitive performance on fine-grained datasets FAIR1M and STAR. 
In addition to remote sensing scenarios, we carry out experiments on SKU110K for densely packed retail scenes. Existing point-supervised methods struggle in this case, whereas Point2RBox-v2 achieves performance on par with HBox-supervised H2RBox (65.64\% vs. 57.15\%).

\begin{table}[!tb]
\fontsize{8.5pt}{10pt}\selectfont
\setlength{\tabcolsep}{1.78mm}
\setlength{\aboverulesep}{0.4ex}
\setlength{\belowrulesep}{0.4ex}
\setlength{\abovecaptionskip}{1.5mm}
\centering
\begin{tabular}{ccccc|cc|cc}
\toprule
\multicolumn{5}{c|}{\textbf{Modules}} & \multicolumn{2}{c|}{\textbf{DOTA}} & \multicolumn{2}{c}{\textbf{HRSC}} \\
$\mathcal{L}_\text{O}$ & $\mathcal{L}_\text{W}$ & $\mathcal{L}_\text{ss}$ & $\mathcal{L}_\text{E}$ & \textit{CP} & {E2E} & {FCOS} & {E2E} & {FCOS} \\ \midrule
\checkmark & & & & & 0.00 & 0.00 & 0.00 & 0.00 \\
\checkmark & \checkmark & & & & 41.54 & 52.98 & 17.96 & 19.64 \\
\checkmark & \checkmark & \checkmark & & & 46.64 & 54.26 & 18.10 & 22.13 \\
\checkmark & \checkmark & \checkmark & \checkmark & & 49.55 & 61.88 & 78.79 & 83.79 \\
& \checkmark & \checkmark & \checkmark & \checkmark & 48.58 & 59.56 & 20.35 & 24.76 \\
\checkmark & & \checkmark & \checkmark & \checkmark & 38.94 & 48.44 & 11.64 & 14.93 \\
\checkmark & \checkmark & \checkmark & & \checkmark & 47.08 & 55.05 & 19.58 & 21.78 \\
\rowcolor{gray!20} \checkmark & \checkmark & \checkmark & \checkmark & \checkmark & \textbf{51.00} & \textbf{62.61} & \textbf{82.67} & \textbf{86.15} \\
\bottomrule
\end{tabular}
\caption{Ablation with incremental addition of modules.}
\label{tab:abl_mod}
\vspace{-4pt}
\end{table}

\begin{table}[!tb]
\fontsize{8.5pt}{10pt}\selectfont
\setlength{\tabcolsep}{2.85mm}
\setlength{\aboverulesep}{0.4ex}
\setlength{\belowrulesep}{0.4ex}
\setlength{\abovecaptionskip}{1.5mm}
\centering
\begin{tabular}{c|c|c||c|c|c}
\toprule
16 & \cellcolor{gray!20}$K\!=\!24$ & 32 & 1.2 & \cellcolor{gray!20}$\beta\!=\!1.6$ & 2.0 \\ \midrule
50.87 & \cellcolor{gray!20}\textbf{51.00} & 48.08 & 48.14 & \cellcolor{gray!20}51.00 & \textbf{51.33} \\
\bottomrule
\end{tabular}
\caption{Ablation with $K$ and $\beta$ in edge loss on DOTA (E2E).}
\label{tab:abl_edgeparam}
\vspace{-4pt}
\end{table}

\begin{table}[!tb]
\fontsize{8.5pt}{10pt}\selectfont
\setlength{\tabcolsep}{1.75mm}
\setlength{\aboverulesep}{0.4ex}
\setlength{\belowrulesep}{0.4ex}
\setlength{\abovecaptionskip}{1.5mm}
\centering
\begin{tabular}{c|cc|cc|cc}
\toprule
\multirow{2}{*}{$\sigma$} & \multicolumn{2}{c|}{Point2RBox} & \multicolumn{2}{c|}{PointOBB-v2} & \multicolumn{2}{c}{Point2RBox-v2} \\
 & {\textbf{DOTA}} & {\textbf{HRSC}} & {\textbf{DOTA}} & {\textbf{HRSC}} & {\textbf{DOTA}} & {\textbf{HRSC}} \\ \midrule
0\%  & 40.27 & 79.40 & 44.85 & - & 62.61 & 86.15 \\
10\% & 39.60 & 78.81 & 42.30 & - & 61.58 & 85.76 \\
30\% & 38.42 & 78.28 & 38.46 & - & 60.31 & 85.71 \\
\bottomrule
\end{tabular}
\caption{Ablation with the inaccuracy in point annotations.}
\label{tab:abl_noise}
\vspace{-10pt}
\end{table}

\subsection{Ablation Studies}
\label{sec:experiment-ablation}

Tables \ref{tab:abl_lo}-\ref{tab:abl_noise} display the ablation studies on DOTA-v1.0 and HRSC. ``E2E'' denotes end-to-end training; ``FCOS'' denotes two-stage training (i.e. generating pseudo labels to train FCOS). The final values adopted are highlighted in gray.

\textbf{Weight of each loss.} Tables \ref{tab:abl_lo}-\ref{tab:abl_le} determine the weights of the proposed losses. Based on these experiments, the weights $(w_\text{O},w_\text{W},w_\text{E},w_\text{ss})$ are set to $(10, 5, 0.3, 1)$.

\textbf{Proportion of augmented views.} Table \ref{tab:abl_pro} studies the proportion between rotation, flip, and scale. The results are reported with two-stage training (FCOS). Based on the results, the proportion is set to 68\%, 7\%, and 25\%.

\textbf{Incremental addition of modules.} Table \ref{tab:abl_mod} demonstrates the constraints from Gaussian and Voronoi achieve an accuracy of 52.98\% on DOTA. Adding consistency loss and edge loss further boosts it to 54.26\% and 61.88\%, respectively, whereas the improvement from copy-paste is 0.73\%. We also demonstrate the impact of omitting each core loss.

\textbf{Edge loss parameters.} We set $K=24$ and $\beta=1.6$ as they are observed to discern the correct edges during code development. Table \ref{tab:abl_edgeparam} provides a more precise ablation.

\textbf{Annotation inaccuracy.} We offset the annotated points by a noise from the uniform distribution $\left[-\sigma H, +\sigma H \right ]$, where $H$ is the height of objects. Table \ref{tab:abl_noise} shows that the AP$_{50}$ of Point2RBox-v2 decreases by less than 3\% when noise is added to point annotations, demonstrating the robustness of the proposed learning mechanisms.

\subsection{More Discussions}
\label{sec:experiment-discussions}

The qualitative analysis on the failed/overlap cases is shown in Fig. \ref{fig:case}. \textbf{1) Failed cases.} Although our method performs well overall, it struggles with certain categories that are sparse and not constrained by other objects. \textbf{2) Overlap cases.} 
Minimizing overlap as a soft constraint during training does not entirely eliminate overlap. Once trained, the model remains robust to some overlap during inference.

\section{Conclusion}

%In this paper, w
We propose a new PEFT method called DiffoRA, which enables efficient and adaptive LLM fine-tuning based on LoRA. 
Instead of adjusting every interior rank, 
%of the decomposition matrices 
%of all modules, 
we argue that adopting LoRA module-wisely is sufficient. 
To achieve this, we construct a DAM to select the modules that are most suitable and essential to fine-tune. We theoretically analyze how the DAM impacts the convergence rate and generalization capability.
%of the pre-trained model. 
Furthermore, we adopt continuous relaxation and discretization to establish DAM.
%for each task. 
To alleviate the issue of discretization discrepancy, we utilize the weight-sharing strategy for optimization. 
%We fully implement our method and t
The experimental results demonstrate that our DiffoRA works consistently better than the baselines across all benchmarks. 

\clearpage
\newpage
\section{Limitations}

Our method imposes certain constraints on its applicability to existing decoder-only large language models (LLMs) due to its reliance on parallel encoding/decoding capabilities during the pre-filling stage. This requirement limits its direct adoption in conventional autoregressive LLMs. However, it is worth noting that many high-performance language models with parallel encoding/decoding capabilities have already become standard choices in various Retrieval-Augmented Generation (RAG) systems, such as FiD~\cite{DBLP:conf/eacl/IzacardG21}, CEPE~\cite{DBLP:conf/acl/YenG024}, and Parallel Windows~\cite{DBLP:conf/acl/RatnerLBRMAKSLS23}. Furthermore, our approach requires such models only during the reranker training phase; once trained, the reranker itself is independent of any specific LLM and can be flexibly adapted to other decoder-only models. Therefore, our method primarily serves as a general training framework rather than imposing architectural constraints on the final inference model. Additionally, our approach introduces extra hyperparameters in the Gumbel-Softmax process, including the temperature parameter $\tau$ and the scaling factor $\kappa$, which require tuning to achieve optimal performance. However, through empirical studies, we find that $\tau=0.5$ and $\kappa=1.0$ provide robust and stable performance across different model architectures and datasets. We provide a further discussion on the effect of $\tau$ and $\kappa$ in \autoref{sec: Effect of hyper-parameters on the Training Process}.

\section{Ethical Considerations}
While our method aims to improve the accuracy of the RAG system, it does not eliminate the inherent risks of biased data or model outputs, as the performance of RAG systems still heavily depends on the quality of training data and underlying models. The potential for bias in the training data, particularly for domain-specific queries, can lead to the amplification of these biases in the retrieved results, which can impact downstream applications.

\clearpage 
\newpage 

\bibliography{references}

\newpage 
\appendix
\onecolumn
\section{Appendix} \label{sec:appendix}

\subsection{Prompts used for generating tabular data} \label{lst:prompt}
This prompt template is used in Section~\ref{method} to generate realistic data that follows the same distribution as the given real data.

% [language=java]
\begin{lstlisting} 
You are a synthetic data generator tasked with creating new tabular data samples that closely mirror the distribution and characteristics of the original dataset.

# Instruction
1. Analyze the provided real samples carefully.
2. Generate synthetic data that maintains the statistical properties of the real data.
3. Ensure all attributes cover their full expected ranges, including less common or extreme values.
4. Maintain the relationships and correlations between different attributes.
5. Preserve the overall distribution of the real data while introducing realistic variations.

# Key points to consider
- Replicate the data types of each column (e.g., numerical, categorical).
- Match the range and distribution of numerical attributes.
- Maintain the frequency distribution of categorical attributes.
- Reflect any patterns or trends present in the original data.
- Introduce realistic variability to avoid exact duplication.

# Real samples
{data}

# Output format:
Please present the generated data in a JSON format, structured as a list of objects, where each object represents a single data point with all attributes.

\end{lstlisting}

\subsection{Dummy Prompt} \label{appendix:dummy_prompt}
The following prompt only contains the column names, but not any actual data in it. It is used to produce the results in Fig.\ref{fig:in-context-examples} (a).
\begin{lstlisting}
You are a synthetic data generator tasked with creating new tabular data samples that closely mirror the distribution and characteristics of the original dataset.
Generate 50 samples of synthetic data.
    
Each sample should include the following attributes:
{attributes_list}

Make sure that the numbers make sense for each attribute. 

Output Format:
Present the generated data in a JSON format, structured as a list of objects, where each object represents a single data point with all attributes.

\end{lstlisting}


\subsection{JSON Schema} \label{lst:json}
The following code define the JSON data class for the structured output function of GPT-4o and GPT-4o-mini.
\begin{lstlisting}
def create_json_model(df: pd.DataFrame, dataname=None) -> BaseModel:
    fields = {}
    
    for column in df.columns:
        if df[column].dtype == 'object':  
            fields[column] = (str, ...)
        elif df[column].dtype == 'int64':
            fields[column] = (int, ...)
        elif df[column].dtype == 'float64':
            fields[column] = (float, ...)
        elif df[column].dtype == 'bool':
            fields[column] = (bool, ...)
        else:
            raise TypeError(f"Unexpected dtype for column {column}: {df[column].dtype}")
    
    JSONModel = create_model(dataname, **fields)
    
    class JSONListModel(BaseModel):
        JSON: List[JSONModel]

    return JSONListModel
\end{lstlisting}

\subsection{Heuristic for computing residual}
In this section, we provide the pseudo-code of our heuristic strategy for computing the residual.
\begin{algorithm}
\caption{Compute residual \label{appendix:res_alg}}
\begin{algorithmic}[1]
\Require current dataset $X$, target dataset $Y$, distribution distance $d$.
\State Randomly select a column index $j$
\If{column $j$ is categorical}
    \State Let $C_j$ be the number of categories in column $j$
    \State Group samples in $Y$ into $C_j$ number of subsets based on its category on column $j$, denote the set of subsets by $(Y_j^i)_{i=1}^{C_j}$
\Else
    \State Quantize column $i$ into 50 bins
    \State $C_j\leftarrow$ 50
    \State Group samples in $Y$ into $C_j$ number of subsets based on its bin index on column $j$, denote the set of subsets by $(Y_j^i)_{i=1}^{C_j}$
\EndIf
\For{$i = 1$ to $C_j$} 
    \State Compute distance between $Y_j^i \cup X$ and $Y$: $d_i = d(Y_j^i \cup X, Y)$
\EndFor
%\State Sort distances in ascending order
\State \Return subset $Y_j^i$ that attains the minimal distance.
\end{algorithmic}
\end{algorithm}

\subsection{Datasets} \label{appendix:datasets}
We use five real-world datasets of varying scales, and all of them are available at Kaggle\footnote{\url{https://www.kaggle.com/}} or the UCI Machine Learning repository\footnote{\url{https://archive.ics.uci.edu/}}. We consider five datasets containing both numerical and catergorical attributes: California\footnote{\url{https://www.kaggle.com/datasets/camnugent/california-housing-prices}}, Magic\footnote{\url{https://archive.ics.uci.edu/dataset/159/magic+gamma+telescope}}, Adult\footnote{\url{https://archive.ics.uci.edu/dataset/2/adult}}, Default\footnote{\url{https://archive.ics.uci.edu/dataset/350/default+of+credit+card+clients}}, Shoppers\footnote{\url{https://archive.ics.uci.edu/dataset/468/online+shoppers+purchasing+intention+dataset}}. The statistics of these datasets are presented in Table~\ref{tbl:stat-dataset}. 
\begin{table}[h] 
    \centering
    \caption{Statistics of datasets. \# Num stands for the number of numerical columns, and \# Cat stands for the number of categorical columns.} 
    \label{tbl:stat-dataset}
    \small
    \begin{threeparttable}
    {
    \scalebox{0.95}
    {
	\begin{tabular}{lccccccccc}
            \toprule[0.8pt]
            \textbf{Dataset}  &  \# Rows  & \# Num & \# Cat & \# Train  & \# Test  \\
            \midrule 
            \textbf{California} Housing  & $20,640$ & $9$ & 1 & $18,390$ & $2,250$   \\
            % \textbf{Letter} Recognition & $20,000$ & $16$ & 1 & $14,000$ & $6,000$ \\
            % \textbf{Gesture} Phase Segmentation & $9,522$ & $32$ & 1 & $6,665$ & $2,857$ \\
            \textbf{Magic} Gamma Telescope & $19,021$ & $10$ & 1 & $17,118$ & $1,903$  \\
            % Dry \textbf{Bean} & $13,610$ & $17$ & - & $9,527$ & $4,083$ \\
            % \midrule
            \textbf{Adult} Income & $32,561$ & $6$ & $8$ & $22,792$ & $9,769$ \\
            \textbf{Default} of Credit Card Clients & $30,000$ & $14$ & $10$ & $27,000$ & $3,000$\\
            Online \textbf{Shoppers} Purchase & $12,330$ & $10$ & $7$ & $11,098$ & $1,232$ \\
            % \textbf{Beijing} PM2.5 data & $43,824$ & $6$ & $5$ & $29,229$ & $12,528$\\
            % Online \textbf{News} Popularity & $39,644$ & $45$ & $2$ & $27,790$ & $11,894$  \\
		\bottomrule[1.0pt] 
		\end{tabular}
   }
  }        
  \end{threeparttable}
\end{table}
%%%%%%%%%%%%%%%%%%%%%%%%%%%%%%%%%%%%%%%%%%%%%%%%%%%%%%%%%%%%

\subsection{Evaluation Metrics} \label{appendix:metrics}
\paragraph{Fidelity}
To evaluate if the generated data can faithfully recover the ground-truth data distribution, we employ the following metrics: 
1) \textbf{Marginal distribution:} The Marginal metric evaluates if each column's marginal distribution is faithfully recovered by the
synthetic data. We use Kolmogorov-Sirnov Test for continuous data and Total Variation Distance for discrete data.
2) \textbf{Pair-wise column correlation}: This metric evaluates if the correlation between every two columns in the real data is captured by the synthetic data. We compute the Pearson Correlation between all pairs of columns then take average. In addition, we present joint density plots for the Longitude and Latitude features in the California Housing data set in Figure \ref{fig:2d-california}.  
3) \textbf{Classifier Two Sample Test (C2ST):} This metric evaluates how difficult it is to distinguish real data from synthetic data. Specifically, we create an augmented table that has all the rows of real data and all the rows of synthetic data. Add an extra column to keep track of whether each original row is real or synthetic. Then we train a Logistic Regression classifier to distinguish real and synthetic rows. 
4) \textbf{Precision and Recall:} Precision measures the quality of generated samples. High precision means the generated samples are realistic and similar to the true data distribution. Recall measures how much of the true data distribution is covered by the generated distribution. High recall means the model captures most modes/variations present in the true data.
5) \textbf{Jensen-Shannon Divergence (JSD):} This metric evaluates the Jensen-Shannon divergence \citep{jsd} between the distributions of real data and synthetic data.


\paragraph{Utility} 
We evaluate the utility of the generated data by assessing their performance in Machine Learning Efficiency (MLE). Following the previous works \cite{tabsyn},  we first split a real table into a real training and a real testing set. The generative models are trained on the real training set, from which a synthetic set of equivalent size is sampled. This synthetic data is then used to train a classification/regression model (XGBoost Classifier and XGBoost Regressor~\citep{xgboost}), which will be evaluated using the real testing set. The performance of MLE is measured by the AUC score for classification tasks and RMSE for regression tasks.

\paragraph{Privacy}
A high-quality synthetic dataset should accurately reflect the underlying distribution of the original data, rather than merely replicating it. To assess this, we employ the Distance to Closest Record (DCR) metric. We begin by splitting the real data into two equal parts: a training set and a holdout set. Using the training set, we generate a synthetic dataset. We then measure the distances between each synthetic data point and its nearest neighbor in both the training and holdout sets. In theory, if both sets are drawn from the same distribution, and if the synthetic data effectively captures this distribution, we should observe an equal proportion (around 50$\%$) of synthetic samples closer to each set. However, if the synthetic data simply copies the training set, a significantly higher percentage would be closer to the training set, well exceeding the expected $50\%$.


\subsection{Scalability of \modelname}
To evaluate the scalability of \modelname, we compare \modelname with CLLM on a large-scale dataset: Covertype dataset. This dataset consists of 581,012 instances and 54 features. \modelname and CLLM use iterative in-context learning to generate samples, thus the running time of these two methods are agnostic to the size of the training dataset, making them scalable to large datasets. In the following table, we compare \modelname with CLLM, employed with both GPT-4o mini and GPT-4o. 
\begin{table}[!t]
\begin{center}
\caption{Performance Comparison on the \textbf{Covertype} Dataset. For all the metrics except AUC, we scale the metrics to a base of 
$10^{-2}$, and reverse it so that lower values indicate better performance. For AUC, the higher the better.}
\begin{tabular}{lrrrrrrr}
\hline
Model & Marginal $\downarrow$ & Corr $\downarrow$ & C2ST $\downarrow$ & Precision $\downarrow$ & Recall $\downarrow$ & JSD $\downarrow$ & AUC $\uparrow$ \\
\hline
%\multicolumn{8}{l}{\textbf{Performance with GPT-4o}} \\
\hline
CLLM w. GPT-4o & 3.28 & 10.70 & 55.67 & 32.32 & 1.95 & 0.6078 & 0.8822 \\
TabGEN w. GPT-4o & 2.83 & 11.10 & 49.91 & 24.03 & 1.27 & 0.5109 & 0.9070 \\
Improvement (\%) & $\redbf{13.72\%}$ & - & $\redbf{10.34\%}$ & $\redbf{25.62\%}$ & $\redbf{34.87\%}$ & $\redbf{15.94\%}$ & $\redbf{2.81\%}$ \\
\hline
%\multicolumn{8}{l}{\textbf{Performance with GPT-4o mini}} \\
\hline
CLLM w. GPT-4o mini & 5.35 & 13.70 & 0.7796 & 0.3225 & 0.0591 & 0.8743 & 0.7113 \\
TabGEN w. GPT-4o mini & 5.09 & 12.62 & 0.7554 & 0.2876 & 0.0436 & 0.9353 & 0.8311 \\
Improvement (\%) & $\redbf{4.86\%}$ & $\redbf{7.88\%}$ & $\redbf{3.11\%}$ & $\redbf{10.81\%}$ & $\redbf{26.25\%}$ & - & $\redbf{16.86\%}$ \\
\hline
\end{tabular}
\end{center}
\end{table}

The results demonstrate that \modelname outperforms CLLM on most of the metrics. Notably, the Recall metric again shows the greatest improvement: 34.85\% on GPT-4o and 26.25\% on GPT-4o mini. This observation is consistent with our original findings in Sec. \ref{sec:bsl}. We believe these results strongly support \modelname's scalability to larger, more complex datasets.

% \subsection{Experimental Results}

% \begin{table}[h] 
%   \centering
%   \caption{Error rate (\%) of \textbf{column-wise density estimation}. Lower values indicate more accurate estimation (superior results). } 
%   \label{tbl:exp-shape}
%   \small
%   \begin{threeparttable}
%   {
%   \scalebox{0.92}
%   {
% \begin{tabular}{lccccc|cccc}
%           \toprule[0.8pt]
%           \textbf{Method} & \textbf{Adult} & \textbf{Default} & \textbf{Shoppers} & \textbf{Magic}  & \textbf{California}&  \textbf{Average}  \\
%           \midrule 
%            SMOTE & $1.60${\tiny$\pm 0.23$} & $1.48${\tiny$\pm0.15$} & $2.68${\tiny$\pm0.19$} & $0.91${\tiny$\pm0.05$}  &$1.15${\tiny$\pm0.14$} &    $2.30$ & \\
%           \midrule
%           CTGAN    & $16.84${\tiny$\pm$ $0.03$} & $16.83${\tiny$\pm$$0.04$} & $21.15${\tiny$\pm0.10$} & $9.81${\tiny$\pm0.08$}  &$12.84${\tiny$\pm0.09$} & $17.02$  \\
%           TVAE     & $14.22${\tiny$\pm0.08$} & $10.17${\tiny$\pm$$0.05$} & $24.51${\tiny$\pm0.06$} & $8.25${\tiny$\pm0.06$}  &  $5.37${\tiny$\pm0.06$}  & $15.49$   \\
%           GReaT     & $12.12${\tiny$\pm$$0.04$} & $19.94${\tiny$\pm$$0.06$}  & $14.51${\tiny$\pm0.12$}  &  $16.16${\tiny$\pm0.09$}   & $10.25${\tiny$\pm0.20$}  & $14.20$ \\
%           STaSy    & $11.29${\tiny$\pm0.06$} & $5.77${\tiny$\pm0.06$} & $9.37${\tiny$\pm0.09$} & $6.29${\tiny$\pm0.13$}   & $30.09${\tiny$\pm0.12$}  &   $7.72$ \\
%           CoDi  & $21.38${\tiny$\pm0.06$}  & $15.77${\tiny$\pm$ $0.07$}  & $31.84${\tiny$\pm0.05$}  & $11.56${\tiny$\pm0.26$}  & $16.94${\tiny$\pm0.02$} &    $21.63$  \\
%           TabDDPM & $1.75${\tiny$\pm0.03$}  & $1.57${\tiny$\pm$ $0.08$}  & $2.72${\tiny$\pm0.13$}  & $1.01${\tiny$\pm0.09$}   & $19.98${\tiny$\pm0.05$}  &  $14.52$ \\
%           TabSyn   & $0.58${\tiny$\pm0.06$} & $0.85${\tiny$\pm0.04$} & $1.43${\tiny$\pm0.24$} & $0.88${\tiny$\pm0.09$}  & $1.12${\tiny$\pm0.05$}  &  $1.08$ \\
%           \midrule
%           \modelname  & $13.19${\tiny$\pm0.16$} & $10.99${\tiny$\pm0.33$} & $5.10${\tiny$\pm0.24$} & $6.58${\tiny$\pm0.55$}  & $4.49${\tiny$\pm0.21$}  &  $8.07$ \\
%           % Improv. &  $\redbf{66.9 \% \downarrow} $ & $ \redbf{45.9\% \downarrow} $ & $\redbf{47.4\% \downarrow}$ & $\redbf{12.9\% \downarrow} $  & $\redbf{13.8\% \downarrow }$  &  $\redbf{76.2\% \downarrow }$ &  $\redbf{86.0\% \downarrow }$ &  \\
%   \bottomrule[1.0pt] 
%   \end{tabular}
%             }
%             }
%     % \begin{tablenotes}
%     %       \item[1] GOGGLE fixes the random seed during sampling in the official codes, and we follow it for consistency.
%     %       \item[2] GReaT cannot be applied on News because of the maximum length limit.
%     % \item[3] TabDDPM fails to generate meaningful content on the News dataset.
%     %   \end{tablenotes}
% \end{threeparttable}

% \end{table}

% \begin{table}[h] 
%   \centering
%   \caption{Error rate (\%) of \textbf{pair-wise column} correlation score. } 
%   \label{tbl:exp-trend}
%   \small
% % \begin{threeparttable}
%   {
%   \scalebox{0.93}
%   {
% \begin{tabular}{lccccc|ccc}
%           \toprule[0.8pt]
%           \textbf{Method} & \textbf{Adult} & \textbf{Default} & \textbf{Shoppers} & \textbf{Magic}  & \textbf{California} & \textbf{Average}  \\
%           \midrule 
%            SMOTE & $3.28${\tiny$\pm0.29$} & $8.41${\tiny$\pm0.38$} & $3.56${\tiny$\pm0.22$} & $3.16${\tiny$\pm0.41$}  &$2.39${\tiny$\pm0.35$} &    $4.36$ & \\
%           \midrule
%           CTGAN    & $20.23${\tiny$\pm1.20$} & $26.95${\tiny$\pm0.93$} & $13.08${\tiny$\pm0.16$} & $7.00${\tiny$\pm0.19$} &  $22.95${\tiny$\pm0.08$}  &   $15.93$     \\
%           TVAE     & $14.15${\tiny$\pm0.88$} & $19.50${\tiny$\pm$$0.95$} & $18.67${\tiny$\pm0.38$} &  $5.82${\tiny$\pm0.49$}  &  $18.01${\tiny$\pm0.08$}  & $13.72$   \\
%           GReaT    & $17.59${\tiny$\pm0.22$} & $70.02${\tiny$\pm$$0.12$} & $45.16${\tiny$\pm0.18$} & $10.23${\tiny$\pm0.40$}  & $59.60${\tiny$\pm0.55$}    & $44.24$  \\
%           STaSy    & $14.51${\tiny$\pm0.25$} & $5.96${\tiny$\pm$$0.26$}  & $8.49${\tiny$\pm0.15$} & $6.61${\tiny$\pm0.53$}  & $8.00${\tiny$\pm0.10$} &  $7.77$     \\
%           CoDi  & $22.49${\tiny$\pm0.08$}  & $68.41${\tiny$\pm$$0.05$}  & $17.78${\tiny$\pm0.11$}  & $6.53${\tiny$\pm0.25$} & $7.07${\tiny$\pm0.15$}  & $22.23$   \\ 
%           TabDDPM & $3.01${\tiny$\pm0.25$}  & $4.89${\tiny$\pm0.10$}  & $6.61${\tiny$\pm0.16$} & $1.70${\tiny$\pm0.22$} & $2.71${\tiny$\pm0.09$} &  $5.34$ \\
%           TabSyn  & $1.54${\tiny$\pm0.27$} & $2.05${\tiny$\pm0.12$} & $2.07${\tiny$\pm0.21$}  & $1.06${\tiny$\pm0.31$}   &  $2.24${\tiny$\pm0.28$}  & $1.73$  \\
%           \midrule
%           \modelname  & $25.70${\tiny$\pm0.27$} & $22.25${\tiny$\pm0.12$} & $20.04${\tiny$\pm0.21$}  & $5.66${\tiny$\pm0.32$}   &  $10.65${\tiny$\pm0.28$}  &  $16.86$  \\
%           % Improve. &  $\redbf{48.8\% \downarrow} $ & $\redbf{58.1\% \downarrow} $ & $\redbf{68.7\% \downarrow}$  & $\redbf{37.6\% \downarrow} $  & $\redbf{17.3\% \downarrow} $  &  $\redbf{53.1\% \downarrow}$ & $\redbf{67.6\% \downarrow}$&  \\
%   \bottomrule[1.0pt] 
%   \end{tabular}
% }
% }
% % \end{threeparttable}
% \end{table}


% % \begin{table}[h] 
% %   \centering
% %   \caption{Wesserstain distance between the empirical distribution of real data and the synthetic data generated by different methods. Lower values indicate better generation performance (superior results). } 
% %   \label{tbl:exp-wasserstein}
% %   \small
% %   \begin{threeparttable}
% %   {
% %   \scalebox{0.92}
% %   {
% % \begin{tabular}{lccccc|cccc}
% %           \toprule[0.8pt]
% %           \textbf{Method} & \textbf{Adult} & \textbf{Default} & \textbf{Shoppers} & \textbf{Magic}  & \textbf{California}&  \textbf{Average}  \\
% %           \midrule 
% %            SMOTE & $0.162$ & $0.096$ & $0.210$ & $0.002$  &$0.001$ &    $0.09$ & \\
% %           \midrule
% %           CTGAN    & $0.832$ & $1.362$ & $0.891$ & $0.014$  &$0.22$ & $0.664$  \\
% %           TVAE     & $1.103$ & $0.478$ & $1.116$ & $0.006$  &  $0.007$  & $0.542$   \\
% %           % GReaT     & $12.12${\tiny$\pm$$0.04$} & $19.94${\tiny$\pm$$0.06$}  & $14.51${\tiny$\pm0.12$}  &  $16.16${\tiny$\pm0.09$}   & $10.25${\tiny$\pm0.20$}  & $14.20$ \\
% %           STaSy    & $0.630$ & $0.417$ & $0.870$ & $0.057$   & $0.701$  &   $0.535$ \\
% %           CoDi  & $1.500$  & $1.201$  & $1.293$  & $0.057$  & $0.049$ &    $0.820$  \\
% %           % TabDDPM & $1.75${\tiny$\pm0.03$}  & $1.57${\tiny$\pm$ $0.08$}  & $2.72${\tiny$\pm0.13$}  & $1.01${\tiny$\pm0.09$}   & $19.98${\tiny$\pm0.05$}  &  $14.52$ \\
% %           TabSyn   & $0.308$ & $0.201$ & $0.4370$ & $0.0032$  & $0.002$  &  $0.190$ \\
% %           \midrule
% %           \modelname  & $3.465$ & $3.478$ & $2.62$ & $0.915$  & $34.71$  &  $8.07$ \\
% %   \bottomrule[1.0pt] 
% %   \end{tabular}
% %             }
% %             }
% % \end{threeparttable}

% % \end{table}


% \begin{table}[h] 
%   \centering
%   \caption{AUC scores of \textbf{Machine Learning Efficiency}. $\uparrow$ indicates that the higher the score, the better the performance.} 
%   \label{tbl:exp-mle}
%   \small
%   \begin{threeparttable}
%   {
%   \scalebox{0.92}
%   {
% \begin{tabular}{lccccc|ccc}
%           \toprule[0.8pt]
%           \multirow{2}{*}{Methods} & {\textbf{Adult}} &{\textbf{Default}} & \textbf{Shoppers} & {\textbf{Magic}} &   {\textbf{California}}$^{1}$ & {\textbf{Average Gap}} \\
%           \cmidrule{2-8} 
%           & AUC $\uparrow$ & AUC $\uparrow$ &  AUC $\uparrow$ &  AUC $\uparrow$  & AUC $\uparrow$ &  $\%$ \\
%           \midrule 
%           Real & $.927${\tiny$\pm.000$} & $.770${\tiny$\pm.005$} & $.926${\tiny$\pm.001$}  & $.946${\tiny$\pm.001$}  & -  & $0\%$  \\
%           \midrule
%           SMOTE & $.899${\tiny$\pm.007$} &$.741${\tiny$\pm.009$} & $.911${\tiny$\pm.012$} & $.934${\tiny$\pm.008$} & -  &  $9.39\%$  \\
%           \midrule
%           CTGAN & $.886${\tiny$\pm.002$} &$.696${\tiny$\pm.005$} & $.875${\tiny$\pm.009$} & $.855${\tiny$\pm.006$}    & - & $8.4\%$ \\
%           TVAE  & $.878${\tiny$\pm.004$} &$.724 ${\tiny$\pm.005$} & $.871${\tiny$\pm.006$} & $.887${\tiny$\pm.003$} & - & $6.9\%$  \\
%           GReaT & $.913${\tiny$\pm.003$} &$.755${\tiny$\pm.006$} & $.902${\tiny$\pm.005$} & $.888${\tiny$\pm.008$}  & - &  $3.3\%$  \\
%           STaSy  & $.906${\tiny$\pm.001$} & $.752${\tiny$\pm.006$} & $.914${\tiny$\pm.005$} & $.934${\tiny$\pm.003$}  & - & $1.9\%$  \\ 
%           CoDi & $.871${\tiny$\pm.006$} & $.525${\tiny$\pm.006$} & $.865${\tiny$\pm.006$} & $.932${\tiny$\pm.003$}   & - &  $10.7\%$  \\
%           TabDDPM & $.907${\tiny$\pm.001$}  & $.758${\tiny$\pm.004$}& $.918${\tiny$\pm.005$} & $.935${\tiny$\pm.003$} & - & $1.65\%$ \\
%           TabSyn & $.915${\tiny$\pm.002$} & $.764${\tiny$\pm.004$} & $.920${\tiny$\pm.005$} & $.938${\tiny$\pm.002$} & - & $1.1\%$ \\
%           \midrule
%           \modelname  & $.894${\tiny$\pm.009$} & $.634${\tiny$\pm.112$} & $.792${\tiny$\pm.005$} & $.896${\tiny$\pm.006$} & - & $10.1\%$ &  
%           \\
%   \bottomrule[1.0pt] 
%   \end{tabular}
% }
% }
%    \begin{tablenotes}
%    \item[1] AUC score for California dataset is not available because the dataset contains extremely rare classes.
%       \end{tablenotes}
% \end{threeparttable}
% \end{table}



% \begin{table}[h]
%   \centering
%   \caption{The absolute difference between Normalized Distance to Closest Records (DCR) and 1. The smaller the better. } 
%   \label{tbl:exp-dcr}
%   \small
%   \begin{threeparttable}
%   {
%   \scalebox{0.92}
%   {
% \begin{tabular}{lccccc|cccc}
%           \toprule[0.8pt]
%           \textbf{Method} & \textbf{Adult} & \textbf{Default} & \textbf{Shoppers} & \textbf{Magic}  & \textbf{California}&  \textbf{Average}  \\
%           \midrule 
%            SMOTE & $0.411$ & $0.093$ & $0.210$ & $0.102$  &$0.103$ &    $0.184$ & \\
%           \midrule
%           CTGAN    & $0.020$ & $0.004$ & $0.017$ & $0.014$  &$0.008$ & $0.013$  \\
%           TVAE     & $0.007$ & $0.002$ & $0.024$ & $0.000$  &  $0.000$  & $0.007$   \\
%           % GReaT     & $12.12${\tiny$\pm$$0.04$} & $19.94${\tiny$\pm$$0.06$}  & $14.51${\tiny$\pm0.12$}  &  $16.16${\tiny$\pm0.09$}   & $10.25${\tiny$\pm0.20$}  & $14.20$ \\
%           STaSy    & $0.009$ & $0.005$ & $0.004$ & $0.010$   & $0.020$  &   $0.010$ \\
%           CoDi  & $0.007$  & $0.011$  & $0.005$  & $0.010$  & $0.005$ &    $0.008$  \\
%           % TabDDPM & $1.75${\tiny$\pm0.03$}  & $1.57${\tiny$\pm$ $0.08$}  & $2.72${\tiny$\pm0.13$}  & $1.01${\tiny$\pm0.09$}   & $19.98${\tiny$\pm0.05$}  &  $14.52$ \\
%           TabSyn   & $0.012$ & $0.001$ & $0.003$ & $0.002$  & $0.004$  &  $0.004$ \\
%           \midrule
%           \modelname  & $0.001$ & $0.004$ & $0.005$ & $0.042$  & $0.011$  &  $0.013$ \\
%   \bottomrule[1.0pt] 
%   \end{tabular}
%             }
%             }
% \end{threeparttable}

% \end{table}

\end{document}
