% This must be in the first 5 lines to tell arXiv to use pdfLaTeX, which is strongly recommended.
\pdfoutput=1
% In particular, the hyperref package requires pdfLaTeX in order to break URLs across lines.

\documentclass[11pt]{article}

% Change "review" to "final" to generate the final (sometimes called camera-ready) version.
% Change to "preprint" to generate a non-anonymous version with page numbers.
\usepackage[preprint]{acl}

% Standard package includes
\usepackage{times}
\usepackage{latexsym}

% For proper rendering and hyphenation of words containing Latin characters (including in bib files)
\usepackage[T1]{fontenc}
% For Vietnamese characters
% \usepackage[T5]{fontenc}
% See https://www.latex-project.org/help/documentation/encguide.pdfac for other character sets

% This assumes your files are encoded as UTF8
\usepackage[utf8]{inputenc}

% This is not strictly necessary, and may be commented out,
% but it will improve the layout of the manuscript,
% and will typically save some space.
\usepackage{microtype}

% This is also not strictly necessary, and may be commented out.
% However, it will improve the aesthetics of text in
% the typewriter font.
\usepackage{inconsolata}

%Including images in your LaTeX document requires adding
%additional package(s)
\usepackage{graphicx}

% If the title and author information does not fit in the area allocated, uncomment the following
%
%\setlength\titlebox{<dim>}
%
% and set <dim> to something 5cm or larger.
\usepackage{hyperref}
% \hypersetup{
%     colorlinks=true,
%     urlcolor=teal  
% }
\usepackage{url}            % simple URL typesetting
\usepackage{booktabs}       % professional-quality tables
\usepackage{amsfonts}       % blackboard math symbols
\usepackage{nicefrac}       % compact symbols for 1/2, etc.
\usepackage{microtype}      % microtypography
\usepackage{amsmath}
\usepackage{threeparttable}
\usepackage{graphicx}
\usepackage{multirow}
\usepackage{subfigure}
\usepackage{listings}
\usepackage{tikz}
\usepackage{colortbl}
\usepackage{xspace}
\usepackage{algorithm}
\usepackage{algpseudocode}
\usepackage{enumitem}
\usepackage{xcolor}         % colors
\usepackage{tabularx} % for better table width handling
\usepackage{booktabs} % for enhanced table styling

\usepackage{amsthm}
\usepackage[english]{babel}
\theoremstyle{definition}
\newtheorem{definition}{Definition}
\newtheorem{remark}{Remark}

\lstdefinestyle{mystyle}{
    backgroundcolor=\color{backcolour},   
    commentstyle=\color{codegreen},
    keywordstyle=\color{magenta},
    numberstyle=\tiny\color{codegray},
    stringstyle=\color{codepurple},
    basicstyle=\ttfamily\footnotesize,
    breakatwhitespace=false,         
    breaklines=true,                 
    captionpos=b,                    
    keepspaces=true,                 
    numbers=left,                    
    numbersep=5pt,                  
    showspaces=false,                
    showstringspaces=false,
    showtabs=false,                  
    tabsize=2
}

\lstset{style=mystyle}

% Define colors
\definecolor{codegreen}{rgb}{0,0.6,0}
\definecolor{codegray}{rgb}{0.5,0.5,0.5}
\definecolor{codepurple}{rgb}{0.58,0,0.82}
\definecolor{backcolour}{rgb}{0.95,0.95,0.92}
\definecolor{brickred}{rgb}{0.8, 0.25, 0.1}
\definecolor{midnightblue}{rgb}{0.1, 0.1, 0.44}
\definecolor{oceanboatblue}{rgb}{0.0, 0.47, 0.75}
\newcommand{\redbf}[1]{\bf{\textcolor{brickred}{#1}}}
\newcommand{\bluebf}[1]{\bf{\textcolor{oceanboatblue}{#1}}}
\newcommand{\grey}{\cellcolor{gray!20}}
\definecolor{lightgray1}{gray}{0.95}  

\newcommand{\circlednumber}[1]{%
    \tikz[baseline=(char.base)]{%
        \node[shape=circle,draw,inner sep=1pt] (char) {\scriptsize#1};%
    }%
}

\newcommand{\modelname}{{\sc TabGen-ICL}\xspace}
\title{\modelname: Residual-Aware In-Context Example Selection for Tabular Data Generation}

\author{
 \textbf{Liancheng Fang \textsuperscript{1}},
 \textbf{Aiwei Liu \textsuperscript{2}},
 \textbf{Hengrui Zhang \textsuperscript{1}},
 \textbf{Henry Peng Zou \textsuperscript{1}},
\\
 \textbf{Weizhi Zhang\textsuperscript{1}},
 \textbf{Philip S. Yu\textsuperscript{1}},
\\
 \textsuperscript{1}University of Illinois Chicago,
 \textsuperscript{2}Tsinghua University,
\\
\href{mailto:zhonghaoli@hkust-gz.edu.cn}{lfang87@uic.edu},
\href{mailto:xuminghu@hkust-gz.edu.cn}{liuaw20@mails.tsinghua.edu.cn},
\href{mailto:psyu@uic.edu}{psyu@uic.edu}
}

\begin{document}
\maketitle
\begin{abstract}
    Large Language models (LLMs) have achieved encouraging results in tabular data generation. However, existing approaches require fine-tuning, which is computationally expensive. This paper explores an alternative: prompting a fixed LLM with in-context examples. We observe that using randomly selected in-context examples hampers the LLM's performance, resulting in sub-optimal generation quality.
    To address this, we propose a novel in-context learning framework: \modelname, to enhance the in-context learning ability of LLMs for tabular data generation. 
    \modelname operates iteratively, retrieving a subset of real samples that represent the \textit{residual} between currently generated samples and true data distributions. This approach serves two purposes: locally, it provides more effective in-context learning examples for the LLM in each iteration; globally, it progressively narrows the gap between generated and real data.
    Extensive experiments on five real-world tabular datasets demonstrate that \modelname significantly outperforms the random selection strategy. Specifically, it reduces the error rate by a margin of $3.5\%-42.2\%$ on fidelity metrics. We demonstrate for the first time that prompting a fixed LLM can yield high-quality synthetic tabular data. 
    The code is provided in the \href{https://github.com/fangliancheng/TabGEN-ICL}{link}.
\end{abstract}

\section{Introduction}

Video generation has garnered significant attention owing to its transformative potential across a wide range of applications, such media content creation~\citep{polyak2024movie}, advertising~\citep{zhang2024virbo,bacher2021advert}, video games~\citep{yang2024playable,valevski2024diffusion, oasis2024}, and world model simulators~\citep{ha2018world, videoworldsimulators2024, agarwal2025cosmos}. Benefiting from advanced generative algorithms~\citep{goodfellow2014generative, ho2020denoising, liu2023flow, lipman2023flow}, scalable model architectures~\citep{vaswani2017attention, peebles2023scalable}, vast amounts of internet-sourced data~\citep{chen2024panda, nan2024openvid, ju2024miradata}, and ongoing expansion of computing capabilities~\citep{nvidia2022h100, nvidia2023dgxgh200, nvidia2024h200nvl}, remarkable advancements have been achieved in the field of video generation~\citep{ho2022video, ho2022imagen, singer2023makeavideo, blattmann2023align, videoworldsimulators2024, kuaishou2024klingai, yang2024cogvideox, jin2024pyramidal, polyak2024movie, kong2024hunyuanvideo, ji2024prompt}.


In this work, we present \textbf{\ours}, a family of rectified flow~\citep{lipman2023flow, liu2023flow} transformer models designed for joint image and video generation, establishing a pathway toward industry-grade performance. This report centers on four key components: data curation, model architecture design, flow formulation, and training infrastructure optimization—each rigorously refined to meet the demands of high-quality, large-scale video generation.


\begin{figure}[ht]
    \centering
    \begin{subfigure}[b]{0.82\linewidth}
        \centering
        \includegraphics[width=\linewidth]{figures/t2i_1024.pdf}
        \caption{Text-to-Image Samples}\label{fig:main-demo-t2i}
    \end{subfigure}
    \vfill
    \begin{subfigure}[b]{0.82\linewidth}
        \centering
        \includegraphics[width=\linewidth]{figures/t2v_samples.pdf}
        \caption{Text-to-Video Samples}\label{fig:main-demo-t2v}
    \end{subfigure}
\caption{\textbf{Generated samples from \ours.} Key components are highlighted in \textcolor{red}{\textbf{RED}}.}\label{fig:main-demo}
\end{figure}


First, we present a comprehensive data processing pipeline designed to construct large-scale, high-quality image and video-text datasets. The pipeline integrates multiple advanced techniques, including video and image filtering based on aesthetic scores, OCR-driven content analysis, and subjective evaluations, to ensure exceptional visual and contextual quality. Furthermore, we employ multimodal large language models~(MLLMs)~\citep{yuan2025tarsier2} to generate dense and contextually aligned captions, which are subsequently refined using an additional large language model~(LLM)~\citep{yang2024qwen2} to enhance their accuracy, fluency, and descriptive richness. As a result, we have curated a robust training dataset comprising approximately 36M video-text pairs and 160M image-text pairs, which are proven sufficient for training industry-level generative models.

Secondly, we take a pioneering step by applying rectified flow formulation~\citep{lipman2023flow} for joint image and video generation, implemented through the \ours model family, which comprises Transformer architectures with 2B and 8B parameters. At its core, the \ours framework employs a 3D joint image-video variational autoencoder (VAE) to compress image and video inputs into a shared latent space, facilitating unified representation. This shared latent space is coupled with a full-attention~\citep{vaswani2017attention} mechanism, enabling seamless joint training of image and video. This architecture delivers high-quality, coherent outputs across both images and videos, establishing a unified framework for visual generation tasks.


Furthermore, to support the training of \ours at scale, we have developed a robust infrastructure tailored for large-scale model training. Our approach incorporates advanced parallelism strategies~\citep{jacobs2023deepspeed, pytorch_fsdp} to manage memory efficiently during long-context training. Additionally, we employ ByteCheckpoint~\citep{wan2024bytecheckpoint} for high-performance checkpointing and integrate fault-tolerant mechanisms from MegaScale~\citep{jiang2024megascale} to ensure stability and scalability across large GPU clusters. These optimizations enable \ours to handle the computational and data challenges of generative modeling with exceptional efficiency and reliability.


We evaluate \ours on both text-to-image and text-to-video benchmarks to highlight its competitive advantages. For text-to-image generation, \ours-T2I demonstrates strong performance across multiple benchmarks, including T2I-CompBench~\citep{huang2023t2i-compbench}, GenEval~\citep{ghosh2024geneval}, and DPG-Bench~\citep{hu2024ella_dbgbench}, excelling in both visual quality and text-image alignment. In text-to-video benchmarks, \ours-T2V achieves state-of-the-art performance on the UCF-101~\citep{ucf101} zero-shot generation task. Additionally, \ours-T2V attains an impressive score of \textbf{84.85} on VBench~\citep{huang2024vbench}, securing the top position on the leaderboard (as of 2025-01-25) and surpassing several leading commercial text-to-video models. Qualitative results, illustrated in \Cref{fig:main-demo}, further demonstrate the superior quality of the generated media samples. These findings underscore \ours's effectiveness in multi-modal generation and its potential as a high-performing solution for both research and commercial applications.
\begin{figure*}[t]
  \centering
    \includegraphics[width=1\linewidth]{visuals/final_registration.png}
    \caption{Target measurement process on low-cost scan data using ICP and Coloured ICP. (1) Initialisation: The source point cloud (checkerboard) is misaligned with the target point cloud. (2) Initial Registration using Point-to-Plane ICP: Standard ICP leads to suboptimal registration. (3) Final Registration using Coloured ICP: Colour information is incorporated after pre-processing with RANSAC and Binarisation with Otsu Thresholding for real data, resulting in improved alignment.}
    \label{fig:Registration_visualisation}
\end{figure*}

\subsection{Iterative Closest Point (ICP) Algorithm}
The Iterative Closest Point (ICP) algorithm has been a fundamental technique in 3D computer vision and robotics for point cloud. Originally proposed by \cite{besl_method_1992}, ICP aims to minimise the distance between two datasets, typically referred to as the source and the target. The algorithm operates in an iterative manner, identifying correspondences by matching each source point with its nearest target point \citep{survey_ICP}. It then computes the rigid transformation, usually involving both rotation and translation, to achieve the best alignment of these matched points \citep{survey_ICP}. This process is repeated until convergence, where the change in the alignment parameters or the overall alignment error becomes smaller than a predefined threshold.

One key advantage of the ICP framework lies in its simplicity: the algorithm is conceptually straightforward, and its basic version is relatively easy to implement. However, traditional ICP can be sensitive to local minima, often requiring a good initial alignment \citep{zhang2021fast}. Furthermore, outliers, noise, and partial overlaps between datasets can significantly degrade its performance \citep{zhang2021fast, bouaziz2013sparse}. Over the years, various modifications and improvements \citep{gelfand2005robust, rusu2009fast, aiger20084, gruen2005least, fitzgibbon2003robust} have been proposed to mitigate these issues. Among the most common strategies are robust cost functions \citep{fitzgibbon2003robust}, weighting schemes for correspondences \citep{rusu2009fast}, and more sophisticated techniques \citep{gelfand2005robust, bouaziz2013sparse} to reject outliers. 

In addition, there is significant interest in integrating additional information into the ICP pipeline. Instead of solely relying on geometric cues such as point coordinates or surface normals, recent approaches have proposed incorporating colour (RGB) or intensity data to enhance correspondence accuracy. These methods \citep{park_colored_2017, 5980407}, commonly known as "Colored ICP" employ differences in pixel intensities or colour values as additional constraints. This is particularly beneficial in situations where geometric attributes alone are inadequate for accurate alignment or where surfaces possess complex texture patterns that can assist in the matching process.

\subsection{Applications of Target Measurement}

One approach relies on the use of physical checkerboard targets for registration. \cite{fryskowska2019} analyse checkerboard target identification for terrestrial laser scanning. They propose a geometric method to determine the target centre with higher precision, demonstrating that their approach can reduce errors by up to 6 mm compared to conventional automatic methods.

\cite{becerik2011assessment} examines data acquisition errors in 3D laser scanning for construction by evaluating how different target types (paper, paddle, and sphere) and layouts impact registration accuracy in both indoor and outdoor environments and presents guidelines for optimal target configuration.

\citet{Liang2024} propose to use Coloured ICP to measure target centres for checkerboard targets, similar to our investigation. They use data from a survey-grade terrestrial laser scanner. Their intended application is structural bridge monitoring purposes. They report an average accuracy of the measurement below 1.3 millimetres.

Where targets cannot be placed in the scene, the intensity information form the scanner can still be used to identify distinctive points. For point cloud data that is captured with a regular pattern, standard image processing can be used in a similar way to target detection. For example, \citet{wendt_automation_2004} proposes to use the SUSAN operator on a co-registered image from a camera, \citet{bohm_automatic_2007} proposes to use the SIFT operator on the LIDAR reflectance directly and \citet{theiler_markerless_2013} propose to use a Difference-of-Gaussian approach on the reflectance information.
Most of these methods extract image features to find reliable 3D correspondences for the purpose of registration.

In the following we describe our approach to the measurement of the target centre. In contrast to most proposed methods above we focus on unordered point clouds, where raster-based methods are not available, and low-cost sensors, where increased measurement noise and outliers are expected. As we are not aware of a commercial reference solution to this problem, we start by conducting a series of synthetic experiments to explore the viability and accuracy potential of the approach.



%The reviewed studies primarily rely on physical targets or target-free methods and do not utilise 3D synthetic point cloud checkerboards. In contrast, our approach introduces synthetic point cloud checkerboards, which offer controlled and consistent target geometry and reduce variability caused by physical targets. This innovation has significant potential for commercialisation and industrial application.

\section{Study Design}
% robot: aliengo 
% We used the Unitree AlienGo quadruped robot. 
% See Appendix 1 in AlienGo Software Guide PDF
% Weight = 25kg, size (L,W,H) = (0.55, 0.35, 06) m when standing, (0.55, 0.35, 0.31) m when walking
% Handle is 0.4 m or 0.5 m. I'll need to check it to see which type it is.
We gathered input from primary stakeholders of the robot dog guide, divided into three subgroups: BVI individuals who have owned a dog guide, BVI individuals who were not dog guide owners, and sighted individuals with generally low degrees of familiarity with dog guides. While the main focus of this study was on the BVI participants, we elected to include survey responses from sighted participants given the importance of social acceptance of the robot by the general public, which could reflect upon the BVI users themselves and affect their interactions with the general population \cite{kayukawa2022perceive}. 

The need-finding processes consisted of two stages. During Stage 1, we conducted in-depth interviews with BVI participants, querying their experiences in using conventional assistive technologies and dog guides. During Stage 2, a large-scale survey was distributed to both BVI and sighted participants. 

This study was approved by the University’s Institutional Review Board (IRB), and all processes were conducted after obtaining the participants' consent.

\subsection{Stage 1: Interviews}
We recruited nine BVI participants (\textbf{Table}~\ref{tab:bvi-info}) for in-depth interviews, which lasted 45-90 minutes for current or former dog guide owners (DO) and 30-60 minutes for participants without dog guides (NDO). Group DO consisted of five participants, while Group NDO consisted of four participants.
% The interview participants were divided into two groups. Group DO (Dog guide Owner) consisted of five participants who were current or former dog guide owners and Group NDO (Non Dog guide Owner) consisted of three participants who were not dog guide owners. 
All participants were familiar with using white canes as a mobility aid. 

We recruited participants in both groups, DO and NDO, to gather data from those with substantial experience with dog guides, offering potentially more practical insights, and from those without prior experience, providing a perspective that may be less constrained and more open to novel approaches. 

We asked about the participants' overall impressions of a robot dog guide, expectations regarding its potential benefits and challenges compared to a conventional dog guide, their desired methods of giving commands and communicating with the robot dog guide, essential functionalities that the robot dog guide should offer, and their preferences for various aspects of the robot dog guide's form factors. 
For Group DO, we also included questions that asked about the participants' experiences with conventional dog guides. 

% We obtained permission to record the conversations for our records while simultaneously taking notes during the interviews. The interviews lasted 30-60 minutes for NDO participants and 45-90 minutes for DO participants. 

\subsection{Stage 2: Large-Scale Surveys} 
After gathering sufficient initial results from the interviews, we created an online survey for distributing to a larger pool of participants. The survey platform used was Qualtrics. 

\subsubsection{Survey Participants}
The survey had 100 participants divided into two primary groups. Group BVI consisted of 42 blind or visually impaired participants, and Group ST consisted of 58 sighted participants. \textbf{Table}~\ref{tab:survey-demographics} shows the demographic information of the survey participants. 

\subsubsection{Question Differentiation} 
Based on their responses to initial qualifying questions, survey participants were sorted into three subgroups: DO, NDO, and ST. Each participant was assigned one of three different versions of the survey. The surveys for BVI participants mirrored the interview categories (overall impressions, communication methods, functionalities, and form factors), but with a more quantitative approach rather than the open-ended questions used in interviews. The DO version included additional questions pertaining to their prior experience with dog guides. The ST version revolved around the participants' prior interactions with and feelings toward dog guides and dogs in general, their thoughts on a robot dog guide, and broad opinions on the aesthetic component of the robot's design. 


\section{Dataset}
\label{sec:dataset}

\subsection{Data Collection}

To analyze political discussions on Discord, we followed the methodology in \cite{singh2024Cross-Platform}, collecting messages from politically-oriented public servers in compliance with Discord's platform policies.

Using Discord's Discovery feature, we employed a web scraper to extract server invitation links, names, and descriptions, focusing on public servers accessible without participation. Invitation links were used to access data via the Discord API. To ensure relevance, we filtered servers using keywords related to the 2024 U.S. elections (e.g., Trump, Kamala, MAGA), as outlined in \cite{balasubramanian2024publicdatasettrackingsocial}. This resulted in 302 server links, further narrowed to 81 English-speaking, politics-focused servers based on their names and descriptions.

Public messages were retrieved from these servers using the Discord API, collecting metadata such as \textit{content}, \textit{user ID}, \textit{username}, \textit{timestamp}, \textit{bot flag}, \textit{mentions}, and \textit{interactions}. Through this process, we gathered \textbf{33,373,229 messages} from \textbf{82,109 users} across \textbf{81 servers}, including \textbf{1,912,750 messages} from \textbf{633 bots}. Data collection occurred between November 13th and 15th, covering messages sent from January 1st to November 12th, just after the 2024 U.S. election.

\subsection{Characterizing the Political Spectrum}
\label{sec:timeline}

A key aspect of our research is distinguishing between Republican- and Democratic-aligned Discord servers. To categorize their political alignment, we relied on server names and self-descriptions, which often include rules, community guidelines, and references to key ideologies or figures. Each server's name and description were manually reviewed based on predefined, objective criteria, focusing on explicit political themes or mentions of prominent figures. This process allowed us to classify servers into three categories, ensuring a systematic and unbiased alignment determination.

\begin{itemize}
    \item \textbf{Republican-aligned}: Servers referencing Republican and right-wing and ideologies, movements, or figures (e.g., MAGA, Conservative, Traditional, Trump).  
    \item \textbf{Democratic-aligned}: Servers mentioning Democratic and left-wing ideologies, movements, or figures (e.g., Progressive, Liberal, Socialist, Biden, Kamala).  
    \item \textbf{Unaligned}: Servers with no defined spectrum and ideologies or opened to general political debate from all orientations.
\end{itemize}

To ensure the reliability and consistency of our classification, three independent reviewers assessed the classification following the specified set of criteria. The inter-rater agreement of their classifications was evaluated using Fleiss' Kappa \cite{fleiss1971measuring}, with a resulting Kappa value of \( 0.8191 \), indicating an almost perfect agreement among the reviewers. Disagreements were resolved by adopting the majority classification, as there were no instances where a server received different classifications from all three reviewers. This process guaranteed the consistency and accuracy of the final categorization.

Through this process, we identified \textbf{7 Republican-aligned servers}, \textbf{9 Democratic-aligned servers}, and \textbf{65 unaligned servers}.

Table \ref{tab:statistics} shows the statistics of the collected data. Notably, while Democratic- and Republican-aligned servers had a comparable number of user messages, users in the latter servers were significantly more active, posting more than double the number of messages per user compared to their Democratic counterparts. 
This suggests that, in our sample, Democratic-aligned servers attract more users, but these users were less engaged in text-based discussions. Additionally, around 10\% of the messages across all server categories were posted by bots. 

\subsection{Temporal Data} 

Throughout this paper, we refer to the election candidates using the names adopted by their respective campaigns: \textit{Kamala}, \textit{Biden}, and \textit{Trump}. To examine how the content of text messages evolves based on the political alignment of servers, we divided the 2024 election year into three periods: \textbf{Biden vs Trump} (January 1 to July 21), \textbf{Kamala vs Trump} (July 21 to September 20), and the \textbf{Voting Period} (after September 20). These periods reflect key phases of the election: the early campaign dominated by Biden and Trump, the shift in dynamics with Kamala Harris replacing Joe Biden as the Democratic candidate, and the final voting stage focused on electoral outcomes and their implications. This segmentation enables an analysis of how discourse responds to pivotal electoral moments.

Figure \ref{fig:line-plot} illustrates the distribution of messages over time, highlighting trends in total messages volume and mentions of each candidate. Prior to Biden's withdrawal on July 21, mentions of Biden and Trump were relatively balanced. However, following Kamala's entry into the race, mentions of Trump surged significantly, a trend further amplified by an assassination attempt on him, solidifying his dominance in the discourse. The only instance where Trump’s mentions were exceeded occurred during the first debate, as concerns about Biden’s age and cognitive abilities temporarily shifted the focus. In the final stages of the election, mentions of all three candidates rose, with Trump’s mentions peaking as he emerged as the victor.
\section{Experimental Methodology}\label{sec:exp}
In this section, we introduce the datasets, evaluation metrics, baselines, and implementation details used in our experiments. More experimental details are shown in Appendix~\ref{app:experiment_detail}.

\textbf{Dataset.}
We utilize various datasets for training and evaluation. Data statistics are shown in Table~\ref{tab:dataset}.

\textit{Training.}
We use the publicly available E5 dataset~\cite{wang2024improving,springer2024repetition} to train both the LLM-QE and dense retrievers. We concentrate on English-based question answering tasks and collect a total of 808,740 queries. From this set, we randomly sample 100,000 queries to construct the DPO training data, while the remaining queries are used for contrastive training. During the DPO preference pair construction, we first prompt LLMs to generate expansion documents, filtering out queries where the expanded documents share low similarity with the query. This results in a final set of 30,000 queries.

\textit{Evaluation.}
We evaluate retrieval effectiveness using two retrieval benchmarks: MS MARCO \cite{bajaj2016ms} and BEIR \cite{thakur2021beir}, in both unsupervised and supervised settings.

\textbf{Evaluation Metrics.}
We use nDCG@10 as the evaluation metric. Statistical significance is tested using a permutation test with $p<0.05$.

\textbf{Baselines.} We compare our LLM-QE model with three unsupervised retrieval models and five query expansion baseline models.
% —

Three unsupervised retrieval models—BM25~\cite{robertson2009probabilistic}, CoCondenser~\cite{gao2022unsupervised}, and Contriever~\cite{izacard2021unsupervised}—are evaluated in the experiments. Among these, Contriever serves as our primary baseline retrieval model, as it is used as the backbone model to assess the query expansion performance of LLM-QE. Additionally, we compare LLM-QE with Contriever in a supervised setting using the same training dataset.

For query expansion, we benchmark against five methods: Pseudo-Relevance Feedback (PRF), Q2Q, Q2E, Q2C, and Q2D. PRF is specifically implemented following the approach in~\citet{yu2021improving}, which enhances query understanding by extracting keywords from query-related documents. The Q2Q, Q2E, Q2C, and Q2D methods~\cite{jagerman2023query,li2024can} expand the original query by prompting LLMs to generate query-related queries, keywords, chains-of-thought~\cite{wei2022chain}, and documents.


\textbf{Implementation Details.} 
For our query expansion model, we deploy the Meta-LLaMA-3-8B-Instruct~\cite{llama3modelcard} as the backbone for the query expansion generator. The batch size is set to 16, and the learning rate is set to $2e-5$. Optimization is performed using the AdamW optimizer. We employ LoRA~\cite{hu2022lora} to efficiently fine-tune the model for 2 epochs. The temperature for the construction of the DPO data varies across $\tau \in \{0.8, 0.9, 1.0, 1.1\}$, with each setting sampled eight times. For the dense retriever, we utilize Contriever~\cite{izacard2021unsupervised} as the backbone. During training, we set the batch size to 1,024 and the learning rate to $3e-5$, with the model trained for 3 epochs.

\section{Conclusion}
We introduce a novel approach, \algo, to reduce human feedback requirements in preference-based reinforcement learning by leveraging vision-language models. While VLMs encode rich world knowledge, their direct application as reward models is hindered by alignment issues and noisy predictions. To address this, we develop a synergistic framework where limited human feedback is used to adapt VLMs, improving their reliability in preference labeling. Further, we incorporate a selective sampling strategy to mitigate noise and prioritize informative human annotations.

Our experiments demonstrate that this method significantly improves feedback efficiency, achieving comparable or superior task performance with up to 50\% fewer human annotations. Moreover, we show that an adapted VLM can generalize across similar tasks, further reducing the need for new human feedback by 75\%. These results highlight the potential of integrating VLMs into preference-based RL, offering a scalable solution to reducing human supervision while maintaining high task success rates. 

\section*{Impact Statement}
This work advances embodied AI by significantly reducing the human feedback required for training agents. This reduction is particularly valuable in robotic applications where obtaining human demonstrations and feedback is challenging or impractical, such as assistive robotic arms for individuals with mobility impairments. By minimizing the feedback requirements, our approach enables users to more efficiently customize and teach new skills to robotic agents based on their specific needs and preferences. The broader impact of this work extends to healthcare, assistive technology, and human-robot interaction. One possible risk is that the bias from human feedback can propagate to the VLM and subsequently to the policy. This can be mitigated by personalization of agents in case of household application or standardization of feedback for industrial applications. 

\clearpage
\newpage
\section{Limitations}

\paragraph{Reliance on a Stronger LLM. }
Our framework relies on a stronger LLM to synthesize data. While this enables the synthesis of high quality data, removing this dependency could help lead to a more robust and independent framework, possibly at the cost of performance degradation. Additionally, LLM-generated data may amplify existing biases or include inappropriate content.

\paragraph{Seed Data Quality. }
The quality of our synthesized data is tied to that of our seed data. We select concise, high-quality datasets from prior works to use as the seed data. A more comprehensive exploration of seed data selection and its impact on synthetic data remains an important direction for future work.

Furthermore, our work does not fully address the scalability our framework. There likely exists a limit to how much data we can synthesize from our seed data, until the synthesized data becomes repetitive and lacks diversity.

\paragraph{LLM-Based Evaluation. }
Our evaluation relies on benchmarks that use LLMs as a judge. Although they correlate highly with human judgments, it is important to acknowledge that they may still have limitations, such as biases towards longer responses or their own outputs.


\section{Acknowledgments}
This work has benefited from the Microsoft Accelerate Foundation Models Research (AFMR) grant program, through which leading foundation models hosted by Microsoft Azure and access to Azure credits were provided to conduct the research.


\clearpage 
\newpage 

\bibliography{references}

\newpage 
\appendix
\clearpage
\section{Appendix}
\label{sec:appendix}

% ===
\subsection{Results Compressor/Limiter}
\setlength{\tabcolsep}{3pt}
\renewcommand{\arraystretch}{1.3}
\begin{table*}[h]
    \small
    \caption{
    \textit{Scaled test loss for non parametric models of compressor/limiter effects. Bold indicates best performing models.}
    }
    \label{tab:testloss_comp}
    \centerline{
        \begin{tabular}{l c >{\columncolor{gray!20}}ccc >{\columncolor{gray!20}}ccc >{\columncolor{gray!20}}ccc >{\columncolor{gray!20}}ccc}
            \midrule
            \midrule
            \multirow{2}{*}{Model} 
                & \multirow{2}{*}{Params.} 
                    & \multicolumn{3}{c}{Ampeg Optocomp} 
                        &  \multicolumn{3}{c}{Flamma AnalogComp} 
                            & \multicolumn{3}{c}{Yuer DynaCompressor} 
                                & \multicolumn{3}{c}{UA 1176LN} \\
            \cmidrule{3-5} 
                \cmidrule(lr){6-8} 
                    \cmidrule(lr){9-11} 
                        \cmidrule(lr){12-14}
                        
            & & 
            Tot. & {\footnotesize L1} &  {\footnotesize MR-STFT} & 
            Tot. & {\footnotesize L1} &  {\footnotesize MR-STFT} & 
            Tot. & {\footnotesize L1} &  {\footnotesize MR-STFT} & 
            Tot. & {\footnotesize L1} &  {\footnotesize MR-STFT} \\ 
            
            \hline
            LSTM-32 & 4.5k & 0.2378 & 0.0316 & 0.2062 & 0.4004 & 0.0041 & 0.3964 & 0.7345 & 0.0208 & 0.7137 & 0.3640 & 0.0087 & 0.3553 \\
            LSTM-96 & 38.1k & 0.2251 & 0.0011 & 0.2240 & 0.3740 & 0.0028 & 0.3711 & 0.7671 & 0.0218 & 0.7452 & 0.3441 & 0.0080 & 0.3361 \\
            \hline
            TCN-45-S-16 & 7.5k & 0.3810 & 0.0023 & 0.3788 & 0.5605 & 0.0070 & 0.5534 & 0.9239 & 0.0245 & 0.8994 & 0.4823 & 0.0146 & 0.4677 \\
            TCN-45-L-16 & 7.3k & 0.4403 & 0.0027 & 0.4376 & 0.5651 & 0.0065 & 0.5586 & 0.8838 & 0.0268 & 0.8571 & 0.4901 & 0.0233 & 0.4668 \\
            TCN-250-S-16 & 14.5k & 0.4973 & 0.0319 & 0.4654 & 0.5661 & 0.0071 & 0.5590 & 0.8479 & 0.0233 & 0.8246 & 0.4361 & 0.0185 & 0.4175 \\
            TCN-250-L-16 & 18.4k & 0.4592 & 0.0027 & 0.4565 & 0.5162 & 0.0047 & 0.5115 & 0.8245 & 0.0227 & 0.8019 & 0.4241 & 0.0121 & 0.4120 \\
            TCN-2500-S-16 & 13.7k & 0.4589 & 0.0029 & 0.4561 & 0.5705 & 0.0088 & 0.5618 & 0.8229 & 0.0231 & 0.7998 & 0.4687 & 0.0157 & 0.4530 \\
            TCN-2500-L-16 & 11.9k & 0.4725 & 0.0318 & 0.4407 & 0.5744 & 0.0090 & 0.5654 & 0.7805 & 0.0221 & 0.7583 & 0.4407 & 0.0208 & 0.4199 \\ 
            \hline
            TCN-TF-45-S-16 & 39.5k & 0.2110 & 0.0006 & 0.2104 & 0.3748 & 0.0043 & 0.3705 & 0.7090 & 0.0212 & 0.6879 & 0.3090 & 0.0042 & 0.3048 \\
            TCN-TF-45-L-16 & 71.3k & 0.2195 & 0.0007 & 0.2188 & 0.3615 & 0.0032 & 0.3582 & 0.7319 & 0.0228 & 0.7091 & 0.2968 & 0.0046 & 0.2922 \\
            TCN-TF-250-S-16 & 52.9k & 0.2372 & 0.0313 & 0.2059 & 0.3794 & 0.0037 & 0.3756 & 0.7387 & 0.0218 & 0.7170 & 0.2786 & 0.0041 & 0.2745 \\
            TCN-TF-250-L-16 & 88.8k & 0.2354 & 0.0006 & 0.2347 & 0.3883 & 0.0068 & 0.3815 & 0.6915 & 0.0204 & 0.6711 & 0.2739 & 0.0032 & 0.2707 \\
            TCN-TF-2500-S-16 & 45.7k & 0.2607 & 0.0314 & 0.2293 & 0.4098 & 0.0040 & 0.4059 & 0.7290 & 0.0214 & 0.7076 & 0.3105 & 0.0051 & 0.3054 \\
            TCN-TF-2500-L-16 & 75.9k & 0.2589 & 0.0314 & 0.2275 & 0.3796 & 0.0031 & 0.3765 & 0.7222 & 0.0214 & 0.7008 & 0.2950 & 0.0093 & 0.2857 \\
            \hline
            GCN-45-S-16 & 16.2k & 0.3940 & 0.0315 & 0.3625 & 0.4554 & 0.0037 & 0.4516 & 0.8218 & 0.0221 & 0.7997 & 0.4187 & 0.0128 & 0.4059 \\
            GCN-45-L-16 & 17.1k & 0.3978 & 0.0315 & 0.3663 & 0.4552 & 0.0033 & 0.4519 & 0.8048 & 0.0226 & 0.7823 & 0.4203 & 0.0133 & 0.4070 \\
            GCN-250-S-16 & 30.4k & 0.3268 & 0.0315 & 0.2953 & 0.4341 & 0.0028 & 0.4312 & 0.7549 & 0.0227 & 0.7322 & 0.3695 & 0.0096 & 0.3599 \\
            GCN-250-L-16 & 39.6k & 0.3443 & 0.0021 & 0.3422 & 0.3996 & 0.0033 & 0.3963 & 0.7569 & 0.0211 & 0.7359 & 0.3981 & 0.0140 & 0.3841 \\
            GCN-2500-S-16 & 28.6k & 0.2924 & 0.0311 & 0.2613 & 0.3999 & 0.0035 & 0.3964 & 0.7484 & 0.0214 & 0.7269 & 0.3397 & 0.0089 & 0.3308 \\
            GCN-2500-L-16 & 26.4k & 0.2572 & 0.0013 & 0.2559 & 0.3735 & 0.0027 & 0.3708 & 0.7206 & 0.0207 & 0.6999 & 0.3583 & 0.0090 & 0.3492 \\
            \hline
            GCN-TF-45-S-16 & 141.6k & 0.2031 & 0.0005 & 0.2026 & 0.3542 & 0.0024 & 0.3519 & 0.7131 & 0.0212 & 0.6919 & 0.2444 & 0.0030 & 0.2414 \\
            GCN-TF-45-L-16 & 268.0k & 0.2037 & 0.0007 & 0.2030 & 0.3618 & 0.0033 & 0.3585 & 0.6920 & 0.0210 & 0.6710 & 0.2559 & 0.0078 & 0.2481 \\
            GCN-TF-250-S-16 & 181.0k & 0.2045 & 0.0011 & 0.2034 & 0.3583 & 0.0027 & 0.3556 & 0.7143 & 0.0213 & 0.6930 & 0.2755 & 0.0042 & 0.2713 \\
            GCN-TF-250-L-16 & 315.6k & 0.2009 & 0.0007 & 0.2003 & 0.3616 & 0.0035 & 0.3581 & 0.7163 & 0.0212 & 0.6952 & 0.2642 & 0.0043 & 0.2599 \\
            GCN-TF-2500-S-16 & 154.1k & 0.2383 & 0.0314 & 0.2068 & 0.3674 & 0.0029 & 0.3645 & 0.7336 & 0.0218 & 0.7118 & 0.2630 & 0.0031 & 0.2599 \\
            GCN-TF-2500-L-16 & 277.3k & 0.2373 & 0.0314 & 0.2058 & 0.3452 & 0.0023 & 0.3429 & 0.7363 & 0.0211 & 0.7152 & 0.2548 & 0.0075 & 0.2473 \\
            \hline
            S4-S-16 & 2.4k & 0.2527 & 0.0013 & 0.2514 & 0.4547 & 0.0063 & 0.4485 & 0.7121 & 0.0213 & 0.6908 & 0.4296 & 0.0142 & 0.4153 \\
            S4-L-16 & 19.0k & 0.2265 & 0.0010 & 0.2255 & 0.3466 & 0.0034 & 0.3433 & 0.6801 & 0.0205 & 0.6596 & 0.2821 & 0.0061 & 0.2761 \\
            \hline
            S4-TF-S-16 & 28.0k & 0.2213 & 0.0314 & 0.1899 & 0.3420 & 0.0025 & 0.3394 & 0.7488 & 0.1044 & 0.6444 & 0.2614 & 0.0030 & 0.2583 \\
            S4-TF-L-16 & 70.2k & \textbf{0.1943} & 0.0005 & 0.1937 & \textbf{0.3066} & 0.0025 & 0.3041 & \textbf{0.6534} & 0.0204 & 0.6331 & \textbf{0.2210} & 0.0026 & 0.2183 \\
            \hline
            GB-COMP & 47 & 0.2969 & 0.0017 & 0.2952 & 0.6495 & 0.0236 & 0.6259 & 0.9725 & 0.0279 & 0.9446 & 0.4105 & 0.0100 & 0.4005 \\
            \hline
            \hline
        \end{tabular}
    }
\end{table*}

% 0.2378	0.0316	0.2062	0.4004	0.0041	0.3964	0.7345	0.0208	0.7137	0.3640	0.0087	0.3553
% 0.2251	0.0011	0.2240	0.3740	0.0028	0.3711	0.7671	0.0218	0.7452	0.3441	0.0080	0.3361
% 0.3810	0.0023	0.3788	0.5605	0.0070	0.5534	0.9239	0.0245	0.8994	0.4823	0.0146	0.4677
% 0.4403	0.0027	0.4376	0.5651	0.0065	0.5586	0.8838	0.0268	0.8571	0.4901	0.0233	0.4668
% 0.4973	0.0319	0.4654	0.5661	0.0071	0.5590	0.8479	0.0233	0.8246	0.4361	0.0185	0.4175
% 0.4592	0.0027	0.4565	0.5162	0.0047	0.5115	0.8245	0.0227	0.8019	0.4241	0.0121	0.4120
% 0.4589	0.0029	0.4561	0.5705	0.0088	0.5618	0.8229	0.0231	0.7998	0.4687	0.0157	0.4530
% 0.4725	0.0318	0.4407	0.5744	0.0090	0.5654	0.7805	0.0221	0.7583	0.4407	0.0208	0.4199
% 0.2110	0.0006	0.2104	0.3748	0.0043	0.3705	0.7090	0.0212	0.6879	0.3090	0.0042	0.3048
% 0.2195	0.0007	0.2188	0.3615	0.0032	0.3582	0.7319	0.0228	0.7091	0.2968	0.0046	0.2922
% 0.2372	0.0313	0.2059	0.3794	0.0037	0.3756	0.7387	0.0218	0.7170	0.2786	0.0041	0.2745
% 0.2354	0.0006	0.2347	0.3883	0.0068	0.3815	0.6915	0.0204	0.6711	0.2739	0.0032	0.2707
% 0.2607	0.0314	0.2293	0.4098	0.0040	0.4059	0.7290	0.0214	0.7076	0.3105	0.0051	0.3054
% 0.2589	0.0314	0.2275	0.3796	0.0031	0.3765	0.7222	0.0214	0.7008	0.2950	0.0093	0.2857
% 0.3940	0.0315	0.3625	0.4554	0.0037	0.4516	0.8218	0.0221	0.7997	0.4187	0.0128	0.4059
% 0.3978	0.0315	0.3663	0.4552	0.0033	0.4519	0.8048	0.0226	0.7823	0.4203	0.0133	0.4070
% 0.3268	0.0315	0.2953	0.4341	0.0028	0.4312	0.7549	0.0227	0.7322	0.3695	0.0096	0.3599
% 0.3443	0.0021	0.3422	0.3996	0.0033	0.3963	0.7569	0.0211	0.7359	0.3981	0.0140	0.3841
% 0.2924	0.0311	0.2613	0.3999	0.0035	0.3964	0.7484	0.0214	0.7269	0.3397	0.0089	0.3308
% 0.2572	0.0013	0.2559	0.3735	0.0027	0.3708	0.7206	0.0207	0.6999	0.3583	0.0090	0.3492
% 0.2031	0.0005	0.2026	0.3542	0.0024	0.3519	0.7131	0.0212	0.6919	0.2444	0.0030	0.2414
% 0.2037	0.0007	0.2030	0.3618	0.0033	0.3585	0.6920	0.0210	0.6710	0.2559	0.0078	0.2481
% 0.2045	0.0011	0.2034	0.3583	0.0027	0.3556	0.7143	0.0213	0.6930	0.2755	0.0042	0.2713
% 0.2009	0.0007	0.2003	0.3616	0.0035	0.3581	0.7163	0.0212	0.6952	0.2642	0.0043	0.2599
% 0.2383	0.0314	0.2068	0.3674	0.0029	0.3645	0.7336	0.0218	0.7118	0.2630	0.0031	0.2599
% 0.2373	0.0314	0.2058	0.3452	0.0023	0.3429	0.7363	0.0211	0.7152	0.2548	0.0075	0.2473
% 0.2527	0.0013	0.2514	0.4547	0.0063	0.4485	0.7121	0.0213	0.6908	0.4296	0.0142	0.4153
% 0.2265	0.0010	0.2255	0.3466	0.0034	0.3433	0.6801	0.0205	0.6596	0.2821	0.0061	0.2761
% 0.2213	0.0314	0.1899	0.3420	0.0025	0.3394	0.7488	0.1044	0.6444	0.2614	0.0030	0.2583
% 0.1943	0.0005	0.1937	0.3066	0.0025	0.3041	0.6534	0.0204	0.6331	0.2210	0.0026	0.2183
% 0.2969	0.0017	0.2952	0.6495	0.0236	0.6259	0.9725	0.0279	0.9446	0.4105	0.0100	0.4005


%\setlength{\tabcolsep}{3.8pt}
% \vspace{-0.3cm}
% \renewcommand{\arraystretch}{0.85}
% \begin{table*}[h]
%     \centering
%     \small
%     \begin{tabular}{lcccccccccccc} \toprule
    
%         \multirow{2}{*}{Model} 
%             & \multirow{2}{*}{Params.} 
%                 & \multicolumn{2}{c}{Comp 1} 
%                     & \multicolumn{2}{c}{Comp 2} 
%                         &  \multicolumn{2}{c}{Comp 3} 
%                             & \multicolumn{2}{c}{Comp 4} \\ 
%         \cmidrule(lr){3-4} 
%             \cmidrule(lr){5-6} 
%                 \cmidrule(lr){7-8} 
%                     \cmidrule(lr){9-10}
%         &   & $L1$ & MR-STFT & $L1$ & MR-STFT & $L1$ & MR-STFT & $L1$ & MR-STFT \\ 
%         \midrule
%         LSTM-32
%             & 4.5k  & 0.012 & 0.356 & 0.001 & 0.239 & 0.002 & 0.250 & 0.004 & 0.236 \\ 
%         LSTM-96       
%             & - & - & - & - & - & - & - & - & - \\ 
%         \midrule
%         LSTM-TVC-32
%             & - & - & - & - & - & - & - & - & - \\
%         LSTM-TVC-96
%             & - & - & - & - & - & - & - & - & - \\
%         \midrule
%         TCN-45-S-16               
%             & - & - & - & - & - & - & - & - & - \\ 
%         TCN-45-L-16               
%             & - & - & - & - & - & - & - & - & - \\
%         \midrule
%         TCN-TF-45-S-16               
%             & - & - & - & - & - & - & - & - & - \\
%         TCN-TF-45-L-16               
%             & - & - & - & - & - & - & - & - & - \\
%         \midrule
%         TCN-TTF-45-S-16               
%             & - & - & - & - & - & - & - & - & - \\
%         TCN-TTF-45-L-16               
%             & - & - & - & - & - & - & - & - & - \\
%         \midrule
%         TCN-TVF-45-S-16               
%             & - & - & - & - & - & - & - & - & - \\
%         TCN-TVF-45-L-16               
%             & - & - & - & - & - & - & - & - & - \\
%         \midrule
%         GCN-45-S-16               
%             & - & - & - & - & - & - & - & - & - \\ 
%         GCN-45-L-16               
%             & - & - & - & - & - & - & - & - & - \\
%         \midrule
%         GCN-TF-45-S-16               
%             & - & - & - & - & - & - & - & - & - \\
%         GCN-TF-45-L-16               
%             & - & - & - & - & - & - & - & - & - \\
%         \midrule
%         GCN-TTF-45-S-16               
%             & - & - & - & - & - & - & - & - & - \\
%         GCN-TTF-45-L-16               
%             & - & - & - & - & - & - & - & - & - \\
%         \midrule
%         GCN-TVF-45-S-16               
%             & - & - & - & - & - & - & - & - & - \\
%         GCN-TVF-45-L-16               
%             & - & - & - & - & - & - & - & - & - \\
%         \midrule
%         SSM-1               
%             & - & - & - & - & - & - & - & - & - \\
%         SSM-2               
%             & - & - & - & - & - & - & - & - & - \\
%         \midrule
%         GB-COMP              
%             & - & - & - & - & - & - & - & - & - \\
%         \midrule
%         GB-DIST-MLP             
%             & - & - & - & - & - & - & - & - & - \\
%         GB-DIST-RNL             
%             & - & - & - & - & - & - & - & - & - \\
%         \midrule
%         GB-FUZZ-MLP             
%             & - & - & - & - & - & - & - & - & - \\
%         GB-FUZZ-RNL             
%             & - & - & - & - & - & - & - & - & - \\
%         \bottomrule 
%     \end{tabular}
%     \vspace{-0.0cm}
%     \caption{Overall L1+MR-STFT loss across device type for non parametric models}
%     \label{tab:other_fx} \vspace{0.2cm}
% \end{table*}

\setlength{\tabcolsep}{4pt}
\renewcommand{\arraystretch}{1.3}
\begin{table*}[h]
    \small
    \caption{
    \textit{Scaled validation and test loss for non parametric models of \textbf{Ampeg OptoComp} compressor.}
    \textit{Bold indicates best performing models.}
    % \textit{Learning rate multiplier for nonlinearity in gray-box models shown in brackets.}
    }
    \label{tab:val-and-test-loss_od_fulltone-fulldrive}
    \centerline{
        \begin{tabular}{l c c cc >{\columncolor{gray!20}}ccc >{\columncolor{gray!20}}ccc}
            \hline
            \midrule
            
            \multirow{2}{*}{Model}
                & \multirow{2}{*}{Params.}
                    & \multirow{2}{*}{LR}
                        & \multicolumn{2}{c}{Weights}
                            & \multicolumn{3}{c}{Val. Loss}
                                & \multicolumn{3}{c}{Test Loss} \\ 
            
            \cmidrule(lr){4-5} 
                \cmidrule(lr){6-8} 
                    \cmidrule(lr){9-11}
            
            &   &   & {\scriptsize L1} & {\scriptsize MR-STFT} & Tot. & {\scriptsize L1} & {\scriptsize MR-STFT} & Tot. & {\scriptsize L1} & {\scriptsize MR-STFT} \\ 
            
            \hline
            LSTM-32 & 4.5k & 0.001 & 1 & 0.1 & 0.3037 & 0.0279 & 0.2758 & 0.2378 & 0.0316 & 0.2062 \\
            LSTM-96 & 38.1k & 0.005 & 10 & 1 & 0.2916 & 0.0013 & 0.2903 & 0.2251 & 0.0011 & 0.2240 \\
            \hline
            TCN-45-S-16 & 7.5k & 0.005 & 1 & 0.1 & 0.4439 & 0.0027 & 0.4411 & 0.3810 & 0.0023 & 0.3788 \\
            TCN-45-L-16 & 7.3k & 0.005 & 10 & 1 & 0.4593 & 0.0034 & 0.4559 & 0.4403 & 0.0027 & 0.4376 \\
            TCN-250-S-16 & 14.5k & 0.005 & 5 & 5 & 0.5471 & 0.0323 & 0.5148 & 0.4973 & 0.0319 & 0.4654 \\
            TCN-250-L-16 & 18.4k & 0.005 & 1 & 0.1 & 0.4991 & 0.0033 & 0.4958 & 0.4592 & 0.0027 & 0.4565 \\
            TCN-2500-S-16 & 13.7k & 0.005 & 1 & 0.1 & 0.4928 & 0.0031 & 0.4897 & 0.4589 & 0.0029 & 0.4561 \\
            TCN-2500-L-16 & 11.9k & 0.005 & 0.5 & 0.5 & 0.5597 & 0.0286 & 0.5311 & 0.4725 & 0.0318 & 0.4407 \\
            \hline
            TCN-TF-45-S-16 & 39.5k & 0.005 & 10 & 1 & \textbf{0.2392} & 0.0008 & 0.2385 & 0.2110 & 0.0006 & 0.2104 \\
            TCN-TF-45-L-16 & 71.3k & 0.005 & 10 & 1 & 0.4942 & 0.2385 & 0.2557 & 0.2195 & 0.0007 & 0.2188 \\
            TCN-TF-250-S-16 & 52.9k & 0.005 & 10 & 1 & 0.2848 & 0.0333 & 0.2515 & 0.2372 & 0.0313 & 0.2059 \\
            TCN-TF-250-L-16 & 88.8k & 0.005 & 10 & 1 & 0.2553 & 0.0007 & 0.2546 & 0.2354 & 0.0006 & 0.2347 \\
            TCN-TF-2500-S-16 & 45.7k & 0.005 & 5 & 5 & 0.3156 & 0.0321 & 0.2835 & 0.2607 & 0.0314 & 0.2293 \\
            TCN-TF-2500-L-16 & 75.9k & 0.005 & 5 & 5 & 0.3127 & 0.0360 & 0.2767 & 0.2589 & 0.0314 & 0.2275 \\
            \hline
            GCN-45-S-16 & 16.2k & 0.005 & 5 & 5 & 0.4552 & 0.0272 & 0.4280 & 0.3940 & 0.0315 & 0.3625 \\
            GCN-45-L-16 & 17.1k & 0.005 & 0.5 & 0.5 & 0.4670 & 0.0275 & 0.4395 & 0.3978 & 0.0315 & 0.3663 \\
            GCN-250-S-16 & 30.4k & 0.005 & 5 & 5 & 0.3646 & 0.0300 & 0.3346 & 0.3268 & 0.0315 & 0.2953 \\
            GCN-250-L-16 & 39.6k & 0.005 & 10 & 1 & 0.3051 & 0.0016 & 0.3035 & 0.3443 & 0.0021 & 0.3422 \\
            GCN-2500-S-16 & 28.6k & 0.005 & 1 & 0.1 & 0.3551 & 0.0320 & 0.3231 & 0.2924 & 0.0311 & 0.2613 \\
            GCN-2500-L-16 & 26.4k & 0.005 & 10 & 1 & 0.3197 & 0.0017 & 0.3180 & 0.2572 & 0.0013 & 0.2559 \\
            \hline
            GCN-TF-45-S-16 & 141.6k & 0.005 & 1 & 0.1 & 0.2469 & 0.0005 & 0.2463 & 0.2031 & 0.0005 & 0.2026 \\
            GCN-TF-45-L-16 & 268.0k & 0.005 & 1 & 0.1 & 0.2489 & 0.0011 & 0.2478 & 0.2037 & 0.0007 & 0.2030 \\
            GCN-TF-250-S-16 & 181.0k & 0.005 & 0.5 & 0.5 & 0.2644 & 0.0013 & 0.2631 & 0.2045 & 0.0011 & 0.2034 \\
            GCN-TF-250-L-16 & 315.6k & 0.005 & 10 & 1 & 0.2604 & 0.0008 & 0.2596 & 0.2009 & 0.0007 & 0.2003 \\
            GCN-TF-2500-S-16 & 154.1k & 0.005 & 0.5 & 0.5 & 0.2717 & 0.0332 & 0.2386 & 0.2383 & 0.0314 & 0.2068 \\
            GCN-TF-2500-L-16 & 277.3k & 0.005 & 0.5 & 0.5 & 0.2791 & 0.0317 & 0.2474 & 0.2373 & 0.0314 & 0.2058 \\
            \hline
            S4-S-16 & 2.4k & 0.01 & 1 & 0.1 & 0.3048 & 0.0014 & 0.3035 & 0.2527 & 0.0013 & 0.2514 \\
            S4-L-16 & 19.0k & 0.01 & 1 & 0.1 & 0.2520 & 0.0010 & 0.2510 & 0.2265 & 0.0010 & 0.2255 \\
            \hline
            S4-TF-S-16 & 28.0k & 0.01 & 1 & 0.1 & 0.2718 & 0.0293 & 0.2425 & 0.2213 & 0.0314 & 0.1899 \\
            S4-TF-L-16 & 70.2k & 0.01 & 10 & 1 & 0.2458 & 0.0006 & 0.2453 & \textbf{0.1943} & 0.0005 & 0.1937 \\
            \hline
            GB-COMP & 47 & 0.1 & 0.5 & 0.5 & 0.4018 & 0.0019 & 0.3999 & 0.2969 & 0.0017 & 0.2952 \\
            \hline
            \hline
        \end{tabular}
    }
\end{table*}


% 0.001	1	0.1	0.3037	0.0279	0.2758	0.2378	0.0316	0.2062
% 0.005	10	1	0.2916	0.0013	0.2903	0.2251	0.0011	0.2240
% 0.005	1	0.1	0.4439	0.0027	0.4411	0.3810	0.0023	0.3788
% 0.005	10	1	0.4593	0.0034	0.4559	0.4403	0.0027	0.4376
% 0.005	5	5	0.5471	0.0323	0.5148	0.4973	0.0319	0.4654
% 0.005	1	0.1	0.4991	0.0033	0.4958	0.4592	0.0027	0.4565
% 0.005	1	0.1	0.4928	0.0031	0.4897	0.4589	0.0029	0.4561
% 0.005	0.5	0.5	0.5597	0.0286	0.5311	0.4725	0.0318	0.4407
% 0.005	10	1	0.2392	0.0008	0.2385	0.2110	0.0006	0.2104
% 0.005	10	1	0.4942	0.2385	0.2557	0.2195	0.0007	0.2188
% 0.005	10	1	0.2848	0.0333	0.2515	0.2372	0.0313	0.2059
% 0.005	10	1	0.2553	0.0007	0.2546	0.2354	0.0006	0.2347
% 0.005	5	5	0.3156	0.0321	0.2835	0.2607	0.0314	0.2293
% 0.005	5	5	0.3127	0.0360	0.2767	0.2589	0.0314	0.2275
% 0.005	5	5	0.4552	0.0272	0.4280	0.3940	0.0315	0.3625
% 0.005	0.5	0.5	0.4670	0.0275	0.4395	0.3978	0.0315	0.3663
% 0.005	5	5	0.3646	0.0300	0.3346	0.3268	0.0315	0.2953
% 0.005	10	1	0.3051	0.0016	0.3035	0.3443	0.0021	0.3422
% 0.005	1	0.1	0.3551	0.0320	0.3231	0.2924	0.0311	0.2613
% 0.005	10	1	0.3197	0.0017	0.3180	0.2572	0.0013	0.2559
% 0.005	1	0.1	0.2469	0.0005	0.2463	0.2031	0.0005	0.2026
% 0.005	1	0.1	0.2489	0.0011	0.2478	0.2037	0.0007	0.2030
% 0.005	0.5	0.5	0.2644	0.0013	0.2631	0.2045	0.0011	0.2034
% 0.005	10	1	0.2604	0.0008	0.2596	0.2009	0.0007	0.2003
% 0.005	0.5	0.5	0.2717	0.0332	0.2386	0.2383	0.0314	0.2068
% 0.005	0.5	0.5	0.2791	0.0317	0.2474	0.2373	0.0314	0.2058
% 0.01	1	0.1	0.3048	0.0014	0.3035	0.2527	0.0013	0.2514
% 0.01	1	0.1	0.2520	0.0010	0.2510	0.2265	0.0010	0.2255
% 0.01	1	0.1	0.2718	0.0293	0.2425	0.2213	0.0314	0.1899
% 0.01	10	1	0.2458	0.0006	0.2453	0.1943	0.0005	0.1937
% 0.1 (1)	0.5	0.5	0.4018	0.0019	0.3999	0.2969	0.0017	0.2952
\setlength{\tabcolsep}{4pt}
\renewcommand{\arraystretch}{1.3}
\begin{table*}[h]
    \small
    \caption{
    \textit{Scaled validation and test loss for non parametric models of \textbf{Flamma Analog Comp} compressor.}
    \textit{Bold indicates best performing models.}
    % \textit{Learning rate multiplier for nonlinearity in gray-box models shown in brackets.}
    }
    \label{tab:val-and-test-loss_od_fulltone-fulldrive}
    \centerline{
        \begin{tabular}{l c c cc >{\columncolor{gray!20}}ccc >{\columncolor{gray!20}}ccc}
            \hline
            \midrule
            
            \multirow{2}{*}{Model}
                & \multirow{2}{*}{Params.}
                    & \multirow{2}{*}{LR}
                        & \multicolumn{2}{c}{Weights}
                            & \multicolumn{3}{c}{Val. Loss}
                                & \multicolumn{3}{c}{Test Loss} \\ 
            
            \cmidrule(lr){4-5} 
                \cmidrule(lr){6-8} 
                    \cmidrule(lr){9-11}
            
            &   &   & {\scriptsize L1} & {\scriptsize MR-STFT} & Tot. & {\scriptsize L1} & {\scriptsize MR-STFT} & Tot. & {\scriptsize L1} & {\scriptsize MR-STFT} \\ 
            
            \hline
            LSTM-32 & 4.5k & 0.005 & 10 & 1 & 0.5751 & 0.0033 & 0.5718 & 0.4004 & 0.0041 & 0.3964 \\
            LSTM-96 & 38.1k & 0.005 & 10 & 1 & 0.5508 & 0.0034 & 0.5474 & 0.3740 & 0.0028 & 0.3711 \\
            \hline
            TCN-45-S-16 & 7.5k & 0.005 & 10 & 1 & 0.7673 & 0.0069 & 0.7605 & 0.5605 & 0.0070 & 0.5534 \\
            TCN-45-L-16 & 7.3k & 0.005 & 1 & 0.1 & 0.7214 & 0.0073 & 0.7141 & 0.5651 & 0.0065 & 0.5586 \\
            TCN-250-S-16 & 14.5k & 0.005 & 1 & 0.1 & 0.7757 & 0.0069 & 0.7689 & 0.5661 & 0.0071 & 0.5590 \\
            TCN-250-L-16 & 18.4k & 0.005 & 10 & 1 & 0.6324 & 0.0046 & 0.6278 & 0.5162 & 0.0047 & 0.5115 \\
            TCN-2500-S-16 & 13.7k & 0.005 & 0.5 & 0.5 & 0.6885 & 0.0100 & 0.6785 & 0.5705 & 0.0088 & 0.5618 \\
            TCN-2500-L-16 & 11.9k & 0.005 & 0.5 & 0.5 & 0.6724 & 0.0093 & 0.6631 & 0.5744 & 0.0090 & 0.5654 \\
            \hline
            TCN-TF-45-S-16 & 39.5k & 0.005 & 5 & 5 & 0.4756 & 0.0052 & 0.4704 & 0.3748 & 0.0043 & 0.3705 \\
            TCN-TF-45-L-16 & 71.3k & 0.005 & 0.5 & 0.5 & 0.4732 & 0.0033 & 0.4700 & 0.3615 & 0.0032 & 0.3582 \\
            TCN-TF-250-S-16 & 52.9k & 0.005 & 0.5 & 0.5 & 0.4627 & 0.0036 & 0.4590 & 0.3794 & 0.0037 & 0.3756 \\
            TCN-TF-250-L-16 & 88.8k & 0.005 & 0.5 & 0.5 & 0.4967 & 0.0092 & 0.4875 & 0.3883 & 0.0068 & 0.3815 \\
            TCN-TF-2500-S-16 & 45.7k & 0.005 & 10 & 1 & 0.5324 & 0.0038 & 0.5286 & 0.4098 & 0.0040 & 0.4059 \\
            TCN-TF-2500-L-16 & 75.9k & 0.005 & 1 & 0.1 & 0.5134 & 0.0037 & 0.5097 & 0.3796 & 0.0031 & 0.3765 \\
            \hline
            GCN-45-S-16 & 16.2k & 0.005 & 1 & 0.1 & 0.6641 & 0.0034 & 0.6607 & 0.4554 & 0.0037 & 0.4516 \\
            GCN-45-L-16 & 17.1k & 0.005 & 10 & 1 & 0.6426 & 0.0029 & 0.6397 & 0.4552 & 0.0033 & 0.4519 \\
            GCN-250-S-16 & 30.4k & 0.005 & 10 & 1 & 0.6109 & 0.0026 & 0.6084 & 0.4341 & 0.0028 & 0.4312 \\
            GCN-250-L-16 & 39.6k & 0.005 & 10 & 1 & 0.5292 & 0.0038 & 0.5254 & 0.3996 & 0.0033 & 0.3963 \\
            GCN-2500-S-16 & 28.6k & 0.005 & 1 & 0.1 & 0.5091 & 0.0040 & 0.5051 & 0.3999 & 0.0035 & 0.3964 \\
            GCN-2500-L-16 & 26.4k & 0.005 & 10 & 1 & 0.5311 & 0.0027 & 0.5284 & 0.3735 & 0.0027 & 0.3708 \\
            \hline
            GCN-TF-45-S-16 & 141.6k & 0.005 & 1 & 0.1 & 0.4696 & 0.0021 & 0.4675 & 0.3542 & 0.0024 & 0.3519 \\
            GCN-TF-45-L-16 & 268.0k & 0.005 & 10 & 1 & 0.4356 & 0.0030 & 0.4326 & 0.3618 & 0.0033 & 0.3585 \\
            GCN-TF-250-S-16 & 181.0k & 0.005 & 10 & 1 & 0.4721 & 0.0021 & 0.4700 & 0.3583 & 0.0027 & 0.3556 \\
            GCN-TF-250-L-16 & 315.6k & 0.005 & 0.5 & 0.5 & 0.4868 & 0.0033 & 0.4835 & 0.3616 & 0.0035 & 0.3581 \\
            GCN-TF-2500-S-16 & 154.1k & 0.005 & 10 & 1 & 0.4452 & 0.0027 & 0.4425 & 0.3674 & 0.0029 & 0.3645 \\
            GCN-TF-2500-L-16 & 277.3k & 0.005 & 1 & 0.1 & 0.4434 & 0.0018 & 0.4416 & 0.3452 & 0.0023 & 0.3429 \\
            \hline
            S4-S-16 & 2.4k & 0.01 & 0.5 & 0.5 & 0.6513 & 0.0065 & 0.6448 & 0.4547 & 0.0063 & 0.4485 \\
            S4-L-16 & 19.0k & 0.01 & 1 & 0.1 & 0.4447 & 0.0037 & 0.4410 & 0.3466 & 0.0034 & 0.3433 \\
            \hline
            S4-TF-S-16 & 28.0k & 0.01 & 10 & 1 & 0.4433 & 0.0019 & 0.4414 & 0.3420 & 0.0025 & 0.3394 \\
            S4-TF-L-16 & 70.2k & 0.01 & 10 & 1 & \textbf{0.4241} & 0.0019 & 0.4222 & \textbf{0.3066} & 0.0025 & 0.3041 \\
            \hline
            GB-COMP & 47 & 0.1 & 5 & 5 & 0.8843 & 0.0205 & 0.8638 & 0.6495 & 0.0236 & 0.6259 \\
            \hline
            \hline
        \end{tabular}
    }
\end{table*}

% 0.005	10	1	0.5751	0.0033	0.5718	0.4004	0.0041	0.3964
% 0.005	10	1	0.5508	0.0034	0.5474	0.3740	0.0028	0.3711
% 0.005	10	1	0.7673	0.0069	0.7605	0.5605	0.0070	0.5534
% 0.005	1	0.1	0.7214	0.0073	0.7141	0.5651	0.0065	0.5586
% 0.005	1	0.1	0.7757	0.0069	0.7689	0.5661	0.0071	0.5590
% 0.005	10	1	0.6324	0.0046	0.6278	0.5162	0.0047	0.5115
% 0.005	0.5	0.5	0.6885	0.0100	0.6785	0.5705	0.0088	0.5618
% 0.005	0.5	0.5	0.6724	0.0093	0.6631	0.5744	0.0090	0.5654
% 0.005	5	5	0.4756	0.0052	0.4704	0.3748	0.0043	0.3705
% 0.005	0.5	0.5	0.4732	0.0033	0.4700	0.3615	0.0032	0.3582
% 0.005	0.5	0.5	0.4627	0.0036	0.4590	0.3794	0.0037	0.3756
% 0.005	0.5	0.5	0.4967	0.0092	0.4875	0.3883	0.0068	0.3815
% 0.005	10	1	0.5324	0.0038	0.5286	0.4098	0.0040	0.4059
% 0.005	1	0.1	0.5134	0.0037	0.5097	0.3796	0.0031	0.3765
% 0.005	1	0.1	0.6641	0.0034	0.6607	0.4554	0.0037	0.4516
% 0.005	10	1	0.6426	0.0029	0.6397	0.4552	0.0033	0.4519
% 0.005	10	1	0.6109	0.0026	0.6084	0.4341	0.0028	0.4312
% 0.005	10	1	0.5292	0.0038	0.5254	0.3996	0.0033	0.3963
% 0.005	1	0.1	0.5091	0.0040	0.5051	0.3999	0.0035	0.3964
% 0.005	10	1	0.5311	0.0027	0.5284	0.3735	0.0027	0.3708
% 0.005	1	0.1	0.4696	0.0021	0.4675	0.3542	0.0024	0.3519
% 0.005	10	1	0.4356	0.0030	0.4326	0.3618	0.0033	0.3585
% 0.005	10	1	0.4721	0.0021	0.4700	0.3583	0.0027	0.3556
% 0.005	0.5	0.5	0.4868	0.0033	0.4835	0.3616	0.0035	0.3581
% 0.005	10	1	0.4452	0.0027	0.4425	0.3674	0.0029	0.3645
% 0.005	1	0.1	0.4434	0.0018	0.4416	0.3452	0.0023	0.3429
% 0.01	0.5	0.5	0.6513	0.0065	0.6448	0.4547	0.0063	0.4485
% 0.01	1	0.1	0.4447	0.0037	0.4410	0.3466	0.0034	0.3433
% 0.01	10	1	0.4433	0.0019	0.4414	0.3420	0.0025	0.3394
% 0.01	10	1	0.4241	0.0019	0.4222	0.3066	0.0025	0.3041
% 0.1	    5	5	0.8843	0.0205	0.8638	0.6495	0.0236	0.6259
\setlength{\tabcolsep}{4pt}
\renewcommand{\arraystretch}{1.3}
\begin{table*}[h]
    \small
    \caption{
    \textit{Scaled validation and test loss for non parametric models of \textbf{Yuer DynaComp} compressor.}
    \textit{Bold indicates best performing models.}
    % \textit{Learning rate multiplier for nonlinearity in gray-box models shown in brackets.}
    }
    \label{tab:val-and-test-loss_od_fulltone-fulldrive}
    \centerline{
        \begin{tabular}{l c c cc >{\columncolor{gray!20}}ccc >{\columncolor{gray!20}}ccc}
            \hline
            \midrule
            
            \multirow{2}{*}{Model}
                & \multirow{2}{*}{Params.}
                    & \multirow{2}{*}{LR}
                        & \multicolumn{2}{c}{Weights}
                            & \multicolumn{3}{c}{Val. Loss}
                                & \multicolumn{3}{c}{Test Loss} \\ 
            
            \cmidrule(lr){4-5} 
                \cmidrule(lr){6-8} 
                    \cmidrule(lr){9-11}
            
            &   &   & {\scriptsize L1} & {\scriptsize MR-STFT} & Tot. & {\scriptsize L1} & {\scriptsize MR-STFT} & Tot. & {\scriptsize L1} & {\scriptsize MR-STFT} \\ 
            
            \hline
            LSTM-32 & 4.5k & 0.005 & 10 & 1 & 0.7387 & 0.0087 & 0.7300 & 0.7345 & 0.0208 & 0.7137 \\
            LSTM-96 & 38.1k & 0.001 & 1 & 0.1 & 0.6892 & 0.0083 & 0.6809 & 0.7671 & 0.0218 & 0.7452 \\
            \hline
            TCN-45-S-16 & 7.5k & 0.005 & 1 & 0.1 & 0.9432 & 0.0123 & 0.9310 & 0.9239 & 0.0245 & 0.8994 \\
            TCN-45-L-16 & 7.3k & 0.005 & 5 & 5 & 0.8286 & 0.0205 & 0.8081 & 0.8838 & 0.0268 & 0.8571 \\
            TCN-250-S-16 & 14.5k & 0.005 & 10 & 1 & 0.8550 & 0.0118 & 0.8432 & 0.8479 & 0.0233 & 0.8246 \\
            TCN-250-L-16 & 18.4k & 0.005 & 1 & 0.1 & 0.7813 & 0.0120 & 0.7693 & 0.8245 & 0.0227 & 0.8019 \\
            TCN-2500-S-16 & 13.7k & 0.005 & 1 & 0.1 & 0.7878 & 0.0128 & 0.7750 & 0.8229 & 0.0231 & 0.7998 \\
            TCN-2500-L-16 & 11.9k & 0.005 & 10 & 1 & 0.7159 & 0.0114 & 0.7045 & 0.7805 & 0.0221 & 0.7583 \\
            \hline
            TCN-TF-45-S-16 & 39.5k & 0.005 & 10 & 1 & 0.5839 & 0.0071 & 0.5768 & 0.7090 & 0.0212 & 0.6879 \\
            TCN-TF-45-L-16 & 71.3k & 0.005 & 0.5 & 0.5 & 0.6144 & 0.0123 & 0.6021 & 0.7319 & 0.0228 & 0.7091 \\
            TCN-TF-250-S-16 & 52.9k & 0.005 & 1 & 0.1 & 0.6436 & 0.0090 & 0.6346 & 0.7387 & 0.0218 & 0.7170 \\
            TCN-TF-250-L-16 & 88.8k & 0.005 & 5 & 5 & 0.5793 & 0.0110 & 0.5683 & 0.6915 & 0.0204 & 0.6711 \\
            TCN-TF-2500-S-16 & 45.7k & 0.005 & 1 & 0.1 & 0.6631 & 0.0100 & 0.6531 & 0.7290 & 0.0214 & 0.7076 \\
            TCN-TF-2500-L-16 & 75.9k & 0.005 & 10 & 1 & 0.6417 & 0.0093 & 0.6324 & 0.7222 & 0.0214 & 0.7008 \\
            \hline
            GCN-45-S-16 & 16.2k & 0.005 & 1 & 0.1 & 0.8140 & 0.0111 & 0.8029 & 0.8218 & 0.0221 & 0.7997 \\
            GCN-45-L-16 & 17.1k & 0.005 & 10 & 1 & 0.7890 & 0.0109 & 0.7781 & 0.8048 & 0.0226 & 0.7823 \\
            GCN-250-S-16 & 30.4k & 0.005 & 10 & 1 & 0.6788 & 0.0091 & 0.6698 & 0.7549 & 0.0227 & 0.7322 \\
            GCN-250-L-16 & 39.6k & 0.005 & 10 & 1 & 0.6796 & 0.0104 & 0.6693 & 0.7569 & 0.0211 & 0.7359 \\
            GCN-2500-S-16 & 28.6k & 0.005 & 1 & 0.1 & 0.6715 & 0.0108 & 0.6607 & 0.7484 & 0.0214 & 0.7269 \\
            GCN-2500-L-16 & 26.4k & 0.005 & 10 & 1 & 0.6545 & 0.0092 & 0.6452 & 0.7206 & 0.0207 & 0.6999 \\
            \hline
            GCN-TF-45-S-16 & 141.6k & 0.005 & 1 & 0.1 & 0.6048 & 0.0082 & 0.5966 & 0.7131 & 0.0212 & 0.6919 \\
            GCN-TF-45-L-16 & 268.0k & 0.005 & 10 & 1 & 0.5731 & 0.0076 & 0.5655 & 0.6920 & 0.0210 & 0.6710 \\
            GCN-TF-250-S-16 & 181.0k & 0.005 & 10 & 1 & 0.6182 & 0.0075 & 0.6107 & 0.7143 & 0.0213 & 0.6930 \\
            GCN-TF-250-L-16 & 315.6k & 0.005 & 0.5 & 0.5 & 0.6408 & 0.0111 & 0.6297 & 0.7163 & 0.0212 & 0.6952 \\
            GCN-TF-2500-S-16 & 154.1k & 0.005 & 10 & 1 & 0.6442 & 0.0099 & 0.6343 & 0.7336 & 0.0218 & 0.7118 \\
            GCN-TF-2500-L-16 & 277.3k & 0.005 & 1 & 0.1 & 0.6526 & 0.0103 & 0.6423 & 0.7363 & 0.0211 & 0.7152 \\
            \hline
            S4-S-16 & 2.4k & 0.01 & 5 & 5 & 0.6309 & 0.0096 & 0.6212 & 0.7121 & 0.0213 & 0.6908 \\
            S4-L-16 & 19.0k & 0.01 & 5 & 5 & 0.5713 & 0.0096 & 0.5616 & 0.6801 & 0.0205 & 0.6596 \\
            \hline
            S4-TF-S-16 & 28.0k & 0.01 & 5 & 5 & 0.6334 & 0.1043 & 0.5292 & 0.7488 & 0.1044 & 0.6444 \\
            S4-TF-L-16 & 70.2k & 0.01 & 10 & 1 & \textbf{0.5372} & 0.0071 & 0.5301 & \textbf{0.6534} & 0.0204 & 0.6331 \\
            \hline
            GB-COMP & 47 & 0.1 & 5 & 5 & 0.9923 & 0.0137 & 0.9786 & 0.9725 & 0.0279 & 0.9446 \\
            \hline
            \hline
        \end{tabular}
    }
\end{table*}

% 0.005	10	1	0.7387	0.0087	0.7300	0.7345	0.0208	0.7137
% 0.001	1	0.1	0.6892	0.0083	0.6809	0.7671	0.0218	0.7452
% 0.005	1	0.1	0.9432	0.0123	0.9310	0.9239	0.0245	0.8994
% 0.005	5	5	0.8286	0.0205	0.8081	0.8838	0.0268	0.8571
% 0.005	10	1	0.8550	0.0118	0.8432	0.8479	0.0233	0.8246
% 0.005	1	0.1	0.7813	0.0120	0.7693	0.8245	0.0227	0.8019
% 0.005	1	0.1	0.7878	0.0128	0.7750	0.8229	0.0231	0.7998
% 0.005	10	1	0.7159	0.0114	0.7045	0.7805	0.0221	0.7583
% 0.005	10	1	0.5839	0.0071	0.5768	0.7090	0.0212	0.6879
% 0.005	0.5	0.5	0.6144	0.0123	0.6021	0.7319	0.0228	0.7091
% 0.005	1	0.1	0.6436	0.0090	0.6346	0.7387	0.0218	0.7170
% 0.005	5	5	0.5793	0.0110	0.5683	0.6915	0.0204	0.6711
% 0.005	1	0.1	0.6631	0.0100	0.6531	0.7290	0.0214	0.7076
% 0.005	10	1	0.6417	0.0093	0.6324	0.7222	0.0214	0.7008
% 0.005	1	0.1	0.8140	0.0111	0.8029	0.8218	0.0221	0.7997
% 0.005	1	0.1	0.7890	0.0109	0.7781	0.8048	0.0226	0.7823
% 0.005	10	1	0.6788	0.0091	0.6698	0.7549	0.0227	0.7322
% 0.005	10	1	0.6796	0.0104	0.6693	0.7569	0.0211	0.7359
% 0.005	10	1	0.6715	0.0108	0.6607	0.7484	0.0214	0.7269
% 0.005	10	1	0.6545	0.0092	0.6452	0.7206	0.0207	0.6999
% 0.005	1	0.1	0.6048	0.0082	0.5966	0.7131	0.0212	0.6919
% 0.005	10	1	0.5731	0.0076	0.5655	0.6920	0.0210	0.6710
% 0.005	1	0.1	0.6182	0.0075	0.6107	0.7143	0.0213	0.6930
% 0.005	5	5	0.6408	0.0111	0.6297	0.7163	0.0212	0.6952
% 0.005	1	0.1	0.6442	0.0099	0.6343	0.7336	0.0218	0.7118
% 0.005	10	1	0.6526	0.0103	0.6423	0.7363	0.0211	0.7152
% 0.01	5	5	0.6309	0.0096	0.6212	0.7121	0.0213	0.6908
% 0.01	5	5	0.5713	0.0096	0.5616	0.6801	0.0205	0.6596
% 0.01	5	5	0.6334	0.1043	0.5292	0.7488	0.1044	0.6444
% 0.01	10	1	0.5372	0.0071	0.5301	0.6534	0.0204	0.6331
% 0.1	5	5	0.9923	0.0137	0.9786	0.9725	0.0279	0.9446
\setlength{\tabcolsep}{4pt}
\renewcommand{\arraystretch}{1.3}
\begin{table*}[h]
    \small
    \caption{
    \textit{Scaled validation and test loss for non parametric models of \textbf{Universal Audio 1176LN} limiter.}
    \textit{Bold indicates best performing models.}
    % \textit{Learning rate multiplier for nonlinearity in gray-box models shown in brackets.}
    }
    \label{tab:val-and-test-loss_od_fulltone-fulldrive}
    \centerline{
        \begin{tabular}{l c c cc >{\columncolor{gray!20}}ccc >{\columncolor{gray!20}}ccc}
            \hline
            \midrule
            
            \multirow{2}{*}{Model}
                & \multirow{2}{*}{Params.}
                    & \multirow{2}{*}{LR}
                        & \multicolumn{2}{c}{Weights}
                            & \multicolumn{3}{c}{Val. Loss}
                                & \multicolumn{3}{c}{Test Loss} \\ 
            
            \cmidrule(lr){4-5} 
                \cmidrule(lr){6-8} 
                    \cmidrule(lr){9-11}
            
            &   &   & {\scriptsize L1} & {\scriptsize MR-STFT} & Tot. & {\scriptsize L1} & {\scriptsize MR-STFT} & Tot. & {\scriptsize L1} & {\scriptsize MR-STFT} \\ 
            
            \hline
            LSTM-32 & 4.5k & 0.001 & 10 & 1 & 0.3702 & 0.0103 & 0.3600 & 0.3640 & 0.0087 & 0.3553 \\
            LSTM-96 & 38.1k & 0.001 & 1 & 0.1 & 0.3303 & 0.0074 & 0.3229 & 0.3441 & 0.0080 & 0.3361 \\
            \hline
            TCN-45-S-16 & 7.5k & 0.005 & 10 & 1 & 0.4461 & 0.0134 & 0.4327 & 0.4823 & 0.0146 & 0.4677 \\
            TCN-45-L-16 & 7.3k & 0.005 & 5 & 5 & 0.4581 & 0.0204 & 0.4378 & 0.4901 & 0.0233 & 0.4668 \\
            TCN-250-S-16 & 14.5k & 0.005 & 5 & 5 & 0.4019 & 0.0163 & 0.3856 & 0.4361 & 0.0185 & 0.4175 \\
            TCN-250-L-16 & 18.4k & 0.005 & 10 & 1 & 0.3961 & 0.0115 & 0.3846 & 0.4241 & 0.0121 & 0.4120 \\
            TCN-2500-S-16 & 13.7k & 0.005 & 10 & 1 & 0.4302 & 0.0141 & 0.4162 & 0.4687 & 0.0157 & 0.4530 \\
            TCN-2500-L-16 & 11.9k & 0.005 & 5 & 5 & 0.3975 & 0.0179 & 0.3796 & 0.4407 & 0.0208 & 0.4199 \\
            \hline
            TCN-TF-45-S-16 & 39.5k & 0.005 & 10 & 1 & 0.2846 & 0.0038 & 0.2809 & 0.3090 & 0.0042 & 0.3048 \\
            TCN-TF-45-L-16 & 71.3k & 0.005 & 1 & 0.1 & 0.2469 & 0.0035 & 0.2434 & 0.2968 & 0.0046 & 0.2922 \\
            TCN-TF-250-S-16 & 52.9k & 0.005 & 10 & 1 & 0.2532 & 0.0037 & 0.2494 & 0.2786 & 0.0041 & 0.2745 \\
            TCN-TF-250-L-16 & 88.8k & 0.005 & 1 & 0.1 & 0.2463 & 0.0030 & 0.2433 & 0.2739 & 0.0032 & 0.2707 \\
            TCN-TF-2500-S-16 & 45.7k & 0.005 & 1 & 0.1 & 0.2788 & 0.0048 & 0.2740 & 0.3105 & 0.0051 & 0.3054 \\
            TCN-TF-2500-L-16 & 75.9k & 0.005 & 5 & 5 & 0.2647 & 0.0084 & 0.2563 & 0.2950 & 0.0093 & 0.2857 \\
            \hline
            GCN-45-S-16 & 16.2k & 0.005 & 10 & 1 & 0.4011 & 0.0129 & 0.3882 & 0.4187 & 0.0128 & 0.4059 \\
            GCN-45-L-16 & 17.1k & 0.005 & 10 & 1 & 0.4198 & 0.0108 & 0.4091 & 0.4203 & 0.0133 & 0.4070 \\
            GCN-250-S-16 & 30.4k & 0.005 & 1 & 0.1 & 0.3333 & 0.0092 & 0.3241 & 0.3695 & 0.0096 & 0.3599 \\
            GCN-250-L-16 & 39.6k & 0.005 & 1 & 0.1 & 0.4088 & 0.0112 & 0.3976 & 0.3981 & 0.0140 & 0.3841 \\
            GCN-2500-S-16 & 28.6k & 0.005 & 10 & 1 & 0.2961 & 0.0082 & 0.2879 & 0.3397 & 0.0089 & 0.3308 \\
            GCN-2500-L-16 & 26.4k & 0.005 & 10 & 1 & 0.3505 & 0.0094 & 0.3412 & 0.3583 & 0.0090 & 0.3492 \\
            \hline
            GCN-TF-45-S-16 & 141.6k & 0.005 & 10 & 1 & 0.2131 & 0.0030 & 0.2101 & 0.2444 & 0.0030 & 0.2414 \\
            GCN-TF-45-L-16 & 268.0k & 0.005 & 5 & 5 & 0.2292 & 0.0062 & 0.2230 & 0.2559 & 0.0078 & 0.2481 \\
            GCN-TF-250-S-16 & 181.0k & 0.005 & 1 & 0.1 & 0.2374 & 0.0040 & 0.2334 & 0.2755 & 0.0042 & 0.2713 \\
            GCN-TF-250-L-16 & 315.6k & 0.005 & 1 & 0.1 & 0.2271 & 0.0035 & 0.2236 & 0.2642 & 0.0043 & 0.2599 \\
            GCN-TF-2500-S-16 & 154.1k & 0.005 & 1 & 0.1 & 0.2227 & 0.0030 & 0.2198 & 0.2630 & 0.0031 & 0.2599 \\
            GCN-TF-2500-L-16 & 277.3k & 0.005 & 5 & 5 & 0.2285 & 0.0058 & 0.2227 & 0.2548 & 0.0075 & 0.2473 \\
            \hline
            S4-S-16 & 2.4k & 0.01 & 5 & 5 & 0.3985 & 0.0130 & 0.3855 & 0.4296 & 0.0142 & 0.4153 \\
            S4-L-16 & 19.0k & 0.01 & 10 & 1 & 0.2551 & 0.0062 & 0.2489 & 0.2821 & 0.0061 & 0.2761 \\
            \hline
            S4-TF-S-16 & 28.0k & 0.01 & 1 & 0.1 & 0.2261 & 0.0027 & 0.2234 & 0.2614 & 0.0030 & 0.2583 \\
            S4-TF-L-16 & 70.2k & 0.01 & 10 & 1 & \textbf{0.1937} & 0.0025 & 0.1911 & \textbf{0.2210} & 0.0026 & 0.2183 \\
            \hline
            GB-COMP & 47 & 0.1 & 0.5 & 0.5 & 0.3839 & 0.0107 & 0.3733 & 0.4105 & 0.0100 & 0.4005 \\
            \hline
            \hline
        \end{tabular}
    }
\end{table*}

% 0.001	10	1	0.3702	0.0103	0.3600	0.3640	0.0087	0.3553
% 0.001	1	0.1	0.3303	0.0074	0.3229	0.3441	0.0080	0.3361
% 0.005	10	1	0.4461	0.0134	0.4327	0.4823	0.0146	0.4677
% 0.005	5	5	0.4581	0.0204	0.4378	0.4901	0.0233	0.4668
% 0.005	5	5	0.4019	0.0163	0.3856	0.4361	0.0185	0.4175
% 0.005	10	1	0.3961	0.0115	0.3846	0.4241	0.0121	0.4120
% 0.005	10	1	0.4302	0.0141	0.4162	0.4687	0.0157	0.4530
% 0.005	5	5	0.3975	0.0179	0.3796	0.4407	0.0208	0.4199
% 0.005	10	1	0.2846	0.0038	0.2809	0.3090	0.0042	0.3048
% 0.005	1	0.1	0.2469	0.0035	0.2434	0.2968	0.0046	0.2922
% 0.005	10	1	0.2532	0.0037	0.2494	0.2786	0.0041	0.2745
% 0.005	1	0.1	0.2463	0.0030	0.2433	0.2739	0.0032	0.2707
% 0.005	1	0.1	0.2788	0.0048	0.2740	0.3105	0.0051	0.3054
% 0.005	5	5	0.2647	0.0084	0.2563	0.2950	0.0093	0.2857
% 0.005	10	1	0.4011	0.0129	0.3882	0.4187	0.0128	0.4059
% 0.005	10	1	0.4198	0.0108	0.4091	0.4203	0.0133	0.4070
% 0.005	1	0.1	0.3333	0.0092	0.3241	0.3695	0.0096	0.3599
% 0.005	1	0.1	0.4088	0.0112	0.3976	0.3981	0.0140	0.3841
% 0.005	10	1	0.2961	0.0082	0.2879	0.3397	0.0089	0.3308
% 0.005	10	1	0.3505	0.0094	0.3412	0.3583	0.0090	0.3492
% 0.005	10	1	0.2131	0.0030	0.2101	0.2444	0.0030	0.2414
% 0.005	5	5	0.2292	0.0062	0.2230	0.2559	0.0078	0.2481
% 0.005	1	0.1	0.2374	0.0040	0.2334	0.2755	0.0042	0.2713
% 0.005	1	0.1	0.2271	0.0035	0.2236	0.2642	0.0043	0.2599
% 0.005	1	0.1	0.2227	0.0030	0.2198	0.2630	0.0031	0.2599
% 0.005	5	5	0.2285	0.0058	0.2227	0.2548	0.0075	0.2473
% 0.01	5	5	0.3985	0.0130	0.3855	0.4296	0.0142	0.4153
% 0.01	10	1	0.2551	0.0062	0.2489	0.2821	0.0061	0.2761
% 0.01	1	0.1	0.2261	0.0027	0.2234	0.2614	0.0030	0.2583
% 0.01	10	1	0.1937	0.0025	0.1911	0.2210	0.0026	0.2183
% 0.1	0.5	0.5	0.3839	0.0107	0.3733	0.4105	0.0100	0.4005

\setlength{\tabcolsep}{4pt}
\renewcommand{\arraystretch}{1.3}
\begin{table*}[h]
    \small
    \caption{
    \textit{Objective metrics for non parametric models of \textbf{Ampeg OptoComp} compressor.}
    \textit{Bold indicates best performing models.}
    % \textit{Learning rate multiplier for nonlinearity in gray-box models shown in brackets.}
    }
    \label{tab:metrics_od_diy-klon-centaur}
    \centerline{
        \begin{tabular}{lccccccccccccc}
            \hline
            \midrule
            
            \multirow{2}{*}{Model}
                & \multirow{2}{*}{Params.}
                    & \multirow{2}{*}{LR}
                        & \multicolumn{2}{c}{Weights}
                            & \multicolumn{3}{c}{}
                                & \multicolumn{3}{c}{FAD} \\ 
            \cmidrule(lr){4-5} 
                % \cmidrule(lr){6-8} 
                    \cmidrule(lr){9-12}
            
            &   &   & L1 & MR-STFT & MSE & ESR & MAPE & VGGish & PANN & CLAP & AFx-Rep \\ 
            \hline
            LSTM-32 & 4.5k & 0.001 & 1 & 0.1 & 2.72e-03 & 4.0779 & 2.5266 & 0.0046 & 4.21e-09 & 0.0068 & 0.0082 \\
            LSTM-96 & 38.1k & 0.005 & 10 & 1 & 2.67e-06 & 0.0052 & 1.5669 & 0.0052 & 2.19e-07 & 0.0070 & 0.0082 \\
            \hline
            TCN-45-S-16 & 7.5k & 0.005 & 1 & 0.1 & 1.31e-05 & 0.0288 & 1.5961 & 0.0536 & 8.43e-06 & 0.0103 & 0.0746 \\
            TCN-45-L-16 & 7.3k & 0.005 & 10 & 1 & 1.73e-05 & 0.0373 & 3.1260 & 0.1518 & 4.87e-05 & 0.0198 & 0.1553 \\
            TCN-250-S-16 & 14.5k & 0.005 & 5 & 5 & 2.72e-03 & 3.9334 & 4.1025 & 0.1730 & 3.28e-05 & 0.0227 & 0.1805 \\
            TCN-250-L-16 & 18.4k & 0.005 & 1 & 0.1 & 1.83e-05 & 0.0388 & 1.4565 & 0.1727 & 4.84e-05 & 0.0201 & 0.1694 \\
            TCN-2500-S-16 & 13.7k & 0.005 & 1 & 0.1 & 2.06e-05 & 0.0442 & 1.2756 & 0.1919 & 6.49e-05 & 0.0239 & 0.1964 \\
            TCN-2500-L-16 & 11.9k & 0.005 & 0.5 & 0.5 & 2.72e-03 & 3.9934 & 3.4717 & 0.1500 & 3.31e-05 & 0.0219 & 0.1366 \\
            \hline
            TCN-TF-45-S-16 & 39.5k & 0.005 & 10 & 1 & 1.81e-06 & 0.0037 & 0.4155 & 0.0049 & 1.20e-08 & 0.0067 & 0.0061 \\
            TCN-TF-45-L-16 & 71.3k & 0.005 & 10 & 1 & 2.13e-06 & 0.0040 & 0.6163 & 0.0053 & 3.32e-09 & 0.0057 & 0.0051 \\
            TCN-TF-250-S-16 & 52.9k & 0.005 & 10 & 1 & 2.68e-03 & 4.0072 & 2.4095 & 0.0042 & 1.05e-07 & 0.0064 & 0.0049 \\
            TCN-TF-250-L-16 & 88.8k & 0.005 & 10 & 1 & 1.87e-06 & 0.0037 & 0.4079 & 0.0054 & 1.73e-07 & 0.0064 & 0.0056 \\
            TCN-TF-2500-S-16 & 45.7k & 0.005 & 5 & 5 & 2.69e-03 & 4.0316 & 2.6948 & 0.0118 & 2.45e-06 & 0.0059 & 0.0066 \\
            TCN-TF-2500-L-16 & 75.9k & 0.005 & 5 & 5 & 2.69e-03 & 4.0319 & 2.5622 & 0.0070 & 5.72e-08 & 0.0050 & 0.0048 \\
            \hline
            GCN-45-S-16 & 16.2k & 0.005 & 5 & 5 & 2.69e-03 & 3.9597 & 2.8336 & 0.0486 & 1.71e-09 & 0.0120 & 0.0822 \\
            GCN-45-L-16 & 17.1k & 0.005 & 0.5 & 0.5 & 2.69e-03 & 3.9538 & 3.1831 & 0.0635 & 1.27e-06 & 0.0116 & 0.0851 \\
            GCN-250-S-16 & 30.4k & 0.005 & 5 & 5 & 2.70e-03 & 4.0120 & 2.9786 & 0.0170 & 3.53e-07 & 0.0080 & 0.0239 \\
            GCN-250-L-16 & 39.6k & 0.005 & 10 & 1 & 1.14e-05 & 0.0220 & 2.3755 & 0.0496 & 4.30e-07 & 0.0097 & 0.0421 \\
            GCN-2500-S-16 & 28.6k & 0.005 & 1 & 0.1 & 2.65e-03 & 3.9405 & 2.9564 & 0.0166 & 4.46e-06 & 0.0076 & 0.0208 \\
            GCN-2500-L-16 & 26.4k & 0.005 & 10 & 1 & 5.71e-06 & 0.0117 & 0.7966 & 0.0175 & 2.02e-06 & 0.0063 & 0.0144 \\
            \hline
            GCN-TF-45-S-16 & 141.6k & 0.005 & 1 & 0.1 & 1.56e-06 & 0.0030 & \textbf{0.3327} & 0.0037 & 2.37e-08 & 0.0068 & 0.0052 \\
            GCN-TF-45-L-16 & 268.0k & 0.005 & 1 & 0.1 & 2.24e-06 & 0.0042 & 0.3917 & 0.0045 & 5.42e-09 & 0.0060 & \textbf{0.0048} \\
            GCN-TF-250-S-16 & 181.0k & 0.005 & 0.5 & 0.5 & 4.97e-06 & 0.0086 & 0.6895 & 0.0038 & 3.86e-07 & 0.0073 & 0.0056 \\
            GCN-TF-250-L-16 & 315.6k & 0.005 & 10 & 1 & 2.22e-06 & 0.0041 & 0.3898 & \textbf{0.0035} & 4.70e-09 & 0.0061 & 0.0048 \\
            GCN-TF-2500-S-16 & 154.1k & 0.005 & 0.5 & 0.5 & 2.70e-03 & 4.0415 & 2.5044 & 0.0047 & 5.02e-07 & 0.0049 & 0.0070 \\
            GCN-TF-2500-L-16 & 277.3k & 0.005 & 0.5 & 0.5 & 2.70e-03 & 4.0372 & 2.3945 & 0.0043 & \textbf{6.33e-10} & 0.0048 & 0.0056 \\
            \hline
            S4-S-16 & 2.4k & 0.01 & 1 & 0.1 & 4.45e-06 & 0.0109 & 1.0636 & 0.0158 & 4.12e-09 & 0.0065 & 0.0092 \\
            S4-L-16 & 19.0k & 0.01 & 1 & 0.1 & 2.70e-06 & 0.0062 & 1.3359 & 0.0067 & 7.24e-08 & 0.0053 & 0.0058 \\
            \hline
            S4-TF-S-16 & 28.0k & 0.01 & 1 & 0.1 & 2.69e-03 & 4.0326 & 2.1828 & 0.0039 & 5.34e-08 & 0.0057 & 0.0050 \\
            S4-TF-L-16 & 70.2k & 0.01 & 10 & 1 & \textbf{1.18e-06} & \textbf{0.0022} & 0.4089 & 0.0047 & 3.12e-09 & \textbf{0.0046} & 0.0054 \\
            \hline
            GB-COMP & 47 & 0.1 & 0.5 & 0.5 & 1.17e-05 & 0.0256 & 1.0919 & 0.0077 & 5.73e-06 & 0.0068 & 0.0185 \\
            \hline
            \hline
        \end{tabular}
    }
\end{table*}

% 0.001	1	0.1	2.72e-03	4.0779	2.5266	0.0046	4.21e-09	0.0068	0.0082
% 0.005	10	1	2.67e-06	0.0052	1.5669	0.0052	2.19e-07	0.0070	0.0082
% 0.005	1	0.1	1.31e-05	0.0288	1.5961	0.0536	8.43e-06	0.0103	0.0746
% 0.005	10	1	1.73e-05	0.0373	3.1260	0.1518	4.87e-05	0.0198	0.1553
% 0.005	5	5	2.72e-03	3.9334	4.1025	0.1730	3.28e-05	0.0227	0.1805
% 0.005	1	0.1	1.83e-05	0.0388	1.4565	0.1727	4.84e-05	0.0201	0.1694
% 0.005	1	0.1	2.06e-05	0.0442	1.2756	0.1919	6.49e-05	0.0239	0.1964
% 0.005	0.5	0.5	2.72e-03	3.9934	3.4717	0.1500	3.31e-05	0.0219	0.1366
% 0.005	10	1	1.81e-06	0.0037	0.4155	0.0049	1.20e-08	0.0067	0.0061
% 0.005	10	1	2.13e-06	0.0040	0.6163	0.0053	3.32e-09	0.0057	0.0051
% 0.005	10	1	2.68e-03	4.0072	2.4095	0.0042	1.05e-07	0.0064	0.0049
% 0.005	10	1	1.87e-06	0.0037	0.4079	0.0054	1.73e-07	0.0064	0.0056
% 0.005	5	5	2.69e-03	4.0316	2.6948	0.0118	2.45e-06	0.0059	0.0066
% 0.005	5	5	2.69e-03	4.0319	2.5622	0.0070	5.72e-08	0.0050	0.0048
% 0.005	5	5	2.69e-03	3.9597	2.8336	0.0486	1.71e-09	0.0120	0.0822
% 0.005	0.5	0.5	2.69e-03	3.9538	3.1831	0.0635	1.27e-06	0.0116	0.0851
% 0.005	5	5	2.70e-03	4.0120	2.9786	0.0170	3.53e-07	0.0080	0.0239
% 0.005	10	1	1.14e-05	0.0220	2.3755	0.0496	4.30e-07	0.0097	0.0421
% 0.005	1	0.1	2.65e-03	3.9405	2.9564	0.0166	4.46e-06	0.0076	0.0208
% 0.005	10	1	5.71e-06	0.0117	0.7966	0.0175	2.02e-06	0.0063	0.0144
% 0.005	1	0.1	1.56e-06	0.0030	0.3327	0.0037	2.37e-08	0.0068	0.0052
% 0.005	1	0.1	2.24e-06	0.0042	0.3917	0.0045	5.42e-09	0.0060	0.0048
% 0.005	0.5	0.5	4.97e-06	0.0086	0.6895	0.0038	3.86e-07	0.0073	0.0056
% 0.005	10	1	2.22e-06	0.0041	0.3898	0.0035	4.70e-09	0.0061	0.0048
% 0.005	0.5	0.5	2.70e-03	4.0415	2.5044	0.0047	5.02e-07	0.0049	0.0070
% 0.005	0.5	0.5	2.70e-03	4.0372	2.3945	0.0043	6.33e-10	0.0048	0.0056
% 0.01	1	0.1	4.45e-06	0.0109	1.0636	0.0158	4.12e-09	0.0065	0.0092
% 0.01	1	0.1	2.70e-06	0.0062	1.3359	0.0067	7.24e-08	0.0053	0.0058
% 0.01	1	0.1	2.69e-03	4.0326	2.1828	0.0039	5.34e-08	0.0057	0.0050
% 0.01	10	1	1.18e-06	0.0022	0.4089	0.0047	3.12e-09	0.0046	0.0054
% 0.1	0.5	0.5	1.17e-05	0.0256	1.0919	0.0077	5.73e-06	0.0068	0.0185
\setlength{\tabcolsep}{4pt}
\renewcommand{\arraystretch}{1.3}
\begin{table*}[h]
    \small
    \caption{
    \textit{Objective metrics for non parametric models of \textbf{Flamma AnalogComp} compressor.}
    \textit{Bold indicates best performing models.}
    % \textit{Learning rate multiplier for nonlinearity in gray-box models shown in brackets.}
    }
    \label{tab:metrics_od_diy-klon-centaur}
    \centerline{
        \begin{tabular}{lccccccccccccc}
            \hline
            \midrule
            
            \multirow{2}{*}{Model}
                & \multirow{2}{*}{Params.}
                    & \multirow{2}{*}{LR}
                        & \multicolumn{2}{c}{Weights}
                            & \multicolumn{3}{c}{}
                                & \multicolumn{3}{c}{FAD} \\ 
            \cmidrule(lr){4-5} 
                % \cmidrule(lr){6-8} 
                    \cmidrule(lr){9-12}
            
            &   &   & L1 & MR-STFT & MSE & ESR & MAPE & VGGish & PANN & CLAP & AFx-Rep \\ 
            \hline
            LSTM-32 & 4.5k & 0.005 & 10 & 1 & 6.11e-05 & 0.0041 & 0.6271 & 0.0620 & 3.15e-06 & 0.0085 & 0.0151 \\
            LSTM-96 & 38.1k & 0.005 & 10 & 1 & 2.76e-05 & 0.0021 & 1.2703 & 0.0303 & 4.37e-07 & 0.0078 & 0.0132 \\
            \hline
            TCN-45-S-16 & 7.5k & 0.005 & 10 & 1 & 2.00e-04 & 0.0145 & 1.1282 & 0.5163 & 1.03e-04 & 0.0351 & 0.2174 \\
            TCN-45-L-16 & 7.3k & 0.005 & 1 & 0.1 & 1.77e-04 & 0.0125 & 1.3288 & 0.3382 & 5.45e-05 & 0.0267 & 0.1848 \\
            TCN-250-S-16 & 14.5k & 0.005 & 1 & 0.1 & 1.97e-04 & 0.0142 & 0.5991 & 0.4504 & 7.37e-05 & 0.0332 & 0.2065 \\
            TCN-250-L-16 & 18.4k & 0.005 & 10 & 1 & 1.11e-04 & 0.0077 & 1.9784 & 0.1169 & 3.43e-06 & 0.0135 & 0.0685 \\
            TCN-2500-S-16 & 13.7k & 0.005 & 0.5 & 0.5 & 2.73e-04 & 0.0206 & 1.2972 & 0.5588 & 1.04e-04 & 0.0354 & 0.2222 \\
            TCN-2500-L-16 & 11.9k & 0.005 & 0.5 & 0.5 & 2.75e-04 & 0.0216 & 3.4443 & 0.3600 & 5.73e-05 & 0.0250 & 0.1563 \\
            \hline
            TCN-TF-45-S-16 & 39.5k & 0.005 & 5 & 5 & 7.66e-05 & 0.0055 & 1.0900 & 0.1093 & 4.43e-07 & 0.0074 & 0.0426 \\
            TCN-TF-45-L-16 & 71.3k & 0.005 & 0.5 & 0.5 & 4.54e-05 & 0.0036 & 0.7004 & 0.0303 & 6.09e-07 & 0.0060 & 0.0205 \\
            TCN-TF-250-S-16 & 52.9k & 0.005 & 0.5 & 0.5 & 5.37e-05 & 0.0041 & 0.7663 & 0.0901 & 7.17e-07 & 0.0072 & 0.0302 \\
            TCN-TF-250-L-16 & 88.8k & 0.005 & 0.5 & 0.5 & 1.53e-04 & 0.0134 & 1.4655 & 0.0303 & 4.72e-07 & 0.0058 & 0.0223 \\
            TCN-TF-2500-S-16 & 45.7k & 0.005 & 10 & 1 & 7.57e-05 & 0.0050 & 1.6560 & 0.2033 & 2.35e-06 & 0.0096 & 0.0441 \\
            TCN-TF-2500-L-16 & 75.9k & 0.005 & 1 & 0.1 & 4.36e-05 & 0.0031 & 1.5167 & 0.0896 & 6.05e-06 & 0.0089 & 0.0320 \\
            \hline
            GCN-45-S-16 & 16.2k & 0.005 & 1 & 0.1 & 7.37e-05 & 0.0052 & 0.6256 & 0.0797 & 6.26e-06 & 0.0113 & 0.0415 \\
            GCN-45-L-16 & 17.1k & 0.005 & 10 & 1 & 6.32e-05 & 0.0044 & 0.4764 & 0.0510 & 3.43e-08 & 0.0116 & 0.0347 \\
            GCN-250-S-16 & 30.4k & 0.005 & 10 & 1 & 4.94e-05 & 0.0035 & 1.1553 & 0.0430 & 3.34e-07 & 0.0092 & 0.0251 \\
            GCN-250-L-16 & 39.6k & 0.005 & 10 & 1 & 5.46e-05 & 0.0039 & 1.4670 & 0.0358 & 2.81e-06 & 0.0088 & 0.0150 \\
            GCN-2500-S-16 & 28.6k & 0.005 & 1 & 0.1 & 6.81e-05 & 0.0050 & 0.4872 & 0.0394 & 5.87e-06 & 0.0047 & 0.0161 \\
            GCN-2500-L-16 & 26.4k & 0.005 & 10 & 1 & 3.95e-05 & 0.0029 & 0.4672 & 0.0346 & 7.07e-06 & 0.0046 & 0.0127 \\
            \hline
            GCN-TF-45-S-16 & 141.6k & 0.005 & 1 & 0.1 & 2.47e-05 & 0.0017 & 1.0051 & 0.1070 & 1.87e-06 & 0.0082 & 0.0311 \\
            GCN-TF-45-L-16 & 268.0k & 0.005 & 10 & 1 & 3.81e-05 & 0.0027 & 0.9231 & 0.0336 & 3.92e-07 & 0.0055 & 0.0214 \\
            GCN-TF-250-S-16 & 181.0k & 0.005 & 10 & 1 & 3.12e-05 & 0.0022 & 0.5085 & 0.0618 & \textbf{1.04e-08} & 0.0065 & 0.0323 \\
            GCN-TF-250-L-16 & 315.6k & 0.005 & 0.5 & 0.5 & 4.78e-05 & 0.0039 & 0.6216 & 0.0731 & 1.89e-06 & 0.0052 & 0.0244 \\
            GCN-TF-2500-S-16 & 154.1k & 0.005 & 10 & 1 & 3.70e-05 & 0.0025 & 0.4120 & 0.0380 & 9.65e-07 & 0.0046 & 0.0235 \\
            GCN-TF-2500-L-16 & 277.3k & 0.005 & 1 & 0.1 & 2.45e-05 & 0.0017 & 0.3605 & 0.0193 & 1.93e-06 & 0.0040 & 0.0099 \\
            \hline
            S4-S-16 & 2.4k & 0.01 & 0.5 & 0.5 & 1.35e-04 & 0.0101 & 1.0429 & 0.0734 & 6.76e-06 & 0.0095 & 0.0409 \\
            S4-L-16 & 19.0k & 0.01 & 1 & 0.1 & 4.33e-05 & 0.0030 & 2.0815 & \textbf{0.0177} & 4.41e-06 & \textbf{0.0020} & \textbf{0.0073} \\
            \hline
            S4-TF-S-16 & 28.0k & 0.01 & 10 & 1 & 3.25e-05 & 0.0023 & \textbf{0.2862} & 0.0302 & 2.02e-07 & 0.0045 & 0.0194 \\
            S4-TF-L-16 & 70.2k & 0.01 & 10 & 1 & \textbf{2.35e-05} & \textbf{0.0017} & 0.3077 & 0.0188 & 5.28e-07 & 0.0041 & 0.0135 \\
            \hline
            GB-COMP & 47 & 0.1 & 5 & 5 & 1.37e-03 & 0.0954 & 2.7457 & 0.2296 & 6.64e-07 & 0.0169 & 0.1098 \\
            \hline
            \hline
        \end{tabular}
    }
\end{table*}

% 0.005	10	1	6.11e-05	0.0041	0.6271	0.0620	3.15e-06	0.0085	0.0151
% 0.005	10	1	2.76e-05	0.0021	1.2703	0.0303	4.37e-07	0.0078	0.0132
% 0.005	10	1	2.00e-04	0.0145	1.1282	0.5163	1.03e-04	0.0351	0.2174
% 0.005	1	0.1	1.77e-04	0.0125	1.3288	0.3382	5.45e-05	0.0267	0.1848
% 0.005	1	0.1	1.97e-04	0.0142	0.5991	0.4504	7.37e-05	0.0332	0.2065
% 0.005	10	1	1.11e-04	0.0077	1.9784	0.1169	3.43e-06	0.0135	0.0685
% 0.005	0.5	0.5	2.73e-04	0.0206	1.2972	0.5588	1.04e-04	0.0354	0.2222
% 0.005	0.5	0.5	2.75e-04	0.0216	3.4443	0.3600	5.73e-05	0.0250	0.1563
% 0.005	5	5	7.66e-05	0.0055	1.0900	0.1093	4.43e-07	0.0074	0.0426
% 0.005	0.5	0.5	4.54e-05	0.0036	0.7004	0.0303	6.09e-07	0.0060	0.0205
% 0.005	0.5	0.5	5.37e-05	0.0041	0.7663	0.0901	7.17e-07	0.0072	0.0302
% 0.005	0.5	0.5	1.53e-04	0.0134	1.4655	0.0303	4.72e-07	0.0058	0.0223
% 0.005	10	1	7.57e-05	0.0050	1.6560	0.2033	2.35e-06	0.0096	0.0441
% 0.005	1	0.1	4.36e-05	0.0031	1.5167	0.0896	6.05e-06	0.0089	0.0320
% 0.005	1	0.1	7.37e-05	0.0052	0.6256	0.0797	6.26e-06	0.0113	0.0415
% 0.005	10	1	6.32e-05	0.0044	0.4764	0.0510	3.43e-08	0.0116	0.0347
% 0.005	10	1	4.94e-05	0.0035	1.1553	0.0430	3.34e-07	0.0092	0.0251
% 0.005	10	1	5.46e-05	0.0039	1.4670	0.0358	2.81e-06	0.0088	0.0150
% 0.005	1	0.1	6.81e-05	0.0050	0.4872	0.0394	5.87e-06	0.0047	0.0161
% 0.005	10	1	3.95e-05	0.0029	0.4672	0.0346	7.07e-06	0.0046	0.0127
% 0.005	1	0.1	2.47e-05	0.0017	1.0051	0.1070	1.87e-06	0.0082	0.0311
% 0.005	10	1	3.81e-05	0.0027	0.9231	0.0336	3.92e-07	0.0055	0.0214
% 0.005	10	1	3.12e-05	0.0022	0.5085	0.0618	1.04e-08	0.0065	0.0323
% 0.005	0.5	0.5	4.78e-05	0.0039	0.6216	0.0731	1.89e-06	0.0052	0.0244
% 0.005	10	1	3.70e-05	0.0025	0.4120	0.0380	9.65e-07	0.0046	0.0235
% 0.005	1	0.1	2.45e-05	0.0017	0.3605	0.0193	1.93e-06	0.0040	0.0099
% 0.01	0.5	0.5	1.35e-04	0.0101	1.0429	0.0734	6.76e-06	0.0095	0.0409
% 0.01	1	0.1	4.33e-05	0.0030	2.0815	0.0177	4.41e-06	0.0020	0.0073
% 0.01	10	1	3.25e-05	0.0023	0.2862	0.0302	2.02e-07	0.0045	0.0194
% 0.01	10	1	2.35e-05	0.0017	0.3077	0.0188	5.28e-07	0.0041	0.0135
% 0.1	5	5	1.37e-03	0.0954	2.7457	0.2296	6.64e-07	0.0169	0.1098
\setlength{\tabcolsep}{4pt}
\renewcommand{\arraystretch}{1.3}
\begin{table*}[h]
    \small
    \caption{
    \textit{Objective metrics for non parametric models of \textbf{Yuer DynaComp} compressor.}
    \textit{Bold indicates best performing models.}
    % \textit{Learning rate multiplier for nonlinearity in gray-box models shown in brackets.}
    }
    \label{tab:metrics_od_diy-klon-centaur}
    \centerline{
        \begin{tabular}{lccccccccccccc}
            \hline
            \midrule
            
            \multirow{2}{*}{Model}
                & \multirow{2}{*}{Params.}
                    & \multirow{2}{*}{LR}
                        & \multicolumn{2}{c}{Weights}
                            & \multicolumn{3}{c}{}
                                & \multicolumn{3}{c}{FAD} \\ 
            \cmidrule(lr){4-5} 
                % \cmidrule(lr){6-8} 
                    \cmidrule(lr){9-12}
            
            &   &   & L1 & MR-STFT & MSE & ESR & MAPE & VGGish & PANN & CLAP & AFx-Rep \\ 
            \hline
            LSTM-32 & 4.5k & 0.005 & 10 & 1 & 2.77e-03 & 0.5049 & 3.5045 & 0.1909 & 5.72e-06 & 0.0234 & 0.0238 \\
            LSTM-96 & 38.1k & 0.001 & 1 & 0.1 & 2.80e-03 & 0.5107 & 3.9581 & 0.2057 & 3.61e-06 & 0.0235 & 0.0308 \\
            \hline
            TCN-45-S-16 & 7.5k & 0.005 & 1 & 0.1 & 2.92e-03 & 0.5142 & 3.6166 & 1.1084 & 2.35e-04 & 0.0755 & 0.2505 \\
            TCN-45-L-16 & 7.3k & 0.005 & 5 & 5 & 3.02e-03 & 0.5290 & 6.2789 & 0.2674 & 6.71e-05 & 0.0306 & 0.1082 \\
            TCN-250-S-16 & 14.5k & 0.005 & 10 & 1 & 2.83e-03 & 0.5022 & 3.1373 & 0.7767 & 2.16e-04 & 0.0545 & 0.1500 \\
            TCN-250-L-16 & 18.4k & 0.005 & 1 & 0.1 & 2.81e-03 & 0.5022 & 3.7181 & 0.3657 & 7.80e-05 & 0.0362 & 0.0870 \\
            TCN-2500-S-16 & 13.7k & 0.005 & 1 & 0.1 & 2.90e-03 & 0.5150 & 3.4373 & 0.3080 & 8.77e-05 & 0.0232 & 0.0593 \\
            TCN-2500-L-16 & 11.9k & 0.005 & 10 & 1 & 2.85e-03 & 0.5113 & 3.2285 & 0.1675 & 3.50e-05 & 0.0131 & 0.0318 \\
            \hline
            TCN-TF-45-S-16 & 39.5k & 0.005 & 10 & 1 & 2.76e-03 & 0.5013 & 3.1074 & 0.1491 & 1.31e-05 & 0.0160 & 0.0192 \\
            TCN-TF-45-L-16 & 71.3k & 0.005 & 0.5 & 0.5 & 2.91e-03 & 0.5240 & 3.7697 & 0.1822 & 4.41e-06 & 0.0181 & 0.0238 \\
            TCN-TF-250-S-16 & 52.9k & 0.005 & 1 & 0.1 & 2.81e-03 & 0.5102 & 3.1086 & 0.1574 & 2.00e-05 & 0.0219 & 0.0277 \\
            TCN-TF-250-L-16 & 88.8k & 0.005 & 5 & 5 & 2.75e-03 & 0.4998 & 2.8870 & 0.1731 & 1.34e-05 & 0.0194 & 0.0189 \\
            TCN-TF-2500-S-16 & 45.7k & 0.005 & 1 & 0.1 & \textbf{2.72e-03} & \textbf{0.4928} & 3.5306 & 0.2158 & 8.90e-06 & 0.0149 & 0.0208 \\
            TCN-TF-2500-L-16 & 75.9k & 0.005 & 10 & 1 & 2.75e-03 & 0.4975 & 3.1162 & 0.1772 & 5.67e-06 & 0.0143 & 0.0229 \\
            \hline
            GCN-45-S-16 & 16.2k & 0.005 & 1 & 0.1 & 2.84e-03 & 0.5087 & 3.1357 & 0.2140 & \textbf{3.85e-07} & 0.0266 & 0.0495 \\
            GCN-45-L-16 & 17.1k & 0.005 & 1 & 0.1 & 2.88e-03 & 0.5157 & 3.3112 & 0.2161 & 4.80e-07 & 0.0248 & 0.0409 \\
            GCN-250-S-16 & 30.4k & 0.005 & 10 & 1 & 2.78e-03 & 0.5051 & 9.9088 & 0.1298 & 6.88e-06 & 0.0188 & 0.0237 \\
            GCN-250-L-16 & 39.6k & 0.005 & 10 & 1 & 2.76e-03 & 0.4991 & 3.5485 & 0.1663 & 2.13e-05 & 0.0178 & 0.0263 \\
            GCN-2500-S-16 & 28.6k & 0.005 & 10 & 1 & 2.81e-03 & 0.5078 & 2.9546 & 0.1478 & 6.36e-06 & \textbf{0.0097} & 0.0230 \\
            GCN-2500-L-16 & 26.4k & 0.005 & 10 & 1 & 2.77e-03 & 0.5035 & 3.0217 & \textbf{0.0931} & 2.79e-06 & 0.0107 & 0.0178 \\
            \hline
            GCN-TF-45-S-16 & 141.6k & 0.005 & 1 & 0.1 & 2.78e-03 & 0.5044 & 3.1127 & 0.1987 & 2.43e-05 & 0.0151 & 0.0259 \\
            GCN-TF-45-L-16 & 268.0k & 0.005 & 10 & 1 & 2.75e-03 & 0.5003 & 3.0603 & 0.2021 & 7.21e-06 & 0.0186 & 0.0187 \\
            GCN-TF-250-S-16 & 181.0k & 0.005 & 1 & 0.1 & 2.77e-03 & 0.5038 & 3.1035 & 0.2323 & 1.63e-05 & 0.0193 & 0.0299 \\
            GCN-TF-250-L-16 & 315.6k & 0.005 & 5 & 5 & 2.77e-03 & 0.5023 & 3.0060 & 0.2340 & 7.38e-06 & 0.0198 & 0.0261 \\
            GCN-TF-2500-S-16 & 154.1k & 0.005 & 1 & 0.1 & 2.76e-03 & 0.4992 & 3.1722 & 0.1872 & 4.05e-06 & 0.0143 & 0.0277 \\
            GCN-TF-2500-L-16 & 277.3k & 0.005 & 10 & 1 & 2.76e-03 & 0.5002 & 3.1549 & 0.1598 & 1.12e-05 & 0.0136 & 0.0240 \\
            \hline
            S4-S-16 & 2.4k & 0.01 & 5 & 5 & 2.80e-03 & 0.5066 & 3.0662 & 0.1657 & 8.36e-06 & 0.0182 & 0.0263 \\
            S4-L-16 & 19.0k & 0.01 & 5 & 5 & 2.75e-03 & 0.5004 & 2.8643 & 0.2143 & 6.60e-06 & 0.0188 & 0.0124 \\
            \hline
            S4-TF-S-16 & 28.0k & 0.01 & 5 & 5 & 2.62e-02 & 3.4814 & 4.0847 & 0.1099 & 8.28e-06 & 0.0193 & \textbf{0.0118} \\
            S4-TF-L-16 & 70.2k & 0.01 & 10 & 1 & 2.72e-03 & 0.4961 & \textbf{2.8315} & 0.1056 & 1.60e-05 & 0.0175 & 0.0137 \\
            \hline
            GB-COMP & 47 & 0.1 & 5 & 5 & 3.44e-03 & 0.5953 & 3.7422 & 0.4163 & 6.29e-05 & 0.0382 & 0.1934 \\
            \hline
            \hline
        \end{tabular}
    }
\end{table*}

% 0.005	10	1	2.77e-03	0.5049	3.5045	0.1909	5.72e-06	0.0234	0.0238
% 0.001	1	0.1	2.80e-03	0.5107	3.9581	0.2057	3.61e-06	0.0235	0.0308
% 0.005	1	0.1	2.92e-03	0.5142	3.6166	1.1084	2.35e-04	0.0755	0.2505
% 0.005	5	5	3.02e-03	0.5290	6.2789	0.2674	6.71e-05	0.0306	0.1082
% 0.005	10	1	2.83e-03	0.5022	3.1373	0.7767	2.16e-04	0.0545	0.1500
% 0.005	1	0.1	2.81e-03	0.5022	3.7181	0.3657	7.80e-05	0.0362	0.0870
% 0.005	1	0.1	2.90e-03	0.5150	3.4373	0.3080	8.77e-05	0.0232	0.0593
% 0.005	10	1	2.85e-03	0.5113	3.2285	0.1675	3.50e-05	0.0131	0.0318
% 0.005	10	1	2.76e-03	0.5013	3.1074	0.1491	1.31e-05	0.0160	0.0192
% 0.005	0.5	0.5	2.91e-03	0.5240	3.7697	0.1822	4.41e-06	0.0181	0.0238
% 0.005	1	0.1	2.81e-03	0.5102	3.1086	0.1574	2.00e-05	0.0219	0.0277
% 0.005	5	5	2.75e-03	0.4998	2.8870	0.1731	1.34e-05	0.0194	0.0189
% 0.005	1	0.1	2.72e-03	0.4928	3.5306	0.2158	8.90e-06	0.0149	0.0208
% 0.005	10	1	2.75e-03	0.4975	3.1162	0.1772	5.67e-06	0.0143	0.0229
% 0.005	1	0.1	2.84e-03	0.5087	3.1357	0.2140	3.85e-07	0.0266	0.0495
% 0.005	1	0.1	2.88e-03	0.5157	3.3112	0.2161	4.80e-07	0.0248	0.0409
% 0.005	10	1	2.78e-03	0.5051	9.9088	0.1298	6.88e-06	0.0188	0.0237
% 0.005	10	1	2.76e-03	0.4991	3.5485	0.1663	2.13e-05	0.0178	0.0263
% 0.005	10	1	2.81e-03	0.5078	2.9546	0.1478	6.36e-06	0.0097	0.0230
% 0.005	10	1	2.77e-03	0.5035	3.0217	0.0931	2.79e-06	0.0107	0.0178
% 0.005	1	0.1	2.78e-03	0.5044	3.1127	0.1987	2.43e-05	0.0151	0.0259
% 0.005	10	1	2.75e-03	0.5003	3.0603	0.2021	7.21e-06	0.0186	0.0187
% 0.005	1	0.1	2.77e-03	0.5038	3.1035	0.2323	1.63e-05	0.0193	0.0299
% 0.005	5	5	2.77e-03	0.5023	3.0060	0.2340	7.38e-06	0.0198	0.0261
% 0.005	1	0.1	2.76e-03	0.4992	3.1722	0.1872	4.05e-06	0.0143	0.0277
% 0.005	10	1	2.76e-03	0.5002	3.1549	0.1598	1.12e-05	0.0136	0.0240
% 0.01	5	5	2.80e-03	0.5066	3.0662	0.1657	8.36e-06	0.0182	0.0263
% 0.01	5	5	2.75e-03	0.5004	2.8643	0.2143	6.60e-06	0.0188	0.0124
% 0.01	5	5	2.62e-02	3.4814	4.0847	0.1099	8.28e-06	0.0193	0.0118
% 0.01	10	1	2.72e-03	0.4961	2.8315	0.1056	1.60e-05	0.0175	0.0137
% 0.1	    5	5	3.44e-03	0.5953	3.7422	0.4163	6.29e-05	0.0382	0.1934
\setlength{\tabcolsep}{4pt}
\renewcommand{\arraystretch}{1.3}
\begin{table*}[h]
    \small
    \caption{
    \textit{Objective metrics for non parametric models of \textbf{Universal Audio 1176LN} limiter.}
    \textit{Bold indicates best performing models.}
    % \textit{Learning rate multiplier for nonlinearity in gray-box models shown in brackets.}
    }
    \label{tab:metrics_od_diy-klon-centaur}
    \centerline{
        \begin{tabular}{lccccccccccccc}
            \hline
            \midrule
            
            \multirow{2}{*}{Model}
                & \multirow{2}{*}{Params.}
                    & \multirow{2}{*}{LR}
                        & \multicolumn{2}{c}{Weights}
                            & \multicolumn{3}{c}{}
                                & \multicolumn{3}{c}{FAD} \\ 
            \cmidrule(lr){4-5} 
                % \cmidrule(lr){6-8} 
                    \cmidrule(lr){9-12}
            
            &   &   & L1 & MR-STFT & MSE & ESR & MAPE & VGGish & PANN & CLAP & AFx-Rep \\ 
            \hline
            LSTM-32 & 4.5k & 0.001 & 10 & 1 & 2.11e-04 & 0.0034 & 33.4556 & 0.1199 & 1.43e-05 & 0.0927 & 0.0531 \\
            LSTM-96 & 38.1k & 0.001 & 1 & 0.1 & 1.84e-04 & 0.0030 & 31.7705 & 0.0537 & 2.79e-05 & 0.0953 & 0.0491 \\
            \hline
            TCN-45-S-16 & 7.5k & 0.005 & 10 & 1 & 7.52e-04 & 0.0123 & 22.6519 & 0.5622 & 5.67e-05 & 0.1789 & 0.1421 \\
            TCN-45-L-16 & 7.3k & 0.005 & 5 & 5 & 2.08e-03 & 0.0351 & 7.1800 & 0.5551 & 3.39e-05 & 0.1781 & 0.1338 \\
            TCN-250-S-16 & 14.5k & 0.005 & 5 & 5 & 1.42e-03 & 0.0238 & 3.6584 & 0.7090 & 3.49e-05 & 0.1215 & 0.1091 \\
            TCN-250-L-16 & 18.4k & 0.005 & 10 & 1 & 6.01e-04 & 0.0098 & 1.7731 & 0.5886 & 3.76e-05 & 0.1323 & 0.1095 \\
            TCN-2500-S-16 & 13.7k & 0.005 & 10 & 1 & 9.33e-04 & 0.0153 & 2.9281 & 1.0565 & 7.22e-05 & 0.1313 & 0.1349 \\
            TCN-2500-L-16 & 11.9k & 0.005 & 5 & 5 & 1.85e-03 & 0.0312 & 6.2843 & 0.7231 & 2.78e-05 & 0.1098 & 0.1037 \\
            \hline
            TCN-TF-45-S-16 & 39.5k & 0.005 & 10 & 1 & 9.46e-05 & 0.0016 & 2.3228 & 0.1738 & 1.52e-05 & 0.0417 & 0.0224 \\
            TCN-TF-45-L-16 & 71.3k & 0.005 & 1 & 0.1 & 9.54e-05 & 0.0015 & 2.3920 & 0.1097 & 2.24e-05 & 0.0293 & 0.0166 \\
            TCN-TF-250-S-16 & 52.9k & 0.005 & 10 & 1 & 8.49e-05 & 0.0014 & 1.6220 & 0.0718 & 2.01e-06 & 0.0201 & 0.0164 \\
            TCN-TF-250-L-16 & 88.8k & 0.005 & 1 & 0.1 & 6.11e-05 & 0.0010 & 2.0004 & 0.0643 & 2.91e-06 & 0.0184 & 0.0114 \\
            TCN-TF-2500-S-16 & 45.7k & 0.005 & 1 & 0.1 & 1.30e-04 & 0.0021 & 1.9660 & 0.2419 & 1.63e-05 & 0.0315 & 0.0271 \\
            TCN-TF-2500-L-16 & 75.9k & 0.005 & 5 & 5 & 4.71e-04 & 0.0080 & 5.8395 & 0.1076 & 2.98e-06 & 0.0229 & 0.0220 \\
            \hline
            GCN-45-S-16 & 16.2k & 0.005 & 10 & 1 & 6.51e-04 & 0.0109 & 6.3400 & 0.4614 & 4.86e-07 & 0.1465 & 0.0761 \\
            GCN-45-L-16 & 17.1k & 0.005 & 10 & 1 & 5.68e-04 & 0.0095 & 42.1807 & 0.3609 & 1.19e-05 & 0.1478 & 0.0763 \\
            GCN-250-S-16 & 30.4k & 0.005 & 1 & 0.1 & 4.00e-04 & 0.0066 & 3.4189 & 0.4703 & 7.88e-06 & 0.0974 & 0.0521 \\
            GCN-250-L-16 & 39.6k & 0.005 & 1 & 0.1 & 5.99e-04 & 0.0101 & 53.2828 & 0.5867 & 1.30e-05 & 0.1051 & 0.0649 \\
            GCN-2500-S-16 & 28.6k & 0.005 & 10 & 1 & 3.50e-04 & 0.0058 & 1.8697 & 0.3144 & 5.04e-06 & 0.0857 & 0.0296 \\
            GCN-2500-L-16 & 26.4k & 0.005 & 10 & 1 & 3.62e-04 & 0.0060 & 3.1052 & 0.4022 & 1.58e-06 & 0.0840 & 0.0444 \\
            \hline
            GCN-TF-45-S-16 & 141.6k & 0.005 & 10 & 1 & 5.64e-05 & 0.0009 & 0.8019 & 0.0317 & \textbf{1.67e-07} & 0.0206 & 0.0054 \\
            GCN-TF-45-L-16 & 268.0k & 0.005 & 5 & 5 & 3.02e-04 & 0.0052 & 1.0718 & 0.0243 & 2.11e-06 & 0.0299 & 0.0052 \\
            GCN-TF-250-S-16 & 181.0k & 0.005 & 1 & 0.1 & 9.48e-05 & 0.0016 & 1.5248 & 0.0628 & 4.35e-06 & 0.0203 & 0.0078 \\
            GCN-TF-250-L-16 & 315.6k & 0.005 & 1 & 0.1 & 9.54e-05 & 0.0016 & 1.7229 & 0.0371 & 4.02e-06 & 0.0191 & 0.0098 \\
            GCN-TF-2500-S-16 & 154.1k & 0.005 & 1 & 0.1 & 5.83e-05 & 0.0010 & 1.4051 & 0.1082 & 8.25e-06 & 0.0187 & 0.0124 \\
            GCN-TF-2500-L-16 & 277.3k & 0.005 & 5 & 5 & 2.71e-04 & 0.0046 & 1.6863 & 0.0476 & 4.82e-07 & 0.0170 & 0.0069 \\
            \hline
            S4-S-16 & 2.4k & 0.01 & 5 & 5 & 7.53e-04 & 0.0123 & 5.3618 & 0.4503 & 5.75e-06 & 0.0812 & 0.0886 \\
            S4-L-16 & 19.0k & 0.01 & 10 & 1 & 1.50e-04 & 0.0025 & 9.5521 & 0.0517 & 8.10e-07 & 0.0253 & 0.0186 \\
            \hline
            S4-TF-S-16 & 28.0k & 0.01 & 1 & 0.1 & 6.56e-05 & 0.0011 & 0.7010 & 0.0810 & 2.41e-06 & 0.0223 & 0.0100 \\
            S4-TF-L-16 & 70.2k & 0.01 & 10 & 1 & \textbf{5.03e-05} & \textbf{0.0008} & 0.9904 & \textbf{0.0215} & 1.17e-05 & \textbf{0.0152} & \textbf{0.0046} \\
            \hline
            GB-COMP & 47 & 0.1 & 0.5 & 0.5 & 4.92e-04 & 0.0081 & \textbf{0.6918} & 0.3773 & 8.98e-07 & 0.1448 & 0.0977 \\
            \hline
            \hline
        \end{tabular}
    }
\end{table*}

% 0.001	10	1	2.11e-04	0.0034	33.4556	0.1199	1.43e-05	0.0927	0.0531
% 0.001	1	0.1	1.84e-04	0.0030	31.7705	0.0537	2.79e-05	0.0953	0.0491
% 0.005	10	1	7.52e-04	0.0123	22.6519	0.5622	5.67e-05	0.1789	0.1421
% 0.005	5	5	2.08e-03	0.0351	7.1800	0.5551	3.39e-05	0.1781	0.1338
% 0.005	5	5	1.42e-03	0.0238	3.6584	0.7090	3.49e-05	0.1215	0.1091
% 0.005	10	1	6.01e-04	0.0098	1.7731	0.5886	3.76e-05	0.1323	0.1095
% 0.005	10	1	9.33e-04	0.0153	2.9281	1.0565	7.22e-05	0.1313	0.1349
% 0.005	5	5	1.85e-03	0.0312	6.2843	0.7231	2.78e-05	0.1098	0.1037
% 0.005	10	1	9.46e-05	0.0016	2.3228	0.1738	1.52e-05	0.0417	0.0224
% 0.005	1	0.1	9.54e-05	0.0015	2.3920	0.1097	2.24e-05	0.0293	0.0166
% 0.005	10	1	8.49e-05	0.0014	1.6220	0.0718	2.01e-06	0.0201	0.0164
% 0.005	1	0.1	6.11e-05	0.0010	2.0004	0.0643	2.91e-06	0.0184	0.0114
% 0.005	1	0.1	1.30e-04	0.0021	1.9660	0.2419	1.63e-05	0.0315	0.0271
% 0.005	5	5	4.71e-04	0.0080	5.8395	0.1076	2.98e-06	0.0229	0.0220
% 0.005	10	1	6.51e-04	0.0109	6.3400	0.4614	4.86e-07	0.1465	0.0761
% 0.005	10	1	5.68e-04	0.0095	42.1807	0.3609	1.19e-05	0.1478	0.0763
% 0.005	1	0.1	4.00e-04	0.0066	3.4189	0.4703	7.88e-06	0.0974	0.0521
% 0.005	1	0.1	5.99e-04	0.0101	53.2828	0.5867	1.30e-05	0.1051	0.0649
% 0.005	10	1	3.50e-04	0.0058	1.8697	0.3144	5.04e-06	0.0857	0.0296
% 0.005	10	1	3.62e-04	0.0060	3.1052	0.4022	1.58e-06	0.0840	0.0444
% 0.005	10	1	5.64e-05	0.0009	0.8019	0.0317	1.67e-07	0.0206	0.0054
% 0.005	5	5	3.02e-04	0.0052	1.0718	0.0243	2.11e-06	0.0299	0.0052
% 0.005	1	0.1	9.48e-05	0.0016	1.5248	0.0628	4.35e-06	0.0203	0.0078
% 0.005	1	0.1	9.54e-05	0.0016	1.7229	0.0371	4.02e-06	0.0191	0.0098
% 0.005	1	0.1	5.83e-05	0.0010	1.4051	0.1082	8.25e-06	0.0187	0.0124
% 0.005	5	5	2.71e-04	0.0046	1.6863	0.0476	4.82e-07	0.0170	0.0069
% 0.01	5	5	7.53e-04	0.0123	5.3618	0.4503	5.75e-06	0.0812	0.0886
% 0.01	10	1	1.50e-04	0.0025	9.5521	0.0517	8.10e-07	0.0253	0.0186
% 0.01	1	0.1	6.56e-05	0.0011	0.7010	0.0810	2.41e-06	0.0223	0.0100
% 0.01	10	1	5.03e-05	0.0008	0.9904	0.0215	1.17e-05	0.0152	0.0046
% 0.1	    0.5	0.5	4.92e-04	0.0081	0.6918	0.3773	8.98e-07	0.1448	0.0977

\clearpage

% ===
\subsection{Results Overdrive}
\setlength{\tabcolsep}{3pt}
\renewcommand{\arraystretch}{1.3}
\begin{table*}[h]
    \small
    \caption{
    \textit{Scaled test loss for non parametric models of overdrive effects. Bold indicates best performing models.}
    % Test scaled-loss for overdrive effects and non parametric models. 
    % Bold indicates best performing models.
    }
    \label{tab:testloss_od}
    \centerline{
        \begin{tabular}{l c >{\columncolor{gray!20}}ccc >{\columncolor{gray!20}}ccc >{\columncolor{gray!20}}ccc >{\columncolor{gray!20}}ccc}
            \hline
            \midrule
            \multirow{2}{*}{Model} 
                & \multirow{2}{*}{Params.} 
                    & \multicolumn{3}{c}{Fulltone Fulldrive 2} 
                        &  \multicolumn{3}{c}{Harley Benton Green Tint} 
                            & \multicolumn{3}{c}{Ibanez TS9} 
                                & \multicolumn{3}{c}{DIY Klon Centaur} \\
            \cmidrule(lr){3-5} 
                \cmidrule(lr){6-8} 
                    \cmidrule(lr){9-11} 
                        \cmidrule(lr){12-14}
                        
            & & 
            Tot. & {\footnotesize L1} &  {\footnotesize MR-STFT} & 
            Tot. & {\footnotesize L1} &  {\footnotesize MR-STFT} & 
            Tot. & {\footnotesize L1} &  {\footnotesize MR-STFT} & 
            Tot. & {\footnotesize L1} &  {\footnotesize MR-STFT} \\ 
            
            \hline
            LSTM-32 & 4.5k & 1.1757 & 0.0275 & 1.1482 & 1.1488 & 0.0404 & 1.1084 & 0.2504 & 0.0029 & 0.2475 & 2.6646 & 0.1623 & 2.5023 \\ 
            LSTM-96 & 38.1k & 0.5053 & 0.0071 & 0.4982 & 1.1921 & 0.0459 & 1.1463 & \textbf{0.2094} & 0.0025 & 0.2068 & 1.6661 & 0.1663 & 1.4998 \\ 
            \hline
            TCN-45-S-16 & 7.5k & 0.4409 & 0.0059 & 0.4350 & 0.5190 & 0.0051 & 0.5138 & 0.4214 & 0.0125 & 0.4089 & 0.9993 & 0.0539 & 0.9454 \\ 
            TCN-45-L-16 & 7.3k & 0.4388 & 0.0059 & 0.4328 & 0.4985 & 0.0061 & 0.4924 & 0.3622 & 0.0088 & 0.3534 & 0.8964 & 0.0492 & 0.8472 \\
            TCN-250-S-16 & 14.5k & 0.4161 & 0.0038 & 0.4122 & 0.5041 & 0.0045 & 0.4996 & 0.4050 & 0.0063 & 0.3987 & 1.0135 & 0.0478 & 0.9657 \\ 
            TCN-250-L-16 & 18.4k & 0.4165 & 0.0027 & 0.4138 & 0.4799 & 0.0058 & 0.4740 & 0.3066 & 0.0057 & 0.3010 & 0.8163 & 0.0244 & 0.7919 \\
            TCN-2500-S-16 & 13.7k & 0.4468 & 0.0039 & 0.4429 & 0.5025 & 0.0054 & 0.4971 & 0.4976 & 0.0167 & 0.4809 & 1.1596 & 0.0670 & 1.0926 \\ 
            TCN-2500-L-16 & 11.9k & 0.4078 & 0.0030 & 0.4048 & 0.4675 & 0.0052 & 0.4623 & 0.3639 & 0.0109 & 0.3531 & 0.8681 & 0.0358 & 0.8323 \\
            \hline
            TCN-TF-45-S-16 & 39.5k & 0.4023 & 0.0029 & 0.3994 & 0.5279 & 0.0054 & 0.5225 & 0.4209 & 0.0072 & 0.4137 & 0.8154 & 0.0466 & 0.7688 \\
            TCN-TF-45-L-16 & 71.3k & 0.3946 & 0.0029 & 0.3917 & 0.4923 & 0.0049 & 0.4874 & 0.3743 & 0.0062 & 0.3681 & 0.6794 & 0.0354 & 0.6440 \\
            TCN-TF-250-S-16 & 52.9k & 0.4056 & 0.0028 & 0.4027 & 0.5242 & 0.0055 & 0.5187 & 0.3088 & 0.0057 & 0.3032 & 0.7251 & 0.0412 & 0.6839 \\
            TCN-TF-250-L-16 & 88.8k & 0.3729 & 0.0024 & 0.3705 & 0.4908 & 0.0116 & 0.4792 & 0.2992 & 0.0037 & 0.2955 & 0.6896 & 0.0286 & 0.6610 \\
            TCN-TF-2500-S-16 & 45.7k & 0.4092 & 0.0035 & 0.4057 & 0.4745 & 0.0047 & 0.4698 & 0.4799 & 0.0076 & 0.4724 & 0.9178 & 0.0608 & 0.8571 \\
            TCN-TF-2500-L-16 & 75.9k & 0.3883 & 0.0029 & 0.3853 & 0.4364 & 0.0061 & 0.4303 & 0.3389 & 0.0054 & 0.3334 & 0.8672 & 0.0352 & 0.8320 \\
            \hline
            GCN-45-S-16 & 16.2k & 0.4385 & 0.0043 & 0.4342 & 0.5215 & 0.0048 & 0.5168 & 0.4147 & 0.0127 & 0.4020 & 1.0857 & 0.0540 & 1.0317 \\ 
            GCN-45-L-16 & 17.1k & 0.4129 & 0.0070 & 0.4059 & 0.5085 & 0.0060 & 0.5025 & 0.3482 & 0.0104 & 0.3378 & 0.8823 & 0.0311 & 0.8512 \\
            GCN-250-S-16 & 30.4k & 0.4290 & 0.0032 & 0.4258 & 0.4827 & 0.0042 & 0.4785 & 0.4121 & 0.0168 & 0.3953 & 1.0585 & 0.0481 & 1.0104 \\ 
            GCN-250-L-16 & 39.6k & 0.3919 & 0.0027 & 0.3892 & 0.4635 & 0.0040 & 0.4594 & 0.3317 & 0.0046 & 0.3271 & 0.7715 & 0.0263 & 0.7452 \\
            GCN-2500-S-16 & 28.6k & 0.4312 & 0.0049 & 0.4263 & 0.4948 & 0.0055 & 0.4894 & 0.4725 & 0.0181 & 0.4544 & 1.0732 & 0.0626 & 1.0106 \\ 
            GCN-2500-L-16 & 26.4k & 0.3958 & 0.0028 & 0.3931 & 0.4625 & 0.0038 & 0.4587 & 0.3559 & 0.0128 & 0.3431 & 0.7494 & 0.0393 & 0.7101 \\
            \hline
            GCN-TF-45-S-16 & 141.6k & 0.3872 & 0.0025 & 0.3847 & 0.5896 & 0.0071 & 0.5824 & 0.3799 & 0.0067 & 0.3732 & 0.7586 & 0.0463 & 0.7123 \\
            GCN-TF-45-L-16 & 268.0k & 0.3704 & 0.0026 & 0.3678 & 0.5029 & 0.0056 & 0.4973 & 0.3128 & 0.0061 & 0.3067 & 0.6668 & 0.0372 & 0.6296 \\
            GCN-TF-250-S-16 & 181.0k & 0.4312 & 0.0034 & 0.4278 & 0.4495 & 0.0043 & 0.4452 & 0.3613 & 0.0042 & 0.3571 & 0.6735 & 0.0367 & 0.6369 \\
            GCN-TF-250-L-16 & 315.6k & 0.3701 & 0.0027 & 0.3674 & 0.5065 & 0.0052 & 0.5013 & 0.2702 & 0.0038 & 0.2664 & 0.6275 & 0.0301 & 0.5974 \\
            GCN-TF-2500-S-16 & 154.1k & 0.3897 & 0.0029 & 0.3868 & 0.4654 & 0.0048 & 0.4606 & 0.4014 & 0.0079 & 0.3936 & 0.8594 & 0.0485 & 0.8109 \\
            GCN-TF-2500-L-16 & 277.3k & 0.3790 & 0.0046 & 0.3743 & 0.4307 & 0.0039 & 0.4267 & 0.3073 & 0.0063 & 0.3009 & 0.6816 & 0.0406 & 0.6410 \\
            \hline
            S4-S-16 & 2.4k & 0.3615 & 0.0019 & 0.3596 & 0.4431 & 0.0035 & 0.4395 & 0.3631 & 0.0106 & 0.3525 & 1.0089 & 0.0312 & 0.9777 \\
            S4-L-16 & 19.0k & 0.3427 & 0.0017 & 0.3409 & \textbf{0.4272} & 0.0035 & 0.4237 & 0.2553 & 0.0026 & 0.2527 & \textbf{0.5605} & 0.0201 & 0.5404 \\
            \hline
            S4-TF-S-16 & 28.0k & 0.3403 & 0.0019 & 0.3384 & 0.4313 & 0.0049 & 0.4264 & 0.3070 & 0.0044 & 0.3026 & 0.8116 & 0.0240 & 0.7876 \\
            S4-TF-L-16 & 70.2k & \textbf{0.3191} & 0.0018 & 0.3173 & 0.4316 & 0.0037 & 0.4278 & 0.2637 & 0.0028 & 0.2608 & 0.5859 & 0.0280 & 0.5579 \\
            \hline
            GB-DIST-MLP & 2.2k & 0.6919 & 0.0165 & 0.6754 & 0.8617 & 0.0239 & 0.8379 & 0.5755 & 0.0802 & 0.4953 & 1.2269 & 0.1200 & 1.1069 \\
            GB-DIST-RNL & 47 & 0.7860 & 0.0170 & 0.7690 & 0.9526 & 0.0252 & 0.9274 & 0.6380 & 0.0774 & 0.5606 & 0.9397 & 0.1090 & 0.8306 \\
            \hline
            GB-FUZZ-MLP & 2.3k & 0.6928 & 0.0158 & 0.6770 & 0.8992 & 0.0229 & 0.8763 & 0.6618 & 0.0846 & 0.5772 & 1.1462 & 0.0876 & 1.0586 \\
            GB-FUZZ-RNL & 62 & 0.7970 & 0.0172 & 0.7798 & 0.8817 & 0.0250 & 0.8567 & 0.5814 & 0.0797 & 0.5017 & 1.1943 & 0.1095 & 1.0849 \\
            \hline
            \hline
        \end{tabular}
    }
\end{table*}

% 1.1757	0.0275	1.1482	1.1488	0.0404	1.1084	0.2504	0.0029	0.2475	2.6646	0.1623	2.5023
% 0.5053	0.0071	0.4982	1.1921	0.0459	1.1463	0.2094	0.0025	0.2068	1.6661	0.1663	1.4998
% 0.4409	0.0059	0.4350	0.5190	0.0051	0.5138	0.4214	0.0125	0.4089	0.9993	0.0539	0.9454
% 0.4388	0.0059	0.4328	0.4985	0.0061	0.4924	0.3622	0.0088	0.3534	0.8964	0.0492	0.8472
% 0.4161	0.0038	0.4122	0.5041	0.0045	0.4996	0.4050	0.0063	0.3987	1.0135	0.0478	0.9657
% 0.4165	0.0027	0.4138	0.4799	0.0058	0.4740	0.3066	0.0057	0.3010	0.8163	0.0244	0.7919
% 0.4468	0.0039	0.4429	0.5025	0.0054	0.4971	0.4976	0.0167	0.4809	1.1596	0.0670	1.0926
% 0.4078	0.0030	0.4048	0.4675	0.0052	0.4623	0.3639	0.0109	0.3531	0.8681	0.0358	0.8323
% 0.4023	0.0029	0.3994	0.5279	0.0054	0.5225	0.4209	0.0072	0.4137	0.8154	0.0466	0.7688
% 0.3946	0.0029	0.3917	0.4923	0.0049	0.4874	0.3743	0.0062	0.3681	0.6794	0.0354	0.6440
% 0.4056	0.0028	0.4027	0.5242	0.0055	0.5187	0.3088	0.0057	0.3032	0.7251	0.0412	0.6839
% 0.3729	0.0024	0.3705	0.4908	0.0116	0.4792	0.2992	0.0037	0.2955	0.6896	0.0286	0.6610
% 0.4092	0.0035	0.4057	0.4745	0.0047	0.4698	0.4799	0.0076	0.4724	0.9178	0.0608	0.8571
% 0.3883	0.0029	0.3853	0.4364	0.0061	0.4303	0.3389	0.0054	0.3334	0.8672	0.0352	0.8320
% 0.4385	0.0043	0.4342	0.5215	0.0048	0.5168	0.4147	0.0127	0.4020	1.0857	0.0540	1.0317
% 0.4129	0.0070	0.4059	0.5085	0.0060	0.5025	0.3482	0.0104	0.3378	0.8823	0.0311	0.8512
% 0.4290	0.0032	0.4258	0.4827	0.0042	0.4785	0.4121	0.0168	0.3953	1.0585	0.0481	1.0104
% 0.3919	0.0027	0.3892	0.4635	0.0040	0.4594	0.3317	0.0046	0.3271	0.7715	0.0263	0.7452
% 0.4312	0.0049	0.4263	0.4948	0.0055	0.4894	0.4725	0.0181	0.4544	1.0732	0.0626	1.0106
% 0.3958	0.0028	0.3931	0.4625	0.0038	0.4587	0.3559	0.0128	0.3431	0.7494	0.0393	0.7101
% 0.3872	0.0025	0.3847	0.5896	0.0071	0.5824	0.3799	0.0067	0.3732	0.7586	0.0463	0.7123
% 0.3704	0.0026	0.3678	0.5029	0.0056	0.4973	0.3128	0.0061	0.3067	0.6668	0.0372	0.6296
% 0.4312	0.0034	0.4278	0.4495	0.0043	0.4452	0.3613	0.0042	0.3571	0.6735	0.0367	0.6369
% 0.3701	0.0027	0.3674	0.5065	0.0052	0.5013	0.2702	0.0038	0.2664	0.6275	0.0301	0.5974
% 0.3897	0.0029	0.3868	0.4654	0.0048	0.4606	0.4014	0.0079	0.3936	0.8594	0.0485	0.8109
% 0.3790	0.0046	0.3743	0.4307	0.0039	0.4267	0.3073	0.0063	0.3009	0.6816	0.0406	0.6410
% 0.3615	0.0019	0.3596	0.4431	0.0035	0.4395	0.3631	0.0106	0.3525	1.0089	0.0312	0.9777
% 0.3427	0.0017	0.3409	0.4272	0.0035	0.4237	0.2553	0.0026	0.2527	0.5605	0.0201	0.5404
% 0.3403	0.0019	0.3384	0.4313	0.0049	0.4264	0.3070	0.0044	0.3026	0.8116	0.0240	0.7876
% 0.3191	0.0018	0.3173	0.4316	0.0037	0.4278	0.2637	0.0028	0.2608	0.5859	0.0280	0.5579
% 0.6919	0.0165	0.6754	0.8617	0.0239	0.8379	0.5755	0.0802	0.4953	1.2269	0.1200	1.1069
% 0.7860	0.0170	0.7690	0.9526	0.0252	0.9274	0.6380	0.0774	0.5606	0.9397	0.1090	0.8306
% 0.6928	0.0158	0.6770	0.8992	0.0229	0.8763	0.6618	0.0846	0.5772	1.1462	0.0876	1.0586
% 0.7970	0.0172	0.7798	0.8817	0.0250	0.8567	0.5814	0.0797	0.5017	1.1943	0.1095	1.0849







% %\setlength{\tabcolsep}{3.8pt}
% % \vspace{-0.3cm}
% % \renewcommand{\arraystretch}{0.85}
% \begin{table*}[h]
%     \centering
%     \small
%     \begin{tabular}{lcccccccccccc} \toprule
    
%         \multirow{2}{*}{Model} 
%             & \multirow{2}{*}{Params.} 
%                 & \multicolumn{2}{c}{Od 1} 
%                     & \multicolumn{2}{c}{Od 2} 
%                         &  \multicolumn{2}{c}{Od 3} 
%                             & \multicolumn{2}{c}{Od 4} \\ 
%         \cmidrule(lr){3-4} 
%             \cmidrule(lr){5-6} 
%                 \cmidrule(lr){7-8} 
%                     \cmidrule(lr){9-10}
%         &   & $L1$ & MR-STFT & $L1$ & MR-STFT & $L1$ & MR-STFT & $L1$ & MR-STFT \\ 
%         \midrule
%         LSTM-32
%             & 4.5k  & 0.012 & 0.356 & 0.001 & 0.239 & 0.002 & 0.250 & 0.004 & 0.236 \\ 
%         LSTM-96       
%             & - & - & - & - & - & - & - & - & - \\ 
%         \midrule
%         LSTM-TVC-32
%             & - & - & - & - & - & - & - & - & - \\
%         LSTM-TVC-96
%             & - & - & - & - & - & - & - & - & - \\
%         \midrule
%         TCN-45-S-16               
%             & - & - & - & - & - & - & - & - & - \\ 
%         TCN-45-L-16               
%             & - & - & - & - & - & - & - & - & - \\
%         \midrule
%         TCN-TF-45-S-16               
%             & - & - & - & - & - & - & - & - & - \\
%         TCN-TF-45-L-16               
%             & - & - & - & - & - & - & - & - & - \\
%         \midrule
%         TCN-TTF-45-S-16               
%             & - & - & - & - & - & - & - & - & - \\
%         TCN-TTF-45-L-16               
%             & - & - & - & - & - & - & - & - & - \\
%         \midrule
%         TCN-TVF-45-S-16               
%             & - & - & - & - & - & - & - & - & - \\
%         TCN-TVF-45-L-16               
%             & - & - & - & - & - & - & - & - & - \\
%         \midrule
%         GCN-45-S-16               
%             & - & - & - & - & - & - & - & - & - \\ 
%         GCN-45-L-16               
%             & - & - & - & - & - & - & - & - & - \\
%         \midrule
%         GCN-TF-45-S-16               
%             & - & - & - & - & - & - & - & - & - \\
%         GCN-TF-45-L-16               
%             & - & - & - & - & - & - & - & - & - \\
%         \midrule
%         GCN-TTF-45-S-16               
%             & - & - & - & - & - & - & - & - & - \\
%         GCN-TTF-45-L-16               
%             & - & - & - & - & - & - & - & - & - \\
%         \midrule
%         GCN-TVF-45-S-16               
%             & - & - & - & - & - & - & - & - & - \\
%         GCN-TVF-45-L-16               
%             & - & - & - & - & - & - & - & - & - \\
%         \midrule
%         SSM-1               
%             & - & - & - & - & - & - & - & - & - \\
%         SSM-2               
%             & - & - & - & - & - & - & - & - & - \\
%         \midrule
%         GB-COMP              
%             & - & - & - & - & - & - & - & - & - \\
%         \midrule
%         GB-DIST-MLP             
%             & - & - & - & - & - & - & - & - & - \\
%         GB-DIST-RNL             
%             & - & - & - & - & - & - & - & - & - \\
%         \midrule
%         GB-FUZZ-MLP             
%             & - & - & - & - & - & - & - & - & - \\
%         GB-FUZZ-RNL             
%             & - & - & - & - & - & - & - & - & - \\
%         \bottomrule 
%     \end{tabular}
%     \vspace{-0.0cm}
%     \caption{Overall L1+MR-STFT loss across device type for non parametric models}
%     \label{tab:other_fx} \vspace{0.2cm}
% \end{table*}

\setlength{\tabcolsep}{4pt}
\renewcommand{\arraystretch}{1.3}
\begin{table*}[h]
    \small
    \caption{
    \textit{Scaled validation and test loss for non parametric models of \textbf{Fulltone Fulldrive 2} overdrive.}
    \textit{Bold indicates best performing models.}
    \textit{Learning rate multiplier for nonlinearity in gray-box models shown in brackets.}
    }
    \label{tab:val-and-test-loss_od_fulltone-fulldrive}
    \centerline{
        \begin{tabular}{l c c cc >{\columncolor{gray!20}}ccc >{\columncolor{gray!20}}ccc}
            \hline
            \midrule
            
            \multirow{2}{*}{Model}
                & \multirow{2}{*}{Params.}
                    & \multirow{2}{*}{LR}
                        & \multicolumn{2}{c}{Weights}
                            & \multicolumn{3}{c}{Val. Loss}
                                & \multicolumn{3}{c}{Test Loss} \\ 
            \cmidrule(lr){4-5} 
                \cmidrule(lr){6-8} 
                    \cmidrule(lr){9-11}
            
            &   &   & {\scriptsize L1} & {\scriptsize MR-STFT} & Tot. & {\scriptsize L1} & {\scriptsize MR-STFT} & Tot. & {\scriptsize L1} & {\scriptsize MR-STFT} \\ 
            
            \hline
            LSTM-32 & 4.5k & 0.005 & 10 & 1 & 0.9025 & 0.0156 & 0.8869 & 1.1757 & 0.0275 & 1.1482 \\
            LSTM-96 & 38.1k & 0.005 & 1 & 0.1 & 0.5337 & 0.0053 & 0.5284 & 0.5053 & 0.0071 & 0.4982 \\
            \hline
            TCN-45-S-16 & 7.5k & 0.005 & 5 & 5 & 0.4692 & 0.0069 & 0.4623 & 0.4409 & 0.0059 & 0.4350 \\
            TCN-45-L-16 & 7.3k & 0.005 & 5 & 5 & 0.4605 & 0.0082 & 0.4523 & 0.4388 & 0.0059 & 0.4328 \\
            TCN-250-S-16 & 14.5k & 0.005 & 0.5 & 0.5 & 0.4669 & 0.0048 & 0.4621 & 0.4161 & 0.0038 & 0.4122 \\
            TCN-250-L-16 & 18.4k & 0.005 & 1 & 0.1 & 0.4385 & 0.0027 & 0.4358 & 0.4165 & 0.0027 & 0.4138 \\
            TCN-2500-S-16 & 13.7k & 0.005 & 10 & 1 & 0.4695 & 0.0037 & 0.4658 & 0.4468 & 0.0039 & 0.4429 \\
            TCN-2500-L-16 & 11.9k & 0.005 & 1 & 0.1 & 0.4430 & 0.0028 & 0.4402 & 0.4078 & 0.0030 & 0.4048 \\
            \hline
            TCN-TF-45-S-16 & 39.5k & 0.005 & 1 & 0.1 & 0.4463 & 0.0028 & 0.4435 & 0.4023 & 0.0029 & 0.3994 \\
            TCN-TF-45-L-16 & 71.3k & 0.005 & 10 & 1 & 0.4273 & 0.0023 & 0.4250 & 0.3946 & 0.0029 & 0.3917 \\
            TCN-TF-250-S-16 & 52.9k & 0.005 & 1 & 0.1 & 0.4215 & 0.0027 & 0.4188 & 0.4056 & 0.0028 & 0.4027 \\
            TCN-TF-250-L-16 & 88.8k & 0.005 & 10 & 1 & 0.3983 & 0.0020 & 0.3962 & 0.3729 & 0.0024 & 0.3705 \\
            TCN-TF-2500-S-16 & 45.7k & 0.005 & 5 & 5 & 0.4496 & 0.0044 & 0.4452 & 0.4092 & 0.0035 & 0.4057 \\
            TCN-TF-2500-L-16 & 75.9k & 0.005 & 10 & 1 & 0.4152 & 0.0026 & 0.4126 & 0.3883 & 0.0029 & 0.3853 \\
            \hline
            GCN-45-S-16 & 16.2k & 0.005 & 5 & 5 & 0.4894 & 0.0047 & 0.4847 & 0.4385 & 0.0043 & 0.4342 \\
            GCN-45-L-16 & 17.1k & 0.005 & 5 & 5 & 0.4805 & 0.0081 & 0.4724 & 0.4129 & 0.0070 & 0.4059 \\
            GCN-250-S-16 & 30.4k & 0.005 & 1 & 0.1 & 0.4569 & 0.0029 & 0.4540 & 0.4290 & 0.0032 & 0.4258 \\
            GCN-250-L-16 & 39.6k & 0.005 & 10 & 1 & 0.4226 & 0.0026 & 0.4200 & 0.3919 & 0.0027 & 0.3892 \\
            GCN-2500-S-16 & 28.6k & 0.005 & 5 & 5 & 0.4800 & 0.0051 & 0.4748 & 0.4312 & 0.0049 & 0.4263 \\
            GCN-2500-L-16 & 26.4k & 0.005 & 10 & 1 & 0.4239 & 0.0028 & 0.4211 & 0.3958 & 0.0028 & 0.3931 \\
            \hline
            GCN-TF-45-S-16 & 141.6k & 0.005 & 10 & 1 & 0.4180 & 0.0021 & 0.4159 & 0.3872 & 0.0025 & 0.3847 \\
            GCN-TF-45-L-16 & 268.0k & 0.005 & 1 & 0.1 & 0.4046 & 0.0020 & 0.4025 & 0.3704 & 0.0026 & 0.3678 \\
            GCN-TF-250-S-16 & 181.0k & 0.005 & 10 & 1 & 0.5026 & 0.0036 & 0.4990 & 0.4312 & 0.0034 & 0.4278 \\
            GCN-TF-250-L-16 & 315.6k & 0.005 & 1 & 0.1 & 0.4084 & 0.0023 & 0.4061 & 0.3701 & 0.0027 & 0.3674 \\
            GCN-TF-2500-S-16 & 154.1k & 0.005 & 1 & 0.1 & 0.3946 & 0.0026 & 0.3919 & 0.3897 & 0.0029 & 0.3868 \\
            GCN-TF-2500-L-16 & 277.3k & 0.005 & 5 & 5 & 0.3932 & 0.0058 & 0.3874 & 0.3790 & 0.0046 & 0.3743 \\
            \hline
            S4-S-16 & 2.4k & 0.01 & 10 & 1 & 0.3945 & 0.0015 & 0.3930 & 0.3615 & 0.0019 & 0.3596 \\
            S4-L-16 & 19.0k & 0.01 & 1 & 0.1 & 0.3585 & 0.0015 & 0.3569 & 0.3427 & 0.0017 & 0.3409 \\
            \hline
            S4-TF-S-16 & 28.0k & 0.01 & 10 & 1 & 0.3726 & 0.0012 & 0.3714 & 0.3403 & 0.0019 & 0.3384 \\
            S4-TF-L-16 & 70.2k & 0.01 & 1 & 0.1 & \textbf{0.3488} & 0.0011 & 0.3477 & \textbf{0.3191} & 0.0018 & 0.3173 \\
            \hline
            GB-DIST-MLP & 2.2k & 0.1 (0.01) & 5 & 5 & 0.7507 & 0.0148 & 0.7360 & 0.6919 & 0.0165 & 0.6754 \\
            GB-DIST-RNL & 47 & 0.1 (1) & 0.5 & 0.5 & 0.8472 & 0.0151 & 0.8321 & 0.7860 & 0.0170 & 0.7690 \\
            \hline
            GB-FUZZ-MLP & 2.3k & 0.1 (0.01) & 1 & 0.1 & 0.7521 & 0.0159 & 0.7362 & 0.6928 & 0.0158 & 0.6770 \\
            GB-FUZZ-RNL & 62 & 0.1 (1) & 0.5 & 0.5 & 0.8666 & 0.0166 & 0.8500 & 0.7970 & 0.0172 & 0.7798 \\
            \hline
            \hline
        \end{tabular}
    }
\end{table*}


% 0.005	10	1	0.9025	0.0156	0.8869	1.1757	0.0275	1.1482
% 0.005	1	0.1	0.5337	0.0053	0.5284	0.5053	0.0071	0.4982
% 0.005	5	5	0.4692	0.0069	0.4623	0.4409	0.0059	0.4350
% 0.005	5	5	0.4605	0.0082	0.4523	0.4388	0.0059	0.4328
% 0.005	0.5	0.5	0.4669	0.0048	0.4621	0.4161	0.0038	0.4122
% 0.005	1	0.1	0.4385	0.0027	0.4358	0.4165	0.0027	0.4138
% 0.005	10	1	0.4695	0.0037	0.4658	0.4468	0.0039	0.4429
% 0.005	1	0.1	0.4430	0.0028	0.4402	0.4078	0.0030	0.4048
% 0.005	1	0.1	0.4463	0.0028	0.4435	0.4023	0.0029	0.3994
% 0.005	10	1	0.4273	0.0023	0.4250	0.3946	0.0029	0.3917
% 0.005	1	0.1	0.4215	0.0027	0.4188	0.4056	0.0028	0.4027
% 0.005	10	1	0.3983	0.0020	0.3962	0.3729	0.0024	0.3705
% 0.005	5	5	0.4496	0.0044	0.4452	0.4092	0.0035	0.4057
% 0.005	10	1	0.4152	0.0026	0.4126	0.3883	0.0029	0.3853
% 0.005	5	5	0.4894	0.0047	0.4847	0.4385	0.0043	0.4342
% 0.005	5	5	0.4805	0.0081	0.4724	0.4129	0.0070	0.4059
% 0.005	1	0.1	0.4569	0.0029	0.4540	0.4290	0.0032	0.4258
% 0.005	10	1	0.4226	0.0026	0.4200	0.3919	0.0027	0.3892
% 0.005	5	5	0.4800	0.0051	0.4748	0.4312	0.0049	0.4263
% 0.005	10	1	0.4239	0.0028	0.4211	0.3958	0.0028	0.3931
% 0.005	10	1	0.4180	0.0021	0.4159	0.3872	0.0025	0.3847
% 0.005	1	0.1	0.4046	0.0020	0.4025	0.3704	0.0026	0.3678
% 0.005	10	1	0.5026	0.0036	0.4990	0.4312	0.0034	0.4278
% 0.005	1	0.1	0.4084	0.0023	0.4061	0.3701	0.0027	0.3674
% 0.005	1	0.1	0.3946	0.0026	0.3919	0.3897	0.0029	0.3868
% 0.005	5	5	0.3932	0.0058	0.3874	0.3790	0.0046	0.3743
% 0.01	10	1	0.3945	0.0015	0.3930	0.3615	0.0019	0.3596
% 0.01	1	0.1	0.3585	0.0015	0.3569	0.3427	0.0017	0.3409
% 0.01	10	1	0.3726	0.0012	0.3714	0.3403	0.0019	0.3384
% 0.01	1	0.1	0.3488	0.0011	0.3477	0.3191	0.0018	0.3173
% 0.1 (0.01)	5	5	0.7507	0.0148	0.7360	0.6919	0.0165	0.6754
% 0.1 (1)	0.5	0.5	0.8472	0.0151	0.8321	0.7860	0.0170	0.7690
% 0.1 (0.01)	1	0.1	0.7521	0.0159	0.7362	0.6928	0.0158	0.6770
% 0.1 (1)	0.5	0.5	0.8666	0.0166	0.8500	0.7970	0.0172	0.7798
\setlength{\tabcolsep}{4pt}
\renewcommand{\arraystretch}{1.3}
\begin{table*}[h]
    \small
    \caption{
    \textit{Scaled validation and test loss for non parametric models of \textbf{Harley Benton Green Tint} overdrive.}
    \textit{Bold indicates best performing models.}
    \textit{Learning rate multiplier for nonlinearity in gray-box models shown in brackets.}
    }
    \label{tab:val-and-test-loss_od_hb-greentint}
    \centerline{    
        \begin{tabular}{l c c cc >{\columncolor{gray!20}}ccc >{\columncolor{gray!20}}ccc}
            \hline
            \midrule
            
            \multirow{2}{*}{Model}
                & \multirow{2}{*}{Params.}
                    & \multirow{2}{*}{LR}
                        & \multicolumn{2}{c}{Weights}
                            & \multicolumn{3}{c}{Val. Loss}
                                & \multicolumn{3}{c}{Test Loss} \\ 
            \cmidrule(lr){4-5} 
                \cmidrule(lr){6-8} 
                    \cmidrule(lr){9-11}
            
            &   &   & {\scriptsize L1} & {\scriptsize MR-STFT} & Tot. & {\scriptsize L1} & {\scriptsize MR-STFT} & Tot. & {\scriptsize L1} & {\scriptsize MR-STFT} \\ 
            
            \hline
            LSTM-32 & 4.5k & 0.001 & 0.5 & 0.5 & 1.2897 & 0.0377 & 1.2520 & 1.1488 & 0.0404 & 1.1084 \\
            LSTM-96 & 38.1k & 0.001 & 0.5 & 0.5 & 1.2992 & 0.0509 & 1.2483 & 1.1921 & 0.0459 & 1.1463 \\
            \hline
            TCN-45-S-16 & 7.5k & 0.005 & 5 & 5 & 0.5695 & 0.0036 & 0.5659 & 0.5190 & 0.0051 & 0.5138 \\
            TCN-45-L-16 & 7.3k & 0.005 & 5 & 5 & 0.5220 & 0.0092 & 0.5128 & 0.4985 & 0.0061 & 0.4924 \\
            TCN-250-S-16 & 14.5k & 0.005 & 5 & 5 & 0.5550 & 0.0051 & 0.5498 & 0.5041 & 0.0045 & 0.4996 \\
            TCN-250-L-16 & 18.4k & 0.005 & 0.5 & 0.5 & 0.5156 & 0.0084 & 0.5073 & 0.4799 & 0.0058 & 0.4740 \\
            TCN-2500-S-16 & 13.7k & 0.005 & 1 & 0.1 & 0.5324 & 0.0044 & 0.5280 & 0.5025 & 0.0054 & 0.4971 \\
            TCN-2500-L-16 & 11.9k & 0.005 & 5 & 5 & 0.5097 & 0.0060 & 0.5037 & 0.4675 & 0.0052 & 0.4623 \\
            \hline
            TCN-TF-45-S-16 & 39.5k & 0.005 & 10 & 1 & 0.5519 & 0.0049 & 0.5470 & 0.5279 & 0.0054 & 0.5225 \\
            TCN-TF-45-L-16 & 71.3k & 0.005 & 10 & 1 & 0.5248 & 0.0038 & 0.5210 & 0.4923 & 0.0049 & 0.4874 \\
            TCN-TF-250-S-16 & 52.9k & 0.005 & 1 & 0.1 & 0.5417 & 0.0045 & 0.5372 & 0.5242 & 0.0055 & 0.5187 \\
            TCN-TF-250-L-16 & 88.8k & 0.005 & 0.5 & 0.5 & 0.5474 & 0.0183 & 0.5291 & 0.4908 & 0.0116 & 0.4792 \\
            TCN-TF-2500-S-16 & 45.7k & 0.005 & 10 & 1 & 0.5060 & 0.0039 & 0.5022 & 0.4745 & 0.0047 & 0.4698 \\
            TCN-TF-2500-L-16 & 75.9k & 0.005 & 5 & 5 & 0.4631 & 0.0090 & 0.4541 & 0.4364 & 0.0061 & 0.4303 \\
            \hline
            GCN-45-S-16 & 16.2k & 0.005 & 5 & 5 & 0.5930 & 0.0049 & 0.5881 & 0.5215 & 0.0048 & 0.5168 \\
            GCN-45-L-16 & 17.1k & 0.005 & 5 & 5 & 0.6377 & 0.0079 & 0.6298 & 0.5085 & 0.0060 & 0.5025 \\
            GCN-250-S-16 & 30.4k & 0.005 & 5 & 5 & 0.5374 & 0.0042 & 0.5332 & 0.4827 & 0.0042 & 0.4785 \\
            GCN-250-L-16 & 39.6k & 0.005 & 5 & 5 & 0.4933 & 0.0044 & 0.4890 & 0.4635 & 0.0040 & 0.4594 \\
            GCN-2500-S-16 & 28.6k & 0.005 & 5 & 5 & 0.5459 & 0.0056 & 0.5404 & 0.4948 & 0.0055 & 0.4894 \\
            GCN-2500-L-16 & 26.4k & 0.005 & 10 & 1 & 0.5013 & 0.0027 & 0.4986 & 0.4625 & 0.0038 & 0.4587 \\
            \hline
            GCN-TF-45-S-16 & 141.6k & 0.005 & 10 & 1 & 0.5053 & 0.0035 & 0.5019 & 0.5896 & 0.0071 & 0.5824 \\
            GCN-TF-45-L-16 & 268.0k & 0.005 & 5 & 5 & 0.6342 & 0.0077 & 0.6265 & 0.5029 & 0.0056 & 0.4973 \\
            GCN-TF-250-S-16 & 181.0k & 0.005 & 0.5 & 0.5 & 0.4921 & 0.0037 & 0.4884 & 0.4495 & 0.0043 & 0.4452 \\
            GCN-TF-250-L-16 & 315.6k & 0.005 & 10 & 1 & 0.5329 & 0.0042 & 0.5287 & 0.5065 & 0.0052 & 0.5013 \\
            GCN-TF-2500-S-16 & 154.1k & 0.005 & 0.5 & 0.5 & 0.4694 & 0.0046 & 0.4648 & 0.4654 & 0.0048 & 0.4606 \\
            GCN-TF-2500-L-16 & 277.3k & 0.005 & 5 & 5 & 0.4556 & 0.0042 & 0.4514 & 0.4307 & 0.0039 & 0.4267 \\
            \hline
            S4-S-16 & 2.4k & 0.01 & 1 & 0.1 & 0.4743 & 0.0022 & 0.4722 & 0.4431 & 0.0035 & 0.4395 \\
            S4-L-16 & 19.0k & 0.01 & 10 & 1 & \textbf{0.4421} & 0.0025 & 0.4396 & \textbf{0.4272} & 0.0035 & 0.4237 \\
            \hline
            S4-TF-S-16 & 28.0k & 0.01 & 0.5 & 0.5 & 0.4598 & 0.0033 & 0.4565 & 0.4313 & 0.0049 & 0.4264 \\
            S4-TF-L-16 & 70.2k & 0.01 & 10 & 1 & 0.4581 & 0.0022 & 0.4559 & 0.4316 & 0.0037 & 0.4278 \\
            \hline
            GB-DIST-MLP & 2.2k & 0.1 (0.01) & 5 & 5 & 0.9715 & 0.0204 & 0.9511 & 0.8617 & 0.0239 & 0.8379 \\
            GB-DIST-RNL & 47 & 0.1 (1) & 0.5 & 0.5 & 1.0295 & 0.0220 & 1.0075 & 0.9526 & 0.0252 & 0.9274 \\
            \hline
            GB-FUZZ-MLP & 2.3k & 0.1 (0.01) & 10 & 1 & 0.9771 & 0.0236 & 0.9535 & 0.8992 & 0.0229 & 0.8763 \\
            GB-FUZZ-RNL & 62 & 0.1 (1) & 5 & 5 & 0.9321 & 0.0226 & 0.9096 & 0.8817 & 0.0250 & 0.8567 \\
            \hline
            \hline
        \end{tabular}
    }
\end{table*}



% 0.001	0.5	0.5	1.2897	0.0377	1.2520	1.1488	0.0404	1.1084
% 0.001	0.5	0.5	1.2992	0.0509	1.2483	1.1921	0.0459	1.1463
% 0.005	5	5	0.5695	0.0036	0.5659	0.5190	0.0051	0.5138
% 0.005	5	5	0.5220	0.0092	0.5128	0.4985	0.0061	0.4924
% 0.005	5	5	0.5550	0.0051	0.5498	0.5041	0.0045	0.4996
% 0.005	0.5	0.5	0.5156	0.0084	0.5073	0.4799	0.0058	0.4740
% 0.005	1	0.1	0.5324	0.0044	0.5280	0.5025	0.0054	0.4971
% 0.005	5	5	0.5097	0.0060	0.5037	0.4675	0.0052	0.4623
% 0.005	10	1	0.5519	0.0049	0.5470	0.5279	0.0054	0.5225
% 0.005	10	1	0.5248	0.0038	0.5210	0.4923	0.0049	0.4874
% 0.005	1	0.1	0.5417	0.0045	0.5372	0.5242	0.0055	0.5187
% 0.005	0.5	0.5	0.5474	0.0183	0.5291	0.4908	0.0116	0.4792
% 0.005	10	1	0.5060	0.0039	0.5022	0.4745	0.0047	0.4698
% 0.005	5	5	0.4631	0.0090	0.4541	0.4364	0.0061	0.4303
% 0.005	5	5	0.5930	0.0049	0.5881	0.5215	0.0048	0.5168
% 0.005	5	5	0.6377	0.0079	0.6298	0.5085	0.0060	0.5025
% 0.005	5	5	0.5374	0.0042	0.5332	0.4827	0.0042	0.4785
% 0.005	5	5	0.4933	0.0044	0.4890	0.4635	0.0040	0.4594
% 0.005	5	5	0.5459	0.0056	0.5404	0.4948	0.0055	0.4894
% 0.005	10	1	0.5013	0.0027	0.4986	0.4625	0.0038	0.4587
% 0.005	10	1	0.5053	0.0035	0.5019	0.5896	0.0071	0.5824
% 0.005	5	5	0.6342	0.0077	0.6265	0.5029	0.0056	0.4973
% 0.005	0.5	0.5	0.4921	0.0037	0.4884	0.4495	0.0043	0.4452
% 0.005	10	1	0.5329	0.0042	0.5287	0.5065	0.0052	0.5013
% 0.005	0.5	0.5	0.4694	0.0046	0.4648	0.4654	0.0048	0.4606
% 0.005	5	5	0.4556	0.0042	0.4514	0.4307	0.0039	0.4267
% 0.01	1	0.1	0.4743	0.0022	0.4722	0.4431	0.0035	0.4395
% 0.01	10	1	0.4421	0.0025	0.4396	0.4272	0.0035	0.4237
% 0.01	0.5	0.5	0.4598	0.0033	0.4565	0.4313	0.0049	0.4264
% 0.01	10	1	0.4581	0.0022	0.4559	0.4316	0.0037	0.4278
% 0.1 (0.01)	5	5	0.9715	0.0204	0.9511	0.8617	0.0239	0.8379
% 0.1 (1)	0.5	0.5	1.0295	0.0220	1.0075	0.9526	0.0252	0.9274
% 0.1 (0.01)	10	1	0.9771	0.0236	0.9535	0.8992	0.0229	0.8763
% 0.1 (1)	5	5	0.9321	0.0226	0.9096	0.8817	0.0250	0.8567
\setlength{\tabcolsep}{4pt}
\renewcommand{\arraystretch}{1.3}
\begin{table*}[h]
    \small
    \caption{
    \textit{Scaled validation and test loss for non parametric models of \textbf{Ibanez TS9} overdrive.}
    \textit{Bold indicates best performing models.}
    \textit{Learning rate multiplier for nonlinearity in gray-box models shown in brackets.}
    }
    \label{tab:val-and-test-loss_od_ibanez-ts9}
    \centerline{
        \begin{tabular}{l c c cc >{\columncolor{gray!20}}ccc >{\columncolor{gray!20}}ccc}
            \hline
            \midrule
            
            \multirow{2}{*}{Model}
                & \multirow{2}{*}{Params.}
                    & \multirow{2}{*}{LR}
                        & \multicolumn{2}{c}{Weights}
                            & \multicolumn{3}{c}{Val. Loss}
                                & \multicolumn{3}{c}{Test Loss} \\ 
            \cmidrule(lr){4-5} 
                \cmidrule(lr){6-8} 
                    \cmidrule(lr){9-11}
            
            &   &   & {\scriptsize L1} & {\scriptsize MR-STFT} & Tot. & {\scriptsize L1} & {\scriptsize MR-STFT} & Tot. & {\scriptsize L1} & {\scriptsize MR-STFT} \\ 
            
            \hline
            LSTM-32 & 4.5k & 0.005 & 10 & 1 & 0.3481 & 0.0029 & 0.3452 & 0.2504 & 0.0029 & 0.2475 \\
            LSTM-96 & 38.1k & 0.005 & 10 & 1 & 0.2932 & 0.0056 & 0.2876 & \textbf{0.2094} & 0.0025 & 0.2068 \\
            \hline
            TCN-45-S-16 & 7.5k & 0.005 & 0.5 & 0.5 & 0.4900 & 0.0145 & 0.4755 & 0.4214 & 0.0125 & 0.4089 \\
            TCN-45-L-16 & 7.3k & 0.005 & 5 & 5 & 0.8015 & 0.0201 & 0.7814 & 0.3622 & 0.0088 & 0.3534 \\
            TCN-250-S-16 & 14.5k & 0.005 & 1 & 0.1 & 0.5158 & 0.0115 & 0.5043 & 0.4050 & 0.0063 & 0.3987 \\
            TCN-250-L-16 & 18.4k & 0.005 & 5 & 5 & 0.6339 & 0.0154 & 0.6185 & 0.3066 & 0.0057 & 0.3010 \\
            TCN-2500-S-16 & 13.7k & 0.005 & 5 & 5 & 0.6698 & 0.0180 & 0.6518 & 0.4976 & 0.0167 & 0.4809 \\
            TCN-2500-L-16 & 11.9k & 0.005 & 0.5 & 0.5 & 0.3801 & 0.0107 & 0.3694 & 0.3639 & 0.0109 & 0.3531 \\
            \hline
            TCN-TF-45-S-16 & 39.5k & 0.005 & 1 & 0.1 & 0.4427 & 0.0080 & 0.4346 & 0.4209 & 0.0072 & 0.4137 \\
            TCN-TF-45-L-16 & 71.3k & 0.005 & 10 & 1 & 0.3814 & 0.0071 & 0.3743 & 0.3743 & 0.0062 & 0.3681 \\
            TCN-TF-250-S-16 & 52.9k & 0.005 & 0.5 & 0.5 & 0.2752 & 0.0063 & 0.2689 & 0.3088 & 0.0057 & 0.3032 \\
            TCN-TF-250-L-16 & 88.8k & 0.005 & 10 & 1 & 0.3149 & 0.0040 & 0.3109 & 0.2992 & 0.0037 & 0.2955 \\
            TCN-TF-2500-S-16 & 45.7k & 0.005 & 10 & 1 & 0.4834 & 0.0084 & 0.4750 & 0.4799 & 0.0076 & 0.4724 \\
            TCN-TF-2500-L-16 & 75.9k & 0.005 & 1 & 0.1 & 0.3483 & 0.0052 & 0.3431 & 0.3389 & 0.0054 & 0.3334 \\
            \hline
            GCN-45-S-16 & 16.2k & 0.005 & 5 & 5 & 0.5817 & 0.0172 & 0.5645 & 0.4147 & 0.0127 & 0.4020 \\
            GCN-45-L-16 & 17.1k & 0.005 & 5 & 5 & 0.4984 & 0.0101 & 0.4884 & 0.3482 & 0.0104 & 0.3378 \\
            GCN-250-S-16 & 30.4k & 0.005 & 0.5 & 0.5 & 0.7055 & 0.0226 & 0.6830 & 0.4121 & 0.0168 & 0.3953 \\
            GCN-250-L-16 & 39.6k & 0.005 & 1 & 0.1 & 0.3505 & 0.0047 & 0.3459 & 0.3317 & 0.0046 & 0.3271 \\
            GCN-2500-S-16 & 28.6k & 0.005 & 5 & 5 & 0.4817 & 0.0167 & 0.4650 & 0.4725 & 0.0181 & 0.4544 \\
            GCN-2500-L-16 & 26.4k & 0.005 & 0.5 & 0.5 & 0.5076 & 0.0225 & 0.4851 & 0.3559 & 0.0128 & 0.3431 \\
            \hline
            GCN-TF-45-S-16 & 141.6k & 0.005 & 10 & 1 & 0.3824 & 0.0052 & 0.3772 & 0.3799 & 0.0067 & 0.3732 \\
            GCN-TF-45-L-16 & 268.0k & 0.005 & 0.5 & 0.5 & 0.3177 & 0.0074 & 0.3103 & 0.3128 & 0.0061 & 0.3067 \\
            GCN-TF-250-S-16 & 181.0k & 0.005 & 10 & 1 & 0.3541 & 0.0048 & 0.3493 & 0.3613 & 0.0042 & 0.3571 \\
            GCN-TF-250-L-16 & 315.6k & 0.005 & 5 & 5 & 0.3109 & 0.0078 & 0.3031 & 0.2702 & 0.0038 & 0.2664 \\
            GCN-TF-2500-S-16 & 154.1k & 0.005 & 1 & 0.1 & 0.4221 & 0.0080 & 0.4141 & 0.4014 & 0.0079 & 0.3936 \\
            GCN-TF-2500-L-16 & 277.3k & 0.005 & 0.5 & 0.5 & 0.3277 & 0.0059 & 0.3218 & 0.3073 & 0.0063 & 0.3009 \\
            \hline
            S4-S-16 & 2.4k & 0.01 & 0.5 & 0.5 & 0.3611 & 0.0114 & 0.3498 & 0.3631 & 0.0106 & 0.3525 \\
            S4-L-16 & 19.0k & 0.01 & 10 & 1 & \textbf{0.2647} & 0.0030 & 0.2617 & 0.2553 & 0.0026 & 0.2527 \\
            \hline
            S4-TF-S-16 & 28.0k & 0.01 & 5 & 5 & 0.2984 & 0.0043 & 0.2941 & 0.3070 & 0.0044 & 0.3026 \\
            S4-TF-L-16 & 70.2k & 0.01 & 10 & 1 & 0.3591 & 0.0030 & 0.3561 & 0.2637 & 0.0028 & 0.2608 \\
            \hline
            GB-DIST-MLP & 2.2k & 0.1 (0.01) & 5 & 5 & 0.5558 & 0.0705 & 0.4854 & 0.5755 & 0.0802 & 0.4953 \\
            GB-DIST-RNL & 47 & 0.1 (1) & 0.5 & 0.5 & 0.8022 & 0.0787 & 0.7235 & 0.6380 & 0.0774 & 0.5606 \\
            \hline
            GB-FUZZ-MLP & 2.3k & 0.1 (0.01) & 5 & 5 & 0.6642 & 0.0922 & 0.5720 & 0.6618 & 0.0846 & 0.5772 \\
            GB-FUZZ-RNL & 62 & 0.1 (1) & 5 & 5 & 0.6213 & 0.0942 & 0.5271 & 0.5814 & 0.0797 & 0.5017 \\
            \hline
            \hline
        \end{tabular}
    }
\end{table*}



% 0.005	10	1	0.3481	0.0029	0.3452	0.2504	0.0029	0.2475
% 0.005	10	1	0.2932	0.0056	0.2876	0.2094	0.0025	0.2068
% 0.005	0.5	0.5	0.4900	0.0145	0.4755	0.4214	0.0125	0.4089
% 0.005	5	5	0.8015	0.0201	0.7814	0.3622	0.0088	0.3534
% 0.005	1	0.1	0.5158	0.0115	0.5043	0.4050	0.0063	0.3987
% 0.005	5	5	0.6339	0.0154	0.6185	0.3066	0.0057	0.3010
% 0.005	5	5	0.6698	0.0180	0.6518	0.4976	0.0167	0.4809
% 0.005	0.5	0.5	0.3801	0.0107	0.3694	0.3639	0.0109	0.3531
% 0.005	1	0.1	0.4427	0.0080	0.4346	0.4209	0.0072	0.4137
% 0.005	10	1	0.3814	0.0071	0.3743	0.3743	0.0062	0.3681
% 0.005	0.5	0.5	0.2752	0.0063	0.2689	0.3088	0.0057	0.3032
% 0.005	10	1	0.3149	0.0040	0.3109	0.2992	0.0037	0.2955
% 0.005	10	1	0.4834	0.0084	0.4750	0.4799	0.0076	0.4724
% 0.005	1	0.1	0.3483	0.0052	0.3431	0.3389	0.0054	0.3334
% 0.005	5	5	0.5817	0.0172	0.5645	0.4147	0.0127	0.4020
% 0.005	5	5	0.4984	0.0101	0.4884	0.3482	0.0104	0.3378
% 0.005	0.5	0.5	0.7055	0.0226	0.6830	0.4121	0.0168	0.3953
% 0.005	1	0.1	0.3505	0.0047	0.3459	0.3317	0.0046	0.3271
% 0.005	5	5	0.4817	0.0167	0.4650	0.4725	0.0181	0.4544
% 0.005	0.5	0.5	0.5076	0.0225	0.4851	0.3559	0.0128	0.3431
% 0.005	10	1	0.3824	0.0052	0.3772	0.3799	0.0067	0.3732
% 0.005	0.5	0.5	0.3177	0.0074	0.3103	0.3128	0.0061	0.3067
% 0.005	10	1	0.3541	0.0048	0.3493	0.3613	0.0042	0.3571
% 0.005	5	5	0.3109	0.0078	0.3031	0.2702	0.0038	0.2664
% 0.005	1	0.1	0.4221	0.0080	0.4141	0.4014	0.0079	0.3936
% 0.005	0.5	0.5	0.3277	0.0059	0.3218	0.3073	0.0063	0.3009
% 0.01	0.5	0.5	0.3611	0.0114	0.3498	0.3631	0.0106	0.3525
% 0.01	10	1	0.2647	0.0030	0.2617	0.2553	0.0026	0.2527
% 0.01	5	5	0.2984	0.0043	0.2941	0.3070	0.0044	0.3026
% 0.01	10	1	0.3591	0.0030	0.3561	0.2637	0.0028	0.2608
% 0.1 (0.01)	5	5	0.5558	0.0705	0.4854	0.5755	0.0802	0.4953
% 0.1 (1)	0.5	0.5	0.8022	0.0787	0.7235	0.6380	0.0774	0.5606
% 0.1 (0.01)	5	5	0.6642	0.0922	0.5720	0.6618	0.0846	0.5772
% 0.1 (1)	5	5	0.6213	0.0942	0.5271	0.5814	0.0797	0.5017
\setlength{\tabcolsep}{4pt}
\renewcommand{\arraystretch}{1.3}
\begin{table*}[h]
    \small
    \caption{
    \textit{Scaled validation and test loss for non parametric models of \textbf{DIY Klon Centaur} overdrive.}
    \textit{Bold indicates best performing models.}
    \textit{Learning rate multiplier for nonlinearity in gray-box models shown in brackets.}
    }
    \label{tab:val-and-test-loss_od_fulltone-fulldrive}
    \centerline{
        \begin{tabular}{l c c cc >{\columncolor{gray!20}}ccc >{\columncolor{gray!20}}ccc}
            \hline
            \midrule
            
            \multirow{2}{*}{Model}
                & \multirow{2}{*}{Params.}
                    & \multirow{2}{*}{LR}
                        & \multicolumn{2}{c}{Weights}
                            & \multicolumn{3}{c}{Val. Loss}
                                & \multicolumn{3}{c}{Test Loss} \\ 
            
            \cmidrule(lr){4-5} 
                \cmidrule(lr){6-8} 
                    \cmidrule(lr){9-11}
            
            &   &   & {\scriptsize L1} & {\scriptsize MR-STFT} & Tot. & {\scriptsize L1} & {\scriptsize MR-STFT} & Tot. & {\scriptsize L1} & {\scriptsize MR-STFT} \\ 
            
            \hline
            LSTM-32 & 4.5k & 0.001 & 10 & 1 & 1.8829 & 0.1647 & 1.7182 & 2.6646 & 0.1623 & 2.5023 \\
            LSTM-96 & 38.1k & 0.001 & 0.5 & 0.5 & 1.8798 & 0.1696 & 1.7102 & 1.6661 & 0.1663 & 1.4998 \\
            \hline
            TCN-45-S-16 & 7.5k & 0.005 & 5 & 5 & 1.0613 & 0.0562 & 1.0051 & 0.9993 & 0.0539 & 0.9454 \\
            TCN-45-L-16 & 7.3k & 0.005 & 0.5 & 0.5 & 1.2864 & 0.0737 & 1.2127 & 0.8964 & 0.0492 & 0.8472 \\
            TCN-250-S-16 & 14.5k & 0.005 & 5 & 5 & 1.0239 & 0.0477 & 0.9762 & 1.0135 & 0.0478 & 0.9657 \\
            TCN-250-L-16 & 18.4k & 0.005 & 10 & 1 & 1.1026 & 0.0372 & 1.0654 & 0.8163 & 0.0244 & 0.7919 \\
            TCN-2500-S-16 & 13.7k & 0.005 & 5 & 5 & 1.3573 & 0.0687 & 1.2886 & 1.1596 & 0.0670 & 1.0926 \\
            TCN-2500-L-16 & 11.9k & 0.005 & 10 & 1 & 0.8868 & 0.0349 & 0.8519 & 0.8681 & 0.0358 & 0.8323 \\
            \hline
            TCN-TF-45-S-16 & 39.5k & 0.005 & 0.5 & 0.5 & 0.8634 & 0.0487 & 0.8147 & 0.8154 & 0.0466 & 0.7688 \\
            TCN-TF-45-L-16 & 71.3k & 0.005 & 0.5 & 0.5 & 0.6667 & 0.0381 & 0.6286 & 0.6794 & 0.0354 & 0.6440 \\
            TCN-TF-250-S-16 & 52.9k & 0.005 & 5 & 5 & 0.7290 & 0.0434 & 0.6855 & 0.7251 & 0.0412 & 0.6839 \\
            TCN-TF-250-L-16 & 88.8k & 0.005 & 5 & 5 & 0.7160 & 0.0279 & 0.6881 & 0.6896 & 0.0286 & 0.6610 \\
            TCN-TF-2500-S-16 & 45.7k & 0.005 & 0.5 & 0.5 & 0.9325 & 0.0648 & 0.8677 & 0.9178 & 0.0608 & 0.8571 \\
            TCN-TF-2500-L-16 & 75.9k & 0.005 & 1 & 0.1 & 0.8690 & 0.0360 & 0.8330 & 0.8672 & 0.0352 & 0.8320 \\
            \hline
            GCN-45-S-16 & 16.2k & 0.005 & 5 & 5 & 1.1328 & 0.0562 & 1.0766 & 1.0857 & 0.0540 & 1.0317 \\
            GCN-45-L-16 & 17.1k & 0.005 & 10 & 1 & 1.0064 & 0.0290 & 0.9773 & 0.8823 & 0.0311 & 0.8512 \\
            GCN-250-S-16 & 30.4k & 0.005 & 0.5 & 0.5 & 1.1857 & 0.0620 & 1.1237 & 1.0585 & 0.0481 & 1.0104 \\
            GCN-250-L-16 & 39.6k & 0.005 & 5 & 5 & 0.8407 & 0.0309 & 0.8097 & 0.7715 & 0.0263 & 0.7452 \\
            GCN-2500-S-16 & 28.6k & 0.005 & 0.5 & 0.5 & 1.1248 & 0.0601 & 1.0647 & 1.0732 & 0.0626 & 1.0106 \\
            GCN-2500-L-16 & 26.4k & 0.005 & 5 & 5 & 0.8338 & 0.0508 & 0.7830 & 0.7494 & 0.0393 & 0.7101 \\
            \hline
            GCN-TF-45-S-16 & 141.6k & 0.005 & 5 & 5 & 0.7624 & 0.0439 & 0.7185 & 0.7586 & 0.0463 & 0.7123 \\
            GCN-TF-45-L-16 & 268.0k & 0.005 & 5 & 5 & 0.7206 & 0.0415 & 0.6791 & 0.6668 & 0.0372 & 0.6296 \\
            GCN-TF-250-S-16 & 181.0k & 0.005 & 5 & 5 & 0.6830 & 0.0400 & 0.6430 & 0.6735 & 0.0367 & 0.6369 \\
            GCN-TF-250-L-16 & 315.6k & 0.005 & 0.5 & 0.5 & 0.6484 & 0.0375 & 0.6109 & 0.6275 & 0.0301 & 0.5974 \\
            GCN-TF-2500-S-16 & 154.1k & 0.005 & 0.5 & 0.5 & 0.9044 & 0.0496 & 0.8548 & 0.8594 & 0.0485 & 0.8109 \\
            GCN-TF-2500-L-16 & 277.3k & 0.005 & 0.5 & 0.5 & 0.6993 & 0.0384 & 0.6610 & 0.6816 & 0.0406 & 0.6410 \\
            \hline
            S4-S-16 & 2.4k & 0.01 & 10 & 1 & 1.0142 & 0.0317 & 0.9824 & 1.0089 & 0.0312 & 0.9777 \\
            S4-L-16 & 19.0k & 0.01 & 1 & 0.1 & \textbf{0.5904} & 0.0211 & 0.5693 & \textbf{0.5605} & 0.0201 & 0.5404 \\
            \hline
            S4-TF-S-16 & 28.0k & 0.01 & 10 & 1 & 0.8279 & 0.0241 & 0.8039 & 0.8116 & 0.0240 & 0.7876 \\
            S4-TF-L-16 & 70.2k & 0.01 & 0.5 & 0.5 & 0.6307 & 0.0261 & 0.6046 & 0.5859 & 0.0280 & 0.5579 \\
            \hline
            GB-DIST-MLP & 2.2k & 0.1 (0.01) & 0.5 & 0.5 & 1.2171 & 0.1173 & 1.0998 & 1.2269 & 0.1200 & 1.1069 \\
            GB-DIST-RNL & 47 & 0.1 (1) & 0.5 & 0.5 & 0.9195 & 0.1014 & 0.8180 & 0.9397 & 0.1090 & 0.8306 \\
            \hline
            GB-FUZZ-MLP & 2.3k & 0.1 (0.01) & 1 & 0.1 & 1.1539 & 0.0871 & 1.0667 & 1.1462 & 0.0876 & 1.0586 \\
            GB-FUZZ-RNL & 62 & 0.1 (1) & 5 & 5 & 1.2280 & 0.1116 & 1.1164 & 1.1943 & 0.1095 & 1.0849 \\
            \hline
            \hline
        \end{tabular}
    }
\end{table*}



% 0.001	10	1	1.8829	0.1647	1.7182	2.6646	0.1623	2.5023
% 0.001	0.5	0.5	1.8798	0.1696	1.7102	1.6661	0.1663	1.4998
% 0.005	5	5	1.0613	0.0562	1.0051	0.9993	0.0539	0.9454
% 0.005	0.5	0.5	1.2864	0.0737	1.2127	0.8964	0.0492	0.8472
% 0.005	5	5	1.0239	0.0477	0.9762	1.0135	0.0478	0.9657
% 0.005	10	1	1.1026	0.0372	1.0654	0.8163	0.0244	0.7919
% 0.005	5	5	1.3573	0.0687	1.2886	1.1596	0.0670	1.0926
% 0.005	10	1	0.8868	0.0349	0.8519	0.8681	0.0358	0.8323
% 0.005	0.5	0.5	0.8634	0.0487	0.8147	0.8154	0.0466	0.7688
% 0.005	0.5	0.5	0.6667	0.0381	0.6286	0.6794	0.0354	0.6440
% 0.005	5	5	0.7290	0.0434	0.6855	0.7251	0.0412	0.6839
% 0.005	5	5	0.7160	0.0279	0.6881	0.6896	0.0286	0.6610
% 0.005	0.5	0.5	0.9325	0.0648	0.8677	0.9178	0.0608	0.8571
% 0.005	1	0.1	0.8690	0.0360	0.8330	0.8672	0.0352	0.8320
% 0.005	5	5	1.1328	0.0562	1.0766	1.0857	0.0540	1.0317
% 0.005	10	1	1.0064	0.0290	0.9773	0.8823	0.0311	0.8512
% 0.005	0.5	0.5	1.1857	0.0620	1.1237	1.0585	0.0481	1.0104
% 0.005	5	5	0.8407	0.0309	0.8097	0.7715	0.0263	0.7452
% 0.005	0.5	0.5	1.1248	0.0601	1.0647	1.0732	0.0626	1.0106
% 0.005	5	5	0.8338	0.0508	0.7830	0.7494	0.0393	0.7101
% 0.005	5	5	0.7624	0.0439	0.7185	0.7586	0.0463	0.7123
% 0.005	5	5	0.7206	0.0415	0.6791	0.6668	0.0372	0.6296
% 0.005	5	5	0.6830	0.0400	0.6430	0.6735	0.0367	0.6369
% 0.005	0.5	0.5	0.6484	0.0375	0.6109	0.6275	0.0301	0.5974
% 0.005	0.5	0.5	0.9044	0.0496	0.8548	0.8594	0.0485	0.8109
% 0.005	0.5	0.5	0.6993	0.0384	0.6610	0.6816	0.0406	0.6410
% 0.01	10	1	1.0142	0.0317	0.9824	1.0089	0.0312	0.9777
% 0.01	1	0.1	0.5904	0.0211	0.5693	0.5605	0.0201	0.5404
% 0.01	10	1	0.8279	0.0241	0.8039	0.8116	0.0240	0.7876
% 0.01	0.5	0.5	0.6307	0.0261	0.6046	0.5859	0.0280	0.5579
% 0.1 (0.01)	0.5	0.5	1.2171	0.1173	1.0998	1.2269	0.1200	1.1069
% 0.1 (1)	0.5	0.5	0.9195	0.1014	0.8180	0.9397	0.1090	0.8306
% 0.1 (0.01)	1	0.1	1.1539	0.0871	1.0667	1.1462	0.0876	1.0586
% 0.1 (1)	5	5	1.2280	0.1116	1.1164	1.1943	0.1095	1.0849

\setlength{\tabcolsep}{4pt}
\renewcommand{\arraystretch}{1.3}
\begin{table*}[h]
    \small
    \caption{
    \textit{Objective metrics for non parametric models of \textbf{Fulltone Fulldrive 2} overdrive.}
    \textit{Bold indicates best performing models.}
    \textit{Learning rate multiplier for nonlinearity in gray-box models shown in brackets.}
    }
    \label{tab:metrics_od_fulltone-fulldrive}
    \centerline{
        \begin{tabular}{lccccccccccccc}
            \hline
            \midrule
            
            \multirow{2}{*}{Model}
                & \multirow{2}{*}{Params.}
                    & \multirow{2}{*}{LR}
                        & \multicolumn{2}{c}{Weights}
                            & \multicolumn{3}{c}{}
                                & \multicolumn{3}{c}{FAD} \\ 
            \cmidrule(lr){4-5} 
                % \cmidrule(lr){6-8} 
                    \cmidrule(lr){9-12}
            
            &   &   & L1 & MR-STFT & MSE & ESR & MAPE & VGGish & PANN & CLAP & AFx-Rep \\ 
            
            \hline
            LSTM-32 & 4.5k & 0.005 & 10 & 1 & 1.68e-03 & 0.1362 & 3.9998 & 1.4119 & 3.43e-05 & 0.1293 & 0.4988 \\
            LSTM-96 & 38.1k & 0.005 & 1 & 0.1 & 8.11e-05 & 0.0082 & 6.9465 & 0.0667 & 2.05e-07 & 0.0133 & 0.0217 \\
            \hline
            TCN-45-S-16 & 7.5k & 0.005 & 5 & 5 & 8.97e-05 & 0.0082 & 1.2545 & 0.0751 & 3.13e-07 & 0.0107 & 0.0159 \\
            TCN-45-L-16 & 7.3k & 0.005 & 5 & 5 & 9.98e-05 & 0.0086 & 1.1470 & 0.0565 & 1.47e-07 & 0.0113 & 0.0209 \\
            TCN-250-S-16 & 14.5k & 0.005 & 0.5 & 0.5 & 3.65e-05 & 0.0034 & 0.7144 & 0.0529 & 2.62e-07 & 0.0106 & 0.0099 \\
            TCN-250-L-16 & 18.4k & 0.005 & 1 & 0.1 & 2.10e-05 & 0.0019 & 0.7343 & 0.0632 & 3.90e-06 & 0.0105 & 0.0485 \\
            TCN-2500-S-16 & 13.7k & 0.005 & 10 & 1 & 3.06e-05 & 0.0029 & 1.1176 & 0.0792 & 4.23e-07 & 0.0106 & 0.0220 \\
            TCN-2500-L-16 & 11.9k & 0.005 & 1 & 0.1 & 2.15e-05 & 0.0020 & 0.5055 & 0.0584 & 1.18e-06 & 0.0094 & 0.0223 \\
            \hline
            TCN-TF-45-S-16 & 39.5k & 0.005 & 1 & 0.1 & 2.30e-05 & 0.0020 & 0.8225 & 0.0686 & 6.78e-07 & 0.0105 & 0.0177 \\
            TCN-TF-45-L-16 & 71.3k & 0.005 & 10 & 1 & 2.26e-05 & 0.0020 & 1.1366 & 0.0555 & 2.68e-07 & 0.0098 & 0.0208 \\
            TCN-TF-250-S-16 & 52.9k & 0.005 & 1 & 0.1 & 2.22e-05 & 0.0020 & 0.5282 & 0.0558 & 1.42e-07 & 0.0110 & 0.0441 \\
            TCN-TF-250-L-16 & 88.8k & 0.005 & 10 & 1 & 1.67e-05 & 0.0014 & 0.6382 & 0.0477 & 2.64e-06 & 0.0124 & 0.0279 \\
            TCN-TF-2500-S-16 & 45.7k & 0.005 & 5 & 5 & 3.48e-05 & 0.0032 & 0.6971 & 0.0419 & 4.81e-07 & 0.0088 & 0.0353 \\
            TCN-TF-2500-L-16 & 75.9k & 0.005 & 10 & 1 & 2.33e-05 & 0.0021 & 0.8519 & 0.0555 & 1.11e-06 & \textbf{0.0074} & 0.0230 \\
            \hline
            GCN-45-S-16 & 16.2k & 0.005 & 5 & 5 & 4.29e-05 & 0.0039 & 0.7902 & 0.0706 & 1.43e-07 & 0.0111 & 0.0161 \\
            GCN-45-L-16 & 17.1k & 0.005 & 5 & 5 & 1.38e-04 & 0.0121 & 1.2566 & 0.0526 & 8.82e-08 & 0.0107 & 0.0152 \\
            GCN-250-S-16 & 30.4k & 0.005 & 1 & 0.1 & 2.58e-05 & 0.0024 & 1.5628 & 0.0659 & 2.79e-07 & 0.0113 & 0.0162 \\
            GCN-250-L-16 & 39.6k & 0.005 & 10 & 1 & 2.01e-05 & 0.0018 & 0.5185 & 0.0504 & 6.07e-07 & 0.0116 & 0.0121 \\
            GCN-2500-S-16 & 28.6k & 0.005 & 5 & 5 & 4.73e-05 & 0.0045 & 1.8441 & 0.0617 & 2.60e-07 & 0.0092 & 0.0122 \\
            GCN-2500-L-16 & 26.4k & 0.005 & 10 & 1 & 2.15e-05 & 0.0020 & 0.4165 & 0.0632 & 8.68e-08 & 0.0086 & 0.0282 \\
            \hline
            GCN-TF-45-S-16 & 141.6k & 0.005 & 10 & 1 & 1.75e-05 & 0.0015 & 0.4074 & 0.0515 & 1.68e-08 & 0.0087 & 0.0210 \\
            GCN-TF-45-L-16 & 268.0k & 0.005 & 1 & 0.1 & 1.88e-05 & 0.0017 & 0.4704 & 0.0487 & 9.30e-07 & 0.0096 & 0.0148 \\
            GCN-TF-250-S-16 & 181.0k & 0.005 & 10 & 1 & 2.83e-05 & 0.0025 & 0.5976 & 0.0897 & 1.19e-06 & 0.0098 & 0.0248 \\
            GCN-TF-250-L-16 & 315.6k & 0.005 & 1 & 0.1 & 2.02e-05 & 0.0018 & 0.7378 & 0.0538 & 9.96e-07 & 0.0097 & 0.0142 \\
            GCN-TF-2500-S-16 & 154.1k & 0.005 & 1 & 0.1 & 2.32e-05 & 0.0021 & 0.4487 & 0.0563 & 7.83e-07 & 0.0085 & 0.0164 \\
            GCN-TF-2500-L-16 & 277.3k & 0.005 & 5 & 5 & 6.88e-05 & 0.0060 & 0.7360 & 0.0452 & 8.33e-08 & 0.0078 & 0.0207 \\
            \hline
            S4-S-16 & 2.4k & 0.01 & 10 & 1 & 1.22e-05 & 0.0011 & 0.6766 & 0.0403 & 8.82e-08 & 0.0108 & 0.0183 \\
            S4-L-16 & 19.0k & 0.01 & 1 & 0.1 & \textbf{1.14e-05} & \textbf{0.0010} & \textbf{0.3585} & 0.0151 & 3.47e-07 & 0.0082 & 0.0090 \\
            \hline
            S4-TF-S-16 & 28.0k & 0.01 & 10 & 1 & 1.25e-05 & 0.0011 & 0.3888 & 0.0212 & \textbf{1.16e-08} & 0.0087 & 0.0108 \\
            S4-TF-L-16 & 70.2k & 0.01 & 1 & 0.1 & 1.19e-05 & 0.0010 & 0.3658 & \textbf{0.0140} & 3.05e-08 & 0.0080 & \textbf{0.0089} \\
            \hline
            GB-DIST-MLP & 2.2k & 0.1 (0.01) & 5 & 5 & 7.68e-04 & 0.0548 & 1.7275 & 0.1413 & 2.75e-07 & 0.0134 & 0.2571 \\
            GB-DIST-RNL & 47 & 0.1 (1) & 0.5 & 0.5 & 7.28e-04 & 0.0542 & 1.6332 & 0.1725 & 4.32e-07 & 0.0259 & 0.3159 \\
            \hline
            GB-FUZZ-MLP & 2.3k & 0.1 (0.01) & 1 & 0.1 & 6.76e-04 & 0.0489 & 1.5442 & 0.1614 & 4.63e-07 & 0.0158 & 0.2621 \\
            GB-FUZZ-RNL & 62 & 0.1 (1) & 0.5 & 0.5 & 7.21e-04 & 0.0557 & 1.9976 & 0.2280 & 2.20e-06 & 0.0276 & 0.3003 \\
            \hline
            \hline
        \end{tabular}
    }
\end{table*}


% 1.68E-03	0.1362	3.9998	1.4119	3.43E-05	0.1293	0.4988
% 8.11E-05	0.0082	6.9465	0.0667	2.05E-07	0.0133	0.0217
% 8.97E-05	0.0082	1.2545	0.0751	3.13E-07	0.0107	0.0159
% 9.98E-05	0.0086	1.1470	0.0565	1.47E-07	0.0113	0.0209
% 3.65E-05	0.0034	0.7144	0.0529	2.62E-07	0.0106	0.0099
% 2.10E-05	0.0019	0.7343	0.0632	3.90E-06	0.0105	0.0485
% 3.06E-05	0.0029	1.1176	0.0792	4.23E-07	0.0106	0.0220
% 2.15E-05	0.0020	0.5055	0.0584	1.18E-06	0.0094	0.0223
% 2.30E-05	0.0020	0.8225	0.0686	6.78E-07	0.0105	0.0177
% 2.26E-05	0.0020	1.1366	0.0555	2.68E-07	0.0098	0.0208
% 2.22E-05	0.0020	0.5282	0.0558	1.42E-07	0.0110	0.0441
% 1.67E-05	0.0014	0.6382	0.0477	2.64E-06	0.0124	0.0279
% 3.48E-05	0.0032	0.6971	0.0419	4.81E-07	0.0088	0.0353
% 2.33E-05	0.0021	0.8519	0.0555	1.11E-06	0.0074	0.0230
% 4.29E-05	0.0039	0.7902	0.0706	1.43E-07	0.0111	0.0161
% 1.38E-04	0.0121	1.2566	0.0526	8.82E-08	0.0107	0.0152
% 2.58E-05	0.0024	1.5628	0.0659	2.79E-07	0.0113	0.0162
% 2.01E-05	0.0018	0.5185	0.0504	6.07E-07	0.0116	0.0121
% 4.73E-05	0.0045	1.8441	0.0617	2.60E-07	0.0092	0.0122
% 2.15E-05	0.0020	0.4165	0.0632	8.68E-08	0.0086	0.0282
% 1.75E-05	0.0015	0.4074	0.0515	1.68E-08	0.0087	0.0210
% 1.88E-05	0.0017	0.4704	0.0487	9.30E-07	0.0096	0.0148
% 2.83E-05	0.0025	0.5976	0.0897	1.19E-06	0.0098	0.0248
% 2.02E-05	0.0018	0.7378	0.0538	9.96E-07	0.0097	0.0142
% 2.32E-05	0.0021	0.4487	0.0563	7.83E-07	0.0085	0.0164
% 6.88E-05	0.0060	0.7360	0.0452	8.33E-08	0.0078	0.0207
% 1.22E-05	0.0011	0.6766	0.0403	8.82E-08	0.0108	0.0183
% 1.14E-05	0.0010	0.3585	0.0151	3.47E-07	0.0082	0.0090
% 1.25E-05	0.0011	0.3888	0.0212	1.16E-08	0.0087	0.0108
% 1.19E-05	0.0010	0.3658	0.0140	3.05E-08	0.0080	0.0089
% 7.68E-04	0.0548	1.7275	0.1413	2.75E-07	0.0134	0.2571
% 7.28E-04	0.0542	1.6332	0.1725	4.32E-07	0.0259	0.3159
% 6.76E-04	0.0489	1.5442	0.1614	4.63E-07	0.0158	0.2621
% 7.21E-04	0.0557	1.9976	0.2280	2.20E-06	0.0276	0.3003
\setlength{\tabcolsep}{4pt}
\renewcommand{\arraystretch}{1.3}
\begin{table*}[h]
    \small
    \caption{
    \textit{Objective metrics for non parametric models of \textbf{Harley Benton Green Tint} overdrive.}
    \textit{Bold indicates best performing models.}
    \textit{Learning rate multiplier for nonlinearity in gray-box models shown in brackets.}
    }
    \label{tab:metrics_od_hb-green-tint}
    \centerline{
        \begin{tabular}{lccccccccccccc}
            \hline
            \midrule
            
            \multirow{2}{*}{Model}
                & \multirow{2}{*}{Params.}
                    & \multirow{2}{*}{LR}
                        & \multicolumn{2}{c}{Weights}
                            & \multicolumn{3}{c}{}
                                & \multicolumn{3}{c}{FAD} \\ 
            \cmidrule(lr){4-5} 
                % \cmidrule(lr){6-8} 
                    \cmidrule(lr){9-12}
            
            &   &   & L1 & MR-STFT & MSE & ESR & MAPE & VGGish & PANN & CLAP & AFx-Rep \\ 
            
            \hline
            LSTM-32 & 4.5k & 0.005 & 10 & 1 & 2.81e-03 & 0.2160 & 5.4028 & 1.8410 & 8.41e-06 & 0.2023 & 0.2814 \\
            LSTM-96 & 38.1k & 0.005 & 1 & 0.1 & 3.86e-03 & 0.2963 & 4.1289 & 2.2347 & 8.33e-05 & 0.1941 & 0.3846 \\
            \hline
            TCN-45-S-16 & 7.5k & 0.005 & 5 & 5 & 7.40e-05 & 0.0062 & 1.3801 & 0.1096 & 7.32e-06 & 0.0284 & 0.0693 \\
            TCN-45-L-16 & 7.3k & 0.005 & 5 & 5 & 1.95e-04 & 0.0123 & 1.1080 & 0.1268 & 3.56e-06 & 0.0179 & 0.0442 \\
            TCN-250-S-16 & 14.5k & 0.005 & 0.5 & 0.5 & 8.19e-05 & 0.0055 & 0.7237 & 0.1133 & 7.10e-06 & 0.0225 & 0.0793 \\
            TCN-250-L-16 & 18.4k & 0.005 & 1 & 0.1 & 1.95e-04 & 0.0123 & 0.9553 & 0.1236 & 5.66e-06 & 0.0160 & 0.0457 \\
            TCN-2500-S-16 & 13.7k & 0.005 & 10 & 1 & 7.98e-05 & 0.0068 & 0.8872 & 0.1394 & 9.40e-06 & 0.0169 & 0.0756 \\
            TCN-2500-L-16 & 11.9k & 0.005 & 1 & 0.1 & 1.09e-04 & 0.0071 & 1.0042 & 0.1034 & 1.04e-05 & 0.0165 & 0.0609 \\
            \hline
            TCN-TF-45-S-16 & 39.5k & 0.005 & 1 & 0.1 & 8.06e-05 & 0.0065 & 1.3401 & 0.1486 & 5.07e-06 & 0.0326 & 0.1232 \\
            TCN-TF-45-L-16 & 71.3k & 0.005 & 10 & 1 & 6.94e-05 & 0.0058 & 0.9144 & 0.1783 & 5.06e-06 & 0.0348 & 0.1291 \\
            TCN-TF-250-S-16 & 52.9k & 0.005 & 1 & 0.1 & 8.43e-05 & 0.0062 & 0.8361 & 0.1649 & 3.04e-06 & 0.0302 & 0.1349 \\
            TCN-TF-250-L-16 & 88.8k & 0.005 & 10 & 1 & 7.42e-04 & 0.0465 & 1.3120 & 0.1064 & 1.01e-06 & 0.0231 & 0.0534 \\
            TCN-TF-2500-S-16 & 45.7k & 0.005 & 5 & 5 & 6.78e-05 & 0.0056 & 0.6322 & 0.1245 & 1.23e-06 & 0.0199 & 0.0789 \\
            TCN-TF-2500-L-16 & 75.9k & 0.005 & 10 & 1 & 2.19e-04 & 0.0137 & 0.8151 & 0.1283 & 2.76e-06 & 0.0136 & 0.0394 \\
            \hline
            GCN-45-S-16 & 16.2k & 0.005 & 5 & 5 & 6.96e-05 & 0.0050 & 0.9516 & 0.1197 & 1.09e-05 & 0.0334 & 0.0925 \\
            GCN-45-L-16 & 17.1k & 0.005 & 5 & 5 & 1.46e-04 & 0.0094 & 1.6361 & 0.0994 & 5.76e-06 & 0.0277 & 0.0670 \\
            GCN-250-S-16 & 30.4k & 0.005 & 1 & 0.1 & 6.89e-05 & 0.0047 & 0.7354 & 0.0902 & 1.10e-05 & 0.0209 & 0.0411 \\
            GCN-250-L-16 & 39.6k & 0.005 & 10 & 1 & 5.74e-05 & 0.0040 & 0.6825 & 0.1075 & 6.10e-06 & 0.0201 & 0.0396 \\
            GCN-2500-S-16 & 28.6k & 0.005 & 5 & 5 & 7.92e-05 & 0.0062 & 0.6932 & 0.0977 & 1.14e-05 & 0.0165 & 0.0350 \\
            GCN-2500-L-16 & 26.4k & 0.005 & 10 & 1 & 5.07e-05 & 0.0043 & 0.5802 & 0.1010 & 4.54e-06 & 0.0155 & 0.0899 \\
            \hline
            GCN-TF-45-S-16 & 141.6k & 0.005 & 10 & 1 & 1.19e-04 & 0.0096 & 1.0578 & 0.1855 & 3.56e-06 & 0.0490 & 0.1563 \\
            GCN-TF-45-L-16 & 268.0k & 0.005 & 1 & 0.1 & 1.58e-04 & 0.0099 & 0.8804 & 0.1111 & 4.11e-06 & 0.0199 & 0.0685 \\
            GCN-TF-250-S-16 & 181.0k & 0.005 & 10 & 1 & 5.84e-05 & 0.0046 & 0.8507 & 0.1358 & 7.14e-07 & 0.0249 & 0.0667 \\
            GCN-TF-250-L-16 & 315.6k & 0.005 & 1 & 0.1 & 7.09e-05 & 0.0056 & 0.6383 & 0.1275 & 2.41e-07 & 0.0378 & 0.1462 \\
            GCN-TF-2500-S-16 & 154.1k & 0.005 & 1 & 0.1 & 6.38e-05 & 0.0050 & 0.9412 & 0.1304 & 5.86e-06 & 0.0165 & 0.0353 \\
            GCN-TF-2500-L-16 & 277.3k & 0.005 & 5 & 5 & 5.79e-05 & \textbf{0.0040} & 0.6065 & 0.1080 & 4.26e-06 & 0.0140 & \textbf{0.0254} \\
            \hline
            S4-S-16 & 2.4k & 0.01 & 10 & 1 & 5.09e-05 & 0.0044 & 0.5667 & 0.0929 & 9.65e-06 & 0.0139 & 0.0320 \\
            S4-L-16 & 19.0k & 0.01 & 1 & 0.1 & \textbf{4.93e-05} & 0.0042 & \textbf{0.5328} & \textbf{0.0833} & 9.98e-06 & \textbf{0.0120} & 0.0309 \\
            \hline
            S4-TF-S-16 & 28.0k & 0.01 & 10 & 1 & 8.69e-05 & 0.0074 & 0.6419 & 0.0953 & 3.92e-06 & 0.0198 & 0.0365 \\
            S4-TF-L-16 & 70.2k & 0.01 & 1 & 0.1 & 5.59e-05 & 0.0049 & 0.5545 & 0.0884 & 1.51e-06 & 0.0202 & 0.0265 \\
            \hline
            GB-DIST-MLP & 2.2k & 0.1 (0.01) & 5 & 5 & 1.64e-03 & 0.1019 & 1.6994 & 0.1594 & 5.57e-07 & 0.0156 & 0.3358 \\
            GB-DIST-RNL & 47 & 0.1 (1) & 0.5 & 0.5 & 1.63e-03 & 0.1058 & 2.3980 & 0.2697 & \textbf{2.36e-07} & 0.0384 & 0.2337 \\
            \hline
            GB-FUZZ-MLP & 2.3k & 0.1 (0.01) & 1 & 0.1 & 1.52e-03 & 0.0937 & 2.0278 & 0.2103 & 2.00e-06 & 0.0169 & 0.2225 \\
            GB-FUZZ-RNL & 62 & 0.1 (1) & 0.5 & 0.5 & 1.67e-03 & 0.1049 & 3.0199 & 0.5670 & 2.28e-06 & 0.0623 & 0.5714 \\
            \hline
            \hline
        \end{tabular}
    }
\end{table*}



% 2.81e-03	0.2160	5.4028	1.8410	8.41e-06	0.2023	0.2814
% 3.86e-03	0.2963	4.1289	2.2347	8.33e-05	0.1941	0.3846
% 7.40e-05	0.0062	1.3801	0.1096	7.32e-06	0.0284	0.0693
% 1.95e-04	0.0123	1.1080	0.1268	3.56e-06	0.0179	0.0442
% 8.19e-05	0.0055	0.7237	0.1133	7.10e-06	0.0225	0.0793
% 1.95e-04	0.0123	0.9553	0.1236	5.66e-06	0.0160	0.0457
% 7.98e-05	0.0068	0.8872	0.1394	9.40e-06	0.0169	0.0756
% 1.09e-04	0.0071	1.0042	0.1034	1.04e-05	0.0165	0.0609
% 8.06e-05	0.0065	1.3401	0.1486	5.07e-06	0.0326	0.1232
% 6.94e-05	0.0058	0.9144	0.1783	5.06e-06	0.0348	0.1291
% 8.43e-05	0.0062	0.8361	0.1649	3.04e-06	0.0302	0.1349
% 7.42e-04	0.0465	1.3120	0.1064	1.01e-06	0.0231	0.0534
% 6.78e-05	0.0056	0.6322	0.1245	1.23e-06	0.0199	0.0789
% 2.19e-04	0.0137	0.8151	0.1283	2.76e-06	0.0136	0.0394
% 6.96e-05	0.0050	0.9516	0.1197	1.09e-05	0.0334	0.0925
% 1.46e-04	0.0094	1.6361	0.0994	5.76e-06	0.0277	0.0670
% 6.89e-05	0.0047	0.7354	0.0902	1.10e-05	0.0209	0.0411
% 5.74e-05	0.0040	0.6825	0.1075	6.10e-06	0.0201	0.0396
% 7.92e-05	0.0062	0.6932	0.0977	1.14e-05	0.0165	0.0350
% 5.07e-05	0.0043	0.5802	0.1010	4.54e-06	0.0155	0.0899
% 1.19e-04	0.0096	1.0578	0.1855	3.56e-06	0.0490	0.1563
% 1.58e-04	0.0099	0.8804	0.1111	4.11e-06	0.0199	0.0685
% 5.84e-05	0.0046	0.8507	0.1358	7.14e-07	0.0249	0.0667
% 7.09e-05	0.0056	0.6383	0.1275	2.41e-07	0.0378	0.1462
% 6.38e-05	0.0050	0.9412	0.1304	5.86e-06	0.0165	0.0353
% 5.79e-05	0.0040	0.6065	0.1080	4.26e-06	0.0140	0.0254
% 5.09e-05	0.0044	0.5667	0.0929	9.65e-06	0.0139	0.0320
% 4.93e-05	0.0042	0.5328	0.0833	9.98e-06	0.0120	0.0309
% 8.69e-05	0.0074	0.6419	0.0953	3.92e-06	0.0198	0.0365
% 5.59e-05	0.0049	0.5545	0.0884	1.51e-06	0.0202	0.0265
% 1.64e-03	0.1019	1.6994	0.1594	5.57e-07	0.0156	0.3358
% 1.63e-03	0.1058	2.3980	0.2697	2.36e-07	0.0384	0.2337
% 1.52e-03	0.0937	2.0278	0.2103	2.00e-06	0.0169	0.2225
% 1.67e-03	0.1049	3.0199	0.5670	2.28e-06	0.0623	0.5714
\setlength{\tabcolsep}{4pt}
\renewcommand{\arraystretch}{1.3}
\begin{table*}[h]
    \small
    \caption{
    \textit{Objective metrics for non parametric models of \textbf{Ibanez TS9} overdrive.}
    \textit{Bold indicates best performing models.}
    \textit{Learning rate multiplier for nonlinearity in gray-box models shown in brackets.}
    }
    \label{tab:metrics_od_hb-green-tint}
    \centerline{
        \begin{tabular}{lccccccccccccc}
            \hline
            \midrule
            
            \multirow{2}{*}{Model}
                & \multirow{2}{*}{Params.}
                    & \multirow{2}{*}{LR}
                        & \multicolumn{2}{c}{Weights}
                            & \multicolumn{3}{c}{}
                                & \multicolumn{3}{c}{FAD} \\ 
            \cmidrule(lr){4-5} 
                % \cmidrule(lr){6-8} 
                    \cmidrule(lr){9-12}
            
            &   &   & L1 & MR-STFT & MSE & ESR & MAPE & VGGish & PANN & CLAP & AFx-Rep \\ 
            
            \hline
            LSTM-32 & 4.5k & 0.005 & 10 & 1 & 2.08e-05 & 0.0004 & 0.1472 & 0.0453 & 4.34e-08 & 0.0043 & 0.0093 \\
            LSTM-96 & 38.1k & 0.005 & 1 & 0.1 & \textbf{1.70e-05} & \textbf{0.0003} & 0.1314 & \textbf{0.0142} & 7.18e-08 & 0.0034 & \textbf{0.0053} \\
            \hline
            TCN-45-S-16 & 7.5k & 0.005 & 5 & 5 & 2.78e-04 & 0.0057 & 0.4798 & 0.1983 & 2.09e-06 & 0.0100 & 0.1609 \\
            TCN-45-L-16 & 7.3k & 0.005 & 5 & 5 & 1.45e-04 & 0.0030 & 0.3891 & 0.1701 & 6.40e-06 & 0.0073 & 0.1142 \\
            TCN-250-S-16 & 14.5k & 0.005 & 0.5 & 0.5 & 8.28e-05 & 0.0017 & 0.2950 & 0.2580 & 2.80e-05 & 0.0229 & 0.1640 \\
            TCN-250-L-16 & 18.4k & 0.005 & 1 & 0.1 & 6.62e-05 & 0.0013 & 0.2789 & 0.1291 & 4.94e-07 & 0.0060 & 0.0642 \\
            TCN-2500-S-16 & 13.7k & 0.005 & 10 & 1 & 5.18e-04 & 0.0106 & 0.6641 & 0.3674 & 3.95e-06 & 0.0073 & 0.1678 \\
            TCN-2500-L-16 & 11.9k & 0.005 & 1 & 0.1 & 2.52e-04 & 0.0051 & 0.5024 & 0.1956 & 5.17e-06 & 0.0069 & 0.1580 \\
            \hline
            TCN-TF-45-S-16 & 39.5k & 0.005 & 1 & 0.1 & 1.22e-04 & 0.0025 & 0.2897 & 0.2762 & 1.50e-05 & 0.0069 & 0.1585 \\
            TCN-TF-45-L-16 & 71.3k & 0.005 & 10 & 1 & 8.81e-05 & 0.0018 & 0.2615 & 0.2927 & 3.54e-05 & 0.0046 & 0.1104 \\
            TCN-TF-250-S-16 & 52.9k & 0.005 & 1 & 0.1 & 7.04e-05 & 0.0014 & 0.2937 & 0.1097 & 2.66e-06 & 0.0030 & 0.0398 \\
            TCN-TF-250-L-16 & 88.8k & 0.005 & 10 & 1 & 3.70e-05 & 0.0007 & 0.1618 & 0.0881 & 1.02e-05 & 0.0039 & 0.0634 \\
            TCN-TF-2500-S-16 & 45.7k & 0.005 & 5 & 5 & 1.32e-04 & 0.0027 & 0.2822 & 0.6526 & 6.00e-05 & 0.0149 & 0.1723 \\
            TCN-TF-2500-L-16 & 75.9k & 0.005 & 10 & 1 & 6.95e-05 & 0.0014 & 0.2202 & 0.1951 & 1.28e-05 & 0.0032 & 0.0972 \\
            \hline
            GCN-45-S-16 & 16.2k & 0.005 & 5 & 5 & 2.99e-04 & 0.0061 & 0.5453 & 0.3169 & 1.84e-07 & 0.0115 & 0.1526 \\
            GCN-45-L-16 & 17.1k & 0.005 & 5 & 5 & 2.07e-04 & 0.0042 & 0.4218 & 0.2012 & 6.56e-07 & 0.0061 & 0.1304 \\
            GCN-250-S-16 & 30.4k & 0.005 & 1 & 0.1 & 5.61e-04 & 0.0114 & 0.8541 & 0.2388 & 1.40e-05 & 0.0114 & 0.1425 \\
            GCN-250-L-16 & 39.6k & 0.005 & 10 & 1 & 4.68e-05 & 0.0009 & 0.2054 & 0.2092 & 1.66e-06 & 0.0099 & 0.0562 \\
            GCN-2500-S-16 & 28.6k & 0.005 & 5 & 5 & 6.24e-04 & 0.0127 & 0.7235 & 0.3332 & 3.85e-07 & 0.0064 & 0.1276 \\
            GCN-2500-L-16 & 26.4k & 0.005 & 10 & 1 & 3.71e-04 & 0.0075 & 0.5852 & 0.1818 & 6.87e-08 & 0.0048 & 0.1353 \\
            \hline
            GCN-TF-45-S-16 & 141.6k & 0.005 & 10 & 1 & 9.72e-05 & 0.0020 & 0.2776 & 0.1940 & 5.53e-06 & 0.0046 & 0.1080 \\
            GCN-TF-45-L-16 & 268.0k & 0.005 & 1 & 0.1 & 7.99e-05 & 0.0016 & 0.2797 & 0.1215 & 6.36e-08 & 0.0048 & 0.0399 \\
            GCN-TF-250-S-16 & 181.0k & 0.005 & 10 & 1 & 4.96e-05 & 0.0010 & 0.1768 & 0.1024 & 4.53e-06 & 0.0080 & 0.0747 \\
            GCN-TF-250-L-16 & 315.6k & 0.005 & 1 & 0.1 & 3.70e-05 & 0.0008 & 0.1779 & 0.0765 & \textbf{2.64e-08} & 0.0032 & 0.0084 \\
            GCN-TF-2500-S-16 & 154.1k & 0.005 & 1 & 0.1 & 1.29e-04 & 0.0026 & 0.3129 & 0.1873 & 3.86e-06 & 0.0062 & 0.1041 \\
            GCN-TF-2500-L-16 & 277.3k & 0.005 & 5 & 5 & 8.39e-05 & 0.0017 & 0.2697 & 0.1372 & 4.13e-06 & 0.0045 & 0.0398 \\
            \hline
            S4-S-16 & 2.4k & 0.01 & 10 & 1 & 2.95e-04 & 0.0059 & 0.4295 & 0.0998 & 1.44e-07 & 0.0061 & 0.1221 \\
            S4-L-16 & 19.0k & 0.01 & 1 & 0.1 & 2.08e-05 & 0.0004 & \textbf{0.1037} & 0.0590 & 6.87e-07 & 0.0072 & 0.0090 \\
            \hline
            S4-TF-S-16 & 28.0k & 0.01 & 10 & 1 & 4.54e-05 & 0.0009 & 0.2013 & 0.0730 & 1.86e-06 & 0.0042 & 0.0149 \\
            S4-TF-L-16 & 70.2k & 0.01 & 1 & 0.1 & 2.41e-05 & 0.0005 & 0.1114 & 0.0711 & 3.40e-07 & \textbf{0.0028} & 0.0084 \\
            \hline
            GB-DIST-MLP & 2.2k & 0.1 (0.01) & 5 & 5 & 1.30e-02 & 0.2593 & 3.2119 & 0.1769 & 2.53e-06 & 0.0070 & 0.0709 \\
            GB-DIST-RNL & 47 & 0.1 (1) & 0.5 & 0.5 & 1.18e-02 & 0.2356 & 2.9920 & 0.4412 & 1.79e-06 & 0.0169 & 0.0382 \\
            \hline
            GB-FUZZ-MLP & 2.3k & 0.1 (0.01) & 1 & 0.1 & 1.40e-02 & 0.2800 & 3.3089 & 0.3732 & 4.45e-06 & 0.0077 & 0.0792 \\
            GB-FUZZ-RNL & 62 & 0.1 (1) & 0.5 & 0.5 & 1.27e-02 & 0.2545 & 3.1520 & 0.2248 & 1.95e-06 & 0.0116 & 0.0484 \\
            \hline
            \hline
        \end{tabular}
    }
\end{table*}


% 2.08e-05	0.0004	0.1472	0.0453	4.34e-08	0.0043	0.0093
% 1.70e-05	0.0003	0.1314	0.0142	7.18e-08	0.0034	0.0053
% 2.78e-04	0.0057	0.4798	0.1983	2.09e-06	0.0100	0.1609
% 1.45e-04	0.0030	0.3891	0.1701	6.40e-06	0.0073	0.1142
% 8.28e-05	0.0017	0.2950	0.2580	2.80e-05	0.0229	0.1640
% 6.62e-05	0.0013	0.2789	0.1291	4.94e-07	0.0060	0.0642
% 5.18e-04	0.0106	0.6641	0.3674	3.95e-06	0.0073	0.1678
% 2.52e-04	0.0051	0.5024	0.1956	5.17e-06	0.0069	0.1580
% 1.22e-04	0.0025	0.2897	0.2762	1.50e-05	0.0069	0.1585
% 8.81e-05	0.0018	0.2615	0.2927	3.54e-05	0.0046	0.1104
% 7.04e-05	0.0014	0.2937	0.1097	2.66e-06	0.0030	0.0398
% 3.70e-05	0.0007	0.1618	0.0881	1.02e-05	0.0039	0.0634
% 1.32e-04	0.0027	0.2822	0.6526	6.00e-05	0.0149	0.1723
% 6.95e-05	0.0014	0.2202	0.1951	1.28e-05	0.0032	0.0972
% 2.99e-04	0.0061	0.5453	0.3169	1.84e-07	0.0115	0.1526
% 2.07e-04	0.0042	0.4218	0.2012	6.56e-07	0.0061	0.1304
% 5.61e-04	0.0114	0.8541	0.2388	1.40e-05	0.0114	0.1425
% 4.68e-05	0.0009	0.2054	0.2092	1.66e-06	0.0099	0.0562
% 6.24e-04	0.0127	0.7235	0.3332	3.85e-07	0.0064	0.1276
% 3.71e-04	0.0075	0.5852	0.1818	6.87e-08	0.0048	0.1353
% 9.72e-05	0.0020	0.2776	0.1940	5.53e-06	0.0046	0.1080
% 7.99e-05	0.0016	0.2797	0.1215	6.36e-08	0.0048	0.0399
% 4.96e-05	0.0010	0.1768	0.1024	4.53e-06	0.0080	0.0747
% 3.70e-05	0.0008	0.1779	0.0765	2.64e-08	0.0032	0.0084
% 1.29e-04	0.0026	0.3129	0.1873	3.86e-06	0.0062	0.1041
% 8.39e-05	0.0017	0.2697	0.1372	4.13e-06	0.0045	0.0398
% 2.95e-04	0.0059	0.4295	0.0998	1.44e-07	0.0061	0.1221
% 2.08e-05	0.0004	0.1037	0.0590	6.87e-07	0.0072	0.0090
% 4.54e-05	0.0009	0.2013	0.0730	1.86e-06	0.0042	0.0149
% 2.41e-05	0.0005	0.1114	0.0711	3.40e-07	0.0028	0.0084
% 1.30e-02	0.2593	3.2119	0.1769	2.53e-06	0.0070	0.0709
% 1.18e-02	0.2356	2.9920	0.4412	1.79e-06	0.0169	0.0382
% 1.40e-02	0.2800	3.3089	0.3732	4.45e-06	0.0077	0.0792
% 1.27e-02	0.2545	3.1520	0.2248	1.95e-06	0.0116	0.0484
\setlength{\tabcolsep}{4pt}
\renewcommand{\arraystretch}{1.3}
\begin{table*}[h]
    \small
    \caption{
    \textit{Objective metrics for non parametric models of \textbf{DIY Klon Centaur} overdrive.}
    \textit{Bold indicates best performing models.}
    \textit{Learning rate multiplier for nonlinearity in gray-box models shown in brackets.}
    }
    \label{tab:metrics_od_diy-klon-centaur}
    \centerline{
        \begin{tabular}{lccccccccccccc}
            \hline
            \midrule
            
            \multirow{2}{*}{Model}
                & \multirow{2}{*}{Params.}
                    & \multirow{2}{*}{LR}
                        & \multicolumn{2}{c}{Weights}
                            & \multicolumn{3}{c}{}
                                & \multicolumn{3}{c}{FAD} \\ 
            \cmidrule(lr){4-5} 
                % \cmidrule(lr){6-8} 
                    \cmidrule(lr){9-12}
            
            &   &   & L1 & MR-STFT & MSE & ESR & MAPE & VGGish & PANN & CLAP & AFx-Rep \\ 
            \hline
            LSTM-32 & 4.5k & 0.005 & 10 & 1 & 4.66e-02 & 0.2111 & 2.9196 & 9.6966 & 9.05e-04 & 0.2174 & 0.8175 \\
            LSTM-96 & 38.1k & 0.005 & 1 & 0.1 & 4.80e-02 & 0.2181 & 2.7740 & 3.9958 & 2.09e-04 & 0.1629 & 0.7006 \\
            \hline
            TCN-45-S-16 & 7.5k & 0.005 & 5 & 5 & 5.16e-03 & 0.0235 & 1.0840 & 0.8840 & 5.03e-05 & 0.0238 & 0.4548 \\
            TCN-45-L-16 & 7.3k & 0.005 & 5 & 5 & 4.36e-03 & 0.0198 & 1.0185 & 0.7198 & 6.48e-05 & 0.0260 & 0.4765 \\
            TCN-250-S-16 & 14.5k & 0.005 & 0.5 & 0.5 & 4.09e-03 & 0.0186 & 1.0556 & 0.8927 & 8.37e-05 & 0.0219 & 0.5331 \\
            TCN-250-L-16 & 18.4k & 0.005 & 1 & 0.1 & 1.18e-03 & 0.0053 & 0.4629 & 0.7519 & 2.62e-06 & 0.0216 & 0.5483 \\
            TCN-2500-S-16 & 13.7k & 0.005 & 10 & 1 & 8.01e-03 & 0.0365 & 1.3660 & 1.4930 & 2.50e-05 & 0.0434 & 0.6563 \\
            TCN-2500-L-16 & 11.9k & 0.005 & 1 & 0.1 & 2.39e-03 & 0.0109 & 0.6854 & 1.0190 & 3.10e-05 & 0.0254 & 0.5104 \\
            \hline
            TCN-TF-45-S-16 & 39.5k & 0.005 & 1 & 0.1 & 4.05e-03 & 0.0185 & 0.8284 & 1.0321 & 5.56e-05 & 0.0226 & 0.2177 \\
            TCN-TF-45-L-16 & 71.3k & 0.005 & 10 & 1 & 2.42e-03 & 0.0110 & 0.6456 & 0.7566 & 8.71e-06 & 0.0158 & 0.1387 \\
            TCN-TF-250-S-16 & 52.9k & 0.005 & 1 & 0.1 & 3.13e-03 & 0.0143 & 0.7445 & 0.8421 & 8.14e-06 & 0.0250 & 0.2000 \\
            TCN-TF-250-L-16 & 88.8k & 0.005 & 10 & 1 & 1.56e-03 & 0.0071 & 0.5315 & 0.8354 & 3.79e-06 & 0.0215 & 0.3536 \\
            TCN-TF-2500-S-16 & 45.7k & 0.005 & 5 & 5 & 6.84e-03 & 0.0312 & 1.1772 & 1.2828 & 6.68e-05 & 0.0306 & 0.4468 \\
            TCN-TF-2500-L-16 & 75.9k & 0.005 & 10 & 1 & 2.43e-03 & 0.0110 & 0.6983 & 0.7679 & 2.21e-05 & 0.0203 & 0.5172 \\
            \hline
            GCN-45-S-16 & 16.2k & 0.005 & 5 & 5 & 5.13e-03 & 0.0234 & 1.1088 & 0.9727 & 9.33e-06 & 0.0295 & 0.5996 \\
            GCN-45-L-16 & 17.1k & 0.005 & 5 & 5 & 1.82e-03 & 0.0083 & 0.6039 & 0.7229 & 3.32e-06 & 0.0248 & 0.5758 \\
            GCN-250-S-16 & 30.4k & 0.005 & 1 & 0.1 & 4.35e-03 & 0.0197 & 0.9686 & 1.1377 & 5.29e-07 & 0.0439 & 0.5762 \\
            GCN-250-L-16 & 39.6k & 0.005 & 10 & 1 & 1.34e-03 & 0.0061 & 0.5192 & 0.5905 & 3.30e-06 & 0.0187 & 0.5209 \\
            GCN-2500-S-16 & 28.6k & 0.005 & 5 & 5 & 7.09e-03 & 0.0323 & 1.2700 & 1.4552 & 7.52e-05 & 0.0379 & 0.7264 \\
            GCN-2500-L-16 & 26.4k & 0.005 & 10 & 1 & 2.85e-03 & 0.0129 & 0.7419 & 0.8812 & 1.82e-05 & \textbf{0.0150} & 0.3475 \\
            \hline
            GCN-TF-45-S-16 & 141.6k & 0.005 & 10 & 1 & 3.95e-03 & 0.0180 & 0.8750 & 1.0308 & 1.90e-05 & 0.0226 & 0.2466 \\
            GCN-TF-45-L-16 & 268.0k & 0.005 & 1 & 0.1 & 2.65e-03 & 0.0120 & 0.7103 & 0.6903 & 2.28e-05 & 0.0247 & 0.1464 \\
            GCN-TF-250-S-16 & 181.0k & 0.005 & 10 & 1 & 2.52e-03 & 0.0115 & 0.7152 & 0.8511 & \textbf{4.07e-07} & 0.0210 & 0.2873 \\
            GCN-TF-250-L-16 & 315.6k & 0.005 & 1 & 0.1 & 1.82e-03 & 0.0082 & 0.6382 & 0.8487 & 9.19e-06 & 0.0224 & 0.2755 \\
            GCN-TF-2500-S-16 & 154.1k & 0.005 & 1 & 0.1 & 4.46e-03 & 0.0203 & 0.8914 & 1.0105 & 1.04e-06 & 0.0248 & 0.2522 \\
            GCN-TF-2500-L-16 & 277.3k & 0.005 & 5 & 5 & 3.20e-03 & 0.0145 & 0.7610 & 0.8648 & 1.20e-05 & 0.0197 & 0.1330 \\
            \hline
            S4-S-16 & 2.4k & 0.01 & 10 & 1 & 2.05e-03 & 0.0093 & 0.6192 & 0.7942 & 6.93e-05 & 0.0221 & 0.4940 \\
            S4-L-16 & 19.0k & 0.01 & 1 & 0.1 & \textbf{8.45e-04} & \textbf{0.0038} & \textbf{0.3859} & \textbf{0.5087} & 3.12e-05 & 0.0268 & \textbf{0.0540} \\
            \hline
            S4-TF-S-16 & 28.0k & 0.01 & 10 & 1 & 1.16e-03 & 0.0053 & 0.4169 & 0.8394 & 2.19e-05 & 0.0290 & 0.1286 \\
            S4-TF-L-16 & 70.2k & 0.01 & 1 & 0.1 & 1.49e-03 & 0.0068 & 0.4804 & 0.6001 & 3.30e-05 & 0.0228 & 0.1001 \\
            \hline
            GB-DIST-MLP & 2.2k & 0.1 (0.01) & 5 & 5 & 3.18e-02 & 0.1442 & 2.7599 & 1.8080 & 1.35e-05 & 0.0500 & 0.3970 \\
            GB-DIST-RNL & 47 & 0.1 (1) & 0.5 & 0.5 & 2.64e-02 & 0.1201 & 2.5006 & 1.7624 & 5.50e-05 & 0.0464 & 0.1012 \\
            \hline
            GB-FUZZ-MLP & 2.3k & 0.1 (0.01) & 1 & 0.1 & 1.56e-02 & 0.0703 & 1.8575 & 2.3194 & 1.37e-05 & 0.0515 & 0.4289 \\
            GB-FUZZ-RNL & 62 & 0.1 (1) & 0.5 & 0.5 & 2.66e-02 & 0.1206 & 2.4504 & 1.9103 & 3.91e-05 & 0.0567 & 0.4804 \\
            \hline
            \hline
        \end{tabular}
    }
\end{table*}



% 4.66e-02	0.2111	2.9196	9.6966	9.05e-04	0.2174	0.8175
% 4.80e-02	0.2181	2.7740	3.9958	2.09e-04	0.1629	0.7006
% 5.16e-03	0.0235	1.0840	0.8840	5.03e-05	0.0238	0.4548
% 4.36e-03	0.0198	1.0185	0.7198	6.48e-05	0.0260	0.4765
% 4.09e-03	0.0186	1.0556	0.8927	8.37e-05	0.0219	0.5331
% 1.18e-03	0.0053	0.4629	0.7519	2.62e-06	0.0216	0.5483
% 8.01e-03	0.0365	1.3660	1.4930	2.50e-05	0.0434	0.6563
% 2.39e-03	0.0109	0.6854	1.0190	3.10e-05	0.0254	0.5104
% 4.05e-03	0.0185	0.8284	1.0321	5.56e-05	0.0226	0.2177
% 2.42e-03	0.0110	0.6456	0.7566	8.71e-06	0.0158	0.1387
% 3.13e-03	0.0143	0.7445	0.8421	8.14e-06	0.0250	0.2000
% 1.56e-03	0.0071	0.5315	0.8354	3.79e-06	0.0215	0.3536
% 6.84e-03	0.0312	1.1772	1.2828	6.68e-05	0.0306	0.4468
% 2.43e-03	0.0110	0.6983	0.7679	2.21e-05	0.0203	0.5172
% 5.13e-03	0.0234	1.1088	0.9727	9.33e-06	0.0295	0.5996
% 1.82e-03	0.0083	0.6039	0.7229	3.32e-06	0.0248	0.5758
% 4.35e-03	0.0197	0.9686	1.1377	5.29e-07	0.0439	0.5762
% 1.34e-03	0.0061	0.5192	0.5905	3.30e-06	0.0187	0.5209
% 7.09e-03	0.0323	1.2700	1.4552	7.52e-05	0.0379	0.7264
% 2.85e-03	0.0129	0.7419	0.8812	1.82e-05	0.0150	0.3475
% 3.95e-03	0.0180	0.8750	1.0308	1.90e-05	0.0226	0.2466
% 2.65e-03	0.0120	0.7103	0.6903	2.28e-05	0.0247	0.1464
% 2.52e-03	0.0115	0.7152	0.8511	4.07e-07	0.0210	0.2873
% 1.82e-03	0.0082	0.6382	0.8487	9.19e-06	0.0224	0.2755
% 4.46e-03	0.0203	0.8914	1.0105	1.04e-06	0.0248	0.2522
% 3.20e-03	0.0145	0.7610	0.8648	1.20e-05	0.0197	0.1330
% 2.05e-03	0.0093	0.6192	0.7942	6.93e-05	0.0221	0.4940
% 8.45e-04	0.0038	0.3859	0.5087	3.12e-05	0.0268	0.0540
% 1.16e-03	0.0053	0.4169	0.8394	2.19e-05	0.0290	0.1286
% 1.49e-03	0.0068	0.4804	0.6001	3.30e-05	0.0228	0.1001
% 3.18e-02	0.1442	2.7599	1.8080	1.35e-05	0.0500	0.3970
% 2.64e-02	0.1201	2.5006	1.7624	5.50e-05	0.0464	0.1012
% 1.56e-02	0.0703	1.8575	2.3194	1.37e-05	0.0515	0.4289
% 2.66e-02	0.1206	2.4504	1.9103	3.91e-05	0.0567	0.4804
\clearpage

% ===
\subsection{Results Distortion}
\setlength{\tabcolsep}{3pt}
\renewcommand{\arraystretch}{1.3}
\begin{table*}[h]
    \small
    \caption{
    \textit{Scaled test loss for non parametric models of distortion effects. Bold indicates best performing models.}
    }
    \label{tab:testloss_dist}
    \centerline{
        \begin{tabular}{l c >{\columncolor{gray!20}}ccc >{\columncolor{gray!20}}ccc >{\columncolor{gray!20}}ccc >{\columncolor{gray!20}}ccc} 
            \hline
            \hline
            \multirow{2}{*}{Model} 
                & \multirow{2}{*}{Params.} 
                    & \multicolumn{3}{c}{Electro Harmonix Big Muff} 
                        &  \multicolumn{3}{c}{Harley Benton DropKick} 
                            & \multicolumn{3}{c}{Harley Benton Plexicon} 
                                & \multicolumn{3}{c}{Harley Benton Rodent} \\
            \cmidrule(lr){3-5} 
                \cmidrule(lr){6-8} 
                    \cmidrule(lr){9-11} 
                        \cmidrule(lr){12-14}
                        
            & & 
            Tot. & {\footnotesize L1} &  {\footnotesize MR-STFT} & 
            Tot. & {\footnotesize L1} &  {\footnotesize MR-STFT} & 
            Tot. & {\footnotesize L1} &  {\footnotesize MR-STFT} & 
            Tot. & {\footnotesize L1} &  {\footnotesize MR-STFT} \\ 

            \hline
            LSTM-32 & 4.5k & 0.7679 & 0.0294 & 0.7385 & 1.7506 & 0.2284 & 1.5222 & 0.3323 & 0.0163 & 0.3161 & 1.5315 & 0.0605 & 1.4710 \\
            LSTM-96 & 38.1k & 0.5265 & 0.0033 & 0.5232 & 1.9392 & 0.2181 & 1.7211 & \textbf{0.1973} & 0.0118 & 0.1855 & 1.6861 & 0.0569 & 1.6292 \\
            \hline
            TCN-45-S-16 & 7.5k & 0.7548 & 0.0069 & 0.7479 & 1.2535 & 0.1964 & 1.0571 & 0.6290 & 0.0361 & 0.5928 & 1.0722 & 0.0254 & 1.0468 \\
            TCN-45-L-16 & 7.3k & 0.7348 & 0.0086 & 0.7262 & 1.2598 & 0.1986 & 1.0612 & 0.8903 & 0.0633 & 0.8270 & 0.9984 & 0.0238 & 0.9746 \\
            TCN-250-S-16 & 14.5k & 0.7035 & 0.0054 & 0.6981 & 1.2548 & 0.1818 & 1.0731 & 0.5389 & 0.0286 & 0.5103 & 0.9287 & 0.0179 & 0.9108 \\
            TCN-250-L-16 & 18.4k & 0.7371 & 0.0114 & 0.7257 & 1.2506 & 0.1749 & 1.0757 & 0.6132 & 0.0306 & 0.5826 & 0.8366 & 0.0154 & 0.8212 \\
            TCN-2500-S-16 & 13.7k & 0.7942 & 0.0293 & 0.7649 & 1.3208 & 0.1875 & 1.1333 & 0.6608 & 0.0366 & 0.6242 & 0.9234 & 0.0210 & 0.9024 \\
            TCN-2500-L-16 & 11.9k & 0.6920 & 0.0081 & 0.6839 & 1.2115 & 0.1862 & 1.0253 & 0.5284 & 0.0259 & 0.5025 & 0.8330 & 0.0158 & 0.8172 \\
            \hline
            TCN-TF-45-S-16 & 39.5k & 0.7316 & 0.0063 & 0.7253 & 1.0231 & 0.1272 & 0.8959 & 0.5479 & 0.0273 & 0.5206 & 0.7775 & 0.0128 & 0.7647 \\
            TCN-TF-45-L-16 & 71.3k & 0.6598 & 0.0051 & 0.6547 & 0.9189 & 0.1068 & 0.8120 & 0.4759 & 0.0212 & 0.4548 & 0.7786 & 0.0130 & 0.7656 \\
            TCN-TF-250-S-16 & 52.9k & 0.6592 & 0.0049 & 0.6543 & 0.9865 & 0.1165 & 0.8699 & 0.4664 & 0.0218 & 0.4447 & 0.7400 & 0.0131 & 0.7269 \\
            TCN-TF-250-L-16 & 88.8k & 0.6370 & 0.0055 & 0.6315 & \textbf{0.8746} & 0.1003 & 0.7743 & 0.3724 & 0.0172 & 0.3552 & 0.7820 & 0.0132 & 0.7688 \\
            TCN-TF-2500-S-16 & 45.7k & 0.7411 & 0.0064 & 0.7346 & 0.9293 & 0.1063 & 0.8230 & 0.5622 & 0.0295 & 0.5327 & 0.7771 & 0.0128 & 0.7643 \\
            TCN-TF-2500-L-16 & 75.9k & 0.5985 & 0.0045 & 0.5940 & 0.8906 & 0.0470 & 0.8437 & 0.4559 & 0.0191 & 0.4368 & 0.6239 & 0.0101 & 0.6138 \\
            \hline
            GCN-45-S-16 & 16.2k & 0.8324 & 0.0071 & 0.8253 & 1.2207 & 0.1935 & 1.0272 & 0.4597 & 0.0224 & 0.4373 & 0.9744 & 0.0213 & 0.9531 \\
            GCN-45-L-16 & 17.1k & 0.7556 & 0.0063 & 0.7493 & 1.2218 & 0.2033 & 1.0186 & 0.4755 & 0.0299 & 0.4456 & 0.9480 & 0.0212 & 0.9267 \\
            GCN-250-S-16 & 30.4k & 0.6634 & 0.0050 & 0.6585 & 1.2195 & 0.1886 & 1.0309 & 0.4842 & 0.0256 & 0.4585 & 0.8881 & 0.0181 & 0.8699 \\
            GCN-250-L-16 & 39.6k & 0.6737 & 0.0051 & 0.6687 & 1.2022 & 0.1843 & 1.0179 & 0.4123 & 0.0220 & 0.3903 & 0.8468 & 0.0141 & 0.8327 \\
            GCN-2500-S-16 & 28.6k & 0.7132 & 0.0056 & 0.7076 & 1.2034 & 0.1873 & 1.0161 & 0.5805 & 0.0340 & 0.5465 & 0.8929 & 0.0175 & 0.8755 \\
            GCN-2500-L-16 & 26.4k & 0.6765 & 0.0093 & 0.6673 & 1.1590 & 0.1874 & 0.9717 & 0.4753 & 0.0221 & 0.4532 & 0.7830 & 0.0132 & 0.7698 \\
            \hline
            GCN-TF-45-S-16 & 141.6k & 0.6238 & 0.0052 & 0.6186 & 0.9847 & 0.1168 & 0.8679 & 0.4612 & 0.0226 & 0.4386 & 0.7742 & 0.0124 & 0.7618 \\
            GCN-TF-45-L-16 & 268.0k & 0.5918 & 0.0063 & 0.5855 & 0.9676 & 0.1141 & 0.8536 & 0.3965 & 0.0168 & 0.3797 & 0.6863 & 0.0097 & 0.6766 \\
            GCN-TF-250-S-16 & 181.0k & 0.6044 & 0.0054 & 0.5990 & 1.0128 & 0.1141 & 0.8987 & 0.4179 & 0.0199 & 0.3980 & 0.7671 & 0.0122 & 0.7549 \\
            GCN-TF-250-L-16 & 315.6k & 0.5846 & 0.0059 & 0.5787 & 0.9204 & 0.1052 & 0.8151 & 0.3275 & 0.0150 & 0.3125 & 0.7992 & 0.0142 & 0.7850 \\
            GCN-TF-2500-S-16 & 154.1k & 0.5978 & 0.0043 & 0.5935 & 0.9940 & 0.1169 & 0.8770 & 0.5037 & 0.0252 & 0.4785 & 0.7182 & 0.0113 & 0.7069 \\
            GCN-TF-2500-L-16 & 277.3k & 0.6238 & 0.0295 & 0.5944 & 0.9613 & 0.1092 & 0.8521 & 0.3872 & 0.0182 & 0.3690 & 0.6247 & 0.0081 & 0.6166 \\
            \hline
            S4-S-16 & 2.4k & 0.6475 & 0.0048 & 0.6427 & 1.1464 & 0.2090 & 0.9374 & 0.4505 & 0.0212 & 0.4292 & 0.7245 & 0.0125 & 0.7120 \\
            S4-L-16 & 19.0k & 0.6004 & 0.0042 & 0.5961 & 1.0607 & 0.2635 & 0.7972 & 0.2982 & 0.0131 & 0.2851 & 0.6367 & 0.0091 & 0.6276 \\
            \hline
            S4-TF-S-16 & 28.0k & 0.5514 & 0.0037 & 0.5477 & 1.0327 & 0.1461 & 0.8865 & 0.4146 & 0.0161 & 0.3985 & 0.6367 & 0.0089 & 0.6279 \\
            S4-TF-L-16 & 70.2k & \textbf{0.5078} & 0.0031 & 0.5046 & 0.9052 & 0.1323 & 0.7729 & 0.2449 & 0.0121 & 0.2329 & \textbf{0.5816} & 0.0077 & 0.5738 \\
            \hline
            GB-DIST-MLP & 2.2k & 0.9649 & 0.0122 & 0.9527 & 1.5608 & 0.3081 & 1.2527 & 0.9509 & 0.0826 & 0.8683 & 1.2826 & 0.0368 & 1.2458 \\
            GB-DIST-RNL & 47 & 0.9608 & 0.0125 & 0.9483 & 1.5215 & 0.3105 & 1.2110 & 0.9273 & 0.0760 & 0.8512 & 1.3654 & 0.0392 & 1.3262 \\
            \hline
            GB-FUZZ-MLP & 2.3k & 0.9671 & 0.0117 & 0.9554 & 1.5600 & 0.3116 & 1.2484 & 0.9657 & 0.0777 & 0.8880 & 1.2751 & 0.0357 & 1.2394 \\
            GB-FUZZ-RNL & 62 & 0.9626 & 0.0121 & 0.9505 & 1.5124 & 0.3115 & 1.2008 & 0.9827 & 0.0883 & 0.8944 & 1.3963 & 0.0450 & 1.3513 \\
            \hline
            \hline
        \end{tabular}
    }
\end{table*}


% 0.7679	0.0294	0.7385	1.7506	0.2284	1.5222	0.3323	0.0163	0.3161	1.5315	0.0605	1.4710
% 0.5265	0.0033	0.5232	1.9392	0.2181	1.7211	0.1973	0.0118	0.1855	1.6861	0.0569	1.6292
% 0.7548	0.0069	0.7479	1.2535	0.1964	1.0571	0.6290	0.0361	0.5928	1.0722	0.0254	1.0468
% 0.7348	0.0086	0.7262	1.2598	0.1986	1.0612	0.8903	0.0633	0.8270	0.9984	0.0238	0.9746
% 0.7035	0.0054	0.6981	1.2548	0.1818	1.0731	0.5389	0.0286	0.5103	0.9287	0.0179	0.9108
% 0.7371	0.0114	0.7257	1.2506	0.1749	1.0757	0.6132	0.0306	0.5826	0.8366	0.0154	0.8212
% 0.7942	0.0293	0.7649	1.3208	0.1875	1.1333	0.6608	0.0366	0.6242	0.9234	0.0210	0.9024
% 0.6920	0.0081	0.6839	1.2115	0.1862	1.0253	0.5284	0.0259	0.5025	0.8330	0.0158	0.8172
% 0.7316	0.0063	0.7253	1.0231	0.1272	0.8959	0.5479	0.0273	0.5206	0.7775	0.0128	0.7647
% 0.6598	0.0051	0.6547	0.9189	0.1068	0.8120	0.4759	0.0212	0.4548	0.7786	0.0130	0.7656
% 0.6592	0.0049	0.6543	0.9865	0.1165	0.8699	0.4664	0.0218	0.4447	0.7400	0.0131	0.7269
% 0.6370	0.0055	0.6315	0.8746	0.1003	0.7743	0.3724	0.0172	0.3552	0.7820	0.0132	0.7688
% 0.7411	0.0064	0.7346	0.9293	0.1063	0.8230	0.5622	0.0295	0.5327	0.7771	0.0128	0.7643
% 0.5985	0.0045	0.5940	0.8906	0.0470	0.8437	0.4559	0.0191	0.4368	0.6239	0.0101	0.6138
% 0.8324	0.0071	0.8253	1.2207	0.1935	1.0272	0.4597	0.0224	0.4373	0.9744	0.0213	0.9531
% 0.7556	0.0063	0.7493	1.2218	0.2033	1.0186	0.4755	0.0299	0.4456	0.9480	0.0212	0.9267
% 0.6634	0.0050	0.6585	1.2195	0.1886	1.0309	0.4842	0.0256	0.4585	0.8881	0.0181	0.8699
% 0.6737	0.0051	0.6687	1.2022	0.1843	1.0179	0.4123	0.0220	0.3903	0.8468	0.0141	0.8327
% 0.7132	0.0056	0.7076	1.2034	0.1873	1.0161	0.5805	0.0340	0.5465	0.8929	0.0175	0.8755
% 0.6765	0.0093	0.6673	1.1590	0.1874	0.9717	0.4753	0.0221	0.4532	0.7830	0.0132	0.7698
% 0.6238	0.0052	0.6186	0.9847	0.1168	0.8679	0.4612	0.0226	0.4386	0.7742	0.0124	0.7618
% 0.5918	0.0063	0.5855	0.9676	0.1141	0.8536	0.3965	0.0168	0.3797	0.6863	0.0097	0.6766
% 0.6044	0.0054	0.5990	1.0128	0.1141	0.8987	0.4179	0.0199	0.3980	0.7671	0.0122	0.7549
% 0.5846	0.0059	0.5787	0.9204	0.1052	0.8151	0.3275	0.0150	0.3125	0.7992	0.0142	0.7850
% 0.5978	0.0043	0.5935	0.9940	0.1169	0.8770	0.5037	0.0252	0.4785	0.7182	0.0113	0.7069
% 0.6238	0.0295	0.5944	0.9613	0.1092	0.8521	0.3872	0.0182	0.3690	0.6247	0.0081	0.6166
% 0.6475	0.0048	0.6427	1.1464	0.2090	0.9374	0.4505	0.0212	0.4292	0.7245	0.0125	0.7120
% 0.6004	0.0042	0.5961	1.0607	0.2635	0.7972	0.2982	0.0131	0.2851	0.6367	0.0091	0.6276
% 0.5514	0.0037	0.5477	1.0327	0.1461	0.8865	0.4146	0.0161	0.3985	0.6367	0.0089	0.6279
% 0.5078	0.0031	0.5046	0.9052	0.1323	0.7729	0.2449	0.0121	0.2329	0.5816	0.0077	0.5738
% 0.9649	0.0122	0.9527	1.5608	0.3081	1.2527	0.9509	0.0826	0.8683	1.2826	0.0368	1.2458
% 0.9608	0.0125	0.9483	1.5215	0.3105	1.2110	0.9273	0.0760	0.8512	1.3654	0.0392	1.3262
% 0.9671	0.0117	0.9554	1.5600	0.3116	1.2484	0.9657	0.0777	0.8880	1.2751	0.0357	1.2394
% 0.9626	0.0121	0.9505	1.5124	0.3115	1.2008	0.9827	0.0883	0.8944	1.3963	0.0450	1.3513






% %\setlength{\tabcolsep}{3.8pt}
% % \vspace{-0.3cm}
% % \renewcommand{\arraystretch}{0.85}
% \begin{table*}[h]
%     \centering
%     \small
%     \begin{tabular}{lcccccccccccc} \toprule
    
%         \multirow{2}{*}{Model} 
%             & \multirow{2}{*}{Params.} 
%                 & \multicolumn{2}{c}{Dist 1} 
%                     & \multicolumn{2}{c}{Dist 2} 
%                         &  \multicolumn{2}{c}{Dist 3} 
%                             & \multicolumn{2}{c}{Dist 4} \\ 
%         \cmidrule(lr){3-4} 
%             \cmidrule(lr){5-6} 
%                 \cmidrule(lr){7-8} 
%                     \cmidrule(lr){9-10}
%         &   & $L1$ & MR-STFT & $L1$ & MR-STFT & $L1$ & MR-STFT & $L1$ & MR-STFT \\ 
%         \midrule
%         LSTM-32
%             & 4.5k  & - & - & - & - & - & - & - & - \\ 
%         LSTM-96       
%             & - & - & - & - & - & - & - & - & - \\ 
%         \midrule
%         LSTM-TVC-32
%             & - & - & - & - & - & - & - & - & - \\
%         LSTM-TVC-96
%             & - & - & - & - & - & - & - & - & - \\
%         \midrule
%         TCN-45-S-16               
%             & - & - & - & - & - & - & - & - & - \\ 
%         TCN-45-L-16               
%             & - & - & - & - & - & - & - & - & - \\
%         \midrule
%         TCN-TF-45-S-16               
%             & - & - & - & - & - & - & - & - & - \\
%         TCN-TF-45-L-16               
%             & - & - & - & - & - & - & - & - & - \\
%         \midrule
%         TCN-TTF-45-S-16               
%             & - & - & - & - & - & - & - & - & - \\
%         TCN-TTF-45-L-16               
%             & - & - & - & - & - & - & - & - & - \\
%         \midrule
%         TCN-TVF-45-S-16               
%             & - & - & - & - & - & - & - & - & - \\
%         TCN-TVF-45-L-16               
%             & - & - & - & - & - & - & - & - & - \\
%         \midrule
%         GCN-45-S-16               
%             & - & - & - & - & - & - & - & - & - \\ 
%         GCN-45-L-16               
%             & - & - & - & - & - & - & - & - & - \\
%         \midrule
%         GCN-TF-45-S-16               
%             & - & - & - & - & - & - & - & - & - \\
%         GCN-TF-45-L-16               
%             & - & - & - & - & - & - & - & - & - \\
%         \midrule
%         GCN-TTF-45-S-16               
%             & - & - & - & - & - & - & - & - & - \\
%         GCN-TTF-45-L-16               
%             & - & - & - & - & - & - & - & - & - \\
%         \midrule
%         GCN-TVF-45-S-16               
%             & - & - & - & - & - & - & - & - & - \\
%         GCN-TVF-45-L-16               
%             & - & - & - & - & - & - & - & - & - \\
%         \midrule
%         SSM-1               
%             & - & - & - & - & - & - & - & - & - \\
%         SSM-2               
%             & - & - & - & - & - & - & - & - & - \\
%         \midrule
%         GB-COMP              
%             & - & - & - & - & - & - & - & - & - \\
%         \midrule
%         GB-DIST-MLP             
%             & - & - & - & - & - & - & - & - & - \\
%         GB-DIST-RNL             
%             & - & - & - & - & - & - & - & - & - \\
%         \midrule
%         GB-FUZZ-MLP             
%             & - & - & - & - & - & - & - & - & - \\
%         GB-FUZZ-RNL             
%             & - & - & - & - & - & - & - & - & - \\
%         \bottomrule 
%     \end{tabular}
%     \vspace{-0.0cm}
%     \caption{Overall L1+MR-STFT loss across device type for non parametric models}
%     \label{tab:other_fx} \vspace{0.2cm}
% \end{table*}

\setlength{\tabcolsep}{4pt}
\renewcommand{\arraystretch}{1.3}
\begin{table*}[h]
    \small
    \caption{
    \textit{Scaled validation and test loss for non parametric models of \textbf{Electro Harmonix Big Muff} distortion.}
    \textit{Bold indicates best performing models.}
    \textit{Learning rate multiplier for nonlinearity in gray-box models shown in brackets.}
    }
    \label{tab:val-and-test-loss_od_fulltone-fulldrive}
    \centerline{
        \begin{tabular}{l c c cc >{\columncolor{gray!20}}ccc >{\columncolor{gray!20}}ccc}
            \hline
            \midrule
            
            \multirow{2}{*}{Model}
                & \multirow{2}{*}{Params.}
                    & \multirow{2}{*}{LR}
                        & \multicolumn{2}{c}{Weights}
                            & \multicolumn{3}{c}{Val. Loss}
                                & \multicolumn{3}{c}{Test Loss} \\ 
            
            \cmidrule(lr){4-5} 
                \cmidrule(lr){6-8} 
                    \cmidrule(lr){9-11}
            
            &   &   & {\scriptsize L1} & {\scriptsize MR-STFT} & Tot. & {\scriptsize L1} & {\scriptsize MR-STFT} & Tot. & {\scriptsize L1} & {\scriptsize MR-STFT} \\ 
            
            \hline
            LSTM-32 & 4.5k & 0.001 & 5 & 5 & 0.6528 & 0.0300 & 0.6229 & 0.7679 & 0.0294 & 0.7385 \\
            LSTM-96 & 38.1k & 0.001 & 5 & 5 & 0.4140 & 0.0018 & 0.4122 & 0.5265 & 0.0033 & 0.5232 \\
            \hline
            TCN-45-S-16 & 7.5k & 0.005 & 5 & 5 & 0.5976 & 0.0051 & 0.5924 & 0.7548 & 0.0069 & 0.7479 \\
            TCN-45-L-16 & 7.3k & 0.005 & 0.5 & 0.5 & 0.6823 & 0.0056 & 0.6767 & 0.7348 & 0.0086 & 0.7262 \\
            TCN-250-S-16 & 14.5k & 0.005 & 1 & 0.1 & 0.5135 & 0.0029 & 0.5106 & 0.7035 & 0.0054 & 0.6981 \\
            TCN-250-L-16 & 18.4k & 0.005 & 0.5 & 0.5 & 0.5437 & 0.0087 & 0.5350 & 0.7371 & 0.0114 & 0.7257 \\
            TCN-2500-S-16 & 13.7k & 0.005 & 0.5 & 0.5 & 0.6246 & 0.0165 & 0.6081 & 0.7942 & 0.0293 & 0.7649 \\
            TCN-2500-L-16 & 11.9k & 0.005 & 0.5 & 0.5 & 0.4922 & 0.0058 & 0.4864 & 0.6920 & 0.0081 & 0.6839 \\
            \hline
            TCN-TF-45-S-16 & 39.5k & 0.005 & 1 & 0.1 & 0.6927 & 0.0053 & 0.6875 & 0.7316 & 0.0063 & 0.7253 \\
            TCN-TF-45-L-16 & 71.3k & 0.005 & 0.5 & 0.5 & 0.4993 & 0.0034 & 0.4959 & 0.6598 & 0.0051 & 0.6547 \\
            TCN-TF-250-S-16 & 52.9k & 0.005 & 10 & 1 & 0.5144 & 0.0036 & 0.5108 & 0.6592 & 0.0049 & 0.6543 \\
            TCN-TF-250-L-16 & 88.8k & 0.005 & 1 & 0.1 & 0.4474 & 0.0035 & 0.4439 & 0.6370 & 0.0055 & 0.6315 \\
            TCN-TF-2500-S-16 & 45.7k & 0.005 & 10 & 1 & 0.6608 & 0.0052 & 0.6555 & 0.7411 & 0.0064 & 0.7346 \\
            TCN-TF-2500-L-16 & 75.9k & 0.005 & 1 & 0.1 & 0.4910 & 0.0025 & 0.4885 & 0.5985 & 0.0045 & 0.5940 \\
            \hline
            GCN-45-S-16 & 16.2k & 0.005 & 10 & 1 & 0.7628 & 0.0048 & 0.7580 & 0.8324 & 0.0071 & 0.8253 \\
            GCN-45-L-16 & 17.1k & 0.005 & 1 & 0.1 & 0.6188 & 0.0071 & 0.6117 & 0.7556 & 0.0063 & 0.7493 \\
            GCN-250-S-16 & 30.4k & 0.005 & 1 & 0.1 & 0.5001 & 0.0030 & 0.4971 & 0.6634 & 0.0050 & 0.6585 \\
            GCN-250-L-16 & 39.6k & 0.005 & 10 & 1 & 0.5116 & 0.0032 & 0.5085 & 0.6737 & 0.0051 & 0.6687 \\
            GCN-2500-S-16 & 28.6k & 0.005 & 10 & 1 & 0.5177 & 0.0036 & 0.5141 & 0.7132 & 0.0056 & 0.7076 \\
            GCN-2500-L-16 & 26.4k & 0.005 & 5 & 5 & 0.5120 & 0.0064 & 0.5056 & 0.6765 & 0.0093 & 0.6673 \\
            \hline
            GCN-TF-45-S-16 & 141.6k & 0.005 & 5 & 5 & 0.4657 & 0.0030 & 0.4627 & 0.6238 & 0.0052 & 0.6186 \\
            GCN-TF-45-L-16 & 268.0k & 0.005 & 0.5 & 0.5 & 0.5398 & 0.0036 & 0.5363 & 0.5918 & 0.0063 & 0.5855 \\
            GCN-TF-250-S-16 & 181.0k & 0.005 & 5 & 5 & 0.4068 & 0.0033 & 0.4035 & 0.6044 & 0.0054 & 0.5990 \\
            GCN-TF-250-L-16 & 315.6k & 0.005 & 0.5 & 0.5 & 0.4586 & 0.0035 & 0.4551 & 0.5846 & 0.0059 & 0.5787 \\
            GCN-TF-2500-S-16 & 154.1k & 0.005 & 1 & 0.1 & 0.4077 & 0.0021 & 0.4056 & 0.5978 & 0.0043 & 0.5935 \\
            GCN-TF-2500-L-16 & 277.3k & 0.005 & 5 & 5 & 0.4007 & 0.0272 & 0.3735 & 0.6238 & 0.0295 & 0.5944 \\
            \hline
            S4-S-16 & 2.4k & 0.01 & 10 & 1 & 0.4113 & 0.0025 & 0.4088 & 0.6475 & 0.0048 & 0.6427 \\
            S4-L-16 & 19.0k & 0.01 & 10 & 1 & 0.4401 & 0.0019 & 0.4382 & 0.6004 & 0.0042 & 0.5961 \\
            \hline
            S4-TF-S-16 & 28.0k & 0.01 & 0.5 & 0.5 & 0.3195 & 0.0016 & 0.3180 & 0.5514 & 0.0037 & 0.5477 \\
            S4-TF-L-16 & 70.2k & 0.01 & 1 & 0.1 & \textbf{0.3032} & 0.0014 & 0.3018 & \textbf{0.5078} & 0.0031 & 0.5046 \\
            \hline
            GB-DIST-MLP & 2.2k & 0.1 (0.01) & 5 & 5 & 0.9669 & 0.0109 & 0.9560 & 0.9649 & 0.0122 & 0.9527 \\
            GB-DIST-RNL & 47 & 0.1 (1) & 5 & 5 & 1.0055 & 0.0107 & 0.9948 & 0.9608 & 0.0125 & 0.9483 \\
            \hline
            GB-FUZZ-MLP & 2.3k & 0.1 (0.01) & 1 & 0.1 & 0.9813 & 0.0088 & 0.9725 & 0.9671 & 0.0117 & 0.9554 \\
            GB-FUZZ-RNL & 62 & 0.1 (1) & 5 & 5 & 1.0276 & 0.0113 & 1.0163 & 0.9626 & 0.0121 & 0.9505 \\
            \hline
            \hline
        \end{tabular}
    }
\end{table*}

% 0.001	5	5	0.6528	0.0300	0.6229	0.7679	0.0294	0.7385
% 0.001	5	5	0.4140	0.0018	0.4122	0.5265	0.0033	0.5232
% 0.005	5	5	0.5976	0.0051	0.5924	0.7548	0.0069	0.7479
% 0.005	0.5	0.5	0.6823	0.0056	0.6767	0.7348	0.0086	0.7262
% 0.005	1	0.1	0.5135	0.0029	0.5106	0.7035	0.0054	0.6981
% 0.005	0.5	0.5	0.5437	0.0087	0.5350	0.7371	0.0114	0.7257
% 0.005	0.5	0.5	0.6246	0.0165	0.6081	0.7942	0.0293	0.7649
% 0.005	0.5	0.5	0.4922	0.0058	0.4864	0.6920	0.0081	0.6839
% 0.005	1	0.1	0.6927	0.0053	0.6875	0.7316	0.0063	0.7253
% 0.005	0.5	0.5	0.4993	0.0034	0.4959	0.6598	0.0051	0.6547
% 0.005	10	1	0.5144	0.0036	0.5108	0.6592	0.0049	0.6543
% 0.005	1	0.1	0.4474	0.0035	0.4439	0.6370	0.0055	0.6315
% 0.005	10	1	0.6608	0.0052	0.6555	0.7411	0.0064	0.7346
% 0.005	1	0.1	0.4910	0.0025	0.4885	0.5985	0.0045	0.5940
% 0.005	10	1	0.7628	0.0048	0.7580	0.8324	0.0071	0.8253
% 0.005	1	0.1	0.6188	0.0071	0.6117	0.7556	0.0063	0.7493
% 0.005	1	0.1	0.5001	0.0030	0.4971	0.6634	0.0050	0.6585
% 0.005	10	1	0.5116	0.0032	0.5085	0.6737	0.0051	0.6687
% 0.005	10	1	0.5177	0.0036	0.5141	0.7132	0.0056	0.7076
% 0.005	5	5	0.5120	0.0064	0.5056	0.6765	0.0093	0.6673
% 0.005	5	5	0.4657	0.0030	0.4627	0.6238	0.0052	0.6186
% 0.005	0.5	0.5	0.5398	0.0036	0.5363	0.5918	0.0063	0.5855
% 0.005	5	5	0.4068	0.0033	0.4035	0.6044	0.0054	0.5990
% 0.005	0.5	0.5	0.4586	0.0035	0.4551	0.5846	0.0059	0.5787
% 0.005	1	0.1	0.4077	0.0021	0.4056	0.5978	0.0043	0.5935
% 0.005	5	5	0.4007	0.0272	0.3735	0.6238	0.0295	0.5944
% 0.01	10	1	0.4113	0.0025	0.4088	0.6475	0.0048	0.6427
% 0.01	10	1	0.4401	0.0019	0.4382	0.6004	0.0042	0.5961
% 0.01	0.5	0.5	0.3195	0.0016	0.3180	0.5514	0.0037	0.5477
% 0.01	1	0.1	0.3032	0.0014	0.3018	0.5078	0.0031	0.5046
% 0.1	5	5	0.9669	0.0109	0.9560	0.9649	0.0122	0.9527
% 0.1	5	5	1.0055	0.0107	0.9948	0.9608	0.0125	0.9483
% 0.1	1	0.1	0.9813	0.0088	0.9725	0.9671	0.0117	0.9554
% 0.1	5	5	1.0276	0.0113	1.0163	0.9626	0.0121	0.9505

\setlength{\tabcolsep}{4pt}
\renewcommand{\arraystretch}{1.3}
\begin{table*}[h]
    \small
    \caption{
    \textit{Scaled validation and test loss for non parametric models of \textbf{Harley Benton Drop Kick} distortion.}
    \textit{Bold indicates best performing models.}
    \textit{Learning rate multiplier for nonlinearity in gray-box models shown in brackets.}
    }
    \label{tab:val-and-test-loss_od_fulltone-fulldrive}
    \centerline{
        \begin{tabular}{l c c cc >{\columncolor{gray!20}}ccc >{\columncolor{gray!20}}ccc}
            \hline
            \midrule
            
            \multirow{2}{*}{Model}
                & \multirow{2}{*}{Params.}
                    & \multirow{2}{*}{LR}
                        & \multicolumn{2}{c}{Weights}
                            & \multicolumn{3}{c}{Val. Loss}
                                & \multicolumn{3}{c}{Test Loss} \\ 
            
            \cmidrule(lr){4-5} 
                \cmidrule(lr){6-8} 
                    \cmidrule(lr){9-11}
            
            &   &   & {\scriptsize L1} & {\scriptsize MR-STFT} & Tot. & {\scriptsize L1} & {\scriptsize MR-STFT} & Tot. & {\scriptsize L1} & {\scriptsize MR-STFT} \\ 
            
            \hline
            LSTM-32 & 4.5k & 0.001 & 5 & 5 & 1.7919 & 0.2177 & 1.5742 & 1.7506 & 0.2284 & 1.5222 \\
            LSTM-96 & 38.1k & 0.001 & 1 & 0.1 & 1.8443 & 0.2305 & 1.6138 & 1.9392 & 0.2181 & 1.7211 \\
            \hline
            TCN-45-S-16 & 7.5k & 0.005 & 5 & 5 & 1.2951 & 0.1939 & 1.1013 & 1.2535 & 0.1964 & 1.0571 \\
            TCN-45-L-16 & 7.3k & 0.005 & 0.5 & 0.5 & 1.2708 & 0.2096 & 1.0612 & 1.2598 & 0.1986 & 1.0612 \\
            TCN-250-S-16 & 14.5k & 0.005 & 5 & 5 & 1.3040 & 0.1877 & 1.1164 & 1.2548 & 0.1818 & 1.0731 \\
            TCN-250-L-16 & 18.4k & 0.005 & 0.5 & 0.5 & 1.2671 & 0.1784 & 1.0887 & 1.2506 & 0.1749 & 1.0757 \\
            TCN-2500-S-16 & 13.7k & 0.005 & 0.5 & 0.5 & 1.2884 & 0.1775 & 1.1109 & 1.3208 & 0.1875 & 1.1333 \\
            TCN-2500-L-16 & 11.9k & 0.005 & 5 & 5 & 1.1899 & 0.1913 & 0.9986 & 1.2115 & 0.1862 & 1.0253 \\
            \hline
            TCN-TF-45-S-16 & 39.5k & 0.005 & 0.5 & 0.5 & 1.0367 & 0.1425 & 0.8941 & 1.0231 & 0.1272 & 0.8959 \\
            TCN-TF-45-L-16 & 71.3k & 0.005 & 5 & 5 & 0.9281 & 0.1222 & 0.8060 & 0.9189 & 0.1068 & 0.8120 \\
            TCN-TF-250-S-16 & 52.9k & 0.005 & 5 & 5 & 0.9899 & 0.1304 & 0.8596 & 0.9865 & 0.1165 & 0.8699 \\
            TCN-TF-250-L-16 & 88.8k & 0.005 & 5 & 5 & \textbf{0.8754} & 0.1077 & 0.7677 & \textbf{0.8746} & 0.1003 & 0.7743 \\
            TCN-TF-2500-S-16 & 45.7k & 0.005 & 0.5 & 0.5 & 0.9781 & 0.1270 & 0.8511 & 0.9293 & 0.1063 & 0.8230 \\
            TCN-TF-2500-L-16 & 75.9k & 0.005 & 10 & 1 & 0.9631 & 0.0565 & 0.9066 & 0.8906 & 0.0470 & 0.8437 \\
            \hline
            GCN-45-S-16 & 16.2k & 0.005 & 0.5 & 0.5 & 1.2409 & 0.1936 & 1.0473 & 1.2207 & 0.1935 & 1.0272 \\
            GCN-45-L-16 & 17.1k & 0.005 & 5 & 5 & 1.3288 & 0.1939 & 1.1349 & 1.2218 & 0.2033 & 1.0186 \\
            GCN-250-S-16 & 30.4k & 0.005 & 5 & 5 & 1.2469 & 0.1921 & 1.0549 & 1.2195 & 0.1886 & 1.0309 \\
            GCN-250-L-16 & 39.6k & 0.005 & 5 & 5 & 1.1658 & 0.1680 & 0.9978 & 1.2022 & 0.1843 & 1.0179 \\
            GCN-2500-S-16 & 28.6k & 0.005 & 5 & 5 & 1.2159 & 0.1885 & 1.0274 & 1.2034 & 0.1873 & 1.0161 \\
            GCN-2500-L-16 & 26.4k & 0.005 & 5 & 5 & 1.1162 & 0.1775 & 0.9387 & 1.1590 & 0.1874 & 0.9717 \\
            \hline
            GCN-TF-45-S-16 & 141.6k & 0.005 & 0.5 & 0.5 & 1.0018 & 0.1303 & 0.8715 & 0.9847 & 0.1168 & 0.8679 \\
            GCN-TF-45-L-16 & 268.0k & 0.005 & 0.5 & 0.5 & 0.9561 & 0.1295 & 0.8267 & 0.9676 & 0.1141 & 0.8536 \\
            GCN-TF-250-S-16 & 181.0k & 0.005 & 5 & 5 & 1.0252 & 0.1241 & 0.9011 & 1.0128 & 0.1141 & 0.8987 \\
            GCN-TF-250-L-16 & 315.6k & 0.005 & 5 & 5 & 0.9382 & 0.1192 & 0.8189 & 0.9204 & 0.1052 & 0.8151 \\
            GCN-TF-2500-S-16 & 154.1k & 0.005 & 5 & 5 & 0.9933 & 0.1264 & 0.8669 & 0.9940 & 0.1169 & 0.8770 \\
            GCN-TF-2500-L-16 & 277.3k & 0.005 & 0.5 & 0.5 & 0.9816 & 0.1329 & 0.8488 & 0.9613 & 0.1092 & 0.8521 \\
            \hline
            S4-S-16 & 2.4k & 0.01 & 0.5 & 0.5 & 1.1228 & 0.2166 & 0.9062 & 1.1464 & 0.2090 & 0.9374 \\
            S4-L-16 & 19.0k & 0.01 & 5 & 5 & 1.0613 & 0.2605 & 0.8008 & 1.0607 & 0.2635 & 0.7972 \\
            \hline
            S4-TF-S-16 & 28.0k & 0.01 & 5 & 5 & 1.0076 & 0.1520 & 0.8556 & 1.0327 & 0.1461 & 0.8865 \\
            S4-TF-L-16 & 70.2k & 0.01 & 5 & 5 & 0.8902 & 0.1374 & 0.7528 & 0.9052 & 0.1323 & 0.7729 \\
            \hline
            GB-DIST-MLP & 2.2k & 0.1 (0.01) & 5 & 5 & 1.6483 & 0.2669 & 1.3814 & 1.5608 & 0.3081 & 1.2527 \\
            GB-DIST-RNL & 47 & 0.1 (1) & 5 & 5 & 1.5509 & 0.2749 & 1.2760 & 1.5215 & 0.3105 & 1.2110 \\
            \hline
            GB-FUZZ-MLP & 2.3k & 0.1 (0.01) & 5 & 5 & 1.5592 & 0.2855 & 1.2738 & 1.5600 & 0.3116 & 1.2484 \\
            GB-FUZZ-RNL & 62 & 0.1 (1) & 0.5 & 0.5 & 1.5981 & 0.2808 & 1.3173 & 1.5124 & 0.3115 & 1.2008 \\
            \hline
            \hline
        \end{tabular}
    }
\end{table*}

% 0.001	5	5	1.7919	0.2177	1.5742	1.7506	0.2284	1.5222
% 0.001	1	0.1	1.8443	0.2305	1.6138	1.9392	0.2181	1.7211
% 0.005	5	5	1.2951	0.1939	1.1013	1.2535	0.1964	1.0571
% 0.005	0.5	0.5	1.2708	0.2096	1.0612	1.2598	0.1986	1.0612
% 0.005	5	5	1.3040	0.1877	1.1164	1.2548	0.1818	1.0731
% 0.005	0.5	0.5	1.2671	0.1784	1.0887	1.2506	0.1749	1.0757
% 0.005	0.5	0.5	1.2884	0.1775	1.1109	1.3208	0.1875	1.1333
% 0.005	5	5	1.1899	0.1913	0.9986	1.2115	0.1862	1.0253
% 0.005	0.5	0.5	1.0367	0.1425	0.8941	1.0231	0.1272	0.8959
% 0.005	5	5	0.9281	0.1222	0.8060	0.9189	0.1068	0.8120
% 0.005	5	5	0.9899	0.1304	0.8596	0.9865	0.1165	0.8699
% 0.005	5	5	0.8754	0.1077	0.7677	0.8746	0.1003	0.7743
% 0.005	0.5	0.5	0.9781	0.1270	0.8511	0.9293	0.1063	0.8230
% 0.005	10	1	0.9631	0.0565	0.9066	0.8906	0.0470	0.8437
% 0.005	0.5	0.5	1.2409	0.1936	1.0473	1.2207	0.1935	1.0272
% 0.005	5	5	1.3288	0.1939	1.1349	1.2218	0.2033	1.0186
% 0.005	5	5	1.2469	0.1921	1.0549	1.2195	0.1886	1.0309
% 0.005	5	5	1.1658	0.1680	0.9978	1.2022	0.1843	1.0179
% 0.005	5	5	1.2159	0.1885	1.0274	1.2034	0.1873	1.0161
% 0.005	5	5	1.1162	0.1775	0.9387	1.1590	0.1874	0.9717
% 0.005	0.5	0.5	1.0018	0.1303	0.8715	0.9847	0.1168	0.8679
% 0.005	0.5	0.5	0.9561	0.1295	0.8267	0.9676	0.1141	0.8536
% 0.005	5	5	1.0252	0.1241	0.9011	1.0128	0.1141	0.8987
% 0.005	5	5	0.9382	0.1192	0.8189	0.9204	0.1052	0.8151
% 0.005	5	5	0.9933	0.1264	0.8669	0.9940	0.1169	0.8770
% 0.005	0.5	0.5	0.9816	0.1329	0.8488	0.9613	0.1092	0.8521
% 0.01	0.5	0.5	1.1228	0.2166	0.9062	1.1464	0.2090	0.9374
% 0.01	5	5	1.0613	0.2605	0.8008	1.0607	0.2635	0.7972
% 0.01	5	5	1.0076	0.1520	0.8556	1.0327	0.1461	0.8865
% 0.01	5	5	0.8902	0.1374	0.7528	0.9052	0.1323	0.7729
% 0.1	    5	5	1.6483	0.2669	1.3814	1.5608	0.3081	1.2527
% 0.1	    5	5	1.5509	0.2749	1.2760	1.5215	0.3105	1.2110
% 0.1	    5	5	1.5592	0.2855	1.2738	1.5600	0.3116	1.2484
% 0.1	    0.5	0.5	1.5981	0.2808	1.3173	1.5124	0.3115	1.2008
\setlength{\tabcolsep}{4pt}
\renewcommand{\arraystretch}{1.3}
\begin{table*}[h]
    \small
    \caption{
    \textit{Scaled validation and test loss for non parametric models of \textbf{Harley Benton Plexicon} distortion.}
    \textit{Bold indicates best performing models.}
    \textit{Learning rate multiplier for nonlinearity in gray-box models shown in brackets.}
    }
    \label{tab:val-and-test-loss_od_fulltone-fulldrive}
    \centerline{
        \begin{tabular}{l c c cc >{\columncolor{gray!20}}ccc >{\columncolor{gray!20}}ccc}
            \hline
            \midrule
            
            \multirow{2}{*}{Model}
                & \multirow{2}{*}{Params.}
                    & \multirow{2}{*}{LR}
                        & \multicolumn{2}{c}{Weights}
                            & \multicolumn{3}{c}{Val. Loss}
                                & \multicolumn{3}{c}{Test Loss} \\ 
            
            \cmidrule(lr){4-5} 
                \cmidrule(lr){6-8} 
                    \cmidrule(lr){9-11}
            
            &   &   & {\scriptsize L1} & {\scriptsize MR-STFT} & Tot. & {\scriptsize L1} & {\scriptsize MR-STFT} & Tot. & {\scriptsize L1} & {\scriptsize MR-STFT} \\ 
            
            \hline
            LSTM-32 & 4.5k & 0.005 & 1 & 0.1 & 0.3677 & 0.0276 & 0.3401 & 0.3323 & 0.0163 & 0.3161 \\
            LSTM-96 & 38.1k & 0.005 & 5 & 5 & \textbf{0.2312} & 0.0253 & 0.2059 & \textbf{0.1973} & 0.0118 & 0.1855 \\
            \hline
            TCN-45-S-16 & 7.5k & 0.005 & 5 & 5 & 0.6778 & 0.0456 & 0.6321 & 0.6290 & 0.0361 & 0.5928 \\
            TCN-45-L-16 & 7.3k & 0.005 & 0.5 & 0.5 & 0.6473 & 0.0496 & 0.5978 & 0.8903 & 0.0633 & 0.8270 \\
            TCN-250-S-16 & 14.5k & 0.005 & 5 & 5 & 0.6074 & 0.0463 & 0.5612 & 0.5389 & 0.0286 & 0.5103 \\
            TCN-250-L-16 & 18.4k & 0.005 & 10 & 1 & 0.6156 & 0.0458 & 0.5698 & 0.6132 & 0.0306 & 0.5826 \\
            TCN-2500-S-16 & 13.7k & 0.005 & 1 & 0.1 & 0.7247 & 0.0543 & 0.6705 & 0.6608 & 0.0366 & 0.6242 \\
            TCN-2500-L-16 & 11.9k & 0.005 & 10 & 1 & 0.5411 & 0.0338 & 0.5072 & 0.5284 & 0.0259 & 0.5025 \\
            \hline
            TCN-TF-45-S-16 & 39.5k & 0.005 & 5 & 5 & 0.5581 & 0.0318 & 0.5264 & 0.5479 & 0.0273 & 0.5206 \\
            TCN-TF-45-L-16 & 71.3k & 0.005 & 0.5 & 0.5 & 0.4844 & 0.0332 & 0.4512 & 0.4759 & 0.0212 & 0.4548 \\
            TCN-TF-250-S-16 & 52.9k & 0.005 & 5 & 5 & 0.4920 & 0.0410 & 0.4510 & 0.4664 & 0.0218 & 0.4447 \\
            TCN-TF-250-L-16 & 88.8k & 0.005 & 0.5 & 0.5 & 0.4147 & 0.0355 & 0.3792 & 0.3724 & 0.0172 & 0.3552 \\
            TCN-TF-2500-S-16 & 45.7k & 0.005 & 5 & 5 & 0.5884 & 0.0411 & 0.5473 & 0.5622 & 0.0295 & 0.5327 \\
            TCN-TF-2500-L-16 & 75.9k & 0.005 & 0.5 & 0.5 & 0.5065 & 0.0383 & 0.4683 & 0.4559 & 0.0191 & 0.4368 \\
            \hline
            GCN-45-S-16 & 16.2k & 0.005 & 0.5 & 0.5 & 0.5852 & 0.0434 & 0.5418 & 0.4597 & 0.0224 & 0.4373 \\
            GCN-45-L-16 & 17.1k & 0.005 & 5 & 5 & 0.5668 & 0.0535 & 0.5133 & 0.4755 & 0.0299 & 0.4456 \\
            GCN-250-S-16 & 30.4k & 0.005 & 0.5 & 0.5 & 0.5219 & 0.0396 & 0.4823 & 0.4842 & 0.0256 & 0.4585 \\
            GCN-250-L-16 & 39.6k & 0.005 & 0.5 & 0.5 & 0.4682 & 0.0398 & 0.4283 & 0.4123 & 0.0220 & 0.3903 \\
            GCN-2500-S-16 & 28.6k & 0.005 & 0.5 & 0.5 & 0.6274 & 0.0454 & 0.5820 & 0.5805 & 0.0340 & 0.5465 \\
            GCN-2500-L-16 & 26.4k & 0.005 & 1 & 0.1 & 0.5159 & 0.0382 & 0.4777 & 0.4753 & 0.0221 & 0.4532 \\
            \hline
            GCN-TF-45-S-16 & 141.6k & 0.005 & 0.5 & 0.5 & 0.5127 & 0.0468 & 0.4659 & 0.4612 & 0.0226 & 0.4386 \\
            GCN-TF-45-L-16 & 268.0k & 0.005 & 5 & 5 & 0.4151 & 0.0360 & 0.3790 & 0.3965 & 0.0168 & 0.3797 \\
            GCN-TF-250-S-16 & 181.0k & 0.005 & 5 & 5 & 0.4606 & 0.0304 & 0.4302 & 0.4179 & 0.0199 & 0.3980 \\
            GCN-TF-250-L-16 & 315.6k & 0.005 & 5 & 5 & 0.3788 & 0.0312 & 0.3476 & 0.3275 & 0.0150 & 0.3125 \\
            GCN-TF-2500-S-16 & 154.1k & 0.005 & 0.5 & 0.5 & 0.5202 & 0.0364 & 0.4837 & 0.5037 & 0.0252 & 0.4785 \\
            GCN-TF-2500-L-16 & 277.3k & 0.005 & 0.5 & 0.5 & 0.4020 & 0.0333 & 0.3687 & 0.3872 & 0.0182 & 0.3690 \\
            \hline
            S4-S-16 & 2.4k & 0.01 & 0.5 & 0.5 & 0.4833 & 0.0336 & 0.4496 & 0.4505 & 0.0212 & 0.4292 \\
            S4-L-16 & 19.0k & 0.01 & 1 & 0.1 & 0.3543 & 0.0389 & 0.3154 & 0.2982 & 0.0131 & 0.2851 \\
            \hline
            S4-TF-S-16 & 28.0k & 0.01 & 1 & 0.1 & 0.4362 & 0.0265 & 0.4098 & 0.4146 & 0.0161 & 0.3985 \\
            S4-TF-L-16 & 70.2k & 0.01 & 0.5 & 0.5 & 0.2766 & 0.0246 & 0.2519 & 0.2449 & 0.0121 & 0.2329 \\
            \hline
            GB-DIST-MLP & 2.2k & 0.1 (0.01) & 0.5 & 0.5 & 1.0038 & 0.0886 & 0.9152 & 0.9509 & 0.0826 & 0.8683 \\
            GB-DIST-RNL & 47 & 0.1 (1) & 5 & 5 & 0.9344 & 0.0711 & 0.8632 & 0.9273 & 0.0760 & 0.8512 \\
            \hline
            GB-FUZZ-MLP & 2.3k & 0.1 (0.01) & 0.5 & 0.5 & 1.0088 & 0.0833 & 0.9255 & 0.9657 & 0.0777 & 0.8880 \\
            GB-FUZZ-RNL & 62 & 0.1 (1) & 5 & 5 & 1.0354 & 0.0872 & 0.9481 & 0.9827 & 0.0883 & 0.8944 \\
            \hline
            \hline
        \end{tabular}
    }
\end{table*}

% 0.005	1	0.1	0.3677	0.0276	0.3401	0.3323	0.0163	0.3161
% 0.005	5	5	0.2312	0.0253	0.2059	0.1973	0.0118	0.1855
% 0.005	5	5	0.6778	0.0456	0.6321	0.6290	0.0361	0.5928
% 0.005	0.5	0.5	0.6473	0.0496	0.5978	0.8903	0.0633	0.8270
% 0.005	5	5	0.6074	0.0463	0.5612	0.5389	0.0286	0.5103
% 0.005	10	1	0.6156	0.0458	0.5698	0.6132	0.0306	0.5826
% 0.005	1	0.1	0.7247	0.0543	0.6705	0.6608	0.0366	0.6242
% 0.005	10	1	0.5411	0.0338	0.5072	0.5284	0.0259	0.5025
% 0.005	5	5	0.5581	0.0318	0.5264	0.5479	0.0273	0.5206
% 0.005	0.5	0.5	0.4844	0.0332	0.4512	0.4759	0.0212	0.4548
% 0.005	5	5	0.4920	0.0410	0.4510	0.4664	0.0218	0.4447
% 0.005	0.5	0.5	0.4147	0.0355	0.3792	0.3724	0.0172	0.3552
% 0.005	5	5	0.5884	0.0411	0.5473	0.5622	0.0295	0.5327
% 0.005	0.5	0.5	0.5065	0.0383	0.4683	0.4559	0.0191	0.4368
% 0.005	0.5	0.5	0.5852	0.0434	0.5418	0.4597	0.0224	0.4373
% 0.005	5	5	0.5668	0.0535	0.5133	0.4755	0.0299	0.4456
% 0.005	0.5	0.5	0.5219	0.0396	0.4823	0.4842	0.0256	0.4585
% 0.005	0.5	0.5	0.4682	0.0398	0.4283	0.4123	0.0220	0.3903
% 0.005	0.5	0.5	0.6274	0.0454	0.5820	0.5805	0.0340	0.5465
% 0.005	1	0.1	0.5159	0.0382	0.4777	0.4753	0.0221	0.4532
% 0.005	0.5	0.5	0.5127	0.0468	0.4659	0.4612	0.0226	0.4386
% 0.005	5	5	0.4151	0.0360	0.3790	0.3965	0.0168	0.3797
% 0.005	5	5	0.4606	0.0304	0.4302	0.4179	0.0199	0.3980
% 0.005	5	5	0.3788	0.0312	0.3476	0.3275	0.0150	0.3125
% 0.005	0.5	0.5	0.5202	0.0364	0.4837	0.5037	0.0252	0.4785
% 0.005	0.5	0.5	0.4020	0.0333	0.3687	0.3872	0.0182	0.3690
% 0.01	0.5	0.5	0.4833	0.0336	0.4496	0.4505	0.0212	0.4292
% 0.01	1	0.1	0.3543	0.0389	0.3154	0.2982	0.0131	0.2851
% 0.01	1	0.1	0.4362	0.0265	0.4098	0.4146	0.0161	0.3985
% 0.01	0.5	0.5	0.2766	0.0246	0.2519	0.2449	0.0121	0.2329
% 0.1	    0.5	0.5	1.0038	0.0886	0.9152	0.9509	0.0826	0.8683
% 0.1	    5	5	0.9344	0.0711	0.8632	0.9273	0.0760	0.8512
% 0.1	    0.5	0.5	1.0088	0.0833	0.9255	0.9657	0.0777	0.8880
% 0.1	    5	5	1.0354	0.0872	0.9481	0.9827	0.0883	0.8944
\setlength{\tabcolsep}{4pt}
\renewcommand{\arraystretch}{1.3}
\begin{table*}[h]
    \small
    \caption{
    \textit{Scaled validation and test loss for non parametric models of \textbf{Harley Benton Rodent} distortion.}
    \textit{Bold indicates best performing models.}
    \textit{Learning rate multiplier for nonlinearity in gray-box models shown in brackets.}
    }
    \label{tab:val-and-test-loss_od_fulltone-fulldrive}
    \centerline{
        \begin{tabular}{l c c cc >{\columncolor{gray!20}}ccc >{\columncolor{gray!20}}ccc}
            \hline
            \midrule
            
            \multirow{2}{*}{Model}
                & \multirow{2}{*}{Params.}
                    & \multirow{2}{*}{LR}
                        & \multicolumn{2}{c}{Weights}
                            & \multicolumn{3}{c}{Val. Loss}
                                & \multicolumn{3}{c}{Test Loss} \\ 
            
            \cmidrule(lr){4-5} 
                \cmidrule(lr){6-8} 
                    \cmidrule(lr){9-11}
            
            &   &   & {\scriptsize L1} & {\scriptsize MR-STFT} & Tot. & {\scriptsize L1} & {\scriptsize MR-STFT} & Tot. & {\scriptsize L1} & {\scriptsize MR-STFT} \\ 
            
            \hline
            LSTM-32 & 4.5k & 0.005 & 0.5 & 0.5 & 1.4877 & 0.0528 & 1.4349 & 1.5315 & 0.0605 & 1.4710 \\
            LSTM-96 & 38.1k & 0.001 & 5 & 5 & 1.6675 & 0.1810 & 1.4865 & 1.6861 & 0.0569 & 1.6292 \\
            \hline
            TCN-45-S-16 & 7.5k & 0.005 & 10 & 1 & 1.0152 & 0.0171 & 0.9981 & 1.0722 & 0.0254 & 1.0468 \\
            TCN-45-L-16 & 7.3k & 0.005 & 5 & 5 & 0.9341 & 0.0221 & 0.9120 & 0.9984 & 0.0238 & 0.9746 \\
            TCN-250-S-16 & 14.5k & 0.005 & 1 & 0.1 & 0.9058 & 0.0153 & 0.8905 & 0.9287 & 0.0179 & 0.9108 \\
            TCN-250-L-16 & 18.4k & 0.005 & 0.5 & 0.5 & 0.8064 & 0.0149 & 0.7914 & 0.8366 & 0.0154 & 0.8212 \\
            TCN-2500-S-16 & 13.7k & 0.005 & 5 & 5 & 0.9117 & 0.0204 & 0.8912 & 0.9234 & 0.0210 & 0.9024 \\
            TCN-2500-L-16 & 11.9k & 0.005 & 10 & 1 & 0.7681 & 0.0133 & 0.7548 & 0.8330 & 0.0158 & 0.8172 \\
            \hline
            TCN-TF-45-S-16 & 39.5k & 0.005 & 1 & 0.1 & 0.7422 & 0.0114 & 0.7307 & 0.7775 & 0.0128 & 0.7647 \\
            TCN-TF-45-L-16 & 71.3k & 0.005 & 10 & 1 & 0.7249 & 0.0117 & 0.7132 & 0.7786 & 0.0130 & 0.7656 \\
            TCN-TF-250-S-16 & 52.9k & 0.005 & 5 & 5 & 0.6696 & 0.0114 & 0.6582 & 0.7400 & 0.0131 & 0.7269 \\
            TCN-TF-250-L-16 & 88.8k & 0.005 & 1 & 0.1 & 0.7318 & 0.0116 & 0.7203 & 0.7820 & 0.0132 & 0.7688 \\
            TCN-TF-2500-S-16 & 45.7k & 0.005 & 10 & 1 & 0.7343 & 0.0119 & 0.7224 & 0.7771 & 0.0128 & 0.7643 \\
            TCN-TF-2500-L-16 & 75.9k & 0.005 & 0.5 & 0.5 & 0.6068 & 0.0105 & 0.5963 & 0.6239 & 0.0101 & 0.6138 \\
            \hline
            GCN-45-S-16 & 16.2k & 0.005 & 5 & 5 & 0.9508 & 0.0172 & 0.9337 & 0.9744 & 0.0213 & 0.9531 \\
            GCN-45-L-16 & 17.1k & 0.005 & 5 & 5 & 1.0460 & 0.0176 & 1.0284 & 0.9480 & 0.0212 & 0.9267 \\
            GCN-250-S-16 & 30.4k & 0.005 & 5 & 5 & 0.8692 & 0.0153 & 0.8539 & 0.8881 & 0.0181 & 0.8699 \\
            GCN-250-L-16 & 39.6k & 0.005 & 1 & 0.1 & 0.7973 & 0.0129 & 0.7844 & 0.8468 & 0.0141 & 0.8327 \\
            GCN-2500-S-16 & 28.6k & 0.005 & 1 & 0.1 & 0.8633 & 0.0155 & 0.8478 & 0.8929 & 0.0175 & 0.8755 \\
            GCN-2500-L-16 & 26.4k & 0.005 & 1 & 0.1 & 0.7319 & 0.0115 & 0.7205 & 0.7830 & 0.0132 & 0.7698 \\
            \hline
            GCN-TF-45-S-16 & 141.6k & 0.005 & 1 & 0.1 & 0.7805 & 0.0117 & 0.7688 & 0.7742 & 0.0124 & 0.7618 \\
            GCN-TF-45-L-16 & 268.0k & 0.005 & 1 & 0.1 & 0.6801 & 0.0095 & 0.6706 & 0.6863 & 0.0097 & 0.6766 \\
            GCN-TF-250-S-16 & 181.0k & 0.005 & 10 & 1 & 0.7481 & 0.0117 & 0.7364 & 0.7671 & 0.0122 & 0.7549 \\
            GCN-TF-250-L-16 & 315.6k & 0.005 & 1 & 0.1 & 0.6019 & 0.0073 & 0.5947 & 0.7992 & 0.0142 & 0.7850 \\
            GCN-TF-2500-S-16 & 154.1k & 0.005 & 10 & 1 & 0.6899 & 0.0107 & 0.6792 & 0.7182 & 0.0113 & 0.7069 \\
            GCN-TF-2500-L-16 & 277.3k & 0.005 & 10 & 1 & 0.6081 & 0.0083 & 0.5998 & 0.6247 & 0.0081 & 0.6166 \\
            \hline
            S4-S-16 & 2.4k & 0.01 & 0.5 & 0.5 & 0.6965 & 0.0113 & 0.6851 & 0.7245 & 0.0125 & 0.7120 \\
            S4-L-16 & 19.0k & 0.01 & 10 & 1 & 0.5865 & 0.0073 & 0.5791 & 0.6367 & 0.0091 & 0.6276 \\
            \hline
            S4-TF-S-16 & 28.0k & 0.01 & 1 & 0.1 & 0.6270 & 0.0080 & 0.6190 & 0.6367 & 0.0089 & 0.6279 \\
            S4-TF-L-16 & 70.2k & 0.01 & 5 & 5 & \textbf{0.5707} & 0.0080 & 0.5627 & \textbf{0.5816} & 0.0077 & 0.5738 \\
            \hline
            GB-DIST-MLP & 2.2k & 0.1 (0.01) & 1 & 0.1 & 1.4323 & 0.0329 & 1.3993 & 1.2826 & 0.0368 & 1.2458 \\
            GB-DIST-RNL & 47 & 0.1 (1) & 1 & 0.1 & 1.4156 & 0.0353 & 1.3803 & 1.3654 & 0.0392 & 1.3262 \\
            \hline
            GB-FUZZ-MLP & 2.3k & 0.1 (0.01) & 10 & 1 & 1.2579 & 0.0333 & 1.2246 & 1.2751 & 0.0357 & 1.2394 \\
            GB-FUZZ-RNL & 62 & 0.1 (1) & 5 & 5 & 1.4987 & 0.0396 & 1.4591 & 1.3963 & 0.0450 & 1.3513 \\
            \hline
            \hline
        \end{tabular}
    }
\end{table*}

% 0.005	0.5	0.5	1.4877	0.0528	1.4349	1.5315	0.0605	1.4710
% 0.001	5	5	1.6675	0.1810	1.4865	1.6861	0.0569	1.6292
% 0.005	10	1	1.0152	0.0171	0.9981	1.0722	0.0254	1.0468
% 0.005	5	5	0.9341	0.0221	0.9120	0.9984	0.0238	0.9746
% 0.005	1	0.1	0.9058	0.0153	0.8905	0.9287	0.0179	0.9108
% 0.005	0.5	0.5	0.8064	0.0149	0.7914	0.8366	0.0154	0.8212
% 0.005	5	5	0.9117	0.0204	0.8912	0.9234	0.0210	0.9024
% 0.005	10	1	0.7681	0.0133	0.7548	0.8330	0.0158	0.8172
% 0.005	1	0.1	0.7422	0.0114	0.7307	0.7775	0.0128	0.7647
% 0.005	10	1	0.7249	0.0117	0.7132	0.7786	0.0130	0.7656
% 0.005	5	5	0.6696	0.0114	0.6582	0.7400	0.0131	0.7269
% 0.005	1	0.1	0.7318	0.0116	0.7203	0.7820	0.0132	0.7688
% 0.005	10	1	0.7343	0.0119	0.7224	0.7771	0.0128	0.7643
% 0.005	0.5	0.5	0.6068	0.0105	0.5963	0.6239	0.0101	0.6138
% 0.005	5	5	0.9508	0.0172	0.9337	0.9744	0.0213	0.9531
% 0.005	5	5	1.0460	0.0176	1.0284	0.9480	0.0212	0.9267
% 0.005	5	5	0.8692	0.0153	0.8539	0.8881	0.0181	0.8699
% 0.005	1	0.1	0.7973	0.0129	0.7844	0.8468	0.0141	0.8327
% 0.005	1	0.1	0.8633	0.0155	0.8478	0.8929	0.0175	0.8755
% 0.005	1	0.1	0.7319	0.0115	0.7205	0.7830	0.0132	0.7698
% 0.005	1	0.1	0.7805	0.0117	0.7688	0.7742	0.0124	0.7618
% 0.005	1	0.1	0.6801	0.0095	0.6706	0.6863	0.0097	0.6766
% 0.005	10	1	0.7481	0.0117	0.7364	0.7671	0.0122	0.7549
% 0.005	1	0.1	0.6019	0.0073	0.5947	0.7992	0.0142	0.7850
% 0.005	10	1	0.6899	0.0107	0.6792	0.7182	0.0113	0.7069
% 0.005	10	1	0.6081	0.0083	0.5998	0.6247	0.0081	0.6166
% 0.01	0.5	0.5	0.6965	0.0113	0.6851	0.7245	0.0125	0.7120
% 0.01	10	1	0.5865	0.0073	0.5791	0.6367	0.0091	0.6276
% 0.01	1	0.1	0.6270	0.0080	0.6190	0.6367	0.0089	0.6279
% 0.01	5	5	0.5707	0.0080	0.5627	0.5816	0.0077	0.5738
% 0.1	    1	0.1	1.4323	0.0329	1.3993	1.2826	0.0368	1.2458
% 0.1	    1	0.1	1.4156	0.0353	1.3803	1.3654	0.0392	1.3262
% 0.1	    10	1	1.2579	0.0333	1.2246	1.2751	0.0357	1.2394
% 0.1	    5	5	1.4987	0.0396	1.4591	1.3963	0.0450	1.3513

\setlength{\tabcolsep}{4pt}
\renewcommand{\arraystretch}{1.3}
\begin{table*}[h]
    \small
    \caption{
    \textit{Objective metrics for non parametric models of \textbf{Electro Harmonix Big Muff} distortion.}
    \textit{Bold indicates best performing models.}
    \textit{Learning rate multiplier for nonlinearity in gray-box models shown in brackets.}
    }
    \label{tab:metrics_od_diy-klon-centaur}
    \centerline{
        \begin{tabular}{lccccccccccccc}
            \hline
            \midrule
            
            \multirow{2}{*}{Model}
                & \multirow{2}{*}{Params.}
                    & \multirow{2}{*}{LR}
                        & \multicolumn{2}{c}{Weights}
                            & \multicolumn{3}{c}{}
                                & \multicolumn{3}{c}{FAD} \\ 
            \cmidrule(lr){4-5} 
                % \cmidrule(lr){6-8} 
                    \cmidrule(lr){9-12}
            
            &   &   & L1 & MR-STFT & MSE & ESR & MAPE & VGGish & PANN & CLAP & AFx-Rep \\ 
            \hline
            LSTM-32 & 4.5k & 0.001 & 5 & 5 & 9.25e-03 & 3.5938 & 7.7534 & 1.0347 & 7.30e-06 & 0.0932 & 0.0262 \\
            LSTM-96 & 38.1k & 0.001 & 5 & 5 & \textbf{1.80e-04} & 0.1205 & 5.1320 & 0.7303 & 1.42e-07 & 0.0872 & 0.0164 \\
            \hline
            TCN-45-S-16 & 7.5k & 0.005 & 5 & 5 & 6.58e-04 & 0.3407 & 5.4885 & 0.8465 & 1.72e-05 & 0.1027 & 0.0642 \\
            TCN-45-L-16 & 7.3k & 0.005 & 0.5 & 0.5 & 9.69e-04 & 0.4160 & 7.4735 & 0.7747 & 2.51e-05 & 0.0955 & 0.0813 \\
            TCN-250-S-16 & 14.5k & 0.005 & 1 & 0.1 & 4.73e-04 & 0.2703 & 4.7165 & 0.7876 & 2.47e-05 & 0.1119 & 0.0553 \\
            TCN-250-L-16 & 18.4k & 0.005 & 0.5 & 0.5 & 1.79e-03 & 0.6723 & 7.1658 & 0.7803 & 9.28e-06 & 0.0968 & 0.0282 \\
            TCN-2500-S-16 & 13.7k & 0.005 & 0.5 & 0.5 & 8.48e-03 & 3.3871 & 8.5297 & 0.8464 & 1.82e-05 & 0.1312 & 0.0572 \\
            TCN-2500-L-16 & 11.9k & 0.005 & 0.5 & 0.5 & 9.43e-04 & 0.3879 & 6.1196 & 0.7373 & 3.43e-06 & 0.1032 & 0.0298 \\
            \hline
            TCN-TF-45-S-16 & 39.5k & 0.005 & 1 & 0.1 & 5.70e-04 & 0.3003 & 4.0923 & 0.7754 & 5.34e-05 & 0.1030 & 0.0535 \\
            TCN-TF-45-L-16 & 71.3k & 0.005 & 0.5 & 0.5 & 3.80e-04 & 0.2007 & 4.1743 & 0.6458 & 2.21e-05 & 0.0992 & 0.0640 \\
            TCN-TF-250-S-16 & 52.9k & 0.005 & 10 & 1 & 3.43e-04 & 0.1803 & 4.1170 & 0.7057 & 3.79e-05 & 0.0823 & 0.0465 \\
            TCN-TF-250-L-16 & 88.8k & 0.005 & 1 & 0.1 & 4.46e-04 & 0.2104 & 3.9814 & 0.6236 & 2.15e-05 & 0.0944 & 0.0539 \\
            TCN-TF-2500-S-16 & 45.7k & 0.005 & 10 & 1 & 5.61e-04 & 0.2789 & 4.7279 & 0.7041 & 6.96e-05 & 0.0807 & 0.1272 \\
            TCN-TF-2500-L-16 & 75.9k & 0.005 & 1 & 0.1 & 3.17e-04 & 0.1615 & 3.4850 & 0.4300 & 1.50e-05 & 0.0779 & 0.0298 \\
            \hline
            GCN-45-S-16 & 16.2k & 0.005 & 10 & 1 & 6.26e-04 & 0.3408 & 5.8670 & 1.1365 & 6.20e-05 & 0.1200 & 0.2347 \\
            GCN-45-L-16 & 17.1k & 0.005 & 1 & 0.1 & 5.47e-04 & 0.2943 & 4.9605 & 0.8029 & 3.90e-05 & 0.1012 & 0.0851 \\
            GCN-250-S-16 & 30.4k & 0.005 & 1 & 0.1 & 3.46e-04 & 0.1889 & 5.4230 & 0.6357 & 1.52e-05 & 0.0971 & 0.0723 \\
            GCN-250-L-16 & 39.6k & 0.005 & 10 & 1 & 3.86e-04 & 0.2126 & 4.8257 & 0.7745 & 2.54e-05 & 0.1011 & 0.0543 \\
            GCN-2500-S-16 & 28.6k & 0.005 & 10 & 1 & 4.32e-04 & 0.2348 & 5.3948 & 0.7070 & 2.06e-05 & 0.1177 & 0.1008 \\
            GCN-2500-L-16 & 26.4k & 0.005 & 5 & 5 & 1.25e-03 & 0.4943 & 7.1149 & 0.7932 & 7.55e-06 & 0.1067 & 0.0473 \\
            \hline
            GCN-TF-45-S-16 & 141.6k & 0.005 & 5 & 5 & 3.58e-04 & 0.1897 & 4.1376 & 0.6413 & 3.32e-06 & 0.0890 & 0.0610 \\
            GCN-TF-45-L-16 & 268.0k & 0.005 & 0.5 & 0.5 & 6.35e-04 & 0.2430 & 4.2569 & 0.4949 & 2.49e-05 & \textbf{0.0738} & 0.0327 \\
            GCN-TF-250-S-16 & 181.0k & 0.005 & 5 & 5 & 4.26e-04 & 0.2096 & 4.2674 & 0.5707 & 5.42e-06 & 0.0836 & 0.0445 \\
            GCN-TF-250-L-16 & 315.6k & 0.005 & 0.5 & 0.5 & 5.45e-04 & 0.2386 & 5.2190 & 0.5263 & 2.26e-05 & 0.0905 & 0.0364 \\
            GCN-TF-2500-S-16 & 154.1k & 0.005 & 1 & 0.1 & 2.72e-04 & 0.1370 & 4.7303 & 0.5106 & 1.02e-05 & 0.0819 & 0.0572 \\
            GCN-TF-2500-L-16 & 277.3k & 0.005 & 5 & 5 & 9.06e-03 & 3.6098 & 5.1530 & 0.4959 & 2.07e-05 & 0.0841 & 0.0297 \\
            \hline
            S4-S-16 & 2.4k & 0.01 & 10 & 1 & 3.94e-04 & 0.2374 & 4.6882 & 0.8045 & 3.61e-06 & 0.1061 & 0.0257 \\
            S4-L-16 & 19.0k & 0.01 & 10 & 1 & 3.27e-04 & 0.1922 & 4.5979 & 0.8086 & \textbf{1.75e-08} & 0.0992 & \textbf{0.0116} \\
            \hline
            S4-TF-S-16 & 28.0k & 0.01 & 0.5 & 0.5 & 2.63e-04 & 0.1478 & 3.0808 & 0.6401 & 2.73e-06 & 0.0885 & 0.0127 \\
            S4-TF-L-16 & 70.2k & 0.01 & 1 & 0.1 & 1.89e-04 & \textbf{0.1076} & \textbf{2.8853} & \textbf{0.2661} & 1.07e-06 & 0.0809 & 0.0139 \\
            \hline
            GB-DIST-MLP & 2.2k & 0.1 (0.01) & 5 & 5 & 1.48e-03 & 0.6912 & 24.8279 & 0.9631 & 1.85e-05 & 0.1068 & 0.2113 \\
            GB-DIST-RNL & 47 & 0.1 (1) & 5 & 5 & 1.37e-03 & 0.6107 & 42.0342 & 1.0088 & 1.05e-05 & 0.0968 & 0.2744 \\
            \hline
            GB-FUZZ-MLP & 2.3k & 0.1 (0.01) & 1 & 0.1 & 1.45e-03 & 0.7042 & 20.8625 & 1.0107 & 2.30e-05 & 0.1051 & 0.1828 \\
            GB-FUZZ-RNL & 62 & 0.1 (1) & 5 & 5 & 1.30e-03 & 0.5869 & 44.9507 & 0.8679 & 5.02e-06 & 0.0946 & 0.2279 \\
            \hline
            \hline
        \end{tabular}
    }
\end{table*}

% 0.001	5	5	9.25e-03	3.5938	7.7534	1.0347	7.30e-06	0.0932	0.0262
% 0.001	5	5	1.80e-04	0.1205	5.1320	0.7303	1.42e-07	0.0872	0.0164
% 0.005	5	5	6.58e-04	0.3407	5.4885	0.8465	1.72e-05	0.1027	0.0642
% 0.005	0.5	0.5	9.69e-04	0.4160	7.4735	0.7747	2.51e-05	0.0955	0.0813
% 0.005	1	0.1	4.73e-04	0.2703	4.7165	0.7876	2.47e-05	0.1119	0.0553
% 0.005	0.5	0.5	1.79e-03	0.6723	7.1658	0.7803	9.28e-06	0.0968	0.0282
% 0.005	0.5	0.5	8.48e-03	3.3871	8.5297	0.8464	1.82e-05	0.1312	0.0572
% 0.005	0.5	0.5	9.43e-04	0.3879	6.1196	0.7373	3.43e-06	0.1032	0.0298
% 0.005	1	0.1	5.70e-04	0.3003	4.0923	0.7754	5.34e-05	0.1030	0.0535
% 0.005	0.5	0.5	3.80e-04	0.2007	4.1743	0.6458	2.21e-05	0.0992	0.0640
% 0.005	10	1	3.43e-04	0.1803	4.1170	0.7057	3.79e-05	0.0823	0.0465
% 0.005	1	0.1	4.46e-04	0.2104	3.9814	0.6236	2.15e-05	0.0944	0.0539
% 0.005	10	1	5.61e-04	0.2789	4.7279	0.7041	6.96e-05	0.0807	0.1272
% 0.005	1	0.1	3.17e-04	0.1615	3.4850	0.4300	1.50e-05	0.0779	0.0298
% 0.005	10	1	6.26e-04	0.3408	5.8670	1.1365	6.20e-05	0.1200	0.2347
% 0.005	1	0.1	5.47e-04	0.2943	4.9605	0.8029	3.90e-05	0.1012	0.0851
% 0.005	1	0.1	3.46e-04	0.1889	5.4230	0.6357	1.52e-05	0.0971	0.0723
% 0.005	10	1	3.86e-04	0.2126	4.8257	0.7745	2.54e-05	0.1011	0.0543
% 0.005	10	1	4.32e-04	0.2348	5.3948	0.7070	2.06e-05	0.1177	0.1008
% 0.005	5	5	1.25e-03	0.4943	7.1149	0.7932	7.55e-06	0.1067	0.0473
% 0.005	5	5	3.58e-04	0.1897	4.1376	0.6413	3.32e-06	0.0890	0.0610
% 0.005	0.5	0.5	6.35e-04	0.2430	4.2569	0.4949	2.49e-05	0.0738	0.0327
% 0.005	5	5	4.26e-04	0.2096	4.2674	0.5707	5.42e-06	0.0836	0.0445
% 0.005	0.5	0.5	5.45e-04	0.2386	5.2190	0.5263	2.26e-05	0.0905	0.0364
% 0.005	1	0.1	2.72e-04	0.1370	4.7303	0.5106	1.02e-05	0.0819	0.0572
% 0.005	5	5	9.06e-03	3.6098	5.1530	0.4959	2.07e-05	0.0841	0.0297
% 0.01	10	1	3.94e-04	0.2374	4.6882	0.8045	3.61e-06	0.1061	0.0257
% 0.01	10	1	3.27e-04	0.1922	4.5979	0.8086	1.75e-08	0.0992	0.0116
% 0.01	0.5	0.5	2.63e-04	0.1478	3.0808	0.6401	2.73e-06	0.0885	0.0127
% 0.01	1	0.1	1.89e-04	0.1076	2.8853	0.2661	1.07e-06	0.0809	0.0139
% 0.1	5	5	1.48e-03	0.6912	24.8279	0.9631	1.85e-05	0.1068	0.2113
% 0.1	5	5	1.37e-03	0.6107	42.0342	1.0088	1.05e-05	0.0968	0.2744
% 0.1	1	0.1	1.45e-03	0.7042	20.8625	1.0107	2.30e-05	0.1051	0.1828
% 0.1	5	5	1.30e-03	0.5869	44.9507	0.8679	5.02e-06	0.0946	0.2279


\setlength{\tabcolsep}{4pt}
\renewcommand{\arraystretch}{1.3}
\begin{table*}[h]
    \small
    \caption{
    \textit{Objective metrics for non parametric models of \textbf{Harley Benton Drop Kick} distortion.}
    \textit{Bold indicates best performing models.}
    \textit{Learning rate multiplier for nonlinearity in gray-box models shown in brackets.}
    }
    \label{tab:metrics_od_diy-klon-centaur}
    \centerline{
        \begin{tabular}{lccccccccccccc}
            \hline
            \midrule
            
            \multirow{2}{*}{Model}
                & \multirow{2}{*}{Params.}
                    & \multirow{2}{*}{LR}
                        & \multicolumn{2}{c}{Weights}
                            & \multicolumn{3}{c}{}
                                & \multicolumn{3}{c}{FAD} \\ 
            \cmidrule(lr){4-5} 
                % \cmidrule(lr){6-8} 
                    \cmidrule(lr){9-12}
            
            &   &   & L1 & MR-STFT & MSE & ESR & MAPE & VGGish & PANN & CLAP & AFx-Rep \\ 
            \hline
            LSTM-32 & 4.5k & 0.001 & 5 & 5 & 8.74e-02 & 1.0047 & 14.4504 & 10.5485 & 7.54e-06 & 0.2205 & 0.6451 \\
            LSTM-96 & 38.1k & 0.001 & 1 & 0.1 & 8.42e-02 & 0.9664 & 6.4870 & 12.3535 & 2.28e-04 & 0.2478 & 0.7762 \\
            \hline
            TCN-45-S-16 & 7.5k & 0.005 & 5 & 5 & 6.98e-02 & 0.8266 & 8.8056 & 1.0275 & 1.44e-05 & 0.0468 & 0.0640 \\
            TCN-45-L-16 & 7.3k & 0.005 & 0.5 & 0.5 & 7.30e-02 & 0.8620 & 8.4297 & 0.9014 & 9.25e-06 & 0.0425 & 0.0375 \\
            TCN-250-S-16 & 14.5k & 0.005 & 5 & 5 & 6.51e-02 & 0.7714 & 6.6630 & 0.8192 & 5.69e-05 & 0.0417 & 0.0313 \\
            TCN-250-L-16 & 18.4k & 0.005 & 0.5 & 0.5 & 6.20e-02 & 0.7351 & 12.6195 & 0.9232 & 1.17e-04 & 0.0562 & 0.0278 \\
            TCN-2500-S-16 & 13.7k & 0.005 & 0.5 & 0.5 & 6.64e-02 & 0.8001 & 40.7961 & 0.7086 & 8.30e-05 & 0.0392 & 0.0573 \\
            TCN-2500-L-16 & 11.9k & 0.005 & 5 & 5 & 6.90e-02 & 0.8203 & 5.3636 & 0.6465 & 1.92e-05 & 0.0270 & 0.0189 \\
            \hline
            TCN-TF-45-S-16 & 39.5k & 0.005 & 0.5 & 0.5 & 3.35e-02 & 0.4090 & 5.0408 & 2.1693 & 3.79e-04 & 0.0661 & 0.0803 \\
            TCN-TF-45-L-16 & 71.3k & 0.005 & 5 & 5 & 2.57e-02 & 0.3175 & 4.0402 & 1.7221 & 3.10e-04 & 0.0342 & 0.0445 \\
            TCN-TF-250-S-16 & 52.9k & 0.005 & 5 & 5 & 2.88e-02 & 0.3540 & 3.7182 & 1.3592 & 2.63e-04 & 0.0397 & 0.0491 \\
            TCN-TF-250-L-16 & 88.8k & 0.005 & 5 & 5 & 2.19e-02 & 0.2713 & 3.2864 & 0.9728 & 2.22e-04 & 0.0256 & 0.0385 \\
            TCN-TF-2500-S-16 & 45.7k & 0.005 & 0.5 & 0.5 & 2.09e-02 & 0.2541 & 4.4876 & 1.0521 & 1.49e-04 & 0.0385 & 0.0775 \\
            TCN-TF-2500-L-16 & 75.9k & 0.005 & 10 & 1 & \textbf{8.15e-03} & \textbf{0.0950} & \textbf{2.6796} & 3.4022 & 7.43e-04 & 0.1095 & 0.0926 \\
            \hline
            GCN-45-S-16 & 16.2k & 0.005 & 0.5 & 0.5 & 7.21e-02 & 0.8479 & 8.3446 & 0.9866 & 4.52e-05 & 0.0482 & 0.0941 \\
            GCN-45-L-16 & 17.1k & 0.005 & 5 & 5 & 7.60e-02 & 0.8967 & 7.2561 & 1.2360 & 5.16e-07 & 0.0500 & 0.0897 \\
            GCN-250-S-16 & 30.4k & 0.005 & 5 & 5 & 6.98e-02 & 0.8266 & 7.2827 & 0.7918 & 2.68e-05 & 0.0346 & 0.0532 \\
            GCN-250-L-16 & 39.6k & 0.005 & 5 & 5 & 6.81e-02 & 0.8115 & 8.5848 & 1.1609 & 1.81e-05 & 0.0528 & 0.0352 \\
            GCN-2500-S-16 & 28.6k & 0.005 & 5 & 5 & 7.04e-02 & 0.8298 & 6.6477 & 0.8074 & 3.31e-05 & 0.0390 & 0.0518 \\
            GCN-2500-L-16 & 26.4k & 0.005 & 5 & 5 & 7.01e-02 & 0.8363 & 5.0842 & 0.8746 & 3.11e-06 & 0.0312 & 0.0515 \\
            \hline
            GCN-TF-45-S-16 & 141.6k & 0.005 & 0.5 & 0.5 & 3.08e-02 & 0.3788 & 4.4110 & 1.2935 & 1.93e-04 & 0.0427 & 0.0784 \\
            GCN-TF-45-L-16 & 268.0k & 0.005 & 0.5 & 0.5 & 2.74e-02 & 0.3366 & 4.2464 & 1.4893 & 4.20e-04 & 0.0428 & 0.0501 \\
            GCN-TF-250-S-16 & 181.0k & 0.005 & 5 & 5 & 3.05e-02 & 0.3765 & 3.6688 & 1.2289 & 3.02e-04 & 0.0435 & 0.0559 \\
            GCN-TF-250-L-16 & 315.6k & 0.005 & 5 & 5 & 2.38e-02 & 0.2948 & 3.9232 & 1.4309 & 3.17e-04 & 0.0268 & 0.0373 \\
            GCN-TF-2500-S-16 & 154.1k & 0.005 & 5 & 5 & 3.00e-02 & 0.3622 & 4.4489 & 1.7656 & 1.81e-04 & 0.0450 & 0.0750 \\
            GCN-TF-2500-L-16 & 277.3k & 0.005 & 0.5 & 0.5 & 2.64e-02 & 0.3268 & 8.1004 & 1.5557 & 3.52e-04 & 0.0362 & 0.0618 \\
            \hline
            S4-S-16 & 2.4k & 0.01 & 0.5 & 0.5 & 7.73e-02 & 0.9161 & 8.7527 & 0.4475 & 5.20e-06 & 0.0191 & 0.0479 \\
            S4-L-16 & 19.0k & 0.01 & 5 & 5 & 1.24e-01 & 1.3801 & 7.6756 & \textbf{0.2166} & 5.14e-06 & \textbf{0.0103} & \textbf{0.0057} \\
            \hline
            S4-TF-S-16 & 28.0k & 0.01 & 5 & 5 & 4.39e-02 & 0.5296 & 4.5282 & 1.5957 & 5.72e-04 & 0.0348 & 0.0327 \\
            S4-TF-L-16 & 70.2k & 0.01 & 5 & 5 & 3.61e-02 & 0.4382 & 4.5640 & 1.3339 & 4.48e-04 & 0.0222 & 0.0367 \\
            \hline
            GB-DIST-MLP & 2.2k & 0.1 & 5 & 5 & 1.51e-01 & 1.7542 & 9.5654 & 1.2074 & \textbf{1.88e-07} & 0.0312 & 0.2084 \\
            GB-DIST-RNL & 47 & 0.1 & 5 & 5 & 1.54e-01 & 1.7786 & 8.4294 & 1.1364 & 3.86e-07 & 0.0336 & 0.4717 \\
            \hline
            GB-FUZZ-MLP & 2.3k & 0.1 & 5 & 5 & 1.55e-01 & 1.7895 & 21.3856 & 1.2245 & 2.95e-07 & 0.0365 & 0.2212 \\
            GB-FUZZ-RNL & 62 & 0.1 & 0.5 & 0.5 & 1.55e-01 & 1.7890 & 10.4823 & 1.0056 & 5.07e-06 & 0.0337 & 0.4870 \\
            \hline
            \hline
        \end{tabular}
    }
\end{table*}

% 0.001	5	5	8.74e-02	1.0047	14.4504	10.5485	7.54e-06	0.2205	0.6451
% 0.001	1	0.1	8.42e-02	0.9664	6.4870	12.3535	2.28e-04	0.2478	0.7762
% 0.005	5	5	6.98e-02	0.8266	8.8056	1.0275	1.44e-05	0.0468	0.0640
% 0.005	0.5	0.5	7.30e-02	0.8620	8.4297	0.9014	9.25e-06	0.0425	0.0375
% 0.005	5	5	6.51e-02	0.7714	6.6630	0.8192	5.69e-05	0.0417	0.0313
% 0.005	0.5	0.5	6.20e-02	0.7351	12.6195	0.9232	1.17e-04	0.0562	0.0278
% 0.005	0.5	0.5	6.64e-02	0.8001	40.7961	0.7086	8.30e-05	0.0392	0.0573
% 0.005	5	5	6.90e-02	0.8203	5.3636	0.6465	1.92e-05	0.0270	0.0189
% 0.005	0.5	0.5	3.35e-02	0.4090	5.0408	2.1693	3.79e-04	0.0661	0.0803
% 0.005	5	5	2.57e-02	0.3175	4.0402	1.7221	3.10e-04	0.0342	0.0445
% 0.005	5	5	2.88e-02	0.3540	3.7182	1.3592	2.63e-04	0.0397	0.0491
% 0.005	5	5	2.19e-02	0.2713	3.2864	0.9728	2.22e-04	0.0256	0.0385
% 0.005	0.5	0.5	2.09e-02	0.2541	4.4876	1.0521	1.49e-04	0.0385	0.0775
% 0.005	10	1	8.15e-03	0.0950	2.6796	3.4022	7.43e-04	0.1095	0.0926
% 0.005	0.5	0.5	7.21e-02	0.8479	8.3446	0.9866	4.52e-05	0.0482	0.0941
% 0.005	5	5	7.60e-02	0.8967	7.2561	1.2360	5.16e-07	0.0500	0.0897
% 0.005	5	5	6.98e-02	0.8266	7.2827	0.7918	2.68e-05	0.0346	0.0532
% 0.005	5	5	6.81e-02	0.8115	8.5848	1.1609	1.81e-05	0.0528	0.0352
% 0.005	5	5	7.04e-02	0.8298	6.6477	0.8074	3.31e-05	0.0390	0.0518
% 0.005	5	5	7.01e-02	0.8363	5.0842	0.8746	3.11e-06	0.0312	0.0515
% 0.005	0.5	0.5	3.08e-02	0.3788	4.4110	1.2935	1.93e-04	0.0427	0.0784
% 0.005	0.5	0.5	2.74e-02	0.3366	4.2464	1.4893	4.20e-04	0.0428	0.0501
% 0.005	5	5	3.05e-02	0.3765	3.6688	1.2289	3.02e-04	0.0435	0.0559
% 0.005	5	5	2.38e-02	0.2948	3.9232	1.4309	3.17e-04	0.0268	0.0373
% 0.005	5	5	3.00e-02	0.3622	4.4489	1.7656	1.81e-04	0.0450	0.0750
% 0.005	0.5	0.5	2.64e-02	0.3268	8.1004	1.5557	3.52e-04	0.0362	0.0618
% 0.01	0.5	0.5	7.73e-02	0.9161	8.7527	0.4475	5.20e-06	0.0191	0.0479
% 0.01	5	5	1.24e-01	1.3801	7.6756	0.2166	5.14e-06	0.0103	0.0057
% 0.01	5	5	4.39e-02	0.5296	4.5282	1.5957	5.72e-04	0.0348	0.0327
% 0.01	5	5	3.61e-02	0.4382	4.5640	1.3339	4.48e-04	0.0222	0.0367
% 0.1	5	5	1.51e-01	1.7542	9.5654	1.2074	1.88e-07	0.0312	0.2084
% 0.1	5	5	1.54e-01	1.7786	8.4294	1.1364	3.86e-07	0.0336	0.4717
% 0.1	5	5	1.55e-01	1.7895	21.3856	1.2245	2.95e-07	0.0365	0.2212
% 0.1	0.5	0.5	1.55e-01	1.7890	10.4823	1.0056	5.07e-06	0.0337	0.4870
\setlength{\tabcolsep}{4pt}
\renewcommand{\arraystretch}{1.3}
\begin{table*}[h]
    \small
    \caption{
    \textit{Objective metrics for non parametric models of \textbf{Harley Benton Plexicon} distortion.}
    \textit{Bold indicates best performing models.}
    \textit{Learning rate multiplier for nonlinearity in gray-box models shown in brackets.}
    }
    \label{tab:metrics_od_diy-klon-centaur}
    \centerline{
        \begin{tabular}{lccccccccccccc}
            \hline
            \midrule
            
            \multirow{2}{*}{Model}
                & \multirow{2}{*}{Params.}
                    & \multirow{2}{*}{LR}
                        & \multicolumn{2}{c}{Weights}
                            & \multicolumn{3}{c}{}
                                & \multicolumn{3}{c}{FAD} \\ 
            \cmidrule(lr){4-5} 
                % \cmidrule(lr){6-8} 
                    \cmidrule(lr){9-12}
            
            &   &   & L1 & MR-STFT & MSE & ESR & MAPE & VGGish & PANN & CLAP & AFx-Rep \\ 
            \hline
            LSTM-32 & 4.5k & 0.005 & 1 & 0.1 & 5.73e-04 & 0.0054 & 2.3736 & 0.0833 & 2.97e-07 & 0.0060 & 0.0036 \\
            LSTM-96 & 38.1k & 0.005 & 5 & 5 & \textbf{4.21e-04} & \textbf{0.0040} & 1.2612 & 0.0483 & \textbf{1.22e-08} & 0.0031 & \textbf{0.0020} \\
            \hline
            TCN-45-S-16 & 7.5k & 0.005 & 5 & 5 & 3.13e-03 & 0.0304 & 3.5541 & 0.1646 & 1.16e-05 & 0.0105 & 0.0093 \\
            TCN-45-L-16 & 7.3k & 0.005 & 0.5 & 0.5 & 7.92e-03 & 0.0757 & 8.8844 & 0.5941 & 2.43e-04 & 0.0468 & 0.0455 \\
            TCN-250-S-16 & 14.5k & 0.005 & 5 & 5 & 2.13e-03 & 0.0199 & 1.9187 & 0.1228 & 3.76e-06 & 0.0087 & 0.0077 \\
            TCN-250-L-16 & 18.4k & 0.005 & 10 & 1 & 2.28e-03 & 0.0225 & 4.6874 & 0.1808 & 1.07e-05 & 0.0164 & 0.0206 \\
            TCN-2500-S-16 & 13.7k & 0.005 & 1 & 0.1 & 3.65e-03 & 0.0341 & 2.3259 & 0.1905 & 1.37e-05 & 0.0114 & 0.0159 \\
            TCN-2500-L-16 & 11.9k & 0.005 & 10 & 1 & 1.88e-03 & 0.0175 & 1.4000 & 0.1218 & 7.60e-06 & 0.0106 & 0.0068 \\
            \hline
            TCN-TF-45-S-16 & 39.5k & 0.005 & 5 & 5 & 2.04e-03 & 0.0188 & 1.3898 & 0.1445 & 6.04e-06 & 0.0106 & 0.0103 \\
            TCN-TF-45-L-16 & 71.3k & 0.005 & 0.5 & 0.5 & 1.24e-03 & 0.0116 & 1.1071 & 0.0958 & 8.65e-06 & 0.0070 & 0.0079 \\
            TCN-TF-250-S-16 & 52.9k & 0.005 & 5 & 5 & 1.24e-03 & 0.0116 & 1.5858 & 0.1158 & 1.09e-05 & 0.0071 & 0.0067 \\
            TCN-TF-250-L-16 & 88.8k & 0.005 & 0.5 & 0.5 & 7.99e-04 & 0.0075 & \textbf{0.9314} & 0.0856 & 8.97e-06 & 0.0050 & 0.0100 \\
            TCN-TF-2500-S-16 & 45.7k & 0.005 & 5 & 5 & 2.19e-03 & 0.0205 & 1.8846 & 0.1664 & 2.29e-05 & 0.0076 & 0.0120 \\
            TCN-TF-2500-L-16 & 75.9k & 0.005 & 0.5 & 0.5 & 1.00e-03 & 0.0095 & 2.1539 & 0.1023 & 1.25e-05 & 0.0074 & 0.0060 \\
            \hline
            GCN-45-S-16 & 16.2k & 0.005 & 0.5 & 0.5 & 1.31e-03 & 0.0123 & 1.4356 & 0.1037 & 6.33e-06 & 0.0071 & 0.0059 \\
            GCN-45-L-16 & 17.1k & 0.005 & 5 & 5 & 2.06e-03 & 0.0211 & 2.3339 & 0.1176 & 5.79e-06 & 0.0069 & 0.0077 \\
            GCN-250-S-16 & 30.4k & 0.005 & 0.5 & 0.5 & 1.54e-03 & 0.0148 & 2.3808 & 0.1215 & 5.08e-06 & 0.0079 & 0.0088 \\
            GCN-250-L-16 & 39.6k & 0.005 & 0.5 & 0.5 & 1.12e-03 & 0.0109 & 3.3324 & 0.1141 & 6.08e-06 & 0.0058 & 0.0043 \\
            GCN-2500-S-16 & 28.6k & 0.005 & 0.5 & 0.5 & 2.74e-03 & 0.0261 & 2.9833 & 0.1606 & 2.82e-06 & 0.0067 & 0.0141 \\
            GCN-2500-L-16 & 26.4k & 0.005 & 1 & 0.1 & 1.23e-03 & 0.0117 & 1.0922 & 0.1185 & 1.33e-05 & 0.0080 & 0.0065 \\
            \hline
            GCN-TF-45-S-16 & 141.6k & 0.005 & 0.5 & 0.5 & 1.32e-03 & 0.0123 & 1.6869 & 0.0974 & 9.96e-06 & 0.0072 & 0.0059 \\
            GCN-TF-45-L-16 & 268.0k & 0.005 & 5 & 5 & 8.05e-04 & 0.0076 & 1.3799 & 0.0870 & 5.71e-06 & 0.0074 & 0.0044 \\
            GCN-TF-250-S-16 & 181.0k & 0.005 & 5 & 5 & 1.04e-03 & 0.0098 & 1.4588 & 0.0933 & 3.04e-06 & 0.0069 & 0.0072 \\
            GCN-TF-250-L-16 & 315.6k & 0.005 & 5 & 5 & 6.86e-04 & 0.0066 & 1.3643 & 0.0832 & 5.00e-06 & 0.0049 & 0.0037 \\
            GCN-TF-2500-S-16 & 154.1k & 0.005 & 0.5 & 0.5 & 1.62e-03 & 0.0152 & 2.0690 & 0.1098 & 1.86e-05 & 0.0057 & 0.0055 \\
            GCN-TF-2500-L-16 & 277.3k & 0.005 & 0.5 & 0.5 & 9.05e-04 & 0.0087 & 1.0505 & 0.0980 & 4.99e-06 & 0.0044 & 0.0039 \\
            \hline
            S4-S-16 & 2.4k & 0.01 & 0.5 & 0.5 & 1.31e-03 & 0.0122 & 1.7591 & 0.0789 & 2.91e-07 & 0.0056 & 0.0066 \\
            S4-L-16 & 19.0k & 0.01 & 1 & 0.1 & 6.00e-04 & 0.0056 & 1.3946 & 0.0489 & 1.12e-07 & 0.0040 & 0.0025 \\
            \hline
            S4-TF-S-16 & 28.0k & 0.01 & 1 & 0.1 & 9.82e-04 & 0.0090 & 0.9568 & 0.1256 & 1.40e-07 & 0.0041 & 0.0044 \\
            S4-TF-L-16 & 70.2k & 0.01 & 0.5 & 0.5 & 5.33e-04 & 0.0047 & 1.3392 & \textbf{0.0409} & 7.18e-07 & \textbf{0.0026} & 0.0020 \\
            \hline
            GB-DIST-MLP & 2.2k & 0.1 & 0.5 & 0.5 & 2.27e-02 & 0.2128 & 4.8038 & 0.4266 & 9.17e-06 & 0.0165 & 0.2288 \\
            GB-DIST-RNL & 47 & 0.1 & 5 & 5 & 1.87e-02 & 0.1731 & 5.3675 & 0.2802 & 4.65e-06 & 0.0157 & 0.2007 \\
            \hline
            GB-FUZZ-MLP & 2.3k & 0.1 & 0.5 & 0.5 & 2.17e-02 & 0.2041 & 4.1903 & 0.4562 & 1.27e-05 & 0.0188 & 0.0886 \\
            GB-FUZZ-RNL & 62 & 0.1 & 5 & 5 & 2.51e-02 & 0.2367 & 4.6740 & 0.4497 & 1.71e-06 & 0.0174 & 0.2193 \\
            \hline
            \hline
        \end{tabular}
    }
\end{table*}

% 0.005	1	0.1	5.73e-04	0.0054	2.3736	0.0833	2.97e-07	0.0060	0.0036
% 0.005	5	5	4.21e-04	0.0040	1.2612	0.0483	1.22e-08	0.0031	0.0020
% 0.005	5	5	3.13e-03	0.0304	3.5541	0.1646	1.16e-05	0.0105	0.0093
% 0.005	0.5	0.5	7.92e-03	0.0757	8.8844	0.5941	2.43e-04	0.0468	0.0455
% 0.005	5	5	2.13e-03	0.0199	1.9187	0.1228	3.76e-06	0.0087	0.0077
% 0.005	10	1	2.28e-03	0.0225	4.6874	0.1808	1.07e-05	0.0164	0.0206
% 0.005	1	0.1	3.65e-03	0.0341	2.3259	0.1905	1.37e-05	0.0114	0.0159
% 0.005	10	1	1.88e-03	0.0175	1.4000	0.1218	7.60e-06	0.0106	0.0068
% 0.005	5	5	2.04e-03	0.0188	1.3898	0.1445	6.04e-06	0.0106	0.0103
% 0.005	0.5	0.5	1.24e-03	0.0116	1.1071	0.0958	8.65e-06	0.0070	0.0079
% 0.005	5	5	1.24e-03	0.0116	1.5858	0.1158	1.09e-05	0.0071	0.0067
% 0.005	0.5	0.5	7.99e-04	0.0075	0.9314	0.0856	8.97e-06	0.0050	0.0100
% 0.005	5	5	2.19e-03	0.0205	1.8846	0.1664	2.29e-05	0.0076	0.0120
% 0.005	0.5	0.5	1.00e-03	0.0095	2.1539	0.1023	1.25e-05	0.0074	0.0060
% 0.005	0.5	0.5	1.31e-03	0.0123	1.4356	0.1037	6.33e-06	0.0071	0.0059
% 0.005	5	5	2.06e-03	0.0211	2.3339	0.1176	5.79e-06	0.0069	0.0077
% 0.005	0.5	0.5	1.54e-03	0.0148	2.3808	0.1215	5.08e-06	0.0079	0.0088
% 0.005	0.5	0.5	1.12e-03	0.0109	3.3324	0.1141	6.08e-06	0.0058	0.0043
% 0.005	0.5	0.5	2.74e-03	0.0261	2.9833	0.1606	2.82e-06	0.0067	0.0141
% 0.005	1	0.1	1.23e-03	0.0117	1.0922	0.1185	1.33e-05	0.0080	0.0065
% 0.005	0.5	0.5	1.32e-03	0.0123	1.6869	0.0974	9.96e-06	0.0072	0.0059
% 0.005	5	5	8.05e-04	0.0076	1.3799	0.0870	5.71e-06	0.0074	0.0044
% 0.005	5	5	1.04e-03	0.0098	1.4588	0.0933	3.04e-06	0.0069	0.0072
% 0.005	5	5	6.86e-04	0.0066	1.3643	0.0832	5.00e-06	0.0049	0.0037
% 0.005	0.5	0.5	1.62e-03	0.0152	2.0690	0.1098	1.86e-05	0.0057	0.0055
% 0.005	0.5	0.5	9.05e-04	0.0087	1.0505	0.0980	4.99e-06	0.0044	0.0039
% 0.01	0.5	0.5	1.31e-03	0.0122	1.7591	0.0789	2.91e-07	0.0056	0.0066
% 0.01	1	0.1	6.00e-04	0.0056	1.3946	0.0489	1.12e-07	0.0040	0.0025
% 0.01	1	0.1	9.82e-04	0.0090	0.9568	0.1256	1.40e-07	0.0041	0.0044
% 0.01	0.5	0.5	5.33e-04	0.0047	1.3392	0.0409	7.18e-07	0.0026	0.0020
% 0.1	    0.5	0.5	2.27e-02	0.2128	4.8038	0.4266	9.17e-06	0.0165	0.2288
% 0.1	    5	5	1.87e-02	0.1731	5.3675	0.2802	4.65e-06	0.0157	0.2007
% 0.1	    0.5	0.5	2.17e-02	0.2041	4.1903	0.4562	1.27e-05	0.0188	0.0886
% 0.1	    5	5	2.51e-02	0.2367	4.6740	0.4497	1.71e-06	0.0174	0.2193
\setlength{\tabcolsep}{4pt}
\renewcommand{\arraystretch}{1.3}
\begin{table*}[h]
    \small
    \caption{
    \textit{Objective metrics for non parametric models of \textbf{Harley Benton Rodent} distortion.}
    \textit{Bold indicates best performing models.}
    \textit{Learning rate multiplier for nonlinearity in gray-box models shown in brackets.}
    }
    \label{tab:metrics_od_diy-klon-centaur}
    \centerline{
        \begin{tabular}{lccccccccccccc}
            \hline
            \midrule
            
            \multirow{2}{*}{Model}
                & \multirow{2}{*}{Params.}
                    & \multirow{2}{*}{LR}
                        & \multicolumn{2}{c}{Weights}
                            & \multicolumn{3}{c}{}
                                & \multicolumn{3}{c}{FAD} \\ 
            \cmidrule(lr){4-5} 
                % \cmidrule(lr){6-8} 
                    \cmidrule(lr){9-12}
            
            &   &   & L1 & MR-STFT & MSE & ESR & MAPE & VGGish & PANN & CLAP & AFx-Rep \\ 
            \hline
            LSTM-32 & 4.5k & 0.005 & 0.5 & 0.5 & 6.83e-03 & 0.4959 & 3.6359 & 6.6231 & 1.35e-04 & 0.1759 & 0.5508 \\
            LSTM-96 & 38.1k & 0.001 & 5 & 5 & 6.83e-03 & 0.4928 & 2.5585 & 8.7789 & 5.36e-04 & 0.2618 & 0.6084 \\
            \hline
            TCN-45-S-16 & 7.5k & 0.005 & 10 & 1 & 2.06e-03 & 0.1556 & 2.6272 & 0.7638 & 5.75e-06 & 0.0701 & 0.0890 \\
            TCN-45-L-16 & 7.3k & 0.005 & 5 & 5 & 1.83e-03 & 0.1367 & 1.5607 & 0.7710 & 1.96e-05 & 0.0376 & 0.0364 \\
            TCN-250-S-16 & 14.5k & 0.005 & 1 & 0.1 & 1.32e-03 & 0.0981 & 1.3677 & 0.6382 & 6.56e-06 & 0.0411 & 0.0624 \\
            TCN-250-L-16 & 18.4k & 0.005 & 0.5 & 0.5 & 9.68e-04 & 0.0728 & 1.3386 & 0.4340 & 6.89e-06 & 0.0285 & 0.0266 \\
            TCN-2500-S-16 & 13.7k & 0.005 & 5 & 5 & 1.58e-03 & 0.1185 & 1.6631 & 0.5459 & 1.52e-05 & 0.0434 & 0.0768 \\
            TCN-2500-L-16 & 11.9k & 0.005 & 10 & 1 & 9.94e-04 & 0.0741 & 1.2711 & 0.5598 & 1.53e-05 & 0.0358 & 0.0316 \\
            \hline
            TCN-TF-45-S-16 & 39.5k & 0.005 & 1 & 0.1 & 7.85e-04 & 0.0584 & 0.9434 & 0.3768 & 3.13e-07 & 0.0426 & 0.0439 \\
            TCN-TF-45-L-16 & 71.3k & 0.005 & 10 & 1 & 8.40e-04 & 0.0625 & 0.9973 & 0.4241 & 1.58e-06 & 0.0491 & 0.0773 \\
            TCN-TF-250-S-16 & 52.9k & 0.005 & 5 & 5 & 7.77e-04 & 0.0578 & 1.0744 & 0.4545 & 5.66e-07 & 0.0348 & 0.0655 \\
            TCN-TF-250-L-16 & 88.8k & 0.005 & 1 & 0.1 & 8.72e-04 & 0.0644 & 0.9415 & 0.3636 & 2.06e-06 & 0.0420 & 0.0658 \\
            TCN-TF-2500-S-16 & 45.7k & 0.005 & 10 & 1 & 7.85e-04 & 0.0583 & 1.0832 & 0.4552 & 9.58e-08 & 0.0423 & 0.0368 \\
            TCN-TF-2500-L-16 & 75.9k & 0.005 & 0.5 & 0.5 & 5.10e-04 & 0.0380 & 0.9725 & 0.3843 & 8.40e-06 & 0.0275 & 0.0555 \\
            \hline
            GCN-45-S-16 & 16.2k & 0.005 & 5 & 5 & 1.53e-03 & 0.1145 & 1.6186 & 0.6044 & 4.31e-06 & 0.0401 & 0.0772 \\
            GCN-45-L-16 & 17.1k & 0.005 & 5 & 5 & 1.50e-03 & 0.1113 & 1.5865 & 0.6837 & 4.47e-06 & 0.0520 & 0.0502 \\
            GCN-250-S-16 & 30.4k & 0.005 & 5 & 5 & 1.16e-03 & 0.0861 & 1.3995 & 0.5936 & 3.11e-07 & 0.0435 & 0.0490 \\
            GCN-250-L-16 & 39.6k & 0.005 & 1 & 0.1 & 7.75e-04 & 0.0574 & 1.1568 & 0.5820 & 4.06e-06 & 0.0636 & 0.0423 \\
            GCN-2500-S-16 & 28.6k & 0.005 & 1 & 0.1 & 1.13e-03 & 0.0838 & 1.3130 & 0.6754 & 6.00e-06 & 0.0432 & 0.0660 \\
            GCN-2500-L-16 & 26.4k & 0.005 & 1 & 0.1 & 7.18e-04 & 0.0533 & 1.0706 & 0.4771 & 1.42e-05 & 0.0546 & 0.0309 \\
            \hline
            GCN-TF-45-S-16 & 141.6k & 0.005 & 1 & 0.1 & 7.44e-04 & 0.0560 & 1.1428 & 0.4954 & 3.14e-06 & 0.0521 & 0.0641 \\
            GCN-TF-45-L-16 & 268.0k & 0.005 & 1 & 0.1 & 4.83e-04 & 0.0362 & 0.8751 & 0.2920 & 5.06e-06 & 0.0385 & 0.0472 \\
            GCN-TF-250-S-16 & 181.0k & 0.005 & 10 & 1 & 7.53e-04 & 0.0559 & 0.9686 & 0.3764 & 5.38e-06 & 0.0367 & 0.0397 \\
            GCN-TF-250-L-16 & 315.6k & 0.005 & 1 & 0.1 & 9.39e-04 & 0.0700 & 1.1924 & 0.3339 & 2.13e-06 & 0.0402 & 0.0287 \\
            GCN-TF-2500-S-16 & 154.1k & 0.005 & 10 & 1 & 6.31e-04 & 0.0474 & 1.0118 & 0.2901 & \textbf{2.67e-08} & 0.0368 & 0.0400 \\
            GCN-TF-2500-L-16 & 277.3k & 0.005 & 10 & 1 & \textbf{3.42e-04} & \textbf{0.0260} & 0.7787 & 0.3571 & 1.98e-05 & 0.0336 & 0.0221 \\
            \hline
            S4-S-16 & 2.4k & 0.01 & 0.5 & 0.5 & 7.85e-04 & 0.0595 & 1.0255 & 0.4654 & 1.65e-05 & 0.0276 & 0.0184 \\
            S4-L-16 & 19.0k & 0.01 & 10 & 1 & 5.43e-04 & 0.0409 & 0.8254 & 0.2271 & 2.16e-05 & \textbf{0.0143} & \textbf{0.0094} \\
            \hline
            S4-TF-S-16 & 28.0k & 0.01 & 1 & 0.1 & 4.87e-04 & 0.0366 & 0.8074 & \textbf{0.2233} & 8.89e-06 & 0.0237 & 0.0289 \\
            S4-TF-L-16 & 70.2k & 0.01 & 5 & 5 & 3.53e-04 & 0.0268 & \textbf{0.7482} & 0.3983 & 1.21e-05 & 0.0264 & 0.0232 \\
            \hline
            GB-DIST-MLP & 2.2k & 0.1 & 1 & 0.1 & 4.01e-03 & 0.2931 & 2.3269 & 0.7087 & 9.12e-06 & 0.0711 & 0.3428 \\
            GB-DIST-RNL & 47 & 0.1 & 1 & 0.1 & 4.09e-03 & 0.3008 & 2.9429 & 1.3022 & 1.52e-04 & 0.0732 & 0.4530 \\
            \hline
            GB-FUZZ-MLP & 2.3k & 0.1 & 10 & 1 & 3.82e-03 & 0.2797 & 2.4281 & 0.9981 & 2.11e-05 & 0.0602 & 0.1769 \\
            GB-FUZZ-RNL & 62 & 0.1 & 5 & 5 & 5.07e-03 & 0.3703 & 3.2628 & 1.4103 & 2.72e-05 & 0.0679 & 0.3126 \\
            \hline
            \hline
        \end{tabular}
    }
\end{table*}

% 0.005	0.5	0.5	6.83e-03	0.4959	3.6359	6.6231	1.35e-04	0.1759	0.5508
% 0.001	5	5	6.83e-03	0.4928	2.5585	8.7789	5.36e-04	0.2618	0.6084
% 0.005	10	1	2.06e-03	0.1556	2.6272	0.7638	5.75e-06	0.0701	0.0890
% 0.005	5	5	1.83e-03	0.1367	1.5607	0.7710	1.96e-05	0.0376	0.0364
% 0.005	1	0.1	1.32e-03	0.0981	1.3677	0.6382	6.56e-06	0.0411	0.0624
% 0.005	0.5	0.5	9.68e-04	0.0728	1.3386	0.4340	6.89e-06	0.0285	0.0266
% 0.005	5	5	1.58e-03	0.1185	1.6631	0.5459	1.52e-05	0.0434	0.0768
% 0.005	10	1	9.94e-04	0.0741	1.2711	0.5598	1.53e-05	0.0358	0.0316
% 0.005	1	0.1	7.85e-04	0.0584	0.9434	0.3768	3.13e-07	0.0426	0.0439
% 0.005	10	1	8.40e-04	0.0625	0.9973	0.4241	1.58e-06	0.0491	0.0773
% 0.005	5	5	7.77e-04	0.0578	1.0744	0.4545	5.66e-07	0.0348	0.0655
% 0.005	1	0.1	8.72e-04	0.0644	0.9415	0.3636	2.06e-06	0.0420	0.0658
% 0.005	10	1	7.85e-04	0.0583	1.0832	0.4552	9.58e-08	0.0423	0.0368
% 0.005	0.5	0.5	5.10e-04	0.0380	0.9725	0.3843	8.40e-06	0.0275	0.0555
% 0.005	5	5	1.53e-03	0.1145	1.6186	0.6044	4.31e-06	0.0401	0.0772
% 0.005	5	5	1.50e-03	0.1113	1.5865	0.6837	4.47e-06	0.0520	0.0502
% 0.005	5	5	1.16e-03	0.0861	1.3995	0.5936	3.11e-07	0.0435	0.0490
% 0.005	1	0.1	7.75e-04	0.0574	1.1568	0.5820	4.06e-06	0.0636	0.0423
% 0.005	1	0.1	1.13e-03	0.0838	1.3130	0.6754	6.00e-06	0.0432	0.0660
% 0.005	1	0.1	7.18e-04	0.0533	1.0706	0.4771	1.42e-05	0.0546	0.0309
% 0.005	1	0.1	7.44e-04	0.0560	1.1428	0.4954	3.14e-06	0.0521	0.0641
% 0.005	1	0.1	4.83e-04	0.0362	0.8751	0.2920	5.06e-06	0.0385	0.0472
% 0.005	10	1	7.53e-04	0.0559	0.9686	0.3764	5.38e-06	0.0367	0.0397
% 0.005	1	0.1	9.39e-04	0.0700	1.1924	0.3339	2.13e-06	0.0402	0.0287
% 0.005	10	1	6.31e-04	0.0474	1.0118	0.2901	2.67e-08	0.0368	0.0400
% 0.005	10	1	3.42e-04	0.0260	0.7787	0.3571	1.98e-05	0.0336	0.0221
% 0.01	0.5	0.5	7.85e-04	0.0595	1.0255	0.4654	1.65e-05	0.0276	0.0184
% 0.01	10	1	5.43e-04	0.0409	0.8254	0.2271	2.16e-05	0.0143	0.0094
% 0.01	1	0.1	4.87e-04	0.0366	0.8074	0.2233	8.89e-06	0.0237	0.0289
% 0.01	5	5	3.53e-04	0.0268	0.7482	0.3983	1.21e-05	0.0264	0.0232
% 0.1	    1	0.1	4.01e-03	0.2931	2.3269	0.7087	9.12e-06	0.0711	0.3428
% 0.1	    1	0.1	4.09e-03	0.3008	2.9429	1.3022	1.52e-04	0.0732	0.4530
% 0.1	    10	1	3.82e-03	0.2797	2.4281	0.9981	2.11e-05	0.0602	0.1769
% 0.1 	5	5	5.07e-03	0.3703	3.2628	1.4103	2.72e-05	0.0679	0.3126

\clearpage

% ===
\subsection{Results Fuzz}
\setlength{\tabcolsep}{3pt}
\renewcommand{\arraystretch}{1.3}
\begin{table*}[h]
    \small
    \caption{
    \textit{Scaled test loss for non parametric models of fuzz effects. Bold indicates best performing models.}
    }
    \label{tab:results_overall}
    \centerline{
        \begin{tabular}{l c >{\columncolor{gray!20}}ccc >{\columncolor{gray!20}}ccc >{\columncolor{gray!20}}ccc >{\columncolor{gray!20}}ccc} 
            \hline
            \hline
            \multirow{2}{*}{Model} 
                & \multirow{2}{*}{Params.} 
                    & \multicolumn{3}{c}{Custom Dynamic Fuzz} 
                        &  \multicolumn{3}{c}{Harley Benton Fuzzy Logic} 
                            & \multicolumn{3}{c}{Harley Benton Silly Fuzz} 
                                & \multicolumn{3}{c}{Arturia Spring636 Preamp} \\
            \cmidrule(lr){3-5} 
                \cmidrule(lr){6-8} 
                    \cmidrule(lr){9-11} 
                        \cmidrule(lr){12-14}
                        
            & & 
            Tot. & {\footnotesize L1} &  {\footnotesize MR-STFT} & 
            Tot. & {\footnotesize L1} &  {\footnotesize MR-STFT} & 
            Tot. & {\footnotesize L1} &  {\footnotesize MR-STFT} & 
            Tot. & {\footnotesize L1} &  {\footnotesize MR-STFT} \\ 

            \hline
            LSTM-32 & 4.5k & 0.5504 & 0.0205 & 0.5299 & 2.2845 & 0.1179 & 2.1666 & 1.9206 & 0.0471 & 1.8735 & 0.2687 & 0.0091 & 0.2595 \\ 
            LSTM-96 & 38.1k & 0.4541 & 0.0146 & 0.4395 & 2.3062 & 0.1168 & 2.1894 & 1.6853 & 0.0422 & 1.6431 & 0.1760 & 0.0123 & 0.1637 \\ 
            \hline
            TCN-45-S-16 & 7.5k & 0.9885 & 0.0523 & 0.9362 & 1.3093 & 0.0682 & 1.2411 & 0.8025 & 0.0107 & 0.7918 & 0.5928 & 0.0292 & 0.5635 \\ 
            TCN-45-L-16 & 7.3k & 0.9854 & 0.0535 & 0.9319 & 0.8215 & 0.0234 & 0.7981 & 0.7405 & 0.0095 & 0.7310 & 0.5882 & 0.0228 & 0.5654 \\
            TCN-250-S-16 & 14.5k & 0.9032 & 0.0408 & 0.8624 & 0.9163 & 0.0245 & 0.8918 & 0.7379 & 0.0090 & 0.7289 & 0.3764 & 0.0109 & 0.3655 \\ 
            TCN-250-L-16 & 18.4k & 0.8851 & 0.0406 & 0.8445 & 0.6819 & 0.0127 & 0.6692 & 0.6666 & 0.0062 & 0.6604 & 0.4225 & 0.0114 & 0.4112 \\
            TCN-2500-S-16 & 13.7k & 0.8728 & 0.0404 & 0.8325 & 1.3715 & 0.0723 & 1.2992 & 0.8418 & 0.0124 & 0.8293 & 0.4032 & 0.0182 & 0.3850 \\ 
            TCN-2500-L-16 & 11.9k & 0.8819 & 0.0420 & 0.8399 & 0.6843 & 0.0150 & 0.6694 & 0.6682 & 0.0076 & 0.6606 & 0.3497 & 0.0192 & 0.3305 \\
            \hline
            TCN-TF-45-S-16 & 39.5k & 0.5797 & 0.0191 & 0.5605 & 0.7721 & 0.0174 & 0.7547 & 0.7451 & 0.0080 & 0.7371 & 0.3211 & 0.0059 & 0.3152 \\
            TCN-TF-45-L-16 & 71.3k & 0.5136 & 0.0158 & 0.4978 & 0.5586 & 0.0090 & 0.5496 & 0.6570 & 0.0057 & 0.6513 & 0.2669 & 0.0069 & 0.2600 \\
            TCN-TF-250-S-16 & 52.9k & 0.5413 & 0.0173 & 0.5241 & 0.6030 & 0.0112 & 0.5918 & 0.7018 & 0.0076 & 0.6942 & 0.3278 & 0.0064 & 0.3215 \\
            TCN-TF-250-L-16 & 88.8k & 0.5265 & 0.0160 & 0.5106 & 0.6077 & 0.0117 & 0.5960 & 0.6634 & 0.0170 & 0.6463 & 0.2505 & 0.0066 & 0.2439 \\
            TCN-TF-2500-S-16 & 45.7k & 0.5513 & 0.0192 & 0.5321 & 0.6829 & 0.0156 & 0.6673 & 1.0854 & 0.0152 & 1.0702 & 0.3203 & 0.0145 & 0.3058 \\
            TCN-TF-2500-L-16 & 75.9k & 0.5073 & 0.0148 & 0.4925 & 0.8263 & 0.0212 & 0.8052 & 0.6428 & 0.0139 & 0.6289 & 0.2588 & 0.0217 & 0.2371 \\
            \hline
            GCN-45-S-16 & 16.2k & 0.9622 & 0.0585 & 0.9037 & 0.8225 & 0.0174 & 0.8050 & 0.7125 & 0.0068 & 0.7057 & 0.5803 & 0.0227 & 0.5576 \\ 
            GCN-45-L-16 & 17.1k & 0.9417 & 0.0556 & 0.8861 & 0.7138 & 0.0209 & 0.6930 & 0.6452 & 0.0056 & 0.6396 & 0.5943 & 0.0384 & 0.5558 \\
            GCN-250-S-16 & 30.4k & 0.8493 & 0.0457 & 0.8036 & 0.7027 & 0.0137 & 0.6890 & 0.6327 & 0.0057 & 0.6270 & 0.3552 & 0.0098 & 0.3453 \\ 
            GCN-250-L-16 & 39.6k & 0.8844 & 0.0401 & 0.8443 & 0.6534 & 0.0113 & 0.6421 & 0.6523 & 0.0054 & 0.6469 & 0.3381 & 0.0089 & 0.3292 \\
            GCN-2500-S-16 & 28.6k & 0.7325 & 0.0324 & 0.7001 & 0.6822 & 0.0157 & 0.6665 & 0.6528 & 0.0087 & 0.6441 & 0.3267 & 0.0147 & 0.3120 \\ 
            GCN-2500-L-16 & 26.4k & 0.7103 & 0.0291 & 0.6812 & 0.7378 & 0.0165 & 0.7212 & 0.6804 & 0.0069 & 0.6735 & 0.3163 & 0.0100 & 0.3063 \\
            \hline
            GCN-TF-45-S-16 & 141.6k & 0.5409 & 0.0228 & 0.5181 & 0.5609 & 0.0089 & 0.5519 & 0.7960 & 0.0098 & 0.7861 & 0.2566 & 0.0047 & 0.2519 \\
            GCN-TF-45-L-16 & 268.0k & 0.4949 & 0.0220 & 0.4729 & 0.4624 & 0.0066 & 0.4558 & 0.6006 & 0.0051 & 0.5955 & 0.2229 & 0.0164 & 0.2066 \\
            GCN-TF-250-S-16 & 181.0k & 0.5082 & 0.0240 & 0.4842 & 0.6930 & 0.0231 & 0.6699 & 1.1097 & 0.0203 & 1.0894 & 0.2183 & 0.0109 & 0.2074 \\
            GCN-TF-250-L-16 & 315.6k & 0.4778 & 0.0146 & 0.4633 & 0.8390 & 0.0178 & 0.8212 & 0.6889 & 0.0067 & 0.6823 & 0.2196 & 0.0073 & 0.2123 \\
            GCN-TF-2500-S-16 & 154.1k & 0.5216 & 0.0246 & 0.4970 & 0.6920 & 0.0211 & 0.6709 & 0.6078 & 0.0073 & 0.6004 & 0.2624 & 0.0123 & 0.2501 \\
            GCN-TF-2500-L-16 & 277.3k & 0.4868 & 0.0214 & 0.4654 & 0.5529 & 0.0137 & 0.5393 & 0.5687 & 0.0065 & 0.5622 & 0.2103 & 0.0123 & 0.1980 \\
            \hline
            S4-S-16 & 2.4k & 0.6468 & 0.0295 & 0.6173 & 0.5937 & 0.0098 & 0.5839 & 0.6539 & 0.0053 & 0.6486 & 0.2290 & 0.0085 & 0.2205 \\
            S4-L-16 & 19.0k & 0.5737 & 0.0241 & 0.5496 & \textbf{0.4336} & 0.0048 & 0.4288 & 0.5362 & 0.0038 & 0.5324 & \textbf{0.1431} & 0.0090 & 0.1340 \\
            \hline
            S4-TF-S-16 & 28.0k & 0.4544 & 0.0173 & 0.4371 & 0.5507 & 0.0079 & 0.5429 & 0.5531 & 0.0041 & 0.5490 & 0.1719 & 0.0069 & 0.1650 \\
            S4-TF-L-16 & 70.2k & \textbf{0.4344} & 0.0153 & 0.4190 & 0.6048 & 0.0120 & 0.5928 & \textbf{0.4980} & 0.0033 & 0.4946 & 0.1469 & 0.0038 & 0.1431 \\
            \hline
            GB-DIST-MLP & 2.2k & 1.3593 & 0.1271 & 1.2322 & 1.8472 & 0.1119 & 1.7353 & 1.7205 & 0.1094 & 1.6111 & 0.8404 & 0.0408 & 0.7996 \\
            GB-DIST-RNL & 47 & 1.4017 & 0.1324 & 1.2692 & 1.9171 & 0.1083 & 1.8088 & 1.8562 & 0.0479 & 1.8083 & 0.8593 & 0.0461 & 0.8132 \\
            \hline
            GB-FUZZ-MLP & 2.3k & 1.1861 & 0.0620 & 1.1241 & 1.4306 & 0.0531 & 1.3776 & 1.4077 & 0.0209 & 1.3868 & 0.7604 & 0.0355 & 0.7249 \\
            GB-FUZZ-RNL & 62 & 1.0867 & 0.0999 & 0.9868 & 1.5059 & 0.0563 & 1.4496 & 1.4646 & 0.0202 & 1.4444 & 0.8582 & 0.0452 & 0.8130 \\
            \hline
            \hline
        \end{tabular}
    }
\end{table*}


% 0.5504	0.0205	0.5299	2.2845	0.1179	2.1666	1.9206	0.0471	1.8735	0.2687	0.0091	0.2595
% 0.4541	0.0146	0.4395	2.3062	0.1168	2.1894	1.6853	0.0422	1.6431	0.1760	0.0123	0.1637
% 0.9885	0.0523	0.9362	1.3093	0.0682	1.2411	0.8025	0.0107	0.7918	0.5928	0.0292	0.5635
% 0.9854	0.0535	0.9319	0.8215	0.0234	0.7981	0.7405	0.0095	0.7310	0.5882	0.0228	0.5654
% 0.9032	0.0408	0.8624	0.9163	0.0245	0.8918	0.7379	0.0090	0.7289	0.3764	0.0109	0.3655
% 0.8851	0.0406	0.8445	0.6819	0.0127	0.6692	0.6666	0.0062	0.6604	0.4225	0.0114	0.4112
% 0.8728	0.0404	0.8325	1.3715	0.0723	1.2992	0.8418	0.0124	0.8293	0.4032	0.0182	0.3850
% 0.8819	0.0420	0.8399	0.6843	0.0150	0.6694	0.6682	0.0076	0.6606	0.3497	0.0192	0.3305
% 0.5797	0.0191	0.5605	0.7721	0.0174	0.7547	0.7451	0.0080	0.7371	0.3211	0.0059	0.3152
% 0.5136	0.0158	0.4978	0.5586	0.0090	0.5496	0.6570	0.0057	0.6513	0.2669	0.0069	0.2600
% 0.5413	0.0173	0.5241	0.6030	0.0112	0.5918	0.7018	0.0076	0.6942	0.3278	0.0064	0.3215
% 0.5265	0.0160	0.5106	0.6077	0.0117	0.5960	0.6634	0.0170	0.6463	0.2505	0.0066	0.2439
% 0.5513	0.0192	0.5321	0.6829	0.0156	0.6673	1.0854	0.0152	1.0702	0.3203	0.0145	0.3058
% 0.5073	0.0148	0.4925	0.8263	0.0212	0.8052	0.6428	0.0139	0.6289	0.2588	0.0217	0.2371
% 0.9622	0.0585	0.9037	0.8225	0.0174	0.8050	0.7125	0.0068	0.7057	0.5803	0.0227	0.5576
% 0.9417	0.0556	0.8861	0.7138	0.0209	0.6930	0.6452	0.0056	0.6396	0.5943	0.0384	0.5558
% 0.8493	0.0457	0.8036	0.7027	0.0137	0.6890	0.6327	0.0057	0.6270	0.3552	0.0098	0.3453
% 0.8844	0.0401	0.8443	0.6534	0.0113	0.6421	0.6523	0.0054	0.6469	0.3381	0.0089	0.3292
% 0.7325	0.0324	0.7001	0.6822	0.0157	0.6665	0.6528	0.0087	0.6441	0.3267	0.0147	0.3120
% 0.7103	0.0291	0.6812	0.7378	0.0165	0.7212	0.6804	0.0069	0.6735	0.3163	0.0100	0.3063
% 0.5409	0.0228	0.5181	0.5609	0.0089	0.5519	0.7960	0.0098	0.7861	0.2566	0.0047	0.2519
% 0.4949	0.0220	0.4729	0.4624	0.0066	0.4558	0.6006	0.0051	0.5955	0.2229	0.0164	0.2066
% 0.5082	0.0240	0.4842	0.6930	0.0231	0.6699	1.1097	0.0203	1.0894	0.2183	0.0109	0.2074
% 0.4778	0.0146	0.4633	0.8390	0.0178	0.8212	0.6889	0.0067	0.6823	0.2196	0.0073	0.2123
% 0.5216	0.0246	0.4970	0.6920	0.0211	0.6709	0.6078	0.0073	0.6004	0.2624	0.0123	0.2501
% 0.4868	0.0214	0.4654	0.5529	0.0137	0.5393	0.5687	0.0065	0.5622	0.2103	0.0123	0.1980
% 0.6468	0.0295	0.6173	0.5937	0.0098	0.5839	0.6539	0.0053	0.6486	0.2290	0.0085	0.2205
% 0.5737	0.0241	0.5496	0.4336	0.0048	0.4288	0.5362	0.0038	0.5324	0.1431	0.0090	0.1340
% 0.4544	0.0173	0.4371	0.5507	0.0079	0.5429	0.5531	0.0041	0.5490	0.1719	0.0069	0.1650
% 0.4344	0.0153	0.4190	0.6048	0.0120	0.5928	0.4980	0.0033	0.4946	0.1469	0.0038	0.1431
% 1.3593	0.1271	1.2322	1.8472	0.1119	1.7353	1.7205	0.1094	1.6111	0.8404	0.0408	0.7996
% 1.4017	0.1324	1.2692	1.9171	0.1083	1.8088	1.8562	0.0479	1.8083	0.8593	0.0461	0.8132
% 1.1861	0.0620	1.1241	1.4306	0.0531	1.3776	1.4077	0.0209	1.3868	0.7604	0.0355	0.7249
% 1.0867	0.0999	0.9868	1.5059	0.0563	1.4496	1.4646	0.0202	1.4444	0.8582	0.0452	0.8130





% %\setlength{\tabcolsep}{3.8pt}
% % \vspace{-0.3cm}
% % \renewcommand{\arraystretch}{0.85}
% \begin{table*}[h]
%     \centering
%     \small
%     \begin{tabular}{lcccccccccccc} \toprule
    
%         \multirow{2}{*}{Model} 
%             & \multirow{2}{*}{Params.} 
%                 & \multicolumn{2}{c}{Fuzz 1} 
%                     & \multicolumn{2}{c}{Fuzz 2} 
%                         &  \multicolumn{2}{c}{Fuzz 3} 
%                             & \multicolumn{2}{c}{Fuzz 4} \\ 
%         \cmidrule(lr){3-4} 
%             \cmidrule(lr){5-6} 
%                 \cmidrule(lr){7-8} 
%                     \cmidrule(lr){9-10}
%         &   & $L1$ & MR-STFT & $L1$ & MR-STFT & $L1$ & MR-STFT & $L1$ & MR-STFT \\ 
%         \midrule
%         LSTM-32
%             & 4.5k  & 0.012 & 0.356 & 0.001 & 0.239 & 0.002 & 0.250 & 0.004 & 0.236 \\ 
%         LSTM-96       
%             & - & - & - & - & - & - & - & - & - \\ 
%         \midrule
%         LSTM-TVC-32
%             & - & - & - & - & - & - & - & - & - \\
%         LSTM-TVC-96
%             & - & - & - & - & - & - & - & - & - \\
%         \midrule
%         TCN-45-S-16               
%             & - & - & - & - & - & - & - & - & - \\ 
%         TCN-45-L-16               
%             & - & - & - & - & - & - & - & - & - \\
%         \midrule
%         TCN-TF-45-S-16               
%             & - & - & - & - & - & - & - & - & - \\
%         TCN-TF-45-L-16               
%             & - & - & - & - & - & - & - & - & - \\
%         \midrule
%         TCN-TTF-45-S-16               
%             & - & - & - & - & - & - & - & - & - \\
%         TCN-TTF-45-L-16               
%             & - & - & - & - & - & - & - & - & - \\
%         \midrule
%         TCN-TVF-45-S-16               
%             & - & - & - & - & - & - & - & - & - \\
%         TCN-TVF-45-L-16               
%             & - & - & - & - & - & - & - & - & - \\
%         \midrule
%         GCN-45-S-16               
%             & - & - & - & - & - & - & - & - & - \\ 
%         GCN-45-L-16               
%             & - & - & - & - & - & - & - & - & - \\
%         \midrule
%         GCN-TF-45-S-16               
%             & - & - & - & - & - & - & - & - & - \\
%         GCN-TF-45-L-16               
%             & - & - & - & - & - & - & - & - & - \\
%         \midrule
%         GCN-TTF-45-S-16               
%             & - & - & - & - & - & - & - & - & - \\
%         GCN-TTF-45-L-16               
%             & - & - & - & - & - & - & - & - & - \\
%         \midrule
%         GCN-TVF-45-S-16               
%             & - & - & - & - & - & - & - & - & - \\
%         GCN-TVF-45-L-16               
%             & - & - & - & - & - & - & - & - & - \\
%         \midrule
%         SSM-1               
%             & - & - & - & - & - & - & - & - & - \\
%         SSM-2               
%             & - & - & - & - & - & - & - & - & - \\
%         \midrule
%         GB-COMP              
%             & - & - & - & - & - & - & - & - & - \\
%         \midrule
%         GB-DIST-MLP             
%             & - & - & - & - & - & - & - & - & - \\
%         GB-DIST-RNL             
%             & - & - & - & - & - & - & - & - & - \\
%         \midrule
%         GB-FUZZ-MLP             
%             & - & - & - & - & - & - & - & - & - \\
%         GB-FUZZ-RNL             
%             & - & - & - & - & - & - & - & - & - \\
%         \bottomrule 
%     \end{tabular}
%     \vspace{-0.0cm}
%     \caption{Overall L1+MR-STFT loss across device type for non parametric models}
%     \label{tab:other_fx} \vspace{0.2cm}
% \end{table*}

\setlength{\tabcolsep}{4pt}
\renewcommand{\arraystretch}{1.3}
\begin{table*}[h]
    \small
    \caption{
    \textit{Scaled validation and test loss for non parametric models of \textbf{Custom Dynamic Fuzz} fuzz.}
    \textit{Bold indicates best performing models.}
    \textit{Learning rate multiplier for nonlinearity in gray-box models shown in brackets.}
    }
    \label{tab:val-and-test-loss_od_fulltone-fulldrive}
    \centerline{
        \begin{tabular}{l c c cc >{\columncolor{gray!20}}ccc >{\columncolor{gray!20}}ccc}
            \hline
            \midrule
            
            \multirow{2}{*}{Model}
                & \multirow{2}{*}{Params.}
                    & \multirow{2}{*}{LR}
                        & \multicolumn{2}{c}{Weights}
                            & \multicolumn{3}{c}{Val. Loss}
                                & \multicolumn{3}{c}{Test Loss} \\ 
            
            \cmidrule(lr){4-5} 
                \cmidrule(lr){6-8} 
                    \cmidrule(lr){9-11}
            
            &   &   & {\scriptsize L1} & {\scriptsize MR-STFT} & Tot. & {\scriptsize L1} & {\scriptsize MR-STFT} & Tot. & {\scriptsize L1} & {\scriptsize MR-STFT} \\ 
            
            \hline
            LSTM-32 & 4.5k & 0.005 & 1 & 0.1 & 0.7049 & 0.0278 & 0.6771 & 0.5504 & 0.0205 & 0.5299 \\
            LSTM-96 & 38.1k & 0.005 & 10 & 1 & \textbf{0.5572} & 0.0213 & 0.5359 & 0.4541 & 0.0146 & 0.4395 \\
            \hline
            TCN-45-S-16 & 7.5k & 0.005 & 10 & 1 & 1.1265 & 0.0655 & 1.0609 & 0.9885 & 0.0523 & 0.9362 \\
            TCN-45-L-16 & 7.3k & 0.005 & 1 & 0.1 & 1.1229 & 0.0657 & 1.0572 & 0.9854 & 0.0535 & 0.9319 \\
            TCN-250-S-16 & 14.5k & 0.005 & 10 & 1 & 1.0447 & 0.0533 & 0.9914 & 0.9032 & 0.0408 & 0.8624 \\
            TCN-250-L-16 & 18.4k & 0.005 & 10 & 1 & 1.0360 & 0.0526 & 0.9835 & 0.8851 & 0.0406 & 0.8445 \\
            TCN-2500-S-16 & 13.7k & 0.005 & 10 & 1 & 1.0334 & 0.0534 & 0.9801 & 0.8728 & 0.0404 & 0.8325 \\
            TCN-2500-L-16 & 11.9k & 0.005 & 10 & 1 & 1.0179 & 0.0507 & 0.9672 & 0.8819 & 0.0420 & 0.8399 \\
            \hline
            TCN-TF-45-S-16 & 39.5k & 0.005 & 10 & 1 & 0.7597 & 0.0309 & 0.7289 & 0.5797 & 0.0191 & 0.5605 \\
            TCN-TF-45-L-16 & 71.3k & 0.005 & 1 & 0.1 & 0.7431 & 0.0320 & 0.7111 & 0.5136 & 0.0158 & 0.4978 \\
            TCN-TF-250-S-16 & 52.9k & 0.005 & 10 & 1 & 0.7633 & 0.0313 & 0.7320 & 0.5413 & 0.0173 & 0.5241 \\
            TCN-TF-250-L-16 & 88.8k & 0.005 & 10 & 1 & 0.7611 & 0.0315 & 0.7296 & 0.5265 & 0.0160 & 0.5106 \\
            TCN-TF-2500-S-16 & 45.7k & 0.005 & 1 & 0.1 & 0.7472 & 0.0313 & 0.7159 & 0.5513 & 0.0192 & 0.5321 \\
            TCN-TF-2500-L-16 & 75.9k & 0.005 & 1 & 0.1 & 0.6930 & 0.0282 & 0.6647 & 0.5073 & 0.0148 & 0.4925 \\
            \hline
            GCN-45-S-16 & 16.2k & 0.005 & 5 & 5 & 1.1000 & 0.0700 & 1.0300 & 0.9622 & 0.0585 & 0.9037 \\
            GCN-45-L-16 & 17.1k & 0.005 & 5 & 5 & 1.0838 & 0.0683 & 1.0156 & 0.9417 & 0.0556 & 0.8861 \\
            GCN-250-S-16 & 30.4k & 0.005 & 5 & 5 & 1.0057 & 0.0574 & 0.9483 & 0.8493 & 0.0457 & 0.8036 \\
            GCN-250-L-16 & 39.6k & 0.005 & 10 & 1 & 1.0129 & 0.0505 & 0.9623 & 0.8844 & 0.0401 & 0.8443 \\
            GCN-2500-S-16 & 28.6k & 0.005 & 10 & 1 & 0.9751 & 0.0481 & 0.9270 & 0.7325 & 0.0324 & 0.7001 \\
            GCN-2500-L-16 & 26.4k & 0.005 & 1 & 0.1 & 0.8555 & 0.0393 & 0.8162 & 0.7103 & 0.0291 & 0.6812 \\
            \hline
            GCN-TF-45-S-16 & 141.6k & 0.005 & 5 & 5 & 0.7316 & 0.0335 & 0.6981 & 0.5409 & 0.0228 & 0.5181 \\
            GCN-TF-45-L-16 & 268.0k & 0.005 & 5 & 5 & 0.6574 & 0.0340 & 0.6234 & 0.4949 & 0.0220 & 0.4729 \\
            GCN-TF-250-S-16 & 181.0k & 0.005 & 0.5 & 0.5 & 0.7418 & 0.0365 & 0.7053 & 0.5082 & 0.0240 & 0.4842 \\
            GCN-TF-250-L-16 & 315.6k & 0.005 & 10 & 1 & 0.6665 & 0.0271 & 0.6393 & 0.4778 & 0.0146 & 0.4633 \\
            GCN-TF-2500-S-16 & 154.1k & 0.005 & 5 & 5 & 0.7319 & 0.0367 & 0.6952 & 0.5216 & 0.0246 & 0.4970 \\
            GCN-TF-2500-L-16 & 277.3k & 0.005 & 5 & 5 & 0.6975 & 0.0350 & 0.6624 & 0.4868 & 0.0214 & 0.4654 \\
            \hline
            S4-S-16 & 2.4k & 0.01 & 10 & 1 & 0.8620 & 0.0426 & 0.8194 & 0.6468 & 0.0295 & 0.6173 \\
            S4-L-16 & 19.0k & 0.01 & 0.5 & 0.5 & 0.8022 & 0.0398 & 0.7624 & 0.5737 & 0.0241 & 0.5496 \\
            \hline
            S4-TF-S-16 & 28.0k & 0.01 & 0.5 & 0.5 & 0.6269 & 0.0272 & 0.5997 & 0.4544 & 0.0173 & 0.4371 \\
            S4-TF-L-16 & 70.2k & 0.01 & 0.5 & 0.5 & 0.6151 & 0.0267 & 0.5884 & \textbf{0.4344} & 0.0153 & 0.4190 \\
            \hline
            GB-DIST-MLP & 2.2k & 0.1 (0.01) & 5 & 5 & 1.4192 & 0.1253 & 1.2938 & 1.3593 & 0.1271 & 1.2322 \\
            GB-DIST-RNL & 47 & 0.1 (1) & 5 & 5 & 1.3941 & 0.1280 & 1.2661 & 1.4017 & 0.1324 & 1.2692 \\
            \hline
            GB-FUZZ-MLP & 2.3k & 0.1 (0.01) & 10 & 1 & 1.3185 & 0.0688 & 1.2498 & 1.1861 & 0.0620 & 1.1241 \\
            GB-FUZZ-RNL & 62 & 0.1 (1) & 10 & 1 & 1.1998 & 0.1003 & 1.0995 & 1.0867 & 0.0999 & 0.9868 \\
            \hline
            \hline
        \end{tabular}
    }
\end{table*}

% 0.005	1	0.1	0.7049	0.0278	0.6771	0.5504	0.0205	0.5299
% 0.005	10	1	0.5572	0.0213	0.5359	0.4541	0.0146	0.4395
% 0.005	10	1	1.1265	0.0655	1.0609	0.9885	0.0523	0.9362
% 0.005	1	0.1	1.1229	0.0657	1.0572	0.9854	0.0535	0.9319
% 0.005	10	1	1.0447	0.0533	0.9914	0.9032	0.0408	0.8624
% 0.005	10	1	1.0360	0.0526	0.9835	0.8851	0.0406	0.8445
% 0.005	10	1	1.0334	0.0534	0.9801	0.8728	0.0404	0.8325
% 0.005	10	1	1.0179	0.0507	0.9672	0.8819	0.0420	0.8399
% 0.005	10	1	0.7597	0.0309	0.7289	0.5797	0.0191	0.5605
% 0.005	1	0.1	0.7431	0.0320	0.7111	0.5136	0.0158	0.4978
% 0.005	10	1	0.7633	0.0313	0.7320	0.5413	0.0173	0.5241
% 0.005	10	1	0.7611	0.0315	0.7296	0.5265	0.0160	0.5106
% 0.005	1	0.1	0.7472	0.0313	0.7159	0.5513	0.0192	0.5321
% 0.005	1	0.1	0.6930	0.0282	0.6647	0.5073	0.0148	0.4925
% 0.005	5	5	1.1000	0.0700	1.0300	0.9622	0.0585	0.9037
% 0.005	5	5	1.0838	0.0683	1.0156	0.9417	0.0556	0.8861
% 0.005	5	5	1.0057	0.0574	0.9483	0.8493	0.0457	0.8036
% 0.005	10	1	1.0129	0.0505	0.9623	0.8844	0.0401	0.8443
% 0.005	10	1	0.9751	0.0481	0.9270	0.7325	0.0324	0.7001
% 0.005	1	0.1	0.8555	0.0393	0.8162	0.7103	0.0291	0.6812
% 0.005	5	5	0.7316	0.0335	0.6981	0.5409	0.0228	0.5181
% 0.005	5	5	0.6574	0.0340	0.6234	0.4949	0.0220	0.4729
% 0.005	0.5	0.5	0.7418	0.0365	0.7053	0.5082	0.0240	0.4842
% 0.005	10	1	0.6665	0.0271	0.6393	0.4778	0.0146	0.4633
% 0.005	5	5	0.7319	0.0367	0.6952	0.5216	0.0246	0.4970
% 0.005	5	5	0.6975	0.0350	0.6624	0.4868	0.0214	0.4654
% 0.01	10	1	0.8620	0.0426	0.8194	0.6468	0.0295	0.6173
% 0.01	0.5	0.5	0.8022	0.0398	0.7624	0.5737	0.0241	0.5496
% 0.01	0.5	0.5	0.6269	0.0272	0.5997	0.4544	0.0173	0.4371
% 0.01	0.5	0.5	0.6151	0.0267	0.5884	0.4344	0.0153	0.4190
% 0.1 (0.01)	5	5	1.4192	0.1253	1.2938	1.3593	0.1271	1.2322
% 0.1 (1)	5	5	1.3941	0.1280	1.2661	1.4017	0.1324	1.2692
% 0.1 (0.01)	10	1	1.3185	0.0688	1.2498	1.1861	0.0620	1.1241
% 0.1 (1)	10	1	1.1998	0.1003	1.0995	1.0867	0.0999	0.9868
\setlength{\tabcolsep}{4pt}
\renewcommand{\arraystretch}{1.3}
\begin{table*}[h]
    \small
    \caption{
    \textit{Scaled validation and test loss for non parametric models of \textbf{Harley Benton Fuzzy Logic} fuzz.}
    \textit{Bold indicates best performing models.}
    \textit{Learning rate multiplier for nonlinearity in gray-box models shown in brackets.}
    }
    \label{tab:val-and-test-loss_od_fulltone-fulldrive}
    \centerline{
        \begin{tabular}{l c c cc >{\columncolor{gray!20}}ccc >{\columncolor{gray!20}}ccc}
            \hline
            \midrule
            
            \multirow{2}{*}{Model}
                & \multirow{2}{*}{Params.}
                    & \multirow{2}{*}{LR}
                        & \multicolumn{2}{c}{Weights}
                            & \multicolumn{3}{c}{Val. Loss}
                                & \multicolumn{3}{c}{Test Loss} \\ 
            
            \cmidrule(lr){4-5} 
                \cmidrule(lr){6-8} 
                    \cmidrule(lr){9-11}
            
            &   &   & {\scriptsize L1} & {\scriptsize MR-STFT} & Tot. & {\scriptsize L1} & {\scriptsize MR-STFT} & Tot. & {\scriptsize L1} & {\scriptsize MR-STFT} \\ 
            
            \hline
            LSTM-32 & 4.5k & 0.001 & 1 & 0.1 & 2.1293 & 0.1288 & 2.0005 & 2.2845 & 0.1179 & 2.1666 \\
            LSTM-96 & 38.1k & 0.001 & 10 & 1 & 2.3521 & 0.1234 & 2.2287 & 2.3062 & 0.1168 & 2.1894 \\
            \hline
            TCN-45-S-16 & 7.5k & 0.005 & 0.5 & 0.5 & 1.0224 & 0.0292 & 0.9931 & 1.3093 & 0.0682 & 1.2411 \\
            TCN-45-L-16 & 7.3k & 0.005 & 5 & 5 & 0.8131 & 0.0259 & 0.7872 & 0.8215 & 0.0234 & 0.7981 \\
            TCN-250-S-16 & 14.5k & 0.005 & 1 & 0.1 & 0.9138 & 0.0215 & 0.8923 & 0.9163 & 0.0245 & 0.8918 \\
            TCN-250-L-16 & 18.4k & 0.005 & 1 & 0.1 & 0.7011 & 0.0135 & 0.6876 & 0.6819 & 0.0127 & 0.6692 \\
            TCN-2500-S-16 & 13.7k & 0.005 & 1 & 0.1 & 0.8944 & 0.0244 & 0.8700 & 1.3715 & 0.0723 & 1.2992 \\
            TCN-2500-L-16 & 11.9k & 0.005 & 1 & 0.1 & 0.6806 & 0.0154 & 0.6652 & 0.6843 & 0.0150 & 0.6694 \\
            \hline
            TCN-TF-45-S-16 & 39.5k & 0.005 & 1 & 0.1 & 0.6230 & 0.0110 & 0.6120 & 0.7721 & 0.0174 & 0.7547 \\
            TCN-TF-45-L-16 & 71.3k & 0.005 & 1 & 0.1 & 0.5809 & 0.0099 & 0.5709 & 0.5586 & 0.0090 & 0.5496 \\
            TCN-TF-250-S-16 & 52.9k & 0.005 & 1 & 0.1 & 0.5822 & 0.0113 & 0.5709 & 0.6030 & 0.0112 & 0.5918 \\
            TCN-TF-250-L-16 & 88.8k & 0.005 & 10 & 1 & 0.5735 & 0.0122 & 0.5614 & 0.6077 & 0.0117 & 0.5960 \\
            TCN-TF-2500-S-16 & 45.7k & 0.005 & 1 & 0.1 & 0.6850 & 0.0160 & 0.6690 & 0.6829 & 0.0156 & 0.6673 \\
            TCN-TF-2500-L-16 & 75.9k & 0.005 & 1 & 0.1 & 0.8690 & 0.0258 & 0.8432 & 0.8263 & 0.0212 & 0.8052 \\
            \hline
            GCN-45-S-16 & 16.2k & 0.005 & 10 & 1 & 0.8763 & 0.0165 & 0.8598 & 0.8225 & 0.0174 & 0.8050 \\
            GCN-45-L-16 & 17.1k & 0.005 & 5 & 5 & 0.9200 & 0.0240 & 0.8960 & 0.7138 & 0.0209 & 0.6930 \\
            GCN-250-S-16 & 30.4k & 0.005 & 10 & 1 & 0.7352 & 0.0133 & 0.7219 & 0.7027 & 0.0137 & 0.6890 \\
            GCN-250-L-16 & 39.6k & 0.005 & 1 & 0.1 & 0.6389 & 0.0114 & 0.6274 & 0.6534 & 0.0113 & 0.6421 \\
            GCN-2500-S-16 & 28.6k & 0.005 & 1 & 0.1 & 0.7191 & 0.0174 & 0.7017 & 0.6822 & 0.0157 & 0.6665 \\
            GCN-2500-L-16 & 26.4k & 0.005 & 10 & 1 & 0.7147 & 0.0159 & 0.6988 & 0.7378 & 0.0165 & 0.7212 \\
            \hline
            GCN-TF-45-S-16 & 141.6k & 0.005 & 1 & 0.1 & 0.6017 & 0.0099 & 0.5918 & 0.5609 & 0.0089 & 0.5519 \\
            GCN-TF-45-L-16 & 268.0k & 0.005 & 1 & 0.1 & 0.5245 & 0.0086 & 0.5159 & 0.4624 & 0.0066 & 0.4558 \\
            GCN-TF-250-S-16 & 181.0k & 0.005 & 5 & 5 & 0.6902 & 0.0267 & 0.6636 & 0.6930 & 0.0231 & 0.6699 \\
            GCN-TF-250-L-16 & 315.6k & 0.005 & 10 & 1 & 0.7624 & 0.0165 & 0.7460 & 0.8390 & 0.0178 & 0.8212 \\
            GCN-TF-2500-S-16 & 154.1k & 0.005 & 5 & 5 & 0.7063 & 0.0159 & 0.6904 & 0.6920 & 0.0211 & 0.6709 \\
            GCN-TF-2500-L-16 & 277.3k & 0.005 & 5 & 5 & 0.7925 & 0.0205 & 0.7721 & 0.5529 & 0.0137 & 0.5393 \\
            \hline
            S4-S-16 & 2.4k & 0.01 & 0.5 & 0.5 & 0.5991 & 0.0106 & 0.5885 & 0.5937 & 0.0098 & 0.5839 \\
            S4-L-16 & 19.0k & 0.01 & 10 & 1 & \textbf{0.4702} & 0.0055 & 0.4647 & \textbf{0.4336} & 0.0048 & 0.4288 \\
            \hline
            S4-TF-S-16 & 28.0k & 0.01 & 1 & 0.1 & 0.5541 & 0.0082 & 0.5459 & 0.5507 & 0.0079 & 0.5429 \\
            S4-TF-L-16 & 70.2k & 0.01 & 1 & 0.1 & 0.4988 & 0.0064 & 0.4925 & 0.6048 & 0.0120 & 0.5928 \\
            \hline
            GB-DIST-MLP & 2.2k & 0.1 (0.01) & 0.5 & 0.5 & 1.9248 & 0.1000 & 1.8248 & 1.8472 & 0.1119 & 1.7353 \\
            GB-DIST-RNL & 47 & 0.1 (1) & 10 & 1 & 1.8934 & 0.0971 & 1.7963 & 1.9171 & 0.1083 & 1.8088 \\
            \hline
            GB-FUZZ-MLP & 2.3k & 0.1 (0.01) & 5 & 5 & 1.4539 & 0.0574 & 1.3965 & 1.4306 & 0.0531 & 1.3776 \\
            GB-FUZZ-RNL & 62 & 0.1 (1) & 0.5 & 0.5 & 1.6511 & 0.0557 & 1.5954 & 1.5059 & 0.0563 & 1.4496 \\
            \hline
            \hline
        \end{tabular}
    }
\end{table*}

% 0.001	1	0.1	2.1293	0.1288	2.0005	2.2845	0.1179	2.1666
% 0.001	10	1	2.3521	0.1234	2.2287	2.3062	0.1168	2.1894
% 0.005	0.5	0.5	1.0224	0.0292	0.9931	1.3093	0.0682	1.2411
% 0.005	5	5	0.8131	0.0259	0.7872	0.8215	0.0234	0.7981
% 0.005	1	0.1	0.9138	0.0215	0.8923	0.9163	0.0245	0.8918
% 0.005	1	0.1	0.7011	0.0135	0.6876	0.6819	0.0127	0.6692
% 0.005	1	0.1	0.8944	0.0244	0.8700	1.3715	0.0723	1.2992
% 0.005	1	0.1	0.6806	0.0154	0.6652	0.6843	0.0150	0.6694
% 0.005	1	0.1	0.6230	0.0110	0.6120	0.7721	0.0174	0.7547
% 0.005	1	0.1	0.5809	0.0099	0.5709	0.5586	0.0090	0.5496
% 0.005	1	0.1	0.5822	0.0113	0.5709	0.6030	0.0112	0.5918
% 0.005	10	1	0.5735	0.0122	0.5614	0.6077	0.0117	0.5960
% 0.005	1	0.1	0.6850	0.0160	0.6690	0.6829	0.0156	0.6673
% 0.005	1	0.1	0.8690	0.0258	0.8432	0.8263	0.0212	0.8052
% 0.005	10	1	0.8763	0.0165	0.8598	0.8225	0.0174	0.8050
% 0.005	5	5	0.9200	0.0240	0.8960	0.7138	0.0209	0.6930
% 0.005	10	1	0.7352	0.0133	0.7219	0.7027	0.0137	0.6890
% 0.005	1	0.1	0.6389	0.0114	0.6274	0.6534	0.0113	0.6421
% 0.005	1	0.1	0.7191	0.0174	0.7017	0.6822	0.0157	0.6665
% 0.005	10	1	0.7147	0.0159	0.6988	0.7378	0.0165	0.7212
% 0.005	1	0.1	0.6017	0.0099	0.5918	0.5609	0.0089	0.5519
% 0.005	1	0.1	0.5245	0.0086	0.5159	0.4624	0.0066	0.4558
% 0.005	5	5	0.6902	0.0267	0.6636	0.6930	0.0231	0.6699
% 0.005	10	1	0.7624	0.0165	0.7460	0.8390	0.0178	0.8212
% 0.005	5	5	0.7063	0.0159	0.6904	0.6920	0.0211	0.6709
% 0.005	5	5	0.7925	0.0205	0.7721	0.5529	0.0137	0.5393
% 0.01	0.5	0.5	0.5991	0.0106	0.5885	0.5937	0.0098	0.5839
% 0.01	10	1	0.4702	0.0055	0.4647	0.4336	0.0048	0.4288
% 0.01	1	0.1	0.5541	0.0082	0.5459	0.5507	0.0079	0.5429
% 0.01	1	0.1	0.4988	0.0064	0.4925	0.6048	0.0120	0.5928
% 0.1	    0.5	0.5	1.9248	0.1000	1.8248	1.8472	0.1119	1.7353
% 0.1	    10	1	1.8934	0.0971	1.7963	1.9171	0.1083	1.8088
% 0.1	    5	5	1.4539	0.0574	1.3965	1.4306	0.0531	1.3776
% 0.1	    0.5	0.5	1.6511	0.0557	1.5954	1.5059	0.0563	1.4496
\setlength{\tabcolsep}{4pt}
\renewcommand{\arraystretch}{1.3}
\begin{table*}[h]
    \small
    \caption{
    \textit{Scaled validation and test loss for non parametric models of \textbf{Harley Benton Silly Fuzz} fuzz.}
    \textit{Bold indicates best performing models.}
    \textit{Learning rate multiplier for nonlinearity in gray-box models shown in brackets.}
    }
    \label{tab:val-and-test-loss_od_fulltone-fulldrive}
    \centerline{
        \begin{tabular}{l c c cc >{\columncolor{gray!20}}ccc >{\columncolor{gray!20}}ccc}
            \hline
            \midrule
            
            \multirow{2}{*}{Model}
                & \multirow{2}{*}{Params.}
                    & \multirow{2}{*}{LR}
                        & \multicolumn{2}{c}{Weights}
                            & \multicolumn{3}{c}{Val. Loss}
                                & \multicolumn{3}{c}{Test Loss} \\ 
            
            \cmidrule(lr){4-5} 
                \cmidrule(lr){6-8} 
                    \cmidrule(lr){9-11}
            
            &   &   & {\scriptsize L1} & {\scriptsize MR-STFT} & Tot. & {\scriptsize L1} & {\scriptsize MR-STFT} & Tot. & {\scriptsize L1} & {\scriptsize MR-STFT} \\ 
            
            \hline
            LSTM-32 & 4.5k & 0.001 & 10 & 1 & 1.9546 & 0.0420 & 1.9126 & 1.9206 & 0.0471 & 1.8735 \\
            LSTM-96 & 38.1k & 0.001 & 5 & 5 & 1.6304 & 0.0419 & 1.5885 & 1.6853 & 0.0422 & 1.6431 \\
            \hline
            TCN-45-S-16 & 7.5k & 0.005 & 0.5 & 0.5 & 0.9053 & 0.0124 & 0.8929 & 0.8025 & 0.0107 & 0.7918 \\
            TCN-45-L-16 & 7.3k & 0.005 & 5 & 5 & 0.7829 & 0.0160 & 0.7670 & 0.7405 & 0.0095 & 0.7310 \\
            TCN-250-S-16 & 14.5k & 0.005 & 10 & 1 & 0.7935 & 0.0094 & 0.7841 & 0.7379 & 0.0090 & 0.7289 \\
            TCN-250-L-16 & 18.4k & 0.005 & 1 & 0.1 & 0.7099 & 0.0068 & 0.7031 & 0.6666 & 0.0062 & 0.6604 \\
            TCN-2500-S-16 & 13.7k & 0.005 & 10 & 1 & 0.8727 & 0.0133 & 0.8594 & 0.8418 & 0.0124 & 0.8293 \\
            TCN-2500-L-16 & 11.9k & 0.005 & 1 & 0.1 & 0.7049 & 0.0074 & 0.6976 & 0.6682 & 0.0076 & 0.6606 \\
            \hline
            TCN-TF-45-S-16 & 39.5k & 0.005 & 10 & 1 & 0.7487 & 0.0100 & 0.7387 & 0.7451 & 0.0080 & 0.7371 \\
            TCN-TF-45-L-16 & 71.3k & 0.005 & 1 & 0.1 & 0.6664 & 0.0071 & 0.6593 & 0.6570 & 0.0057 & 0.6513 \\
            TCN-TF-250-S-16 & 52.9k & 0.005 & 1 & 0.1 & 0.7066 & 0.0109 & 0.6956 & 0.7018 & 0.0076 & 0.6942 \\
            TCN-TF-250-L-16 & 88.8k & 0.005 & 5 & 5 & 0.6873 & 0.0268 & 0.6605 & 0.6634 & 0.0170 & 0.6463 \\
            TCN-TF-2500-S-16 & 45.7k & 0.005 & 10 & 1 & 0.7812 & 0.0113 & 0.7699 & 1.0854 & 0.0152 & 1.0702 \\
            TCN-TF-2500-L-16 & 75.9k & 0.005 & 0.5 & 0.5 & 0.6473 & 0.0216 & 0.6258 & 0.6428 & 0.0139 & 0.6289 \\
            \hline
            GCN-45-S-16 & 16.2k & 0.005 & 10 & 1 & 0.8102 & 0.0063 & 0.8039 & 0.7125 & 0.0068 & 0.7057 \\
            GCN-45-L-16 & 17.1k & 0.005 & 10 & 1 & 0.8518 & 0.0055 & 0.8463 & 0.6452 & 0.0056 & 0.6396 \\
            GCN-250-S-16 & 30.4k & 0.005 & 10 & 1 & 0.7105 & 0.0051 & 0.7055 & 0.6327 & 0.0057 & 0.6270 \\
            GCN-250-L-16 & 39.6k & 0.005 & 1 & 0.1 & 0.6694 & 0.0068 & 0.6626 & 0.6523 & 0.0054 & 0.6469 \\
            GCN-2500-S-16 & 28.6k & 0.005 & 0.5 & 0.5 & 0.7229 & 0.0118 & 0.7111 & 0.6528 & 0.0087 & 0.6441 \\
            GCN-2500-L-16 & 26.4k & 0.005 & 1 & 0.1 & 0.7042 & 0.0078 & 0.6964 & 0.6804 & 0.0069 & 0.6735 \\
            \hline
            GCN-TF-45-S-16 & 141.6k & 0.005 & 10 & 1 & 0.7290 & 0.0094 & 0.7196 & 0.7960 & 0.0098 & 0.7861 \\
            GCN-TF-45-L-16 & 268.0k & 0.005 & 1 & 0.1 & 0.6273 & 0.0064 & 0.6209 & 0.6006 & 0.0051 & 0.5955 \\
            GCN-TF-250-S-16 & 181.0k & 0.005 & 10 & 1 & 0.7950 & 0.0100 & 0.7850 & 1.1097 & 0.0203 & 1.0894 \\
            GCN-TF-250-L-16 & 315.6k & 0.005 & 1 & 0.1 & 0.6954 & 0.0080 & 0.6874 & 0.6889 & 0.0067 & 0.6823 \\
            GCN-TF-2500-S-16 & 154.1k & 0.005 & 5 & 5 & 0.6180 & 0.0101 & 0.6079 & 0.6078 & 0.0073 & 0.6004 \\
            GCN-TF-2500-L-16 & 277.3k & 0.005 & 0.5 & 0.5 & 0.5797 & 0.0093 & 0.5704 & 0.5687 & 0.0065 & 0.5622 \\
            \hline
            S4-S-16 & 2.4k & 0.01 & 1 & 0.1 & 0.6905 & 0.0055 & 0.6849 & 0.6539 & 0.0053 & 0.6486 \\
            S4-L-16 & 19.0k & 0.01 & 10 & 1 & 0.5679 & 0.0033 & 0.5646 & 0.5362 & 0.0038 & 0.5324 \\
            \hline
            S4-TF-S-16 & 28.0k & 0.01 & 1 & 0.1 & 0.5886 & 0.0041 & 0.5845 & 0.5531 & 0.0041 & 0.5490 \\
            S4-TF-L-16 & 70.2k & 0.01 & 10 & 1 & \textbf{0.5272} & 0.0028 & 0.5244 & \textbf{0.4980} & 0.0033 & 0.4946 \\
            \hline
            GB-DIST-MLP & 2.2k & 0.1 & 5 & 5 & 1.8176 & 0.0991 & 1.7185 & 1.7205 & 0.1094 & 1.6111 \\
            GB-DIST-RNL & 47 & 0.1 & 5 & 5 & 1.8785 & 0.0442 & 1.8343 & 1.8562 & 0.0479 & 1.8083 \\
            \hline
            GB-FUZZ-MLP & 2.3k & 0.1 & 5 & 5 & 1.4506 & 0.0219 & 1.4287 & 1.4077 & 0.0209 & 1.3868 \\
            GB-FUZZ-RNL & 62 & 0.1 & 1 & 0.1 & 1.6310 & 0.0207 & 1.6103 & 1.4646 & 0.0202 & 1.4444 \\
            \hline
            \hline
        \end{tabular}
    }
\end{table*}

% 0.001	10	1	1.9546	0.0420	1.9126	1.9206	0.0471	1.8735
% 0.001	5	5	1.6304	0.0419	1.5885	1.6853	0.0422	1.6431
% 0.005	0.5	0.5	0.9053	0.0124	0.8929	0.8025	0.0107	0.7918
% 0.005	5	5	0.7829	0.0160	0.7670	0.7405	0.0095	0.7310
% 0.005	10	1	0.7935	0.0094	0.7841	0.7379	0.0090	0.7289
% 0.005	1	0.1	0.7099	0.0068	0.7031	0.6666	0.0062	0.6604
% 0.005	10	1	0.8727	0.0133	0.8594	0.8418	0.0124	0.8293
% 0.005	1	0.1	0.7049	0.0074	0.6976	0.6682	0.0076	0.6606
% 0.005	10	1	0.7487	0.0100	0.7387	0.7451	0.0080	0.7371
% 0.005	1	0.1	0.6664	0.0071	0.6593	0.6570	0.0057	0.6513
% 0.005	1	0.1	0.7066	0.0109	0.6956	0.7018	0.0076	0.6942
% 0.005	5	5	0.6873	0.0268	0.6605	0.6634	0.0170	0.6463
% 0.005	10	1	0.7812	0.0113	0.7699	1.0854	0.0152	1.0702
% 0.005	0.5	0.5	0.6473	0.0216	0.6258	0.6428	0.0139	0.6289
% 0.005	10	1	0.8102	0.0063	0.8039	0.7125	0.0068	0.7057
% 0.005	10	1	0.8518	0.0055	0.8463	0.6452	0.0056	0.6396
% 0.005	10	1	0.7105	0.0051	0.7055	0.6327	0.0057	0.6270
% 0.005	1	0.1	0.6694	0.0068	0.6626	0.6523	0.0054	0.6469
% 0.005	0.5	0.5	0.7229	0.0118	0.7111	0.6528	0.0087	0.6441
% 0.005	1	0.1	0.7042	0.0078	0.6964	0.6804	0.0069	0.6735
% 0.005	10	1	0.7290	0.0094	0.7196	0.7960	0.0098	0.7861
% 0.005	1	0.1	0.6273	0.0064	0.6209	0.6006	0.0051	0.5955
% 0.005	10	1	0.7950	0.0100	0.7850	1.1097	0.0203	1.0894
% 0.005	1	0.1	0.6954	0.0080	0.6874	0.6889	0.0067	0.6823
% 0.005	5	5	0.6180	0.0101	0.6079	0.6078	0.0073	0.6004
% 0.005	0.5	0.5	0.5797	0.0093	0.5704	0.5687	0.0065	0.5622
% 0.01	1	0.1	0.6905	0.0055	0.6849	0.6539	0.0053	0.6486
% 0.01	10	1	0.5679	0.0033	0.5646	0.5362	0.0038	0.5324
% 0.01	1	0.1	0.5886	0.0041	0.5845	0.5531	0.0041	0.5490
% 0.01	10	1	0.5272	0.0028	0.5244	0.4980	0.0033	0.4946
% 0.1 (0.01)	5	5	1.8176	0.0991	1.7185	1.7205	0.1094	1.6111
% 0.1 (1)	5	5	1.8785	0.0442	1.8343	1.8562	0.0479	1.8083
% 0.1 (0.01)	5	5	1.4506	0.0219	1.4287	1.4077	0.0209	1.3868
% 0.1 (1)	1	0.1	1.6310	0.0207	1.6103	1.4646	0.0202	1.4444
\setlength{\tabcolsep}{4pt}
\renewcommand{\arraystretch}{1.3}
\begin{table*}[h]
    \small
    \caption{
    \textit{Scaled validation and test loss for non parametric models of \textbf{Arturia Rev Spring 636 Preamp} fuzz.}
    \textit{Bold indicates best performing models.}
    \textit{Learning rate multiplier for nonlinearity in gray-box models shown in brackets.}
    }
    \label{tab:val-and-test-loss_od_fulltone-fulldrive}
    \centerline{
        \begin{tabular}{l c c cc >{\columncolor{gray!20}}ccc >{\columncolor{gray!20}}ccc}
            \hline
            \midrule
            
            \multirow{2}{*}{Model}
                & \multirow{2}{*}{Params.}
                    & \multirow{2}{*}{LR}
                        & \multicolumn{2}{c}{Weights}
                            & \multicolumn{3}{c}{Val. Loss}
                                & \multicolumn{3}{c}{Test Loss} \\ 
            
            \cmidrule(lr){4-5} 
                \cmidrule(lr){6-8} 
                    \cmidrule(lr){9-11}
            
            &   &   & {\scriptsize L1} & {\scriptsize MR-STFT} & Tot. & {\scriptsize L1} & {\scriptsize MR-STFT} & Tot. & {\scriptsize L1} & {\scriptsize MR-STFT} \\ 
            
            \hline
            LSTM-32 & 4.5k & 0.005 & 1 & 0.1 & 0.3070 & 0.0110 & 0.2960 & 0.2687 & 0.0091 & 0.2595 \\
            LSTM-96 & 38.1k & 0.001 & 5 & 5 & 0.2131 & 0.0150 & 0.1981 & 0.1760 & 0.0123 & 0.1637 \\
            \hline
            TCN-45-S-16 & 7.5k & 0.005 & 5 & 5 & 0.5907 & 0.0282 & 0.5625 & 0.5928 & 0.0292 & 0.5635 \\
            TCN-45-L-16 & 7.3k & 0.005 & 10 & 1 & 0.5918 & 0.0225 & 0.5693 & 0.5882 & 0.0228 & 0.5654 \\
            TCN-250-S-16 & 14.5k & 0.005 & 10 & 1 & 0.4128 & 0.0115 & 0.4014 & 0.3764 & 0.0109 & 0.3655 \\
            TCN-250-L-16 & 18.4k & 0.005 & 1 & 0.1 & 0.4516 & 0.0121 & 0.4395 & 0.4225 & 0.0114 & 0.4112 \\
            TCN-2500-S-16 & 13.7k & 0.005 & 0.5 & 0.5 & 0.4729 & 0.0236 & 0.4493 & 0.4032 & 0.0182 & 0.3850 \\
            TCN-2500-L-16 & 11.9k & 0.005 & 0.5 & 0.5 & 0.4499 & 0.0247 & 0.4252 & 0.3497 & 0.0192 & 0.3305 \\
            \hline
            TCN-TF-45-S-16 & 39.5k & 0.005 & 1 & 0.1 & 0.3642 & 0.0072 & 0.3570 & 0.3211 & 0.0059 & 0.3152 \\
            TCN-TF-45-L-16 & 71.3k & 0.005 & 1 & 0.1 & 0.3308 & 0.0085 & 0.3223 & 0.2669 & 0.0069 & 0.2600 \\
            TCN-TF-250-S-16 & 52.9k & 0.005 & 10 & 1 & 0.3484 & 0.0066 & 0.3419 & 0.3278 & 0.0064 & 0.3215 \\
            TCN-TF-250-L-16 & 88.8k & 0.005 & 10 & 1 & 0.2823 & 0.0079 & 0.2744 & 0.2505 & 0.0066 & 0.2439 \\
            TCN-TF-2500-S-16 & 45.7k & 0.005 & 0.5 & 0.5 & 0.3790 & 0.0178 & 0.3612 & 0.3203 & 0.0145 & 0.3058 \\
            TCN-TF-2500-L-16 & 75.9k & 0.005 & 5 & 5 & 0.2875 & 0.0261 & 0.2614 & 0.2588 & 0.0217 & 0.2371 \\
            \hline
            GCN-45-S-16 & 16.2k & 0.005 & 10 & 1 & 0.6270 & 0.0237 & 0.6034 & 0.5803 & 0.0227 & 0.5576 \\
            GCN-45-L-16 & 17.1k & 0.005 & 5 & 5 & 0.6836 & 0.0463 & 0.6373 & 0.5943 & 0.0384 & 0.5558 \\
            GCN-250-S-16 & 30.4k & 0.005 & 10 & 1 & 0.4430 & 0.0136 & 0.4294 & 0.3552 & 0.0098 & 0.3453 \\
            GCN-250-L-16 & 39.6k & 0.005 & 1 & 0.1 & 0.3420 & 0.0089 & 0.3330 & 0.3381 & 0.0089 & 0.3292 \\
            GCN-2500-S-16 & 28.6k & 0.005 & 5 & 5 & 0.3454 & 0.0171 & 0.3283 & 0.3267 & 0.0147 & 0.3120 \\
            GCN-2500-L-16 & 26.4k & 0.005 & 10 & 1 & 0.3495 & 0.0115 & 0.3380 & 0.3163 & 0.0100 & 0.3063 \\
            \hline
            GCN-TF-45-S-16 & 141.6k & 0.005 & 5 & 5 & 0.2924 & 0.0056 & 0.2867 & 0.2566 & 0.0047 & 0.2519 \\
            GCN-TF-45-L-16 & 268.0k & 0.005 & 5 & 5 & 0.2716 & 0.0222 & 0.2495 & 0.2229 & 0.0164 & 0.2066 \\
            GCN-TF-250-S-16 & 181.0k & 0.005 & 10 & 1 & 0.2566 & 0.0125 & 0.2441 & 0.2183 & 0.0109 & 0.2074 \\
            GCN-TF-250-L-16 & 315.6k & 0.005 & 5 & 5 & 0.2462 & 0.0079 & 0.2383 & 0.2196 & 0.0073 & 0.2123 \\
            GCN-TF-2500-S-16 & 154.1k & 0.005 & 0.5 & 0.5 & 0.3237 & 0.0161 & 0.3076 & 0.2624 & 0.0123 & 0.2501 \\
            GCN-TF-2500-L-16 & 277.3k & 0.005 & 0.5 & 0.5 & 0.2297 & 0.0148 & 0.2149 & 0.2103 & 0.0123 & 0.1980 \\
            \hline
            S4-S-16 & 2.4k & 0.01 & 5 & 5 & 0.2495 & 0.0086 & 0.2409 & 0.2290 & 0.0085 & 0.2205 \\
            S4-L-16 & 19.0k & 0.01 & 5 & 5 & 0.1861 & 0.0130 & 0.1731 & \textbf{0.1431} & 0.0090 & 0.1340 \\
            \hline
            S4-TF-S-16 & 28.0k & 0.01 & 5 & 5 & 0.2258 & 0.0082 & 0.2176 & 0.1719 & 0.0069 & 0.1650 \\
            S4-TF-L-16 & 70.2k & 0.01 & 10 & 1 & \textbf{0.1660} & 0.0044 & 0.1616 & 0.1469 & 0.0038 & 0.1431 \\
            \hline
            GB-DIST-MLP & 2.2k & 0.1 & 10 & 1 & 0.7976 & 0.0373 & 0.7603 & 0.8404 & 0.0408 & 0.7996 \\
            GB-DIST-RNL & 47 & 0.1 & 0.5 & 0.5 & 0.8532 & 0.0458 & 0.8074 & 0.8593 & 0.0461 & 0.8132 \\
            \hline
            GB-FUZZ-MLP & 2.3k & 0.1 & 0.5 & 0.5 & 0.8098 & 0.0360 & 0.7738 & 0.7604 & 0.0355 & 0.7249 \\
            GB-FUZZ-RNL & 62 & 0.1 & 0.5 & 0.5 & 0.8857 & 0.0467 & 0.8390 & 0.8582 & 0.0452 & 0.8130 \\
            \hline
            \hline
        \end{tabular}
    }
\end{table*}

% 0.005	1	0.1	0.3070	0.0110	0.2960	0.2687	0.0091	0.2595
% 0.001	5	5	0.2131	0.0150	0.1981	0.1760	0.0123	0.1637
% 0.005	5	5	0.5907	0.0282	0.5625	0.5928	0.0292	0.5635
% 0.005	10	1	0.5918	0.0225	0.5693	0.5882	0.0228	0.5654
% 0.005	10	1	0.4128	0.0115	0.4014	0.3764	0.0109	0.3655
% 0.005	1	0.1	0.4516	0.0121	0.4395	0.4225	0.0114	0.4112
% 0.005	0.5	0.5	0.4729	0.0236	0.4493	0.4032	0.0182	0.3850
% 0.005	0.5	0.5	0.4499	0.0247	0.4252	0.3497	0.0192	0.3305
% 0.005	1	0.1	0.3642	0.0072	0.3570	0.3211	0.0059	0.3152
% 0.005	1	0.1	0.3308	0.0085	0.3223	0.2669	0.0069	0.2600
% 0.005	10	1	0.3484	0.0066	0.3419	0.3278	0.0064	0.3215
% 0.005	10	1	0.2823	0.0079	0.2744	0.2505	0.0066	0.2439
% 0.005	0.5	0.5	0.3790	0.0178	0.3612	0.3203	0.0145	0.3058
% 0.005	5	5	0.2875	0.0261	0.2614	0.2588	0.0217	0.2371
% 0.005	10	1	0.6270	0.0237	0.6034	0.5803	0.0227	0.5576
% 0.005	5	5	0.6836	0.0463	0.6373	0.5943	0.0384	0.5558
% 0.005	10	1	0.4430	0.0136	0.4294	0.3552	0.0098	0.3453
% 0.005	1	0.1	0.3420	0.0089	0.3330	0.3381	0.0089	0.3292
% 0.005	5	5	0.3454	0.0171	0.3283	0.3267	0.0147	0.3120
% 0.005	10	1	0.3495	0.0115	0.3380	0.3163	0.0100	0.3063
% 0.005	10	1	0.2924	0.0056	0.2867	0.2566	0.0047	0.2519
% 0.005	5	5	0.2716	0.0222	0.2495	0.2229	0.0164	0.2066
% 0.005	5	5	0.2566	0.0125	0.2441	0.2183	0.0109	0.2074
% 0.005	10	1	0.2462	0.0079	0.2383	0.2196	0.0073	0.2123
% 0.005	5	5	0.3237	0.0161	0.3076	0.2624	0.0123	0.2501
% 0.005	0.5	0.5	0.2297	0.0148	0.2149	0.2103	0.0123	0.1980
% 0.01	5	5	0.2495	0.0086	0.2409	0.2290	0.0085	0.2205
% 0.01	5	5	0.1861	0.0130	0.1731	0.1431	0.0090	0.1340
% 0.01	5	5	0.2258	0.0082	0.2176	0.1719	0.0069	0.1650
% 0.01	10	1	0.1660	0.0044	0.1616	0.1469	0.0038	0.1431
% 0.1 (0.01)	10	1	0.7976	0.0373	0.7603	0.8404	0.0408	0.7996
% 0.1 (1)	0.5	0.5	0.8532	0.0458	0.8074	0.8593	0.0461	0.8132
% 0.1 (0.01)	0.5	0.5	0.8098	0.0360	0.7738	0.7604	0.0355	0.7249
% 0.1 (1)	0.5	0.5	0.8857	0.0467	0.8390	0.8582	0.0452	0.8130

\setlength{\tabcolsep}{4pt}
\renewcommand{\arraystretch}{1.3}
\begin{table*}[h]
    \small
    \caption{
    \textit{Objective metrics for non parametric models of \textbf{Custom Dynamic Fuzz} fuzz.}
    \textit{Bold indicates best performing models.}
    \textit{Learning rate multiplier for nonlinearity in gray-box models shown in brackets.}
    }
    \label{tab:metrics_od_diy-klon-centaur}
    \centerline{
        \begin{tabular}{lccccccccccccc}
            \hline
            \midrule
            
            \multirow{2}{*}{Model}
                & \multirow{2}{*}{Params.}
                    & \multirow{2}{*}{LR}
                        & \multicolumn{2}{c}{Weights}
                            & \multicolumn{3}{c}{}
                                & \multicolumn{3}{c}{FAD} \\ 
            \cmidrule(lr){4-5} 
                % \cmidrule(lr){6-8} 
                    \cmidrule(lr){9-12}
            
            &   &   & L1 & MR-STFT & MSE & ESR & MAPE & VGGish & PANN & CLAP & AFx-Rep \\ 
            \hline
            LSTM-32 & 4.5k & 0.005 & 1 & 0.1 & 3.56e-03 & 0.0315 & 0.8384 & 0.0931 & 3.04e-06 & 0.0086 & 0.0080 \\
            LSTM-96 & 38.1k & 0.005 & 10 & 1 & 2.05e-03 & 0.0191 & 0.7337 & 0.0894 & 1.09e-06 & 0.0058 & 0.0036 \\
            \hline
            TCN-45-S-16 & 7.5k & 0.005 & 10 & 1 & 1.01e-02 & 0.0974 & 5.1123 & 0.5420 & 2.94e-05 & 0.0556 & 0.0447 \\
            TCN-45-L-16 & 7.3k & 0.005 & 1 & 0.1 & 1.04e-02 & 0.1002 & 4.5405 & 0.5082 & 9.95e-06 & 0.0541 & 0.0417 \\
            TCN-250-S-16 & 14.5k & 0.005 & 10 & 1 & 8.39e-03 & 0.0784 & 2.3234 & 0.4771 & 7.08e-05 & 0.0403 & 0.0473 \\
            TCN-250-L-16 & 18.4k & 0.005 & 10 & 1 & 8.22e-03 & 0.0776 & 2.9425 & 0.2847 & 7.42e-05 & 0.0338 & 0.0312 \\
            TCN-2500-S-16 & 13.7k & 0.005 & 10 & 1 & 7.09e-03 & 0.0662 & 1.9213 & 1.0353 & 1.06e-04 & 0.0668 & 0.1189 \\
            TCN-2500-L-16 & 11.9k & 0.005 & 10 & 1 & 8.46e-03 & 0.0793 & 2.5310 & 0.3263 & 1.27e-04 & 0.0422 & 0.0336 \\
            \hline
            TCN-TF-45-S-16 & 39.5k & 0.005 & 10 & 1 & 2.67e-03 & 0.0256 & 0.7979 & 0.0863 & 2.61e-05 & 0.0380 & 0.0172 \\
            TCN-TF-45-L-16 & 71.3k & 0.005 & 1 & 0.1 & 2.01e-03 & 0.0189 & 0.6915 & 0.0792 & 1.89e-05 & 0.0129 & 0.0553 \\
            TCN-TF-250-S-16 & 52.9k & 0.005 & 10 & 1 & 2.20e-03 & 0.0202 & 0.8998 & 0.0935 & 1.41e-05 & 0.0190 & 0.0138 \\
            TCN-TF-250-L-16 & 88.8k & 0.005 & 10 & 1 & 2.21e-03 & 0.0198 & 0.6452 & 0.0828 & 1.12e-05 & 0.0112 & 0.0184 \\
            TCN-TF-2500-S-16 & 45.7k & 0.005 & 1 & 0.1 & 2.40e-03 & 0.0220 & 0.7754 & 0.1007 & 1.71e-05 & 0.0149 & 0.0141 \\
            TCN-TF-2500-L-16 & 75.9k & 0.005 & 1 & 0.1 & \textbf{1.98e-03} & \textbf{0.0175} & \textbf{0.4391} & 0.0697 & 2.79e-06 & 0.0120 & 0.0313 \\
            \hline
            GCN-45-S-16 & 16.2k & 0.005 & 5 & 5 & 1.04e-02 & 0.0985 & 5.0496 & 0.4863 & 8.73e-05 & 0.0342 & 0.0380 \\
            GCN-45-L-16 & 17.1k & 0.005 & 5 & 5 & 1.03e-02 & 0.0976 & 4.6198 & 0.3822 & 5.09e-05 & 0.0399 & 0.0286 \\
            GCN-250-S-16 & 30.4k & 0.005 & 5 & 5 & 8.25e-03 & 0.0761 & 3.2831 & 0.3989 & 1.56e-04 & 0.0351 & 0.0415 \\
            GCN-250-L-16 & 39.6k & 0.005 & 10 & 1 & 8.57e-03 & 0.0818 & 2.8326 & 0.4045 & 1.11e-04 & 0.0400 & 0.0336 \\
            GCN-2500-S-16 & 28.6k & 0.005 & 10 & 1 & 4.75e-03 & 0.0446 & 2.1462 & 0.6218 & 1.38e-04 & 0.0481 & 0.0454 \\
            GCN-2500-L-16 & 26.4k & 0.005 & 1 & 0.1 & 4.49e-03 & 0.0437 & 1.9491 & 0.4803 & 8.45e-05 & 0.0362 & 0.0445 \\
            \hline
            GCN-TF-45-S-16 & 141.6k & 0.005 & 5 & 5 & 2.66e-03 & 0.0238 & 1.2796 & 0.1062 & 1.23e-05 & 0.0275 & 0.0113 \\
            GCN-TF-45-L-16 & 268.0k & 0.005 & 5 & 5 & 3.42e-03 & 0.0317 & 1.0079 & 0.0703 & 1.56e-06 & 0.0079 & \textbf{0.0034} \\
            GCN-TF-250-S-16 & 181.0k & 0.005 & 0.5 & 0.5 & 2.45e-03 & 0.0227 & 1.5625 & 0.0928 & 2.51e-06 & 0.0233 & 0.0077 \\
            GCN-TF-250-L-16 & 315.6k & 0.005 & 10 & 1 & 2.17e-03 & 0.0193 & 0.5536 & 0.0714 & 1.27e-06 & 0.0065 & 0.0057 \\
            GCN-TF-2500-S-16 & 154.1k & 0.005 & 5 & 5 & 2.68e-03 & 0.0242 & 0.9486 & 0.0976 & 1.35e-06 & 0.0076 & 0.0103 \\
            GCN-TF-2500-L-16 & 277.3k & 0.005 & 5 & 5 & 2.66e-03 & 0.0239 & 0.9135 & 0.0641 & 1.39e-06 & 0.0066 & 0.0137 \\
            \hline
            S4-S-16 & 2.4k & 0.01 & 10 & 1 & 4.92e-03 & 0.0459 & 1.5261 & 0.1436 & 1.21e-05 & 0.0137 & 0.0258 \\
            S4-L-16 & 19.0k & 0.01 & 0.5 & 0.5 & 3.94e-03 & 0.0369 & 1.2131 & 0.1101 & 8.76e-06 & 0.0086 & 0.0054 \\
            \hline
            S4-TF-S-16 & 28.0k & 0.01 & 0.5 & 0.5 & 2.21e-03 & 0.0200 & 0.8430 & 0.0876 & 1.08e-06 & 0.0057 & 0.0056 \\
            S4-TF-L-16 & 70.2k & 0.01 & 0.5 & 0.5 & 1.99e-03 & 0.0178 & 0.6776 & \textbf{0.0636} & \textbf{2.76e-07} & \textbf{0.0050} & 0.0037 \\
            \hline
            GB-DIST-MLP & 2.2k & 0.1 & 5 & 5 & 2.94e-02 & 0.2591 & 6.7693 & 1.6367 & 4.71e-05 & 0.0904 & 0.3448 \\
            GB-DIST-RNL & 47 & 0.1 & 5 & 5 & 2.97e-02 & 0.2605 & 6.7492 & 2.1777 & 2.60e-05 & 0.0966 & 0.3214 \\
            \hline
            GB-FUZZ-MLP & 2.3k & 0.1 & 10 & 1 & 1.20e-02 & 0.1104 & 2.7967 & 1.5592 & 1.69e-05 & 0.0843 & 0.3571 \\
            GB-FUZZ-RNL & 62 & 0.1 & 10 & 1 & 1.64e-02 & 0.1772 & 13.0024 & 0.3961 & 1.57e-05 & 0.0309 & 0.3512 \\
            \hline
            \hline
        \end{tabular}
    }
\end{table*}

% 0.005	1	0.1	3.56e-03	0.0315	0.8384	0.0931	3.04e-06	0.0086	0.0080
% 0.005	10	1	2.05e-03	0.0191	0.7337	0.0894	1.09e-06	0.0058	0.0036
% 0.005	10	1	1.01e-02	0.0974	5.1123	0.5420	2.94e-05	0.0556	0.0447
% 0.005	1	0.1	1.04e-02	0.1002	4.5405	0.5082	9.95e-06	0.0541	0.0417
% 0.005	10	1	8.39e-03	0.0784	2.3234	0.4771	7.08e-05	0.0403	0.0473
% 0.005	10	1	8.22e-03	0.0776	2.9425	0.2847	7.42e-05	0.0338	0.0312
% 0.005	10	1	7.09e-03	0.0662	1.9213	1.0353	1.06e-04	0.0668	0.1189
% 0.005	10	1	8.46e-03	0.0793	2.5310	0.3263	1.27e-04	0.0422	0.0336
% 0.005	10	1	2.67e-03	0.0256	0.7979	0.0863	2.61e-05	0.0380	0.0172
% 0.005	1	0.1	2.01e-03	0.0189	0.6915	0.0792	1.89e-05	0.0129	0.0553
% 0.005	10	1	2.20e-03	0.0202	0.8998	0.0935	1.41e-05	0.0190	0.0138
% 0.005	10	1	2.21e-03	0.0198	0.6452	0.0828	1.12e-05	0.0112	0.0184
% 0.005	1	0.1	2.40e-03	0.0220	0.7754	0.1007	1.71e-05	0.0149	0.0141
% 0.005	1	0.1	1.98e-03	0.0175	0.4391	0.0697	2.79e-06	0.0120	0.0313
% 0.005	5	5	1.04e-02	0.0985	5.0496	0.4863	8.73e-05	0.0342	0.0380
% 0.005	5	5	1.03e-02	0.0976	4.6198	0.3822	5.09e-05	0.0399	0.0286
% 0.005	5	5	8.25e-03	0.0761	3.2831	0.3989	1.56e-04	0.0351	0.0415
% 0.005	10	1	8.57e-03	0.0818	2.8326	0.4045	1.11e-04	0.0400	0.0336
% 0.005	10	1	4.75e-03	0.0446	2.1462	0.6218	1.38e-04	0.0481	0.0454
% 0.005	1	0.1	4.49e-03	0.0437	1.9491	0.4803	8.45e-05	0.0362	0.0445
% 0.005	5	5	2.66e-03	0.0238	1.2796	0.1062	1.23e-05	0.0275	0.0113
% 0.005	5	5	3.42e-03	0.0317	1.0079	0.0703	1.56e-06	0.0079	0.0034
% 0.005	0.5	0.5	2.45e-03	0.0227	1.5625	0.0928	2.51e-06	0.0233	0.0077
% 0.005	10	1	2.17e-03	0.0193	0.5536	0.0714	1.27e-06	0.0065	0.0057
% 0.005	5	5	2.68e-03	0.0242	0.9486	0.0976	1.35e-06	0.0076	0.0103
% 0.005	5	5	2.66e-03	0.0239	0.9135	0.0641	1.39e-06	0.0066	0.0137
% 0.01	10	1	4.92e-03	0.0459	1.5261	0.1436	1.21e-05	0.0137	0.0258
% 0.01	0.5	0.5	3.94e-03	0.0369	1.2131	0.1101	8.76e-06	0.0086	0.0054
% 0.01	0.5	0.5	2.21e-03	0.0200	0.8430	0.0876	1.08e-06	0.0057	0.0056
% 0.01	0.5	0.5	1.99e-03	0.0178	0.6776	0.0636	2.76e-07	0.0050	0.0037
% 0.1 (0.01)	5	5	2.94e-02	0.2591	6.7693	1.6367	4.71e-05	0.0904	0.3448
% 0.1 (1)	5	5	2.97e-02	0.2605	6.7492	2.1777	2.60e-05	0.0966	0.3214
% 0.1 (0.01)	10	1	1.20e-02	0.1104	2.7967	1.5592	1.69e-05	0.0843	0.3571
% 0.1 (1)	10	1	1.64e-02	0.1772	13.0024	0.3961	1.57e-05	0.0309	0.3512
\setlength{\tabcolsep}{4pt}
\renewcommand{\arraystretch}{1.3}
\begin{table*}[h]
    \small
    \caption{
    \textit{Objective metrics for non parametric models of \textbf{Harley Benton Fuzzy Logic} fuzz.}
    \textit{Bold indicates best performing models.}
    \textit{Learning rate multiplier for nonlinearity in gray-box models shown in brackets.}
    }
    \label{tab:metrics_od_diy-klon-centaur}
    \centerline{
        \begin{tabular}{lccccccccccccc}
            \hline
            \midrule
            
            \multirow{2}{*}{Model}
                & \multirow{2}{*}{Params.}
                    & \multirow{2}{*}{LR}
                        & \multicolumn{2}{c}{Weights}
                            & \multicolumn{3}{c}{}
                                & \multicolumn{3}{c}{FAD} \\ 
            \cmidrule(lr){4-5} 
                % \cmidrule(lr){6-8} 
                    \cmidrule(lr){9-12}
            
            &   &   & L1 & MR-STFT & MSE & ESR & MAPE & VGGish & PANN & CLAP & AFx-Rep \\ 
            \hline
            LSTM-32 & 4.5k & 0.001 & 1 & 0.1 & 2.83e-02 & 0.9203 & 11.3948 & 6.1461 & 1.29e-03 & 0.5026 & 0.5827 \\
            LSTM-96 & 38.1k & 0.001 & 10 & 1 & 2.60e-02 & 0.8763 & 15.8152 & 6.7294 & 1.50e-04 & 0.3734 & 0.8195 \\
            \hline
            TCN-45-S-16 & 7.5k & 0.005 & 0.5 & 0.5 & 1.13e-02 & 0.3754 & 15.4948 & 0.8415 & 2.84e-04 & 0.1034 & 0.1431 \\
            TCN-45-L-16 & 7.3k & 0.005 & 5 & 5 & 2.10e-03 & 0.0753 & 7.2155 & 0.2363 & 2.49e-06 & 0.0353 & 0.0534 \\
            TCN-250-S-16 & 14.5k & 0.005 & 1 & 0.1 & 2.33e-03 & 0.0866 & 6.8113 & 0.2551 & 4.52e-06 & 0.0443 & 0.0716 \\
            TCN-250-L-16 & 18.4k & 0.005 & 1 & 0.1 & 7.82e-04 & 0.0289 & 1.7976 & 0.1547 & 1.13e-07 & 0.0285 & 0.0345 \\
            TCN-2500-S-16 & 13.7k & 0.005 & 1 & 0.1 & 9.33e-03 & 0.3630 & 20.1439 & 0.7083 & 4.74e-05 & 0.0609 & 0.2203 \\
            TCN-2500-L-16 & 11.9k & 0.005 & 1 & 0.1 & 9.44e-04 & 0.0345 & 2.2698 & 0.2444 & \textbf{4.23e-08} & 0.0186 & 0.0302 \\
            \hline
            TCN-TF-45-S-16 & 39.5k & 0.005 & 1 & 0.1 & 1.26e-03 & 0.0438 & 3.1692 & 0.7610 & 4.96e-05 & 0.0530 & 0.1156 \\
            TCN-TF-45-L-16 & 71.3k & 0.005 & 1 & 0.1 & 5.96e-04 & 0.0208 & 1.0964 & 0.1435 & 2.72e-06 & 0.0217 & 0.0638 \\
            TCN-TF-250-S-16 & 52.9k & 0.005 & 1 & 0.1 & 9.00e-04 & 0.0314 & 1.4806 & 0.2281 & 1.55e-06 & 0.0220 & 0.0913 \\
            TCN-TF-250-L-16 & 88.8k & 0.005 & 10 & 1 & 1.11e-03 & 0.0387 & 1.1759 & 0.2281 & 1.29e-06 & 0.0257 & 0.0269 \\
            TCN-TF-2500-S-16 & 45.7k & 0.005 & 1 & 0.1 & 9.66e-04 & 0.0330 & 2.3828 & 0.2138 & 1.61e-05 & 0.0195 & 0.0709 \\
            TCN-TF-2500-L-16 & 75.9k & 0.005 & 1 & 0.1 & 3.03e-03 & 0.1031 & 2.0202 & 0.5470 & 1.30e-04 & 0.0331 & 0.1535 \\
            \hline
            GCN-45-S-16 & 16.2k & 0.005 & 10 & 1 & 1.30e-03 & 0.0473 & 4.6385 & 0.1845 & 9.64e-07 & 0.0343 & 0.1529 \\
            GCN-45-L-16 & 17.1k & 0.005 & 5 & 5 & 2.83e-03 & 0.0975 & 3.4353 & 0.1250 & 3.13e-07 & 0.0315 & 0.0391 \\
            GCN-250-S-16 & 30.4k & 0.005 & 10 & 1 & 9.27e-04 & 0.0341 & 2.3576 & 0.1283 & 9.24e-07 & 0.0265 & 0.1072 \\
            GCN-250-L-16 & 39.6k & 0.005 & 1 & 0.1 & 6.41e-04 & 0.0236 & 2.2444 & 0.1208 & 3.04e-06 & 0.0219 & 0.0248 \\
            GCN-2500-S-16 & 28.6k & 0.005 & 1 & 0.1 & 1.01e-03 & 0.0369 & 2.4683 & 0.1470 & 1.61e-05 & 0.0184 & 0.0249 \\
            GCN-2500-L-16 & 26.4k & 0.005 & 10 & 1 & 1.42e-03 & 0.0510 & 3.5339 & 0.2557 & 3.10e-07 & 0.0488 & 0.0330 \\
            \hline
            GCN-TF-45-S-16 & 141.6k & 0.005 & 1 & 0.1 & 5.65e-04 & 0.0196 & 1.0334 & 0.1470 & 6.96e-07 & 0.0173 & 0.0929 \\
            GCN-TF-45-L-16 & 268.0k & 0.005 & 1 & 0.1 & 3.07e-04 & 0.0107 & \textbf{0.7361} & 0.1626 & 1.40e-06 & 0.0160 & 0.0106 \\
            GCN-TF-250-S-16 & 181.0k & 0.005 & 5 & 5 & 3.32e-03 & 0.1123 & 3.5458 & 0.2811 & 1.60e-05 & 0.0321 & 0.1033 \\
            GCN-TF-250-L-16 & 315.6k & 0.005 & 10 & 1 & 2.12e-03 & 0.0711 & 1.9047 & 0.4196 & 1.76e-04 & 0.0475 & 0.1372 \\
            GCN-TF-2500-S-16 & 154.1k & 0.005 & 5 & 5 & 2.83e-03 & 0.0957 & 3.0513 & 0.2763 & 1.65e-05 & 0.0318 & 0.1025 \\
            GCN-TF-2500-L-16 & 277.3k & 0.005 & 5 & 5 & 1.39e-03 & 0.0475 & 1.5987 & 0.1967 & 1.13e-06 & 0.0189 & 0.0224 \\
            \hline
            S4-S-16 & 2.4k & 0.01 & 0.5 & 0.5 & 7.96e-04 & 0.0293 & 1.6396 & 0.1253 & 8.78e-07 & 0.0120 & 0.0227 \\
            S4-L-16 & 19.0k & 0.01 & 10 & 1 & \textbf{2.00e-04} & \textbf{0.0073} & 0.8161 & \textbf{0.0697} & 1.89e-06 & \textbf{0.0114} & \textbf{0.0055} \\
            \hline
            S4-TF-S-16 & 28.0k & 0.01 & 1 & 0.1 & 4.41e-04 & 0.0156 & 1.1197 & 0.1996 & 7.60e-08 & 0.0183 & 0.0207 \\
            S4-TF-L-16 & 70.2k & 0.01 & 1 & 0.1 & 7.82e-04 & 0.0258 & 1.7880 & 0.1861 & 3.40e-07 & 0.0187 & 0.0112 \\
            \hline
            GB-DIST-MLP & 2.2k & 0.1 & 0.5 & 0.5 & 2.30e-02 & 0.7771 & 19.7928 & 2.3912 & 7.05e-04 & 0.1384 & 0.4256 \\
            GB-DIST-RNL & 47 & 0.1 & 10 & 1 & 2.04e-02 & 0.6886 & 21.6368 & 1.6225 & 1.51e-04 & 0.0967 & 0.4251 \\
            \hline
            GB-FUZZ-MLP & 2.3k & 0.1 & 5 & 5 & 9.74e-03 & 0.3289 & 7.8629 & 0.5924 & 3.52e-05 & 0.0546 & 0.1954 \\
            GB-FUZZ-RNL & 62 & 0.1 & 0.5 & 0.5 & 1.01e-02 & 0.3537 & 8.8434 & 0.8787 & 1.06e-04 & 0.0623 & 0.3390 \\
            \hline
            \hline
        \end{tabular}
    }
\end{table*}

% 0.001	1	0.1	2.83e-02	0.9203	11.3948	6.1461	1.29e-03	0.5026	0.5827
% 0.001	10	1	2.60e-02	0.8763	15.8152	6.7294	1.50e-04	0.3734	0.8195
% 0.005	0.5	0.5	1.13e-02	0.3754	15.4948	0.8415	2.84e-04	0.1034	0.1431
% 0.005	5	5	2.10e-03	0.0753	7.2155	0.2363	2.49e-06	0.0353	0.0534
% 0.005	1	0.1	2.33e-03	0.0866	6.8113	0.2551	4.52e-06	0.0443	0.0716
% 0.005	1	0.1	7.82e-04	0.0289	1.7976	0.1547	1.13e-07	0.0285	0.0345
% 0.005	1	0.1	9.33e-03	0.3630	20.1439	0.7083	4.74e-05	0.0609	0.2203
% 0.005	1	0.1	9.44e-04	0.0345	2.2698	0.2444	4.23e-08	0.0186	0.0302
% 0.005	1	0.1	1.26e-03	0.0438	3.1692	0.7610	4.96e-05	0.0530	0.1156
% 0.005	1	0.1	5.96e-04	0.0208	1.0964	0.1435	2.72e-06	0.0217	0.0638
% 0.005	1	0.1	9.00e-04	0.0314	1.4806	0.2281	1.55e-06	0.0220	0.0913
% 0.005	10	1	1.11e-03	0.0387	1.1759	0.2281	1.29e-06	0.0257	0.0269
% 0.005	1	0.1	9.66e-04	0.0330	2.3828	0.2138	1.61e-05	0.0195	0.0709
% 0.005	1	0.1	3.03e-03	0.1031	2.0202	0.5470	1.30e-04	0.0331	0.1535
% 0.005	10	1	1.30e-03	0.0473	4.6385	0.1845	9.64e-07	0.0343	0.1529
% 0.005	5	5	2.83e-03	0.0975	3.4353	0.1250	3.13e-07	0.0315	0.0391
% 0.005	10	1	9.27e-04	0.0341	2.3576	0.1283	9.24e-07	0.0265	0.1072
% 0.005	1	0.1	6.41e-04	0.0236	2.2444	0.1208	3.04e-06	0.0219	0.0248
% 0.005	1	0.1	1.01e-03	0.0369	2.4683	0.1470	1.61e-05	0.0184	0.0249
% 0.005	10	1	1.42e-03	0.0510	3.5339	0.2557	3.10e-07	0.0488	0.0330
% 0.005	1	0.1	5.65e-04	0.0196	1.0334	0.1470	6.96e-07	0.0173	0.0929
% 0.005	1	0.1	3.07e-04	0.0107	0.7361	0.1626	1.40e-06	0.0160	0.0106
% 0.005	5	5	3.32e-03	0.1123	3.5458	0.2811	1.60e-05	0.0321	0.1033
% 0.005	10	1	2.12e-03	0.0711	1.9047	0.4196	1.76e-04	0.0475	0.1372
% 0.005	5	5	2.83e-03	0.0957	3.0513	0.2763	1.65e-05	0.0318	0.1025
% 0.005	5	5	1.39e-03	0.0475	1.5987	0.1967	1.13e-06	0.0189	0.0224
% 0.01	0.5	0.5	7.96e-04	0.0293	1.6396	0.1253	8.78e-07	0.0120	0.0227
% 0.01	10	1	2.00e-04	0.0073	0.8161	0.0697	1.89e-06	0.0114	0.0055
% 0.01	1	0.1	4.41e-04	0.0156	1.1197	0.1996	7.60e-08	0.0183	0.0207
% 0.01	1	0.1	7.82e-04	0.0258	1.7880	0.1861	3.40e-07	0.0187	0.0112
% 0.1	0.5	0.5	2.30e-02	0.7771	19.7928	2.3912	7.05e-04	0.1384	0.4256
% 0.1	10	1	2.04e-02	0.6886	21.6368	1.6225	1.51e-04	0.0967	0.4251
% 0.1	5	5	9.74e-03	0.3289	7.8629	0.5924	3.52e-05	0.0546	0.1954
% 0.1	0.5	0.5	1.01e-02	0.3537	8.8434	0.8787	1.06e-04	0.0623	0.3390
\setlength{\tabcolsep}{4pt}
\renewcommand{\arraystretch}{1.3}
\begin{table*}[h]
    \small
    \caption{
    \textit{Objective metrics for non parametric models of \textbf{Harley Benton Silly Fuzz} fuzz.}
    \textit{Bold indicates best performing models.}
    \textit{Learning rate multiplier for nonlinearity in gray-box models shown in brackets.}
    }
    \label{tab:metrics_od_diy-klon-centaur}
    \centerline{
        \begin{tabular}{lccccccccccccc}
            \hline
            \midrule
            
            \multirow{2}{*}{Model}
                & \multirow{2}{*}{Params.}
                    & \multirow{2}{*}{LR}
                        & \multicolumn{2}{c}{Weights}
                            & \multicolumn{3}{c}{}
                                & \multicolumn{3}{c}{FAD} \\ 
            \cmidrule(lr){4-5} 
                % \cmidrule(lr){6-8} 
                    \cmidrule(lr){9-12}
            
            &   &   & L1 & MR-STFT & MSE & ESR & MAPE & VGGish & PANN & CLAP & AFx-Rep \\ 
            \hline
            LSTM-32 & 4.5k & 0.001 & 10 & 1 & 6.74e-03 & 0.5091 & 15.6365 & 3.9768 & 9.32e-05 & 0.3342 & 0.7288 \\
            LSTM-96 & 38.1k & 0.001 & 5 & 5 & 6.08e-03 & 0.4479 & 13.4749 & 1.8205 & 6.77e-05 & 0.1799 & 0.3917 \\
            \hline
            TCN-45-S-16 & 7.5k & 0.005 & 0.5 & 0.5 & 5.47e-04 & 0.0404 & 4.2415 & 0.1861 & 3.99e-05 & 0.0305 & 0.0444 \\
            TCN-45-L-16 & 7.3k & 0.005 & 5 & 5 & 7.33e-04 & 0.0508 & 2.6337 & 0.1814 & 8.60e-05 & 0.0373 & 0.0483 \\
            TCN-250-S-16 & 14.5k & 0.005 & 10 & 1 & 4.41e-04 & 0.0333 & 3.3895 & 0.1018 & 1.25e-05 & 0.0195 & 0.0286 \\
            TCN-250-L-16 & 18.4k & 0.005 & 1 & 0.1 & 2.74e-04 & 0.0214 & 2.1916 & 0.0828 & 8.70e-06 & 0.0110 & 0.0134 \\
            TCN-2500-S-16 & 13.7k & 0.005 & 10 & 1 & 8.01e-04 & 0.0582 & 4.1357 & 0.2226 & 1.81e-05 & 0.0292 & 0.0387 \\
            TCN-2500-L-16 & 11.9k & 0.005 & 1 & 0.1 & 3.63e-04 & 0.0282 & 2.0592 & 0.1304 & 3.85e-06 & 0.0135 & 0.0149 \\
            \hline
            TCN-TF-45-S-16 & 39.5k & 0.005 & 10 & 1 & 6.38e-04 & 0.0456 & 2.3212 & 0.8912 & 8.29e-05 & 0.0894 & 0.1407 \\
            TCN-TF-45-L-16 & 71.3k & 0.005 & 1 & 0.1 & 3.82e-04 & 0.0279 & 1.5769 & 0.3794 & 5.63e-05 & 0.0597 & 0.0556 \\
            TCN-TF-250-S-16 & 52.9k & 0.005 & 1 & 0.1 & 6.63e-04 & 0.0466 & 1.6720 & 0.5500 & 1.59e-04 & 0.0555 & 0.0972 \\
            TCN-TF-250-L-16 & 88.8k & 0.005 & 5 & 5 & 3.96e-03 & 0.2689 & 3.1497 & 0.8495 & 4.75e-05 & 0.1017 & 0.0931 \\
            TCN-TF-2500-S-16 & 45.7k & 0.005 & 10 & 1 & 1.10e-03 & 0.0826 & 7.2219 & 0.4671 & 1.88e-04 & 0.0592 & 0.1902 \\
            TCN-TF-2500-L-16 & 75.9k & 0.005 & 0.5 & 0.5 & 2.52e-03 & 0.1716 & 2.5158 & 0.4645 & 1.18e-04 & 0.0669 & 0.0267 \\
            \hline
            GCN-45-S-16 & 16.2k & 0.005 & 10 & 1 & 3.16e-04 & 0.0245 & 2.4492 & 0.1132 & 1.58e-05 & 0.0220 & 0.0515 \\
            GCN-45-L-16 & 17.1k & 0.005 & 10 & 1 & 2.48e-04 & 0.0196 & 1.9068 & 0.1304 & 1.71e-05 & 0.0243 & 0.0230 \\
            GCN-250-S-16 & 30.4k & 0.005 & 10 & 1 & 2.35e-04 & 0.0189 & 2.5600 & \textbf{0.0602} & 3.68e-06 & \textbf{0.0092} & 0.0131 \\
            GCN-250-L-16 & 39.6k & 0.005 & 1 & 0.1 & 2.38e-04 & 0.0182 & 1.9500 & 0.0680 & 5.88e-06 & 0.0102 & 0.0152 \\
            GCN-2500-S-16 & 28.6k & 0.005 & 0.5 & 0.5 & 6.07e-04 & 0.0421 & 2.4410 & 0.0927 & 1.35e-06 & 0.0169 & 0.0135 \\
            GCN-2500-L-16 & 26.4k & 0.005 & 1 & 0.1 & 3.06e-04 & 0.0232 & 2.2419 & 0.0801 & 2.70e-06 & 0.0141 & 0.0304 \\
            \hline
            GCN-TF-45-S-16 & 141.6k & 0.005 & 10 & 1 & 5.85e-04 & 0.0418 & 2.8098 & 0.5357 & 4.52e-05 & 0.0883 & 0.1347 \\
            GCN-TF-45-L-16 & 268.0k & 0.005 & 1 & 0.1 & 2.33e-04 & \textbf{0.0176} & 1.9511 & 0.7671 & 5.88e-05 & 0.0943 & 0.0604 \\
            GCN-TF-250-S-16 & 181.0k & 0.005 & 10 & 1 & 2.07e-03 & 0.1478 & 7.2714 & 0.9028 & 9.65e-05 & 0.1337 & 0.2381 \\
            GCN-TF-250-L-16 & 315.6k & 0.005 & 1 & 0.1 & 4.76e-04 & 0.0344 & 2.2700 & 0.5768 & 1.09e-04 & 0.0729 & 0.1104 \\
            GCN-TF-2500-S-16 & 154.1k & 0.005 & 5 & 5 & 4.78e-04 & 0.0331 & 2.2827 & 0.3036 & 4.31e-05 & 0.0395 & 0.0395 \\
            GCN-TF-2500-L-16 & 277.3k & 0.005 & 0.5 & 0.5 & 5.87e-04 & 0.0400 & 1.6689 & 0.6217 & 5.00e-05 & 0.0985 & 0.0661 \\
            \hline
            S4-S-16 & 2.4k & 0.01 & 1 & 0.1 & 4.32e-04 & 0.0328 & 1.7129 & 0.1179 & \textbf{6.43e-07} & 0.0327 & 0.0315 \\
            S4-L-16 & 19.0k & 0.01 & 10 & 1 & 2.59e-04 & 0.0205 & 1.2290 & 0.2815 & 5.38e-06 & 0.0424 & \textbf{0.0123} \\
            \hline
            S4-TF-S-16 & 28.0k & 0.01 & 1 & 0.1 & 2.66e-04 & 0.0207 & 1.3235 & 0.4832 & 9.70e-05 & 0.0690 & 0.0430 \\
            S4-TF-L-16 & 70.2k & 0.01 & 10 & 1 & \textbf{2.29e-04} & 0.0181 & \textbf{1.0153} & 0.5121 & 7.88e-05 & 0.0700 & 0.0246 \\
            \hline
            GB-DIST-MLP & 2.2k & 0.1 & 5 & 5 & 4.17e-02 & 2.9266 & 9.1623 & 1.2965 & 4.67e-04 & 0.1155 & 0.3451 \\
            GB-DIST-RNL & 47 & 0.1 & 5 & 5 & 6.98e-03 & 0.5341 & 16.7509 & 2.1911 & 7.50e-05 & 0.1835 & 0.3865 \\
            \hline
            GB-FUZZ-MLP & 2.3k & 0.1 & 5 & 5 & 2.56e-03 & 0.1938 & 11.5878 & 0.8343 & 1.02e-05 & 0.1215 & 0.2632 \\
            GB-FUZZ-RNL & 62 & 0.1 & 1 & 0.1 & 2.47e-03 & 0.1867 & 10.4652 & 0.6758 & 3.85e-05 & 0.1154 & 0.3534 \\
            \hline
            \hline
        \end{tabular}
    }
\end{table*}

% 0.001	10	1	6.74e-03	0.5091	15.6365	3.9768	9.32e-05	0.3342	0.7288
% 0.001	5	5	6.08e-03	0.4479	13.4749	1.8205	6.77e-05	0.1799	0.3917
% 0.005	0.5	0.5	5.47e-04	0.0404	4.2415	0.1861	3.99e-05	0.0305	0.0444
% 0.005	5	5	7.33e-04	0.0508	2.6337	0.1814	8.60e-05	0.0373	0.0483
% 0.005	10	1	4.41e-04	0.0333	3.3895	0.1018	1.25e-05	0.0195	0.0286
% 0.005	1	0.1	2.74e-04	0.0214	2.1916	0.0828	8.70e-06	0.0110	0.0134
% 0.005	10	1	8.01e-04	0.0582	4.1357	0.2226	1.81e-05	0.0292	0.0387
% 0.005	1	0.1	3.63e-04	0.0282	2.0592	0.1304	3.85e-06	0.0135	0.0149
% 0.005	10	1	6.38e-04	0.0456	2.3212	0.8912	8.29e-05	0.0894	0.1407
% 0.005	1	0.1	3.82e-04	0.0279	1.5769	0.3794	5.63e-05	0.0597	0.0556
% 0.005	1	0.1	6.63e-04	0.0466	1.6720	0.5500	1.59e-04	0.0555	0.0972
% 0.005	5	5	3.96e-03	0.2689	3.1497	0.8495	4.75e-05	0.1017	0.0931
% 0.005	10	1	1.10e-03	0.0826	7.2219	0.4671	1.88e-04	0.0592	0.1902
% 0.005	0.5	0.5	2.52e-03	0.1716	2.5158	0.4645	1.18e-04	0.0669	0.0267
% 0.005	10	1	3.16e-04	0.0245	2.4492	0.1132	1.58e-05	0.0220	0.0515
% 0.005	10	1	2.48e-04	0.0196	1.9068	0.1304	1.71e-05	0.0243	0.0230
% 0.005	10	1	2.35e-04	0.0189	2.5600	0.0602	3.68e-06	0.0092	0.0131
% 0.005	1	0.1	2.38e-04	0.0182	1.9500	0.0680	5.88e-06	0.0102	0.0152
% 0.005	0.5	0.5	6.07e-04	0.0421	2.4410	0.0927	1.35e-06	0.0169	0.0135
% 0.005	1	0.1	3.06e-04	0.0232	2.2419	0.0801	2.70e-06	0.0141	0.0304
% 0.005	10	1	5.85e-04	0.0418	2.8098	0.5357	4.52e-05	0.0883	0.1347
% 0.005	1	0.1	2.33e-04	0.0176	1.9511	0.7671	5.88e-05	0.0943	0.0604
% 0.005	10	1	2.07e-03	0.1478	7.2714	0.9028	9.65e-05	0.1337	0.2381
% 0.005	1	0.1	4.76e-04	0.0344	2.2700	0.5768	1.09e-04	0.0729	0.1104
% 0.005	5	5	4.78e-04	0.0331	2.2827	0.3036	4.31e-05	0.0395	0.0395
% 0.005	0.5	0.5	5.87e-04	0.0400	1.6689	0.6217	5.00e-05	0.0985	0.0661
% 0.01	1	0.1	4.32e-04	0.0328	1.7129	0.1179	6.43e-07	0.0327	0.0315
% 0.01	10	1	2.59e-04	0.0205	1.2290	0.2815	5.38e-06	0.0424	0.0123
% 0.01	1	0.1	2.66e-04	0.0207	1.3235	0.4832	9.70e-05	0.0690	0.0430
% 0.01	10	1	2.29e-04	0.0181	1.0153	0.5121	7.88e-05	0.0700	0.0246
% 0.1 (0.01)	5	5	4.17e-02	2.9266	9.1623	1.2965	4.67e-04	0.1155	0.3451
% 0.1 (1)	5	5	6.98e-03	0.5341	16.7509	2.1911	7.50e-05	0.1835	0.3865
% 0.1 (0.01)	5	5	2.56e-03	0.1938	11.5878	0.8343	1.02e-05	0.1215	0.2632
% 0.1 (1)	1	0.1	2.47e-03	0.1867	10.4652	0.6758	3.85e-05	0.1154	0.3534
\setlength{\tabcolsep}{4pt}
\renewcommand{\arraystretch}{1.3}
\begin{table*}[h]
    \small
    \caption{
    \textit{Objective metrics for non parametric models of \textbf{Arturia Rev Spring 636 Preamp} fuzz.}
    \textit{Bold indicates best performing models.}
    \textit{Learning rate multiplier for nonlinearity in gray-box models shown in brackets.}
    }
    \label{tab:metrics_od_diy-klon-centaur}
    \centerline{
        \begin{tabular}{lccccccccccccc}
            \hline
            \midrule
            
            \multirow{2}{*}{Model}
                & \multirow{2}{*}{Params.}
                    & \multirow{2}{*}{LR}
                        & \multicolumn{2}{c}{Weights}
                            & \multicolumn{3}{c}{}
                                & \multicolumn{3}{c}{FAD} \\ 
            \cmidrule(lr){4-5} 
                % \cmidrule(lr){6-8} 
                    \cmidrule(lr){9-12}
            
            &   &   & L1 & MR-STFT & MSE & ESR & MAPE & VGGish & PANN & CLAP & AFx-Rep \\ 
            \hline
            LSTM-32 & 4.5k & 0.005 & 1 & 0.1 & 2.88e-04 & 0.0119 & 0.7834 & 0.0221 & 1.45e-06 & 0.0018 & 0.0109 \\
            LSTM-96 & 38.1k & 0.001 & 5 & 5 & 1.28e-03 & 0.0514 & 0.6530 & 0.0053 & 1.71e-07 & 0.0009 & 0.0073 \\
            \hline
            TCN-45-S-16 & 7.5k & 0.005 & 5 & 5 & 2.71e-03 & 0.1112 & 2.3335 & 0.0975 & 1.15e-06 & 0.0110 & 0.0562 \\
            TCN-45-L-16 & 7.3k & 0.005 & 10 & 1 & 1.33e-03 & 0.0565 & 1.7421 & 0.0853 & 4.33e-06 & 0.0112 & 0.0405 \\
            TCN-250-S-16 & 14.5k & 0.005 & 10 & 1 & 4.14e-04 & 0.0169 & 0.6300 & 0.0406 & 4.09e-06 & 0.0075 & 0.0188 \\
            TCN-250-L-16 & 18.4k & 0.005 & 1 & 0.1 & 4.15e-04 & 0.0171 & 0.7136 & 0.0441 & 3.99e-07 & 0.0065 & 0.0207 \\
            TCN-2500-S-16 & 13.7k & 0.005 & 0.5 & 0.5 & 1.85e-03 & 0.0743 & 0.9756 & 0.0597 & 3.29e-07 & 0.0080 & 0.0504 \\
            TCN-2500-L-16 & 11.9k & 0.005 & 0.5 & 0.5 & 2.43e-03 & 0.0978 & 1.1197 & 0.0386 & 1.51e-06 & 0.0055 & 0.0509 \\
            \hline
            TCN-TF-45-S-16 & 39.5k & 0.005 & 1 & 0.1 & 1.30e-04 & 0.0054 & 0.3339 & 0.0277 & 2.73e-06 & 0.0116 & 0.0351 \\
            TCN-TF-45-L-16 & 71.3k & 0.005 & 1 & 0.1 & 3.18e-04 & 0.0132 & 0.3950 & 0.0216 & 3.05e-06 & 0.0024 & 0.0374 \\
            TCN-TF-250-S-16 & 52.9k & 0.005 & 10 & 1 & 1.58e-04 & 0.0065 & 0.3430 & 0.0244 & 2.48e-06 & 0.0071 & 0.0681 \\
            TCN-TF-250-L-16 & 88.8k & 0.005 & 10 & 1 & 3.34e-04 & 0.0137 & 0.3805 & 0.0197 & 1.03e-06 & 0.0025 & 0.0255 \\
            TCN-TF-2500-S-16 & 45.7k & 0.005 & 0.5 & 0.5 & 1.78e-03 & 0.0712 & 0.7379 & 0.0254 & 3.30e-07 & 0.0042 & 0.0664 \\
            TCN-TF-2500-L-16 & 75.9k & 0.005 & 5 & 5 & 3.91e-03 & 0.1568 & 0.9180 & 0.0122 & 1.95e-07 & 0.0020 & 0.0565 \\
            \hline
            GCN-45-S-16 & 16.2k & 0.005 & 10 & 1 & 1.25e-03 & 0.0529 & 1.5329 & 0.0890 & 9.36e-07 & 0.0104 & 0.0374 \\
            GCN-45-L-16 & 17.1k & 0.005 & 5 & 5 & 5.52e-03 & 0.2232 & 2.1635 & 0.0936 & 2.52e-07 & 0.0096 & 0.0577 \\
            GCN-250-S-16 & 30.4k & 0.005 & 10 & 1 & 3.69e-04 & 0.0152 & 0.6251 & 0.0394 & 4.16e-06 & 0.0055 & 0.0269 \\
            GCN-250-L-16 & 39.6k & 0.005 & 1 & 0.1 & 2.42e-04 & 0.0101 & 0.5882 & 0.0265 & 7.26e-06 & 0.0035 & 0.0464 \\
            GCN-2500-S-16 & 28.6k & 0.005 & 5 & 5 & 1.37e-03 & 0.0552 & 0.8173 & 0.0537 & 5.92e-06 & 0.0061 & 0.0398 \\
            GCN-2500-L-16 & 26.4k & 0.005 & 10 & 1 & 5.56e-04 & 0.0229 & 0.5805 & 0.0358 & 5.40e-06 & 0.0047 & 0.0467 \\
            \hline
            GCN-TF-45-S-16 & 141.6k & 0.005 & 10 & 1 & 1.16e-04 & 0.0049 & 0.2786 & 0.0091 & 6.32e-07 & 0.0096 & 0.0169 \\
            GCN-TF-45-L-16 & 268.0k & 0.005 & 5 & 5 & 2.43e-03 & 0.0979 & 0.8554 & 0.0098 & 1.03e-07 & 0.0019 & 0.0101 \\
            GCN-TF-250-S-16 & 181.0k & 0.005 & 5 & 5 & 1.15e-03 & 0.0461 & 0.5897 & 0.0072 & 5.08e-07 & 0.0054 & 0.0404 \\
            GCN-TF-250-L-16 & 315.6k & 0.005 & 10 & 1 & 4.88e-04 & 0.0202 & 0.4354 & 0.0105 & 6.52e-07 & 0.0016 & 0.0064 \\
            GCN-TF-2500-S-16 & 154.1k & 0.005 & 5 & 5 & 1.29e-03 & 0.0521 & 0.6616 & 0.0195 & \textbf{1.34e-09} & 0.0022 & 0.0354 \\
            GCN-TF-2500-L-16 & 277.3k & 0.005 & 0.5 & 0.5 & 1.47e-03 & 0.0591 & 0.6384 & 0.0074 & 2.57e-07 & 0.0014 & 0.0370 \\
            \hline
            S4-S-16 & 2.4k & 0.01 & 5 & 5 & 4.35e-04 & 0.0175 & 0.5423 & 0.0082 & 1.15e-07 & 0.0011 & 0.0180 \\
            S4-L-16 & 19.0k & 0.01 & 5 & 5 & 7.06e-04 & 0.0283 & 0.6099 & \textbf{0.0043} & 2.80e-08 & \textbf{0.0008} & 0.0015 \\
            \hline
            S4-TF-S-16 & 28.0k & 0.01 & 5 & 5 & 3.50e-04 & 0.0141 & 0.3821 & 0.0049 & 2.16e-07 & 0.0011 & 0.0140 \\
            S4-TF-L-16 & 70.2k & 0.01 & 10 & 1 & \textbf{1.01e-04} & \textbf{0.0042} & \textbf{0.2493} & 0.0043 & 9.17e-08 & 0.0008 & \textbf{0.0012} \\
            \hline
            GB-DIST-MLP & 2.2k & 0.1 & 10 & 1 & 3.52e-03 & 0.1451 & 2.1582 & 0.2588 & 6.97e-06 & 0.0210 & 0.0469 \\
            GB-DIST-RNL & 47 & 0.1 & 0.5 & 0.5 & 4.00e-03 & 0.1651 & 3.5724 & 0.3442 & 4.58e-05 & 0.0235 & 0.0707 \\
            \hline
            GB-FUZZ-MLP & 2.3k & 0.1 & 0.5 & 0.5 & 2.87e-03 & 0.1171 & 1.8932 & 0.1243 & 2.47e-05 & 0.0143 & 0.0376 \\
            GB-FUZZ-RNL & 62 & 0.1 & 0.5 & 0.5 & 3.97e-03 & 0.1636 & 3.4919 & 0.3373 & 4.41e-05 & 0.0200 & 0.0750 \\
            \hline
            \hline
        \end{tabular}
    }
\end{table*}

% 0.005	1	0.1	2.88e-04	0.0119	0.7834	0.0221	1.45e-06	0.0018	0.0109
% 0.001	5	5	1.28e-03	0.0514	0.6530	0.0053	1.71e-07	0.0009	0.0073
% 0.005	5	5	2.71e-03	0.1112	2.3335	0.0975	1.15e-06	0.0110	0.0562
% 0.005	10	1	1.33e-03	0.0565	1.7421	0.0853	4.33e-06	0.0112	0.0405
% 0.005	10	1	4.14e-04	0.0169	0.6300	0.0406	4.09e-06	0.0075	0.0188
% 0.005	1	0.1	4.15e-04	0.0171	0.7136	0.0441	3.99e-07	0.0065	0.0207
% 0.005	0.5	0.5	1.85e-03	0.0743	0.9756	0.0597	3.29e-07	0.0080	0.0504
% 0.005	0.5	0.5	2.43e-03	0.0978	1.1197	0.0386	1.51e-06	0.0055	0.0509
% 0.005	1	0.1	1.30e-04	0.0054	0.3339	0.0277	2.73e-06	0.0116	0.0351
% 0.005	1	0.1	3.18e-04	0.0132	0.3950	0.0216	3.05e-06	0.0024	0.0374
% 0.005	10	1	1.58e-04	0.0065	0.3430	0.0244	2.48e-06	0.0071	0.0681
% 0.005	10	1	3.34e-04	0.0137	0.3805	0.0197	1.03e-06	0.0025	0.0255
% 0.005	0.5	0.5	1.78e-03	0.0712	0.7379	0.0254	3.30e-07	0.0042	0.0664
% 0.005	5	5	3.91e-03	0.1568	0.9180	0.0122	1.95e-07	0.0020	0.0565
% 0.005	10	1	1.25e-03	0.0529	1.5329	0.0890	9.36e-07	0.0104	0.0374
% 0.005	5	5	5.52e-03	0.2232	2.1635	0.0936	2.52e-07	0.0096	0.0577
% 0.005	10	1	3.69e-04	0.0152	0.6251	0.0394	4.16e-06	0.0055	0.0269
% 0.005	1	0.1	2.42e-04	0.0101	0.5882	0.0265	7.26e-06	0.0035	0.0464
% 0.005	5	5	1.37e-03	0.0552	0.8173	0.0537	5.92e-06	0.0061	0.0398
% 0.005	10	1	5.56e-04	0.0229	0.5805	0.0358	5.40e-06	0.0047	0.0467
% 0.005	10	1	1.16e-04	0.0049	0.2786	0.0091	6.32e-07	0.0096	0.0169
% 0.005	5	5	2.43e-03	0.0979	0.8554	0.0098	1.03e-07	0.0019	0.0101
% 0.005	5	5	1.15e-03	0.0461	0.5897	0.0072	5.08e-07	0.0054	0.0404
% 0.005	10	1	4.88e-04	0.0202	0.4354	0.0105	6.52e-07	0.0016	0.0064
% 0.005	5	5	1.29e-03	0.0521	0.6616	0.0195	1.34e-09	0.0022	0.0354
% 0.005	0.5	0.5	1.47e-03	0.0591	0.6384	0.0074	2.57e-07	0.0014	0.0370
% 0.01	5	5	4.35e-04	0.0175	0.5423	0.0082	1.15e-07	0.0011	0.0180
% 0.01	5	5	7.06e-04	0.0283	0.6099	0.0043	2.80e-08	0.0008	0.0015
% 0.01	5	5	3.50e-04	0.0141	0.3821	0.0049	2.16e-07	0.0011	0.0140
% 0.01	10	1	1.01e-04	0.0042	0.2493	0.0043	9.17e-08	0.0008	0.0012
% 0.1 (0.01)	10	1	3.52e-03	0.1451	2.1582	0.2588	6.97e-06	0.0210	0.0469
% 0.1 (1)	0.5	0.5	4.00e-03	0.1651	3.5724	0.3442	4.58e-05	0.0235	0.0707
% 0.1 (0.01)	0.5	0.5	2.87e-03	0.1171	1.8932	0.1243	2.47e-05	0.0143	0.0376
% 0.1 (1)	0.5	0.5	3.97e-03	0.1636	3.4919	0.3373	4.41e-05	0.0200	0.0750

\clearpage
% ===
\subsection{Results Parametric Models}
\setlength{\tabcolsep}{4pt}
\renewcommand{\arraystretch}{1.3}
\begin{table*}[h]
    \small
    \caption{
    \textit{Scaled validation and test loss for non parametric models of \textbf{Marshall JVM410H - Ch. OD1} amplifier.}
    \textit{Bold indicates best performing models.}
    \textit{Learning rate multiplier for nonlinearity in gray-box models shown in brackets.}
    }
    \label{tab:val-and-test-loss_od_ibanez-ts9}
    \centerline{
        \begin{tabular}{l c c cc >{\columncolor{gray!20}}ccc >{\columncolor{gray!20}}ccc}
            \hline
            \midrule
            
            \multirow{2}{*}{Model}
                & \multirow{2}{*}{Params.}
                    & \multirow{2}{*}{LR}
                        & \multicolumn{2}{c}{Weights}
                            & \multicolumn{3}{c}{Val. Loss}
                                & \multicolumn{3}{c}{Test Loss} \\ 
            \cmidrule(lr){4-5} 
                \cmidrule(lr){6-8} 
                    \cmidrule(lr){9-11}
            
            &   &   & {\scriptsize L1} & {\scriptsize MR-STFT} & Tot. & {\scriptsize L1} & {\scriptsize MR-STFT} & Tot. & {\scriptsize L1} & {\scriptsize MR-STFT} \\ 
            
            \hline
            LSTM-C-32 & 5.0k & 0.001 & 0.5 & 0.5 & 0.4634 & 0.0131 & 0.4503 & 1.1216 & 0.0376 & 1.0839 \\
            LSTM-TVC-32 & 8.0k & 0.001 & 1 & 0.1 & 0.4521 & 0.0103 & 0.4419 & 1.1610 & 0.0351 & 1.1259 \\
            \hline
            LSTM-C-96 & 39.7k & 0.001 & 10 & 1 & 0.4686 & 0.0126 & 0.4560 & 1.1423 & 0.0347 & 1.1075 \\
            LSTM-TVC-96 & 4.6k & 0.001 & 0.5 & 0.5 & 0.4345 & 0.0098 & 0.4247 & 1.1142 & 0.0344 & 1.0798 \\
            \hline
            \hline
            TCN-F-45-S-16 & 15.0k & 0.005 & 5 & 5 & 0.6137 & 0.0224 & 0.5913 & 1.2636 & 0.0519 & 1.2117 \\
            TCN-TF-45-S-16 & 42.0k & 0.005 & 10 & 1 & 0.4624 & 0.0115 & 0.4509 & 1.2087 & 0.0333 & 1.1754 \\
            TCN-TTF-45-S-16 & 17.3k & 0.005 & 1 & 0.1 & 0.5220 & 0.0163 & 0.5057 & 1.2982 & 0.0449 & 1.2533 \\
            TCN-TVF-45-S-16 & 17.7k & 0.005 & 0.5 & 0.5 & 0.4809 & 0.0164 & 0.4645 & 1.4137 & 0.0534 & 1.3603 \\
            \hline
            TCN-F-45-L-16 & 20.1k & 0.005 & 0.5 & 0.5 & 0.5962 & 0.0230 & 0.5733 & 1.5938 & 0.0638 & 1.5300 \\
            TCN-TF-45-L-16 & 76.4k & 0.005 & 1 & 0.1 & 0.3976 & 0.0090 & 0.3886 & 1.1138 & 0.0319 & 1.0819 \\
            TCN-TTF-45-L-16 & 27.0k & 0.005 & 1 & 0.1 & 0.4759 & 0.0134 & 0.4626 & 1.2022 & 0.0385 & 1.1637 \\
            TCN-TVF-45-L-16 & 22.8k & 0.005 & 5 & 5 & 0.4476 & 0.0134 & 0.4343 & 1.1835 & 0.0393 & 1.1442 \\
            \hline
            \hline
            S4-F-S-16 & 8.9k & 0.01 & 1 & 0.1 & 0.4571 & 0.0089 & 0.4482 & 1.2783 & 0.0465 & 1.2317 \\
            S4-TF-S-16 & 30.0k & 0.01 & 10 & 1 & 0.3864 & 0.0073 & 0.3791 & 1.0965 & 0.0290 & 1.0675 \\
            S4-TTF-S-16 & 10.2k & 0.01 & 1 & 0.1 & 0.4227 & 0.0102 & 0.4125 & 1.2164 & 0.0368 & 1.1796 \\
            S4-TVF-S-16 & 11.6k & 0.01 & 5 & 5 & 0.3778 & 0.0095 & 0.3683 & 1.0991 & 0.0351 & 1.0640 \\
            \hline
            S4-F-L-16 & 29.7k & 0.01 & 5 & 5 & 0.3503 & 0.0084 & 0.3419 & 1.1745 & 0.0374 & 1.1370 \\
            S4-TF-L-16 & 74.3k & 0.01 & 1 & 0.1 & 0.3109 & 0.0049 & 0.3060 & 1.1490 & 0.0321 & 1.1169 \\
            S4-TTF-L-16 & 34.8k & 0.01 & 5 & 5 & 0.4066 & 0.0132 & 0.3933 & 1.0984 & 0.0349 & 1.0635 \\
            S4-TVF-L-16 & 32.4k & 0.01 & 10 & 1 & \textbf{0.2965} & 0.0048 & 0.2917 & \textbf{1.0133} & 0.0257 & 0.9876 \\
            \hline
            \hline
            GB-C-DIST-MLP & 4.5k & 0.1 (0.01) & 0.5 & 0.5 & 0.8563 & 0.0413 & 0.8151 & 1.5072 & 0.0741 & 1.4331 \\
            GB-C-DIST-RNL & 2.3k & 0.1 (1) & 0.5 & 0.5 & 0.8395 & 0.0407 & 0.7988 & 1.5113 & 0.0667 & 1.4445 \\
            \hline
            GB-C-FUZZ-MLP & 4.2k & 0.1 (0.01) & 5 & 5 & 0.8208 & 0.0401 & 0.7807 & 1.5009 & 0.0686 & 1.4323 \\
            GB-C-FUZZ-RNL & 2.0k & 0.1 (1) & 5 & 5 & 0.8123 & 0.0438 & 0.7685 & 1.5428 & 0.0720 & 1.4708 \\
            \hline
            \hline
        \end{tabular}
    }
\end{table*}

% 5.00E+03	0.001	0.5	0.5	0.4634	0.0131	0.4503	1.1216	0.0376	1.0839
% 8.00E+03	0.001	1	0.1	0.4521	0.0103	0.4419	1.1610	0.0351	1.1259
% 3.97E+04	0.001	10	1	0.4686	0.0126	0.4560	1.1423	0.0347	1.1075
% 4.57E+04	0.001	0.5	0.5	0.4345	0.0098	0.4247	1.1142	0.0344	1.0798
% 1.50E+04	0.005	5	5	0.6137	0.0224	0.5913	1.2636	0.0519	1.2117
% 4.20E+04	0.005	10	1	0.4624	0.0115	0.4509	1.2087	0.0333	1.1754
% 1.73E+04	0.005	1	0.1	0.5220	0.0163	0.5057	1.2982	0.0449	1.2533
% 1.77E+04	0.005	0.5	0.5	0.4809	0.0164	0.4645	1.4137	0.0534	1.3603
% 2.01E+04	0.005	0.5	0.5	0.5962	0.0230	0.5733	1.5938	0.0638	1.5300
% 7.64E+04	0.005	1	0.1	0.3976	0.0090	0.3886	1.1138	0.0319	1.0819
% 2.70E+04	0.005	1	0.1	0.4759	0.0134	0.4626	1.2022	0.0385	1.1637
% 2.28E+04	0.005	5	5	0.4476	0.0134	0.4343	1.1835	0.0393	1.1442
% 8.90E+03	0.01	1	0.1	0.4571	0.0089	0.4482	1.2783	0.0465	1.2317
% 3.00E+04	0.01	10	1	0.3864	0.0073	0.3791	1.0965	0.0290	1.0675
% 1.02E+04	0.01	1	0.1	0.4227	0.0102	0.4125	1.2164	0.0368	1.1796
% 1.16E+04	0.01	5	5	0.3778	0.0095	0.3683	1.0991	0.0351	1.0640
% 2.97E+04	0.01	5	5	0.3503	0.0084	0.3419	1.1745	0.0374	1.1370
% 7.43E+04	0.01	1	0.1	0.3109	0.0049	0.3060	1.1490	0.0321	1.1169
% 3.48E+04	0.01	5	5	0.4066	0.0132	0.3933	1.0984	0.0349	1.0635
% 3.24E+04	0.01	10	1	0.2965	0.0048	0.2917	1.0133	0.0257	0.9876
% 4.50E+03	0.1 (0.01)	0.5	0.5	0.8563	0.0413	0.8151	1.5072	0.0741	1.4331
% 2.30E+03	0.1 (1)	0.5	0.5	0.8395	0.0407	0.7988	1.5113	0.0667	1.4445
% 4.20E+03	0.1 (0.01)	5	5	0.8208	0.0401	0.7807	1.5009	0.0686	1.4323
% 2.00E+03	0.1 (1)	5	5	0.8123	0.0438	0.7685	1.5428	0.0720	1.4708

\setlength{\tabcolsep}{4pt}
\renewcommand{\arraystretch}{1.3}
\begin{table*}[h]
    \small
    \caption{
    \textit{Scaled validation and test loss for non parametric models of \textbf{Multidrive F-Fuzz} fuzz.}
    \textit{Bold indicates best performing models.}
    \textit{Learning rate multiplier for nonlinearity in gray-box models shown in brackets.}
    }
    \label{tab:val-and-test-loss_od_ibanez-ts9}
    \centerline{
        \begin{tabular}{l c c cc >{\columncolor{gray!20}}ccc >{\columncolor{gray!20}}ccc}
            \hline
            \midrule
            
            \multirow{2}{*}{Model}
                & \multirow{2}{*}{Params.}
                    & \multirow{2}{*}{LR}
                        & \multicolumn{2}{c}{Weights}
                            & \multicolumn{3}{c}{Val. Loss}
                                & \multicolumn{3}{c}{Test Loss} \\ 
            \cmidrule(lr){4-5} 
                \cmidrule(lr){6-8} 
                    \cmidrule(lr){9-11}
            
            &   &   & {\scriptsize L1} & {\scriptsize MR-STFT} & Tot. & {\scriptsize L1} & {\scriptsize MR-STFT} & Tot. & {\scriptsize L1} & {\scriptsize MR-STFT} \\ 
            
            \hline
            LSTM-C-32 & 5.0k & 0.001 & 5 & 5 & 0.3234 & 0.0050 & 0.3184 & 0.3123 & 0.0051 & 0.3072 \\
            LSTM-TVC-32 & 8.0k & 0.001 & 1 & 0.1 & 0.2071 & 0.0027 & 0.2044 & 0.1652 & 0.0023 & 0.1628 \\
            \hline
            LSTM-C-96 & 39.7k & 0.001 & 1 & 0.1 & 0.2405 & 0.0052 & 0.2353 & 0.1689 & 0.0024 & 0.1665 \\
            LSTM-TVC-96 & 45.7k & 0.001 & 5 & 5 & \textbf{0.2026} & 0.0053 & 0.1973 & \textbf{0.1560} & 0.0040 & 0.1521 \\
            \hline
            \hline
            TCN-F-45-S-16 & 15.0k & 0.005 & 1 & 0.1 & 0.6664 & 0.0163 & 0.6501 & 0.7095 & 0.0217 & 0.6878 \\
            TCN-TF-45-S-16 & 42.0k & 0.005 & 10 & 1 & 0.5147 & 0.0082 & 0.5065 & 0.4886 & 0.0077 & 0.4809 \\
            TCN-TTF-45-S-16 & 17.3k & 0.005 & 10 & 1 & 0.5594 & 0.0106 & 0.5488 & 0.5324 & 0.0102 & 0.5223 \\
            TCN-TVF-45-S-16 & 17.7k & 0.005 & 10 & 1 & 0.5491 & 0.0117 & 0.5374 & 0.5356 & 0.0115 & 0.5241 \\
            \hline
            TCN-F-45-L-16 & 20.1k & 0.005 & 1 & 0.1 & 0.6496 & 0.0176 & 0.6320 & 0.6681 & 0.0185 & 0.6495 \\
            TCN-TF-45-L-16 & 76.4k & 0.005 & 10 & 1 & 0.4026 & 0.0062 & 0.3964 & 0.3553 & 0.0058 & 0.3495 \\
            TCN-TTF-45-L-16 & 27.0k & 0.005 & 5 & 5 & 0.5052 & 0.0263 & 0.4790 & 0.4788 & 0.0225 & 0.4563 \\
            TCN-TVF-45-L-16 & 22.8k & 0.005 & 5 & 5 & 0.5872 & 0.0171 & 0.5701 & 0.5835 & 0.0164 & 0.5671 \\
            \hline
            \hline
            S4-F-S-16 & 89.0k & 0.01 & 1 & 0.1 & 0.5171 & 0.0107 & 0.5064 & 0.7687 & 0.0243 & 0.7444 \\
            S4-TF-S-16 & 30.0k & 0.01 & 1 & 0.1 & 0.3710 & 0.0055 & 0.3655 & 0.4034 & 0.0075 & 0.3959 \\
            S4-TTF-S-16 & 10.2k & 0.01 & 10 & 1 & 0.4264 & 0.0066 & 0.4198 & 0.3816 & 0.0066 & 0.3749 \\
            S4-TVF-S-16 & 11.6k & 0.01 & 1 & 0.1 & 0.3673 & 0.0055 & 0.3618 & 0.3354 & 0.0058 & 0.3296 \\
            \hline
            S4-F-L-16 & 29.7k & 0.01 & 5 & 5 & 0.3811 & 0.0262 & 0.3549 & 0.4973 & 0.0225 & 0.4748 \\
            S4-TF-L-16 & 74.3k & 0.01 & 10 & 1 & 0.2907 & 0.0054 & 0.2853 & 0.2619 & 0.0042 & 0.2577 \\
            S4-TTF-L-16 & 34.8k & 0.01 & 5 & 5 & 0.3506 & 0.0071 & 0.3436 & 0.3683 & 0.0075 & 0.3608 \\
            S4-TVF-L-16 & 32.4k & 0.01 & 10 & 1 & 0.2476 & 0.0041 & 0.2435 & 0.2673 & 0.0045 & 0.2628 \\
            \hline
            \hline
            GB-C-DIST-MLP & 45.0k & 0.1 (0.01) & 5 & 5 & 1.1759 & 0.0631 & 1.1128 & 1.2104 & 0.0611 & 1.1492 \\
            GB-C-DIST-RNL & 23.0k & 0.1 (1) & 0.5 & 0.5 & 1.2355 & 0.0683 & 1.1671 & 1.2531 & 0.0672 & 1.1858 \\
            \hline
            GB-C-FUZZ-MLP & 42.0k & 0.1 (0.01) & 0.5 & 0.5 & 0.9809 & 0.0363 & 0.9446 & 0.9303 & 0.0345 & 0.8958 \\
            GB-C-FUZZ-RNL & 20.0k & 0.1 (1) & 0.5 & 0.5 & 1.0043 & 0.0398 & 0.9645 & 0.9395 & 0.0355 & 0.9040 \\
            \hline
            \hline
        \end{tabular}
    }
\end{table*}

% 5.0k	0.001	5	5	0.3234	0.0050	0.3184	0.3123	0.0051	0.3072
% 8.0k	0.001	1	0.1	0.2071	0.0027	0.2044	0.1652	0.0023	0.1628
% 39.7k	0.001	1	0.1	0.2405	0.0052	0.2353	0.1689	0.0024	0.1665
% 45.7k	0.001	5	5	0.2026	0.0053	0.1973	0.1560	0.0040	0.1521
% 15.0k	0.005	1	0.1	0.6664	0.0163	0.6501	0.7095	0.0217	0.6878
% 42.0k	0.005	10	1	0.5147	0.0082	0.5065	0.4886	0.0077	0.4809
% 17.3k	0.005	10	1	0.5594	0.0106	0.5488	0.5324	0.0102	0.5223
% 17.7	0.005	10	1	0.5491	0.0117	0.5374	0.5356	0.0115	0.5241
% 20.1	0.005	1	0.1	0.6496	0.0176	0.6320	0.6681	0.0185	0.6495
% 76.4	0.005	10	1	0.4026	0.0062	0.3964	0.3553	0.0058	0.3495
% 27.0	0.005	5	5	0.5052	0.0263	0.4790	0.4788	0.0225	0.4563
% 22.8	0.005	5	5	0.5872	0.0171	0.5701	0.5835	0.0164	0.5671
% 89.0	0.01	1	0.1	0.5171	0.0107	0.5064	0.7687	0.0243	0.7444
% 30.0	0.01	1	0.1	0.3710	0.0055	0.3655	0.4034	0.0075	0.3959
% 10.2	0.01	10	1	0.4264	0.0066	0.4198	0.3816	0.0066	0.3749
% 11.6	0.01	1	0.1	0.3673	0.0055	0.3618	0.3354	0.0058	0.3296
% 29.7	0.01	5	5	0.3811	0.0262	0.3549	0.4973	0.0225	0.4748
% 74.3	0.01	10	1	0.2907	0.0054	0.2853	0.2619	0.0042	0.2577
% 34.8	0.01	5	5	0.3506	0.0071	0.3436	0.3683	0.0075	0.3608
% 32.4	0.01	10	1	0.2476	0.0041	0.2435	0.2673	0.0045	0.2628
% 45.0	0.1	5	5	1.1759	0.0631	1.1128	1.2104	0.0611	1.1492
% 23.0	0.1	0.5	0.5	1.2355	0.0683	1.1671	1.2531	0.0672	1.1858
% 42.0	0.1	0.5	0.5	0.9809	0.0363	0.9446	0.9303	0.0345	0.8958
% 20.0	0.1	0.5	0.5	1.0043	0.0398	0.9645	0.9395	0.0355	0.9040

\setlength{\tabcolsep}{4pt}
\renewcommand{\arraystretch}{1.3}
\begin{table*}[h]
    \small
    \caption{
    \textit{Objective metrics for parametric models of \textbf{Marshall JVM410H - Ch. OD1} amplifier.}
    \textit{Bold indicates best performing models.}
    \textit{Learning rate multiplier for nonlinearity in gray-box models shown in brackets.}
    }
    \label{tab:metrics_od_hb-green-tint}
    \centerline{
        \begin{tabular}{lccccccccccccc}
            \hline
            \midrule
            
            \multirow{2}{*}{Model}
                & \multirow{2}{*}{Params.}
                    & \multirow{2}{*}{LR}
                        & \multicolumn{2}{c}{Weights}
                            & \multicolumn{3}{c}{}
                                & \multicolumn{3}{c}{FAD} \\ 
            \cmidrule(lr){4-5} 
                % \cmidrule(lr){6-8} 
                    \cmidrule(lr){9-12}
            
            &   &   & L1 & MR-STFT & MSE & ESR & MAPE & VGGish & PANN & CLAP & AFx-Rep \\ 

            \hline
            LSTM-C-32 & 5.0k & 0.001 & 0.5 & 0.5 & 5.53e-03 & 0.2178 & 6.4798 & 0.6575 & 2.14e-05 & 0.0670 & 0.1319 \\
            LSTM-TVC-32 & 8.0k & 0.001 & 1 & 0.1 & 5.23e-03 & 0.2097 & 6.3634 & 0.6531 & 3.09e-05 & 0.0750 & 0.1340 \\
            \hline
            LSTM-C-96 & 39.7k & 0.001 & 10 & 1 & 4.65e-03 & 0.1742 & 4.9881 & 0.6790 & 4.34e-05 & 0.0646 & 0.1348 \\
            LSTM-TVC-96 & 4.6k & 0.001 & 0.5 & 0.5 & 5.61e-03 & 0.2240 & 5.4487 & 0.6408 & 2.00e-05 & 0.0672 & 0.1354 \\
            \hline
            \hline
            TCN-F-45-S-16 & 15.0k & 0.005 & 5 & 5 & 6.47e-03 & 0.2436 & 17.5463 & 1.0410 & 1.43e-05 & 0.0744 & 0.1798 \\
            TCN-TF-45-S-16 & 42.0k & 0.005 & 10 & 1 & 4.13e-03 & 0.1688 & 3.7782 & 0.7006 & \textbf{2.67e-07} & 0.0686 & 0.1616 \\
            TCN-TTF-45-S-16 & 17.3k & 0.005 & 1 & 0.1 & 4.67e-03 & 0.1788 & 10.3527 & 0.9590 & 5.78e-07 & 0.0764 & 0.2189 \\
            TCN-TVF-45-S-16 & 17.7k & 0.005 & 0.5 & 0.5 & 7.14e-03 & 0.2883 & 14.5766 & 0.7157 & 3.37e-05 & 0.0768 & 0.1750 \\
            \hline
            TCN-F-45-L-16 & 20.1k & 0.005 & 0.5 & 0.5 & 8.08e-03 & 0.3148 & 37.1117 & 1.3193 & 2.88e-05 & 0.1108 & 0.2619 \\
            TCN-TF-45-L-16 & 76.4k & 0.005 & 1 & 0.1 & 3.32e-03 & 0.1359 & 3.5274 & 0.7486 & 1.02e-06 & 0.0674 & 0.1414 \\
            TCN-TTF-45-L-16 & 27.0k & 0.005 & 1 & 0.1 & 4.93e-03 & 0.2091 & 4.3162 & 0.7804 & 8.67e-07 & 0.0707 & 0.1604 \\
            TCN-TVF-45-L-16 & 22.8k & 0.005 & 5 & 5 & 4.68e-03 & 0.1893 & 5.7824 & 0.7282 & 2.10e-05 & 0.0697 & 0.1440 \\
            \hline
            \hline
            S4-F-S-16 & 8.9k & 0.01 & 1 & 0.1 & 5.68e-03 & 0.2242 & 7.2561 & 0.7921 & 5.93e-05 & 0.0776 & \textbf{0.0995} \\
            S4-TF-S-16 & 30.0k & 0.01 & 10 & 1 & 2.78e-03 & 0.1145 & 3.8495 & 0.7167 & 1.09e-05 & 0.0672 & 0.1593 \\
            S4-TTF-S-16 & 10.2k & 0.01 & 1 & 0.1 & 4.16e-03 & 0.1710 & 4.8166 & 0.7697 & 4.99e-06 & 0.0756 & 0.1685 \\
            S4-TVF-S-16 & 11.6k & 0.01 & 5 & 5 & 3.96e-03 & 0.1525 & 5.0032 & 0.6239 & 4.45e-05 & 0.0670 & 0.1245 \\
            \hline
            S4-F-L-16 & 29.7k & 0.01 & 5 & 5 & 4.23e-03 & 0.1786 & 4.5815 & 0.7637 & 2.02e-05 & 0.0721 & 0.1495 \\
            S4-TF-L-16 & 74.3k & 0.01 & 1 & 0.1 & 3.92e-03 & 0.1778 & \textbf{3.4974} & 0.6139 & 4.56e-06 & 0.0644 & 0.1153 \\
            S4-TTF-L-16 & 34.8k & 0.01 & 5 & 5 & 4.39e-03 & 0.1781 & 3.8163 & 0.7574 & 3.11e-06 & 0.0706 & 0.1292 \\
            S4-TVF-L-16 & 32.4k & 0.01 & 10 & 1 & \textbf{2.62e-03} & \textbf{0.1084} & 3.7071 & \textbf{0.5880} & 1.02e-05 & \textbf{0.0585} & 0.1228 \\
            \hline
            \hline
            GB-C-DIST-MLP & 4.5k & 0.1 (0.01) & 0.5 & 0.5 & 1.77e-02 & 0.6223 & 4.5876 & 1.8688 & 1.74e-05 & 0.0813 & 0.1987 \\
            GB-C-DIST-RNL & 2.3k & 0.1 (1) & 0.5 & 0.5 & 1.75e-02 & 0.6220 & 5.4532 & 1.8288 & 8.63e-06 & 0.0806 & 0.1848 \\
            \hline
            GB-C-FUZZ-MLP & 4.2k & 0.1 (0.01) & 5 & 5 & 1.70e-02 & 0.6005 & 4.6540 & 1.8778 & 2.57e-06 & 0.0868 & 0.2099 \\
            GB-C-FUZZ-RNL & 2.0k & 0.1 (1) & 5 & 5 & 1.74e-02 & 0.6193 & 4.9578 & 1.5210 & 1.19e-05 & 0.0854 & 0.2405 \\
            \hline
            \hline
        \end{tabular}
    }
\end{table*}

% 0.001	0.5	0.5	5.53e-03	0.2178	6.4798	0.6575	2.14e-05	0.0670	0.1319
% 0.001	1	0.1	5.23e-03	0.2097	6.3634	0.6531	3.09e-05	0.0750	0.1340
% 0.001	10	1	4.65e-03	0.1742	4.9881	0.6790	4.34e-05	0.0646	0.1348
% 0.001	0.5	0.5	5.61e-03	0.2240	5.4487	0.6408	2.00e-05	0.0672	0.1354
% 0.005	5	5	6.47e-03	0.2436	17.5463	1.0410	1.43e-05	0.0744	0.1798
% 0.005	10	1	4.13e-03	0.1688	3.7782	0.7006	2.67e-07	0.0686	0.1616
% 0.005	1	0.1	4.67e-03	0.1788	10.3527	0.9590	5.78e-07	0.0764	0.2189
% 0.005	0.5	0.5	7.14e-03	0.2883	14.5766	0.7157	3.37e-05	0.0768	0.1750
% 0.005	0.5	0.5	8.08e-03	0.3148	37.1117	1.3193	2.88e-05	0.1108	0.2619
% 0.005	1	0.1	3.32e-03	0.1359	3.5274	0.7486	1.02e-06	0.0674	0.1414
% 0.005	1	0.1	4.93e-03	0.2091	4.3162	0.7804	8.67e-07	0.0707	0.1604
% 0.005	5	5	4.68e-03	0.1893	5.7824	0.7282	2.10e-05	0.0697	0.1440
% 0.01	1	0.1	5.68e-03	0.2242	7.2561	0.7921	5.93e-05	0.0776	0.0995
% 0.01	10	1	2.78e-03	0.1145	3.8495	0.7167	1.09e-05	0.0672	0.1593
% 0.01	1	0.1	4.16e-03	0.1710	4.8166	0.7697	4.99e-06	0.0756	0.1685
% 0.01	5	5	3.96e-03	0.1525	5.0032	0.6239	4.45e-05	0.0670	0.1245
% 0.01	5	5	4.23e-03	0.1786	4.5815	0.7637	2.02e-05	0.0721	0.1495
% 0.01	1	0.1	3.92e-03	0.1778	3.4974	0.6139	4.56e-06	0.0644	0.1153
% 0.01	5	5	4.39e-03	0.1781	3.8163	0.7574	3.11e-06	0.0706	0.1292
% 0.01	10	1	2.62e-03	0.1084	3.7071	0.5880	1.02e-05	0.0585	0.1228
% 0.1 (0.01)	0.5	0.5	1.77e-02	0.6223	4.5876	1.8688	1.74e-05	0.0813	0.1987
% 0.1 (1)	0.5	0.5	1.75e-02	0.6220	5.4532	1.8288	8.63e-06	0.0806	0.1848
% 0.1 (0.01)	5	5	1.70e-02	0.6005	4.6540	1.8778	2.57e-06	0.0868	0.2099
% 0.1 (1)	5	5	1.74e-02	0.6193	4.9578	1.5210	1.19e-05	0.0854	0.2405
\setlength{\tabcolsep}{4pt}
\renewcommand{\arraystretch}{1.3}
\begin{table*}[h]
    \small
    \caption{
    \textit{Objective metrics for parametric models of \textbf{Multidrive F-Fuzz} fuzz.}
    \textit{Bold indicates best performing models.}
    \textit{Learning rate multiplier for nonlinearity in gray-box models shown in brackets.}
    }
    \label{tab:metrics_od_hb-green-tint}
    \centerline{
        \begin{tabular}{lccccccccccccc}
            \hline
            \midrule
            
            \multirow{2}{*}{Model}
                & \multirow{2}{*}{Params.}
                    & \multirow{2}{*}{LR}
                        & \multicolumn{2}{c}{Weights}
                            & \multicolumn{3}{c}{}
                                & \multicolumn{3}{c}{FAD} \\ 
            \cmidrule(lr){4-5} 
                % \cmidrule(lr){6-8} 
                    \cmidrule(lr){9-12}
            
            &   &   & L1 & MR-STFT & MSE & ESR & MAPE & VGGish & PANN & CLAP & AFx-Rep \\ 

            \hline
            LSTM-C-32 & 5.0k & 0.001 & 5 & 5 & 1.42e-04 & 0.0050 & 6.5954 & 0.0325 & 1.97e-06 & 0.0036 & 0.0041 \\
            LSTM-TVC-32 & 8.0k & 0.001 & 1 & 0.1 & 4.94e-05 & 0.0018 & 11.0315 & 0.0108 & 2.45e-07 & 0.0022 & 0.0019 \\
            \hline
            LSTM-C-96 & 39.7k & 0.001 & 1 & 0.1 & \textbf{4.07e-05} & \textbf{0.0015} & 4.5250 & 0.0119 & 1.70e-07 & 0.0020 & 0.0015 \\
            LSTM-TVC-96 & 4.6k & 0.001 & 5 & 5 & 7.76e-05 & 0.0029 & 18.7066 & 0.0155 & 1.55e-07 & 0.0022 & 0.0016 \\
            \hline
            \hline
            TCN-F-45-S-16 & 15.0k & 0.005 & 1 & 0.1 & 1.63e-03 & 0.0627 & 156.4299 & 0.3258 & 1.17e-05 & 0.0247 & 0.0658 \\
            TCN-TF-45-S-16 & 42.0k & 0.005 & 10 & 1 & 2.99e-04 & 0.0107 & 4.5291 & 0.1596 & 1.75e-05 & 0.0197 & 0.0383 \\
            TCN-TTF-45-S-16 & 17.3k & 0.005 & 10 & 1 & 5.49e-04 & 0.0191 & 7.1355 & 0.1041 & 8.50e-06 & 0.0138 & 0.0112 \\
            TCN-TVF-45-S-16 & 17.7k & 0.005 & 10 & 1 & 5.98e-04 & 0.0212 & 9.9470 & 0.2359 & 1.02e-07 & 0.0166 & 0.0373 \\
            \hline
            TCN-F-45-L-16 & 20.1k & 0.005 & 1 & 0.1 & 1.25e-03 & 0.0467 & 79.8801 & 0.2689 & 1.12e-07 & 0.0257 & 0.0805 \\
            TCN-TF-45-L-16 & 76.4k & 0.005 & 10 & 1 & 1.55e-04 & 0.0055 & 7.1107 & 0.0623 & 6.67e-08 & 0.0072 & 0.0046 \\
            TCN-TTF-45-L-16 & 27.0k & 0.005 & 5 & 5 & 3.63e-03 & 0.1340 & 16.5099 & 0.0890 & 9.66e-07 & 0.0145 & 0.0127 \\
            TCN-TVF-45-L-16 & 22.8k & 0.005 & 5 & 5 & 1.41e-03 & 0.0505 & 6.8365 & 0.1635 & 1.56e-06 & 0.0151 & 0.0117 \\
            \hline
            \hline
            S4-F-S-16 & 8.9k & 0.01 & 1 & 0.1 & 2.55e-03 & 0.0922 & 18.0766 & 0.2028 & 5.39e-08 & 0.0238 & 0.0179 \\
            S4-TF-S-16 & 30.0k & 0.01 & 1 & 0.1 & 2.16e-04 & 0.0079 & 29.8031 & 0.0568 & 7.36e-07 & 0.0049 & 0.0085 \\
            S4-TTF-S-16 & 10.2k & 0.01 & 10 & 1 & 2.15e-04 & 0.0076 & 11.0442 & 0.1101 & 2.20e-05 & 0.0123 & 0.0235 \\
            S4-TVF-S-16 & 11.6k & 0.01 & 1 & 0.1 & 1.45e-04 & 0.0052 & 28.9949 & 0.0550 & 5.23e-07 & 0.0056 & 0.0097 \\
            \hline
            S4-F-L-16 & 29.7k & 0.01 & 5 & 5 & 3.89e-03 & 0.1428 & 7.7912 & 0.0745 & 2.12e-06 & 0.0070 & 0.0068 \\
            S4-TF-L-16 & 74.3k & 0.01 & 10 & 1 & 8.86e-05 & 0.0032 & 4.5734 & 0.0441 & 2.45e-07 & 0.0048 & 0.0041 \\
            S4-TTF-L-16 & 34.8k & 0.01 & 5 & 5 & 2.55e-04 & 0.0090 & 9.2914 & 0.0570 & 9.88e-07 & 0.0049 & 0.0041 \\
            S4-TVF-L-16 & 32.4k & 0.01 & 10 & 1 & 1.44e-04 & 0.0050 & \textbf{2.7653} & 0.0271 & 1.47e-08 & 0.0047 & 0.0030 \\
            \hline
            \hline
            GB-C-DIST-MLP & 4.5k & 0.1 (0.01) & 5 & 5 & 9.26e-03 & 0.3370 & 56.1451 & \textbf{0.5493} & \textbf{1.32e-04} & \textbf{0.0557} & 0.1264 \\
            GB-C-DIST-RNL & 2.3k & 0.1 (1) & 0.5 & 0.5 & 9.73e-03 & 0.3560 & 105.7389 & 0.4071 & 4.94e-06 & 0.0448 & 0.0649 \\
            \hline
            GB-C-FUZZ-MLP & 4.2k & 0.1 (0.01) & 0.5 & 0.5 & 5.93e-03 & 0.2149 & 33.0963 & 0.2608 & 4.51e-05 & 0.0213 & 0.0604 \\
            GB-C-FUZZ-RNL & 2.0k & 0.1 (1) & 0.5 & 0.5 & 5.90e-03 & 0.2143 & 17.0522 & 0.3807 & 5.00e-05 & 0.0546 & \textbf{0.2399} \\
            \hline
            \hline
        \end{tabular}
    }
\end{table*}

% 0.001	5	5	1.42e-04	0.0050	6.5954	0.0325	1.97e-06	0.0036	0.0041
% 0.001	1	0.1	4.94e-05	0.0018	11.0315	0.0108	2.45e-07	0.0022	0.0019
% 0.001	1	0.1	4.07e-05	0.0015	4.5250	0.0119	1.70e-07	0.0020	0.0015
% 0.001	5	5	7.76e-05	0.0029	18.7066	0.0155	1.55e-07	0.0022	0.0016
% 0.005	1	0.1	1.63e-03	0.0627	156.4299	0.3258	1.17e-05	0.0247	0.0658
% 0.005	10	1	2.99e-04	0.0107	4.5291	0.1596	1.75e-05	0.0197	0.0383
% 0.005	10	1	5.49e-04	0.0191	7.1355	0.1041	8.50e-06	0.0138	0.0112
% 0.005	10	1	5.98e-04	0.0212	9.9470	0.2359	1.02e-07	0.0166	0.0373
% 0.005	1	0.1	1.25e-03	0.0467	79.8801	0.2689	1.12e-07	0.0257	0.0805
% 0.005	10	1	1.55e-04	0.0055	7.1107	0.0623	6.67e-08	0.0072	0.0046
% 0.005	5	5	3.63e-03	0.1340	16.5099	0.0890	9.66e-07	0.0145	0.0127
% 0.005	5	5	1.41e-03	0.0505	6.8365	0.1635	1.56e-06	0.0151	0.0117
% 0.01	1	0.1	2.55e-03	0.0922	18.0766	0.2028	5.39e-08	0.0238	0.0179
% 0.01	1	0.1	2.16e-04	0.0079	29.8031	0.0568	7.36e-07	0.0049	0.0085
% 0.01	10	1	2.15e-04	0.0076	11.0442	0.1101	2.20e-05	0.0123	0.0235
% 0.01	1	0.1	1.45e-04	0.0052	28.9949	0.0550	5.23e-07	0.0056	0.0097
% 0.01	5	5	3.89e-03	0.1428	7.7912	0.0745	2.12e-06	0.0070	0.0068
% 0.01	10	1	8.86e-05	0.0032	4.5734	0.0441	2.45e-07	0.0048	0.0041
% 0.01	5	5	2.55e-04	0.0090	9.2914	0.0570	9.88e-07	0.0049	0.0041
% 0.01	10	1	1.44e-04	0.0050	2.7653	0.0271	1.47e-08	0.0047	0.0030
% 0.1	5	5	9.26e-03	0.3370	56.1451	0.5493	1.32e-04	0.0557	0.1264
% 0.1	0.5	0.5	9.73e-03	0.3560	105.7389	0.4071	4.94e-06	0.0448	0.0649
% 0.1	0.5	0.5	5.93e-03	0.2149	33.0963	0.2608	4.51e-05	0.0213	0.0604
% 0.1	0.5	0.5	5.90e-03	0.2143	17.0522	0.3807	5.00e-05	0.0546	0.2399


\end{document}
