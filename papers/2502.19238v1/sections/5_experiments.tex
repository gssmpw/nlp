\section{Experimental Results} \label{sec:experiments}

In this section, we discuss about our implementation specific details, qualitative and quantitative results of proposed algorithms. We provide detailed comparison between our two algorithms for dense and sparse LFs.

\subsection{Datasets}


% \textcolor{red}{make brief introductions to each datasets. What is it, format of availability, any specialities, and number of images/LFs.}
We tested our proposed algorithms on LFs from stanford dataset~\cite{stanfordnewdataset}, HCI dataset~\cite{hcidataset} and EPFL dataset~\cite{epfldataset} to evaluate their performances.
For LFs in EPFL dataset~\cite{epfldataset} acquired using an LF camera, we remove the SAIs at the boarder of the grid, due to vignetting effect. 
The LF dimensions are shown in Table \ref{tab:Datasets}.

\begin{table}[H]
    \centering
    \caption{Dataset details}
    \begin{tabularx}{\linewidth}{ >{\raggedright\arraybackslash}m{2.0cm} |
    >{\centering\arraybackslash}m{2.0cm} |
    >{\centering\arraybackslash}m{2.0cm}
    }
        \noalign{\vskip 1.5pt}
        \hlineB{3}
        \noalign{\vskip 1.5pt}
        % \hlineB{3}
        \textbf{Dataset} & \textbf{SAIs} & \textbf{Resolution} \\ 
        \noalign{\vskip 1.5pt}
        \hlineB{3}
        \noalign{\vskip 1.5pt}
        EPFL~\cite{epfldataset} & $15\times15$ & $434\times625$ \\
        HCI~\cite{hcidataset} & $9\times9$ & $512\times512$ \\
        Stanford~\cite{stanfordnewdataset} & $17\times17$ & $1024\times1024$ \\
        \noalign{\vskip 1.5pt}
        \hlineB{3}
    \end{tabularx}
    
    \label{tab:Datasets}
\end{table}


\subsection{Implementation Details} \label{sec:implementationDetails}


ROIs $\in \mathfrak{R}_{w}$ are divided into $20\times20$ pixels patches $(p^i)$ and find a suitable $\alpha$ value separately. The $\alpha$ value search for sparse LF refocusing algorithm finds the nearest $\alpha$ value to the first decimal point $(\Delta\alpha=0.1)$.
% The $\alpha$ value search for sparse LF refocusing algorithm is implemented as a binary search in \hl{$\alpha$ space} and find the nearest $\alpha$ value to the first decimal point.
Our proposed algorithms are implemented using PyTorch to facilitate from GPU acceleration. Furthermore, the SAIs are shifted to the nearest pixel instead of shifting to sub-pixel level using interpolation to speed-up the process. We empirically found that the quality reduction in this approximation is negligible. 

The U-Net~\cite{unet} which is used for aliasing artifact removal in sparse LF image refocusing, is trained upto 30 epochs with a batch-size of 256. The weights are initialized with Xavier initialization~\cite{xavier} and used RMSProp optimizer. The learning rate is kept at 0.001.
% For each LF, by varying $\alpha$ value from 0.1 to 2.0 with step size of 0.1, 20 input, ground truth image pairs were generated. 
As the network is fully convolutional, inferencing was done to whole image at once. The training and inferencing was done on Google Colaboratory (Tesla T4 GPU and Intel(R) Xeon(R) CPU @ 2.30GHz).
% \textcolor{red}{Mention the GPU used for training as well.}

\subsection{Experimental Results}

\begin{table*}[!t]
    % \vspace{-20cm}
    \centering
    \caption{Dense and sparse LF multi arbitrary-volume refocusing algorithm computational time and refocused images' BRISQUE~\cite{brisque} scores. All the time values are in seconds.}
    % \scriptsize
    \begin{tabularx}{\textwidth}{m{2.5cm} c|c c c c >{\centering\arraybackslash}m{1.1cm}|c c c c c >{\centering\arraybackslash}m{0.8cm} }
        \hlineB{3}
      \noalign{\vskip 1.5pt} 
      \multirow{3}{*}{\textbf{Light Field}} & \multirow{3}{*}{$N$} & \multicolumn{5}{c|}{\textbf{Dense LF refocusing}} & \multicolumn{6}{c}{\textbf{Sparse LF refocussing}} \\ 
      \noalign{\vskip 1.5pt} 
      \cline{3-13}
      \noalign{\vskip 1.5pt} 
      & & $N^{d}_{\alpha}$ &  $T^{d}_{mask}$ & $T^{d}_{refocus}$ &  $T^{d}_{total}$ & BRISQUE value~\cite{brisque} & $N^{s}_{\alpha}$ &  $T^{s}_{mask}$ & $T^{s}_{refocus}$ & $T^{s}_{artifact}$ & $T^{s}_{total}$ & BRISQUE value~\cite{brisque}\\
      \noalign{\vskip 1.5pt} \hlineB{3}
      \noalign{\vskip 1.5pt}
      Lego Knights & 4 & 81 & 8.98 & 10.96 & 19.94 & 44.63 & 66 & 2.37 & 8.99 & 0.35 & 11.71 & 69.11 \\
      Mirablelle Prune Tree & 4 & 19 & 2.87 & 3.27 & 6.15 & 47.58 & 20 & 0.91 & 3.81 & 0.09 & 4.81 & 46.59 \\
      Books & 3 & 33 & 2.86 & 5.68 & 8.54 & 60.30 & 33 & 1.31 & 6.26 & 0.09 & 7.66 & 47.31 \\
      Sideboard & 3 & 34 & 3.23 & 0.40 & 3.64 & 14.96 & 34 & 0.29 & 0.35 & 0.08 & 0.71 & 42.83 \\ \hlineB{3}
    \end{tabularx}
    \label{tab:dense_sparse_resultls}
\end{table*}



% \textcolor{red}{Would prefer these two to be two-column tables with three rows (for Boxes, Prune Tree, Lego) and attributes as columns?}
To generate the results, several ROIs were selected on the middle SAI of LFs. Their coordinates and their visible depth ranges (whether the ROI  $\in \mathfrak{R}_{s}$ or $\in \mathfrak{R}_{w}$) were provided as mentioned in the section \ref{sec:methodology}. 

Table \ref{tab:dense_sparse_resultls} shows the computational time and the refocused images' quality measured using BIRSQUE~\cite{brisque} score of the proposed LF refocusing algorithms.
Here, we use $N^{d}_{\alpha}$, $T^{d}_{mask}$, $T^{d}_{refocus}$, $T^{d}_{total}$ to denote the  $\alpha$ value count, $\mathcal{M}_{\alpha}(n_x,n_y)$ generation time, refocused image generation time and total computational time for dense LF refocusing algorithm, respectively. $N^{s}_{\alpha}$, $T^{s}_{mask}$, $T^{s}_{refocus}$, $T^{s}_{total}$ are the counterparts for those in sparse LF refocusing algorithm while $T^{s}_{artifact}$ denotes the time for aliasing artifact removal in sparse LF refocusing algorithm.
% by running on Google Colaboratory (Tesla T4 GPU and Intel(R) Xeon(R) CPU @ 2.30GHz).
$T^{d}_{total}$ and $T^{s}_{total}$ depends on the size (SAI count and resolution) of the LF, $N$, ROI sizes and types ($\in \mathfrak{R}_{w}$ or $\in \mathfrak{R}_{s}$). Lego Knights LF from stanford dataset~\cite{stanfordnewdataset} has has the highest number of SAIs with highest resolution. Therefore, it requires the highest computational time among all tested LFs.  The number of individual refocusings are equal to $N^{d}_{\alpha}$ or $N^{s}_{\alpha}$, and that is quite high due to applying Gaussian filter on $\mathcal{M}_{\alpha}(n_x,n_y)$ for smoothing the transition between regions with different $\alpha$ values. If smooth transition between different regions in $\mathcal{I}_f$ is not applied, $N^{d}_{\alpha}$ and $N^{s}_{\alpha}$ will be limited only to $N$. Therefore, by varying the smoothness according to the user preference, computational time can be reduced. Furthermore, the proposed algorithms utilize the GPU support, significantly reducing the computational time for a single refocusing. As an example, for Lego Knights dense LF, 81 different refocusing requires only 11 seconds limiting single refocusing to less than 0.2 seconds.  For dense LF refocusing algorithm, $T^{d}_{mask}$ is more than $30\%$ of $T^{d}_{total}$. However, $D(n_x,n_y)$ generation is a one time process. Therefore, in applications with dense LFs, $D(n_x,n_y)$ can be generated during setup time and during the processing time, $\mathcal{M}_{\alpha}(n_x,n_y)$ can be generated quickly, reducing the $T^{d}_{mask}$ significantly. The qualitative results are shown in the \figurename~\ref{fig:qualitative}. While dense LF refocusing algorithm perform quite well, the $\mathcal{I}_f$ from sparse LF refocusing algorithm shows some aliasing artifact in Mirabelle Prune Tree LF (\figurename~\ref{fig:mirabelle_prune_tree}). Also there is a slight difference in color and brightness in $\mathcal{I}_f$ of sparse LF refocusing algorithm due to the less number of SAIs and the image restoration CNN. We believe that, these effects can be minimized by training the image restoration CNN further with larger datasets and better augmentations.

\begin{table}[!t]
    \centering
    \caption{Measure refocused image's quality and the computational time of sparse LF refocusing algorithm with respect to that of dense LF refocusing algorithm.}
    \begin{tabular}{l|c >{\centering\arraybackslash}m{1.5cm} l l}
        \hlineB{3}
        \noalign{\vskip 1.5pt}
        \textbf{Light Field}  &  \textbf{SSIM~\cite{ssim}}    &  \textbf{PSNR} & \textbf{Duration}\\ 
        \noalign{\vskip 1.5pt}
        \hlineB{3}
        \noalign{\vskip 1.5pt}
        Lego Knights          & 0.7738  &  24.4413 & 58.7\%\\ 
        Mirablelle Prune Tree & 0.9071  &  24.5378 & 78.2\%\\ 
        Books                 & 0.9342  &  27.9272 & 89.7\%\\ 
        Sideboard             & 0.9230  &  27.8050 & 19.5\%\\ 
        \hlineB{3}
    \end{tabular}
    \label{tab:denseVsSparseCompare}
\end{table}

To evaluate the sparse LF refocusing algorithm, we compared the refocused image quality and the computational time of sparse LF refocusing algorithm with respect to those of dense LF refocusing algorithms, as shown in Table \ref{tab:denseVsSparseCompare}. SSIM value for Lego Knights LF from Standford dataset~\cite{stanfordnewdataset} (captured using a camera array) is relatively low as its SAIs have a larger baseline than LFs from lenselet-based camera, whereas other LFs have SSIM~\cite{ssim} values higher than 0.9. Note that, this much of quality is achieved in sparse LF refocusing algorithm by using less than $20\%$ of SAIs of dense LFs. Furthermore, sparse LF refocusing algorithm takes less computational time than dense LF refocusing algorithm. In the presented LFs and ROIs here, $T^{s}_{total}$ always less than $90\%$ of $T^{d}_{total}$. This is mainly due to the fact that the $\alpha$ search for sparse LFs is much faster than $D(n_x,n_y)$ generation for dense LFs during $\mathcal{M}_{\alpha}(n_x,n_y)$ generation. 

\begin{figure*}[!p]
    \centering
    
    \begin{subfigure}[c]{0.29\textwidth}
         \centering
        %  \includegraphics[width=\textwidth]{}
         \caption*{Middle SAI}
        %  \label{fig:y equals x}
     \end{subfigure}
     \hspace{0.1cm}
    \begin{subfigure}[c]{0.29\textwidth}
         \centering
        %  \includegraphics[width=\textwidth]{}
         \caption*{\centering Dense LF multi arbitrary-volume refocused image}
        %  \label{fig:y equals x}
     \end{subfigure}
     \hspace{0.1cm}
     \begin{subfigure}[c]{0.29\textwidth}
         \centering
        %  \includegraphics[width=\textwidth]{}
         \caption*{\centering Sparse LF multi arbitrary-volume refocused image}
        %  \label{fig:three sin x}
     \end{subfigure}

    \begin{subfigure}[t]{0.29\textwidth}
         \centering
         \includegraphics[width=\textwidth]{images/middle_SAIs/lego_knights_middle_SAI.jpg}
        %  \caption{\small Evetar Lens M13B0618W}
        %  \label{fig:y equals x}
     \end{subfigure}
     \hspace{0.1cm}
    \begin{subfigure}[t]{0.29\textwidth}
         \centering
         \includegraphics[width=\textwidth]{images/dense_multi_refocus/lego_knights_multi_refocusd_img.jpg}
         \caption{}
        %  \label{fig:y equals x}
     \end{subfigure}
     \hspace{0.1cm}
     \begin{subfigure}[t]{0.29\textwidth}
         \centering
         \includegraphics[width=\textwidth]{images/sparse_multi_refocus/lego_knights_multi_refocusd_img.jpg}
        %  \caption{ Basler daA1920-30uc (S-Mount)}
        %  \label{fig:three sin x}
     \end{subfigure}
     
    \vspace{0.2cm}

% \end{figure*}
% \begin{figure*}[]\ContinuedFloat
     
    \begin{subfigure}[t]{0.29\textwidth}
         \centering
         \includegraphics[width=\textwidth]{images/middle_SAIs/Mirabelle_Prune_Tree_middle_SAI.jpg}
        %  \caption{\small Evetar Lens M13B0618W}
        %  \label{fig:y equals x}
     \end{subfigure}
     \hspace{0.1cm}
    \begin{subfigure}[t]{0.29\textwidth}
         \centering
         \includegraphics[width=\textwidth]{images/dense_multi_refocus/Mirabelle_Prune_Tree_multi_refocusd_img.jpg}
         \caption{}
         \label{fig:mirabelle_prune_tree}
     \end{subfigure}
     \hspace{0.1cm}
     \begin{subfigure}[t]{0.29\textwidth}
         \centering
         \includegraphics[width=\textwidth]{images/sparse_multi_refocus/Mirabelle_Prune_Tree_multi_refocusd_img.jpg}
        %  \caption{ Basler daA1920-30uc (S-Mount)}
        %  \label{fig:three sin x}
     \end{subfigure}
    \vspace{.2cm}
     
    \begin{subfigure}[t]{0.29\textwidth}
         \centering
         \includegraphics[width=\textwidth]{images/middle_SAIs/Books_middle_SAI.jpg}
        %  \caption{\small Evetar Lens M13B0618W}
        %  \label{fig:y equals x}
     \end{subfigure}
     \hspace{0.1cm}
    \begin{subfigure}[t]{0.29\textwidth}
         \centering
         \includegraphics[width=\textwidth]{images/dense_multi_refocus/Books_multi_refocusd_img.jpg}
         \caption{}
        %  \label{fig:y equals x}
     \end{subfigure}
     \hspace{0.1cm}
     \begin{subfigure}[t]{0.29\textwidth}
         \centering
         \includegraphics[width=\textwidth]{images/sparse_multi_refocus/Books_multi_refocusd_img.jpg}
        %  \caption{ Basler daA1920-30uc (S-Mount)}
        %  \label{fig:three sin x}
     \end{subfigure}
     
    \vspace{.2cm}
     
    \begin{subfigure}[t]{0.29\textwidth}
         \centering
         \includegraphics[width=\textwidth]{images/middle_SAIs/sideboard_middle_SAI.jpg}
        %  \caption{\small Evetar Lens M13B0618W}
        %  \label{fig:y equals x}
     \end{subfigure}
     \hspace{0.1cm}
    \begin{subfigure}[t]{0.29\textwidth}
         \centering
         \includegraphics[width=\textwidth]{images/dense_multi_refocus/sideboard_multi_refocusd_img.jpg}
         \caption{}
        %  \label{fig:y equals x}
     \end{subfigure}
     \hspace{0.1cm}
     \begin{subfigure}[t]{0.29\textwidth}
         \centering
         \includegraphics[width=\textwidth]{images/sparse_multi_refocus/sideboard_multi_refocusd_img.jpg}
        %  \caption{ Basler daA1920-30uc (S-Mount)}
        %  \label{fig:three sin x}
     \end{subfigure}
     
     
    \caption{Multi arbitrary-volume refocusing qualitative results. (a) Lego Knights LF from Stanford LF dataset~\cite{stanfordnewdataset}, (b) Mirabelle Prune Tree, (c) Books LFs from EPFL LF dataset~\cite{epfldataset}, (d) Sideboard LF from HCI LF dataset~\cite{hcidataset}. ROIs in green color boxes $\in \mathfrak{R}_{s}$ and ROIs in red color boxes $\in \mathfrak{R}_{w}$.}
    \label{fig:qualitative}
\end{figure*}

%%%%%%%%%%%%%%%%%%%%%%%%%%%%%%%%%%%%%%%%%%%%%%
% We use the BRISQUE no reference image quality assessment method \cite{brisque} to evaluate the effectiveness of the image restoration CNN used for sparse LF multi-volume refocus algorithm. Unexpectedly, as shown in table \ref{tab:ablation}, the quality of the refocused image is reduced after the CNN. That is mainly due to the reduction of sharp lines due to aliasing artifact and increase the blurriness in out-of-focus regions. But the improvement of quality is visible as shown in figure \ref{fig:denseSparsPostProcess}. 


% \begin{table}[!h]
%     \small
%     \centering
%     \caption{Comparison of refocused image quality before and after image restoration CNN.}
%     \begin{tabular}{m{0.35\columnwidth}|p{0.2\columnwidth}|p{0.2\columnwidth}}
%         \hline
%         \multirow{2}{*}{} & \multicolumn{2}{c}{BRISQUE value} \\ \hhline{~|--}
%          LF name & before aliasing removal  & after aliasing removal\\ \hline
%         Mirabelle Prune Tree  & \hl{28.34} & \hl{46.43} \\
%         Sideboard & \hl{27.36} & \hl{51.20}\\
%         Lego Knights & \hl{50.98} & \hl{71.95}\\
        
%     \end{tabular}
%     \label{tab:ablation}
% \end{table}