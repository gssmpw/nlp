\section{Review of Light Field Refocusing using the Shift-and-Sum Algorithm}
\label{sec:shiftSumReview}
We briefly review the shift-and-sum algorithm~\cite{ng2005light} employed for LF refocusing in this section. To this end, we first consider the two-plane parameterization~\cite{levoy1996light} of a continuous-domain LF $L_c(u,v,x,y)$ as shown in Fig~\ref{fig:refocus}, where $(u,v,x,y)\in\mathbb{R}^4$. Here, the $uv$ plane and the $xy$ plane denote the camera plane and the image plane of an LF camera, respectively. The distance between the $uv$ and $xy$ planes is $F$. In generating a focused 2-D image using the captured LF $L_c(u,v,x,y)$, the intensity $\mathcal{I}_c(x,y)$ of the 2-D image at the image plane can be expressed as~\cite{ng2005light}
\begin{align}
\label{eq:LF_intensity}
\mathcal{I}_c(x,y) =\frac{1}{F^{2}}\iint L_c(u,v,x,y) A(u,v)\cos^{4}(\theta )\: \mathrm{d}u\mathrm{d}v,
\end{align}
where $A(u,v)$ is the aperture function (unity inside and zero outside the aperture). $\mathcal{I}_c(x,y)$ represents a 2-D image focused at a depth $F$, and the depth range corresponding to focused region is determined by the size of $A(u,v)$, where a larger aperture leads to a narrower depth range. The refocusing of the LF $L_c(u,v,x,y)$ is equivalent to computing the intensity $\overline{\mathcal{I}_c}(x',y')$ of a 2-D image with a shifted image plane $(x',y')$ located at a distance $\alpha F$ from the camera plane $uv$. Note that, when $\alpha>1$, the LF is refocused to distant depths than the original focused depth, and when $\alpha<1$, the LF is refocused to near depths than the original focused depth. The intensity $\overline{\mathcal{I}_c}(x',y')$ of the refocused 2-D image can be computed as~\cite{ng2005light}
\begin{align}
\label{eq:LF_refocus}
\overline{\mathcal{I}_c}\left( x',y'\right) =\iint L_c\left( u,v,u+\frac{x' -u}{\alpha } ,v+\frac{y' -v}{\alpha }\right) \: \mathrm{d}u\mathrm{d}v,
\end{align}
where, the constant scaling factor $\frac{1}{F^{2}}$ is ignored, and integration is limited $A(u,v)=1$. Note that, the pixels of an SAI of the LF $L( u_0,v_0,x,y)$ at $(u,v)=(u_0,v_0)$ need to be shifted by $(u_0+\frac{x' -u_0}{\alpha})-x'$ and $(v_0+\frac{y' -v_0}{\alpha})-y'$ pixels before integrating with respect to to $(x',y')$, and the shifts are determined by $\alpha$ that depends on the depth of the refocused plane. We now consider a discrete-domain LF $L(n_u,n_v,n_x,n_y)=L_c(n_u\Delta u,n_v\Delta v,n_x\Delta x,n_y\Delta y)$, where $(n_u,n_v,n_x,n_y)\in\mathbb{Z}^4$ and $\Delta i$ ($i=u,v,x,y$) is the sampling interval corresponding to the dimension $i$. In this case, $\overline{\mathcal{I}}\left( n_x' ,n_y'\right)$  can be expressed for a dense LF consisting of $N_u\times N_v$ SAIs, from~\eqref{eq:LF_refocus}, as~\cite{ng2005light}
\begin{align}
\label{eq:LF_refocus_discrete}
\overline{\mathcal{I}}\left( n_x' ,n_y'\right) =&\sum_{n_u=1}^{N_u}\sum_{n_v=1}^{N_v} L\left( n_u,n_v,\frac{n_x'+(\alpha-1) n_u}{\alpha}, \right. \notag \\ 
& \left. \hspace{3cm} \frac{n_y'+(\alpha-1)n_v}{\alpha}\right).
\end{align}
Note that the factor $1/\alpha$ can be ignored because it just scale the refocused LF. This simplifies \eqref{eq:LF_refocus_discrete} to   
\begin{align}
\label{eq:LF_refocus_simple}
\overline{\mathcal{I}}\left( n_x' ,n_y'\right) =&\sum_{n_u=1}^{N_u}\sum_{n_v=1}^{N_v} L\left( n_u,n_v,n_x'+(\alpha-1) n_u, \right. \notag \\ 
& \left. \hspace{3cm} n_y'+(\alpha-1)n_v\right),
\end{align}
where refocused 2-D image is generated by \emph{shifting and summing} the SAIs of the LF. 
%constant Following \cite{ng2005light}, the original LF refocusing algorithm can be derived by substituting $(x^{'},y^{'})$ for $(x,y)$ considering geometry and ignoring $\frac{1}{F^{2}}$ constant. The new refocus plane is parameterized by $\alpha$ as shown in the \figurename~\ref{fig:refocus}. The intensity of refocused image $\overline{E}( x^{'},y^{'})$ can be calculated from original LF as shown in the Eq. \ref{eq:LF_refocus}. When $\alpha<1$, the LF will be refocused to distant depths and when $\alpha>1$, LF will be refocused to near depths.
 



%This equation can be further simplified by removing the $1/\alpha$ term to remove the scaling artefact due to the refocusing. By approximating integral by summation, we get the final simplified shift-and-sum LF refocusing equation (Eq.\ref{eq:LF_refocus_simple}).



\begin{figure}[!t]
    \centering
    \includegraphics[width=0.48\textwidth]{images/LF_dimensions.pdf}
    \caption{The two-plane parameterization of an LF. The $uv$ plane is called the camera plane, and the $xy$ plane is called the image plane. The distance between the two planes are $F$, and the scene at a distance $F$ is in focus while the rest are out of focus. The image plane is moved artificially to the $x'y'$ plane (called refocused plane) to refocus to a new depth $\alpha F$.}
    %the principal plane of the lens. The $xy$ plane is the sensor plane. Sensor plane is moved artificially to the $x'y'$ plane which is parameterized by $\alpha$, to refocus to a new depth.
    \label{fig:refocus}
\end{figure}

% Here, $(\alpha-1)$ is equal to the slope of the epipolar lines $d_{x,u}$ and $d_{y,v}$  which goes through $(u,x)$ and $(v,y)$ respectively \cite{wanner2013variational}. Hence it is equal to the disparity of point $(x,y)$ as shown in the equation \ref{eq:disparity}.


% \begin{equation}\label{eq:disparity}
% d_{x,u} =d_{y,v} =\frac{\Delta x}{\Delta u} =\frac{\Delta y}{\Delta v}=\alpha -1
% \end{equation}
