Data diversity is crucial for the instruction tuning of large language models. 
Existing studies have explored various diversity-aware data selection methods to construct high-quality datasets and enhance model performance. 
However, the fundamental problem of precisely defining and measuring data diversity remains underexplored, limiting clear guidance for data engineering. 
To address this, we systematically analyze 11 existing diversity measurement methods by evaluating their correlation with model performance through extensive fine-tuning experiments. 
Our results indicate that a reliable diversity measure should properly account for both inter-sample differences and the information distribution in the sample space.
Building on this, we propose \textit{NovelSum}, a new diversity metric based on sample-level "novelty." 
Experiments on both simulated and real-world data show that \textit{NovelSum} accurately captures diversity variations and achieves a 0.97 correlation with instruction-tuned model performance, highlighting its value in guiding data engineering practices. 
With \textit{NovelSum} as an optimization objective, we further develop a greedy, diversity-oriented data selection strategy that outperforms existing approaches, validating both the effectiveness and practical significance of our metric.

