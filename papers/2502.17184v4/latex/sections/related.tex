\section{Related Work}
\paragraph{Measuring Dataset Diversity}


Dataset diversity is essential for training generalizable machine learning models, drawing significant research interest \cite{sun2024diversity-CV, zhao2024measuring, qin2024unleashing}. In NLP, numerous lexical diversity metrics have been proposed to measure text diversity through vocabulary usage \cite{richards1987type-TTR, malvern2004lexical-vocd}. Recently, semantic embeddings have enabled more flexible diversity measurement from distance \cite{stasaski2022semantic-KNN, du2019boosting-Inertia, dang2024data} or distribution perspectives \cite{shao2024balanced}. Focusing on instruction tuning, while some studies have explored the assessment of IT data diversity \cite{wang2024diversity-logD, bukharin2023data-QDIT}, the proposed metrics lack sufficient validation of their correlation with IT performance; thus, reliable metrics for guiding data engineering remain underexplored.

\paragraph{Data Selection for Instruction Tuning}
Instruction tuning trains LLMs to follow human instructions using instruction-response pairs \cite{zhang2023instruction}. While earlier work focused on large-scale IT datasets \cite{longpre2023flan, chiang2023vicuna-ShareGPT}, recent studies show that small, high-quality data sets can reduce costs and improve performance \cite{chen2023maybe-Kcentergreedy, zhou2024lima, dou2024loramoe, ye2024empirical}. This has led to the development of data selection strategies to identify subsets that boost IT performance \cite{liu2023makes, du2023mods-Kcentergreedy, wu2023self-Kcentergreedy, song2024iterselecttune-Kmeans, yang2024beyond}. However, the lack of clear definitions and reliable diversity metrics for IT datasets hinders effective optimization. Consequently, some selection methods fail to generalize or perform worse than random selection \cite{xia2024rethinking,diddee2024chasing}. Our work seeks to provide a more reliable diversity metric, based on comprehensive analysis, that accurately reflects the diversity of IT datasets and their instruction tuning performance.
