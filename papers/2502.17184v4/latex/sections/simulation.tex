\section{Simulation Study}
\begin{figure*}[!t]
    \centering
        \includegraphics[width=\linewidth]{latex/figures/simulation_p.pdf}
    \caption{Simulating data selection in a 2D sample space: Selection A represents datasets with redundancy, Selection B optimizes inter-sample distances, and Selection C accounts for both distances and density, which prior analysis suggests yields the highest diversity.}
    \label{fig:sim_points}
    \vspace{-4mm}
\end{figure*}

\begin{figure}[!t]
    \centering
        \includegraphics[width=\linewidth]{latex/figures/simulation_subplots.pdf}
    \caption{Measuring the diversity of simulated selection A/B/C with various metrics. \textit{NovelSum} accurately captures dataset diversity, exhibiting expected behaviors.}
    \label{fig:sim_res}
    \vspace{-2mm}
\end{figure}


To validate whether the proposed metric aligns with our design principles and accurately captures dataset diversity, we create a visualizable simulation environment. We generate 150 points in 2D space as the data source and select 20 samples to form a dataset, simulating the data selection process for instruction tuning. As shown in Figure \ref{fig:sim_points}, we analyze three data selection scenarios to examine the behavior of our diversity metric. "Selection A" contains samples from two clusters, with most points close to each other, simulating datasets with redundancy. "Selection B", constructed using K-Center-Greedy, consists of samples far apart, simulating datasets optimized for inter-sample semantic distances. "Selection C" considers both inter-sample distances and information density, simulating datasets that best represent the sample space with unique points. Based on prior analysis, the dataset diversity of the three selections should follow $A<B<C$ order intuitively.

Figure \ref{fig:sim_res} presents the diversity measurement results using DistSum, a proximity-weighted version of DistSum, and \textit{NovelSum}. From left to right, we see that DistSum counterintuitively considers $\mathcal{M}(A) \simeq \mathcal{M}(C)$, failing to reflect sample uniqueness. Incorporating the proximity-weighted sum improves uniqueness capture but still exhibits $\mathcal{M}(B) > \mathcal{M}(C)$, overlooking information density. \textbf{\textit{NovelSum} resolves these issues, accurately capturing diversity variations in alignment with design principles}, yielding $\mathcal{M}(A) < \mathcal{M}(B) < \mathcal{M}(C)$. This study further validates the necessity of the proximity-weighted sum and density-aware distance for precise diversity measurement.





