\section{Related Work}
IC3 was originally introduced as a SAT-based bit-level model checking algorithm. Since its inception, numerous tools have been developed based on this foundational approach. \textbf{IC3ref} \cite{IC3ref}, created by the algorithm’s inventor, stands out for its efficiency and relatively simple implementation, making it a baseline for many IC3-related studies \cite{DeepIC3,Progress,PredictingLemma,IGoodLemma}. The \textbf{ABC} framework \cite{ABC} includes an implementation of the PDR algorithm \cite{PDR}, while the \textbf{nuXmv} model checker \cite{NUXMV} also integrates IC3. Additionally, \textbf{SimpleCAR} \cite{SimpleCAR} implements the CAR algorithm \cite{CAR}, which excels at bug detection, and \textbf{Avy} \cite{Avy} combines sequence interpolants with IC3.

The IC3 algorithm has also been extended to address word-level problems by leveraging SMT solvers. \textbf{AVR} \cite{AVR} implements IC3sa \cite{IC3SA}, combining IC3 with syntax-guided abstraction to enable scalable word-level model checking. Furthermore, \textbf{Pono} \cite{PONO} supports both IC3sa and IC3ia \cite{IC3IA}, which extends IC3 to modulo theories through implicit predicate abstraction.

The Hardware Model Checking Competition (HWMCC) \cite{HWMCC} has been instrumental in driving advancements in hardware model checkers. Submissions to the competition are typically executed in a 16-thread portfolio mode, prompting many model checkers to develop portfolio versions that integrate various configurations of IC3, BMC, and K-Induction engines. Examples include \textbf{Superprove} \cite{SuperProve} for ABC, \textbf{SuperCAR} \cite{SuperCAR} for SimpleCAR, and \textbf{Pavy} \cite{Pavy} for Avy. AVR and Pono also have their own portfolio versions.

\textbf{rIC3}, like these tools, is used for hardware model checking. It incorporates several optimization techniques proposed in recent years to achieve better scalability, delivering superior performance compared to existing tools. It can also be used to verify industrial models through SymbiYosys. Moreover, it is implemented in the modern programming language Rust, featuring a modular and concise codebase, making it an excellent platform for academic research as well.