
\documentclass{article} % For LaTeX2e
\usepackage{iclr2025_conference,times}

% Optional math commands from https://github.com/goodfeli/dlbook_notation.
%%%%% NEW MATH DEFINITIONS %%%%%

\usepackage{amsmath,amsfonts,bm}
\usepackage{derivative}
% Mark sections of captions for referring to divisions of figures
\newcommand{\figleft}{{\em (Left)}}
\newcommand{\figcenter}{{\em (Center)}}
\newcommand{\figright}{{\em (Right)}}
\newcommand{\figtop}{{\em (Top)}}
\newcommand{\figbottom}{{\em (Bottom)}}
\newcommand{\captiona}{{\em (a)}}
\newcommand{\captionb}{{\em (b)}}
\newcommand{\captionc}{{\em (c)}}
\newcommand{\captiond}{{\em (d)}}

% Highlight a newly defined term
\newcommand{\newterm}[1]{{\bf #1}}

% Derivative d 
\newcommand{\deriv}{{\mathrm{d}}}

% Figure reference, lower-case.
\def\figref#1{figure~\ref{#1}}
% Figure reference, capital. For start of sentence
\def\Figref#1{Figure~\ref{#1}}
\def\twofigref#1#2{figures \ref{#1} and \ref{#2}}
\def\quadfigref#1#2#3#4{figures \ref{#1}, \ref{#2}, \ref{#3} and \ref{#4}}
% Section reference, lower-case.
\def\secref#1{section~\ref{#1}}
% Section reference, capital.
\def\Secref#1{Section~\ref{#1}}
% Reference to two sections.
\def\twosecrefs#1#2{sections \ref{#1} and \ref{#2}}
% Reference to three sections.
\def\secrefs#1#2#3{sections \ref{#1}, \ref{#2} and \ref{#3}}
% Reference to an equation, lower-case.
\def\eqref#1{equation~\ref{#1}}
% Reference to an equation, upper case
\def\Eqref#1{Equation~\ref{#1}}
% A raw reference to an equation---avoid using if possible
\def\plaineqref#1{\ref{#1}}
% Reference to a chapter, lower-case.
\def\chapref#1{chapter~\ref{#1}}
% Reference to an equation, upper case.
\def\Chapref#1{Chapter~\ref{#1}}
% Reference to a range of chapters
\def\rangechapref#1#2{chapters\ref{#1}--\ref{#2}}
% Reference to an algorithm, lower-case.
\def\algref#1{algorithm~\ref{#1}}
% Reference to an algorithm, upper case.
\def\Algref#1{Algorithm~\ref{#1}}
\def\twoalgref#1#2{algorithms \ref{#1} and \ref{#2}}
\def\Twoalgref#1#2{Algorithms \ref{#1} and \ref{#2}}
% Reference to a part, lower case
\def\partref#1{part~\ref{#1}}
% Reference to a part, upper case
\def\Partref#1{Part~\ref{#1}}
\def\twopartref#1#2{parts \ref{#1} and \ref{#2}}

\def\ceil#1{\lceil #1 \rceil}
\def\floor#1{\lfloor #1 \rfloor}
\def\1{\bm{1}}
\newcommand{\train}{\mathcal{D}}
\newcommand{\valid}{\mathcal{D_{\mathrm{valid}}}}
\newcommand{\test}{\mathcal{D_{\mathrm{test}}}}

\def\eps{{\epsilon}}


% Random variables
\def\reta{{\textnormal{$\eta$}}}
\def\ra{{\textnormal{a}}}
\def\rb{{\textnormal{b}}}
\def\rc{{\textnormal{c}}}
\def\rd{{\textnormal{d}}}
\def\re{{\textnormal{e}}}
\def\rf{{\textnormal{f}}}
\def\rg{{\textnormal{g}}}
\def\rh{{\textnormal{h}}}
\def\ri{{\textnormal{i}}}
\def\rj{{\textnormal{j}}}
\def\rk{{\textnormal{k}}}
\def\rl{{\textnormal{l}}}
% rm is already a command, just don't name any random variables m
\def\rn{{\textnormal{n}}}
\def\ro{{\textnormal{o}}}
\def\rp{{\textnormal{p}}}
\def\rq{{\textnormal{q}}}
\def\rr{{\textnormal{r}}}
\def\rs{{\textnormal{s}}}
\def\rt{{\textnormal{t}}}
\def\ru{{\textnormal{u}}}
\def\rv{{\textnormal{v}}}
\def\rw{{\textnormal{w}}}
\def\rx{{\textnormal{x}}}
\def\ry{{\textnormal{y}}}
\def\rz{{\textnormal{z}}}

% Random vectors
\def\rvepsilon{{\mathbf{\epsilon}}}
\def\rvphi{{\mathbf{\phi}}}
\def\rvtheta{{\mathbf{\theta}}}
\def\rva{{\mathbf{a}}}
\def\rvb{{\mathbf{b}}}
\def\rvc{{\mathbf{c}}}
\def\rvd{{\mathbf{d}}}
\def\rve{{\mathbf{e}}}
\def\rvf{{\mathbf{f}}}
\def\rvg{{\mathbf{g}}}
\def\rvh{{\mathbf{h}}}
\def\rvu{{\mathbf{i}}}
\def\rvj{{\mathbf{j}}}
\def\rvk{{\mathbf{k}}}
\def\rvl{{\mathbf{l}}}
\def\rvm{{\mathbf{m}}}
\def\rvn{{\mathbf{n}}}
\def\rvo{{\mathbf{o}}}
\def\rvp{{\mathbf{p}}}
\def\rvq{{\mathbf{q}}}
\def\rvr{{\mathbf{r}}}
\def\rvs{{\mathbf{s}}}
\def\rvt{{\mathbf{t}}}
\def\rvu{{\mathbf{u}}}
\def\rvv{{\mathbf{v}}}
\def\rvw{{\mathbf{w}}}
\def\rvx{{\mathbf{x}}}
\def\rvy{{\mathbf{y}}}
\def\rvz{{\mathbf{z}}}

% Elements of random vectors
\def\erva{{\textnormal{a}}}
\def\ervb{{\textnormal{b}}}
\def\ervc{{\textnormal{c}}}
\def\ervd{{\textnormal{d}}}
\def\erve{{\textnormal{e}}}
\def\ervf{{\textnormal{f}}}
\def\ervg{{\textnormal{g}}}
\def\ervh{{\textnormal{h}}}
\def\ervi{{\textnormal{i}}}
\def\ervj{{\textnormal{j}}}
\def\ervk{{\textnormal{k}}}
\def\ervl{{\textnormal{l}}}
\def\ervm{{\textnormal{m}}}
\def\ervn{{\textnormal{n}}}
\def\ervo{{\textnormal{o}}}
\def\ervp{{\textnormal{p}}}
\def\ervq{{\textnormal{q}}}
\def\ervr{{\textnormal{r}}}
\def\ervs{{\textnormal{s}}}
\def\ervt{{\textnormal{t}}}
\def\ervu{{\textnormal{u}}}
\def\ervv{{\textnormal{v}}}
\def\ervw{{\textnormal{w}}}
\def\ervx{{\textnormal{x}}}
\def\ervy{{\textnormal{y}}}
\def\ervz{{\textnormal{z}}}

% Random matrices
\def\rmA{{\mathbf{A}}}
\def\rmB{{\mathbf{B}}}
\def\rmC{{\mathbf{C}}}
\def\rmD{{\mathbf{D}}}
\def\rmE{{\mathbf{E}}}
\def\rmF{{\mathbf{F}}}
\def\rmG{{\mathbf{G}}}
\def\rmH{{\mathbf{H}}}
\def\rmI{{\mathbf{I}}}
\def\rmJ{{\mathbf{J}}}
\def\rmK{{\mathbf{K}}}
\def\rmL{{\mathbf{L}}}
\def\rmM{{\mathbf{M}}}
\def\rmN{{\mathbf{N}}}
\def\rmO{{\mathbf{O}}}
\def\rmP{{\mathbf{P}}}
\def\rmQ{{\mathbf{Q}}}
\def\rmR{{\mathbf{R}}}
\def\rmS{{\mathbf{S}}}
\def\rmT{{\mathbf{T}}}
\def\rmU{{\mathbf{U}}}
\def\rmV{{\mathbf{V}}}
\def\rmW{{\mathbf{W}}}
\def\rmX{{\mathbf{X}}}
\def\rmY{{\mathbf{Y}}}
\def\rmZ{{\mathbf{Z}}}

% Elements of random matrices
\def\ermA{{\textnormal{A}}}
\def\ermB{{\textnormal{B}}}
\def\ermC{{\textnormal{C}}}
\def\ermD{{\textnormal{D}}}
\def\ermE{{\textnormal{E}}}
\def\ermF{{\textnormal{F}}}
\def\ermG{{\textnormal{G}}}
\def\ermH{{\textnormal{H}}}
\def\ermI{{\textnormal{I}}}
\def\ermJ{{\textnormal{J}}}
\def\ermK{{\textnormal{K}}}
\def\ermL{{\textnormal{L}}}
\def\ermM{{\textnormal{M}}}
\def\ermN{{\textnormal{N}}}
\def\ermO{{\textnormal{O}}}
\def\ermP{{\textnormal{P}}}
\def\ermQ{{\textnormal{Q}}}
\def\ermR{{\textnormal{R}}}
\def\ermS{{\textnormal{S}}}
\def\ermT{{\textnormal{T}}}
\def\ermU{{\textnormal{U}}}
\def\ermV{{\textnormal{V}}}
\def\ermW{{\textnormal{W}}}
\def\ermX{{\textnormal{X}}}
\def\ermY{{\textnormal{Y}}}
\def\ermZ{{\textnormal{Z}}}

% Vectors
\def\vzero{{\bm{0}}}
\def\vone{{\bm{1}}}
\def\vmu{{\bm{\mu}}}
\def\vtheta{{\bm{\theta}}}
\def\vphi{{\bm{\phi}}}
\def\va{{\bm{a}}}
\def\vb{{\bm{b}}}
\def\vc{{\bm{c}}}
\def\vd{{\bm{d}}}
\def\ve{{\bm{e}}}
\def\vf{{\bm{f}}}
\def\vg{{\bm{g}}}
\def\vh{{\bm{h}}}
\def\vi{{\bm{i}}}
\def\vj{{\bm{j}}}
\def\vk{{\bm{k}}}
\def\vl{{\bm{l}}}
\def\vm{{\bm{m}}}
\def\vn{{\bm{n}}}
\def\vo{{\bm{o}}}
\def\vp{{\bm{p}}}
\def\vq{{\bm{q}}}
\def\vr{{\bm{r}}}
\def\vs{{\bm{s}}}
\def\vt{{\bm{t}}}
\def\vu{{\bm{u}}}
\def\vv{{\bm{v}}}
\def\vw{{\bm{w}}}
\def\vx{{\bm{x}}}
\def\vy{{\bm{y}}}
\def\vz{{\bm{z}}}

% Elements of vectors
\def\evalpha{{\alpha}}
\def\evbeta{{\beta}}
\def\evepsilon{{\epsilon}}
\def\evlambda{{\lambda}}
\def\evomega{{\omega}}
\def\evmu{{\mu}}
\def\evpsi{{\psi}}
\def\evsigma{{\sigma}}
\def\evtheta{{\theta}}
\def\eva{{a}}
\def\evb{{b}}
\def\evc{{c}}
\def\evd{{d}}
\def\eve{{e}}
\def\evf{{f}}
\def\evg{{g}}
\def\evh{{h}}
\def\evi{{i}}
\def\evj{{j}}
\def\evk{{k}}
\def\evl{{l}}
\def\evm{{m}}
\def\evn{{n}}
\def\evo{{o}}
\def\evp{{p}}
\def\evq{{q}}
\def\evr{{r}}
\def\evs{{s}}
\def\evt{{t}}
\def\evu{{u}}
\def\evv{{v}}
\def\evw{{w}}
\def\evx{{x}}
\def\evy{{y}}
\def\evz{{z}}

% Matrix
\def\mA{{\bm{A}}}
\def\mB{{\bm{B}}}
\def\mC{{\bm{C}}}
\def\mD{{\bm{D}}}
\def\mE{{\bm{E}}}
\def\mF{{\bm{F}}}
\def\mG{{\bm{G}}}
\def\mH{{\bm{H}}}
\def\mI{{\bm{I}}}
\def\mJ{{\bm{J}}}
\def\mK{{\bm{K}}}
\def\mL{{\bm{L}}}
\def\mM{{\bm{M}}}
\def\mN{{\bm{N}}}
\def\mO{{\bm{O}}}
\def\mP{{\bm{P}}}
\def\mQ{{\bm{Q}}}
\def\mR{{\bm{R}}}
\def\mS{{\bm{S}}}
\def\mT{{\bm{T}}}
\def\mU{{\bm{U}}}
\def\mV{{\bm{V}}}
\def\mW{{\bm{W}}}
\def\mX{{\bm{X}}}
\def\mY{{\bm{Y}}}
\def\mZ{{\bm{Z}}}
\def\mBeta{{\bm{\beta}}}
\def\mPhi{{\bm{\Phi}}}
\def\mLambda{{\bm{\Lambda}}}
\def\mSigma{{\bm{\Sigma}}}

% Tensor
\DeclareMathAlphabet{\mathsfit}{\encodingdefault}{\sfdefault}{m}{sl}
\SetMathAlphabet{\mathsfit}{bold}{\encodingdefault}{\sfdefault}{bx}{n}
\newcommand{\tens}[1]{\bm{\mathsfit{#1}}}
\def\tA{{\tens{A}}}
\def\tB{{\tens{B}}}
\def\tC{{\tens{C}}}
\def\tD{{\tens{D}}}
\def\tE{{\tens{E}}}
\def\tF{{\tens{F}}}
\def\tG{{\tens{G}}}
\def\tH{{\tens{H}}}
\def\tI{{\tens{I}}}
\def\tJ{{\tens{J}}}
\def\tK{{\tens{K}}}
\def\tL{{\tens{L}}}
\def\tM{{\tens{M}}}
\def\tN{{\tens{N}}}
\def\tO{{\tens{O}}}
\def\tP{{\tens{P}}}
\def\tQ{{\tens{Q}}}
\def\tR{{\tens{R}}}
\def\tS{{\tens{S}}}
\def\tT{{\tens{T}}}
\def\tU{{\tens{U}}}
\def\tV{{\tens{V}}}
\def\tW{{\tens{W}}}
\def\tX{{\tens{X}}}
\def\tY{{\tens{Y}}}
\def\tZ{{\tens{Z}}}


% Graph
\def\gA{{\mathcal{A}}}
\def\gB{{\mathcal{B}}}
\def\gC{{\mathcal{C}}}
\def\gD{{\mathcal{D}}}
\def\gE{{\mathcal{E}}}
\def\gF{{\mathcal{F}}}
\def\gG{{\mathcal{G}}}
\def\gH{{\mathcal{H}}}
\def\gI{{\mathcal{I}}}
\def\gJ{{\mathcal{J}}}
\def\gK{{\mathcal{K}}}
\def\gL{{\mathcal{L}}}
\def\gM{{\mathcal{M}}}
\def\gN{{\mathcal{N}}}
\def\gO{{\mathcal{O}}}
\def\gP{{\mathcal{P}}}
\def\gQ{{\mathcal{Q}}}
\def\gR{{\mathcal{R}}}
\def\gS{{\mathcal{S}}}
\def\gT{{\mathcal{T}}}
\def\gU{{\mathcal{U}}}
\def\gV{{\mathcal{V}}}
\def\gW{{\mathcal{W}}}
\def\gX{{\mathcal{X}}}
\def\gY{{\mathcal{Y}}}
\def\gZ{{\mathcal{Z}}}

% Sets
\def\sA{{\mathbb{A}}}
\def\sB{{\mathbb{B}}}
\def\sC{{\mathbb{C}}}
\def\sD{{\mathbb{D}}}
% Don't use a set called E, because this would be the same as our symbol
% for expectation.
\def\sF{{\mathbb{F}}}
\def\sG{{\mathbb{G}}}
\def\sH{{\mathbb{H}}}
\def\sI{{\mathbb{I}}}
\def\sJ{{\mathbb{J}}}
\def\sK{{\mathbb{K}}}
\def\sL{{\mathbb{L}}}
\def\sM{{\mathbb{M}}}
\def\sN{{\mathbb{N}}}
\def\sO{{\mathbb{O}}}
\def\sP{{\mathbb{P}}}
\def\sQ{{\mathbb{Q}}}
\def\sR{{\mathbb{R}}}
\def\sS{{\mathbb{S}}}
\def\sT{{\mathbb{T}}}
\def\sU{{\mathbb{U}}}
\def\sV{{\mathbb{V}}}
\def\sW{{\mathbb{W}}}
\def\sX{{\mathbb{X}}}
\def\sY{{\mathbb{Y}}}
\def\sZ{{\mathbb{Z}}}

% Entries of a matrix
\def\emLambda{{\Lambda}}
\def\emA{{A}}
\def\emB{{B}}
\def\emC{{C}}
\def\emD{{D}}
\def\emE{{E}}
\def\emF{{F}}
\def\emG{{G}}
\def\emH{{H}}
\def\emI{{I}}
\def\emJ{{J}}
\def\emK{{K}}
\def\emL{{L}}
\def\emM{{M}}
\def\emN{{N}}
\def\emO{{O}}
\def\emP{{P}}
\def\emQ{{Q}}
\def\emR{{R}}
\def\emS{{S}}
\def\emT{{T}}
\def\emU{{U}}
\def\emV{{V}}
\def\emW{{W}}
\def\emX{{X}}
\def\emY{{Y}}
\def\emZ{{Z}}
\def\emSigma{{\Sigma}}

% entries of a tensor
% Same font as tensor, without \bm wrapper
\newcommand{\etens}[1]{\mathsfit{#1}}
\def\etLambda{{\etens{\Lambda}}}
\def\etA{{\etens{A}}}
\def\etB{{\etens{B}}}
\def\etC{{\etens{C}}}
\def\etD{{\etens{D}}}
\def\etE{{\etens{E}}}
\def\etF{{\etens{F}}}
\def\etG{{\etens{G}}}
\def\etH{{\etens{H}}}
\def\etI{{\etens{I}}}
\def\etJ{{\etens{J}}}
\def\etK{{\etens{K}}}
\def\etL{{\etens{L}}}
\def\etM{{\etens{M}}}
\def\etN{{\etens{N}}}
\def\etO{{\etens{O}}}
\def\etP{{\etens{P}}}
\def\etQ{{\etens{Q}}}
\def\etR{{\etens{R}}}
\def\etS{{\etens{S}}}
\def\etT{{\etens{T}}}
\def\etU{{\etens{U}}}
\def\etV{{\etens{V}}}
\def\etW{{\etens{W}}}
\def\etX{{\etens{X}}}
\def\etY{{\etens{Y}}}
\def\etZ{{\etens{Z}}}

% The true underlying data generating distribution
\newcommand{\pdata}{p_{\rm{data}}}
\newcommand{\ptarget}{p_{\rm{target}}}
\newcommand{\pprior}{p_{\rm{prior}}}
\newcommand{\pbase}{p_{\rm{base}}}
\newcommand{\pref}{p_{\rm{ref}}}

% The empirical distribution defined by the training set
\newcommand{\ptrain}{\hat{p}_{\rm{data}}}
\newcommand{\Ptrain}{\hat{P}_{\rm{data}}}
% The model distribution
\newcommand{\pmodel}{p_{\rm{model}}}
\newcommand{\Pmodel}{P_{\rm{model}}}
\newcommand{\ptildemodel}{\tilde{p}_{\rm{model}}}
% Stochastic autoencoder distributions
\newcommand{\pencode}{p_{\rm{encoder}}}
\newcommand{\pdecode}{p_{\rm{decoder}}}
\newcommand{\precons}{p_{\rm{reconstruct}}}

\newcommand{\laplace}{\mathrm{Laplace}} % Laplace distribution

\newcommand{\E}{\mathbb{E}}
\newcommand{\Ls}{\mathcal{L}}
\newcommand{\R}{\mathbb{R}}
\newcommand{\emp}{\tilde{p}}
\newcommand{\lr}{\alpha}
\newcommand{\reg}{\lambda}
\newcommand{\rect}{\mathrm{rectifier}}
\newcommand{\softmax}{\mathrm{softmax}}
\newcommand{\sigmoid}{\sigma}
\newcommand{\softplus}{\zeta}
\newcommand{\KL}{D_{\mathrm{KL}}}
\newcommand{\Var}{\mathrm{Var}}
\newcommand{\standarderror}{\mathrm{SE}}
\newcommand{\Cov}{\mathrm{Cov}}
% Wolfram Mathworld says $L^2$ is for function spaces and $\ell^2$ is for vectors
% But then they seem to use $L^2$ for vectors throughout the site, and so does
% wikipedia.
\newcommand{\normlzero}{L^0}
\newcommand{\normlone}{L^1}
\newcommand{\normltwo}{L^2}
\newcommand{\normlp}{L^p}
\newcommand{\normmax}{L^\infty}

\newcommand{\parents}{Pa} % See usage in notation.tex. Chosen to match Daphne's book.

\DeclareMathOperator*{\argmax}{arg\,max}
\DeclareMathOperator*{\argmin}{arg\,min}

\DeclareMathOperator{\sign}{sign}
\DeclareMathOperator{\Tr}{Tr}
\let\ab\allowbreak


\usepackage{hyperref}
\usepackage{url}
\usepackage{optidef}
\usepackage{bm}
\usepackage{xcolor}
\usepackage{caption}
\usepackage{subcaption}


%%%%%%%%%%%%%%%%%%%% USER DEFS %%%%%%%%%%%%%%%%%%%%

\usepackage{multicol}
\setlength{\multicolsep}{6.0pt plus 2.0pt minus 1.5pt}
\usepackage{amsthm, amsmath, amssymb}
\usepackage{graphicx}
\usepackage{wrapfig}
\usepackage{enumitem}
\newtheorem{definition}{Definition}
\newtheorem{theorem}{Theorem}
\newtheorem{corollary}{Corollary}
\newtheorem{lemma}{Lemma}
% \DeclareMathOperator*{\argmax}{arg\,max}
% \DeclareMathOperator*{\argmin}{arg\,min}
\newcommand{\norm}[1]{\left\lVert#1\right\rVert_2}
\newcommand{\normop}[1]{\left\lVert#1\right\rVert_{op}}
\newcommand{\todo}[1]{\textcolor{red}{[TODO: #1]}}
\newcommand{\mc}[1]{\mathcal{#1}}
\newcommand{\appropto}{\mathrel{\vcenter{
  \offinterlineskip\halign{\hfil$##$\cr
    \propto\cr\noalign{\kern2pt}\sim\cr\noalign{\kern-2pt}}}}}

\newtheorem{innercustomthm}{Theorem}
\newenvironment{customthm}[1]
  {\renewcommand\theinnercustomthm{#1}\innercustomthm}
  {\endinnercustomthm}

%%% Define relation symbol labeling
\newcounter{relctr} %% <- counter for relations
\everydisplay\expandafter{\the\everydisplay\setcounter{relctr}{0}} %% <- reset every eq
\renewcommand*\therelctr{\alph{relctr}} %% <- label format

\newcommand\labelrel[2]{%
  \begingroup
    \refstepcounter{relctr}%
    \stackrel{\textnormal{(\alph{relctr})}}{\mathstrut{#1}}%
    \originallabel{#2}%
  \endgroup
}
\AtBeginDocument{\let\originallabel\label}
\renewcommand{\baselinestretch}{0.99}
%%%%%%%%%%%%%%%%%%%% END DEFS %%%%%%%%%%%%%%%%%%%%


%%%%%%%%%%%%%%%%%%%% USER DEFS %%%%%%%%%%%%%%%%%%%%
\newcommand{\until}[1]{\{1,\dots, #1\}}
\newcommand{\subscr}[2]{#1_{\textup{#2}}}
\newcommand{\supscr}[2]{#1^{\textup{#2}}}
% \newcommand{\new}[1]{\textcolor{blue}{#1}}
\newcommand{\new}[1]{\textcolor{black}{#1}}

%%%%%%%%%%%%%%%%%%%% END DEFS %%%%%%%%%%%%%%%%%%%%



\title{Behavioral Entropy-Guided Dataset Generation for Offline Reinforcement Learning}


% Authors must not appear in the submitted version. They should be hidden
% as long as the \iclrfinalcopy macro remains commented out below.
% Non-anonymous submissions will be rejected without review.

\author{Wesley A. Suttle\thanks{Equal contribution.} , Aamodh Suresh$^*$, Carlos Nieto-Granda \\
U.S. Army Research Laboratory\\
Adelphi, MD 20783, USA \\
% \texttt{\{wesley.a.suttle.ctr, carlos.p.nieto2.civ\}@army.mil, aamodh@gmail.com} \\
\texttt{wesley.a.suttle.ctr@army.mil}, \\
\texttt{aamodh@gmail.com} ,\\
\texttt{carlos.p.nieto2.civ@army.mil}
}

% The \author macro works with any number of authors. There are two commands
% used to separate the names and addresses of multiple authors: \And and \AND.
%
% Using \And between authors leaves it to \LaTeX{} to determine where to break
% the lines. Using \AND forces a linebreak at that point. So, if \LaTeX{}
% puts 3 of 4 authors names on the first line, and the last on the second
% line, try using \AND instead of \And before the third author name.

\newcommand{\fix}{\marginpar{FIX}}
% \newcommand{\new}{\marginpar{NEW}}

\iclrfinalcopy % Uncomment for camera-ready version, but NOT for submission.
\begin{document}


\maketitle

\begin{abstract}

Entropy-based objectives are widely used to perform state space exploration in reinforcement learning (RL) and dataset generation for offline RL. Behavioral entropy (BE), a rigorous generalization of classical entropies that incorporates cognitive and perceptual biases of agents, was recently proposed for discrete settings and shown to be a promising metric for robotic exploration problems. In this work, we propose using BE as a principled exploration objective for systematically generating datasets that provide diverse state space coverage in complex, continuous, potentially high-dimensional domains. To achieve this, we extend the notion of BE to continuous settings, derive tractable $k$-nearest neighbor estimators, provide theoretical guarantees for these estimators, and develop practical reward functions that can be used with standard RL methods to learn BE-maximizing policies. Using standard MuJoCo environments, we experimentally compare the performance of offline RL algorithms for a variety of downstream tasks on datasets generated using BE, R\'{e}nyi, and Shannon entropy-maximizing policies, \new{as well as the SMM and RND algorithms}. We find that offline RL algorithms trained on datasets collected using BE outperform those trained on datasets collected using Shannon entropy, SMM, and RND on all tasks considered, and on 80\% of the tasks compared to datasets collected using R\'{e}nyi entropy.

\end{abstract}



\section{Introduction}

Large language models (LLMs) have achieved remarkable success in automated math problem solving, particularly through code-generation capabilities integrated with proof assistants~\citep{lean,isabelle,POT,autoformalization,MATH}. Although LLMs excel at generating solution steps and correct answers in algebra and calculus~\citep{math_solving}, their unimodal nature limits performance in plane geometry, where solution depends on both diagram and text~\citep{math_solving}. 

Specialized vision-language models (VLMs) have accordingly been developed for plane geometry problem solving (PGPS)~\citep{geoqa,unigeo,intergps,pgps,GOLD,LANS,geox}. Yet, it remains unclear whether these models genuinely leverage diagrams or rely almost exclusively on textual features. This ambiguity arises because existing PGPS datasets typically embed sufficient geometric details within problem statements, potentially making the vision encoder unnecessary~\citep{GOLD}. \cref{fig:pgps_examples} illustrates example questions from GeoQA and PGPS9K, where solutions can be derived without referencing the diagrams.

\begin{figure}
    \centering
    \begin{subfigure}[t]{.49\linewidth}
        \centering
        \includegraphics[width=\linewidth]{latex/figures/images/geoqa_example.pdf}
        \caption{GeoQA}
        \label{fig:geoqa_example}
    \end{subfigure}
    \begin{subfigure}[t]{.48\linewidth}
        \centering
        \includegraphics[width=\linewidth]{latex/figures/images/pgps_example.pdf}
        \caption{PGPS9K}
        \label{fig:pgps9k_example}
    \end{subfigure}
    \caption{
    Examples of diagram-caption pairs and their solution steps written in formal languages from GeoQA and PGPS9k datasets. In the problem description, the visual geometric premises and numerical variables are highlighted in green and red, respectively. A significant difference in the style of the diagram and formal language can be observable. %, along with the differences in formal languages supported by the corresponding datasets.
    \label{fig:pgps_examples}
    }
\end{figure}



We propose a new benchmark created via a synthetic data engine, which systematically evaluates the ability of VLM vision encoders to recognize geometric premises. Our empirical findings reveal that previously suggested self-supervised learning (SSL) approaches, e.g., vector quantized variataional auto-encoder (VQ-VAE)~\citep{unimath} and masked auto-encoder (MAE)~\citep{scagps,geox}, and widely adopted encoders, e.g., OpenCLIP~\citep{clip} and DinoV2~\citep{dinov2}, struggle to detect geometric features such as perpendicularity and degrees. 

To this end, we propose \geoclip{}, a model pre-trained on a large corpus of synthetic diagram–caption pairs. By varying diagram styles (e.g., color, font size, resolution, line width), \geoclip{} learns robust geometric representations and outperforms prior SSL-based methods on our benchmark. Building on \geoclip{}, we introduce a few-shot domain adaptation technique that efficiently transfers the recognition ability to real-world diagrams. We further combine this domain-adapted GeoCLIP with an LLM, forming a domain-agnostic VLM for solving PGPS tasks in MathVerse~\citep{mathverse}. 
%To accommodate diverse diagram styles and solution formats, we unify the solution program languages across multiple PGPS datasets, ensuring comprehensive evaluation. 

In our experiments on MathVerse~\citep{mathverse}, which encompasses diverse plane geometry tasks and diagram styles, our VLM with a domain-adapted \geoclip{} consistently outperforms both task-specific PGPS models and generalist VLMs. 
% In particular, it achieves higher accuracy on tasks requiring geometric-feature recognition, even when critical numerical measurements are moved from text to diagrams. 
Ablation studies confirm the effectiveness of our domain adaptation strategy, showing improvements in optical character recognition (OCR)-based tasks and robust diagram embeddings across different styles. 
% By unifying the solution program languages of existing datasets and incorporating OCR capability, we enable a single VLM, named \geovlm{}, to handle a broad class of plane geometry problems.

% Contributions
We summarize the contributions as follows:
We propose a novel benchmark for systematically assessing how well vision encoders recognize geometric premises in plane geometry diagrams~(\cref{sec:visual_feature}); We introduce \geoclip{}, a vision encoder capable of accurately detecting visual geometric premises~(\cref{sec:geoclip}), and a few-shot domain adaptation technique that efficiently transfers this capability across different diagram styles (\cref{sec:domain_adaptation});
We show that our VLM, incorporating domain-adapted GeoCLIP, surpasses existing specialized PGPS VLMs and generalist VLMs on the MathVerse benchmark~(\cref{sec:experiments}) and effectively interprets diverse diagram styles~(\cref{sec:abl}).

\iffalse
\begin{itemize}
    \item We propose a novel benchmark for systematically assessing how well vision encoders recognize geometric premises, e.g., perpendicularity and angle measures, in plane geometry diagrams.
	\item We introduce \geoclip{}, a vision encoder capable of accurately detecting visual geometric premises, and a few-shot domain adaptation technique that efficiently transfers this capability across different diagram styles.
	\item We show that our final VLM, incorporating GeoCLIP-DA, effectively interprets diverse diagram styles and achieves state-of-the-art performance on the MathVerse benchmark, surpassing existing specialized PGPS models and generalist VLM models.
\end{itemize}
\fi

\iffalse

Large language models (LLMs) have made significant strides in automated math word problem solving. In particular, their code-generation capabilities combined with proof assistants~\citep{lean,isabelle} help minimize computational errors~\citep{POT}, improve solution precision~\citep{autoformalization}, and offer rigorous feedback and evaluation~\citep{MATH}. Although LLMs excel in generating solution steps and correct answers for algebra and calculus~\citep{math_solving}, their uni-modal nature limits performance in domains like plane geometry, where both diagrams and text are vital.

Plane geometry problem solving (PGPS) tasks typically include diagrams and textual descriptions, requiring solvers to interpret premises from both sources. To facilitate automated solutions for these problems, several studies have introduced formal languages tailored for plane geometry to represent solution steps as a program with training datasets composed of diagrams, textual descriptions, and solution programs~\citep{geoqa,unigeo,intergps,pgps}. Building on these datasets, a number of PGPS specialized vision-language models (VLMs) have been developed so far~\citep{GOLD, LANS, geox}.

Most existing VLMs, however, fail to use diagrams when solving geometry problems. Well-known PGPS datasets such as GeoQA~\citep{geoqa}, UniGeo~\citep{unigeo}, and PGPS9K~\citep{pgps}, can be solved without accessing diagrams, as their problem descriptions often contain all geometric information. \cref{fig:pgps_examples} shows an example from GeoQA and PGPS9K datasets, where one can deduce the solution steps without knowing the diagrams. 
As a result, models trained on these datasets rely almost exclusively on textual information, leaving the vision encoder under-utilized~\citep{GOLD}. 
Consequently, the VLMs trained on these datasets cannot solve the plane geometry problem when necessary geometric properties or relations are excluded from the problem statement.

Some studies seek to enhance the recognition of geometric premises from a diagram by directly predicting the premises from the diagram~\citep{GOLD, intergps} or as an auxiliary task for vision encoders~\citep{geoqa,geoqa-plus}. However, these approaches remain highly domain-specific because the labels for training are difficult to obtain, thus limiting generalization across different domains. While self-supervised learning (SSL) methods that depend exclusively on geometric diagrams, e.g., vector quantized variational auto-encoder (VQ-VAE)~\citep{unimath} and masked auto-encoder (MAE)~\citep{scagps,geox}, have also been explored, the effectiveness of the SSL approaches on recognizing geometric features has not been thoroughly investigated.

We introduce a benchmark constructed with a synthetic data engine to evaluate the effectiveness of SSL approaches in recognizing geometric premises from diagrams. Our empirical results with the proposed benchmark show that the vision encoders trained with SSL methods fail to capture visual \geofeat{}s such as perpendicularity between two lines and angle measure.
Furthermore, we find that the pre-trained vision encoders often used in general-purpose VLMs, e.g., OpenCLIP~\citep{clip} and DinoV2~\citep{dinov2}, fail to recognize geometric premises from diagrams.

To improve the vision encoder for PGPS, we propose \geoclip{}, a model trained with a massive amount of diagram-caption pairs.
Since the amount of diagram-caption pairs in existing benchmarks is often limited, we develop a plane diagram generator that can randomly sample plane geometry problems with the help of existing proof assistant~\citep{alphageometry}.
To make \geoclip{} robust against different styles, we vary the visual properties of diagrams, such as color, font size, resolution, and line width.
We show that \geoclip{} performs better than the other SSL approaches and commonly used vision encoders on the newly proposed benchmark.

Another major challenge in PGPS is developing a domain-agnostic VLM capable of handling multiple PGPS benchmarks. As shown in \cref{fig:pgps_examples}, the main difficulties arise from variations in diagram styles. 
To address the issue, we propose a few-shot domain adaptation technique for \geoclip{} which transfers its visual \geofeat{} perception from the synthetic diagrams to the real-world diagrams efficiently. 

We study the efficacy of the domain adapted \geoclip{} on PGPS when equipped with the language model. To be specific, we compare the VLM with the previous PGPS models on MathVerse~\citep{mathverse}, which is designed to evaluate both the PGPS and visual \geofeat{} perception performance on various domains.
While previous PGPS models are inapplicable to certain types of MathVerse problems, we modify the prediction target and unify the solution program languages of the existing PGPS training data to make our VLM applicable to all types of MathVerse problems.
Results on MathVerse demonstrate that our VLM more effectively integrates diagrammatic information and remains robust under conditions of various diagram styles.

\begin{itemize}
    \item We propose a benchmark to measure the visual \geofeat{} recognition performance of different vision encoders.
    % \item \sh{We introduce geometric CLIP (\geoclip{} and train the VLM equipped with \geoclip{} to predict both solution steps and the numerical measurements of the problem.}
    \item We introduce \geoclip{}, a vision encoder which can accurately recognize visual \geofeat{}s and a few-shot domain adaptation technique which can transfer such ability to different domains efficiently. 
    % \item \sh{We develop our final PGPS model, \geovlm{}, by adapting \geoclip{} to different domains and training with unified languages of solution program data.}
    % We develop a domain-agnostic VLM, namely \geovlm{}, by applying a simple yet effective domain adaptation method to \geoclip{} and training on the refined training data.
    \item We demonstrate our VLM equipped with GeoCLIP-DA effectively interprets diverse diagram styles, achieving superior performance on MathVerse compared to the existing PGPS models.
\end{itemize}

\fi 

% \section{Differential Behavioral Entropy and its Estimation}

\section{Behavioral Entropy in Continuous Spaces} \label{sec:be_in_continuous_spaces}

\begin{figure}[tp]
    \begin{subfigure}[b]{0.48\textwidth}
        \includegraphics[width=\linewidth]{figs/prelec_cividis.png}
        \caption{\small \textbf{Perception of normal distribution ($\mathcal{N}(0,1)$) under Prelec weighting function.} Black dotted line shows standard normal density, $f$. For $\alpha \approx 0$ the perceived density is uniform, indicating \textit{over-weighting} of uncertainty over entire support of $f$. For $\alpha \gg 0$ the perceived density approaches a step function around the mean, indicating \textit{under-weighting} of uncertainty near $f$'s tails. $\alpha = 1$ recovers original $f$.
        %
        }
        \label{fig:prelec_perception}
    \end{subfigure}
    \centering
    ~
    \begin{subfigure}[b]{0.48\textwidth}
        \centering
        \includegraphics[width=\linewidth]{figs/bernoulli.png}
        \caption{\small\textbf{Diversity of Shannon, R\'{e}nyi and BE values as a function of Bernoulli trial parameter $p$.} Behavioral entropy $H^B$ captures entire behavior spectrum from overvaluing uncertainty (light blue, $\alpha \approx 0$) to highly undervaluing uncertainty (dark blue, $\alpha \gg 0$). R\'{e}nyi, $H^R$, captures the former (light red, $q \approx 0$), but cannot capture the latter. Dotted red curve shows $H^R$ as $q \rightarrow \infty$. Shannon, $\supscr{H}{S}$, is dotted black curve.
        %
        %
        }
    \label{fig:entropy}
    \end{subfigure}
    %
    \caption{\small Visualizations of probability weightings (left) and superior expressiveness of BE (right).}
    \label{fig:prob_ent}\
    
\vspace{-4mm}
\end{figure}



\textbf{Behavioral Entropy.} The various notions of entropy that have been studied in the literature since the initial work by \cite{shannon1948entropy} quantify the uncertainty inherent in a random variable by measuring how evenly distributed its associated probability density is over its support. Let $X$ be a discrete random variable over a finite set of $M$ elements, and let $p$ denote its probability mass function (p.m.f.). Two classical and widely used entropies are the Shannon and R\'{e}nyi entropies \citep{shannon1948entropy, renyi1961measures}, given respectively by
%
\begin{multicols}{2}
    \noindent
    \begin{equation} \label{eqn:shannon}
        H^S(X) = - \sum_{i=1}^M \log(p_i) p_i,
    \end{equation}
    %
    % \break
    %
    \noindent
    \begin{equation} \label{eqn:renyi}
        H^R_q(X) = \frac{1}{1-q} \log \sum_{i=1}^M p_i^q, \ q > 0, q \neq 1.
    \end{equation}
\end{multicols}
%
These entropy functionals, along with others such as Tsallis entropy \citep{tsallis1988possible}, belong to the class of \textit{admissible generalized entropies} satisfying the first three Shannon-Khinchin axioms (see \citep{Amigo2018entropy} for details). These axioms ensure that an entropy functional is well-behaved by ensuring their continuity, stability with respect to addition or removal of known outcomes, and maximality of the uniform distribution.

As mentioned in the introduction, the recent work \citep{suresh2024robotic} proposed composing Shannon's entropy with the probability weighting functions widely used in behavioral economics to encode human cognitive and perceptual biases. The composition of Shannon's entropy with Prelec's probability weighting function \cite{prelec1998probability} yielded BE, an admissible generalized entropy that provides a tractable mathematical formalism for incorporating helpful human biases into uncertainty quantification via entropy. The probability weighting functions used to encode human biases are defined as follows (see \citep[Ch. 2.2]{dhami2016foundations}).
%
\begin{definition} \label{def:prob_weighting}
    A function $w : [0, 1] \rightarrow [0, 1]$ is said to be a \textbf{probability weighting function} if it is continuous, strictly increasing, satisfies $w(0) = 0$ and $w(1) = 1$, and has a unique, continuous, and strictly increasing inverse.
\end{definition}
%
The most widely used probability weighting function, and that which was the focus of \cite{suresh2024robotic}, is Prelec's probability weighting, given by
%
\begin{equation} \label{eqn:prelec_w}
    w(x) = e^{-\beta(-\log x)^{\alpha}}, \quad \alpha, \beta > 0.
\end{equation}
%
Prelec's function is smooth in both the probability and parameter space and has the ability to control the fixed point and shape of the weighting function (see \citep{prelec1998probability, dhami2016foundations} for details). Figure \ref{fig:prelec_perception} illustrates its effect on a Gaussian density.
%
Equipped with \eqref{eqn:prelec_w}, \cite{suresh2024robotic} proposed and studied the following.
%
\begin{definition} Letting $w$ be as in \eqref{eqn:prelec_w}, \textbf{behavioral entropy} is given by
    \begin{equation} \label{eqn:behavioral_entropy}
        \supscr{H}{B}(X) = -\sum_{i=1}^M w(p_i) \log(w(p_i)).
    \end{equation}
\end{definition}
%
The parameter $\alpha$ controls the shape of the perceived probability curve, enabling a wide range of probability perceptions (see Figure \ref{fig:prelec_perception}) and the resulting perceived entropies, as illustrated in Figure \ref{fig:entropy}. Under the condition that $\beta = e^{(1-\alpha)\log(\log(M))}$, \eqref{eqn:behavioral_entropy} was shown in \cite{suresh2024robotic} to belong to the class of admissible generalized entropies. 
%
With this conditioning, \eqref{eqn:prelec_w} allows control of the third fixed point of $w$, which is critical for ensuring BE remains an admissible entropy, whereas other probability weighting functions lack this property \citep{dhami2016foundations} and do not generate meaningful, admissible generalized entropies.

Interestingly, it was shown in \citep{suresh2024robotic} that using BE over other approaches led to significant acceleration of robotic exploration tasks as well as emergent search behaviors similar to breadth-first and depth-first search, depending on choice of $\alpha$. These results indicate that BE holds promise as an objective for a broad range of exploration tasks in complex environments, yet \cite{suresh2024robotic} only applied BE to discrete, binary random variables. To pave the way for the application of BE to more complex problems, we now extend it to continuous settings.
%



\textbf{Differential Behavioral Entropy.} In this subsection we extend the definition of BE to that of differential BE, providing a novel entropy functional that is applicable in continuous, potentially high-dimensional spaces.
%
Let $f \in \Delta(\mc{X})$ be a p.d.f. over $\mc{X} \subset \mathbb{R}^d$, where $d \in \mathbb{N}^+$. We first recall the differential versions of Shannon's entropy and R\'{e}nyi entropy of order $q$, where $q > 0, q \neq 1$, which are defined, respectively, by
%
\begin{multicols}{2}
    \noindent
    \begin{equation} \label{eqn:diff_shannon}
        H^S(f) = - \int_{\mc{X}} \log (f(x)) f(x) dx,
    \end{equation}
    %
    % \break
    %
    \noindent
    \begin{equation} \label{eqn:diff_renyi}
        H^R_q(f) = \frac{1}{1 - q} \log \int_{\mc{X}} f^q(x) dx.
    \end{equation}
\end{multicols}
%
We will use these expressions in our experimental comparisons below and thus include them here for easy reference. We next state the definitions of our continuous-spaces analogues of \eqref{eqn:behavioral_entropy}.
%
\begin{definition}
    For an arbitrary probability weighting function $w$, \textbf{differential generalized behavioral entropy} is given by
    %
    \begin{equation} \label{eqn:diff_gbe}
        H^{B,w}(f) = - \int_{\mc{X}} \log(w(f(x))) w(f(x)) dx.
    \end{equation}
% \end{definition}
%
%
%
In particular, substituting Prelec's probability weighting from \eqref{eqn:prelec_w} into \eqref{eqn:diff_gbe} yields \textbf{differential behavioral entropy}, given by
%
\begin{equation} \label{eqn:diff_be}
    H^{B,\alpha,\beta}(f) = \beta \int_{\mc{X}} e^{-\beta ( - \log (f(x)))^{\alpha}} (-\log f(x))^{\alpha} dx.
\end{equation}
\end{definition}
%
It is important to note that, unlike Definition \ref{def:prob_weighting} for probability weightings in the discrete setting, where $w : [0, 1] \rightarrow [0, 1]$ in the continuous setting $w$ must be generalized to $w : [0, \infty) \rightarrow [0, \infty)$ to accommodate arbitrary densities, $f$. Desirable structural properties of $w$ are described in the detailed statement of Theorem \ref{thm:main_bound} in the appendix. 
%
We will henceforth abuse both terminology and notation by omitting ``differential'' when referring to differential entropies and by suppressing the dependence of \eqref{eqn:diff_be} on $\alpha, \beta$ when these are clear from context.



\section{$k$-Nearest Neighbor Behavioral Entropy Estimation}

We next turn to the problem of BE estimation in continuous, potentially high-dimensional spaces. To accomplish this, we derive $k$-nearest neighbor ($k$-NN) estimates of BE along the lines of the estimates studied in \citep{kozachenko1987sample, singh2003nearest, leonenko2008class, sricharan2012estimation, singh2016finite} and others for Shannon and R\'{e}nyi entropy. The $k$-NN family of nonparametric estimators enables estimation of arbitrary densities from a finite number of i.i.d. samples in continuous and potentially high-dimensional settings, making them particularly well-suited to the RL context considered in the following section.
%
%
Let $f \in \Delta(\mathbb{R}^d)$ be a probability density function (p.d.f.) over $\mathbb{R}^d$, where $d \in \mathbb{R}^+$. Let $X_1, X_2, \ldots, X_n \sim f(\cdot)$ be $n \in \mathbb{N}^+$ i.i.d. samples drawn from $f$. In \citep{loftsgaarden1965nonparametric, devroye1977strong} and subsequent works it was established that, for suitably chosen $n$ and $k$, a reasonable approximation of $f$ is provided by the $k$-NN density estimator
%
\begin{equation} \label{eqn:knn_f}
    \hat{f}(x) = \frac{ k \Gamma(d / 2 + 1) }{ n \pi^{d / 2} R^d_{k, n}(x) },
\end{equation}
%
where $R_{k, n}(x) = \norm{x - NN_k(x)}$ is the Euclidean distance between $x$ and its $k$th nearest neighbor among $\{ X_1, \ldots, X_n \}$ and $\Gamma(x) = \int_0^{\infty} t^{x-1} e^{-t} dt$ is the gamma function. When $x = X_i$, we will write $R_{i, k, n} = R_{k, n}(X_i)$ for simplicity. A natural first approximation to Shannon entropy of $f$ is given by the plug-in estimator
%
\begin{equation} \label{eqn:shannon_knn_approx_1}
    \widehat{H}^{S}_{k,n}(f) = - \frac{1}{n} \sum_{i=1}^n \log \hat{f}(X_i) \approx \mathbb{E}_{X_i \sim f(\cdot)} \left[ - \log \hat{f}(X_i) \right] = - \int_{\mathbb{R}^n} \log \hat{f}(x) \cdot f(x) dx.
\end{equation}
%
With this in mind, the na\"{i}ve approach to estimating \eqref{eqn:diff_gbe} for general $w$ is via
%
\begin{equation} \label{eqn:be_knn_wrong_1}
    \widetilde{H}^{B,w}_{k, n}(f) = - \frac{1}{n} \sum_{i=1}^n w(\hat{f}(X_i)) \log w(\hat{f}(X_i)).
\end{equation}
%
Since $X_i \sim f(\cdot)$, however, the estimator of \eqref{eqn:be_knn_wrong_1} is biased, since
%
\begin{equation} \label{eqn:be_knn_wrong_2}
    \widetilde{H}^{B,w}_{k,n}(f) \approx \mathbb{E}_{X_i \sim f(\cdot)} \left[ - w(\hat{f}(X_i)) \log w(\hat{f}(X_i)) \right] = - \int_{\mathbb{R}^n} w(\hat{f}(x)) \log w(\hat{f}(x)) \cdot f(x) dx.
\end{equation}
%
Dividing by the approximation $\hat{f}$ yields the alternative, importance sampling-corrected estimator
%
\begin{align}
    \widehat{H}^{B,w}_{k,n}(f) &= - \frac{1}{n} \sum_{i=1}^n \frac{1}{\hat{f}(X_i)} w(\hat{f}(X_i)) \log w(\hat{f}(X_i)) \label{eqn:be_knn_approx} \\
    %
    &\approx \mathbb{E}_{X_i \sim f(\cdot)} \left[ \frac{1}{\hat{f}(X_i)} w(\hat{f}(X_i)) \log w(\hat{f}(X_i)) \right] = - \int_{\mathbb{R}^n} w(\hat{f}(x)) \log w(\hat{f}(x)) \frac{f(x)}{\hat{f}(x)} dx. \nonumber
\end{align}
%
Though this importance sampling-corrected plug-in estimator will be biased for the same reasons detailed in \cite[Thm. 8]{singh2003nearest} for $\widehat{H}^S_{k,n}(f)$, for large $n$ and suitable $k$ \eqref{eqn:be_knn_approx} provides a reasonable estimator of \eqref{eqn:diff_be}, as characterized by the following results.

\begin{theorem} \label{thm:convergence}
    Suppose that $k := k_n \rightarrow \infty, \frac{k_n}{n} \rightarrow 0$, and $\frac{k_n}{\log n} \rightarrow \infty$ as $n \rightarrow \infty$. Assume that $w$ is Lipschitz, that $f$ is absolutely continuous, and that there exist $c_1, c_2 > 0$ such that $0 < c_1 \leq f(x) \leq c_2 < \infty$, for all $x \in \mc{X}$. Then $\widehat{H}^{B,w}_{k,n}(f) \rightarrow H^{B,w}(f)$ both uniformly and in probability.
\end{theorem}
%
The result follows from the strong uniform consistency of $\hat{f}$ \citep{devroye1977strong}.

Though asymptotic guarantees like Theorem \ref{thm:convergence} are somewhat reassuring, in practice we will use finite $k$ in our $k$-NN estimators and therefore need a more fine-grained characterization of the bias and variance.
%
Unfortunately, for finite $k$, the approximator $\widehat{H}^{B,w}_{k,n}(f)$ remains biased as $n \rightarrow \infty$ due to the biasedness of $\hat{f}$ for fixed $k$ and the lack of a known bias correction procedure for our BE approximator $\widehat{H}^{B,w}_{k,n}(f)$. This contrasts with the situation for simpler estimators like those for Shannon and R\'{e}nyi entropies, for which explicit bias correction terms are known (see \cite{singh2003nearest, leonenko2008class, singh2016finite}). Nonetheless, we establish probabilistic guarantees on the bias and variance of our proposed BE estimator in the following main result.
%
\begin{theorem} \label{thm:main_bound}
    Suppose $f$ and $w$ satisfy suitable differentiability and continuity conditions. Then, for $\varepsilon > 0$, it holds with probability $1 - \varepsilon$ that
    %
    \begin{align} \left| \mathbb{E} \left[ \widehat{H}^{B,w}_{k,n}(f) \right] - H^{B,w}(f) \right| &= 
        \begin{cases}
            \mc{O}\left( \frac{k}{n} \right)^{\frac{\xi}{d}} + \mc{O}\left( \left( \frac{k}{n} \right)^{\frac{2}{d}} \log n + \sqrt{ \frac{ \log ( n / \varepsilon ) }{k} } \right) & \text{ if } d > 2, \\
            %
            \mc{O}\left( \frac{k}{n} \right)^{\frac{\xi}{d}} + \mc{O}\left( \frac{k}{n} \log n + \sqrt{ \frac{ \log ( n / \varepsilon ) }{k} } \right) & \text{ if } d = 1, 2,
        \end{cases}
        %
        \\
        %
        \text{Var} \left( \widehat{H}^{B,w}_{k,n}(f) \right) &= \mc{O}\left( \frac{1}{n} \right).
    \end{align}
    %
\end{theorem}
%
The proof and a precise statement are provided in the appendix. Equipped with the $k$-NN estimators and theoretical guarantees derived above, we next turn to their application in the RL setting.
\section{Behavioral Entropy in the Reinforcement Learning Context}

In this section we leverage the $k$-nearest neighbor generalized behavioral entropy estimates developed in the previous section for general probability weightings to derive a practical reward function that can be used in conjunction with standard RL methods to maximize behavioral entropy of an RL agent's state occupancy measure.



\textbf{Behavioral Entropy as an RL Objective.} State occupancy measure entropy has been used as an exploration objective for RL in a wide range of previous works \citep{hazan2019provably, liu2021behavior, yarats2021reinforcement, zhang2021exploration, yuan2022renyi}. In order to formally define behavioral entropy for state occupancy measures in this context, we first provide preliminary background on Markov decision processes (MDPs).
%
Let an average-reward MDP $\mathcal{M} = (\mc{S}, \mc{A}, p, r)$ be given, where $\mc{S}$ is the state space, $\mc{A}$ is the action space, $p: \mc{S} \times \mc{A} \rightarrow \Delta(\mc{S})$ is the transition probability kernel mapping state-action pairs $(s, a) \in \mc{S} \times \mc{A}$ to probability distributions $p(\cdot | s, a) \in \Delta(\mc{S})$ over the state space, and $r : \mc{S} \times \mc{A} \rightarrow \mathbb{R}$ is the reward function. Given a policy $\pi : \mc{S} \rightarrow \Delta(\mc{A})$ mapping states to probability distributions over the action space, $\mc{M}$ evolves as follows: at timestep $t \in \mathbb{N}$, the system is in state $s_t$, action $a_t \sim \pi(\cdot | s_t)$ is selected and executed, a reward $r_t = r(s_t, a_t)$ is received, the state transitions according to $s_{t+1} \sim p(\cdot | s_t, a_t)$, and the process repeats. To each policy $\pi$ is associated the long-run average reward $J(\pi) = \lim_{T \rightarrow \infty} \frac{1}{T} \mathbb{E}_{\pi} [ \sum_{i=0}^{T-1} r_i ]$, and the goal is to determine an optimal policy $\pi^* = \argmax_{\pi} J(\pi)$.

Under mild conditions on the transition kernel and policy (see \citep{puterman2014markov}), each policy $\pi$ induces a state occupancy measure $d_{\pi}(\cdot) \in \Delta(\mc{S})$ capturing the long-run state visitation behavior induced by $\pi$ over $\mc{S}$.
%
When $\mc{S}$ is continuous, for a measurable subset $B \subset \mc{S}$ we have $d_{\pi}(B) = \lim_{t \rightarrow \infty} P(s_t \in B)$.
%
We will henceforth assume that each measure $d_{\pi}$ has a corresponding p.d.f. and abuse notation by denoting the value of this p.d.f at $s$ by $d_{\pi}(s)$.
%
\footnote{Similarly, $\pi$ induces a state-action occupancy measure $\lambda_{\pi}(s, a) = d_{\pi}(s) \pi(a | s)$. In this work we focus on state occupancy measures, but all results and methods can be extended to apply to state-action occupancy measures in a straightforward manner.}
%
State occupancy measure entropies can be obtained by directly substituting $f = d_{\pi}$ and $\mc{X} = \mc{S}$ in the entropy definitions in \eqref{eqn:diff_shannon}, \eqref{eqn:diff_renyi}, and \eqref{eqn:diff_be}. In particular, the behavioral entropy induced by $\pi$ resulting from this substitution in \eqref{eqn:diff_be} is given by
%
\begin{equation} \label{eqn:be_occ_meas}
    H^{B,\alpha,\beta}(d_{\pi}) = \beta \int_{\mc{S}} e^{-\beta (-\log(d_{\pi}(s)))^\alpha} (-\log d_{\pi}(s))^{\alpha} ds.
\end{equation}
%
We propose to use \eqref{eqn:be_occ_meas} as an exploration objective. To achieve this, we leverage the $k$-NN estimator of \eqref{eqn:be_knn_approx} developed in the preceding section to derive a reward function $r$ such that $J(\pi) \approx H^{B,\alpha,\beta}(d_{\pi})$ in the following subsection.



\textbf{Behavioral Entropy Reward Derivation.} We next build on the $k$-NN estimator of \eqref{eqn:be_knn_approx} to derive a practical reward function $r$ such that $J(\pi) \approx H^{B,\alpha,\beta}(d_{\pi})$. Our derivation is similar to the reward derivations followed in \citep{liu2021behavior, yarats2021reinforcement} for Shannon entropy and \citep{yuan2022renyi} for R\'{e}nyi entropy. Once we are equipped with this reward, we can leverage existing RL methods to learn behavioral entropy-maximizing exploration policies.
%
Let $s_1, \ldots, s_n \sim d_{\pi}(\cdot)$. Substituting $f = d_{\pi}$ and $s_i = X_i$ into \eqref{eqn:knn_f}, for $i = 1, \ldots, n$, we have that $\widehat{d}_{\pi}(s_i) = k \Gamma(d/2 + 1) / n \pi^{d/2} R^d_{i,k,n}$, where $R_{i, k, n} = \norm{s_i - NN_k(s_i)}$ and $NN_k(s)$ denotes the $k$-NN of $s_i$ within $\{s_i\}_{i = 1, \ldots, n}$. Recalling that $w(x) = e^{-\beta(-\log(x))^{\alpha}}$, we can write
%
\begin{align}
    w(\widehat{d}_{\pi}(s_i)) &= e^{-\beta \left( -\left[ \log(k \Gamma(d/2 + 1)) - \log(n \pi^{d/2} R_{i, k, n}^d) \right] \right)^{\alpha}} \label{eqn:r_deriv:1} \\
    %
    &= e^{-\beta \left( d \log R_{i, k, n} + D_{k, n} \right)^\alpha}, \label{eqn:r_deriv:2}
\end{align}
%
where $D_{k,n} = - \log (k \Gamma(d/2 + 1)) + \log(n \pi^{d/2}) = \log \left( (n \pi^{d/2}) / (k \Gamma(d/2 + 1)) \right)$. Substituting \eqref{eqn:r_deriv:2} into \eqref{eqn:be_knn_approx} gives
%
\begin{align}
    \widehat{H}^{B,\alpha,\beta}_{k,n}(d_{\pi}) &= - \frac{1}{n} \sum_{i=1}^n \frac{1}{\widehat{d}_{\pi}(s_i)} w(\widehat{d}_{\pi}(s_i)) \log w(\widehat{d}_{\pi}(s_i)) \label{eqn:r_deriv:3} \\
    %
    &= \frac{\beta}{n} \sum_{i=1}^n \frac{n \pi^{d/2} R^d_{i,k,n}}{k \Gamma(d/2 + 1)} e^{-\beta (d \log R_{i,k,n} + D_{k,n})^{\alpha}} \left( d \log R_{i,k,n} + D_{k,n} \right)^{\alpha} \label{eqn:r_deriv:4} \\
    %
    &\propto \frac{1}{n} \sum_{i=1}^n R_{i,k,n}^d e^{-\beta ( d \log R_{i,k,n} + D_{k,n} )^{\alpha}} \left( d \log R_{i,k,n} + D_{k,n} \right)^{\alpha} \label{eqn:r_deriv:5} \\
    %
    &\appropto \frac{1}{n} \sum_{i=1}^n R_{i,k,n}^d e^{-\beta ( d \log R_{i,k,n} )^{\alpha}} \left( d \log R_{i,k,n} \right)^{\alpha}.\label{eqn:r_deriv:6}
\end{align}
%
where the approximate proportionality in \eqref{eqn:r_deriv:6} follows from the fact that, under suitable conditions on $n, k$ (see, e.g., Theorem \ref{thm:convergence}) the contribution of $D_{k,n}$ to the value of \eqref{eqn:r_deriv:5} is negligible.
%
Since $\mathbb{E}_{\pi} [ \widehat{H}^{B,\alpha,\beta}_{k,n}(d_{\pi}) ] \approx H^{B, \alpha, \beta} (d_{\pi})$ by Theorems \ref{thm:convergence} and \ref{thm:main_bound}, and since \eqref{eqn:r_deriv:6} is approximately proportional to $\widehat{H}^{B,\alpha,\beta}_{k,n}(d_{\pi})$, \eqref{eqn:r_deriv:6} suggests
%
\begin{equation} \label{eqn:r_deriv:7}
    \widetilde{r}(s, a) = \norm{s - NN_k(s)}^d e^{-\beta(d \log \norm{s - NN_k(s)})^{\alpha}} \left( d \log \norm{s - NN_k(s)} \right)^\alpha
\end{equation}
%
as a suitable proxy reward for maximizing behavioral entropy in an RL context. For numerical stability, we follow \citep{yarats2021reinforcement, liu2021behavior} by making the additional simplification of setting $d=1$ and adding a constant $c > 0$ inside the logarithms to obtain
%
\begin{equation} \label{eqn:r_final}
    r(s, a) = \norm{s - NN_k(s)} e^{-\beta(\log( \norm{s - NN_k(s)} + c ) )^{\alpha}} \left( \log( \norm{s - NN_k(s)} + c ) \right)^\alpha.
\end{equation}
%
A visualization of \eqref{eqn:r_final} with a comparison to the SE reward function is provided in Fig.~\ref{fig:be_rwrd_fn} in the appendix. Armed with this reward, any standard RL method can be applied to learn exploration policies approximately maximizing BE using \eqref{eqn:be_occ_meas}. We illustrate its application in data generation for offline RL in the next section.
%
\new{We note that implementing the $k$-NN estimator in \eqref{eqn:r_final} can be computationally challenging in high dimensions for large $k$ values due to the well-known curse of dimensionality of suffered by $k$-NN estimators \cite{beyer1999nearest}.
%
%
To address this, in practice $k \leq 15$ is selected and dimension reduction to a feature space of manageable dimensions is performed before $k$-NN estimation is carried out, thereby limiting computational costs \cite{liu2021behavior, yarats2021reinforcement}. 
%
}
\section{Experimental Results}
%
%
%
%
\begin{wraptable}{r}{0.75\textwidth}
    \vspace{-4mm}
    %
    \centering
    \begin{tabular}{|cc|ccccc|}
        \hline
        %
        Environment & Task & BE & RE & SE & RND & SMM \\
        %
        \hline
        %
        Walker & Stand & \textbf{990.38} & 988.93 & 954.93 & 947.89 & 496.09 \\
        %
        & Walk & \textbf{904.66} & 878.20 & 895.89 & 735.77 & 409.46  \\
        %
        & Run & 385.07 & \textbf{440.53} & 360.64 & 341.03 & 140.29  \\
        %
        \hline
        %
        Quadruped & Walk & \textbf{845.31} & 776.64 & 755.79 & 699.22 & 425.11  \\
        %
        & Run & \textbf{522.32} & 490.75 & 490.46 & 490.66 & 275.38  \\
        %
        \hline
    \end{tabular}
    \caption{\small \new{ Comparison of best offline RL performance across all datasets, training seeds, and offline RL algorithms.} }
    %
    %
    \label{tab:orl_performance}
    %
    \vspace{-3mm}
\end{wraptable}
%
% \vspace{5mm}
%
The experiments presented in this section (i) provide qualitative insights into the state space coverage achieved by policies that maximize BE, RE, and SE, and (ii) examine the utility of BE-maximizing policies for performing offline dataset generation for subsequent offline RL \new{compared with datasets generated using the SE and RE objectives and datasets generated using the RND and SMM algorithms}. The state space coverage visualizations that we present suggest that BE-generated datasets achieve a wider variety of coverage than RE- and SE-generated datasets, and that the RE objective is unstable as a function of $q$ and provides poor coverage for $q > 1$. \new{We provide coverage visualizations for RND and SMM in the appendix.} In our offline RL experiments, we demonstrate that offline learning on \new{BE datasets leads to superior performance over SE, RND, and SMM datasets on all five tasks,} and superior performance to RE datasets on four out of five tasks (see Table \ref{tab:orl_performance}).



\textbf{Experimental Setup.} For our experiments, we generated BE, RE, SE, RND, and SMM datasets for the Walker and Quadruped environments using the Unsupervised Reinforcement Learning Benchmark (URLB) framework \citep{laskin2021urlb}\footnote{\url{https://github.com/rll-research/url_benchmark}}. We subsequently generated t-SNE plots \citep{hinton2002stochastic} and PHATE plots \citep{moon2019visualizing} from the BE, RE, SE, RND, and SMM datasets to visualize their varying state space coverage. Finally, we performed offline RL training on all datasets using the Exploratory Data for Offline RL (ExORL) framework \citep{yarats2022don}\footnote{\url{https://github.com/denisyarats/exorl}}. We emphasize that the datasets we generated contained just 500K elements, only 5\% as many as the 10M-element datasets considered in the ExORL framework \citep{yarats2022don}, and that we performed just 100K offline training steps, only 20\% of the 500K performed in ExORL. Despite these limitations, we achieved comparable performance to that achieved in \citep{yarats2022don}, indicating that using BE-generated data for subsequent offline RL leads to significant improvements in both data- and sample-efficiency.



\begin{wrapfigure}{r}{0.45\textwidth}
    %
    \vspace{-3mm}
    % \begin{figure}[]
    \begin{subfigure}[b]{0.98\linewidth}
        \includegraphics[width=\linewidth, height=0.9\linewidth]{figs/phate_gbe_walker_full_v2.png}
        \caption{PHATE plots for BE for Walker.}
        \label{fig:phate_gbe_walker_full}
    \end{subfigure}
    \centering

    \begin{subfigure}[b]{0.98\linewidth}
        \includegraphics[width=\linewidth, height=0.6\linewidth]{figs/phate_renyi_walker_full_v2.png}
        \caption{PHATE plots for Renyi for Walker. Coverage for $q = 3.0, 5.0$ similar to $q = 2.0$.}
        \label{fig:phate_renyi_walker_full}
    \end{subfigure}
    \caption{\small PHATE plots for Walker tasks.}
    \label{fig:phate_walker_full}
    % \end{figure}
    \vspace{-5mm}
\end{wrapfigure}
%
\noindent\textbf{Dataset Generation and Visualization.} For dataset generation, we used the Active Pre-Training (APT) algorithm \citep{liu2021behavior} implemented in the URLB framework to maximize BE using the reward proposed in \eqref{eqn:r_final} for various values of $\alpha$, RE using the reward proposed in \cite{zhang2021exploration} for various values of $q$, and SE using the default reward from \cite{liu2021behavior}. Specifically, for behavioral and RE we considered $\alpha \in \{ 0.2, 0.5, 0.7, 0.9, 1.5, 2.0, 3.0, 5.0 \}$ and $q \in \{ 0.2, 0.5, 0.7, 0.9, 1.1, 2.0, 3.0, 5.0 \}$.
%
\new{To ensure admissibility of the behavioral entropies we considered, we used the conditioning $\beta=e^{(1-\alpha)\log(\log(M))}$ from \cite{suresh2024robotic}, where $M$ is the dimensions of the representation space. See the discussion following \eqref{eqn:behavioral_entropy} in Section \ref{sec:be_in_continuous_spaces} for details.}
%
%
For each of the $\alpha$ and $q$ values, as well as for SE, we trained APT on the corresponding reward for 500K pretraining steps on both the Walker and Quadruped environments, collecting the resulting trajectories to form our datasets. This resulted in 17 datasets for each environment, for a total of 34 datasets. To ensure a fair comparison across entropies, for the APT hyperparameters we used the default URLB pretraining hyperparameter values across all datasets (see Table \ref{tab:data_gen_hyperparams} in the appendix). \new{For the datasets generated using the Random Network Distillation (RND) \citep{burda2019exploration} and State Marginal Matching (SMM) \cite{lee2019efficient} algorithms, we similarly trained for 500K pretraining steps and for consistency we used the same RND and SMM hyperparameters considered in URLB.}

%
To provide qualitative insight into the state space coverage of the SE, RE, and BE datasets, we generated two-dimensional $t$-SNE \citep{hinton2002stochastic} and PHATE \citep{moon2019visualizing} plots of the trajectories they contain.\footnote{ \new{ 10K representative samples from each dataset were gathered uniformly, totaling 170K samples for each domain from $17$ datasets. $t$-SNE and PHATE were implemented on this aggregated 170K-element dataset to ensure uniformity in projections for each dataset and then correspondingly represented individually for clarity.} } \new{ We also generated $t$-SNE and PHATE plots for the RND and SMM datasets, pictured in Figure \ref{fig:smm_rnd_full} in the appendix.}
%
% The $t$-SNE and PHATE techniques are well-known non-linear dimensionality reduction techniques for visualizing high-dimensional data.
%
While $t$-SNE has been previously used to visualize RL trajectory data \citep{zhang2021exploration}, to our knowledge this is the first time PHATE plots have been used to visualize such data. PHATE tends to better retain global structure such as the temporal nature of trajectory data, while $t$-SNE obscures it \citep{moon2019visualizing}. Figure \ref{fig:phate_walker_full} indicates that while both BE and RE-generated datasets provide more flexible levels of state space coverage than SE, BE-generated datasets achieve a wider variety of coverage than RE. See the appendix for $t$-SNE and PHATE plots for the remaining datasets. Importantly, these plots indicate that the RE objective is unstable as a function of $q$ and provides poor coverage for $q > 1$, while the level of coverage provided by BE varies smoothly in $\alpha$. We provide experimental support for this in the following section, where highly unstable offline RL performance on RE datasets and the contrasting stability on BE datasets is illustrated in Figure \ref{fig:orl_performance}.
%



\begin{figure}[!htp]
    %
    \begin{subfigure}{.95\textwidth}
    % \captionsetup{font=footnotesize,labelfont=scriptsize,textfont=scriptsize}
    \includegraphics[width=\textwidth]{figs/quadruped_walk_barplot_with_shannon3M_SMM_RND.png}
    % \caption{TEST}
    \label{fig:orl:quad_walk}
    \end{subfigure}
    %
    \vspace{-4mm}
    %
    \begin{subfigure}{.95\textwidth}
    % \captionsetup{font=footnotesize,labelfont=scriptsize,textfont=scriptsize}
    \includegraphics[width=\textwidth]{figs/quadruped_run_barplot_with_shannon3M_SMM_RND.png}
    % \caption{TEST}
    \label{fig:orl:quad_run}
    \end{subfigure}
    %
    \vspace{-4mm}
    %
    \begin{subfigure}{.95\textwidth}
    % \captionsetup{font=footnotesize,labelfont=scriptsize,textfont=scriptsize}
    \includegraphics[width=\textwidth]{figs/walker_stand_barplot_with_shannon3M_SMM_RND.png}
    % \caption{TEST}
    \label{fig:orl:walker_stand}
    \end{subfigure}
    %
    \vspace{-4mm}
    %
    \begin{subfigure}{.95\textwidth}
    % \captionsetup{font=footnotesize,labelfont=scriptsize,textfont=scriptsize}
    \includegraphics[width=\textwidth]{figs/walker_walk_barplot_with_shannon3M_SMM_RND.png}
    % \caption{TEST}
    \label{fig:orl:walker_walk}
    \end{subfigure}
    %
    \vspace{-4mm}
    %
    \begin{subfigure}{.95\textwidth}
    % \captionsetup{font=footnotesize,labelfont=scriptsize,textfont=scriptsize}
    \includegraphics[width=\textwidth]{figs/walker_run_barplot_with_shannon3M_SMM_RND.png}
    % \caption{TEST}
    \label{fig:orl:walker_run}
    \end{subfigure}
    %
    \vspace{-4mm}
    %
    \caption{\small \new{Comparison of offline RL performance over the entropy objectives used in dataset generation. Plots show mean and standard deviation over five seeds. Dotted line shows performance of RL policy trained online until approximate optimality.} }
    \label{fig:orl_performance}
\end{figure} 




%
\noindent\textbf{Offline RL Experiments.} We compared the TD3, CQL, and CRR offline RL algorithms \citep{fujimoto2018addressing, kumar2020conservative, wang2020critic} implemented in the ExORL framework on the datasets generated as described above for all eight $\alpha$ values and for $q \in \{ 0.2, 0.5, 0.7, 0.9, 1.1\}$. We omitted offline RL training for RE datasets with $q \in \{2.0, 3.0, 5.0\}$ after observing in initial trials that performance was no better than for $q = 1.1$ and typically worse (see poor performance on $q = 1.1$ datasets in Figure \ref{fig:orl_performance}). \new{To gain insight into the effect of using larger datasets, we also considered a 3M-element SE-generated dataset. Altogether we considered 17 datasets: eight BE, five BE, two SE, and one each for RND and SMM.} % We omitted the behavioral cloning (BC) and BC+TD3 algorithms implemented in ExORL due to the poor performance reported by \cite{yarats2022don}.
%
%
On the Walker datasets we considered the Stand, Walk, and Run tasks, while on Quadruped we considered the Walk and Run tasks. \new{ For each of the $5 \times 17 = 85$ task-dataset combinations, we trained each of TD3, CQL, and CRR for 100K offline training steps, evaluating performance every 10K training steps. For each of the $255$ task-dataset-algorithm combinations, we repeated this training process for a total of 5 different seeds, resulting in our training $1275$ offline RL policies altogether. To ensure a fair comparison across all entropies, we used default ExORL hyperparameter values across all datasets (see Table \ref{tab:orl_hyperparams} in the appendix). }

As summarized in Table \ref{tab:orl_performance}, offline RL training on BE-generated datasets leads to superior performance over SE-, RND-, and SMM-generated datasets on all five tasks we considered, and superior performance to RE-generated datasets on four out of five tasks. Figure \ref{fig:orl_performance} provides a detailed overview of the experimental results for $\alpha \in \{0.2, 0.5, 0.7, 0.9, 1.5\}$ and $q \in \{0.2, 0.5, 0.7, 0.9, 1.1\}$ (complete results for all $\alpha$ values are shown in the appendix). This figure illustrates that BE-generated datasets lead to significantly better performance over the other methods on Quadruped Walk and Walker Walk for the best-performing $\alpha$ values, while offline RL performance on RE datasets for the best-performing values of $q$ is only slightly below that of BE datasets in Quadruped Run and Walker Stand. These trends hold across all algorithms for each of the tasks. On Walker Run, the best-performing RE parameter clearly leads to superior offline RL performance over both BE and SE datasets in the TD3 and CQL trials, but performance on BE datasets is again better than on RE datasets in the CRR trials. \new{Performance on SMM datasets is clearly inferior across all tasks considered. Performance on RND datasets is inferior on Walker tasks, but is almost competitive with BE on Quadruped tasks. Interestingly, the 3M-element SE datasets lead to strong downstream performance on Walker Stand and improved downstream performance over the 500K-element SE datasets on the Walker Stand and Run tasks, but the 3M-element SE datasets actually lead to worse performance compared with the 500K-element SE datasets on both Quadruped tasks and TD3 performance on Walker Walk.} Overall, best offline RL performance on BE-generated datasets clearly exceeds best performance on RE datasets on 13 out of 15 task-algorithm combinations and best performance on SE, RND, and SMM datasets on 15 out of 15 task-algorithm combinations.

We observed sensitivity of performance to parameters $\alpha$ and $q$ as well as choice of offline RL algorithm. Regarding the latter, notice in Figure \ref{fig:orl_average_appendix} in the appendix that on Walker Walk the SE-generated datasets are competitive with the average and best-performing BE datasets in the TD3 trials, while BE datasets significantly outperform SE ones in both the CQL and CRR trials. On Quadruped Run, on the other hand, performance on BE and SE datasets remains roughly the same across all algorithms, while average RE performance is significantly worse (see appendix for average performance plots for all tasks). These results suggest that offline RL algorithm performance depends in a complex way on the choice of exploration objective used in dataset generation. Well-performing, flexible objectives such as BE -- and to a lesser but still significant extent, RE -- therefore merit additional study as tools for dataset generation for offline RL.
We present RiskHarvester, a risk-based tool to compute a security risk score based on the value of the asset and ease of attack on a database. We calculated the value of asset by identifying the sensitive data categories present in a database from the database keywords. We utilized data flow analysis, SQL, and Object Relational Mapper (ORM) parsing to identify the database keywords. To calculate the ease of attack, we utilized passive network analysis to retrieve the database host information. To evaluate RiskHarvester, we curated RiskBench, a benchmark of 1,791 database secret-asset pairs with sensitive data categories and host information manually retrieved from 188 GitHub repositories. RiskHarvester demonstrates precision of (95\%) and recall (90\%) in detecting database keywords for the value of asset and precision of (96\%) and recall (94\%) in detecting valid hosts for ease of attack. Finally, we conducted an online survey to understand whether developers prioritize secret removal based on security risk score. We found that 86\% of the developers prioritized the secrets for removal with descending security risk scores.

\bibliographystyle{iclr2025_conference}
\bibliography{iclr2025_conference}



\newpage
\appendix
\section{Appendix}

\subsection{Proofs}

Fix a probability weighting function $w$ and let $g(y) = -\frac{1}{y} \log(w(y)) w(y)$. Fix a p.d.f. $f \in \Delta(\mc{X})$, where $\mc{X} \subset \mathbb{R}^d$ is compact. Fix $n, k \in \mathbb{N}$, and let $X_1, \ldots, X_n \sim f(\cdot)$. Recall the definition of $\hat{f}$ from \eqref{eqn:knn_f}. Let $\mu$ denote the Lebesgue measure and $B_r(x) = \{ x' \in \mathbb{R}^d \ | \ \norm{x' - x} < r \}$. Define
%
\begin{align}
    H^{B,w}(f) &= - \int_{\mc{X}} \log(w(f(x))) w(f(x)) dx \int_{\mc{X}} g(f(x)) f(x) dx, \\
    %
    H^{B,w}_n(f) &= - \sum_{i=1}^n \frac{1}{f(x)} \log(w(f(X_i))) w(f(X_i)) = \frac{1}{n} \sum_{i=1}^n g(f(X_i)), \\
    %
    \widehat{H}^{B,w}_{k,n}(f) &= - \sum_{i=1}^n \frac{1}{\hat{f}(x)} \log(w(\hat{f}(X_i))) w(\hat{f}(X_i)) = \frac{1}{n} \sum_{i=1}^n g(\hat{f}(X_i)).
\end{align}
%
Our goal is to establish a bound on the error
%
\begin{equation} \label{eqn:goal_to_bound}
    \left| \mathbb{E} \left[ \widehat{H}^{B,w}_{k,n}(f) \right] - H^{B,w}(f) \right|.
\end{equation}
%
In general, for finite $k$, even as $n \rightarrow \infty$ the approximator $\widehat{H}^{B,w}_{k,n}(f)$ will remain biased due to the biasedness of $\hat{f}$ for fixed $k$ and the lack of a known bias correction procedure for our BE approximator $\widehat{H}^{B,w}_{k,n}(f)$. This contrasts with the situation for simpler estimators like Shannon and R\'{e}nyi entropies, for which explicit bias correction terms are known (see \cite{singh2003nearest, leonenko2008class, singh2016finite}). Nonetheless, in Theorem \ref{thm:main_bound} we are able to build on existing results to establish a probabilistic bound on \eqref{eqn:goal_to_bound}. We first recall the following result.
%
\begin{lemma}[\citep{singh2016finite}] \label{lem:singh_poczos}
    Suppose that, for some $\xi \in (0, 2]$, $f$ is $\xi$-H\"{o}lder continuous and strictly positive on $\mc{X}$. Suppose furthermore that there exists a function $f_* : \mc{X} \rightarrow \mathbb{R}^+$ and a constant $f^*$ such that $0 < f_*(x) \leq \int_{B_r(x)} f(y) dy / \mu(B_r(x)) \leq f^* < \infty$, for all $x \in \mc{X}, r \in (0, \sqrt{d}]$, and assume that $\int_0^{\infty} e^{-x} x^k f(x) dx < \infty$. Then
    %
    % \begin{align}
    %     \left| \mathbb{E} \left[ \widehat{H}^{B,w}_{k,n}(f) \right] - H^{B,w}(f) \right| &= \mc{O}\left( \frac{k}{n} \right)^{\frac{\xi}{d}}, \\
    %     %
    %     \text{Var} \left( \widehat{H}^{B,w}_{k,n}(f) \right) &= \mc{O}\left( \frac{C_V}{n} \right).
    % \end{align}
    %
    \begin{multicols}{2}
        \noindent
        \small
        \vspace{-8mm}
        \begin{equation} \label{eqn:bias_bound}
            \left| \mathbb{E} \left[ H^{B,w}_n(f) \right] - H^{B,w}(f) \right| = \mc{O}\left( \frac{k}{n} \right)^{\frac{\xi}{d}},
        \end{equation}
        \normalsize
        %
        % \break
        %
        \noindent
        \small
        \begin{equation} \label{eqn:var_bound}
            \text{Var} \left( H^{B,w}_n(f) \right) = \mc{O}\left( \frac{1}{n} \right).
        \end{equation}
        \normalsize
    \end{multicols}
\end{lemma}
%
The proof of this result follows directly from that of \citep[Thm. 5]{singh2016finite} due to the fact that $H^{B,w}_n(f)$ is an unbiased estimator of $H^{B,w}(f)$. Also note that the variance bound can be trivially strengthened to apply to $\widehat{H}^{B,w}_{k,n}(f)$ due to the fact that the latter is simply the sample average of $n$ i.i.d., bounded random variables:
%
\begin{corollary}
    Under the conditions of Lemma \ref{lem:singh_poczos}, $\text{Var} \left( \widehat{H}^{B,w}_{k,n}(f) \right) = \mc{O}\left( \frac{1}{n} \right).$
\end{corollary}

It remains to characterize \eqref{eqn:goal_to_bound}. We first recall another useful result from the literature. For a given set $S \subset \mc{X}$, radius $r$, and $m > 0$, let $\mc{N} \left( S, r \right)$ denote the covering number, the minimum number of balls of radius $r$ needed to cover $S$. Let $\normop{\cdot}$ denote the operator norm.
%
\begin{lemma}[\citep{zhao2022analysis}] \label{lem:zhao_lai}
    Suppose there exist $C_1, C_2, C_3, \mc{N}_0 > 0$ and $\beta \in (0, 1]$ such that the following conditions hold:
    \begin{align*}
        &(i) \quad \frac{\norm{ \nabla f(x) }}{f(x)} \leq C_1; \quad (ii) \quad \frac{ \normop{ \nabla^2 f(x) } }{f(x)} \leq C_2; \quad (iii) \quad \forall t > 0, P(f(x) < t) \leq C_3 t^{\beta}; \\
        %
        &(iv) \quad \mc{N} \left( \{ x | f(x) > m \}, r \right) \leq \frac{\mc{N}_0}{m^{\gamma} r^d}, \text{ for some } \gamma > 0 \text{ and all } m > 0.
    \end{align*} 
    %
    Then, for $\varepsilon > 0$, it holds with probability (w.p.) $1 - \varepsilon$ that
    %
    \begin{equation} \sup_x \left| \hat{f}(x) - f(x) \right| =
        \begin{cases}
            \mc{O}\left( \left( \frac{k}{n} \right)^{\frac{2}{d}} \log n + \sqrt{ \frac{ \log ( n / \varepsilon ) }{k} } \right) & \text{ if } d > 2, \\
            %
            \mc{O}\left( \frac{k}{n} \log n + \sqrt{ \frac{ \log ( n / \varepsilon ) }{k} } \right) & \text{ if } d = 1, 2.
        \end{cases}
    \end{equation}
\end{lemma}

We are now in a position to prove our main result.
%
\begin{customthm}{2} \label{eqn:main_bound}
    Suppose $f$ satisfies the conditions of Lemmas \ref{lem:singh_poczos} and \ref{lem:zhao_lai}. Assume $w$ is Lipschitz continuous. Then, for $\varepsilon > 0$, it holds w.p. $1 - \varepsilon$ that
    %
    \begin{equation} \left| \mathbb{E} \left[ \widehat{H}^{B,w}_{k,n}(f) \right] - H^{B,w}(f) \right| = 
        \begin{cases}
            \mc{O}\left( \frac{k}{n} \right)^{\frac{\xi}{d}} + \mc{O}\left( \left( \frac{k}{n} \right)^{\frac{2}{d}} \log n + \sqrt{ \frac{ \log ( n / \varepsilon ) }{k} } \right) & \text{ if } d > 2, \\
            %
            \mc{O}\left( \frac{k}{n} \right)^{\frac{\xi}{d}} + \mc{O}\left( \frac{k}{n} \log n + \sqrt{ \frac{ \log ( n / \varepsilon ) }{k} } \right) & \text{ if } d = 1, 2.
        \end{cases}
    \end{equation}
\end{customthm}
%
\begin{proof}
    First notice that
    %
    \begin{equation} \label{eq:0}
        \left| \mathbb{E} \left[ \widehat{H}^{B,w}_{k,n}(f) \right] - H^{B,w}(f) \right| \leq \left| \mathbb{E} \left[ \widehat{H}^{B,w}_{k,n}(f) - H^{B,w}_n(f) \right] \right| + \left| \mathbb{E} \left[ H^{B,w}_n(f) \right] - H^{B,w}(f) \right|.
    \end{equation}
    %
    The second term can be bounded using Lemma \ref{lem:singh_poczos}, so it just remains to bound the first term. Recall that $\mc{X}$ is compact, $f$ is bounded strictly away from 0 on $\mc{X}$, and $w$ is Lipschitz. We therefore have that $g$ is the product of Lipschitz, bounded functions and is therefore itself Lipschitz on its domain. Let $K$ denote the minimal Lipschitz parameter of $g$. Rewriting \eqref{eq:0} in terms of $g$, we obtain
    %
    \begin{align}
        \left| \mathbb{E} \left[ \widehat{H}^{B,w}_{k,n}(f) - H^{B,w}_n(f) \right] \right| &= \left| \frac{1}{n} \sum_{i=1}^n \mathbb{E} \left[ g(\hat{f}(X_i)) - g(f(X_i)) \right] \right| \\
        %
        &\leq \frac{1}{n} \sum_{i=1}^n \mathbb{E} \left[ \left| g(\hat{f}(X_i)) - g(f(X_i)) \right| \right] \\
        %
        &\labelrel={eq:2} \mathbb{E} \left[ \left| g(\hat{f}(X_1)) - g(f(X_1)) \right| \right] \\
        %
        &\leq K \mathbb{E} \left[ \left| \hat{f}(X_1) - f(X_1) \right| \right] \\
        %
        &\leq K \mathbb{E} \left[ \sup_x \left| \hat{f}(x) - f(x) \right| \right] % \\
        %
        % &\leq K \sup_x \left| \hat{f}(x) - f(x) \right|,
    \end{align}
    %
    where \eqref{eq:2} follows from the fact that the $X_1, \ldots, X_n$ are i.i.d. An application of the law of total probability and Lemma \ref{lem:zhao_lai} to the last term completes the proof.
\end{proof}
\newpage
\subsection{Hyperparameters}

\begin{table}[ht]
    \centering
    \begin{tabular}{|c|c|}
        \hline
        %
        Optimization hyperparameter & Value \\
        %
        \hline
        %
        replay buffer capacity & $10^6$ \\
        %
        mini-batch size & 1024 \\
        %
        agent update frequency & 2 \\
        %
        discount factor ($\gamma$) & 0.99 \\
        %
        optimizer & Adam \\
        %
        learning rate & $10^{-4}$ \\
        %
        critic target rate ($\tau$) & 0.01 \\
        %
        exploration stddev clip & 0.3 \\
        %
        exploration stddev value & 0.2 \\
        %
        \hline
        %
        APT hyperparameter & \\
        %
        \hline
        %
        forward net architecture & $(512 + \text{dim}(\mc{A})) \rightarrow 1024 \rightarrow 512$ ReLU MLP \\
        %
        inverse net architecture & $(2 \times 512) \rightarrow 1024 \rightarrow \text{dim}(\mc{A})$ ReLU MLP \\
        %
        representation dimension & 512 \\
        %
        $k$ in NN approximator & 12 \\
        %
        average top $k$ in NN & True \\
        %
        \hline
        %
        RND hyperparameter & \\
        %
        \hline
        %
        representation dimension & 512 \\
        %
        predictor, target network architecture & $\text{dim}(\mathcal{S}) \rightarrow 1024 \rightarrow 1024 \rightarrow 512$ ReLU MLP \\
        %
        normalized observation clipping & 5 \\
        %
        \hline
        %
        SMM hyperparameter & \\
        %
        \hline
        %
        skill dimension & 4 \\
        %
        skill discriminator learning rate & $10^{-3}$ \\
        %
        VAE learning rate & $10^{-2}$ \\
        %
        \hline
    \end{tabular}
    \caption{Data generation hyperparameters}
    \label{tab:data_gen_hyperparams}
\end{table}



\begin{table}[h]
    \centering
    \begin{tabular}{|c|c|}
        \hline
        %
        Shared hyperparameter & Value \\
        %
        \hline
        %
        replay buffer capacity & $10^6$ \\
        %
        mini-batch size & 1024 \\
        %
        agent update frequency & 2 \\
        %
        discount factor ($\gamma$) & 0.99 \\
        %
        optimizer & Adam \\
        %
        number of hidden layers & 2 \\
        %
        hidden dimension & 1024 \\
        %
        learning rate & $10^{-4}$ \\
        %
        critic target rate ($\tau$) & 0.01 \\
        %
        training steps & $10^5$ \\
        %
        \hline
        %
        TD3 hyperparameter &  \\
        %
        \hline
        %
        stddev clip & 0.3 \\
        %
        \hline
        %
        CQL hyperparameter &  \\
        %
        \hline
        %
        CQL-specific $\alpha$ & 0.01 \\
        %
        Lagrange & False \\
        %
        number of sample actions & 3 \\
        %
        \hline
        %
        CRR hyperparameter &  \\
        %
        \hline
        %
        number of samples to estimate $V$ & 10 \\
        %
        transformation & indicator \\
        %
        \hline
    \end{tabular}
    \caption{Offline RL hyperparameters}
    \label{tab:orl_hyperparams}
\end{table}
\newpage
\subsection{Additional Offline RL Experiments}

% \subsubsection{Complete Offline RL Results}

\begin{figure}[htp]
    \begin{subfigure}{\textwidth}
    % \captionsetup{font=footnotesize,labelfont=scriptsize,textfont=scriptsize}
    \includegraphics[width=\textwidth]{figs/offline_RL_all_alphas/quadruped_walk_barplot_with_shannon3M_SMM_RND_all_alphas.png}
    % \caption{TEST}
    \label{fig:orl_appendix:quad_walk}
    \end{subfigure}
    %
    \vspace{-4mm}
    %
    \begin{subfigure}{\textwidth}
    % \captionsetup{font=footnotesize,labelfont=scriptsize,textfont=scriptsize}
    \includegraphics[width=\textwidth]{figs/offline_RL_all_alphas/quadruped_run_barplot_with_shannon3M_SMM_RND_all_alphas.png}
    % \caption{TEST}
    \label{fig:orl_appendix:quad_run}
    \end{subfigure}
    %
    \vspace{-4mm}
    %
    \begin{subfigure}{\textwidth}
    % \captionsetup{font=footnotesize,labelfont=scriptsize,textfont=scriptsize}
    \includegraphics[width=\textwidth]{figs/offline_RL_all_alphas/walker_stand_barplot_with_shannon3M_SMM_RND_all_alphas.png}
    % \caption{TEST}
    \label{fig:orl_appendix:walker_stand}
    \end{subfigure}
    %
    \vspace{-4mm}
    %
    \begin{subfigure}{\textwidth}
    % \captionsetup{font=footnotesize,labelfont=scriptsize,textfont=scriptsize}
    \includegraphics[width=\textwidth]{figs/offline_RL_all_alphas/walker_walk_barplot_with_shannon3M_SMM_RND_all_alphas.png}
    % \caption{TEST}
    \label{fig:orl_appendix:walker_walk}
    \end{subfigure}
    %
    \vspace{-4mm}
    %
    \begin{subfigure}{\textwidth}
    % \captionsetup{font=footnotesize,labelfont=scriptsize,textfont=scriptsize}
    \includegraphics[width=\textwidth]{figs/offline_RL_all_alphas/walker_run_barplot_with_shannon3M_SMM_RND_all_alphas.png}
    % \caption{TEST}
    \label{fig:orl_appendix:walker_run}
    \end{subfigure}
    %
    \vspace{-4mm}
    %
    \caption{ \new{Offline RL results for all $\alpha$ and $q$ values evaluated. Initial trials showed $q \in \{2.0, 3.0, 5.0\}$ led to performance no better (and usually worse) than $q = 1.1$, so offline RL training for these $q$ values was not performed.} }
    \label{fig:orl_appendix}
\end{figure}



\begin{figure}
    \begin{subfigure}{0.49\textwidth}
        \includegraphics[width=\textwidth]{figs/quadruped_walk_barplot_100K_vs_200K_Shannon.png}
        % \caption{TEST}
        \label{fig:orl_appendix:quad_walk_average}
    \end{subfigure}
    \hfill
    \begin{subfigure}{0.49\textwidth}
        \includegraphics[width=\textwidth]{figs/quadruped_run_barplot_100K_vs_200K_Shannon.png}
        % \caption{TEST}
        \label{fig:orl_appendix:quad_run_average}
    \end{subfigure}
    
    \medskip
    \begin{subfigure}{0.49\textwidth}
        \includegraphics[width=\textwidth]{figs/walker_stand_barplot_100K_vs_200K_Shannon.png}
        % \caption{TEST}
        \label{fig:orl_appendix:walker_stand_average}
    \end{subfigure}
    \hfill
    \begin{subfigure}{0.49\textwidth}
        \includegraphics[width=\textwidth]{figs/walker_walk_barplot_100K_vs_200K_Shannon.png}
        % \caption{TEST}
        \label{fig:orl_appendix:walker_walk_average}
    \end{subfigure}
    
    \medskip
    \centering
    \begin{subfigure}{0.49\textwidth}
        \includegraphics[width=\textwidth]{figs/walker_run_barplot_100K_vs_200K_Shannon.png}
        % \caption{TEST}
    \end{subfigure}
    \caption{ \new{ Ablation result comparing the effect of performing 100K vs. 200K offline RL training steps on a 3M-element dataset generated using Shannon entropy as exploration objective. These results suggest that performing additional offline RL training has only a marginal effect on downstream task performance.} }
    \label{fig:orl_average_appendix}
\end{figure}



\begin{figure}
    \begin{subfigure}{0.49\textwidth}
        \includegraphics[width=\textwidth]{figs/quadruped_walk_average_barplot.png}
        % \caption{TEST}
        \label{fig:orl_appendix:quad_walk_average}
    \end{subfigure}
    \hfill
    \begin{subfigure}{0.49\textwidth}
        \includegraphics[width=\textwidth]{figs/quadruped_run_average_barplot.png}
        % \caption{TEST}
        \label{fig:orl_appendix:quad_run_average}
    \end{subfigure}
    
    \medskip
    \begin{subfigure}{0.49\textwidth}
        \includegraphics[width=\textwidth]{figs/walker_stand_average_barplot.png}
        % \caption{TEST}
        \label{fig:orl_appendix:walker_stand_average}
    \end{subfigure}
    \hfill
    \begin{subfigure}{0.49\textwidth}
        \includegraphics[width=\textwidth]{figs/walker_walk_average_barplot.png}
        % \caption{TEST}
        \label{fig:orl_appendix:walker_walk_average}
    \end{subfigure}
    
    \medskip
    \centering
    \begin{subfigure}{0.49\textwidth}
        \includegraphics[width=\textwidth]{figs/walker_run_average_barplot.png}
        % \caption{TEST}
    \end{subfigure}
    \caption{Offline RL results averaged over all $\alpha, q$ values.}
    \label{fig:orl_average_appendix}
\end{figure}










\newpage
\subsection{Quantitative Coverage Experiments}

\begin{figure}[]
    \begin{subfigure}[b]{0.98\textwidth}
        \includegraphics[width=\linewidth]{figs/coverage_radii_walker.png}
        \caption{ \new{Volumetric coverage for data generation on Walker} }
        \label{fig:tsne_gbe_walker_full}
    \end{subfigure}
    \centering
    
    \begin{subfigure}[b]{0.98\textwidth}
        \includegraphics[width=\linewidth]{figs/coverage_radii_quadruped.png}
        \caption{ \new{Volumetric coverage for data generation on Quadruped} }
        \label{fig:tsne_renyi_quadruped_full}
    \end{subfigure}
    \caption{ \new{Visualization of evolution of smallest hypersphere radius $\overline{r}$ (normalized by the maximum radius achieved over all datasets) over the course of data generation training step $T$ for the Walker and Quadruped domains. We refer to this coverage metric as \textit{volumetric coverage}. Welzl's algorithm was used to determine the radius $\overline{r}$. 10K data points were sampled uniformly from every 50K iteration increment and cumulatively added to get a total of 100K samples for 500K iterations. Volumetric coverage varies considerably with the choice of parameters and data generation methods. For the range of parameters $\alpha$ and $q$ that we considered, BE exhibits higher volumetric coverage than RE on average on the problems under consideration. SE volumetric coverage was about average, while RND and SMM volumetric coverage differed sharply across domains: RND outperformed all other data generation methods on Walker, while SMM was not far behind; on Quadruped, on the other hand, both underperformed. Values of $q < 1$ enjoy higher volumetric coverage for RE on both tasks, while values of $\alpha < 1$ enjoy higher volumetric coverage for BE on Quadruped; since $q < 1, \alpha < 1$ tend to correspond to superior downstream offline RL performance (see Fig. \ref{fig:orl_appendix}), this suggests volumetric may be positively correlated with performance on downstream tasks. We also note note that RE with $q>1$ in general shows both poor volumetric coverage and poor qualitative coverage (PHATE and t-SNE), which might correspond to its poor performance in all tasks. These relationships are not conclusive, however, and further investigation is needed.} }
    \label{fig:tsne_quadruped_full}
\end{figure}


\newpage
\subsection{Additional Qualitative Visualizations}

\begin{figure}[ht]
    \begin{subfigure}[b]{0.98\textwidth}
        \includegraphics[width=\linewidth]{figs/tsne_gbe_walker_full.png}
        \caption{TSNE plots for GBE for Walker}
        \label{fig:tsne_gbe_walker_full}
    \end{subfigure}
    \centering
    
    \begin{subfigure}[b]{0.98\textwidth}
        \includegraphics[width=\linewidth,height=3in]{figs/tsne_renyi_walker_full.png}
        \caption{TSNE plots for Renyi for Walker}
        \label{fig:tsne_renyi_walker_full}
    \end{subfigure}
    \caption{TSNE plots for Walker }
    \label{fig:TSNE_walker_full}

\end{figure}



\begin{figure}[]
    \begin{subfigure}[b]{0.98\textwidth}
        \includegraphics[width=\linewidth]{figs/phate_gbe_quadruped_full.png}
        \caption{PHATE plots for GBE for Quadruped}
        \label{fig:phate_gbe_walker_full}
    \end{subfigure}
    \centering
    
    \begin{subfigure}[b]{0.98\textwidth}
        \includegraphics[width=\linewidth,height=3in]{figs/phate_renyi_quadruped_full.png}
        \caption{PHATE plots for Renyi for Quadruped}
        \label{fig:phate_renyi_quadruped_full}
    \end{subfigure}
    \caption{PHATE plots for Quadruped }
    \label{fig:phate_quadruped_full}
\end{figure}

\begin{figure}[]
    \begin{subfigure}[b]{0.98\textwidth}
        \includegraphics[width=\linewidth]{figs/tsne_gbe_quadruped_full.png}
        \caption{TSNE plots for GBE for Quadruped}
        \label{fig:tsne_gbe_walker_full}
    \end{subfigure}
    \centering
    
    \begin{subfigure}[b]{0.98\textwidth}
        \includegraphics[width=\linewidth,height=3in]{figs/tsne_renyi_quadruped_full.png}
        \caption{TSNE plots for Renyi for Quadruped}
        \label{fig:tsne_renyi_quadruped_full}
    \end{subfigure}
    \caption{TSNE plots for Quadruped }
    \label{fig:tsne_quadruped_full}
\end{figure}


\begin{figure}[ht]
    \begin{subfigure}[b]{0.48\textwidth}
        \includegraphics[width=\linewidth]{figs/tsne_walker_misc_v2.png}
        \caption{t-SNE plots for RND and SMM for Walker}
        \label{fig:tsne_misc_walker_full}
    \end{subfigure}
    \centering
    ~
    \begin{subfigure}[b]{0.48\textwidth}
        \includegraphics[width=\linewidth]{figs/tsne_quadruped_misc.png}
        \caption{t-SNE plots for RND and SMM for Quadruped}
        \label{fig:tsne_misc_quadruped_full}
    \end{subfigure}

    \begin{subfigure}[b]{0.48\textwidth}
        \includegraphics[width=\linewidth]{figs/phate_walker_misc_v2.png}
        \caption{PHATE plots for RND and SMM for Walker}
        \label{fig:phate_misc_walker_full}
    \end{subfigure}
    \centering
    ~
    \begin{subfigure}[b]{0.48\textwidth}
        \includegraphics[width=\linewidth]{figs/phate_misc_quadruped.png}
        \caption{PHATE plots for RND and SMM for Quadruped}
        \label{fig:phate_misc_quadruped_full}
    \end{subfigure}
    \caption{\new{Qualitative visualization of SMM and RND for data generation}}
    \label{fig:smm_rnd_full}

\end{figure}



\newpage
\subsection{BE Reward Function Visualization}
%
\begin{figure}[ht]
    \centering
    \includegraphics[width=0.99\linewidth]{figs/reward_fn_BE.png}
    \caption{ \new{ Visualization of the BE reward function \eqref{eqn:r_final} by varying the parameter $\alpha$ with $\beta$ conditioned according to (4) from \citep{suresh2024robotic} with $M=512$, denoting the representation dimensions. These visualizations highlight the diversity and variety of rewards that can be obtained by a BE-maximizing reward function (blue region) as compared to the single SE objective (dotted black line).} }
    \label{fig:be_rwrd_fn}
\end{figure}






% %%%%%%%%%%%%%% ICLR formatting guidelines %%%%%%%%%%%%%%

% \newpage

% \section{Submission of conference papers to ICLR 2025}

% ICLR requires electronic submissions, processed by
% \url{https://openreview.net/}. See ICLR's website for more instructions.

% If your paper is ultimately accepted, the statement {\tt
%   {\textbackslash}iclrfinalcopy} should be inserted to adjust the
% format to the camera ready requirements.

% The format for the submissions is a variant of the NeurIPS format.
% Please read carefully the instructions below, and follow them
% faithfully.

% \subsection{Style}

% Papers to be submitted to ICLR 2025 must be prepared according to the
% instructions presented here.

% %% Please note that we have introduced automatic line number generation
% %% into the style file for \LaTeXe. This is to help reviewers
% %% refer to specific lines of the paper when they make their comments. Please do
% %% NOT refer to these line numbers in your paper as they will be removed from the
% %% style file for the final version of accepted papers.

% Authors are required to use the ICLR \LaTeX{} style files obtainable at the
% ICLR website. Please make sure you use the current files and
% not previous versions. Tweaking the style files may be grounds for rejection.

% \subsection{Retrieval of style files}

% The style files for ICLR and other conference information are available online at:
% \begin{center}
%    \url{http://www.iclr.cc/}
% \end{center}
% The file \verb+iclr2025_conference.pdf+ contains these
% instructions and illustrates the
% various formatting requirements your ICLR paper must satisfy.
% Submissions must be made using \LaTeX{} and the style files
% \verb+iclr2025_conference.sty+ and \verb+iclr2025_conference.bst+ (to be used with \LaTeX{}2e). The file
% \verb+iclr2025_conference.tex+ may be used as a ``shell'' for writing your paper. All you
% have to do is replace the author, title, abstract, and text of the paper with
% your own.

% The formatting instructions contained in these style files are summarized in
% sections \ref{gen_inst}, \ref{headings}, and \ref{others} below.

% \section{General formatting instructions}
% \label{gen_inst}

% The text must be confined within a rectangle 5.5~inches (33~picas) wide and
% 9~inches (54~picas) long. The left margin is 1.5~inch (9~picas).
% Use 10~point type with a vertical spacing of 11~points. Times New Roman is the
% preferred typeface throughout. Paragraphs are separated by 1/2~line space,
% with no indentation.

% Paper title is 17~point, in small caps and left-aligned.
% All pages should start at 1~inch (6~picas) from the top of the page.

% Authors' names are
% set in boldface, and each name is placed above its corresponding
% address. The lead author's name is to be listed first, and
% the co-authors' names are set to follow. Authors sharing the
% same address can be on the same line.

% Please pay special attention to the instructions in section \ref{others}
% regarding figures, tables, acknowledgments, and references.


% There will be a strict upper limit of 10 pages for the main text of the initial submission, with unlimited additional pages for citations. 

% \section{Headings: first level}
% \label{headings}

% First level headings are in small caps,
% flush left and in point size 12. One line space before the first level
% heading and 1/2~line space after the first level heading.

% \subsection{Headings: second level}

% Second level headings are in small caps,
% flush left and in point size 10. One line space before the second level
% heading and 1/2~line space after the second level heading.

% \subsubsection{Headings: third level}

% Third level headings are in small caps,
% flush left and in point size 10. One line space before the third level
% heading and 1/2~line space after the third level heading.

% \section{Citations, figures, tables, references}
% \label{others}

% These instructions apply to everyone, regardless of the formatter being used.

% \subsection{Citations within the text}

% Citations within the text should be based on the \texttt{natbib} package
% and include the authors' last names and year (with the ``et~al.'' construct
% for more than two authors). When the authors or the publication are
% included in the sentence, the citation should not be in parenthesis using \verb|\citet{}| (as
% in ``See \citet{Hinton06} for more information.''). Otherwise, the citation
% should be in parenthesis using \verb|\citep{}| (as in ``Deep learning shows promise to make progress
% towards AI~\citep{Bengio+chapter2007}.'').

% The corresponding references are to be listed in alphabetical order of
% authors, in the \textsc{References} section. As to the format of the
% references themselves, any style is acceptable as long as it is used
% consistently.

% \subsection{Footnotes}

% Indicate footnotes with a number\footnote{Sample of the first footnote} in the
% text. Place the footnotes at the bottom of the page on which they appear.
% Precede the footnote with a horizontal rule of 2~inches
% (12~picas).\footnote{Sample of the second footnote}

% \subsection{Figures}

% All artwork must be neat, clean, and legible. Lines should be dark
% enough for purposes of reproduction; art work should not be
% hand-drawn. The figure number and caption always appear after the
% figure. Place one line space before the figure caption, and one line
% space after the figure. The figure caption is lower case (except for
% first word and proper nouns); figures are numbered consecutively.

% Make sure the figure caption does not get separated from the figure.
% Leave sufficient space to avoid splitting the figure and figure caption.

% You may use color figures.
% However, it is best for the
% figure captions and the paper body to make sense if the paper is printed
% either in black/white or in color.
% \begin{figure}[h]
% \begin{center}
% %\framebox[4.0in]{$\;$}
% \fbox{\rule[-.5cm]{0cm}{4cm} \rule[-.5cm]{4cm}{0cm}}
% \end{center}
% \caption{Sample figure caption.}
% \end{figure}

% \subsection{Tables}

% All tables must be centered, neat, clean and legible. Do not use hand-drawn
% tables. The table number and title always appear before the table. See
% Table~\ref{sample-table}.

% Place one line space before the table title, one line space after the table
% title, and one line space after the table. The table title must be lower case
% (except for first word and proper nouns); tables are numbered consecutively.

% \begin{table}[t]
% \caption{Sample table title}
% \label{sample-table}
% \begin{center}
% \begin{tabular}{ll}
% \multicolumn{1}{c}{\bf PART}  &\multicolumn{1}{c}{\bf DESCRIPTION}
% \\ \hline \\
% Dendrite         &Input terminal \\
% Axon             &Output terminal \\
% Soma             &Cell body (contains cell nucleus) \\
% \end{tabular}
% \end{center}
% \end{table}

% \section{Default Notation}

% In an attempt to encourage standardized notation, we have included the
% notation file from the textbook, \textit{Deep Learning}
% \cite{goodfellow2016deep} available at
% \url{https://github.com/goodfeli/dlbook_notation/}.  Use of this style
% is not required and can be disabled by commenting out
% \texttt{math\_commands.tex}.


% \centerline{\bf Numbers and Arrays}
% \bgroup
% \def\arraystretch{1.5}
% \begin{tabular}{p{1in}p{3.25in}}
% $\displaystyle a$ & A scalar (integer or real)\\
% $\displaystyle \va$ & A vector\\
% $\displaystyle \mA$ & A matrix\\
% $\displaystyle \tA$ & A tensor\\
% $\displaystyle \mI_n$ & Identity matrix with $n$ rows and $n$ columns\\
% $\displaystyle \mI$ & Identity matrix with dimensionality implied by context\\
% $\displaystyle \ve^{(i)}$ & Standard basis vector $[0,\dots,0,1,0,\dots,0]$ with a 1 at position $i$\\
% $\displaystyle \text{diag}(\va)$ & A square, diagonal matrix with diagonal entries given by $\va$\\
% $\displaystyle \ra$ & A scalar random variable\\
% $\displaystyle \rva$ & A vector-valued random variable\\
% $\displaystyle \rmA$ & A matrix-valued random variable\\
% \end{tabular}
% \egroup
% \vspace{0.25cm}

% \centerline{\bf Sets and Graphs}
% \bgroup
% \def\arraystretch{1.5}

% \begin{tabular}{p{1.25in}p{3.25in}}
% $\displaystyle \sA$ & A set\\
% $\displaystyle \R$ & The set of real numbers \\
% $\displaystyle \{0, 1\}$ & The set containing 0 and 1 \\
% $\displaystyle \{0, 1, \dots, n \}$ & The set of all integers between $0$ and $n$\\
% $\displaystyle [a, b]$ & The real interval including $a$ and $b$\\
% $\displaystyle (a, b]$ & The real interval excluding $a$ but including $b$\\
% $\displaystyle \sA \backslash \sB$ & Set subtraction, i.e., the set containing the elements of $\sA$ that are not in $\sB$\\
% $\displaystyle \gG$ & A graph\\
% $\displaystyle \parents_\gG(\ervx_i)$ & The parents of $\ervx_i$ in $\gG$
% \end{tabular}
% \vspace{0.25cm}


% \centerline{\bf Indexing}
% \bgroup
% \def\arraystretch{1.5}

% \begin{tabular}{p{1.25in}p{3.25in}}
% $\displaystyle \eva_i$ & Element $i$ of vector $\va$, with indexing starting at 1 \\
% $\displaystyle \eva_{-i}$ & All elements of vector $\va$ except for element $i$ \\
% $\displaystyle \emA_{i,j}$ & Element $i, j$ of matrix $\mA$ \\
% $\displaystyle \mA_{i, :}$ & Row $i$ of matrix $\mA$ \\
% $\displaystyle \mA_{:, i}$ & Column $i$ of matrix $\mA$ \\
% $\displaystyle \etA_{i, j, k}$ & Element $(i, j, k)$ of a 3-D tensor $\tA$\\
% $\displaystyle \tA_{:, :, i}$ & 2-D slice of a 3-D tensor\\
% $\displaystyle \erva_i$ & Element $i$ of the random vector $\rva$ \\
% \end{tabular}
% \egroup
% \vspace{0.25cm}


% \centerline{\bf Calculus}
% \bgroup
% \def\arraystretch{1.5}
% \begin{tabular}{p{1.25in}p{3.25in}}
% % NOTE: the [2ex] on the next line adds extra height to that row of the table.
% % Without that command, the fraction on the first line is too tall and collides
% % with the fraction on the second line.
% $\displaystyle\frac{d y} {d x}$ & Derivative of $y$ with respect to $x$\\ [2ex]
% $\displaystyle \frac{\partial y} {\partial x} $ & Partial derivative of $y$ with respect to $x$ \\
% $\displaystyle \nabla_\vx y $ & Gradient of $y$ with respect to $\vx$ \\
% $\displaystyle \nabla_\mX y $ & Matrix derivatives of $y$ with respect to $\mX$ \\
% $\displaystyle \nabla_\tX y $ & Tensor containing derivatives of $y$ with respect to $\tX$ \\
% $\displaystyle \frac{\partial f}{\partial \vx} $ & Jacobian matrix $\mJ \in \R^{m\times n}$ of $f: \R^n \rightarrow \R^m$\\
% $\displaystyle \nabla_\vx^2 f(\vx)\text{ or }\mH( f)(\vx)$ & The Hessian matrix of $f$ at input point $\vx$\\
% $\displaystyle \int f(\vx) d\vx $ & Definite integral over the entire domain of $\vx$ \\
% $\displaystyle \int_\sS f(\vx) d\vx$ & Definite integral with respect to $\vx$ over the set $\sS$ \\
% \end{tabular}
% \egroup
% \vspace{0.25cm}

% \centerline{\bf Probability and Information Theory}
% \bgroup
% \def\arraystretch{1.5}
% \begin{tabular}{p{1.25in}p{3.25in}}
% $\displaystyle P(\ra)$ & A probability distribution over a discrete variable\\
% $\displaystyle p(\ra)$ & A probability distribution over a continuous variable, or over
% a variable whose type has not been specified\\
% $\displaystyle \ra \sim P$ & Random variable $\ra$ has distribution $P$\\% so thing on left of \sim should always be a random variable, with name beginning with \r
% $\displaystyle  \E_{\rx\sim P} [ f(x) ]\text{ or } \E f(x)$ & Expectation of $f(x)$ with respect to $P(\rx)$ \\
% $\displaystyle \Var(f(x)) $ &  Variance of $f(x)$ under $P(\rx)$ \\
% $\displaystyle \Cov(f(x),g(x)) $ & Covariance of $f(x)$ and $g(x)$ under $P(\rx)$\\
% $\displaystyle H(\rx) $ & Shannon entropy of the random variable $\rx$\\
% $\displaystyle \KL ( P \Vert Q ) $ & Kullback-Leibler divergence of P and Q \\
% $\displaystyle \mathcal{N} ( \vx ; \vmu , \mSigma)$ & Gaussian distribution %
% over $\vx$ with mean $\vmu$ and covariance $\mSigma$ \\
% \end{tabular}
% \egroup
% \vspace{0.25cm}

% \centerline{\bf Functions}
% \bgroup
% \def\arraystretch{1.5}
% \begin{tabular}{p{1.25in}p{3.25in}}
% $\displaystyle f: \sA \rightarrow \sB$ & The function $f$ with domain $\sA$ and range $\sB$\\
% $\displaystyle f \circ g $ & Composition of the functions $f$ and $g$ \\
%   $\displaystyle f(\vx ; \vtheta) $ & A function of $\vx$ parametrized by $\vtheta$.
%   (Sometimes we write $f(\vx)$ and omit the argument $\vtheta$ to lighten notation) \\
% $\displaystyle \log x$ & Natural logarithm of $x$ \\
% $\displaystyle \sigma(x)$ & Logistic sigmoid, $\displaystyle \frac{1} {1 + \exp(-x)}$ \\
% $\displaystyle \zeta(x)$ & Softplus, $\log(1 + \exp(x))$ \\
% $\displaystyle || \vx ||_p $ & $\normlp$ norm of $\vx$ \\
% $\displaystyle || \vx || $ & $\normltwo$ norm of $\vx$ \\
% $\displaystyle x^+$ & Positive part of $x$, i.e., $\max(0,x)$\\
% $\displaystyle \1_\mathrm{condition}$ & is 1 if the condition is true, 0 otherwise\\
% \end{tabular}
% \egroup
% \vspace{0.25cm}



% \section{Final instructions}
% Do not change any aspects of the formatting parameters in the style files.
% In particular, do not modify the width or length of the rectangle the text
% should fit into, and do not change font sizes (except perhaps in the
% \textsc{References} section; see below). Please note that pages should be
% numbered.

% \section{Preparing PostScript or PDF files}

% Please prepare PostScript or PDF files with paper size ``US Letter'', and
% not, for example, ``A4''. The -t
% letter option on dvips will produce US Letter files.

% Consider directly generating PDF files using \verb+pdflatex+
% (especially if you are a MiKTeX user).
% PDF figures must be substituted for EPS figures, however.

% Otherwise, please generate your PostScript and PDF files with the following commands:
% \begin{verbatim}
% dvips mypaper.dvi -t letter -Ppdf -G0 -o mypaper.ps
% ps2pdf mypaper.ps mypaper.pdf
% \end{verbatim}

% \subsection{Margins in LaTeX}

% Most of the margin problems come from figures positioned by hand using
% \verb+\special+ or other commands. We suggest using the command
% \verb+\includegraphics+
% from the graphicx package. Always specify the figure width as a multiple of
% the line width as in the example below using .eps graphics
% \begin{verbatim}
%    \usepackage[dvips]{graphicx} ...
%    \includegraphics[width=0.8\linewidth]{myfile.eps}
% \end{verbatim}
% or % Apr 2009 addition
% \begin{verbatim}
%    \usepackage[pdftex]{graphicx} ...
%    \includegraphics[width=0.8\linewidth]{myfile.pdf}
% \end{verbatim}
% for .pdf graphics.
% See section~4.4 in the graphics bundle documentation (\url{http://www.ctan.org/tex-archive/macros/latex/required/graphics/grfguide.ps})

% A number of width problems arise when LaTeX cannot properly hyphenate a
% line. Please give LaTeX hyphenation hints using the \verb+\-+ command.

% \subsubsection*{Author Contributions}
% If you'd like to, you may include  a section for author contributions as is done
% in many journals. This is optional and at the discretion of the authors.

% \subsubsection*{Acknowledgments}
% Use unnumbered third level headings for the acknowledgments. All
% acknowledgments, including those to funding agencies, go at the end of the paper.


% \bibliography{iclr2025_conference}
% \bibliographystyle{iclr2025_conference}

% \appendix
% \section{Appendix}
% You may include other additional sections here.


\end{document}