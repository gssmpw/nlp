\newpage
\subsection{Additional Offline RL Experiments}

% \subsubsection{Complete Offline RL Results}

\begin{figure}[htp]
    \begin{subfigure}{\textwidth}
    % \captionsetup{font=footnotesize,labelfont=scriptsize,textfont=scriptsize}
    \includegraphics[width=\textwidth]{figs/offline_RL_all_alphas/quadruped_walk_barplot_with_shannon3M_SMM_RND_all_alphas.png}
    % \caption{TEST}
    \label{fig:orl_appendix:quad_walk}
    \end{subfigure}
    %
    \vspace{-4mm}
    %
    \begin{subfigure}{\textwidth}
    % \captionsetup{font=footnotesize,labelfont=scriptsize,textfont=scriptsize}
    \includegraphics[width=\textwidth]{figs/offline_RL_all_alphas/quadruped_run_barplot_with_shannon3M_SMM_RND_all_alphas.png}
    % \caption{TEST}
    \label{fig:orl_appendix:quad_run}
    \end{subfigure}
    %
    \vspace{-4mm}
    %
    \begin{subfigure}{\textwidth}
    % \captionsetup{font=footnotesize,labelfont=scriptsize,textfont=scriptsize}
    \includegraphics[width=\textwidth]{figs/offline_RL_all_alphas/walker_stand_barplot_with_shannon3M_SMM_RND_all_alphas.png}
    % \caption{TEST}
    \label{fig:orl_appendix:walker_stand}
    \end{subfigure}
    %
    \vspace{-4mm}
    %
    \begin{subfigure}{\textwidth}
    % \captionsetup{font=footnotesize,labelfont=scriptsize,textfont=scriptsize}
    \includegraphics[width=\textwidth]{figs/offline_RL_all_alphas/walker_walk_barplot_with_shannon3M_SMM_RND_all_alphas.png}
    % \caption{TEST}
    \label{fig:orl_appendix:walker_walk}
    \end{subfigure}
    %
    \vspace{-4mm}
    %
    \begin{subfigure}{\textwidth}
    % \captionsetup{font=footnotesize,labelfont=scriptsize,textfont=scriptsize}
    \includegraphics[width=\textwidth]{figs/offline_RL_all_alphas/walker_run_barplot_with_shannon3M_SMM_RND_all_alphas.png}
    % \caption{TEST}
    \label{fig:orl_appendix:walker_run}
    \end{subfigure}
    %
    \vspace{-4mm}
    %
    \caption{ \new{Offline RL results for all $\alpha$ and $q$ values evaluated. Initial trials showed $q \in \{2.0, 3.0, 5.0\}$ led to performance no better (and usually worse) than $q = 1.1$, so offline RL training for these $q$ values was not performed.} }
    \label{fig:orl_appendix}
\end{figure}



\begin{figure}
    \begin{subfigure}{0.49\textwidth}
        \includegraphics[width=\textwidth]{figs/quadruped_walk_barplot_100K_vs_200K_Shannon.png}
        % \caption{TEST}
        \label{fig:orl_appendix:quad_walk_average}
    \end{subfigure}
    \hfill
    \begin{subfigure}{0.49\textwidth}
        \includegraphics[width=\textwidth]{figs/quadruped_run_barplot_100K_vs_200K_Shannon.png}
        % \caption{TEST}
        \label{fig:orl_appendix:quad_run_average}
    \end{subfigure}
    
    \medskip
    \begin{subfigure}{0.49\textwidth}
        \includegraphics[width=\textwidth]{figs/walker_stand_barplot_100K_vs_200K_Shannon.png}
        % \caption{TEST}
        \label{fig:orl_appendix:walker_stand_average}
    \end{subfigure}
    \hfill
    \begin{subfigure}{0.49\textwidth}
        \includegraphics[width=\textwidth]{figs/walker_walk_barplot_100K_vs_200K_Shannon.png}
        % \caption{TEST}
        \label{fig:orl_appendix:walker_walk_average}
    \end{subfigure}
    
    \medskip
    \centering
    \begin{subfigure}{0.49\textwidth}
        \includegraphics[width=\textwidth]{figs/walker_run_barplot_100K_vs_200K_Shannon.png}
        % \caption{TEST}
    \end{subfigure}
    \caption{ \new{ Ablation result comparing the effect of performing 100K vs. 200K offline RL training steps on a 3M-element dataset generated using Shannon entropy as exploration objective. These results suggest that performing additional offline RL training has only a marginal effect on downstream task performance.} }
    \label{fig:orl_average_appendix}
\end{figure}



\begin{figure}
    \begin{subfigure}{0.49\textwidth}
        \includegraphics[width=\textwidth]{figs/quadruped_walk_average_barplot.png}
        % \caption{TEST}
        \label{fig:orl_appendix:quad_walk_average}
    \end{subfigure}
    \hfill
    \begin{subfigure}{0.49\textwidth}
        \includegraphics[width=\textwidth]{figs/quadruped_run_average_barplot.png}
        % \caption{TEST}
        \label{fig:orl_appendix:quad_run_average}
    \end{subfigure}
    
    \medskip
    \begin{subfigure}{0.49\textwidth}
        \includegraphics[width=\textwidth]{figs/walker_stand_average_barplot.png}
        % \caption{TEST}
        \label{fig:orl_appendix:walker_stand_average}
    \end{subfigure}
    \hfill
    \begin{subfigure}{0.49\textwidth}
        \includegraphics[width=\textwidth]{figs/walker_walk_average_barplot.png}
        % \caption{TEST}
        \label{fig:orl_appendix:walker_walk_average}
    \end{subfigure}
    
    \medskip
    \centering
    \begin{subfigure}{0.49\textwidth}
        \includegraphics[width=\textwidth]{figs/walker_run_average_barplot.png}
        % \caption{TEST}
    \end{subfigure}
    \caption{Offline RL results averaged over all $\alpha, q$ values.}
    \label{fig:orl_average_appendix}
\end{figure}










\newpage
\subsection{Quantitative Coverage Experiments}

\begin{figure}[]
    \begin{subfigure}[b]{0.98\textwidth}
        \includegraphics[width=\linewidth]{figs/coverage_radii_walker.png}
        \caption{ \new{Volumetric coverage for data generation on Walker} }
        \label{fig:tsne_gbe_walker_full}
    \end{subfigure}
    \centering
    
    \begin{subfigure}[b]{0.98\textwidth}
        \includegraphics[width=\linewidth]{figs/coverage_radii_quadruped.png}
        \caption{ \new{Volumetric coverage for data generation on Quadruped} }
        \label{fig:tsne_renyi_quadruped_full}
    \end{subfigure}
    \caption{ \new{Visualization of evolution of smallest hypersphere radius $\overline{r}$ (normalized by the maximum radius achieved over all datasets) over the course of data generation training step $T$ for the Walker and Quadruped domains. We refer to this coverage metric as \textit{volumetric coverage}. Welzl's algorithm was used to determine the radius $\overline{r}$. 10K data points were sampled uniformly from every 50K iteration increment and cumulatively added to get a total of 100K samples for 500K iterations. Volumetric coverage varies considerably with the choice of parameters and data generation methods. For the range of parameters $\alpha$ and $q$ that we considered, BE exhibits higher volumetric coverage than RE on average on the problems under consideration. SE volumetric coverage was about average, while RND and SMM volumetric coverage differed sharply across domains: RND outperformed all other data generation methods on Walker, while SMM was not far behind; on Quadruped, on the other hand, both underperformed. Values of $q < 1$ enjoy higher volumetric coverage for RE on both tasks, while values of $\alpha < 1$ enjoy higher volumetric coverage for BE on Quadruped; since $q < 1, \alpha < 1$ tend to correspond to superior downstream offline RL performance (see Fig. \ref{fig:orl_appendix}), this suggests volumetric may be positively correlated with performance on downstream tasks. We also note note that RE with $q>1$ in general shows both poor volumetric coverage and poor qualitative coverage (PHATE and t-SNE), which might correspond to its poor performance in all tasks. These relationships are not conclusive, however, and further investigation is needed.} }
    \label{fig:tsne_quadruped_full}
\end{figure}


\newpage
\subsection{Additional Qualitative Visualizations}

\begin{figure}[ht]
    \begin{subfigure}[b]{0.98\textwidth}
        \includegraphics[width=\linewidth]{figs/tsne_gbe_walker_full.png}
        \caption{TSNE plots for GBE for Walker}
        \label{fig:tsne_gbe_walker_full}
    \end{subfigure}
    \centering
    
    \begin{subfigure}[b]{0.98\textwidth}
        \includegraphics[width=\linewidth,height=3in]{figs/tsne_renyi_walker_full.png}
        \caption{TSNE plots for Renyi for Walker}
        \label{fig:tsne_renyi_walker_full}
    \end{subfigure}
    \caption{TSNE plots for Walker }
    \label{fig:TSNE_walker_full}

\end{figure}



\begin{figure}[]
    \begin{subfigure}[b]{0.98\textwidth}
        \includegraphics[width=\linewidth]{figs/phate_gbe_quadruped_full.png}
        \caption{PHATE plots for GBE for Quadruped}
        \label{fig:phate_gbe_walker_full}
    \end{subfigure}
    \centering
    
    \begin{subfigure}[b]{0.98\textwidth}
        \includegraphics[width=\linewidth,height=3in]{figs/phate_renyi_quadruped_full.png}
        \caption{PHATE plots for Renyi for Quadruped}
        \label{fig:phate_renyi_quadruped_full}
    \end{subfigure}
    \caption{PHATE plots for Quadruped }
    \label{fig:phate_quadruped_full}
\end{figure}

\begin{figure}[]
    \begin{subfigure}[b]{0.98\textwidth}
        \includegraphics[width=\linewidth]{figs/tsne_gbe_quadruped_full.png}
        \caption{TSNE plots for GBE for Quadruped}
        \label{fig:tsne_gbe_walker_full}
    \end{subfigure}
    \centering
    
    \begin{subfigure}[b]{0.98\textwidth}
        \includegraphics[width=\linewidth,height=3in]{figs/tsne_renyi_quadruped_full.png}
        \caption{TSNE plots for Renyi for Quadruped}
        \label{fig:tsne_renyi_quadruped_full}
    \end{subfigure}
    \caption{TSNE plots for Quadruped }
    \label{fig:tsne_quadruped_full}
\end{figure}


\begin{figure}[ht]
    \begin{subfigure}[b]{0.48\textwidth}
        \includegraphics[width=\linewidth]{figs/tsne_walker_misc_v2.png}
        \caption{t-SNE plots for RND and SMM for Walker}
        \label{fig:tsne_misc_walker_full}
    \end{subfigure}
    \centering
    ~
    \begin{subfigure}[b]{0.48\textwidth}
        \includegraphics[width=\linewidth]{figs/tsne_quadruped_misc.png}
        \caption{t-SNE plots for RND and SMM for Quadruped}
        \label{fig:tsne_misc_quadruped_full}
    \end{subfigure}

    \begin{subfigure}[b]{0.48\textwidth}
        \includegraphics[width=\linewidth]{figs/phate_walker_misc_v2.png}
        \caption{PHATE plots for RND and SMM for Walker}
        \label{fig:phate_misc_walker_full}
    \end{subfigure}
    \centering
    ~
    \begin{subfigure}[b]{0.48\textwidth}
        \includegraphics[width=\linewidth]{figs/phate_misc_quadruped.png}
        \caption{PHATE plots for RND and SMM for Quadruped}
        \label{fig:phate_misc_quadruped_full}
    \end{subfigure}
    \caption{\new{Qualitative visualization of SMM and RND for data generation}}
    \label{fig:smm_rnd_full}

\end{figure}



\newpage
\subsection{BE Reward Function Visualization}
%
\begin{figure}[ht]
    \centering
    \includegraphics[width=0.99\linewidth]{figs/reward_fn_BE.png}
    \caption{ \new{ Visualization of the BE reward function \eqref{eqn:r_final} by varying the parameter $\alpha$ with $\beta$ conditioned according to (4) from \citep{suresh2024robotic} with $M=512$, denoting the representation dimensions. These visualizations highlight the diversity and variety of rewards that can be obtained by a BE-maximizing reward function (blue region) as compared to the single SE objective (dotted black line).} }
    \label{fig:be_rwrd_fn}
\end{figure}



