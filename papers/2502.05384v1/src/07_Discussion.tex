\section{Limitations \& Challenges}
%From the initial CAD design to real-world deployment of CavePI, we identify several engineering challenges and practical limitations. Key observations and potential improvements in the mechanical design, perception, planning, and control subsystems are outlined below.

\subsection{Design Aspects}
CavePI’s onboard components are packed within a single pressure-sealed tube, leaving minimal space for additional hardware. Notably, major housing volume is occupied by the LiPo battery, which can be replaced with a more compact alternative to create space for future add-ons. We also plan to upgrade the perception SBC from the existing Jetson Nano to a Jetson Orin Nano that offers $4\times$ memory with more advanced GPU resources. The redesigned system will also incorporate a magnetic switching mechanism for seamless power control. Outside of the enclosure, the 4 thrusters are arranged to allow control over surge, heave, roll, and yaw motions, however, their adjacent placement induces mutual turbulence and reduces thrust efficiency. To address this, we are investigating alternative thruster placements that will enhance motion dynamics without affecting the robot's buoyancy properties and dynamic stability.

\begin{figure}[t]
    \centering
    % \vspace{2mm}
    \includegraphics[width=\columnwidth]{figures/PerceptionFailures.png}%
    %\vspace{-1mm}
    \caption{A few perception failure modes are shown. The down-facing camera falsely detects (a) the edge of the laboratory tank; (b) tree roots as the caveline, resulting in incorrect tracking.
    }%
    \vspace{-2mm}
    \label{fig:perception_failure}
\end{figure}

% \begin{figure*}[ht]
% \centering
% \begin{subfigure}[]{0.24\linewidth}
% \includegraphics[width=\linewidth]{figures/Incorrect_Detection.jpg}%
% % \vspace{-1mm}
% \caption{}
% \label{fig:Incorrect_Detection}
% % \vspace{-2mm}
% \end{subfigure}
% \begin{subfigure}[]{0.24\linewidth}
% \includegraphics[width=\linewidth]{figures/Drifting.jpg}%
% % \vspace{-1mm}
% \caption{}
% \label{fig:Drifting}
% % \vspace{-2mm}
% \end{subfigure}
% \begin{subfigure}[]{0.24\linewidth}
% \includegraphics[width=\linewidth]{figures/Pitching.jpg}%
% % \vspace{-1mm}
% \caption{}
% \label{fig:Pitching}
% % \vspace{-2mm}
% \end{subfigure}
% \begin{subfigure}[]{0.24\linewidth}
% \includegraphics[width=\linewidth]{figures/Overshooting.jpg}%
% % \vspace{-1mm}
% \caption{}
% \label{fig:Overshooting}
% % \vspace{-2mm}
% \end{subfigure}
% \caption{A few tracking failure modes, observed during the field experiment, are illustrated. CavePI is (a) Falsely detecting a tree's root as the caveline; (b) Experiencing significant lateral movement under strong water currents; (c) Pitching upward in the presence of water currents, attributed to its slightly back-heavy design; (d) Overshooting its intended trajectory while moving downstream in water currents.
% }
% \label{fig:failures}
% \vspace{-3mm}
% \end{figure*}


% Talk about: limited space for sensor add-ons and a larger more powerful computer, more visual commands (gesture) for diver-robot co-op, buoyancy adjustment and better thruster config, 

\subsection{Perception Challenges}
The current onboard sensor suite offers a limited understanding of the surrounding 3D environment. For instance, the Ping2 sonar provides only 1D depth measurements, which we will replace with an advanced $360^\circ$ scanning sonar system for enhanced spatial awareness. The scanning sonar, combined with other state estimation sensors, will map the environment and effectively avoid obstacles during navigation. Moreover, the two cameras currently operate independently for different purposes without synchronization. In the next iteration, we propose incorporating a $45^\circ$ slanted camera and combining all the visual feeds into a mosaic vision. This advanced vision system will offer wider FOV with more peripheral information and improve visual servoing performance. Additionally, the current caveline detection model occasionally misidentifies objects such as submerged tree roots as part of the caveline as shown in Fig.~\ref{fig:perception_failure}b, and it sometimes fails to detect the line under low-light conditions. These issues lead to significant tracking inaccuracies. Although more computationally intensive models might address these shortcomings, they are currently infeasible due to hardware limitations. With the proposed design modifications for the next iteration, we plan to adopt a more robust detection model to improve CavePI's tracking performance. Furthermore, hand gesture recognition~\cite{xu2008natural} will be integrated to enable seamless cooperation between divers and CavePI during underwater cave operations. With this advanced sensor setup, we will deploy advanced SLAM algorithms, such as SVIn2~\cite{rahman2022svin2} for robust navigation in GPS-denied underwater cave environments. 

%\JI{I assumed there will be some challenging cases for detection/segmentation - shown here}


\begin{figure}[h]
    \centering
    % \vspace{2mm}
    \includegraphics[width=\columnwidth]{figures/Tracking_Failure.png}%
    %\vspace{-1mm}
    \caption{A few tracking failure modes are shown: (a) CavePI is correctly following the line; (b) lateral drift under strong currents; (c) pitching upward due to currents, attributed to its back-heavy design; (d) overshooting its intended trajectory while moving downstream.
    }%
    \vspace{-2mm}
    \label{fig:tracking_failure}
\end{figure}


\subsection{Smooth 6-DOF Control}

Although CavePI is a 6-DOF AUV capable of maneuvering in a 3D environment, it currently offers active control over only four DOFs -- surge, heave, roll, and yaw. In future iterations, we intend to reposition its thrusters to enable control over the remaining two DOFs -- pitch and sway -- thereby enhancing the robot’s maneuverability in complex underwater environments. Furthermore, CavePI's autonomous control primarily utilizes a proportional-derivative (PD) controller, which performs effectively under conditions with minimal environmental disturbances. However, this straightforward control strategy becomes unstable in more complex scenarios, such as when water currents are present. In such conditions, lower proportional gains are insufficient for adjusting CavePI’s yaw to align with the cave line, while higher proportional gains cause overshooting in the robot's trajectory during sharp turns (see Fig.~\ref{fig:tracking_failure}\,d), attributed to both perception latency and the limitations of the current control method. Additionally, nonlinear (primarily quadratic) drag forces significantly impact the robot's stability and must be incorporated into the control system design. To overcome these challenges, we are developing a more robust nonlinear adaptive control system to enhance stability. Concurrently, we are optimizing the communication between the perception and control modules to reduce latency and improve overall responsiveness.




%\JI{No images showing the effects of current or difficulty in failure cases? If you are showing some failure cases in video - perhaps use a frame or two to show here}

% CavePI's autonomous control primarily relies on a proportional-derivative (PD) controller, which is effective for planar motions such as forward movement, yaw, and depth holding. However, this simple control strategy proves unstable for more complex maneuvers such as roll and pitch. In narrow underwater caves, caveline is often placed along the sidewall rather than the floor, which requires roll adjustments to maintain visibility in the down-facing camera’s field of view. Additionally, during sharp turns, we observe frequent overshoots in the robot's trajectory (see Fig.~\ref{fig:heading_control}\,d), attributed to both perception latency and the limitations of the current control method. To address these challenges, we are developing a more robust PID control system for better stability. Simultaneously, we are optimizing communication between the perception and control bays to reduce latency and improve overall responsiveness.

