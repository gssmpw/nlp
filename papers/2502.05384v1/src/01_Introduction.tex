\section{Introduction}
% what is the problem why is it important
Autonomous underwater cave exploration is an important area of research, providing valuable insights into some of the planet's least explored ecosystems to understand and manage marine resources. Underwater caves, in particular, hold immense potential for advancing our knowledge of archaeology, hydrology, and geology~\cite{lace2013biological}. The formations and sediments within these caves serve as critical records of historical climate and geology events while playing a key role in monitoring groundwater flow within Karst topographies, which supply freshwater to nearly a quarter of the global population~\cite{karstbook}. However, these environments are difficult to access and often hazardous for human divers due to their confined spaces, complex geometries, and lack of natural light~\cite{potts2016thirty,buzzacott2009american}. To this end, AUVs present a promising solution, offering safe and efficient exploration while minimizing risks and improving  precision~\cite{abdullah2023caveseg}.

\begin{figure}[t]
    \centering
    % \vspace{2mm}
    \includegraphics[width=\columnwidth]{figures/fig1_ginnie.png}%
    \vspace{-3mm}
    \caption{The CavePI AUV navigates by leveraging the semantic guidance of a caveline from its down-facing camera. A deep visual perception module extracts the semantic cues, which are processed by an onboard planner to make visual servoing decisions.
    }%
    \vspace{-2mm}
    \label{fig:vehicle}
\end{figure}


However, fully autonomous exploration inside underwater caves presents significant challenges, navigation is difficult without GPS, while poor visibility, backscattering, and silt disturbances hinder perception. Thus, uninformed exploration, as often done by AUVs in open-water ecological surveying and mapping tasks~\cite{manderson2020visionbasedgoalconditionedpoliciesunderwater}, is quite challenging. For autonomous navigation inside underwater caves~\cite{richmond2020autonomous,yu2023weakly}, it is important to be aware of the entry and exit directions as well as a semantic understanding of existing navigational markers, when available. More specifically, underwater caves explored by humans are augmented by a line (string), referred to as the \emph{caveline}~\cite{exley1986basic}, which extends from the entrance through all the major sections of the cave. Other, navigational markers such as \emph{arrows} and \emph{cookies}~\cite{abdullah2023caveseg} indicate the closest path to the exit (arrows' directions) and the presence of landmarks and/or divers (cookies). The caveline, together with the navigational markers, provides a 1D retraction of the 3D environment, indicating the orientation and distance from the entrance of the primary passages. 








% needs to be a separate paragraph
In this paper, we demonstrate the system design and algorithmic integration of CavePI, a novel AUV specifically developed for navigating underwater caves using semantic guidance from cavelines and other navigational markers; see Fig.~\ref{fig:vehicle}. The compact design and 4-degree-of-freedom (4-DOF) motion model enable safe traversal through narrow passages with minimal silt disturbance. Designed for single-person deployment, CavePI features a forward-facing camera for visual servoing and a downward-facing camera for caveline detection and tracking, effectively minimizing blind spots around the robot. A Ping sonar provides sparse range data to ensure robust obstacle avoidance and safe navigation within the caves. Additionally, two onboard lights enhance visibility in low-light conditions, complemented by visual filters for improved perception. The computational framework is powered by two single-board computers (SBCs): a Jetson Nano for perception, and a Raspberry Pi-5 for planning and control. 



For visual servoing, we integrate a caveline detector that performs real-time pixel-level segmentation of the caveline using the robot's down-facing camera stream. While prior studies have proposed vision transformer (ViT)-based frameworks such as CL-ViT~\cite{yu2023weakly} and CaveSeg~\cite{abdullah2023caveseg} for caveline detection, these approaches are computationally intensive and unsuitable for the resource-constrained SBCs. To address this, we adopt more efficient feature-extractor networks: MobileNetV3~\cite{howard2019searching} and ResNet101~\cite{he2016deep} -- combined with a lightweight DeeplabV3~\cite{chen2017rethinking} as the segmentation head. By leveraging GPU-accelerated computation on a Jetson Nano SBC, our system achieves a maximum inference rate of $18.2$ frames per second (FPS) while ensuring robust segmentation performance. The resulting segmentation map is post-processed to extract caveline contours, which are used to guide a PID-based Pure Pursuit controller for precise heading adjustments for smooth tracking-by-detection~\cite{shkurti2017underwater}.

%For visual servoing, we integrate a caveline detector that performs real-time pixel-level segmentation of the caveline using the robot's down-facing camera stream. While previous research has presented vision transformer (ViT)-based~\cite{dosovitskiy2020image} caveline detection frameworks such as CL-ViT~\cite{yu2023weakly} and CaveSeg~\cite{abdullah2023caveseg}, they are computationally expensive and unsuitable for CavePI's SBCs. Therefore, we adapt more efficient feature extractor networks -- MobileNetV3~\cite{howard2019searching} and ResNet101~\cite{he2016deep} -- paired with a light DeeplabV3~\cite{chen2017rethinking} segmentation head. We achieve a maximum of $18.2$\,fps inference rate by utilizing the GPU accelerated computation of Jetson Nano SBC. The segmentation map is further processed to extract caveline contours and their directions, followed by a PID controller that adjusts the robot's heading accordingly. 

%\JI{talk about the caveline detection and tracking, how the model was integrated to the system. Mention that CLVIT or CaveSeg models are too big for SBCs}


%tracking a predefined caveline as illustrated in Fig~\ref{fig:vehicle}. It provides autonomous path-planning capabilities which makes it easier to use for different missions without any skilled operators. CavePI offers a light-weight and compact design which enables it to deploy by only one person. CavePI is equipped with specialized perception subsystems that record critical environmental data during missions. It offers two cameras-one front-facing and the other downward facing. The downward-facing camera enables vision-based tracking of the caveline for navigation while also capturing detailed visual data beneath the AUV, effectively eliminating blind spots below CavePI. The complete system is designed to achieve positive buoyancy such that the AUV surfaces in the absence of a control signal. 

%Therefore, detecting and following the caveline is essential for safe and effective robot navigation. 

% why a new robotic system is needed: 
 
%To meet these demands, such robots must feature compact designs to traverse narrow passages and be equipped with advanced acoustic-visual sensors tailored to the mission requirements.


%In addition, we develop a digital twin of CavePI that facilitates pre-mission planning and performance testing, offering a cost-effective platform for validating mission concepts. Developed using the Robot Operating System (ROS) and simulated in an underwater environment via Gazebo, the digital twin addresses the challenges associated with expensive and delicate underwater technologies that require high safety standards for testing. Even verified systems can exhibit unexpected behavior due to new technological elements; for instance, a novel caveline detection algorithm might mistakenly identify an unrelated object as the caveline, prompting CavePI to follow it and potentially creating hazardous conditions. By rigorously testing and refining new underwater technologies, algorithms, or system modifications on the digital twin before physical deployment, we ensure a safer and more economical evaluation process. Furthermore, as marine exploration missions demand extensive logistical planning and significant time investments, initial mission planning and performance assessments conducted on the digital twin enable quicker feedback and improvements.


In addition, we develop a digital twin (DT) of CavePI to support pre-mission planning and testing, providing a cost-effective platform for validating mission concepts. Built using the Robot Operating System (ROS) and simulated in an underwater environment via Gazebo, it addresses the challenges posed by expensive and delicate underwater technologies that require stringent safety standards for testing. Even verified systems can exhibit unforeseen behavior due to new technological components; for example, a novel caveline detection algorithm might erroneously identify unrelated objects as cavelines, leading CavePI to follow incorrect paths and potentially creating hazardous conditions. The DT allows for prototyping, rigorous testing, and refinement of new underwater technologies, algorithms, or system modifications before physical deployment, ensuring safer and more economical evaluations. Moreover, as marine exploration missions require substantial logistical planning and time investment, initial mission planning and performance assessments conducted on the DT facilitate rapid feedback and iterative improvements, enhancing overall mission efficiency.


% identify the sw and hw components and strengths 

% how you have evaluated it? What are you demonstrating
CavePI's guided navigation capabilities are initially evaluated by line-following experiments conducted in a controlled environment: a $2$\,m$\times3$\,m water tank with a maximum depth of $1.5$\,m. Cavelines are arranged in both regular geometric patterns (\eg, circles, rectangles) and irregular configurations such as curves, spirals, and vertical slopes. The \textit{line-following} accuracy is quantified by measuring the \textit{tracking error}, defined as the distance between the line and the optical center of the down-facing camera on the plane of the line. Similar evaluations are performed in the Gazebo simulation environment using the DT. Additionally, the delay in decision-making and its impact on an overshoot at sharp corners are analyzed. Specifically, the proportional and differential gains of the PID controller are fine-tuned via exhaustive search, and overshoots are minimized for subsequent runs. 
% The navigation control smoothness and maneuvering capabilities in confined spaces are further assessed within the Gazebo simulation environment by the CavePI's DT. \JI{how? what parameters} 

Following extensive in-house testing and refinement, CavePI is deployed in diverse real-world environments, including $1.5$\,m-$6$\,m %$5'$-$20'$  
 deep spring waters and $12$\,m-$30$\,m %40'$-$100'$ 
deep natural underwater grottos and caves. These outdoor settings pose unique perception and navigation challenges, such as strong currents and turbidity in open waters, as well as noisy, low-light conditions within overhead environments. These deployments validate the robustness of CavePI, demonstrating its capability to perform long-term autonomous missions in complex underwater environments.

% \JI{talk about trials in Ginnie and blue grotto in low-light conditions. Make some basic statements for now, we can revise it later.}

%CavePI's guided navigation capabilities are initially tested through line following experiments in a laboratory tank with a maximum depth of $1.5$\,m. The test lines are arranged in both regular geometric patterns (\eg, circles and rectangles) and more complex configurations, including irregular curves, spirals, and vertical slopes. Line-following accuracy is assessed in terms of deviation error, defined as the distance of the line from the optical center of the down-facing camera. Moreover, we investigate the delay in decision-making, how it influences overshoot at sharp corners, and fine-tune the PID controller accordingly. The control smoothness, line-following accuracy, and maneuvering capability inside confined spaces are further evaluated with the DT in the Gazebo simulation environment. After extensive iterative testing, we deploy the CavePI in spring-water and natural underwater caves to demonstrate its robustness and applicability for real-world autonomous robotic missions. 



% and in an open-water spring at a depth of XX m. The tracking accuracy is evaluated in terms of normal deviation between the caveline and the vertical axis center of the robot. Additionally, we analyze the depth holding capability and overall motion smoothness along sharp corner turns. Similar evaluations are performed for the DT in Gazebo simulation. 


% we present a graph plotting the normal distance to the caveline from the longitudinal axis of the CavePI.


