\section{System Evaluation}



 
\subsection{Mechanical Stability under Hydrostatic Pressure}
\label{sub:Stability}
The CavePI system is evaluated for structural robustness and functional durability to support long-term autonomous operation at depths. Critical components of AUVs generally include the interfacing joints of the actuators (thrusters), sensors (\eg, cameras and sonar), and enclosure doors~\cite{macias2024numerical}. As shown earlier in Fig.~\ref{fig:system_design}, CavePI’s \textit{head} section is connected to the \textit{main body} via a custom-designed aluminum plate, referred to as the dome connector, which ensures a watertight interface for the forward-facing camera. Similarly, the clamps used in the thruster and sonar subassemblies, as well as the plates for electronics mounting within the computational enclosure, are custom-designed to enhance mechanical efficiency and dynamic stability. Components exposed directly to the underwater environment, including thrusters, the Ping2 sonar, the dome connector, and external cables, are identified as sensitive elements requiring careful design and validation to ensure system reliability under challenging conditions.

\begin{figure}[t]
% \vspace{-1mm}
     \centering
     \includegraphics[width=0.96\columnwidth]{figures/Fig5_a.png}
     \includegraphics[width=0.96\columnwidth]{figures/Fig5_b.png}%
     % \vspace{-1mm}
     \caption{(a) FEA mesh of the dome connector; (b) its total deformation due to hydrostatic forces at $65$\,m water depth; (c) equivalent stresses as per Von-Mises yield criterion; and (d) maximum stresses at the zoomed-in area (best viewed digitally at $2\times$ zoom).}%
     \vspace{-2mm}
     \label{fig:dome_connector_stress}
 \end{figure}


\begin{figure*}[t]
     \centering
     \includegraphics[width=\linewidth]{figures/tank_setup.png}%
     \vspace{-1.5mm}
     \caption{The laboratory setup used for tracking accuracy evaluation is shown. CavePI detects and follows the line laid on the tank floor, completing a loop around the $6$\,meter perimeter in approximately $2$\,minutes.}%
     \vspace{-2mm}
     \label{fig:tank_setup}
 \end{figure*}
%Stress figures

 
 
%With the broader objective of long-term autonomous operation, we evaluate the CavePI system based on structural robustness and functional durability. For autonomous underwater robots, critical components are the actuators (thrusters in this case), sensor interfaces (camera, sonar), and a watertight enclosure~\cite{macias2024numerical,estarki2021design}. As shown in Fig.~\ref{fig:system_design}, CavePI's \textit{head} section is connected to the \textit{main body} through a custom-designed aluminum plate called the dome connector, ensuring watertight camera interfacing. The clamps used for thrusters- and sonar-assembly, and the plates used for electronics assembly inside the computational enclosure are likewise custom-designed to ensure an efficient and dynamically stable robotic system. Thrusters, Ping2 sonar, the dome connector, and cables are the sensitive components as they are exposed to underwater environment. 
%; Thus, we conduct their mechanical analyses to ensure the structural integrity and performance safety margins. 

The thrusters and Ping2 sonar have depth ratings of more than $500$ meters and $300$ meters, respectively, along with their cables. To evaluate the structural strength of the dome connector of CavePI, we use finite element analysis (FEA) methods~\cite{bieze2018finite}. Tetrahedral elements are utilized to generate the finite element mesh from the 3D model of the dome connector, while the Von Mises yield criterion is applied to assess the \textbf{maximum stresses}, \textbf{total deformation}, and \textbf{factor of safety ({\tt FoS})}. The analysis considered hydrostatic pressures~\cite{macias2024numerical,li2023soft} at the maximum water depth of $65$\,m.


The number of nodes and finite elements in the mesh indicates the mathematical model's quality: a higher count typically reflects a more accurate model. In the analysis of the dome connector, the mesh contained $4,636,236$ nodes and $3,300,641$ elements; see Fig.~\ref{fig:dome_connector_stress}. Based on the Von-Mises criterion, the maximum equivalent stress was calculated to be $15.3$\,MPa. The maximum deformation, observed at the inner most circumference of the dome connector, was $0.005$\,mm. The minimum {\tt FoS} is calculated using the yield strength ($\sigma_y$) of the material (Aluminium 6061-T6), and maximum stress ($\sigma_{max}$) as follows: 
\begin{equation*}
    \text{{\tt FoS}} = \frac{\sigma_y}{\sigma_{\text{max}}} = \frac{240 \text{ MPa}}{15.3 \text{ MPa}} \approx 15.76.
\end{equation*}
Note that a {\tt FoS} greater than $2$ for hydrostatic applications and $4$ for hydrodynamics applications is considered sufficient~\cite{kazemi2004reliability,SafetyCulture}. Thus, an {\tt FoS} over $15$ indicates that the dome connector is sufficiently stable for the intended operation pressure.



 




 
\subsection{Caveline Following Performance}\label{sec_5b}
\noindent
\textbf{Setup.} The line following experiments are initially conducted in a  $2$\,m$\times3$\,m laboratory water tank with a maximum depth of $1.5$\,m. 
%$10'\times6'$ laboratory water tank with a maximum depth of $5'$ ($1.52$\,m). 
As a first setup, a line is laid on the tank floor in a rectangular loop, without any depth variation; see Fig.~\ref{fig:tank_setup}. Subsequently, we add small obstacles and vertical slopes in the tank to simulate the rough underwater terrains; see Fig.~\ref{fig:all_setup}\,(a). Upon detecting the line via the down-facing camera, CavePI autonomously follows the loop while maintaining a specified depth. The system's line-following accuracy, depth-holding capabilities, and control smoothness are evaluated and fine-tuned to prepare for field deployments.

% \begin{figure}[h]
% \centering
% \includegraphics[width=\linewidth]{figures/tank_results.png}%
% \vspace{-1mm}
% \caption{Line-following and depth-holding accuracy are reported for a laboratory experiment: (a) The target depth was set to $0.35$\,m; (b) the tracking error ($\delta$), \ie, the offset of the optical center of the camera from the detected line is plotted for a rectangular loop; (c) the trajectory is visualized on the lab tank setup. \JI{Where are you putting the a.b.c? why is it at the top for this image only? The images and fonts seem stretched - check aspect ratio}
% }
% \label{fig:deviation}
% \vspace{-2mm}
% \end{figure}

\begin{figure}[ht]
\centering
\begin{subfigure}[]{0.9\linewidth}
\includegraphics[width=\linewidth]{figures/tank_depth2.png}%
% \vspace{-1mm}
\caption{Depth Control Accuracy}
% \vspace{-1mm}
\end{subfigure}
\begin{subfigure}[]{0.9\linewidth}
\includegraphics[width=\linewidth]
{figures/tank_deviation2.png}%
% \vspace{-1mm}
\caption{Tracking Accuracy}
\end{subfigure}
\caption{Line-following and depth-holding accuracy are reported for a laboratory experiment: (a) The target depth was set to $0.35$\,m; (b) the tracking error ($\delta$), \ie, the offset of the optical center of the camera from the detected line is plotted for a rectangular loop following task.  
}
\label{fig:deviation}
\vspace{-2mm}
\end{figure}


\vspace{1mm}
\noindent
\textbf{Evaluation process.}  CavePI's caveline-following performance is evaluated by measuring the \textit{tracking error}, defined as the perpendicular distance between the optical center of the downward-facing camera and the nearest point on the caveline in the plane of caveline; see Fig.~\ref{fig:deviation}. This measurement leverages pose information from CavePI's IMU, depth readings from the sonar, and visual information from the segmentation mask of the caveline captured by the down-facing camera.

To formalize the calculation, we define several reference frames. The initial pose of the robot assumes its front axis is aligned with the caveline and its downward axis perpendicular to the caveline plane. This reference frame, denoted as \{N\}, has its origin at the camera. A corresponding frame \{S\}, with the same orientation as \{N\} but with its origin at the sonar, is also defined. Additionally, we define the global reference frame \{G\}, and the CavePI body frame \{C\} with its origin at the down-facing camera. The transformation matrix from any frame $a$ to $b$ is defined as ${}^{b}_{a}\mathbf{T} = [ {}^{b}_{a}\mathbf{R}_{3\times3} \,|\, {}^{b}_{a}\mathbf{t}_{3\times1}]$. The roll, pitch, and yaw angle measurements from IMU are expressed in the global frame \{G\} at any time instance.

For a given image $I$, we extract the caveline contour closest to the camera center and calculate the distance between the camera center and the pixel ${}^IP$ on the contour edge, then convert it to a 3D point ${}^C\boldsymbol{P}$ using:

\[
  {}^I{P} = 
  \begin{bmatrix}
      u_{P} & v_{P} & 1
  \end{bmatrix}^T, 
  \quad
  {}^C\boldsymbol{P} = \lambda \, K^{-1} ({}^IP);
\] where $K$ is the camera intrinsic matrix and  $\lambda$ is the depth scale factor.
The rotation of the CavePI's frame with respect to the sonar frame is calculated as follows:
\[
      {}^{C}_{S}\boldsymbol{R} 
      = \left({}^{S}_{G}\boldsymbol{R}\right)^{-1} {}^{C}_{G}\boldsymbol{R} 
      = \left({}^{N}_{G}\boldsymbol{R}\right)^{-1} {}^{C}_{G}\boldsymbol{R};
    \]
where ${}^{C}_{G}\boldsymbol{R}$ and ${}^{N}_{G}\boldsymbol{R}$ are computed from instantaneous and initial IMU measurements, respectively. %Note that, ${}^{S}_{G}\boldsymbol{R}$ and ${}^{N}_{G}\boldsymbol{R}$ are same because of the same orientation of frame \{S\} and \{N\}.
Subsequently, ${}^C\boldsymbol{P}$ is transformed to the sonar frame using:   
\[
      \begin{bmatrix}
          {}^S\boldsymbol{P} \\
          1
      \end{bmatrix}
      =
      \begin{bmatrix}
          {}^{C}_{S}\boldsymbol{R} & {}^{C}_{S}\boldsymbol{t} \\
          0 & 1
      \end{bmatrix}
      \begin{bmatrix}
          {}^C\boldsymbol{P} \\
          1
      \end{bmatrix},
    \]
    \[
      {}^S\boldsymbol{P} 
      = \lambda
      \begin{bmatrix}
          a & b & c
      \end{bmatrix}^T,
    \]
where the scalars \(a\), \(b\), and \(c\) are obtained from the above equation. The raw depth measurement from sonar ${}^Cd$ is converted to the sonar frame ${}^Sd$ and $\lambda$ is calculated as:
    % \[
    %   {}^Sd = {}^Cd \cos\phi \cos\theta, ~ \text{and}~\lambda = \frac{{}^Sd}{c}.
    % \]
    \[
        {{}^{S}d} = {}^{C}_{S}\boldsymbol{R}_{(3,3)} \cdot {}^{C}d, ~ \text{and}~\lambda = \frac{{}^Sd}{c}.
    \]
The 3D coordinates of ${}^N\boldsymbol{P}$ is calculated using:
    \[
      {}^N\boldsymbol{P} = -{}^{C}_{S}\boldsymbol{t} + {}^S\boldsymbol{P}
    \]
Finally, $\delta$ is obtained from the \(y\)-coordinate of \({}^N\boldsymbol{P}\).





    % where ${}^Cd$ is given by the Ping2 sonar measurement.
    % \[
    %   \lambda = \frac{(d_{S})}{c}
    % \]
    
% The \(z\)-coordinate of \({}^S\boldsymbol{P}\) is also derived from sonar measurements.
% ${}^Cd$: Distance of the caveline plane from the sonar’s top center in frame \(\{C\}\), given by the Ping2 sonar measurement.



    % \textbf{Assumption:} During initialization, CavePI’s front axis is aligned with the caveline, and its downward axis is perpendicular to the caveline plane.
    
    % \item \(\{N\}\): Fixed reference frame at the optical center of the downward-facing camera at any time during the motion, with the x-axis along the caveline and the z-axis perpendicular to the caveline plane.
    % \item \({}^F\boldsymbol{R}^{N}\): Rotation matrix of frame \(\{N\}\) with respect to frame \(\{F\}\), computed from the initial Euler angle measurements from the IMU.
    % \item \(\{S\}\): Reference frame at the sonar’s top center, with axes parallel to frame \(\{N\}\) at any time during the motion.
    % \item \({}^F\boldsymbol{R}^{S}\): Rotation matrix of frame \(\{S\}\) with respect to \(\{F\}\).

% \begin{itemize}
    % \centerline{\({}^F\boldsymbol{R}^{S} = {}^F\boldsymbol{R}^{N}\)}
    % \item \(\{C\}\): CavePI's base frame
    % \item \(\{I\}\): Image frame of the downward-facing camera.
    % \item \(K\): Intrinsic matrix of the downward-facing camera.
    % \item \(P\): Closest point on the caveline, computed as the center of the contour having the least distance from the center pixel of the image.
    % \item \({}^I\boldsymbol{P}\): Homogeneous 2D coordinates of \(P\) in frame \(\{I\}\).
    % \item \({}^C\boldsymbol{P}\): 3D coordinates of \(P\) in reference frame \(\{C\}\).
    % \[
    %   {}^I\boldsymbol{P} = 
    %   \begin{bmatrix}
    %       u_{P} & v_{P} & 1
    %   \end{bmatrix}^T, 
    %   \quad
    %   {}^C\boldsymbol{P} = \lambda \, K^{-1} ({}^IP),
    % \]
    % where \(\lambda\) is a constant.
    % \item \
    % \item \({}^F\boldsymbol{R}^{C}\): Rotation matrix of frame \(\{C\}\) in reference frame \(\{F\}\), computed from the current Euler angle measurements from the IMU.
    % \item \({}^S\boldsymbol{T}^{C}\): Position vector of the origin of frame \(\{C\}\) in reference frame \(\{S\}\).
    % \item \({}^S\boldsymbol{R}^{C}\): Rotation matrix of frame \(\{C\}\) in frame \(\{S\}\).
    % \[
    %   {}^S\boldsymbol{R}^{C} 
    %   = \left({}^F\boldsymbol{R}^{S}\right)^{-1} {}^F\boldsymbol{R}^{C} 
    %   = \left({}^F\boldsymbol{R}^{N}\right)^{-1} {}^F\boldsymbol{R}^{C}.
    % \]
    % \item \({}^S\boldsymbol{P}\): 3D coordinates of \(P\) in reference frame \(\{S\}\).
 
    
% \end{itemize}






 

\subsection{PID Controller Tuning}
\vspace{-1 mm}
To ensure stable operation in turbulent water conditions, CavePI’s PID-based Pure Pursuit controller is designed and fine-tuned through extensive experimentation. Initial trials reveal challenges, including high overshoot, difficulty maintaining depth, and instability at sharp corners. These issues are iteratively calibrated to achieve robust and reliable motion control. A grid search technique is employed to optimize the proportional gain ($K_p$) and differential gain ($K_d$) for the two onboard controllers -- depth and heading. For each tested parameter pair, the robot traverses a complete 6-meter loop in the testbed at a depth of $0.35$ meters. Tracking ($\delta$) and depth errors for each trial are recorded at $0.6$\,Hz. Table~\ref{tab:performance_metrics} and~\ref{tab:depth_controller_tuning} summarize the observations of the tuning parameters combinations for heading and depth controllers, respectively. We identified that a controller tuned with $K_p = 3.4$ and $K_d = 0.9$ minimized the tracking error, and therefore employed these values as the heading controller gains in all subsequent experiments. Similarly, $K_p = 600$ and $K_d = 50$ are chosen for the depth controller.

\setlength{\extrarowheight}{2pt}
\begin{table}[h]
\centering
\caption{Tracking errors from the line following experiments are compared for various choices of $K_p$ and $K_d$ values of the \textit{heading} controller. Here, $\delta$: mean tracking error (cm), $\sigma$: standard deviation.}%
\vspace{-1mm}
\resizebox{\linewidth}{!}{
\begin{tabular}{c||cccccccccc}
\Xhline{2\arrayrulewidth}
\textbf{$K_p$} & 1 & 2 & 3 & 3 & 3.5 & 3.5 & 3.4 & \textbf{3.4} & 3.5 & 3.4 \\ \hline
\textbf{$K_d$} & 0 & 0 & 0 & 0.5 & 0.5 & 0.7 & 0.7 & \textbf{0.9} & 0.9 & 1.0 \\ \hline
\textbf{$\delta$} & 19.8 & 17.2 & 15.9 & 14.7 & 14.1 & 13.9 & 13.6 & \textbf{13.1}  & 13.4 & 13.3 \\ \hline
\textbf{$\sigma$} & 26.5 & 24.1 & 22.3 & 21.6 & 20.9 & 21.2 & 20.3 & \textbf{19.8} & 21.5 & 20.0 \\
\Xhline{2\arrayrulewidth}
\end{tabular}
}
\label{tab:performance_metrics}
\vspace{-2mm}
\end{table}


% \setlength{\extrarowheight}{2pt}
\begin{table}[h]
\centering
\caption{Depth errors from the line following experiments are compared for various choices of $K_p$ and $K_d$ of the depth controller. Here, $\mu$: mean depth error (cm), $\sigma$: standard deviation.}%
\vspace{-1mm}
\resizebox{\linewidth}{!}{
\begin{tabular}{c||cccccccccc}
\Xhline{2\arrayrulewidth}
\textbf{$K_p$} & 500 & 500 & 550 & 600 & 600 & 600 & \textbf{600} & 600 & 650 & 720 \\ \hline
\textbf{$K_d$} & 0 & 10 & 10 & 10 & 20 & 30 & \textbf{50} & 100 & 200 & 300 \\ \hline
\textbf{$\mu$} & 3.09 & 2.98 & 2.47 & 2.39 & 2.06 & 2.61 & \textbf{1.91} & 1.80 & 2.98 & 1.92 \\ \hline
\textbf{$\sigma$} & 2.14 & 1.97 & 1.72 & 1.63 & 1.38 & 1.84 & \textbf{1.38} & 1.70 & 2.02 & 1.44 \\
\Xhline{2\arrayrulewidth}
\end{tabular}
}
\label{tab:depth_controller_tuning}
\vspace{-2mm}
\end{table}
% where,  represent the mean depth error and the standard deviation of the depth error in cm, respectively.

%\JI{Is this subsection and tables incomplete? seems very weird to me. Parameters are not introduces, captions are not formatted. Check please}

\begin{figure}[b]
    \centering
    % \vspace{2mm}
    \includegraphics[width=\columnwidth]{figures/CavePi_gazebo.png}%
    %\vspace{-1mm}
    \caption{The digital twin (DT) of CavePI, modeled in ROS, is used for virtual testing in Gazebo underwater environment. (a) An isometric view of the DT in Gazebo; (b,c) Line following experiments in a hexagonal loop and a lawn-mower pattern, respectively.
    }%
    \vspace{-4mm}
    \label{fig:gazebo}
\end{figure}

\subsection{ROS Prototyping and Gazebo Simulation}
The digital twin of CavePI is developed using ROS and simulated within Gazebo’s underwater environment to evaluate its line-tracking performance; see Fig.~\ref{fig:gazebo}. As a simplified representation of the physical platform, it enables extensive pre-deployment testing of mission-critical functionalities. While the actual CavePI is positively buoyant -- automatically resurfacing in the absence of control signals -- the digital twin is designed to be neutrally buoyant to minimize continuous depth-control inputs. Simulating the digital twin under realistic underwater conditions allows for detailed analysis of hydrostatic and hydrodynamic forces, informing component placement for stable buoyancy and achieving near-zero roll and pitch angles when stationary. A complete sensor suite is integrated to confirm optimal camera placement, particularly for the downward-facing camera essential for caveline detection. Additionally, it incorporates a PID controller to simulate CavePI’s line-tracking functionality, serving as a platform for preliminary control algorithm design and validation. Nevertheless, final controller tuning is performed on the physical system through experimental trials due to factors such as nonlinear drag forces, caveline detection dependencies, and design parameter variations between the CAD model and the actual hardware, which cannot be replicated in simulation.





%The digital twin is developed using the ROS and simulated within Gazebo’s underwater environment to evaluate CavePI’s line-tracking performance (see Fig.~\ref{fig:gazebo}). As a simplified representation of the physical CavePI, the digital twin facilitates extensive pre-deployment testing of mission-critical functionalities. While the actual CavePI is engineered to be positively buoyant—enabling automatic resurfacing in the absence of control signals—the digital twin is designed to be neutrally buoyant, thereby reducing the need for continuous depth-control inputs. Simulating the digital twin under realistic underwater conditions enables a detailed examination of hydrostatic and hydrodynamic forces, guiding component placement in the assembly to ensure stable buoyancy and yielding near-zero roll and pitch angles when the robot is stationary. Furthermore, incorporating a complete sensor suite confirms optimal camera positioning for enhanced visual perception, particularly for the downward-facing camera essential to caveline detection. In addition, the digital twin integrates a PID controller to simulate CavePI’s line-tracking functionality, serving as a platform for preliminary design and validation of the control algorithm. Nevertheless, final controller tuning is conducted on the physical system through experimental trials, as complexities, such as nonlinear drag forces, dependency on caveline detection, and differences in design parameters between the CAD model and actual system, cannot be fully replicated in simulation.

%\JI{talk about ROS2 simulation here. This is the systems evaluation section, so talk about any evaluation or tuning that you did with the simulator}






