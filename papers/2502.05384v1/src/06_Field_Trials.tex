
\vspace{1mm}
\section{Field Experimental Demonstration}
\subsection{Field Trials and Experimental Setups}
The guided navigation capability of CavePI was demonstrated in two distinct real-world environments: (1) shallow riverine areas ($2$\,m - $6$\,m depth) near springs' outlets; and (2) deep natural underwater grottos and caves ($15$\,m - $30$\,m depth).  
The experiments included $15$ open-water trials in spring areas and $10$ trials inside underwater grottos. In the open-water trials, a $20$-meter rope line was laid in both irregular loop patterns and linear configurations along the uneven riverbed. In contrast, cave trials used an actual caveline, which varied in color, texture, and thickness. Each environment posed unique challenges. Strong currents near the spring outlets caused significant drift, particularly when navigating across or against the flow. Additionally, low-light conditions inside underwater caves hindered accurate detection of the semantic markers.


\begin{figure}[b]
    \centering
    % \vspace{2mm}
    \includegraphics[width=\columnwidth]{figures/divers.png}%
    \vspace{-1mm}
    \caption{(a) A support diver is placing the robot on a caveline to initiate tracking; (b) two divers are following the robot once tracking is initiated. Note that the tether is used only for remotely aborting the mission in case of emergency.
    }%
    %\vspace{-2mm}
    \label{fig:divers}
\end{figure}

Field deployments were conducted under the supervision of two support divers responsible for initiating and concluding the missions. Divers carried QR-code tags to issue visual commands to CavePI via its front-facing camera. Upon receiving the \texttt{start} command, the robot activated depth-hold mode and began line-following while maintaining the specified depth. A surface station operator monitored each mission remotely. For cave deployments, a tether connected the robot to the surface station to enable remote emergency intervention if necessary (see Fig.~\ref{fig:divers}). The trials highlighted both the strengths and potential areas for improvement in CavePI's perception, planning, and control mechanisms. 
% Representative snapshots of these trials are shown in Fig.~\ref{fig:all_setup}. 


\begin{figure}[h]
\centering
\begin{subfigure}[]{0.9\linewidth}
\includegraphics[width=\linewidth]{figures/field_depth2.png}%
% \vspace{-1mm}
\caption{Depth Control Accuracy}
% \vspace{-2mm}
\end{subfigure}
\begin{subfigure}[]{0.9\linewidth}
\includegraphics[width=\linewidth]
{figures/field_deviation.png}%
% \vspace{-1mm}
\caption{Tracking Accuracy}
\end{subfigure}
\caption{Line-following and depth-holding accuracy are reported for an open-water experiment: (a) the target depth was set to $0.525$\,m; (b) the higher tracking error is caused by strong currents, leading to loss of tracking on one occasion.
}
\label{fig:field_results}
\vspace{-3mm}
\end{figure}



\begin{figure*}[t]
     \centering
     \includegraphics[width=\linewidth]{figures/all_trials.jpeg}%
     \vspace{-2mm}
     \caption{A few snapshots of caveline-tracking experiments with CavePI are shown. The setups include: (a) rectangular loops and slopes in a water tank; (b) irregular shapes in Spring water, and (c) low-light underwater cave scenarios (note that the tether is for safety). Images are best viewed digitally at $2\times$ zoom; video demonstrations can be seen here: \url{https://robopi.ece.ufl.edu/cavepi.html}. 
     }%
     \vspace{-1mm}
     \label{fig:all_setup}
 \end{figure*}

 
\subsection{Results and Observations}

\vspace{1mm}
\noindent
\textbf{Laboratory tank tests.} Sec.~\ref{sec_5b} details the setup and evaluation procedures for assessing CavePI's line-following performance in laboratory tests. During these tests, CavePI operated at a depth of $0.35$ meters below the water surface. The depth controller demonstrated robust performance, maintaining the desired depth with a mean error of approximately $2$\,cm. At the beginning of each trial, heading control signals were disabled for $10$\,seconds to allow CavePI to stabilize at the target depth after its descent. Consequently, higher tracking errors were observed during this initial period, as shown in Fig.~\ref{fig:deviation}. Additionally, occasional perception failures were encountered, where the tank edges were misidentified as caveline segments (see Fig.~\ref{fig:perception_failure}\,a), leading to \textit{overshooting}. Despite these transient deviations, CavePI successfully completed multiple loops, demonstrating the robustness of its control strategy.

\vspace{1mm}
\noindent
\textbf{Open-water tests (Spring riverine environments).} These trials were evaluated using the same methodology described in Sec.~\ref{sec_5b}, focusing on both line-tracking and depth-regulation accuracy. Fig.~\ref{fig:field_results} presents a representative result from a 10-minute open-water experiment. As expected, the field results exhibit greater variation compared to the controlled laboratory experiments due to environmental disturbances. Despite these challenges, the depth controller demonstrated robust performance, maintaining a depth within $\pm 10$\,cm without requiring additional trial-specific parameter tuning.

In contrast, the heading controller proved less resilient under strong currents, resulting in much higher tracking errors. As illustrated in Fig.~\ref{fig:field_results}\,b, the system experienced significant lateral drift, occasionally causing CavePI to lose the caveline. This limitation became particularly pronounced when traveling perpendicular to the current (see Fig.~\ref{fig:tracking_failure}\,b), as there was no active lateral control to counter crossflow. Additionally, the vehicle’s slightly back-heavy design induced a pitch-up motion during upstream travel, as shown in Fig.~\ref{fig:tracking_failure}\,c, further compromising the stability and navigation accuracy of the CavePI. %Nevertheless, such strong currents and waves are less common inside underwater caves. 



% \vspace{1mm}
% \noindent
% \textbf{Open-water tests.} These trials are evaluated using the same process outlined in Sec.~\ref{sec_5b}, \ie, in terms of line tracking and depth regulation accuracy. Fig.~\ref{fig:field_results} illustrates a sample result from a $10$-minute open-water experiment. As expected, the field results exhibit more variation compared to controlled laboratory experiments (Fig.~\ref{fig:deviation}) due to environmental disturbances. Despite these challenges, the depth controller demonstrates a consistently robust performance, maintaining depth within $\pm 10$\,cm of the target level without the need for trial-specific parameter tuning.



% \vspace{1mm}
% \noindent
% \textbf{Effects of water currents.} As opposed to the depth controller, which remains less affected by environmental variations, the heading controller struggles against strong currents, leading to significantly higher deviation errors. As illustrated in Fig.~\ref{fig:field_results}\,b, substantial drift occurs, occasionally causing the robot to lose track of the caveline. This issue is particularly pronounced when CavePI navigates perpendicular to the current as shown in Fig.~\ref{fig:Drifting}, as it lacks lateral motion control to actively compensate for drift. Additionally, the vehicle's slightly back-heavy design induces a pitch-up motion when traveling upstream as illustrated in Fig.~\ref{fig:Pitching}, further impacting stability and navigation accuracy.

\vspace{1mm}
\noindent
\textbf{Low-light tests (Nighttime cave environments).} These trials utilized an earlier (ablation) iteration of CavePI equipped with a three-thruster configuration, in contrast to the newly proposed four-thruster design. The three-thruster setup required precise orientation of the downward thruster to achieve proper heave motion; otherwise, thrust forces could inadvertently induce \textit{roll}. Additionally, the absence of a dedicated roll-control mechanism posed challenges for stabilization. In contrast, the four-thruster configuration incorporates two thrusters for heave motion, enabling dedicated roll control and significantly improving overall stability during navigation.

Despite being equipped with two onboard lights, CavePI faced considerable challenges in low-light environments due to glare and backscattering effects, which impeded accurate semantic perception. Nighttime trials conducted in an underwater cave/grotto system revealed that the lightweight segmentation models struggled to detect the caveline from camera images. During two one-hour dives, support divers reported instances where submerged tree roots, mosses, and other thin structures were mistakenly identified as the caveline. However, once manually repositioned onto the caveline by the divers, CavePI was able to maintain tracking for up to one minute. These findings underscore the need for integrating a more powerful computational platform capable of running robust segmentation models to enhance CavePI’s perception capabilities in challenging environments.

%\JI{I wrote three sample points from the top of my head last night. You should also think more and find some other aspects to discuss or you observed during the experimetns!}


% Moreover, the segmentation pipeline demonstrates robust performance across varying water colors and lighting conditions, consistently identifying the caveline with high reliability. 

% However, in some instances, CavePI successfully regains the path through $360^\circ$ look-arounds, while in a few cases, intervention from support divers was required to correctly reposition the robot.





% \begin{figure}[h]
%     \centering
%     % \vspace{2mm}
%     \includegraphics[width=\columnwidth]{figures/field_results.png}%
%     %\vspace{-1mm}
%     \caption{Line-following and depth-holding accuracy are reported for an open-water experiment: (a) the target depth was set to $0.525$\,m; (b) the higher deviation error is caused by strong currents, leading to loss of tracking on one occasion. \JI{what fonts are you using in this image? something random?}
%     }%
%     \vspace{-2mm}
%     \label{fig:field_results}
% \end{figure}


