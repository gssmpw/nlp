\section{CavePI System Design}
The CavePI AUV design includes three major subsystems: sensory bay, computational bay, and locomotion bay. Our proposed system and its components are shown in Fig.~\ref{fig:system_design}.  
% \vspace{-1 mm}


\subsection{Sensor Bay: Acoustic-Optic Perception Subsystem}
The CavePI platform includes visual and acoustic sensors: a front-facing fisheye camera, a downward-facing low-light camera, and a Ping2 active sonar. The fisheye camera, housed within a transparent dome at the \textit{head} of the AUV, captures forward-facing visuals with a $160^\circ$ field-of-view (FOV) and outputs a video feed at $1920\times1080$ resolution. It is worth noting that the cylindrical enclosure is a $6''$ tube while the dome has a $4''$ diameter. A custom interface is built to connect the two; see section \ref{sub:Stability} for more details. The low-light camera, mounted inside the computational enclosure, captures downward-facing visuals with an $80^\circ \times 64^\circ$ FOV, also at the same resolution. Additionally, a Ping2 sonar altimeter-echosounder from Blue Robotics\texttrademark{} is mounted on the underside of the robot; the sonar has a range of $100$ meters, a depth rating of $300$ meters, and a resolution of $0.5\%$ of the range, allowing it to detect obstacles in the surrounding environment beneath CavePI. These sensory components collectively provide robust environmental awareness for autonomous navigation in challenging underwater environments.
%This camera provides critical imagery for tracking cavelines, enabling autonomous operation. 
% \vspace{-1mm}




\subsection{Computational Bay}
\vspace{-1 mm}
As illustrated in Fig.~\ref{fig:system_design}, the computational and electronic components of CavePI are housed within an acrylic cylindrical enclosure. This enclosure, with a thickness of $6.35$\,mm and a depth rating of 65 meters, forms the \textit{main body} of the robot, providing mechanical stability, buoyancy, and waterproof protection for the electronics. The computational elements include a Raspberry Pi-5, a Nvidia\texttrademark{} Jetson Nano, and a Pixhawk\texttrademark{} flight controller. The Jetson Nano is dedicated to processing visual data from the cameras, performing image processing tasks critical for scene perception and state estimation. The Raspberry Pi-5 manages planning and control modules, ensuring real-time underwater navigation. The Pixhawk flight controller acts as a bridge between hardware and software, receiving actuation commands from the Raspberry Pi-5 and transmitting them to the thrusters and lights via the MAVLink communication protocol. Additionally, the Pixhawk integrates a 9-DOF IMU, offering 3-axis gyroscope, accelerometer, and magnetometer measurements, which are used to calculate the attitude of CavePI during underwater operations.

The enclosure also contains the battery compartment, voltage regulators, electronic speed controllers (ESCs), and a Bar-30 pressure sensor. The battery compartment holds a $14.8$\,V ($18$\,Ah) battery pack, regulated to power internal components (\eg, cameras, computers) and external components (\eg, thrusters, sonar). Each thruster is controlled by an ESC, which drives the three-phase brushless motor using PWM signals from the Pixhawk. The Bar-30 sensor provides high-precision pressure readings with a resolution of $0.2$\,mbar and an accuracy of $2$\,mm, with a working depth of up to $300$ meters. This pressure data is processed to determine CavePI’s underwater depth, ensuring reliable and accurate interoceptive perception during operations.

%Furthermore, it houses the battery compartment, voltage regulators, electronic speed controllers (ESCs), and a Bar-30 sensor. The battery compartment features a $14.8$V, $18$Ah pack -- which is regulated to individual components inside (e.g. cameras, computers) and outside (e.g. thrusters, sonar) of the enclosure. The ESCs, one for each thruster, controls a three-phase brushless motor inside a thruster by receiving an input PWM signal from Pixhawk. The Bar-30 sensor provides pressure readings with a resolution of $0.2$\,mbar and $2$\,mm accuracy up to a working depth of $300$\,m.  This pressure data is then processed to compute the underwater depth of the CavePI ensuring robust and accurate interoceptive perception.



% \begin{figure*}[t]
%      \centering
%      \begin{subfigure}[]{0.49\textwidth}
%          \centering
%          \includegraphics[width=\linewidth]{figures/Fig4_Electronics.png}%
%          \vspace{-1.5 mm}
%          \caption{Major electronics and sensor-actuator connections.}
%          \label{fig:electronics}
%      \end{subfigure}~     
%      \begin{subfigure}[]{0.47\textwidth}
%          \centering
%          \includegraphics[width=\linewidth]{figures/Fig4_ROS.png}
%          \vspace{-1.5 mm}
%          \caption{Data flow of major computational modules in the form of \textit{ROS topics}: red and blue arrows represent \textit{subscribed} and \textit{published} topics, respectively.}
%          \label{fig:ROS}
%      \end{subfigure}
%      \vspace{-1 mm}
%         \caption{Simplified outlines of the end-to-end hardware, software, and ROS2 middleware integration of NemoGator are shown.}%
%      \label{fig:hw_mw_sw}
% \vspace{-4 mm}
% \end{figure*}




% \subsection{Locomotion Subsystem}
% \vspace{-1 mm}
% As opposed to traditional thruster-based AUV systems (eg, CUREE~\cite{girdhar2023curee}, ReefGlider~\cite{macauley2024reefglider}, LoCO~\cite{edge2020design}), NemoGator employs a bio-inspired propulsion system~\cite{zhang2010biologically} driven by three servo motors that actuate a \textit{caudal} tail (BCF) and two \textit{pectoral} fins (MPF). This design is inspired by the carangiform design~\cite{macias2024numerical,raj2016fish,costa2018design} (see Sec~\ref{sec:background}), combining the benefits of BCF and MPF for propulsion and maneuverability control, respectively~\cite{zhang2021development,marchese2013towards}. Specifically, the tail generates forward thrust and contributes to yaw motions, powered by a single servo motor; the pectoral fins ensures the stability of the system, regulating pitch within the water column, and assisting in roll and yaw motions~\cite{zhang2021design}. They are powered by respective servo motors, ensuring low-power operation with only three $35$\,Kg-cm torque ($7.4$V) motors. 

% When the fins are angled upward, the resulting hydrodynamic lift combined with tail propulsion, enabling ascent. Conversely, downward angling of the fins induces controlled descent, allowing depth modulation in the water column. We make sure that these fins oscillate synchronously for forward propulsion, while independent actuation of a single fin modulates yaw, enabling precise directional control. 

% The caudal fin of the NemoGator is directly connected to a servo motor to control the oscillations of the caudal fin. Forward movement is achieved through symmetric oscillations of the caudal fin around the NemoGator's longitudinal axis, while the yaw control is managed through asymmetric oscillations.
% cite:Towards a Self-contained Soft Robotic Fish: On-Board Pressure Generation and Embedded Electro-permanent Magnet Valves 

% The two side fins are engineered to replicate the functional dynamics of pectoral fins in fish, contributing to the stability of the system, regulating pitch within the water column, and assisting in roll and yaw motions. The pectoral fins of the NemoGator are directly connected to their respective servo motors, which are mounted on the outer body.%cite:Design and Locomotion Control of a Dactylopteridae-Inspired Biomimetic Underwater Vehicle With Hybrid Propulsion

% Specifically, oscillating only the left fin generates a yaw moment, inducing a left turn, whereas oscillating the right fin similarly facilitates a right turn.
% \vspace{-1mm}


 

% CavePI AUV is intended for low-power autonomous operation with portable ROS support for application-specific perception, planner, and control modules. As shown in Fig.~\ref{fig:ROS}, we incorporate SVIn (sonar-visual-inertial navigation)~\cite{rahman2022svin2} packages for general-purpose state estimation and waypoint-based trajectory planning. NemoGator can also be used as an underwater ROV with an optional tether-based TeleOp module. Depending on \textit{ROV mode} or \textit{autonomous mode}, a unified AutoPilot package is designed to control the servo motor commands for navigation.

\begin{figure}[t]
% \vspace{-1 mm}
    \centering
    \includegraphics[width=0.98\linewidth]{figures/Fig4_Electronics.png}%
    \vspace{-1mm}
    \caption{Major electronics and sensor-actuator connections of CavePI.}
    \label{fig:electronics}
    \vspace{-3mm}
\end{figure}


\subsection{Locomotion Bay: Middleware Integration}
\vspace{-1 mm}
The end-to-end integration of CavePI ensures that each computational component operates in sync, tied to a ROS2 Humble-based middleware backbone. The modular design also allows for future upgrades, ensuring that the CavePI can be tailored to meet evolving research in marine ecosystem exploration and monitoring. The sensor-actuator signal communication graph is illustrated in Fig.~\ref{fig:electronics}.


The CavePI autonomous underwater vehicle (AUV) is designed for low-power operation and integrates a portable ROS2 framework to support application-specific perception, planning, and control modules. As depicted in Fig.~\ref{fig:ROS}, the \textit{detector} node acquires visual data from the two cameras to identify the caveline for navigation. In the absence of GPS underwater, the system employs SVIn (sonar-visual-inertial navigation)~\cite{rahman2022svin2} packages to estimate the AUV’s position relative to the detected caveline. The \textit{mission planner} node then integrates the caveline information with the estimated position data to generate subsequent waypoints for the mission. Finally, the \textit{autopilot controller} node utilizes these waypoints, along with the detected caveline, positional data, and depth readings from the Bar-30 sensor, to generate precise actuation signals for the thrusters, enabling accurate movements and depth control. Additionally, CavePI can function as an ROV through an optional tether-based teleoperation module. This module transmits user commands from a joystick to the onboard Raspberry Pi-5, which processes the inputs and relays them to the thrusters for manual control.

\begin{figure}[t]
    \centering
    \includegraphics[width=\linewidth]{figures/Fig4_ROS.png}%
    \vspace{-1mm}
    \caption{Data flow among major computational modules of CavePI is shown in the form of \texttt{ROS Topics}: red and blue arrows represent \textit{subscribed} and \textit{published} topics in the ROS graph, respectively.}
    \label{fig:ROS}
    \vspace{-4mm}
\end{figure}


\subsection{CavePI Digital Twin}
%\vspace{-1 mm}
% The digital twin of CavePI is created in ROS, contains a similar structure as presented in Fig.~\ref{fig:ROS}. 
We develop a digital twin (DT) model of CavePI by using the Unified Robot Description Format (URDF), with links and joints carefully assigned to represent the various CAD components designed in SolidWorks. To replicate the sensor suite of the physical CavePI, Gazebo plugins are integrated to simulate the front-facing camera, down-facing camera, IMU, pressure sensor, and sonar. Additional plugins are employed to simulate environmental forces, including buoyancy, thrust, and hydrodynamic drag, thereby enhancing the physical realism.

A controlled open-water scenario is created in Gazebo to simulate realistic missions, featuring a thin line arranged in a rectangular loop to mimic a caveline. Since the simulated environment lacks real-world perception challenges such as low light or turbid water conditions, the perception subsystem remains simplified. Instead of deploying computationally intensive deep visual learning models, simpler edge detection and contour extraction techniques~\cite{SUZUKI198532} are used to identify the caveline from the down-facing camera feed. The remaining navigation and control subsystems mirror the real-world implementation and operate via two ROS nodes. The first node processes the extracted contours to make navigation decisions and publishes high-level control commands (\eg, yaw angle). The second node subscribes to these commands, computes the required thrust and hydrodynamic drag forces, and publishes them as ROS topics to control the simulated robot model.

Beyond replicating caveline following experiments, we utilize the DT system for preliminary testing and fine-tuning of new control algorithms. It also enables the simulation of complex cave scenes, such as narrow passages, dead ends, and sharp turns. Conducting repeated real-world experiments in such scenarios to improve the control system can be logistically demanding where the simulation offers an efficient alternative for extensive evaluation and fine-tuning.  

% These force commands are then subscribed to by the robot model in Gazebo, enabling it to execute the desired motion accurately. Finally, the URDF and Gazebo world files, along with the three control nodes, are integrated into a ROS launch file. Executing this launch file initiates a simulation of CavePI's digital twin in Gazebo, demonstrating its ability to follow the caveline within an underwater environment.

% To control the digital twin to follow the cave line in the underwater world, three ROS nodes are created for different applications. In the first ROS node, subscribed frames from the downward-facing camera are utilized by a caveline detection algorithm~\cite{SUZUKI198532} that publishes contours along the caveline. In the second node, control decisions are taken based on the detected contours along the caveline and high-level control signals are published. In the third node, these high-level control signals are subscribed, and thrust and hydrodynamic drag forces on the robot model are calculated. These forces are then published to ROS topics and later subscribed by the robot model in Gazebo to achieve the desired motion.

% \subsubsection{Integration}
% Finally, the URDF and Gazebo world files, along with the three control nodes, are integrated into a ROS launch file. Executing this launch file initiates a simulation of CavePI's digital twin in Gazebo, demonstrating its ability to follow the caveline within an underwater environment.