\newpage
\appendix
\onecolumn

\section{Proof of the optimal mask $\bp^*$}\label{sec_appx:proof_mask}
\begin{definition}\label{def:tops}
The operator TOP-$C: \mathbb{R}^d \rightarrow \mathbb{R}^d$, for $1 \leq C \leq d$ is defined as
\begin{equation}\
    \left( \text{TOP-}C(\bx) \right)_{\pi(i)} := \left\{
\begin{array}{ll}
      x_{\pi(i)}, & i \leq C \\
      0, & \text{otherwise},\\
\end{array} 
\right. \nonumber
\end{equation}
where $\bx = (x_1, \ldots, x_d)^{\top} \in \RR^d$ and $\pi$ is a permutation of $\{1, 2, \ldots, d \}$ such that $|x_{\pi(i)}| \geq |x_{\pi(i+1)}|$, for $i=1, \ldots, d-1$, i.e. the TOP-$S$ operator keeps only the $S$ largest elements of $\bx$ in magnitude and truncates the rest to zero.
\end{definition}

\begin{lemma}\label{lemma:max_norm}
For any $\bx \in \mathbb{R}^d-\{ \mathbf{0}\}$, $1 \leq C \leq d$, the optimal mask
\begin{align}
     \bp^* & =  \argmax_{\bp \in \{0, 1 \}^d} \frac{\norm{\bp \odot \bx }^2}{\norm{\bx}^2},~~\text{s.t.}~\norm{\bp}_0 \leq C,  \nonumber
\end{align}
has zeros everywhere except the $C$ largest elements of $\bx$ in magnitude.
\begin{proof}
    Rewriting the optimization problem as 
    \begin{equation}
        \max_{\bp \in \{0, 1 \}^d} \sum_{i=1}^d p_i x_i^2,~~ \text{s.t.}~\sum_{i=1}^d p_i \leq C \nonumber,
    \end{equation}
    Notice that this is a trivial binary knapsack problem with maximum weight capacity $C$ and weights equal to one. Hence, the maximum is attained when we pick the top $C$ maximal $x_i^2$ elements. 
\end{proof}
\end{lemma}

\begin{remark}

\end{remark} It holds that $\text{TOP-}S(\bx) = \bp^* \odot \bx$.


\begin{corollary}
The optimal mask $\bp^*$ in \eqref{eq:maxim} has zeros everywhere except for the indices $i \in \{j: \exists \ell \in \{1, \ldots, G \},~\text{such that}~j \in \{ \pi_{\ell}(1), \ldots,  \pi_{\ell}(c_{\ell})\}    \}$, where $\pi_{\ell}$ is the same permutation as in Definition $\ref{def:tops}$ for the set of indices $I_{\ell}
$.
\end{corollary}
\begin{proof}
    The result follows from the mutual exclusiveness of $I_{\ell}$ in the constraints of \eqref{eq:maxim} and Lemma \ref{lemma:max_norm}.
\end{proof}

\section{Implementation Details}\label{sec_appx:implementation_details}
We describe below the implementation details of section~\ref{sec:experiments}.
\begin{itemize}
    \item All the experiments are conducted on a single NVIDIA A100 GPU.
    \item We have included error bars over three runs for all experiments.
    \item We use PyTorch~\cite{paszke2019pytorch} to implement all the algorithms.
    \item We use Adam~\citep{kingma2014adam} as an optimizer for the methods that utilize the CLIP loss for fine tuning and AdamW~\citep{loshchilov2017decoupled} for those ones that use the perplexity loss. 
    \item A learning rate scheduler of Cosine Annealing with Warmup is employed for all methods.
    \item For all experiments, we set the learning rate $1 \times 10^{-5}$ and $2 \times 10^{-5}$, for LoRSU and LoRSU-Ppl, respectively.
    \item We set batch size to 16 for all methods that fine-tune the vision encoder through CLIP loss. We reduce the batch size to 8 for those methods that fine-tune the vision encoder through perplexity loss or those that fine-tune the LLM. This was due to GPU memory limitations.
    \item All methods run for 20, 15, and 10 epochs for the CL-5, CL-10, and CL-50 settings, respectively.
    \item For LoRA (-Ppl), we set rank $r=64$ while LoRA-L and LoRA-F use $r=8$, for all experiments.
    \item For AdaLoRA, we set the initial rank to 70 and the final average rank to 64.
    \item The adapters of LoRA and AdaLoRA are applied to all weight matrices of each of the transformer blocks.
    \item For SPU, we use sparsity=15\% for all experiments.
    \item For \ours (-Ppl) we use sparsity=10\%, rank=64, and we pick the top-2 attention heads for all experiments.   
\end{itemize}
The choice of the above hyperparameters ensures that LoRA (-Ppl), LoRA-L, LoRA-F, AdaLoRA. SPU, and LoRSU (-Ppl) have similar number of trainable parameters. 

\section{Datasets}\label{sec_appx:datasets}
Details on all datasets used in section~\ref{sec:experiments} are presented here. 

\subsection{TSI \& DALLE}
We start with the description of how we constructed our newly introduced VQA datasets \emph{TSI} and \emph{DALLE}.

\paragraph{TSI.}  To extract  images from the videos of the Toyota Smart Home dataset (TSI), we discretized each video clip into 2 frames per second and then selected the frame in the middle of the total time duration of the video clip. In Table~\ref{table:tsi_class_names} we describe the actions that were selected and the corresponding prompt used for CLIP classification. We also note dropping few actions to avoid ambiguous classes.
\begin{table}
\caption{The original action names of the Toyota Smarthome dataset and their corresponding captions used to create the Toyota Smarthome Images (TSI) dataset. We use~\xmark~to denore the actions that are ambiguous and were not used to build the TSI dataset. The final prompt is created as ``\textit{The person in this image is \{caption\}}''.}
\label{table:tsi_class_names}
\vskip 0.15in
\begin{center}
\begingroup
%\setlength{\tabcolsep}{2.1pt} % Default value: 6pt
%\renewcommand{\arraystretch}{.9}
%\begin{small}
%\begin{sc}
\begin{tabular}{l c }
\toprule
\textbf{Original Class name/Action} & \textbf{Generated Caption}  \\
\midrule
Cook.Cleandishes & washing dishes \\
Cook.Cleanup & cleaning up \\
Cook.Cut & cutting food \\
Cook.Stir & stirring the pot \\
Cook.Usestove & \xmark \\
Cook.Cutbread & cutting bread \\
Drink.Frombottle & holding a bottle \\
Drink.Fromcan & holding a can \\
Drink.Fromcup & holding a cup \\
Drink.Fromglass & holding a glass \\
Eat.Attable & eating \\
Eat.Snack & \xmark \\
Enter & walking \\
Getup & \xmark \\
Laydown & lying down \\
Leave & walking \\
Makecoffee.Pourgrains & using a white coffee machine \\
Makecoffee.Pourwater & using a white coffee machine \\
Maketea.Boilwater & boiling water in a black kettle \\
Maketea.Insertteabag & making tea \\
Pour.Frombottle & holding a bottle \\
Pour.Fromcan & holding a can \\
Pour.Fromkettle & holding a black kettle \\
Readbook & reading a book \\
Sitdown & sitting down \\
Takepills & \xmark \\
Uselaptop & using a laptop \\
Usetablet & using a tablet \\
Usetelephone & using a cordless phone \\
Walk & walking \\
WatchTV & watching TV \\
\bottomrule
\end{tabular}
%\end{sc}
%\end{small}
\endgroup
\end{center}
\vskip -0.1in
\end{table}

\paragraph{DALLE.} We generated images from DALL·E 2 using OpenAI python package and we used the prompt
 ``\textit{A person} $\{a\}$'' where $a \in $ \{ \textit{using a white coffee machine, 
                 eating, 
                 cutting bread, 
                 stirring the pot, 
                 holding a glass, 
                 watching TV, 
                 holding a bottle, 
                 walking, 
                 making tea, 
                 cutting food, 
                 holding a cup, 
                 using a laptop, 
                 lying down, 
                 holding a can, 
                 person holding a black kettle, 
                 reading a book, 
                 cleaning up, 
                 sitting down, 
                 using a tablet, 
                 boiling water in a black kettle, 
                 using a cordless phone, 
                 washing dishes}\}.

In Table~\ref{table:datasets_num_data}, we present the average number of images per session used to update the model for each CL setting. Finally, Table~\ref{table:datasets_num_data_evaluation} provides characteristics of the datasets used for evaluating performance.

\subsection{Continual Learning Splits}
For the continual learning settings of section~\ref{sec:experiments}, we split all datasets into five non-overlapping continual learning (CL) splits based on the classes/categories of each dataset. Unless stated otherwise, we use the training split of each dataset to construct these CL splits.

\paragraph{GTS~\cite{stallkamp2012man}.} We split the 43 classes of GTS as follows:
\begin{itemize}
    \item \emph{Session 1:} $\left[ 25, 2, 11,  1, 40, 27,  5,  9, 17 \right]$.
    \item \emph{Session 2:} $\left[ 32, 29, 20, 39, 21, 15, 23, 10, 3 \right]$.
    \item \emph{Session 3:} $\left[ 18, 38, 42, 14, 22, 35, 34, 19, 33 \right]$.
    \item \emph{Session 4:} $\left[ 12, 26, 41, 0, 37, 6, 13, 24 \right]$.
    \item \emph{Session 5:} $\left[ 30, 28, 31, 7, 16, 4, 36, 8 \right]$.
\end{itemize}

\paragraph{TSI~\cite{das2019toyota}.} We split the 27 action categories of TSI as follows:
\begin{itemize}
    \item \emph{Session 1:} [\textit{WatchTV}, \textit{Laydown}, \textit{Sitdown}, \textit{Pour.Fromkettle}, \textit{Enter}, \textit{Drink.Frombottle}].
    \item \emph{Session 2:} [\textit{Eat.Attable}, \textit{Pour.Frombottle}, \textit{Cook.Cleandishes}, \textit{Maketea.Boilwater}, \textit{Leave}, \textit{Cook.Cleanup}].
    \item \emph{Session 3:} [\textit{Maketea.Insertteabag}, \textit{Makecoffee.Pourwater}, \textit{Drink.Fromcan}, \textit{Readbook}, \textit{Cutbread}].
    \item \emph{Session 4:} [\textit{Drink.Fromcup}, \textit{Drink.Fromglass}, \textit{Usetablet}, \textit{Pour.Fromcan}, \textit{Usetelephone}].
    \item \emph{Session 5:} [\textit{Walk}, \textit{Cook.Stir}, \textit{Makecoffee.Pourgrains}, \textit{Cook.Cut}, \textit{Uselaptop}].
\end{itemize}

\paragraph{CAn~\cite{wang2024clips}.} The 45 classes of CAn are split as follows:
\begin{itemize}
    \item \emph{Session 1:} $\left[ 102, 9, 20, 56, 23, 30, 357, 291, 144 \right]$.
    \item \emph{Session 2:} $\left[ 41, 293, 42, 49, 54, 57, 70, 279, 305 \right]$.
    \item \emph{Session 3:} $\left[ 71, 10, 76, 79, 349, 16, 81, 83, 100 \right]$.
    \item \emph{Session 4:} $\left[ 130, 30, 133, 150, 275, 276, 58, 277, 80 \right]$.
    \item \emph{Session 5:} $\left[ 39, 290, 37, 296, 316, 337, 89, 360, 128 \right]$.
\end{itemize}
The indices of CAn correspond to those of ImageNet~\cite{imagenet} since the dataset was built based on these 45 animal classes of ImageNet.

\paragraph{AIR~\cite{maji13fine-grained}.} We split the 100 aircraft types of AIR as follows:
\begin{itemize}
    \item \emph{Session 1:} $\left[ 23, 8, 11, 7, 48, 13, 1, 91, 94, 54, 16, 63, 52, 41, 80, 2, 47, 87, 78, 66 \right]$.
    \item \emph{Session 2:} $\left[ 19, 6, 24, 10, 59, 30, 22, 29, 83, 37, 93, 81, 43, 99, 86, 28, 34, 88, 44, 14 \right]$.
    \item \emph{Session 3:} $\left[ 84, 70, 4, 20, 15, 21, 31, 76, 57, 67, 73, 50, 69, 25, 98, 46, 96, 0, 72, 35 \right]$.
    \item \emph{Session 4:} $\left[ 58, 92, 3, 95, 56, 90, 26, 40, 55, 89, 75, 71, 60, 42, 9, 82, 39, 18, 77, 68 \right]$.
    \item \emph{Session 5:} $\left[ 32, 79, 12, 85, 36, 17, 64, 27, 74, 45, 61, 38, 51, 62, 65, 33, 5, 53, 97, 49 \right]$.
\end{itemize}

\paragraph{ESAT~\cite{helber2019eurosat}.} We split the 10 different land terrain classes of ESAT as follows:
\begin{itemize}
    \item \emph{Session 1:} $\left[ 0, 1 \right]$.
    \item \emph{Session 2:} $\left[ 2, 3 \right]$.
    \item \emph{Session 3:} $\left[ 4, 5 \right]$.
    \item \emph{Session 4:} $\left[ 6, 7 \right]$.
    \item \emph{Session 5:} $\left[ 8, 9 \right]$.
\end{itemize}

\paragraph{DALLE.} This dataset was only used for performance evaluation (control dataset), and not fine-tuning.

\paragraph{VSR~\cite{Liu2022VisualSR}.} The images of this VQA dataset are labeled according to 36 different categories that describe the dominant object of the image. We create the CL splits as follows:
\begin{itemize}
    \item \emph{Session 1:} [\textit{oven}, \textit{dining table}, \textit{spoon}, \textit{boat}, \textit{cake}, \textit{donut}, \textit{sandwich}].
    \item \emph{Session 2:} [\textit{fire hydrant}, \textit{elephant}, \textit{airplane}, \textit{truck}, \textit{apple}, \textit{hot dog}, \textit{sheep}].
    \item \emph{Session 3:} [\textit{kite}, \textit{baseball glove}, \textit{cow}, \textit{tie}, \textit{scissors}, \textit{toaster}, \textit{tv}].
    \item \emph{Session 4:} [\textit{bicycle}, \textit{banana}, \textit{couch}, \textit{teddy bear}, \textit{bus}, \textit{umbrella}, \textit{bird}].
    \item \emph{Session 5:} [\textit{potted plant}, \textit{bowl}, \textit{broccoli}, \textit{bottle}, \textit{knife}, \textit{orange}, \textit{person}, \textit{pizza}].
\end{itemize}

\paragraph{HM~\cite{kiela2020hateful}.} For the hateful memes dataset, since there was not any labeling information of the images so we can spli the images in a meaningful way, we randomly split the training images into five disjoint sets to create our final CL splits.

\paragraph{MMVP~\cite{tong2024eyes}.} This is the only dataset where no training split is available and it is comprised of just 300 images. For this reason, we only used it for evaluation in our experiments in the main paper. However, for completeness, we included results in Table~\ref{table:fine_tune_llm_mmvp} where we fine-tune on it. We use 150 images for training which are equally split into five sessions and the rest of the 150 images are used for evaluation. Thus, the setting can be considered as a 30-shot CL setting. 

\paragraph{VisOnly~\cite{kamoi2024visonlyqa}.} This dataset categorizes its samples into seven categories describing the nature of the geometric and numerical information in scientific figures. We created the splits as follows:
\begin{itemize}
    \item \emph{Session 1:} \textit{Geometry-Triangle}.
    \item \emph{Session 2:} \textit{Geometry-Quadrilateral}.
    \item \emph{Session 3:} \textit{Geometry-Length}
    \item \emph{Session 4:} \textit{Geometry-Angle}.
    \item \emph{Session 5:} [\textit{Geometry-Area}, \textit{3D-Size}, \textit{3D-Angle}].
\end{itemize}

\begin{table}
\caption{Average number of images per session (5 sessions in total) for each dataset used for fine-tuning.}
 \label{table:datasets_num_data}
\vskip 0.15in
\begin{center}
%\begin{small}
\begingroup
\setlength{\tabcolsep}{9.7pt}
\begin{tabular}{l c c c c c c c c c}
\toprule
 & \multicolumn{8}{c}{\textbf{FT Dataset}}  \\
\cmidrule(lr){2-9}
\textbf{Setting} & \textbf{GTS} & \textbf{TSI} & \textbf{CAn} & \textbf{AIR} & \textbf{ESAT}  & \textbf{VSR} & \textbf{HM} & \textbf{VisOnly} \\
\midrule
\textbf{CL-5} & $43.0$ & $27.0$ & $45.0$ & $100.0$ & $10.0$ & $100.0$ & $100.0$ & $7.0$ \\
\midrule
 \textbf{CL-20} & $170.0$ & $84.0$ & $180.0$ & $400.0$ & $40.0$ & $274.6$ & $300.0$ & $28.0$ \\
\midrule
 \textbf{CL-50} & $430.0$ & $253.8$ & $450.0$ & $1000.0$ & $100.0$ & $485.2$ & $600.0$ & $70.0$ \\
\bottomrule
\end{tabular}
\endgroup
%\end{small}
\end{center}
\end{table}

\begin{table}
\caption{Characteristics of the datasets used for performance  evaluation in section~\ref{sec:experiments}.}
 \label{table:datasets_num_data_evaluation}
\vskip 0.15in
\begin{center}
\begingroup
\setlength{\tabcolsep}{6.7pt}
\begin{tabular}{l c c c c c c c c c c c c c c}
\toprule
\textbf{Eval Datasets} & \textbf{GTS} & 
 \textbf{TSI} & \textbf{CAn} & \textbf{AIR} & \textbf{ESAT} & \textbf{DALLE} & \textbf{VSR} & \textbf{HM} & \textbf{MMVP} & \textbf{VisOnly} \\
\midrule
\textbf{\# Samples} & $3,990$ & $4,908$ & $1,796$ & $3,333$ & $17,000$ & $660$ & $1,222$ & $2,000$ & $150$ & $1,150$ \\
\textbf{\# Classes} & $43$ & $27$ & $45$ & $100$ & $10$ & $27$ & $36$ & NaN & NaN & $7$ \\
\bottomrule
\end{tabular}
\endgroup
\end{center}
\vskip -0.1in
\end{table}

\section{Detailed Results}\label{sec_appx:detailed_res}

\subsection{CLIP-based Updates+}
The detailed accuracies for all baselines and datasets used to create Table~\ref{table:clip_baselines_summary} of the main paper can be found in Tables~\ref{table:vlm_vqa_acc_gtsrb_clip} through~\ref{table:vlm_vqa_acc_eurosat_clip}.
\begin{table}
\caption{Accuracy scores (\%) for LLaVA with the pretrained (\emph{Zr-Shot}) or fine-tuned image encoder. All baselines use \emph{GTS} dataset for fine-tuning the image encoder~(the LLM remains frozen) via CLIP loss. We include error bars over 3 runs.}
 \label{table:vlm_vqa_acc_gtsrb_clip}
\vskip 0.15in
\begin{center}
\begin{small}
\begingroup
\setlength{\tabcolsep}{3.9pt}
\begin{tabular}{l c c c c c c c c c c c}
\toprule
 & & \multicolumn{9}{c}{\textbf{VQA Datasets (Acc \%)}}  \\
\cmidrule(lr){3-12}
\textbf{Setting} & \textbf{FT Method}  & \textbf{GTS} & \textbf{TSI} & \textbf{CAn} & \textbf{AIR} & \textbf{ESAT} & \textbf{DALLE} & \textbf{VSR} & \textbf{HM} & \textbf{MMVP} & \textbf{VisOnly} \\
\midrule
 & \textbf{Zr-Shot} & $75.6$ & $53.1$ & $82.7$ & $60.4$ & $76.1$ & $91.1$ & $51.5$ & $61.2$ & $58.0$ & $31.3$ \\
\midrule
\multirow{7}{*}{\textbf{CL-5}} & \textbf{LN} & $79.1 \mtiny{\pm 1.2}$ & $53.6 \mtiny{\pm 0.5}$ & $81.2 \mtiny{\pm 0.6}$ & $61.0 \mtiny{\pm 1.2}$ & $58.9 \mtiny{\pm 0.9}$ & $91.1 \mtiny{\pm 1.3}$ & $51.9 \mtiny{\pm 1.5}$ & $62.7 \mtiny{\pm 1.1}$ & $59.6 \mtiny{\pm 0.2}$ & $31.8 \mtiny{\pm 0.4}$ \\
& \textbf{F-FT} & $79.3 \mtiny{\pm 0.6}$ & $55.1 \mtiny{\pm 0.8}$ & $76.8 \mtiny{\pm 1.3}$ & $58.8 \mtiny{\pm 1.0}$ & $25.6 \mtiny{\pm 0.9}$ & $89.2 \mtiny{\pm 1.2}$ & $51.7 \mtiny{\pm 0.9}$ & $62.1 \mtiny{\pm 0.8}$ & $56.4 \mtiny{\pm 0.4}$ & $30.9 \mtiny{\pm 0.2}$ \\
& \textbf{F-EWC} & $80.6 \mtiny{\pm 0.6}$ & $37.4 \mtiny{\pm 1.3}$ & $63.2 \mtiny{\pm 0.7}$ & $55.8 \mtiny{\pm 1.4}$ & $26.1 \mtiny{\pm 1.4}$ & $81.5 \mtiny{\pm 1.1}$ & $51.8 \mtiny{\pm 1.4}$ & $61.2 \mtiny{\pm 0.6}$ & $53.8 \mtiny{\pm 0.4}$ & $31.2 \mtiny{\pm 0.4}$ \\
& \textbf{LoRA} & $76.3 \mtiny{\pm 0.8}$ & $52.6 \mtiny{\pm 1.4}$ & $73.3 \mtiny{\pm 0.6}$ & $56.7 \mtiny{\pm 1.2}$ & $49.3 \mtiny{\pm 0.8}$ & $87.1 \mtiny{\pm 1.3}$ & $51.8 \mtiny{\pm 1.2}$ & $61.3 \mtiny{\pm 1.2}$ & $58.1 \mtiny{\pm 0.3}$ & $31.6 \mtiny{\pm 0.4}$ \\
& \textbf{AdaLoRA} & $74.7 \mtiny{\pm 0.9}$ & $49.7 \mtiny{\pm 0.7}$ & $79.6 \mtiny{\pm 0.9}$ & $56.3 \mtiny{\pm 0.8}$ & $42.5 \mtiny{\pm 0.8}$ & $91.6 \mtiny{\pm 1.1}$ & $52.0 \mtiny{\pm 0.8}$ & $60.9 \mtiny{\pm 1.2}$ & $57.1 \mtiny{\pm 0.3}$ & $31.7 \mtiny{\pm 0.2}$ \\
& \textbf{SPU} & $81.0 \mtiny{\pm 1.4}$ & $53.7 \mtiny{\pm 1.5}$ & $82.5 \mtiny{\pm 0.7}$ & $61.0 \mtiny{\pm 1.0}$ & $67.8 \mtiny{\pm 0.6}$ & $91.6 \mtiny{\pm 1.3}$ & $52.0 \mtiny{\pm 0.6}$ & $62.0 \mtiny{\pm 1.3}$ & $58.2 \mtiny{\pm 0.2}$ & $31.6 \mtiny{\pm 0.2}$ \\
& \textbf{LoRSU} & $82.0 \mtiny{\pm 1.3}$ & $53.5 \mtiny{\pm 1.3}$ & $82.4 \mtiny{\pm 0.8}$ & $60.8 \mtiny{\pm 1.4}$ & $66.6 \mtiny{\pm 0.9}$ & $91.5 \mtiny{\pm 1.4}$ & $51.6 \mtiny{\pm 0.7}$ & $61.7 \mtiny{\pm 1.4}$ & $59.8 \mtiny{\pm 0.2}$ & $31.6 \mtiny{\pm 0.2}$ \\
\midrule
\multirow{7}{*}{\textbf{CL-20}} & \textbf{LN} & $80.8 \mtiny{\pm 0.6}$ & $49.5 \mtiny{\pm 0.7}$ & $77.7 \mtiny{\pm 1.0}$ & $59.7 \mtiny{\pm 0.5}$ & $32.7 \mtiny{\pm 0.6}$ & $89.8 \mtiny{\pm 0.9}$ & $51.8 \mtiny{\pm 0.7}$ & $62.3 \mtiny{\pm 0.3}$ & $57.5 \mtiny{\pm 0.1}$ & $31.2 \mtiny{\pm 0.2}$ \\
& \textbf{F-FT} & $80.2 \mtiny{\pm 0.8}$ & $54.5 \mtiny{\pm 0.7}$ & $74.9 \mtiny{\pm 0.8}$ & $57.2 \mtiny{\pm 1.0}$ & $23.2 \mtiny{\pm 0.7}$ & $86.7 \mtiny{\pm 0.4}$ & $51.9 \mtiny{\pm 0.9}$ & $61.6 \mtiny{\pm 1.0}$ & $58.3 \mtiny{\pm 0.2}$ & $31.7 \mtiny{\pm 0.3}$ \\
& \textbf{F-EWC} & $82.3 \mtiny{\pm 0.9}$ & $35.5 \mtiny{\pm 0.9}$ & $55.7 \mtiny{\pm 0.4}$ & $35.4 \mtiny{\pm 0.3}$ & $28.7 \mtiny{\pm 0.9}$ & $72.4 \mtiny{\pm 0.8}$ & $51.6 \mtiny{\pm 0.7}$ & $60.9 \mtiny{\pm 0.8}$ & $53.5 \mtiny{\pm 0.2}$ & $31.0 \mtiny{\pm 0.3}$ \\
& \textbf{LoRA} & $78.1 \mtiny{\pm 0.8}$ & $55.6 \mtiny{\pm 0.3}$ & $59.0 \mtiny{\pm 0.9}$ & $47.6 \mtiny{\pm 0.4}$ & $26.0 \mtiny{\pm 0.6}$ & $83.6 \mtiny{\pm 0.8}$ & $52.1 \mtiny{\pm 0.5}$ & $62.1 \mtiny{\pm 1.0}$ & $53.7 \mtiny{\pm 0.3}$ & $30.8 \mtiny{\pm 0.2}$ \\
& \textbf{AdaLoRA} & $75.8 \mtiny{\pm 0.8}$ & $51.9 \mtiny{\pm 0.5}$ & $79.3 \mtiny{\pm 0.9}$ & $59.3 \mtiny{\pm 0.4}$ & $62.1 \mtiny{\pm 0.4}$ & $90.7 \mtiny{\pm 1.0}$ & $51.6 \mtiny{\pm 0.5}$ & $61.1 \mtiny{\pm 0.6}$ & $57.7 \mtiny{\pm 0.2}$ & $31.7 \mtiny{\pm 0.2}$ \\
& \textbf{SPU} & $83.5 \mtiny{\pm 0.6}$ & $53.1 \mtiny{\pm 0.6}$ & $82.2 \mtiny{\pm 0.7}$ & $60.7 \mtiny{\pm 0.8}$ & $62.0 \mtiny{\pm 0.4}$ & $91.5 \mtiny{\pm 0.4}$ & $51.9 \mtiny{\pm 0.5}$ & $61.8 \mtiny{\pm 0.7}$ & $58.8 \mtiny{\pm 0.2}$ & $31.5 \mtiny{\pm 0.2}$ \\
& \textbf{LoRSU} & $84.2 \mtiny{\pm 0.9}$ & $52.9 \mtiny{\pm 0.6}$ & $82.2 \mtiny{\pm 0.5}$ & $60.7 \mtiny{\pm 0.6}$ & $64.7 \mtiny{\pm 0.6}$ & $90.8 \mtiny{\pm 0.5}$ & $51.9 \mtiny{\pm 0.4}$ & $61.7 \mtiny{\pm 0.5}$ & $59.5 \mtiny{\pm 0.1}$ & $31.6 \mtiny{\pm 0.2}$ \\
\midrule
\multirow{7}{*}{\textbf{CL-50}} & \textbf{LN} & $80.4 \mtiny{\pm 0.2}$ & $50.4 \mtiny{\pm 0.1}$ & $74.9 \mtiny{\pm 0.1}$ & $58.3 \mtiny{\pm 0.0}$ & $30.4 \mtiny{\pm 0.3}$ & $89.0 \mtiny{\pm 0.1}$ & $51.8 \mtiny{\pm 0.0}$ & $62.0 \mtiny{\pm 0.3}$ & $58.7 \mtiny{\pm 0.1}$ & $31.4 \mtiny{\pm 0.1}$ \\
& \textbf{F-FT} & $79.0 \mtiny{\pm 0.1}$ & $48.9 \mtiny{\pm 0.2}$ & $65.0 \mtiny{\pm 0.2}$ & $55.0 \mtiny{\pm 0.3}$ & $23.5 \mtiny{\pm 0.0}$ & $86.8 \mtiny{\pm 0.2}$ & $52.0 \mtiny{\pm 0.1}$ & $60.8 \mtiny{\pm 0.1}$ & $54.9 \mtiny{\pm 0.1}$ & $30.7 \mtiny{\pm 0.1}$ \\
& \textbf{F-EWC} & $80.9 \mtiny{\pm 0.2}$ & $45.2 \mtiny{\pm 0.4}$ & $60.5 \mtiny{\pm 0.4}$ & $43.2 \mtiny{\pm 0.0}$ & $26.9 \mtiny{\pm 0.3}$ & $78.5 \mtiny{\pm 0.1}$ & $52.0 \mtiny{\pm 0.0}$ & $58.7 \mtiny{\pm 0.1}$ & $52.9 \mtiny{\pm 0.0}$ & $31.7 \mtiny{\pm 0.1}$ \\
& \textbf{LoRA} & $78.7 \mtiny{\pm 0.0}$ & $50.7 \mtiny{\pm 0.0}$ & $62.1 \mtiny{\pm 0.2}$ & $47.4 \mtiny{\pm 0.1}$ & $24.2 \mtiny{\pm 0.2}$ & $82.9 \mtiny{\pm 0.3}$ & $51.7 \mtiny{\pm 0.3}$ & $61.0 \mtiny{\pm 0.2}$ & $54.3 \mtiny{\pm 0.1}$ & $30.8 \mtiny{\pm 0.0}$ \\
& \textbf{AdaLoRA} & $76.6 \mtiny{\pm 0.4}$ & $50.4 \mtiny{\pm 0.0}$ & $79.0 \mtiny{\pm 0.2}$ & $57.4 \mtiny{\pm 0.1}$ & $58.3 \mtiny{\pm 0.1}$ & $90.4 \mtiny{\pm 0.2}$ & $51.6 \mtiny{\pm 0.2}$ & $61.8 \mtiny{\pm 0.3}$ & $55.4 \mtiny{\pm 0.1}$ & $31.8 \mtiny{\pm 0.1}$ \\
& \textbf{SPU} & $83.3 \mtiny{\pm 0.3}$ & $53.8 \mtiny{\pm 0.2}$ & $81.8 \mtiny{\pm 0.2}$ & $61.1 \mtiny{\pm 0.4}$ & $58.8 \mtiny{\pm 0.0}$ & $91.0 \mtiny{\pm 0.2}$ & $51.8 \mtiny{\pm 0.4}$ & $62.1 \mtiny{\pm 0.1}$ & $59.5 \mtiny{\pm 0.1}$ & $32.2 \mtiny{\pm 0.1}$ \\
& \textbf{LoRSU} & $85.3 \mtiny{\pm 0.1}$ & $54.2 \mtiny{\pm 0.1}$ & $81.9 \mtiny{\pm 0.2}$ & $60.5 \mtiny{\pm 0.2}$ & $61.4 \mtiny{\pm 0.3}$ & $91.0 \mtiny{\pm 0.1}$ & $51.7 \mtiny{\pm 0.2}$ & $62.2 \mtiny{\pm 0.4}$ & $58.9 \mtiny{\pm 0.1}$ & $31.8 \mtiny{\pm 0.1}$ \\
\bottomrule
\end{tabular}
\endgroup
\end{small}
\end{center}
\vskip -0.1in
\end{table}


\begin{table}
\caption{Accuracy scores (\%) for LLaVA with the pretrained (\emph{Zr-Shot}) or fine-tuned image encoder. All baselines use \emph{TSI} dataset for fine-tuning the image encoder~(the LLM remains frozen) via CLIP loss. We include error bars over 3 runs.}
 \label{table:vlm_vqa_acc_tsi_clip}
\vskip 0.15in
\begin{center}
\begin{small}
\begingroup
\setlength{\tabcolsep}{3.9pt}
\begin{tabular}{l c c c c c c c c c c c}
\toprule
 & & \multicolumn{9}{c}{\textbf{VQA Datasets (Acc \%)}}  \\
\cmidrule(lr){3-12}
\textbf{Setting} & \textbf{FT Method}  & \textbf{GTS} & \textbf{TSI} & \textbf{CAn} & \textbf{AIR} & \textbf{ESAT} & \textbf{DALLE} & \textbf{VSR} & \textbf{HM} & \textbf{MMVP} & \textbf{VisOnly} \\
\midrule
 & \textbf{Zr-Shot} & $75.6$ & $53.1$ & $82.7$ & $60.4$ & $76.1$ & $91.1$ & $51.5$ & $61.2$ & $58.0$ & $31.3$ \\
\midrule
\multirow{7}{*}{\textbf{CL-5}} & \textbf{LN} & $75.4 \mtiny{\pm 1.0}$ & $53.9 \mtiny{\pm 0.6}$ & $82.6 \mtiny{\pm 1.3}$ & $60.0 \mtiny{\pm 1.0}$ & $75.9 \mtiny{\pm 0.8}$ & $91.1 \mtiny{\pm 1.3}$ & $51.7 \mtiny{\pm 1.4}$ & $61.9 \mtiny{\pm 1.0}$ & $58.4 \mtiny{\pm 0.3}$ & $30.9 \mtiny{\pm 0.3}$ \\
& \textbf{F-FT} & $73.8 \mtiny{\pm 0.5}$ & $60.5 \mtiny{\pm 1.1}$ & $81.6 \mtiny{\pm 0.9}$ & $59.5 \mtiny{\pm 1.5}$ & $70.4 \mtiny{\pm 1.0}$ & $91.1 \mtiny{\pm 1.2}$ & $51.8 \mtiny{\pm 0.9}$ & $61.5 \mtiny{\pm 1.3}$ & $56.9 \mtiny{\pm 0.2}$ & $31.3 \mtiny{\pm 0.3}$ \\
& \textbf{F-EWC} & $74.9 \mtiny{\pm 1.1}$ & $61.6 \mtiny{\pm 1.0}$ & $82.1 \mtiny{\pm 1.1}$ & $58.8 \mtiny{\pm 0.9}$ & $72.3 \mtiny{\pm 1.2}$ & $89.9 \mtiny{\pm 1.4}$ & $51.9 \mtiny{\pm 0.9}$ & $62.4 \mtiny{\pm 1.4}$ & $55.5 \mtiny{\pm 0.4}$ & $31.5 \mtiny{\pm 0.3}$ \\
& \textbf{LoRA} & $73.4 \mtiny{\pm 1.0}$ & $53.0 \mtiny{\pm 0.9}$ & $80.2 \mtiny{\pm 0.6}$ & $58.8 \mtiny{\pm 0.7}$ & $59.1 \mtiny{\pm 1.4}$ & $90.2 \mtiny{\pm 1.1}$ & $51.6 \mtiny{\pm 1.3}$ & $61.2 \mtiny{\pm 1.4}$ & $56.7 \mtiny{\pm 0.4}$ & $31.7 \mtiny{\pm 0.4}$ \\
& \textbf{AdaLoRA} & $75.6 \mtiny{\pm 0.8}$ & $54.2 \mtiny{\pm 0.6}$ & $82.6 \mtiny{\pm 1.1}$ & $60.0 \mtiny{\pm 1.3}$ & $75.7 \mtiny{\pm 1.3}$ & $91.1 \mtiny{\pm 1.2}$ & $51.6 \mtiny{\pm 0.9}$ & $62.1 \mtiny{\pm 1.0}$ & $59.5 \mtiny{\pm 0.3}$ & $31.7 \mtiny{\pm 0.2}$ \\
& \textbf{SPU} & $75.4 \mtiny{\pm 0.7}$ & $54.0 \mtiny{\pm 1.1}$ & $83.0 \mtiny{\pm 1.3}$ & $60.1 \mtiny{\pm 0.6}$ & $75.7 \mtiny{\pm 1.5}$ & $91.3 \mtiny{\pm 1.3}$ & $51.9 \mtiny{\pm 1.4}$ & $61.7 \mtiny{\pm 0.9}$ & $58.5 \mtiny{\pm 0.4}$ & $31.6 \mtiny{\pm 0.4}$ \\
& \textbf{LoRSU} & $75.9 \mtiny{\pm 0.9}$ & $56.3 \mtiny{\pm 0.7}$ & $82.7 \mtiny{\pm 0.9}$ & $60.8 \mtiny{\pm 1.0}$ & $76.2 \mtiny{\pm 1.4}$ & $91.3 \mtiny{\pm 1.2}$ & $51.6 \mtiny{\pm 0.9}$ & $61.7 \mtiny{\pm 0.8}$ & $57.7 \mtiny{\pm 0.3}$ & $31.2 \mtiny{\pm 0.3}$ \\
\midrule
\multirow{7}{*}{\textbf{CL-20}} & \textbf{LN} & $72.9 \mtiny{\pm 0.5}$ & $58.2 \mtiny{\pm 0.5}$ & $78.9 \mtiny{\pm 0.9}$ & $56.8 \mtiny{\pm 0.4}$ & $69.3 \mtiny{\pm 0.9}$ & $91.4 \mtiny{\pm 0.8}$ & $51.6 \mtiny{\pm 0.8}$ & $62.6 \mtiny{\pm 0.5}$ & $56.3 \mtiny{\pm 0.3}$ & $31.3 \mtiny{\pm 0.2}$ \\
& \textbf{F-FT} & $72.1 \mtiny{\pm 0.7}$ & $68.4 \mtiny{\pm 0.4}$ & $80.0 \mtiny{\pm 0.7}$ & $55.4 \mtiny{\pm 0.4}$ & $58.8 \mtiny{\pm 0.8}$ & $88.4 \mtiny{\pm 0.3}$ & $51.8 \mtiny{\pm 0.6}$ & $62.3 \mtiny{\pm 0.5}$ & $56.9 \mtiny{\pm 0.2}$ & $31.2 \mtiny{\pm 0.3}$ \\
& \textbf{F-EWC} & $23.3 \mtiny{\pm 0.6}$ & $69.1 \mtiny{\pm 0.6}$ & $20.4 \mtiny{\pm 0.7}$ & $20.1 \mtiny{\pm 0.6}$ & $24.2 \mtiny{\pm 0.6}$ & $17.7 \mtiny{\pm 0.7}$ & $51.7 \mtiny{\pm 0.7}$ & $56.9 \mtiny{\pm 0.8}$ & $49.6 \mtiny{\pm 0.3}$ & $31.1 \mtiny{\pm 0.1}$ \\
& \textbf{LoRA} & $68.5 \mtiny{\pm 0.7}$ & $61.6 \mtiny{\pm 0.3}$ & $76.7 \mtiny{\pm 0.9}$ & $55.3 \mtiny{\pm 0.7}$ & $55.6 \mtiny{\pm 0.6}$ & $88.8 \mtiny{\pm 0.8}$ & $51.9 \mtiny{\pm 0.3}$ & $61.4 \mtiny{\pm 0.6}$ & $59.1 \mtiny{\pm 0.3}$ & $31.1 \mtiny{\pm 0.3}$ \\
& \textbf{AdaLoRA} & $70.3 \mtiny{\pm 0.5}$ & $54.4 \mtiny{\pm 0.4}$ & $72.4 \mtiny{\pm 0.5}$ & $43.6 \mtiny{\pm 0.8}$ & $34.6 \mtiny{\pm 0.7}$ & $77.0 \mtiny{\pm 0.3}$ & $52.2 \mtiny{\pm 0.9}$ & $62.6 \mtiny{\pm 0.4}$ & $57.0 \mtiny{\pm 0.1}$ & $31.9 \mtiny{\pm 0.3}$ \\
& \textbf{SPU} & $75.5 \mtiny{\pm 0.7}$ & $60.9 \mtiny{\pm 0.8}$ & $82.3 \mtiny{\pm 0.4}$ & $59.2 \mtiny{\pm 0.5}$ & $73.7 \mtiny{\pm 1.0}$ & $91.2 \mtiny{\pm 0.7}$ & $51.7 \mtiny{\pm 0.8}$ & $61.8 \mtiny{\pm 0.9}$ & $58.2 \mtiny{\pm 0.3}$ & $32.0 \mtiny{\pm 0.2}$ \\
& \textbf{LoRSU} & $75.9 \mtiny{\pm 0.6}$ & $63.7 \mtiny{\pm 0.4}$ & $82.8 \mtiny{\pm 0.8}$ & $60.4 \mtiny{\pm 0.3}$ & $73.4 \mtiny{\pm 0.6}$ & $90.9 \mtiny{\pm 0.6}$ & $51.7 \mtiny{\pm 0.4}$ & $61.5 \mtiny{\pm 0.7}$ & $58.8 \mtiny{\pm 0.2}$ & $31.9 \mtiny{\pm 0.2}$ \\
\midrule
\multirow{7}{*}{\textbf{CL-50}} & \textbf{LN} & $73.0 \mtiny{\pm 0.2}$ & $60.1 \mtiny{\pm 0.2}$ & $79.6 \mtiny{\pm 0.3}$ & $57.7 \mtiny{\pm 0.4}$ & $61.3 \mtiny{\pm 0.4}$ & $89.6 \mtiny{\pm 0.4}$ & $51.9 \mtiny{\pm 0.0}$ & $61.3 \mtiny{\pm 0.0}$ & $55.5 \mtiny{\pm 0.1}$ & $31.3 \mtiny{\pm 0.1}$ \\
& \textbf{F-FT} & $72.5 \mtiny{\pm 0.4}$ & $70.3 \mtiny{\pm 0.1}$ & $78.3 \mtiny{\pm 0.4}$ & $53.4 \mtiny{\pm 0.0}$ & $50.6 \mtiny{\pm 0.2}$ & $89.1 \mtiny{\pm 0.3}$ & $52.3 \mtiny{\pm 0.3}$ & $61.1 \mtiny{\pm 0.2}$ & $57.1 \mtiny{\pm 0.1}$ & $31.7 \mtiny{\pm 0.0}$ \\
& \textbf{F-EWC} & $48.0 \mtiny{\pm 0.3}$ & $75.5 \mtiny{\pm 0.2}$ & $59.5 \mtiny{\pm 0.4}$ & $38.8 \mtiny{\pm 0.1}$ & $42.6 \mtiny{\pm 0.3}$ & $82.5 \mtiny{\pm 0.0}$ & $52.5 \mtiny{\pm 0.1}$ & $56.4 \mtiny{\pm 0.3}$ & $55.4 \mtiny{\pm 0.1}$ & $31.3 \mtiny{\pm 0.1}$ \\
& \textbf{LoRA} & $66.1 \mtiny{\pm 0.2}$ & $71.3 \mtiny{\pm 0.3}$ & $76.0 \mtiny{\pm 0.1}$ & $56.0 \mtiny{\pm 0.1}$ & $44.5 \mtiny{\pm 0.2}$ & $88.9 \mtiny{\pm 0.3}$ & $51.8 \mtiny{\pm 0.1}$ & $60.4 \mtiny{\pm 0.2}$ & $56.3 \mtiny{\pm 0.1}$ & $31.6 \mtiny{\pm 0.1}$ \\
& \textbf{AdaLoRA} & $73.1 \mtiny{\pm 0.2}$ & $61.0 \mtiny{\pm 0.0}$ & $80.6 \mtiny{\pm 0.0}$ & $52.0 \mtiny{\pm 0.4}$ & $72.2 \mtiny{\pm 0.3}$ & $88.9 \mtiny{\pm 0.3}$ & $51.7 \mtiny{\pm 0.2}$ & $62.0 \mtiny{\pm 0.4}$ & $59.1 \mtiny{\pm 0.0}$ & $31.2 \mtiny{\pm 0.1}$ \\
& \textbf{SPU} & $75.4 \mtiny{\pm 0.0}$ & $65.3 \mtiny{\pm 0.1}$ & $81.8 \mtiny{\pm 0.1}$ & $59.7 \mtiny{\pm 0.2}$ & $72.3 \mtiny{\pm 0.1}$ & $90.8 \mtiny{\pm 0.2}$ & $51.9 \mtiny{\pm 0.1}$ & $61.9 \mtiny{\pm 0.4}$ & $58.0 \mtiny{\pm 0.1}$ & $31.8 \mtiny{\pm 0.0}$ \\
& \textbf{LoRSU} & $75.3 \mtiny{\pm 0.2}$ & $72.2 \mtiny{\pm 0.4}$ & $82.4 \mtiny{\pm 0.3}$ & $59.7 \mtiny{\pm 0.3}$ & $72.5 \mtiny{\pm 0.3}$ & $90.8 \mtiny{\pm 0.3}$ & $51.7 \mtiny{\pm 0.2}$ & $61.7 \mtiny{\pm 0.4}$ & $58.5 \mtiny{\pm 0.1}$ & $31.7 \mtiny{\pm 0.0}$ \\
\bottomrule
\end{tabular}
\endgroup
\end{small}
\end{center}
\vskip -0.1in
\end{table}


\begin{table}
\caption{Accuracy scores (\%) for LLaVA with the pretrained (\emph{Zr-Shot}) or fine-tuned image encoder. All baselines use \emph{CAn} dataset for fine-tuning the image encoder~(the LLM remains frozen) via CLIP loss. We include error bars over 3 runs.}
 \label{table:vlm_vqa_acc_counteranimal_clip}
\vskip 0.15in
\begin{center}
\begin{small}
\begingroup
\setlength{\tabcolsep}{3.9pt}
\begin{tabular}{l c c c c c c c c c c c}
\toprule
 & & \multicolumn{9}{c}{\textbf{VQA Datasets (Acc \%)}}  \\
\cmidrule(lr){3-12}
\textbf{Setting} & \textbf{FT Method}  & \textbf{GTS} & \textbf{TSI} & \textbf{CAn} & \textbf{AIR} & \textbf{ESAT} & \textbf{DALLE} & \textbf{VSR} & \textbf{HM} & \textbf{MMVP} & \textbf{VisOnly} \\
\midrule
 & \textbf{Zr-Shot} & $75.6$ & $53.1$ & $82.7$ & $60.4$ & $76.1$ & $91.1$ & $51.5$ & $61.2$ & $58.0$ & $31.3$ \\
\midrule
\multirow{7}{*}{\textbf{CL-5}} & \textbf{LN} & $74.3 \mtiny{\pm 1.5}$ & $52.9 \mtiny{\pm 1.4}$ & $80.3 \mtiny{\pm 1.4}$ & $58.9 \mtiny{\pm 0.7}$ & $72.4 \mtiny{\pm 1.2}$ & $91.1 \mtiny{\pm 0.8}$ & $52.0 \mtiny{\pm 0.9}$ & $61.5 \mtiny{\pm 1.2}$ & $61.7 \mtiny{\pm 0.3}$ & $32.1 \mtiny{\pm 0.4}$ \\
& \textbf{F-FT} & $73.5 \mtiny{\pm 1.1}$ & $50.6 \mtiny{\pm 0.9}$ & $80.3 \mtiny{\pm 0.8}$ & $56.5 \mtiny{\pm 0.6}$ & $63.1 \mtiny{\pm 0.6}$ & $91.3 \mtiny{\pm 1.5}$ & $51.7 \mtiny{\pm 1.4}$ & $61.8 \mtiny{\pm 0.8}$ & $58.4 \mtiny{\pm 0.2}$ & $31.3 \mtiny{\pm 0.4}$ \\
& \textbf{F-EWC} & $65.9 \mtiny{\pm 1.5}$ & $39.1 \mtiny{\pm 0.7}$ & $66.0 \mtiny{\pm 1.3}$ & $40.0 \mtiny{\pm 0.9}$ & $41.7 \mtiny{\pm 0.7}$ & $86.2 \mtiny{\pm 0.8}$ & $51.8 \mtiny{\pm 1.3}$ & $59.9 \mtiny{\pm 1.0}$ & $57.6 \mtiny{\pm 0.4}$ & $31.3 \mtiny{\pm 0.2}$ \\
& \textbf{LoRA} & $69.7 \mtiny{\pm 1.4}$ & $44.8 \mtiny{\pm 1.1}$ & $81.4 \mtiny{\pm 0.7}$ & $56.9 \mtiny{\pm 1.0}$ & $50.7 \mtiny{\pm 1.3}$ & $92.9 \mtiny{\pm 1.3}$ & $52.0 \mtiny{\pm 1.0}$ & $61.8 \mtiny{\pm 1.5}$ & $56.5 \mtiny{\pm 0.4}$ & $31.3 \mtiny{\pm 0.4}$ \\
& \textbf{AdaLoRA} & $75.5 \mtiny{\pm 1.4}$ & $53.2 \mtiny{\pm 0.7}$ & $81.7 \mtiny{\pm 0.6}$ & $60.1 \mtiny{\pm 0.7}$ & $72.0 \mtiny{\pm 1.2}$ & $92.1 \mtiny{\pm 0.9}$ & $51.9 \mtiny{\pm 1.4}$ & $61.8 \mtiny{\pm 1.5}$ & $59.0 \mtiny{\pm 0.3}$ & $31.9 \mtiny{\pm 0.3}$ \\
& \textbf{SPU} & $76.0 \mtiny{\pm 0.9}$ & $53.2 \mtiny{\pm 0.6}$ & $82.3 \mtiny{\pm 1.1}$ & $60.3 \mtiny{\pm 1.3}$ & $75.7 \mtiny{\pm 0.9}$ & $91.3 \mtiny{\pm 1.3}$ & $51.7 \mtiny{\pm 0.8}$ & $61.5 \mtiny{\pm 1.2}$ & $58.4 \mtiny{\pm 0.3}$ & $31.4 \mtiny{\pm 0.4}$ \\
& \textbf{LoRSU} & $75.2 \mtiny{\pm 0.8}$ & $52.7 \mtiny{\pm 0.9}$ & $83.0 \mtiny{\pm 1.0}$ & $60.1 \mtiny{\pm 0.7}$ & $76.8 \mtiny{\pm 1.0}$ & $91.8 \mtiny{\pm 1.4}$ & $51.6 \mtiny{\pm 1.1}$ & $62.3 \mtiny{\pm 1.2}$ & $58.7 \mtiny{\pm 0.3}$ & $31.4 \mtiny{\pm 0.4}$ \\
\midrule
\multirow{7}{*}{\textbf{CL-20}} & \textbf{LN} & $72.9 \mtiny{\pm 0.5}$ & $54.0 \mtiny{\pm 0.9}$ & $80.3 \mtiny{\pm 0.6}$ & $57.3 \mtiny{\pm 0.4}$ & $73.3 \mtiny{\pm 0.4}$ & $90.7 \mtiny{\pm 0.4}$ & $51.8 \mtiny{\pm 0.8}$ & $61.9 \mtiny{\pm 0.9}$ & $61.0 \mtiny{\pm 0.1}$ & $31.4 \mtiny{\pm 0.1}$ \\
& \textbf{F-FT} & $72.9 \mtiny{\pm 0.5}$ & $47.9 \mtiny{\pm 0.6}$ & $83.0 \mtiny{\pm 0.7}$ & $56.9 \mtiny{\pm 0.9}$ & $62.7 \mtiny{\pm 0.9}$ & $90.6 \mtiny{\pm 0.9}$ & $51.9 \mtiny{\pm 0.4}$ & $61.3 \mtiny{\pm 0.4}$ & $56.5 \mtiny{\pm 0.2}$ & $31.5 \mtiny{\pm 0.3}$ \\
& \textbf{F-EWC} & $70.1 \mtiny{\pm 1.0}$ & $48.7 \mtiny{\pm 0.4}$ & $82.8 \mtiny{\pm 0.5}$ & $51.1 \mtiny{\pm 0.8}$ & $54.8 \mtiny{\pm 0.9}$ & $88.3 \mtiny{\pm 0.7}$ & $51.8 \mtiny{\pm 1.0}$ & $57.0 \mtiny{\pm 0.8}$ & $59.6 \mtiny{\pm 0.3}$ & $31.2 \mtiny{\pm 0.3}$ \\
& \textbf{LoRA} & $67.5 \mtiny{\pm 0.6}$ & $48.9 \mtiny{\pm 0.6}$ & $80.4 \mtiny{\pm 0.4}$ & $57.3 \mtiny{\pm 0.9}$ & $39.7 \mtiny{\pm 0.4}$ & $91.1 \mtiny{\pm 0.6}$ & $51.8 \mtiny{\pm 0.9}$ & $61.7 \mtiny{\pm 0.3}$ & $60.1 \mtiny{\pm 0.2}$ & $31.9 \mtiny{\pm 0.3}$ \\
& \textbf{AdaLoRA} & $72.5 \mtiny{\pm 1.0}$ & $51.5 \mtiny{\pm 1.0}$ & $79.2 \mtiny{\pm 0.4}$ & $54.1 \mtiny{\pm 1.0}$ & $65.5 \mtiny{\pm 0.7}$ & $90.6 \mtiny{\pm 0.8}$ & $51.7 \mtiny{\pm 0.9}$ & $61.9 \mtiny{\pm 0.9}$ & $56.5 \mtiny{\pm 0.3}$ & $31.7 \mtiny{\pm 0.3}$ \\
& \textbf{SPU} & $75.0 \mtiny{\pm 0.5}$ & $53.5 \mtiny{\pm 0.3}$ & $82.8 \mtiny{\pm 0.8}$ & $59.9 \mtiny{\pm 0.6}$ & $76.1 \mtiny{\pm 0.9}$ & $91.6 \mtiny{\pm 0.9}$ & $51.6 \mtiny{\pm 0.6}$ & $61.9 \mtiny{\pm 0.4}$ & $61.8 \mtiny{\pm 0.2}$ & $31.6 \mtiny{\pm 0.3}$ \\
& \textbf{LoRSU} & $75.3 \mtiny{\pm 0.8}$ & $53.1 \mtiny{\pm 0.9}$ & $83.8 \mtiny{\pm 0.9}$ & $58.8 \mtiny{\pm 1.0}$ & $75.5 \mtiny{\pm 0.7}$ & $92.0 \mtiny{\pm 0.3}$ & $51.9 \mtiny{\pm 0.4}$ & $62.3 \mtiny{\pm 0.6}$ & $60.4 \mtiny{\pm 0.2}$ & $31.6 \mtiny{\pm 0.2}$ \\
\midrule
\multirow{7}{*}{\textbf{CL-50}} & \textbf{LN} & $71.1 \mtiny{\pm 0.1}$ & $50.4 \mtiny{\pm 0.3}$ & $77.0 \mtiny{\pm 0.3}$ & $57.5 \mtiny{\pm 0.3}$ & $57.9 \mtiny{\pm 0.1}$ & $89.7 \mtiny{\pm 0.1}$ & $51.6 \mtiny{\pm 0.1}$ & $62.4 \mtiny{\pm 0.3}$ & $56.1 \mtiny{\pm 0.1}$ & $31.9 \mtiny{\pm 0.0}$ \\
& \textbf{F-FT} & $70.1 \mtiny{\pm 0.1}$ & $48.9 \mtiny{\pm 0.3}$ & $81.7 \mtiny{\pm 0.0}$ & $56.2 \mtiny{\pm 0.2}$ & $47.5 \mtiny{\pm 0.1}$ & $89.9 \mtiny{\pm 0.3}$ & $52.0 \mtiny{\pm 0.1}$ & $61.2 \mtiny{\pm 0.1}$ & $57.7 \mtiny{\pm 0.1}$ & $31.1 \mtiny{\pm 0.1}$ \\
& \textbf{F-EWC} & $61.7 \mtiny{\pm 0.0}$ & $43.9 \mtiny{\pm 0.3}$ & $83.3 \mtiny{\pm 0.4}$ & $46.2 \mtiny{\pm 0.3}$ & $38.9 \mtiny{\pm 0.2}$ & $87.5 \mtiny{\pm 0.1}$ & $51.8 \mtiny{\pm 0.3}$ & $55.8 \mtiny{\pm 0.3}$ & $54.7 \mtiny{\pm 0.1}$ & $30.7 \mtiny{\pm 0.1}$ \\
& \textbf{LoRA} & $66.8 \mtiny{\pm 0.2}$ & $47.8 \mtiny{\pm 0.3}$ & $82.3 \mtiny{\pm 0.2}$ & $55.7 \mtiny{\pm 0.0}$ & $52.0 \mtiny{\pm 0.3}$ & $91.0 \mtiny{\pm 0.3}$ & $51.7 \mtiny{\pm 0.3}$ & $61.6 \mtiny{\pm 0.2}$ & $60.2 \mtiny{\pm 0.0}$ & $31.6 \mtiny{\pm 0.1}$ \\
& \textbf{AdaLoRA} & $73.5 \mtiny{\pm 0.0}$ & $49.9 \mtiny{\pm 0.1}$ & $80.9 \mtiny{\pm 0.4}$ & $55.7 \mtiny{\pm 0.4}$ & $77.8 \mtiny{\pm 0.1}$ & $93.1 \mtiny{\pm 0.0}$ & $51.5 \mtiny{\pm 0.1}$ & $61.4 \mtiny{\pm 0.3}$ & $56.9 \mtiny{\pm 0.0}$ & $31.6 \mtiny{\pm 0.1}$ \\
& \textbf{SPU} & $75.2 \mtiny{\pm 0.2}$ & $53.2 \mtiny{\pm 0.0}$ & $83.3 \mtiny{\pm 0.3}$ & $59.3 \mtiny{\pm 0.2}$ & $73.1 \mtiny{\pm 0.3}$ & $91.4 \mtiny{\pm 0.4}$ & $51.7 \mtiny{\pm 0.3}$ & $61.7 \mtiny{\pm 0.1}$ & $58.5 \mtiny{\pm 0.1}$ & $31.6 \mtiny{\pm 0.1}$ \\
& \textbf{LoRSU} & $75.0 \mtiny{\pm 0.2}$ & $51.8 \mtiny{\pm 0.1}$ & $84.0 \mtiny{\pm 0.4}$ & $58.5 \mtiny{\pm 0.2}$ & $72.7 \mtiny{\pm 0.3}$ & $91.9 \mtiny{\pm 0.3}$ & $51.7 \mtiny{\pm 0.1}$ & $62.3 \mtiny{\pm 0.4}$ & $58.1 \mtiny{\pm 0.0}$ & $31.7 \mtiny{\pm 0.1}$ \\
\bottomrule
\end{tabular}
\endgroup
\end{small}
\end{center}
\vskip -0.1in
\end{table}


\begin{table}
\caption{Accuracy scores (\%) for LLaVA with the pretrained (\emph{Zr-Shot}) or fine-tuned image encoder. All baselines use \emph{AIR} dataset for fine-tuning the image encoder~(the LLM remains frozen) via CLIP loss. We include error bars over 3 runs.}
 \label{table:vlm_vqa_acc_aircraft_clip}
\vskip 0.15in
\begin{center}
\begin{small}
\begingroup
\setlength{\tabcolsep}{3.9pt}
\begin{tabular}{l c c c c c c c c c c c}
\toprule
 & & \multicolumn{9}{c}{\textbf{VQA Datasets (Acc \%)}}  \\
\cmidrule(lr){3-12}
\textbf{Setting} & \textbf{FT Method}  & \textbf{GTS} & \textbf{TSI} & \textbf{CAn} & \textbf{AIR} & \textbf{ESAT} & \textbf{DALLE} & \textbf{VSR} & \textbf{HM} & \textbf{MMVP} & \textbf{VisOnly} \\
\midrule
 & \textbf{Zr-Shot} & $75.6$ & $53.1$ & $82.7$ & $60.4$ & $76.1$ & $91.1$ & $51.5$ & $61.2$ & $58.0$ & $31.3$ \\
\midrule
\multirow{7}{*}{\textbf{CL-5}} & \textbf{LN} & $73.4 \mtiny{\pm 0.8}$ & $51.3 \mtiny{\pm 1.2}$ & $80.2 \mtiny{\pm 0.6}$ & $60.7 \mtiny{\pm 1.5}$ & $66.9 \mtiny{\pm 0.7}$ & $91.3 \mtiny{\pm 0.6}$ & $51.9 \mtiny{\pm 0.9}$ & $62.4 \mtiny{\pm 1.2}$ & $58.5 \mtiny{\pm 0.2}$ & $30.6 \mtiny{\pm 0.2}$ \\
& \textbf{F-FT} & $72.5 \mtiny{\pm 1.2}$ & $50.5 \mtiny{\pm 0.5}$ & $79.9 \mtiny{\pm 0.9}$ & $62.4 \mtiny{\pm 0.9}$ & $60.7 \mtiny{\pm 1.4}$ & $90.6 \mtiny{\pm 0.5}$ & $51.7 \mtiny{\pm 0.9}$ & $60.9 \mtiny{\pm 1.1}$ & $58.3 \mtiny{\pm 0.4}$ & $31.4 \mtiny{\pm 0.3}$ \\
& \textbf{F-EWC} & $74.9 \mtiny{\pm 1.2}$ & $52.4 \mtiny{\pm 0.8}$ & $71.5 \mtiny{\pm 1.2}$ & $63.3 \mtiny{\pm 1.0}$ & $63.8 \mtiny{\pm 1.0}$ & $90.7 \mtiny{\pm 1.5}$ & $51.2 \mtiny{\pm 0.5}$ & $61.2 \mtiny{\pm 0.8}$ & $58.1 \mtiny{\pm 0.4}$ & $31.4 \mtiny{\pm 0.4}$ \\
& \textbf{LoRA} & $70.9 \mtiny{\pm 0.9}$ & $52.7 \mtiny{\pm 0.6}$ & $79.0 \mtiny{\pm 0.7}$ & $61.7 \mtiny{\pm 0.5}$ & $48.8 \mtiny{\pm 0.7}$ & $90.6 \mtiny{\pm 0.6}$ & $52.0 \mtiny{\pm 0.9}$ & $62.5 \mtiny{\pm 0.8}$ & $60.0 \mtiny{\pm 0.3}$ & $31.1 \mtiny{\pm 0.2}$ \\
& \textbf{AdaLoRA} & $75.0 \mtiny{\pm 1.0}$ & $53.3 \mtiny{\pm 0.8}$ & $83.7 \mtiny{\pm 0.9}$ & $60.8 \mtiny{\pm 0.8}$ & $75.2 \mtiny{\pm 1.5}$ & $91.7 \mtiny{\pm 1.0}$ & $51.6 \mtiny{\pm 0.8}$ & $61.6 \mtiny{\pm 0.8}$ & $56.9 \mtiny{\pm 0.3}$ & $31.9 \mtiny{\pm 0.4}$ \\
& \textbf{SPU} & $76.2 \mtiny{\pm 0.6}$ & $53.0 \mtiny{\pm 1.3}$ & $83.0 \mtiny{\pm 0.8}$ & $63.5 \mtiny{\pm 0.8}$ & $75.3 \mtiny{\pm 0.7}$ & $91.5 \mtiny{\pm 1.5}$ & $51.5 \mtiny{\pm 0.6}$ & $61.5 \mtiny{\pm 0.8}$ & $58.1 \mtiny{\pm 0.3}$ & $31.5 \mtiny{\pm 0.4}$ \\
& \textbf{LoRSU} & $76.2 \mtiny{\pm 0.8}$ & $53.4 \mtiny{\pm 1.4}$ & $82.5 \mtiny{\pm 1.0}$ & $65.2 \mtiny{\pm 1.3}$ & $76.0 \mtiny{\pm 0.9}$ & $91.8 \mtiny{\pm 0.8}$ & $51.6 \mtiny{\pm 0.8}$ & $62.1 \mtiny{\pm 1.1}$ & $59.0 \mtiny{\pm 0.4}$ & $31.2 \mtiny{\pm 0.3}$ \\
\midrule
\multirow{7}{*}{\textbf{CL-20}} & \textbf{LN} & $70.3 \mtiny{\pm 0.9}$ & $53.7 \mtiny{\pm 0.6}$ & $77.9 \mtiny{\pm 1.0}$ & $60.2 \mtiny{\pm 0.4}$ & $56.3 \mtiny{\pm 0.7}$ & $90.6 \mtiny{\pm 0.3}$ & $51.7 \mtiny{\pm 1.0}$ & $62.8 \mtiny{\pm 0.7}$ & $58.1 \mtiny{\pm 0.1}$ & $31.8 \mtiny{\pm 0.3}$ \\
& \textbf{F-FT} & $73.0 \mtiny{\pm 0.6}$ & $54.1 \mtiny{\pm 0.6}$ & $80.3 \mtiny{\pm 0.9}$ & $69.7 \mtiny{\pm 0.5}$ & $62.7 \mtiny{\pm 0.5}$ & $90.0 \mtiny{\pm 0.4}$ & $51.9 \mtiny{\pm 0.3}$ & $61.8 \mtiny{\pm 0.4}$ & $58.9 \mtiny{\pm 0.1}$ & $31.4 \mtiny{\pm 0.1}$ \\
& \textbf{F-EWC} & $71.2 \mtiny{\pm 0.5}$ & $53.9 \mtiny{\pm 1.0}$ & $79.3 \mtiny{\pm 0.4}$ & $70.6 \mtiny{\pm 1.0}$ & $64.6 \mtiny{\pm 0.7}$ & $89.7 \mtiny{\pm 0.6}$ & $51.7 \mtiny{\pm 0.4}$ & $61.5 \mtiny{\pm 0.5}$ & $58.9 \mtiny{\pm 0.3}$ & $31.4 \mtiny{\pm 0.2}$ \\
& \textbf{LoRA} & $71.8 \mtiny{\pm 0.9}$ & $51.1 \mtiny{\pm 0.8}$ & $78.6 \mtiny{\pm 0.3}$ & $65.7 \mtiny{\pm 0.4}$ & $63.4 \mtiny{\pm 0.8}$ & $89.9 \mtiny{\pm 1.0}$ & $51.7 \mtiny{\pm 0.3}$ & $62.3 \mtiny{\pm 0.3}$ & $56.2 \mtiny{\pm 0.2}$ & $31.5 \mtiny{\pm 0.2}$ \\
& \textbf{AdaLoRA} & $73.4 \mtiny{\pm 0.8}$ & $51.6 \mtiny{\pm 0.6}$ & $81.2 \mtiny{\pm 0.9}$ & $63.1 \mtiny{\pm 0.6}$ & $73.8 \mtiny{\pm 0.8}$ & $90.8 \mtiny{\pm 0.5}$ & $52.1 \mtiny{\pm 0.4}$ & $62.7 \mtiny{\pm 0.8}$ & $57.7 \mtiny{\pm 0.2}$ & $31.2 \mtiny{\pm 0.1}$ \\
& \textbf{SPU} & $75.7 \mtiny{\pm 0.4}$ & $52.2 \mtiny{\pm 0.7}$ & $82.0 \mtiny{\pm 0.8}$ & $63.4 \mtiny{\pm 0.9}$ & $72.6 \mtiny{\pm 0.6}$ & $91.7 \mtiny{\pm 0.6}$ & $51.8 \mtiny{\pm 0.6}$ & $62.2 \mtiny{\pm 0.5}$ & $59.0 \mtiny{\pm 0.2}$ & $31.4 \mtiny{\pm 0.2}$ \\
& \textbf{LoRSU} & $75.7 \mtiny{\pm 0.9}$ & $52.6 \mtiny{\pm 0.9}$ & $81.4 \mtiny{\pm 0.7}$ & $66.3 \mtiny{\pm 0.7}$ & $73.0 \mtiny{\pm 0.8}$ & $90.9 \mtiny{\pm 0.8}$ & $51.9 \mtiny{\pm 0.8}$ & $61.8 \mtiny{\pm 0.8}$ & $56.9 \mtiny{\pm 0.1}$ & $31.6 \mtiny{\pm 0.3}$ \\
\midrule
\multirow{7}{*}{\textbf{CL-50}} & \textbf{LN} & $69.6 \mtiny{\pm 0.4}$ & $54.0 \mtiny{\pm 0.1}$ & $76.9 \mtiny{\pm 0.3}$ & $62.2 \mtiny{\pm 0.2}$ & $50.9 \mtiny{\pm 0.2}$ & $90.2 \mtiny{\pm 0.0}$ & $52.0 \mtiny{\pm 0.3}$ & $62.8 \mtiny{\pm 0.4}$ & $57.7 \mtiny{\pm 0.1}$ & $31.5 \mtiny{\pm 0.1}$ \\
& \textbf{F-FT} & $71.2 \mtiny{\pm 0.3}$ & $50.3 \mtiny{\pm 0.3}$ & $78.3 \mtiny{\pm 0.2}$ & $70.4 \mtiny{\pm 0.4}$ & $59.9 \mtiny{\pm 0.0}$ & $90.1 \mtiny{\pm 0.1}$ & $51.9 \mtiny{\pm 0.1}$ & $61.8 \mtiny{\pm 0.3}$ & $57.5 \mtiny{\pm 0.1}$ & $31.3 \mtiny{\pm 0.1}$ \\
& \textbf{F-EWC} & $71.8 \mtiny{\pm 0.2}$ & $51.6 \mtiny{\pm 0.1}$ & $78.3 \mtiny{\pm 0.0}$ & $71.3 \mtiny{\pm 0.2}$ & $57.6 \mtiny{\pm 0.2}$ & $90.2 \mtiny{\pm 0.2}$ & $51.7 \mtiny{\pm 0.1}$ & $61.1 \mtiny{\pm 0.2}$ & $57.4 \mtiny{\pm 0.1}$ & $31.5 \mtiny{\pm 0.0}$ \\
& \textbf{LoRA} & $69.8 \mtiny{\pm 0.0}$ & $54.7 \mtiny{\pm 0.0}$ & $77.0 \mtiny{\pm 0.3}$ & $68.2 \mtiny{\pm 0.3}$ & $51.6 \mtiny{\pm 0.1}$ & $90.0 \mtiny{\pm 0.1}$ & $52.0 \mtiny{\pm 0.4}$ & $62.4 \mtiny{\pm 0.0}$ & $57.1 \mtiny{\pm 0.1}$ & $31.5 \mtiny{\pm 0.1}$ \\
& \textbf{AdaLoRA} & $74.2 \mtiny{\pm 0.3}$ & $52.0 \mtiny{\pm 0.2}$ & $82.4 \mtiny{\pm 0.1}$ & $65.0 \mtiny{\pm 0.2}$ & $72.6 \mtiny{\pm 0.0}$ & $91.9 \mtiny{\pm 0.1}$ & $51.7 \mtiny{\pm 0.2}$ & $60.7 \mtiny{\pm 0.1}$ & $55.6 \mtiny{\pm 0.0}$ & $31.3 \mtiny{\pm 0.0}$ \\
& \textbf{SPU} & $75.2 \mtiny{\pm 0.2}$ & $52.2 \mtiny{\pm 0.4}$ & $82.6 \mtiny{\pm 0.3}$ & $66.6 \mtiny{\pm 0.4}$ & $70.0 \mtiny{\pm 0.2}$ & $91.6 \mtiny{\pm 0.3}$ & $51.9 \mtiny{\pm 0.2}$ & $62.0 \mtiny{\pm 0.3}$ & $57.6 \mtiny{\pm 0.0}$ & $31.8 \mtiny{\pm 0.0}$ \\
& \textbf{LoRSU} & $75.4 \mtiny{\pm 0.4}$ & $52.7 \mtiny{\pm 0.3}$ & $81.6 \mtiny{\pm 0.2}$ & $68.6 \mtiny{\pm 0.3}$ & $69.7 \mtiny{\pm 0.3}$ & $91.5 \mtiny{\pm 0.2}$ & $51.7 \mtiny{\pm 0.4}$ & $62.2 \mtiny{\pm 0.1}$ & $58.7 \mtiny{\pm 0.1}$ & $31.1 \mtiny{\pm 0.1}$ \\
\bottomrule
\end{tabular}
\endgroup
\end{small}
\end{center}
\vskip -0.1in
\end{table}


\begin{table}
\caption{Accuracy scores (\%) for LLaVA with the pretrained (\emph{Zr-Shot}) or fine-tuned image encoder. All baselines use \emph{ESAT} dataset for fine-tuning the image encoder~(the LLM remains frozen) via CLIP loss. We include error bars over 3 runs.}
 \label{table:vlm_vqa_acc_eurosat_clip}
\vskip 0.15in
\begin{center}
\begin{small}
\begingroup
\setlength{\tabcolsep}{3.9pt}
\begin{tabular}{l c c c c c c c c c c c}
\toprule
 & & \multicolumn{9}{c}{\textbf{VQA Datasets (Acc \%)}}  \\
\cmidrule(lr){3-12}
\textbf{Setting} & \textbf{FT Method}  & \textbf{GTS} & \textbf{TSI} & \textbf{CAn} & \textbf{AIR} & \textbf{ESAT} & \textbf{DALLE} & \textbf{VSR} & \textbf{HM} & \textbf{MMVP} & \textbf{VisOnly} \\
\midrule
 & \textbf{Zr-Shot} & $75.6$ & $53.1$ & $82.7$ & $60.4$ & $76.1$ & $91.1$ & $51.5$ & $61.2$ & $58.0$ & $31.3$ \\
\midrule
\multirow{7}{*}{\textbf{CL-5}} & \textbf{LN} & $75.8 \mtiny{\pm 0.9}$ & $53.2 \mtiny{\pm 0.6}$ & $82.6 \mtiny{\pm 1.1}$ & $60.0 \mtiny{\pm 1.3}$ & $80.3 \mtiny{\pm 1.0}$ & $92.7 \mtiny{\pm 1.0}$ & $51.9 \mtiny{\pm 0.7}$ & $61.7 \mtiny{\pm 0.5}$ & $60.4 \mtiny{\pm 0.4}$ & $31.8 \mtiny{\pm 0.3}$ \\
& \textbf{F-FT} & $69.1 \mtiny{\pm 0.8}$ & $50.5 \mtiny{\pm 0.6}$ & $80.8 \mtiny{\pm 1.1}$ & $57.7 \mtiny{\pm 1.5}$ & $65.8 \mtiny{\pm 0.6}$ & $91.3 \mtiny{\pm 1.5}$ & $51.8 \mtiny{\pm 0.7}$ & $62.0 \mtiny{\pm 0.7}$ & $58.8 \mtiny{\pm 0.3}$ & $30.4 \mtiny{\pm 0.2}$ \\
& \textbf{F-EWC} & $66.3 \mtiny{\pm 0.9}$ & $52.1 \mtiny{\pm 1.4}$ & $79.3 \mtiny{\pm 1.0}$ & $56.8 \mtiny{\pm 1.3}$ & $67.7 \mtiny{\pm 1.3}$ & $90.9 \mtiny{\pm 0.8}$ & $51.9 \mtiny{\pm 1.3}$ & $62.0 \mtiny{\pm 1.2}$ & $55.4 \mtiny{\pm 0.2}$ & $30.9 \mtiny{\pm 0.4}$ \\
& \textbf{LoRA} & $73.2 \mtiny{\pm 1.3}$ & $49.3 \mtiny{\pm 1.2}$ & $80.6 \mtiny{\pm 0.9}$ & $60.4 \mtiny{\pm 1.1}$ & $74.5 \mtiny{\pm 0.8}$ & $92.3 \mtiny{\pm 1.3}$ & $52.0 \mtiny{\pm 1.1}$ & $61.6 \mtiny{\pm 1.1}$ & $57.4 \mtiny{\pm 0.4}$ & $31.4 \mtiny{\pm 0.3}$ \\
& \textbf{AdaLoRA} & $75.9 \mtiny{\pm 0.5}$ & $52.4 \mtiny{\pm 1.4}$ & $82.4 \mtiny{\pm 0.5}$ & $60.5 \mtiny{\pm 0.8}$ & $78.0 \mtiny{\pm 1.3}$ & $91.5 \mtiny{\pm 0.9}$ & $51.6 \mtiny{\pm 0.8}$ & $61.5 \mtiny{\pm 1.3}$ & $59.0 \mtiny{\pm 0.4}$ & $30.9 \mtiny{\pm 0.2}$ \\
& \textbf{SPU} & $75.8 \mtiny{\pm 0.8}$ & $53.2 \mtiny{\pm 1.4}$ & $82.8 \mtiny{\pm 1.4}$ & $60.5 \mtiny{\pm 1.5}$ & $80.6 \mtiny{\pm 0.9}$ & $91.5 \mtiny{\pm 1.1}$ & $51.7 \mtiny{\pm 0.6}$ & $61.7 \mtiny{\pm 1.5}$ & $57.5 \mtiny{\pm 0.4}$ & $31.5 \mtiny{\pm 0.2}$ \\
& \textbf{LoRSU} & $76.2 \mtiny{\pm 1.0}$ & $53.6 \mtiny{\pm 1.1}$ & $82.5 \mtiny{\pm 1.2}$ & $60.8 \mtiny{\pm 0.8}$ & $82.9 \mtiny{\pm 1.0}$ & $91.5 \mtiny{\pm 0.9}$ & $51.6 \mtiny{\pm 0.9}$ & $61.3 \mtiny{\pm 0.7}$ & $57.7 \mtiny{\pm 0.4}$ & $31.9 \mtiny{\pm 0.4}$ \\
\midrule
\multirow{7}{*}{\textbf{CL-20}} & \textbf{LN} & $74.5 \mtiny{\pm 0.5}$ & $52.6 \mtiny{\pm 0.7}$ & $82.5 \mtiny{\pm 0.5}$ & $58.8 \mtiny{\pm 0.7}$ & $77.0 \mtiny{\pm 0.4}$ & $92.4 \mtiny{\pm 0.5}$ & $51.9 \mtiny{\pm 1.0}$ & $62.5 \mtiny{\pm 0.5}$ & $58.0 \mtiny{\pm 0.3}$ & $31.2 \mtiny{\pm 0.1}$ \\
& \textbf{F-FT} & $66.5 \mtiny{\pm 0.8}$ & $51.1 \mtiny{\pm 0.7}$ & $79.1 \mtiny{\pm 0.4}$ & $56.7 \mtiny{\pm 0.6}$ & $51.2 \mtiny{\pm 0.7}$ & $92.0 \mtiny{\pm 0.4}$ & $51.6 \mtiny{\pm 0.6}$ & $61.4 \mtiny{\pm 0.8}$ & $60.1 \mtiny{\pm 0.1}$ & $31.5 \mtiny{\pm 0.2}$ \\
& \textbf{F-EWC} & $69.3 \mtiny{\pm 0.3}$ & $51.2 \mtiny{\pm 1.0}$ & $60.5 \mtiny{\pm 0.8}$ & $57.1 \mtiny{\pm 0.6}$ & $54.1 \mtiny{\pm 0.6}$ & $89.7 \mtiny{\pm 0.6}$ & $51.9 \mtiny{\pm 0.6}$ & $60.9 \mtiny{\pm 0.7}$ & $58.4 \mtiny{\pm 0.2}$ & $31.8 \mtiny{\pm 0.2}$ \\
& \textbf{LoRA} & $71.1 \mtiny{\pm 0.7}$ & $50.9 \mtiny{\pm 0.5}$ & $80.3 \mtiny{\pm 1.0}$ & $59.4 \mtiny{\pm 0.7}$ & $64.6 \mtiny{\pm 0.7}$ & $91.1 \mtiny{\pm 0.7}$ & $52.0 \mtiny{\pm 0.4}$ & $62.3 \mtiny{\pm 0.6}$ & $62.3 \mtiny{\pm 0.2}$ & $31.3 \mtiny{\pm 0.1}$ \\
& \textbf{AdaLoRA} & $70.0 \mtiny{\pm 0.6}$ & $47.3 \mtiny{\pm 0.8}$ & $78.4 \mtiny{\pm 0.9}$ & $51.7 \mtiny{\pm 0.4}$ & $69.3 \mtiny{\pm 0.5}$ & $91.3 \mtiny{\pm 0.7}$ & $51.7 \mtiny{\pm 0.9}$ & $60.8 \mtiny{\pm 0.9}$ & $58.1 \mtiny{\pm 0.2}$ & $31.6 \mtiny{\pm 0.1}$ \\
& \textbf{SPU} & $75.6 \mtiny{\pm 0.9}$ & $53.1 \mtiny{\pm 0.3}$ & $82.8 \mtiny{\pm 0.9}$ & $59.9 \mtiny{\pm 0.8}$ & $81.5 \mtiny{\pm 0.6}$ & $92.3 \mtiny{\pm 0.4}$ & $51.9 \mtiny{\pm 0.5}$ & $61.5 \mtiny{\pm 0.8}$ & $58.8 \mtiny{\pm 0.2}$ & $31.7 \mtiny{\pm 0.1}$ \\
& \textbf{LoRSU} & $75.3 \mtiny{\pm 1.0}$ & $53.7 \mtiny{\pm 0.8}$ & $82.8 \mtiny{\pm 0.4}$ & $60.7 \mtiny{\pm 0.8}$ & $82.7 \mtiny{\pm 0.7}$ & $91.6 \mtiny{\pm 0.6}$ & $51.6 \mtiny{\pm 0.4}$ & $61.5 \mtiny{\pm 0.4}$ & $58.4 \mtiny{\pm 0.2}$ & $31.4 \mtiny{\pm 0.2}$ \\
\midrule
\multirow{7}{*}{\textbf{CL-50}} & \textbf{LN} & $73.1 \mtiny{\pm 0.3}$ & $53.0 \mtiny{\pm 0.2}$ & $82.0 \mtiny{\pm 0.1}$ & $59.1 \mtiny{\pm 0.2}$ & $80.7 \mtiny{\pm 0.0}$ & $92.4 \mtiny{\pm 0.2}$ & $51.8 \mtiny{\pm 0.3}$ & $62.0 \mtiny{\pm 0.1}$ & $60.4 \mtiny{\pm 0.0}$ & $32.0 \mtiny{\pm 0.0}$ \\
& \textbf{F-FT} & $58.0 \mtiny{\pm 0.4}$ & $50.3 \mtiny{\pm 0.0}$ & $76.8 \mtiny{\pm 0.1}$ & $57.2 \mtiny{\pm 0.2}$ & $34.7 \mtiny{\pm 0.1}$ & $89.7 \mtiny{\pm 0.0}$ & $51.7 \mtiny{\pm 0.2}$ & $61.6 \mtiny{\pm 0.2}$ & $58.1 \mtiny{\pm 0.0}$ & $31.6 \mtiny{\pm 0.1}$ \\
& \textbf{F-EWC} & $59.0 \mtiny{\pm 0.1}$ & $64.5 \mtiny{\pm 0.1}$ & $77.2 \mtiny{\pm 0.1}$ & $56.3 \mtiny{\pm 0.1}$ & $38.0 \mtiny{\pm 0.2}$ & $87.3 \mtiny{\pm 0.2}$ & $51.9 \mtiny{\pm 0.2}$ & $60.7 \mtiny{\pm 0.2}$ & $58.2 \mtiny{\pm 0.1}$ & $31.8 \mtiny{\pm 0.0}$ \\
& \textbf{LoRA} & $62.8 \mtiny{\pm 0.3}$ & $47.2 \mtiny{\pm 0.4}$ & $72.4 \mtiny{\pm 0.4}$ & $54.4 \mtiny{\pm 0.2}$ & $61.6 \mtiny{\pm 0.4}$ & $90.2 \mtiny{\pm 0.3}$ & $51.7 \mtiny{\pm 0.2}$ & $62.0 \mtiny{\pm 0.1}$ & $60.8 \mtiny{\pm 0.0}$ & $30.9 \mtiny{\pm 0.1}$ \\
& \textbf{AdaLoRA} & $67.2 \mtiny{\pm 0.2}$ & $49.3 \mtiny{\pm 0.3}$ & $78.8 \mtiny{\pm 0.3}$ & $56.9 \mtiny{\pm 0.3}$ & $58.8 \mtiny{\pm 0.3}$ & $89.6 \mtiny{\pm 0.3}$ & $51.8 \mtiny{\pm 0.1}$ & $61.9 \mtiny{\pm 0.2}$ & $56.0 \mtiny{\pm 0.1}$ & $31.6 \mtiny{\pm 0.0}$ \\
& \textbf{SPU} & $75.1 \mtiny{\pm 0.3}$ & $53.4 \mtiny{\pm 0.2}$ & $82.5 \mtiny{\pm 0.2}$ & $60.2 \mtiny{\pm 0.3}$ & $81.9 \mtiny{\pm 0.1}$ & $92.3 \mtiny{\pm 0.3}$ & $51.8 \mtiny{\pm 0.1}$ & $61.6 \mtiny{\pm 0.1}$ & $57.1 \mtiny{\pm 0.1}$ & $31.9 \mtiny{\pm 0.0}$ \\
& \textbf{LoRSU} & $75.4 \mtiny{\pm 0.3}$ & $53.9 \mtiny{\pm 0.1}$ & $83.1 \mtiny{\pm 0.2}$ & $60.3 \mtiny{\pm 0.1}$ & $83.1 \mtiny{\pm 0.1}$ & $92.1 \mtiny{\pm 0.1}$ & $51.6 \mtiny{\pm 0.2}$ & $61.2 \mtiny{\pm 0.0}$ & $57.6 \mtiny{\pm 0.0}$ & $31.1 \mtiny{\pm 0.0}$ \\
\bottomrule
\end{tabular}
\endgroup
\end{small}
\end{center}
\vskip -0.1in
\end{table}

\subsection{Extra ACC and BWT results}\label{sec_appx:extra_bwt_results}
\begin{table*}
\caption{\emph{Average accuracy} (ACC) and \emph{backward transfer} (BWT) scores (\%) for LLaVA with the fine-tuned CLIP-L-14. Each column indicates the setting and fine-tuning method. We include error bars over 3 runs.}
\label{table:bwt_metrics_clip_full}
\vskip 0.15in
\begin{center}
%\begin{small}
\begingroup
\setlength{\tabcolsep}{5.6pt}
\begin{tabular}{l c c c c c c c c c}
\toprule
 & & & & \multicolumn{6}{c}{\textbf{FT Method}}  \\
\cmidrule(lr){5-10}
\multirow{2}{*}{\textbf{Setting}} & \multirow{2}{*}{\textbf{FT Dataset}}  & \multicolumn{2}{c}{\textbf{Zr-Shot}} & \multicolumn{2}{c}{\textbf{LoRA}} & \multicolumn{2}{c}{\textbf{SPU}} &  \multicolumn{2}{c}{\textbf{LoRSU}} \\
\cmidrule(lr){3-4} \cmidrule(lr){5-6} \cmidrule(lr){7-8} \cmidrule(lr){9-10} & & \textbf{ACC ($\uparrow)$} & \textbf{BWT ($\uparrow)$} & \textbf{ACC ($\uparrow)$} & \textbf{BWT ($\uparrow)$} & \textbf{ACC ($\uparrow)$} & \textbf{BWT ($\uparrow)$} & \textbf{ACC ($\uparrow)$} & \textbf{BWT ($\uparrow)$} \\
\midrule
\multirow{4}{*}{\textbf{CL-5}} & \textbf{GTS} & $75.4$ & $0.0$ & $79.2 \mtiny{\pm 0.7}$ & $-7.1 \mtiny{\pm 0.8}$ & $80.8 \mtiny{\pm 0.5}$ & $0.5 \mtiny{\pm 0.6}$ & $81.1 \mtiny{\pm 0.6}$ & $0.4 \mtiny{\pm 0.7}$ \\
& \textbf{TSI} & $54.0$ & $0.0$ & $55.5 \mtiny{\pm 0.9}$ & $-2.5 \mtiny{\pm 0.6}$ & $55.5 \mtiny{\pm 0.6}$ & $0.2 \mtiny{\pm 0.5}$ & $57.0 \mtiny{\pm 0.8}$ & $0.5 \mtiny{\pm 0.6}$ \\
& \textbf{AIR} & $60.4$ & $0.0$ & $59.2 \mtiny{\pm 0.8}$ & $-2.1 \mtiny{\pm 0.7}$ & $64.7 \mtiny{\pm 0.5}$ & $2.8 \mtiny{\pm 0.6}$ & $65.0 \mtiny{\pm 0.7}$ & $2.5 \mtiny{\pm 0.6}$ \\
& \textbf{ESAT} & $76.4$ & $0.0$ & $73.8 \mtiny{\pm 0.9}$ & $-3.4 \mtiny{\pm 0.6}$ & $79.8 \mtiny{\pm 0.6}$ & $1.5 \mtiny{\pm 0.7}$ & $82.2 \mtiny{\pm 0.7}$ & $2.0 \mtiny{\pm 0.6}$ \\
\midrule
\multirow{4}{*}{\textbf{CL-20}} & \textbf{GTS} & $75.4$ & $0.0$ & $77.2 \mtiny{\pm 0.4}$ & $-9.1 \mtiny{\pm 0.5}$ & $82.8 \mtiny{\pm 0.4}$ & $-0.6 \mtiny{\pm 0.3}$ & $83.5 \mtiny{\pm 0.6}$ & $-0.4 \mtiny{\pm 0.3}$ \\
& \textbf{TSI} & $54.0$ & $0.0$ & $60.6 \mtiny{\pm 0.3}$ & $-7.2 \mtiny{\pm 0.4}$ & $60.1 \mtiny{\pm 0.5}$ & $-1.7 \mtiny{\pm 0.3}$ & $62.1 \mtiny{\pm 0.3}$ & $-0.9 \mtiny{\pm 0.4}$ \\
& \textbf{AIR} & $60.4$ & $0.0$ & $64.3 \mtiny{\pm 0.4}$ & $-3.6 \mtiny{\pm 0.6}$ & $65.2 \mtiny{\pm 0.7}$ & $1.1 \mtiny{\pm 0.4}$ & $65.4 \mtiny{\pm 0.3}$ & $0.9 \mtiny{\pm 0.4}$ \\
& \textbf{ESAT} & $76.4$ & $0.0$ & $64.1 \mtiny{\pm 0.5}$ & $-18.3 \mtiny{\pm 0.7}$ & $82.0 \mtiny{\pm 0.4}$ & $2.0 \mtiny{\pm 0.2}$ & $82.7 \mtiny{\pm 0.5}$ & $0.1 \mtiny{\pm 0.3}$ \\
\midrule
\multirow{4}{*}{\textbf{CL-50}} & \textbf{GTS} & $75.4$ & $0.0$ & $79.3 \mtiny{\pm 0.3}$ & $-10.3 \mtiny{\pm 0.5}$ & $83.8 \mtiny{\pm 0.2}$ & $-0.7 \mtiny{\pm 0.1}$ & $84.7 \mtiny{\pm 0.3}$ & $-0.5 \mtiny{\pm 0.2}$ \\
& \textbf{TSI} & $54.0$ & $0.0$ & $67.0 \mtiny{\pm 0.3}$ & $-8.1 \mtiny{\pm 0.6}$ & $61.8 \mtiny{\pm 0.2}$ & $-1.9 \mtiny{\pm 0.3}$ & $67.9 \mtiny{\pm 0.2}$ & $-1.1 \mtiny{\pm 0.3}$ \\
& \textbf{AIR} & $60.4$ & $0.0$ & $65.6 \mtiny{\pm 0.4}$ & $-6.1 \mtiny{\pm 0.3}$ & $67.1 \mtiny{\pm 0.3}$ & $0.5 \mtiny{\pm 0.2}$ & $67.7 \mtiny{\pm 0.3}$ & $0.7 \mtiny{\pm 0.3}$ \\
& \textbf{ESAT} & $76.4$ & $0.0$ & $61.4 \mtiny{\pm 0.3}$ & $-27.8 \mtiny{\pm 0.4}$ & $81.2 \mtiny{\pm 0.3}$ & $-2.4 \mtiny{\pm 0.2}$ & $82.1 \mtiny{\pm 0.4}$ & $-0.8 \mtiny{\pm 0.2}$ \\
\bottomrule
\end{tabular}
\endgroup
%\end{small}
\end{center}
\vskip -0.1in
\end{table*}




In Table~\ref{table:bwt_metrics_clip_full} we present results of the ACC and BWT on extra datasets plus the ones in the main paper. The results follow the same patterns as in section~\ref{sec:experiments} with LoRSU demonstrating the most consistent performance in both ACC and BWT compared to the other two baselines. SPU is close to \ours in terms of BWT but it significantly lacks behind in ACC.

\subsection{CLIP-based vs. Perplexity-based Updates+}
The detailed accuracies for all baselines and datasets used to create Table~\ref{table:ppl_vs_clip_summary} of the main paper can be found in Tables~\ref{table:fine_tune_llm_gtsrb} through~\ref{table:fine_tune_llm_eurosat}. We have also included results on fine-tuning the model using \emph{MMVP} dataset in Table~\ref{table:fine_tune_llm_mmvp}.
\begin{table}
\caption{Exact accuracy scores (\%) for each baseline used to fine-tune the model on the \emph{GTS} dataset under three different continual learning (5, 10, 50 shots)  settings. We include error bars over 3 runs.}
 \label{table:fine_tune_llm_gtsrb}
\vskip 0.15in
\begin{center}
\begin{small}
\begingroup
\setlength{\tabcolsep}{3.6pt}
\begin{tabular}{l c c c c c c c c c c c}
\toprule
 & & \multicolumn{10}{c}{\textbf{VQA Datasets (Acc \%)}}  \\
\cmidrule(lr){3-12}
\textbf{Setting} & \textbf{PEFT Method}  & \textbf{GTS} & \textbf{TSI} & \textbf{CAn} & \textbf{AIR} & \textbf{ESAT} & \textbf{DALLE} & \textbf{VSR} & \textbf{HM} & \textbf{MMVP} & \textbf{VisOnly} \\
\midrule
 & \textbf{Zr-Shot} & $75.6$ & $53.1$ & $82.7$ & $60.4$ & $76.1$ & $91.1$ & $51.5$ & $61.2$ & $58.0$ & $31.3$ \\
\midrule
\multirow{6}{*}{\textbf{CL-5}} & \textbf{LoRA-L} & $71.5 \mtiny{\pm 1.2}$ & $52.3 \mtiny{\pm 0.5}$ & $81.2 \mtiny{\pm 0.6}$ & $60.0 \mtiny{\pm 1.2}$ & $75.5 \mtiny{\pm 0.9}$ & $91.5 \mtiny{\pm 1.3}$ & $51.9 \mtiny{\pm 1.5}$ & $61.2 \mtiny{\pm 1.1}$ & $57.6 \mtiny{\pm 0.3}$ & $32.2 \mtiny{\pm 0.5}$ \\
& \textbf{LoRA} & $76.3 \mtiny{\pm 0.8}$ & $52.6 \mtiny{\pm 1.4}$ & $73.3 \mtiny{\pm 0.6}$ & $56.7 \mtiny{\pm 1.2}$ & $49.3 \mtiny{\pm 0.8}$ & $87.1 \mtiny{\pm 1.3}$ & $51.8 \mtiny{\pm 1.2}$ & $61.3 \mtiny{\pm 1.2}$ & $58.1 \mtiny{\pm 0.3}$ & $31.6 \mtiny{\pm 0.4}$ \\
& \textbf{LoRSU} & $82.0 \mtiny{\pm 1.3}$ & $53.5 \mtiny{\pm 1.3}$ & $82.4 \mtiny{\pm 0.8}$ & $60.8 \mtiny{\pm 1.4}$ & $66.6 \mtiny{\pm 0.9}$ & $91.5 \mtiny{\pm 1.4}$ & $51.6 \mtiny{\pm 0.7}$ & $61.7 \mtiny{\pm 1.4}$ & $59.8 \mtiny{\pm 0.2}$ & $31.6 \mtiny{\pm 0.2}$ \\
& \textbf{LoRA-Ppl} & $68.1 \mtiny{\pm 0.8}$ & $54.5 \mtiny{\pm 1.4}$ & $80.7 \mtiny{\pm 0.6}$ & $59.3 \mtiny{\pm 1.2}$ & $52.8 \mtiny{\pm 0.8}$ & $90.7 \mtiny{\pm 1.3}$ & $51.7 \mtiny{\pm 1.2}$ & $60.7 \mtiny{\pm 1.2}$ & $54.8 \mtiny{\pm 0.4}$ & $33.4 \mtiny{\pm 0.5}$ \\
& \textbf{LoRA-F} & $72.9 \mtiny{\pm 0.9}$ & $54.0 \mtiny{\pm 0.7}$ & $81.5 \mtiny{\pm 0.9}$ & $59.6 \mtiny{\pm 0.8}$ & $61.9 \mtiny{\pm 0.8}$ & $90.3 \mtiny{\pm 1.1}$ & $51.9 \mtiny{\pm 0.8}$ & $60.9 \mtiny{\pm 1.2}$ & $58.4 \mtiny{\pm 0.4}$ & $31.1 \mtiny{\pm 0.3}$ \\
& \textbf{LoRSU-Ppl} & $77.2 \mtiny{\pm 1.4}$ & $55.1 \mtiny{\pm 1.5}$ & $82.1 \mtiny{\pm 0.7}$ & $58.9 \mtiny{\pm 1.0}$ & $67.0 \mtiny{\pm 0.6}$ & $90.9 \mtiny{\pm 1.3}$ & $51.8 \mtiny{\pm 0.6}$ & $61.6 \mtiny{\pm 1.3}$ & $58.7 \mtiny{\pm 0.3}$ & $30.4 \mtiny{\pm 0.3}$ \\
\midrule
\multirow{6}{*}{\textbf{CL-20}} & \textbf{LoRA-L} & $74.2 \mtiny{\pm 0.9}$ & $52.2 \mtiny{\pm 0.9}$ & $82.1 \mtiny{\pm 0.5}$ & $59.6 \mtiny{\pm 1.0}$ & $75.9 \mtiny{\pm 0.6}$ & $91.8 \mtiny{\pm 1.0}$ & $51.6 \mtiny{\pm 0.4}$ & $62.1 \mtiny{\pm 0.9}$ & $59.1 \mtiny{\pm 0.2}$ & $31.8 \mtiny{\pm 0.2}$ \\
& \textbf{LoRA} & $78.1 \mtiny{\pm 0.8}$ & $55.6 \mtiny{\pm 0.3}$ & $59.0 \mtiny{\pm 0.9}$ & $47.6 \mtiny{\pm 0.4}$ & $26.0 \mtiny{\pm 0.6}$ & $83.6 \mtiny{\pm 0.8}$ & $52.1 \mtiny{\pm 0.5}$ & $62.1 \mtiny{\pm 1.0}$ & $53.7 \mtiny{\pm 0.3}$ & $30.8 \mtiny{\pm 0.2}$ \\
& \textbf{LoRSU} & $84.2 \mtiny{\pm 0.9}$ & $52.9 \mtiny{\pm 0.6}$ & $82.2 \mtiny{\pm 0.5}$ & $60.7 \mtiny{\pm 0.6}$ & $64.7 \mtiny{\pm 0.6}$ & $90.8 \mtiny{\pm 0.5}$ & $51.9 \mtiny{\pm 0.4}$ & $61.7 \mtiny{\pm 0.5}$ & $59.5 \mtiny{\pm 0.1}$ & $31.6 \mtiny{\pm 0.2}$ \\
& \textbf{LoRA-Ppl} & $75.1 \mtiny{\pm 0.9}$ & $50.4 \mtiny{\pm 0.9}$ & $75.8 \mtiny{\pm 0.4}$ & $56.5 \mtiny{\pm 0.3}$ & $40.1 \mtiny{\pm 0.9}$ & $89.7 \mtiny{\pm 0.8}$ & $51.6 \mtiny{\pm 0.7}$ & $57.8 \mtiny{\pm 0.8}$ & $54.2 \mtiny{\pm 0.2}$ & $31.5 \mtiny{\pm 0.4}$ \\
& \textbf{LoRA-F} & $74.2 \mtiny{\pm 0.8}$ & $52.7 \mtiny{\pm 0.3}$ & $80.1 \mtiny{\pm 0.9}$ & $59.5 \mtiny{\pm 0.4}$ & $66.0 \mtiny{\pm 0.6}$ & $90.1 \mtiny{\pm 0.8}$ & $52.1 \mtiny{\pm 0.5}$ & $64.7 \mtiny{\pm 1.0}$ & $60.4 \mtiny{\pm 0.4}$ & $32.3 \mtiny{\pm 0.2}$ \\
& \textbf{LoRSU-Ppl} & $79.5 \mtiny{\pm 0.8}$ & $56.1 \mtiny{\pm 0.5}$ & $82.1 \mtiny{\pm 0.9}$ & $59.8 \mtiny{\pm 0.4}$ & $66.1 \mtiny{\pm 0.4}$ & $90.8 \mtiny{\pm 1.0}$ & $51.7 \mtiny{\pm 0.5}$ & $62.1 \mtiny{\pm 0.6}$ & $59.0 \mtiny{\pm 0.3}$ & $31.5 \mtiny{\pm 0.3}$ \\
\midrule
\multirow{6}{*}{\textbf{CL-50}} & \textbf{LoRA-L} & $74.9 \mtiny{\pm 0.2}$ & $51.7 \mtiny{\pm 0.2}$ & $81.8 \mtiny{\pm 0.2}$ & $59.8 \mtiny{\pm 0.3}$ & $75.8 \mtiny{\pm 0.1}$ & $91.5 \mtiny{\pm 0.0}$ & $52.0 \mtiny{\pm 0.1}$ & $61.1 \mtiny{\pm 0.2}$ & $57.4 \mtiny{\pm 0.1}$ & $31.8 \mtiny{\pm 0.1}$ \\
& \textbf{LoRA} & $78.7 \mtiny{\pm 0.0}$ & $50.7 \mtiny{\pm 0.0}$ & $62.1 \mtiny{\pm 0.2}$ & $47.4 \mtiny{\pm 0.1}$ & $24.2 \mtiny{\pm 0.2}$ & $82.9 \mtiny{\pm 0.3}$ & $51.7 \mtiny{\pm 0.3}$ & $61.0 \mtiny{\pm 0.2}$ & $54.3 \mtiny{\pm 0.1}$ & $30.8 \mtiny{\pm 0.0}$ \\
& \textbf{LoRSU} & $85.3 \mtiny{\pm 0.1}$ & $54.2 \mtiny{\pm 0.1}$ & $81.9 \mtiny{\pm 0.2}$ & $60.5 \mtiny{\pm 0.2}$ & $61.4 \mtiny{\pm 0.3}$ & $91.0 \mtiny{\pm 0.1}$ & $51.7 \mtiny{\pm 0.2}$ & $62.2 \mtiny{\pm 0.4}$ & $58.9 \mtiny{\pm 0.1}$ & $31.8 \mtiny{\pm 0.1}$ \\
& \textbf{LoRA-Ppl} & $74.2 \mtiny{\pm 0.1}$ & $49.4 \mtiny{\pm 0.2}$ & $76.0 \mtiny{\pm 0.2}$ & $57.9 \mtiny{\pm 0.3}$ & $37.2 \mtiny{\pm 0.0}$ & $89.5 \mtiny{\pm 0.2}$ & $51.7 \mtiny{\pm 0.1}$ & $57.7 \mtiny{\pm 0.1}$ & $55.6 \mtiny{\pm 0.1}$ & $29.8 \mtiny{\pm 0.1}$ \\
& \textbf{LoRA-F} & $71.7 \mtiny{\pm 0.2}$ & $51.7 \mtiny{\pm 0.4}$ & $80.8 \mtiny{\pm 0.4}$ & $58.3 \mtiny{\pm 0.0}$ & $60.9 \mtiny{\pm 0.3}$ & $90.8 \mtiny{\pm 0.1}$ & $52.1 \mtiny{\pm 0.0}$ & $63.3 \mtiny{\pm 0.1}$ & $57.5 \mtiny{\pm 0.0}$ & $30.9 \mtiny{\pm 0.1}$ \\
& \textbf{LoRSU-Ppl} & $82.5 \mtiny{\pm 0.0}$ & $55.8 \mtiny{\pm 0.0}$ & $82.1 \mtiny{\pm 0.2}$ & $59.9 \mtiny{\pm 0.1}$ & $65.4 \mtiny{\pm 0.2}$ & $91.0 \mtiny{\pm 0.3}$ & $51.6 \mtiny{\pm 0.3}$ & $61.7 \mtiny{\pm 0.2}$ & $62.3 \mtiny{\pm 0.1}$ & $32.2 \mtiny{\pm 0.0}$ \\
\bottomrule
\end{tabular}
\endgroup
\end{small}
\end{center}
\vskip -0.1in
\end{table}


\begin{table}
\caption{Exact accuracy scores (\%) for each baseline used to fine-tune the model on the \emph{TSI} dataset under three different continual learning (5, 10, 50 shots)  settings. We include error bars over 3 runs.}
 \label{table:fine_tune_llm_tsi}
\vskip 0.15in
\begin{center}
\begin{small}
\begingroup
\setlength{\tabcolsep}{3.6pt}
\begin{tabular}{l c c c c c c c c c c c}
\toprule
 & & \multicolumn{10}{c}{\textbf{VQA Datasets (Acc \%)}}  \\
\cmidrule(lr){3-12}
\textbf{Setting} & \textbf{PEFT Method}  & \textbf{GTS} & \textbf{TSI} & \textbf{CAn} & \textbf{AIR} & \textbf{ESAT} & \textbf{DALLE} & \textbf{VSR} & \textbf{HM} & \textbf{MMVP} & \textbf{VisOnly} \\
\midrule
 & \textbf{Zr-Shot} & $75.6$ & $53.1$ & $82.7$ & $60.4$ & $76.1$ & $91.1$ & $51.5$ & $61.2$ & $58.0$ & $31.3$ \\
\midrule
\multirow{6}{*}{\textbf{CL-5}} & \textbf{LoRA-L} & $76.0 \mtiny{\pm 1.5}$ & $59.1 \mtiny{\pm 0.6}$ & $82.7 \mtiny{\pm 0.9}$ & $60.7 \mtiny{\pm 0.7}$ & $75.9 \mtiny{\pm 0.9}$ & $91.5 \mtiny{\pm 1.0}$ & $51.5 \mtiny{\pm 0.9}$ & $63.6 \mtiny{\pm 1.2}$ & $54.1 \mtiny{\pm 0.4}$ & $31.2 \mtiny{\pm 0.4}$ \\
& \textbf{LoRA} & $73.4 \mtiny{\pm 1.0}$ & $53.0 \mtiny{\pm 0.9}$ & $80.2 \mtiny{\pm 0.6}$ & $58.8 \mtiny{\pm 0.7}$ & $59.1 \mtiny{\pm 1.4}$ & $90.2 \mtiny{\pm 1.1}$ & $51.6 \mtiny{\pm 1.3}$ & $61.2 \mtiny{\pm 1.4}$ & $56.7 \mtiny{\pm 0.4}$ & $31.7 \mtiny{\pm 0.4}$ \\
& \textbf{LoRSU} & $75.9 \mtiny{\pm 0.9}$ & $56.3 \mtiny{\pm 0.7}$ & $82.7 \mtiny{\pm 0.9}$ & $60.8 \mtiny{\pm 1.0}$ & $76.2 \mtiny{\pm 1.4}$ & $91.3 \mtiny{\pm 1.2}$ & $51.6 \mtiny{\pm 0.9}$ & $61.7 \mtiny{\pm 0.8}$ & $57.7 \mtiny{\pm 0.3}$ & $31.2 \mtiny{\pm 0.3}$ \\
& \textbf{LoRA-Ppl} & $75.0 \mtiny{\pm 1.0}$ & $64.0 \mtiny{\pm 0.6}$ & $82.8 \mtiny{\pm 1.3}$ & $58.4 \mtiny{\pm 1.0}$ & $60.8 \mtiny{\pm 0.8}$ & $88.7 \mtiny{\pm 1.3}$ & $51.6 \mtiny{\pm 1.4}$ & $61.5 \mtiny{\pm 1.0}$ & $55.0 \mtiny{\pm 0.4}$ & $32.2 \mtiny{\pm 0.4}$ \\
& \textbf{LoRA-F} & $75.3 \mtiny{\pm 0.5}$ & $45.1 \mtiny{\pm 1.1}$ & $82.5 \mtiny{\pm 0.9}$ & $57.2 \mtiny{\pm 1.5}$ & $73.2 \mtiny{\pm 1.0}$ & $83.9 \mtiny{\pm 1.2}$ & $53.8 \mtiny{\pm 0.9}$ & $64.3 \mtiny{\pm 1.3}$ & $45.6 \mtiny{\pm 0.3}$ & $30.9 \mtiny{\pm 0.4}$ \\
& \textbf{LoRSU-Ppl} & $76.1 \mtiny{\pm 1.1}$ & $66.2 \mtiny{\pm 1.0}$ & $83.9 \mtiny{\pm 1.1}$ & $66.1 \mtiny{\pm 0.9}$ & $76.1 \mtiny{\pm 1.2}$ & $91.1 \mtiny{\pm 1.4}$ & $52.0 \mtiny{\pm 0.9}$ & $64.4 \mtiny{\pm 1.4}$ & $60.8 \mtiny{\pm 0.5}$ & $31.1 \mtiny{\pm 0.4}$ \\
\midrule
\multirow{6}{*}{\textbf{CL-20}} & \textbf{LoRA-L} & $76.1 \mtiny{\pm 0.7}$ & $59.0 \mtiny{\pm 0.6}$ & $82.4 \mtiny{\pm 0.4}$ & $60.8 \mtiny{\pm 0.4}$ & $75.7 \mtiny{\pm 0.9}$ & $91.3 \mtiny{\pm 0.7}$ & $51.5 \mtiny{\pm 0.9}$ & $63.9 \mtiny{\pm 1.0}$ & $55.4 \mtiny{\pm 0.3}$ & $30.8 \mtiny{\pm 0.3}$ \\
& \textbf{LoRA} & $68.5 \mtiny{\pm 0.7}$ & $61.6 \mtiny{\pm 0.3}$ & $76.7 \mtiny{\pm 0.9}$ & $55.3 \mtiny{\pm 0.7}$ & $55.6 \mtiny{\pm 0.6}$ & $88.8 \mtiny{\pm 0.8}$ & $51.9 \mtiny{\pm 0.3}$ & $61.4 \mtiny{\pm 0.6}$ & $59.1 \mtiny{\pm 0.3}$ & $31.1 \mtiny{\pm 0.3}$ \\
& \textbf{LoRSU} & $75.9 \mtiny{\pm 0.6}$ & $63.7 \mtiny{\pm 0.4}$ & $82.8 \mtiny{\pm 0.8}$ & $60.4 \mtiny{\pm 0.3}$ & $73.4 \mtiny{\pm 0.6}$ & $90.9 \mtiny{\pm 0.6}$ & $51.7 \mtiny{\pm 0.4}$ & $61.5 \mtiny{\pm 0.7}$ & $58.8 \mtiny{\pm 0.2}$ & $31.9 \mtiny{\pm 0.2}$ \\
& \textbf{LoRA-Ppl} & $62.1 \mtiny{\pm 0.6}$ & $59.6 \mtiny{\pm 0.5}$ & $71.9 \mtiny{\pm 0.6}$ & $48.3 \mtiny{\pm 0.7}$ & $42.5 \mtiny{\pm 1.0}$ & $75.8 \mtiny{\pm 0.8}$ & $51.6 \mtiny{\pm 0.6}$ & $49.0 \mtiny{\pm 0.5}$ & $49.7 \mtiny{\pm 0.3}$ & $32.4 \mtiny{\pm 0.2}$ \\
& \textbf{LoRA-F} & $76.1 \mtiny{\pm 0.5}$ & $56.0 \mtiny{\pm 0.5}$ & $82.8 \mtiny{\pm 0.9}$ & $58.2 \mtiny{\pm 0.4}$ & $67.7 \mtiny{\pm 0.9}$ & $87.5 \mtiny{\pm 0.8}$ & $51.6 \mtiny{\pm 0.8}$ & $64.4 \mtiny{\pm 0.5}$ & $40.3 \mtiny{\pm 0.4}$ & $31.2 \mtiny{\pm 0.2}$ \\
& \textbf{LoRSU-Ppl} & $76.4 \mtiny{\pm 0.7}$ & $67.0 \mtiny{\pm 0.4}$ & $83.0 \mtiny{\pm 0.7}$ & $57.4 \mtiny{\pm 0.4}$ & $74.0 \mtiny{\pm 0.8}$ & $88.1 \mtiny{\pm 0.3}$ & $51.8 \mtiny{\pm 0.6}$ & $63.6 \mtiny{\pm 0.5}$ & $57.6 \mtiny{\pm 0.2}$ & $30.8 \mtiny{\pm 0.3}$ \\
\midrule
\multirow{6}{*}{\textbf{CL-50}} & \textbf{LoRA-L} & $76.4 \mtiny{\pm 0.2}$ & $63.0 \mtiny{\pm 0.2}$ & $81.9 \mtiny{\pm 0.2}$ & $60.5 \mtiny{\pm 0.2}$ & $75.6 \mtiny{\pm 0.2}$ & $91.1 \mtiny{\pm 0.2}$ & $51.7 \mtiny{\pm 0.2}$ & $64.1 \mtiny{\pm 0.3}$ & $55.6 \mtiny{\pm 0.2}$ & $30.9 \mtiny{\pm 0.0}$ \\
& \textbf{LoRA} & $66.1 \mtiny{\pm 0.2}$ & $71.3 \mtiny{\pm 0.3}$ & $76.0 \mtiny{\pm 0.1}$ & $56.0 \mtiny{\pm 0.1}$ & $44.5 \mtiny{\pm 0.2}$ & $88.9 \mtiny{\pm 0.3}$ & $51.8 \mtiny{\pm 0.1}$ & $60.4 \mtiny{\pm 0.2}$ & $56.3 \mtiny{\pm 0.1}$ & $31.6 \mtiny{\pm 0.1}$ \\
& \textbf{LoRSU} & $75.3 \mtiny{\pm 0.2}$ & $72.2 \mtiny{\pm 0.4}$ & $82.4 \mtiny{\pm 0.3}$ & $59.7 \mtiny{\pm 0.3}$ & $72.5 \mtiny{\pm 0.3}$ & $90.8 \mtiny{\pm 0.3}$ & $51.7 \mtiny{\pm 0.2}$ & $61.7 \mtiny{\pm 0.4}$ & $58.5 \mtiny{\pm 0.1}$ & $31.7 \mtiny{\pm 0.0}$ \\
& \textbf{LoRA-Ppl} & $46.3 \mtiny{\pm 0.3}$ & $51.5 \mtiny{\pm 0.3}$ & $63.4 \mtiny{\pm 0.1}$ & $40.1 \mtiny{\pm 0.1}$ & $41.3 \mtiny{\pm 0.4}$ & $73.9 \mtiny{\pm 0.2}$ & $51.7 \mtiny{\pm 0.3}$ & $49.5 \mtiny{\pm 0.3}$ & $40.2 \mtiny{\pm 0.1}$ & $32.7 \mtiny{\pm 0.1}$ \\
& \textbf{LoRA-F} & $74.0 \mtiny{\pm 0.2}$ & $68.2 \mtiny{\pm 0.1}$ & $81.6 \mtiny{\pm 0.3}$ & $59.2 \mtiny{\pm 0.0}$ & $75.1 \mtiny{\pm 0.2}$ & $88.5 \mtiny{\pm 0.2}$ & $56.8 \mtiny{\pm 0.1}$ & $65.0 \mtiny{\pm 0.3}$ & $50.8 \mtiny{\pm 0.1}$ & $30.4 \mtiny{\pm 0.1}$ \\
& \textbf{LoRSU-Ppl} & $75.8 \mtiny{\pm 0.2}$ & $75.1 \mtiny{\pm 0.2}$ & $82.1 \mtiny{\pm 0.3}$ & $56.0 \mtiny{\pm 0.4}$ & $74.2 \mtiny{\pm 0.4}$ & $86.0 \mtiny{\pm 0.4}$ & $52.0 \mtiny{\pm 0.0}$ & $63.2 \mtiny{\pm 0.0}$ & $58.1 \mtiny{\pm 0.1}$ & $30.2 \mtiny{\pm 0.1}$ \\
\bottomrule
\end{tabular}
\endgroup
\end{small}
\end{center}
\vskip -0.1in
\end{table}


\begin{table}
\caption{Exact accuracy scores (\%) for each baseline used to fine-tune the model on the \emph{CAn} dataset under three different continual learning (5, 10, 50 shots)  settings. We include error bars over 3 runs.}
 \label{table:fine_tune_llm_counteranimal}
\vskip 0.15in
\begin{center}
\begin{small}
\begingroup
\setlength{\tabcolsep}{3.6pt}
\begin{tabular}{l c c c c c c c c c c c}
\toprule
 & & \multicolumn{10}{c}{\textbf{VQA Datasets (Acc \%)}}  \\
\cmidrule(lr){3-12}
\textbf{Setting} & \textbf{PEFT Method}  & \textbf{GTS} & \textbf{TSI} & \textbf{CAn} & \textbf{AIR} & \textbf{ESAT} & \textbf{DALLE} & \textbf{VSR} & \textbf{HM} & \textbf{MMVP} & \textbf{VisOnly} \\
\midrule
 & \textbf{Zr-Shot} & $75.6$ & $53.1$ & $82.7$ & $60.4$ & $76.1$ & $91.1$ & $51.5$ & $61.2$ & $58.0$ & $31.3$ \\
\midrule
\multirow{6}{*}{\textbf{CL-5}} & \textbf{LoRA-L} & $75.5 \mtiny{\pm 1.4}$ & $53.1 \mtiny{\pm 0.8}$ & $79.4 \mtiny{\pm 1.4}$ & $59.2 \mtiny{\pm 0.6}$ & $75.2 \mtiny{\pm 0.9}$ & $91.5 \mtiny{\pm 1.1}$ & $52.4 \mtiny{\pm 1.3}$ & $60.2 \mtiny{\pm 1.1}$ & $57.7 \mtiny{\pm 0.5}$ & $32.1 \mtiny{\pm 0.3}$ \\
& \textbf{LoRA} & $69.7 \mtiny{\pm 1.4}$ & $44.8 \mtiny{\pm 1.1}$ & $81.4 \mtiny{\pm 0.7}$ & $56.9 \mtiny{\pm 1.0}$ & $50.7 \mtiny{\pm 1.3}$ & $92.9 \mtiny{\pm 1.3}$ & $52.0 \mtiny{\pm 1.0}$ & $61.8 \mtiny{\pm 1.5}$ & $56.5 \mtiny{\pm 0.4}$ & $31.3 \mtiny{\pm 0.4}$ \\
& \textbf{LoRSU} & $75.2 \mtiny{\pm 0.8}$ & $52.7 \mtiny{\pm 0.9}$ & $83.0 \mtiny{\pm 1.0}$ & $60.1 \mtiny{\pm 0.7}$ & $76.8 \mtiny{\pm 1.0}$ & $91.8 \mtiny{\pm 1.4}$ & $51.6 \mtiny{\pm 1.1}$ & $62.3 \mtiny{\pm 1.2}$ & $58.7 \mtiny{\pm 0.3}$ & $31.4 \mtiny{\pm 0.4}$ \\
& \textbf{LoRA-Ppl} & $65.8 \mtiny{\pm 1.1}$ & $50.7 \mtiny{\pm 0.6}$ & $79.2 \mtiny{\pm 0.5}$ & $48.4 \mtiny{\pm 1.4}$ & $63.0 \mtiny{\pm 1.2}$ & $86.7 \mtiny{\pm 1.3}$ & $51.8 \mtiny{\pm 1.0}$ & $57.2 \mtiny{\pm 1.4}$ & $52.5 \mtiny{\pm 0.3}$ & $32.4 \mtiny{\pm 0.4}$ \\
& \textbf{LoRA-F} & $70.1 \mtiny{\pm 0.6}$ & $52.2 \mtiny{\pm 0.7}$ & $78.6 \mtiny{\pm 0.7}$ & $50.9 \mtiny{\pm 0.9}$ & $73.4 \mtiny{\pm 0.8}$ & $91.3 \mtiny{\pm 1.0}$ & $54.7 \mtiny{\pm 0.8}$ & $62.2 \mtiny{\pm 1.4}$ & $58.0 \mtiny{\pm 0.5}$ & $31.3 \mtiny{\pm 0.3}$ \\
& \textbf{LoRSU-Ppl} & $74.6 \mtiny{\pm 0.9}$ & $51.3 \mtiny{\pm 1.4}$ & $82.9 \mtiny{\pm 1.2}$ & $58.4 \mtiny{\pm 1.2}$ & $77.7 \mtiny{\pm 1.2}$ & $91.8 \mtiny{\pm 1.3}$ & $51.5 \mtiny{\pm 1.1}$ & $64.7 \mtiny{\pm 1.4}$ & $56.5 \mtiny{\pm 0.6}$ & $29.8 \mtiny{\pm 0.3}$ \\
\midrule
\multirow{6}{*}{\textbf{CL-20}} & \textbf{LoRA-L} & $73.6 \mtiny{\pm 1.0}$ & $52.2 \mtiny{\pm 0.9}$ & $80.8 \mtiny{\pm 0.9}$ & $56.7 \mtiny{\pm 0.4}$ & $74.7 \mtiny{\pm 0.8}$ & $91.7 \mtiny{\pm 0.5}$ & $52.2 \mtiny{\pm 0.6}$ & $60.9 \mtiny{\pm 0.8}$ & $59.1 \mtiny{\pm 0.3}$ & $31.9 \mtiny{\pm 0.4}$ \\
& \textbf{LoRA} & $67.5 \mtiny{\pm 0.6}$ & $48.9 \mtiny{\pm 0.6}$ & $80.4 \mtiny{\pm 0.4}$ & $57.3 \mtiny{\pm 0.9}$ & $39.7 \mtiny{\pm 0.4}$ & $91.1 \mtiny{\pm 0.6}$ & $51.8 \mtiny{\pm 0.9}$ & $61.7 \mtiny{\pm 0.3}$ & $60.1 \mtiny{\pm 0.2}$ & $31.9 \mtiny{\pm 0.3}$ \\
& \textbf{LoRSU} & $75.3 \mtiny{\pm 0.8}$ & $53.1 \mtiny{\pm 0.9}$ & $83.8 \mtiny{\pm 0.9}$ & $58.8 \mtiny{\pm 1.0}$ & $75.5 \mtiny{\pm 0.7}$ & $92.0 \mtiny{\pm 0.3}$ & $51.9 \mtiny{\pm 0.4}$ & $62.3 \mtiny{\pm 0.6}$ & $60.4 \mtiny{\pm 0.2}$ & $31.6 \mtiny{\pm 0.2}$ \\
& \textbf{LoRA-Ppl} & $65.6 \mtiny{\pm 0.9}$ & $47.0 \mtiny{\pm 0.7}$ & $79.0 \mtiny{\pm 0.4}$ & $46.0 \mtiny{\pm 0.6}$ & $58.9 \mtiny{\pm 0.8}$ & $82.5 \mtiny{\pm 0.8}$ & $51.9 \mtiny{\pm 0.7}$ & $43.9 \mtiny{\pm 1.0}$ & $52.5 \mtiny{\pm 0.4}$ & $30.4 \mtiny{\pm 0.3}$ \\
& \textbf{LoRA-F} & $69.4 \mtiny{\pm 0.9}$ & $54.9 \mtiny{\pm 0.4}$ & $80.6 \mtiny{\pm 0.4}$ & $50.4 \mtiny{\pm 0.5}$ & $72.0 \mtiny{\pm 0.8}$ & $91.2 \mtiny{\pm 0.5}$ & $51.9 \mtiny{\pm 0.9}$ & $64.3 \mtiny{\pm 1.0}$ & $57.0 \mtiny{\pm 0.3}$ & $31.6 \mtiny{\pm 0.3}$ \\
& \textbf{LoRSU-Ppl} & $72.4 \mtiny{\pm 0.6}$ & $49.2 \mtiny{\pm 0.4}$ & $83.2 \mtiny{\pm 0.7}$ & $56.4 \mtiny{\pm 0.9}$ & $75.5 \mtiny{\pm 0.6}$ & $91.8 \mtiny{\pm 0.9}$ & $51.6 \mtiny{\pm 0.5}$ & $61.0 \mtiny{\pm 0.8}$ & $57.7 \mtiny{\pm 0.3}$ & $31.6 \mtiny{\pm 0.3}$ \\
\midrule
\multirow{6}{*}{\textbf{CL-50}} & \textbf{LoRA-L} & $73.8 \mtiny{\pm 0.1}$ & $51.6 \mtiny{\pm 0.2}$ & $80.9 \mtiny{\pm 0.2}$ & $56.9 \mtiny{\pm 0.1}$ & $74.9 \mtiny{\pm 0.2}$ & $91.3 \mtiny{\pm 0.3}$ & $51.7 \mtiny{\pm 0.2}$ & $61.2 \mtiny{\pm 0.3}$ & $58.0 \mtiny{\pm 0.1}$ & $32.4 \mtiny{\pm 0.1}$ \\
& \textbf{LoRA} & $66.8 \mtiny{\pm 0.2}$ & $47.8 \mtiny{\pm 0.3}$ & $82.3 \mtiny{\pm 0.2}$ & $55.7 \mtiny{\pm 0.0}$ & $52.0 \mtiny{\pm 0.3}$ & $91.0 \mtiny{\pm 0.3}$ & $51.7 \mtiny{\pm 0.3}$ & $61.6 \mtiny{\pm 0.2}$ & $60.2 \mtiny{\pm 0.0}$ & $31.6 \mtiny{\pm 0.1}$ \\
& \textbf{LoRSU} & $75.0 \mtiny{\pm 0.2}$ & $51.8 \mtiny{\pm 0.1}$ & $84.0 \mtiny{\pm 0.4}$ & $58.5 \mtiny{\pm 0.2}$ & $72.7 \mtiny{\pm 0.3}$ & $91.9 \mtiny{\pm 0.3}$ & $51.7 \mtiny{\pm 0.1}$ & $62.3 \mtiny{\pm 0.4}$ & $58.1 \mtiny{\pm 0.0}$ & $31.7 \mtiny{\pm 0.1}$ \\
& \textbf{LoRA-Ppl} & $56.2 \mtiny{\pm 0.4}$ & $36.4 \mtiny{\pm 0.0}$ & $80.9 \mtiny{\pm 0.1}$ & $48.5 \mtiny{\pm 0.3}$ & $54.1 \mtiny{\pm 0.3}$ & $78.1 \mtiny{\pm 0.2}$ & $53.6 \mtiny{\pm 0.4}$ & $62.3 \mtiny{\pm 0.3}$ & $48.4 \mtiny{\pm 0.1}$ & $32.4 \mtiny{\pm 0.1}$ \\
& \textbf{LoRA-F} & $69.2 \mtiny{\pm 0.2}$ & $52.0 \mtiny{\pm 0.2}$ & $80.6 \mtiny{\pm 0.1}$ & $53.7 \mtiny{\pm 0.3}$ & $74.4 \mtiny{\pm 0.1}$ & $90.7 \mtiny{\pm 0.2}$ & $51.8 \mtiny{\pm 0.4}$ & $66.5 \mtiny{\pm 0.0}$ & $58.7 \mtiny{\pm 0.1}$ & $31.4 \mtiny{\pm 0.1}$ \\
& \textbf{LoRSU-Ppl} & $74.9 \mtiny{\pm 0.4}$ & $49.7 \mtiny{\pm 0.4}$ & $83.7 \mtiny{\pm 0.0}$ & $42.5 \mtiny{\pm 0.4}$ & $74.9 \mtiny{\pm 0.2}$ & $91.2 \mtiny{\pm 0.3}$ & $51.2 \mtiny{\pm 0.3}$ & $52.2 \mtiny{\pm 0.4}$ & $58.5 \mtiny{\pm 0.2}$ & $32.3 \mtiny{\pm 0.2}$ \\
\bottomrule
\end{tabular}
\endgroup
\end{small}
\end{center}
\vskip -0.1in
\end{table}


\begin{table}
\caption{Exact accuracy scores (\%) for each baseline used to fine-tune the model on the \emph{AIR} dataset under three different continual learning (5, 10, 50 shots)  settings. We include error bars over 3 runs.}
 \label{table:fine_tune_llm_aircraft}
 \vskip 0.15in
\begin{center}
\begin{small}
\begingroup
\setlength{\tabcolsep}{3.6pt}
\begin{tabular}{l c c c c c c c c c c c}
\toprule
 & & \multicolumn{10}{c}{\textbf{VQA Datasets (Acc \%)}}  \\
\cmidrule(lr){3-12}
\textbf{Setting} & \textbf{PEFT Method}  & \textbf{GTS} & \textbf{TSI} & \textbf{CAn} & \textbf{AIR} & \textbf{ESAT} & \textbf{DALLE} & \textbf{VSR} & \textbf{HM} & \textbf{MMVP} & \textbf{VisOnly} \\
\midrule
 & \textbf{Zr-Shot} & $75.6$ & $53.1$ & $82.7$ & $60.4$ & $76.1$ & $91.1$ & $51.5$ & $61.2$ & $58.0$ & $31.3$ \\
\midrule
\multirow{6}{*}{\textbf{CL-5}} & \textbf{LoRA-L} & $75.6 \mtiny{\pm 0.7}$ & $54.4 \mtiny{\pm 0.5}$ & $81.8 \mtiny{\pm 1.1}$ & $58.7 \mtiny{\pm 0.9}$ & $75.7 \mtiny{\pm 1.4}$ & $92.0 \mtiny{\pm 1.4}$ & $51.6 \mtiny{\pm 0.9}$ & $61.0 \mtiny{\pm 0.6}$ & $59.1 \mtiny{\pm 0.3}$ & $32.2 \mtiny{\pm 0.5}$ \\
& \textbf{LoRA} & $70.9 \mtiny{\pm 0.9}$ & $52.7 \mtiny{\pm 0.6}$ & $79.0 \mtiny{\pm 0.7}$ & $61.7 \mtiny{\pm 0.5}$ & $48.8 \mtiny{\pm 0.7}$ & $90.6 \mtiny{\pm 0.6}$ & $52.0 \mtiny{\pm 0.9}$ & $62.5 \mtiny{\pm 0.8}$ & $60.0 \mtiny{\pm 0.3}$ & $31.1 \mtiny{\pm 0.2}$ \\
& \textbf{LoRSU} & $76.2 \mtiny{\pm 0.8}$ & $53.4 \mtiny{\pm 1.4}$ & $82.5 \mtiny{\pm 1.0}$ & $65.2 \mtiny{\pm 1.3}$ & $76.0 \mtiny{\pm 0.9}$ & $91.8 \mtiny{\pm 0.8}$ & $51.6 \mtiny{\pm 0.8}$ & $62.1 \mtiny{\pm 1.1}$ & $59.0 \mtiny{\pm 0.4}$ & $31.2 \mtiny{\pm 0.3}$ \\
& \textbf{LoRA-Ppl} & $74.9 \mtiny{\pm 0.8}$ & $54.2 \mtiny{\pm 1.2}$ & $79.1 \mtiny{\pm 0.5}$ & $59.7 \mtiny{\pm 0.9}$ & $68.5 \mtiny{\pm 0.9}$ & $90.8 \mtiny{\pm 1.3}$ & $51.8 \mtiny{\pm 0.7}$ & $62.0 \mtiny{\pm 0.7}$ & $55.1 \mtiny{\pm 0.5}$ & $31.1 \mtiny{\pm 0.5}$ \\
& \textbf{LoRA-F} & $72.3 \mtiny{\pm 0.5}$ & $50.6 \mtiny{\pm 1.3}$ & $78.7 \mtiny{\pm 1.4}$ & $70.0 \mtiny{\pm 1.3}$ & $64.4 \mtiny{\pm 0.9}$ & $90.9 \mtiny{\pm 0.6}$ & $54.9 \mtiny{\pm 1.3}$ & $57.7 \mtiny{\pm 1.1}$ & $62.0 \mtiny{\pm 0.6}$ & $32.2 \mtiny{\pm 0.5}$ \\
& \textbf{LoRSU-Ppl} & $75.6 \mtiny{\pm 1.0}$ & $54.6 \mtiny{\pm 1.2}$ & $79.8 \mtiny{\pm 1.0}$ & $66.2 \mtiny{\pm 0.5}$ & $76.4 \mtiny{\pm 1.1}$ & $90.6 \mtiny{\pm 1.3}$ & $51.7 \mtiny{\pm 1.3}$ & $60.1 \mtiny{\pm 0.9}$ & $58.8 \mtiny{\pm 0.4}$ & $31.1 \mtiny{\pm 0.4}$ \\
\midrule
\multirow{6}{*}{\textbf{CL-20}} & \textbf{LoRA-L} & $75.4 \mtiny{\pm 0.3}$ & $53.6 \mtiny{\pm 0.4}$ & $82.2 \mtiny{\pm 1.0}$ & $64.1 \mtiny{\pm 1.0}$ & $75.7 \mtiny{\pm 0.5}$ & $92.2 \mtiny{\pm 0.3}$ & $51.5 \mtiny{\pm 0.5}$ & $61.5 \mtiny{\pm 0.8}$ & $58.9 \mtiny{\pm 0.2}$ & $31.9 \mtiny{\pm 0.3}$ \\
& \textbf{LoRA} & $71.8 \mtiny{\pm 0.9}$ & $51.1 \mtiny{\pm 0.8}$ & $78.6 \mtiny{\pm 0.3}$ & $65.7 \mtiny{\pm 0.4}$ & $63.4 \mtiny{\pm 0.8}$ & $89.9 \mtiny{\pm 1.0}$ & $51.7 \mtiny{\pm 0.3}$ & $62.3 \mtiny{\pm 0.3}$ & $56.2 \mtiny{\pm 0.2}$ & $31.5 \mtiny{\pm 0.2}$ \\
& \textbf{LoRSU} & $75.7 \mtiny{\pm 0.9}$ & $52.6 \mtiny{\pm 0.9}$ & $81.4 \mtiny{\pm 0.7}$ & $66.3 \mtiny{\pm 0.7}$ & $73.0 \mtiny{\pm 0.8}$ & $90.9 \mtiny{\pm 0.8}$ & $51.9 \mtiny{\pm 0.8}$ & $61.8 \mtiny{\pm 0.8}$ & $56.9 \mtiny{\pm 0.1}$ & $31.6 \mtiny{\pm 0.3}$ \\
& \textbf{LoRA-Ppl} & $72.1 \mtiny{\pm 0.5}$ & $48.0 \mtiny{\pm 0.8}$ & $72.7 \mtiny{\pm 0.4}$ & $65.2 \mtiny{\pm 1.0}$ & $65.1 \mtiny{\pm 0.5}$ & $90.4 \mtiny{\pm 0.3}$ & $51.8 \mtiny{\pm 0.6}$ & $61.5 \mtiny{\pm 0.8}$ & $55.8 \mtiny{\pm 0.1}$ & $31.7 \mtiny{\pm 0.1}$ \\
& \textbf{LoRA-F} & $74.5 \mtiny{\pm 0.8}$ & $53.0 \mtiny{\pm 0.3}$ & $82.0 \mtiny{\pm 0.6}$ & $76.7 \mtiny{\pm 0.6}$ & $74.9 \mtiny{\pm 0.9}$ & $91.1 \mtiny{\pm 0.3}$ & $52.4 \mtiny{\pm 0.6}$ & $59.3 \mtiny{\pm 0.8}$ & $59.6 \mtiny{\pm 0.4}$ & $31.3 \mtiny{\pm 0.3}$ \\
& \textbf{LoRSU-Ppl} & $76.1 \mtiny{\pm 0.8}$ & $55.5 \mtiny{\pm 0.5}$ & $78.7 \mtiny{\pm 0.8}$ & $66.4 \mtiny{\pm 0.6}$ & $75.7 \mtiny{\pm 0.6}$ & $91.6 \mtiny{\pm 1.0}$ & $51.5 \mtiny{\pm 0.3}$ & $59.8 \mtiny{\pm 0.5}$ & $58.1 \mtiny{\pm 0.4}$ & $31.2 \mtiny{\pm 0.4}$ \\
\midrule
\multirow{6}{*}{\textbf{CL-50}} & \textbf{LoRA-L} & $75.6 \mtiny{\pm 0.2}$ & $53.8 \mtiny{\pm 0.1}$ & $83.5 \mtiny{\pm 0.1}$ & $65.0 \mtiny{\pm 0.0}$ & $75.7 \mtiny{\pm 0.1}$ & $92.0 \mtiny{\pm 0.0}$ & $51.8 \mtiny{\pm 0.2}$ & $61.1 \mtiny{\pm 0.1}$ & $58.7 \mtiny{\pm 0.1}$ & $32.3 \mtiny{\pm 0.0}$ \\
& \textbf{LoRA} & $69.8 \mtiny{\pm 0.0}$ & $54.7 \mtiny{\pm 0.0}$ & $77.0 \mtiny{\pm 0.3}$ & $68.2 \mtiny{\pm 0.3}$ & $51.6 \mtiny{\pm 0.1}$ & $90.0 \mtiny{\pm 0.1}$ & $52.0 \mtiny{\pm 0.4}$ & $62.4 \mtiny{\pm 0.0}$ & $57.1 \mtiny{\pm 0.1}$ & $31.5 \mtiny{\pm 0.1}$ \\
& \textbf{LoRSU} & $75.4 \mtiny{\pm 0.4}$ & $52.7 \mtiny{\pm 0.3}$ & $81.6 \mtiny{\pm 0.2}$ & $68.6 \mtiny{\pm 0.3}$ & $69.7 \mtiny{\pm 0.3}$ & $91.5 \mtiny{\pm 0.2}$ & $51.7 \mtiny{\pm 0.4}$ & $62.2 \mtiny{\pm 0.1}$ & $58.7 \mtiny{\pm 0.1}$ & $31.1 \mtiny{\pm 0.1}$ \\
& \textbf{LoRA-Ppl} & $74.4 \mtiny{\pm 0.1}$ & $50.9 \mtiny{\pm 0.4}$ & $76.8 \mtiny{\pm 0.2}$ & $66.6 \mtiny{\pm 0.3}$ & $65.4 \mtiny{\pm 0.2}$ & $91.3 \mtiny{\pm 0.1}$ & $51.6 \mtiny{\pm 0.1}$ & $57.2 \mtiny{\pm 0.2}$ & $53.7 \mtiny{\pm 0.1}$ & $31.5 \mtiny{\pm 0.1}$ \\
& \textbf{LoRA-F} & $74.6 \mtiny{\pm 0.3}$ & $53.2 \mtiny{\pm 0.2}$ & $80.7 \mtiny{\pm 0.4}$ & $78.3 \mtiny{\pm 0.1}$ & $71.4 \mtiny{\pm 0.2}$ & $91.4 \mtiny{\pm 0.0}$ & $52.9 \mtiny{\pm 0.4}$ & $60.0 \mtiny{\pm 0.2}$ & $57.4 \mtiny{\pm 0.0}$ & $31.1 \mtiny{\pm 0.2}$ \\
& \textbf{LoRSU-Ppl} & $75.1 \mtiny{\pm 0.2}$ & $54.5 \mtiny{\pm 0.1}$ & $78.0 \mtiny{\pm 0.4}$ & $69.3 \mtiny{\pm 0.1}$ & $75.7 \mtiny{\pm 0.1}$ & $91.5 \mtiny{\pm 0.1}$ & $51.7 \mtiny{\pm 0.0}$ & $61.5 \mtiny{\pm 0.1}$ & $58.2 \mtiny{\pm 0.0}$ & $30.8 \mtiny{\pm 0.0}$ \\
\bottomrule
\end{tabular}
\endgroup
\end{small}
\end{center}
\vskip -0.1in
\end{table}


\begin{table}
\caption{Exact accuracy scores (\%) for each baseline used to fine-tune the model on the \emph{ESAT} dataset under three different continual learning (5, 10, 50 shots)  settings. We include error bars over 3 runs.}
 \label{table:fine_tune_llm_eurosat}
 \vskip 0.15in
\begin{center}
\begin{small}
\begingroup
\setlength{\tabcolsep}{3.6pt}
\begin{tabular}{l c c c c c c c c c c c}
\toprule
 & & \multicolumn{10}{c}{\textbf{VQA Datasets (Acc \%)}}  \\
\cmidrule(lr){3-12}
\textbf{Setting} & \textbf{PEFT Method}  & \textbf{GTS} & \textbf{TSI} & \textbf{CAn} & \textbf{AIR} & \textbf{ESAT} & \textbf{DALLE} & \textbf{VSR} & \textbf{HM} & \textbf{MMVP} & \textbf{VisOnly} \\
\midrule
 & \textbf{Zr-Shot} & $75.6$ & $53.1$ & $82.7$ & $60.4$ & $76.1$ & $91.1$ & $51.5$ & $61.2$ & $58.0$ & $31.3$ \\
\midrule
\multirow{6}{*}{\textbf{CL-5}} & \textbf{LoRA-L} & $75.4 \mtiny{\pm 0.7}$ & $52.2 \mtiny{\pm 1.4}$ & $82.8 \mtiny{\pm 0.6}$ & $60.6 \mtiny{\pm 1.5}$ & $75.9 \mtiny{\pm 1.1}$ & $91.7 \mtiny{\pm 0.9}$ & $51.5 \mtiny{\pm 0.6}$ & $60.2 \mtiny{\pm 0.8}$ & $57.6 \mtiny{\pm 0.5}$ & $31.6 \mtiny{\pm 0.4}$ \\
& \textbf{LoRA} & $73.2 \mtiny{\pm 1.3}$ & $49.3 \mtiny{\pm 1.2}$ & $80.6 \mtiny{\pm 0.9}$ & $60.4 \mtiny{\pm 1.1}$ & $74.5 \mtiny{\pm 0.8}$ & $92.3 \mtiny{\pm 1.3}$ & $52.0 \mtiny{\pm 1.1}$ & $61.6 \mtiny{\pm 1.1}$ & $57.4 \mtiny{\pm 0.4}$ & $31.4 \mtiny{\pm 0.3}$ \\
& \textbf{LoRSU} & $76.2 \mtiny{\pm 1.0}$ & $53.6 \mtiny{\pm 1.1}$ & $82.5 \mtiny{\pm 1.2}$ & $60.8 \mtiny{\pm 0.8}$ & $82.9 \mtiny{\pm 1.0}$ & $91.5 \mtiny{\pm 0.9}$ & $51.6 \mtiny{\pm 0.9}$ & $61.3 \mtiny{\pm 0.7}$ & $57.7 \mtiny{\pm 0.4}$ & $31.9 \mtiny{\pm 0.4}$ \\
& \textbf{LoRA-Ppl} & $76.0 \mtiny{\pm 0.7}$ & $52.6 \mtiny{\pm 1.0}$ & $82.6 \mtiny{\pm 1.3}$ & $60.4 \mtiny{\pm 1.4}$ & $75.5 \mtiny{\pm 0.9}$ & $91.9 \mtiny{\pm 1.0}$ & $51.8 \mtiny{\pm 0.9}$ & $62.8 \mtiny{\pm 0.8}$ & $59.0 \mtiny{\pm 0.4}$ & $31.6 \mtiny{\pm 0.5}$ \\
& \textbf{LoRA-F} & $74.3 \mtiny{\pm 1.3}$ & $51.5 \mtiny{\pm 1.4}$ & $81.1 \mtiny{\pm 1.0}$ & $60.3 \mtiny{\pm 1.1}$ & $81.5 \mtiny{\pm 1.2}$ & $90.8 \mtiny{\pm 1.2}$ & $51.9 \mtiny{\pm 1.2}$ & $61.9 \mtiny{\pm 1.2}$ & $57.7 \mtiny{\pm 0.2}$ & $31.3 \mtiny{\pm 0.5}$ \\
& \textbf{LoRSU-Ppl} & $75.6 \mtiny{\pm 1.4}$ & $52.3 \mtiny{\pm 0.6}$ & $82.0 \mtiny{\pm 1.2}$ & $60.5 \mtiny{\pm 1.0}$ & $79.8 \mtiny{\pm 1.1}$ & $92.3 \mtiny{\pm 0.5}$ & $51.8 \mtiny{\pm 1.2}$ & $62.2 \mtiny{\pm 1.4}$ & $57.7 \mtiny{\pm 0.4}$ & $31.3 \mtiny{\pm 0.4}$ \\
\midrule
\multirow{6}{*}{\textbf{CL-20}} & \textbf{LoRA-L} & $75.9 \mtiny{\pm 0.8}$ & $52.4 \mtiny{\pm 0.9}$ & $82.7 \mtiny{\pm 0.7}$ & $60.8 \mtiny{\pm 1.0}$ & $76.8 \mtiny{\pm 0.3}$ & $91.3 \mtiny{\pm 0.5}$ & $51.7 \mtiny{\pm 0.5}$ & $60.4 \mtiny{\pm 0.9}$ & $61.5 \mtiny{\pm 0.3}$ & $31.6 \mtiny{\pm 0.3}$ \\
& \textbf{LoRA} & $71.1 \mtiny{\pm 0.7}$ & $50.9 \mtiny{\pm 0.5}$ & $80.3 \mtiny{\pm 1.0}$ & $59.4 \mtiny{\pm 0.7}$ & $64.6 \mtiny{\pm 0.7}$ & $91.1 \mtiny{\pm 0.7}$ & $52.0 \mtiny{\pm 0.4}$ & $62.3 \mtiny{\pm 0.6}$ & $62.3 \mtiny{\pm 0.2}$ & $31.3 \mtiny{\pm 0.1}$ \\
& \textbf{LoRSU} & $75.3 \mtiny{\pm 1.0}$ & $53.7 \mtiny{\pm 0.8}$ & $82.8 \mtiny{\pm 0.4}$ & $60.7 \mtiny{\pm 0.8}$ & $82.7 \mtiny{\pm 0.7}$ & $91.6 \mtiny{\pm 0.6}$ & $51.6 \mtiny{\pm 0.4}$ & $61.5 \mtiny{\pm 0.4}$ & $58.4 \mtiny{\pm 0.2}$ & $31.4 \mtiny{\pm 0.2}$ \\
& \textbf{LoRA-Ppl} & $75.5 \mtiny{\pm 0.9}$ & $51.6 \mtiny{\pm 0.7}$ & $82.0 \mtiny{\pm 0.4}$ & $59.3 \mtiny{\pm 0.6}$ & $74.9 \mtiny{\pm 0.3}$ & $91.6 \mtiny{\pm 0.5}$ & $51.7 \mtiny{\pm 0.6}$ & $62.8 \mtiny{\pm 0.5}$ & $57.0 \mtiny{\pm 0.1}$ & $32.1 \mtiny{\pm 0.1}$ \\
& \textbf{LoRA-F} & $74.8 \mtiny{\pm 0.7}$ & $52.7 \mtiny{\pm 1.0}$ & $81.6 \mtiny{\pm 0.8}$ & $59.4 \mtiny{\pm 0.9}$ & $71.5 \mtiny{\pm 0.7}$ & $91.0 \mtiny{\pm 0.8}$ & $51.7 \mtiny{\pm 0.7}$ & $63.4 \mtiny{\pm 0.8}$ & $58.9 \mtiny{\pm 0.2}$ & $31.0 \mtiny{\pm 0.2}$ \\
& \textbf{LoRSU-Ppl} & $74.1 \mtiny{\pm 1.0}$ & $52.0 \mtiny{\pm 0.9}$ & $82.5 \mtiny{\pm 0.7}$ & $59.8 \mtiny{\pm 0.8}$ & $79.0 \mtiny{\pm 0.7}$ & $92.1 \mtiny{\pm 0.7}$ & $51.8 \mtiny{\pm 0.9}$ & $61.8 \mtiny{\pm 0.4}$ & $58.7 \mtiny{\pm 0.4}$ & $31.6 \mtiny{\pm 0.3}$ \\
\midrule
\multirow{6}{*}{\textbf{CL-50}} & \textbf{LoRA-L} & $75.6 \mtiny{\pm 0.2}$ & $53.0 \mtiny{\pm 0.1}$ & $82.7 \mtiny{\pm 0.3}$ & $60.6 \mtiny{\pm 0.3}$ & $77.1 \mtiny{\pm 0.2}$ & $91.5 \mtiny{\pm 0.2}$ & $51.7 \mtiny{\pm 0.1}$ & $60.7 \mtiny{\pm 0.0}$ & $59.8 \mtiny{\pm 0.1}$ & $31.4 \mtiny{\pm 0.1}$ \\
& \textbf{LoRA} & $62.8 \mtiny{\pm 0.3}$ & $47.2 \mtiny{\pm 0.4}$ & $72.4 \mtiny{\pm 0.4}$ & $54.4 \mtiny{\pm 0.2}$ & $61.6 \mtiny{\pm 0.4}$ & $90.2 \mtiny{\pm 0.3}$ & $51.7 \mtiny{\pm 0.2}$ & $62.0 \mtiny{\pm 0.1}$ & $60.8 \mtiny{\pm 0.0}$ & $30.9 \mtiny{\pm 0.1}$ \\
& \textbf{LoRSU} & $75.4 \mtiny{\pm 0.3}$ & $53.9 \mtiny{\pm 0.1}$ & $83.1 \mtiny{\pm 0.2}$ & $60.3 \mtiny{\pm 0.1}$ & $83.1 \mtiny{\pm 0.1}$ & $92.1 \mtiny{\pm 0.1}$ & $51.6 \mtiny{\pm 0.2}$ & $61.2 \mtiny{\pm 0.0}$ & $57.6 \mtiny{\pm 0.0}$ & $31.1 \mtiny{\pm 0.0}$ \\
& \textbf{LoRA-Ppl} & $74.9 \mtiny{\pm 0.3}$ & $51.7 \mtiny{\pm 0.3}$ & $81.9 \mtiny{\pm 0.2}$ & $59.8 \mtiny{\pm 0.2}$ & $77.8 \mtiny{\pm 0.1}$ & $92.1 \mtiny{\pm 0.3}$ & $51.8 \mtiny{\pm 0.2}$ & $62.9 \mtiny{\pm 0.3}$ & $59.4 \mtiny{\pm 0.2}$ & $31.9 \mtiny{\pm 0.1}$ \\
& \textbf{LoRA-F} & $73.6 \mtiny{\pm 0.0}$ & $51.8 \mtiny{\pm 0.3}$ & $81.2 \mtiny{\pm 0.0}$ & $58.1 \mtiny{\pm 0.1}$ & $66.6 \mtiny{\pm 0.3}$ & $90.7 \mtiny{\pm 0.1}$ & $51.6 \mtiny{\pm 0.1}$ & $63.7 \mtiny{\pm 0.3}$ & $58.4 \mtiny{\pm 0.1}$ & $30.5 \mtiny{\pm 0.0}$ \\
& \textbf{LoRSU-Ppl} & $72.9 \mtiny{\pm 0.1}$ & $51.1 \mtiny{\pm 0.3}$ & $81.3 \mtiny{\pm 0.4}$ & $59.4 \mtiny{\pm 0.4}$ & $75.4 \mtiny{\pm 0.2}$ & $91.6 \mtiny{\pm 0.2}$ & $51.7 \mtiny{\pm 0.1}$ & $62.7 \mtiny{\pm 0.4}$ & $57.5 \mtiny{\pm 0.1}$ & $32.1 \mtiny{\pm 0.0}$ \\
\bottomrule
\end{tabular}
\endgroup
\end{small}
\end{center}
\vskip -0.1in
\end{table}


\begin{table}
\caption{Exact accuracy scores (\%) for each baseline used to fine-tune the model on the \emph{HM} dataset under three different continual learning (5, 10, 50 shots) settings. We include error bars over 3 runs.}
 \label{table:fine_tune_llm_hm}
 \vskip 0.15in
\begin{center}
\begin{small}
\begingroup
\setlength{\tabcolsep}{3.6pt}
\begin{tabular}{l c c c c c c c c c c c}
\toprule
 & & \multicolumn{10}{c}{\textbf{VQA Datasets (Acc \%)}}  \\
\cmidrule(lr){3-12}
\textbf{Setting} & \textbf{PEFT Method}  & \textbf{GTS} & \textbf{TSI} & \textbf{CAn} & \textbf{AIR} & \textbf{ESAT} & \textbf{DALLE} & \textbf{VSR} & \textbf{HM} & \textbf{MMVP} & \textbf{VisOnly} \\
\midrule
 & \textbf{Zr-Shot} & $75.6$ & $53.1$ & $82.7$ & $60.4$ & $76.1$ & $91.1$ & $51.5$ & $61.2$ & $58.0$ & $31.3$ \\
\midrule
\multirow{6}{*}{\textbf{CL-5}} & \textbf{LoRA-L} & $76.5 \mtiny{\pm 1.0}$ & $51.5 \mtiny{\pm 1.1}$ & $83.2 \mtiny{\pm 1.2}$ & $60.5 \mtiny{\pm 0.8}$ & $75.7 \mtiny{\pm 1.0}$ & $90.9 \mtiny{\pm 0.9}$ & $51.6 \mtiny{\pm 0.9}$ & $68.6 \mtiny{\pm 0.7}$ & $34.4 \mtiny{\pm 0.5}$ & $31.1 \mtiny{\pm 0.5}$ \\
& \textbf{LoRA} & $68.8 \mtiny{\pm 0.8}$ & $47.0 \mtiny{\pm 1.0}$ & $70.5 \mtiny{\pm 0.8}$ & $51.7 \mtiny{\pm 1.1}$ & $54.1 \mtiny{\pm 0.6}$ & $89.1 \mtiny{\pm 0.8}$ & $52.2 \mtiny{\pm 1.5}$ & $60.8 \mtiny{\pm 0.8}$ & $54.7 \mtiny{\pm 0.6}$ & $30.5 \mtiny{\pm 0.3}$ \\
& \textbf{LoRSU} & $75.7 \mtiny{\pm 1.2}$ & $54.1 \mtiny{\pm 1.1}$ & $82.9 \mtiny{\pm 0.6}$ & $60.7 \mtiny{\pm 1.0}$ & $76.3 \mtiny{\pm 1.1}$ & $92.2 \mtiny{\pm 0.6}$ & $51.5 \mtiny{\pm 0.9}$ & $61.8 \mtiny{\pm 1.2}$ & $58.1 \mtiny{\pm 0.2}$ & $31.9 \mtiny{\pm 0.5}$ \\
& \textbf{LoRA-Ppl} & $76.2 \mtiny{\pm 0.6}$ & $48.4 \mtiny{\pm 1.4}$ & $82.5 \mtiny{\pm 1.2}$ & $57.2 \mtiny{\pm 0.9}$ & $72.8 \mtiny{\pm 0.9}$ & $90.9 \mtiny{\pm 0.9}$ & $51.8 \mtiny{\pm 1.0}$ & $60.0 \mtiny{\pm 1.0}$ & $56.4 \mtiny{\pm 0.4}$ & $33.1 \mtiny{\pm 0.4}$ \\
& \textbf{LoRA-F} & $71.8 \mtiny{\pm 1.1}$ & $47.8 \mtiny{\pm 0.8}$ & $79.9 \mtiny{\pm 1.5}$ & $57.6 \mtiny{\pm 1.0}$ & $63.2 \mtiny{\pm 1.1}$ & $90.1 \mtiny{\pm 1.0}$ & $48.0 \mtiny{\pm 0.7}$ & $67.2 \mtiny{\pm 0.9}$ & $49.0 \mtiny{\pm 0.3}$ & $31.5 \mtiny{\pm 0.2}$ \\
& \textbf{LoRSU-Ppl} & $76.6 \mtiny{\pm 1.0}$ & $51.7 \mtiny{\pm 1.3}$ & $83.6 \mtiny{\pm 1.4}$ & $60.3 \mtiny{\pm 0.6}$ & $75.2 \mtiny{\pm 0.8}$ & $90.8 \mtiny{\pm 1.0}$ & $51.7 \mtiny{\pm 1.3}$ & $60.4 \mtiny{\pm 1.4}$ & $60.7 \mtiny{\pm 0.5}$ & $31.2 \mtiny{\pm 0.2}$ \\
\midrule
\multirow{6}{*}{\textbf{CL-20}} & \textbf{LoRA-L} & $75.1 \mtiny{\pm 0.9}$ & $50.5 \mtiny{\pm 0.3}$ & $82.1 \mtiny{\pm 0.9}$ & $59.3 \mtiny{\pm 0.8}$ & $65.1 \mtiny{\pm 0.6}$ & $91.8 \mtiny{\pm 0.4}$ & $51.9 \mtiny{\pm 0.5}$ & $71.8 \mtiny{\pm 0.8}$ & $52.8 \mtiny{\pm 0.3}$ & $31.7 \mtiny{\pm 0.2}$ \\
& \textbf{LoRA} & $68.1 \mtiny{\pm 1.0}$ & $46.8 \mtiny{\pm 0.8}$ & $76.3 \mtiny{\pm 0.4}$ & $56.4 \mtiny{\pm 0.8}$ & $49.6 \mtiny{\pm 0.7}$ & $87.3 \mtiny{\pm 0.6}$ & $51.7 \mtiny{\pm 0.4}$ & $59.4 \mtiny{\pm 0.4}$ & $59.7 \mtiny{\pm 0.3}$ & $31.4 \mtiny{\pm 0.3}$ \\
& \textbf{LoRSU} & $76.1 \mtiny{\pm 0.8}$ & $53.0 \mtiny{\pm 0.7}$ & $82.7 \mtiny{\pm 0.5}$ & $60.4 \mtiny{\pm 0.6}$ & $75.7 \mtiny{\pm 0.4}$ & $92.1 \mtiny{\pm 0.7}$ & $51.8 \mtiny{\pm 0.8}$ & $61.9 \mtiny{\pm 0.5}$ & $58.4 \mtiny{\pm 0.2}$ & $31.5 \mtiny{\pm 0.2}$ \\
& \textbf{LoRA-Ppl} & $77.0 \mtiny{\pm 0.9}$ & $52.0 \mtiny{\pm 0.4}$ & $83.9 \mtiny{\pm 0.5}$ & $63.6 \mtiny{\pm 0.7}$ & $73.4 \mtiny{\pm 0.5}$ & $90.5 \mtiny{\pm 0.3}$ & $53.1 \mtiny{\pm 0.7}$ & $71.9 \mtiny{\pm 0.7}$ & $54.1 \mtiny{\pm 0.2}$ & $31.1 \mtiny{\pm 0.4}$ \\
& \textbf{LoRA-F} & $75.6 \mtiny{\pm 0.4}$ & $50.9 \mtiny{\pm 0.5}$ & $80.6 \mtiny{\pm 0.5}$ & $60.8 \mtiny{\pm 0.5}$ & $71.2 \mtiny{\pm 0.7}$ & $90.9 \mtiny{\pm 0.7}$ & $52.2 \mtiny{\pm 0.7}$ & $72.9 \mtiny{\pm 0.7}$ & $53.6 \mtiny{\pm 0.3}$ & $31.6 \mtiny{\pm 0.1}$ \\
& \textbf{LoRSU-Ppl} & $76.1 \mtiny{\pm 0.8}$ & $49.8 \mtiny{\pm 0.9}$ & $83.5 \mtiny{\pm 0.9}$ & $59.8 \mtiny{\pm 0.6}$ & $76.1 \mtiny{\pm 0.9}$ & $91.0 \mtiny{\pm 0.9}$ & $51.7 \mtiny{\pm 0.6}$ & $72.1 \mtiny{\pm 0.4}$ & $59.5 \mtiny{\pm 0.2}$ & $30.5 \mtiny{\pm 0.4}$ \\
\midrule
\multirow{6}{*}{\textbf{CL-50}} & \textbf{LoRA-L} & $75.8 \mtiny{\pm 0.2}$ & $49.5 \mtiny{\pm 0.3}$ & $83.4 \mtiny{\pm 0.3}$ & $59.9 \mtiny{\pm 0.3}$ & $71.1 \mtiny{\pm 0.3}$ & $89.9 \mtiny{\pm 0.3}$ & $51.7 \mtiny{\pm 0.1}$ & $71.4 \mtiny{\pm 0.2}$ & $48.7 \mtiny{\pm 0.1}$ & $31.1 \mtiny{\pm 0.0}$ \\
& \textbf{LoRA} & $72.7 \mtiny{\pm 0.3}$ & $47.1 \mtiny{\pm 0.2}$ & $72.6 \mtiny{\pm 0.2}$ & $56.7 \mtiny{\pm 0.3}$ & $60.4 \mtiny{\pm 0.1}$ & $89.7 \mtiny{\pm 0.3}$ & $51.9 \mtiny{\pm 0.1}$ & $61.9 \mtiny{\pm 0.1}$ & $57.1 \mtiny{\pm 0.2}$ & $31.1 \mtiny{\pm 0.0}$ \\
& \textbf{LoRSU} & $75.3 \mtiny{\pm 0.3}$ & $53.2 \mtiny{\pm 0.1}$ & $83.3 \mtiny{\pm 0.2}$ & $60.5 \mtiny{\pm 0.1}$ & $74.9 \mtiny{\pm 0.1}$ & $92.2 \mtiny{\pm 0.1}$ & $51.6 \mtiny{\pm 0.2}$ & $61.5 \mtiny{\pm 0.0}$ & $58.9 \mtiny{\pm 0.1}$ & $31.3 \mtiny{\pm 0.0}$ \\
& \textbf{LoRA-Ppl} & $76.6 \mtiny{\pm 0.2}$ & $49.3 \mtiny{\pm 0.4}$ & $81.9 \mtiny{\pm 0.3}$ & $60.3 \mtiny{\pm 0.4}$ & $72.7 \mtiny{\pm 0.2}$ & $89.8 \mtiny{\pm 0.3}$ & $52.5 \mtiny{\pm 0.2}$ & $73.7 \mtiny{\pm 0.3}$ & $52.7 \mtiny{\pm 0.0}$ & $30.9 \mtiny{\pm 0.1}$ \\
& \textbf{LoRA-F} & $74.1 \mtiny{\pm 0.1}$ & $52.0 \mtiny{\pm 0.3}$ & $80.6 \mtiny{\pm 0.2}$ & $57.0 \mtiny{\pm 0.1}$ & $63.5 \mtiny{\pm 0.3}$ & $88.7 \mtiny{\pm 0.1}$ & $53.0 \mtiny{\pm 0.4}$ & $73.5 \mtiny{\pm 0.2}$ & $46.0 \mtiny{\pm 0.1}$ & $31.8 \mtiny{\pm 0.0}$ \\
& \textbf{LoRSU-Ppl} & $76.0 \mtiny{\pm 0.1}$ & $50.4 \mtiny{\pm 0.1}$ & $83.4 \mtiny{\pm 0.1}$ & $60.6 \mtiny{\pm 0.4}$ & $76.4 \mtiny{\pm 0.1}$ & $91.4 \mtiny{\pm 0.2}$ & $51.9 \mtiny{\pm 0.1}$ & $73.4 \mtiny{\pm 0.4}$ & $59.8 \mtiny{\pm 0.1}$ & $32.0 \mtiny{\pm 0.1}$ \\
\bottomrule
\end{tabular}
\endgroup
\end{small}
\end{center}
\vskip -0.1in
\end{table}


\begin{table}
\caption{Exact accuracy scores (\%) for each baseline used to fine-tune the model on the \emph{VSR} dataset under three different continual learning (5, 10, 50 shots)  settings. We include error bars over 3 runs.}
 \label{table:fine_tune_llm_vsr}
 \vskip 0.15in
\begin{center}
\begin{small}
\begingroup
\setlength{\tabcolsep}{3.6pt}
\begin{tabular}{l c c c c c c c c c c c}
\toprule
 & & \multicolumn{10}{c}{\textbf{VQA Datasets (Acc \%)}}  \\
\cmidrule(lr){3-12}
\textbf{Setting} & \textbf{PEFT Method}  & \textbf{GTS} & \textbf{TSI} & \textbf{CAn} & \textbf{AIR} & \textbf{ESAT} & \textbf{DALLE} & \textbf{VSR} & \textbf{HM} & \textbf{MMVP} & \textbf{VisOnly} \\
\midrule
 & \textbf{Zr-Shot} & $75.6$ & $53.1$ & $82.7$ & $60.4$ & $76.1$ & $91.1$ & $51.5$ & $61.2$ & $58.0$ & $31.3$ \\
\midrule
\multirow{6}{*}{\textbf{CL-5}} & \textbf{LoRA-L} & $75.3 \mtiny{\pm 0.7}$ & $59.9 \mtiny{\pm 1.4}$ & $81.0 \mtiny{\pm 1.1}$ & $56.2 \mtiny{\pm 0.5}$ & $66.8 \mtiny{\pm 1.3}$ & $90.1 \mtiny{\pm 1.3}$ & $68.3 \mtiny{\pm 1.1}$ & $65.0 \mtiny{\pm 1.4}$ & $57.6 \mtiny{\pm 0.3}$ & $32.5 \mtiny{\pm 0.4}$ \\
& \textbf{LoRA} & $72.6 \mtiny{\pm 1.3}$ & $49.5 \mtiny{\pm 1.5}$ & $78.2 \mtiny{\pm 0.8}$ & $57.5 \mtiny{\pm 1.5}$ & $55.0 \mtiny{\pm 0.9}$ & $88.8 \mtiny{\pm 0.7}$ & $52.0 \mtiny{\pm 1.0}$ & $61.9 \mtiny{\pm 1.5}$ & $59.7 \mtiny{\pm 0.3}$ & $30.4 \mtiny{\pm 0.5}$ \\
& \textbf{LoRSU} & $75.6 \mtiny{\pm 0.7}$ & $52.2 \mtiny{\pm 1.4}$ & $82.2 \mtiny{\pm 0.6}$ & $60.1 \mtiny{\pm 0.9}$ & $77.9 \mtiny{\pm 0.6}$ & $91.1 \mtiny{\pm 1.1}$ & $51.9 \mtiny{\pm 1.3}$ & $62.2 \mtiny{\pm 1.5}$ & $58.4 \mtiny{\pm 0.3}$ & $31.7 \mtiny{\pm 0.3}$ \\
& \textbf{LoRA-Ppl} & $65.8 \mtiny{\pm 0.7}$ & $48.7 \mtiny{\pm 0.8}$ & $65.4 \mtiny{\pm 1.3}$ & $33.8 \mtiny{\pm 1.4}$ & $48.8 \mtiny{\pm 0.5}$ & $81.7 \mtiny{\pm 1.2}$ & $61.7 \mtiny{\pm 0.5}$ & $56.2 \mtiny{\pm 0.7}$ & $43.6 \mtiny{\pm 0.2}$ & $32.8 \mtiny{\pm 0.4}$ \\
& \textbf{LoRA-F} & $76.0 \mtiny{\pm 0.9}$ & $64.5 \mtiny{\pm 0.8}$ & $81.2 \mtiny{\pm 1.3}$ & $57.6 \mtiny{\pm 0.6}$ & $69.7 \mtiny{\pm 1.5}$ & $89.4 \mtiny{\pm 0.8}$ & $69.5 \mtiny{\pm 1.0}$ & $12.8 \mtiny{\pm 0.5}$ & $30.3 \mtiny{\pm 0.5}$ & $13.0 \mtiny{\pm 0.3}$ \\
& \textbf{LoRSU-Ppl} & $73.6 \mtiny{\pm 0.7}$ & $57.5 \mtiny{\pm 1.1}$ & $80.3 \mtiny{\pm 1.1}$ & $57.8 \mtiny{\pm 1.3}$ & $73.1 \mtiny{\pm 1.3}$ & $90.7 \mtiny{\pm 1.1}$ & $62.0 \mtiny{\pm 1.5}$ & $57.4 \mtiny{\pm 0.5}$ & $57.9 \mtiny{\pm 0.6}$ & $30.3 \mtiny{\pm 0.4}$ \\
\midrule
\multirow{6}{*}{\textbf{CL-20}} & \textbf{LoRA-L} & $77.1 \mtiny{\pm 0.8}$ & $54.7 \mtiny{\pm 0.9}$ & $84.5 \mtiny{\pm 0.9}$ & $61.4 \mtiny{\pm 0.5}$ & $75.5 \mtiny{\pm 0.7}$ & $90.9 \mtiny{\pm 0.8}$ & $73.7 \mtiny{\pm 0.5}$ & $64.5 \mtiny{\pm 0.8}$ & $56.9 \mtiny{\pm 0.2}$ & $32.6 \mtiny{\pm 0.4}$ \\
& \textbf{LoRA} & $72.6 \mtiny{\pm 0.7}$ & $54.5 \mtiny{\pm 0.9}$ & $76.6 \mtiny{\pm 0.8}$ & $57.4 \mtiny{\pm 0.7}$ & $57.3 \mtiny{\pm 0.4}$ & $87.9 \mtiny{\pm 0.8}$ & $51.9 \mtiny{\pm 0.7}$ & $59.0 \mtiny{\pm 0.5}$ & $57.6 \mtiny{\pm 0.2}$ & $31.3 \mtiny{\pm 0.4}$ \\
& \textbf{LoRSU} & $74.9 \mtiny{\pm 0.6}$ & $54.6 \mtiny{\pm 0.5}$ & $82.1 \mtiny{\pm 0.8}$ & $58.5 \mtiny{\pm 0.7}$ & $75.5 \mtiny{\pm 0.5}$ & $91.6 \mtiny{\pm 0.5}$ & $51.6 \mtiny{\pm 0.6}$ & $62.4 \mtiny{\pm 0.7}$ & $57.5 \mtiny{\pm 0.2}$ & $30.9 \mtiny{\pm 0.2}$ \\
& \textbf{LoRA-Ppl} & $74.9 \mtiny{\pm 0.4}$ & $62.2 \mtiny{\pm 0.4}$ & $82.4 \mtiny{\pm 0.3}$ & $58.2 \mtiny{\pm 0.7}$ & $70.5 \mtiny{\pm 0.7}$ & $89.0 \mtiny{\pm 0.6}$ & $71.0 \mtiny{\pm 0.8}$ & $64.8 \mtiny{\pm 0.5}$ & $55.8 \mtiny{\pm 0.2}$ & $28.6 \mtiny{\pm 0.2}$ \\
& \textbf{LoRA-F} & $75.4 \mtiny{\pm 0.5}$ & $60.6 \mtiny{\pm 0.5}$ & $80.9 \mtiny{\pm 0.9}$ & $56.6 \mtiny{\pm 0.9}$ & $63.1 \mtiny{\pm 0.7}$ & $88.2 \mtiny{\pm 0.6}$ & $74.8 \mtiny{\pm 0.5}$ & $48.7 \mtiny{\pm 0.9}$ & $50.1 \mtiny{\pm 0.4}$ & $20.2 \mtiny{\pm 0.2}$ \\
& \textbf{LoRSU-Ppl} & $72.6 \mtiny{\pm 0.8}$ & $52.7 \mtiny{\pm 0.5}$ & $81.6 \mtiny{\pm 0.8}$ & $60.3 \mtiny{\pm 0.5}$ & $69.4 \mtiny{\pm 0.7}$ & $89.6 \mtiny{\pm 0.5}$ & $74.4 \mtiny{\pm 0.9}$ & $62.5 \mtiny{\pm 0.8}$ & $57.1 \mtiny{\pm 0.3}$ & $29.7 \mtiny{\pm 0.4}$ \\
\midrule
\multirow{6}{*}{\textbf{CL-50}} & \textbf{LoRA-L} & $77.2 \mtiny{\pm 0.3}$ & $56.5 \mtiny{\pm 0.1}$ & $84.5 \mtiny{\pm 0.0}$ & $61.4 \mtiny{\pm 0.2}$ & $76.4 \mtiny{\pm 0.2}$ & $91.5 \mtiny{\pm 0.3}$ & $73.4 \mtiny{\pm 0.1}$ & $65.3 \mtiny{\pm 0.2}$ & $54.4 \mtiny{\pm 0.1}$ & $31.5 \mtiny{\pm 0.1}$ \\
& \textbf{LoRA} & $73.4 \mtiny{\pm 0.1}$ & $53.8 \mtiny{\pm 0.0}$ & $74.6 \mtiny{\pm 0.4}$ & $56.7 \mtiny{\pm 0.1}$ & $56.2 \mtiny{\pm 0.1}$ & $87.0 \mtiny{\pm 0.2}$ & $51.9 \mtiny{\pm 0.0}$ & $59.2 \mtiny{\pm 0.2}$ & $57.6 \mtiny{\pm 0.1}$ & $30.8 \mtiny{\pm 0.0}$ \\
& \textbf{LoRSU} & $75.3 \mtiny{\pm 0.1}$ & $54.7 \mtiny{\pm 0.1}$ & $81.6 \mtiny{\pm 0.1}$ & $58.3 \mtiny{\pm 0.2}$ & $75.7 \mtiny{\pm 0.1}$ & $91.4 \mtiny{\pm 0.4}$ & $53.8 \mtiny{\pm 0.2}$ & $62.1 \mtiny{\pm 0.3}$ & $57.3 \mtiny{\pm 0.1}$ & $30.8 \mtiny{\pm 0.0}$ \\
& \textbf{LoRA-Ppl} & $71.7 \mtiny{\pm 0.1}$ & $48.7 \mtiny{\pm 0.1}$ & $75.1 \mtiny{\pm 0.2}$ & $46.3 \mtiny{\pm 0.4}$ & $64.6 \mtiny{\pm 0.3}$ & $87.9 \mtiny{\pm 0.2}$ & $71.7 \mtiny{\pm 0.4}$ & $61.9 \mtiny{\pm 0.2}$ & $55.1 \mtiny{\pm 0.1}$ & $30.9 \mtiny{\pm 0.0}$ \\
& \textbf{LoRA-F} & $76.3 \mtiny{\pm 0.3}$ & $64.2 \mtiny{\pm 0.2}$ & $84.5 \mtiny{\pm 0.4}$ & $58.1 \mtiny{\pm 0.3}$ & $69.6 \mtiny{\pm 0.1}$ & $90.1 \mtiny{\pm 0.1}$ & $72.5 \mtiny{\pm 0.3}$ & $64.6 \mtiny{\pm 0.1}$ & $61.4 \mtiny{\pm 0.1}$ & $30.6 \mtiny{\pm 0.1}$ \\
& \textbf{LoRSU-Ppl} & $72.1 \mtiny{\pm 0.2}$ & $49.8 \mtiny{\pm 0.1}$ & $74.8 \mtiny{\pm 0.3}$ & $57.6 \mtiny{\pm 0.0}$ & $71.0 \mtiny{\pm 0.4}$ & $88.2 \mtiny{\pm 0.1}$ & $74.9 \mtiny{\pm 0.1}$ & $58.3 \mtiny{\pm 0.2}$ & $55.4 \mtiny{\pm 0.2}$ & $30.0 \mtiny{\pm 0.2}$ \\
\bottomrule
\end{tabular}
\endgroup
\end{small}
\end{center}
\vskip -0.1in
\end{table}


\begin{table}
\caption{Exact accuracy scores (\%) for each baseline used to fine-tune the model on the \emph{MMVP} dataset under three different continual learning (5, 10, 50 shots)  settings. We include error bars over 3 runs.}
 \label{table:fine_tune_llm_mmvp}
 \vskip 0.15in
\begin{center}
%\begin{small}
\begingroup
\setlength{\tabcolsep}{6.4pt}
\begin{tabular}{l c c c c c c c c c c c}
\toprule
 & & \multicolumn{10}{c}{\textbf{VQA Datasets (Acc \%)}}  \\
\cmidrule(lr){3-12}
\textbf{Setting} & \textbf{PEFT Method}  & \textbf{GTS} & \textbf{TSI} & \textbf{CAn} & \textbf{AIR} & \textbf{ESAT} & \textbf{DALLE} & \textbf{VSR} & \textbf{HM} & \textbf{MMVP} & \textbf{VisOnly} \\
\midrule
 & \textbf{Zr-Shot} & $75.6$ & $53.1$ & $82.7$ & $60.4$ & $76.1$ & $91.1$ & $51.5$ & $61.2$ & $58.0$ & $31.3$ \\
\midrule
\multirow{4}{*}{\textbf{CL}} & \textbf{LoRA-L} & $75.5$ & $52.8$ & $82.0$ & $60.5$ & $76.0$ & $91.5$ & $51.5$ & $63.6$ & $57.7$ & $30.6$ \\
& \textbf{LoRA-Ppl} & $75.5$ & $53.6$ & $83.0$ & $60.3$ & $75.6$ & $91.1$ & $51.5$ & $63.1$ & $60.7$ & $31.7$ \\
& \textbf{LoRA-F} & $75.2$ & $52.9$ & $81.3$ & $60.5$ & $74.3$ & $90.4$ & $51.6$ & $63.6$ & $60.0$ & $31.4$ \\
& \textbf{LoRSU-Ppl} & $75.1$ & $52.0$ & $81.2$ & $57.4$ & $75.2$ & $90.2$ & $51.7$ & $63.9$ & $60.3$ & $30.8$ \\
\bottomrule
\end{tabular}
\endgroup
%\end{small}
\end{center}
\vskip -0.1in
\end{table}


\begin{table}
\caption{Exact accuracy scores (\%) for each baseline used to fine-tune the model on the \emph{VisOnly} dataset under three different continual learning (5, 10, 50 shots)  settings. We include error bars over 3 runs.}
 \label{table:fine_tune_llm_visonlyqa}
 \vskip 0.15in
\begin{center}
\begin{small}
\begingroup
\setlength{\tabcolsep}{3.6pt}
\begin{tabular}{l c c c c c c c c c c c}
\toprule
 & & \multicolumn{10}{c}{\textbf{VQA Datasets (Acc \%)}}  \\
\cmidrule(lr){3-12}
\textbf{Setting} & \textbf{PEFT Method}  & \textbf{GTS} & \textbf{TSI} & \textbf{CAn} & \textbf{AIR} & \textbf{ESAT} & \textbf{DALLE} & \textbf{VSR} & \textbf{HM} & \textbf{MMVP} & \textbf{VisOnly} \\
\midrule
 & \textbf{Zr-Shot} & $75.6$ & $53.1$ & $82.7$ & $60.4$ & $76.1$ & $91.1$ & $51.5$ & $61.2$ & $58.0$ & $31.3$ \\
\midrule
\multirow{6}{*}{\textbf{CL-5}} & \textbf{LoRA-L} & $76.5 \mtiny{\pm 1.2}$ & $51.9 \mtiny{\pm 0.7}$ & $82.4 \mtiny{\pm 1.4}$ & $60.5 \mtiny{\pm 1.5}$ & $76.1 \mtiny{\pm 1.0}$ & $91.5 \mtiny{\pm 0.6}$ & $51.6 \mtiny{\pm 0.9}$ & $60.3 \mtiny{\pm 1.0}$ & $57.6 \mtiny{\pm 0.3}$ & $31.3 \mtiny{\pm 0.4}$ \\
& \textbf{LoRA} & $70.9 \mtiny{\pm 1.4}$ & $52.1 \mtiny{\pm 1.2}$ & $77.5 \mtiny{\pm 1.3}$ & $55.6 \mtiny{\pm 0.6}$ & $52.6 \mtiny{\pm 0.8}$ & $89.3 \mtiny{\pm 0.6}$ & $51.7 \mtiny{\pm 0.8}$ & $61.7 \mtiny{\pm 0.7}$ & $56.9 \mtiny{\pm 0.6}$ & $30.9 \mtiny{\pm 0.5}$ \\
& \textbf{LoRSU} & $75.9 \mtiny{\pm 0.7}$ & $53.1 \mtiny{\pm 0.8}$ & $82.5 \mtiny{\pm 0.7}$ & $60.4 \mtiny{\pm 1.0}$ & $76.1 \mtiny{\pm 1.5}$ & $91.9 \mtiny{\pm 0.8}$ & $51.5 \mtiny{\pm 1.3}$ & $61.3 \mtiny{\pm 1.2}$ & $58.9 \mtiny{\pm 0.4}$ & $31.5 \mtiny{\pm 0.2}$ \\
& \textbf{LoRA-Ppl} & $76.3 \mtiny{\pm 1.1}$ & $50.7 \mtiny{\pm 1.1}$ & $82.2 \mtiny{\pm 0.9}$ & $61.0 \mtiny{\pm 1.3}$ & $73.4 \mtiny{\pm 0.9}$ & $91.7 \mtiny{\pm 1.3}$ & $52.1 \mtiny{\pm 1.1}$ & $59.3 \mtiny{\pm 1.3}$ & $58.0 \mtiny{\pm 0.2}$ & $35.0 \mtiny{\pm 0.5}$ \\
& \textbf{LoRA-F} & $76.0 \mtiny{\pm 0.8}$ & $51.1 \mtiny{\pm 1.4}$ & $82.9 \mtiny{\pm 1.1}$ & $59.9 \mtiny{\pm 0.7}$ & $71.2 \mtiny{\pm 1.2}$ & $91.7 \mtiny{\pm 1.1}$ & $51.6 \mtiny{\pm 1.3}$ & $60.8 \mtiny{\pm 0.7}$ & $58.4 \mtiny{\pm 0.2}$ & $34.9 \mtiny{\pm 0.4}$ \\
& \textbf{LoRSU-Ppl} & $76.2 \mtiny{\pm 1.1}$ & $53.0 \mtiny{\pm 0.9}$ & $83.4 \mtiny{\pm 0.7}$ & $61.3 \mtiny{\pm 1.4}$ & $76.6 \mtiny{\pm 0.8}$ & $92.3 \mtiny{\pm 0.5}$ & $52.0 \mtiny{\pm 1.0}$ & $61.6 \mtiny{\pm 0.7}$ & $60.7 \mtiny{\pm 0.3}$ & $32.0 \mtiny{\pm 0.5}$ \\
\midrule
\multirow{6}{*}{\textbf{CL-20}} & \textbf{LoRA-L} & $77.8 \mtiny{\pm 1.0}$ & $53.0 \mtiny{\pm 0.8}$ & $83.4 \mtiny{\pm 0.4}$ & $62.1 \mtiny{\pm 0.6}$ & $75.5 \mtiny{\pm 0.8}$ & $91.6 \mtiny{\pm 0.4}$ & $52.4 \mtiny{\pm 0.9}$ & $61.2 \mtiny{\pm 0.6}$ & $55.6 \mtiny{\pm 0.3}$ & $32.5 \mtiny{\pm 0.3}$ \\
& \textbf{LoRA} & $73.3 \mtiny{\pm 0.9}$ & $49.3 \mtiny{\pm 0.4}$ & $77.9 \mtiny{\pm 0.6}$ & $56.4 \mtiny{\pm 0.6}$ & $47.7 \mtiny{\pm 0.8}$ & $91.2 \mtiny{\pm 0.6}$ & $51.8 \mtiny{\pm 0.8}$ & $61.5 \mtiny{\pm 0.6}$ & $57.0 \mtiny{\pm 0.3}$ & $32.8 \mtiny{\pm 0.1}$ \\
& \textbf{LoRSU} & $75.7 \mtiny{\pm 0.5}$ & $53.3 \mtiny{\pm 0.7}$ & $82.0 \mtiny{\pm 0.5}$ & $60.0 \mtiny{\pm 0.5}$ & $76.1 \mtiny{\pm 0.6}$ & $91.9 \mtiny{\pm 0.9}$ & $51.7 \mtiny{\pm 0.6}$ & $61.6 \mtiny{\pm 0.3}$ & $58.2 \mtiny{\pm 0.3}$ & $31.5 \mtiny{\pm 0.4}$ \\
& \textbf{LoRA-Ppl} & $78.0 \mtiny{\pm 0.4}$ & $52.8 \mtiny{\pm 0.4}$ & $83.7 \mtiny{\pm 0.6}$ & $60.9 \mtiny{\pm 0.7}$ & $74.3 \mtiny{\pm 0.4}$ & $91.5 \mtiny{\pm 0.7}$ & $51.9 \mtiny{\pm 0.5}$ & $61.7 \mtiny{\pm 0.7}$ & $56.0 \mtiny{\pm 0.2}$ & $32.8 \mtiny{\pm 0.3}$ \\
& \textbf{LoRA-F} & $77.4 \mtiny{\pm 0.6}$ & $51.7 \mtiny{\pm 0.9}$ & $83.7 \mtiny{\pm 0.6}$ & $59.7 \mtiny{\pm 0.7}$ & $73.9 \mtiny{\pm 0.9}$ & $91.2 \mtiny{\pm 0.5}$ & $53.4 \mtiny{\pm 0.4}$ & $62.0 \mtiny{\pm 0.9}$ & $56.9 \mtiny{\pm 0.4}$ & $31.0 \mtiny{\pm 0.3}$ \\
& \textbf{LoRSU-Ppl} & $76.7 \mtiny{\pm 0.5}$ & $53.7 \mtiny{\pm 0.4}$ & $83.8 \mtiny{\pm 0.6}$ & $61.4 \mtiny{\pm 0.3}$ & $75.5 \mtiny{\pm 0.6}$ & $91.2 \mtiny{\pm 0.8}$ & $51.8 \mtiny{\pm 0.3}$ & $61.9 \mtiny{\pm 0.9}$ & $59.6 \mtiny{\pm 0.4}$ & $31.3 \mtiny{\pm 0.2}$ \\
\midrule
\multirow{6}{*}{\textbf{CL-50}} & \textbf{LoRA-L} & $76.4 \mtiny{\pm 0.4}$ & $54.5 \mtiny{\pm 0.3}$ & $84.1 \mtiny{\pm 0.3}$ & $61.3 \mtiny{\pm 0.0}$ & $73.9 \mtiny{\pm 0.1}$ & $91.5 \mtiny{\pm 0.1}$ & $51.9 \mtiny{\pm 0.3}$ & $62.8 \mtiny{\pm 0.1}$ & $55.4 \mtiny{\pm 0.0}$ & $32.1 \mtiny{\pm 0.1}$ \\
& \textbf{LoRA} & $70.0 \mtiny{\pm 0.1}$ & $46.8 \mtiny{\pm 0.0}$ & $70.5 \mtiny{\pm 0.1}$ & $51.0 \mtiny{\pm 0.2}$ & $50.9 \mtiny{\pm 0.0}$ & $88.1 \mtiny{\pm 0.0}$ & $52.0 \mtiny{\pm 0.3}$ & $61.2 \mtiny{\pm 0.3}$ & $57.8 \mtiny{\pm 0.2}$ & $31.7 \mtiny{\pm 0.1}$ \\
& \textbf{LoRSU} & $75.6 \mtiny{\pm 0.4}$ & $53.1 \mtiny{\pm 0.1}$ & $81.7 \mtiny{\pm 0.3}$ & $58.2 \mtiny{\pm 0.1}$ & $75.3 \mtiny{\pm 0.2}$ & $91.8 \mtiny{\pm 0.3}$ & $51.7 \mtiny{\pm 0.1}$ & $62.1 \mtiny{\pm 0.1}$ & $58.3 \mtiny{\pm 0.1}$ & $31.9 \mtiny{\pm 0.0}$ \\
& \textbf{LoRA-Ppl} & $76.9 \mtiny{\pm 0.4}$ & $54.6 \mtiny{\pm 0.1}$ & $84.1 \mtiny{\pm 0.3}$ & $60.5 \mtiny{\pm 0.2}$ & $74.9 \mtiny{\pm 0.4}$ & $91.2 \mtiny{\pm 0.3}$ & $51.8 \mtiny{\pm 0.3}$ & $62.5 \mtiny{\pm 0.3}$ & $56.0 \mtiny{\pm 0.1}$ & $33.0 \mtiny{\pm 0.0}$ \\
& \textbf{LoRA-F} & $77.1 \mtiny{\pm 0.0}$ & $53.0 \mtiny{\pm 0.2}$ & $83.9 \mtiny{\pm 0.4}$ & $60.9 \mtiny{\pm 0.1}$ & $73.1 \mtiny{\pm 0.1}$ & $92.2 \mtiny{\pm 0.3}$ & $51.9 \mtiny{\pm 0.2}$ & $61.4 \mtiny{\pm 0.4}$ & $58.0 \mtiny{\pm 0.0}$ & $32.5 \mtiny{\pm 0.1}$ \\
& \textbf{LoRSU-Ppl} & $76.1 \mtiny{\pm 0.3}$ & $51.5 \mtiny{\pm 0.2}$ & $81.6 \mtiny{\pm 0.1}$ & $60.2 \mtiny{\pm 0.0}$ & $75.6 \mtiny{\pm 0.2}$ & $92.2 \mtiny{\pm 0.3}$ & $52.0 \mtiny{\pm 0.2}$ & $61.2 \mtiny{\pm 0.3}$ & $58.3 \mtiny{\pm 0.0}$ & $33.5 \mtiny{\pm 0.1}$ \\
\bottomrule
\end{tabular}
\endgroup
\end{small}
\end{center}
\vskip -0.1in
\end{table}


\section{Extra Ablation Studies}\label{sec_appx:extra_ablations}

\subsection{Ablation on the rank $r$ of LoRSU}
In Table~\ref{table:ablation_ranks}, we investigate the effect on performance of using different ranks for LoRSU. As the rank $r$ increases, the VQA accuracy on the target dataset slightly improves, peaking at $r=64$. Beyond 
that, performance slightly decreases. Performance on other datasets remains relatively stable with small fluctuations.
\begin{table}
\caption{Ablation study over the effect of the rank $r$ used by \emph{LoRSU} to fine-tune the image encoder, CLIP-L-14. We report the VQA accuracies of the last session in the \emph{50-shot} CL setting. The accuracies on the target dataset are in red color. For this experiment, we use two attention heads to fine-tune with LoRSU.}
 \label{table:ablation_ranks}
\vskip 0.15in
\begin{center}
%\begin{small}
\begingroup
\setlength{\tabcolsep}{6.7pt}
\begin{tabular}{l c c c c c c c c c c c}
\toprule
 & & \multicolumn{9}{c}{\textbf{VQA Datasets (Acc \%)}}  \\
\cmidrule(lr){3-12}
\textbf{FT Dataset} & \textbf{rank ($r$)}  & \textbf{GTS} & \textbf{TSI} & \textbf{CAn} & \textbf{AIR} & \textbf{ESAT} & \textbf{DALLE} & \textbf{VSR} & \textbf{HM} & \textbf{MMVP} & \textbf{VisOnly} \\
\midrule
\multirow{6}{*}{\textbf{GTS}} & \textbf{8} & \textcolor{red}{$83.0$} & $53.2$ & $81.3$ & $60.9$ & $61.0$ & $91.2$ & $51.5$ & $61.6$ & $60.0$ & $31.6$ \\
 & \textbf{16} & \textcolor{red}{$83.9$} & $53.4$ & $81.5$ & $60.2$ & $54.0$ & $91.4$ & $51.5$ & $62.1$ & $60.7$ & $31.6$ \\
 & \textbf{32} & \textcolor{red}{$84.8$} & $53.1$ & $81.9$ & $60.5$ & $58.0$ & $90.6$ & $51.6$ & $61.8$ & $58.7$ & $31.5$ \\
 & \textbf{64} & \textcolor{red}{$84.9$} &  $53.2$ & $81.3$ & $60.7$ & $61.7$ & $90.9$ & $51.5$ & $61.9$ & $59.3$ & $31.3$ \\
 & \textbf{128} & \textcolor{red}{$84.3$} & $53.2$ & $81.8$ & $60.6$ & $56.8$ & $91.5$ & $51.6$ & $61.8$ & $58.7$ & $31.2$ \\
 & \textbf{256} & \textcolor{red}{$84.5$} & $53.1$ & $81.5$ & $61.1$ & $51.5$ & $90.3$ & $51.6$ & $62.0$ & $58.7$ & $31.6$ \\
\midrule
\multirow{6}{*}{\textbf{TSI}} & \textbf{8} & $75.2$ & \textcolor{red}{$67.2$} & $82.0$ & $59.2$ & $71.6$ & $91.1$ & $51.5$ & $61.6$ & $58.0$ & $31.5$ \\
 & \textbf{16} & $75.4$ & \textcolor{red}{$68.0$} & $82.3$ & $59.1$ & $71.0$ & $90.6$ & $51.6$ & $61.6$ & $56.7$ & $31.2$ \\
 & \textbf{32} & $74.9$ & \textcolor{red}{$68.9$} & $81.8$ & $59.3$ & $70.1$ & $91.2$ & $51.5$ & $61.6$ & $58.0$ & $31.6$ \\
 & \textbf{64} & $75.3$ & \textcolor{red}{$72.1$} & $82.0$ & $59.3$ & $72.3$ & $90.5$ & $51.6$ & $61.4$ & $58.0$ & $31.6$ \\
 & \textbf{128} & $75.1$ & \textcolor{red}{$65.8$} & $81.7$ & $59.0$ & $70.0$ & $90.6$ & $51.5$ & $62.1$ & $56.7$ & $31.6$ \\
 & \textbf{256} & $75.4$ & \textcolor{red}{$66.4$} & $82.3$ & $59.6$ & $72.0$ & $91.2$ & $51.5$ & $62.1$ & $56.7$ & $31.5$ \\
\midrule
\midrule
\textbf{Zr-Shot} & & $75.6$ & $53.1$ & $82.7$ & $60.4$ & $76.1$ & $91.1$ & $51.5$ & $61.2$ & $58.0$ & $31.3$ \\
\bottomrule
\end{tabular}
\endgroup
%\end{small}
\end{center}
\vskip -0.1in
\end{table}

\subsection{Ablation on the number of optimal attention heads of LoRSU}\label{sec_appx:ablation_heads_general}
\begin{table}
\caption{Ablation study over the effect of the number of attention heads used by \emph{LoRSU} to fine-tune the image encoder. We report the VQA accuracies of the last session in the \emph{50-shot} CL setting. The accuracies on the target dataset are in red color. For this experiment, we use $r=64$ for the rank of LoRSU.}
 \label{table:ablation_heads}
 \vskip 0.15in
\begin{center}
%\begin{small}
\begingroup
\setlength{\tabcolsep}{6.7pt}
\begin{tabular}{l c c c c c c c c c c c}
\toprule
 & & \multicolumn{9}{c}{\textbf{VQA Datasets (Acc \%)}}  \\
\cmidrule(lr){3-12}
\textbf{FT Dataset} & \textbf{\# heads}  & \textbf{GTS} & \textbf{TSI} & \textbf{CAn} & \textbf{AIR} & \textbf{ESAT} & \textbf{DALLE} & \textbf{VSR} & \textbf{HM} & \textbf{MMVP} & \textbf{VisOnly} \\
\midrule
\multirow{6}{*}{\textbf{GTS}} & \textbf{0} & \textcolor{red}{$83.1$} & $52.7$ & $82.2$ & $60.8$ & $60.6$ & $91.1$ & $51.6$ & $61.7$ & $59.3$ & $31.6$ \\
 & \textbf{1} &  \textcolor{red}{$83.9$} & $53.8$ & $82.0$ & $60.7$ & $55.4$ & $91.2$ & $51.6$ & $61.6$ & $60.0$ & $31.8$ \\
 & \textbf{2} &  \textcolor{red}{$84.9$} & $53.2$ & $81.3$ & $60.7$ & $61.7$ & $90.9$ & $51.5$ & $61.9$ & $59.3$ & $31.3$ \\
 & \textbf{4} &  \textcolor{red}{$84.7$} & $53.5$ & $81.0$ & $60.5$ & $60.5$ & $90.6$ & $51.5$ & $61.8$ & $58.7$ & $31.5$ \\
 & \textbf{8} &  \textcolor{red}{$84.9$} & $52.9$ & $81.2$ & $60.5$ & $58.8$ & $90.5$ & $51.5$ & $61.6$ & $59.3$ & $31.5$ \\
 & \textbf{16} &  \textcolor{red}{$85.0$} & $53.1$ & $81.3$ & $60.0$ & $59.2$ & $90.6$ & $51.5$ & $61.6$ & $56.7$ & $31.3$ \\
\midrule
\multirow{6}{*}{\textbf{TSI}} & \textbf{0} & $75.1$ &  \textcolor{red}{$64.2$} & $82.1$ & $59.3$ & $72.2$ & $90.8$ & $51.5$ & $61.8$ & $57.3$ & $31.5$ \\
 & \textbf{1} & $75.3$ &  \textcolor{red}{$64.8$} & $81.9$ & $59.5$ & $74.0$ & $90.5$ & $51.5$ & $61.6$ & $58.0$ & $32.0$ \\
 & \textbf{2} & $75.3$ &  \textcolor{red}{$72.1$} & $82.0$ & $59.3$ & $72.3$ & $90.5$ & $51.6$ & $61.4$ & $58.0$ & $31.6$ \\
 & \textbf{4} & $74.9$ &  \textcolor{red}{$66.8$} & $82.2$ & $58.9$ & $74.0$ & $90.5$ & $51.5$ & $62.1$ & $58.0$ & $31.4$ \\
 & \textbf{8} & $74.7$ &  \textcolor{red}{$67.4$} & $81.7$ & $59.1$ & $71.5$ & $91.2$ & $51.5$ & $62.2$ & $58.0$ & $31.7$ \\
 & \textbf{16} & $75.3$ &  \textcolor{red}{$65.2$} & $81.8$ & $59.9$ & $69.1$ & $90.5$ & $51.5$ & $61.6$ & $58.0$ & $31.3$ \\
\midrule
\midrule
\textbf{Zr-Shot} & & $75.6$ & $53.1$ & $82.7$ & $60.4$ & $76.1$ & $91.1$ & $51.5$ & $61.2$ & $58.0$ & $31.3$ \\
\bottomrule
\end{tabular}
\endgroup
%\end{small}
\end{center}
\vskip -0.1in
\end{table}
In Table~\ref{table:ablation_heads}, we examine how the number of attention heads chosen to be fine-tuned affects LoRSU's performance. We notice that more attention heads marginally improve the performance of the model while the extra flexibility can cause more forgetting, e.g. ESAT.

\subsection{Robustness on the Choice of Attention Heads}\label{sec_appx:robustness}
\begin{table}
\caption{Robustness comparison of LoRSU with respect to the number of training epochs. We consider LoRSU, \emph{LoRSU-Rand} where the $k$ attention heads are chosen randomly and \emph{LoRSU-AAH} where all the attention heads are chosen for fine tuning. We use \emph{50 shots} on the \emph{GTS} for each method and we report the Target Improvement (\emph{TI}) on this dataset and the Control Change (\emph{CC}) using only ESAT as a control dataset.  We include error bars over 3 runs.}
 \label{table:lorsu_attn_rand_epochs}
\vskip 0.15in
\begin{center}
%\begin{small}
\begingroup
\setlength{\tabcolsep}{7.9pt}
\begin{tabular}{l c c c c c c}
\toprule
\multirow{2}{*}{\textbf{\# Epochs}} &  \multicolumn{2}{c}{\textbf{LoRSU-Rand}} & \multicolumn{2}{c}{\textbf{LoRSU-AAH}}  &  \multicolumn{2}{c}{\textbf{LoRSU}}  \\
 \cmidrule(lr){2-3}  \cmidrule(lr){4-5}  \cmidrule(lr){6-7} & \textbf{TI ($\uparrow)$} & \textbf{CC ($\uparrow)$} & \textbf{TI ($\uparrow)$} & \textbf{CC ($\uparrow)$} & \textbf{TI ($\uparrow)$} & \textbf{CC ($\uparrow)$} \\
\midrule
\textbf{2} & $5.2 \mtiny{\pm 0.9}$ & $-11.1 \mtiny{\pm 1.1}$ & $6.1 \mtiny{\pm 0.3}$   & $-11.6 \mtiny{\pm 0.7}$ & $5.6 \mtiny{\pm 0.4}$ & $-9.7 \mtiny{\pm 0.8}$  \\
\textbf{5} & $7.6 \mtiny{\pm 0.8}$ & $-15.0 \mtiny{\pm 0.9}$ & $9.3 \mtiny{\pm 0.4}$ & $-15.6 \mtiny{\pm 0.6}$ & $8.6 \mtiny{\pm 0.3}$ & $-12.6 \mtiny{\pm 0.5}$  \\
\textbf{10} & $7.8 \mtiny{\pm 0.5}$ & $-18.1 \mtiny{\pm 0.8}$ & $9.1 \mtiny{\pm 0.1}$ & $-19.6 \mtiny{\pm 0.5}$ & $9.7 \mtiny{\pm 0.1}$ & $-14.3 \mtiny{\pm 0.7}$  \\
\textbf{20} & $5.9 \mtiny{\pm 0.6}$ & $-20.0 \mtiny{\pm 0.7}$ & $8.1 \mtiny{\pm 0.1}$ & $-21.5 \mtiny{\pm 0.6}$ & $7.4 \mtiny{\pm 0.2}$ & $-15.7 \mtiny{\pm 0.6}$  \\
\bottomrule
\end{tabular}
\endgroup
%\end{small}
\end{center}
\vskip -0.1in
\end{table}
We show in Table~\ref{table:lorsu_attn_rand_epochs} that LoRSU's mechanism of choosing the most important attention heads provides a clear advantage in terms of robustness over the other two LoRSU's variants, LoRSU-Rand and LoRSU-AAH. We can see that TI and CC decline in a lower rate compared to that of LoRSU-RAnd and LoRSU-AAH, as we increase the number of training epochs.. As expected, LoRSU-Rand appears to be the least robust method since the random choice of the attention heads constitute it more unstable.

\section{TSI vs. DALLE}\label{sec_appx:tsi_vs_dalle}
In Figures~\ref{fig:tsi_example_1} through~\ref{fig:tsi_example_4}, we present examples of images from TSI and DALLE for different actions. In general, we observe that TSI comprised of   natural, unposed images of senior individuals performing daily tasks, reflecting real-life scenarios. The images are broader, showing the surrounding environment, which is crucial for context. On the other hand, DALLE images are idealized or stylized images. The focus is narrower, with emphasis on the object of the action (e.g. tablet, glass, etc.).

\begin{figure}%
\vskip 0.2in
\centering
    \subfigure[TSI]{\includegraphics[width=.4\linewidth,height=5.2cm]{figures/Uselaptop_p14_r07_v02_c05_sec-45}}
    \subfigure[DALLE]{\includegraphics[width=.4\linewidth,height=5.2cm]{figures/using-a-laptop_batch-4_3}}
    \caption{Instances of the `Use Laptop' action.}%
    \label{fig:tsi_example_1}
\vskip -0.2in
\end{figure}

\begin{figure}%
\vskip 0.2in
\centering
    \subfigure[TSI]{\includegraphics[width=.4\linewidth,height=5.2cm]{figures/WatchTV_p03_r00_v19_c05_sec-18}}
    \subfigure[DALLE]{\includegraphics[width=.4\linewidth,height=5.2cm]{figures/watching-TV_batch-1_0.png}}
    \caption{Instances of the `Watching TV' action.}%
    \label{fig:tsi_example_2}
\vskip -0.2in
\end{figure}

\begin{figure}%
\vskip 0.2in
\centering
    \subfigure[TSI]{\includegraphics[width=.4\linewidth,height=5.2cm]{figures/Usetablet_p03_r00_v13_c02_sec-72.jpg}}
    \subfigure[DALLE]{\includegraphics[width=.4\linewidth,height=5.2cm]{figures/using-a-tablet_batch-3_1.png}}
    \caption{Instances of the `Use Tablet' action.}%
    \label{fig:tsi_example_3}
\vskip -0.2in
\end{figure}

\begin{figure}%
\vskip 0.2in
\centering
    \subfigure[TSI]{\includegraphics[width=.4\linewidth,height=5.2cm]{figures/Usetelephone_p07_r00_v06_c05_sec-216.jpg}}
    \subfigure[DALLE]{\includegraphics[width=.4\linewidth,height=5.2cm]{figures/using-a-cordless-phone_batch-3_1.png}}
    \caption{Instances of the `Use  a telephone' action.}%
    \label{fig:tsi_example_4}
\vskip -0.2in
\end{figure}

