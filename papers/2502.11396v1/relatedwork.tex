\section{RELATED WORK}
SH theory was first studied by Burt \cite{burt2009structural} to identify the critical personnel in the company to integrate various operations. Goyal et al. \cite{goyal2007structural} designed a model to illustrate how a node act as a SH spanner in the network. They formulated the SH spanner problem as a set of nodes that pass through numerous shortest paths between distinct pair of nodes. Tang et al. \cite{tang2012inferring} designed a 2-step mechanism to identify SH spanners. For every node, the model only considers the shortest paths having length two, on which the node resides, and the rest of the paths are ignored. Rezvani et al. \cite{rezvani2015identifying} designed several heuristics and argued that eliminating those nodes that bridge multiple communities results in an increase in the sum of all-pair shortest distance in the network. Based on \cite{rezvani2015identifying}, Xu et al. \cite{xu2017efficient} designed fast and scalable algorithms for identifying spanner nodes. Gong et al. \cite{gong2019identifying} designed a machine learning model to discover SH spanners in the online social network. Burt \cite{ burt2011structural} studied the correlation between the strength of the links with which a node is connected to its bridged communities and the bridging advantage of that node. Based on \cite{burt2011structural}, Xu et al. \cite{xu2019identifying} designed maxBlock algorithm to discover SH spanners that connect many communities and have substantial relations with these communities. Lou et al. \cite{lou2013mining} designed a model to discover SH spanners and argued that eliminating the SH spanners from the network leads to a decrease in the minimal cut of the communities. The model requires prior community information however, community identification is an expensive process. He et al. \cite{he2016joint} designed a model that jointly discovers communities and SH spanners.