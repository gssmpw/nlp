\documentclass[11pt, reqno]{amsart}
\usepackage{amsmath, amsthm, amscd, amsfonts, amssymb, graphicx, color, mathtools, mathrsfs}
\usepackage[bookmarksnumbered, colorlinks, plainpages]{hyperref}
\usepackage{enumerate}
\usepackage{setspace}
\usepackage{multicol}
\usepackage[margin=2cm]{geometry}
% \usepackage{refcheck}
\usepackage{comment}
\usepackage{booktabs}
\usepackage{caption}
\usepackage{subcaption}
\usepackage{pgfplots}

\theoremstyle{definition}
\newtheorem{theorem}{Theorem}[section]
\newtheorem{lemma}[theorem]{Lemma}
\newtheorem{proposition}[theorem]{Proposition}
\newtheorem{corollary}[theorem]{Corollary}
\newtheorem{definition}[theorem]{Definition}
\newtheorem{example}[theorem]{Example}
\newtheorem{exercise}[theorem]{Exercise}
\newtheorem{conclusion}[theorem]{Conclusion}
\newtheorem{conjecture}[theorem]{Conjecture}
\newtheorem{criterion}[theorem]{Criterion}
\newtheorem{problem}[theorem]{Problem}
\newtheorem{facts}[theorem]{Facts}
%\theoremstyle{remark}
\newtheorem{remark}[theorem]{Remark}
\numberwithin{equation}{section}

% NEW COMMAND
\newcommand{\cal}{\mathcal}
\newcommand{\bff}{\boldsymbol}
\newcommand{\bb}{\mathbb}
\newcommand{\leqs}{\lesssim}
\newcommand{\geqs}{\gtrsim}
\newcommand{\eqs}{\eqsim}
\newcommand{\dt}{\mathrm{d}t}
\newcommand{\ddt}{\frac{\mathrm{d}}{\mathrm{d}t}}
\newcommand{\dx}{\mathrm{d}x}
\newcommand{\ds}{\mathrm{d}s}
\newcommand{\dy}{\mathrm{d}y}
\newcommand{\dz}{\mathrm{d}z}
\newcommand{\dtau}{\mathrm{d}\tau}
\newcommand{\norm}[2]{\left\|{#1}\right\|_{#2}}
\newcommand{\inpro}[2]{\left\langle#1,#2\right\rangle}
\newcommand{\abs}[1]{\left|{#1}\right|}

\newcommand{\rd}[1]{{\color{red}{#1}}}
\newcommand{\bl}[1]{{\color{blue}{#1}}}

\allowdisplaybreaks

\begin{document}
	\setcounter{page}{1}
	
	\title[Numerical analysis of the LLB equation with spin-torques]
	{Numerical analysis of the Landau--Lifshitz--Bloch equation with spin-torques}
	
	\author[Agus L. Soenjaya]{Agus L. Soenjaya}
	\address{School of Mathematics and Statistics, The University of New South Wales, Sydney 2052, Australia}
	\email{\textcolor[rgb]{0.00,0.00,0.84}{a.soenjaya@unsw.edu.au}}
	
	\date{November 4, 2024}
	
	\keywords{}
	\subjclass{}
	
	\begin{abstract}
		The current-induced magnetisation dynamics in a ferromagnet at elevated temperatures can be described by the Landau--Lifshitz--Bloch (LLB) equation with spin-torque terms. First, we establish the existence and uniqueness of the global strong solution to the model in spatial dimensions $d=1,2,3$, with an additional smallness assumption on the initial data if $d=3$. We then propose a fully discrete linearly implicit finite element scheme for the problem and prove that it approximates the solution with an optimal order of convergence, provided the exact solution possesses adequate regularity. Furthermore, we propose an unconditionally energy-stable finite element method to approximate the LLB equation without spin current. This scheme also converges to the exact solution at an optimal order, and is shown to be energy-dissipative at the discrete level. Finally, some numerical simulations to complement the theoretical analysis are included.
	\end{abstract}
	\maketitle
	
\section{Introduction}

The theory of micromagnetism is a widely used framework to analyse the behaviour of a ferromagnetic material on the micron scale. In contrast with the atomistic approach, continuum models in micromagnetics can be solved numerically for a much larger system. A standard model to describe the time evolution of magnetic configuration of a ferromagnetic body subject to some non-steady conditions is given by the Landau--Lifshitz equation~\cite{Cim08, GuoDing08, LL35}. However, numerous theoretical and experimental studies show that this model is not suitable at elevated temperatures since it cuts off all contributions from high frequency spin waves responsible for longitudinal magnetisation fluctuations~\cite{ChuNie20, GriKoc03}. This problem is especially crucial in applications since many modern magnetic devices such as the HAMR (heat-assisted magnetic recording) works at very high temperatures, often exceeding the so-called Curie temperature~\cite{MeoPan20, VogAbeBru16, ZhuLi13}.

A variant of the Landau--Lifshitz model widely used in the literature to describe the magnetisation dynamics at high temperatures is the Landau--Lifshitz--Bloch (LLB) equation, which is a second-order vector-valued quasilinear PDE. This model takes into account longitudinal dynamics and is valid below and above the critical temperature (the Curie temperature $T_c$)~\cite{AtxHinNow16, ChuNowChaGar06, Gar97}. In this paper, we focus on the LLB equation above $T_c$ and a simplified form of the model which is valid slightly below $T_c$ (the latter case also coincides with the simplified Landau--Lifshitz--Baryakhtar equation at moderate temperatures~\cite{DvoVanVan13, WanDvo15}). The general case below the Curie temperature will be addressed in a forthcoming paper. We also include spin-torque terms due to currents~\cite{ThiNakMilSuz05} in our analysis, which are important for applications in spintronics~\cite{AyoKotMouZak21, YasFasIvaMak22}.
The LLB equation with spin-torques posed in a bounded domain $\mathscr{D}\subset \bb{R}^d$ (for $d=1,2,3$) reads:
\begin{subequations}\label{equ:llb a}
	\begin{alignat}{2}
		\label{equ:llb eq1}
		&\partial_t \bff{u}
		=
		-
		\gamma \bff{u} \times \bff{H}
		+
		\alpha \bff{H}
		+
		\beta_1 (\bff{\nu}\cdot\nabla) \bff{u}
		+
		\beta_2 \bff{u}\times (\bff{\nu}\cdot\nabla) \bff{u}
		\,
		\qquad && \text{for $(t,\bff{x})\in(0,T)\times\mathscr{D}$,}
		\\
		\label{equ:llb eq2}
		&\bff{H}=\sigma \Delta \bff{u} + \kappa\mu \bff{u} - \kappa |\bff{u}|^2 \bff{u} + \lambda \bff{e} (\bff{e}\cdot \bff{u}),
		\qquad && \text{for $(t,\bff{x})\in(0,T)\times\mathscr{D}$,}
		\\
		\label{equ:llb init}
		&\bff{u}(0,\bff{x})= \bff{u_0}(\bff{x}) 
		\qquad && \text{for } \bff{x}\in \mathscr{D},
		\\
		\label{equ:llb bound}
		&\displaystyle{
			\frac{\partial \bff{u}}{\partial \bff{n}}= \bff{0}}
		\qquad && \text{for } (t,\bff{x})\in (0,T) \times \partial \mathscr{D},
	\end{alignat}
\end{subequations}
where $\gamma>0$ is the electron gyromagnetic ratio, $\alpha>0$ is the longitudinal damping constant, while $\beta_1$ and $\beta_2$ are, respectively, the adiabatic and the non-adiabatic torques constants. Moreover, $\sigma>0$ is the exchange damping constant, $\kappa=(2\chi)^{-1}$ is a positive constant related to the longitudinal susceptibility of the material, $\mu$ is a constant related to the equilibrium magnetisation, $\lambda$ is the uniaxial anisotropy constant, and $\bff{e}$ is a given unit vector denoting the easy axis of the material. The constant $\lambda$ can be positive or negative depending on the orientation of the easy axis, but for simplicity we take it as positive here. Physical considerations dictate that $\mu>0$ for temperatures above $T_c$, and $\mu<0$ for temperatures below $T_c$. Subsequently, we take $\mu>0$ and set $\sigma=\kappa=1$ for ease of presentation.


The given vector field $\bff{\nu}:[0,T]\times \mathscr{D} \to \bb{R}^d$ is the current density. The effective field $\bff{H}$ in~\eqref{equ:llb eq2} is the negative variational derivative of the micromagnetic energy functional $\mathcal{E}$, i.e. $\bff{H}=-\nabla_{\bff{u}} \mathcal{E}(\bff{u})$, where
\begin{equation}\label{equ:energy}
	\mathcal{E}(\bff{u})
	:=
	\frac{\sigma}{2} \norm{\nabla \bff{u}}{\bb{L}^2}^2
	+
	\frac{\kappa}{4} \norm{|\bff{u}|^2-\mu}{\bb{L}^2}^2
	-
	\frac{\lambda}{2} \int_\mathscr{D} (\bff{e}\cdot \bff{u})^2\, \dx.
\end{equation}
Here, we consider the exchange field (given by the term $\sigma\Delta \bff{u}$), the Ginzburg--Landau/phase transition field (given by the term $\kappa \mu \bff{u}-\kappa |\bff{u}|^2 \bff{u}$), and the anisotropy field (given by the term $\lambda \bff{e}(\bff{e}\cdot \bff{u})$). The analysis done in this paper will continue to hold if we add a time-varying applied magnetic field $\bff{B}(t)$ to $\bff{H}$, but for simplicity of presentation we take $\bff{B}(t)\equiv \bff{0}$.

Existing results in the literature which are relevant to the present paper will be reviewed next. We remark that many variants of the standard Landau--Lifshitz equation attract significant interest in the mathematical literature, a non-exhaustive list includes~\cite{AbeHrkPagPra14, DavDiPraRug22, FeiTra17a, MelPta13, Rug22, SoeTra23}, which take into account various effects not included in the standard model. Specifically for the LLB equation, a rigorous analysis of the model (without spin-torques) is initiated in~\cite{Le16}, where the existence of a weak solution is obtained. The existence, uniqueness, and decay of the strong solution are discussed in~\cite{LeSoeTra24}. These papers, however, do not take into account the spin-torque terms in the analysis. The existence of a weak solution (but without uniqueness in the general case) for the LLB equation with spin-torques is shown in~\cite{AyoKotMouZak21}, but the existence of a strong solution is not known yet. Spin-torque effects have also been considered for the Landau--Lifshitz--Gilbert equation~\cite{VinTra24}, where a local-in-time strong solution is obtained in that case.

On the aspect of numerical analysis, several fully discrete finite element methods to approximate the solution of the LLB equation without spin-torques are proposed in~\cite{LeSoeTra24} and~\cite{Soe24}. However, numerical schemes developed in these papers actually approximate the regularised version of the equation, not the LLB equation itself. The strong solution of the regularised problem is shown to converge to that of~\eqref{equ:llb a} as the regularisation parameter goes to zero. As such, the order of convergence of the numerical scheme is also dependent on this parameter, which theoretically blow up as the regularisation parameter vanishes. Similar approach using regularisation is also taken in the stochastic case~\cite{GolJiaLe24}, where a suboptimal strong order of convergence is obtained for $d=1$ and $2$. In~\cite{BenEssAyo24}, a finite element scheme is proposed to approximate the LLB equation without spin-torques directly, but no order of convergence is given and the energy dissipativity is not guaranteed. Moreover, this scheme is nonlinear, in that a system of nonlinear equations needs to be solved at each time-step. In any case, numerical scheme to approximate the LLB equation with spin-torques~\eqref{equ:llb a} has not been developed yet in the literature.

This paper aims to address the aforementioned gaps. Firstly, we show the global existence and uniqueness of strong solution to~\eqref{equ:llb a} for $d=1,2,3$. We then propose a fully discrete linearly implicit numerical scheme which apply to~\eqref{equ:llb a} directly without the need for regularisation. In the absence of spin current ($\bff{\nu}=\bff{0}$), a special fully discrete scheme is also provided, which maintains the energy dissipativity at the discrete level. Indeed, it is desirable to have a numerical method that preserves discrete analogues of the energy dissipation law or other invariants of the PDEs, as this leads to an approximation which behave qualitatively similar to the exact solution. These schemes are shown to converge to the exact solution at an optimal order. As far as we know, these results are new even for the case where the spin-torques are absent. Some numerical experiments to corroborate the analysis are also provided.

To summarise, our main contributions in this paper are:
\begin{enumerate}
	\item Proving the existence and uniqueness of global strong solution to the LLB equation with spin-torques, unconditionally when $d=1$ or $2$, and for small initial data when $d=3$, which build on the results in~\cite{AyoKotMouZak21, Le16}. Further smoothing estimates are also provided.
	\item Proposing a fully discrete linearly implicit finite element scheme to approximate~\eqref{equ:llb a} with an optimal order of convergence, which removes the need for regularisation in~\cite{LeSoeTra24} while also taking spin-torque terms into account.
	\item Proposing a fully discrete finite element scheme to approximate the LLB equation without spin current, which maintains energy dissipativity at the discrete level. An optimal order of convergence is also shown, improving on the results in~\cite{BenEssAyo24}.
\end{enumerate}
The existence of a global strong solution is shown by deriving further a priori estimates in stronger norms. These, in turn, depend on a uniform estimate in $\bb{L}^\infty$-norm which, unlike the case of the standard Landau--Lifshitz equation, is not immediate since the magnetisation length is not conserved. Some smoothing estimates are also derived to reflect the instantaneous regularisation of the strong solution. We then proceed to propose some finite element schemes. By using an elliptic projection adapted to the problem at hand, we show an optimal order of convergence for the linear finite element scheme to approximate~\eqref{equ:llb a}. In the absence of spin current, a nonlinear scheme is devised to preserve energy dissipativity. Detailed analysis is done to show optimal order of convergence in $\bb{L}^2$, $\bb{L}^\infty$, and $\bb{H}^1$-norms. Finally, some numerical simulations are performed to support the theoretical results.




\section{Preliminaries}

\subsection{Notations}
We begin by defining some notations used in this paper. The function space $\bb{L}^p := \bb{L}^p(\mathscr{D}; \bb{R}^3)$ denotes the usual space of $p$-th integrable functions taking values in $\bb{R}^3$ and $\bb{W}^{k,p} := \bb{W}^{k,p}(\mathscr{D}; \bb{R}^3)$ denotes the usual Sobolev space of 
functions on $\mathscr{D} \subset \bb{R}^d$ taking values in $\bb{R}^3$. We
write $\bb{H}^k := \bb{W}^{k,2}$. Here, $\mathscr{D}\subset \bb{R}^d$ for $d=1,2,3$
is an open domain with $C^2$-smooth boundary. The Laplacian operator acting on $\bb{R}^3$-valued functions is denoted by $\Delta$ with domain $\bb{H}^2_{\bff{n}}$ given by
\begin{equation*}
	\bb{H}^2_{\bff{n}}:= \left\{\bff{v}\in \bb{H}^2 : \frac{\partial \bff{v}}{\partial \bff{n}} = \bff{0} \text{ on } \partial\mathscr{D} \right\}.
\end{equation*}

If $X$ is a Banach space, the spaces $L^p(0,T; X)$ and $W^{k,p}(0,T;X)$ denote respectively the usual Lebesgue and Sobolev spaces of functions on $(0,T)$ taking values in $X$. The space $C([0,T];X)$ denotes the space of continuous functions on $[0,T]$ taking values in $X$. For simplicity, we will write $L^p(\bb{W}^{m,r}) := L^p(0,T; \bb{W}^{m,r})$ and $L^p(\bb{L}^q) := L^p(0,T; \bb{L}^q)$. Throughout this paper, we denote the scalar product in a Hilbert space $H$ by $\inpro{\cdot}{\cdot}_H$ and its corresponding norm by $\|\cdot\|_H$. We will not distinguish between the scalar product of $\bb{L}^2$ vector-valued functions taking values in $\bb{R}^3$ and the scalar product of $\bb{L}^2$ matrix-valued functions taking values in $\bb{R}^{3\times 3}$, and denote them by $\langle\cdot,\cdot\rangle$.

Finally, throughout this paper, the constant $C$ in the estimate denotes a generic constant which takes different values at different occurrences. If
the dependence of $C$ on some variable, e.g.~$T$, is highlighted, we will write
$C(T)$.


\subsection{Formulations and assumptions}

Here, we formulate the problem more precisely and state several assumptions which will be used throughout the paper. For simplicity, we set $\sigma=\kappa=1$ in~\eqref{equ:llb a} and assume that $\bff{\nu}$ is a given vector field such that $\norm{\bff{\nu}}{L^\infty(0,T;\bb{L}^\infty(\mathscr{D};\,\bb{R}^d))} \leq \nu_\infty$, where $\nu_\infty$ is a positive constant.

\begin{definition}\label{def:strong sol}
Given $T>0$ and $\bff{u}_0\in \bb{H}^1(\mathscr{D})$, a \emph{strong solution} to \eqref{equ:llb a} is a function
\[
\bff{u}\in H^1(0,T;\bb{L}^2) \cap C([0,T];\bb{H}^1) \cap L^2(0,T; \bb{H}^2)
\]
such that for all $\bff{\chi}\in \bb{H}^1$ and $t\in [0,T]$,
\begin{align}\label{equ:weakform}
	\inpro{\partial_t \bff{u}(t)}{\bff{\chi}}
	&=
	-
	\gamma \inpro{\bff{u}(t)\times \big(\Delta \bff{u}(t)+ \lambda \bff{e}(\bff{e}\cdot \bff{u}(t) \big)}{\bff{\chi}}
	+
	\alpha \inpro{\big(\Delta \bff{u}(t)+ \bff{w}(t)\big)}{\bff{\chi}}
	\nonumber \\
	&\quad
	+
	\beta_1 \inpro{(\bff{\nu}(t) \cdot\nabla) \bff{u}(t)}{\bff{\chi}}
	+
	\beta_2 \inpro{\bff{u}(t)\times (\bff{\nu}(t)\cdot \nabla) \bff{u}(t)}{\bff{\chi}},
\end{align}
where for almost every $t\in [0,T]$,
\begin{align}\label{equ:w}
	\bff{w}(t)
	= 
	\mu \bff{u}(t) 
	- 
	 |\bff{u}(t)|^2 \bff{u}(t) 
	+ 
	\lambda \bff{e} \big(\bff{e}\cdot \bff{u}(t) \big).
\end{align}
In this case, \eqref{equ:llb a} is satisfied for almost every $(t,x)\in (0,T)\times \mathscr{D}$.
\end{definition}
At times, we write $\bff{H}:=\Delta \bff{u} + \bff{w}$, where $\bff{w}$ in~\eqref{equ:w} represents the remaining terms in the effective field.
Note that by the vector identity~\eqref{equ:div ab}, we can write
\begin{align}
	\label{equ:otimes 1}
	\beta_1 (\bff{\nu}\cdot \nabla) \bff{u}
	&=
	\beta_1 \nabla \cdot (\bff{u} \otimes \bff{\nu}) 
	- \beta_1 (\nabla \cdot \bff{\nu}) \bff{u},
	\\
	\label{equ:otimes 2}
	\beta_2 \bff{u}\times (\bff{\nu}\cdot \nabla)\bff{u} 
	&=
	\beta_2 \bff{u} \times \nabla \cdot (\bff{u} \otimes \bff{\nu})
	- \beta_2 \bff{u} \times (\nabla \cdot \bff{\nu}) \bff{u},
\end{align}
where the last term on the right-hand side of~\eqref{equ:otimes 2} is a zero vector.
Thus, if we assume that $\bff{\nu}$ is either tangential to the boundary ($\bff{\nu}\cdot \bff{n}=0$ on $\partial \mathscr{D}$) or vanishing at the boundary ($\bff{\nu}=\bff{0}$ on $\partial\mathscr{D}$), which is physically reasonable, then by~\eqref{equ:otimes 1}, \eqref{equ:otimes 2}, and the divergence theorem,
\begin{align*}
	\beta_1 \inpro{(\bff{\nu}(t) \cdot\nabla) \bff{u}(t)}{\bff{\chi}}
	&=
	-
	\beta_1 \inpro{\bff{u}(t) \otimes \bff{\nu}(t)}{\nabla \bff{\chi}}
	-
	\beta_1 \inpro{(\nabla \cdot \bff{\nu}(t)) \bff{u}(t)}{\bff{\chi}}
	\\
	\beta_2 \inpro{\bff{u}(t)\times (\bff{\nu}(t)\cdot \nabla) \bff{u}(t)}{\bff{\chi}}
	&=
	-
	\beta_2 \inpro{\bff{u}(t) \otimes \bff{\nu}(t)}{\nabla(\bff{u}(t) \times \bff{\chi})}.
\end{align*}
Therefore, in this case we can also write~\eqref{equ:weakform} as
\begin{align}\label{equ:weak special}
	\inpro{\partial_t \bff{u}(t)}{\bff{\chi}}
	&=
	\gamma \inpro{\bff{u}(t) \times \nabla \bff{u}(t)}{\nabla \bff{\chi}}
	-
	\lambda\gamma \inpro{\bff{u}(t)\times \bff{e}(\bff{e}\cdot \bff{u}(t))}{\bff{\chi}}
	-
	\alpha \inpro{\nabla \bff{u}(t)}{\nabla \bff{\chi}}
	+
	\alpha \inpro{\bff{w}(t)}{\bff{\chi}} 
	\nonumber \\
	&\quad
	-
	\beta_1 \inpro{\bff{u}\otimes \bff{\nu}}{\nabla \bff{\chi}}
	-
	\beta_1 \inpro{(\nabla \cdot \bff{\nu}) \bff{u}}{\bff{\chi}}
	-
	\beta_2 \inpro{\bff{u}\otimes \bff{\nu}}{\nabla(\bff{u}\times \bff{\chi})},
\end{align}
which is the weak formulation that we will use in Section~\ref{sec:llb spin}.


Throughout this paper, to ensure optimal order of convergence for the underlying approximation, we assume that the problem~\eqref{equ:llb a} admits a sufficiently regular solution~$\bff{u}$ which satisfies
\begin{equation}\label{equ:ass 1}
	\norm{\bff{u}}{L^\infty(\bb{H}^{r+1})}
	+ \norm{\partial_t \bff{u}}{L^\infty(\bb{H}^{r+1})}
	+ \norm{\partial_t^2 \bff{u}}{L^\infty(\bb{L}^2)}
	\leq K_0,
\end{equation}
for some positive constant $K_0$ depending on the initial data and $T$. Here, $r$ is the degree of piecewise polynomials in the finite element space~\eqref{equ:Vh}.

Note that the existence of a strong solution satisfying~\eqref{equ:ass 1} for $r=1$ is given by Theorem~\ref{the:main existence}. The existence of a more regular solution can be shown in a similar manner.


\subsection{Finite element approximation}

Let $\mathscr{D}\subset \bb{R}^d$, $d=1,2, 3$, be a smooth or convex polygonal (or convex polyhedral) domain. Let $\mathcal{T}_h$ be a quasi-uniform triangulation of $\mathscr{D}$
into intervals (in 1D), triangles (in 2D), or tetrahedra (in 3D)
with maximal mesh-size $h$.
To discretise the LLB equation, we
introduce the conforming finite element space $\bb{V}_h \subset \bb{H}^1$ given by
\begin{equation}\label{equ:Vh}
\bb{V}_h := \{\bff{\phi}_h \in \bff{C}(\overline{\mathscr{D}}; \bb{R}^3): \bff{\phi}|_K \in \cal{P}_r(K;\bb{R}^3), \; \forall K \in \cal{T}_h\},
\end{equation}
where $\cal{P}_r(K; \bb{R}^3)$ denotes the space of polynomials of degree $r$ on $K$ taking values in $\bb{R}^3$. The case $r=1$ (linear polynomials) or $r=2$ (quadratic polynomials) are most commonly used.

As a consequence of the Bramble--Hilbert lemma, for $p\in [1,\infty]$, there exists a constant $C$ independent of $h$ such that for any $\bff{v} \in \bb{W}^{r+1,p}$, we have
\begin{align}\label{equ:fin approx}
	\inf_{\chi \in {\bb{V}}_h} \left\{ \norm{\bff{v} - \bff{\chi}}{\bb{L}^p} 
	+ 
	h \norm{\nabla (\bff{v}-\bff{\chi})}{\bb{L}^p} 
	\right\} 
	\leq 
	C h^{r+1} \norm{\bff{v}}{\bb{W}^{r+1,p}}.
\end{align}
Moreover, if the triangulation $\mathcal{T}_h$ is quasi-uniform, we have the following inverse estimate:
\begin{align}\label{equ:inverse}
	\norm{\bff{\chi}}{\bb{W}^{1,p}} \leq Ch^{-d \left(\frac12-\frac{1}{p}\right)} \norm{\bff{\chi}}{\bb{H}^1}, \quad \forall \bff{\chi}\in \bb{V}_h.
\end{align}


In the analysis, we will use several projection and interpolation operators. The existence of such operators and the properties that they possess will be described below (also see~\cite{BreSco08, CroTho87, DouDupWah74}). Sufficient regularity conditions on the mesh will be assumed as needed so that the following stability and approximation properties hold.

Firstly, there exists an orthogonal projection operator $P_h: \bb{L}^2 \to \bb{V}_h$ such that
\begin{align}\label{equ:orth proj}
	\inpro{P_h \bff{v}-\bff{v}}{\bff{\chi}}=0,
	\quad
	\forall \bff{\chi}\in \bb{V}_h,
\end{align}
with the property that for any $\bff{v}\in \bb{W}^{r+1,p}$,
\begin{align}\label{equ:proj approx}
	\norm{\bff{v}- P_h\bff{v}}{\bb{L}^p}
	+
	h \norm{\nabla( \bff{v}-P_h\bff{v})}{\bb{L}^p}
	\leq
	Ch^{r+1} \norm{\bff{v}}{\bb{W}^{r+1,p}}.
\end{align}
Furthermore, there exists a nodal interpolation operator $\mathcal{I}_h:\bb{H}^1 \to \bb{V}_h$ that satisfies
\begin{align}\label{equ:interp approx}
	\norm{\bff{v}- \mathcal{I}_h \bff{v}}{\bb{L}^p}
	+
	h \norm{\nabla \left(\bff{v}-\mathcal{I}_h \bff{v}\right)}{\bb{L}^p}
	\leq
	Ch^{r+1} \norm{\bff{v}}{\bb{W}^{r+1,p}}.
\end{align}
Next, we introduce the discrete Laplacian operator $\Delta_h: \bb{V}_h \to \bb{V}_h$ defined by
\begin{align}\label{equ:disc laplacian}
	\inpro{\Delta_h \bff{v}_h}{\bff{\chi}}
	=
	- \inpro{\nabla \bff{v}_h}{\nabla \bff{\chi}},
	\quad 
	\forall \bff{v}_h, \bff{\chi} \in \bb{V}_h,
\end{align}
as well as the Ritz projection $R_h: \bb{H}^1 \to \bb{V}_h$ defined by
\begin{align}\label{equ:Ritz}
	\inpro{\nabla R_h \bff{v}- \nabla \bff{v}}{\nabla \bff{\chi}}=0,
	\quad
	\forall \bff{\chi}\in \bb{V}_h.
\end{align}
If $\bff{v}\in \bb{H}^2_{\bff{n}}$, then we have $\Delta_h R_h\bff{v}=P_h\Delta \bff{v}$.
For any $\bff{v}\in \bb{H}^2$, let $\bff{\omega}:= \bff{v}-R_h\bff{v}$. The approximation property for the Ritz projection is assumed to hold \cite{LeyLi21, LinThoWah91, RanSco82}, namely for $s=0$ or $1$ and $p\in (1,\infty)$:
\begin{align}\label{equ:Ritz ineq}
	\norm{\bff{\omega}(t)}{\bb{W}^{s,p}} + \norm{\partial_t \bff{\omega}(t)}{\bb{W}^{s,p}}
	&\leq
	C h^{r+1-s} \norm{\bff{v}(t)}{\bb{W}^{r+1, p}},
	\\
	\label{equ:Ritz ineq L infty}
	\norm{\bff{\omega}(t)}{\bb{L}^\infty} 
	&\leq 
	Ch^{r+1} \abs{\ln h} \norm{\bff{v}(t)}{\bb{W}^{r+1,\infty}}.
\end{align}
We also have the $\bb{W}^{1,\infty}$ stability property of the operator $R_h$, namely~\cite{DemLeySchWah12, Li22, Sco76}:
\begin{align}
	\label{equ:Ritz stab u infty}
	\norm{R_h \bff{v}(t)}{\bb{W}^{1,\infty}}
	&\leq 
	C \norm{\bff{v}(t)}{\bb{W}^{1,\infty}}.
\end{align}
A condition which would guarantee all the stability and approximation properties listed above to hold is quasi-uniformity of the triangulation (which we assumed for $\mathcal{T}_h$). However, they are also known to hold for mesh which is less regular (only locally quasi-uniform or mildly graded). We will not discuss these further, but instead refer interested readers to the references cited above.


\subsection{Auxiliary results}

We need some auxiliary results in our analysis. Firstly, the following vector identities will be used: For all $\bff{a},\bff{b}\in \bb{R}^3$,
\begin{align}
	\label{equ:div ab}
	\nabla \cdot (\bff{a}\otimes \bff{b})
	&=
	(\nabla \cdot \bff{b}) \bff{a}
	+
	(\bff{b}\cdot \nabla) \bff{a},
	\\
	\label{equ:aab}
	2\bff{a} \cdot (\bff{a}-\bff{b}) 
	&= |\bff{a}|^2 - |\bff{b}|^2 + |\bff{a}-\bff{b}|^2.
\end{align}

Next, some frequently used interpolation inequalities are collected in the following lemma.

\begin{lemma}
Let $\epsilon>0$ be given and $d\in \{1,2,3\}$. Then there exists a positive constant $C$ such that the following inequalities hold.
\begin{enumerate}
	\renewcommand{\labelenumi}{\theenumi}
	\renewcommand{\theenumi}{{\rm (\roman{enumi})}}	
	\item For any $\bff{v}\in \bb{H}^1$,
	\begin{align}\label{equ:gal nir uh L4}
		\norm{\bff{v}}{\bb{L}^4}
		&\leq
		C \norm{\bff{v}}{\bb{H}^1}^{\frac{d}{4}}
		\norm{\bff{v}}{\bb{L}^2}^{1-\frac{d}{4}}
		\leq
		C \norm{\bff{v}}{\bb{L}^2}^2
		+
		\epsilon \norm{\nabla \bff{v}}{\bb{L}^2}^2.
	\end{align}
	\item Let $\mathscr{D}$ be a convex polygonal or polyhedral domain with globally quasi-uniform triangulation. For any $\bff{v}_h\in \bb{V}_h$,
	\begin{align}
		\label{equ:disc lapl L infty}
		\norm{\bff{v}_h}{\bb{L}^\infty}
		&\leq
		C \norm{\bff{v}_h}{\bb{L}^2}^{1-\frac{d}{4}} \left(\norm{\bff{v}_h}{\bb{L}^2}^\frac{d}{4} + \norm{\Delta_h \bff{v}_h}{\bb{L}^2}^\frac{d}{4} \right).
	\end{align}
\end{enumerate}
\end{lemma}

\begin{proof}
Estimate \eqref{equ:gal nir uh L4} follows from the Gagliardo--Nirenberg and the Young inequalities. Inequality \eqref{equ:disc lapl L infty} is shown in \cite[Appendix A]{GuiLiWan22}).
\end{proof}




\section{Existence and uniqueness of strong solution}

To show the global existence and uniqueness of solution to the problem~\eqref{equ:llb a}, one could apply the Faedo--Galerkin method and establish appropriate uniform estimates on the approximate solution. In order to simplify presentation, we will work directly with the solution $\bff{u}$ (instead of its approximation). These a priori estimates can be made rigorous by working with the Galerkin approximation as in~\cite{AyoKotMouZak21, Le16}.

\begin{proposition}
Let $\bff{u}(t)$ be a solution of~\eqref{equ:llb a} and let $p\in [2,\infty)$. For all $t\in [0,T]$,
\begin{align}\label{equ:ut Lp}
	\norm{\bff{u}(t)}{\bb{L}^p}^p
	+
	\int_0^t p\norm{|\bff{u}(s)|^{\frac{p-2}{2}} |\nabla \bff{u}(s)|}{\bb{L}^2}^2 \ds
	+
	\int_0^t p\norm{\bff{u}(s)}{\bb{L}^{p+2}}^{p+2} \ds 
	\leq
	C_p (1+t),
\end{align}
where $C_p$ is a constant depending on $p$, $\norm{\bff{u}_0}{\bb{L}^p}$, and the coefficients of \eqref{equ:llb a}. Furthermore,
\begin{equation}\label{equ:u Linfty}
	\norm{\bff{u}(t)}{\bb{L}^\infty}
	\leq 
	\norm{\bff{u}_0}{\bb{L}^\infty} + \sqrt{\beta},
\end{equation}
where $\beta:=\mu+\lambda+(\beta_1\nu_\infty)^2 \alpha^{-1}$.
\end{proposition}

\begin{proof}
Taking the inner product of \eqref{equ:llb eq1} with $|\bff{u}|^{p-2} \bff{u}$, we obtain
\begin{align*}
	\frac{1}{p} \ddt \norm{\bff{u}}{\bb{L}^p}^p
	+
	\alpha
	\inpro{\nabla \bff{u}}{\nabla (\abs{\bff{u}}^{p-2} \bff{u})}
	+
	\norm{\bff{u}}{\bb{L}^{p+2}}^{p+2}
	=
	\mu \norm{\bff{u}}{\bb{L}^p}^p
	+
	\lambda \norm{|\bff{u}|^{\frac{p-2}{2}} (\bff{e}\cdot \bff{u})}{\bb{L}^2}^2
	+
	\beta_1 \inpro{(\bff{\nu}\cdot \nabla)\bff{u}}{|\bff{u}|^{p-2} \bff{u}},
\end{align*}
which implies by Young's inequality,
\begin{align*}
	&\frac{1}{p} \ddt \norm{\bff{u}}{\bb{L}^p}^p
	+
	\alpha \norm{|\bff{u}|^{\frac{p-2}{2}} |\nabla \bff{u}|}{\bb{L}^2}^2
	+
	\alpha (p-2) \norm{|\bff{u}|^{\frac{p-4}{2}} |\bff{u}\cdot\nabla\bff{u}|}{\bb{L}^2}^2
	\\
	&\leq
	(\mu+\lambda) \norm{\bff{u}}{\bb{L}^p}^p
	+
	\beta_1 \nu_\infty \norm{|\bff{u}|^{\frac{p-2}{2}} |\nabla \bff{u}|}{\bb{L}^2} \norm{|\bff{u}|^{\frac{p}{2}}}{\bb{L}^2}
	-
	\norm{\bff{u}}{\bb{L}^{p+2}}^{p+2}
	\\
	&\leq
	\left(\mu+\lambda+ \frac{(\beta_1 \nu_\infty)^2}{\alpha} \right) \norm{\bff{u}}{\bb{L}^p}^p
	-
	\norm{\bff{u}}{\bb{L}^{p+2}}^{p+2}
	+
	\frac{\alpha}{4} \norm{|\bff{u}|^{\frac{p-2}{2}} |\nabla \bff{u}|}{\bb{L}^2}^2.
\end{align*}
After rearranging the terms, we have
\begin{equation}\label{equ:ut Lpp}
\ddt \norm{\bff{u}}{\bb{L}^p}^p 
+
\frac{\alpha p}{2} \norm{|\bff{u}|^{\frac{p-2}{2}} |\nabla \bff{u}|}{\bb{L}^2}^2
\leq
\beta p \norm{\bff{u}}{\bb{L}^p}^p - p\norm{\bff{u}}{\bb{L}^{p+2}}^{p+2}.
\end{equation}
By considering the maximum value of the function $x\mapsto \beta x^p-x^{p+2}$ for $p\geq 2$, we have the inequality
\begin{align*}
	\beta \norm{\bff{u}}{\bb{L}^p}^p
	-
	\norm{\bff{u}}{\bb{L}^{p+2}}^{p+2}
	\leq
	\left(\frac{2\beta |\mathscr{D}|}{p+2}\right) \left(\frac{\beta p}{p+2}\right)^{\frac{p}{2}}.
\end{align*}
Thus, integrating~\eqref{equ:ut Lpp} over $(0,t)$, we obtain~\eqref{equ:ut Lp}. In particular, rearranging the terms and taking the $p$-th root, we have
\begin{equation*}
	\norm{\bff{u}(t)}{\bb{L}^p}
	\leq
	\norm{\bff{u}_0}{\bb{L}^p}
	+
	\left(\frac{2\beta |\mathscr{D}|t}{p+2}\right)^{\frac{1}{p}} \left(\frac{\beta p}{p+2}\right)^{\frac{1}{2}}.
\end{equation*}
Letting $p\to\infty$ then yield~\eqref{equ:u Linfty}. This completes the proof of the proposition.
\end{proof}


\begin{remark}\label{rem:coeff above Tc}
If $\mu$ and $\lambda$ are negative, and the current density $\bff{\nu}$ is divergence-free and vanishing on the boundary of $\mathscr{D}$, then we can obtain an exponential decay estimate
\[
	\norm{\bff{u}(t)}{\bb{L}^\infty} 
	\leq
	e^{-|\mu| t} \norm{\bff{u}_0}{\bb{L}^\infty},
\]
as in~\cite{LeSoeTra24}.
\end{remark}

\begin{proposition}
	Let $\bff{u}(t)$ be a solution of~\eqref{equ:llb a}. For all $t\in [0,T]$,
	\begin{align}\label{equ:nab u L2}
		\norm{\nabla \bff{u}(t)}{\bb{L}^2}^2
		+
		\int_0^t \norm{\Delta \bff{u}(s)}{\bb{L}^2}^2 \ds
		\leq
		C(1+t),
	\end{align}
	where $C$ is a constant depending on $\norm{\bff{u}_0}{\bb{H}^1}$ and the coefficients of~\eqref{equ:llb a}.
\end{proposition}

\begin{proof}
Taking the inner product of~\eqref{equ:llb eq1} with $-\Delta \bff{u}$, we have
\begin{align*}
	\frac12 \ddt \norm{\nabla \bff{u}}{\bb{L}^2}^2
	+
	\alpha \norm{\Delta \bff{u}}{\bb{L}^2}^2
	+
	\mu \norm{|\bff{u}||\nabla\bff{u}|}{\bb{L}^2}^2
	&\leq
	\mu \norm{\nabla \bff{u}}{\bb{L}^2}^2
	+
	\gamma \lambda \inpro{\bff{u}\times \bff{e}(\bff{e}\cdot \bff{u})}{\Delta \bff{u}}
	\\
	&\quad
	+
	\beta_1 \inpro{(\bff{\nu}\cdot \nabla)\bff{u}}{\Delta \bff{u}}
	+
	\beta_2 \inpro{\bff{u}\times (\bff{\nu}\cdot \nabla) \bff{u}}{\Delta \bff{u}}
	\\
	&=:
	I_1+I_2+I_3+I_4.
\end{align*}
The term $I_1$ is left as is. For the rest of the terms, by Young's inequality we have
\begin{align*}
	\abs{I_2}
	&\leq
	C\norm{\bff{u}}{\bb{L}^4}^4
	+
	\frac{\alpha}{4} \norm{\Delta \bff{u}}{\bb{L}^2}^2,
	\\
	\abs{I_3}
	&\leq
	C\nu_\infty \norm{\nabla \bff{u}}{\bb{L}^2}^2
	+
	\frac{\alpha}{4} \norm{\Delta \bff{u}}{\bb{L}^2}^2,
	\\
	\abs{I_4}
	&\leq
	C\nu_\infty \norm{|\bff{u}| |\nabla \bff{u}|}{\bb{L}^2}^2
	+
	\frac{\alpha}{4} \norm{\Delta \bff{u}}{\bb{L}^2}^2.
\end{align*}
Rearranging and integrating over $(0,t)$, noting~\eqref{equ:ut Lp} and~\eqref{equ:u Linfty}, we obtain the required inequality.
\end{proof}


\begin{proposition}\label{pro:Delta u L2}
	Let $\bff{u}(t)$ be a solution of~\eqref{equ:llb a}. If $d=1$ or $2$, then for all $t\in [0,T]$,
	\begin{align}\label{equ:Delta u L2}
		\norm{\Delta \bff{u}(t)}{\bb{L}^2}^2
		+
		\int_0^t \norm{\nabla \Delta \bff{u}(s)}{\bb{L}^2}^2 \ds
		\leq
		Ce^{C t},
	\end{align}
	where $C$ is a constant depending on the coefficients of~\eqref{equ:llb a} and $\norm{\bff{u}_0}{\bb{H}^2}$. 
	
	If $d=3$, then the inequality~\eqref{equ:Delta u L2} holds under an additional assumption that $\norm{\bff{u}_0}{\bb{L}^\infty}+\beta \lesssim \sqrt{\alpha/\gamma}$, where $\beta$ was defined in~\eqref{equ:u Linfty}.
\end{proposition}

\begin{proof}
Taking the inner product of~\eqref{equ:llb eq1} with $\Delta^2 \bff{u}$ and integrating by parts as necessary, we obtain
\begin{align*}
	&\frac12 \ddt \norm{\Delta \bff{u}}{\bb{L}^2}^2
	+
	\alpha \norm{\nabla\Delta \bff{u}}{\bb{L}^2}^2
	\\
	&=
	\gamma \inpro{\nabla\bff{u}\times \Delta \bff{u}}{\nabla\Delta \bff{u}}
	+
	\mu \norm{\Delta \bff{u}}{\bb{L}^2}^2
	-
	\lambda\gamma \inpro{\nabla \big(\bff{u}\times \bff{e}(\bff{e}\cdot \bff{u})\big)}{\nabla\Delta \bff{u}}
	+
	\lambda \norm{\bff{e}\cdot \Delta \bff{u}}{\bb{L}^2}^2
	\\
	&\quad
	+
	\inpro{\nabla(|\bff{u}|^2 \bff{u})}{\nabla\Delta \bff{u}}
	+
	\beta_1 \inpro{\nabla\big((\bff{\nu} \cdot \nabla)\bff{u}\big)}{\nabla\Delta \bff{u}}
	-
	\beta_2 \inpro{\nabla\big(\bff{u}\times (\bff{\nu}\cdot\nabla)\bff{u}\big)}{\nabla\Delta \bff{u}}
	\\
	&=: J_1+J_2+\cdots+J_7.
\end{align*}
The term $J_1$ will be handled last, while the term $J_2$ is kept as is. For the term $J_3$, noting~\eqref{equ:u Linfty} and~\eqref{equ:nab u L2}, by H\"older's and Young's inequalities we have
\begin{align*}
	\abs{J_3}
	&\leq
	2\lambda\gamma \norm{\bff{u}}{\bb{L}^\infty} \norm{\nabla\bff{u}}{\bb{L}^2} \norm{\nabla\Delta \bff{u}}{\bb{L}^2}
	\leq
	C(1+t) + \frac{\alpha}{8} \norm{\nabla\Delta \bff{u}}{\bb{L}^2}^2.
\end{align*}
For the next two terms, we easily have~$\abs{J_4}\leq \lambda \norm{\Delta \bff{u}}{\bb{L}^2}^2$ and
\begin{align*}
	\abs{J_5}
	&\leq
	C\norm{\nabla(|\bff{u}|^2 \bff{u})}{\bb{L}^2}^2
	+
	\frac{\alpha}{8} \norm{\nabla\Delta \bff{u}}{\bb{L}^2}^2
	\leq
	C(1+t)+ \frac{\alpha}{8} \norm{\nabla\Delta \bff{u}}{\bb{L}^2}^2.
\end{align*}
For the term $J_6$, by similar argument we have
\begin{align*}
	\abs{J_6}
	&\leq
	C\nu_\infty (1+t) \left(1+ \norm{\Delta \bff{u}}{\bb{L}^2}^2 \right) 
	+
	\frac{\alpha}{8} \norm{\nabla\Delta \bff{u}}{\bb{L}^2}^2.
\end{align*}
For the last term, by the H\"older and the Gagliardo--Nirenberg inequalities we have
\begin{align*}
	\abs{J_7}
	&\leq
	C\nu_\infty \norm{\nabla \bff{u}}{\bb{L}^4}^2 \norm{\nabla\Delta \bff{u}}{\bb{L}^2}
	+
	C\nu_\infty \norm{\bff{u}}{\bb{L}^\infty} \norm{\nabla \bff{u}}{\bb{L}^2} \norm{\nabla\Delta \bff{u}}{\bb{L}^2}
	+
	C\nu_\infty \norm{\bff{u}}{\bb{L}^\infty} \norm{\bff{u}}{\bb{H}^2} \norm{\nabla\Delta \bff{u}}{\bb{L}^2}
	\\
	&\leq
	C\nu_\infty \norm{\bff{u}}{\bb{L}^\infty} \norm{\bff{u}}{\bb{H}^2} \norm{\nabla\Delta \bff{u}}{\bb{L}^2}
	\\
	&\leq
	C(1+t) \left(1+\norm{\Delta \bff{u}}{\bb{L}^2}^2 \right)
	+
	\frac{\alpha}{8} \norm{\nabla\Delta \bff{u}}{\bb{L}^2}^2.
\end{align*}
It remains to estimate $J_1$. To this end, we will consider the cases $d=1,2$, and $3$ separately.

\smallskip
\underline{Case 1 ($d=1$):} By H\"older's and Agmon's inequalities (noting~\eqref{equ:nab u L2}), we have
\begin{align*}
	\abs{J_1}
	&\leq
	\gamma \norm{\nabla \bff{u}}{\bb{L}^2} \norm{\Delta \bff{u}}{\bb{L}^\infty} \norm{\nabla\Delta \bff{u}}{\bb{L}^2}
	\\
	&\leq
	C \norm{\nabla \bff{u}}{\bb{L}^2} \norm{\Delta \bff{u}}{\bb{L}^2}^{\frac12} \norm{\Delta \bff{u}}{\bb{H}^1}^{\frac12} \norm{\nabla\Delta \bff{u}}{\bb{L}^2}
	\leq
	C(1+t)\norm{\Delta \bff{u}}{\bb{L}^2}^2
	+
	\frac{\alpha}{8} \norm{\nabla\Delta \bff{u}}{\bb{L}^2}^2.
\end{align*}

\smallskip
\underline{Case 2 ($d=2$):} By the Gagliardo--Nirenberg inequalities (noting~\eqref{equ:nab u L2}), we have
\begin{align*}
	\abs{J_1}
	&\leq
	\gamma \norm{\nabla\bff{u}}{\bb{L}^4} \norm{\Delta \bff{u}}{\bb{L}^4} \norm{\nabla\Delta \bff{u}}{\bb{L}^2}
	\\
	&\leq
	C \norm{\nabla \bff{u}}{\bb{L}^2}^{\frac12} \norm{\Delta \bff{u}}{\bb{L}^2} \norm{\nabla\Delta \bff{u}}{\bb{L}^2}^{\frac32}
	\leq
	C(1+t)\norm{\Delta \bff{u}}{\bb{L}^2}^4
	+
	\frac{\alpha}{8} \norm{\nabla\Delta \bff{u}}{\bb{L}^2}^2.
\end{align*}

\smallskip
\underline{Case 3 ($d=3$):} 
We note the Gagliardo--Nirenberg inequalities
\begin{align}
	\label{equ:BGL}
	\norm{\nabla \bff{u}}{\bb{L}^6} 
	&\leq
	B_{\mathrm{GL}} \norm{\bff{u}}{\bb{L}^\infty}^{\frac23} \norm{\bff{u}}{\bb{H}^3}^{\frac13}
	\\
	\label{equ:CGL}
	\norm{\Delta \bff{u}}{\bb{L}^3}
	&\leq
	C_{\mathrm{GL}} \norm{\bff{u}}{\bb{L}^\infty}^{\frac13} \norm{\bff{u}}{\bb{H}^3}^{\frac23},
\end{align} 
where $B_{\mathrm{GL}}$ and $C_{\mathrm{GL}}$ are constants depending on $\mathscr{D}$.
By \eqref{equ:BGL}, \eqref{equ:CGL}, and~\eqref{equ:u Linfty}, we have
\begin{align}\label{equ:J1 3d}
	\abs{J_1}
	&\leq
	\gamma \norm{\nabla\bff{u}}{\bb{L}^6} \norm{\Delta \bff{u}}{\bb{L}^3} \norm{\nabla\Delta \bff{u}}{\bb{L}^2}
	\leq
	\gamma B_{\mathrm{GL}} C_{\mathrm{GL}} \norm{\bff{u}}{\bb{L}^\infty} \norm{\bff{u}}{\bb{H}^3} \norm{\nabla\Delta \bff{u}}{\bb{L}^2}
	\nonumber\\
	&\leq
	\gamma B_{\mathrm{GL}} C_{\mathrm{GL}} \left(\norm{\bff{u}_0}{\bb{L}^\infty} +\sqrt{\beta}\right)^2
	\big(\norm{\bff{u}}{\bb{L}^2} + \norm{\Delta \bff{u}}{\bb{L}^2} + \norm{\nabla\Delta \bff{u}}{\bb{L}^2} \big) \norm{\nabla\Delta \bff{u}}{\bb{L}^2}
	\nonumber\\
	&\leq
	C(T)
	+
	\frac{\alpha}{8} \norm{\nabla\Delta \bff{u}}{\bb{L}^2}^2
	+
	\gamma B_{\mathrm{GL}} C_{\mathrm{GL}} \left(\norm{\bff{u}_0}{\bb{L}^\infty} +\sqrt{\beta}\right)^2 \norm{\nabla\Delta \bff{u}}{\bb{L}^2}^2,
\end{align}
where in the last step we also used~\eqref{equ:ut Lp}.
As such, if $d=1$ or $2$, we obtain
\begin{align*}
	\ddt \norm{\Delta \bff{u}}{\bb{L}^2}^2
	+
	\norm{\nabla\Delta \bff{u}}{\bb{L}^2}^2
	\leq
	C(1+t) \left(1+\norm{\Delta \bff{u}}{\bb{L}^2}^4 \right).
\end{align*}
Therefore, \eqref{equ:Delta u L2} follows from the Gronwall lemma in these cases. On the other hand, for~$d=3$, if
\begin{equation}\label{equ:small alpha}
		\gamma B_{\mathrm{GL}} C_{\mathrm{GL}} \left(\norm{\bff{u}_0}{\bb{L}^\infty} +\sqrt{\beta}\right)^2
		\leq 7\alpha/8,
\end{equation}
then we can absorb the term containing $\norm{\nabla\Delta \bff{u}}{\bb{L}^2}^2$ from~\eqref{equ:J1 3d}, thus~\eqref{equ:Delta u L2} also follows in this case.
\end{proof}


\begin{remark}
Proposition~\ref{pro:Delta u L2} implies the existence of a global solution~$\bff{u}\in C([0,T];\bb{H}^2)\cap L^2(0,T;\bb{H}^4)$ for an initial data $\bff{u}_0\in \bb{H}^2$ for $d=1$ or $2$. For $d=3$, this holds under an additional assumption that $\alpha$ is sufficiently large, or $\norm{\bff{u}_0}{\bb{L}^\infty}+\sqrt{\beta}$ is sufficiently small (see~\eqref{equ:small alpha}). If the coefficients of~\eqref{equ:llb a} satisfy the assumptions described in Remark~\ref{rem:coeff above Tc}, then $\beta$ can be taken to be zero.
Note that a \emph{local} solution with the aforementioned regularity exists by similar argument as in~\cite{LeSoeTra24}.
\end{remark}


\begin{proposition}
Let $\bff{u}(t)$ be a solution of~\eqref{equ:llb a} such that Proposition~\ref{pro:Delta u L2} holds. For all $t\in [0,T]$,
\begin{align}\label{equ:nab Delta u L2}
	t\norm{\nabla \Delta \bff{u}(t)}{\bb{L}^2}^2
	+
	\int_0^t s\norm{\Delta^2 \bff{u}(s)}{\bb{L}^2}^2 \ds
	\leq
	Ce^{Ct},
\end{align}
where $C$ is a constant depending on $\norm{\bff{u}_0}{\bb{H}^2}$ and the coefficients of~\eqref{equ:llb a}.
\end{proposition}

\begin{proof}
Applying the operator $-\Delta$ on~\eqref{equ:llb eq1}, then taking the inner product of the result with $t\Delta^2 \bff{u}$ and integrating by parts as necessary, we obtain
\begin{align}\label{equ:ddt t nab Delta u}
	&\frac12 \ddt \left(t \norm{\nabla \Delta \bff{u}}{\bb{L}^2}^2\right)
	+
	\alpha t\norm{\Delta^2 \bff{u}}{\bb{L}^2}^2
	\nonumber\\
	&=
	\frac12 \norm{\nabla\Delta \bff{u}}{\bb{L}^2}^2
	+
	\alpha\gamma t\inpro{\nabla\bff{u}\times \nabla\Delta \bff{u}}{\Delta^2 \bff{u}}
	+
	\mu t \norm{\nabla \Delta \bff{u}}{\bb{L}^2}^2
	-
	\lambda\gamma t\inpro{\Delta \big(\bff{u}\times \bff{e}(\bff{e}\cdot \bff{u})\big)}{\Delta^2 \bff{u}}
	+
	\lambda t \norm{\bff{e}\cdot \nabla\Delta \bff{u}}{\bb{L}^2}^2
	\nonumber\\
	&\quad
	+
	t \inpro{\Delta(|\bff{u}|^2 \bff{u})}{\Delta^2 \bff{u}}
	-
	\beta_1 t \inpro{\Delta\big((\bff{\nu} \cdot \nabla)\bff{u}\big)}{\Delta^2 \bff{u}}
	+
	\beta_2 t \inpro{\Delta\big(\bff{u}\times (\bff{\nu}\cdot\nabla)\bff{u}\big)}{\Delta^2 \bff{u}}
	\nonumber\\
	&=: J_1+J_2+\cdots+J_8.
\end{align}
We will estimate each term on the last line, noting the estimates~\eqref{equ:nab u L2} and~\eqref{equ:Delta u L2} obtained previously. The first term is left as is. For the second term, by the Gagliardo--Nirenberg inequalities, we have
\begin{align*}
	\abs{J_2}
	\leq
	\alpha\gamma t \norm{\nabla\bff{u}}{\bb{L}^6} \norm{\nabla\Delta \bff{u}}{\bb{L}^3} \norm{\Delta^2 \bff{u}}{\bb{L}^2}
	&\leq
	Ct \norm{\Delta \bff{u}}{\bb{L}^2} \norm{\nabla\Delta \bff{u}}{\bb{L}^2}^{\frac12}
	\norm{\Delta^2 \bff{u}}{\bb{L}^2}^{\frac32}
	\\
	&\leq
	Ct e^{Ct} \norm{\nabla \Delta \bff{u}}{\bb{L}^2}^2
	+
	\frac{\alpha t}{9} \norm{\Delta^2 \bff{u}}{\bb{L}^2}^2.
\end{align*}
The term $J_3$ is left as is. For the term $J_4$, using the fact that $\bb{H}^2$ is an algebra for $d\leq 3$, we obtain
\begin{align*}
	\abs{J_4}
	&\leq
	Ct \norm{\bff{u}}{\bb{H}^2}^2 \norm{\Delta^2 \bff{u}}{\bb{L}^2}
	\leq
	Cte^{Ct} + \frac{\alpha t}{9} \norm{\Delta^2 \bff{u}}{\bb{L}^2}^2.
\end{align*}
Similarly, for the term $J_6$, we have
\begin{align*}
	\abs{J_6}
	&\leq
	Ct \norm{\bff{u}}{\bb{H}^2}^3 \norm{\Delta^2 \bff{u}}{\bb{L}^2}
	\leq
	Cte^{Ct} + \frac{\alpha t}{9} \norm{\Delta^2 \bff{u}}{\bb{L}^2}^2.
\end{align*}
The term $J_5$ can be estimated as $\abs{J_5} \leq \lambda t\norm{\nabla\Delta \bff{u}}{\bb{L}^2}^2$. For the terms $J_7$ and $J_8$, by H\"older's and Young's inequalities we have
\begin{align*}
	\abs{J_7} 
	&\leq 
	Ct \norm{\nabla\bff{u}}{\bb{H}^2} \norm{\Delta^2 \bff{u}}{\bb{L}^2}
	\leq
	Cte^{Ct} \left(1+\norm{\nabla\Delta \bff{u}}{\bb{L}^2}^2 \right)
	+
	\frac{\alpha t}{9} \norm{\Delta^2 \bff{u}}{\bb{L}^2}^2,
	\\
	\abs{J_8}
	&\leq
	Ct \norm{\bff{u}}{\bb{H}^2} \norm{\nabla \bff{u}}{\bb{H}^2} \norm{\Delta^2 \bff{u}}{\bb{L}^2}
	\leq
	Cte^{Ct} \left(1+\norm{\nabla\Delta \bff{u}}{\bb{L}^2}^2 \right)
	+
	\frac{\alpha t}{9} \norm{\Delta^2 \bff{u}}{\bb{L}^2}^2.
\end{align*}
Altogether, from~\eqref{equ:ddt t nab Delta u} we obtain
\begin{align*}
	&\ddt \left(t \norm{\nabla \Delta \bff{u}}{\bb{L}^2}^2\right)
	+
	\alpha t\norm{\Delta^2 \bff{u}}{\bb{L}^2}^2
	\leq
	\norm{\nabla\Delta \bff{u}}{\bb{L}^2}^2
	+
	Cte^{Ct} \left(1+\norm{\nabla\Delta \bff{u}}{\bb{L}^2}^2 \right).
\end{align*}
Integrating both sides over $(0,t)$ and applying~\eqref{equ:Delta u L2}, we obtain the required estimate.
\end{proof}


\begin{proposition}
	Let $\bff{u}(t)$ be a solution of~\eqref{equ:llb a} such that Proposition~\ref{pro:Delta u L2} holds. For all $t\in [0,T]$,
	\begin{align}\label{equ:Delta22 u L2}
		t^2 \norm{\Delta^2 \bff{u}(t)}{\bb{L}^2}^2
		+
		\int_0^t s^2 \norm{\nabla \Delta^2 \bff{u}(s)}{\bb{L}^2}^2 \ds
		\leq
		Ce^{Ct},
	\end{align}
	where $C$ is a constant depending on $\norm{\bff{u}_0}{\bb{H}^2}$ and the coefficients of~\eqref{equ:llb a}.
\end{proposition}

\begin{proof}
Applying the operator $\Delta^2$ on~\eqref{equ:llb eq1}, then taking the inner product of the result with $t^2 \Delta^2 \bff{u}$ and integrating by parts as necessary, we obtain	
\begin{align}\label{equ:ddt t2 Delta2u}
	&\frac12 \ddt\left(t^2\norm{\Delta^2 \bff{u}}{\bb{L}^2}^2\right)
	+
	\alpha t^2 \norm{\nabla\Delta^2 \bff{u}}{\bb{L}^2}^2
	\nonumber\\
	&=
	t\norm{\Delta^2 \bff{u}}{\bb{L}^2}^2
	+
	\mu t^2 \norm{\Delta^2 \bff{u}}{\bb{L}^2}^2
	+
	\alpha\gamma t^2 \inpro{\nabla\Delta\big(\bff{u}\times \Delta\bff{u}\big)}{\nabla\Delta^2 \bff{u}}
	\nonumber\\
	&\quad
	+
	\lambda\gamma t^2 \inpro{\nabla\Delta \big(\bff{u}\times \bff{e}(\bff{e}\cdot \bff{u})\big)}{\nabla\Delta^2 \bff{u}}
	+
	\inpro{\nabla\Delta\big(|\bff{u}|^2 \bff{u}\big)}{\nabla\Delta^2 \bff{u}}
	+
	\lambda t^2 \norm{\bff{e}\cdot \Delta^2 \bff{u}}{\bb{L}^2}^2
	\nonumber\\
	&\quad
	-
	\beta_1 t^2 \inpro{\Delta^2 \big((\bff{\nu}\cdot\nabla)\bff{u} \big)}{\Delta^2 \bff{u}}
	+
	\beta_2 t^2 \inpro{\Delta^2 \big(\bff{u}\times (\bff{\nu}\cdot\nabla)\bff{u} \big)}{\Delta^2 \bff{u}}
	\nonumber\\
	&=
	I_1+I_2+\cdots+I_8.
\end{align}
We will estimate each term on the last line. The first two terms are kept as is. For the terms $I_3, I_4$, and $I_5$, arguing in a manner similar to that in~\cite[Lemma 2.9]{LeSoeTra24}, we have
\begin{align*}
	\abs{I_3}
	&\leq
	Ct^2 \norm{\bff{u}}{\bb{H}^3}^2 \norm{\bff{u}}{\bb{H}^4}^2
	+
	\frac{\alpha t^2}{9} \norm{\nabla\Delta^2 \bff{u}}{\bb{L}^2}^2,
	\\
	\abs{I_4}
	&\leq
	Ct^2 \norm{\bff{u}}{\bb{H}^3}^4
	+
	\frac{\alpha t^2}{9} \norm{\nabla\Delta^2 \bff{u}}{\bb{L}^2}^2,
	\\
	\abs{I_5}
	&\leq
	Ct^2 \norm{\bff{u}}{\bb{H}^2}^2 \norm{\bff{u}}{\bb{H}^3}^2
	+
	\frac{\alpha t^2}{9} \norm{\nabla\Delta^2 \bff{u}}{\bb{L}^2}^2.
\end{align*}
For the next term, it is easy to see that $\abs{I_6}\leq \lambda t^2 \norm{\Delta^2 \bff{u}}{\bb{L}^2}^2$. Similarly, by the Leibniz formula for derivatives and Young's inequality, we have
\begin{align*}
	\abs{I_7}
	&\leq
	Ct^2 \norm{\bff{u}}{\bb{H}^4}^2
	+
	\frac{\alpha t^2}{9} \norm{\nabla\Delta^2 \bff{u}}{\bb{L}^2}^2,
	\\
	\abs{I_8}
	&\leq
	Ct^2 \norm{\bff{u}}{\bb{H}^3}^2 \norm{\bff{u}}{\bb{H}^4}^2
	+
	\frac{\alpha t^2}{9} \norm{\nabla\Delta^2 \bff{u}}{\bb{L}^2}^2.
\end{align*}
Altogether, substituting these into~\eqref{equ:ddt t2 Delta2u}, we have
\begin{align*}
	\ddt\left(t^2\norm{\Delta^2 \bff{u}}{\bb{L}^2}^2\right)
	+
	\alpha t^2 \norm{\nabla\Delta^2 \bff{u}}{\bb{L}^2}^2
	\leq
	Ct^2 \left(1+\norm{\bff{u}}{\bb{H}^3}^2\right) \norm{\bff{u}}{\bb{H}^4}^2.
\end{align*}
Integrating over $(0,t)$ and using~\eqref{equ:nab Delta u L2}, we obtain the required estimate.
\end{proof}


\begin{theorem}\label{the:main existence}
Let the initial data $\bff{u}_0\in \bb{H}^2$ be given, with an additional condition that $\norm{\bff{u}_0}{\bb{L}^\infty}$ is sufficiently small (as given by Proposition~\ref{pro:Delta u L2}) if $d=3$. Then there exists a unique global strong solution $\bff{u}$ to~\eqref{equ:llb a} in the sense of Definition~\ref{def:strong sol}.

Furthermore, this solution satisfies
\begin{align}\label{equ:smooth H4}
	\norm{\bff{u}(t)}{\bb{H}^4}^2 \leq Ce^{Ct}(1+t^{-2}),
\end{align}
where $C$ is a constant depending on $\norm{\bff{u}_0}{\bb{H}^2}$ and the coefficients of~\eqref{equ:llb a}.
\end{theorem}

\begin{proof}
The existence of a global strong solution follows from a standard compactness argument and the Aubin--Lions lemma, making use of the uniform a priori estimates in~\eqref{equ:u Linfty}, \eqref{equ:nab u L2}, and~\eqref{equ:Delta u L2}. Inequality~\eqref{equ:smooth H4} follows from \eqref{equ:u Linfty}, \eqref{equ:nab u L2}, \eqref{equ:Delta u L2}, \eqref{equ:nab Delta u L2}, and~\eqref{equ:Delta22 u L2}.

Finally, we show the uniqueness of strong solution by deriving a stability estimate. Let $\bff{u}$ and $\bff{v}$ be strong solutions of \eqref{equ:llb a} corresponding to initial data $\bff{u}_0$ and $\bff{v}_0$, respectively. Let $\bff{w}:=\bff{u}-\bff{v}$ and $\bff{w}_0:= \bff{u}_0-\bff{v}_0$. Then for almost every $(t,x)\in (0,T)\times \mathscr{D}$, we have
\begin{align*}
	\partial_t \bff{w}
	&=
%	\alpha \Delta \bff{w}
%	+
%	\alpha \mu \bff{w}
%	-
%	\alpha \kappa \left(|\bff{u}|^2 \bff{u}- |\bff{v}|^2 \bff{v}\right) 
%	+
%	\alpha \lambda \bff{e}(\bff{e}\cdot \bff{w})
%	-
%	\gamma \left(\bff{u}\times \Delta \bff{u}- \bff{v}\times \Delta \bff{v}\right) 
%	\\
%	&\quad
%	-
%	\gamma \left(\bff{u} \times \bff{e}(\bff{e}\cdot \bff{u})- \bff{v}\times \bff{e}(\bff{e}\cdot \bff{v})\right) 
%	+
%	\beta_1 (\bff{\nu}\cdot \nabla) \bff{w}
%	+
%	\beta_2 \left(\bff{u}\times (\bff{\nu}\cdot \nabla) \bff{u} - \bff{v}\times (\bff{\nu}\cdot \nabla)\bff{v}\right)
%	\\
%	&=
	\alpha \Delta \bff{w}
	+
	\alpha \mu \bff{w}
	-
	\alpha\kappa \left(|\bff{u}|^2\bff{w} + ((\bff{u}+\bff{v})\cdot \bff{w})\bff{v}\right)
	+
	\alpha \lambda \bff{e}(\bff{e}\cdot \bff{w})
	-
	\gamma \left(\bff{u}\times \Delta \bff{w} + \bff{w}\times \Delta \bff{v}\right)
	\\
	&\quad 
	-
	\gamma \left(\bff{u}\times \bff{e}(\bff{e}\cdot \bff{w}) + \bff{w}\times \bff{e}(\bff{e}\cdot \bff{v})\right)
	+
	\beta_1 (\bff{\nu}\cdot \nabla) \bff{w}
	+
	\beta_2 \left(\bff{u}\times (\bff{\nu}\cdot \nabla) \bff{w} + \bff{w} \times (\bff{\nu}\cdot\nabla)\bff{v} \right).
\end{align*}
Taking the inner product of this equation with $\bff{w}$, we obtain
\begin{align}\label{equ:ineq J1 J7}
	&\frac12 \ddt \norm{\bff{w}}{\bb{L}^2}^2
	+
	\alpha \norm{\nabla \bff{w}}{\bb{L}^2}^2
	+
	\alpha \kappa \norm{|\bff{u}| |\bff{w}|}{\bb{L}^2}^2
	+
	\alpha\kappa \norm{\bff{v}\cdot \bff{w}}{\bb{L}^2}^2
	\nonumber\\
	&=
	\alpha\mu \norm{\bff{w}}{\bb{L}^2}^2
	+
	\alpha\kappa \inpro{(\bff{u}\cdot \bff{w})\bff{v}}{\bff{w}}
	+
	\alpha \lambda \norm{\bff{e}\cdot \bff{w}}{\bb{L}^2}^2
	-
	\gamma \inpro{\nabla\bff{u}\times \bff{w}}{\nabla \bff{w}}
	-
	\gamma \inpro{\bff{u}\times \bff{e}(\bff{e}\cdot \bff{w})}{\bff{w}}
	\nonumber\\
	&\quad
	+
	\beta_1 \inpro{(\bff{\nu}\cdot \nabla) \bff{w}}{\bff{w}}
	+
	\beta_2 \inpro{\bff{u}\times (\bff{\nu}\cdot \nabla) \bff{w}}{\bff{w}}
	\nonumber\\
	&=:
	J_1+J_2+\cdots+J_7.
\end{align}
Straightforward application of Young's inequality and Sobolev embedding yields
\begin{align*}
	\abs{J_2} 
	&\leq
	\frac{\alpha\kappa}{2} \norm{|\bff{u}||\bff{w}|}{\bb{L}^2}^2
	+
	\frac{\alpha\kappa}{2} \norm{\bff{v}\cdot \bff{w}}{\bb{L}^2}^2,
	\\
	\abs{J_3}
	&\leq
	\alpha\lambda \norm{\bff{w}}{\bb{L}^2}^2,
	\\
	\abs{J_4}
	&\leq
	\gamma \norm{\nabla \bff{u}}{\bb{L}^\infty} \norm{\bff{w}}{\bb{L}^2} \norm{\nabla \bff{w}}{\bb{L}^2}
	\leq
	C\norm{\bff{u}}{\bb{H}^3}^2 \norm{\bff{w}}{\bb{L}^2}^2 
	+
	\frac{\alpha}{4} \norm{\nabla \bff{w}}{\bb{L}^2}^2,
	\\
	\abs{J_5}
	&\leq
	C\norm{\bff{u}}{\bb{L}^\infty}^2 \norm{\bff{w}}{\bb{L}^2}^2
	+
	C\norm{\bff{u}}{\bb{L}^2}^2,
	\\
	\abs{J_6}
	&\leq
	C\norm{\bff{w}}{\bb{L}^2}^2
	+
	\frac{\alpha}{4} \norm{\nabla \bff{w}}{\bb{L}^2}^2,
	\\
	\abs{J_7}
	&\leq
	C\norm{\bff{u}}{\bb{L}^\infty}^2 \norm{\bff{w}}{\bb{L}^2}^2
	+
	\frac{\alpha}{4} \norm{\nabla \bff{w}}{\bb{L}^2}^2.
\end{align*}
We substitute the above estimates into~\eqref{equ:ineq J1 J7} to obtain
\begin{align*}
	\ddt \norm{\bff{w}}{\bb{L}^2}^2
	+
	\norm{\nabla \bff{w}}{\bb{L}^2}^2
	&\leq
	C\left(1+ \norm{\bff{u}}{\bb{L}^\infty}^2 + \norm{\bff{u}}{\bb{H}^3}^2\right) \norm{\bff{w}}{\bb{L}^2}^2.
\end{align*}
Note that $\bff{u}\in L^2(\bb{H}^3)$ by~\eqref{equ:u Linfty}, \eqref{equ:nab u L2}, \eqref{equ:Delta u L2}, and the elliptic regularity. Applying the Gronwall lemma, we then have
\begin{align*}
	\norm{\bff{w}(t)}{\bb{L}^2} \leq Ce^{Ct} \norm{\bff{w}_0}{\bb{L}^2}.
\end{align*}
Uniqueness of the solution then follows immediately from this stability estimate.
\end{proof}



\section{A linear fully discrete scheme for the LLB equation with spin-torques}\label{sec:llb spin}

In this section, we propose a linear fully discrete finite element scheme for the LLB equation with spin-torques. For the analysis, we need to first define several multilinear forms. 

\begin{definition}[multilinear forms]\label{def:forms}
	Given~$\delta>0$, $\bff{\nu}\in L^\infty(\bb{L}^\infty)$, and~$\bff{\phi},\bff{\xi}\in \bb{H}^1\cap \bb{L}^\infty$, we define the maps
	\begin{align*}
		&\mathcal{A}_\delta (\cdot\,,\,\cdot): \bb{H}^1\times \bb{H}^1 \to \bb{R},
		\\ 
		&\mathcal{B}(\bff{\phi},\bff{\xi};\,\cdot\, , \,\cdot): \bb{H}^1\times \bb{H}^1 \to \bb{R},
		\\
		&\mathcal{C}(\bff{\phi};\,\cdot\, ,\, \cdot): \bb{H}^1\times \bb{H}^1 \to \bb{R},
		\\
		&\mathcal{D}(\cdot\, , \,\cdot): \bb{H}^1 \times \bb{H}^1 \to \bb{R},
	\end{align*}
	by
	\begin{align*}
		\mathcal{A}_\delta (\bff{v},\bff{w}) &:= 
		\alpha \inpro{\nabla \bff{v}}{\nabla \bff{w}} + \delta \inpro{\bff{v}}{\bff{w}},
		\\
		\mathcal{B}(\bff{\phi},\bff{\xi};\bff{v},\bff{w}) &:= 
		\alpha \inpro{(\bff{\phi}\cdot\bff{\xi})\bff{v}}{\bff{w}},
		\\
		\mathcal{C}(\bff{\phi};\bff{v},\bff{w})
		&:= -\gamma\inpro{\bff{\phi}\times \nabla \bff{v}}{\nabla \bff{w}}
		+\lambda\gamma \inpro{\bff{v}\times \bff{e}(\bff{e}\cdot \bff{\phi})}{\bff{w}}
		-\beta_2 \inpro{\bff{v}\times (\bff{\nu}\cdot \nabla)\bff{\phi}}{\bff{w}},
		\\
		\mathcal{D}(\bff{v},\bff{w})
		&:= 
		(\delta+\alpha \mu) \inpro{\bff{v}}{\bff{w}}
		+
		\alpha\lambda \inpro{\bff{e}(\bff{e}\cdot \bff{v})}{\bff{w}}
		+
		\beta_1 \inpro{(\bff{\nu}\cdot \nabla)\bff{v}}{\bff{w}}.
	\end{align*}
	If either $\bff{\nu}\cdot \bff{n}=0$ on $\partial \mathscr{D}$ or $\bff{\nu}=\bff{0}$ on $\partial \mathscr{D}$, then by~\eqref{equ:div ab} and the divergence theorem, we can write:
	\begin{align}
		\label{equ:C div}
		\mathcal{C}(\bff{\phi};\bff{v},\bff{w})
		&= 
		-\gamma\inpro{\bff{\phi}\times \nabla \bff{v}}{\nabla \bff{w}}
		+\lambda\gamma \inpro{\bff{v}\times \bff{e}(\bff{e}\cdot \bff{\phi})}{\bff{w}}
		-
		\beta_2 \inpro{\bff{\phi}\otimes \bff{\nu}}{\nabla(\bff{v}\times \bff{w})}
		\nonumber\\
		&\quad
		+
		\beta_2 \inpro{\bff{v}\times (\nabla \cdot \bff{\nu})\bff{\phi}}{\bff{w}},
		\\
		\label{equ:D div}
		\mathcal{D}(\bff{v},\bff{w})
		&= 
		(\delta+\alpha \mu) \inpro{\bff{v}}{\bff{w}}
		+
		\alpha\lambda \inpro{\bff{e}(\bff{e}\cdot \bff{v})}{\bff{w}}
		-
		\beta_1 \inpro{\bff{v}\otimes \bff{\nu}}{\nabla \bff{w}}
		-
		\beta_1 \inpro{(\nabla \cdot \bff{\nu}) \bff{v}}{\bff{w}}.
	\end{align}
	Furthermore, let
	\begin{align}\label{equ:bilinear A}
		\mathcal{A}(\bff{\phi}; \bff{v},\bff{w})
		:=
		\mathcal{A}_\delta(\bff{v},\bff{w})
		+
		\mathcal{B}(\bff{\phi},\bff{\phi};\bff{v},\bff{w})
		+
		\mathcal{C}(\bff{\phi}; \bff{v},\bff{w}).
	\end{align}
\end{definition}

Some properties of the multilinear forms defined above are gathered in the following lemma.

\begin{lemma}
	Let $\bff{\phi}, \bff{\xi} \in \bb{H}^1\cap \bb{L}^\infty$, and let the maps $\mathcal{A}_\delta,\mathcal{B},\mathcal{C},\mathcal{D}$, and $\mathcal{A}$ be as defined in Definition~\ref{def:forms}. The following statements hold true:
	\begin{enumerate}[(i)]
		\item \label{item:bdd} There exists a constant $C>0$ such that for all $\bff{v},\bff{w}\in \bb{H}^1$,
		\begin{align}
			\label{equ:A delta bdd}
			\left|\mathcal{A}_\delta (\bff{v},\bff{w}) \right|
			&\leq
			(\alpha+\delta) \norm{\bff{v}}{\bb{H}^1} \norm{\bff{w}}{\bb{H}^1},
			\\
			\label{equ:B bdd}
			\left|\mathcal{B} (\bff{\phi},\bff{\xi}; \bff{v},\bff{w}) \right|
			&\leq
			\norm{\bff{\phi}}{\bb{L}^\infty} \norm{\bff{\xi}}{\bb{L}^\infty} \norm{\bff{v}}{\bb{L}^2} \norm{\bff{w}}{\bb{L}^2},
			\\
			\label{equ:C bdd}
			\left|\mathcal{C} (\bff{\phi}; \bff{v},\bff{w}) \right|
			&\leq
			C \norm{\bff{\phi}}{\bb{H}^1} \norm{\bff{v}}{\bb{H}^1} \norm{\bff{w}}{\bb{H}^1}.
		\end{align}
		\item \label{item:2} For all $\bff{v}\in \bb{H}^1$, $\mathcal{C}(\bff{\phi}; \bff{v},\bff{v})=0$.
		\item $\mathcal{A}(\bff{\phi};\, \cdot\, , \, \cdot)$ is bounded, i.e. there exists a constant $\beta>0$ depending only on the coefficients of the equation, $\norm{\bff{\phi}}{\bb{L}^\infty}$, and~$\norm{\bff{\phi}}{\bb{H}^1}$, such that
		\begin{align}\label{equ:A bounded}
			\abs{\mathcal{A}(\bff{\phi};\bff{v},\bff{w})}
			\leq
			\beta \norm{\bff{v}}{\bb{H}^1} \norm{\bff{w}}{\bb{H}^1},
			\quad
			\forall \bff{v},\bff{w}\in \bb{H}^1.
		\end{align}
		\item $\mathcal{A}(\bff{\phi};\, \cdot\, , \, \cdot)$ is coercive, i.e. there exists a constant $\mu>0$ \emph{independent} of $\bff{\phi}$ such that
		\begin{align}\label{equ:A coercive}
			\mathcal{A}(\bff{\phi};\bff{v},\bff{v}) \geq \mu \norm{\bff{v}}{\bb{H}^1}^2,
			\quad \forall \bff{v}\in \bb{H}^1.
		\end{align}
		\item If additionally $\bff{\phi}\in \bb{W}^{1,4}$ and $\bff{w}\in \bb{H}^2_{\bff{n}}$, then there exists a constant $C>0$ such that
		\begin{align}\label{equ:C ineq W14}
			\left|\mathcal{C} (\bff{\phi}; \bff{v},\bff{w}) \right|
			&\leq
			C \norm{\bff{\phi}}{\bb{W}^{1,4}} \norm{\bff{v}}{\bb{L}^2} \norm{\bff{w}}{\bb{H}^2},
			\\
			\label{equ:C Delta w w}
			\left|\mathcal{C} (\bff{\phi}; \Delta \bff{w},\bff{w}) \right|
			&\leq
			C \norm{\bff{\phi}}{\bb{W}^{1,4}} \norm{\bff{w}}{\bb{W}^{1,4}} \norm{\Delta \bff{w}}{\bb{L}^2},
		\end{align}
	\end{enumerate}
\end{lemma}

\begin{proof}
	Inequalities~\eqref{equ:A delta bdd}, \eqref{equ:B bdd}, and \eqref{equ:C bdd} follow immediately by H\"older's inequality. Statement~\eqref{item:2} is obvious since $\bff{a}\times\bff{a}=\bff{0}$ and $(\bff{a}\times \bff{b})\cdot \bff{b}=0$ for any $\bff{a},\bff{b}\in \bb{R}^3$. Inequality~\eqref{equ:A bounded} follows from the estimates in part~\eqref{item:bdd} and the triangle inequality.
	Moreover, note that
	\begin{align*}
		\mathcal{A}(\bff{\phi}; \bff{v},\bff{v})
		&=
		\alpha \norm{\nabla \bff{v}}{\bb{L}^2}^2
		+
		\delta \norm{\bff{v}}{\bb{L}^2}^2
		+
		\alpha \norm{|\bff{\phi}| |\bff{v}|}{\bb{L}^2}^2
		\geq
		\min\{\alpha,\delta\} \norm{\bff{v}}{\bb{H}^1}^2,
	\end{align*}
	showing~\eqref{equ:A coercive}.
	
	Next, we show~\eqref{equ:C ineq W14} and~\eqref{equ:C Delta w w}. Firstly, integrating by parts the first term in the definition of $\mathcal{C}$, and noting the assumptions, we have
	\begin{align}\label{equ:int parts C}
		\mathcal{C} (\bff{\phi}; \bff{v},\bff{w})
		&=
		\gamma\inpro{\bff{\phi}\times \bff{v}}{\Delta \bff{w}}
		+
		\gamma\inpro{\nabla \bff{\phi}\times \bff{v}}{\nabla \bff{w}}
		+\lambda\gamma \inpro{\bff{v}\times \bff{e}(\bff{e}\cdot \bff{\phi})}{\bff{w}}
		-\beta_2 \inpro{\bff{v}\times (\bff{\nu}\cdot \nabla)\bff{\phi}}{\bff{w}}.
	\end{align}
	Inequality~\eqref{equ:C ineq W14} then follows from H\"older's inequality and the Sobolev embeddings $\bb{H}^2\hookrightarrow \bb{W}^{1,4} \hookrightarrow \bb{L}^\infty$.
	Similarly, inequality~\eqref{equ:C Delta w w} follows from~\eqref{equ:int parts C} with $\bff{v}=\Delta \bff{w}$ and H\"older's inequality.
\end{proof}



In terms of the forms introduced in Definition~\ref{def:forms}, the weak formulation~\eqref{equ:weak special} of the problem can be written as
\begin{align}\label{equ:weak form AB}
	\inpro{\partial_t \bff{u}}{\bff{\chi}} 
	+
	\mathcal{A}(\bff{u};\bff{u},\bff{\chi})
	=
	\mathcal{D}(\bff{u},\bff{\chi}), \quad
	\forall \bff{\chi} \in \bb{H}^1.
\end{align}

Let $k$ be the time step and $\bb{V}_h$ be the finite element space~\eqref{equ:Vh}. Let $\bff{u}_h^n$ be the approximation in $\bb{V}_h$ of $\bff{u}(t)$ at time $t=t_n=nk\in [0,T]$, where $n=0,1,2,\ldots, \lfloor T/k \rfloor$. We denote $\bff{u}^n:= \bff{u}(t_n)$. For any discrete function $\bff{v}$, define for $n\in \bb{N}$,
\begin{align*}
	\mathrm{d}_t \bff{v}^{n} 
	:=
	\frac{\bff{v}^{n}-\bff{v}^{n-1}}{k}.
\end{align*}
We now describe a linear fully discrete scheme to solve~\eqref{equ:weak form AB}. We start with $\bff{u}_h^0= P_h \bff{u}_0 \in \bb{V}_h$. For $t_n\in [0,T]$, $n\in \bb{N}$, given $\bff{u}_h^{n-1}\in \bb{V}_h$, define $\bff{u}_h^n$ by
\begin{align}\label{equ:scheme spin}
	\inpro{\mathrm{d}_t \bff{u}_h^n}{\bff{\chi}}
	+
	\mathcal{A} (\bff{u}_h^{n-1}; \bff{u}_h^n, \bff{\chi})
	=
	\mathcal{D}(\bff{u}_h^n, \bff{\chi}),
	\quad 
	\forall \bff{\chi}\in\bb{V}_h.
\end{align}

To facilitate the proof of the error analysis, we split the approximation error as:
\begin{align}\label{equ:split theta eta}
	\bff{u}_h^n- \bff{u}^n
	= 
	\left(\bff{u}_h^n- \Pi_h \bff{u}^n\right) 
	+ 
	\left(\Pi_h \bff{u}^n- \bff{u}^n\right) =: \bff{\theta}^n+ \bff{\eta}^n,
\end{align}
where for any $t_n\in [0,T]$, $\Pi_h \bff{u}^n=: \Pi_h \bff{u}(t_n)$ is the elliptic projection of the solution $\bff{u}(t_n)$ defined by
\begin{align}\label{equ:elliptic proj}
	\mathcal{A}(\bff{u}(t_n); \bff{\eta}^n, \bff{\chi})=0, \quad \forall \bff{\chi}\in\bb{V}_h.
\end{align}
Note that the scheme~\eqref{equ:scheme spin} and the elliptic projection are well-defined by the Lax--Milgram lemma. 
We study some properties of this projection in the following section, while the error analysis is performed in Section~\ref{subsec:error analysis}. For a technical reason (see Lemma~\ref{lem:stab elliptic}), we assume that in~\eqref{equ:Vh} the polynomial degree $r$ in $\bb{V}_h$ is:
\begin{equation}\label{equ:deg r}
	\begin{cases}
		r\geq 1, &\text{if } d\in \{1,2\},
		\\
		r\geq 2, &\text{if } d=3.
	\end{cases}
\end{equation}



\subsection{Properties of elliptic projection}\label{subsec:elliptic}

Relevant approximation and stability properties for the elliptic projection defined by~\eqref{equ:elliptic proj} are derived in this section.

\begin{proposition}\label{pro:eta t}
Let $\Pi_h \bff{u}$ be the elliptic projection of $\bff{u}$ defined by~\eqref{equ:elliptic proj}, $\bff{u}$ be the solution of~\eqref{equ:llb a} which satisfies~\eqref{equ:ass 1}, and $\bff{\eta}(t)=\Pi_h \bff{u}(t)- \bff{u}(t)$ be as defined in~\eqref{equ:split theta eta}. Then for any $t\in [0,T]$,
\begin{align}
	\label{equ:eta t}
	\norm{\bff{\eta}(t)}{\bb{L}^2} + h \norm{\nabla \bff{\eta}(t)}{\bb{L}^2}
	&\leq
	Ch^{r+1} \norm{\bff{u}}{L^\infty(\bb{H}^{r+1})},
	\\
	\label{equ:dt eta t}
	\norm{\partial_t \bff{\eta}(t)}{\bb{L}^2} + h \norm{\nabla \partial_t \bff{\eta}(t)}{\bb{L}^2}
	&\leq
	Ch^{r+1} \norm{\partial_t \bff{u}}{L^\infty(\bb{H}^{r+1})},
\end{align}
where $r$ is the degree of polynomials in the finite element space $\bb{V}_h$.
The constant $C$ depends on $\beta,\mu$, and $T$, where $\beta$ and $\mu$ were defined in~\eqref{equ:A bounded} and \eqref{equ:A coercive}, respectively.
\end{proposition}

\begin{proof}
For all $\bff{\chi}\in \bb{V}_h$, by the coercivity and the boundedness of $\mathcal{A}$, and the definition of $\bff{\eta}$,
\begin{align*}
	\mu \norm{\bff{\eta}(t)}{\bb{H}^1}^2
	&\leq
	\mathcal{A}(\bff{u}(t); \bff{\eta}(t), \bff{\eta}(t))
	\\
	&=
	\mathcal{A}(\bff{u}(t); \bff{\eta}(t), \Pi_h\bff{u}(t)- \bff{\chi})
	-
	\mathcal{A}(\bff{u}(t); \bff{\eta}(t), \bff{u}(t)-\bff{\chi})
	\\
	&\leq
	\beta \norm{\bff{\eta}(t)}{\bb{H}^1} \norm{\bff{u}(t)-\bff{\chi}}{\bb{H}^1},
\end{align*}
since the first term in the second step is zero. Therefore, by~\eqref{equ:fin approx},
\begin{align}\label{equ:eta H1}
	\norm{\bff{\eta}(t)}{\bb{H}^1}
	\leq
	(\beta/\mu) \inf_{\bff{\chi}\in \bb{V}_h} \norm{\bff{u}(t)-\bff{\chi}}{\bb{H}^1}
	\leq
	Ch^r \norm{\bff{u}}{L^\infty(\bb{H}^{r+1})}.
\end{align}
To show the $\bb{L}^2$-estimate, we use duality argument. For each $\bff{u}(t)\in \bb{H}^2$, let $\bff{\psi}(t)\in \bb{H}^2_{\bff{n}}$ satisfy
\begin{align}\label{equ:A u zeta}
	\mathcal{A}(\bff{u}(t); \bff{\zeta}, \bff{\psi}(t)) = \inpro{\bff{\eta}(t)}{\bff{\zeta}},
	\quad \forall \bff{\zeta}\in \bb{H}^1.
\end{align}
For any $t\in [0,T]$, such $\bff{\psi}(t)$ exists by Lemma~\ref{lem:dual exist} under the assumed conditions on $\mathscr{D}$. Moreover,
\begin{align*}
	\norm{\bff{\psi}(t)}{\bb{H}^2} \leq
	C\norm{\bff{\eta}(t)}{\bb{L}^2}.
\end{align*}
Therefore, taking $\bff{\zeta}= \bff{\eta}(t)$ in~\eqref{equ:A u zeta} and noting~\eqref{equ:elliptic proj}, we have for all $\bff{\chi}\in \bb{V}_h$,
\begin{align*}
	\norm{\bff{\eta}(t)}{\bb{L}^2}^2
	&=
	\mathcal{A}(\bff{u}(t); \bff{\eta}(t), \bff{\psi}(t))
	=
	\mathcal{A}(\bff{u}(t); \bff{\eta}(t), \bff{\psi}(t)-\bff{\chi})
	\\
	&\leq
	\beta \norm{\bff{\eta}(t)}{\bb{H}^1} \inf_{\bff{\chi}\in \bb{V}_h} \norm{\bff{\psi}(t)-\bff{\chi}}{\bb{H}^1}
	\leq
	Ch^{r+1} \norm{\bff{u}(t)}{\bb{H}^{r+1}} \norm{\bff{\eta}(t)}{\bb{L}^2},
\end{align*}
where in the last step we used~\eqref{equ:eta H1} and~\eqref{equ:fin approx}. This and~\eqref{equ:eta H1} then implies inequality~\eqref{equ:eta t}.

Next, we show~\eqref{equ:dt eta t}. For ease of presentation, we omit the dependence of the functions on $t$. By the coercivity of $\mathcal{A}$ and the definition of $\bff{\eta}$,
\begin{align}\label{equ:leq A dt eta}
	\mu \norm{\partial_t \bff{\eta}(t)}{\bb{H}^1}^2
	\leq
	\mathcal{A}(\bff{u}; \partial_t \bff{\eta}, \partial_t \bff{\eta})
	=
	\mathcal{A}\left(\bff{u}; \partial_t \bff{\eta}, \mathcal{I}_h(\partial_t \bff{\eta})\right) 
	+
	\mathcal{A}\left(\bff{u}; \partial_t \bff{\eta}, \mathcal{I}_h(\partial_t \bff{u}) -\partial_t \bff{u}\right).
\end{align}
We will estimate each term on the last line. To this end, noting~\eqref{equ:bilinear A} and differentiating~\eqref{equ:elliptic proj} with respect to $t$, we have for all $\bff{\chi}\in \bb{V}_h$,
\begin{align}\label{equ:A dt eta}
	\mathcal{A} (\bff{u}; \partial_t \bff{\eta},\bff{\chi})
	+
	2 \mathcal{B}(\bff{u},\partial_t \bff{u}; \bff{\eta}, \bff{\chi})
	+
	\mathcal{C}(\partial_t \bff{u}; \bff{\eta}, \bff{\chi})
	= 0.
\end{align}
Thus, for the first term on the right-hand side of~\eqref{equ:leq A dt eta}, by the boundedness of $\mathcal{B}, \mathcal{C}$, and $\mathcal{I}_h$ we have
\begin{align}\label{equ:A dt leq 1}
	\big| \mathcal{A}(\bff{u}; \partial_t \bff{\eta}, \mathcal{I}_h(\partial_t \bff{\eta})) \big| 
	&\leq
	\big| 2 \mathcal{B}(\bff{u},\partial_t \bff{u}; \bff{\eta}, \mathcal{I}_h(\partial_t \bff{\eta})) \big|
	+
	\big| \mathcal{C}(\partial_t \bff{u}; \bff{\eta}, \mathcal{I}_h(\partial_t \bff{\eta})) \big|
	\nonumber\\
	&\leq
	C \norm{\bff{\eta}}{\bb{H}^1} \norm{\mathcal{I}_h (\partial_t \bff{\eta})}{\bb{H}^1}
	\leq
	Ch^r \norm{\partial_t \bff{\eta}}{\bb{H}^1} \norm{\partial_t \bff{u}}{L^\infty(\bb{H}^{r+1})},
\end{align}
where in the last step we also used~\eqref{equ:eta t}. For the second term on the right-hand side of~\eqref{equ:leq A dt eta}, by the boundedness of $\mathcal{A}$ and \eqref{equ:interp approx} we have
\begin{align}\label{equ:A dt leq 2}
	\big| \mathcal{A}(\bff{u}; \partial_t \bff{\eta}, \mathcal{I}_h(\partial_t \bff{u})- \partial_t \bff{u}) \big| 
	\leq
	C \norm{\partial_t \bff{\eta}}{\bb{H}^1} \norm{\mathcal{I}_h(\partial_t \bff{u})- \partial_t \bff{u}}{\bb{H}^1}
	\leq
	Ch^r \norm{\partial_t \bff{\eta}}{\bb{H}^1} \norm{\partial_t \bff{u}}{L^\infty(\bb{H}^{r+1})}.
\end{align}
The estimates~\eqref{equ:A dt leq 1} and \eqref{equ:A dt leq 2}, together with~\eqref{equ:leq A dt eta} imply
\begin{align}\label{equ:dt eta H1}
	\norm{\partial_t \bff{\eta}(t)}{\bb{H}^1}
	\leq
	Ch^r \norm{\partial_t \bff{u}}{L^\infty(\bb{H}^{r+1})}.
\end{align}
To estimate $\norm{\partial_t \bff{\eta}}{\bb{L}^2}$, we use duality argument as before. For each $\bff{u}(t) \in \bb{H}^2$, let $\bff{\psi}(t)\in \bb{H}^2_{\bff{n}}$ satisfy
\begin{align}\label{equ:A pa t zeta}
	\mathcal{A}(\bff{u}(t); \bff{\zeta}, \bff{\psi}(t)) = \inpro{\partial_t \bff{\eta}(t)}{\bff{\zeta}}, \quad \forall \bff{\zeta}\in \bb{H}^1. 
\end{align}
The existence of $\bff{\psi}(t)$ is conferred by Lemma~\ref{lem:dual exist}, and furthermore we have
\begin{align}\label{equ:psi dt eta}
	\norm{\bff{\psi}(t)}{\bb{H}^2} \leq C \norm{\partial_t \bff{\eta}(t)}{\bb{L}^2}.
\end{align}
Taking $\bff{\zeta}= \partial_t \bff{\eta}(t)$ in~\eqref{equ:A pa t zeta}, we have
\begin{align*}
	\mathcal{A}(\bff{u}(t); \partial_t \bff{\eta}(t), \bff{\psi}(t))
	=
	\norm{\partial_t \bff{\eta}(t)}{\bb{L}^2}^2.
\end{align*}
This equation and~\eqref{equ:A dt eta} yield for all $\bff{\chi}\in \bb{V}_h$,
\begin{align*}
	\norm{\partial_t \bff{\eta}}{\bb{L}^2}^2
	&=
	\mathcal{A} (\bff{u}; \partial_t \bff{\eta},\bff{\psi}-\bff{\chi})
	+
	2 \mathcal{B}(\bff{u},\partial_t \bff{u}; \bff{\eta}, \bff{\chi})
	+
	\mathcal{C}(\partial_t \bff{u}; \bff{\eta}, \bff{\chi})
	\\
	&=
	\mathcal{A} (\bff{u}; \partial_t \bff{\eta},\bff{\psi}-\bff{\chi})
	+
	2\mathcal{B} (\bff{u},\partial_t \bff{u}; \bff{\eta}, \bff{\psi}-\bff{\chi})
	-
	2 \mathcal{B}(\bff{u},\partial_t \bff{u}; \bff{\eta}, \bff{\psi})
	\\
	&\quad
	+
	\mathcal{C}(\partial_t \bff{u}; \bff{\eta},\bff{\psi}-\bff{\chi})
	-
	\mathcal{C}(\partial_t \bff{u}; \bff{\eta}, \bff{\psi}).
\end{align*}
We estimate the first four terms on the right-hand side above by using the boundedness of $\mathcal{A}$, $\mathcal{B}$, and $\mathcal{C}$ (cf.~\eqref{equ:A bounded}, \eqref{equ:B bdd}, and~\eqref{equ:C bdd}). The fourth term is bounded using~\eqref{equ:C bdd}, while the last term is bounded using~\eqref{equ:C ineq W14}. Thus, we obtain
\begin{align*}
	\norm{\partial_t \bff{\eta}}{\bb{L}^2}^2
	&\leq
	C \norm{\partial_t \bff{\eta}}{\bb{H}^1} \norm{\bff{\psi}-\bff{\chi}}{\bb{H}^1}
	+
	2 \norm{\bff{u}}{\bb{L}^\infty} \norm{\partial_t \bff{u}}{\bb{L}^\infty} \norm{\bff{\eta}}{\bb{L}^2} \norm{\bff{\psi}-\bff{\chi}}{\bb{L}^2}
	+
	2 \norm{\bff{u}}{\bb{L}^\infty} \norm{\partial_t \bff{u}}{\bb{L}^\infty} \norm{\bff{\eta}}{\bb{L}^2} \norm{\bff{\psi}}{\bb{L}^2}
	\\
	&\quad
	+
	C \norm{\partial_t \bff{u}}{\bb{H}^1} \norm{\bff{\eta}}{\bb{H}^1} \norm{\bff{\psi}-\bff{\chi}}{\bb{H}^1}
	+
	C \norm{\partial_t \bff{u}}{\bb{W}^{1,4}} \norm{\bff{\eta}}{\bb{L}^2} \norm{\bff{\psi}}{\bb{H}^2}.
\end{align*}
We now choose $\bff{\chi}= \mathcal{I}_h \bff{\psi}$. Successively using~\eqref{equ:dt eta H1}, \eqref{equ:interp approx}, and~\eqref{equ:eta t}, noting the assumption~\eqref{equ:ass 1} we have
\begin{align*}
	\norm{\partial_t \bff{\eta}(t)}{\bb{L}^2}^2
	\leq
	(Ch^{r+1}+Ch^{r+3}) \norm{\bff{\psi}(t)}{\bb{H}^2}
	\leq
	Ch^{r+1} \norm{\partial_t \bff{\eta}(t)}{\bb{L}^2},
\end{align*}
where in the last step we used~\eqref{equ:psi dt eta}. This implies
\begin{align*}
	\norm{\partial_t \bff{\eta}(t)}{\bb{L}^2}
	\leq
	Ch^{r+1}.
\end{align*}
This inequality and~\eqref{equ:dt eta H1} together implies~\eqref{equ:dt eta t}. The proof of this proposition will then be complete once we show the following regularity result in Lemma~\ref{lem:dual exist}.
\end{proof}


\begin{lemma}\label{lem:dual exist}
Let $\mathscr{D}$ be a smooth or a convex polyhedral domain. Assume that the exact solution $\bff{u}$ of~\eqref{equ:llb a} belongs to $L^\infty(\bb{H}^2_{\bff{n}})$. For any $\bff{\varphi}\in \bb{L}^2$ and for each $t\in [0,T]$, there exists $\bff{\psi}(t)\in \bb{H}^2_{\bff{n}}$ such that
\begin{align}\label{equ:A ut zeta}
	\mathcal{A}(\bff{u}(t); \bff{\zeta}, \bff{\psi}(t))
	=
	\inpro{\bff{\varphi}}{\bff{\zeta}}, \quad
	\forall \bff{\zeta}\in \bb{H}^1.
\end{align}
Moreover,
\begin{align}\label{equ:psi H2}
	\norm{\bff{\psi}(t)}{\bb{H}^2} \leq C\norm{\bff{\varphi}}{\bb{L}^2},
\end{align}
where the constant $C$ depends on $\mathscr{D}$, $T$ and $\norm{\bff{u}}{L^\infty(\bb{H}^2)}$.
\end{lemma}

\begin{proof}
We use the Faedo--Galerkin method. Let $\{\bff{e}_i\}_{i=1}^\infty$ be an orthonormal basis of $\bb{L}^2$ consisting of eigenfunctions for $-\Delta$ such that
\begin{align*}
	-\Delta \bff{e}_i = \lambda_i \bff{e}_i \text{ in } \mathscr{D} \quad
	\text{and} \quad \frac{\partial \bff{e}_i}{\partial \bff{n}}=\bff{0} \text{ on } \partial \mathscr{D},
\end{align*}
where $\lambda_i \geq 0$ are the eigenvalues of $-\Delta$ associated with $\bff{e}_i$.
Let $\bb{W}_n:= \text{span}\{\bff{e}_1,\ldots,\bff{e}_n\}$. 

Given $\bff{\varphi}\in \bb{L}^2$, for each $t\in [0,T]$, we approximate the solution to~\eqref{equ:A ut zeta} by $\bff{\psi}_n(t)\in \bb{W}_n$ satisfying 
\begin{align}\label{equ:A ut zeta psi n}
	\mathcal{A}(\bff{u}(t); \bff{\zeta}, \bff{\psi}_n(t))
	=
	\inpro{\bff{\varphi}}{\bff{\zeta}}, \quad
	\forall \bff{\zeta}\in \bb{W}_n.
\end{align}
We need the following uniform bound:
\begin{align}\label{equ:psi n phi H2}
	\norm{\bff{\psi}_n(t)}{\bb{H}^2} \leq C \norm{\bff{\varphi}}{\bb{L}^2}.
\end{align}
Once this bound is established, we can apply the standard compactness argument: For each $t\in [0,T]$, the Banach--Alaoglu theorem and a compactness argument imply the existence of a subsequence, which is still denoted by $\{\bff{\psi}_n(t)\}$, such that
\begin{align*}
	&\bff{\psi}_n(t) \rightharpoonup \bff{\psi}(t) \quad \text{weakly in } \bb{H}^2_{\bff{n}},
	\\
	&\bff{\psi}_n(t) \to \bff{\psi}(t) \quad \text{strongly in } \bb{W}^{1,4}.
\end{align*}
A standard argument then shows $\bff{\psi}(t)$ satisfies~\eqref{equ:A ut zeta}, while \eqref{equ:psi H2} follows from~\eqref{equ:psi n phi H2}.

It remains to prove~\eqref{equ:psi n phi H2}. Setting $\bff{\zeta}= \bff{\psi}_n$ in~\eqref{equ:A ut zeta psi n} and using the coercivity of $\mathcal{A}$, we have
\begin{align*}
	\mu \norm{\bff{\psi}_n}{\bb{H}^1}^2
	\leq
	\mathcal{A}(\bff{u}; \bff{\psi}_n, \bff{\psi}_n)
	=
	\inpro{\bff{\varphi}}{\bff{\psi}_n}
	\leq
	\norm{\bff{\varphi}}{\bb{L}^2} \norm{\bff{\psi}_n}{\bb{L}^2},
\end{align*}
which implies
\begin{align}\label{equ:psi n H1}
	\norm{\bff{\psi}_n(t)}{\bb{H}^1} \leq C \norm{\bff{\varphi}}{\bb{L}^2}.
\end{align}
Next, taking $\bff{\zeta}= -\Delta \bff{\psi}_n$, integrating by parts as necessary, and applying~\eqref{equ:B bdd} and~\eqref{equ:C Delta w w} we have
\begin{align*}
	\alpha \norm{\Delta \bff{\psi}_n}{\bb{L}^2}^2
	+
	\delta \norm{\nabla \bff{\psi}_n}{\bb{L}^2}^2
	&=
	\mathcal{B}(\bff{u},\bff{u}; \Delta \bff{\psi}_n, \bff{\psi}_n)
	+
	\mathcal{C}(\bff{u}; \Delta \bff{\psi}_n, \bff{\psi}_n)
	-
	\inpro{\bff{\varphi}}{\Delta \bff{\psi}_n}
	\\
	&\leq
	C\norm{\bff{u}}{\bb{L}^\infty}^2 \norm{\Delta \bff{\psi}_n}{\bb{L}^2} \norm{\bff{\psi}_n}{\bb{L}^2}
	+
	C \norm{\bff{u}}{\bb{W}^{1,4}} \norm{\bff{\psi}_n}{\bb{W}^{1,4}} \norm{\Delta \bff{\psi}_n}{\bb{L}^2}
	+
	\norm{\bff{\varphi}}{\bb{L}^2} \norm{\Delta \bff{\psi}_n}{\bb{L}^2}
	\\
	&\leq
	C \norm{\bff{\psi}_n}{\bb{H}^1}^2
	+
	\frac{\alpha}{2} \norm{\Delta \bff{\psi}_n}{\bb{L}^2}^2
	+
	\norm{\bff{\varphi}}{\bb{L}^2}^2,
\end{align*}
where we in the last step we also used the Gagliardo--Nirenberg inequality, Sobolev embedding, and Young's inequality. This, together with~\eqref{equ:psi n H1} and the elliptic regularity result, implies~\eqref{equ:psi n phi H2}. The proof is now complete.
\end{proof}

Next, we have the following result on the stability of the elliptic projection.

\begin{lemma}\label{lem:stab elliptic}
Let $\Pi_h\bff{u}$ be the elliptic projection of $\bff{u}$, where $\bff{u}$ is the solution of~\eqref{equ:llb a} satisfying the assumption~\eqref{equ:ass 1}. Suppose that the polynomial degree $r$ in $\bb{V}_h$ satisfies~\eqref{equ:deg r}. Then for all $t\in [0,T]$,
\begin{align*}
	\norm{\Pi_h \bff{u}(t)}{\bb{W}^{1,\infty}}
	&\leq
	C \norm{\bff{u}}{L^\infty(\bb{H}^3)}.
\end{align*}
The constant $C$ depends on $T$, but is independent of $h$ and $\bff{u}$.
\end{lemma}

\begin{proof}
By the triangle inequality, the inverse estimate~\eqref{equ:inverse}, \eqref{equ:eta t}, and the stability of $\mathcal{I}_h$, we obtain
\begin{align*}
	\norm{\Pi_h \bff{u}(t)}{\bb{W}^{1,\infty}}
	&\leq
	\norm{\Pi_h \bff{u}(t)- \mathcal{I}_h(\bff{u}(t))}{\bb{W}^{1,\infty}} 
	+
	\norm{\mathcal{I}_h(\bff{u}(t))}{\bb{W}^{1,\infty}}
	\\
	&\leq
	Ch^{-d/2} \big(\norm{\bff{\eta}(t)}{\bb{H}^1} + \norm{\mathcal{I}_h(\bff{u}(t))- \bff{u}(t)}{\bb{H}^1} \big) 
	+
	C\norm{\bff{u}(t)}{\bb{W}^{1,\infty}}
	\\
	&\leq
	Ch^{-d/2} \big(Ch^r \norm{\bff{u}}{L^\infty(\bb{H}^2)} \big) 
	+
	C\norm{\bff{u}}{L^\infty(\bb{H}^3)}
	\\
	&\leq
	C(1+h^{r-d/2}) \norm{\bff{u}}{L^\infty(\bb{H}^3)},
\end{align*}
where we used the fact that $r\geq d/2$ by the assumption~\eqref{equ:deg r}, and the embedding $\bb{H}^3\hookrightarrow \bb{W}^{1,\infty}$.
\end{proof}



\subsection{Error analysis} \label{subsec:error analysis}

We perform the error analysis for the scheme~\eqref{equ:scheme spin} in this section. Recall that we split the error as a sum of two terms in~\eqref{equ:split theta eta}. We also remark that for $p\in [1,\infty]$
\begin{align}\label{equ:norm delta un Lp}
	\norm{\mathrm{d}_t \bff{u}^n}{\bb{L}^p}
	=
	\norm{\frac{1}{k} \int_{t_{n-1}}^{t_n} \partial_t \bff{u}(t)\, \dt}{\bb{L}^p}
	\leq 
	\norm{\partial_t \bff{u}(t)}{\bb{L}^p}.
\end{align}
First, we have the following stability estimate.

\begin{lemma}
Let $T>0$ be given and let $\bff{u}_h^n$ be defined by~\eqref{equ:scheme spin} with initial data $\bff{u}^0\in \bb{V}_h$. Then for any $n\in \{1,2,\ldots, \lfloor T/k \rfloor\}$,
\begin{align}\label{equ:stab L2 lin scheme}
	\norm{\bff{u}_h^n}{\bb{L}^2}^2
	+
	\sum_{j=1}^n \norm{\bff{u}_h^j- \bff{u}_h^{j-1}}{\bb{L}^2}^2
	+
	k\sum_{j=1}^n \norm{\nabla \bff{u}_h^j}{\bb{L}^2}^2
	\leq
	C \norm{\bff{u}^0}{\bb{L}^2}^2,
\end{align}
where the constant $C$ depends on $T$, but is independent of $n$ and $k$.
\end{lemma}

\begin{proof}
We set $\bff{\chi}=\bff{u}_h^n$ in~\eqref{equ:scheme spin} and note the vector identity~\eqref{equ:aab}.
The required estimate then follows from similar argument as in~\cite[Lemma~3.1]{LeSoeTra24}.
\end{proof}

The following estimates on the nonlinear terms are needed later.

\begin{lemma}\label{lem:ineq forms}
Let $\mathcal{A}, \mathcal{B}, \mathcal{C}$, and $\mathcal{D}$ be as defined in Definition~\ref{def:forms}, and let $\bff{u}$ be the solution of~\eqref{equ:llb a} that satisfies~\eqref{equ:ass 1}. Then for any $\bff{\chi}\in \bb{V}_h$,
\begin{align}
	\label{equ:B nonlinear est}
	\big| \mathcal{B}(\bff{u}_h^{n-1}, \bff{u}_h^{n-1}; \Pi_h \bff{u}^n, \bff{\chi}) - \mathcal{B}(\bff{u}^n, \bff{u}^n; \Pi_h \bff{u}^n, \bff{\chi}) \big| 
	&\leq
	C\left(1+ \norm{\bff{u}_h^{n-1}}{\bb{L}^4}^2 \right) \norm{\bff{\theta}^{n-1}}{\bb{L}^2}^2
	\nonumber\\
	&\quad
	+
	C\left(1+ \norm{\bff{u}_h^{n-1}}{\bb{L}^4}^2 \right) (h^{2(r+1)}+k^2)
	+
	\epsilon \norm{\bff{\chi}}{\bb{H}^1}^2,
	\\
	\label{equ:C nonlinear est}
	\big| \mathcal{C}(\bff{u}_h^{n-1}; \Pi_h \bff{u}^n, \bff{\chi}) - \mathcal{C}(\bff{u}^n; \Pi_h \bff{u}^n, \bff{\chi}) \big| 
	&\leq
	Ch^{2(r+1)} + Ck^2 + C\norm{\bff{\theta}^{n-1}}{\bb{L}^2}^2
	+
	\epsilon \norm{\bff{\chi}}{\bb{H}^1}^2,
	\\
	\label{equ:D nonlinear est}
	\big| \mathcal{D}(\bff{u}_h^n, \bff{\chi}) - \mathcal{D}(\bff{u}^n, \bff{\chi}) \big| 
	&\leq
	Ch^{2(r+1)} + \epsilon \norm{\bff{\chi}}{\bb{H}^1}^2.
\end{align}
Consequently, 
\begin{align}\label{equ:A nonlinear}
	\big|\mathcal{A}(\bff{u}_h^{n-1}; \Pi_h \bff{u}_h^n, \bff{\chi})- \mathcal{A}(\bff{u}^n;\Pi_h \bff{u}^n,\bff{\chi}) \big|
	&\leq
	C\left(1+ \norm{\bff{u}_h^{n-1}}{\bb{L}^4}^2 \right) \norm{\bff{\theta}^{n-1}}{\bb{L}^2}^2
	\nonumber\\
	&\quad
	+
	C\left(1+ \norm{\bff{u}_h^{n-1}}{\bb{L}^4}^2 \right) (h^{2(r+1)}+k^2)
	+
	\epsilon \norm{\bff{\chi}}{\bb{H}^1}^2.
\end{align}
The constant $C$ depends on the coefficients of~\eqref{equ:llb a}, $T$, $K_0$, and $\mathscr{D}$ (but is independent of $n$, $h$, or $k$).
\end{lemma}

\begin{proof}
Firstly, note that we have
\begin{align}\label{equ:uhn1 min un}
	\bff{u}_h^{n-1}- \bff{u}^n = \bff{\theta}^{n-1} + \bff{\eta}^{n-1} - k\cdot \mathrm{d}_t \bff{u}^n.
\end{align}
Thus, by H\"older's inequality, \eqref{equ:norm delta un Lp}, and
\begin{align}\label{equ:un2 L43}
	\norm{|\bff{u}_h^{n-1}|^2 - |\bff{u}^n|^2}{\bb{L}^{4/3}}
	&=
	\norm{(\bff{u}_h^{n-1}+\bff{u}^n) \cdot (\bff{u}_h^{n-1}-\bff{u}^n)}{\bb{L}^{4/3}}
	\nonumber\\
	&\leq
	\norm{\bff{u}_h^{n-1}+\bff{u}^n}{\bb{L}^4} \norm{\bff{\theta}^{n-1} + \bff{\eta}^{n-1} - k\cdot \mathrm{d}_t \bff{u}^n}{\bb{L}^2}
	\nonumber\\
	&\leq
	C\left(1+ \norm{\bff{u}_h^{n-1}}{\bb{L}^4} \right) \norm{\bff{\theta}^{n-1}}{\bb{L}^2}
	+
	C\left(1+ \norm{\bff{u}_h^{n-1}}{\bb{L}^4} \right) (h^2+k).
\end{align}
We can then estimate
\begin{align*}
	&\big| \mathcal{B}(\bff{u}_h^{n-1}, \bff{u}_h^{n-1}; \Pi_h \bff{u}^n, \bff{\chi}) - \mathcal{B}(\bff{u}^n, \bff{u}^n; \Pi_h \bff{u}^n, \bff{\chi}) \big|
	\\
	&=
	\big| \alpha \inpro{(|\bff{u}_h^{n-1}|^2- |\bff{u}^{n}|^2) \Pi_h \bff{u}^n}{\bff{\chi}} \big|
	\\
	&\leq
	\alpha \norm{|\bff{u}_h^{n-1}|^2 - |\bff{u}^n|^2}{\bb{L}^{4/3}} \norm{\Pi_h \bff{u}^n}{\bb{L}^\infty} \norm{\bff{\chi}}{\bb{L}^4}
	\\
	&\leq
	C\left(1+ \norm{\bff{u}_h^{n-1}}{\bb{L}^4}^2 \right) \norm{\bff{\theta}^{n-1}}{\bb{L}^2}^2
	+
	C\left(1+ \norm{\bff{u}_h^{n-1}}{\bb{L}^4}^2 \right) (h^4+k^2)
	+
	\epsilon \norm{\bff{\chi}}{\bb{H}^1}^2,
\end{align*}
where in the last step we used~\eqref{equ:un2 L43}, Young's inequality, and Lemma~\ref{lem:stab elliptic} (noting the Sobolev embedding $\bb{W}^{1,4}\hookrightarrow \bb{L}^\infty$ and $\bb{H}^1\hookrightarrow \bb{L}^4$). This proves~\eqref{equ:B nonlinear est}.

Next, noting~\eqref{equ:C div}, by H\"older's inequality we have
\begin{align*}
	&\big| \mathcal{C}(\bff{u}_h^{n-1}; \Pi_h \bff{u}^n, \bff{\chi}) - \mathcal{C}(\bff{u}^n; \Pi_h \bff{u}^n, \bff{\chi}) \big| 
	\\
	&\leq
	\big| \gamma \inpro{(\bff{u}_h^{n-1}-\bff{u}^n) \times \nabla \Pi_h \bff{u}^n}{\nabla \bff{\chi}}
	+
	\big| \lambda\gamma \inpro{\Pi_h\bff{u}^n \times \bff{e} \big( \bff{e}\cdot (\bff{u}_h^{n-1}-\bff{u}^n)\big)}{\bff{\chi}}
	\\
	&\quad
	+
	\big| \beta_2 \inpro{(\bff{u}_h^{n-1}-\bff{u}^n) \otimes \bff{\nu}}{\nabla(\Pi_h \bff{u}^n \times \bff{\chi})}
	+
	\big| \beta_2 \inpro{\Pi_h \bff{u}^n \times (\nabla\cdot\bff{\nu}) (\bff{u}_h^{n-1}-\bff{u}^n)}{\bff{\chi}}
	\\
	&\leq
	C \norm{\bff{u}_h^{n-1}- \bff{u}^n}{\bb{L}^2} \norm{\Pi_h \bff{u}^n}{\bb{W}^{1,\infty}} \norm{\bff{\chi}}{\bb{H}^1}
	\\
	&\leq
	Ch^4 + Ck^2 + C\norm{\bff{\theta}^{n-1}}{\bb{L}^2}^2
	+
	\epsilon \norm{\bff{\chi}}{\bb{H}^1}^2,
\end{align*}
where in the last step we used~\eqref{equ:uhn1 min un}, \eqref{equ:eta t}, Lemma~\ref{lem:stab elliptic}, and Young's inequality. This proves~\eqref{equ:C nonlinear est}.

Noting~\eqref{equ:D div}, we have by H\"older's inequality,
\begin{align*}
	\big| \mathcal{D}(\bff{u}_h^n, \bff{\chi}) - \mathcal{D}(\bff{u}^n, \bff{\chi}) \big| 
	&\leq
	\big|(\delta+\alpha \mu) \inpro{\bff{u}_h^n-\bff{u}^n}{\bff{\chi}} \big| 
	+
	\big| \alpha\lambda \inpro{\bff{e}\big(\bff{e}\cdot (\bff{u}_h^n-\bff{u}^n)\big)}{\bff{\chi}} \big| 
	\\
	&\quad
	+
	\big| \beta_1 \inpro{(\bff{u}_h^n-\bff{u}^n) \otimes \bff{\nu}}{\nabla \bff{\chi}}
	+
	\big| \beta_1 \inpro{(\nabla \cdot \bff{\nu}) (\bff{u}_h^n-\bff{u}^n)}{\bff{\chi}} \big| 
	\\
	&\leq
	C\norm{\bff{u}_h^n-\bff{u}^n}{\bb{L}^2} \norm{\bff{\chi}}{\bb{H}^1}
	\\
	&\leq
	Ch^4 + \epsilon \norm{\bff{\chi}}{\bb{H}^1}^2,
\end{align*}
thus proving~\eqref{equ:D nonlinear est}. 

Finally, inequality~\eqref{equ:A nonlinear} follows from~\eqref{equ:B nonlinear est}, \eqref{equ:C nonlinear est}, and the triangle inequality. This completes the proof of the lemma.
\end{proof}

We can now prove an error estimate for the scheme~\eqref{equ:scheme spin}.

\begin{proposition}\label{pro:est theta n L2}
	Let $\bff{\theta}^n$ be as defined in \eqref{equ:split theta eta}, where $\bff{u}_h^n$ and $\bff{u}^n$ solve~\eqref{equ:scheme spin} and~\eqref{equ:weak form AB}, respectively. Then for $n\in \{1,2,\ldots,\lfloor T/k \rfloor\}$,
	\begin{align*}
		\norm{\bff{\theta}^n}{\bb{L}^2}^2
		+
		k \sum_{m=1}^n \norm{\nabla \bff{\theta}^m}{\bb{L}^2}^2
		\leq
		C (h^{2(r+1)}+k^2),
	\end{align*}
	where $C$ depends on the coefficients of the equation, $K_0$, $T$, and $\mathscr{D}$ (but is independent of $n$, $h$ or $k$).
\end{proposition}

\begin{proof}
Noting the definition of $\bff{\theta}^n=\bff{u}_h^n-\Pi_h \bff{u}^n$ and $\bff{\eta}^n= \Pi_h\bff{u}^n-\bff{u}^n$, the scheme~\eqref{equ:scheme spin}, and the weak formulation~\eqref{equ:weak special}, we have for all $\bff{\chi}\in \bb{V}_h$,
\begin{align}\label{equ:A theta n}
	&\inpro{\mathrm{d}_t \bff{\theta}^n}{\bff{\chi}}
	+
	\mathcal{A}(\bff{u}_h^{n-1};\bff{\theta}^n, \bff{\chi})
	\nonumber\\
	&=
	\inpro{\mathrm{d}_t \bff{u}_h^n}{\bff{\chi}}
	+
	\mathcal{A}(\bff{u}_h^{n-1};\bff{u}_h^n, \bff{\chi})
	-
	\inpro{\mathrm{d}_t \Pi_h \bff{u}^n}{\bff{\chi}}
	-
	\mathcal{A}(\bff{u}_h^{n-1}; \Pi_h \bff{u}^n, \bff{\chi})
	\nonumber\\
	&=
	\mathcal{D}(\bff{u}_h^n,\bff{\chi})
	-
	\inpro{\mathrm{d}_t \Pi_h\bff{u}^n- \partial_t \bff{u}^n}{\bff{\chi}}
	-
	\inpro{\partial_t \bff{u}^n}{\bff{\chi}}
	\nonumber\\
	&\quad
	-
	\left(\mathcal{A}(\bff{u}_h^{n-1}; \Pi_h \bff{u}_h^n, \bff{\chi})- \mathcal{A}(\bff{u}^n;\Pi_h \bff{u}^n,\bff{\chi})\right) 
	-
	\mathcal{A}(\bff{u}^n; \Pi_h\bff{u}^n, \bff{\chi})
	\nonumber\\
	&=
	\mathcal{D}(\bff{u}_h^n- \bff{u}^n, \bff{\chi})
	-
	\inpro{\mathrm{d}_t \bff{\eta}^n}{\bff{\chi}}
	-
	\inpro{\mathrm{d}_t \bff{u}^n-\partial_t \bff{u}^n}{\bff{\chi}}
	-
	\left(\mathcal{A}(\bff{u}_h^{n-1}; \Pi_h \bff{u}_h^n, \bff{\chi})- \mathcal{A}(\bff{u}^n;\Pi_h \bff{u}^n,\bff{\chi})\right).
\end{align}
Now, we set $\bff{\chi}=\bff{\theta}^n$ in~\eqref{equ:A theta n}. Applying Lemma~\ref{lem:ineq forms} and Proposition~\ref{pro:eta t}, noting~\eqref{equ:norm delta un Lp}, the identity~\eqref{equ:aab} and the coercivity of $\mathcal{A}$, we obtain
\begin{align*}
	&\frac{1}{2k} \left(\norm{\bff{\theta}^n}{\bb{L}^2}^2 - \norm{\bff{\theta}^{n-1}}{\bb{L}^2}^2 \right)
	+
	\frac{1}{2k} \norm{\bff{\theta}^n - \bff{\theta}^{n-1}}{\bb{L}^2}^2
	+
	\mu \norm{\bff{\theta}^n}{\bb{H}^1}^2
	\\
	&\leq 
	\big| \mathcal{D}(\bff{u}_h^n-\bff{u}^n, \bff{\theta}^n) \big|
	+
	\big| \inpro{\mathrm{d}_t \bff{\eta}^n}{\bff{\theta}^n} \big| 
	+
	\big|\inpro{\mathrm{d}_t \bff{u}^n-\partial_t \bff{u}^n}{\bff{\theta}^n}\big|
	+
	\big|\mathcal{A}(\bff{u}_h^{n-1}; \Pi_h \bff{u}_h^n, \bff{\theta}^n)- \mathcal{A}(\bff{u}^n;\Pi_h \bff{u}^n,\bff{\theta}^n) \big|
	\\
	&\leq
	C\left(1+ \norm{\bff{u}_h^{n-1}}{\bb{L}^4}^2 \right) \norm{\bff{\theta}^{n-1}}{\bb{L}^2}^2
	+
	C\left(1+ \norm{\bff{u}_h^{n-1}}{\bb{L}^4}^2 \right) (h^{2(r+1)}+k^2)
	+
	\frac{\mu}{2} \norm{\bff{\theta}^n}{\bb{H}^1}^2.
\end{align*}
Summing over $m\in \{1,2,\ldots,n\}$, using the embedding $\bb{H}^1\hookrightarrow \bb{L}^4$ and noting~\eqref{equ:stab L2 lin scheme}, we obtain
\begin{align*}
	\norm{\bff{\theta}^n}{\bb{L}^2}^2
	+
	k \sum_{m=1}^n \norm{\bff{\theta}^m}{\bb{H}^1}^2
	\leq
	\norm{\bff{\theta}^0}{\bb{L}^2}^2
	+
	C \left(h^{2(r+1)} + k^2 \right)
	+
	C \sum_{m=1}^n \left(1+ \norm{\bff{u}_h^{m-1}}{\bb{H}^1}^2 \right) \norm{\bff{\theta}^{m-1}}{\bb{L}^2}^2.
\end{align*}
Note that since we take $\bff{u}_h^0= P_h\bff{u}^0$, by~\eqref{equ:proj approx} and~\eqref{equ:eta t},
\begin{align*}
\norm{\bff{\theta}^0}{\bb{L}^2}= \norm{\bff{u}_h^0- \Pi_h \bff{u}^0}{\bb{L}^2}
\leq 
\norm{P_h \bff{u}^0- \bff{u}^0}{\bb{L}^2} + \norm{\bff{u}^0- \Pi_h \bff{u}^0}{\bb{L}^2}
\leq
Ch^{r+1}.
\end{align*}
Thus, the required inequality follows from the discrete Gronwall lemma, noting again~\eqref{equ:stab L2 lin scheme}.
\end{proof}


\begin{theorem}\label{the:spin torq error}
	Let $\bff{u}_h^n$ and $\bff{u}$ be the solution of \eqref{equ:scheme spin} and \eqref{equ:weak form AB}, respectively. For $n\in \{1,2,\ldots,\lfloor T/k \rfloor\}$,
	\begin{align*}
		\norm{\bff{u}_h^n-\bff{u}(t_n)}{\bb{L}^2}
		\leq 
		C(h^{r+1} +k),
	\end{align*}
	where $C$ depends on the coefficients of the equation, $K_0$, $T$, and $\mathscr{D}$ (but is independent of $n$, $h$ or $k$).
\end{theorem}

\begin{proof}
	This follows from Proposition \ref{pro:est theta n L2} and the triangle inequality (noting \eqref{equ:split theta eta} and~\eqref{equ:eta t}).
\end{proof}




\section{An energy-dissipative fully discrete scheme for the LLB equation}

Now, we consider a special case of~\eqref{equ:llb a} where $\beta_1=\beta_2=0$. In the absence of spin current, it is known that the physical system described by the LLB equation dissipates energy~\cite{ChuNie20} (see also~\cite{DiInnPra20} for the standard Landau--Lifshitz--Gilbert equation). Indeed, assuming adequate regularity, taking the inner product of~\eqref{equ:llb eq1} and \eqref{equ:llb eq2} with $\bff{H}$ and $\partial_t \bff{u}$ respectively, then adding the resulting equations, we obtain for any $t_1<t_2$,
\begin{equation*}
	\mathcal{E}(\bff{u}(t_2)) + \alpha \int_{t_1}^{t_2} \norm{\bff{H}(s)}{\bb{L}^2}^2 \ds 
	=
	\mathcal{E}(\bff{u}(t_1)),
\end{equation*}
which implies
\begin{equation*}
	\mathcal{E}(\bff{u}(t_2)) \leq \mathcal{E}(\bff{u}(t_1)), \quad \forall t_1\leq t_2.
\end{equation*}
Therefore, it is desirable to have a numerical scheme which preserves this property at the discrete level. 
Here, we propose an energy-dissipative fully discrete finite element scheme for the LLB equation in the absence of spin torques (i.e. $\beta_1=\beta_2=0$) for $d=1,2$, or $3$. Recall that the weak formulation for the problem in this special case can be written as
\begin{align}\label{equ:weak form no spin}
	\inpro{\partial_t \bff{u}(t)}{\bff{\chi}}
	=
	-\gamma \inpro{\bff{u}(t)\times \bff{H}(t)}{\bff{\chi}}
	+
	\alpha \inpro{\bff{H}(t)}{\bff{\chi}},
\end{align}
where $\bff{H}=\Delta \bff{u}+\bff{w}$, and $\bff{w}$ was defined in~\eqref{equ:w}.

For ease of presentation, we assume $d=3$, noting that similar (and simpler) argument will also hold for $d=1$ or $2$. Let $\bb{V}_h$ be the finite element space defined in~\eqref{equ:Vh} with $r\geq 1$. Let $k$ be the time step and $\bff{u}_h^n$ be the approximation in $\bb{V}_h$ of $\bff{u}(t)$ at time $t=t_n=nk\in [0,T]$, where $n=0,1,2,\ldots, \lfloor T/k \rfloor$.
A fully discrete scheme to solve~\eqref{equ:llb a} can be described as follows. We start with $\bff{u}_h^0= R_h \bff{u}_0 \in \bb{V}_h$ for simplicity. For $t_n\in [0,T]$, $n\in \bb{N}$, given $\bff{u}_h^{n-1}\in \bb{V}_h$, define $\bff{u}_h^n$ by
\begin{align}\label{equ:scheme 2}
	\inpro{\mathrm{d}_t \bff{u}_h^n}{\bff{\chi}}
	&=
	-
	\gamma \inpro{\bff{u}_h^n \times \bff{H}_h^n}{\bff{\chi}}
	+
	\alpha \inpro{\bff{H}_h^n}{\bff{\chi}}, \quad \forall \bff{\chi}\in \bb{V}_h,
\end{align}
where
\begin{align}
	\label{equ:Hhn}
	\bff{H}_h^n
	&:=
	\Delta_h \bff{u}_h^n
	+
	\bff{w}_h^n,
	\\
	\label{equ:whn}
	\bff{w}_h^n
	&:=
	\mu \bff{u}_h^{n-1}
	-
	P_h \big(|\bff{u}_h^n|^2 \bff{u}_h^n\big)
	+
	\lambda \bff{e}\left(\bff{e}\cdot \bff{u}_h^{n-1}\right).
\end{align}
In the analysis, we will assume $\mu>0$ and $\lambda>0$ for ease of presentation. To maintain energy dissipativity, if $\mu<0$, then replace the term $\mu \bff{u}_h^{n-1}$ by $\mu \bff{u}_h^n$. If $\lambda<0$, then replace the term $\lambda \bff{e} \left(\bff{e} \cdot \bff{u}_h^{n-1}\right)$ by $\lambda \bff{e} \left(\bff{e} \cdot \bff{u}_h^{n}\right)$. The analysis will not change significantly in these cases. 

The following proposition shows that the scheme \eqref{equ:scheme 2} is well-defined.

\begin{proposition}
	Given $k>0$ and $\bff{u}_h^{n-1}\in \bb{V}_h$, there exists $\bff{u}_h^n\in \bb{V}_h$ that solves the fully discrete scheme \eqref{equ:scheme 2}.
\end{proposition}

\begin{proof}
	Given $\bff{u}_h^{n-1}\in \bb{V}_h$, let $\mathcal{A}_n:\bb{V}_h\to \bb{V}_h$ be a nonlinear operator defined by
	\begin{equation*}
		\mathcal{A}_n(\bff{v}):=
		\mu \bff{u}_h^{n-1} 
		-
		P_h\big(|\bff{v}|^2 \bff{v}\big)
		+
		\lambda \bff{e}(\bff{e}\cdot \bff{u}_h^{n-1}).
	\end{equation*}
	Define a map $G_h:\bb{V}_h\to \bb{V}_h$ by
	\begin{align*}
		G_h(\bff{v})
		&:=
		\bff{v}
		-
		\bff{u}_h^{n-1}
		+
		k\gamma P_h \big( \bff{v}\times (\Delta_h \bff{v}+ \mathcal{A}_n(\bff{v})) \big)
		-
		k \alpha \big( \Delta_h \bff{v}+ \mathcal{A}_n(\bff{v}) \big).
	\end{align*}
	The scheme~\eqref{equ:scheme 2} is equivalent to solving~$G_h\big(\bff{u}_h^n \big)= \bff{0}$. The existence of solution to this equation will be shown using the Brouwer fixed point theorem. To this end, let~$B_\rho:= \{\bff{v}\in \bb{V}_h: \norm{\bff{v}}{\bb{L}^2} \leq \rho\}$. For any $\bff{v}\in \partial B_\rho$, we have
	\begin{align*}
		\inpro{G_h(\bff{v})}{\bff{v}}
		&=
		\norm{\bff{v}}{\bb{L}^2}^2
		-
		\inpro{\bff{u}_h^{n-1}}{\bff{v}}
		+
		k\alpha \norm{\nabla \bff{v}}{\bb{L}^2}^2
		-
		k\alpha \inpro{\mathcal{A}_n(\bff{v})}{\bff{v}}
		\\
		&=
		\frac12 \left(\norm{\bff{v}}{\bb{L}^2}^2 - \norm{\bff{u}_h^{n-1}}{\bb{L}^2}^2 \right) 
		+
		\frac12 
		\norm{\bff{v}-\bff{u}_h^{n-1}}{\bb{L}^2}^2
		+
		k\alpha \norm{\nabla \bff{v}}{\bb{L}^2}^2
		-
		k\alpha\mu \inpro{\bff{u}_h^{n-1}}{\bff{v}}
		+
		k\alpha \norm{\bff{v}}{\bb{L}^4}^4
		\\
		&\quad
		-
		k\alpha \lambda \inpro{\bff{e}(\bff{e}\cdot \bff{u}_h^{n-1})}{\bff{v}}
		\\
		&\geq
		\frac12 \left(\norm{\bff{v}}{\bb{L}^2}^2 - \norm{\bff{u}_h^{n-1}}{\bb{L}^2}^2 \right) 
		+
		k\alpha \norm{\nabla \bff{v}}{\bb{L}^2}^2
		+
		k\alpha \norm{\bff{v}}{\bb{L}^4}^4
		-
		k\alpha(\mu+\lambda) \norm{\bff{u}_h^{n-1}}{\bb{L}^2} \norm{\bff{v}}{\bb{L}^2}
		\\
		&\geq
		\frac12 \left(\rho^2 - C_{\rm{S}}\right) + k\alpha \Big( \rho^4 - (\mu+\lambda)C_{\rm{S}}^{\frac12} \rho \Big),
	\end{align*}
	where $C_{\rm{S}}$ is the constant in the stability estimate~\eqref{equ:stab H1}. Therefore, for sufficiently large $\rho$, precisely
	\[
		\rho > \max \Big\{ C_{\rm{S}}^{\frac12}, (\mu+\lambda)^{\frac13} C_{\rm{S}}^{\frac16} \Big\},
	\]
	we have $\inpro{G_h(\bff{v})}{\bff{v}}>0$. Thus, we infer the existence of $\bff{u}_h^n\in \bb{V}_h$ solving~\eqref{equ:scheme 2} by the Brouwer's fixed point theorem.
\end{proof}


Next, we show the energy dissipativity and the long-time unconditional stability of the scheme.

\begin{lemma}
Let $T>0$ be given and let $\bff{u}_h^n$ be defined by \eqref{equ:scheme 2}. Then for any initial data $\bff{u}_h^0\in \bb{V}_h$ and $n\in \{1,2,\ldots, \lfloor T/k \rfloor\}$,
\begin{align}\label{equ:stab ene}
	\mathcal{E}(\bff{u}_h^{n}) 
	\leq
	\mathcal{E}(\bff{u}_h^{n-1}),
\end{align}
where $\mathcal{E}$ was defined in \eqref{equ:energy}. Consequently,
\begin{align}\label{equ:stab H1}
	\norm{\bff{u}_h^n}{\bb{L}^4}^4
	+
	\norm{\bff{u}_h^n}{\bb{H}^1}^2
	+
	k\sum_{m=1}^n \norm{\bff{H}_h^{m}}{\bb{L}^2}^2
	\leq
	C_{\rm{S}},
\end{align}
where $C_{\rm{S}}$ depends only on the coefficients of the equation, $\norm{\bff{u}_h^0}{\bb{H}^1}$, and $\mathscr{D}$.
\end{lemma}

\begin{proof}
Let $\bff{H}_h^n$ be as defined in~\eqref{equ:Hhn}. Taking $\bff{\chi}= \bff{H}_h^n$ in~\eqref{equ:scheme 2}, we have
\begin{align}\label{equ:dtuhn H}
	\inpro{\mathrm{d}_t \bff{u}_h^n}{\bff{H}_h^n}
	&=
	\alpha \norm{\bff{H}_h^n}{\bb{L}^2}^2.
\end{align}
On the other hand, using the identity\eqref{equ:aab} we obtain
\begin{align}\label{equ:H du mu}
	\inpro{\bff{H}_h^n}{\mathrm{d}_t \bff{u}_h^n}
	&=
	-
	\frac{\alpha}{2k} \big(\norm{\nabla \bff{u}_h^{n}}{\bb{L}^2}^2 - \norm{\nabla \bff{u}_h^{n-1}}{\bb{L}^2}^2 \big)
	-
	\frac{\alpha}{2k}\norm{\nabla \bff{u}_h^{n}- \nabla \bff{u}_h^{n-1}}{\bb{L}^2}^2
	\nonumber\\
	\nonumber
	&\quad
	-
	\frac{1}{4k} \left(\norm{\abs{\bff{u}_h^{n}}^2 - \mu}{\bb{L}^2}^2 - \norm{\abs{\bff{u}_h^{n-1}}^2 -\mu}{\bb{L}^2}^2 \right)
	-
	\frac{1}{4k} \norm{\abs{\bff{u}_h^{n}}^2 - \abs{\bff{u}_h^{n-1}}^2}{\bb{L}^2}^2
	\\
	&\quad
	-
	\frac{k}{2} \norm{\abs{\bff{u}_h^{n}} \abs{\mathrm{d}_t \bff{u}_h^{n}}}{\bb{L}^2}^2
	-
	\frac{\mu k}{2} \norm{\bff{u}_h^{n}-\bff{u}_h^{n-1}}{\bb{L}^2}^2
	\nonumber \\
	&\quad
	+
	\frac{\lambda}{2k} \left(\norm{\bff{e}\cdot \bff{u}_h^{n}}{L^2}^2 - \norm{\bff{e}\cdot \bff{u}_h^{n-1}}{L^2}^2 \right)
	-
	\frac{\lambda}{2k} \norm{\bff{e}\cdot (\bff{u}_h^{n}-\bff{u}_h^{n-1})}{L^2}^2.
\end{align}
Substituting~\eqref{equ:H du mu} into \eqref{equ:dtuhn H} and rearranging the terms immediately imply~\eqref{equ:stab ene}. Thus, by splitting the second term in~\eqref{equ:energy}, we have
\begin{align*}
	&\frac{\alpha}{2} \norm{\nabla \bff{u}_h^n}{\bb{L}^2}^2
	+
	\frac{1}{8} \norm{\bff{u}_h^n}{\bb{L}^4}^4
	+
	\frac{\mu+\lambda}{2} \norm{\bff{u}_h^n}{\bb{L}^2}^2
	+
	\frac{\mu^2 \abs{\mathscr{D}}}{4} 
	+
	\int_\mathscr{D} \left(\frac{1}{8} \abs{\bff{u}_h^n}^4
	-
	(\mu+\lambda) \abs{\bff{u}_h^n}^2\right) \, \dx
	\leq
	\mathcal{E}(\bff{u}_h^0).
\end{align*}
By considering the minimum value of the function $x\mapsto Ax^4-Bx^2$ to bound the integral term, we obtain
\begin{align*}
	&\frac{\alpha}{2} \norm{\nabla \bff{u}_h^n}{\bb{L}^2}^2
	+
	\frac{1}{8} \norm{\bff{u}_h^n}{\bb{L}^4}^4
	+
	\frac{\mu+\lambda}{2} \norm{\bff{u}_h^n}{\bb{L}^2}^2
	\leq
	\mathcal{E}(\bff{u}_h^0)
	+
	\left(2(\mu+\lambda)^2 - \frac{\mu^2}{4} \right) \abs{\mathscr{D}} 
	\leq C,
\end{align*}
proving~\eqref{equ:stab H1}. This completes the proof of the lemma.
\end{proof}


\begin{remark}
%In fact, for the case $0\leq\tau <1$, similar to \eqref{equ:ineq H1 ener}, more generally we have for any $\delta>0$,
%\begin{align}\label{equ:ineq E neg}
%	\frac{1}{2} \norm{\nabla \bff{u}_h^n}{\bb{L}^2}^2
%	+
%	\delta \norm{\bff{u}_h^n}{\bb{L}^4}^4
%	+
%	\delta \norm{\bff{u}_h^n}{\bb{L}^2}^2
%	\leq
%	\mathcal{E}(\bff{u}_h^{n-1})
%	+
%	\left( \frac{(\kappa\mu+\lambda-2\delta)^2}{4(\kappa-4\delta)} - \frac{\kappa\mu}{4}\right) \abs{\mathscr{D}}.
%\end{align}
%One could see that if at some iteration the energy $\mathcal{E}(\bff{u}_h^{n-1})$ is below certain threshold, then for some sufficiently small $\lambda$ and $\delta$, there exists $\mu\in (0,1)$ such that the right-hand side of \eqref{equ:ineq E neg} becomes negative. This implies $\bff{u}_h^m\equiv \bff{0}$ for all $m\geq n$.

More can be said if $\mu<0$. To emphasise the sign of $\mu$, we write $\abs{\mu}=-\mu$. In addition, suppose that the anisotropy constant $\lambda<\alpha \abs{\mu}$.  Then taking $\bff{\chi}= \bff{u}_h^n$ in \eqref{equ:scheme 2} and noting \eqref{equ:Hhn} yield
\begin{align*}
	\frac{1}{2k} \left(\norm{\bff{u}_h^{n}}{\bb{L}^2}^2 - \norm{\bff{u}_h^{n-1}}{\bb{L}^2}^2 \right) 
	+
	\alpha \norm{\nabla \bff{u}_h^{n}}{\bb{L}^2}^2
	+
	\alpha \abs{\mu} \norm{\bff{u}_h^{n}}{\bb{L}^2}^2
	+
	\norm{\bff{u}_h^{n}}{\bb{L}^4}^4
	\leq
	\lambda \norm{\bff{u}_h^{n}}{\bb{L}^2}^2.
\end{align*}
Rearranging the terms, we have
\begin{align*}
	\frac{1}{2k} \left(\norm{\bff{u}_h^{n}}{\bb{L}^2}^2 - \norm{\bff{u}_h^{n-1}}{\bb{L}^2}^2 \right) 
	+
	(\alpha\abs{\mu}-\lambda) \norm{\bff{u}_h^{n}}{\bb{L}^2}^2
	\leq 0.
\end{align*}
By a version of the discrete Gronwall lemma~\cite[equation (3.4)]{Emm99}, we obtain an exponential decay of the finite element approximation for the mean square magnetisation to zero:
\begin{align*}
	\norm{\bff{u}_h^n}{\bb{L}^2} 
	\leq
	e^{-\beta t_n} \norm{\bff{u}_h^0}{\bb{L}^2},
\end{align*}
for some positive constant $\beta< \alpha\abs{\mu}-\lambda$, in agreement with the qualitative behaviour for the continuous problem~\cite{LeSoeTra24}.
\end{remark}


We also have the following discrete $\bb{L}^\infty$ stability of the scheme.

\begin{lemma}
Let $T>0$ be given and let $\bff{u}_h^n$ be defined by \eqref{equ:scheme 2}. Then for any initial data $\bff{u}_h^0\in \bb{V}_h$ and $n\in \{1,2,\ldots, \lfloor T/k \rfloor\}$,
\begin{align}\label{equ:stab disc lap}
	k \sum_{m=1}^n \norm{\Delta_h \bff{u}_h^{m}}{\bb{L}^2}^2
	\leq
	C_{\Delta}.
\end{align}
Consequently, if $\mathscr{D}$ is a convex polygonal or polyhedral domain with quasi-uniform triangulation, then
\begin{align}\label{equ:stab L infty}
	k \sum_{m=1}^n \norm{\bff{u}_h^m}{\bb{L}^\infty}^2
	\leq
	C_{\infty},
\end{align}
where constants $C_{\Delta}$ and $C_\infty$ depend only on the coefficients of the equation, $\norm{\bff{u}_h^0}{\bb{H}^1}$, $\mathscr{D}$, and $T$.
\end{lemma}

\begin{proof}
Noting~\eqref{equ:Hhn}, by H\"older's inequality we have
\begin{align*}
	\norm{\Delta_h \bff{u}_h^{n}}{\bb{L}^2}
	&\leq
	\norm{\bff{H}_h^{n}}{\bb{L}^2}
	+
	\mu \norm{\bff{u}_h^{n-1}}{\bb{L}^2}
	+
	\norm{\bff{u}_h^{n}}{\bb{H}^1}^3
	+
	\lambda \norm{\bff{e}}{\bb{L}^\infty}^2 \norm{\bff{u}_h^{n-1}}{\bb{L}^2},
\end{align*}
where in the last step we used the Sobolev embedding $\bb{H}^1 \hookrightarrow \bb{L}^6$. Multiplying both sides by $k$, then summing over $m\in \{1,2,\ldots, n\}$, and using \eqref{equ:stab H1} give \eqref{equ:stab disc lap}. Applying \eqref{equ:disc lapl L infty} then yields \eqref{equ:stab L infty}.
\end{proof}

In the next few lemmas, we will derive some estimates for the nonlinear terms to aid in the analysis, analogous to those in the previous section. To this end, we split the approximation error by writing
\begin{align}\label{equ:theta rho split 2}
	\bff{u}_h^n-\bff{u}^n
	=
	\left(\bff{u}_h^n - R_h \bff{u}^n\right)
	+
	\left(R_h \bff{u}^n - \bff{u}^n\right)
	=
	\bff{\theta}^n + \bff{\rho}^n,
\end{align}
where $R_h$ is the Ritz projection defined in~\eqref{equ:Ritz}, while $\bff{\rho}^n$ enjoys the estimates~\eqref{equ:Ritz ineq}, \eqref{equ:Ritz ineq L infty}, and \eqref{equ:Ritz stab u infty}. 
We also remark that since $\Delta_h R_h \bff{v}=P_h \Delta \bff{v}$ for any $\bff{v}\in \bb{H}^2_{\mathrm{N}}$, we have
\begin{align}\label{equ:Delta uhn Delta un}
	\Delta_h \bff{u}_h^n- \Delta \bff{u}^n
	&=
	\Delta_h \bff{\theta}^n + (P_h-I) \Delta \bff{u}^n.
\end{align}

Noting \eqref{equ:Hhn}, \eqref{equ:Ritz}, and~\eqref{equ:Delta uhn Delta un}, we can write
\begin{align}\label{equ:Hh minus Hn}
	\nonumber
	\bff{H}_h^{n}-\bff{H}^{n}
	&=
	\Delta_h \bff{\theta}^{n} 
	+ 
	(P_h-I) \Delta \bff{u}^{n} 
	+
	\mu (\bff{\theta}^{n-1}+ \bff{\rho}^{n-1})
	+
	\mu (\bff{u}^{n-1}-\bff{u}^{n})
	-
	P_h \left(|\bff{u}_h^{n}|^2 \bff{u}_h^{n}-|\bff{u}^{n}|^2 \bff{u}^{n}\right) 
	\nonumber \\
	&\quad
	-
	(P_h-I) \left(|\bff{u}^{n}|^2 \bff{u}^{n}\right)
	+
	\lambda\bff{e} \left(\bff{e}\cdot (\bff{\theta}^{n-1}+\bff{\rho}^{n-1})\right)
	+
	\lambda\bff{e} \left(\bff{e}\cdot (\bff{u}^{n-1}-\bff{u}^n)\right),
\end{align}
where $I$ is the identity operator. Moreover, we have
\begin{align}\label{equ:split cubic}
	|\bff{u}_h^{n}|^2 \bff{u}_h^{n}-|\bff{u}^{n}|^2 \bff{u}^{n}
	&=
	\abs{\bff{u}_h^n}^2 \left(\bff{\theta}^n+\bff{\rho}^n\right)
	+
	\big((\bff{\theta}^n+\bff{\rho}^n)\cdot (\bff{u}_h^n+\bff{u}^n)\big) \bff{u}^n.
\end{align}

\begin{lemma}\label{lem:inpro theta n}
Let $\epsilon >0$ be given. Let $\bff{u}_h^n$ and $\bff{H}_h^n$ be defined by \eqref{equ:scheme 2} and \eqref{equ:Hhn}, respectively. Then for any initial data $\bff{u}_h^0\in \bb{V}_h$ and $n\in \{1,2,\ldots, \lfloor T/k \rfloor\}$,
\begin{align}
	\label{equ:inpro uhn theta n}
	\big| \inpro{\bff{u}_h^n \times \bff{H}_h^n - \bff{u}^n \times \bff{H}^n}{\bff{\theta}^n} \big| 
	&\leq
	C \left(1+ \norm{\bff{u}_h^n}{\bb{L}^\infty}^2\right) h^{2(r+1)} + Ck^2 
	+ 
	C\norm{\bff{\theta}^n}{\bb{L}^2}^2
	+
	C\norm{\bff{\theta}^{n-1}}{\bb{L}^2}^2
 \nonumber \\
	&\quad
	+
	\epsilon \norm{\nabla \bff{\theta}^n}{\bb{L}^2}^2
	+
	\epsilon \norm{\Delta_h \bff{\theta}^n}{\bb{L}^2}^2,
	\\
	\label{equ:inpro uhn Delta theta n}
	\big| \inpro{\bff{u}_h^n \times \bff{H}_h^n - \bff{u}^n \times \bff{H}^n}{\Delta_h \bff{\theta}^n} \big| 
	&\leq 
	C \left(1+ \norm{\bff{u}_h^n}{\bb{L}^\infty}^2\right) h^{2(r+1)} + Ck^2 
	+ 
	C\norm{\bff{\theta}^n}{\bb{H}^1}^2
     +
	C\norm{\bff{\theta}^{n-1}}{\bb{L}^2}^2
 \nonumber \\
	&\quad
	+
	\epsilon \norm{\Delta_h \bff{\theta}^n}{\bb{L}^2}^2,
\end{align}
where $C$ depends on the coefficients of the equation, $K_0$, $T$, and $\mathscr{D}$ (but is independent of $n$, $h$ or $k$).
\end{lemma}

\begin{proof}
Noting~\eqref{equ:Hh minus Hn}, we can write
\begin{align*}
	&\bff{u}_h^n \times \bff{H}_h^n- \bff{u}^n\times \bff{H}^n
	\\
	&=
	\bff{u}_h^n \times \left(\bff{H}_h^n-\bff{H}^n\right) 
	+
	\left(\bff{u}_h^n- \bff{u}^n\right) \times \bff{H}^n
	\\
	&=
	\bff{u}_h^n \times \Delta_h \bff{\theta}^n
	+
	\bff{u}_h^n \times (P_h-I) \Delta \bff{u}^n
	+
	\mu \bff{u}_h^n\times \left(\bff{\theta}^{n-1}+\bff{\rho}^{n-1}\right) 
	-
	\mu k \left( \bff{u}_h^n \times \mathrm{d}_t \bff{u}^n \right)
	\\
	&\quad
	-
	\bff{u}_h^n \times 	P_h \left(|\bff{u}_h^{n}|^2 \bff{u}_h^{n}-|\bff{u}^{n}|^2 \bff{u}^{n}\right) 
	-
	\bff{u}_h^n \times (P_h-I) \left(|\bff{u}^{n}|^2 \bff{u}^{n}\right)
	+
	\lambda \bff{u}_h^n \times \left(\bff{e}\cdot (\bff{\theta}^{n}+\bff{\rho}^{n})\right)
	\\
	&\quad
	+
	\lambda k \big(\bff{u}_h^n \times \bff{e} \left( \bff{e} \cdot \mathrm{d}_t \bff{u}^n \right)\big)
	+
	\left( \bff{\theta}^n + \bff{\rho}^n \right) \times \bff{H}^n
	\\
	&=:
	I_1+I_2+\cdots+I_9.
\end{align*}
We will take the inner product of each term on the last line with $\bff{\theta}^n$ and estimate it in the following. For the first term, by H\"older's and Young's inequalities, as well as estimates~\eqref{equ:Ritz ineq} and~\eqref{equ:gal nir uh L4}, we have
\begin{align*}
	\abs{\inpro{I_1}{\bff{\theta}^n}}
	&\leq
	\norm{\bff{u}_h^n}{\bb{L}^4} \norm{\Delta_h \bff{\theta}^n}{\bb{L}^2} \norm{\bff{\theta}^n}{\bb{L}^4}
	\leq
	C \norm{\bff{\theta}^n}{\bb{L}^2}^2
	+
	\epsilon \norm{\nabla \bff{\theta}^n}{\bb{L}^2}^2
	+
	\epsilon \norm{\Delta_h \bff{\theta}^n}{\bb{L}^2}^2.
\end{align*}
Similarly, for the terms $I_3$, $I_4$, $I_7$, $I_8$, and $I_9$, we have
\begin{align*}
	\abs{\inpro{I_3}{\bff{\theta}^n}}
	&\leq
	\mu \norm{\bff{u}_h^n}{\bb{L}^4} \norm{\bff{\theta}^{n-1}+\bff{\rho}^{n-1}}{\bb{L}^2} \norm{\bff{\theta}^n}{\bb{L}^4}
	\\
	&\leq
	Ch^{2(r+1)}
	+
	C\norm{\bff{\theta}^n}{\bb{L}^2}^2
	+
	C\norm{\bff{\theta}^{n-1}}{\bb{L}^2}^2
	+
	\epsilon \norm{\nabla \bff{\theta}^n}{\bb{L}^2}^2,
	\\
	\abs{\inpro{I_4}{\bff{\theta}^n}}
	&\leq
	\mu k \norm{\bff{u}_h^n}{\bb{L}^4} \norm{\mathrm{d}_t \bff{u}^n}{\bb{L}^4} \norm{\bff{\theta}^n}{\bb{L}^2}
	\leq
	Ck^2 + \epsilon \norm{\bff{\theta}^n}{\bb{L}^2}^2,
	\\
	\abs{\inpro{I_7}{\bff{\theta}^n}}
	&\leq
	\lambda \norm{\bff{u}_h^n}{\bb{L}^4} \norm{\bff{\theta}^n+\bff{\rho}^n}{\bb{L}^2} \norm{\bff{\theta}^n}{\bb{L}^4}
	\leq
	Ch^{2(r+1)}
	+
	C\norm{\bff{\theta}^n}{\bb{L}^2}^2
	+
	\epsilon \norm{\nabla \bff{\theta}^n}{\bb{L}^2}^2,
	\\
	\abs{\inpro{I_8}{\bff{\theta}^n}}
	&\leq
	\lambda k \norm{\bff{u}_h^n}{\bb{L}^4} \norm{\mathrm{d}_t \bff{u}^n}{\bb{L}^4} \norm{\bff{\theta}^n}{\bb{L}^2}
	\leq
	Ck^2 + \epsilon \norm{\bff{\theta}^n}{\bb{L}^2}^2,
	\\
	\abs{\inpro{I_9}{\bff{\theta}^n}}
	&\leq
	\norm{\bff{\rho}^n}{\bb{L}^2} \norm{\bff{H}^n}{\bb{L}^\infty} \norm{\bff{\theta}^n}{\bb{L}^2}
	\leq
	Ch^{2(r+1)}
	+
	\epsilon\norm{\bff{\theta}^n}{\bb{L}^2}^2.
\end{align*}
For the terms $I_2$ and $I_6$, we use \eqref{equ:proj approx} and Young's inequality to obtain
\begin{align*}
	\abs{\inpro{I_2}{\bff{\theta}^n}} + \abs{\inpro{I_6}{\bff{\theta}^n}}
	&\leq
	Ch^{2(r+1)} + \epsilon \norm{\bff{\theta}^n}{\bb{L}^2}^2.
\end{align*}
Finally, for the term $I_5$, we use~\eqref{equ:split cubic}, \eqref{equ:stab H1}, the stability of $P_h$, and the fact that $\bff{\theta}^n= \bff{u}_h^n- R_h\bff{u}^n$ to obtain
\begin{align}\label{equ:I5 theta n}
	\abs{\inpro{I_5}{\bff{\theta}^n}}
	&\leq
	C \norm{\bff{u}_h^n}{\bb{L}^\infty} \norm{\bff{u}_h^n}{\bb{L}^6}^2 \norm{\bff{\theta}^n + \bff{\rho}^n}{\bb{L}^2} \norm{\bff{\theta}^n}{\bb{L}^6}
	\nonumber\\
	&\quad
	+
	C \norm{\bff{u}_h^n}{\bb{L}^6} \norm{\bff{u}_h^n+\bff{u}^n}{\bb{L}^6} \norm{\bff{u}^n}{\bb{L}^\infty} \norm{\bff{\theta}^n+\bff{\rho}^n}{\bb{L}^2} \norm{\bff{\theta}^n}{\bb{L}^6}
	\nonumber \\
	&\leq
	C \norm{\bff{u}_h^n}{\bb{L}^\infty}^2 h^{2(r+1)} 
    + 
    C \norm{\bff{\theta}^n}{\bb{L}^2}^2
    +
    C \norm{\bff{u}_h^n}{\bb{L}^\infty}^2 \norm{\bff{\theta}^n}{\bb{L}^2}^2 
    + \epsilon \norm{\bff{\theta}^n}{\bb{H}^1}^2
    \nonumber \\
    &\leq
    C \norm{\bff{u}_h^n}{\bb{L}^\infty} h^{2(r+1)}
    +
    C \norm{\bff{\theta}^n}{\bb{L}^2}^2
    +
    C \norm{\bff{\theta}^n + R_h \bff{u}^n}{\bb{L}^\infty}^2 \norm{\bff{\theta}^n}{\bb{L}^2}^2 
    +
    \epsilon \norm{\bff{\theta}^n}{\bb{H}^1}^2
     \nonumber \\
    &\leq
    C \norm{\bff{u}_h^n}{\bb{L}^\infty} h^{2(r+1)}
    +
    C \norm{\bff{\theta}^n}{\bb{L}^2}^2
    +
    C \norm{\bff{\theta}^n}{\bb{L}^\infty}^2
    +
    \epsilon \norm{\bff{\theta}^n}{\bb{H}^1}^2,
    \nonumber \\
    &\leq
    C \norm{\bff{u}_h^n}{\bb{L}^\infty} h^{2(r+1)}
    +
    C \norm{\bff{\theta}^n}{\bb{L}^2}^2
    +
    \epsilon \norm{\bff{\theta}^n}{\bb{H}^1}^2
    +
    \epsilon \norm{\Delta_h \bff{\theta}^n}{\bb{L}^2}^2,
\end{align}
where in the penultimate step we used the fact that $\norm{\bff{\theta}^n}{\bb{L}^2} \leq C$ and~\eqref{equ:Ritz stab u infty}, while in the last step we used~\eqref{equ:disc lapl L infty} and Young's inequality.
Altogether, we obtain~\eqref{equ:inpro uhn theta n} from the above estimates.

In a similar manner, we obtain the following bounds:
\begin{align*}
	\abs{\inpro{I_1}{\Delta_h \bff{\theta}^n}}
	&=
	0,
	\\
	\abs{\inpro{I_2}{\Delta_h \bff{\theta}^n}}
	&\leq
	\norm{\bff{u}_h^n}{\bb{L}^\infty} \norm{(P_h-I)\Delta \bff{u}^n}{\bb{L}^2} \norm{\Delta_h \bff{\theta}^n}{\bb{L}^2}
	\leq
	C \norm{\bff{u}_h^n}{\bb{L}^\infty}^2 h^{2(r+1)} 
	+
	\epsilon \norm{\Delta_h \bff{\theta}^n}{\bb{L}^2}^2,
	\\
	\abs{\inpro{I_3}{\Delta_h \bff{\theta}^n}}
	&\leq
	\norm{\bff{u}_h^n}{\bb{L}^4} \norm{\bff{\theta}^{n-1}+\bff{\rho}^{n-1}}{\bb{L}^4} \norm{\Delta_h \bff{\theta}^n}{\bb{L}^2}
	\\
	&\leq
	Ch^{2(r+1)}
	+
	C \norm{\bff{\theta}^{n-1}}{\bb{L}^2}^2
	+
	\epsilon \norm{\nabla \bff{\theta}^{n-1}}{\bb{L}^2}^2
	+
	\epsilon \norm{\Delta_h \bff{\theta}^n}{\bb{L}^2}^2,
	\\
	\abs{\inpro{I_4}{\Delta_h \bff{\theta}^n}}
	&\leq
	\mu \norm{\bff{u}_h^n}{\bb{L}^4} \norm{\mathrm{d}_t \bff{u}^n}{\bb{L}^4} \norm{\Delta_h \bff{\theta}^n}{\bb{L}^2}
	\leq 
	Ck^2 + \epsilon \norm{\Delta_h \bff{\theta}^n}{\bb{L}^2}^2,
	\\
	\abs{\inpro{I_6}{\Delta_h \bff{\theta}^n}}
	&\leq
	C \norm{\bff{u}_h^n}{\bb{L}^4} \norm{(P_h-I)\left(\abs{\bff{u}^n}^2 \bff{u}^n \right)}{\bb{L}^4} \norm{\Delta_h \bff{\theta}^n}{\bb{L}^2}
    \leq
	C k^2 
	+
	\epsilon \norm{\Delta_h \bff{\theta}^n}{\bb{L}^2}^2,
	\\
	\abs{\inpro{I_7}{\Delta_h \bff{\theta}^n}}
	&\leq
	\lambda \norm{\bff{u}_h^n}{\bb{L}^4} \norm{\bff{\theta}^n+\bff{\rho}^n}{\bb{L}^4} \norm{\Delta_h \bff{\theta}^n}{\bb{L}^2}
	\\
	&\leq
	Ch^{2(r+1)} + \norm{\bff{\theta}^n}{\bb{L}^2} +
	\epsilon \norm{\nabla \bff{\theta}^n}{\bb{L}^2}^2
	+
	\epsilon \norm{\Delta_h \bff{\theta}^n}{\bb{L}^2}^2,
	\\
	\abs{\inpro{I_8}{\Delta_h \bff{\theta}^n}}
	&\leq
	\lambda k \norm{\bff{u}_h^n}{\bb{L}^4} \norm{\mathrm{d}_t \bff{u}^n}{\bb{L}^4} \norm{\Delta_h \bff{\theta}^n}{\bb{L}^2}
	\leq
	Ck^2 + \epsilon \norm{\Delta_h \bff{\theta}^n}{\bb{L}^2}^2,
	\\
	\abs{\inpro{I_9}{\Delta_h \bff{\theta}^n}}
	&\leq
	\norm{\bff{\theta}^n+\bff{\rho}^n}{\bb{L}^2} \norm{\bff{H}^n}{\bb{L}^\infty} \norm{\Delta_h \bff{\theta}^n}{\bb{L}^2}
	\leq
	Ch^{2(r+1)} + C\norm{\bff{\theta}^n}{\bb{L}^2}^2 
	+
	\epsilon \norm{\Delta_h \bff{\theta}^n}{\bb{L}^2}^2.
\end{align*}
Finally, for the term $\inpro{I_5}{\Delta_h \bff{\theta}^n}$, we apply similar argument as in~\eqref{equ:I5 theta n} to obtain
\begin{align}\label{equ:I5 Delta theta n}
	\abs{\inpro{I_5}{\Delta_h \bff{\theta}^n}}
	&\leq
	C \norm{\bff{u}_h^n}{\bb{L}^\infty} \norm{\bff{u}_h^n}{\bb{L}^6}^2 \norm{\bff{\theta}^n + \bff{\rho}^n}{\bb{L}^6} \norm{\Delta_h \bff{\theta}^n}{\bb{L}^2}
	\nonumber\\
	&\quad
	+
	C \norm{\bff{u}_h^n}{\bb{L}^6} \norm{\bff{u}_h^n+\bff{u}^n}{\bb{L}^6} \norm{\bff{u}^n}{\bb{L}^\infty} \norm{\bff{\theta}^n+\bff{\rho}^n}{\bb{L}^6} \norm{\Delta_h \bff{\theta}^n}{\bb{L}^2}
	\nonumber \\
	&\leq
	C \norm{\bff{u}_h^n}{\bb{L}^\infty}^2 h^{2(r+1)} 
    + 
    C \norm{\bff{\theta}^n}{\bb{H}^1}^2 
    + \epsilon \norm{\Delta_h \bff{\theta}^n}{\bb{L}^2}^2.
\end{align}
Altogether, we obtain the inequality~\eqref{equ:inpro uhn Delta theta n}. This completes the proof of the lemma.
\end{proof}

\begin{remark}\label{rem:1d 2d}
If $d=1$ or $d=2$, then we have the Sobolev embedding $\bb{H}^1\hookrightarrow \bb{L}^8$. In these cases, one could remove the term $\norm{\bff{u}_h^n}{\bb{L}^\infty}^2$ in \eqref{equ:inpro uhn theta n} and \eqref{equ:inpro uhn Delta theta n} by estimating  $\abs{\inpro{I_5}{\bff{\theta}^n}}$ in \eqref{equ:I5 theta n} and $\abs{\inpro{I_5}{\Delta_h \bff{\theta}^n}}$ in \eqref{equ:I5 Delta theta n} differently. Indeed, we have
\begin{align*}
	\nonumber
	\abs{\inpro{I_5}{\Delta_h \bff{\theta}^n}}
	&\leq
	\norm{\bff{u}_h^n}{\bb{L}^8}
	\left( \norm{\bff{u}_h^n}{\bb{L}^8}^2 \norm{\bff{\theta}^n+\bff{\rho}^n}{\bb{L}^8}
	+
	\norm{\bff{\theta}^n}{\bb{L}^8}
	\norm{\bff{u}_h^n+\bff{u}^{n}}{\bb{L}^8}
	\norm{\bff{u}^{n}}{\bb{L}^8} \right)
	\norm{\Delta_h \bff{\theta}^{n}}{\bb{L}^2}
	\\
	&\leq
	C h^{2(r+1)}
	+
	C \norm{\bff{\theta}^n}{\bb{H}^1}^2
	+
	\epsilon \norm{\Delta_h \bff{\theta}^n}{\bb{L}^2}^2,
\end{align*}
where in the last step we also used~\eqref{equ:Ritz ineq}. Similar bound also holds for $\abs{\inpro{I_5}{\bff{\theta}^n}}$. Furthermore, if we assume more regularity on the exact solution $\bff{u}$, say $\bff{u}\in L^\infty(\bb{W}^{4,4})$, then
\begin{align*}
	\abs{\inpro{I_2}{\Delta_h \bff{\theta}^n}}
	&\leq 
	\norm{\bff{u}_h^n}{\bb{L}^4}
	\norm{(P_h-I) \Delta \bff{u}^n}{\bb{L}^4}
	\norm{\Delta_h\bff{\theta}^{n}}{\bb{L}^2}
	\leq
	Ch^{2(r+1)} \norm{\Delta \bff{u}^n}{\bb{W}^{2,4}} + \epsilon \norm{\Delta_h \bff{\theta}^{n}}{\bb{L}^2}^2.
\end{align*}
These stronger estimates allow us to avoid using \eqref{equ:Ritz stab u infty}, \eqref{equ:disc lapl L infty}, and~\eqref{equ:stab L infty}. Thus, we could remove the term $\norm{\bff{u}_h^n}{\bb{L}^\infty}^2$ in the estimates \eqref{equ:inpro uhn theta n} and \eqref{equ:inpro uhn Delta theta n} if $d=1$ or $2$, even without necessarily assuming global quasi-uniformity of the triangulation.
\end{remark}


\begin{lemma}
Let $\epsilon>0$ be given. Let $\bff{w}_h^n$ and $\bff{w}^n$ be defined by~\eqref{equ:whn} and~\eqref{equ:w}, respectively.
Then for $n\in \{1,2,\ldots, \lfloor T/k \rfloor\}$,
\begin{equation}\label{equ:wh mu pos}
	\norm{\bff{w}_h^n-\bff{w}^n}{\bb{L}^2}^2
	\leq
	Ch^{2(r+1)} + Ck^2 
	+ 
	C \norm{\bff{\theta}^{n-1}}{\bb{L}^2}^2
	+
	C \norm{\bff{\theta}^n}{\bb{H}^1}^2,
\end{equation} 
where $C$ depends on the coefficients of the equation, $K_0$, $T$, and $\mathscr{D}$ (but is independent of $n$, $h$ or $k$).
\end{lemma}

\begin{proof}
Note that
\begin{align*}
	\bff{w}_h^{n}-\bff{w}^{n}
	&=
	\mu (\bff{\theta}^{n-1}+ \bff{\rho}^{n-1})
	+
	\mu (\bff{u}^{n-1}-\bff{u}^{n})
	-
	P_h \left(|\bff{u}_h^{n}|^2 \bff{u}_h^{n}-|\bff{u}^{n}|^2 \bff{u}^{n}\right) 
	-
	(P_h-I) \left(|\bff{u}^{n}|^2 \bff{u}^{n}\right)
	\nonumber \\
	&\quad
	+
	\lambda\bff{e} \left(\bff{e}\cdot (\bff{\theta}^{n-1}+\bff{\rho}^{n-1})\right)
	+
	\lambda\bff{e} \left(\bff{e}\cdot (\bff{u}^{n-1}-\bff{u}^n)\right),
\end{align*}
where $I$ is the identity operator.
Noting \eqref{equ:ass 1}, \eqref{equ:proj approx}, \eqref{equ:Ritz ineq}, the stability estimate \eqref{equ:stab H1}, and~\eqref{equ:split cubic}, we obtain
\begin{align*}
	\nonumber
	\norm{\bff{w}_h^{n}-\bff{w}^{n}}{\bb{L}^2}^2
	&\leq
	\mu \norm{\bff{\theta}^{n-1} + \bff{\rho}^{n-1}}{\bb{L}^2}^2
	+
	Ck^2
	+
	C \norm{\bff{u}_h^{n}}{\bb{L}^6}^4
	\norm{\bff{\theta}^{n}+\bff{\rho}^{n}}{\bb{L}^6}^2
	\\
	\nonumber
	&\quad
	+
	C
	\norm{\bff{u}_h^{n}+\bff{u}^{n}}{\bb{L}^6}^2
	\norm{\bff{u}^{n}}{\bb{L}^6}^2
	\norm{\bff{\theta}^{n}+\bff{\rho}^{n}}{\bb{L}^6}^2
	+
	Ch^{2(r+1)} \norm{\bff{u}^{n}}{\bb{H}^2}^6
	+
	C\lambda \norm{\bff{\theta}^{n-1}+\bff{\rho}^{n-1}}{\bb{L}^2}^2
	\\
	&\leq	
	Ch^{2(r+1)} + Ck^2
	+
	C \norm{\bff{\theta}^{n-1}}{\bb{L}^2}^2
	+
	C \norm{\bff{\theta}^n}{\bb{H}^1}^2.
\end{align*}
This proves \eqref{equ:wh mu pos}.
\end{proof}




%
%Next, we prove \eqref{equ:dot Delta theta}.
%By similar argument as before,
%\begin{align*}
%	&\left| \inpro{\left(\frac{\bff{u}_h^{n+1} \cdot \bff{w}_h^{n+1}}{\abs{\bff{u}_h^{n+1}}^2}\right) \bff{u}_h^{n+1}- \left(\frac{\bff{u}^{n+1} \cdot \bff{w}^{n+1}}{\abs{\bff{u}^{n+1}}^2}\right) \bff{u}^{n+1}}{\Delta_h \bff{\theta}^{n+1}} \right| 
%	\\
%	&\leq 
%	\norm{\widehat{\bff{u}}_h^{n+1}}{\bb{L}^\infty}^2
%	\norm{\bff{w}_h^{n+1}-\bff{w}^{n+1}}{\bb{L}^2}
%	\norm{\bff{\theta}^{n+1}}{\bb{L}^2}
%	\\
%	&\quad
%	+
%	\norm{\widehat{\bff{u}}_h^{n+1}-\widehat{\bff{u}}^{n+1}}{\bb{L}^2}
%	\norm{\bff{w}^{n+1}}{\bb{L}^\infty}
%	\left(\norm{\widehat{\bff{u}}_h^{n+1}}{\bb{L}^\infty} + \norm{\widehat{\bff{u}}^{n+1}}{\bb{L}^\infty}\right)
%	\norm{\bff{\theta}^{n+1}}{\bb{L}^2}
%	\\
%	&\leq 
%	Ch^{2(r+1)} 
%	+ 
%	Ck^2 
%	+ 
%	C\norm{\bff{\theta}^n}{\bb{L}^2}^2
%	+ 
%	C \norm{\bff{\theta}^{n+1}}{\bb{H}^1}^2
%	+
%	C\delta^{-2} \norm{\bff{\theta}^{n+1}}{\bb{L}^2}^2
%	+
%	C\delta^{-2}h^{2(r+1)}
%	+
%	\epsilon\norm{\Delta_h \bff{\theta}^{n+1}}{\bb{L}^2}^2,
%\end{align*}
%where in the last step we used \eqref{equ:wh wn mu pos}, \eqref{equ:wh mu pos}, Young's inequality, and Lemma~\ref{lem:norm}. This proves \eqref{equ:dot Delta theta}, thus completing the proof of the lemma.





%\begin{align*}
%	&\inpro{\left(\frac{\bff{u}_h^{n+1} \cdot \bff{H}_h^{n+1}}{\abs{\bff{u}_h^{n+1}}^2}\right) \bff{u}_h^{n+1}- \left(\frac{\bff{u}^{n+1} \cdot \bff{H}^{n+1}}{\abs{\bff{u}^{n+1}}^2}\right) \bff{u}^{n+1}}{\Delta_h \bff{\theta}^{n+1}}
%	\\
%	&=
%	\left\langle\big(\widehat{\bff{u}}_h^{n+1} \cdot (\Delta_h \bff{\theta}^{n+1} + (P_h-I)\Delta \bff{u}^{n+1} + \bff{w}_h^{n+1}-\bff{w}^{n+1}) \big) \widehat{\bff{u}}_h^{n+1} \right.
%	\\
%	&\qquad
%	+
%	\left.\big( (\widehat{\bff{u}}_h^{n+1}-\widehat{\bff{u}}^{n+1}) \cdot \bff{H}^{n+1}\big) \widehat{\bff{u}}_h^{n+1}
%	+
%	\big(\widehat{\bff{u}}^{n+1} \cdot \bff{H}^{n+1}\big) (\widehat{\bff{u}}_h^{n+1}-\widehat{\bff{u}}^{n+1}), \Delta_h \bff{\theta}^{n+1} \right\rangle
%\end{align*}

We are now ready to prove an error estimate for the numerical scheme. The following proposition shows a superconvergence estimate for $\bff{\theta}^n$.

\begin{proposition}\label{pro:est theta no cur}
	Let $\bff{\theta}^n$ be as defined in \eqref{equ:theta rho split 2}. Then for $n\in \{0,1,\ldots,\lfloor T/k \rfloor\}$,
	\begin{align}\label{equ:superconv theta}
		\norm{\bff{\theta}^n}{\bb{H}^1}^2
		+
		k \sum_{m=1}^n \norm{\Delta_h \bff{\theta}^m}{\bb{L}^2}^2
		+
		k \sum_{m=1}^n \norm{\bff{\theta}^n}{\bb{L}^\infty}^2
		&\leq
		C (h^{2(r+1)}+k^2).
	\end{align}
	In particular, if $\mathscr{D}\subset \bb{R}^2$ and the triangulation is globally quasi-uniform, then
	\begin{equation}\label{equ:theta L infty}
		\norm{\bff{\theta}^m}{\bb{L}^\infty}^2
		\leq
		C\big(h^{2(r+1)} +k^2\big) \abs{\ln h},
	\end{equation}
	where $C$ depends on the coefficients of the equation, $K_0$, $T$, and $\mathscr{D}$ (but is independent of $n$, $h$ or $k$).
\end{proposition}

\begin{proof}
Subtracting~\eqref{equ:weak form no spin} from~\eqref{equ:scheme 2} and noting~\eqref{equ:Delta uhn Delta un}, we have
\begin{align*}
	&\inpro{\mathrm{d}_t \bff{\theta}^n+ \mathrm{d}_t \bff{\rho}^n+ \mathrm{d}_t \bff{u}^n- \partial_t \bff{u}^n}{\bff{\chi}}
	\\
	&=
	-\gamma \inpro{\bff{u}_h^n\times \bff{H}_h^n- \bff{u}^n\times \bff{H}^n}{\bff{\chi}}
	+
	\alpha \inpro{\Delta_h \bff{\theta}^n + (P_h-I) \Delta \bff{u}^n}{\bff{\chi}}
	+
	\alpha \inpro{\bff{w}_h^n-\bff{w}^n}{\bff{\chi}}.
\end{align*}
Now, we put $\bff{\chi}=\bff{\theta}^n- \Delta_h \bff{\theta}^n$ and rearrange the terms (noting the identity~\eqref{equ:aab}) to obtain
\begin{align}\label{equ:ineq theta H1}
	&\frac{1}{2k} \left(\norm{\bff{\theta}^n}{\bb{H}^1}^2 - \norm{\bff{\theta}^{n-1}}{\bb{H}^1}^2 \right)
	+
	\frac{1}{2k} \norm{\bff{\theta}^n-\bff{\theta}^{n-1}}{\bb{H}^1}^2
	+
	\alpha \norm{\nabla \bff{\theta}^n}{\bb{L}^2}^2
	+
	\alpha \norm{\Delta_h \bff{\theta}^n}{\bb{L}^2}^2
	\nonumber\\
	&=
	-\gamma \inpro{\bff{u}_h^n\times \bff{H}_h^n- \bff{u}^n\times \bff{H}^n}{\bff{\theta}^n}
	+
	\gamma \inpro{\bff{u}_h^n\times \bff{H}_h^n- \bff{u}^n\times \bff{H}^n}{\Delta_h \bff{\theta}^n}
	\nonumber\\
	&\quad
	+
	\alpha \inpro{(P_h-I)\Delta \bff{u}^n}{\bff{\theta}^n- \Delta_h \bff{\theta}^n}
	+
	\alpha \inpro{\bff{w}_h^n-\bff{w}^n}{\bff{\theta}^n- \Delta_h \bff{\theta}^n}
	\nonumber\\
	&=: J_1+J_2+J_3+J_4.
\end{align}
It remains to estimate each term on the last line. For the first two terms, we apply Lemma~\ref{lem:inpro theta n}. For the last term, we use~\eqref{equ:wh mu pos} and Young's inequality. For the term $J_3$, by Young's inequality and~\eqref{equ:proj approx}, we have
\begin{align*}
	\abs{J_3}
	\leq
	Ch^{2(r+1)} 
	+
	\frac{\alpha}{4} \norm{\bff{\theta}^n}{\bb{L}^2}^2
	+
	\frac{\alpha}{4} \norm{\Delta_h \bff{\theta}^n}{\bb{L}^2}^2.
\end{align*}
Altogether, from \eqref{equ:ineq theta H1}, after rearranging the terms we obtain
\begin{align*}
	&\frac{1}{2k} \left(\norm{\bff{\theta}^n}{\bb{H}^1}^2 - \norm{\bff{\theta}^{n-1}}{\bb{H}^1}^2 \right)
	+
	\frac{1}{2k} \norm{\bff{\theta}^n-\bff{\theta}^{n-1}}{\bb{H}^1}^2
	+
	\frac{\alpha}{2} \norm{\nabla \bff{\theta}^n}{\bb{L}^2}^2
	+
	\frac{\alpha}{2} \norm{\Delta_h \bff{\theta}^n}{\bb{L}^2}^2
	\\
	&\leq
	C \left(1+ \norm{\bff{u}_h^n}{\bb{L}^\infty}^2\right) h^{2(r+1)} + Ck^2 
	+ 
	C\norm{\bff{\theta}^n}{\bb{H}^1}^2
	+
	C\norm{\bff{\theta}^{n-1}}{\bb{L}^2}^2,
\end{align*}
from which an estimate for the first two terms in~\eqref{equ:superconv theta} follows by the discrete Gronwall lemma (noting~\eqref{equ:stab L infty}). For the last term in~\eqref{equ:superconv theta}, the estimate follows by applying~\eqref{equ:disc lapl L infty}.

Finally, inequality~\eqref{equ:theta L infty} follows from~\eqref{equ:superconv theta} and the discrete Sobolev inequality~\cite{Bre04}.
\end{proof}

\begin{theorem}\label{the:without spin error}
	Let $\bff{u}_h^n$ and $\bff{u}$ be the solution of \eqref{equ:scheme 2} and \eqref{equ:weak form no spin}, respectively. For $n\in \{0,1,\ldots,\lfloor T/k \rfloor\}$, and $s=0$ or $1$,
	\begin{align*}
		\norm{\bff{u}_h^n-\bff{u}(t_n)}{\bb{H}^s}
		&\leq 
		C(h^{r+1-s} +k),
		\\
		k \sum_{m=1}^n \norm{\bff{u}_h^n-\bff{u}(t_n)}{\bb{L}^\infty}^2 
		&\leq
		C(h^{2(r+1)} +k^2).
	\end{align*}
	If $\mathscr{D}\subset \bb{R}^d$, where $d=1,2$, and the triangulation is globally quasi-uniform, then
	\begin{align*}
		\norm{\bff{u}_h^n-\bff{u}(t_n)}{\bb{L}^\infty} 
		\leq
		C\big(h^{r+1} +k \big) \abs{\ln h}^{\frac12},
	\end{align*}
	where $C$ depends on the coefficients of the equation, $K_0$, $T$, and $\mathscr{D}$ (but is independent of $n$, $h$ or $k$).
\end{theorem}

\begin{proof}
	This follows from Proposition \ref{pro:est theta no cur} and the triangle inequality (noting \eqref{equ:theta rho split 2}, \eqref{equ:Ritz ineq}, and \eqref{equ:Ritz ineq L infty}).
\end{proof}




\section{Numerical Experiments}

Numerical simulations for the schemes~\eqref{equ:scheme spin} and \eqref{equ:scheme 2} are performed using the open-source package~\textsc{FEniCS}~\cite{AlnaesEtal15}. Since the exact solution of the equation is not known, we use extrapolation to verify the spatial order of convergence experimentally. To this end, let $\bff{u}_h^n$ be the finite element solution with spatial step size $h$ and time-step size $k=\lfloor T/n\rfloor$. For $s=0$ or $1$, define the extrapolated order of convergence
\begin{equation*}
	\text{rate}_s :=  \log_2 \left[\frac{\max_n \norm{\bff{e}_{2h}}{\bb{H}^s}}{\max_n \norm{\bff{e}_{h}}{\bb{H}^s}}\right],
\end{equation*}
where $\bff{e}_h := \bff{u}_{h}^n-\bff{u}_{h/2}^n$. We expect that for both schemes, when $k$ is sufficiently small,~$\text{rate}_s \approx h^{r+1-s}$. In these simulations, we take the domain $\mathscr{D}= [-1,1]^2\subset \bb{R}^2$.

\subsection{Simulation 1}
We consider the scheme~\eqref{equ:scheme spin} and take $k=1.0\times 10^{-6}$. The coefficients in~\eqref{equ:llb a} are~$\gamma=2.2\times 10^5, \alpha=1.0\times 10^5, \beta_1=1.0, \beta_2=-0.1, \sigma=1.3\times 10^{-6}, \kappa=1.0, \mu=1.0\times 10^{-6}$, and $\lambda=0$, which are of typical order of magnitude for a micromagnetic simulation. The current density is $\bff{\nu}=(200,0,0)^\top$. The initial data $\bff{u}_0$ is given by
\begin{equation*}
	\bff{u}_0(x,y)= \big(-y, x, 0.5 \big).
\end{equation*}
Snapshots of the magnetic spin field $\bff{u}$ at selected times are shown in Figure~\ref{fig:snapshots field 2d}, which shows a vortex state being carried by the current. The colours indicate the relative magnitude of the value of the $z$-component. Plots of $\bff{e}_h$ against $1/h$ are shown in Figure~\ref{fig:order u 1}.

\begin{figure}[!htb]
	\centering
	\begin{subfigure}[b]{0.3\textwidth}
		\centering
		\includegraphics[width=\textwidth]{u0a.png}
		\caption{$t=0$}
	\end{subfigure}
	\begin{subfigure}[b]{0.3\textwidth}
		\centering
		\includegraphics[width=\textwidth]{u1a.png}
		\caption{$t=5\times 10^{-5}$}
	\end{subfigure}
	\begin{subfigure}[b]{0.3\textwidth}
		\centering
		\includegraphics[width=\textwidth]{u2a.png}
		\caption{$t=1 \times 10^{-3}$}
	\end{subfigure}
	\begin{subfigure}[b]{0.3\textwidth}
		\centering
		\includegraphics[width=\textwidth]{u3a.png}
		\caption{$t=3\times 10^{-3}$}
	\end{subfigure}
	\begin{subfigure}[b]{0.3\textwidth}
		\centering
		\includegraphics[width=\textwidth]{u4a.png}
		\caption{$t=6\times 10^{-3}$}
	\end{subfigure}
	\begin{subfigure}[b]{0.3\textwidth}
		\centering
		\includegraphics[width=\textwidth]{u5a.png}
		\caption{$t=1\times 10^{-2}$}
	\end{subfigure}
	\caption{Snapshots of the spin field $\bff{u}$ (projected onto $\bb{R}^2$) for simulation 1.}
	\label{fig:snapshots field 2d}
\end{figure}


\begin{figure}[!htb]
		\begin{tikzpicture}
			\begin{axis}[
				title=Plot of $\bff{e}_h(\bff{u})$ against $1/h$,
				height=0.5\textwidth,
				width=0.5\textwidth,
				xlabel= $1/h$,
				ylabel= $\bff{e}_h$,
				xmode=log,
				ymode=log,
				legend pos=south west,
				legend cell align=left,
				]
				\addplot+[mark=*,blue] coordinates {(4,0.225)(8,0.106)(16,0.0518)(32,0.0258)(64,0.0136)};
				\addplot+[mark=*,red] coordinates {(4,0.0184)(8,0.00352)(16,0.000816)(32,0.000201)(64,0.00006)};
				\addplot+[dashed,no marks,blue,domain=20:65]{1.5/x};
				\addplot+[dashed,no marks,red,domain=20:65]{0.45/x^2};
				\legend{\small{$\max_n \norm{\bff{e}_h}{\bb{H}_0^1}$}, \small{$\max_n \norm{\bff{e}_h}{\bb{L}^2}$}, \small{order 1 line}, \small{order 2 line}}
			\end{axis}
		\end{tikzpicture}
		\caption{Error order of $\bff{u}$ for simulation 1.}
		\label{fig:order u 1}
\end{figure}


\subsection{Simulation 2}
We consider the scheme~\eqref{equ:scheme 2} with positive $\mu$ and take $k=1.0\times 10^{-6}$. The coefficients in~\eqref{equ:llb a} are $\gamma=2.2\times 10^5, \alpha=1.0\times 10^5, \sigma=1.0\times 10^{-6}, \kappa=1.0, \mu=1.3\times 10^{-6}$, and $\lambda=0.05$. The anisotropy axis vector $\bff{e}=(0,1,0)^\top$. The initial data $\bff{u}_0$ is given by
\begin{equation*}
	\bff{u}_0(x,y)= \big(-y, x, 0.5 \big).
\end{equation*}
Snapshots of the magnetic spin field $\bff{u}$ at selected times are shown in Figure~\ref{fig:snapshots field 2d 2}, which shows the formation of magnetic domains. The colours indicate the relative magnitude of the value of the $z$-component. Plots of $\bff{e}_h$ against $1/h$ are shown in Figure~\ref{fig:order u 2}. The graphs of energy vs time which show energy dissipativity are displayed in Figure~\ref{fig:energy}.


\begin{figure}[!htb]
	\centering
	\begin{subfigure}[b]{0.3\textwidth}
		\centering
		\includegraphics[width=\textwidth]{u0b.png}
		\caption{$t=0$}
	\end{subfigure}
	\begin{subfigure}[b]{0.3\textwidth}
		\centering
		\includegraphics[width=\textwidth]{u1b.png}
		\caption{$t=2\times 10^{-4}$}
	\end{subfigure}
	\begin{subfigure}[b]{0.3\textwidth}
		\centering
		\includegraphics[width=\textwidth]{u2b.png}
		\caption{$t=5 \times 10^{-4}$}
	\end{subfigure}
	\begin{subfigure}[b]{0.3\textwidth}
		\centering
		\includegraphics[width=\textwidth]{u3b.png}
		\caption{$t=1\times 10^{-3}$}
	\end{subfigure}
	\begin{subfigure}[b]{0.3\textwidth}
		\centering
		\includegraphics[width=\textwidth]{u4b.png}
		\caption{$t=5\times 10^{-3}$}
	\end{subfigure}
	\begin{subfigure}[b]{0.3\textwidth}
		\centering
		\includegraphics[width=\textwidth]{u5b.png}
		\caption{$t=1\times 10^{-2}$}
	\end{subfigure}
	\caption{Snapshots of the spin field $\bff{u}$ (projected onto $\bb{R}^2$) for simulation 2.}
	\label{fig:snapshots field 2d 2}
\end{figure}


\begin{figure}[!htb]
	\begin{tikzpicture}
		\begin{axis}[
			title=Plot of $\bff{e}_h(\bff{u})$ against $1/h$,
			height=0.5\textwidth,
			width=0.5\textwidth,
			xlabel= $1/h$,
			ylabel= $\bff{e}_h$,
			xmode=log,
			ymode=log,
			legend pos=south west,
			legend cell align=left,
			]
			\addplot+[mark=*,blue] coordinates {(4,0.252)(8,0.12)(16,0.056)(32,0.028)(64,0.015)};
			\addplot+[mark=*,green] coordinates {(4,0.0196)(8,0.009)(16,0.0027)(32,0.00066)(64,0.00016)};
			\addplot+[mark=*,red] coordinates {(4,0.028)(8,0.0054)(16,0.00097)(32,0.00022)(64,0.00006)};
			\addplot+[dashed,no marks,blue,domain=20:65]{1.5/x};
			\addplot+[dashed,no marks,red,domain=20:65]{0.4/x^2};
			\legend{\small{$\max_n \norm{\bff{e}_h}{\bb{H}_0^1}$}, \small{$\max_n \norm{\bff{e}_h}{\bb{L}^\infty}$}, \small{$\max_n \norm{\bff{e}_h}{\bb{L}^2}$}, \small{order 1 line}, \small{order 2 line}}
		\end{axis}
	\end{tikzpicture}
	\caption{Error order of $\bff{u}$ for simulation 2.}
	\label{fig:order u 2}
\end{figure}


\begin{figure}[!htb]
	\centering
	\begin{subfigure}[b]{0.45\textwidth}
		\centering
		\includegraphics[width=\textwidth]{energy.png}
		\caption{Fixed $k$, varying $h$}
	\end{subfigure}
	\begin{subfigure}[b]{0.45\textwidth}
		\centering
		\includegraphics[width=\textwidth]{energy2.png}
		\caption{Fixed $h$, varying $k$}
	\end{subfigure}
	\caption{Graph of energy vs time for simulation 2}
	\label{fig:energy}
\end{figure}



\subsection{Simulation 3}
We consider the scheme~\eqref{equ:scheme 2} with negative $\mu$ and take $k=1.0\times 10^{-6}$. The coefficients in~\eqref{equ:llb a} are $\gamma=2.2\times 10^5, \alpha=1.0\times 10^5, \sigma=1.0\times 10^{-6}, \kappa=1.0, \mu=-0.1$, and $\lambda=0.05$. The anisotropy axis vector $\bff{e}=(0,1,0)^\top$. The initial data $\bff{u}_0$ is given by
\begin{equation*}
	\bff{u}_0(x,y)= \big(\sin(2\pi y), \sin(2\pi x), 0.5 \big).
\end{equation*}
Snapshots of the magnetic spin field $\bff{u}$ at selected times are shown in Figure~\ref{fig:snapshots field 2d 3}, which shows the formation of magnetic domains. The colours indicate the relative magnitude of the value of the $z$-component. Plots of $\bff{e}_h$ against $1/h$ are shown in Figure~\ref{fig:order u 3}.


\begin{figure}[!htb]
	\centering
	\begin{subfigure}[b]{0.3\textwidth}
		\centering
		\includegraphics[width=\textwidth]{u3_0.png}
		\caption{$t=0$}
	\end{subfigure}
	\begin{subfigure}[b]{0.3\textwidth}
		\centering
		\includegraphics[width=\textwidth]{u3_1.png}
		\caption{$t=3\times 10^{-4}$}
	\end{subfigure}
	\begin{subfigure}[b]{0.3\textwidth}
		\centering
		\includegraphics[width=\textwidth]{u3_2.png}
		\caption{$t=5 \times 10^{-4}$}
	\end{subfigure}
	\begin{subfigure}[b]{0.3\textwidth}
		\centering
		\includegraphics[width=\textwidth]{u3_3.png}
		\caption{$t=1\times 10^{-3}$}
	\end{subfigure}
	\begin{subfigure}[b]{0.3\textwidth}
		\centering
		\includegraphics[width=\textwidth]{u3_4.png}
		\caption{$t=2\times 10^{-3}$}
	\end{subfigure}
	\begin{subfigure}[b]{0.3\textwidth}
		\centering
		\includegraphics[width=\textwidth]{u3_5.png}
		\caption{$t=1\times 10^{-2}$}
	\end{subfigure}
	\caption{Snapshots of the spin field $\bff{u}$ (projected onto $\bb{R}^2$) for simulation 3.}
	\label{fig:snapshots field 2d 3}
\end{figure}


\begin{figure}[!htb]
	\begin{tikzpicture}
		\begin{axis}[
			title=Plot of $\bff{e}_h(\bff{u})$ against $1/h$,
			height=0.5\textwidth,
			width=0.5\textwidth,
			xlabel= $1/h$,
			ylabel= $\bff{e}_h$,
			xmode=log,
			ymode=log,
			legend pos=south west,
			legend cell align=left,
			]
			\addplot+[mark=*,blue] coordinates {(4,5.02)(8,2.95)(16,1.47)(32,0.71)(64,0.35)};
			\addplot+[mark=*,green] coordinates {(4,0.080)(8,0.055)(16,0.04)(32,0.01)(64,0.003)};
			\addplot+[mark=*,red] coordinates {(4,0.8)(8,0.16)(16,0.04)(32,0.0083)(64,0.002)};
			\addplot+[dashed,no marks,blue,domain=20:65]{35/x};
			\addplot+[dashed,no marks,red,domain=20:65]{5/x^2};
			\legend{\small{$\max_n \norm{\bff{e}_h}{\bb{H}_0^1}$}, \small{$\max_n \norm{\bff{e}_h}{\bb{L}^\infty}$}, \small{$\max_n \norm{\bff{e}_h}{\bb{L}^2}$}, \small{order 1 line}, \small{order 2 line}}
		\end{axis}
	\end{tikzpicture}
	\caption{Error order of $\bff{u}$ for simulation 3.}
	\label{fig:order u 3}
\end{figure}


\section*{Acknowledgements}
The author is supported by the Australian Government through the Research Training Program (RTP) Scholarship awarded at the University of New South Wales, Sydney.

Financial support from the Australian Research Council under grant number DP200101866 (awarded to Prof. Thanh Tran) is gratefully acknowledged.

\bibliographystyle{myabbrv}
\bibliography{mybib}

\end{document}


