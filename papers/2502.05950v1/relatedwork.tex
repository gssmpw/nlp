\section{Related work}
\textbf{Concept-based learning models.} Many CBL models were proposed in
recent years \cite{kim2018interpretability,Sheth-Kahou-23,yeh2020completeness}
in order to improve the classification and regression performance of machine
learning models and to interpret their predictions in terms of the high-level
concepts. A large part of the models are CBMs \cite{koh2020concept} which have
attracted special attention due to a number of remarkable properties. In
particular, we point out stochastic CBMs \cite{vandenhirtz2024stochastic},
interactive CBMs \cite{chauhan2023interactive}, editable CBMs
\cite{hu2024editable}, semi-supervised CBMs \cite{hu2024semi}, probabilistic
CBMs \cite{kim2023probabilistic}, label-free CBMs \cite{oikarinen2023label},
CBMs without predefined concepts, \cite{Wang-Junlin-Chen-24}
\cite{schrodi2024concept}, post-hoc CBMs \cite{yuksekgonul2022post},
incremental residual CBMs \cite{shang2024incremental}, concept bottleneck
generative models \cite{ismail2023concept}, any CBMs
\cite{dominici2024anycbms}. concept complement bottleneck models
\cite{Wang-Junlin-Chen-24}. The above CBMs are a small part of various CBL
models proposed in literature.

Concept-based models, their advantages and disadvantages are considered in the
survey papers
\cite{Gupta-Narayanan-24,poeta2023concept,Aysel-etal-25,lee2023neural,mahinpei2021promises}%
.

\textbf{Survival analysis in machine learning}. Many survival machine learning
models have been developed \cite{Wang-Li-Reddy-2019} due to their importance
in several application areas, for example, in medicine, safety, reliability,
economics. Detailed reviews of many survival models can be found in
\cite{Wang-Li-Reddy-2019,salerno2023high,Wiegrebe:2024aa}. Deep survival
machine learning models were reviewed in \cite{chen2024introduction}. A
practical introduction to survival analysis was provided by Emmert-Streib and
Dehmer in \cite{EmmertStreib-Dehmer-19}.

A large part of survival models can be regarded as extensions of conventional
machine learning models under condition of censored data. For example, several
survival models are based on applying neural networks and deep learning
\cite{chen2024introduction,Katzman-etal-2018,Luck-etal-2017,Nezhad-etal-2018,ren2019deep,Steingrimsson-Morrison-20,Tarkhan-etal-21,Yao-Zhu-Zhu-Huang-2017,Zhong-Mueller-Wang-21}%
, several survival models are based on the transformer architectures
\cite{Chatha-etal-22,hu2021transformer,Li-Zhu-Yao-Huang-22,Lv-Lin-etal-22,Shen-liu-etal-22,tang2023explainable,Wang-Sun-22}%
, attention-based deep survival models were proposed in
\cite{Li-Krivtsov-Arora-22,Sun-Dong-etal-21}. A part of models is based on
extending the random forest \cite{Ibrahim-etal-2008,Wright-etal-2017}.
Convolutional neural networks also used in survival analysis
\cite{Haarburger-etal-2018}. Many machine learning survival models extend the
Cox model \cite{Cox-1972}. They mainly relax or modify the linear relationship
assumption used in the Cox model
\cite{Widodo-Yang-2011,Witten-Tibshirani-2010}.

Despite the intensive development of survival models, their application to
concept-based learning is currently not reflected in the literature.
Therefore, this work can be considered as the first attempt to create a
survival concept-based model.