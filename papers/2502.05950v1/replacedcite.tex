\section{Related work}
\textbf{Concept-based learning models.} Many CBL models were proposed in
recent years ____
in order to improve the classification and regression performance of machine
learning models and to interpret their predictions in terms of the high-level
concepts. A large part of the models are CBMs ____ which have
attracted special attention due to a number of remarkable properties. In
particular, we point out stochastic CBMs ____,
interactive CBMs ____, editable CBMs
____, semi-supervised CBMs ____, probabilistic
CBMs ____, label-free CBMs ____,
CBMs without predefined concepts, ____
____, post-hoc CBMs ____,
incremental residual CBMs ____, concept bottleneck
generative models ____, any CBMs
____. concept complement bottleneck models
____. The above CBMs are a small part of various CBL
models proposed in literature.

Concept-based models, their advantages and disadvantages are considered in the
survey papers
____%
.

\textbf{Survival analysis in machine learning}. Many survival machine learning
models have been developed ____ due to their importance
in several application areas, for example, in medicine, safety, reliability,
economics. Detailed reviews of many survival models can be found in
____. Deep survival
machine learning models were reviewed in ____. A
practical introduction to survival analysis was provided by Emmert-Streib and
Dehmer in ____.

A large part of survival models can be regarded as extensions of conventional
machine learning models under condition of censored data. For example, several
survival models are based on applying neural networks and deep learning
____%
, several survival models are based on the transformer architectures
____%
, attention-based deep survival models were proposed in
____. A part of models is based on
extending the random forest ____.
Convolutional neural networks also used in survival analysis
____. Many machine learning survival models extend the
Cox model ____. They mainly relax or modify the linear relationship
assumption used in the Cox model
____.

Despite the intensive development of survival models, their application to
concept-based learning is currently not reflected in the literature.
Therefore, this work can be considered as the first attempt to create a
survival concept-based model.