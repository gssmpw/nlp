\section{New Result for Peer Prediction}

\begin{table}[hbtp]
\centering
\begin{tabular}{@{}llll@{}}
\toprule
Truthful reporting & is an equilibrium & \multicolumn{2}{l}{is NOT an equilibrium} \\ \midrule
ex-ante & \begin{tabular}[c]{@{}l@{}}$\varepsilon = 0$\\ $\kd = \kde$\end{tabular} & \begin{tabular}[c]{@{}l@{}}$\varepsilon = 0$\\ $\kd = \kde + 1$\end{tabular} & \begin{tabular}[c]{@{}l@{}}$\varepsilon = \Theta(1)$\\ $\kd = \kde + 2$\\ Under sufficient condition\end{tabular} \\ \midrule
quasi-interim & \begin{tabular}[c]{@{}l@{}}$\varepsilon = 0$\\ $\kd = \kdq$\end{tabular} & \multicolumn{2}{l}{\begin{tabular}[c]{@{}l@{}}$\varepsilon = \Theta(1)$\\ $\kd = \kdq + 1$\end{tabular}} \\ \bottomrule
\end{tabular}
\caption{Summary of results for PP. }
\end{table}

For ex-ante equilibria, we give a necessary condition where truthful reporting is NOT an $\varepsilon$-ex-ante Bayesian $\kde+2$ strong equilibrium for some constant $\varepsilon$. \Qishen{My initial plan is to give a necessary-sufficient condition. However, such condition seems to be highly non-trivial while giving not much insight.}
Interestingly, in such deviation, the deviator with $\varepsilon$ gain is also a truthful reporter! 
This reveals that, a large collusive group will even benefits the truthful reporters!

\begin{prop}
    Let $\kde^h$, $\kde^\ell$, and $\kde$ be defined as in Theorem~\ref{thm:pp_exante}. If either of the following to conditions hold: (1)$\kde = \kde^h$ and $\pr(h\mid \ell) \cdot (\ps(h, \vpr_h) - \ps(\ell, \vpr_h)) + \pr(\ell \mid \ell)\cdot (\ps(h, \vpr_\ell) - \ps(\ell, \vpr_{\ell}) > 0$, or (2) $\kde = \kde^\ell$ and $\pr(h\mid h) \cdot (\ps(\ell, \vpr_h) - \ps(h, \vpr_h)) + \pr(\ell \mid h)\cdot (\ps(\ell, \vpr_\ell) - \ps(h, \vpr_{\ell})) > 0$, then for all $\ag$, truthful reporting is NOT an $\varepsilon$-ex-ante Bayesian $\kde + 2$-strong equilibrium, where $\varepsilon$ is a constant. 
\end{prop}

\begin{proof}
    We show the case where $\kde = \kde^h$ and $\pr(h\mid \ell) \cdot (\ps(h, \vpr_h) - \ps(\ell, \vpr_h)) + \pr(\ell \mid \ell)\cdot (\ps(h, \vpr_\ell) - \ps(\ell, \vpr_{\ell})) > 0$ hold. The other case goes through by symmetricity.

    Firstly, notice that Lemma~\ref{lem:pr_psr} guarantee that at least one of $\ps(h, \vpr_h) - \ps(\ell, \vpr_h) > 0$ and $\ps(h, \vpr_\ell) - \ps(\ell, \vpr_{\ell}) < 0$ must hold. If $\ps(h, \vpr_h) - \ps(\ell, \vpr_h) \le 0$, then $\ps(h, \vpr_\ell) - \ps(\ell, \vpr_{\ell}) < 0$, which contradict with the condition. Therefore, $\ps(h, \vpr_h) - \ps(\ell, \vpr_h) > 0$. This guarantee that $\kde^h < n$. 

    Now consider the following strategies for a group of $\kde + 2$ deviators. $\kde + 1$ deviators always report $h$, and one deviator report truthfully. For the deviators reporting $h$, their expected utility is the same as if only $\kde + 1$ of them are in the deviating group, which is shown to exceed truthful reporting in Theorem~\ref{thm:pp_exante}. Now we show that the deviator report truthfully (denoted by agent $i$) has their utility increased by some constant $\varepsilon$.
    Since $i$ reports truthfully, the expected utility $i$ gets from truthful is the same with $\stgp^*$, where everyone reports truthfully. This means that $\dut_t = 0$. Then we consider $i$'s utility from other deviators. When $i$'s peer always report $h$, $i$'s expected utility is
    \begin{equation*}
        \ut_i(\stgp \mid \jdeviate) = \pr(h)\cdot \ps(h, \vpr_h) + \pr(\ell)\cdot \ps(h, \vpr_\ell)
    \end{equation*}
    Therefore, the difference on the expected utility is 
    \begin{align*}
        \ut_i(\stgp) - \ut_i(\stgp^*) =&\ \pr(h)\cdot\pr(\ell \mid h)\cdot (\ps(h, \vpr_h) - \ps(\ell, \vpr_h)) + \pr(\ell)\cdot \pr(\ell\mid \ell) \cdot (\ps(h, \vpr_\ell) - \ps(\ell, \vpr_\ell))\\
        =&\ \pr(\ell) \cdot (\pr(h \mid \ell)\cdot (\ps(h, \vpr_h) - \ps(\ell, \vpr_h)) + \pr(\ell\mid \ell) \cdot (\ps(h, \vpr_\ell) - \ps(\ell, \vpr_\ell))).
    \end{align*}
    The condition (1) guarantees that $\ut_i(\stgp) - \ut_i(\stgp^*) > 0$ and is a constant on all $\ag$. Therefore, $i$'s expected utility gain through deviation is
    \begin{equation*}
        \frac{\kde + 1}{\ag - 1}\cdot (\ut_i(\stgp) - \ut_i(\stgp^*)) > \frac{\Ex_{\sigi\sim \vpr_{\ell}}[\ps(\sigi, \vpr_{\ell}) - \ps(\sigi, \vpr_{h})]}{\ps(h, \vpr_h) - \ps(\ell, \vpr_h)} \cdot (\ut_i(\stgp) - \ut_i(\stgp^*)). 
    \end{equation*}
    Therefore, by setting $\varepsilon$ to be the right-hand-side, we show that truthful reporting is NOT an $\varepsilon$-ex-ante Bayesian $\kde + 2$-strong equilibrium. 
\end{proof}

For Bayesian $\kd$-strong equilibrium, agents with signal $h$ and signal $\ell$ have different interim expected utility. Therefore, truthful reporting is NOT an $\varepsilon$-Bayesian $(\kdq + 1)$-strong equilibrium, for some constant $\varepsilon$, as shown in the following proposition. 

\begin{prop}
    Let $\kdq$ defined as in Theorem~\ref{thm:pp_qi}. Truthful reporting is NOT an $\varepsilon$-Bayesian $\kdq + 1$-strong equilibrium, where $\varepsilon = \frac{\kdq}{\ag}\cdot \pr(\ell \mid h) \cdot (\ps(h, \vpr_h) - \ps(\ell, \vpr_h))$ or $\frac{\kdq}{\ag}\cdot \pr(h \mid \ell) \cdot  (\ps(\ell, \vpr_\ell) - \ps(h, \vpr_\ell))$. 
\end{prop}

% \begin{prop}
%     If the condition in the Proposition does not hold, then for $\ag$ and any $0 < kc \le  |\kde^h - \kde^{\ell}| $, there exists a $\varepsilon = \Theta( \kc / \ag)$ such that truthful reporting is an $\varepsilon$-ex-ante Bayesian $\kde + \kc$-strong equilibrium. 
% \end{prop}

% \begin{proof}
%     Suppose there are $\kde + \kc$ deviators among $\ag$ agents. Let $i$ denote the deviator with the highest expected utility, whose strategy is $(\bpl^i, \bph^i)$. Let $(\abpl, \abph)$ be the average strategy of all other agents (including deviators and truthful agents). The proof consists of two parts. First, we show that $\ut_i$ is maximized when $(\bpl^i, \bph^i) \in \{0, 1\}^2$, i.e., agent $i$ plays a pure strategy. Then, we show that for all four pure strategy, $i$'s expected utility gain in deviation cannot exceed $\Theta(kc / \ag)$. 

%     The expected utility of $i$ in the deviation can be viewed as a function of $(\bpl^i, \bph^i)$ and $(\abpl, \abph)$. Let $\func(\bpl^i, \bph^i, \abpl, \abph) = \ut_i(\stgp)$. Notice that $\func$ is linear on $\bpl^i$ and linear on $\bph^i$. Therefore, $\func$ is maximized on the corner of $(\bpl^i, \bph^i)$'s feasibility set, which is $\{0, 1\}^2$. Now we discuss four possible pure strategy of $i$ one by one. 

%     \paragraph{Case 1: $(\bpl^i, \bph^i) = (1, 1)$.} In this case, the expected utility of $i$ is 
%     \begin{align*}
%         \ut_i(\stgp) =&\ \ps(h, \vpr_h) \cdot (\pr(h)\cdot (\pr(h \mid h)\cdot \abph + \pr(\ell \mid h) \cdot \abpl) + \pr(\ell) \cdot (\pr( h \mid \ell) \cdot \abph + \pr(\ell\mid \ell)\cdot \abpl))\\
%         &\ + \ps(\ell, \vpr_h) \cdot (\pr(h)\cdot (1 - \pr(h \mid h)\cdot \abph - \pr(\ell \mid h) \cdot \abpl) + \pr(\ell) \cdot (1 - \pr( h \mid \ell) \cdot \abph - \pr(\ell\mid \ell)\cdot \abpl)).
%     \end{align*}
%     And the derivative 
%     \begin{align*}
%         \frac{\partial \ut_i(\stgp)}{\partial \abph} =&\ (\ps(h, \vpr_h) - \ps(\ell, \vpr_h))\cdot (\pr(h) \cdot \pr(h\mid h) + \pr(\ell) \cdot \pr(h \mid \ell)) = \pr(h) \cdot (\ps(h, \vpr_h) - \ps(\ell, \vpr_h)). \\
%         \frac{\partial \ut_i(\stgp)}{\partial \abpl}=&\ \pr(\ell) \cdot (\ps(h, \vpr_h) - \ps(\ell, \vpr_h)). 
%     \end{align*}

%     Therefore, when $\ps(h, \vpr_h) - \ps(\ell, \vpr_h) > 0$, $\ut_i(\stgp)$ is uniquely maximized when $\abph$ and $\abpl$ are maximized. This will be achieved when all other deviators also play $(1, 1)$. Now let $\stgp$ be the strategy profile where all $\kd + \kc$ deviatior (including $i$) always report $h$. Following the proof of Theorem~\ref{thm:pp_exante}, we have 
%     \begin{align*}
%         \ut_i(\stgp) - \ut_i(\stgp^*) =&\ -( \frac{\kd +\kc - 1}{\ag - 1}\dut_d + \frac{\ag - \kd - \kc}{\ag - 1}\dut_t)\\
%         \le &\ \frac{\kc}{\ag - 1}(\dut_t - \dut_d)\\
%         =&\ \frac{\kc}{\ag - 1}\cdot \pr(\ell)\cdot (\ps(h, \vpr_h) - \ps(\ell, \vpr_h))\\
%         =&\ \Theta(\kc / \ag). 
%     \end{align*}

%     On the other hand, when $\ps(h, \vpr_h) - \ps(\ell, \vpr_h) \le 0$, $\kde^h = \ag$, and the second condition in the previous proposition holds, which is beyond the discussion of this proposition. 

%     \paragraph{Case 2: $(\bpl^i, \bph^i) = (0, 0)$.} Due to symmetricity, this case resembles the case where $(\bpl^i, \bph^i) = (1, 1)$. When $\ps(\ell, \vpr_\ell) - \ps(h , \vpr_\ell) > 0$, the expected utility gain of $i$ is at most $\Theta(\kc / \ag)$. When $\ps(\ell, \vpr_\ell) - \ps(h , \vpr_\ell) \le 0$, the condition of this proposition does not hold. 

%     \paragraph{Case 3: $(\bpl^i, \bph^i) = (0, 1)$.} In this case agent $i$ also report informatively. 
%     In this case, 
%     \begin{align*}
%         \ut_i(\stgp) =&\ \pr(h) \cdot ((\pr(h \mid h) \cdot \abph + \pr(\ell \mid h)\cdot \abpl) \cdot \ps(h, \vpr_h)\\
%         &\  +( 1- \pr(h \mid h) \cdot \abph - \pr(\ell \mid h)\cdot \abpl) \cdot \ps(\ell, \vpr_h))\\
%         +&\ \pr(\ell) \cdot ((\pr(h \mid \ell) \cdot \abph + \pr(\ell \mid \ell)\cdot \abpl) \cdot \ps(h, \vpr_\ell)\\
%         &\  +( 1- \pr(h \mid \ell) \cdot \abph - \pr(\ell \mid \ell)\cdot \abpl) \cdot \ps(\ell, \vpr_\ell)). 
%     \end{align*}
%     The derivatives
%     \begin{align*}
%         \frac{\partial \ut_i(\stgp)}{\partial \abph} =&\ \pr(h)\cdot (\pr(h \mid h) \cdot (\ps(h, \vpr_h) - \ps(\ell, \vpr_h)) + \pr(\ell \mid h) \cdot (\ps(h, \vpr_\ell) - \ps(\ell, \vpr_\ell))).\\  
%         \frac{\partial \ut_i(\stgp)}{\partial \abpl}=&\ \pr(\ell)\cdot (\pr(h \mid \ell) \cdot (\ps(h, \vpr_h) - \ps(\ell, \vpr_h)) + \pr(\ell \mid \ell) \cdot (\ps(h, \vpr_\ell) - \ps(\ell, \vpr_\ell))). 
%     \end{align*}
%     Note that the derivates are closely related to the conditions in the previous proposition. Note that $\frac{\partial \ut_i(\stgp)}{\partial \abph} +\frac{\partial \ut_i(\stgp)}{\partial \abpl} < 0$.

%     If $\frac{\partial \ut_i(\stgp)}{\partial \abph} \ge 0$ and $\frac{\partial \ut_i(\stgp)}{\partial \abpl} \le 0$, then $\ut_i(\stgp)$ is maximized at $(\abpl, \abph) = (0, 1)$, which is achieved by all the deviators report truthfully. In this case, agent $i$'s expected utility does not increase under any deviation. 

%     Suppose $\frac{\partial \ut_i(\stgp)}{\partial \abpl} > 0$, In this case, $\kde^h < n$ according to the previous proposition. Therefore, there must be $\kde = \kde^\ell < \kde^h$ so that the condition (1) does not hold. We show that, if $(\abpl, \abph)$ is close to $(1, 1)$, other deviators suffer loss in expected utility; if $(\abpl, \abph)$ is close to $(0, 0)$, agent $i$ suffers loss in expected utility. 

%     Let $\gunc_h = \frac{\partial \ut_i(\stgp)}{\partial \abph}$ and $\gunc_{\ell} = \frac{\partial \ut_i(\stgp)}{\partial \abpl}$.
%     Firstly, the feasibility set of $(\abpl, \abph)$ is $[0, \frac{\kde + \kc - 1}{\ag - 1}]\times [1 - \frac{\kde + \kc - 1}{\ag - 1}, 1]$. Secondly, when $\abpl = \frac{\gunc_h}{\gunc_{\ell}} \cdot (1 - \abph)$, $\ut_i(\stgp)$ is equal or all feasible $(\abpl, \abph)$. When $\abpl < \frac{\gunc_h}{\gunc_{\ell}} \cdot (1 - \abph)$, $\ut_i(\stgp) < \ut(\stgp^*)$, and agent $i$ has no incentives to join the deviate group. On the other hand, when $\abpl \ge \frac{\gunc_h}{\gunc_{\ell}} \cdot (1 - \abph)$, we show that the average expected utility of all other agents will be strictly lower than $\ut(\stgp^*)$, and at least one of them have no incentives to deviate. 

%     We first characterized show that the area $\abpl \ge \frac{\gunc_h}{\gunc_{\ell}} \cdot (1 - \abph)$ a triangle with vertices $(0, 1)$, $(\frac{\kde + \kc - 1}{\ag - 1})$, and $(\frac{\kde + \kc - 1}{\ag - 1}, 1 - \frac{\gunc_{\ell}}{\gunc_h} \cdot \frac{\kd + \kc - 1}{\ag - 1})$. The first two points is determined by the feasibility set of $\abpl$ and $\abph$. For the third point, it is sufficiently to show that $\gunc_h > \gunc_\ell > 0$ so that $\abpl = \frac{\gunc_h}{\gunc_{\ell}} \cdot (1 - \abph)$ intersects the feasibility boundary at $\abpl = \frac{\kde + \kc - 1}{\ag - 1}$ rather than $\abph = 1 - \frac{\kde + \kc - 1}{\ag - 1}$. 
%     \begin{align*}
%         \gunc_h - \gunc_{\ell} =&\ \pr(h) \cdot (\pr(h \mid h) - \pr(\ell \mid h))\cdot (\ps(h, \vpr_h) - \ps(\ell,  \vpr_h))\\
%         &\ + \pr(\ell)\cdot (\pr(h\mid \ell) - \pr(\ell \mid \ell))\cdot (\ps(h, \vpr_{\ell}) - \ps(\ell, \vpr_{\ell}))
%     \end{align*}

%     Firstly, notice that $\pr(h \mid h) - \pr(\ell \mid h) > \pr(h\mid \ell) - \pr(\ell \mid \ell)$. Therefore, when $\pr(h \mid h) - \pr(\ell \mid h) \ge 0$, 
%     \begin{equation*}
%         \gunc_h - \gunc_{\ell} >(\pr(h \mid h) - \pr(\ell \mid h))\cdot (\pr(h) \cdot  (\ps(h, \vpr_h) - \ps(\ell, \vpr_h)) - \pr(\ell) \cdot (\ps(h, \vpr_{\ell}) - \ps(\ell, \vpr_{\ell}))). 
%     \end{equation*}

%     Then notice that $\gunc_{\ell} > 0$ implies that $\ps(h, \vpr_h) - \ps(\ell, \vpr_h) > 0$. Combined with the fact that $\pr(h \mid \ell) < \pr(h)$ and $\pr(\ell \mid \ell) > \pr(\ell)$, 
%     \begin{equation*}
%         \pr(h) \cdot  (\ps(h, \vpr_h) - \ps(\ell, \vpr_h)) - \pr(\ell) \cdot (\ps(h, \vpr_{\ell}) - \ps(\ell, \vpr_{\ell})) > 0. 
%     \end{equation*}
%     Therefore, we show that when $\pr(h \mid h) - \pr(\ell \mid h) \ge 0$, $\gunc_h, \gunc_{\ell} > 0$. 
    
%     When $\pr(h \mid h) - \pr(\ell \mid h) < 0$. Note that $\kde^l < \ag$ guarantees that $\ps(h, \vpr_{\ell}) - \ps(\ell, \vpr_{\ell}) < 0$. Therefore, 
%     \begin{equation*}
%         \gunc_h - \gunc_{\ell} > (\pr(\ell\mid \ell) - \pr(h \mid \ell))\cdot (\pr(\ell) \cdot  (\ps(\ell, \vpr_\ell) - \ps(h, \vpr_\ell)) - \pr(h) \cdot (\ps(h, \vpr_h) - \ps(\ell, \vpr_h))). 
%     \end{equation*}
% \end{proof}