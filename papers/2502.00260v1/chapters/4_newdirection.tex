Although our theoretical results focus on peer prediction, we believe that our solution concept of (ex-ante) Bayesian $\kd$-strong equilibrium is a powerful tool to characterize coalitional strategic behaviors (with bounded group size) and predict the outcome in a wide range of real-world scenarios. Here we give two more scenarios in which our solution can be applied: voting with partially informed voters and the Private Blotto game. 

\subsection{Voting with Partially Informed Voters}
Starting from the Condorcet Jury Theorem~\cite{Condorcet1785:Essai}, voting with partially informed voters has been extensively studied in the past literature~\cite{austen1996information,feddersen1997voting,wit1998rational, mclennan1998consequences, myerson1998extended, duggan2001bayesian, pesendorfer1996swing,feddersen1998convicting,martinelli2002convergence,gerardi2000jury,meirowitz2002informative,coughlan2000defense, han2023wisdom,acharya2016information,kim2007swing,bhattacharya2013preference, bhattacharya2023condorcet,ali2018adverse}.
In the voting setting where voters are partially informed, each voter only has partial information about the alternatives and his/her preference over the alternatives is not immediately clear.
This happens in myriad scenarios such as presidential elections (where the performance of each president candidate is not fully known) and voting for or against a certain policy or a certain decision (where the effect of the policy/decision is unclear at the moment of the voting).
The goal of a voting scheme is to aggregate the information of the voters and uncover the alternative favored by the majority  \emph{ex-post}.
This is already a non-trivial task in the information aggregation aspect~\cite{prelec2017solution}, and the situation is even more complex with strategic agents.
In fact, this problem is highly non-trivial even with two alternatives.

Consider the following typical model.
A set of $n$ voters are voting between two alternatives $\{\mathcal{A},\mathcal{B}\}$.
There are two world states $\{X,Y\}$.
One of them is \emph{the actual world}, but this is unknown to the voters.
A voter's preference over $\{\mathcal{A},\mathcal{B}\}$ may or may not depend on the world state.
For example, a voter may prefer $\mathcal{A}$ over $\mathcal{B}$ if the actual world is $X$ and $\mathcal{B}$ over $\mathcal{A}$ if the actual world is $Y$, while another voter may always prefer $\mathcal{A}$ over $\mathcal{B}$ regardless of the world state.
The goal of a voting mechanism is to identify the alternative that is favored by the majority if the actual world state were revealed.
Although voters do not know the actual world state, each voter receives a signal that is correlated to the world states.
A voter, after receiving the signal, forms beliefs over the likelihood of each world state, infers the signals received by other voters, and casts a vote according to these pieces of information.
This naturally formulates the problem as a Bayesian game.
In addition, when the number of voters $n$ is large, an individual voter's change of strategy is unlikely to affect the outcome of the election, and thus his/her expected utility is almost unrelated to his/her strategy.
On the other hand, voters form coalitions and jointly decide their votes in many practical scenarios.
This motivates the study of strong Bayes-Nash equilibria.

% \Biaoshuai{The next two paragraphs about \citet{han2023wisdom} and \citet{deng2024aggregation} can be removed if we are running out of space, although I personally prefer to keep them as they give the readers a concrete sense of the problem.}

% \gs{a little cheeky calling our own results "celebrated", but will throw them off our trail.}
The celebrated result from~\citet{han2023wisdom} shows that, when voters' preferences are \emph{aligned}, under the \emph{majority vote mechanism} (each voter votes for either $\mathcal{A}$ or $\mathcal{B}$; the alternative voted by more than half of the voters wins), the alternative favored by the majority (in \emph{ex-post}) is almost surely identified \emph{if and only if} the strategy profile is a strong Bayes Nash equilibrium.
Here, by saying aligned preferences, we mean that all voters' utilities for alternative $\mathcal{A}$ are higher in world state $X$ than in world state $Y$ and their utilities for alternative $\mathcal{B}$ are higher in world state $Y$ than in world state $X$.
That is, all voters' preferences are aligned in that they agree $X$ ``corresponds to'' $\mathcal{A}$ and $Y$ ``corresponds to'' $\mathcal{B}$, although the extent the voters' preferences are aligned with this correspondence can be different and due to which voters can be classified into three types:
\begin{itemize}
    \item the ``left-wing voters'' who always prefer $\mathcal{A}$: $v(\mathcal{A},X)>v(\mathcal{A},Y)>v(\mathcal{B},Y)>v(\mathcal{B},X)$
    \item the ``right-wing voters'' who always prefer $\mathcal{B}$: $v(\mathcal{B},Y)>v(\mathcal{B},X)>v(\mathcal{A},X)>v(\mathcal{A},Y)$
    \item  ``swing voters'' who prefer the alternative corresponding to the world state: $v(\mathcal{A},X)>v(\mathcal{B},X)$ and $v(\mathcal{B},Y)>v(\mathcal{A},Y)$
\end{itemize}
where $v(a,s)$ denotes the utility for alternative $a\in\{\mathcal{A},\mathcal{B}\}$ given the actual world state $s\in\{X,Y\}$.

The story is much more complicated with general utilities $v(\cdot,\cdot)$ that are not necessarily aligned.
In~\citet{deng2024aggregation}, it is proved that strong Bayes Nash equilibria may not exist even with only two types of voters with antagonistic preferences.
In particular, ``good'' equilibria that identify the majority-favored alternative only exist when the voters from one type significantly outnumber the voters from the other type.
When the population sizes of the two types of voters are close, \citet{deng2024aggregation} show that no strong Bayes Nash equilibrium exists.

However, only strong equilibria with unrestrictive group sizes are considered in the above-mentioned work.
A typical deviation group in a strategy profile consists of all voters of the same type, and the existence of this kind of large deviation group prevents many strategy profiles from being equilibria.
When considering more practical scenarios with bounded coalition sizes, more ``good'' equilibria are attainable.
Given the large size of deviation groups in the non-equilibria found in~\citet{deng2024aggregation}, it is likely there is an interpolation between the deviation group size $k$ and the distribution of voters from different types where ``good'' equilibria exist.
This provides a fine-grained structure to the problem compared with the ``all-or-nothing'' result in~\citet{deng2024aggregation}.

Strong equilibria are even less likely to exist for more general utilities.
It is appealing to apply our new equilibrium concepts with bounded deviating group sizes to characterize voters' strategic behaviors and obtain more positive and fine-grained results.
We believe this is a challenging yet exciting future research direction.

\subsection{Private Blotto Game}
Private Blotto game~\citep{donahue2023private} is a decentralized variation of the classic Colonel Blotto game~\citep{Borel1953:Blotto}.
It is proposed in order to model the conflict in the crowdsourcing social media annotation. For example, the Community Notes on X.com~\citep{wojcik2022birdwatch} allows users to vote for/against posts to identify misinformation and toxic speech with the wisdom of the crowd. 
\begin{example}
    \label{ex:anno} Suppose there are $\ag$ platform users and $m$ posts on a topic (for example, whether restrictions should be made for the COVID pandemic). Users obtain different private information from different sources, which can be generally categorized into two types, pros and cons. Each user simultaneously chooses and labels one post based on their type. The labels on each post will eventually determine the influence on the readers. A post with more supporters spreads more widely, and a post with more opponents will be announced as misinformation. Each user aims to maximize the influence of their type and plays the game strategically. What will be a stable status in such a scenario?
\end{example}

The traditional Colonel Blotto game models this scenario as a centralized game, where two opposite ``colonels'' (for example, campaign groups) control all the users. In the Private Blotto game, on the other hand, users make their own decisions on where to deploy. This better simulates the modern social media environment where a central coordinator is generally lacking. 

% While previous work models the type of each user as predetermined, it is natural to extend the game to the Bayesian setting where the type of each user is drawn from a prior. This allows us to characterize modern Internet scenarios where users receive different noisy information and form different preferences. 

% \gs{made this not from a prior distribution.}

\begin{definition}[Private Blotto game.] 
    $\ag$ agents are competing over $m$ items. Each agent has a type ($pro$ or $con$). Every agent (simultaneously) chooses exactly one item to label. The outcome of each item is determined by some outcome function. The disutility of each agent is the distance from the agent's type to each item's outcome. 
\end{definition}

The results in the Private Blotto game~\citep{donahue2023private} appear to heavily rely on the complete lack of coordination, which is also not entirely realistic. While a central coordinator is lacking, an agent can still locally coordinate with a few others. This allows (small) strategic groups and local campaigns to emerge in real-world scenarios. Moreover, these settings nearly always lack complete information and might be more faithfully modeled by agents receiving different information about various topics.   

In this setting, our new solution concept of (ex-ante) Bayesian $\kd$-strong equilibrium seamlessly interpolates between these two extremes of complete centralization and complete decentralization.  
The bound $\kd$ can characterize how well-organized the agents are. When $\kd = 1$, agents are fully decentralized. A larger $\kd$ characterizes scenarios where agents coordinate with friends, neighborhoods, or campaigns on relevant issues. Finally, when $\kd$ is large enough, agents can be viewed as commanded by two opposite centralized ``colonels'', and the game becomes closer to the traditional Colonel Blotto game. 
Moreover, our definition will also naturally extend to the setting where agents have more than two sides (for example, different political factions that are more or less aligned) and the scenario where the agent's utilities are related to an underlying ground truth rather than peer partisanship. 


% We are interested in the following questions. 

% \paragraph{Does there exist a $\kd$-equilibrium?} \citet{donahue2023private} shows that even a Nash equilibrium is not guaranteed to exist in the private Blotto game with pre-determined agent types. A natural question is, under different prior and outcome functions, what is the largest $\kd$ to make an $\kd$-equilibrium exists? This evaluates the game's robustness against strategic group campaigns of a certain type and serves as the basis for studying the other two questions. 

% \paragraph{What does a $\kd$-equilibrium looks like?} \citet{donahue2023private} finds two types of equilibrium. In the first case, almost all agents tug on one item while leaving other items largely empty. In the second case, both types of agents are spread evenly among all items. It is an interesting future work to generalize these characterizations to the notion of (ex-ante) Bayesian $\kd$-strong equilibrium. We conjecture that for large $\kd$, agents are more likely to spread among items in an equilibrium (if exists). When a large number of agents tug on one item (that is not much more important than other items), a group of deviators (likely with the same type) are willing to deviate to be spread on empty items, which makes them lose the tugging item but win more back on empty models. It is also an interesting topic to discover whether other types of equilibria exist. 

% \paragraph{Does this process help us identify misinformation?} Suppose the type prior is determined by an underlying world state that assigns one type as ``correct'' and the other as ``incorrect'', and the game designer (the social media platform, for example) aims to maximize the influence of the correct type, which can be evaluated by the outcomes of all the items. (For example, the platform wants to maximize the accuracy of identifying misinformation.) Do the equilibria in this decentralized Private Blotto game lead to better results than those in the centralized Blotto game? A price-of-anarchy or price-of-stability-like research shall be conducted for this problem. 