\section{Background}
\label{sec:back}

\subsection{2.5D Integration}
\label{ssec:back-25D}

\ps{Introduce types of 2.5D stacked chips}

The most prominent difference between monolithic chips (\Cref{fig:back-monolithic}) and 2.5D stacked chips (\Cref{fig:back-substrate,fig:back-bridge,fig:back-interposer}) is that the former only contain a single silicon die while the later contain multiple dies which are called chiplets.
We categorize 2.5D stacked chips into those that only use an organic substrate (\Cref{fig:back-substrate}), those that use silicon bridges (\Cref{fig:back-bridge}), and those that use a silicon interposer (\Cref{fig:back-interposer}).
There are passive silicon interposers \cite{psi, ops-vs-psi} only containing metal layers fabricated in the \gls{beol} and active silicon interposers \cite{intact, first-act-int} also containing transistors fabricated in the \gls{feol}.
\Cref{tab:back-chips} summarizes the packaging options.

\begin{table}[h]
\setlength{\tabcolsep}{3pt}
\centering
\captionsetup{justification=centering}
\begin{tabular}{lcccc}
\toprule
Technology & 
Cost & 
\makecell{Link\\Bandwidth} & 
\makecell{Maximum\\Link Length$^*$} &
\makecell{Package-Level\\Routers}
\vspace{-0.3em}\\
\midrule
Monolithic Chip		& High	& High	& unlimited						& n/a	\\
Organic Substrate 	& Low	& Low 	& 10-25mm$^{**}$ \cite{bow-spec}& No 	\\
Silicon Bridge		& Mid	& Mid	& 2-4mm	\cite{emib, sib-2}		& No 	\\
Passive Interposer 	& Mid	& Mid	& 2-4mm	\cite{bow-spec}			& No 	\\
Active Interposer 	& High	& Mid	& unlimited						& Yes	\\
\bottomrule
\end{tabular}
\vspace{-0.5em}
\caption{\textbf{(\textsection\ref{ssec:back-25D}) Overview of packaging technologies.} *Maximum link length depends on data rate, bump pitch, etc. **Source termination (up to 50 mm for double termination).}
\vspace{-2.5em}
\label{tab:back-chips}
\end{table}


\ps{Properties and challenges of organic substrate}

\subsubsection{2.5D Stacked Chips using an Organic Substrate}
\label{sssec:back-substrate}

In 2.5D stacked chips without a silicon bridge or interposer (Fig.  \ref{fig:back-substrate}), the chiplets are directly connected to the organic substrate using \gls{c4} bumps. 
The rather large bump-pitch of $150$-$200$ $\mu$m severely limits the number of bumps that can be used to connect a chiplet to the substrate.
Therefore, only a limited number of wires is available for the construction of \gls{d2d} links. 
Consequently, wide, high-bandwidth links that are commonly used in monolithic chips cannot be used to connect different chiplets.
Instead, the data is serialized and transmitted over narrow \gls{d2d} links, which can quickly become a bottleneck.

\ps{Properties and challenges of silicon bridges}

\subsubsection{2.5D Stacked Chips using a Silicon Bridge}
\label{sssec:back-bridge}

To counteract the bandwidth bottlenecks formed by \gls{d2d} links in organic substrates, silicon bridges (Fig. \ref{fig:back-bridge}) such as Intel's \gls{emib} \cite{emib} or IBM's \gls{dbhi} \cite{dbhi} were introduced.
For off-chip communication, chiplets are connected to the package substrate using \gls{c4} bumps.
For high-bandwidth die-to-die communication, chiplets are connected to the silicon bridge using microbumps.
Since these microbumps have a pitch of $30$-$60$ $\mu$m, the number of wires and hence the bandwidth of \gls{d2d} links is about $10\times$ higher than in chips with a package substrate only.
However, these \gls{d2d} links still deliver lower bandwidth than on-die links in monolithic chips.
Furthermore, silicon bridges limit the link length \cite{emib, sib-2}.

\ps{Properties and challenges of passive interposer}

\subsubsection{2.5D Stacked Chips using a Passive Silicon Interposer}
\label{sssec:back-interposer-passive}

Another way of improving the bandwidth of \gls{d2d} links are passive silicon interposers, such as TSMC's \gls{cowos} \cite{cowos}.
Here, chiplets are connected to the interposer using microbumps and the interposer is connected to the package substrate using \gls{c4} bumps.
\gls{Tsvs} are used to provide connectivity between chiplets and the substrate.
One major limitation of passive interposers is the limited length of \gls{d2d} links.
Since passive interposers do not contain any transistors, it is not possible to build buffered links.
Therefore, unbuffered links are used, and their length cannot exceed some millimeters \cite{ucie, bow, bow-spec}.

\ps{Properties and challenges of active interposer}

\subsubsection{2.5D Stacked Chips using an Active Silicon Interposer}
\label{sssec:back-interposer-active}

Active interposers \cite{intact, first-act-int} allow the construction of buffered \gls{d2d} links, therefore, links in active interposers can be arbitrarily long.
Another advantage of active interposers is that they allow the integration of package-level routers.
However, active interposers come with their own challenges.
Since active interposers require the \gls{feol} process, they are about ten times more expensive than passive interposers \cite{coskun-2}.
Furthermore, they suffer from a lower yield than passive interposers, which is problematic as interposers are usually large.
Another challenge is the fact that active interposers generate heat, which is hard to remove since the interposer sits below the chiplets.
Due to these drawbacks of active silicon interposers, our work focuses on the remaining 2.5D integration technologies.


\subsection{Optimization Algorithms Suitable for Chiplet Placements}
\label{ssec:back-opt}

\ps{Quick intro of best random}

\subsubsection{Best Random}
\label{sssec:back-opt-br}

This algorithm generates random solutions to an optimization problem until the time budget is exhausted, then, it returns the best solution that was found.
We use this na\"ive algorithm as a baseline to determine whether the more advanced algorithms perform better than random.

\ps{Quick intro of the genetic algorithm}

\subsubsection{The Genetic Algorithm}
\label{sssec:back-opt-ga}

The \gls{ga} mimics biological evolution, where a solution to an optimization problem is viewed as an individual.
Each individual has a fitness score, corresponding to the quality of the solution.
The algorithm maintains a population of $P$ individuals, which changes through a series of generations.
In each generation, individuals with an insufficient fitness score are eliminated from the population and existing individuals with good fitness scores are merged to produce new individuals.
The key idea is to merge two solutions with good properties to achieve a new solution with even better properties.
For the genetic algorithm to work properly, we need to formulate a merge function that enables combining the strengths of two solutions while eliminating their weaknesses.

\ps{Quick intro of simulated annealing}

\subsubsection{Simulated Annealing}
\label{sssec:back-opt-sa}

The \gls{sa} algorithm is based on the annealing process, in which a material is heated up and slowly cooled down in order to alter its properties.
We start with a randomly selected solution and alter this solution through a series of iterations.
In each iteration, we explore a \textit{neighboring} solution.
If this solution is better than the current one, we accept it, i.e., we set it as our current solution.
Otherwise, we only accept it with a certain probability. 
This probability decreases over time since we assume that the longer our algorithm is running, the closer to the optimum we are and hence the less favorable it is to accept an inferior solution.
When formulating a problem for \gls{sa}, we need a method to generate \textit{neighboring} solutions with a similar quality.
If there is no sufficient correlation between the quality of a solution and the quality of its neighbors, then simulated annealing can not work properly.



