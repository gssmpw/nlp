\section{Artifact Description}
\label{app:artifacts}

\paragraph{Requirements}
Hardware: At least 25GB of disk space (mainly for traces) and the capability of running 16 threads in parallel (for the evaluation).
Software: python version 3.11.5 with matplotlib version 3.7.1 and numpy version 1.26.1.

\paragraph{Setup}

Download our framework using the link on the first page.
Download the traces listed in \Cref{tab:eval-traces} from the Netrace website \cite{netrace-traces} and moved them to \textit{./Rapid Chiplet/booksim2/src/netrace/traces/}.
Build BookSim2 by executing \textit{cd ./RapidChiplet/booksim2/src/} and \textit{make}.

\paragraph{Execution}

The script \textit{./reproduce\_placeit\_results.py} reproduces all results. It performs the following steps: 

\begin{enumerate}
	\item	\textbf{Optimization}: Optimize chiplet placements and \gls{ici} topologies for the four architectures. This step reads \textit{./config.py} and stores the results in \textit{./results/}. It is single-threaded and runs for approximately 100 hours.
	\item	\textbf{Evaluation}: Evaluate the optimized architectures using the RapidChiplet \cite{rapidchiplet} toolchain. This step reads the files from \textit{./results/} and traces from \textit{./RapidChiplet/booksim2/ src/netrace/traces/}. It stores evaluation results in \textit{./Rapid Chiplet/results/}. This step spawns up to 16 concurrent threads and runs for approximately 24 hours.
	\item	\textbf{Plotting}: Create all figures in the paper that contain data. This step reads data from \textit{./results/} and \textit{./RapidChiplet/ results/}, and stores the figures in \textit{./plots/}.
\end{enumerate}

