\section{Conclusion}
\label{sec:conclusion}
\vspace{0.7em}

\ps{Highlight key idea/novelty of PlaceIT}

In this work, we present \name, a novel methodology to jointly optimize the chiplet placement and \gls{ici} topology for chips with heterogeneous chiplet shapes and silicon bridges or passive silicon interposers.
The main novelty of our approach is that we perform optimization on the chiplet placement itself, where we infer a custom, placement-based \gls{ici} topology for each placement produced by an optimization algorithm.
We use the placement and its inferred \gls{ici} topology to compute proxies for \gls{ici} latency and throughput of different traffic types, which we combine into a user-defined quality metric that is returned to the optimization algorithm.

\ps{Highlight the framework/code}

The open-source \name~framework is modular and allows adding custom optimization algorithms or placement representations.
\name~offers a wide range of configurable parameters, making it applicable for a variety of designs with different chiplet dimensions, PHY-counts, and \gls{d2d} links.

\ps{Summarize evaluation}

Our evaluation on synthetic traffic shows that \name~produces \gls{ici}s with vastly lower \gls{c2m}, \gls{c2i}, and \gls{m2i} latency (reduced by up to 62\%) compared to a 2D mesh baseline.
On real traffic traces, \name~reduces the average packet latency in almost all traces and architectures considered.
The average packet latency is reduced by up to $18\%$ on average.

\ps{Concluding sentence}

By using our open-source \name~framework, architects can co-optimize their chiplet-placement and \gls{ici} topology to build 2.5D stacked chips with low-latency interconnects.

