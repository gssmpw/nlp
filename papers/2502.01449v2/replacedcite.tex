\section{Related Work}
\label{sec:related-work}

\ps{Optimize placement for existing topology}

Multiple recent studies have focused on placing chiplets on a passive silicon interposer.
These works usually assume that the \gls{ici} topology is given as an input.
Most of them optimize the \gls{twl} ____ or a combination of \gls{twl} and thermal properties ____.
Some also consider the \gls{ici} performance or the cost ____.
Many of these works can be combined with \name. 
We could, e.g., use \name~to find a placement and \gls{ici} topology, and then apply TAP-2.5D ____ to fine-tune the placement for thermal properties. 

\ps{Explain Coskun in detail, since they kind of optimize the topology}

An interesting line of work is that of Coskun et al. ____.
They apply a cross-layer co-optimization approach to jointly optimize a 2.5D stacked chip across the logical-, physical- and circuit layer.
They consider a predetermined set of well-known \gls{ici} topologies out of which they select the most suitable one.
This is in contrast to \name, where completely new \gls{ici} topologies are created.
We see potential in combining the two approaches by first finding an \gls{ici} topology and placement using \name~and then applying the cross-layer co-optimization approach to optimize the remaining layers or to select the placement found by either \gls{br}, the \gls{ga} or \gls{sa}.

\ps{Optimized topologies for active interposers}

Research on \gls{ici} topologies focuses on active interposers, since they offer longer links and package-level routers.
Such works assume a regular 2D grid of compute-chiplets with memory- or IO-chiplets on the side.
\gls{ici} topologies such as ButterDonut ____, ClusCross ____, or Kite ____ are optimized for low \gls{ici} latency and high \gls{ici} throughput.

\ps{HexaMesh and its shortcomings}

One of the few works focussing on \gls{ici} topologies for passive silicon interposers is HexaMesh ____.
They propose a hexagonal arrangement of chiplets where each non-border chiplet has six \gls{d2d} links to other chiplets.
However, this approach is only applicable to homogeneous architectures.

\ps{Explain how our approach fills a gap}

\name~is the first work known to us that jointly optimizes \gls{ici} topology and chiplet placement.
Furthermore, it is the first work on \gls{ici} topologies for heterogeneously shaped chiplets.
Table \ref{tab:related-work} compares \name~to its related work.