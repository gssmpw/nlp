\section{Experiments with Relaxed Constraints}
\label{app:relaxed-constraints}

\ps{Introduce experiments with relaxed constraints}

In \Cref{ssec:homo-opt,ssec:hetero-opt}, we use chiplets that are optimized for architectures similar to our baseline, i.e., compute-chiplets have four \gls{phys} and relay-capability while memory-, and IO-chiplets only have a single PHY.
We use these chiplets because architectures like our baseline are the most common today and therefore, such chiplets are most likely to be readily available.
However, this introduces an unfavorable bias against \name~as implementing multiple PHYs and relay-capability in memory- and IO-chiplets would vastly improve the benefits of architectures like the ones found by \name.
Therefore, in this appendix, we repeat all of our experiments with revised chiplets specifications where each compute-, memory-, and IO-chiplet has four \gls{phys} and relay capability.
All remaining parameters are exactly as in the main body of the paper.

\begin{figure}[h]
\centering
\captionsetup{justification=centering}
\includegraphics[width=1.0\columnwidth]{img/appendix/homo_results_appendix.pdf}
\caption{\textbf{(\textsection \ref{app:relaxed-constraints}) Results for homogeneously shaped chiplets.}
We show the evolution of the result over time (left) and the distribution of the final result over 10 repetitions (right).
See \Cref{fig:app-eval-placements} for the placements found by the best algorithm.}
\label{fig:app-homo-results}
\end{figure}


\begin{figure}[h]
\centering
\captionsetup{justification=centering}
\vspace{-1.0em}
\includegraphics[width=1.0\columnwidth]{img/appendix/hetero_results_appendix.pdf}
%
\caption{\textbf{(\textsection \ref{app:relaxed-constraints}) Results for heterogeneous chiplet shapes.}
We show the evolution of the result over time (left) and the distribution of the final result over 10 repetitions (right).
See \Cref{fig:app-eval-placements} for the placements found by the best algorithm.}
\label{fig:app-hetero-results}
\vspace{-1.75em}
\end{figure}


\begin{figure}[h]
\centering
\vspace{-0.5em}
\captionsetup{justification=centering}
\begin{subfigure}{0.99 \columnwidth}
\centering
\includegraphics[width=1.0\columnwidth]{img/appendix/eval_synthetic_appendix.pdf}
\end{subfigure}
\caption{\textbf{(\textsection \ref{ssec:eval-synthetic}) Results on synthetic traffic using the \textit{\name} configuration}.}

\label{fig:app-eval-synthetic}
\vspace{-2.2em}
\end{figure}


\begin{figure}[h]
\centering
\captionsetup{justification=centering}
\includegraphics[width=1.0\columnwidth]{img/appendix/eval_full_trace_appendix.pdf}
\caption{\textbf{(\textsection \ref{ssec:eval-trace-full}) speedup over baseline in average packet latency} (blackscholes trace, \textit{\name} configuration).}
\label{fig:app-eval-trace-full}
\end{figure}


\begin{figure*}[t]
\centering
\captionsetup{justification=centering}
%
\includegraphics[width=0.85\textwidth]{img/appendix/eval_placements_appendix.pdf}
%
\vspace{-0.5em}
\caption{\textbf{(\textsection \ref{app:relaxed-constraints}) Baseline placements/topologies (top) and optimized placements/topologies found by \name~(bottom).}}
\label{fig:app-eval-placements}
\vspace{-0.5em}
\end{figure*}


\begin{figure*}[t]
\centering
\captionsetup{justification=centering}
\includegraphics[width=0.85\textwidth]{img/appendix/eval_partial_trace_appendix.pdf}
\vspace{-0.5em}
\caption{\textbf{(\textsection \ref{app:relaxed-constraints}) Results for the partial trace regions:} speedup in average packet latency compared to the baseline.\\
We do not show data for the vips trace due to issues with the simulator when simulating that trace.}
\label{fig:app-eval-trace-partial}
\vspace{-1.0em}
\end{figure*}


\newpage
\Cref{fig:app-homo-results,fig:app-hetero-results} show the results of the optimization process for homogeneous and heterogeneous chiplet shapes, respectively.
We observe that with the revised chiplet specifications, the gap between the three optimization algorithms and the baseline grow considerably. 
The performance differences between the three optimization algorithms themselves are very similar to those in the main body of the paper.


\Cref{fig:app-eval-placements} shows the baseline architectures and the best architectures found by \name.
\Cref{fig:app-eval-synthetic} shows the evaluation on synthetic traffic. 
The benefits of \name~for \gls{c2m}, \gls{c2i}, and \gls{m2i} traffic grew significantly compared to the experiments in the main body of the paper while the disadvantages for \gls{c2c} traffic became less severe.

\Cref{fig:app-eval-trace-full} shows the speedup in average packet latency for the full blackscholes trace. 
We observe that on average, the speedup increased compared to the experiments in the main paper body.
Finally, \Cref{fig:app-eval-trace-partial} shows our results for the simulation of partial trace regions.
Using the revised chiplet specifications, our results are significantly better than with the chiplets tailored to the baseline.
We now reduce the average packet latency for an even larger fraction of the trace-regions, and we reduce it by more, on average, by $18\%$ instead of $8\%$.



