\section{\name~for Heterogeneous Chiplet Shapes}
\label{sec:hetero}

\ps{Section introduction}

We explain our method to represent and optimize placements of heterogeneously shaped chiplets with arbitrary, rectangular shapes and arbitrary PHY counts and positions.

\subsection{Placement Representation}
\label{ssec:hetero-repr}


\ps{Introduce the first idea one might have and why we don't use it}

An intuitive representation of a placement of heterogeneously shaped chiplets would be a list of chiplets where each chiplet has a location and a rotation.
However, with such an approach, the \texttt{mutate()} and \texttt{merge()} operations could result in overlapping chiplets.
Only allowing operations that do not result in overlaps would severely restrict the number of operations and prohibit an exploration of the full solution space.
Allowing operations that result in overlaps and using a legalization step to make the chiplets non-overlapping again makes the placement quality deteriorate over time (we tried this as our first approach).
Therefore, in the remainder of this section, we present our more elaborate placement representation for chiplets with heterogeneous shapes.

\ps{Introduce the key idea of our representation of heterogeneous chiplets}

Instead of optimizing the location of chiplets directly, we optimize the \emph{order} and \emph{rotations} in which our custom placement algorithm places the chiplets.
Like this, every possible combination of \emph{order} and \emph{rotations} results in a placement with non-overlapping chiplets.
Our custom chiplet placement algorithm iterates through the chiplets in the specified order and places each of them by performing the following steps:


\begin{enumerate}[leftmargin=*, label={Step \arabic*:}]
	\item	Draw a line along the perimeter of the current placement (see the blue, dashed line in \Cref{fig:hetero-plac}).			
	\item	Identify all L-shaped corners of the perimeter-line (see the red solid corners in \Cref{fig:hetero-plac}).
	\item	Place the chiplet in the corner, that minimizes the area of a minimum square enclosing the whole placement.
	\item	The third step can result in overlaps (see \Cref{fig:hetero-plac-3}).
			If the newly placed chiplet has an overlap on the right, move it to the top and vice versa (see \Cref{fig:hetero-plac-4}).
\end{enumerate}

\begin{figure}[H]
\centering
\vspace{-1em}
\captionsetup{justification=centering}
%
\begin{subfigure}{1.0 \columnwidth}
\centering
\includegraphics[width=1.0\columnwidth]{img/heterogeneous/hetero_place_legend.pdf}
\vspace{-1.0em}
\end{subfigure}
%
\begin{subfigure}{0.24 \columnwidth}
\centering
\includegraphics[width=0.95\columnwidth]{img/heterogeneous/hetero_place_1.pdf}
\caption{Placing $C_1$.\\~}
\label{fig:hetero-plac-1}
\end{subfigure}
%
\begin{subfigure}{0.24 \columnwidth}
\centering
\includegraphics[width=0.95\columnwidth]{img/heterogeneous/hetero_place_2.pdf}
\caption{Placing $C_2$.\\~}
\label{fig:hetero-plac-2}
\end{subfigure}
%
\begin{subfigure}{0.24 \columnwidth}
\centering
\includegraphics[width=0.95\columnwidth]{img/heterogeneous/hetero_place_3.pdf}
\caption{Placing $C_3$\\creates overlap.}
\label{fig:hetero-plac-3}
\end{subfigure}
%
\begin{subfigure}{0.24 \columnwidth}
\centering
\includegraphics[width=0.95\columnwidth]{img/heterogeneous/hetero_place_4.pdf}
\caption{Move $C_3$ to\\fix overlap.}
\label{fig:hetero-plac-4}
\end{subfigure}
\vspace{-2em}
%
\caption{\textbf{(\textsection \ref{ssec:hetero-repr}) Our custom placement algorithm.}}
\label{fig:hetero-plac}
\vspace{-1em}
\end{figure}


\ps{Explain additional issues with our approach and how we solve them}

Notice that the first-class citizen on which the optimization algorithms operate are the \emph{order} and \emph{rotations} in which the chiplets are placed, not the placement itself.
Due to this indirection, we need to avoid isomorphic representations (different (order, rotations)-pairs that result in the same placement). 
If this is not the case, an optimization algorithm might use a set of supposedly different solutions that all result in the same placement.
To avoid isomorphic representations, we represent the order by chiplet types and not by chiplet IDs since two different orders by ID can result in equivalent placements, but two different orders by type cannot (see \Cref{fig:hetero-issues} left).
Furthermore, notice that a chiplet can be rotation-invariant, rotation-sensitive, or rotation-hybrid depending on whether the chiplet shape and PHY locations change upon rotation (see \Cref{fig:hetero-issues} right).
To prevent isomorphic representations, we disallow rotations of rotation-invariant chiplets, and we only allow $90^\circ$ rotations for rotation-hybrid chiplets.

\begin{figure}[H]
\centering
\vspace{-0.5em}
\captionsetup{justification=centering}
%
\includegraphics[width=1.0\columnwidth]{img/heterogeneous/hetero_issues.pdf}
%
\caption{\textbf{(\textsection \ref{ssec:hetero-repr}) Prevent multiple (order, rotations)-pairs from producing the same placement} by using order by type and dissallowing rotations based on the rotation behavior.}
\label{fig:hetero-issues}
\end{figure}

\vspace{-1em}
\begin{figure*}[h]
\centering
\captionsetup{justification=centering}
%
\begin{subfigure}{0.94 \textwidth}
\centering
\includegraphics[width=1.0\textwidth]{img/heterogeneous/hetero_nw_legend.pdf}
\vspace{-1.0em}
\end{subfigure}
%
\\
%
\begin{subfigure}{0.16 \textwidth}
\centering
\includegraphics[width=0.9\textwidth]{img/heterogeneous/hetero_nw_1.pdf}
\caption{Input: Placemet.}
\label{fig:hetero-nw-1}
\end{subfigure}
%
\begin{subfigure}{0.025\textwidth}
\centering
\includegraphics[width=0.9\textwidth]{img/heterogeneous/hetero_nw_arrow.pdf}
\vspace{0.9em}
\end{subfigure}
%
\begin{subfigure}{0.16 \textwidth}
\centering
\includegraphics[width=0.9\textwidth]{img/heterogeneous/hetero_nw_2.pdf}
\caption{Create graph.}
\label{fig:hetero-nw-2}
\end{subfigure}
%
\begin{subfigure}{0.025\textwidth}
\centering
\includegraphics[width=0.9\textwidth]{img/heterogeneous/hetero_nw_arrow.pdf}
\vspace{0.9em}
\end{subfigure}
%
\begin{subfigure}{0.16 \textwidth}
\centering
\includegraphics[width=0.9\textwidth]{img/heterogeneous/hetero_nw_3.pdf}
\caption{Compute MST.}
\label{fig:hetero-nw-3}
\end{subfigure}
%
\begin{subfigure}{0.025\textwidth}
\centering
\includegraphics[width=0.9\textwidth]{img/heterogeneous/hetero_nw_arrow.pdf}
\vspace{0.9em}
\end{subfigure}
%
\begin{subfigure}{0.16 \textwidth}
\centering
\includegraphics[width=0.9\textwidth]{img/heterogeneous/hetero_nw_4.pdf}
\caption{Add more edges.}
\label{fig:hetero-nw-4}
\end{subfigure}
%
\begin{subfigure}{0.025\textwidth}
\centering
\includegraphics[width=0.9\textwidth]{img/heterogeneous/hetero_nw_arrow.pdf}
\vspace{0.9em}
\end{subfigure}
%
\begin{subfigure}{0.16 \textwidth}
\centering
\includegraphics[width=0.9\textwidth]{img/heterogeneous/hetero_nw_5.pdf}
\caption{Output: Topology.}
\label{fig:hetero-nw-5}
\end{subfigure}
%
\caption{\textbf{(\textsection \ref{ssec:hetero-repr}) Inferring the placement-based \gls{ici} topology} from our representation for heterogeneous chiplet placements.}
\label{fig:hetero-nw}
\end{figure*}


\ps{Explain the details of our representation of heterogeneous chiplets}

Our placement representation for heterogeneous chiplets implements the same five functions as the representation of homogeneous chiplets.
Instead of generating, mutating or merging placements directly, we perform these operations on the chiplet \emph{order} and \emph{rotations} (see \Cref{fig:hetero-mutate-merge}).
To implement the \texttt{get\_area()} function, we compute the area of a minimal rectangle that encloses the whole placement.

\begin{figure}[h]
\centering
\captionsetup{justification=centering}
%
\begin{subfigure}{0.99 \columnwidth}
\centering
\includegraphics[width=1.0\columnwidth]{img/heterogeneous/hetero_mutate_merge.pdf}
\end{subfigure}
%
\caption{\textbf{(\textsection \ref{ssec:hetero-repr}) Mutate \& merge for  heterogeneous chip- lets.} C: Compute-chiplet, M: Memory-chiplet, I: IO-chiplet.}
\label{fig:hetero-mutate-merge}
\end{figure}


Extracting the placement-based \gls{ici} topology from a placement with heterogeneously shaped chiplets is more involved than for the homogeneous case: 
Given a placement (see \Cref{fig:hetero-nw-1}), we create a graph where each vertex represents a PHY.	
We add "internal" edges between all \gls{phys} of a chiplet with relay-capabilities (the black, solid edges in \Cref{fig:hetero-nw-2}).
Furthermore, we add a "candidate" edge between any two PHYs of different chiplets that are at most the maximum link length apart (the black, dotted edges in \Cref{fig:hetero-nw-2}).
We set the weight of "candidate" edges based on the link length and compute a \gls{mst} \cite{mst} on top of this graph (the red, dashed edges in \Cref{fig:hetero-nw-3}).
Each edge in the \gls{mst} corresponds to a \gls{d2d} link.
If some \gls{phys} of otherwise connected chiplets remain unconnected, we ignore them.	
If a whole chiplet remains unconnected, we abort and the operation that created this placement (random generation, mutate, or merge) is repeated.
Next, we iterate through the remaining candidate-edges, ordered by increasing weight.
If an edge is connected to two otherwise unused \gls{phys}, we add that edge to the \gls{ici} topology (the purple, fat edges in \Cref{fig:hetero-nw-4}).
\Cref{fig:hetero-nw-5} shows the resulting 	\gls{ici} topology.

\subsection{Optimization Results}
\label{ssec:hetero-opt}

\ps{Explain the Experiment Configuration}

We run \name~on one core of an Intel Xeon X7550 running Debian 11 to optimize the same two architectures as in \Cref{ssec:homo-opt}, but with heterogeneous chiplets (see \Cref{fig:hetero-chiplets}).

\begin{figure}[H]
\centering
\captionsetup{justification=centering}
\begin{subfigure}{0.99 \columnwidth}
\centering
\includegraphics[width=1.0\columnwidth]{img/heterogeneous/hetero_chiplets.pdf}
\end{subfigure}
\caption{\textbf{(\textsection \ref{ssec:hetero-opt}) Chiplet dimensions and PHY locations.}}
\label{fig:hetero-chiplets}
\vspace{-1em}
\end{figure}


Since we still want to optimize the \gls{ici} topology for cache coherency traffic, we use the same cost function weights as in \Cref{ssec:homo-opt}.
The remaining parameters are shown in \Cref{tab:hetero-params}.

\begin{table}[H]
\setlength{\tabcolsep}{3pt}
\centering
\captionsetup{justification=centering}
\vspace{-0.5em}
\begin{tabular}{llcc}
\hline
&Parameter					& 32 cores heterogeneous 	& 64 cores heterogeneous	\vspace{-0.3em}\\
\midrule
\multirow{4}{*}{\rotatebox[origin = c]{90}{General\qquad}}
&Time Budget				& 3600 s					& 3600 s					\\	
&Repetitions				& 10						& 10 						\\
&Norm. Samples				& 500 						& 500 						\\	
&Mutation Mode				& any-one					& any-one					\\	
&$L_\text{R}$,$L_\text{P}$, and $L_\text{L}$		& 10, 12, and 1 cycles 		& 10, 12, and 1 cycles 				\\	
&Distance Type				& Eucledian					& Eucledian					\\
&Max. Length				& 3mm						& 3mm						\\
\midrule
\multirow{4}{*}{\rotatebox[origin = c]{90}{GA}}
&Population size (P)			& 30  						& 20 						\\	
&Elitism size (E)			& 6  						& 5							\\	
&Tournament size (T)			& 6  						& 5 						\\	
&Mutation prob. ($p_m$) 		& 0.5						& 0.5 						\\	
\midrule
\multirow{4}{*}{\rotatebox[origin = c]{90}{SA}}
&Initial Temp. ($T_0$) 		& 33 						& 28 						\\	
&Iterations ($L$) 			& 50 						& 45 						\\	
&Cooling param. ($\alpha$) 	& 1 						& 1 						\\	
&Adaptive param. ($\beta$) 	& 5 						& 5 						\\	
\hline
\end{tabular}
\caption{\textbf{(\textsection \ref{ssec:hetero-opt}) Experiment configuration}\\for heterogeneously shaped chiplets.}
\label{tab:hetero-params}
\vspace{-1em}
\end{table}


\setcounter{table}{5}
\begin{table*}[b]
\setlength{\tabcolsep}{3pt}
\centering
\captionsetup{justification=centering}
\begin{tabular}{lccccccccccccccc}
\toprule
\vspace{-0.25em}
\multirow{2}{*}{Trace}		& \multicolumn{3}{c}{Region 1} & \multicolumn{3}{c}{Region 2} & \multicolumn{3}{c}{Region 3} & \multicolumn{3}{c}{Region 4} & \multicolumn{3}{c}{Region 5}\\
\cmidrule(lr){2-4} \cmidrule(lr){5-7} \cmidrule(lr){8-10} \cmidrule(lr){11-13} \cmidrule(lr){14-16}
\vspace{-0.25em}
						& P 	& C 	& I  		& P 	& C 	& I  		& P 	& C 	& I  		& P 	& C 	& I  		& P 	& C 	& I		\\
\midrule
blackscholes\_64c\_simsmall	& 189k 	& 5.6M 	& 0.0337	& 1.2M 	& 219M 	& 0.0056	& 4.9M 	& 75M 	& 0.0655	& 195k 	& 10M 	& 0.0019	& 129k 	& 5.7M 	& 0.0228	\\
bodytrack\_64c\_simlarge	& 189k 	& 5.6M 	& 0.0337	& 30M 	& 654M 	& 0.0453	& 355M 	& 3.9B 	& 0.0914	& 429k 	& 24M 	& 0.0176	& 161k 	& 5.7M 	& 0.0283	\\
canneal\_64c\_simmedium		& 189k 	& 5.6M 	& 0.0337	& 240M 	& 20B 	& 0.0121	& 74M 	& 300M 	& 0.2473	& 58M 	& 2.9B 	& 0.0198	& 133k 	& 5.7M 	& 0.0235	\\
dedup\_64c\_simmedium		& 189k 	& 5.6M 	& 0.0337	& 37M 	& 838M 	& 0.0201	& 379M 	& 2.6B 	& 0.1477	& 16M 	& 1.0B 	& 0.0153	& 160k 	& 5.7M 	& 0.0282	\\
ferret\_64c\_simmedium		& 189k 	& 5.6M 	& 0.0337	& 8.6M 	& 648M 	& 0.0133	& 273M 	& 7.5B 	& 0.0365	& 5.8M 	& 145M 	& 0.0402	& 220k 	& 5.7M 	& 0.0387	\\
fluidanimate\_64c\_simsmall	& 189k 	& 5.6M 	& 0.0337	& 6.8M 	& 777M 	& 0.0087	& 21M 	& 499M 	& 0.0420	& 6.1M 	& 599M 	& 0.0103	& 139k 	& 5.7M 	& 0.0245	\\
swaptions\_64c\_simlarge	& 189k 	& 5.6M 	& 0.0337	& 247k 	& 9.7M 	& 0.0254	& 310M 	& 1.7B 	& 0.1800	& 194k 	& 14M 	& 0.0141	& 113k 	& 5.7M 	& 0.0199	\\
x264\_64c\_simsmall			& 189k 	& 5.6M 	& 0.0337	& 1.8M 	& 82M 	& 0.0220	& 31M 	& 1.5B 	& 0.0212	& 102M 	& 12B 	& 0.0084	& 129k 	& 5.7M 	& 0.0227	\\
\bottomrule
\end{tabular}
\caption{\textbf{(\textsection \ref{ssec:eval-methodology}) Trace-regions used in the evaluation.} \textbf{P}: Number of packets, \textbf{C}: number of cycles, \textbf{I}: injection rate.}
\label{tab:eval-traces}
\end{table*}


					% Fix table placement

\ps{Discuss Reduction of solution space size}

Even though the problem of placing heterogeneously shaped chiplets is significantly more complex than that of placing homogeneous ones, we were able the achieve a similar solution space size of approximately $10^{14}$ solutions for the 32 core architecture and $10^{30}$ for the 64 core architecture.
We achieve this by optimizing the order and rotations, in which our custom placement algorithm places the chiplets, not the chiplet placement itself.
Like this, we are able to remove many unfavorable or invalid placements from the solution space.
However, it is possible that we also removed some good placements or even the best one.
In our opinion, this is not problematic as finding the single best solutions in a solution space of that scale is highly unlikely, and our approach preserves enough good solutions to achieve good results.

\ps{Compare different algorithms}

\Cref{fig:hetero-results} shows our results for the two heterogeneous architectures.
All optimization algorithms outperform the 2D mesh baseline (see \Cref{fig:eval-placements}) and both the \gls{ga} and \gls{sa} outperform \gls{br}.
As in the homogeneous setting, the \gls{ga} performs better than \gls{sa}, however, for the 32 core architecture with heterogeneous chiplet shapes, \gls{sa} performs comparably to the \gls{ga}.

\begin{figure}[h]
\centering
\captionsetup{justification=centering}
\vspace{-0.5em}
\includegraphics[width=1.0\columnwidth]{img/heterogeneous/hetero_results.pdf}
%
\caption{\textbf{(\textsection \ref{ssec:hetero-opt}) Results for heterogeneous chiplet shapes.}
We show the evolution of the result over time (left) and the distribution of the final result over 10 repetitions (right).
See \Cref{fig:eval-placements} for the placements found by the best algorithm.}
\label{fig:hetero-results}
\vspace{-1em}
\end{figure}




\ps{compare this setting to homogeneous one}

Comparing our results for the heterogeneous setting to those of the homogeneous ones reveals that even though the solution spaces of both settings have the same size, reaching convergence in the heterogeneous setting takes significantly longer than in the homogeneous one.
The reason for this is that the heterogeneous setting is more complex: For each new (order, rotations)-pair that is generated, we need to run our placement algorithm and the function to infer the placement-based \gls{ici} topology.
Therefore, in the heterogeneous setting, the number of placements that an optimization algorithm can create within the time budget is almost an order of magnitude lower than in the homogeneous setting (see \Cref{tab:hetero-ninstance}).

\setcounter{table}{4}
\begin{table}[H]
\setlength{\tabcolsep}{5pt}
\centering
\captionsetup{justification=centering}
\begin{tabular}{lcccc}
\hline
Algorithm					& \makecell{32 cores\\Homog.}		& \makecell{64 cores\\Homog.} 	& \makecell{32 cores\\Heterog.} 	& \makecell{64 cores\\Heterog.}
\vspace{-0.3em}\\
\midrule
Best Random (BR)			& 87.0k 							& 17.3k							& 8.5k							& 1.2k					\\
Genetic Alforithm (GA)		& 41.3k								& 11.5k							& 8.3k							& 1.7k					\\
Simulated Annealing (SA)	& 92.6k								& 20.4k							& 14.1k							& 3.1k					\\
\hline
\end{tabular}
\caption{\textbf{(\textsection \ref{ssec:hetero-opt}) Number of placements generated} by our optimization algorithms for different architectures.}
\label{tab:hetero-ninstance}
\end{table}
\vspace{-1em}
\setcounter{table}{6}




