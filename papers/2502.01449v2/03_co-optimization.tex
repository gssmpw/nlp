\section{Placement and Topology Co-Optimization}
\label{sec:coopt}

\ps{How performance depends on combination of topology and placement}

2.5D stacked chips require low-latency and high-throughput \gls{ici}s.
The latency mainly depends on the number of chiplet-to-chiplet hops and on the link latency.
The throughput primarily depends on the frequency, at which the links can be operated and on the congestion, i.e. how many different flows compete for the same link.
While hop count and congestion both depend on the \gls{ici} topology, link latency and link frequency both depend on the link length, which depends on the combination of \gls{ici} topology and chiplet placement.
To maximize the performance of the \gls{ici}, chiplet placement and \gls{ici} topology must be perfectly aligned.
Selecting a topology first and then optimizing the placement for that topology might not yield satisfying results, since the choice of a suboptimal topology can prevent us from finding a good placement.
Selecting a placement first and the optimizing the topology results in a similar problem.
The solution to this problem is to co-optimize the chiplet placement and \gls{ici} topology.

\begin{figure}[H]
\vspace{-1em}
\centering
\captionsetup{justification=centering}
\includegraphics[width=1.0\columnwidth]{img/cooptimization/cooptimization_idea.pdf}
\caption{\textbf{(\textsection\ref{sec:coopt}) Placement and topology co-optimization.}}
\label{fig:coopt-idea}
\vspace{-0.5em}
\end{figure}


\ps{Explain PlaceIT}

\Cref{fig:coopt-idea} visualizes our proposed chiplet placement and \gls{ici} topology co-optimization methodology.
An optimization algorithm is used to optimize the chiplet placement.
For each placement that the optimization algorithm produces, a placement-based \gls{ici} topology is inferred.
In this inference process, we minimize the length of \gls{d2d} links.
We then assess the quality of the combined chiplet placement and \gls{ici} topology, which we return to the optimization algorithm.
Using this methodology, the optimization algorithm does not only optimize metrics of the placement itself, like, e.g., the total area, but it also optimizes the placement in a way that enables us to construct good \gls{ici} topologies on top of it.
This is comparable to placers in the \gls{vlsi} place \& route step that optimize the placement of macros not only for area but also for routability of wires.
The difference to our methodology is that placers optimize the macro placement for the routability of predetermined nets, while we optimize the chiplet placement for the inference of an \gls{ici} topology with a yet unknown connectivity pattern.






