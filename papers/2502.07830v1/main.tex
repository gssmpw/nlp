\documentclass{article}
\usepackage{iclr2024_conference,times}
% if you need to pass options to natbib, use, e.g.:
%     \PassOptionsToPackage{numbers, compress}{natbib}
% before loading neurips_2023

% if you need to pass options to natbib, use, e.g.:
%\PassOptionsToPackage{numbers, compress}{natbib}
% before loading neurips_2021

% ready for submission
%\usepackage{neurips_2022}
%\usepackage{algorithmic_custom}
%\usepackage{algorithm}
\usepackage{algcompatible}
%\usepackage[algcompatible]{algpseudocode}

% ready for submission
%\usepackage{neurips_2024}


% to avoid loading the natbib package, add option nonatbib:
%\usepackage[nonatbib]{neurips_2021}
\usepackage[utf8]{inputenc} % allow utf-8 input
\usepackage[T1]{fontenc}    % use 8-bit T1 fonts
% \usepackage{accents}
\usepackage{amsmath}
\usepackage{amsfonts}       % blackboard math symbols
\usepackage{url}            % simple URL typesetting
\usepackage{booktabs}       % professional-quality tables
\usepackage{nicefrac}       % compact symbols for 1/2, etc.
\usepackage{microtype}      % microtypography
\usepackage{xcolor}         % colors
\usepackage{bm}
\usepackage{diagbox}
\usepackage{oplotsymbl}
\usepackage{pgfplots}
\usepackage{wrapfig}
\usepackage{multirow}
\usepackage{tablefootnote}
\usepackage{makecell}
\usetikzlibrary{patterns}

\usepackage[export]{adjustbox} % to align subfloats next to each other
\usepackage{tikz}
\usepackage{pifont} %\cmark and \xmark
\usepackage{circledsteps} % encircled numbers
\newcommand{\mycirc}[1]{\Circled[/csteps/fill color=black,/csteps/inner color=white]{\sffamily \small #1} }


\usepackage[ruled,vlined]{algorithm2e}

\SetKwInput{KwInput}{Input}                % Set the Input
\SetKwInput{KwOutput}{Output}              % set the Output

\DeclareMathOperator*{\argmax}{arg\,max}
\DeclareMathOperator*{\argmin}{arg\,min}

% for bibliography

%\usepackage{achemso}
%\mciteErrorOnUnknownfalse

% Comments Switch
\newif\ifdraft
%\drafttrue
\draftfalse

\newcommand{\thought}[1]{{\color[rgb]{0.2,0.39,0.66}(#1)}}
\newcommand{\todo}[1]{{\color[rgb]{1.0,0.0,0.0}(#1)}}
\newcommand{\hsh}[1]{{\color{green!50!black} Henrik: #1}}
\newcommand{\st}[1]{{\color{red!50!black} Sebastian: #1}}

\newcommand{\ulm}[1]{_{\scaleto{\mathrm{#1}}{3pt}}}
\newcommand\at[2]{\left.#1\right|_{#2}}











\newtheorem{assumption}{Assumption}

\DeclareMathOperator*{\argmax}{arg\,max}
\DeclareMathOperator*{\argmin}{arg\,min}

\newcommand{\swname}[1]{\texttt{#1}}
\newcommand{\ie}{i\/.\/e\/.,\/~}
\newcommand{\eg}{e\/.\/g\/.,\/~}
\newcommand{\cf}{cf\/.\/~}

\newcommand{\fig}{Fig\/.\/~}
\newcommand{\defn}{Def\/.\/~}
\newcommand{\sect}{Sec\/.\/~}
\newcommand{\tabl}{Tab\/.\/~}
\newcommand{\algo}{Algorithm~}
\newcommand{\theo}{Theorem~}

\newcommand{\bnnl}{3 hidden layers}
\newcommand{\bnnn}{50 neurons}
\newcommand{\bnna}{tanh activations}

\newcommand{\capt}[1]{\mdseries{\emph{#1}}}

\newcommand{\videolink}{at \url{https://youtu.be/_d7AqTRjz6g}}
\newcommand{\codelink}{\url{https://github.com/wheelbot/mini-wheelbot}}

\newcommand{\fakepar}[1]{\vspace{0mm}\noindent\textbf{#1.}}

\newcommand{\needref}{\textcolor{red}{[REF]}}

\newcommand{\plotfontsize}{9pt}

%%%%% NEW MATH DEFINITIONS %%%%%

% \usepackage{amsmath,amsfonts,bm}
\usepackage{amsmath,amsfonts}

\usepackage{pifont}


\newcommand{\R}{\mathbb{R}}


\def\va{{\mathbf{a}}}
\def\vg{{\mathbf{g}}}

% Sets
\def\sR{\mathbb{R}}
\def\sC{\mathbb{C}}
\def\sZ{\mathbb{Z}}
\def\sN{\mathbb{N}}
\def\sQ{\mathbb{Q}}

\def\sS{\mathcal{S}}



% Vectors
\def\vzero{{\mathbf{0}}}
\def\vone{{\mathbf{1}}}
\def\vmu{{\mathbf{\mu}}}
\def\vtheta{{\mathbf{\theta}}}
\def\va{{\mathbf{a}}}
\def\vb{{\mathbf{b}}}
\def\vc{{\mathbf{c}}}
\def\vd{{\mathbf{d}}}
\def\ve{{\mathbf{e}}}
\def\vf{{\mathbf{f}}}
\def\vg{{\mathbf{g}}}
\def\vh{{\mathbf{h}}}
\def\vi{{\mathbf{i}}}
\def\vj{{\mathbf{j}}}
\def\vk{{\mathbf{k}}}
\def\vl{{\mathbf{l}}}
\def\vm{{\mathbf{m}}}
\def\vn{{\mathbf{n}}}
\def\vo{{\mathbf{o}}}
\def\vp{{\mathbf{p}}}
\def\vq{{\mathbf{q}}}
\def\vr{{\mathbf{r}}}
\def\vs{{\mathbf{s}}}
\def\vt{{\mathbf{t}}}
\def\vu{{\mathbf{u}}}
\def\vv{{\mathbf{v}}}
\def\vw{{\mathbf{w}}}
\def\vx{{\mathbf{x}}}
\def\vy{{\mathbf{y}}}
\def\vz{{\mathbf{z}}}
\def\vzeta{{\mathbf{\zeta}}}

% Matrix
\def\mA{{\mathbf{A}}}
\def\mB{{\mathbf{B}}}
\def\mC{{\mathbf{C}}}
\def\mD{{\mathbf{D}}}
\def\mE{{\mathbf{E}}}
\def\mF{{\mathbf{F}}}
\def\mG{{\mathbf{G}}}
\def\mH{{\mathbf{H}}}
\def\mI{{\mathbf{I}}}
\def\mJ{{\mathbf{J}}}
\def\mK{{\mathbf{K}}}
\def\mL{{\mathbf{L}}}
\def\mM{{\mathbf{M}}}
\def\mN{{\mathbf{N}}}
\def\mO{{\mathbf{O}}}
\def\mP{{\mathbf{P}}}
\def\mQ{{\mathbf{Q}}}
\def\mR{{\mathbf{R}}}
\def\mS{{\mathbf{S}}}
\def\mT{{\mathbf{T}}}
\def\mU{{\mathbf{U}}}
\def\mV{{\mathbf{V}}}
\def\mW{{\mathbf{W}}}
\def\mX{{\mathbf{X}}}
\def\mY{{\mathbf{Y}}}
\def\mZ{{\mathbf{Z}}}
\def\mBeta{{\mathbf{\beta}}}
\def\mPhi{{\mathbf{\Phi}}}
\def\mLambda{{\mathbf{\Lambda}}}
\def\mSigma{{\mathbf{\Sigma}}}


% Expectation
% \def\eE{\mathop{\mathbb{E}}\limits}
\def\eE{\mathbb{E}}

% Probability
\def\pP{\mathbb{P}}

% Tilde
\def\tf{\tilde{f}}
\def\tS{\tilde{S}}
\def\wtF{\widetilde{\mathcal{F}}}
\def\whR{\widehat{R}}
\def\tvx{\tilde{\mathbf{x}}}
\def\ty{\tilde{y}}


\def\defeq{\overset{\textup{def}}{=}}
% \def\defeq{\overset{.}{=}}
\def\defone{\overset{\text{\ding{172}}}{=}}
\def\deftwo{\overset{\text{\ding{173}}}{=}}
\def\leqone{\overset{\text{\ding{172}}}{\leq}}
\def\leqtwo{\overset{\text{\ding{173}}}{\leq}}
\def\leqthree{\overset{\text{\ding{174}}}{\leq}}
\def\leqfour{\overset{\text{\ding{175}}}{\leq}}
\def\eqone{\overset{\text{\ding{172}}}{=}}
\def\eqtwo{\overset{\text{\ding{173}}}{=}}
\def\eqthree{\overset{\text{\ding{174}}}{=}}
\def\eqfour{\overset{\text{\ding{175}}}{=}}
\def\geqfive{\overset{\text{\ding{176}}}{\geq}}

\renewcommand{\paragraph}[1]{\noindent\textbf{#1}~~}

\newcommand{\ClipMem}{\textit{ClipMem}\@\xspace}

\newcommand{\reb}[1]{{\textcolor{black}{#1}}}
%\newcommand{\new}[1]{{\textcolor{black}{#1}}
%\newcommand{\rebut}[1]{\textcolor{black}{#1}}

\usepackage{hyperref}       % hyperlinks
%\newcommand{\ourtitle}{Open Large Language Models for Privacy Protection}
%\newcommand{\ourtitle}{Open Large Language Models for High-Utility Privacy-Preserving Adaptations}
%\newcommand{\ourtitle}{Differentially Private Adaptations of Large Language Models}
%\newcommand{\ourtitle}{How to Privately Adapt Large Language Models to Downstream Tasks?}
%\newcommand{\ourtitle}{More Private LLM Adaptations that Yield High Performance and }
%\newcommand{\ourtitle}{Private and Local Adaptations of Open LLMs Outperform their Closed Alternatives}
%\newcommand{\ourtitle}{Open and Local LLMs are Required for Private Adaptations and Outperform their Closed Alternatives}
%\newcommand{\ourtitle}{Memorization in CLIP}
\newcommand{\ourtitle}{Captured by Captions: On Memorization and its Mitigation in CLIP Models}
\title{\ourtitle}


% The \author macro works with any number of authors. There are two commands
% used to separate the names and addresses of multiple authors: \And and \AND.
%
% Using \And between authors leaves it to LaTeX to determine where to break the
% lines. Using \AND forces a line break at that point. So, if LaTeX puts 3 of 4
% authors names on the first line, and the last on the second line, try using
% \AND instead of \And before the third author name.


\author{
Wenhao Wang$^{1}$, Adam Dziedzic$^{1}$, Grace C. Kim$^{2}$, Michael Backes$^{1}$, Franziska Boenisch$^{1}$\thanks{Correspondence to boenisch@cispa.de}\\
$^{1}$CISPA, $^{2}$Georgia Institute of Technology
}


\iclrfinalcopy
\begin{document}


\maketitle

% {\color{purple}
% \textbf{Storyline:} \\
% - There is a problem with sharing private data with the LLM provider \\
% - Other works focus on protecting the private training data against the querying party but still have to share the data with LLM provider \\

% \textbf{- We show that local LLMs are always better, anyways.} \\
% 1. in the small epsilon case, local PATE would work best \{Vicuna13B, Openllama13B\} \\
% 2. in the larger epsilon case, DPSGD methods work best \\
% 3. promptDPSGD can handle generation tasks
% }


Vision-Language Models (VLMs) occasionally generate outputs that contradict input images, constraining their reliability in real-world applications. While visual prompting is reported to suppress hallucinations by augmenting prompts with relevant area inside an image, the effectiveness in terms of the area remains uncertain. This study analyzes success and failure cases of Attention-driven visual prompting in object hallucination, revealing that preserving background context is crucial for mitigating object hallucination.

Gaussian Processes (GPs) \citep{kolmogorov1940wienersche,rasmussen2006gaussian} are an important class of stochastic processes used in machine learning and statistics, with use cases including spatial data analysis \citep{liu2021missing}, time series forecasting \citep{girard2002gaussian}, bioinformatics \citep{luo2023diseasegps} and Bayesian optimization \citep{frazier2018tutorial}. GPs offer a non-parametric framework for modeling distributions over functions, enabling both flexibility and uncertainty quantification. These capabilities, combined with the ability to incorporate prior knowledge and specify relationships by choice of kernel function, make Gaussian Processes effective for both regression and classification.

However, GPs have substantial computational and memory bottlenecks. Both training and inference require computing the action of the inverse kernel Gram matrix, while training requires computing its log-determinant: both are $O(n^3)$ operations with sample size $n$. Further, storing the full Gram matrix requires $O(n^2)$ memory. These bottlenecks require scalable approximations for larger datasets.

Structured Kernel Interpolation (SKI) \cite{wilson2015kernel} helps scale Gaussian Processes (GPs) to large datasets by approximating the kernel matrix using interpolation on a set of inducing points. For stationary kernels, this requires $O(n+m \log m)$ computational complexity. The core idea is to express the original kernel as a combination of interpolation functions and a kernel matrix defined on a set of inducing points. However, despite its effectiveness, popularity (over $600$ citations, a large number for a GP paper) and high quality software availability (\cite{gardner2018gpytorch} has 3.5k stars on github), it currently lacks theoretical analysis. A key initial question is, given a fixed error bound for the SKI Gram matrix and use of cubic convolutional interpolation, how many inducing points are required to achieve that error bound? Given the required value of $m$ as a function of $n$, for what error tolerance is $O(n+m\log m)$ still linear? Following this, what do these errors imply for hyperparameter estimation and posterior inference?

\begin{table*}[h]
\centering
\begin{tabular}{|l|l|}
\hline
\textbf{Quantity} & \textbf{Bound} \\
\hline
SKI kernel error & $O(\frac{c^{2d}}{m^{3/d}})$ \\
\hline
SKI Gram matrix error & $O(\frac{nc^{2d}}{m^{3/d}})$ \\
\hline
SKI cross-kernel matrix error & $O(\frac{\max(n,T)c^{2d}}{m^{3/d}})$ \\
\hline
SKI score function error & $O(\frac{\sqrt{p}n^{2}c^{4d}}{m^{3/d}})$ \\
\hline
SKI posterior mean error & $O(c^{2d}\frac{\max(T,n)+\sqrt{Tn}n}{m^{3/d}})$ \\
\hline
SKI posterior covariance error & $O(\frac{Tn^{2}mc^{4d}+\sqrt{Tn}mc^{4d}\max(T,n)}{m^{3/d}})$ \\
\hline
\end{tabular}
\caption{Summary of Theoretical Results when using SKI with convolutional cubic interpolation. This shows the rate at which the error of using SKI (vs the exact kernel) grows as a function of important variables. Here $n$ and $T$ are the train/test sample sizes, $d$ is the dimensionality, $m$ the number of inducing points, $p$ is the number of hyperparameters and $c>0$ is a constant. Most importantly, the Gram matrix error grows linearly with the sample size, exponentially with the dimension while decaying at an $m^{3/d}$ rate in the inducing points.}
\label{table:theoretical_results}
\end{table*}

In this paper, we begin to bridge the gap between practice and a theoretical understanding of SKI. We have three primary contributions: 1) The first error analysis for the SKI kernel and relevant quantities, including the SKI gram matrix's spectral norm error. Based on this we provide \textit{a practical guide to select the number of inducing points}: they should grow as $n^{d/3}$ to control error. 2) SKI hyperparameter estimation analysis. 3) SKI inference analysis: the error of the GP posterior means and variances at test points. We find two interesting results: 1) we identify two dimensionality regimes relating SKI Gram matrix error to computational complexity. For $d\leq 3$, for \textit{any} fixed spectral norm error, we can achieve it in linear time using SKI with a sufficient sample size. For $d>3$, the error must \textit{increase} with the sample size to maintain our guarantee of linear time. 2) For a $\mu$-smooth log-likelihood, gradient ascent on the SKI log-likelihood will approach a neighborhood of a stationary point of the true log-likelihood at a $O\left(\frac{1}{K}\right)$ rate, with the neighborhood size determined by the SKI score function's error, which aside from the response variables grows \textit{linearly} with the sample size when increasing inducing points as we suggested. To obtain this, we leverage a recent result \cite{stonyakin2023stopping} from the inexact gradient descent \cite{daspremont2008smooth,devolder2014first} literature.

% By our downstream analysis, this implies that for small dimensionality and kernels exhibiting desirable regularity conditions, we can have arbitrarily small error in the SKI Gram matrix, in our estimated parameters and in posterior inference all in linear time as long as the sample size is sufficiently large.

In section \ref{sec:related} we describe related work. In section \ref{sec:ski-background} we give a brief background on SKI. In section \ref{sec:important-quantities} we bound the error of important quantities: specifically the SKI kernel, Gram matrix and cross-kernel matrix errors. In section \ref{sec:gp-applications} we use these to analyze the error of the SKI MLE and posteriors. We conclude in section \ref{sec:discussion} by summarizing our results and discussing limitations and future work.
\section{Background and Related Work}
\label{sec:background}
% \subsection{Text Data Summarization}
% % Text summarization is a basic task in NLP domains that aims to compress original articles into a few sentences while retaining essential information and meaning. Overall, text summarization can be divided into extractive summarization and abstractive summarization.


% Text data summarization significantly enhances the efficiency of data processing, storing, and retrieving by extracting and generating concise summaries. For example, it helps search engines better determine the relevance between queries and results~\cite{MarketBrew2024SEO}. Overall, text summarization can be divided into extractive summarization and abstractive summarization.

% % Over the past few years, various approaches have been developed to enhance the performance of news summarization~\cite{zhou-etal-2018-neural-document,dong-etal-2018-banditsum,zhang-etal-2018-neural,t5,pegasus,brio,lewis-etal-2020-bart}. 

% % \subsubsection{Extractive Summarization}
% Extractive summarization is selecting the most relevant sentences from the original text to form a summary. This can ensure the relevance and faithfulness of the summary but may result in a lack of coherence or some redundant information. Early methods often use statistical or graph-based approaches to measure sentence importance. Luhn~\cite{luhn1958automatic} proposed a method based on word frequency statistics to select the most important sentences from a text. And TextRank~\cite{mihalcea2004textrank} is a graph-based ranking algorithm that represents the text as a graph, where nodes represent basic units of the text (such as words or sentences), and edges represent relationships between these units. Through the structure of the graph and the strength of connections between nodes, the TextRank algorithm can evaluate the importance of each node. BanditSum~\cite{dong-etal-2018-banditsum} addressed sentence selection by framing it as a contextual bandit problem. With the development of word embeddings~\cite{mikolov2013word2vec}, many works have started using semantic information to select sentences. For instance, NEUSUM~\cite{zhou-etal-2018-neural-document} and LATENT~\cite{zhang-etal-2018-neural} all encode
% the sentences and documents. Recently, some efforts have also considered improving extractive summarization based on LLMs. LaMSUM~\cite{chhikara2024lamsum} is a multi-level framework or extractive summarization. Parmar et al.~\cite{parmar2024enhancingcoherenceextractivesummarization} improved the extractive summarization capabilities of LLMs through fine-tuning training.


% Abstractive summarization, on the other hand, generates new sentences that cover the key information of the source text. Although this method offers better readability than extractive summarization, it faces challenges in maintaining the relevance and faithfulness of the summary. Initially, abstractive summarization often relies on techniques such as concatenation~\cite{jing2000cut} and rule matching~\cite{genest-lapalme-2012-fully} to generate summaries. With the rise of the Transformer architecture and pre-trained models, some efforts have achieved promising results. Pointer Generator~\cite{see-etal-2017-get}, based on the encoder-decoder architecture, can flexibly copy words from the source text or generate new words, thereby enhancing the accuracy and readability of summaries. T5~\cite{t5} is a pre-trained model that unifies all text processing tasks into a text-to-text format. Through pre-training on large-scale data and multi-task learning, it improves the performance of abstractive summarization. Pegasus~\cite{pegasus} is a pre-trained model designed for text summarization. It achieves pre-training by simulating the deletion of sentences to generate summaries. Brio~\cite{brio} improves summary quality by incorporating evaluation models during training. Recent works~\cite{dead_summarization,zhang2024benchmarking} indicate that LLMs have surpassed traditional sequence-to-sequence neural networks in abstractive summarization and can even achieve human-level performance.


% algo work
% evaluation work
% system level
\subsection{Small Language Model}
% Although LLMs have surpassed traditional neural network models in many NLP tasks, their enormous parameter sizes and computational requirements make them nearly impractical for consumer-grade devices. Additionally, using commercial APIs raises issues related to cost and privacy protection. Therefore, Some studies such as TinyLalma~\cite{zhang2024tinyllama}, Phi2~\cite{microsoft2023phi2}, etc. aim to achieve LLM-like text generation capabilities using smaller models and less data, which we call small language models (SLM). There is no clear definition specifying the parameter count for SLMs. Generally, SLMs are considered capable of running on mobile devices. Therefore, in this paper, we primarily explore models smaller than 8GB in FP16 precision, equivalent to fewer than 4B parameters.


Small language models (SLMs) typically refer to models with fewer parameters and a smaller computational footprint compared to larger LLMs. These include general-purpose and specialized models that can typically be deployed for inference on edge devices, such as mobile phones and PCs. With the increasing focus on general-purpose SLMs, this paper centers on their evaluation, hereafter referred to simply as SLMs. Structurally, both SLMs and LLMs consist of stacked decoder layers from the transformer architecture~\cite{attention_all_you_need}, generating output autoregressively. However, SLMs have fewer decoder layers, attention heads, and smaller hidden dimensions, and they are trained on smaller, high-quality datasets. There is no clear definition specifying the parameter count for SLMs, but they are generally considered capable of running on consumer-grade devices. This paper focuses on models smaller than 8GB in FP16 precision, corresponding to fewer than 4B parameters.

% Like LLMs, SLMs also have instruction-following capabilities. Thus, when using SLMs for text summarization, we need to input the original text along with prompts related to the summary. The model then generates the briefly summary. Fig~\ref{fig:diff_prompt_example} shows some prompt templates. By modifying the prompts, we can obtain summaries in different styles, making SLMs more flexible than traditional text summarization models.


%可能需要增加内容
\subsection{Text Summarization Evaluation}
Currently, there are many works evaluating models for text summary generation, covering aspects such as relevance, coherence, and factual consistency~\cite{huang-etal-2020-achieved,fabbri2021summeval,dead_summarization,zhang2024benchmarking,liu2023benchmarking}. \citet{huang-etal-2020-achieved} used ROUGE and human evaluation to evaluate 10 representative summarization models, finding that extractive summarizers performed better in human evaluations. In the Summeval benchmark~\cite{fabbri2021summeval}, the authors analyzed existing metrics and used human evaluation methods to assess 23 models. \citet{dead_summarization} evaluated the text summarization performance of LLMs and found that they were more favored by evaluators. \citet{zhang2024benchmarking} used human evaluation to evaluate LLMs, discovering that instruction tuning had a greater impact on performance than model size. INSTRUSUM~\cite{liu2023benchmarking} evaluated instruction controllable summarization of LLMs and found them still unsatisfactory in summarizing text based on complex instructions.

Although SLMs have shown strong capabilities, comprehensive comparisons of their text summarization abilities are still lacking. Tiny Titans~\cite{tiny_titan} assessed the summarization performance of SLMs in meeting summarization, but it focused only on older models, excluding newer series like the Phi3 and Llama3. Moreover, its evaluation was limited to the ROUGE metric, which is not sufficiently comprehensive. While \citet{lu2024smalllanguagemodelssurvey} evaluated SLMs across various tasks, news summarization was notably absent. Therefore, this paper aims to provide a thorough evaluation of SLM performance in news summarization.




% 没有完整定义
% sml
%system level
% news summarization
\begin{figure}[t!]
    \centering
    \includegraphics[width=0.45\textwidth]{images/method_visual-crop.pdf}
    \caption{Process of Visual Attention Evaluation.}
    \label{img1}
\end{figure}

\subsection{API Prompting}
API Prompting~(sketched in Figure~\ref{api}) is a Visual Prompting method that highlights important parts in an image using a Visual Attention Heatmap derived from a Vision-Language Model~\citep{api}. The Attribution Map, representing the contribution of image tokens to model outputs, is extracted from a VLM (referred to as Heatmap VLM or H-VLM), convolved, resized to match the image size, and then overlaid on the original image.

Following the study by \citet{api}, Vision-Transformer-based CLIP and LLaVA are employed as Heatmap VLMs. The methods for extracting Visual Attention Attribution Maps from each model are described below.

\paragraph{CLIP Attribution Map}
CLIP computes similarity between text and image representations, and the Attribution Map \( \Psi \) is obtained by decomposing the similarity function \( \text{sim}(\hat{I}, \hat{T}) \). %The image representation \( \hat{I} \) is expressed as:
% \begin{equation}
%    \begin{aligned}
%        \hat{I} &= \mathcal{L}([Z^{0}_{\text{cls}}])
%         + \sum_{\ell=1}^{L} \mathcal{L}([\text{MSA}^{\ell}(Z^{\ell-1})]_{\text{cls}}) \\
%        &\quad + \sum_{\ell=1}^{L} \mathcal{L}([\text{MLP}^{\ell}(\hat{Z}^{\ell})]_{\text{cls}}).
%    \end{aligned}
% \end{equation}

Since later MSA layers greatly impact image representation~\citep{clipdec}, the similarity function is approximated as:
\begin{equation}
   \begin{aligned}
\text{sim}(\hat{I}, \hat{T}) \approx \text{sim}\left(\sum_
{\ell=L'}^{L} \mathcal{L}(\text{MSA}^{\ell}([Z^{\ell-1}]))_{\text{cls}}, \hat{T}\right).
\end{aligned}
\end{equation}
To filter out irrelevant regions, a complementary Attribution Map \( \Psi^{\text{comp}} \) is introduced:
\begin{equation}
   \begin{aligned}
   \Psi_{i,j}^{\text{comp}} &\triangleq 1 - \text{sim}(\mathcal{L}(Z_{t}^{L}), \hat{T}), \\
&\quad \text{where} \quad t = 1 + j + P \cdot (i - 1).
\end{aligned}
\end{equation}
Combining both maps, the final CLIP Attribution Map is defined as:
\begin{equation}
   \begin{aligned}
   \Psi = \Psi^{\text{cls}} + \Psi^{\text{comp}} - \Psi^{\text{comp}} \odot \Psi^{\text{cls}}.
\end{aligned}
\end{equation}

\paragraph{LLaVA Attribution Map}
LLaVA can provide an Attribution Map \( \Psi \) using Multi-Head Self-Attention (MSA) weights between output text tokens and image tokens. The Attribution Map is computed by averaging over all output tokens and attention heads:
\begin{equation}
   \begin{aligned}
\Psi_{i,j} &\triangleq \frac{1}{MH} 
  \sum_{m=1}^{M} 
  \sum_{h=1}^{H} 
  A_{m,t}^{(\bar{L},h)}, \\
&\quad \text{where} \quad 
t = j + P \cdot (i - 1).
\end{aligned}
\end{equation}
Here, \(M\) is the number of output tokens, \(H\) is the number of attention heads, \(P\) is the number of patches per image side, and \(A^{(\bar{L},h)}\) represents cross-attention between output text and image tokens at layer \(\bar{L}\) and attention head \(h\).

\subsection{Background Role Examination}
To assess the necessity of background information for object recognition, ground truth segmentation data is used as a Heatmap during API Prompting and the accuracy of output is evaluated (hereafter referred to as API - Seg.). Binary segmentation masks, overlaid in gray are input into VLMs to evaluate their impact on output accuracy. If POPE response accuracy remains unchanged, background information is deemed unnecessary.

\subsection{Minimum Cutoff}
Minimum cutoff redefines the minimum value in segmentation or Visual Attention Heatmap based on a threshold. Since a threshold of 0.5 showed improvement in Table~\ref{table2}, values below 0.5 in the cutoff are replaced with 0.5, refining segmentation granularity.

\begin{table*}[t]
\centering
\begin{tabular}{lllrrrrr}
\toprule
\textbf{Dataset} & \textbf{Model} & \textbf{Prompting} & \textbf{Acc.} & \textbf{Prec.} & \textbf{Rec.} & \textbf{TNR} & \textbf{F1} %& \textbf{Yes (\%)} 
\\ \cmidrule(lr){1-8}
\multirow{7}{*}{\textbf{MSCOCO}} & \multirow{7}{*}{LLaVA} & w/o prpt. & 86.23 & 84.21  & 89.19 & 83.27 & 86.63 \\ %& 52.96 \\
& & API~(CLIP) & \ensuremath{\blacktriangle}86.52 & \ensuremath{\blacktriangle}84.78 & \ensuremath{\triangledown}89.02 & \ensuremath{\blacktriangle}84.02 & \ensuremath{\blacktriangle}86.85 \\ %& \ensuremath{\blacktriangle}52.50  \\
& & API~(CLIP) w Cutoff & \ensuremath{\blacktriangle}88.59 & \ensuremath{\blacktriangle}85.84 & \ensuremath{\blacktriangle}92.43 & \ensuremath{\blacktriangle}84.75 & \ensuremath{\blacktriangle}89.01 \\
 & & API~(LLaVA) & \ensuremath{\triangledown}86.11 & \ensuremath{\blacktriangle}84.72 & \ensuremath{\triangledown}88.12 & \ensuremath{\blacktriangle}84.10 & \ensuremath{\triangledown}86.39 \\ %&  \ensuremath{\blacktriangle}52.01 \\
 & & API~(LLaVA) w Cutoff & \ensuremath{\blacktriangle}87.98 & \ensuremath{\blacktriangle}85.10 & \ensuremath{\blacktriangle}92.09 & \ensuremath{\blacktriangle}83.88 & \ensuremath{\blacktriangle}88.46 \\
  & & API - Seg. & - & - & \ensuremath{\triangledown}71.78 & - &  - \\ %& - \\
  & & API - Seg. w Cutoff & - & - & \ensuremath{\blacktriangle}89.24 & - &  - \\ %& - \\
 \bottomrule
\end{tabular}
\caption{POPE results on MSCOCO datasets with API Prompting.}
\label{table1}
\end{table*}

\begin{table*}[h!]
\centering
\begin{tabular}{llrrrr}
\toprule
%\toprule
\multirow{2}{*}{\textbf{H-VLM}} & \multirow{2}{*}{\textbf{Output}} & \multicolumn{4}{c}{\textbf{Visual Attention Alignment}} \\ %\cline{2-9} 
& & \textbf{Prec.} & \textbf{Rec.}  & \textbf{IoU}    & \textbf{MSE}   \\ 
%\cmidrule(lr){1-9}
\cmidrule(lr){1-6}
%Overall   & 0.1270   & 0.8318 & 0.1092 & 0.2920  & 0.1010   & 0.6523 & 0.0767 & 0.1337 \\ \hline
%\multirow{3}{*}{CLIP} & - & 12.70 & 83.18 & 10.92 & 29.20 \\
\multirow{2}{*}{CLIP}& Correct~(87\%)  &\ensuremath{\blacktriangle}13.52   & \ensuremath{\blacktriangle}83.94 & \ensuremath{\blacktriangle}11.68 & \ensuremath{\blacktriangle}28.46  \\
& Incorrect~(13\%) & 6.09   & 77.02 & 4.78 & 35.22 \\
%\multirow{3}{*}{LLaVA} & - & 10.10 & 65.23 & 7.67 & 13.37 \\
\multirow{2}{*}{LLaVA}& Correct~(86\%)  & \ensuremath{\blacktriangle}10.62  & 64.62 & \ensuremath{\blacktriangle}8.10 & 13.60 \\ 
& Incorrect~(14\%) & 6.17   & \ensuremath{\blacktriangle}69.78 & 4.44 & \ensuremath{\blacktriangle}11.60 \\
\bottomrule
%\bottomrule
\end{tabular}
\caption{Alignment of CLIP/LLaVA Visual Attention.}
\label{table3}
\end{table*}

\subsection{Evaluation of Visual Attention Alignment}
To assess how well visual attention focuses on target objects, Precision, Recall, Intersection over Union~(IoU), and Mean Squared Error~(MSE) are computed between the object segmentation data and the Visual Attention Heatmap as depicted in Figure~\ref{img1}. 
Visual Attention Heatmaps are converted into binary arrays using thresholds set to the average value of each heatmap.


% \begin{figure}[t]
%      \centering
%     \includegraphics[width=0.47\textwidth]{images/cutoff-crop.pdf}
%      \caption{Example of Cutoff Segmentation Annotation}    \label{clip}
%  \end{figure}
\section{Empirical Evaluation}

\subsection{Experimental Setup}

\paragraph{Models and training.}
We build our experiments on OpenCLIP~\citep{cherti2023}, an open-source Python version of Open-CLIP~\citep{ilharco_gabriel_2021_5143773}. \reb{The standard architecture used for the experiments builds on ViT-Base, but we also include experiments using ViT-Large.}
We train the model on the COCO dataset~\citep{cocodataset}.
Since COCO is much smaller than OpenCLIP's standard training datasets, we reduce the training batch size to 128 and increase the epoch number from 32 to 100 to achieve similar performance. All other settings strictly follow OpenCLIP. 
For training DINO, as an example of an SSL vision encoder, we follow the default setting of~\citet{caron2021dino}. The supervised model is trained as  a multi-label classifier, also based on ViT-Base (with an additional fully connection layer) based on the first-level annotation captions in the COCO dataset.
A full specification of our experimental setup is detailed in \Cref{app:setup}. 
Additional experiments for measuring memorization on the BLIP~\citep{li2022blip} model are presented in \Cref{app:BLIP}.

\paragraph{Datasets.} We use COCO~\citep{cocodataset}, CC3M~\citep{sharma2018conceptual}, \reb{and the YFCC100M~\citep{thomee2016yfcc}} datasets to pre-train the OpenCLIP models.
% \todo{@Wenhao, we don't use the whole set but a subset, right? Please specify here (using track changes) what you used.}\wenhao{What I used is cc3m-wds, provided on hugging face here (https://huggingface.co/datasets/pixparse/cc3m-wds). It contains all 2.91M images and captions. I randomly generated a 75000 list to pick up 75000 samples out of 2.91M samples}  
For the CC3M dataset, we randomly sample 75000 examples from the total of 2.91M data points. %\todo{@Franzi: check}
We evaluate the models by testing the linear probing accuracy on ImageNet~\citep{deng2009imagenet} with an added classification layer trained on top of the output representations. 
\reb{We use the YFCC100M dataset to simulate an infinite data regime, \ie using a single training run where no data point is repeated whereas we train iteratively using CC3M and COCO.}

\paragraph{Measuring memorization.} We follow~\citet{wang2024memorization} to approximate our \ours. Since training a separate pair of models for every data point whose memorization we aim to measure would be computationally intractable, we measure memorization of multiple data points at the same time. Therefore, we divide the original training set in four subsets: (1) $S_S$, data points that both model $f$ and $g$ were trained on, (2) $S_C$, data points used only for training $f$, (3) $S_I$, data points used only for training $g$, and (4) $S_E$, external "test" data points that none of the models was trained on.
Note that $|S_C|=|S_I|$, such that $f$ and $g$ have the same number of training data points in total. 
For our experiments, following a similar approach to~\citet{wang2024memorization}, we want to strike a balance when choosing the size of $S_C$. If the size is too large, then $f$ and $g$ might differ too much and not yield a strong memorization signal, but if it is too small, we would only have a memorization signal for too few data points.
Concretely, for COCO and CC3M, we set $|S_S|=65000$ and $|S_C|=|S_I|=|S_E|=5000$. Memorization is reported as an average over all data points in $S_C$ for model $f$, or per individual data point in $S_C$.
% we  80,000 samples are randomly extracted from the CC3M and COCO datasets respectively as the train data. We divide the training data set $S$ into four disjoint partitions and use 81.25\% of the train data, \ie 65000 samples as shared training samples $S_S$ between model $f$ and $g$. The next 6.25\% of samples, \ie 5000 are used as candidates $S_C$ to evaluate memorization. We add those to the training data of $f$ only, and another 6.25\%, \ie 5000 samples, are used as an independent set $S_I$, on which we only train $g$. We also use the remaining 6.25\%, \ie 5000 samples that neither $f$ nor $g$ are trained on as extra set $S_E$.
% We measure our \ours on the candidates $S_C$ and report their average memorization scores as an aggregate metric.

\paragraph{Generating captions and images.}
For generating additional captions for the training images, we rely on GPT-3.5-turbo. For each input image, we provide the representation produced by our trained OpenCLIP model and ask GPT to generate five new captions.
Generated sample captions are presented in \Cref{fig:image_text_sample}. 
To generate additional images for the COCO dataset, we use Stable Diffusion v1.5 to generate five new images, one corresponding to each of the five per-image captions in the COCO dataset. Sample generated images are presented in \Cref{fig:text_image_sample}.

\begin{figure}[t]
    \centering
        % Subfigure with the image
    \begin{subfigure}[b]{0.45\textwidth}
        \centering
        \includegraphics[width=1.0\textwidth]{image/4_set_hist.pdf} % Replace with your image file
        \caption{Memorization scores across data subsets.}
        \label{fig:memorization_subsets}
    \end{subfigure}
    %\hfill
    % Subfigure with the table
    \begin{subfigure}[b]{0.45\textwidth}
        \centering
        \tiny
        \begin{tabular}{cccc}
\toprule
 &  Clean $S_C$ & Poisoned $S_C$ \\
\midrule
Clean Model &  0.438 & N/A\\
Poisoned Model &  0.440 & 0.586\\
%Poisoned Model (Mis-captioned)   \\
\bottomrule
\end{tabular}    %         \begin{tabular}{cccc}
    % \toprule
    %  & Clean Model & Poisoned Model (Clean) & Poisoned Model (Mislabeled) \\
    % \midrule
    %   Top 10\% Most Memorized  & 0.678 & 0.671& 0.707\\
    %   $S_C$   & 0.438 & 0.440&0.586\\
    %   \bottomrule
    % \end{tabular}
    \caption{\vspace{3em}Average \ours scores.}
    \label{tab:memorization_poisoning}
    \end{subfigure}
    \caption{\textbf{Memorization with \ours}. We train a CLIP model on COCO using standard image cropping and no text augmentations. (a) We present memorization scores according to \ours per data subset. The significantly higher scores for $S_C$ compared to $S_S$ indicate that $f$ memorizes $S_C$. (b)~We also study how inserting training samples with imprecise or incorrect captions ("mis-captioned") affects memorization. We refer to the model trained with correct captions as \textbf{Clean Model}, and the model trained with $S_C$ containing 4500 standard canaries (\textbf{Clean}) and 500 mis-captioned (\textbf{Mis-captioned}) as \textbf{Poisoned Model}.    
    We report \ours over the different subsets of candidates. We observe that the mis-captioned samples experience a significantly higher memorization while the memorization of the clean data points is (almost) not affected.}
    % We first measure \ours scores for a ``clean'' model trained on correctly labeled data, reporting the average for the top 10\% most memorized samples and the overall average for all 5000 candidates in $S_C$. \textbf{(Clean Model)}. Next, we assign incorrect captions to 500 data points in $S_C$ and retrain the model. For this \textit{poisoned} model, we measure \ours scores for both the remaining 4500 clean (\ie correctly labeled) samples and the 500 mislabeled samples. We report the results for the top 10\% most memorized clean data, consisting of 450 samples in \textbf{Poisoned Model (Clean)}, and mislabeled data, consisting of 50 samples in \textbf{Poisoned Model (Mislabeled)}. Additionally, we provide the overall average for both clean and mislabeled data (4500 and 500 data points, respectively). \textcolor{red}{We observe that... @Wenhao, in this figure, we have to make Figure (a) and fill the values into Table (b), maybe some data points can already be obtained from your Figure 6 in the report (standard training, so no custom batching).} $S_C$, $S_I$, $S_E$, and $S_S$.}
    % \label{fig:memorization_insights}
\end{figure}
\subsection{Studying Memorization using \ours}
We first set out to analyze the general memorization in CLIP in order to identify which data points are memorized.
To do this, we quantify \ours over the different training subsets. Our results are presented in \Cref{fig:memorization_subsets}. 
%highlight that our \ours reports sensible memorization. \adam{you should not use words like "sensible"}
In particular, we observe that \ours for $S_C$, the data points only used to train model $f$, is significantly higher than for $S_S$, the data points shared between the two models. Memorization for $S_S$ is comparable to that for $S_E$, \ie the external data not seen during training, indicating that $f$ does not memorize these samples. The data in $S_I$ causes negative \ours scores, indicating that this data is memorized by $g$, not by $f$. This is the expected behavior according to the definition of our metric.
\reb{In \Cref{app:modelsize}, we additionally highlight that memorization increases with model size, \ie CLIP based on ViT-Large has a higher overall memorization with an average of $0.457$ while CLIP based on ViT-Base only reaches $0.438$ on average.}

Additionally, we analyze individual data points according to their reported CLIPMem.
We give examples of highly memorized data points in CLIP in \Cref{fig:examples} and more highly vs. little memorized samples in Figures~\ref{fig:examples_memorized_5_caption_least},\ref{fig:examples_memorized_5_caption_most},\ref{fig:examples_memorized_1_caption_least}, and \ref{fig:examples_memorized_1_caption_most}
% \Cref{fig:examples_memorized_5_caption_least}, \Cref{fig:examples_memorized_5_caption_most}, \Cref{fig:examples_memorized_1_caption_least}, and \Cref{fig:examples_memorized_1_caption_most} 
in \Cref{app:examples}. 
Overall, the samples with high \ours, \eg in \Cref{fig:examples} seem to be difficult examples and examples with imprecise or incorrect captions whereas the samples with low \ours are simpler and (potentially consequently) more precisely captioned.
\reb{In \Cref{app:onerun}, we show that these findings also hold when we operate in the \textit{infinite data regime}, \ie when we perform only a single training run where no data point is repeated.}

Motivated by this insight and by observations from supervised learning where it was shown that models can memorize random labels~\citep{zhang2016understanding} and where mislabeled data experiences highest memorization~\citep{feldman2020does}, we test if the same effect can also be observed in CLIP. 
Therefore, we "poison" our CLIP's training data by randomly shuffling the captions among 500 of the 5000 candidate data points in $S_C$.
Thereby, these 500 data points are "mis-captioned". We train a model based on this data and see that the mis-captioned examples experience significantly higher memorization (\ours of 0.586) compared to the "clean" data points  (\ours of 0.440). Even though CLIP trains using a contrastive training objective, the memorization of clean data points is not significantly affected by training the model with the mis-captioned examples, as we can see by their \ours that is 0.438 on the clean model and 0.440 on the poisoned model.




\begin{figure}[t]
    \centering
    \begin{subfigure}[b]{0.32\textwidth}
        \centering
        \includegraphics[width=\textwidth]{image/sslmem_sc_ss.pdf}
        \caption{SSLMem (Img).}
    \end{subfigure}
    \begin{subfigure}[b]{0.32\textwidth}
        \centering
        \includegraphics[width=\textwidth]{image/sslmem_sc_ss_text_1.pdf}
        \caption{SSLMem (Text).}
    \end{subfigure}
        \begin{subfigure}[b]{0.32\textwidth}
        \centering
        \includegraphics[width=\textwidth]{image/clipmem_sc_ss.pdf}
        \caption{\ours.}
    \end{subfigure}
    % \begin{subfigure}[b]{0.24\textwidth}
    %     \centering
    %    \includegraphics[width=\textwidth]{image/ssl_sslnaive_clip.pdf}
    %     \caption{All.}
    % \end{subfigure}
    \caption{\textbf{Measuring memorization on individual modalities is not able to extract a strong signal.} (a)--(b) We measure SSLMem~\citep{wang2024memorization} on the individual encoders of our CLIP model trained on COCO. (c) Our \ours extracts a stronger memorization signal by using both modalities in CLIP jointly. 
    %(d) We compare all scores, including the naive adaptation of SSLMem to CLIP where we sum up the SSLMem for each sample over both modalities. 
    %In neither case, strong memorization can be measured for $S_C$ with SSLMem. In contrast, our \ours reports a strong signal for memorization.
    }
    \label{fig:ssl_vs_clip}
\end{figure}
\subsection{Measuring Memorization in one Modality does not Yield a Strong Signal}
\label{sub:sslmem_not_for_clip}
To understand how important it is to take both modalities into account in our definition of \ours, we set out to evaluate whether existing practical methods to measure memorization over uni-modal encoders~\citep{wang2024memorization} yield a sufficiently strong memorization signal in CLIP.
Therefore, we apply their SSLMem to the individual encoder parts of CLIP. Since SSLMem relies on augmentations of the encoder input, we use image crops, like during CLIP training for the vision encoder, and the 5 COCO captions as augmentations for the text, like in~\citep{fan2023}. 
% We also combine, \ie sum up, the signal from the text and image encoder as a naive baseline on how one could implement a memorization metric in CLIP. 
Our results in \Cref{fig:ssl_vs_clip} 
highlight that SSLMem and its naive adaptation to CLIP fail to yield a strong signal for memorization. In particular, there is a high overlap in scores between the non-memorized samples from $S_S$, and candidate examples for memorization $S_C$. Additionally, the highest reported memorization scores for $S_C$ go up to around 0.65 (for SSLMem on the vision encoder) and 0.73 (for SSLMem on the text encoder). 
In contrast, our new \ours is able to get a distinct signal for the candidates $S_C$ with respect to $S_S$ and reports a much higher memorization of 0.91. Thereby, our \ours prevents under-reporting the actual memorization in CLIP.

% Furthermore, we present the overlap in the number of samples reported as memorized by SSLMem on the different modalities with the highest 500 memorized samples reported by CLIPMem, see for example \Cref{fig:examples}.
% In total SSLMem on the vision encoder identifies 105/500 samples that are highly memorized according to CLIPMem, whereas SSLMem identifies 143/500 samples when applied to the text encoder.
% The overlap between the 500 samples reported as highly memorized by SSLMem on the text and on the image encoder is 389/500. This again indicates that using only SSLMem on an individual modality is insufficient. 
% \adam{However, it even leads to different samples being selected by CLIPMem vs SSLMem while the overlap between the selected data points by SSLMem for vision and text encoder is quite substantial.}
% \textcolor{red}{This indicates that...}

\subsection{Memorization between Modalities}
Our results in \Cref{fig:ssl_vs_clip} indicate that memorization is higher in CLIP's text encoder than in the image encoder (the average SSLMem on $S_C$ in the text encoder is $0.209$ vs. $0.168$ in the image encoder).
To provide further insights into how memorization behaves between the modalities in CLIP, we first analyze the use of augmentations.
We compare five cases: (1) no additional augmentations beyond the baseline (image cropping), \reb{(2) generating one image using a diffusion model for a given original caption,} 
(3) generating five variations of each image using a diffusion model and randomly selecting one for each training iteration while keeping the caption fixed, (4) using the original image but randomly selecting one of the five COCO captions for each training iteration, and (5) randomly pairing each of the five generated images with one of the five COCO captions.

\begin{wraptable}{r}{0.5\textwidth}
\vspace{-0.3cm}
\tiny
\addtolength{\tabcolsep}{0pt}
    \centering
        \caption{Impact of augmentations.}
        \vspace{-0.1cm}
   \scalebox{0.9}{\begin{tabular}{ccc}
    \toprule
    Case & \ours & Lin. Prob. Acc. (ImageNet)  \\
    \midrule
      1 Image, 1 Caption & 0.438 & 63.11\% $\pm$ 0.91\%\\
      %5 Images (generated), 1 Caption & 0.477& 59.92\% $\pm$ 1.04\%\\
      \reb{1 Image (generated), 1 Caption} & \reb{0.428} & \reb{63.97\% $\pm$ 0.79\%} \\
      \reb{5 Images (generated), 1 Caption} & \reb{0.424} & \reb{64.60\% $\pm$ 0.82\%} \\
      1 Image, 5 Captions & 0.423 & 64.88\% $\pm$ 0.83\%\\
      5 Images (generated), 5 Captions & 0.417& 64.79\% $\pm$ 0.99\%\\
      %\begin{tabular}[c]{@{}c@{}} 5 Images (generated), 1 Caption\\ (6000 mis-captioned sample removed)\end{tabular}&0.424&64.60\% $\pm$ 0.82\%\\
      \bottomrule
    \end{tabular}}
    \label{tab:augmentations}
\vspace{-0.4cm}
\addtolength{\tabcolsep}{0pt}
\end{wraptable}
As shown in \Cref{fig:examples_memorized_5_caption_most}, there is quite a variability in the COCO captions for the same sample. Hence, some images might not fit well with the chosen training caption. This imprecise captioning can cause an increase in memorization.
We observe that the effect is mitigated when using the 5 images with the 5 captions (5th case, see \Cref{tab:augmentations}). 
This phenomenon results most likely from the increased number of possible image-text pairs (25), such that individual incorrect or imprecise pairs are not seen so often during training. \reb{For the third case, \ie row three in \Cref{tab:augmentations}, we generate five images with a diffusion model based on all five captions per image from the COCO dataset. However, as we only use the first caption during training, this would introduce many mis-captioned images which significantly lowers performance and increases memorization. To avoid this problem, we removed 6000 mis-captioned samples.}

Our results in \Cref{tab:augmentations} highlight that augmenting text during training reduces memorization and increases performance more than augmenting images. However, applying augmentations of both text and images strikes the right balance between the reduction in memorization and the increase in performance. In fact, applying both augmentations reduces memorization most significantly. 
Overall, these results indicate that memorization in CLIP's is tightly coupled to the captions assigned to the training images with imprecise captions having a destructive effect on CLIP performance and memorization.

% In contrast, memorization increases when using 5 diffusion model-generated images with one caption. 
% This is quite surprising since in the latter case, the linear probing accuracy drops by more than 4\% and one would usually expect that with lower model performance, there should also be lower memorization~\citep{feldman2020does}.
% The drop in ImageNet linear probing accuracy stems most likely from a distribution shift introduced by using diffusion generated instead of natural images.
% Regarding the increase in memorization, we hypothesize that it is linked to image-caption mis-matches during training:
% while we generate the 5 images with the diffusion model based on all 5 COCO captions, we only use the first caption during training. 
% To avoid these problems, we implement extra experiments that remove 6000 mis-captioned samples from the images generated by the diffusion model. The results in the last 2 row of \Cref{tab:augmentations} align with the trend that augmenting either text or image during training reduces memorization significantly.
% \adam{The image is generated per caption, however, in the training of the CLIP model we only select a single caption.
% The fewer times we see the mis-match image-caption pair, the less we memorize it in the case 5 images and 5 captions. Also, in this setup, we train for the same number of iterations, but in the 5 images and 5 captions case, we train on more samples, which improves generalization and leads to higher linear probing accuracy.
% Potential change in the experiment for future: for the case 5 images 1 caption - take the embedding for the caption from CLIP and then noise it 5 times and give these 5 slightly different version to generate from Stable Diffusion.
% }
% We hypothesize that using multiple images for the same caption further enforces memorization due to CLIP's contrastive learning objective. 
% When the exact same caption is learned with multiple images, they will still be considered as negative pairs and pushed away, hence, creating a wider region in the embedding space for the particular sample, causing higher memorization. 
% The same does not happen in the \textit{5 caption, 1 image} case since the image is still augmented through the cropping, and hence not exactly the same with the same caption.\todo{@Adam, please review if the last sentences make sense. Actually, it's really hard to process them.}
%%%% THE FOLLOWING ARE INTERESING RESULTS. BUT I HAVE NO WAY TO EXPLAIN THEM.....
% We also analyze the overlap between the highest memorized samples between case 1 and case 2 and case 1 and case 3, by looking at what is the percentage of overlap within the data points that are in the 10\% highest memorized ones in the respective cases.
% The overlap in the 10\% highest memorized samples between case 1 and case 2 is 66 out of 500 (13.2\%), and the overlap between case 1 and case 3 is 107 out of 500 (21.4\%).
% \textcolor{red}{This indicates that...}
%We show further insights on memorization between modalities in \Cref{app:modality}.
%Finally, in \Cref{app:captions}, we show that captions generated by GPT3.5 have the same effect like the original COCO captions on memorization and linear probing accuracy (\Cref{tab:coco_cpt_train}), and that using each caption at an equal frequency during training improves downstream performance (\Cref{tab:multi_caption_training}).


 


\subsection{Relation to CLIP Memorization to (Self-)Supervised Memorization}\label{chpt:clip_ssl_sl}
We further provide insights on whether CLIP's memorization behavior is more alike to the one of supervised learning or SSL.
This question is highly interesting since the captions in CLIP can be considered as a form of labels, like in supervised learning, whereas the contrastive training objective on the dataset resembles more SSL.
We perform two experiments to gain a better understanding of the memorization behavior of CLIP with respect to supervised learning and SSL.

First, we compare an SSL vision encoder pair $f$ and $g$ with the same architecture as CLIP's vision encoder but trained from scratch on COCO using DINO, 
%\todo{@Wenhao, check if correct: in the appendix you talk about DINO, so I am wondering if this SSL encoder is not trained with DINO?}\wenhao{This is trained with DINO, I correct this},
\ie standard SSL training. We train $f$ and $g$ using the same candidates as the pair of CLIP models in our previous experiments.
Then, we use the SSLMem metric from~\citet{wang2024memorization} to quantify memorization in the CLIP vision encoder and the SSL encoder, respectively.
The CLIP vision encoder has a significantly lower SSLMem than the SSL encoder (0.209 vs. 0.279). 
%\todo{these below were the numbers first, but this would not fit with \Cref{fig:ssl_vs_clip}(a), hence I took the number from there!}
%using the same candidates as for the CLIP models. Then, we used the SSLMem metric~\cite{wang2024memorization} to detect the highest memorized samples in both encoders (the 0.168 and 0.279 for the CLIP and the SSL encoder respectively. 
Hence, CLIP vision encoders experience lower SSL memorization than SSL trained encoders.
%since this metric focuses primarily on atypical examples whereas the memorization in CLIP also stems from mislabeling (caused by, \eg incorrect or imprecise captions).\todo{@Franzi: check the last sentence.}
%\todo{@Adam, help me think of this finding. One might argue that CLIP is just not made for this metric... SSLMem might just be under-reporting on CLIP. Adam: here, we just have to state that with respect to the CLIP vision encoder and not the whole CLIP model.}
%This suggests that overall memorization is \textcolor{red}{higher/lower/same} for CLIP over SSL-trained models. Looking at the top 10\% memorized samples, we can report an average SSLMem of 0.421 and 0.798 for the CLIP and the SSL encoder respectively.
To further investigate the difference, we also report the overlap between the top 10\% memorized samples between the two models, measured according to SSLMem. With an overlap of only 47 out of 500 (9.4\%) samples, we find that CLIP memorizes significantly different samples than SSL encoders.
\citet{wang2024memorization} had performed a similar experiment on SSL vs. supervised learning and found that the two paradigms also lead to different samples being memorized. 
While this is, on the one hand, an effect of the different objective function, the difference between the memorized samples in CLIP and SSL is likely also closely connected to the additional captions that CLIP takes into account. While SSL-trained encoders can memorize atypical images, CLIP encoders can memorize typical images when they have an atypical, imprecise, or incorrect caption.

\begin{wrapfigure}{r}{0.4\textwidth}
\vspace{-0.5cm}
\begin{center}
\centerline{\includegraphics[width=0.95\linewidth]{image/unitmem_vit_clip_dino_supervised.pdf}}
\vspace{-0.2cm}
\caption{
\label{fig:unitmem}
\textbf{UnitMem metric: CLIP is between supervised and SSL models.
}
}
\end{center}
\vspace{-0.8cm}
\end{wrapfigure}
Additionally, we compare the memorization behavior of CLIP against supervised and SSL-trained models on the neuron-level.
Therefore, we train two additional ViT-Base models on COCO using supervised training and SSL training with DINO. Then, we apply the UnitMem metric~\citep{wang2024localizing} to measures how much individual neurons memorize individual samples from the training data. A high UnitMem suggests that neurons highly memorize individual data points instead of groups/classes of points. 
It had been shown that supervised learning causes neurons in lower layers to experience low UnitMem, \ie being responsible for learning joint groups of data points, while neurons in later layers highly memorize individual data points.
In contrast, for SSL, UnitMem was shown to remain relatively constant over layers with neurons in lower layers also being able to memorize individual data points. This difference was attributed to the different objective functions where supervised learning's cross entropy loss pulls together data points from the same class, whereas SSL's contrastive loss leads to individual data points being pushed away from each other~\citep{wang2024localizing}.
Our results in \Cref{fig:unitmem} highlight that CLIP, in terms of its memorization behavior, is between supervised learning and SSL. At the lower layers, it is much less selective than models trained with SSL, \ie it focuses on groups of data points rather than memorizing individual data points, similar to supervised learning.
%\todo{Write about the training of the SSL Model with COCO and the supervised model with COCO.}\wenhao{I am not sure why we need to write about the training of the SSL Model with COCO and the supervised model with COCO, this is already mentioned at beginning of chapter 4.1}
Yet, in later layers, CLIP becomes more selective than SSL, \ie it memorizes individual data points more in individual neurons, but still less than supervised learning which there has a very high average per-layer UnitMem.



% \begin{figure}[t]
%     \centering
%     \begin{subfigure}[b]{0.45\textwidth}
%         \centering
%         \centering
%     \includegraphics[width=0.95\linewidth]{image/caption_num.pdf}
%     %\caption{\textbf{More captions lowers memorization and increases performance.}}
%     \caption{
%     Influence of number of captions.
%     }
%     \label{fig:num_captions}
%     \end{subfigure}
%         \hfill
%     \    \begin{subfigure}{0.49\columnwidth}
%     \centering
%     \includegraphics[width=0.95\linewidth]{image/sample_remove_random_clipmem.pdf}
%     %\caption{\textbf{Removing memorized samples increases performance.}}  
%     \caption{Removing memorized samples.}  
%     \label{fig:remove_samples}
%     \end{subfigure}
%     \caption{\textbf{Relationship between memorization and generalization.} (a) We train CLIP models based on the COCO dataset and analyze the relationship between linear probing accuracy and memorization, based on the number of captions used during training. With more captions, memorization decreases while accuracy increases. \textbf{(b)} We compare the effects of removing random samples versus removing the highest memorized ones according to \ours. Removing according to CLIPMem has a stronger influence on the linear probing accuracy than removing random data points.}
%     \label{fig:generalization}
% \end{figure}



\subsection{Mitigating Memorization while Maintaining Generalization}


%\textcolor{red}{@Franziska: Write here about the generalization, describe the peak in Figure 5(b) which results from us being able to remove first mis-labeled samples (improves utility) and then atypical samples (that are required for learning).}
The experiments from \Cref{tab:augmentations} suggest that using augmentations during training can improve generalization while also reducing memorization.
This is an unexpected synergy since for both supervised learning~\citep{feldman2020does} and SSL~\citep{wang2024memorization}, generalization was shown to decline when memorization decreases.
\begin{figure}[t]
\begin{subfigure}{0.48\columnwidth}
        \centering
    \includegraphics[width=0.95\linewidth]{image/caption_num.pdf}
    \caption{Different numbers of captions.}
    \label{fig:num_captions}
\end{subfigure}
\hfill
\begin{subfigure}{0.48\columnwidth}
\centering
\includegraphics[width=0.9\textwidth]{image/noise_strength.pdf}
        \caption{Noising text embedding during training.}
    \label{fig:noising}
\end{subfigure}
    \caption{\textbf{Mitigating memorization in CLIP improves downstream generalization.} We train CLIP models with different "augmentations" in the textual domain. (a) We use multiple captions for the same image during training. (b) We directly noise the text embeddings during the training using Gaussian noise with a mean of 0 and different standard deviations \reb{(adding the Gaussian noise $\mathcal{N}(0,0.15)$ gives us the sweet spot with the smallest memorization and highest performance)}. Both strategies successfully reduce memorization while improving performance.}
    \label{fig:mitigations}
\end{figure}
% \begin{wrapfigure}{r}{0.4\textwidth}
% \vspace{-0.5cm}
% \begin{center}
% \centerline{\includegraphics[width=0.95\linewidth]{image/caption_num.pdf}}
% \vspace{-0.2cm}
% \caption{
% \label{fig:num_captions}
% \textbf{Memorization vs Generalization.}
% }
% \end{center}
% \vspace{-0.8cm}
% \end{wrapfigure}
To further study the impact of mitigating memorization in CLIP on downstream generalization, we explore two orthogonal strategies for "augmenting" the text modality during CLIP training, first in the input space and second directly in the embedding space. Additionally, we analyze the effect of removing memorized samples from training.

\begin{figure}[t]
    \centering
    \begin{subfigure}{0.48\columnwidth}
    \centering
    \includegraphics[width=0.95\linewidth]{image/sample_remove_random_clipmem_new.pdf}
    %\caption{\textbf{Removing memorized samples increases performance.}}  
    \caption{CLIP trained on COCO.}  
    \label{fig:remove_samples_cooc}
    \end{subfigure}
    \hfill
    \begin{subfigure}[b]{0.48\textwidth}
        \centering
        \centering
    \includegraphics[width=0.95\linewidth]{image/sample_remove_random_clipmem_cc3m_new.pdf}
    %\caption{\textbf{More captions lowers memorization and increases performance.}}
    \caption{CLIP trained on CC3M.}
    \label{fig:remove_samples_cc3m}
    \end{subfigure}
    \caption{
    \textbf{Removing memorized samples according to \ours has a stronger influence on the linear probing accuracy than removing random data points.} 
    Removing the mislabeled samples based on \ours improves the performance significantly, followed by a sharper drop when removing atypical samples.
    %We compare the effects of removing random samples versus removing the highest memorized ones according to \ours.
    \vspace{-0.5cm}
    }
    \label{fig:generalization}
\end{figure}


\textbf{Multiple captions.}
We vary the number of captions used during training and report fine-grained insights into the resulting memorization and downstream performance in \Cref{fig:num_captions}. 
Our results highlight the trend that the more captions are used during training, the lower memorization and the higher the linear probing accuracy.
%with more captions, memorization decreases while accuracy increases.
%the trend of utility improving while memorization drops is consistent: 
Our additional results in \Cref{tab:multi_caption_training} highlight also that choosing all captions equally often is beneficial for utility while keeping memorization roughly the same.
%This suggests that we should use multiple captions during CLIP training, similar to LaCLIP \citep{fan2023}, to benefit from higher utility and lower memorization. 
Since not in every dataset, multiple captions are available, we experiment with generating these captions with a language model. Our results in \Cref{tab:coco_gpt3_captions_train} where we train CLIP with captions generated by GPT3.5 show that the results both in terms of utility and memorization are extremely similar to the original captions, making this improved training strategy widely applicable. 
Our findings that modifying the text during training can reduce memorization align with the insights presented by~\citet{jayaraman2024}.
For datasets where only single captions are available, they proposed \textit{text randomization}, \ie masking out a fraction of tokens during training as a mitigation for their Déjà Vu memorization.
In contrast to our GPT3.5-generated captions, this masking, however, causes a drop in performance when mitigating memorization. We hypothesize that this is due to the higher distribution shift introduced by the masked tokens. %\todo{Here would be space for the noise experiment.}

\textbf{Noising the text embedding during training.} 
To overcome such shortcomings altogether and avoid any inherent distribution shifts, we propose to perform the "augmentations" directly in the embedding space.
More precisely, we experiment with an approach where, during training, before calculating the cosine similarity between text and image embeddings for the contrastive loss, we add small amounts of Gaussian noise to the text embeddings.
Our results in \Cref{fig:noising} \reb{and \Cref{tab:noising}} highlight that this strategy is highly effective in reducing memorization while improving downstream generalization. %, making it a highly practical mitigation strategy.

% \begin{table}[t]
%     \centering
%    \begin{tabular}{ccc}
%     \toprule
%     Noise & & (ImageNet) \\
%     \midrule
%       NULL & 0.438 & 63.11\% $\pm$ 0.91\%\\
%       0 / 0.01 & 0.435 & 63.36\% $\pm$ 0.88\%\\
%       0 / 0.05 & 0.428 & 64.02\% $\pm$ 1.12\%\\
%       0 / 0.10 & 0.421 & 64.95\% $\pm$ 0.96\%\\
%       \bottomrule
%     \end{tabular}
%     \caption{\textbf{Noising the text representation during training.} We add Gaussian noise to the text representation during training of CLIP on COCO and measure \our and ImageNet linear probing accuracy. Adding noise reduces memorization while improving generalization.}
%     \label{tab:text_augmentations}

% \end{table}


\textbf{Removing memorized samples.}
Finally, we investigate the effect of removing memorized samples to understand how it impacts downstream performance. We
perform an additional experiment where we first train a CLIP model, then identify the highest memorized training data points, remove them, and retrain on the remaining data points only. \reb{We compare this method to two baselines where we either randomly remove samples or filter out the samples with the lowest CLIP similarity between the training data points' two modalities.}
We showcase the effect on the downstream linear probing accuracy on ImageNet in \Cref{fig:generalization} with CLIP models trained on COCO and on the CCM3 dataset. For the COCO dataset,
when removing up to 100 most memorized data points, we first observe a sharp increase in downstream performance in comparison to removing random samples. 
Then, the downstream performance starts dropping significantly more when removing memorized instead of random samples, until between 400 and 800 removed samples, the cutoff point is reached where model performance is worse when removing according to highest memorization instead of randomly.
For the CC3M dataset, this cutoff occurs later, between 1600 and 3200 removed samples.
% \reb{Compared with the naive similarity used by CLIP model, \ours is more effective when noise samples are not fully removed.}
\reb{While the CLIP similarity also manages to increase performance through removal, it is not as effective as \ours, highlighting the value of considering memorization as a lens to identify noisy samples.}
This finding is significantly different than for supervised learning and SSL, where the removal of highly memorized samples \textit{constantly} harms performance more than the removal of random samples~\citep{feldman2020does,wang2024memorization}.
We hypothesize that the effect observed in CLIP might result from the distinction between "mis-captioned" and atypical samples, where the former harm generalization while the latter help the model learn from smaller sub-populations~\citep{feldman2020does}. We empirically support this hypothesis in \Cref{app:generalization}.
%, where we find that mis-captioned samples harm utility, in particular through the contrastive training, whereas atypical samples help the model learn from smaller sub-populations
% are required to learn smaller sub-populations and thereby support training
The finding that CLIP generalization can be improved by identifying inaccurately captioned data points using our \ours and removing them from training is of high practical impact, given that state-of-the-art CLIP models are usually trained on large, uncurated datasets sourced from the internet with no guarantees regarding the correctness of the text-image pairs.
%Our results highlight that this practice not only exposes imprecise or incorrect data pairs to more memorization, often recognized as a cause for increased privacy leakage~\citep{carlini2019secret, carlini2021extracting, carlini2022privacy,song2017machine,liu2021encodermi}, but that it also negatively affects model performance. 
% To test this hypothesis, we run an experiment where we train a supervised model (ViT-Base) on CIFAR10 but flip the label of 200 candidate data points in $S_C$ prior to training. Then, we use our setup to approximate memorization in supervised learning as defined by~\citet{feldman2020does} and report generalization while removing most memorized vs. random samples.
% Our results in \Cref{fig:sample_remove_poison_cifar_imagnet_slt} highlight a similar trend of a first increase in generalization when removing most memorized samples. The peak is at roughly 200, \ie the number of inserted mislabeled samples. Afterwards, the removal of memorized samples harms generalization more as we can see a sharper drop than when removing random samples.
% For more details on the experiment, see \Cref{app:generalization}.
Overall, our results suggest that \ours can help reduce memorization in CLIP while improving downstream generalization.
%\todo{@Adam, can you review this last sentence? Is it strong enough to end the empirical part of the paper on it?}
%\textcolor{red}{I still need to go through this in detail.}


% \begin{enumerate}
%     \item How CLIP memorization relates to SL and SSL memorization. 1
%     \item Which modality (text or vision) is more responsible for memorization? 2
%     \item Which samples are memorized and why? 3
%     \item How does the CLIP memorization impact generalization? 4
%     \item How can we prevent memorization and maybe, at the same time, get better CLIP models? 5
% \end{enumerate}














    

% \begin{figure}[h]
%     \centering
%    \includegraphics[width=0.5\columnwidth]{image/caption_num.pdf}
%     \caption{\textbf{More captions used per image increase performance while reducing memorization.}
%     We present results for linear probing accuracy for ImageNet vs the memorization score from our \ClipMem when using different numbers of captions during training. Each caption is used to calculate the \ClipMem, and the final result is obtained by averaging the scores.}
%    \label{fig:clipmem_acc_caption}
% \end{figure}

% \begin{figure}[h]
%     \centering
%    \includegraphics[width=1.0\columnwidth]{}
%     \caption{\textbf{} We remove the memorized samples computed based on the text and image embeddings. The CLIP model is retrained from scratch and we report the linear probing accuracy using ImageNet.\adam{remove the grouping.}
%     }
%    \label{fig:clipmem_sample_remove}
% \end{figure}
\section{Conclusion}
We presented \ours, a formal measure to capture memorization in multi-modal models, such as CLIP.
By not only quantifying memorization but also identifying \textit{which} data points are memorized and \textit{why}, we provide deeper insights into the underlying mechanisms of CLIP.
Our findings highlight that memorization behavior of CLIP models falls between that of supervised and self-supervised models.
In particular, CLIP highly memorizes data points with incorrect and imprecise captions, much like supervised models memorize mislabeled samples, but it also memorizes atypical examples.
%Furthermore, the definition allows us to study which data points are memorized between the modalities and how the modalities contribute to this memorization.
Furthermore, we find that memorization in CLIP happens mainly within the text encoder, which motivates instantiating mitigation strategies there.
By doing so, we can not only \textit{reduce memorization} in CLIP but also \textit{improve} downstream generalization, a result that challenges the typical trade-offs seen in both supervised and self-supervised learning.

% By doing so, in contrast to supervised and self-supervised learning, we can \textit{reduce memorization} in CLIP while at the same time \textit{improving} downstream generalization.

%While our metric focuses on CLIP-like vision-language models, it can, in principle, be extended to more modalities. By doing so, identifying where memorization happens between more than two modalities, and how downstream generalization will be affected presents an interesting avenue for future work.

% -- can we say something about even more modalities? would CLIPMem transfer there?
% -- can we link to other downstream tasks like retrieval?
% -- can we show for more models?
% -- 
\section*{Acknowledgements}
This research was funded by the Deutsche Forschungsgemeinschaft (DFG, German Research Foundation), Project number 550224287.


% \bibliographystyle{plainnat}
% \bibliography{main}
\bibliography{main}
\bibliographystyle{iclr2024_conference}

\appendix
\newpage
\section{Appendix}


%\todo{to be finished!}
\subsection{Extended Background}
\label{app:deja_vu_comparison}
\paragraph{Déjà Vu Memorization in CLIP.}
The Déjà Vu memorization framework \citep{jayaraman2024} is the only existing other work that attempts to quantify memorization in vision-language models. It uses the text embedding of a training image caption to retrieve relevant images from a public dataset of images. It then measures the fraction of ground-truth objects from the original image that are present in the retrieved images. If the training pair is memorized, retrieved images have a higher overlap in ground truth objects, beyond the simple correlation.
While valuable, several aspects warrant further consideration for broader applicability of the framework. First, its focus on object-level memorization ignores non-object information like spatial relationships or visual patterns that can also influence memorization~\citep{feldman2020does,wang2024memorization}.
To perform object retrieval, the framework also relies on object detection and annotation tools, which may introduce variability based on the accuracy and robustness of these tools.
Additionally, the assumption that public datasets with similar distributions to the training data are readily available may not always hold, necessitating alternative approaches. 
Moreover, the framework does not analyze why certain images are memorized limiting detailed analysis. 
%The framework uses a guess-and-check approach to measure memorization, which may not fully leverage CLIP's multimodal nature or its contrastive learning objective. A more direct measurement could take advantage of these features to enhance evaluation.
Finally, while Déjà Vu must address the challenge of distinguishing between memorization and spurious correlations, \ours avoids this by directly assessing memorization on the output representations of the model.
One notable difference between the results of our approach and Déjà Vu's is that their findings show that their mitigation strategies can reduce memorization, but at the cost of decreased model utility. \ours, in contrast, does not observe trade-offs between memorization and performance.

    % \item \textbf{Lack of layer-wise insights.} Does not provide insights into how memorization occurs across different model layers.


% \paragraph{Shortcomings of Déjà Vu Memorization}
% \begin{enumerate}
%     \item \textbf{Limited to object-level memorization.} Focuses on object retrieval, neglecting other forms of memorization that pertain to non-object information like spatial relationships or visual patterns.
%     \item \textbf{Dependence on external models.} Relies on (inaccurate) object annotation tools, constraining the evaluation by the accuracy and robustness of these tools.
%     \item \textbf{Assumption of public dataset.} Assumes the availability of public datasets with similar distributions to the training data, which may not always be accessible. 
%     \item \textbf{No identification of memorized samples.} Does not specify which images are memorized or why.
%     \item \textbf{Lack of layer-wise insights.} Does not provide insights into how memorization occurs across different model layers.
%     \item \textbf{Indirect measurement of memorization.} Relies on a guess-and-check approach for measuring memorization that does not account for CLIP's multimodal nature or contrastive learning objective.
%     \item \textbf{Challenge with correlation vs. memorization.} Needs to distinguish between true memorization and spurious correlations, complicating the evaluation.
%     \item \textbf{Performance trade-offs.} Implements mitigation strategies that reduce memorization, but at the cost of decreased model utility.
% \end{enumerate}



\subsection{Extended Experimental Setup}
\label{app:setup}

\paragraph{General Setup.} 
All the experiments in the paper are done on a server with 4 A100 (80 GB) GPUs and a work station with one RTX 4090 GPU(24 GB).
We detail the setup for our model training, both CLIP and SSL (relying on DINO) in \Cref{tab:settings}.

\addtolength{\tabcolsep}{-2.5 pt}
\begin{table}[h]
\caption{\textbf{Experimental Setup.} We provide details on our setup for encoder training and evaluation.}        
          \label{tab:settings}
    \centering
    \scriptsize 
\begin{tabular}{cccccccc}
\toprule
                       & \multicolumn{3}{c}{Model Training}                                       &  & \multicolumn{3}{c}{Linear Probing}                      \\ \cmidrule{2-4} \cmidrule{6-8} 
                           &  CLIP  & DINO & Supervised ViT          &   & CLIP    &DINO     & Supervised ViT           \\ \midrule
Training Epoch          &  100     & 300          & 100 &   &   45    & 45          &45       \\
Warm-up Epoch          & 5  & 30        & 5& & 5          & 5          &5     \\
Batch Size             &  128  & 1024          &128  &   & 4096    & 4096 &4096      \\
Optimizer              &     Adam              & AdamW             & Adam      &  & LARS         & LARS & LARS        \\
Learning rate          &     1.2e-3               & 2e-3                 & 1e-3     &   & 1.6          & 1.6  & 1.6        \\
Learning rate Schedule &         Cos. Decay        & Cos. Decay & Cos. Decay & & Cos. Decay & Cos. Decay & Cos. Decay\\ \bottomrule 
\end{tabular}
\end{table}
\addtolength{\tabcolsep}{2.5 pt}


\paragraph{Experimental Setup for SSLMem.}
To experimentally evaluate memorization using the SSLMem framework~\citep{wang2024memorization}, the training dataset $S$ is split into four sets: \textit{shared set} ($S_S$) used for training both encoders $f$ and $g$; \textit{candidate set} ($S_C$) used only for training encoder $f$; \textit{independent set} ($S_I$) data used only for training encoder $g$; and an additional \textit{extra set} ($S_I$) from the test set not used for training either $f$ or $g$. For training encoders, encoder $f$ is trained on $S_S \cup S_C$, while encoder $g$ is trained on $S_S \cup S_I$. The alignment losses $\lalign(f, x)$ and $\lalign(g, x)$ are computed for both encoders, and the memorization score $m(x)$ for each data point is derived as the difference between these alignment losses, normalized to a range between $-1$ and $1$. A score of $0$ indicates no memorization, $+1$ indicates the strongest memorization by $f$, and $-1$ indicates the strongest memorization by $g$.

\paragraph{Normalization on \ours.}
For improved interpretability, we normalize our \ours scores to a range of $[-1,1]$. A memorization score of $0$ indicates no memorization, $+1$ indicates the strongest memorization on CLIP model f, and $-1$ indicates the strongest memorization on CLIP model g. We find the normalized CLIPMem score for a dataset using the following process:
For each image-text pair $(I,T)$, we first calculate the CLIPMem score as the difference in alignment scores between two CLIP models $f$ and $g$.
Once CLIPMem scores are computed for all data points, we normalize them by dividing each score by the range, which is the difference between the maximum and minimum scores in the dataset.
Finally, we report the normalized \ours score for a dataset as the average of these normalized values.
% Therefore, we calculate the differences in alignment loss per data sample $(I,T)$ over both CLIP encoder pairs $f$ and $g$.
% Afterwards, we normalize these differences by dividing them by the range (largest minus smallest difference),

% Finally, we report the \ours score as the average of the normalized scores over all data points.
% in $S_C$. 

\subsection{Additional Experiments}




% \subsubsection{Studying Memorization per Modality}
% \label{app:modality}
% \todo{@Adam, I find it very hard to justify the experiment below. Can you please have a look? The reviewers might just be confused.}
% We also perform an experiment where we remove memorized samples, retrain the model, and report the resulting linear probing accuracy.
% The more the linear probing accuracy is affected by the removal of a memorized sample, the more important the samples.
% Rather than taking the overall highest memorized samples and removing them, once the model is trained, we first cluster the training data in the embedding space into 50 clusters, once according to the text and once according to the image embeddings.
% Then, we remove from each cluster the highest memorized data point.
% This semantically corresponds to finding the highest memorized data points per-class in supervised learning (with the difference that we have no class information, and therefore, have to revert to a default number of clusters, in our case 50).
% Our results in \Cref{fig:modality} highlight that removing the highest memorized samples according to the text embeddings' grouping and then re-training boosts downstream performance more than removing based on image embeddings.\todo{@Adam, here, for sure something is missing, an explanation or so. But I have none. In the worst case, we need to remove this paragraph.} 
% Hence, the overall observed trend indicates that memorization in CLIP depends more on the text than on the image.
% \begin{figure}[h]
%     \centering
%     \includegraphics[width=0.55\linewidth]{image/sample_remove_text_image.pdf}
%     \caption{\textbf{Modality-based removal.} We train a CLIP models with standard image cropping on the COCO dataset. 
%     After training, we cluster training samples according to their similarity based on CLIP's image or text embeddings into 50 clusters, remove the highest memorized sample from each cluster, and retrain with the remaining points. The linear probing accuracy on ImageNet suggests that removing based on text embeddings' clustering memorization has higher influence on model performance.}
%     \label{fig:modality}
% \end{figure}

\subsubsection{Memorization vs. Generalization in CLIP}
\label{app:generalization}


\begin{figure}[t]
    \centering
    \includegraphics[width=0.55\linewidth]{image/sample_remove_random_clipmem_5_caption.pdf}
    \caption{\textbf{Removing memorized samples.} We show the effect on downstream performance in terms of ImageNet linear probing accuracy and \ours for a CLIP model trained on COCO using 5 text captions instead of 1, like done in \Cref{fig:generalization}. We observe the same trend, with the difference that the peak is at roughly 500 removed samples rather than 100. This is likely due to the increase in captions (by factor 5) that causes increase in mis-captioned samples.}
    \label{fig:image_text_sample5}
\end{figure}
\begin{figure}[t]
    \centering
    \includegraphics[width=0.55\linewidth]{image/sample_remove_poison_cifar_imagnet_sl.pdf}
    \caption{\textbf{Removing memorized samples in supervised learning.} We train a ViT-tiny on CIFAR10~\citep{krizhevsky2009learning} using supervised learning. We use our evaluation setup with $S_C$, $S_S$, $S_I$, and $S_E$ to approximate the memorization metric from~\citet{feldman2020does}. We use 5000 samples in $S_C$, but before training, we flip the labels of 200 samples. We calculate memorization over all samples in $S_C$ and test the linear probing accuracy with ImageNet resized to 32*32 on the representations output before the original classification layer.}
    \label{fig:sample_remove_poison_cifar_imagnet_slt}
\end{figure}
\textbf{Extending evaluation.}
In \Cref{fig:image_text_sample5}, we perform the same experiment as in \Cref{fig:generalization}, but on a CLIP model trained with 5 captions instead of 1.
We observe the same trend, with the difference that the peak is at roughly 500 removed samples rather than 100. This is likely due to the increase in captions (by factor 5) that causes increase in mis-captioned samples.

\textbf{Verifying the hypothesis on memorizing mis-captioned samples through supervised learning.}
We repeat the same experiment in the supervised learning setup to understand where the increase and then decrease in linear probing accuracy stems from. To test our hypothesis that it stems from "mis-captioned" samples, we "poison" our supervised model by flipping the labels of 200 data points before training. 
Then, we approximate the memorization metric from~\citet{feldman2020does} in our setup and remove highly memorized vs. random data points.
In the same vein as in \Cref{app:generalization}, we first observe an increase in linear probing accuracy when removing memorized samples (instead of random samples). The peak is at roughly 200 data points, \ie the number of deliberately mislabeled samples.
Until the cutoff point at roughly 3200 examples, linear probing accuracy is still higher when removing most memorized rather than random samples, which might suggest that there are other outliers or inherently mislabeled samples whose removal improves model performance. After the cutoff, we observe the behavior as observed in prior work~\citep{wang2024memorization, feldman2020does} that reducing memorization harms generalization more than reducing random data points from training.

%\textcolor{red}{TBD: refer to \Cref{tab:reduce_memorization}}

%\textcolor{red}{We need to describe \Cref{fig:image_text_sample5}}








\subsubsection{The Effect of Captions}
\label{app:captions}

\begin{table}[t]
    \centering
           \caption{\textbf{Using different/multiple captions during training.} 
           We evaluate \ours how memorization on different data subsets and linear probing accuracy on ImageNet differ when using 1 caption (baseline), 5 COCO captions, one chosen at random at every round (random), and 5 COCO captions, but all chosen equally often, \ie 20 out of 100 training epochs (balanced).
           We observe that increasing the number of captions reduces highest memorization. Yet, only when we balance the usage of caption, also model performance increases.}        
          \label{tab:multi_caption_training}
    \scriptsize
    \begin{tabular}{cccc}
    \toprule
&baseline & random & balanced \\
    \midrule
Avg. \ours (Top 10 samples)&0.792&0.788 &0.790 \\
Avg. \ours (Top 20\%) &0.552&0.531&0.540\\
%Avg. \ours ($S_C$) &0.438&0.409&0.423\\
Linear Probing Acc. &63.11\% $\pm$ 0.91\%&62.44\% $\pm$ 1.18\%& 64.88\% $\pm$ 0.83\%\\
      \bottomrule     
      \end{tabular}
\end{table}
\begin{table}[t]
          \caption{\textbf{The CLIPMem and linear probing accuracy of model trained with original coco captions and captions generated by GPT3.5.} For 'Single Caption', only one caption is used during training. For 'Five Caption', all five caption are used equally during training (every caption trained for 20 epoch out of 100). The linear probing accuracy is tested on ImageNet}        
          \label{tab:coco_cpt_train}
    \centering
    \scriptsize
    \begin{tabular}{cccccc}
    \toprule 
    & \multicolumn{2}{c}{COCO} && \multicolumn{2}{c}{GPT3.5}\\
    &Single Caption &Five Caption&&Single Caption &Five Caption\\
    \midrule
    CLIPMem &0.438&0.423& &0.430&0.411\\
    LP. Acc. &63.11\% $\pm$ 0.91\%&64.88\% $\pm$ 0.83\%&&63.09\% $\pm$ 1.12\%&64.47 $\pm$ 0.72\%\\
      \bottomrule     
      \end{tabular}
\end{table}
In \Cref{tab:multi_caption_training}, we show that using more captions during training reduces memorization and that by using each caption at the same frequency over the training epochs, we can additionally improve model performance.
Additionally, we show that captions generated by GPT3.5 have the same effect as the original COCO captions on memorization and linear probing accuracy in \Cref{tab:coco_cpt_train}.

\subsection{The Effect of Model Size}
\label{app:modelsize}
In \Cref{tab:model_size}, we present how the model size affects the memorization level of CLIP models. Both models are trained using the same dataset and settings. We observe that with more parameters (larger model size), encoders have higher memorization capacity. This aligns with findings from previous research~\citep{wang2024memorization, feldman2020does, meehan2023ssl}.



\begin{table}[t]
          \caption{\textbf{
           CLIPMem and linear probing accuracy of models with different sizes.} The models are trained using identical settings and the same subset of the COCO dataset. Linear probing accuracy is tested on the ImageNet dataset as the downstream task.}        
          % The CLIPMem and linear probing accuracy of model with different size} We train the models with same settings as well as same subset of COCO dataset. The linear probing accuracy is tested on ImageNet dataset as the downstream task.
          \label{tab:model_size}
    \centering
    \scriptsize
    \begin{tabular}{ccc}
    \toprule 
     Model& \ours&Lin. Prob. Acc. (ImageNet) \\
     \midrule
    ViT-base (Baseline in main paper) & 0.438 & 63.11\% $\pm$ 0.91\%\\
    ViT-large & 0.457 &  67.04\% $\pm$ 1.05\%\\
      \bottomrule     
      \end{tabular}
\end{table}

\subsection{Verification of infinite data regimes}
\label{app:onerun}
To evaluate CLIPMem over infinite data regimes (\ie using a single training run where no data point is repeated), we use a subset $D$ (containing 7050000 samples) of YFCC100M dataset~\citep{thomee2016yfcc100m} to train another pair of ViT-Base models for only 1 epoch. Following our definition of \ours, we further divide $D$ into $S_S$ with 6950000 samples, $S_C$ with 50000 samples, and $S_I$ with 50000 samples. The reason we use 7M (6950000+50000) samples to train either model $f$ or model $g$ is to make sure the newly trained model has the same number of training samples as the model trained with K-epoch runs (70000 samples/epoch * 100 epoch). The results in \Cref{tab:YFCC7M} show that the model trained with infinite data regimes has higher linear probing accuracy on ImageNet as a downstream task and lower memorization scores, as measured by \ours. This aligns with the fact that duplicated data points increase the memorization level and make the model over-fit, hence reducing the generalization~\citep{wang2024memorization, feldman2020does}. The results in \Cref{fig:most_YFCC} show that the most memorized samples according to \ours in the model trained with infinite data regimes are also samples with imprecise or incorrect captions. This aligns with our statements in \Cref{chpt:clip_ssl_sl}. 

\begin{figure}[t]
    \centering
    \includegraphics[width=0.95\linewidth]{image/most_memorized.pdf}
    \caption{\textbf{Top 10 memorized samples according to \ours in the model trained under infinite data regimes on YFCC100M.} The model is trained for one epoch, \ie seeing each training data point exactly once. Even in this setup, the most memorized samples are still the ones with imprecise or incorrect captions.}
    \label{fig:most_YFCC}
\end{figure}

\begin{table}[t]
\centering
\scriptsize
\caption{\textbf{Evaluation of CLIPMem under infinite data regimes, \ie seeing every data point only once during training vs training with 100 epochs.} We observe that both setups reach comparable downstream accuracy and memorization.}
\label{tab:YFCC7M}
\begin{tabular}{ccc}
      \toprule 
      Model & \ours & Lin. Prob. Acc. (ImageNet) \\
      \midrule
     ViT-Base (YFCC 7M, 1 epoch) &0.425&64.83\% $\pm$ 1.04\%\\
     ViT-Base (COCO 70K, 100 epochs)&0.438&63.11\% $\pm$ 0.91\%\\
      \bottomrule
\end{tabular}
\end{table}


\subsection{Evaluation on BLIP}
\label{app:BLIP}
To verify the effectiveness of \ours over other similar multi-modal models, we train a BLIP model on COCO dataset following the same settings as the baseline model in the main paper. We present the results for \ours on BLIP over all four data subsets in \Cref{fig:4_set_hist_blip}, which is in agreement with the results of the BLIP model in \Cref{fig:memorization_subsets}. We also present the results for UnitMem in \Cref{fig:unitmem}, which is also very similar to the results of CLIP models

\begin{figure}[t]
    \centering
    \includegraphics[width=0.55\linewidth]{image/4_set_hist_blip.pdf}
    \caption{\textbf{Memorization scores across data subsets on BLIP models} We train a BLIP model on COCO standard image cropping and no text augmentation. We present the results for \ours over all 4 data subsets, which is in agreement with the results of the CLIP model in \Cref{fig:memorization_subsets}}
    \label{fig:4_set_hist_blip}
\end{figure}

\subsection{Memorization distribution During training}
We present the distributions of neurons with highest UnitMem during training in \Cref{fig:clipmem_neuron_train}. These results highly consistently indicate that in the early stages of training, neuronal memory occurs mainly in the lower layer of the clip model, while in the middle and later stages of training, neuronal memory is more concentrated in the later layer of the model.
\begin{figure*}
    \centering
    
    \begin{subfigure}[b]{0.7\textwidth}  
        \centering 
        \includegraphics[width=\textwidth]{image/clipmem_neuron_train_1.pdf}
        \caption[]%
        {\textbf{Top 1\% neurons}}    
        \label{fig:clipmem_neuron_train_1}
    \end{subfigure}
    
    \begin{subfigure}[b]{0.7\textwidth}  
        \centering 
        \includegraphics[width=\textwidth]{image/clipmem_neuron_train_3.pdf}
        \caption[]%
        {\textbf{Top 3\% neurons}}    
        \label{fig:clipmem_neuron_train_3}
    \end{subfigure}
    
        \begin{subfigure}[b]{0.7\textwidth}  
        \centering 
        \includegraphics[width=\textwidth]{image/clipmem_neuron_train_5.pdf}
        \caption[]%
        {\textbf{Top 5\% neurons}}    
        \label{fig:clipmem_neuron_train_5}
        
    \end{subfigure}
    \caption{\textbf{Distribution of top 1\%, 3\%, and 5\% neurons with highest UnitMem during training.} We train a CLIP model on COCO standard image cropping and no text augmentation following the settings of baseline model in main paper. We record the neurons with top 1\%, 3\%, and 5\% of highest UnitMem during training (every epoch). } 
    \label{fig:clipmem_neuron_train}
\end{figure*}

\subsection{Human vs Machine Generated Captions}
\label{app:GPTcaptions}

For each image in the COCO dataset, we use GPT 3.5 (specifically, gpt-3.5-turbo) to generate 5 captions (from scratch). We use the following instruction in the OpenAI API: %\todo{@Wenhao, can you update it with your code of "prompt = f"Generate a detailed descri..."}
\begin{verbatim}
def generate_description_for_image(image_caption, clip_features):
    prompt = f"Here is an image with the caption: '{image_caption}'. "
    prompt += f"Based on this caption and the visual features
    represented by this embedding '{clip_features}', 
    please generate a new detailed description."
    response = openai.ChatCompletion.create(
        model="gpt-3.5-turbo",
        messages=[
            {"role": "system", "content": "You are a helpful 
            assistant that generates captions for images."},
            {"role": "user", "content": prompt}
        ]
    )
    return response['choices'][0]['message']['content']
\end{verbatim}
We present the obtained captions in \Cref{fig:image_text_sample}.
\begin{table}[t]
          \caption{\textbf{The machine generated captions provide similar performance to the original human-generated captions.} We report the \ours and linear probing accuracy of model trained with original COCO captions and captions generated by GPT 3.5. For the 'Single Caption', only a single caption is used during training. For 'Five Captions', all five captions are used equally during training (every caption trained for 20 epochs out of 100). The linear probing accuracy is tested on the ImageNet dataset as the downstream task.}        
          \label{tab:coco_gpt3_captions_train}
    \centering
    \scriptsize
    \begin{tabular}{cccccc}
    \toprule 
    & \multicolumn{2}{c}{COCO} && \multicolumn{2}{c}{GPT 3.5}\\
    &Single Caption &Five Captions&&Single Caption &Five Captions\\
    \midrule
    \ours &0.438&0.423& &0.430&0.411\\
    Linear Probing Accuracy (ImageNet) &63.11\% $\pm$ 0.91\%&64.88\% $\pm$ 0.83\%&&63.09\% $\pm$ 1.12\%&64.47 $\pm$ 0.72\%\\
      \bottomrule     
      \end{tabular}
\end{table}
\begin{figure}[t]
    \centering
\begin{subfigure}{0.45\columnwidth}
   \includegraphics[width=1.0\columnwidth]{image/coco_cosine_old}
    \caption{\textbf{COCO (Average: 0.798)}}
   \label{fig:coco_cosine}
\end{subfigure}
\begin{subfigure}{0.45\columnwidth}
     \includegraphics[width=1.0\columnwidth]{image/gpt_cosine_new.pdf}
   \caption{\textbf{GPT3.5 (Average: 0.851)}}
   \label{fig:GPT3.5_cosine}
\end{subfigure}
\caption{\textbf{Pairwise cosine similarity of 5 captions from COCO and generated by GPT3.5.}}
     \label{fig:cosine_single}
\end{figure}
In \Cref{fig:GPT3.5_cosine}, we analyze the pairwise cosine similarity in the original COCO and the GPT3.5 generated captions. We find that the GPT3.5 generated captions are slightly more uniform than the original COCO captions, reflecting in a higher pairwise cosine similarity.



\begin{table}[ht]
\scriptsize
\centering
   \begin{tabular}{ccc}
    \toprule
    Noise & \ours & Lin. Prob. Acc. (ImageNet) \\
    \midrule
      None & 0.438 & 63.11\% $\pm$ 0.91\%\\
      $\mathcal{N}(0.01)$  & 0.435 & 63.36\% $\pm$ 0.88\%\\
      $\mathcal{N}(0.05)$ & 0.428 & 64.02\% $\pm$ 1.12\%\\
      $\mathcal{N}(0.10)$ & 0.421 & 64.95\% $\pm$ 0.96\%\\
      \boldsymbol{$\mathcal{N}(0.15)$}  & \boldsymbol{$0.417$} &  \boldsymbol{$65.34\% \ \pm \  0.84\%$}\\
      $\mathcal{N}(0.20)$ & 0.422 & 64.83\% $\pm$ 0.92\%\\
      $\mathcal{N}(0.25)$ & 0.436 & 63.28\% $\pm$ 0.79\%\\
      $\mathcal{N}(0.30)$ & 0.447 & 61.50\% $\pm$ 0.86\%\\
      $\mathcal{N}(0.50)$ & 0.491 & 57.04\% $\pm$ 1.11\%\\
      $\mathcal{N}(0.75)$ & 0.501 & 52.28\% $\pm$ 0.98\%\\
      $\mathcal{N}(1.00)$ & 0.504 & 51.92\% $\pm$ 1.03\%\\
      \bottomrule
    \end{tabular}
        \caption{\textbf{Noising text embedding during training.} We present the impact of adding noise to the text embedding during training for the ViT-base trained on COCO.}
    \label{tab:noising}
\end{table}


\subsection{Examples for Memorized Samples}
\label{app:examples}
\begin{figure}[t]
    \centering
    \includegraphics[width=0.95\linewidth]{image/10_least_5_caption.pdf}
    \caption{\textbf{The 10 samples with lowest \ours in the CLIP model trained with all 5 captions.} We can see that these samples contain clear concepts and precise captions.}
    \label{fig:examples_memorized_5_caption_least}
\end{figure}

\begin{figure}[t]
    \centering
    \includegraphics[width=0.95\linewidth]{image/10_most_5_caption.pdf}
    \caption{\textbf{The 10 samples with highest \ours in the CLIP model trained with all 5 captions.} We can see that these samples contain atypical, difficult samples with imprecise or incorrect captions.}
    \label{fig:examples_memorized_5_caption_most}
\end{figure}

\begin{figure}[t]
    \centering
    \includegraphics[width=0.95\linewidth]{image/10_least_1_caption.pdf}
    \caption{\textbf{The 10 samples with lowest \ours in the CLIP model trained with 1 caption.} We can see that these samples contain clear concepts and precise captions.}
    \label{fig:examples_memorized_1_caption_least}
\end{figure}

\begin{figure}[t]
    \centering
    \includegraphics[width=0.95\linewidth]{image/10_most_1_caption.pdf}
    \caption{\textbf{The 10 samples with highest \ours in the CLIP model trained with 1 caption.} We can see that these samples contain atypical, difficult samples with imprecise or incorrect captions.}
    \label{fig:examples_memorized_1_caption_most}
\end{figure}

\begin{figure}[t]
    \centering
    \includegraphics[width=0.95\linewidth]{image/text_image_sample.pdf}
    \caption{\textbf{Samples of images generated by Stable Diffusion.} We present the generated images based on the COCO captions.}
    \label{fig:text_image_sample}
\end{figure}

\begin{figure}[t]
    \centering
    \includegraphics[width=0.95\linewidth]{image/image_text_sample.pdf}
    \caption{\textbf{Sample captions generated by GPT3.5.} We present the generated captions and the original image and captions from COCO.}
    \label{fig:image_text_sample}
\end{figure}


\begin{figure*}
    \centering
    \begin{subfigure}[b]{0.475\textwidth}
        \centering
        \includegraphics[width=\textwidth]{image/sslmem_sc_ss_old.pdf}
        \caption[]{{SSLMem Image Encoder (1 caption)}}
        \label{fig:sslmem_sc_ss}
    \end{subfigure}
    \hfill
    \begin{subfigure}[b]{0.475\textwidth}  
        \centering 
        \includegraphics[width=\textwidth]{image/sslmem_sc_ss_5.pdf}
        \caption[]%
        {{SSLMem Image Encoder (5 captions)}}    
        \label{fig:sslmem_sc_ss_5}
    \end{subfigure}
    \begin{subfigure}[b]{0.475\textwidth}  
        \centering 
        \includegraphics[width=\textwidth]{image/sslmem_sc_ss_text_1_old.pdf}
        \caption[]%
        {{SSLMem Text Encoder (1 caption)}}    
        \label{fig:sslmem_sc_ss_text_1}
    \end{subfigure}
    \hfill
    \begin{subfigure}[b]{0.475\textwidth}  
        \centering 
        \includegraphics[width=\textwidth]{image/sslmem_sc_ss_text_5.pdf}
        \caption[]%
        {{SSLMem Text Encoder (5 captions)}}    
        \label{fig:sslmem_sc_ss_text_5}
    \end{subfigure}
        \begin{subfigure}[b]{0.475\textwidth}  
        \centering 
        \includegraphics[width=\textwidth]{image/clipmem_sc_ss_old.pdf}
        \caption[]%
        {{CLIPMem}}    
        \label{fig:clipmem_sc_ss}
    \end{subfigure}
    \hfill
        \begin{subfigure}[b]{0.475\textwidth}  
        \centering 
        \includegraphics[width=\textwidth]{image/ssl_sslnaive_clip.pdf}
        \caption[]%
        {{SSLMem, Naive Sum of SSLMem and CLIPMem}}    
        \label{fig:ssl_sslnaive_clip}
        
    \end{subfigure}
    \caption{\textbf{Evaluation of SSLMem and \ours on a CLIP model trained on COCO.}
    Extended version of \Cref{fig:ssl_vs_clip} where we also include SSLMem calculated on encoders trained with 5 captions instead of 1.
    The trends in both cases are the same. SSLMem for the CLIP Models trained with the 5 captions is slightly higher since SSLMem uses the captions as augmentations for the calculation of the memorization. Overall, our \ours reports the strongest memorization signal for CLIP.
    %Results are obtained on ViT-Base, trained with COCO
    } 
    \label{fig:mean and std of nets}
\end{figure*}




% \paragraph{General Setup.} 
% All the experiments in the paper are done on a server with 4 A100 (80 GB) GPUs and a work station with one RTX 4090 GPU(24 GB).
% We detail the setup for our model training, both CLIP and SSL (relying on DINO) in \Cref{tab:settings}.


\end{document}