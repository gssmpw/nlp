\section{Related Work}
\label{section:related_work}
% \textcolor{red}{TODO}: @Mehdi @Robert
% TODO1: direct optimization vs. indirect optimization?
% TODO2: simulation-based vs. non-simulation based
% TODO3: describe stochastic approach

Solutions the MRP algorithm provides a heuristic approach for computing a sub-optimal but feasible solution for optimization problem, given SSV, ST, demand forecast, and other ordering parameters. To account for such limitations of MRP, new variations of MRP has been introduced. For example, \citet{ptak2011orlicky} introduced a new type of MRP called demand driven MRP (DDMRP). DDMRP leverages knowledge from theory of constraints, traditional MRP \& distribution resource planning (DRP), Six Sigma and lean manufacturing. It leverages MRP for planning, and kan-ban techniques for execution across multi-echelon supply chains, which means that it has the strengths of both but also the weaknesses of both and it remains a niche solution \citep{velasco2020applicability, smith2014demand}.

Since the multi-commodity distribution system design was first introduced by the work of \citet{inbook}, multiple optimization-based approaches have been proposed for the design of supply chain networks~\citep{AIKENS1985263, Geoffrion1995TwentyYO, VIDAL19971}. Most of these research, however, made the assumption that the supply chain are deterministic without considering the uncertainties from different sources such as demands and resource capacity. Supply chain disruption due to the ignorance of uncertain operating conditions will cause huge economic impact. A stochastic programming model was proposed by~\citet{SANTOSO200596} to solve the supply chain network design problem.

Since then, many attempts have been made to solve stochastic nature of supply chain network by leveraging mathematical methods. formulated A two-stage stochastic program targeting at an optimal operation plan was formulated by \citet{Ierapetritou1994NovelOA} and a decomposition-based optimization approach was proposed to solve the program. Further improvement on the approach was made by incorporating a MILP planning model~\citep{Gupta2000ATM} . A non-linear mixed integer formulation was proposed by \citet{Gupta2000ATM} with non-convex objective and constraint functions.

The main issue with the methods above is that they have oversimplified the supply chain in order to model it as a mathematical optimization, which has resulted in the omission of critical real-world operational complexities~\citep{agarwal2019multi}. Furthermore, it order to make the problem solvable additional relaxing assumptions have been made which cannot be made on real examples in real world operations. Simulation-based method was brought into the picture because of its capability of taking into account specific business restrictions, control variables and uncertainties of the problem~\citep{KOCHEL2005505}. However, a pure simulation-based approach won't lead to identification of best parameters to operate the system with, which could lead to optimal performance. As a result, a simulation-optimization approach has has been adopted recently to adequately model the complex system-level interactions and constraints and solve for optimal re-order parameters which leads to optimized inventory level and cost~\citep{10.5555/2700739.2700741}.


\citet{1172871} provided an overview of simulation optimization methods which were later adopted to solve the supply chain optimization problems. Given the involvement of multiple decision makers in the whole supply chain network. \citet{swaminathan1998modeling} proposed a multi-agent approach to represent critical supply chain components (e.g., retails, manufacturers) and simulate the supply chain based on their decisions. \citet{mele2006simulation} further extended agent-based  simulation-optimization approach by incorporating genetic algorithm. \citet{1166395} leveraged a multi-objective GA methodology with an existing supply chain simulator to optimize the system parameters. Besides directly optimize the inventory which may be limited by operation constraints, a simulation-optimization framework was proposed by \citet{JUNG20082570} to optimize the safety stock levels with consideration of production capacity.

The majority of methods mentioned above did not prioritize treatment of the stochastic nature of inventory above the optimization. For the ones that formulate the issue as stochastic programming problem, they lack the details in the modeling to properly represent the dynamics of inventory in supply chain. Those drawbacks make it difficult to productionize those solutions at a large scale.