

\section{DATASETS}
\label{datasets}
\label{sec:datasets}

%The detectors on each TESS camera were read out entirely every 30-minutes and 10-minutes for the primary mission and first extended mission respectively, and currently every 200-seconds in the second extension mision, along with 2-minute cadence observations of 200,000 to 400,000 selected stars.

%We used the pre-processed TESS light curves from the Mikulski Archive for Space Telescopes (MAST) \footnote{https://archive.stsci.edu/tess/bulk\_downloads/bulk\_downloads\_ffi-tp-lc-dv.html}, which have 2-minute cadence observations of 200,000 to 400,000 selected stars. We use two datasets to collect the light curves. The first dataset corresponds to the TESS TOI catalog (ExoFOP-TESS)\footnote{https://exofop.ipac.caltech.edu/tess/}, which contains the TIC id of the confirmed and known planets to date, and all the parameters of the planet. In this database, there are roughly 147 systems labeled as Confirmed Planets (CP), which were confirmed as such after the TESS observations were performed; and 420 systems labeled as Known Planets (KP), which were discovered previous to the TESS observations. As second dataset, we use the light curves previously labeled as planet candidate (PC) by DL models, which correspond to the data labeled by \citet{yu2019identifying} and \citet{rao2021nigraha}(Nigraha), that were disclosed in a public repository\footnote{https://github.com/yuliang419/Astronet-Triage/blob/master/astronet/tces.csv}. We filter the light curves with greater confidence with a threshold $>0.7$, of which we discarded 23\% of them since they corresponded to light curves previously labeled as FP. For this second dataset we collected 153 TCEs. \par 

%We filter the light curves with greater confidence with a threshold $>0.7$, \textbf{out of which 23\% were previously labeled FP.} For this second dataset we collected 153 TCEs. \par 

