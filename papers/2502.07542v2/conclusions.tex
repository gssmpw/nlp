
\section{Conclusions}
\label{sec:conclusions}

This work demonstrates the Transformer's ability to capture long-range dependencies and focus on relevant segments of light curves, combined with a CNN embedding that effectively aids the encoder in learning short-range patterns and variations. This combination allows for the detection of both multi-transit and single-transit light curves without prior transit parameters. Moreover, incorporating background and centroid time series data significantly enhanced the performance of our model, allowing it to better represent variations in the light curves, including stellar variability and instrumental effects. Although the use of Transformers in exoplanet detection, and in astronomy in general, is still in its early stages, the success of this approach suggests that Transformers could offer a valuable alternative to traditional methods.  \par

%Furthermore, the combination of a CNN embedding with a self-attention mechanism allows to capture both short- and long-range dependencies.


Based on our findings, our proposed model learned the signal of an exoplanet transit independently of its periodicity, allowing it to identify transit events even in the absence of regular repeating patterns. This ability to generalize beyond periodic signals is particularly important for detecting single transiters, where only one transit event is observed due to long orbital periods. However, through our analysis, we found that our model did not identify any Earth-sized or super-Earth candidates. Instead, it successfully detected exoplanets with radii greater than 0.25 Jupiter radii. The Earth-sized and super-Earth planets typically produce shallower transits, which makes their detection more reliant on periodicity. In these cases, the recurring transit events allow traditional period-searching algorithms, such as BLS, to identify the smaller, more subtle signals that our model may not yet effectively detect. In addition to detecting multi-transit light curve candidates with radii greater than 0.25 Jupiter radii, our model also effectively identified single transiters and multi-planet systems. In total, we identified 214 planetary system candidates, which include 122 single-planet multi-transit light curve candidates, 88 single-transit candidates and 4 associated with multi-planet systems. \par

%122 periodic, 88 single, 4 multi-planet(9)

As a result, the strength of our NN is that it is able to detect exoplanet transit signals within light curves without relying on their periodicity. This is achieved through training the NN to recognize the distinctive shape of the planetary transit signals, even in the presence of stellar variability. While the detection of smaller, less prominent planets remains a challenge, the success in identifying long-period candidates and complex systems underscores the potential of Transformer-based models in expanding the scope of exoplanet detection. In particular, our model operates without the need for prior transit parameters or phase folding, which are typically required by the previous approach of DL, enabling a more direct analysis of the data. This capability allows for the identification of diverse transit signals that may be overlooked by conventional techniques.  \par


Our future work aims to analyze data from TESS sectors $>$ 26 to identify new candidates. Additionally, we plan to improve our detection models to achieve more accurate identification of smaller exoplanets, such as Earth-sized and super-Earth candidates, while maintaining our capability to detect transits regardless of their periodicity to detect multi-planet systems or single transiters. We also consider that upcoming missions, such as PLATO \citep{rauer2016plato} and the Nancy Grace Roman Space Telescope Roman, will present new challenges in the exoplanet search. The Roman mission, in particular, is predicted to discover between $\sim$60,000 and $\sim$200,000 transiting planets \citep{wilson2023transiting}. The approach we proposed provides a base for the development of new architectures.