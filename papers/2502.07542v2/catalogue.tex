\section{Searching New Candidates}
\label{sec:search_candidates}
In this section, we describe the process of identifying new exoplanet candidates using our trained model. We made predictions on SPOC data from sectors 1 to 26 with the aim of discovering previously undetected exoplanet candidates. Given that these sectors have already been extensively analyzed by numerous candidate search pipelines, employing both traditional methods and machine learning approaches, many possible candidates have already been catalogued. We filtered out all candidates that were already listed in the ExoFOP catalogue, as it is considered the authoritative source for previously identified candidates.
\par
%had been previously identified by these pipelines. 

We developed a pipeline for the identification process, beginning with the inference of light curves to extract potential candidates. Our model evaluated a total of 4.1 million light curves, identifying 0.1\% with a probability of containing planetary transit signals. Among the light curves identified, both single-planet multi-transit and single-transit events were detected, each requiring a specific vetting approach. Figure~\ref{fig:pipeline} shows the general procedure for identifying new candidates. Once our trained model performs inference on new light curves and identifies those that contain potential planetary transits, we calculate the transit parameters for these candidates. We used various techniques for vetting the light curves identified. For detailed information on these examinations, refer to Section \ref{sec:periodic_candidates}, Section \ref{sec:single_candidates} and Section \ref{sec:multiplanet_candidates}. \par
% check sectors with total>150000 LC


\begin{figure}
\begin{center}
\vspace{5mm}
\includegraphics[width=0.48\textwidth]{figures/pipeline_colors.pdf}
\vspace{-2mm} 
\caption{\label{fig:pipeline}
General diagram for the identification of new exoplanet candidates based on model predictions. The blue boxes represent the inputs to the NN and the outputs after inference (candidates and transit vetting). The grey boxes are related to the NN, which includes the creation of the input representation, used to train the model and subsequently perform inference to find candidates. After inference, the resulting candidates are evaluated through transit vetting to identify potential exoplanets for follow-up observations.}
\vspace{-1mm}
\end{center}
\end{figure}


\subsection{Single-planet multi-transit light curve candidates}
\label{sec:periodic_candidates}
Single-planet multi-transit light curve candidates are those where multiple transit events are observed at regular intervals in the light curve, indicating a consistent orbital period around the host star. Some planetary systems may not show multiple transits due to observational time, such as long orbital periods that exceed the duration of the observations. Additionally, systems with TTVs can cause variations in the timing of transits, making them deviate from strict periodicity. The multiple transits in the light curve at regular intervals not only help confirm the presence of the exoplanet candidate but also enable more precise estimations of its orbital parameters. \par

To verify these candidates, we first apply the Box-Least Squares 
\citep[BLS;][]{kovacs2002box} periodogram to compute the transit parameters. BLS is an algorithm designed to search for periodic signals in light curves, specifically focusing on those that have a box-shaped transit shape. After computing the transit parameters, we calculate the radius of the planet $r_{\mathrm{p}} = r_{\ast} \sqrt{d(c+1)}$, using the radius of the target star $r_{\ast}$ and the target's background contamination $c$. The stellar radius $r_{\ast}$ and the $c$ are obtained from the TIC, accessed through the Mikulski Archive for Space Telescopes (MAST) using \texttt{Astroquery} package. When the stellar radius is not available in the TIC, we use the  Gaia mission's \citep{vallenari2023gaia} data release 3 as a secondary source. We discard planet candidates with a predicted radius greater than 1.8$R_{\mathrm{Jupiter}}$. This threshold is set to focus on identifying planets that lie within the typical size range of gas giants and smaller exoplanets, as larger objects are more likely to be low-mass stars, brown dwarfs, or other non-planetary astrophysical phenomena \citep{hodvzic2018wasp, carmichael2020two}. According to the BLS outputs, approximately $20\%$ of the signals have physical parameters consistent with a planetary transit. \par

%$r_p= r_{s}\sqrt{d(c +1)$

After using BLS, we used the Discovery and Vetting of Exoplanets (DAVE) tool, developed by \cite{kostov2019discovery}. Only those signals that passed the initial BLS filter proceeded to this subsequent examination. DAVE evaluates the candidates by examining several factors, among them checking for secondary transit signals, evaluating changes in photometric centroid positions during and outside of the transit events, and comparing odd against even transits. The check for secondary transit signals helps identify binary star systems that might mimic planetary transits, as these typically exhibit a detectable secondary eclipse. The examination of photometric centroid shifts ensures that the detected signal is originated from the target star and not from nearby stars. In some cases, DAVE classified a TIC as a candidate in one sector but flagged it as a false positive in another. In these instances, we consider the TIC to be a false positive. After human vetting and reviewing the output of DAVE for candidates with a positive disposition, $\sim 25\%$ of the cases correspond to valid candidates. \par


The results from our model predictions, after completing the vetting process, are summarized in Table~\ref{tab:periodictransit_candidates}. The table presents the candidates along with their respective transit parameters, including target magnitude (TMag), transit depth, transit epoch, orbital period, transit duration, and radius. These parameters provide a detailed overview of the characteristics of the candidate. Figure~\ref{fig:periodic_transiters_radii} shows the distribution of radii for the identified candidates measured in Jupiter radii, providing insight into the range and frequency of detected planet sizes. The majority of the candidates are concentrated within a radius range between 0.75 and 1.25 Jupiter radii, with a significant peak around 1 Jupiter radius. We did not detect any Earth-sized exoplanets or smaller candidates with radii below 0.3 Jupiter radii. Figure \ref{fig:targetr_radii} shows that stars with radii between 0 and 1 solar radii host a large number of confirmed planets with radii between 0 and 2 Earth radii (upper x-axis), suggesting that smaller planets are more commonly found around smaller stars.  In contrast, our detected candidates (blue points) are concentrated in the 0.75–1.25 Jupiter radii range and mostly around stars between 0.8 and 2 solar radii. The absence of smaller planets in our findings can be attributed to several factors. For example, smaller planets, such as those near the size of Earth, typically produce shallower transits. Traditional methods such as BLS depend heavily on periodicity and may be better to detect shallower transit signals, since it considers multiple potential periods, thereby increasing detection likelihood. Our NN still has challenges in identifying these signals, especially in the presence of noise or stellar variability. Unlike BLS, our approach does not depend on periodicity, which could explain why our NN detects smaller candidates less frequently. \par

%right ascension (RA), declination (DEC)

We compared our exoplanet candidates with ExoFOP-TESS-confirmed planets. Figure~\ref{fig:tmag_vs_depth} shows the relationship between TESS magnitude and transit depth for confirmed exoplanets and for our candidates, indicating that our candidates are concentrated in an intermediate region of depth and magnitude compared to the confirmed planets. The relationship between the radius of exoplanets measured in Jupiter radii and their corresponding orbital periods is illustrated in Figure~\ref{fig:radius_vs_orbital_period}. The majority of confirmed exoplanets have orbital periods of less than $\sim$ 5 days, but there are also many with longer periods, showing great diversity. Among our candidates showing multi-transits, we also find some with periods shorter than 5 days, extending up to 25 days. \par

%single-planet multi-transit light curve

We identified two candidates with radii slightly exceeding the threshold of 1.8 $R_{\mathrm{Jupiter}}$. For candidates that exceed this threshold, we calculated their equilibrium temperature of the planet $T_{\mathrm{eq}}$ using Equation 4 from \cite{mendez2017equilibrium}, which is calculated from the balance between the stellar flux received from the host star and the flux absorbed by the planet \citep{selsis2007habitable}. For candidate TIC 385921743, we calculated an equilibrium temperature of $T_{\mathrm{eq}}$ = 2449.11 $\mathrm{K}$, and a radius of 1.85 $R_{\mathrm{Jupiter}}$. The radial velocity error (RV error) obtained from Gaia is 0.91 $\text{km s}^{-1}$. The second candidate, TIC 275614831, has an $T_{\mathrm{eq}}$ of 3870.08 $\mathrm{K}$, and a calculated radius of 1.82 $R_{\mathrm{Jupiter}}$, and an RV error of 1.50 $\text{km s}^{-1}$. 
Despite being slightly above the threshold, these candidates were included for further consideration due to their close proximity to the limit. The $T_{\mathrm{eq}}$ for both candidates are near the upper range observed for known hot Jupiters but remain within it, suggesting they could be gas giants in tight orbits around their host stars, making them interesting targets for additional follow-up observations. Moreover, the radial velocity errors are relatively low, which suggests a reasonable level of confidence in their classification as planetary candidates.

%$ r_p= r_{s} \sqrt{d(c +1)}$. 

\subsection{Single-transit light curve candidates}
\label{sec:single_candidates}
Single transiters are exoplanet candidates where only one transit has been observed in the light curve of a single TESS sector. This typically happens because TESS observes each sector for 27 days, and the orbital period of these candidates is longer than the duration of a single sector light curve. Additional transits may appear in other sectors depending on the orbital period and the timing of observations. Longer-period transiting exoplanets are particularly valuable as they enable us to measure the densities of warm and cool planets \citep{wang2021aligned}. They often present unique opportunities for discovery, as they can represent long-period planets that may not be detectable through traditional methods, such as BLS. Furthermore, once these transits are detected, they will require further ground-based follow-up observations to determine the orbital period and differentiate from potential astrophysical or statistical false positives \citep{sullivan2015transiting}. Detecting and validating these signals is more challenging compared to multi-transiters, which show multiple transits within the light curve. \par

From the NN predictions, we perform a visual inspection and identify those that have only one transit signal corresponding to $\sim$ 400 TIC IDs, with the remaining cases evaluated as multi-transiters (see Section \ref{sec:periodic_candidates}). We verified that the detected transit signal originates from the target star itself and not from nearby stars that could contaminate the observation. Next, we discard those TIC IDs that have an RV error $>$ 2 $\text{km s}^{-1}$, with the RV measurements obtained from Gaia. This criterion allows us to eliminate potential false positives resulting from measurement uncertainties in radial velocity data. This filter was consistent with the radius of the candidates, as those with an RV error $>$ 2 $\text{km s}^{-1}$ generally had radii larger than 1.8 $R_{\mathrm{Jupiter}}$, with the exception of TIC 438122862, which has a radius of 0.93 $R_{\mathrm{Jupiter}}$ and presents only one single transit in Sector 6. After applying these filters, we verify whether the transits were detected in sectors beyond 26 to ensure consistency across multiple observations of the same transit event. Finally, for the TIC IDs identified, we extend the review to sectors from subsequent sectors, confirming the persistence of the signals or identifying any new ones. As a result, $\sim$ 20\% passed the vetting process, leaving 82 validated as single transiters.

Following our model predictions and the subsequent vetting process, we present in Table~\ref{tab:singletransit_candidates} a detailed list of candidates identified as single transiters. The distribution of single transit candidates as a function of their radii is shown in Figure~\ref{fig:single_transiters_radii} where most candidates have radii between 0.5 and 1.0 Jupiter radii, with a peak around 0.75 Jupiter radii (approximately 8.5 Earth radii). 

An example of a candidate identified by our NN in two different sectors is TIC 232616346. Figure~\ref{fig:232616346_lc} shows the light curves of this candidate identified in sectors 20 and 23. Both sectors show the same transit event with the same depth, which validates that the candidate is the same in both sectors. Similarly, other candidates were also detected in multiple sectors, while some were identified in only a single sector because additional transits may have occurred when TESS was not observing that sector. Figure~\ref{fig:single_transit_candidates} presents three additional examples of single transit candidates: TIC 233577004 in sector 14, TIC 341687821 in sector 21, and TIC 122522333 in sector 3. Each of these candidates displays a clear single transit event, with some having detections in other sectors and others identified in only one sector. These light curves show the type of signals identified by our model, suggesting that these candidates may correspond to longer-period exoplanets. \par
%highlighting how these transits were detected across different TESS sectors

Moreover, these examples underscore the challenges in analyzing single transit events. Our model demonstrates the ability to detect the presence of a transit signal, but it is essential to perform careful validation to discard false positives and ensure the reliability of the identified candidates. In this sense, continued monitoring and follow-up observations will be crucial for confirming these candidates.  \par

\subsection{Multi-planet system candidates}
\label{sec:multiplanet_candidates}
Multi-planet systems are planetary systems that host more than one planet orbiting the same star. These systems usually show multiple transits with different depths in their light curves. The varying depths can be attributed to the planets of different sizes from the host star. Additionally, the orbital period of the transits differs due to the different distances of the planets from the host star. 

Our NN is trained to identity transit signals in light curves, regardless of whether these signals have identical depths or orbital periods. The NN distinguishes between transit signals through the analysis of key features of the light curve, such as the shape of the dip, the variation in the brightness fo the star and the timing of the signal. As a result, the NN is able to identify multi-planet systems. Among these, TIC 193096383, TIC 221567884, TIC 118798035, and TIC 400049903 were detected across sectors 1 - 26. TIC 193096383 is illustrated in Figure~\ref{fig:multiplanet_system}, which shows the transits of two potential exoplanet candidates, ``b''(orange) and ``c' (blue). To validate these candidates, we implemented an approach that combines analyses of single-planet multi-transit and single-transit light curve candidates, as detailed in sections ~\ref{sec:periodic_candidates} and ~\ref{sec:single_candidates}. Although a detailed examination of multi-planet systems is beyond the scope of this work, we underscore the importance of further investigation of these TIC IDs. \par

\begin{figure}
\begin{center}
\vspace{5mm}
\includegraphics[width=0.48\textwidth]{figures/periodic_transiters_radii.pdf}
\vspace{-2mm} 
\caption{\label{fig:periodic_transiters_radii}
Distribution of single-planet multi-transit light curve candidate radii measured in Jupiter raidii.}
\vspace{-1mm}
\end{center}
\end{figure}


\begin{figure}
\begin{center}
\vspace{5mm}
\includegraphics[width=0.48\textwidth]{figures/single_transiters_radii.pdf}
\vspace{-2mm} 
\caption{\label{fig:single_transiters_radii}
Distribution of single-transit light curve candidate radii, measured in Jupiter radii.}
\vspace{-1mm}
\end{center}
\end{figure}

\subsection{Significant candidates of special interest}

As a result of this work, certain candidates emerge as particularly remarkable in the search for exoplanets due to their unique characteristics or potential for further study. We identify these candidates while noting that a more detailed analysis of them is beyond the focus of this work. We highlight the need for continued investigation and validation to uncover insights into planetary formation and system dynamics. \par

\subsubsection{TIC 221567884}
Our model initially detected TIC 221567884  as a candidate in sector 12. After further analysis using BLS, we identified two transits with different depths, suggesting it could be a multi-planet system. To discard the possibility that the TIC 221567884 belongs to an EB, we verified additional sectors $>$ 26 for the same TIC 221567884 to ensure the consistency of the signals across multiple observations. We identified transits in sectors 39 and 66, with a particular possible mutual transit event in sector 66. In systems with more than one planet, there is a unique opportunity for photometric mutual events, where planets pass in front of or obscure each other as seen from Earth. A mutual event specifically refers to this overlap between two planets from our perspective on Earth \citep{ragozzine2010value}. If confirmed, this mutual event would be of particular interest. \par

Figure~\ref{fig:221567884_s_lc} shows light curves for the TIC 221567884 across three different sectors from the TESS mission (sectors 12, 39, and 66). In the top panel (sector 12), there are slight flux variations with two transit events marked as ``b'' (green) and ``c'' (blue); both transit events are described in Table~\ref{tab:multiplanet_candidates}. In the middle panel (sector 39), the transit of candidate ``c'' (blue) is observed again, showing the same depth as in sector 12, along with the transit of candidate `d'' (orange). Finally, the bottom panel (sector 66) highlights the previously mentioned mutual transit event of candidates ``c'' and ``d'', with depths consistent with those observed for candidate c in sectors 12 and 39 and for candidate d in sector 39. This consistency supports their identification as potential exoplanets and suggests a potential indication of planetary interactions within the system.  \par


\subsubsection{TIC 118798035}
TIC 118798035 is a significant candidate within a multi-planet system, showing noteworthy TTVs. TTVs indicate deviations from the expected transit times based on a basic Keplerian model, reflecting the gravitational interactions between the planets in the system. This suggests that exoplanets in multi-planet systems often follow non-Keplerian orbits, leading to variations in the timing and duration of their transits \citep{agol2017transit}. \par

TIC 118798035 was identified as a candidate by our model in sector 13. To further investigate this system, we extended our analysis to sectors beyond 26. In sector 39, additional transits from potential candidates were observed, providing further evidence of a multi-planet system with TTVs. Figure~\ref{fig:118798035_lc} presents the light curves for this possible multi-planet system, with the top panel showing the light curve from sector 13 and the bottom panel showing the light curve from sector 39. In sector 13, transits for candidates ``b'' and ``c'' are clearly visible. In sector 39, the transit of both candidates ``b'' and ``c'' appear again with consistent depths. Furthermore, TTVs for candidate ``c'' is showed in sector 39. \par

\subsection{Analysis of false positives}

Despite incorporating information on centroids and background time series to help the model learn to distinguish false positives, some false positives are still detected by our NN. An example of this is TIC 296945443 from TESS Sector 12, which was detected by our NN as an exoplanet candidate but was later identified as a false positive by DAVE. The analysis of DAVE showed a shift in the photometric centroids, which was consistent with background star contamination. \par

Our NN was trained on SPOC light curves, which are calculated using various photometric apertures. In the case of TIC 296945443, the aperture selected for Sector 12 did not capture the brightness variation of a nearby contaminating star, resulting in no detected centroid shift and leading to a false positive. We also examined light curves from other sectors for the same target, and in Sector 38, SPOC detected a centroid shift that confirmed the presence of a nearby object causing the transit-like signal. Our NN detected this target in Sector 38 as a candidate but with a lower probability compared to the high probability assigned in Sector 12. This analysis is consistent with the results presented in Section \ref{sec:results_nn}, demonstrating that incorporating centroid information does improve detection probability. However, it also highlights that further improvement can be made to reduce false positives. For example, one possible solution is to increase the training set, including more examples of false positives with centroid shifts. This additional data could help the model better recognize cases where background stars or centroid shifts cause false positives and, thus, improve the ability to distinguish true exoplanet candidates from false positives.


%table for single-planet multi-transit light curve candidates

\begin{table}
    \centering 
    \caption{List of single-planet multi-transit light curve candidates.}
    \label{tab:periodictransit_candidates}
    \begin{tabular}{lccccc} 
        \hline 
        \hline
        TIC ID & Transit & Radius & Depth & Orbital& Duration \\ 
        & Epoch & (Jupiter &  & Period& \\
        &  & radii) & (ppm) & (days)& (hours)\\
        \hline 
        \csvreader[head to column names, late after line=\\] {candidates/periodictransitcandidatepart1.csv}{}{
            \TICID & \epoch & \radius & \tdepth & \orbitalperiod & \duration
        }
        %\hline 
    \end{tabular}
    %\caption*{Continued)} 
\end{table}


\begin{table}
    \centering 
    \caption*{(Continued) List of single-planet multi-transit light curve candidates.}
    \begin{tabular}{lccccc} 
        \hline 
        \hline
        TIC ID & Transit & Radius & Depth & Orbital& Duration \\ 
        & Epoch & (Jupiter &  & Period& \\
        &  & radii) & (ppm) & (days)& (hours)\\
        \hline
        \csvreader[head to column names, late after line=\\] {candidates/periodictransitcandidatepart2.csv}{}{
            \TICID & \epoch & \radius & \tdepth & \orbitalperiod & \duration
        }
        \hline 
    \end{tabular}
\end{table}


% table for single transiters
%\onecolumn
\begin{table}
    \centering 
    \caption{List of single-transit light curve candidates.}
    \label{tab:singletransit_candidates}
    \begin{tabular}{lccccc} 
        \hline 
        \hline
        TIC ID & Transit & Radius & Depth & RV Error & Sector \\ 
        & Epoch & (Jupiter &  & & \\
        &  & radii) & (ppm) & $\text{km s}^{-1}$ & \\
        \hline 
        \csvreader[head to column names, late after line=\\] {candidates/singletransitercandidatepart1d.csv}{}{
            \TICID & \epoch & \radius & \tdepth & \RVerror & \sector
        }
        %\hline 
    \end{tabular}
    %\caption*{Continued)} 
\end{table}

\begin{table}
    \centering 
    \caption*{(Continued) List of single-transit lighr curve candidates.}
    \begin{tabular}{lccccc} 
        \hline 
        \hline
        TIC ID & Transit & Radius & Depth & RV Error & Sector \\ 
        & Epoch & (Jupiter &  & & \\
        &  & radii) & (ppm) & $\text{km s}^{-1}$ & \\
        \hline 
        \csvreader[head to column names, late after line=\\] {candidates/singletransitercandidatepart2d.csv}{}{
            \TICID & \epoch & \radius & \tdepth & \RVerror & \sector
        }
        \hline 
    \end{tabular}
\end{table}
%\twocolumn



\begin{table}
    \centering
	\vspace{5mm}
	\caption{List of candidates identified as multi-planet system.}
	\label{tab:multiplanet_candidates}
	\vspace{4mm}
	\begin{tabular}{lcccc} 
		\hline
		\hline
        TIC ID  & TessMag  & Transit & Depth &  Radius\\
		 & & Epoch& (ppm) & (Jupiter\\
		&&&& radii)\\
        \hline
        TIC 193096383 (b)&12.23&1546.82&2870&0.57\\
        TIC 193096383 (c)&12.23&1559.01&9564&1.04\\
        TIC 221567884 (b)&9.09&1629.92&868&0.48\\
        TIC 221567884 (c)&9.09&1648.66&2169&0.75\\
        TIC 221567884 (d)&9.09&2372.65&3037&0.91\\
        TIC 118798035 (b)&11.29&1656.35&5970&0.86\\
        TIC 118798035 (c)&11.29&1663.91&10660&1.16\\
        TIC 400049903 (b)&11.61&1669.72&4041&0.61\\
        TIC 400049903 (c)&11.61&1778.60&2424&0.47\\
        \hline
        \hline		
	\end{tabular}
	\vspace{3mm}
\end{table}




\begin{figure}
\begin{center}
\vspace{5mm}
\includegraphics[width=0.48\textwidth]{figures/tmag_vs_depth2.pdf}
\vspace{-2mm} 
\caption{\label{fig:tmag_vs_depth}
Relationship between TESS magnitude and transit depth. Each point represents a confirmed planet from ExoFOP-TESS or our candidates.}
\vspace{-1mm}
\end{center}
\end{figure}


\begin{figure}
\begin{center}
\vspace{5mm}
\includegraphics[width=0.48\textwidth]{figures/log_radius_vs_orbital_period2.pdf}
\vspace{-2mm} 
\caption{\label{fig:log_radius_vs_orbital_period}
Relationship between radius and log orbital period. Each point represents a confirmed planet from ExoFOP-TESS or our candidates.}
\vspace{-1mm}
\end{center}
\end{figure}

  
%%%%
\begin{figure}
\begin{center}
\vspace{5mm}
\includegraphics[width=0.48\textwidth]{figures/targetr_radii.pdf}
\vspace{-2mm} 
\caption{\label{fig:targetr_radii}
Relationship between radius candidate and stellar radius. Each point represents a confirmed planet from ExoFOP-TESS or our candidates.}
\vspace{-1mm}
\end{center}
\end{figure}

%%%%



\begin{figure*}
\begin{center}
\vspace{5mm}
\subfloat{\includegraphics[width=0.93\textwidth]{figures/lightcurves/232616346_s20_lc.pdf}\label{fig:232616346_s20_lc}}\\
\vspace{2mm}
\subfloat{\includegraphics[width=0.93\textwidth]{figures/lightcurves/232616346_s23_lc.pdf}\label{fig:232616346_s23_lc}}
\vspace{2mm} 
\caption{Light curves for TIC 232616346 in sectors 20 and 23, showing the same transit event with consistent depth, validating the detection of the same candidate in both sectors. \label{fig:232616346_lc}}
\vspace{-1mm}
\end{center}
\end{figure*}



\begin{figure*}
\begin{center}
\vspace{5mm}
\subfloat{\includegraphics[width=0.93\textwidth]{figures/lightcurves/122522333_s03_lc.pdf}\label{fig:122522333_s03_lc}}\\
\vspace{2mm}
\subfloat{\includegraphics[width=0.93\textwidth]{figures/lightcurves/341687821_s21_lc.pdf}\label{fig:341687821_s21_lc}}
\vspace{2mm}
\subfloat{\includegraphics[width=0.93\textwidth]{figures/lightcurves/233577004_s14_lc.pdf}\label{fig:233577004_s14_lc}}
\vspace{2mm} 
\caption{
Three examples of single transiters from TESS observations for TIC 122522333, TIC 341687821 and TIC 233577004. Each candidate shows a single-transit event in the respective sector. While these candidates were also detected in other sectors (TIC 122522333 in sector 4, TIC 341687821 in sector 14, and TIC 233577004 in sectors 21), the figure shows only the transit events from the sectors presented.
\label{fig:single_transit_candidates}}
\vspace{-1mm}
\end{center}
\end{figure*}




\begin{figure*}
\begin{center}
\vspace{5mm}
\subfloat{\includegraphics[width=0.93\textwidth]{figures/lightcurves/221567884_s12_lc.pdf}\label{fig:221567884_s12_lc}}\\
\vspace{2mm}
\subfloat{\includegraphics[width=0.93\textwidth]{figures/lightcurves/221567884_s39_lc.pdf}\label{fig:221567884_s39_lc}}\\
\vspace{2mm}
\subfloat{\includegraphics[width=0.93\textwidth]{figures/lightcurves/221567884_s66_lc.pdf}\label{fig:221567884_s66_lc}}\\
\vspace{2mm} 
\caption{
Lightcurves for TIC 221567884 across sectors 12, 39, and 66, illustrating transit events ``b'', ``c'' and ``d'', and a potential mutual transit event between candidates ``c'' and ``d'' in sector 66. \label{fig:221567884_s_lc}}
\vspace{-1mm}
\end{center}
\end{figure*}

\begin{figure*}
\begin{center}
\vspace{5mm}
\subfloat{\includegraphics[width=0.93\textwidth]{figures/lightcurves/118798035_s13_lc.pdf}\label{fig:118798035_s13_lc}}\\
\vspace{2mm}
\subfloat{\includegraphics[width=0.93\textwidth]{figures/lightcurves/118798035_s39_lc.pdf}\label{fig:118798035_s39_lc}}
\vspace{2mm} 
\caption{Light curves for TIC 118798035 illustrating a system with two exoplanet candidates. In sector 13, transit signals for candidates ``b'' (orange line) and ``c'' (blue line) are detected. The depths are consistent across both sectors, with TTVs for candidate ``c'' in sector 39. \label{fig:118798035_lc}}
\vspace{-1mm}
\end{center}
\end{figure*}





\begin{figure*}
\begin{center}
\vspace{5mm}
\subfloat{\includegraphics[width=0.93\textwidth]{figures/lightcurves/193096383_s09_lc.pdf}\label{fig:193096383_s09_lc}}\\
\vspace{2mm} 
\caption{
Light curve for TIC 193096383 in sector 9, showing the transits of candidates b (orange) and c (blue). The transits of candidate b have the same depth, which differs from the depth observed in the transit of candidate c, suggesting a potential multi-planet system.\label{fig:multiplanet_system}}
\vspace{-1mm}
\end{center}
\end{figure*}



\if 0

\begin{figure*}
\begin{center}
\vspace{5mm}
\subfloat[]{\includegraphics[width=\textwidth]{figures/lightcurves/193096383_lc.pdf}\label{fig:193096383_lc}}\\
\vspace{2mm}
\subfloat[]{\includegraphics[width=\textwidth]{figures/lightcurves/221567884_lc_b.pdf}\label{fig:221567884_lc_b}}
\vspace{2mm}
\vspace{2mm}
\subfloat[]{\includegraphics[width=\textwidth]{figures/lightcurves/221567884_lc_d.pdf}\label{fig:193096383_lc_d}}
\vspace{2mm} 
\caption{
 multiplanet system TIC 221567884 (a) y 193096383 (b).}
\vspace{-1mm}
\end{center}
\end{figure*}


\begin{figure*}
\begin{center}
\vspace{5mm}
\includegraphics[width=1\textwidth]{figures/lightcurves/193096383_lc.pdf}
\vspace{-2mm} 
\caption{\label{fig:193096383_lc}
...}
\vspace{-1mm}
\end{center}
\end{figure*}

\begin{figure*}
    \centering
    \begin{subfigure}[t]{0.5\textwidth}
        \centering
        \includegraphics[width=\textwidth]{figures/radius_vs_orbital_period.pdf}
        %\caption{}
        \label{fig:radius_vs_orbital_period}
    \end{subfigure}
    \hspace{-0.5em} 
    \begin{subfigure}[t]{0.5\textwidth}
        \centering
        \includegraphics[width=\textwidth]{figures/radius_vs_orbital_period_zoom.pdf}
        %\caption{}
        \label{fig:radius_vs_orbital_period_zoom}
    \end{subfigure}
    \caption{...}
    \label{fig:radiusvspperiod}
\end{figure*}
\fi

