\section{Introduction}

The amount of astronomical data collected from both space and ground-based telescopes has increased rapidly. Given the vast amount of data, the task of analysis and examination becomes quite costly and is also prone to human error. Process automation plays an important role in helping to improve the efficiency and accuracy of detection and classification. \par


During the last decade, more than a million stars have been observed in the search for transiting exoplanets. Among the most notable contributors is NASA’s Kepler space telescope, launched in 2009 \citep{borucki2010kepler}. Kepler has identified nearly 4,000 potential planet candidates, of which approximately 2,700 have been confirmed as exoplanets to date. Kepler's successor, NASA's Transiting Exoplanet Survey Satellite \citep[TESS;][]{ricker2014transiting} was launched in April 2018 and is currently monitoring most of the sky searching for transiting exoplanet candidates. To date, TESS has discovered 561 confirmed exoplanets and identified thousands of additional planet candidates, significantly expanding our understanding of planetary systems beyond our solar system. TESS monitors millions of stars across approximately 90\% of the sky, emphasizing the detection of exoplanets around nearby and bright stars. TESS observes each sector for \approximately{}27 days before reorienting to observe the next sector. The data products produced by the TESS mission include two types of data: small summed image subarrays (``postage stamps'') centred on pre-selected targets, also known as target pixel files (TPFs), and summed full-frame images (FFIs), which capture collections of pixels observed simultaneously \citep{guerrero2021tess}. From these data products, the primary goal of TESS is to discover hundreds of transiting planets smaller than Neptune, including dozens comparable in size to Earth \citep{Ricker:2015}. \par


The current standardized identification process of transiting exoplanet candidates from photometric time series involves two main steps. First, long-term flux variations are removed, and periodic transit-like signals are identified by phase-folding the light curve to different orbital periods, times of transit and transit duration \citep[e.g.][]{tenenbaum:2012}. This process yields Threshold Crossing Events (TCEs), which include both potential exoplanets and false positives like eclipsing binaries and instrumental artifacts. The second step involves vetting TCEs by rejecting false positives based on the properties of the phase-folded transiting light curves, stellar properties, and follow-up observations. Typically, experts manually examine possible exoplanet transit signals, a labour-intensive process prone to human error and increasingly demanding due to large data volumes. To address this, several efforts have been made to automate the classification of light curves. \par


Meanwhile, with the advent of Deep Learning (DL) \cite {lecun2015deep}, neural network (NN) architectures have revolutionized numerous fields of scientific research, leading to substantial advancements in image processing, classification, facial recognition, voice recognition, and Natural Language Processing \cite[NLP;][]{krizhevsky2017imagenet, ciregan2012multi, he2016deep, voulodimos2018deep, vaswani2017attention}. DL techniques, particularly convolutional neural networks \citep[CNN;][]{krizhevsky2012imagenet}, have been notably applied to classification tasks, where classification in machine learning refers to the task of predicting the class to which the input data belongs. In particular, it was used to classify transit signals for the validation of exoplanet candidates and has also been applied in the transit search process, though to a smaller extent. In the field of exoplanet transit signal classification, 1D CNNs have been commonly used \citep[e.g.][]{osborn2020rapid, rao2021nigraha, tey2023identifying}. Specifically, the approaches focus on validating transit signals obtained through phase-folding light curves, which depend on prior transit parameters such as period, duration and depth derived from initial detection processes. These methods utilize both local and global views of folded light curves for distinguishing exoplanet transits from false positives, but they do not perform the initial transit search or parameter extraction themselves. Regarding the search for new exoplanet candidates, \cite{olmschenk2021identifying} developed a pipeline for detecting and vetting periodic transiting exoplanets. Their approach utilizes the TESS FFIs light curves and employs a 1D-CNN for candidate identification. Another approach, as demonstrated by \cite{cui2021identify}, employs a 2D object detection algorithm based on YOLOv3 network \citep{redmon2018yolov3incrementalimprovement}, which was trained on Kepler data to detect exoplanet signals. \par


As mentioned above, proposed DL pipelines have primarily focused on classifying transiting exoplanet signals, where transit detections are confirmed by aligning several repeated transits with the orbital period, resulting in enhanced signal detection. In contrast, few efforts have explored new techniques for discovering candidates without requiring prior transit parameters. Consequently, while current standardized pipelines for detecting transiting exoplanets have reached a high level of efficiency, they have at least two important limitations, namely: i) they require de-trending processes to remove long-term flux variations of stellar origin, which could vary the trend of the sequence, introduce artifacts into the light curve, or remove exoplanet transits; and ii) they rely on the periodicity of the transit signal in the light curve. These pipelines could be therefore missing exoplanets that present strong transit timing variations (TTV) and/or those that present just one transit in the light curve (single transiters). \par


Regarding single transiters, many signals in observed light curves arise from stochastic processes—whether instrumental or astrophysical—which can often mimic single transit events \citep{foreman2016population}. To accurately determine the orbital period of this single transit, detecting additional transits in multiple sectors is necessary, given that interruptions in the light curves typically require at least three transit events to reliably constrain the orbital period. This is because the presence of a single transit event does not provide sufficient information to confirm the periodic nature of the signal. This leads to considerable challenges that hinder a search for single transit events. For instance, the transit probability for long-period planets could be lower, particularly for those with orbital periods that exceed the duration of continuous observations \citep{hawthorn2024tess}. This is also due to the fact that transit probability $P_{\mathrm{tr}}$ is proportional to ${r_\ast}/a$ and for long-period planets, the semi-major axis $a$ is larger, resulting in a lower $P_{\mathrm{tr}}$. Despite the challenges associated with detecting single transiters, a small number of named ``monotransiters'' have been confirmed, where a single transit-like feature appears in the light curve but does not repeat \citep{gill2020ngts1, gill2020ngts2, lendl2020toi}.  \par

%it is crucial for the signal to be detected in another sector.

Due to observational biases, the vast majority of well-characterized transiting giant planets have orbital periods shorter than 10 days [e.g.]\citep{hartman:2019, magliano2023tess}. The existence of these planets, known as hot Jupiters, has challenged theories regarding the formation and evolution of giant planets in general \citep{pollack:96}. To solve these open questions, it is crucial to discover and characterize giant planets with longer periods (warm Jupiters) that are not extremely irradiated by their host stars, and their physical and orbital properties can be compared to outcomes of different formation and migration processes \citep{dawson:2018}. While TESS has significantly increased the sample of well-characterized transiting warm Jupiters \citep[e.g.][]{dawson:2021,grieves:2022,brahm:2023,battley:2024}, an important fraction of them might still be undiscovered because they are presented as single transiters in TESS data \citep[e.g.][]{gill:2020}. \par


In this work, we propose an approach to identify exoplanet transit signals within a light curve without requiring prior transit parameters. This enables us to discover new exoplanet candidates directly from the complete light curve, avoiding the initial dependence on traditional techniques that phase-fold the light curve to identify periodic transit-like signals. We propose an architecture based on a sequential model inspired by the Transformer \citep{vaswani2017attention}. This type of architecture has demonstrated its effectiveness in handling complex scenarios, outperforming other model types such as recurrent neural networks \citep[RNN;][]{cho2014learning} and CNNs, in particular for sequential data \citep{hawkins2004problem, lakew2018comparison, karita2019comparative}. Our architecture learns the characteristics of a planetary transit signal, regardless of whether the signal is periodic throughout the light curve. While instrumental systematics are typically removed during preprocessing, to compute light curves, stellar-origin variations remain in the data. Our NN is therefore exposed to both long-term and short-term stellar variability, including activity-related fluctuations on timescales of days. Specifically, the architecture is designed to identify and differentiate between transit signals and the variability of stars. This type of architecture has been explored and applied by \cite{10.1093/mnras/stad1173} for classifying exoplanet transit signals and distinguishing them from false positives, and our NN extends this approach to the search for new exoplanet candidates. In particular, our architecture is designed to make predictions directly from the light curve. The NN captures temporal patterns and long-range dependencies within light curves without depending on prior transit parameters, such as transit depth, transit period and transit duration. This independence from periodicity allows our NN to detect both multi-transit and single-transit signals, making it particularly effective for identifying single transiters that have often been missed by previous approaches focused primarily on light curves with multiple transits of the same planet. \par

%, even amid stellar variability.

This article is organized as follows: Section~\ref{sec:background} introduces the key concepts that inspired this work, where we briefly describe the Transformer encoder related to the original architecture \citep{vaswani2017attention} in order to understand our proposal for exoplanet transit signal identification. Section~\ref{sec:dataset_tess} describes the photometric data used in our work. Section~\ref{sec:ourmethod} explains the proposed methodology in detail, including the NN training process. The model analysis and results are described in Section ~\ref{sec:model_analysis}. In Section ~\ref{sec:search_candidates}, we describe the exoplanet candidates found. Finally, in Section~\ref{sec:conclusions} we present overall conclusions and future work. 

