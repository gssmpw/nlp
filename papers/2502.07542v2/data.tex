\section{TESS Data}
\label{sec:dataset_tess}

For the present work, we have used light curve data obtained from the Mikulski Archive for Space Telescopes (MAST). The light curves are extracted from the TESS full-frame images (FFIs), which were captured at a cadence of 30 minutes during its primary mission. The time stamps for these measurements are given in Barycentric TESS Julian Date (BTJD) format, which is specific to the TESS mission. BTJD is defined as $\mathrm{BTJD = BJD - 2457000.0}$ days, where BJD refers to the Barycentric Julian Date. The use of BTJD ensures precise timing by taking into account the motion of Earth and other relativistic effects. To obtain the light curves, we use the outputs of the TESS Science Processing Operations Center (SPOC) pipeline \citep{caldwell2020tess} (see Section~\ref{sec:tess_spoc_ffi}). SPOC pipeline processes the data of targets selected from the FFIs to create target pixel and light curve files for up to 150,000 targets per sector. By selecting a specific set of target stars from these FFIs, SPOC generates the same outputs as for the pre-selected 2-min cadence targets, including calibrated target pixel files and simple aperture photometry (SAP) flux.  \par

%, which is derived by subtracting 2,457,000.0 from the Julian Date in the Barycentric Dynamical Time (BJD) standard

\subsection{TESS-SPOC FFI light curves}
\label{sec:tess_spoc_ffi}

For the primary mission of TESS, the FFIs are captured at a 30-minute cadence and are crucial for large-scale surveys, enabling the simultaneous monitoring of thousands of stars. To optimize the utility of FFI data, SPOC employs a selection process that prioritizes targets based mainly on their crowding metric \citep{bryson2020kepler} and brightness. Specifically, targets with a crowding metric of $\geq$ 0.5 are selected, ensuring that at least 50\% of the flux within the photometric aperture originates from the target star itself. Additionally, SPOC focuses on TESS Input Catalog (TIC) objects with TESS magnitudes of T $\leq$ 16, balancing the need for both bright and isolated targets to maximize the scientific return from the FFI data \citep{caldwell2020tess}. Building on this, our architecture is trained with the light curves processed by SPOC from TESS sectors 1-26. Each sector was observed continuously for about 27 days, with FFIs captured at a 30-minute cadence. \par

The light curves computed by SPOC from the FFIs include the stellar flux (PDCSAP\_FLUX), which has been corrected for instrumental systematics and shows fewer systematic trends. In addition to the flux, SPOC also provides detailed information on the centroid position of the target star and the background flux. The centroids are a time series that represents the pixel position of the centre of the light, which varies throughout the observation. This information allows us to determine the location of the transit within the pixels. When the centroid shifts during a transit event, we can identify whether the observed flux variations are coming from the target star or from nearby sources. The background time series represents the estimated flux contribution from nearby sources, such as nearby stars or unresolved objects, to the target aperture. It is derived from pixels located outside the photometric aperture but within the same CCD frame. This estimation aims to minimize the influence of nearby stars, though it is not always possible to completely exclude their effect, but their contribution is minimized. Both time series allow the distinction between true transits and other astrophysical phenomena that might affect the measured light curve caused by background contamination or blended stellar sources. We use these data sets; PDCSAP\_FLUX, centroids, and background time series as inputs to our architecture, enhancing the accuracy and reliability of exoplanet transit detection. \par

%This information allows us to know the location of the transit in the pixels.
%The background time series captures the light from sources other than the target star