\section{Related Work}
NLP research for AAE have explored social media use \citep{blodgett-etal-2018-twitter}, POS-tagging \citep{dacon2022deep,JorgensenHS16}, hate speech classification \citep{Harrisetalhatespeechgrammar2022,SapCGCS19}, ASR \citep{koenecke2020racial,Martin_Tang_RacialBias_HabBe_Interspeech2020_2020}, dialectal analysis
\citep{BlodgettGO16,dacon2022deep,Stewart14} and feature detection \citep{masis-etal-2022-corpus,santiago_disambiguation_2022,previlon-etal-2024-leveraging}. Our work highlights the lack of attention to AAE's distinctive grammatical structure that can be used to identify its usage accurately. Methods mitigating bias in NLP often neglect AAE's grammatical features in favor of lexical choice 
\citep{BarikeriLVG20,ChengK022,GarimellaMA22,hwang-etal-2020-towards, KiritchenkoM18,maronikolakis-etal-2022-analyzing,silva-etal-2021-towards}. Some research removes AAE's morphological features \citep{tan-etal-2020-mind} or translates between MAE and AAE \citep{ziems-etal-2023-multi}.