%%
%% This is file `sample-sigconf.tex',
%% generated with the docstrip utility.
%%
%% The original source files were:
%%
%% samples.dtx  (with options: `all,proceedings,bibtex,sigconf')
%% 
%% IMPORTANT NOTICE:
%% 
%% For the copyright see the source file.
%% 
%% Any modified versions of this file must be renamed
%% with new filenames distinct from sample-sigconf.tex.
%% 
%% For distribution of the original source see the terms
%% for copying and modification in the file samples.dtx.
%% 
%% This generated file may be distributed as long as the
%% original source files, as listed above, are part of the
%% same distribution. (The sources need not necessarily be
%% in the same archive or directory.)
%%
%%
%% Commands for TeXCount
%TC:macro \cite [option:text,text]
%TC:macro \citep [option:text,text]
%TC:macro \citet [option:text,text]
%TC:envir table 0 1
%TC:envir table* 0 1
%TC:envir tabular [ignore] word
%TC:envir displaymath 0 word
%TC:envir math 0 word
%TC:envir comment 0 0
%%
%% The first command in your LaTeX source must be the \documentclass
%% command.
%%
%% For submission and review of your manuscript please change the
%% command to \documentclass[manuscript, screen, review]{acmart}.
%%
%% When submitting camera ready or to TAPS, please change the command
%% to \documentclass[sigconf]{acmart} or whichever template is required
%% for your publication.
%%
%%
\documentclass[sigconf]{acmart}
% \documentclass[sigconf]{acmart}
%%
%% \BibTeX command to typeset BibTeX logo in the docs
\AtBeginDocument{%
  \providecommand\BibTeX{{%
    Bib\TeX}}}

%% Rights management information.  This information is sent to you
%% when you complete the rights form.  These commands have SAMPLE
%% values in them; it is your responsibility as an author to replace
%% the commands and values with those provided to you when you
%% complete the rights form.
\setcopyright{acmlicensed}
\copyrightyear{2025}
\acmYear{2025}
\acmDOI{XXXXXXX.XXXXXXX}
%% These commands are for a PROCEEDINGS abstract or paper.
\acmConference[Conference acronym 'XX]{Make sure to enter the correct
  conference title from your rights confirmation email}{June 03--05,
  2018}{Woodstock, NY}
%%
%%  Uncomment \acmBooktitle if the title of the proceedings is different
%%  from ``Proceedings of ...''!
%%
%%\acmBooktitle{Woodstock '18: ACM Symposium on Neural Gaze Detection,
%%  June 03--05, 2018, Woodstock, NY}
\acmISBN{978-1-4503-XXXX-X/2018/06}


%%
%% Submission ID.
%% Use this when submitting an article to a sponsored event. You'll
%% receive a unique submission ID from the organizers
%% of the event, and this ID should be used as the parameter to this command.
%%\acmSubmissionID{123-A56-BU3}

%%
%% For managing citations, it is recommended to use bibliography
%% files in BibTeX format.
%%
%% You can then either use BibTeX with the ACM-Reference-Format style,
%% or BibLaTeX with the acmnumeric or acmauthoryear sytles, that include
%% support for advanced citation of software artefact from the
%% biblatex-software package, also separately available on CTAN.
%%
%% Look at the sample-*-biblatex.tex files for templates showcasing
%% the biblatex styles.
%%

%%
%% The majority of ACM publications use numbered citations and
%% references.  The command \citestyle{authoryear} switches to the
%% "author year" style.
%%
%% If you are preparing content for an event
%% sponsored by ACM SIGGRAPH, you must use the "author year" style of
%% citations and references.
%% Uncommenting
%% the next command will enable that style.
%%\citestyle{acmauthoryear}

\usepackage[utf8]{inputenc} % allow utf-8 input
\usepackage[T1]{fontenc}    % use 8-bit T1 fonts
% \usepackage{newtxtext,newtxmath}
% \usepackage{lmodern}
% \setlength{\parskip}{0pt}
% \usepackage{mathptmx}
\usepackage{hyperref}       % hyperlinks
\usepackage{url}            % simple URL typesetting
\usepackage{booktabs}       % professional-quality tables
\usepackage{amsfonts}       % blackboard math symbols
\usepackage{nicefrac}       % compact symbols for 1/2, etc.
\usepackage{microtype}      % microtypography
\usepackage{xcolor} 

\usepackage{subfigure}
% \usepackage{subfig} % for subfigures

\usepackage{multirow}
\usepackage{tabularx}
% \newtheorem{theorem}{Theorem}[Chapter]
\usepackage{adjustbox}
\usepackage{tikz}
% \usepackage{booktabs,subcaption,amsmath}

\usepackage{wrapfig}
\usepackage{mathtools}
\usepackage[normalem]{ulem}
\usepackage{tabularx}
\usepackage{threeparttable}
\usepackage{amsmath,bm}
\usepackage{changepage}
\usepackage{mathtools}
\usepackage{makecell}
\usepackage{colortbl}
% \theoremstyle{definition}
\newtheorem{exmp}{Example}

%%
%% end of the preamble, start of the body of the document source.
\begin{document}

%%
%% The "title" command has an optional parameter,
%% allowing the author to define a "short title" to be used in page headers.
\title{MIM: Multi-modal Content Interest Modeling Paradigm for User Behavior Modeling}

%%
%% The "author" command and its associated commands are used to define
%% the authors and their affiliations.
%% Of note is the shared affiliation of the first two authors, and the
%% "authornote" and "authornotemark" commands
%% used to denote shared contribution to the research.

\author{
    Bencheng Yan$^{1*}$, Si Chen$^{1*}$, Shichang Jia$^{2}$, Jianyu Liu$^{1}$, 
    Yueran Liu$^{1}$, Chenghan Fu$^{1}$, Wanxian Guan$^{1}$, Hui Zhao$^{1}$, 
    Xiang Zhang$^{1}$, Kai Zhang$^{1}$, Wenbo Su$^{1}$, Pengjie Wang$^{1}$, 
    Jian Xu$^{1}$, Bo Zheng$^{1\dagger}$, Baolin Liu$^{2}$ \\
}

\thanks{$^{*}$ Bencheng Yan and Si Chen contributed equally to this work.}
\thanks{$^{\dagger}$ Bo Zheng(bozheng@alibaba-inc.com) is the corresponding author.}

\affiliation{%
  \institution{$^{1}$Alibaba Group, Beijing, China\\$^{2}$University of Science and Technology Beijing, Beijing, China}
  \city{}
  \state{}
  \country{}
}


% \email{bozheng@alibaba-inc.com} % 仅列出通讯作者的邮箱

% \author{Bencheng Yan$^{1}$}
% \authornote{These authors contributed equally to this work.} % 标注共同一作
% % \email{bencheng.ybc@taobao.com}
% \affiliation{%
%   \institution{}
%   \city{}
%   \state{}
%   \country{}
% }

% \author{Si Chen$^{1}$}
% \authornotemark[1] % 复用共同一作的标注
% % \email{yasui.cs@taobao.com}
% \affiliation{%
%   \institution{}
%   \city{}
%   \state{}
%   \country{}
% }

% \author{Shichang Jia$^{2}$}
% % \email{m202320951@xs.ustb.edu.cn}
% \affiliation{%
%   \institution{}
%   \city{}
%   \state{}
%   \country{}
% }

% \author{Jianyu Liu$^{1}$}
% % \email{liuyu.ljy@taobao.com}
% \affiliation{%
%   \institution{}
%   \city{}
%   \state{}
%   \country{}
% }

% \author{Yueran Liu$^{1}$}
% % \email{tianer.lyr@taobao.com}
% \affiliation{%
%   \institution{}
%   \city{}
%   \state{}
%   \country{}
% }

% \author{Chenghan Fu$^{1}$}
% % \email{fuchenghan.fch@taobao.com}
% \affiliation{%
%   \institution{}
%   \city{}
%   \state{}
%   \country{}
% }

% \author{Wanxian Guan$^{1}$}
% % \email{wanxian.gwx@taobao.com}
% \affiliation{%
%   \institution{}
%   \city{}
%   \state{}
%   \country{}
% }

% \author{Hui Zhao$^{1}$}
% % \email{shuqian.zh@taobao.com}
% \affiliation{%
%   \institution{}
%   \city{}
%   \state{}
%   \country{}
% }

% \author{Xiang Zhang$^{1}$}
% % \email{genshen.zx@taobao.com}
% \affiliation{%
%   \institution{}
%   \city{}
%   \state{}
%   \country{}
% }

% \author{Kai Zhang$^{1}$}
% % \email{victorlanger.zk@taobao.com}
% \affiliation{%
%   \institution{}
%   \city{}
%   \state{}
%   \country{}
% }

% \author{Wenbo Su$^{1}$}
% % \email{vincent.swb@taobao.com}
% \affiliation{%
%   \institution{}
%   \city{}
%   \state{}
%   \country{}
% }

% \author{Pengjie Wang$^{1}$}
% % \email{pengjie.wpj@alibaba-inc.com}
% \affiliation{%
%   \institution{}
%   \city{}
%   \state{}
%   \country{}
% }

% \author{Jian Xu$^{1}$}
% % \email{xiyu.xj@taobao.com}
% \affiliation{%
%   \institution{}
%   \city{}
%   \state{}
%   \country{}
% }

% \author{Bo Zheng$^{1}$}
% \authornote{Corresponding author.} % 标注通讯作者
% \email{bozheng@alibaba-inc.com}
% \affiliation{%
%   \institution{}
%   \city{}
%   \state{}
%   \country{}
% }

% \author{Baolin Liu$^{2}$}
% % \email{liubaolin@ustb.edu.cn}
% \affiliation{%
%   \institution{}
%   \city{}
%   \state{}
%   \country{}
% }

% \author{}
% \affiliation{%
%   \institution{$^{1}$Alibaba Group}
%   \city{}
%   \state{Beijing}
%   \country{China}
% }

% \author{}
% \affiliation{%
%   \institution{$^{2}$University of Science and Technology Beijing}
%   \city{}
%   \state{Beijing}
%   \country{China}
% }


% \author{Bencheng Yan}
% \authornote{These authors contributed equally to this work.} % 标注共同一作
% \email{bencheng.ybc@taobao.com}
% \affiliation{%
%   \institution{Alibaba Group}
%   \city{}
%   \state{Beijing}
%   \country{China}
% }

% \author{Si Chen}
% \authornotemark[1] % 复用共同一作的标注
% \email{yasui.cs@taobao.com}
% \affiliation{%
%   \institution{Alibaba Group}
%   \city{}
%   \state{Beijing}
%   \country{China}
% }

% \author{Shichang Jia}
% \email{m202320951@xs.ustb.edu.cn}
% \affiliation{%
%   \institution{University of Science and Technology Beijing}
%   \city{}
%   \state{Beijing}
%   \country{China}
% }

% \author{Jianyu Liu}
% \email{liuyu.ljy@taobao.com}
% \affiliation{%
%   \institution{Alibaba Group}
%   \city{}
%   \state{Beijing}
%   \country{China}
% }

% \author{Yueran Liu}
% \email{tianer.lyr@taobao.com}
% \affiliation{%
%   \institution{Alibaba Group}
%   \city{}
%   \state{Beijing}
%   \country{China}
% }

% \author{Chenghan Fu}
% \email{fuchenghan.fch@taobao.com}
% \affiliation{%
%   \institution{Alibaba Group}
%   \city{}
%   \state{Beijing}
%   \country{China}
% }

% \author{Wanxian Guan}
% \email{wanxian.gwx@taobao.com}
% \affiliation{%
%   \institution{Alibaba Group}
%   \city{}
%   \state{Beijing}
%   \country{China}
% }

% \author{Hui Zhao}
% \email{shuqian.zh@taobao.com}
% \affiliation{%
%   \institution{Alibaba Group}
%   \city{}
%   \state{Beijing}
%   \country{China}
% }

% \author{Xiang Zhang}
% \email{genshen.zx@taobao.com}
% \affiliation{%
%   \institution{Alibaba Group}
%   \city{}
%   \state{Beijing}
%   \country{China}
% }

% \author{Kai Zhang}
% \email{victorlanger.zk@taobao.com}
% \affiliation{%
%   \institution{Alibaba Group}
%   \city{}
%   \state{Beijing}
%   \country{China}
% }

% \author{Wenbo Su}
% \email{vincent.swb@taobao.com}
% \affiliation{%
%   \institution{Alibaba Group}
%   \city{}
%   \state{Beijing}
%   \country{China}
% }

% \author{Pengjie Wang}
% \email{pengjie.wpj@alibaba-inc.com}
% \affiliation{%
%   \institution{Alibaba Group}
%   \city{}
%   \state{Beijing}
%   \country{China}
% }

% \author{Jian Xu}
% \email{xiyu.xj@taobao.com}
% \affiliation{%
%   \institution{Alibaba Group}
%   \city{}
%   \state{Beijing}
%   \country{China}
% }

% \author{Bo Zheng}
% \authornote{Corresponding author.} % 标注通讯作者
% \email{bozheng@alibaba-inc.com}
% \affiliation{%
%   \institution{Alibaba Group}
%   \city{}
%   \state{Beijing}
%   \country{China}
% }

% \author{Baolin Liu}
% \email{liubaolin@ustb.edu.cn}
% \affiliation{%
%   \institution{University of Science and Technology Beijing}
%   \city{}
%   \state{Beijing}
%   \country{China}
% }

%%
%% By default, the full list of authors will be used in the page
%% headers. Often, this list is too long, and will overlap
%% other information printed in the page headers. This command allows
%% the author to define a more concise list
%% of authors' names for this purpose.
\renewcommand{\shortauthors}{Yan et al.}

%%
%% The abstract is a short summary of the work to be presented in the
%% article.
\begin{abstract}
  Click-Through Rate (CTR) prediction is a crucial task in recommendation systems, online searches, and advertising platforms, where accurately capturing users' real interests in content is essential for performance. However, existing methods heavily rely on ID embeddings, which fail to reflect users' true preferences for content such as images and titles. This limitation becomes particularly evident in cold-start and long-tail scenarios, where traditional approaches struggle to deliver effective results.
  To address these challenges, we propose a novel \textbf{M}ulti-modal Content \textbf{I}nterest \textbf{M}odeling paradigm (\textbf{MIM}), which consists of three key stages: Pre-training, Content-Interest-Aware Supervised Fine-Tuning (C-SFT), and Content-Interest-Aware UBM (CiUBM). 
  The pre-training stage adapts foundational models to domain-specific data, enabling the extraction of high-quality multi-modal embeddings. 
  The C-SFT stage bridges the semantic gap between content and user interests by leveraging user behavior signals to guide the alignment of embeddings with user preferences. 
  Finally, the CiUBM stage integrates multi-modal embeddings and ID-based collaborative filtering signals into a unified framework. Comprehensive offline experiments and online A/B tests conducted on the Taobao, one of the world's largest e-commerce platforms, demonstrated the effectiveness and efficiency of MIM method. The method has been successfully deployed online, achieving a significant increase of +14.14\% in CTR and +4.12\% in RPM, showcasing its industrial applicability and substantial impact on platform performance. To promote further research, we have publicly released the code and dataset at https://pan.quark.cn/s/8fc8ec3e74f3.  
\end{abstract}

%%
%% The code below is generated by the tool at http://dl.acm.org/ccs.cfm.
%% Please copy and paste the code instead of the example below.
%%
%% CCS Concepts
\begin{CCSXML}
<ccs2012>
    <concept>
        <concept_id>10002951.10003317.10003347.10003350</concept_id>
        <concept_desc>Information systems~Recommender systems</concept_desc>
        <concept_significance>500</concept_significance>
        </concept>
  </ccs2012>
\end{CCSXML}

\ccsdesc[500]{Information systems~Recommender systems}


%%
%% Keywords. The author(s) should pick words that accurately describe
%% the work being presented. Separate the keywords with commas.
\keywords{Multi-modal Representations, Click-Through Rate Prediction, Recommendation System, E-commerce Search}

%% A "teaser" image appears between the author and affiliation
%% information and the body of the document, and typically spans the
%% page.
% \begin{teaserfigure}
%   \includegraphics[width=\textwidth]{sampleteaser}
%   \caption{Seattle Mariners at Spring Training, 2010.}
%   \Description{Enjoying the baseball game from the third-base
%   seats. Ichiro Suzuki preparing to bat.}
%   \label{fig:teaser}
% \end{teaserfigure}

% \received{20 February 2007}
% \received[revised]{12 March 2009}
% \received[accepted]{5 June 2009}

%%
%% This command processes the author and affiliation and title
%% information and builds the first part of the formatted document.
\maketitle

\section{Introduction}
\label{sec:Introduction}
Click-Through Rate (CTR) prediction plays a vital role in applications such as recommendation systems, web searches, and online advertising\cite{covington2016Deep,yan2022apg}, as it directly impacts user engagement and platform revenue. Among its key components, User Behavior Modeling (UBM) has emerged as a critical optimization direction. By leveraging users' historical interactions, UBM effectively captures their underlying preferences, enabling more accurate predictions and enhancing overall system performance\cite{zhang2021deep,zhou2018deep,pi2020search}.

Traditional UBM methods predominantly rely on ID embeddings to represent items and user behaviors(see Figure\ref{figure:An example of ID interest and content interest modeling in UBM.} blue part). While effective in some scenarios, ID-based approaches face inherent limitations. First, ID embeddings primarily encode collaborative filtering (CF) signals but fail to effectively capture user preferences for content, such as images and titles, resulting in a misalignment between representations and actual user interests. Second, these methods require abundant user-item interaction data, resulting in poor performance in cold-start and long-tail scenarios\cite{yuan2021One,yuan2020ParameterEfficient}.

\begin{figure}[t]
\centering
\includegraphics[width = .4\textwidth]{imgs/intro.png}
\caption{An example of ID interest and content interest modeling in UBM.}
\label{figure:An example of ID interest and content interest modeling in UBM.}
\vspace{-2em}
\end{figure}

To overcome these limitations, there is a growing need for content-based multi-modal UBMs(see Figure\ref{figure:An example of ID interest and content interest modeling in UBM.} green part). This shift is motivated by two key factors. First, user interactions with recommendation systems are predominantly mediated through visual and textual content, such as product images and descriptions. Modeling these features is essential for accurately reflecting user preferences. Second, recent advancements in multi-modal foundation models (FoMs), such as Vision Transformers (ViT)\cite{dosovitskiy2020Image,liu2021Swin}, LLaMA\cite{touvron2023llamaopenefficientfoundation}, Vicuna\cite{vicuna2023}, BEiT-3\cite{wang2022image}, GPT-4\cite{openai2024gpt4technicalreport}, and so on, provide powerful tools for extracting rich semantic information from the visual and textual content.

Existing multi-modal user behavior modeling methods can be broadly categorized into two classes: two-stage pretraining methods and end-to-end training methods. Two-stage pretraining methods leverage pre-trained foundation models to extract multi-modal features, which are then integrated into user behavior models. End-to-end methods, on the other hand, jointly optimize multi-modal models and user behavior modeling modules, demonstrating their potential in better aligning user interests with content semantics. Despite achieving considerable success, these approaches also have limitations: two-stage methods lack deep alignment between multi-modal content and user interests, which constrains their effectiveness; end-to-end methods, while offering better alignment, come with high training costs, complex deployment requirements, and limited generalizability, making them less adaptable to diverse recommendation tasks. These limitations underscore the need for a universal and efficient approach to construct high-quality representations that align content with user interests.To address these challenges, three core questions must be answered. 

\noindent \textbf{$\bullet$ Q1: what kind of multi-modal embeddings are needed?} Pre-trained FoMs often fail to fully understand domain-specific content, and a semantic gap exists between content embeddings and user interests. 

\noindent \textbf{$\bullet$ Q2: how can multi-modal embeddings be effectively utilized in UBMs?} A unified framework is needed to integrate ID-based CF signals and multi-modal embeddings seamlessly. 

\noindent \textbf{$\bullet$ Q3: how can efficiency be ensured in large-scale applications?} Multi-modal embedding extraction often introduces high computational and memory costs, posing challenges for practical deployment.

To solve these challenges, this paper introduces MIM (Multi-modal Content Interest Modeling), a universal and efficient paradigm for multi-modal UBM. MIM adopts a three-stage framework to address the core questions comprehensively. In the pretraining stage, foundation models are adapted to downstream data to capture domain-specific content and align multi-modal features. In the content-interest-aware supervised fine-tuning (C-SFT) stage, user interest signals, such as purchase behaviors, are introduced to bridge the semantic gap and guide embeddings to reflect user preferences. In the CiUBM stage, a modular content-interest-aware UBM integrates ID-based CF signals, multi-modal content embeddings, and their interactions, providing an effective and flexible framework for user behavior modeling. To ensure efficiency, MIM introduces a representation center that precomputes and stores embeddings for fast retrieval, significantly reducing training and inference costs.

In summary, this paper makes the following contributions: (1) We propose a universal and effective multi-modal UBM method, MIM, which shifts the paradigm from ID-based to content-based interest modeling, enabling broad applicability and performance improvements across existing UBMs. (2) MIM demonstrates significant performance advantages, achieving lightweight training and efficient inference, making it scalable for industrial applications. (3) MIM has been successfully deployed in large-scale industrial scenarios, achieving 14.14\% CTR gains and 4.12\% RPM gains. To foster further research, we have publicly released the code and datasets.

\begin{figure*}[ht]
  \centering
  \includegraphics[width = 0.95\textwidth]{imgs/T1-Par_framework3.png}
  
  \caption{The framework of MIM. There are a total of three stages, including Pre-training, C-SFT, and CiUBM.
  Besides, a representation center is built for efficiency consideration.}
  
  \label{figure:The framework of MIM}
  \end{figure*}

\section{Related Works}
\label{sec:Related Work}
\subsection{User Behavior Modeling}
User Behavior Modeling (UBM) has been widely studied as a critical component of Click-Through Rate (CTR) prediction due to its ability to capture users' historical preferences and predict their future interactions. Traditional UBM methods, such as DIN\cite{zhou2018deep}, use attention mechanisms to selectively focus on relevant behaviors, while extensions like DIEN\cite{zhou2019deep} incorporate sequential modeling to capture the temporal evolution of user interests. Despite their success, these methods heavily rely on ID-based embeddings, which primarily encode collaborative filtering (CF) signals but fail to effectively model content-based preferences, particularly for items such as images and texts. This limitation becomes pronounced in cold-start and long-tail scenarios where interaction data is sparse\cite{yuan2021One,yuan2020ParameterEfficient}. Recent efforts, such as the introduction of hybrid UBM approaches, have attempted to integrate content features but still lack a unified framework for aligning content semantics with user-specific interests.
\subsection{Multi-modal Recommendation}
Multi-modal recommendation integrates diverse content features such as images, texts, and videos to improve user understanding and prediction accuracy. Traditional methods often augment CTR models with multi-modal (MM) features as additional attributes, directly feeding them into the models to enhance representation power \cite{mo2015image}. While effective to some extent, these approaches neglect the critical need to align MM features with user-specific interests, resulting in a limited understanding of user preferences.Recent advancements focus on bridging the domain gap between pre-trained foundation models and downstream recommendation tasks\cite{yuan2023Where, wang2023missrec, wu2021empowering, liu2023Multimodal}. However, these methods largely ignore the role of explicit user interest modeling in refining MM embeddings, limiting their ability to capture sophisticated user preferences.
Moreover, the practicality of existing methods in industrial applications remains a challenge. Many end-to-end frameworks incur high computational and memory costs, making them inefficient for large-scale deployment \cite{yuan2023Where}. 
To address these issues, our proposed MIM paradigm systematically bridges the semantic gap between MM features and user interests while maintaining scalability and efficiency, achieving significant performance gains in real-world applications.


\subsection{Foundation Models}
Pre-trained Foundation Models have significantly advanced the fields of Computer Vision (CV) and Natural Language Processing (NLP) by learning transferable representations from large-scale data. 
In CV, models like iGPT\cite{chen2020generative} and ViT\cite{dosovitskiy2020Image} have pioneered the application of transformer architectures for image recognition, leveraging self-supervised tasks such as masked patch prediction to capture rich visual semantics.
Further advancements, such as Swin Transformer\cite{liu2021Swin} and BEiT\cite{bao2021beit}, improved efficiency and scalability, making transformers a dominant paradigm in vision tasks.In NLP, foundational models like BERT\cite{devlin2019BERT},GPT-2\cite{radford2019language}, and XLNet\cite{yang2019xlnet} introduced pre-training on large corpora followed by task-specific fine-tuning, establishing a new standard for natural language understanding. These models have been extended to handle complex tasks, such as question answering, summarization, and sentiment analysis.
The integration of multi-modal learning has further broadened the applicability of foundation models. Methods like CLIP\cite{radford2021learning} propose contrastive pre-training techniques to align visual and textual modalities, enabling robust joint representations. 
Meanwhile, ViL-BERT\cite{lu2019vilbert} and LXMERT\cite{tan2019lxmert} extend the transformer framework to learn cross-modal interactions, achieving state-of-the-art performance on vision-language tasks. More recent models such as Flamingo\cite{alayrac2022flamingo} and BEiT-3\cite{wang2022image} demonstrate the effectiveness of scaling multi-modal architectures for a variety of downstream applications.

Despite these advancements, challenges remain in achieving efficient training and deployment of multi-modal foundation models, particularly in scenarios requiring alignment between diverse modalities and scalability in large-scale applications.




\section{MIM}

\subsection{Preliminaries and Method Overview}
\label{sec:Preliminaries and Method Overview}


\subsubsection{Preliminaries.}

\noindent \textbf{\newline CTR Prediction.} Considering a typical item search scenario\footnote{Note that in this section, we take the item search scenario as an example. Our proposed paradigm can also be applied in other industrial applications. Experiments all show the effectiveness of MIM on various scenarios \ref{sec:Evaluation on Industrial Applications}).}, given a set of users $\mathcal{U}$, a set of items $\mathcal{I}$, and a set of queries $\mathcal{Q}$, the CTR prediction task is to predict whether a user $u \in \mathcal{U}$ will click the item $i \in \mathcal{I}$ when $u$ searches a query $q \in \mathcal{Q}$.
It can be formulated as:
% \begin{align}
$\hat{y}=f(u,i,q)$
% \label{eqn:y_uvq}
% \end{align}
where $\hat{y}$ is the predicted click-through rate score.
% between user $u$ and item $i$, given query $q$.
% $f$ refers to the architecture (e.g., Embedding+MLP) of ID-CTR.
Note that query $q$ can be a text query and can also be an image query, which refers to a user wanting to search for an item with similar content to the image query.
% Finally, a cross-entropy loss is applied to optimize ID-CTR, i.e.,
% \begin{align}
% L=-\sum ylog\hat{y}+(1-y)log(1-\hat{y}).
% \label{eqn:cross-entropy}
% \end{align}

\noindent \textbf{User Behavior Modeling. }
Given user behavior $B=\{b_1,b_2,...,b_l\}$, where $l$ is the behavior length, a general UBM is expressed as 
\begin{align}
h_{B}^{ID} = \sum_{i=1}\alpha_{t,i}^{ID}h^{ID}_i
\label{eqn:UBM}
\end{align}
where $h^{ID}_i$ is the ID embedding of item $b_i$, $\alpha_{t,i}$ is the relevant score between $b_i$ and target item $t$ via attention
mechanism based on item ID embeddings.
Then $h^{ID}_B$ is fed into deepCTR.
In this way, the relative user interest to target item $t$ can be captured.



\subsubsection{Method Overview.}
The overall framework is illustrated in Figure \ref{figure:The framework of MIM}. The proposed MIM paradigm comprises three key stages designed to address the limitations of existing approaches and enhance user behavior modeling. In the pre-training stage, the primary focus is on adapting foundational models to domain-specific data, ensuring the extraction of high-quality multi-modal embeddings capable of understanding diverse item content. Subsequently, the Content-Interest-Aware Supervised Fine-Tuning (C-SFT) stage bridges the semantic gap between user interests and content representations by leveraging explicit user behavior signals, such as purchase actions, to guide embeddings in aligning with user preferences. Finally, the CiUBM stage integrates the refined multi-modal embeddings with ID-based collaborative filtering signals into a unified and flexible framework, enhancing the overall representation power. To address efficiency challenges in large-scale industrial applications, a representation center is introduced, which precomputes and stores embeddings for efficient retrieval, significantly reducing both training and inference costs.



\subsection{Pre-training}
\label{sec:Pre-train}
The pre-training stage focuses on equipping the multi-modal embeddings with the ability to understand item content effectively. This stage leverages foundational models (FoMs) pre-trained on public datasets (e.g., ImageNet and Wikipedia), which are further adapted to downstream domains. We define vision-based and language-based pre-trained FoMs as $F_V$ and $F_L$ respectively. To address the limitations of general pre-trained models, two aspects of adaptation need to be considered.: 

\noindent \textbf{Downstream Data Adaptation (DDA).} Knowledge learned from generic datasets may not align well with domain-specific item content. To overcome this, $F_V$ and $F_L$ are continually pre-trained on downstream datasets containing item images and textual descriptions. This process ensures that the resulting models are fine-tuned to the unique characteristics of domain-specific data, enabling a deeper understanding of visual and textual content.

\noindent \textbf{Different Modal Alignment (DMA).} Effective multi-modal representation requires a seamless alignment between embeddings from different modalities. Following the contrastive learning framework proposed in \cite{radford2021learning}, we perform image-text contrastive pre-training on downstream data. This process aligns image and text embeddings, ensuring a coherent representation that bridges the semantic gap between modalities.

By achieving these adaptation, the pre-training stage lays the foundation for robust multi-modal embeddings capable of effectively representing diverse item content.


\subsection{Content-Interest-Aware Supervised Fine-tuning} 
\label{sec:C-SFT}


After the pre-training stage, FoM can understand what content an item has.
However, To enhance its effectiveness in user behavior modeling, the focus of embeddings must shift from representing "what the content is" to "what users are interested in.
Ideally, multi-modal embeddings for items with similar user interests should be drawn closer together in the latent space, while embeddings for unrelated items should be pushed farther apart.
Inspired by the recent advancements in contrastive learning \cite{DBLP:journals/mta/LongDL24,DBLP:conf/cvpr/WuXYL18,DBLP:conf/iclr/ClarkLLM20,DBLP:conf/nips/MikolovSCCD13}, we propose a content-interest-aware supervised contrastive fine-tuning (C-SFT) method to effectively model user interest similarity within the multi-modal (MM) embedding space.

To design such a contrastive fine-tuning, we need to answer the following three questions:
\textbf{Q1}: How to define user interest pairs?
\textbf{Q2}: How to encode multi-modal features?
\textbf{Q3}: What is a proper learning objective?
Next, we present the design strategies for C-SFT by correspondingly answering the aforementioned questions.

\noindent \textbf{Q1: User Interest Pairs Definition. }

The success of the contrastive learning framework largely relies on the definition of the user interest pairs\cite{DBLP:conf/kdd/QiuCDZYDWT20}.
It drives us to seek a strong and direct user interest signal.

In this paper, we leveraged data from visual search scenario, we define the user interest pair as <an image query $q$, an item $i$>, which refers to a user who has searched an image query $q$ and purchased an item $i$. Compared to textual queries, image queries carry richer semantic meanings and can more accurately express users' intentions and interests

The reason to use purchase behaviors rather than click behaviors is that purchase actions provide a clearer and more reliable indication of user interests.
We also conduct experiments to show the impacts of different signals in Section \ref{sec:Effectiveness of C-SFT}.

\noindent \textbf{Q2: Multi-modal Encoder. }
% \subsubsection{Q2: Multi-modal Encoder}
Given the user interest pairs <$q$, $i$>, a multi-modal encoder extracts embeddings for both the query and the item. The query image $q$ and item image $img_i$ are processed using $F_V$, while the item title $txt_i$ is encoded by $F_L$:
\begin{equation}
h^{img}_q = F_V(q), \quad h^{img}_i = F_V(img_i), \quad h^{txt}_i = F_L(txt_i)
\end{equation}

To integrate the item's multi-modal content, we fuse the image and title embeddings using a tensor-based approach inspired by TFN \cite{zadeh2017tensor}, capturing both intra- and inter-modal interactions. The fused embedding is defined as:
\begin{equation}
\hat{h}^{MM}_{i} = \begin{bmatrix} h^{img}_i \\ 1 \end{bmatrix} \otimes \begin{bmatrix} h^{txt}_i \\ 1 \end{bmatrix}
\end{equation}

where $\otimes$ indicates the outer product. Finally, the fused representation is refined using an MLP:
\begin{equation}
h^{MM}_i = MLP(\hat{h}^{MM}_{i})
\end{equation}
producing a comprehensive multi-modal embedding $h^{MM}_i$ that effectively represents the item's content.



\noindent \textbf{Q3: Loss Function Design. }
To model the similarity between the pair <$q$,$i$>, we adopt the standard InfoNCE loss function:
\begin{equation}
\label{equ:contrastive loss}
L_{q2item} = - \log \left( \frac{\exp(sim(h^{img}_q, (h^{MM}_i)^+))}{\sum_{j=0}^{K}\exp(sim(h^{img}_q, (h^{MM}_j)^-))} \right)
\end{equation}
where \( sim(h^{img}_q, (h^{MM}_i)^+) = \frac{(h^{img}_q)^\top (h^{MM}_i)^+}{\|h^{img}_q\| \| (h^{MM}_i)^+ \|} \), and \( K \) is the number of negative samples taken from other pairs in the same batch.

However, there are two shortcomings in directly using the basic InfoNCE.: (1) The number of negative samples is limited by the batch size, which is insufficient for industrial-scale applications where billions of items exist. (2) Single-modal embeddings, such as \( h^{img}_i \) and \( h^{txt}_i \), are ignored, which may result in a loss of modality-specific details.To address these, we propose two extensions:


\noindent \textbf{Extension 1: Space-time-based negative sample generation (ST-NSG).}
As shown in Figure \ref{figure:The example of negative sample generation.}, to address the limitation of limited negative samples, we propose the Space-Time Negative Sampling (ST-NSG) method. First, for each sample $<q, i>$, we add a hard negative item $i^-$ from the same category, forming the triplet $<q, i, i^->$. To expand the negative sample set, we use two strategies: (1) Time-based strategy: Negative items are collected from the current batch and the last $k$ batches, increasing the number of negatives to $2N(k+1)-1$, where $N$ is the batch size. (2) Space-based strategy: In distributed training, items from $P$ GPUs across $k$ recent batches are added as negatives, resulting in $2NP(k+1)-1$ total negatives. Combined, these strategies provide nearly $200\times$ more negative samples compared to basic InfoNCE, significantly enhancing model performance in practice.



\noindent \textbf{Extension 2: Multi-level InfoNCE.}
Although MM embeddings provide a holistic view of item content across modalities, they may lack details specific to each modality. To address this, we propose \textbf{multi-level InfoNCE}, which incorporates signals from both single-modality and multi-modal embeddings. The overall loss function is defined as:

\begin{equation}
\label{equ:sft loss}
L_{C-SFT} = L_{q2item} + \alpha L_{q2txt} + \beta L_{q2img}
\end{equation}

where $L_{q2txt}$ and $L_{q2img}$ represent the InfoNCE loss for the pairs $<h^{img}_q, h^{txt}_i>$ and $<h^{img}_q, h^{img}_i>$, respectively. The weights $\alpha$ and $\beta$ balance the contributions of each modality, ensuring a more comprehensive modeling of user interests.







\begin{figure}[t]
% {0.5\columnwidth}
\centering
\includegraphics[width = .45\textwidth]{imgs/ST-NGS.png}

\caption{The example of negative sample generation. (a) Negative items are obtained from other items in the same batch. (b) adding a hard negative item (b) Taking the sample $s_1$ as an example, with the help of ST-NGS, the amount of negative items (i.e., the red items) can be added.}

\label{figure:The example of negative sample generation.}
\end{figure}






\subsection{CiUBM: Content-interest-aware UBM}
% 可以考虑改成MM enhanced UBM
\label{Sec:CiUBM}
After obtaining the multi-modal (MM) embedding $h^{MM}_i$ for item $i$, we design a content-interest-aware user behavior model (CiUBM) to enhance the existing UBMs by integrating ID and content-based signals. CiUBM consists of three components:

\noindent \textbf{ID Interest Module. } This module uses existing UBM techniques to model user preferences based on ID embeddings, capturing collaborative filtering (CF) signals. It provides a straightforward integration of prior UBM methods.

\noindent \textbf{Content Interest Module. } This module leverages the high-quality MM embeddings to calculate user content interest relative to the target item $t$. Similarity scores $\alpha_{t,i}^{MM}$ between the target and historical behavior items are used to weight the MM embeddings, creating a representation $h_{B}^{MM}$ that reflects user content preferences. Formally:
\begin{equation}
\alpha_{t,i}^{MM} = \frac{(h^{MM}_t)^\top h^{MM}_i}{\|h^{MM}_t\| \|h^{MM}_i\|}, \quad h_{B}^{MM} = \sum_{i=1}^l \alpha_{t,i}^{MM} h^{MM}_i
\end{equation}
where $h^{MM}_i$ is the MM embedding of the historical item $b_i$, and $l$ is the number of user behaviors.

\noindent \textbf{Fusion Interest Module. } This module combines ID and content interest. By using $\alpha_{t,i}^{MM}$ to weight ID embeddings $h_{i}^{ID}$, it produces $h_{B}^{Fusion}$, which aligns content-based and collaborative filtering signals:
\begin{equation}
h_{B}^{Fusion} = \sum_{i=1}^l \alpha_{t,i}^{MM} h_{i}^{ID}
\end{equation}

\noindent \textbf{Final Representation. } The overall user behavior representation $h_B$ is obtained by concatenating the outputs of the three modules:
\begin{equation}
h_B = \textit{Concat}(h_{B}^{ID}, h_{B}^{MM}, h_{B}^{Fusion})
\end{equation}
and is then fed into the deepCTR model for improved prediction.










\subsection{Representation Center}
\label{sec:Representation Center}
In general, the decomposed paradigm allows MIM to train different stage models in parallel to save time cost.
However, the MM embedding extraction in CiUBM stage will definitely consume more time and GPU memory.
Even worse, all items in user behavior $B$ are needed to be extracted, aggravating the time and GPU memory cost (see Section \ref{sec:Efficiency Evaluation}).
To address this problem, we proposed the representation center, which contains three key components (Figure \ref{figure:The framework of representation_center.}).

\noindent \textbf{MM Features Memorization (MFM). }
Rather than directly inferring the MM embedding during deepCTR training or inferencing, we pre-infer the MM embeddings of all items.
Then we memorize these features as an embedding table, where the key is the item ID and the value is the MM embedding.
In this way, we can directly obtain MM embedding from the embedding table without any extra time and GPU memory cost.

\noindent \textbf{Real-time Inference Module (RIM). }
The key problem of maintaining the MM embedding table is that it cannot obtain the MM features of new items in time.
However, in practice, it is a very common situation where new items are shown all the time.
Thus, RIM is designed to address this problem.
Specifically, when a new item is added to industrial applications, RIM will infer the MM embedding of this new item in real-time.
Then, this inferred MM embedding will be collected as a window message, which is sent to the MM embedding table for offline training and to the parameter machine (see the following part for details) for online serving in a short window time.

\noindent \textbf{Computation and Memory Decomposition (CMD). }
Although introducing MFM will not cost too much GPU memory, the embedding table still needs a large amount of random access memory (e.g., nearly 300G in our experiments), which also influences the efficiency (e.g., the computation of some operations in CPU and I/O between GPU and CPU).
To solve this problem, we propose a separate strategy where the model parameters (including embedding tables and dense layer parameters) and the computation of inference are divided into different machines, i.e., the parameter machine and the computation machine.
In this way, the computation machine only needs to care about the memory cost during the feed-forward and is not restricted by the huge embedding tables\footnote{During inference, the computation machine can only request the embedding vectors of the corresponding keys rather than the whole embedding table to save memory}.
Meanwhile, with the help of MFM, there are no additional parameters needed for loading and maintaining a number of temporary tensors of MM feature extraction by large-scale FoM.

\begin{figure}[t]
\centering
\includegraphics[trim=10 0 0 0, clip,width = .45\textwidth]{imgs/representaion_center.png}
\vspace{-1em}
\caption{The framework of representation center.}
\vspace{-1em}
\label{figure:The framework of representation_center.}
\end{figure}


\section{Experiments}


\subsection{Experimental Settings}



\subsubsection{Datasets.}
% \begin{itemize}
% \item 
\emph{(1) Pre-training Datasets.} 
As introduced in Section \ref{sec:Pre-train}, a downstream-related dataset is applied to align the style and modal distribution.
% The statistic of this dataset is summarized in Table \ref{tab:The statistic of datasets.}.
We collect images and titles of items from Taobao, one of the world's largest online retail platforms.
There are a total of 1.42+B items.
% \item 
\emph{(2) C-SFT Datasets.}
We collect purchase logs in the last 6 months from the visual search scenario in Taobao and construct the C-SFT dataset as introduced in Section \ref{sec:C-SFT}.
There are a total of 0.14 B user interest pairs.
% \item 
\emph{(3) Downstream Datasets.}
The dataset is collected from the real-world traffic logs in Taobao.
Each sample refers to a user who searches a query on this platform and observes an item and contains the necessary information about the queries and items, including multi-modal embedding, product IDs, categories, brands, prices, sales, etc.
The label is defined as whether the user clicks this item.
Besides, we further adopt 7 subsets to sufficiently evaluate the performance of MIM.
% including Home Decoration, Clothes, Sports, Toys, Fast-Moving Consumer Goods(FMCG), Pets, and Cars) of the whole, to sufficiently evaluate the performance of MIM.
More detailed information about the dataset is summarized in Table \ref{tab:The statistic of datasets.}.
Note that, due to the limited space, apart from Section \ref{sec:Performance on CTR Prediction Tasks}, we report the results on the whole downstream dataset by default, and similar conclusions can be obtained from other sub-datasets.


\begin{table}
\caption{The statistic of datasets.
% Note the pre-training dataset only needs items (i.e., item images and titles).
B refers to billion, and M refers to million.}
% 
\label{tab:The statistic of datasets.}
\centering
\resizebox{\columnwidth}{!}{
\begin{tabular}{c|cccccccc}
\toprule
% \multirow{2}{*}{} & \multirow{2}{*}{\textbf{Pre-training Dataset}} & \multicolumn{8}{c}{\textbf{Downstream Datasets}} \\ \cmidrule{3-10} 
                                                     & \multicolumn{1}{c|}{\textbf{ALL}} & \multicolumn{1}{c|}{\textbf{Home  Decoration}} & \multicolumn{1}{c|}{\textbf{Clothes}} & \multicolumn{1}{c|}{\textbf{Sports}} & \multicolumn{1}{c|}{\textbf{Toys}} & \multicolumn{1}{c|}{\textbf{FMCG}} & \multicolumn{1}{c|}{\textbf{Pets}} & \textbf{Cars} \\ \midrule
\#Data                                              & \multicolumn{1}{c|}{1.83+ B}    & \multicolumn{1}{c|}{0.15+ B}                  & \multicolumn{1}{c|}{0.91+ B}        & \multicolumn{1}{c|}{0.1+ B}       & \multicolumn{1}{c|}{51.6+ M}     & \multicolumn{1}{c|}{0.21+ B}               & \multicolumn{1}{c|}{24.81+ M}     &      60,75+ M
\\ 
\#Query                                              & \multicolumn{1}{c|}{56.34+ M}    & \multicolumn{1}{c|}{5.32+ M}                  & \multicolumn{1}{c|}{15.07+ M}        & \multicolumn{1}{c|}{3.56+ M}       & \multicolumn{1}{c|}{2.32+ M}     & \multicolumn{1}{c|}{6.72+ M}               & \multicolumn{1}{c|}{0.73+ M}     &      3,66+ M
\\ 
\#Users                                             & \multicolumn{1}{c|}{0.12+ B}    & \multicolumn{1}{c|}{14.04+ M}                  & \multicolumn{1}{c|}{55.77+ M}        & \multicolumn{1}{c|}{16.34+ M}       & \multicolumn{1}{c|}{7.04+ M}     & \multicolumn{1}{c|}{25.95+ M}               & \multicolumn{1}{c|}{2.63+ M}     &      6.62+ M
\\ 
\#Items            &                                     \multicolumn{1}{c|}{16.16+ M}    & \multicolumn{1}{c|}{1.50+ M}                  & \multicolumn{1}{c|}{3.74+ M}        & \multicolumn{1}{c|}{0.65+ M}       & \multicolumn{1}{c|}{0.42+ M}     & \multicolumn{1}{c|}{1.11+ M}               & \multicolumn{1}{c|}{0.22+ M}     &      0.91+ M\\ \bottomrule
\end{tabular}
}
\end{table}


% \end{itemize}

\subsubsection{Baselines} Here, we compare our method with various UBM methods, including Avg Pooling\footnote{averaging the embedding of different items in user behaviors}, DIN\cite{zhou2018deep}, DIEN\cite{zhou2019deep}, BST\cite{chen2019behavior}, and SIM\cite{pi2020search}, TWIN\cite{chang2023twin}.
By default, the base backbone of UBM is set as SIM \cite{pi2020search}.




\subsubsection{Training Details.}
\label{sec:Training Details.}
We use the Adam optimizer with a learning rate of 0.005 for all methods.
The batch size N is 1024 for all methods.

By default, we take EVA-2 \cite{woo2023convnext} for images and BGE\cite{bge_embedding} for texts.
We set hyper-parameters $\alpha=0.5$, $\beta=0.5$ for loss weights and $k=10$ for negative samples by grid searching.
Besides, we also evaluate the impact of different FoM, including ResNet50\cite{he2019bag}, SwinTransformer\cite{liu2021Swin}, ConvNext v1\cite{liu2022ConvNeta}, ConvNext V2\cite{woo2023convnext}, ConvNext v2 large\cite{woo2023convnext}, EVA-2\cite{fang2023eva02}, Bert\cite{devlin2018bert}, GPT-2\cite{radford2019language}, BEiT-3\cite{wang2022image} and BGE\cite{bge_embedding} in Section\ref{sec:Effectiveness Evaluation}.

MIM is trained by 64 A100 GPUs in the pre-training stage, 40 A100 GPUs in the C-SFT stage, and 100 V100 GPUs in the CiUBM training stage.
We run all experiments multiple times with different random seeds and report the average results.





\begin{table*}[t] \footnotesize
\caption{The AUC results of the CTR prediction task.
Note Base refers to the original results of the corresponding methods and Base+MIM refers to the results
with the help of MIM.
$\Delta$ refers to the improvement of Base+MIM compared to Base.
}
\vspace{-1em} 
\label{tab:Performance on CTR Prediction Tasks}
\resizebox{0.9\textwidth}{!}{%
\centering
\begin{tabular}{l|l|c|c|c|c|c|c|c|c}
\toprule
\multicolumn{1}{l|}{\textbf{Model}}         & \multicolumn{1}{l|}{\textbf{Version}} & \textbf{ALL}             & \textbf{Toys}            & \textbf{FMCG}            & \textbf{Sports}          & \textbf{Cars}            & \textbf{Clothes}         & \textbf{Pets}            & \textbf{Home Decoration} \\ \midrule
\multirow{3}{*}{Avg Pooling} & Base                 & 0.6965          & 0.6958          & 0.7044          & 0.7045          & 0.6965          & 0.6810          & 0.6844          & 0.6748          \\
                             & Base+MIM             & 0.7030          & 0.6989          & 0.7168          & 0.7148          & 0.7065          & 0.6828          & 0.6911          & 0.6796          \\
                             & $\Delta$                & \textbf{+0.0066} & \textbf{+0.0032} & \textbf{+0.0123} & \textbf{+0.0104} & \textbf{+0.0100} & \textbf{+0.0017} & \textbf{+0.0067} & \textbf{+0.0048} \\ \midrule
\multirow{3}{*}{DIN}         & Base                 & 0.7004          & 0.6969          & 0.7187          & 0.6986          & 0.6928          & 0.6753          & 0.7099          & 0.6802          \\
                             & Base+MIM             & 0.7047          & 0.7006          & 0.7225          & 0.7027          & 0.6963          & 0.6812          & 0.7126          & 0.6864          \\
                             & $\Delta$                & \textbf{+0.0042} & \textbf{+0.0037} & \textbf{+0.0039} & \textbf{+0.0041} & \textbf{+0.0036} & \textbf{+0.0058} & \textbf{+0.0028} & \textbf{+0.0062} \\ \midrule
\multirow{3}{*}{DIEN}        & Base                 & 0.7010          & 0.6975          & 0.7190          & 0.6989          & 0.6934          & 0.6761          & 0.7102          & 0.6809          \\
                             & Base+MIM             & 0.7055          & 0.7013          & 0.7233          & 0.7033          & 0.6970          & 0.6822          & 0.7131          & 0.6872          \\
                             & $\Delta$                & \textbf{+0.0046} & \textbf{+0.0038} & \textbf{+0.0044} & \textbf{+0.0045} & \textbf{+0.0036} & \textbf{+0.0061} & \textbf{+0.0029} & \textbf{+0.0063} \\ \midrule
\multirow{3}{*}{BST}         & Base                 & 0.6995          & 0.6965          & 0.7177          & 0.6973          & 0.6924          & 0.6740          & 0.7093          & 0.6796          \\
                             & Base+MIM             & 0.7049          & 0.7000          & 0.7229          & 0.7026          & 0.6964          & 0.6807          & 0.7129          & 0.6861          \\
                             & $\Delta$                & \textbf{+0.0054} & \textbf{+0.0035} & \textbf{+0.0053} & \textbf{+0.0053} & \textbf{+0.0040} & \textbf{+0.0068} & \textbf{+0.0036} & \textbf{+0.0065} \\ \midrule
\multirow{3}{*}{TWIN}        & Base                 & 0.7028          & 0.6992          & 0.7214          & 0.7013          & 0.6951          & 0.6776          & 0.7126          & 0.6825          \\
                             & Base+MIM             & 0.7051          & 0.7010          & 0.7229          & 0.7032          & 0.6968          & 0.6813          & 0.7134          & 0.6867          \\
                             & $\Delta$                & \textbf{+0.0023} & \textbf{+0.0018} & \textbf{+0.0016} & \textbf{+0.0019} & \textbf{+0.0017} & \textbf{+0.0037} & \textbf{+0.0009} & \textbf{+0.0042} \\ \midrule
\multirow{3}{*}{SIM}         & Base                 & 0.7024          & 0.7017          & 0.7111          & 0.7024          & 0.6945          & 0.6869          & 0.7058          & 0.6928          \\
                             & Base+MIM             & 0.7062          & 0.7067          & 0.7153          & 0.7065          & 0.6993          & 0.6893          & 0.7112          & 0.6951          \\
                             & $\Delta$                & \textbf{+0.0038} & \textbf{+0.0050} & \textbf{+0.0042} & \textbf{+0.0041} & \textbf{+0.0048} & \textbf{+0.0024} & \textbf{+0.0054} & \textbf{+0.0023} \\
\bottomrule
\end{tabular}
}
\end{table*}


\subsection{Performance on CTR Prediction Tasks}
\label{sec:Performance on CTR Prediction Tasks}
We evaluated the performance of MIM on CTR prediction tasks by applying it to various existing UBM methods. As a universal paradigm, MIM is compatible with most UBM frameworks. To assess its effectiveness, we compared the AUC scores of the original models (denoted as Base) with those enhanced by MIM. The AUC metric \cite{fawcett2006introduction} is reported.
As shown in Table \ref{tab:Performance on CTR Prediction Tasks}, integrating MIM leads to consistent performance improvements across all baseline methods and datasets. For example, SIM achieves AUC gains of 0.23pt to 0.54pt, and similar gains are observed for other UBMs. While these improvements may appear modest, in large-scale scenarios such as Taobao, a 0.1pt improvement in AUC can result in several percentage points of uplift in online CTR. This uplift can bring billions in revenue to the platform, highlighting the economic value of our approach.
These results validate the effectiveness and generalizability of MIM. Which can significantly enhances user behavior modeling by bridging the gap between content and user interest embeddings. Furthermore, its universal design ensures compatibility with a wide range of existing UBM frameworks, making it a practical solution for industrial-scale CTR prediction tasks.
\vspace{-1em}


\subsection{Effectiveness Evaluation}
\label{sec:Effectiveness Evaluation}
In this section, we conduct various experiments to detailedly evaluate the effectiveness of MIM.

\subsubsection{Effectiveness of high-quality MM embeddings.}
\label{sec:Effectiveness of high-quality MM embeddings.}

\begin{figure}[t]

  \centering
      \subfigure[The impact of different $F_V$]{
     \includegraphics[trim=5 5 5 5, clip,width = .22\textwidth]{imgs/figure5_1.png}
     \label{fig:subfigure2}
   }
        \subfigure[The impact of different $F_L$]{
     \includegraphics[trim=5 5 5 5, clip,width = .22\textwidth]{imgs/figure5_2.png}
     \label{fig:subfigure2}
   }
 \vspace{-1em}
 \caption{Evaluation of the impact of different FoMs including $F_V$ and $F_L$.
 From left to right, a more powerful FoM is adopted.}
 \vspace{-1em}
 \label{figure:Evaluation of the impact of different FoMs}
 \end{figure}

\noindent \textbf{\newline The impact of FoM. }
The experimental results in Figure \ref{figure:Evaluation of the impact of different FoMs} demonstrate the effectiveness of upgrading FoM. For $F_V$, the shift from ResNet50 to EVA-2 yields constant performance improvements, demonstrating the ability of stronger visual FoMs to capture richer content semantics. Similarly, for $F_L$, upgrading from smaller to larger models (e.g., BEiT-3 base to large) improves AUC, highlighting the benefits of scaling parameter sizes. These findings confirm that enhancing FoM, both in terms of model strength and parameter size, significantly improves the representation and alignment of user interests across modalities.


\begin{table}[t] \footnotesize
  \centering
  % \begin{minipage}{.\linewidth}
  \caption{The impact of pre-training.
  }
  \vspace{-1em}
  \label{tab:The impact of pre-training}
  \resizebox{0.8\columnwidth}{!}{%
  \centering
  \begin{tabular}{l|c|c}
  \toprule
  \textbf{Version}                          & \textbf{AUC}    & \textbf{$\Delta$} \\ \midrule
  Base                      & 0.7024 &  - \\ \midrule
  Plain FoM                & 0.7030 & +0.0060          \\
  +DDA                      & 0.7034 & +0.0010    \\
  +DDA+DMA                  & 0.7040 & +0.0016    \\
  +DDA+DMA+C-SFT(i.e., MIM) & 0.7062 & +0.0038   \\
  \bottomrule
  \end{tabular}
  }
  \end{table}

\noindent \textbf{The impact of pre-training. }
We evaluate the impact of different strategies used in the pre-training stage.
The results are presented in Table \ref{tab:The impact of pre-training}.
% Here, DDA refers to downstream data adaption, DMA refers to different modal alignment.
Here, Plain FoM refers to the FoM that is only pre-trained on general datasets.
From Table \ref{tab:The impact of pre-training}, it shows (1) directly applying Plain FoM only achieves limited improvement due to the sub-effectiveness in item content understanding.
(2) Both downstream data adaption and different modal alignment can help FoM better capture the item content and provide a significant improvement.

\begin{table}[t]
  \footnotesize % 设置表格字体大小
  \caption{The results of C-SFT evaluation. LF, CE, and CL refer to loss function, cross-entropy loss, and contrastive loss respectively.}
  \vspace{-1em}
  \centering
  \resizebox{0.8\columnwidth}{!}{%  
  \label{tab:Effectiveness Evaluation}
  \begin{tabular}{l|l|l|c|c}
  \toprule
  \textbf{Version}  & \textbf{LF} & \textbf{Signal} & \textbf{AUC} & \textbf{$\Delta$} \\ \midrule
  Base              & -           & -               & 0.7024       & -                  \\ \midrule
  v1                & CE          & Category        & 0.7043       & +0.0019            \\
  v2                & CE          & Click           & 0.7053       & +0.0029            \\
  v3                & CE          & Purchase        & 0.7059       & +0.0035            \\
  v4                & CL          & Click           & 0.7055       & +0.0031            \\
  v5                & CL          & Purchase        & 0.7062       & +0.0038   \\ \bottomrule
  \end{tabular}
  }
  \end{table}
  \vspace{-1em}


\subsubsection{Effectiveness of C-SFT}
\label{sec:Effectiveness of C-SFT}
As presented in Section \ref{sec:C-SFT}, there are two important factors that influence C-SFT, i.e., the user interest signal and loss function.
Thus, we train different versions of C-SFT (see Table \ref{tab:Effectiveness Evaluation}) and evaluate the performance of these versions in CTR prediction tasks.
Specifically, for the user interest signal, we take the users' clicking  and users' purchase as the expression of user interest.
Furthermore, we also take an interest-irrelevant signal (item categories) with Cross Entropy loss as a comparison.
For the loss function, the contrastive loss (i.e., Equ \ref{equ:contrastive loss}) and Cross Entropy loss are compared.
The AUC results are reported in Table \ref{tab:Effectiveness Evaluation}.
% Due to the limited space, the results on the whole dataset is reported here, and similar conclusions can be obtained from other sub-datasets.

\noindent \textbf{The impact of user interest signal. }
Compared with the performance of version v1, v2 and v3 achieves a better performance.
It demonstrates the importance of introducing the user interest signal.
Without such signals, MM features can hardly achieve a positive effect on CTR prediction.
Furthermore, v3 (or v5) obtains a better performance than v2 (or v4).
One of the possible reasons is that the purchase behavior is more strong and precise, resulting in a better expression of user interest.
It also shows that understanding user interest plays an important role in MIM.


\noindent \textbf{The impact of loss function. }
Compared with the baselines, both of the two loss functions perform better via bridging the semantic gap.
Furthermore, compared with the performance of version v2 (or v3), it shows that adopting contrastive loss (i.e., version v4 or v5) performs better, which indicates contrastive loss can better model the alignment between user interest and the target item, thereby further improving prediction performance.


\begin{table}[t] 
  \footnotesize
  \centering
  \caption{The results of ablation study.}
  \vspace{-1em}
  \label{tab:The results of ablation study.}
  \resizebox{0.65\columnwidth}{!}{
  \begin{tabular}{l|c}
  \toprule
  \textbf{Version}                      & \textbf{AUC}     \\ \midrule
  MIM                        & 0.7062  \\ \midrule
  \multicolumn{2}{l}{C-SFT stage}           \\  \midrule
  w/o TFN                       & 0.7058  \\
  w/o ST-NSG                    & 0.7056  \\
  w/o Multi-level InfoNCE       & 0.7052  \\
  w/o C-SFT                     & 0.7040  \\ \midrule
  \multicolumn{2}{l}{CiUBM} \\ \midrule
  w/o ID interest module        &  0.7057 \\
  w/o content interest module   &  0.7052 \\
  w/o fusion interest module    &  0.7054 \\
  \bottomrule
  \end{tabular}
  }
  \end{table}

\subsection{Ablation Study}
\label{sec:Ablation Study}
In this section, we conduct an ablation study to analyze the impact of different components.
The results are presented in Table \ref{tab:The results of ablation study.}.
Some observations are summarized as follows:  
1) For the C-SFT stage, it shows that C-SFT takes an important role and improves AUC from 0.7040 to 0.7062.
Besides, different components in C-SFT contribute to the final performance.
2) For the CiUBM, different interest modules contribute to the final improvement. 
It demonstrates that it is important to take both ID interest and content interest into consideration.
Note the analysis for the pre-training stage has been discussed in Section \ref{sec:Effectiveness of high-quality MM embeddings.}.


\subsection{Efficiency Evaluation}
\label{sec:Efficiency Evaluation}
In this section, we analyze the efficiency of the proposed paradigm, including time and GPU memory efficiency.
Detailedly, we develop two versions:
(1) MIM (w/o RC) refers to MIM without representation center, i.e., MM embedding is real-time inferred by FoM.
(2) MIM (E2E) refers to jointly training FoM and CiUBM in an E2E manner.
We report the GFLOPs (the number of giga floating-point operations) per sample to reflect the train and inference time cost and GPU memory usage in deepCTR \footnote{Since the pre-training stage can be pre-processed and C-SFT can be trained in parallel, the cost of these two stages can be ignored and is not included here.}.
The results are presented in Table \ref{tab:Efficiency Evaluation}.
We can find that 
% (1) For MIM, thanks to the decomposed paradigm and representation center, it is significantly reduced cost introduced by FoM, and only needs 0.59$\times$ and 0.387$\times$ extra computation in training and inference respectively, 0.01$\times$ and 0.81$\times$ extra GPU memory cost in training and inference respectively.
% (1) Without representation center, although MIM(w/o RC) can obtain the same AUC, it needs a large number of extra computations (773.55$\times$ $\sim$ 2106.37$\times$  higher than ID-CTR) and is GPU memory costly (8.23$\times$ $\sim$ 13.49$\times$  higher than ID-CTR).
(1) Compared with Base, both of MIM(w/o RC) and MIM(E2E)  need a large number of extra computations (773.55$\times$ $\sim$ 6318.12$\times$  higher than Base) and is GPU memory costly (8.23$\times$ $\sim$ 26.41$\times$  higher than Base).
% Such inefficient paradigms can hardly be accepted in web-scale applications.
Note the architecture of the MIM(w/o RC) and MIM(E2E) is the same, resulting in the same GFLOPs and GPU memory cost during inference.
(2) Compared with MIM(E2E), MIM(w/o RC) does not need to jointly train FoM and Base, which saves computation and GPU memory cost by the backward of FoM.
(3) For MIM, thanks to the decomposed paradigm and representation center, it has significantly reduced the cost introduced by FoM and only needs 0.59$\times$ and 0.387$\times$ extra computation in training and inference, respectively, and 0.01$\times$ and 0.81$\times$ extra GPU memory cost in training and inference, respectively.

Overall, such nice properties in terms of high performance, efficient GPU memory usage, and low time requirements of our proposed model are welcomed for web-scale applications.



  
  \begin{figure}[t]
  \centering
  \includegraphics[width=0.9\columnwidth]{imgs/figure7_1.png}
  \caption{AUC gains by different item sets.}
  \label{fig:Auc gap by item sets}
  \end{figure}

  \begin{table}[t] \footnotesize
    \centering
    % TODO:fixme
    \caption{The results of efficiency evaluation. $\Delta=(Cost_A-Cost_B)/Cost_B$ is the relative difference with respect to Base.
    }
    \vspace{-1em}
    % \begin{adjustwidth}{-1.1cm}{}
    \centering
    \resizebox{\columnwidth}{!}{
    % \resizebox{\columnwidth}{!}{
        
    \label{tab:Efficiency Evaluation}
    \centering
    % \resizebox{\columnwidth}{!}{
    \begin{tabular}{l|cc|cc|cc|cc}
    \toprule
                        & \multicolumn{4}{c|}{\textbf{GFLOPs}}& \multicolumn{4}{c}{\textbf{GPU Memory (G)}} \\ \cline{2-9} 
    \multirow{3}{*}{{}} & \multicolumn{2}{c|}{\textbf{Train}}   & \multicolumn{2}{c|}{\textbf{Inference}}     & \multicolumn{2}{c|}{\textbf{Train}} & \multicolumn{2}{c}{\textbf{Inference}} \\ \cline{2-9}
                        & {Cost} & {$\Delta$} & {Cost} & {$\Delta$}& {Cost} & {$\Delta$}& {Cost} & {$\Delta$} \\ \hline
    Base              & {{0.114}}     & {{-}}           & {{0.31}}     & {-}          & {{1.1}}   & {{-}}        & {{0.68}} & {-}        \\ 
    Base+MIM(w/o RC)                 & 240.24& +2106.37& 240.11& +773.55 & {{10.15}} & {{+8.23}}  & {{9.85}} & {+13.49} \\
    Base+MIM(E2E)                 & 720.38& +6318.12& 240.11& +773.55 & {{30.15}} & {{+26.41}} & {{9.85}} & {+13.49} \\ 
    Base+MIM              & {{0.182}}     & {{+0.59}}       & {{0.43}}     & {+0.38.7}      & {{1.11}}  & {{+0.01}}    & {{0.81}} & {+0.19}   \\ \bottomrule
    \end{tabular}
    }
    \end{table}



 
\subsection{Generalization in Cold Start Scenarios}
\label{sec:Generalization in Cold Start Scenarios}
% As discussed in Section \ref{sec:Introduction}, existing ID-CTR suffers from the cold start problem.
Here we analyze the performance of the cold start items to show the generalization of our proposed paradigm.
Detailedly, we divide different items into 10 sets (denoted as S1 to S10), where S1$\sim$S10 refers to the newest$\sim$oldest item sets.
Then we evaluate the improvement of different sets respectively.
The results are presented in Figure \ref{fig:Auc gap by item sets}.
It shows that MIM contributes more to the newest items (e.g., S1) and achieves more AUC gains. 
This demonstrates that MM features enable new items to better represent their features, leading to better performance.



\vspace{-1em}
\subsection{Evaluation of Industrial Applications}
\label{sec:Evaluation on Industrial Applications}
% 
Here, we first develop MIM on a real-world industrial application, i.e., the sponsored search system in Taobao, and have continually achieved three releases of MIM in our industrial application in the past year.

\noindent \textbf{Online A/B Testing for CTR Prediction Tasks. } 
Compared with the online model, these three releases achieved 0.81pt AUC gains in total, and 0.26pt, 0.25pt, and 0.30pt AUC gains for each.
During the online A/B test, we observed 14.14\% CTR gains in total, and 5.1\%, 4.75\%, and 4.29\% CTR gains for each release.
It also achieves 4.12\% RPM (Revenue Per Mile) gain in total, and 1.5\%, 1.19\% and 1.43\% RPM gains for each release.
% we observe a 5\% CTR gain and a 1\% RPM (Revenue Per Mile) gain respectively.
% Note this is a significant improvement in the industrial-sponsored search system.

\noindent \textbf{Impact on Content-Sensitive Categories. }
We also report AUC, CTR, and RPM gains in different item categories in one of the releases (others have similar results).
Table \ref{tab:The improvement on different items.} shows more improvement can be archived on the content-sensitive categories (e.g., Clothes), where users more care about the styles of items and can be easily attracted by item images or titles.
On the contrary, lower improvement is obtained from content-insensitive categories (e.g., Cars).
It demonstrates user interest can be well modeled by MIM and improve the performance of CTR prediction.

\begin{table}[t]
  \footnotesize % 设置表格字体大小
  \caption{The improvement on different item categories.}
  \vspace{-1em}
  \centering
  \resizebox{0.9\columnwidth}{!}{%
  \label{tab:The improvement on different items.}
  \begin{tabular}{l|c|c|c}
  \toprule
   \textbf{Category}                     & \textbf{AUC Gains} & \textbf{CTR Gains} & \textbf{RPM Gains} \\ \midrule
  \textbf{All}          & +0.0038            & +5\%               & +1\%               \\ \midrule
  \textbf{Home Decoration} & +0.0047            & +4.63\%            & +3.04\%            \\
  \textbf{Clothes}         & +0.0037            & +8.69\%            & +4.44\%            \\
  \textbf{Sports}          & +0.003             & +3.48\%            & +1.86\%            \\
  \textbf{Toys}            & +0.0024            & +4.44\%            & +1.89\%            \\
  \textbf{FMCG}            & +0.0028            & +3.98\%            & +2.74\%            \\
  \textbf{Pets}            & +0.0024            & +3.02\%            & +2.69\%            \\
  \textbf{Cars}            & +0.0028            & +2.54\%            & +0.02\%            \\ \bottomrule
  \end{tabular}%
  }
  \vspace{-1em}
  \end{table}
  
\noindent \textbf{Universality in Recommendation Tasks. }
Furthermore, we also try to apply MIM in a recommendation system, i.e., the display advertising system in Taobao, and achieve similar improvements.
Inspired by this method, the final deployed version achieved 0.4pt offline AUC gains, 3.5\% CTR gain and 1.5\% RPM gain online. From the success of different systems, we believe MIM is a universal and industrial welcomed multi-modal content interest modeling paradigm.



% \vspace{-5pt}
\section{Method}
\label{sec:method}
\section{Overview}

\revision{In this section, we first explain the foundational concept of Hausdorff distance-based penetration depth algorithms, which are essential for understanding our method (Sec.~\ref{sec:preliminary}).
We then provide a brief overview of our proposed RT-based penetration depth algorithm (Sec.~\ref{subsec:algo_overview}).}



\section{Preliminaries }
\label{sec:Preliminaries}

% Before we introduce our method, we first overview the important basics of 3D dynamic human modeling with Gaussian splatting. Then, we discuss the diffusion-based 3d generation techniques, and how they can be applied to human modeling.
% \ZY{I stopp here. TBC.}
% \subsection{Dynamic human modeling with Gaussian splatting}
\subsection{3D Gaussian Splatting}
3D Gaussian splatting~\cite{kerbl3Dgaussians} is an explicit scene representation that allows high-quality real-time rendering. The given scene is represented by a set of static 3D Gaussians, which are parameterized as follows: Gaussian center $x\in {\mathbb{R}^3}$, color $c\in {\mathbb{R}^3}$, opacity $\alpha\in {\mathbb{R}}$, spatial rotation in the form of quaternion $q\in {\mathbb{R}^4}$, and scaling factor $s\in {\mathbb{R}^3}$. Given these properties, the rendering process is represented as:
\begin{equation}
  I = Splatting(x, c, s, \alpha, q, r),
  \label{eq:splattingGA}
\end{equation}
where $I$ is the rendered image, $r$ is a set of query rays crossing the scene, and $Splatting(\cdot)$ is a differentiable rendering process. We refer readers to Kerbl et al.'s paper~\cite{kerbl3Dgaussians} for the details of Gaussian splatting. 



% \ZY{I would suggest move this part to the method part.}
% GaissianAvatar is a dynamic human generation model based on Gaussian splitting. Given a sequence of RGB images, this method utilizes fitted SMPLs and sampled points on its surface to obtain a pose-dependent feature map by a pose encoder. The pose-dependent features and a geometry feature are fed in a Gaussian decoder, which is employed to establish a functional mapping from the underlying geometry of the human form to diverse attributes of 3D Gaussians on the canonical surfaces. The parameter prediction process is articulated as follows:
% \begin{equation}
%   (\Delta x,c,s)=G_{\theta}(S+P),
%   \label{eq:gaussiandecoder}
% \end{equation}
%  where $G_{\theta}$ represents the Gaussian decoder, and $(S+P)$ is the multiplication of geometry feature S and pose feature P. Instead of optimizing all attributes of Gaussian, this decoder predicts 3D positional offset $\Delta{x} \in {\mathbb{R}^3}$, color $c\in\mathbb{R}^3$, and 3D scaling factor $ s\in\mathbb{R}^3$. To enhance geometry reconstruction accuracy, the opacity $\alpha$ and 3D rotation $q$ are set to fixed values of $1$ and $(1,0,0,0)$ respectively.
 
%  To render the canonical avatar in observation space, we seamlessly combine the Linear Blend Skinning function with the Gaussian Splatting~\cite{kerbl3Dgaussians} rendering process: 
% \begin{equation}
%   I_{\theta}=Splatting(x_o,Q,d),
%   \label{eq:splatting}
% \end{equation}
% \begin{equation}
%   x_o = T_{lbs}(x_c,p,w),
%   \label{eq:LBS}
% \end{equation}
% where $I_{\theta}$ represents the final rendered image, and the canonical Gaussian position $x_c$ is the sum of the initial position $x$ and the predicted offset $\Delta x$. The LBS function $T_{lbs}$ applies the SMPL skeleton pose $p$ and blending weights $w$ to deform $x_c$ into observation space as $x_o$. $Q$ denotes the remaining attributes of the Gaussians. With the rendering process, they can now reposition these canonical 3D Gaussians into the observation space.



\subsection{Score Distillation Sampling}
Score Distillation Sampling (SDS)~\cite{poole2022dreamfusion} builds a bridge between diffusion models and 3D representations. In SDS, the noised input is denoised in one time-step, and the difference between added noise and predicted noise is considered SDS loss, expressed as:

% \begin{equation}
%   \mathcal{L}_{SDS}(I_{\Phi}) \triangleq E_{t,\epsilon}[w(t)(\epsilon_{\phi}(z_t,y,t)-\epsilon)\frac{\partial I_{\Phi}}{\partial\Phi}],
%   \label{eq:SDSObserv}
% \end{equation}
\begin{equation}
    \mathcal{L}_{\text{SDS}}(I_{\Phi}) \triangleq \mathbb{E}_{t,\epsilon} \left[ w(t) \left( \epsilon_{\phi}(z_t, y, t) - \epsilon \right) \frac{\partial I_{\Phi}}{\partial \Phi} \right],
  \label{eq:SDSObservGA}
\end{equation}
where the input $I_{\Phi}$ represents a rendered image from a 3D representation, such as 3D Gaussians, with optimizable parameters $\Phi$. $\epsilon_{\phi}$ corresponds to the predicted noise of diffusion networks, which is produced by incorporating the noise image $z_t$ as input and conditioning it with a text or image $y$ at timestep $t$. The noise image $z_t$ is derived by introducing noise $\epsilon$ into $I_{\Phi}$ at timestep $t$. The loss is weighted by the diffusion scheduler $w(t)$. 
% \vspace{-3mm}

\subsection{Overview of the RTPD Algorithm}\label{subsec:algo_overview}
Fig.~\ref{fig:Overview} presents an overview of our RTPD algorithm.
It is grounded in the Hausdorff distance-based penetration depth calculation method (Sec.~\ref{sec:preliminary}).
%, similar to that of Tang et al.~\shortcite{SIG09HIST}.
The process consists of two primary phases: penetration surface extraction and Hausdorff distance calculation.
We leverage the RTX platform's capabilities to accelerate both of these steps.

\begin{figure*}[t]
    \centering
    \includegraphics[width=0.8\textwidth]{Image/overview.pdf}
    \caption{The overview of RT-based penetration depth calculation algorithm overview}
    \label{fig:Overview}
\end{figure*}

The penetration surface extraction phase focuses on identifying the overlapped region between two objects.
\revision{The penetration surface is defined as a set of polygons from one object, where at least one of its vertices lies within the other object. 
Note that in our work, we focus on triangles rather than general polygons, as they are processed most efficiently on the RTX platform.}
To facilitate this extraction, we introduce a ray-tracing-based \revision{Point-in-Polyhedron} test (RT-PIP), significantly accelerated through the use of RT cores (Sec.~\ref{sec:RT-PIP}).
This test capitalizes on the ray-surface intersection capabilities of the RTX platform.
%
Initially, a Geometry Acceleration Structure (GAS) is generated for each object, as required by the RTX platform.
The RT-PIP module takes the GAS of one object (e.g., $GAS_{A}$) and the point set of the other object (e.g., $P_{B}$).
It outputs a set of points (e.g., $P_{\partial B}$) representing the penetration region, indicating their location inside the opposing object.
Subsequently, a penetration surface (e.g., $\partial B$) is constructed using this point set (e.g., $P_{\partial B}$) (Sec.~\ref{subsec:surfaceGen}).
%
The generated penetration surfaces (e.g., $\partial A$ and $\partial B$) are then forwarded to the next step. 

The Hausdorff distance calculation phase utilizes the ray-surface intersection test of the RTX platform (Sec.~\ref{sec:RT-Hausdorff}) to compute the Hausdorff distance between two objects.
We introduce a novel Ray-Tracing-based Hausdorff DISTance algorithm, RT-HDIST.
It begins by generating GAS for the two penetration surfaces, $P_{\partial A}$ and $P_{\partial B}$, derived from the preceding step.
RT-HDIST processes the GAS of a penetration surface (e.g., $GAS_{\partial A}$) alongside the point set of the other penetration surface (e.g., $P_{\partial B}$) to compute the penetration depth between them.
The algorithm operates bidirectionally, considering both directions ($\partial A \to \partial B$ and $\partial B \to \partial A$).
The final penetration depth between the two objects, A and B, is determined by selecting the larger value from these two directional computations.

%In the Hausdorff distance calculation step, we compute the Hausdorff distance between given two objects using a ray-surface-intersection test. (Sec.~\ref{sec:RT-Hausdorff}) Initially, we construct the GAS for both $\partial A$ and $\partial B$ to utilize the RT-core effectively. The RT-based Hausdorff distance algorithms then determine the Hausdorff distance by processing the GAS of one object (e.g. $GAS_{\partial A}$) and set of the vertices of the other (e.g. $P_{\partial B}$). Following the Hausdorff distance definition (Eq.~\ref{equation:hausdorff_definition}), we compute the Hausdorff distance to both directions ($\partial A \to \partial B$) and ($\partial B \to \partial A$). As a result, the bigger one is the final Hausdorff distance, and also it is the penetration depth between input object $A$ and $B$.


%the proposed RT-based penetration depth calculation pipeline.
%Our proposed methods adopt Tang's Hausdorff-based penetration depth methods~\cite{SIG09HIST}. The pipeline is divided into the penetration surface extraction step and the Hausdorff distance calculation between the penetration surface steps. However, since Tang's approach is not suitable for the RT platform in detail, we modified and applied it with appropriate methods.

%The penetration surface extraction step is extracting overlapped surfaces on other objects. To utilize the RT core, we use the ray-intersection-based PIP(Point-In-Polygon) algorithms instead of collision detection between two objects which Tang et al.~\cite{SIG09HIST} used. (Sec.~\ref{sec:RT-PIP})
%RT core-based PIP test uses a ray-surface intersection test. For purpose this, we generate the GAS(Geometry Acceleration Structure) for each object. RT core-based PIP test takes the GAS of one object (e.g. $GAS_{A}$) and a set of vertex of another one (e.g. $P_{B}$). Then this computes the penetrated vertex set of another one (e.g. $P_{\partial B}$). To calculate the Hausdorff distance, these vertex sets change to objects constructed by penetrated surface (e.g. $\partial B$). Finally, the two generated overlapped surface objects $\partial A$ and $\partial B$ are used in the Hausdorff distance calculation step.

Our goal is to increase the robustness of T2I models, particularly with rare or unseen concepts, which they struggle to generate. To do so, we investigate a retrieval-augmented generation approach, through which we dynamically select images that can provide the model with missing visual cues. Importantly, we focus on models that were not trained for RAG, and show that existing image conditioning tools can be leveraged to support RAG post-hoc.
As depicted in \cref{fig:overview}, given a text prompt and a T2I generative model, we start by generating an image with the given prompt. Then, we query a VLM with the image, and ask it to decide if the image matches the prompt. If it does not, we aim to retrieve images representing the concepts that are missing from the image, and provide them as additional context to the model to guide it toward better alignment with the prompt.
In the following sections, we describe our method by answering key questions:
(1) How do we know which images to retrieve? 
(2) How can we retrieve the required images? 
and (3) How can we use the retrieved images for unknown concept generation?
By answering these questions, we achieve our goal of generating new concepts that the model struggles to generate on its own.

\vspace{-3pt}
\subsection{Which images to retrieve?}
The amount of images we can pass to a model is limited, hence we need to decide which images to pass as references to guide the generation of a base model. As T2I models are already capable of generating many concepts successfully, an efficient strategy would be passing only concepts they struggle to generate as references, and not all the concepts in a prompt.
To find the challenging concepts,
we utilize a VLM and apply a step-by-step method, as depicted in the bottom part of \cref{fig:overview}. First, we generate an initial image with a T2I model. Then, we provide the VLM with the initial prompt and image, and ask it if they match. If not, we ask the VLM to identify missing concepts and
focus on content and style, since these are easy to convey through visual cues.
As demonstrated in \cref{tab:ablations}, empirical experiments show that image retrieval from detailed image captions yields better results than retrieval from brief, generic concept descriptions.
Therefore, after identifying the missing concepts, we ask the VLM to suggest detailed image captions for images that describe each of the concepts. 

\vspace{-4pt}
\subsubsection{Error Handling}
\label{subsec:err_hand}

The VLM may sometimes fail to identify the missing concepts in an image, and will respond that it is ``unable to respond''. In these rare cases, we allow up to 3 query repetitions, while increasing the query temperature in each repetition. Increasing the temperature allows for more diverse responses by encouraging the model to sample less probable words.
In most cases, using our suggested step-by-step method yields better results than retrieving images directly from the given prompt (see 
\cref{subsec:ablations}).
However, if the VLM still fails to identify the missing concepts after multiple attempts, we fall back to retrieving images directly from the prompt, as it usually means the VLM does not know what is the meaning of the prompt.

The used prompts can be found in \cref{app:prompts}.
Next, we turn to retrieve images based on the acquired image captions.

\vspace{-3pt}
\subsection{How to retrieve the required images?}

Given $n$ image captions, our goal is to retrieve the images that are most similar to these captions from a dataset. 
To retrieve images matching a given image caption, we compare the caption to all the images in the dataset using a text-image similarity metric and retrieve the top $k$ most similar images.
Text-to-image retrieval is an active research field~\cite{radford2021learning, zhai2023sigmoid, ray2024cola, vendrowinquire}, where no single method is perfect.
Retrieval is especially hard when the dataset does not contain an exact match to the query \cite{biswas2024efficient} or when the task is fine-grained retrieval, that depends on subtle details~\cite{wei2022fine}.
Hence, a common retrieval workflow is to first retrieve image candidates using pre-computed embeddings, and then re-rank the retrieved candidates using a different, often more expensive but accurate, method \cite{vendrowinquire}.
Following this workflow, we experimented with cosine similarity over different embeddings, and with multiple re-ranking methods of reference candidates.
Although re-ranking sometimes yields better results compared to simply using cosine similarity between CLIP~\cite{radford2021learning} embeddings, the difference was not significant in most of our experiments. Therefore, for simplicity, we use cosine similarity between CLIP embeddings as our similarity metric (see \cref{tab:sim_metrics}, \cref{subsec:ablations} for more details about our experiments with different similarity metrics).

\vspace{-3pt}
\subsection{How to use the retrieved images?}
Putting it all together, after retrieving relevant images, all that is left to do is to use them as context so they are beneficial for the model.
We experimented with two types of models; models that are trained to receive images as input in addition to text and have ICL capabilities (e.g., OmniGen~\cite{xiao2024omnigen}), and T2I models augmented with an image encoder in post-training (e.g., SDXL~\cite{podellsdxl} with IP-adapter~\cite{ye2023ip}).
As the first model type has ICL capabilities, we can supply the retrieved images as examples that it can learn from, by adjusting the original prompt.
Although the second model type lacks true ICL capabilities, it offers image-based control functionalities, which we can leverage for applying RAG over it with our method.
Hence, for both model types, we augment the input prompt to contain a reference of the retrieved images as examples.
Formally, given a prompt $p$, $n$ concepts, and $k$ compatible images for each concept, we use the following template to create a new prompt:
``According to these examples of 
$\mathord{<}c_1\mathord{>:<}img_{1,1}\mathord{>}, ... , \mathord{<}img_{1,k}\mathord{>}, ... , \mathord{<}c_n\mathord{>:<}img_{n,1}\mathord{>}, ... , $
$\mathord{<}img_{n,k}\mathord{>}$,
generate $\mathord{<}p\mathord{>}$'', 
where $c_i$ for $i\in{[1,n]}$ is a compatible image caption of the image $\mathord{<}img_{i,j}\mathord{>},  j\in{[1,k]}$. 

This prompt allows models to learn missing concepts from the images, guiding them to generate the required result. 

\textbf{Personalized Generation}: 
For models that support multiple input images, we can apply our method for personalized generation as well, to generate rare concept combinations with personal concepts. In this case, we use one image for personal content, and 1+ other reference images for missing concepts. For example, given an image of a specific cat, we can generate diverse images of it, ranging from a mug featuring the cat to a lego of it or atypical situations like the cat writing code or teaching a classroom of dogs (\cref{fig:personalization}).
\vspace{-2pt}
\begin{figure}[htp]
  \centering
   \includegraphics[width=\linewidth]{Assets/personalization.pdf}
   \caption{\textbf{Personalized generation example.}
   \emph{ImageRAG} can work in parallel with personalization methods and enhance their capabilities. For example, although OmniGen can generate images of a subject based on an image, it struggles to generate some concepts. Using references retrieved by our method, it can generate the required result.
}
   \label{fig:personalization}\vspace{-10pt}
\end{figure}

% \section{Experiments}

\subsection{Experimental Setup}

\textbf{Datasets.} We use three categories from the Amazon Reviews dataset~\cite{mcauley2015image} for our experiments: ``Sports and Outdoors'' (\textbf{Sports}), ``Beauty'' (\textbf{Beauty}), and ``CDs and Vinyl'' (\textbf{CDs}). Each user’s historical reviews are considered ``actions'' and are sorted chronologically as action sequences, with earlier reviews appearing first. To evaluate the models, we adopt the widely used leave-last-out protocol~\cite{kang2018sasrec,zhao2022revisiting,rajput2023tiger}, where the last item and second-to-last item in each action sequence are used for testing and validation, respectively. More details about the datasets can be found in~\Cref{app:datasets}.

\textbf{Compared methods.} We compare the performance of ActionPiece with the following methods: (1)~ID-based sequential recommendation methods, including BERT4Rec~\cite{sun2019bert4rec}, and SASRec~\cite{kang2018sasrec}; (2)~feature-enhanced sequential recommendation methods, such as FDSA~\cite{zhang2019fdsa}, S$^3$-Rec~\cite{zhou2020s3}, and VQ-Rec~\cite{hou2023vqrec}; and (3)~generative recommendation methods, including P5-CID~\cite{hua2023p5cid}, TIGER~\cite{rajput2023tiger}, LMIndexer~\cite{jin2024lmindexer}, HSTU~\cite{zhai2024hstu}, and SPM-SID~\cite{singh2024spmsid}, each representing a different action tokenization method (\Cref{tab:act_tokenization}). A detailed description of these baselines is provided in~\Cref{appendix:baselines}.

\textbf{Evaluation settings.} Following~\citet{rajput2023tiger}, we use Recall@$K$ and NDCG@$K$ as metrics to evaluate the methods, where $K \in \{5, 10\}$. Model checkpoints with the best performance on the validation set are used for evaluation on the test set. We run the experiments with five random seeds and report the average metrics.

\textbf{Implementation details.} Please refer to~\Cref{appendix:implementation} for detailed implementation and hyperparameter settings.


\subsection{Overall Performance}

We compare ActionPiece with sequential recommendation and generative recommendation baselines, which use various action tokenization methods, across three public datasets. The results are shown in~\Cref{tab:performance}. 

For the compared methods, we observe that those using item features generally outperform item ID-only methods. This indicates that incorporating features enhances recommendation performance. Among the methods leveraging item features (``Feature + ID'' and ``Generative''), generative recommendation models achieve better performance. These results further confirm that injecting semantics into item indexing and optimizing at a sub-item level enables generative models to better use semantic information and improve recommendation performance. Among all the baselines, SPM-SID achieves the best results. By incorporating the SentencePiece model~\cite{kudo2018sentencepiece}, SPM-SID replaces popular semantic ID patterns within each item with new tokens, benefiting from a larger vocabulary.

\begin{table}[t!]
    \small
    \centering
	\caption{Ablation analysis of ActionPiece. The recommendation performance is measured using NDCG@$10$. The best performance is denoted in \textbf{bold} fonts.}
	\label{tab:ablation}
	\vskip 0.1in
% 	\setlength{\tabcolsep}{1mm}{
% \resizebox{2.1\columnwidth}{!}{
    \begin{tabular}{lccc}
	\toprule
	\multicolumn{1}{c}{\textbf{Variants}} & \textbf{Sports} & \textbf{Beauty} & \textbf{CDs} \\
	\midrule
	\midrule
    \multicolumn{4}{@{}c}{\textit{TIGER with larger vocabularies}} \\
    \midrule
    (1.1) TIGER\ -\ 1k ($4 \times 2^8$) & 0.0225 & 0.0384 & 0.0411 \\
    (1.2) TIGER-49k ($6 \times 2^{13}$) & 0.0162 & 0.0317 & 0.0338 \\
    (1.3) TIGER-66k ($4 \times 2^{14}$) & 0.0194 & N/A$^\dag$ & 0.0319 \\
    \midrule
    \multicolumn{4}{@{}c}{\textit{Vocabulary construction}} \\
    \midrule
    (2.1) \emph{w/o} tokenization & 0.0215 & 0.0389 & 0.0346 \\
    (2.2) \emph{w/o} context-aware & 0.0258 & 0.0416 & 0.0429 \\
    (2.3) \emph{w/o} weighted counting & 0.0257 & 0.0412 & 0.0435 \\
    \midrule
    \multicolumn{4}{@{}c}{\textit{Set permutation regularization}} \\
    \midrule
    (3.1) only for inference & 0.0192 & 0.0316 & 0.0329 \\
    (3.2) only for training & 0.0244 & 0.0387 & 0.0422 \\
    \midrule
    ActionPiece (40k) & \textbf{0.0264} & \textbf{0.0424} & \textbf{0.0451} \\
    \bottomrule
	\end{tabular}
	\vspace{0.05cm}
	\begin{flushleft}
        $^\dag$ not applicable as $2^{14}$ is larger than \#items in Beauty.
    \end{flushleft}
% 	}}
    \vskip -0.2in
\end{table}

\begin{figure*}[t!]
    \begin{center}
    \includegraphics[width=\linewidth]{fig/vocab_size.pdf}
    \vskip -0.1in
    \caption{Analysis of recommendation performance (NDCG@10, $\uparrow$) and average tokenized sequence length (NSL, $\downarrow$) \wrt vocabulary size across three datasets.
    % NSL refers to the normalized sequence length, calculated relative to the number of initial tokens.
    ``N/A’’ indicates that ActionPiece is not applied, \ie action sequences are represented solely by initial tokens.}
    \label{fig:vocab_size}
    \end{center}
    \vskip -0.2in
\end{figure*}

Our proposed ActionPiece consistently outperforms all baselines across three datasets, achieving a significant improvement in NDCG@$10$. It surpasses the best-performing baseline method by $6.00\%$ to $12.82\%$. Unlike existing methods, ActionPiece is the first context-aware action sequence tokenizer, \ie the same action can be tokenized into different tokens depending on its surrounding context. This allows ActionPiece to capture important sequence-level feature patterns that enhance recommendation performance.

% \begin{figure}[t]
% % \vskip 0.2in
% \begin{center}
% \centerline{\includegraphics[width=0.85\columnwidth]{fig/ndcg_vs_vocab_size.pdf}}
% \end{center}
% % \vskip -0.3in
% \vspace{-0.3in}
% \caption{Comparison of performance and vocabulary size (\#token for TIGER, SPM-SID, and ActionPiece; \#item for SASRec; and \#item+\#attribute for S$^3$-Rec) on ``Sports'' dataset.
% % By adjusting the vocabulary size, ActionPiece is shown to balance memory efficiency and recommendation performance.
% }
% \label{fig:intro}
% % \vskip -0.2in
% \vspace{-0.1in}
% \end{figure}


\subsection{Ablation Study}\label{sec:ablation}

We conduct ablation analyses in~\Cref{tab:ablation} to study how each proposed technique contributes to ActionPiece.\\
\hspace*{3mm} (1)~We increase the vocabulary size of TIGER, to determine whether the performance gain of ActionPiece is solely due to scaling up the number of tokens in the vocabulary. By increasing the number of semantic ID digits per item~($4 \rightarrow 6$) and the number of candidate semantic IDs per digit~($2^8 \rightarrow 2^{13}\ \text{or}\ 2^{14}$), we create two variants with vocabularies larger than ActionPiece. However, these TIGER variants perform worse than ActionPiece, and even the original TIGER with only $1024$ tokens. The experimental results suggest that scaling up the vocabulary size for generative recommendation models is challenging, consistent with the observations from~\citet{zhang2024moc}.\\
\hspace*{3mm} (2)~To evaluate the effectiveness of the proposed vocabulary construction techniques, we introduce the following variants: \emph{(2.1)~w/o tokenization}, which skips vocabulary construction, using item features directly as tokens; \emph{(2.2)~w/o context-aware}, which only considers co-occurrences and merges tokens within each action during vocabulary construction and segmentation; and \emph{(2.3)~w/o weighted counting}, which treats all token pairs equally rather than using the weights defined in~\Cref{eq:p_one_set,eq:p_two_sets}. The results indicate that removing any of these techniques reduces performance, demonstrating the importance of these methods for building a context-aware tokenizer.\\
\hspace*{3mm} (3)~To evaluate the effectiveness of SPR, we revert to naive segmentation, as described in~\Cref{subsubsec:segmentation}, during model training and inference, respectively. The results show that replacing SPR with naive segmentation in either training or inference degrades performance.

\begin{figure}[t!]
    \begin{center}
    \includegraphics[width=0.95\columnwidth]{fig/token_util.pdf}
    \vskip -0.1in
    \caption{Analysis of token utilization rate (\%) during model training \wrt segmentation strategy.
    % ``SPR'' denotes set permutation regularization.
    }
    \label{fig:token_util}
    \end{center}
    % \vskip -0.3in
    \vskip -0.3in
\end{figure}

\subsection{Further Analysis}

% In this section, we analyze the impact of key hyperparameters in vocabulary construction and segmentation.

\subsubsection{Performance and Efficiency \wrt Vocabulary Size}

Vocabulary size is a key hyperparameter for language tokenizers~\cite{meta2024llama3,dagan2024getting}. In this study, we investigate how adjusting vocabulary size affects the generative recommendation models. We use the normalized sequence length (NSL)~\cite{dagan2024getting} to measure the length of tokenized sequences, where a smaller NSL indicates fewer tokens per tokenized sequence. We experiment with vocabulary sizes in \{N/A, 5k, 10k, 20k, 30k, 40k\}, where ``N/A'' represents the direct use of item features as tokens. As shown in~\Cref{fig:vocab_size}, increasing the vocabulary size improves recommendation performance and reduces the tokenized sequence length. Conversely, reducing the vocabulary size lowers the number of model parameters, improving memory efficiency. This analysis demonstrates that adjusting vocabulary size enables a trade-off between model performance, sequence length, and memory efficiency.

\subsubsection{Token Utilization Rate \wrt Segmentation Strategy}\label{sec:token_utilization}

As described in~\Cref{subsubsec:training}, applying SPR augments the training corpus by producing multiple token sequences that share the same semantics. In~\Cref{tab:ablation}, we observe that incorporating SPR significantly improves recommendation performance. One possible reason is that SPR increases token utilization rates. To validate this assumption, we segment the action sequences in each training epoch using two strategies: naive segmentation and SPR. As shown in~\Cref{fig:token_util}, naive segmentation uses only $56.89\%$ of tokens for model training, limiting the model's ability to generalize to unseen action sequences. In contrast, SPR achieves a token utilization rate of $87.01\%$ after the first training epoch, with further increases as training progresses. These results demonstrate that the proposed SPR segmentation strategy improves the utilization of ActionPiece tokens, enabling better generalization and enhanced performance.


\subsubsection{Performance \wrt Inference-Time Ensembles}\label{sec:inference_time_ensemble}

As described in~\Cref{subsubsec:inference}, ActionPiece supports inference-time ensembling by using SPR segmentation. We vary the number of ensembled segments, $q$, in \{N/A, 1, 3, 5, 7\}, where ``N/A'' indicates using naive segmentation during model inference. As shown in~\Cref{fig:ensemble}, ensembling more tokenized sequences improves ActionPiece's recommendation performance. However, the performance gains slow down as $q$ increases to $5$ and $7$. Since a higher $q$ also increases the computational cost of inference, this creates a trade-off between performance and computational budget in practice.

\begin{figure}[t!]
    \begin{center}
    \includegraphics[width=\columnwidth]{fig/ensemble.pdf}
    \vskip -0.15in
    \caption{Analysis of performance (NDCG@10, $\uparrow$) \wrt the number of ensembled segments $q$ during model inference.}
    \label{fig:ensemble}
    \end{center}
    \vskip -0.25in
\end{figure}

\subsection{Case Study}\label{subsec:case}

To understand how GR models benefit from the unordered feature setting and context-aware action sequence tokenization, we present an illustrative example in~\Cref{fig:case}.

Each item in the action sequence is represented as a feature set, with each item consisting of five features. The features within an item do not require a specific order. The first step of tokenization leverages the unordered nature of the feature set and applies set permutation regularization~(\Cref{subsubsec:segmentation}). This process arranges each feature set into a specific permutation and iteratively groups features based on the constructed vocabulary~(\Cref{subsubsec:vocab_construct}). This results in different segments that convey the same semantics. Each segment is represented as a sequence of sets, where each set corresponds to a token in the vocabulary.

By examining the segments and their corresponding token sequences, we identify four types of tokens, as annotated in~\Cref{fig:case}: (1) a subset of features from a single item (token {\setlength{\fboxsep}{0pt}\colorbox{myblue}{14844}} corresponds to features {\setlength{\fboxsep}{0pt}\colorbox{myblue}{747}} and {\setlength{\fboxsep}{0pt}\colorbox{myblue}{923}} of the T-shirt); (2) a set containing a single feature (feature {\setlength{\fboxsep}{0pt}\colorbox{mygreen}{76}} of the socks); (3) all features of a single item (token {\setlength{\fboxsep}{0pt}\colorbox{myyellow}{7995}} corresponds to all features of the shorts); and (4) features from multiple items (\eg token {\setlength{\fboxsep}{0pt}\colorbox{myblue}{83}\colorbox{mygreen}{16}} includes feature {\setlength{\fboxsep}{0pt}\colorbox{myblue}{923}} from the T-shirt and feature {\setlength{\fboxsep}{0pt}\colorbox{mygreen}{679}} from the socks, while token {\setlength{\fboxsep}{0pt}\colorbox{mygreen}{19}\colorbox{myyellow}{895}} includes feature {\setlength{\fboxsep}{0pt}\colorbox{mygreen}{1100}} from the socks as well as features {\setlength{\fboxsep}{0pt}\colorbox{myyellow}{560}} and {\setlength{\fboxsep}{0pt}\colorbox{myyellow}{943}} from the shorts). Notably, the fourth type of token demonstrates that the features of one action can be segmented and grouped with features from adjacent actions. This results in different tokens for the same action depending on the surrounding context, showcasing the context-aware tokenization process of ActionPiece.


% \section{Conclusion}

In this paper, we propose a sample weight averaging strategy to address variance inflation of previous independence-based sample reweighting algorithms. 
We prove its validity and benefits with theoretical analyses. 
Extensive experiments across synthetic and multiple real-world datasets demonstrate its superiority in mitigating variance inflation and improving covariate-shift generalization.  


% \section{Related Works}

\textbf{Enhancing LLMs' Theory of Mind.} There has been systematic evaluation that revealed LLMs' limitations in achieving robust Theory of Mind inference \citep{ullman2023large, shapira2023clever}. To enhance LLMs' Theory of Mind capacity, recent works have proposed various prompting techniques. For instance, SimToM \citep{wilf2023think} encourages LLMs to adopt perspective-taking, PercepToM \citep{jung2024perceptions} improves perception-to-belief inference by extracting relevant contextual details, and \citet{huang2024notion} utilize an LLM as a world model to track environmental changes and refine prompts. Explicit symbolic modules also seem to improve LLM's accuracy through dynamic updates based on inputs. Specifically, TimeToM \citep{hou2024timetom} constructs a temporal reasoning framework to support inference, while SymbolicToM \citep{sclar2023minding} uses graphical representations to track characters' beliefs. Additionally, \citet{wagner2024mind} investigates ToM's necessity and the level of recursion required for specific tasks. However, these approaches continue to exhibit systematic errors in long contexts, complex behaviors, and recursive reasoning due to inherent limitations in inference and modeling \citep{jin2024mmtom,shi2024muma}. Most of them rely on domain-specific designs, lacking open-endedness.


\textbf{Model-based Theory of Mind inference.} Model-based Theory of Mind inference, in particular, Bayesian inverse planning (BIP) \citep{baker2009action,ullman2009help,baker2017rational,zhi2020online}, explicitly constructs representations of agents' mental states and how mental states guide agents' behavior via Bayesian Theory of Mind (BToM) models. These methods can reverse engineer human ToM inference in simple domains \citep[e.g.,][]{baker2017rational,netanyahu2021phase,shu2021agent}. Recent works have proposed to combine BIP with LLMs to achieve robust ToM inference in more realistic settings \citep{ying2023neuro, jin2024mmtom, shi2024muma}. However, these methods require manual specification of the BToM models as well as rigid, domain-specific implementations of Bayesian inference, limiting their adaptability to open-ended scenarios. To overcome this limitation, we propose \ours, a method capable of automatically modeling mental variables across diverse conditions and conducting automated BIP without domain-specific knowledge or implementations.


\begin{figure*}[ht]
  \centering
  \includegraphics[width=\linewidth]{figures/benchmarks_and_models.pdf}
    \vspace{-15pt}
  \caption{Examples questions (top panels) and the necessary Bayesian Theory of Mind (BToM) model for Bayesian inverse planning (bottom panels) in diverse Theory of Mind benchmarks. \ours aims to answer any Theory of Mind question in a variety of benchmarks, encompassing different mental variables, observable contexts, numbers of agents, the presence or absence of utterances, wording styles, and modalities. It proposes and iteratively adjusts an appropriate BToM and conducts automated Bayesian inverse planning based on the model.
  There can be more types of questions/models in each benchmark beyond the examples shown in this figure.}
  \label{fig:benchmarks_and_models}
  %\vspace{-0.75em}
  \vspace{-10pt}
\end{figure*}



\textbf{Automated Modeling with LLMs.} There has been an increasing interest in integrating LLMs with inductive reasoning and probabilistic inference for automated modeling. \citet{piriyakulkij2024doing} combine LLMs with Sequential Monte Carlo to perform probabilistic inference about underlying rules. Iterative hypothesis refinement techniques \citep{qiu2023phenomenal} further enhance LLM-based inductive reasoning by iteratively proposing, selecting, and refining textual hypotheses of rules. Beyond rule-based hypotheses, \citet{wang2023hypothesis} prompt LLMs to generate natural language hypotheses that are then implemented as verifiable programs, while \citet{li2024automated} propose a method in which LLMs construct, critique, and refine statistical models represented as probabilistic programs for data modeling. \citet{cross2024hypothetical} leverage LLMs to propose and evaluate agent strategies for multi-agent planning but do not specifically infer individual mental variables. Our method also aims to achieve automated modeling with LLMs. Unlike prior works, we propose a novel automated model discovery approach for Bayesian inverse planning, where the objective is to confidently infer any mental variable given any context via constructing a suitable Bayesian Theory of Mind model.



\section{Conclusion}
\label{sec:Conclusion}

In this paper, we propose MIM, a novel and universal multi-modal content interest modeling paradigm for industrial-scale applications. By introducing a decomposed training paradigm and a representation center, MIM effectively integrates multi-modal features into User Behavior Modeling (UBM), addressing the limitations of traditional ID-based methods and existing multi-modal approaches. The proposed method aligns user interests with content embeddings, significantly enhancing prediction performance while maintaining efficiency in large-scale real-world applications.
Extensive experiments validate the effectiveness and generalizability of MIM, showing consistent improvements across diverse baselines and datasets. These results demonstrate MIM's practical value and its potential to drive further innovation in multi-modal user modeling for industrial applications.



% \section{Sectioning Commands}

% Your work should use standard \LaTeX\ sectioning commands:
% \verb|\section|, \verb|\subsection|, \verb|\subsubsection|,
% \verb|\paragraph|, and \verb|\subparagraph|. The sectioning levels up to
% \verb|\subsusection| should be numbered; do not remove the numbering
% from the commands.

% Simulating a sectioning command by setting the first word or words of
% a paragraph in boldface or italicized text is {\bfseries not allowed.}

% Below are examples of sectioning commands.

% \subsection{Subsection}
% \label{sec:subsection}

% This is a subsection.

% \subsubsection{Subsubsection}
% \label{sec:subsubsection}

% This is a subsubsection.

% \paragraph{Paragraph}

% This is a paragraph.

% \subparagraph{Subparagraph}

% This is a subparagraph.

% \section{Tables}

% The ``\verb|acmart|'' document class includes the ``\verb|booktabs|''
% package --- \url{https://ctan.org/pkg/booktabs} --- for preparing
% high-quality tables.

% Table captions are placed {\itshape above} the table.

% Because tables cannot be split across pages, the best placement for
% them is typically the top of the page nearest their initial cite.  To
% ensure this proper ``floating'' placement of tables, use the
% environment \textbf{table} to enclose the table's contents and the
% table caption.  The contents of the table itself must go in the
% \textbf{tabular} environment, to be aligned properly in rows and
% columns, with the desired horizontal and vertical rules.  Again,
% detailed instructions on \textbf{tabular} material are found in the
% \textit{\LaTeX\ User's Guide}.

% Immediately following this sentence is the point at which
% Table~\ref{tab:freq} is included in the input file; compare the
% placement of the table here with the table in the printed output of
% this document.

% \begin{table}
%   \caption{Frequency of Special Characters}
%   \label{tab:freq}
%   \begin{tabular}{ccl}
%     \toprule
%     Non-English or Math&Frequency&Comments\\
%     \midrule
%     \O & 1 in 1,000& For Swedish names\\
%     $\pi$ & 1 in 5& Common in math\\
%     \$ & 4 in 5 & Used in business\\
%     $\Psi^2_1$ & 1 in 40,000& Unexplained usage\\
%   \bottomrule
% \end{tabular}
% \end{table}

% To set a wider table, which takes up the whole width of the page's
% live area, use the environment \textbf{table*} to enclose the table's
% contents and the table caption.  As with a single-column table, this
% wide table will ``float'' to a location deemed more
% desirable. Immediately following this sentence is the point at which
% Table~\ref{tab:commands} is included in the input file; again, it is
% instructive to compare the placement of the table here with the table
% in the printed output of this document.

% \begin{table*}
%   \caption{Some Typical Commands}
%   \label{tab:commands}
%   \begin{tabular}{ccl}
%     \toprule
%     Command &A Number & Comments\\
%     \midrule
%     \texttt{{\char'134}author} & 100& Author \\
%     \texttt{{\char'134}table}& 300 & For tables\\
%     \texttt{{\char'134}table*}& 400& For wider tables\\
%     \bottomrule
%   \end{tabular}
% \end{table*}

% Always use midrule to separate table header rows from data rows, and
% use it only for this purpose. This enables assistive technologies to
% recognise table headers and support their users in navigating tables
% more easily.

% \section{Math Equations}
% You may want to display math equations in three distinct styles:
% inline, numbered or non-numbered display.  Each of the three are
% discussed in the next sections.

% \subsection{Inline (In-text) Equations}
% A formula that appears in the running text is called an inline or
% in-text formula.  It is produced by the \textbf{math} environment,
% which can be invoked with the usual
% \texttt{{\char'134}begin\,\ldots{\char'134}end} construction or with
% the short form \texttt{\$\,\ldots\$}. You can use any of the symbols
% and structures, from $\alpha$ to $\omega$, available in
% \LaTeX~\cite{Lamport:LaTeX}; this section will simply show a few
% examples of in-text equations in context. Notice how this equation:
% \begin{math}
%   \lim_{n\rightarrow \infty}x=0
% \end{math},
% set here in in-line math style, looks slightly different when
% set in display style.  (See next section).

% \subsection{Display Equations}
% A numbered display equation---one set off by vertical space from the
% text and centered horizontally---is produced by the \textbf{equation}
% environment. An unnumbered display equation is produced by the
% \textbf{displaymath} environment.

% Again, in either environment, you can use any of the symbols and
% structures available in \LaTeX\@; this section will just give a couple
% of examples of display equations in context.  First, consider the
% equation, shown as an inline equation above:
% \begin{equation}
%   \lim_{n\rightarrow \infty}x=0
% \end{equation}
% Notice how it is formatted somewhat differently in
% the \textbf{displaymath}
% environment.  Now, we'll enter an unnumbered equation:
% \begin{displaymath}
%   \sum_{i=0}^{\infty} x + 1
% \end{displaymath}
% and follow it with another numbered equation:
% \begin{equation}
%   \sum_{i=0}^{\infty}x_i=\int_{0}^{\pi+2} f
% \end{equation}
% just to demonstrate \LaTeX's able handling of numbering.

% \section{Figures}

% The ``\verb|figure|'' environment should be used for figures. One or
% more images can be placed within a figure. If your figure contains
% third-party material, you must clearly identify it as such, as shown
% in the example below.
% \begin{figure}[h]
%   \centering
%   \includegraphics[width=\linewidth]{sample-franklin}
%   \caption{1907 Franklin Model D roadster. Photograph by Harris \&
%     Ewing, Inc. [Public domain], via Wikimedia
%     Commons. (\url{https://goo.gl/VLCRBB}).}
%   \Description{A woman and a girl in white dresses sit in an open car.}
% \end{figure}

% Your figures should contain a caption which describes the figure to
% the reader.

% Figure captions are placed {\itshape below} the figure.

% Every figure should also have a figure description unless it is purely
% decorative. These descriptions convey what's in the image to someone
% who cannot see it. They are also used by search engine crawlers for
% indexing images, and when images cannot be loaded.

% A figure description must be unformatted plain text less than 2000
% characters long (including spaces).  {\bfseries Figure descriptions
%   should not repeat the figure caption – their purpose is to capture
%   important information that is not already provided in the caption or
%   the main text of the paper.} For figures that convey important and
% complex new information, a short text description may not be
% adequate. More complex alternative descriptions can be placed in an
% appendix and referenced in a short figure description. For example,
% provide a data table capturing the information in a bar chart, or a
% structured list representing a graph.  For additional information
% regarding how best to write figure descriptions and why doing this is
% so important, please see
% \url{https://www.acm.org/publications/taps/describing-figures/}.

% \subsection{The ``Teaser Figure''}

% A ``teaser figure'' is an image, or set of images in one figure, that
% are placed after all author and affiliation information, and before
% the body of the article, spanning the page. If you wish to have such a
% figure in your article, place the command immediately before the
% \verb|\maketitle| command:
% \begin{verbatim}
%   \begin{teaserfigure}
%     \includegraphics[width=\textwidth]{sampleteaser}
%     \caption{figure caption}
%     \Description{figure description}
%   \end{teaserfigure}
% \end{verbatim}

% \section{Citations and Bibliographies}

% The use of \BibTeX\ for the preparation and formatting of one's
% references is strongly recommended. Authors' names should be complete
% --- use full first names (``Donald E. Knuth'') not initials
% (``D. E. Knuth'') --- and the salient identifying features of a
% reference should be included: title, year, volume, number, pages,
% article DOI, etc.

% The bibliography is included in your source document with these two
% commands, placed just before the \verb|\end{document}| command:
% \begin{verbatim}
%   \bibliographystyle{ACM-Reference-Format}
%   \bibliography{bibfile}
% \end{verbatim}
% where ``\verb|bibfile|'' is the name, without the ``\verb|.bib|''
% suffix, of the \BibTeX\ file.

% Citations and references are numbered by default. A small number of
% ACM publications have citations and references formatted in the
% ``author year'' style; for these exceptions, please include this
% command in the {\bfseries preamble} (before the command
% ``\verb|\begin{document}|'') of your \LaTeX\ source:
% \begin{verbatim}
%   \citestyle{acmauthoryear}
% \end{verbatim}


%   Some examples.  A paginated journal article \cite{Abril07}, an
%   enumerated journal article \cite{Cohen07}, a reference to an entire
%   issue \cite{JCohen96}, a monograph (whole book) \cite{Kosiur01}, a
%   monograph/whole book in a series (see 2a in spec. document)
%   \cite{Harel79}, a divisible-book such as an anthology or compilation
%   \cite{Editor00} followed by the same example, however we only output
%   the series if the volume number is given \cite{Editor00a} (so
%   Editor00a's series should NOT be present since it has no vol. no.),
%   a chapter in a divisible book \cite{Spector90}, a chapter in a
%   divisible book in a series \cite{Douglass98}, a multi-volume work as
%   book \cite{Knuth97}, a couple of articles in a proceedings (of a
%   conference, symposium, workshop for example) (paginated proceedings
%   article) \cite{Andler79, Hagerup1993}, a proceedings article with
%   all possible elements \cite{Smith10}, an example of an enumerated
%   proceedings article \cite{VanGundy07}, an informally published work
%   \cite{Harel78}, a couple of preprints \cite{Bornmann2019,
%     AnzarootPBM14}, a doctoral dissertation \cite{Clarkson85}, a
%   master's thesis: \cite{anisi03}, an online document / world wide web
%   resource \cite{Thornburg01, Ablamowicz07, Poker06}, a video game
%   (Case 1) \cite{Obama08} and (Case 2) \cite{Novak03} and \cite{Lee05}
%   and (Case 3) a patent \cite{JoeScientist001}, work accepted for
%   publication \cite{rous08}, 'YYYYb'-test for prolific author
%   \cite{SaeediMEJ10} and \cite{SaeediJETC10}. Other cites might
%   contain 'duplicate' DOI and URLs (some SIAM articles)
%   \cite{Kirschmer:2010:AEI:1958016.1958018}. Boris / Barbara Beeton:
%   multi-volume works as books \cite{MR781536} and \cite{MR781537}. A
%   couple of citations with DOIs:
%   \cite{2004:ITE:1009386.1010128,Kirschmer:2010:AEI:1958016.1958018}. Online
%   citations: \cite{TUGInstmem, Thornburg01, CTANacmart}.
%   Artifacts: \cite{R} and \cite{UMassCitations}.

% \section{Acknowledgments}

% Identification of funding sources and other support, and thanks to
% individuals and groups that assisted in the research and the
% preparation of the work should be included in an acknowledgment
% section, which is placed just before the reference section in your
% document.

% This section has a special environment:
% \begin{verbatim}
%   \begin{acks}
%   ...
%   \end{acks}
% \end{verbatim}
% so that the information contained therein can be more easily collected
% during the article metadata extraction phase, and to ensure
% consistency in the spelling of the section heading.

% Authors should not prepare this section as a numbered or unnumbered {\verb|\section|}; please use the ``{\verb|acks|}'' environment.

% \section{Appendices}

% If your work needs an appendix, add it before the
% ``\verb|\end{document}|'' command at the conclusion of your source
% document.

% Start the appendix with the ``\verb|appendix|'' command:
% \begin{verbatim}
%   \appendix
% \end{verbatim}
% and note that in the appendix, sections are lettered, not
% numbered. This document has two appendices, demonstrating the section
% and subsection identification method.

% \section{Multi-language papers}

% Papers may be written in languages other than English or include
% titles, subtitles, keywords and abstracts in different languages (as a
% rule, a paper in a language other than English should include an
% English title and an English abstract).  Use \verb|language=...| for
% every language used in the paper.  The last language indicated is the
% main language of the paper.  For example, a French paper with
% additional titles and abstracts in English and German may start with
% the following command
% \begin{verbatim}
% \documentclass[sigconf, language=english, language=german,
%                language=french]{acmart}
% \end{verbatim}

% The title, subtitle, keywords and abstract will be typeset in the main
% language of the paper.  The commands \verb|\translatedXXX|, \verb|XXX|
% begin title, subtitle and keywords, can be used to set these elements
% in the other languages.  The environment \verb|translatedabstract| is
% used to set the translation of the abstract.  These commands and
% environment have a mandatory first argument: the language of the
% second argument.  See \verb|sample-sigconf-i13n.tex| file for examples
% of their usage.

% \section{SIGCHI Extended Abstracts}

% The ``\verb|sigchi-a|'' template style (available only in \LaTeX\ and
% not in Word) produces a landscape-orientation formatted article, with
% a wide left margin. Three environments are available for use with the
% ``\verb|sigchi-a|'' template style, and produce formatted output in
% the margin:
% \begin{description}
% \item[\texttt{sidebar}:]  Place formatted text in the margin.
% \item[\texttt{marginfigure}:] Place a figure in the margin.
% \item[\texttt{margintable}:] Place a table in the margin.
% \end{description}

%%
%% The acknowledgments section is defined using the "acks" environment
%% (and NOT an unnumbered section). This ensures the proper
%% identification of the section in the article metadata, and the
%% consistent spelling of the heading.
% \begin{acks}
% To Robert, for the bagels and explaining CMYK and color spaces.
% \end{acks}

%%
%% The next two lines define the bibliography style to be used, and
%% the bibliography file.
\bibliographystyle{ACM-Reference-Format}
\bibliography{mim}


%%
%% If your work has an appendix, this is the place to put it.
% \appendix

% \section{Research Methods}

% \subsection{Part One}

% Lorem ipsum dolor sit amet, consectetur adipiscing elit. Morbi
% malesuada, quam in pulvinar varius, metus nunc fermentum urna, id
% sollicitudin purus odio sit amet enim. Aliquam ullamcorper eu ipsum
% vel mollis. Curabitur quis dictum nisl. Phasellus vel semper risus, et
% lacinia dolor. Integer ultricies commodo sem nec semper.

% \subsection{Part Two}

% Etiam commodo feugiat nisl pulvinar pellentesque. Etiam auctor sodales
% ligula, non varius nibh pulvinar semper. Suspendisse nec lectus non
% ipsum convallis congue hendrerit vitae sapien. Donec at laoreet
% eros. Vivamus non purus placerat, scelerisque diam eu, cursus
% ante. Etiam aliquam tortor auctor efficitur mattis.

% \section{Online Resources}

% Nam id fermentum dui. Suspendisse sagittis tortor a nulla mollis, in
% pulvinar ex pretium. Sed interdum orci quis metus euismod, et sagittis
% enim maximus. Vestibulum gravida massa ut felis suscipit
% congue. Quisque mattis elit a risus ultrices commodo venenatis eget
% dui. Etiam sagittis eleifend elementum.

% Nam interdum magna at lectus dignissim, ac dignissim lorem
% rhoncus. Maecenas eu arcu ac neque placerat aliquam. Nunc pulvinar
% massa et mattis lacinia.

\end{document}
\endinput
%%
%% End of file `sample-sigconf.tex'.
