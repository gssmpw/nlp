\documentclass[letterpaper, 10 pt, conference]{ieeeconf}  % Comment this line out if you need a4paper
                                                          
%\documentclass[a4paper, 10pt, conference]{ieeeconf}      % Use this line for a4 paper

\IEEEoverridecommandlockouts                              % This command is only needed if 
                                                          % you want to use the \thanks command
                                                          
\overrideIEEEmargins                                      % Needed to meet printer requirements.
% The preceding line is only needed to identify funding in the first footnote. If that is unneeded, please comment it out.
\usepackage{bm}
\usepackage{cite}
\usepackage{amsmath,amssymb,amsfonts,mathcomp}
\usepackage{algorithmic}
\usepackage{float}
\usepackage{here}
\usepackage{textcomp}
% \usepackage{ulem}
\usepackage[pdftex]{graphicx}

\usepackage{algorithmic}
\usepackage{algorithm}
\usepackage{booktabs}
\usepackage{comment}
\usepackage{url}
\usepackage{hyperref}
\newcommand{\ken}[1]{{\color{red}Ken: #1}}
\newcommand{\takami}[1]{{\color{magenta}Takami: #1}}
\newcommand{\newpart}[1]{{\color{blue}#1}}
\newcommand{\koide}[1]{{\color{blue}Koide: #1}}
\newcommand{\koidebf}[2]{{\bf #1}{\color{blue}Koide: #2}}
\newcommand{\Nagata}[1]{{\color{green}Nagata: #1}}

\DeclareMathOperator*{\argmax}{arg\,max}
\DeclareMathOperator*{\argmin}{arg\,min}

% 図と本文の余白を切り詰める (ちょっと詰めすぎかも?)
\setlength\floatsep{8pt}
\setlength\textfloatsep{8pt}

% \def\BibTeX{{\rm B\kern-.05em{\sc i\kern-.025em b}\kern-.08em
%     T\kern-.1667em\lower.7ex\hbox{E}\kern-.125emX}}
\begin{document}

\title{SLAMSpoof: Practical LiDAR Spoofing Attacks on Localization Systems Guided by Scan Matching Vulnerability Analysis
}

\author{Rokuto Nagata$^{1}$, Kenji Koide$^{2}$, Yuki Hayakawa$^{1}$, Ryo Suzuki$^{1}$, Kazuma Ikeda$^{1}$, Ozora Sako$^{1}$, 
\\Qi Alfred Chen$^{3}$, Takami Sato$^{3}$, and Kentaro Yoshioka$^{1}$% <-this % stops a space
\thanks{This research was supported in part by the NSF CNS-2145493, CNS-1929771, USDOT under Grant 69A3552348327 for the CARMEN+ University Transportation Center, JST CREST JPMJCR23M4, JST PRESTO JPMJPR22PA, JSPS KAKENHI 24K02940, and Amano Institute of Technology.}
% <-this % stops a space
\thanks{$^{1}$ Department of Electronics and Electrical Engineering, Keio University}
\thanks{$^{2}$ Department of Information Technology
and Human Factors, the National Institute of Advanced Industrial Science and Technology}
\thanks{$^{3}$ Department of Computer Science, University of California, Irvine}
}

\maketitle

\begin{abstract}
Accurate localization is essential for enabling modern full self-driving services. These services heavily rely on map-based traffic information to reduce uncertainties in recognizing lane shapes, traffic light locations, and traffic signs. Achieving this level of reliance on map information requires centimeter-level localization accuracy, which is currently only achievable with LiDAR sensors. However, LiDAR is known to be vulnerable to spoofing attacks that emit malicious lasers against LiDAR to overwrite its measurements. Once localization is compromised, the attack could lead the victim off roads or make them ignore traffic lights. Motivated by these serious safety implications, we design \textit{SLAMSpoof}, the first practical LiDAR spoofing attack on localization systems for self-driving to assess the actual attack significance on autonomous vehicles. SLAMSpoof can effectively find the effective attack location based on our scan matching vulnerability score (SMVS), a point-wise metric representing the potential vulnerability to spoofing attacks. To evaluate the effectiveness of the attack, we conduct real-world experiments on ground vehicles and confirm its high capability in real-world scenarios, inducing position errors of $\geq$4.2 meters (more than typical lane width) for all 3 popular LiDAR-based localization algorithms. We finally discuss the potential countermeasures of this attack. \textbf{Code is available at} \href{https://github.com/Keio-CSG/slamspoof}{\textit{https://github.com/Keio-CSG/slamspoof}}

%\ken{TBD: countermeasureの入れ方}
%\Nagata{Abst英語化}
%The societal demand for autonomous driving is increasing year by year. To achieve safe autonomous driving, functions such as obstacle detection for collision avoidance, map generation via SLAM, and self-positioning estimation are indispensable. LiDAR sensors play a crucial role in realizing these functions. However, attack methods targeting LiDAR sensors, particularly spoofing attacks, have been proposed, posing a serious threat to the security of autonomous driving systems. While it is known that spoofing attacks have a significant impact on object detection functions, their effects on SLAM have not been sufficiently examined.
%In this study, we analyze the impact of spoofing attacks on LiDAR-based SLAM systems and propose a new metric called "Attackability" to quantify their vulnerabilities.The robustness of pose estimation against attacks heavily depends on the surrounding environmental structure. Therefore, we define Attackability based on the importance of each point in pose estimation and develop a method to identify locations that are more susceptible to attacks. Furthermore, we construct a SLAM Spoofing Framework centered around Attackability and propose a method to optimize the placement of attack devices. Additionally, we conducted real-world attack experiments on mobile robots, demonstrating for the first time the ability to attack SLAM in a real-world scenario. As a result, we confirmed that optimized attacks could cause the estimated position of the robot to deviate by up to 8.4 meters in KISS-ICP, 6.1 meters in A-LOAM, and 4.2 meters in hdl\_localization, revealing the potential for significant impacts on the safety of autonomous driving systems.
%\Nagata{abstのciteを削除}

%自律走行の社会的需要は年々高まっている。安全な自律走行を実現するためには障害物検知による衝突回避機能やSLAMによる地図生成、自己位置推定機能が欠かせない。これらの機能を実現するためにLiDARセンサは非常に重要な役割を果たしている。しかし、LiDARセンサに対する攻撃手法、特にspoofing攻撃が提案されており、自律走行システムのセキュリティに深刻な脅威をもたらしている。Spoofing攻撃は物体検出機能に重大な影響を及ぼすことが知られているが、SLAMへの影響に関しては検証が不十分である。

%本研究では、LiDARベースのSLAMシステムに対するspoofing攻撃の影響を分析し、その脆弱性を定量化する新しい指標「Attackability」を提案する。

%姿勢推定の攻撃に対する頑強性は周囲環境構造に大きく依存する。そのため、姿勢推定における各点の重要度に基づいてAttackabilityを定義し、攻撃の影響を受けやすい場所を特定する手法を開発した。さらに、Attackabilityを核としたSLAM Spoofingフレームワークを構築し、攻撃装置の設置位置を最適化する手法を提案する。また、実世界で移動ロボットに対する攻撃実験を行い、spoofingによるSLAMへの攻撃能力を初めて現実世界で示した。その結果、最適化された攻撃により、ロボットの推定位置がKISS-ICPでは最大8.4m、A-LOAMでは最大6.1m、hdl\_localizationでは最大4.2m偏差することを確認し、自律走行システムの安全性に重大な影響を与える可能性を明らかにした。
\end{abstract}

%\begin{IEEEkeywords}
%component, formatting, style, styling, insert
%\end{IEEEkeywords}

\section{Introduction}
Autonomous driving is one of the most significant technological breakthroughs in the last decades. We can easily take a self-driving taxi service in several cities. The key sensor to enable full self-driving is LiDAR (Light Detection And Ranging) which can obtain the surrounding 3D environment as point clouds. After LiDAR outperformed in the 2007 DARPA Urban Challenge~\cite{urmson2007tartan}, all full self-driving vehicles permitted on public roads in California equip at least one LiDAR sensor on their rooftop~\cite{permitted_ad}.

LiDAR sensing technology is crucial in various aspects of the self-driving pipeline, from obstacle detection to localization systems. Its role is particularly indispensable in localization, which is essential for enabling full self-driving capabilities. Whether using prior map-based methods or Simultaneous Localization and Mapping (SLAM) techniques, these systems heavily rely on LiDAR data to achieve the centimeter-level accuracy required for safe autonomous navigation. This high precision is critical, as the margin between a vehicle and lane markings is typically less than 1 meter~\cite{sato2021dirty}. Such accuracy ensures that self-driving vehicles can operate safely and reliably in complex urban environments.

\begin{figure}[tb]
  \centering
  \includegraphics[trim=0mm 40mm 0mm 15mm,clip,scale=0.30]{header.pdf}
  \caption{
  %\ken{永田くん: (a)は後ろに同じ図があるため、ここでは消去. (b)だけlinewidthギリギリまで拡大.左下の空いている所に表を作り、各手法でどれくらいズレが生じたかまとめる\Nagata{(a)の写真を削除し、左下に表を追加}}
  \textbf{Real-world attack demo of SLAMspoof on driving vehicle.}
  SLAMSpoof attack successfully deviates the victim vehicle from the planned benign trajectory (while line) to the attack-influenced trajectories (dotted red lines) corresponding to the three major localization algorithms. The induced position errors are $\geq$4.2 meter, which is wider than the typical lane width as shown in the lower-left table.
  % The white line represents the robot's trajectory without the attack, while the red line shows the significantly displaced path resulting from our attack, as observed across three major localization algorithms. The table (lower left) presents the APE in meters for each tested localization method under attack. All methods experienced localization errors of several meters, proving the attack's effectiveness in physical-world scenarios.
  }
  \label{header_fig}
\end{figure}

%左下の表は本研究で採用した位置推定手法に対して実世界で攻撃した場合の位置ずれである。指標はAPEで、単位はメートルである。いずれの手法でも数メートルレベルの推定誤差を誘発出来ており、現実世界での効果的な攻撃を実現したといえる。

Despite the critical role of LiDAR in self-driving, recent lines of research have reported the vulnerabilities of LiDAR against spoofing attacks~\cite{petit2015remote, cao2019adversarial, sato2024lidar}, which shoot malicious lasers with more power than legitimate ones to overwrite the measurements. With LiDAR spoofing attacks, the adversary can inject ghost points into the point cloud or remove a part of the point cloud.
Specifically, prior work~\cite{sato2024lidar} demonstrates that removal attack is effective on state-of-the-art LiDARs, removing almost all points in 80$^\circ$ horizontal range.

Nevertheless, none of the prior works have successfully demonstrated LiDAR spoofing attacks on LiDAR-based localization for driving autonomous vehicles~\cite{xu2023sok}. Yoshida et al,~\cite{yoshida2022adversarial} shows that LiDAR-based SLAM could be vulnerable to malicious manipulation on the point cloud, but this work targets only 2D-LiDAR and does not mention how to achieve such malicious manipulation. Fukunaga et al.~\cite{fukunagarandom} demonstrate that random LiDAR spoofing attacks can compromise the point cloud environmental map, but this work does not target the online localization for the driving vehicle. Furthermore, this work admits that their random attack cannot cause major attack impacts on the x (longitudinal) and y (lateral) axes.

\begin{comment}
    
\end{comment}
Motivated by this, we conduct the first security analysis of LiDAR-based localization systems against LiDAR spoofing attack for driving autonomous vehicles, and design \textit{SLAMSpoof}, the first practical LiDAR spoofing attacks on LiDAR-based localization for self-driving to assess the actual attack significance on autonomous vehicles.
% Ken:繰り返しのためまとめた。
%SLAMSpoof can effectively find the effective attack location based on our \textit{Scan Matching Vulnerability Score (SMVS)}, a metric representing the potential vulnerability to spoofing attacks. To evaluate the attack effectiveness, we conduct real-world attack experiments on ground vehicles and confirm the high attack capability ability in a real-world scenario with the induced position errors of $\geq$4.2 meters, which is even wider than the typical lane width, for three popular LiDAR-based localization algorithms.
Our study has the following contributions:
\begin{itemize}
    \item We introduce \textit{Scan Matching Vulnerability Score (SMVS)}, a metric to quantify the scan matching's vulnerability to spoofing, establishing a new standard for security evaluation in localization systems.
    \item We develop SLAMspoof, a framework centered on SMVS, which significantly enhances real-world attack feasibility compared to conventional methods.
    \item We successfully demonstrate the first real-world spoofing attack on the self-positioning system of a LiDAR-equipped robot during long-distance travel.
    \item We discuss potential defenses against SLAMspoof.
    %\item \newpart{We discuss potential defenses against SLAMspoof.} %This reveals specific vulnerabilities in various LiDAR-based localization systems with different architectures in real-world scenarios and highlights new challenges for improving the safety of autonomous driving technology.
\end{itemize}

%本研究では、これらの課題を克服し、現実的な自動運転システムに対して有効な攻撃手法を提案することで、LiDAR spoofingの脆弱性を体系的に解明し、セキュアなSLAM技術への道を確立する。
%本研究の主な貢献は以下の通りである:
%In this study, we aim to overcome these challenges by proposing an effective attack method against real-world autonomous driving systems, guided by scan matching vulnerability analysis. The main contributions of this study are as follows:
%\begin{enumerate}
%\item \textbf{Introduction of the "Attackability" metric:} \koide{ダブルクォートは ``ABC" (左側はShift+@)} We propose a novel metric to quantify scan matching's vulnerability to spoofing. This metric evaluates the likelihood of position shifts by focusing on each point's contribution to pose estimation, identifying areas susceptible to spoofing along the path. This establishes a new standard for security evaluation in SLAM, recognizing that environments with rich geometric features may naturally resist spoofing attempts and \textit{vice versa}.
%\item  \textbf{「Attackability」指標の導入:} SLAMアルゴリズムのspoofingに対する脆弱性を定量化する新指標を提案する。十分な幾何特徴を持つ環境ではspoofing範囲外の点群によって正しく姿勢推定がなされるので攻撃の効果が薄いと考えられる。そこで各点の姿勢推定への寄与度に着目して位置ずれの起きやすさを評価し、経路上のspoofingに対し脆弱な箇所を特定する。これにより、SLAMにおける新たなセキュリティ評価の基準を確立する。
%\item \textbf{Construction of a SLAM spoofing framework:} We construct an Attackability-based framework that enables attackers to optimize attack device placement and efficiently tamper with critical point clouds for pose estimation. By determining optimal placement through simulation, this approach eliminates the need for real-world trial and error, significantly enhancing real-world attack feasibility compared to conventional methods. The framework also provides valuable insights for developing robust SLAM defense strategies.
%\item \textbf{SLAM Spoofingフレームワークの構築:} Attackabilityを核とした新たなフレームワークを構築し、攻撃者は攻撃装置の設置位置を最適化し姿勢推定を行う上で重要な点群を効率よく改ざん可能となる。シミュレーションより最適な設置位置が求まり、実世界の試行錯誤は不要となるため従来手法に比べ実世界の攻撃実現性に優れる。同時に、この知見は防御策の開発にも活用できる。
%\item \textbf{Real-world demonstration of SLAM spoofing:} Using our proposed method, we successfully carried out the first real-world attack on the self-positioning system of a LiDAR-equipped robot during long-distance travel. This reveals specific vulnerabilities in LiDAR-based SLAM systems in real-world scenarios and highlights new challenges for improving the safety of autonomous driving technology.
%\item \textbf{実世界でのSLAM spoofing攻撃の実証:} 提案手法を用いて長距離走行するLiDAR搭載ロボットの自己位置推定システムに対する攻撃を実世界で初めて成功した。現実世界におけるLiDARベースのSLAMシステムの具体的な脆弱性を明らかにし、自動運転技術の安全性向上に向けた新たな課題を提起する。
%\end{enumerate}

\section{Related work and Background}
%\Nagata{佐藤さんが追加した引用を.bibに追加}
\subsection{LiDAR Spoofing Attacks} \label{sec:lidar_spoofing}
LiDAR is fundamentally vulnerable to other malicious laser sources as the nature of the LiDAR mechanism that senses the laser reflection. This type of attacks is known as LiDAR spoofing attacks~\cite{shin2017illusion, petit2015remote, cao2019adversarial}, where Fig.\ref{example_spoofing} shows the point cloud under attack. Based on the attack effects, there are two major types of LiDAR spoofing attacks: injection and removal attacks. For injection attacks, once initial works~\cite{shin2017illusion, cao2019adversarial} demonstrated that their attacks can inject a few hundred points on the point cloud, the state-of-the-art attack~\cite{sato2024lidar} can compromise 99\% of points within 83$^\circ$ ($\sim$7k points). 
On the other hand, removal attacks~\cite{shin2017illusion, cao2023you,sato2024lidar} try to erase the legitimate points from the detected point clouds. The state-of-the-art removal attack, HFR attack~\cite{sato2024lidar}, can remove almost all points within 80$^\circ$ range. 

\begin{figure}[tb]
  \centering
  \includegraphics[trim=25mm 40mm 0mm 30mm,scale=0.36]{Fig19.pdf}
  \caption{The effects of spoofing attacks on LiDAR. In an injection attack (top-right), false point cloud data representing a non-existent wall is inserted into the LiDAR scan. In a removal attack (bottom-right), injected noise obscures the pedestrians' point cloud, effectively erasing them from the LiDAR scan.}
  \label{example_spoofing}
\end{figure}

%LiDARは自律走行において最も重要なセンサの1つである。自律走行分野では光を照射し、その反射光を受光するまでの時間から物体との距離を測定するdirect Time of Flight (dToF)方式のLiDARが広く用いられている。この測定により、周囲の環境が3次元的な点群として得られ、この点群データは障害物検知による衝突回避やSLAMによる地図生成に利用される。しかしdToF LiDARはその測定原理上、悪意あるレーザ光によって点群を操作される脆弱性を持つ。

%この脆弱性を突いた攻撃として、先行研究では大きく2つの攻撃が提案されている。連続的なレーザパルス波形を照射することで偽の物体を注入する攻撃\cite{petit2015remote, cao2019adversarial}と、ジャミングによって広範囲の点群を消失させる攻撃\cite{shin2017illusion, cao2023you,sato2024lidar}である。
%提案されているLiDAR spoofing攻撃の中で、最も広範囲かつ多様なLiDARセンサにおいて有効性を確認しているのが佐藤ら\cite{sato2024lidar}の手法である。\cite{sato2024lidar}では高周波レーザパルスをLiDARに照射し、測距データを偽のノイズで上書きする手法HFRを提案し、障害物の点群を消失させ、衝突事故の発生を実世界で実証している。加えて、\cite{sato2024lidar}では偽の車や歩行者といった任意形状の偽物体注入攻撃も実証している。
%この論文でも佐藤ら\cite{sato2024lidar}が提案したセットアップを用いて注入、消失攻撃を行う。

\subsection{LiDAR-based Localization} \label{sec:lidar_slam}
Localization is a methodology to identify the ego location on a map with sensor inputs. While the multi-sensor fusion (e.g., LiDAR-IMU fusion) approach has been gaining popularity owing to its accuracy and robustness~\cite{fastlio2,koide2024_02}, the localization solely with LiDAR is still popular due to its simplicity and applicability~\cite{vizzo2023ral, cticp, mulls}. In this study, we focus on the three popular LiDAR-based localization algorithms, LiDAR-based localization methods, namely A-LOAM~\cite{zhang2014loam}, KISS-ICP~\cite{vizzo2023ral}, and hdl\_localization~\cite{koide2019portable}, covering three major point distance calculations (plane and edge feature matching~\cite{zhang2014loam}, ICP~\cite{zhang1994iterative}, NDT~\cite{magnusson2009three}) and two major scan matching schemes (online local map and using prior-map).
This study is the first to explore the vulnerability of these major LiDAR-based localization methods against LiDAR spoofing attacks in driving vehicles.
% Localization methods based on multi-sensor fusion (e.g., LiDAR-IMU fusion) have been gaining popularity owing to their accuracy and robustness \cite{fastlio2,koide2024_02}. However, methods that rely solely on LiDAR still offer advantages in terms of system simplicity and applicability, and they have been actively studied \cite{vizzo2023ral, cticp, mulls} and used in many robotic applications. As these LiDAR-only methods share the basic principles of point cloud registration with the aforementioned sensor-fusion-based methods, in this work, we focus on analyzing the vulnerability of three popular LiDAR-based localization methods, namely A-LOAM \cite{zhang2014loam}, KISS-ICP \cite{vizzo2023ral}, and hdl\_localization \cite{koide2019portable}. These selected algorithms employ distinct point distance metrics (plane and edge feature matching, ICP, NDT) and scan matching schemes (creating a local map vs. using a prior-map). We consider demonstrating the vulnerability of these methods highlights the generality of our proposed attack method and provides insights into the vulnerability of a wide range of LiDAR-based localization systems.

% LiDAR-based localization methods can be categorized by whether using a prior map or not (i.e., SLAM). 
% スペースもないので各手法の説明は外しました
% A-LOAM is a re-implementation of LOAM \cite{zhang2014loam}, which is one of the most influential LiDAR SLAM algorithms based on the matching of edge and plane feature points. KISS-ICP is the most popular algorithm among recent LiDAR-based SLAM methods that is based on point-to-point ICP combined with adaptive thresholding and robust estimation. hdl\_localization \cite{koide2019portable} is a prior-map-based localization algorithm that uses NDT scan matching to estimate the sensor pose on the map. 

\subsection{Prior Attack Attempts on LiDAR-based Localization} \label{sec:prior_attack}
As also discussed in~\cite{xu2023sok}, no successful LiDAR spoofing attack was demonstrated on LiDAR-based SLAM for driving autonomous vehicles. Yoshida et al.~\cite{yoshida2022adversarial} showed that LiDAR-based SLAM could be vulnerable to a malicious point injection. However, this work only evaluates their attack on 2D-LiDAR, which cannot support autonomous driving, in a lab-level scenario. %Furthermore, the injection attack does not work on recent lines of LiDAR as discussed in~\S\ref{sec:lidar_spoofing}.
Fukunaga et al.~\cite{fukunagarandom} demonstrate random LiDAR spoofing attacks to compromise the prior environment map generation. This work does not target to attack the online localization systems in the driving vehicle and admits that their random attack cannot cause major attack impacts on the x (longitudinal) and y (lateral) axes. %which is a 5-meter lateral deviation per 160-meter longitudinal driving. While attack on such a long distance could be possible in the map generation process, but should not be easy to keep attacking fastly driving vehicles. At least, no prior works have demonstrated such attacks.

%Previous research on spoofing attacks against LiDAR has mainly focused on deceiving object detectors, with only a few studies evaluating the impact of spoofing attacks on SLAM systems. Fukunaga et al.\cite{fukunagarandom} evaluated the impact on SLAM systems by conducting a spoofing attack that causes the disappearance of point clouds\cite{sato2024lidar}. However, this study attacked during the pre-map creation phase, causing position deviations by performing route planning with the contaminated map. Since pre-map creation is infrequent and involves human inspection, this approach does not represent a realistic attack on autonomous driving systems. Additionally, the reported results are primarily based on simulations, and attacks on long-distance autonomous driving systems in real-world scenarios have not been demonstrated.
%従来のLiDARへのspoofing攻撃研究は主に物体検出器を騙すことに主眼が置かれており、SLAMに関してspoofing攻撃の影響を評価した研究は少ない。福永ら\cite{fukunagarandom}は点群を消失させるspoofing攻撃\cite{sato2024lidar}を行ってSLAMシステムへの影響を評価している。しかしながら、この研究では事前地図作成中に攻撃し、汚染された地図を用いて経路計画を行うことで位置ずれを生じさせている。事前地図作成の頻度は低く、人によるチェックも入るため、現実的な自動運転システムへの攻撃ではない。また報告された結果はシミュレーションが主であり、長距離走行する実世界での自動運転システムに対する攻撃は実証されていない。

\subsection{Threat Model and Attack Goal}
We assume that the attacker can place a LiDAR spoofing device on a roadside where the victim robot will pass through, which is known as ``Spoofer placed in environment'' threat model~\cite{hallyburton2022security}. We assume that the victim uses a 3D LiDAR such as Velodyne VLP-16~\cite{VLP16} and Livox Horizon~\cite{livox_horizon} and solely relies on LiDAR-based localization without sensor fusion with other sensors such as IMU and GNSS.  
We assume robots that follow a fixed route (e.g., autonomous bus) as the attack target, and the attacker can know the route where the victim will pass by.
The attacker can know which LiDAR is used in the victim vehicle based on the appearance or some documents. However, we assume that the attacker cannot know which localization algorithm is used by the victim. The attacker aims to deviate the victim from their driving lane. According to~\cite{sato2021dirty}, the deviation needs 0.29 meters on local roads and 0.74 meters on highways to touch the lane line.

% 吉岡:こちらはexperiment sectionに移動
%\noindent \textbf{Attacker Capabilities:}
%The attacker places an attack device on the roadside and irradiates the target vehicle with a laser when it passes by to tamper with the target's LiDAR point cloud. In this study, we consider two types of spoofing attacks that have been physically demonstrated in current LiDAR spoofing research: injecting wall-like point clouds and deleting point clouds within a certain range \cite{sato2024lidar}. Examples of affected point clouds are shown in Figure \ref{example_spoofing}.
%With a setup similar to \cite{sato2024lidar}, it is possible to inject or delete point clouds within a range of approximately 80 degrees for the VLP-16, allowing for the tampering of approximately 5000 point clouds. \koide{何点中の5000点?}\Nagata{28800点中の5000点です。spoofing範囲内に限れば6800点中の5000点です。}
%To realize attacks on robots moving outdoors, we use a spoofer with tracking using an infrared camera, as proposed by Suzuki \cite{suzukiwip}. This attack device can track a vehicle traveling at 35 km/h from a distance of 45 meters and continuously perform spoofing attacks on the LiDAR sensor.

\section{Methodology: SLAMSpoof Attack}

To systematically evaluate the security of LiDAR-based localization in driving vehicles, we design the SLAMSpoof attack. As discussed in~\S\ref{sec:prior_attack}, prior attack attempts with random spoofing attacks were not successful due to the robustness of LiDAR-based localization. This result motivates us to design an effective but lightweight methodology to prioritize the potential attack locations based on their vulnerability level because reckless attack attempts are likely to be a waste of time and effort.. Fig.~\ref{proposed_method_concept} illustrate the overview of SLAMSpoof consisting of 3 steps: the attacker \textbf{(1)} first collects point cloud data of the entire potential routes where the victim will likely drive and calculate point-wise SMVS representating of the point importance on scan matching, \textbf{(2)} calculates the frame-wise SMVS, which quantifies the potential vulnerability with the point cloud of the frame, and \textbf{(3)} finally finds the best location to launch LiDAR spoofing attack. We will explain each step in the rest of this section.

%In this study, we propose a framework for effectively attacking LiDAR SLAM in autonomous driving systems. The overall flow is shown in Fig. \ref{proposed_method_concept}. The attacker first acquires a 3D map of the environment on which the target travels. Next, an Attackability map is generated from the acquired map, which quantifies the spoofing vulnerability based on the importance of each point cloud. Finally, by selecting a location with high Attackability, the optimal installation location for the attack device is determined to tamper with point clouds that are important for the SLAM system at that location.
%The main advantage of this method is that it can efficiently identify point clouds that will have a significant impact when tampered with and determine the optimal installation location for an attack device that can cause serious consequences. As a result, effective attacks can be carried out even against autonomous driving systems that travel long distances and complex routes, which was difficult with conventional methods. Details of each step are described below.

\begin{figure*}[h]
  \includegraphics[width=0.85\linewidth]{imgs/fig3.png}
  \centering
  \vspace{-0.1in}
  \caption{The overview of the \textit{SLAMSpoof} framework, which is based on Scan Matching Vulnerability Score (SMVS). First, the attacker replicates the target's route to acquire map data. The SMVS distribution is generated from the map data to identify the optimal attack location. The calculation methods for point-wise SMVS are described in III-B-1, for frame-wise SMVS in  \S \ref{sec:Frame-wise}, and the process of determining the spoofer placement is detailed in \S \ref{sec:attack_loc_sel}.}
  \label{proposed_method_concept}
  \vspace{-0.1in}
\end{figure*}

\subsection{Point-Wise SMVS Calculation}

Localization systems are designed to be robust against partial occlusions, making random point cloud alterations ineffective. Efficient attacks require identifying and manipulating critical areas that play key roles in scan matching. Scan matching determines pose by estimating optimal translation and rotation parameters between two point clouds, controlled by geometric constraints. We introduce point-wise SMVS, a metric that identifies feature points significantly affecting geometric consistency. It quantifies each point's influence on scan matching to calculate its importance.
%localizationシステムは部分的な遮蔽に対して頑健に設計されているため、点群のランダムな改ざんでは効果的な攻撃とならない。localizationシステムを効率的に攻撃するには、scan matchingにおいて重要な役割を果たす点群領域を特定し改ざんする必要がある。
%scan matchingは、2つの点群間の最適な並進・回転パラメータを推定することで姿勢を決定する。この推定プロセスは、点群間の幾何学的拘束によって制御される。我々はPoint-wise SMVSというメトリックで、この幾何学的整合性に大きな影響を与える特徴点を識別する。具体的には、各点がscan matchingに与える影響力を定量化し、それに基づいて重要度を算出する。

For the point-wise SMVS, analyzing the Hessian matrix of the scan matching objective function is crucial to identify vulnerabilities. The eigenvector corresponding to the minimum eigenvalue of this Hessian indicates the weakest constraint direction in pose estimation. Conversely, the eigenvector of each point's Hessian's maximum eigenvalue shows the direction of its strongest constraint.
Combining these insights quantifies each point's contribution to overall scan matching, revealing vulnerable directions and points strongly influencing constraints in those directions. Strategic manipulation of these critical points to weaken constraints can maximize attack effectiveness on scan matching.
To analyze point-wise SMVS for a frame, two point clouds are generated by applying noise and random sampling. These are then treated as target and source point clouds for analysis.
%Point-wise SMVSにおいてスキャンマッチングにおける脆弱性を特定し、効果的に攻撃するためには、目的関数のヘッセ行列の分析が不可欠である。スキャンマッチングの目的関数のヘッセ行列の最小固有値に対応する固有ベクトルは、姿勢推定における拘束が最も弱い方向を示す。一方、各点のヘッセ行列の最大固有値に対応する固有ベクトルは、その点が最も強く拘束をかけている方向を表す。
%これらの情報を組み合わせることで、各点が全体のscan matchingにどれだけ寄与しているか定量的に求めることができる。具体的には脆弱な方向と、その方向の拘束に強く影響している点が明らかになる。したがって、これらの重要な点を戦略的に改ざんし、拘束を弱める方向に操作することで、scan matchingへの攻撃効果を最大化できる。具体的には、point-wise SMVSを解析したいフレームにノイズ付加とランダムサンプリングを適用することで2つの点群を生成し、それらをターゲット・ソース点群として扱う。

Consider a point cloud $\mathcal{P} = \{ \mathbf{p}_i \in \mathbb{R}^3 |_{i=1, 2, \cdots N} \}$. Pose estimation is performed by estimating the pose $\mathbf{T}$ that minimizes the distance residual $e(\mathbf{T} \mathbf{p}_i)$ between the source point $\mathbf{p}_i$ and its nearest target point. In general, the minimization of the residual is achieved through iterative optimization. The update $\delta \mathbf{T}$ to the pose $\mathbf{T}$ is obtained using the Gauss-Newton method as follows.
\begin{align}
  \mathbf{J}_i &= \frac{\partial}{\partial {\mathbf T}} e(\mathbf{T} {\mathbf p}_i), \\
  \mathbf{H}_{\text{global}} &= \sum {^i\mathbf{H}_\text{local}} = \sum \mathbf{J}_i^T \mathbf{J}_i, \\
  \mathbf{b}_{\text{global}} &= \sum \mathbf{J}_i^T e(\mathbf{T} {\mathbf p}_i), \\
  \delta \mathbf{T} &= \mathbf{H}_{\text{global}}^{-1} {\mathbf b}_{\text{global}}.
\end{align}

Here, when $\mathbf{H}_{\text{global}}$ is not invertible, i.e., when the smallest eigenvalue is sufficiently small, the update vector $\delta \mathbf{T}$ diverges, making the pose estimation unstable. By finding the smallest eigenvalue and its corresponding eigenvector from the point cloud, and determining the strength of the constraint in that direction, the influence of each point on the update of the estimated pose can be quantified. Although the specific objective function varies depending on the scan matching method, it is assumed that most of them show a similar trend in the influence of each point since they are based on the matching of local geometric shapes. Here, we use the representative distribution-to-distribution matching function of Generalized ICP \cite{segal2009generalized}.

Let the eigenvalues and eigenvectors of the Hessian matrix $\mathbf{H}_\text{global}$ of a scan matching result be $\lambda^{j}_{\text{global}}, \mathbf{x}^{j}_{\text{global}}$. Also, let $\mathbf{x}^{\text{min}}_{\text{global}}$ be the eigenvector corresponding to the smallest eigenvalue. Let the Hessian matrix of point $\mathbf{p}_{i}$ be $^{i}\mathbf{H}_{\text{local}}$, and its eigenvalues and eigenvectors be $\lambda_{\mathbf{p}_i}^j, \mathbf{x}_{\mathbf{p}_i}^j$. The eigenvector $\mathbf{x}^{\text{max}}_{\mathbf{p}_{i}}$ corresponding to the largest eigenvalue represents the direction in which the point is most strongly constrained. By combining this information, the importance of point $\mathbf{p}_i$ is obtained as $I_{\mathbf{p}_i} = |\mathbf{x}^{\text{max}}_{\text{global}} \cdot \mathbf{x}^{\text{min}}_{\mathbf{p}_{i}}|$. The procedure for calculating the point-wise SMVS is summarized in Algorithm \ref{alg1}.
%ここで、$\mathbf{H}_{\text{global}}$ が invertible でないとき、すなわち最小固有値が小さいとき、更新量ベクトル $\delta \mathbf{T}$ が発散して姿勢推定が不安定となるので点群内から最小固有値と対応する固有ベクトルを求め、その方向への拘束の強さを求めることで各点が推定姿勢の更新に及ぼす影響の大きさを定量化することができる。具体的な目的関数はSLAM手法によって異なるがいずれも局所的な幾何形状のマッチングに基づいているため、各点ごとの影響度は同じような傾向を見せると仮定し、ここでは代表的な Generalized ICP の分布対分布マッチング関数を使用する。

%全体のヘッセ行列 $\mathbf{H}_\text{global}$ の固有値および固有ベクトルを$\lambda^{j}_{\text{global}}, \mathbf{x}^{j}_{\text{global}}$とする。また、最小固有値に対応する固有ベクトルを$\mathbf{x}^{\text{min}}_{\text{global}}$とする。点$p_{i}$のヘッセ行列を$^{i}\mathbf{H}_{\text{local}}$、固有値および固有ベクトルを$\lambda_{p_i}^j, \mathbf{x}_{\mathbf{p}_i}^j$とする。特に、最大固有値に対応する固有ベクトル$\mathbf{x}^{\text{max}}_{p_{i}}$は、その点が最も強く拘束をかけている方向を表す。これらの情報を組み合わせることで、各点の重要度を定量化できる。
%具体的には、点$p_i$の重要度を$I_{\mathbf{p}_i} = |\mathbf{x}^{\text{max}}_{\text{global}} \cdot \mathbf{x}^{\text{min}}_{p_{i}}|$ 
%と求める。各点重要度を求める手続きはアルゴリズム\ref{alg1}のように表せる。

\begin{algorithm}[tb]
    \caption{Calculation of point-wise SMVS}%\Nagata{表記を修正}}
    \label{alg1}
    \begin{algorithmic}[1] 
    \STATE $\mathbf{H}_{\text{global}} \leftarrow  \text{Hessian matrix of a scan matching result}$
    \STATE $\mathbf{x}^{\text{min}}_{\text{global}}, {\lambda}^{\text{min}}_{\text{global}} \leftarrow \text{Minimum eigenvector/value of } \mathbf{H}_{\text{global}}$
    \FOR{$\mathbf{p}_i \in \mathcal{P}$}
    \STATE $^{i}\mathbf{H}_{\text{local}} \leftarrow  \text{Local Hessian matrix of} \, \mathbf{p}_i$
    \STATE $ \mathbf{x}^{\text{max}}_{\mathbf{p}_i}, {\lambda}^{\text{max}}_{\mathbf{p}_i} \leftarrow \text{Maximum eigenvector/value of } {^{i}\mathbf{H}_{\text{local}}}$
    \STATE $ I_{\mathbf{p}_i} \leftarrow | \mathbf{x}^{\text{min}}_{\text{global}} \cdot \mathbf{x}^{\text{max}}_{\mathbf{p}_i}|$ 
    \ENDFOR
    \end{algorithmic}
\end{algorithm}


\subsection{Frame-Wise SMVS Calculation}\label{sec:Frame-wise}

\begin{figure}[tb]
  \centering
  \includegraphics[width=0.9\linewidth]{imgs/fig4.png}
  \vspace{-0.1in}
  \caption{Frame-wise SMVS calculation. We create an angular polar plot from point-wise SMVS ($score_{Rk}$), and then the frame-wise SMVS is calculated if the region is within the attack range or not, for each angular region.}
  \label{}
\end{figure}

In real-world LiDAR spoofing attacks, deploying multiple attack devices is impractical, limiting horizontal tampering range. This makes attacking localization systems difficult when high-importance points are widely dispersed (e.g., robots moving between buildings). To quantify this vulnerability, we propose the \textit{frame-wise} SMVS metric.
Frame-wise SMVS is designed to yield high scores when critical point clusters are concentrated in one area. Specifically, it outputs high values when a large number of point-wise SMVS-weighted points are concentrated within the spoofing range, and low values when dispersed.
By continuously evaluating frame-wise SMVS along a trajectory, we can identify the most effective attack points for a single spoofer. This approach accounts for real-world constraints and enables strategic targeting of localization systems.

To calculate frame-wise SMVS, the point cloud is segmented based on the horizontal azimuth angle of each point. The horizontal azimuth angle $\theta_{\mathbf{p}_i}$ of point $\mathbf{p}_i$ is
\begin{align}
\theta{\mathbf{p}_i} = \arctan{\frac{y_{\mathbf{p}_i}}{x_{\mathbf{p}_i}}},
\end{align}
where $x_{\mathbf{p}_i}$ and $y_{\mathbf{p}_i}$ are the $x$- and $y$-coordinates of point $\mathbf{p}_i$, respectively. The point cloud is divided into $n$ small regions, and each region is denoted as $R_{k}$. Each region contains points $\tilde{\mathcal{P}}_k$ whose horizontal angles are in the range $\frac{2 \pi k }{n}$ to $\frac{2 \pi (k + 1)}{n}$ radians, where $k = 0, \cdots, n - 1$. Then, the score of $R_k$ is computed as
\begin{align}
  \mathrm{score}_{R_k} = \sum_{\mathbf{p}_m \in \tilde{\mathcal{P}}_k} I_{\mathbf{p}_m}.
\end{align}
where $I_{\mathbf{p}_m}$ represents the point-wise SMVS.

To evaluate vulnerability with physical spoofing constraints, we calculate angular differences from a central spoofing angle. This central angle corresponds to the region with the highest score. We define the distance $d_k$ between a region $R_k$ and the center as:
$$d_k = |k_{\text{center}}-k| \pmod n$$
where $k_{\text{center}}$ is the index of the region with the highest score.

Frame-wise SMVS is determined by combining each region's score with its attack feasibility. The feasibility primarily depends on the horizontal angular constraint, with larger $d_k$ indicating increased attack difficulty. We designed a distance function $f(d_k)$ to reflect this relationship:
$$f(d_k) = -d_k + d_{\text{th}}$$
where $d_{\text{th}}$ is a threshold distance.
Consequently, we define frame-wise SMVS as:
$$ \text{Frame-wise SMVS} = \sum_{k=0}^{n-1} \mathrm{score}_{R_k} \cdot f(d_k)$$

% Attackability is considered to represent vulnerable areas by considering spoofing characteristics and assigning rewards and penalties according to the ease of a successful attack.

%LiDAR spoofingにおける攻撃の実行可能性は、物理的制約によって大きく制限される。攻撃装置を複数設置することは現実的ではないため、水平方向の改ざん範囲に強い制約がある。そのため、高重要度の点が広範囲に分散しているフレームでは、効果的な攻撃が困難となる。この課題に対処するため、我々は自動運転車のtrajectory上で最もspoofingに脆弱な領域を特定する新たな指標「Attackability」を提案する。Attackabilityは重要度の一極集中度合いを表し、重要度で重み付けされた点群がspoofing範囲内に集中する場合に高値を、分散する場合に低値を出力するよう設計された。この指標により、trajectoryに沿って連続的に脆弱性を評価し、1つのspooferで最も効果的に攻撃可能なポイントを特定することが可能となる。

%具体的にAttackablityを求める手続きを説明する。各点の水平方位角を求め、小領域に分割を行う。点$p_i$の水平方位角$\theta_{p_i}$は
%$$\theta_{p_i} = \arctan{\frac{y_{p_i}}{x_{p_i}}}$$
%である。ただし、$x_{p_i}$は点$p_i$の$x$座標、$y_{p_i}$は点$p_i$の$y$座標である。また、点群を$n$個の小領域に分割したとき小領域を$R_{k}$とする。この領域は水平角が$k$度から$k+\frac{360}{n}$度までの点を含む。ただし、$k = 0, \frac{360}{n}, \cdots \frac{360(n-1)}{n} \; (\mathrm{deg})$。
%小領域に$N'$点含まれているとき、$R_{k}$のスコアは
%$$\mathrm{score}_{R_k} = \sum_{m=0}^{N'-1} I_{p_m}$$

%改ざん範囲の制約を考慮し、点群が攻撃可能かどうか判断するためにspoofing中心角との角度差を求める。本研究では中心角は最もscoreが高い小領域とし%\Nagata{中心の決め方を説明}
%、他の小領域と最もscoreが高い小領域がどれだけ離れているかを求めることで角度差計算を実現している。
%ある小領域$R_k$の中心角との距離$d_k$は
%$$d_k = |k_{center}-k| \pmod n$$
%と定義される。ただし、$k_\text{center}$は中心角に相当する小領域のインデックス番号である。

%最終的に、Attackablityは小領域$R_{k}$のscoreと攻撃の成功しやすさから求められる。点群が改ざん可能であるかは主に水平方向の角度制約によって決まり、中心角との距離$d_k$が大きくなるほど攻撃は困難になる。したがって、攻撃が困難になったときにAttackablityが減少する、攻撃の成功しやすさを表す距離$d_k$の関数を設計した。本研究では$f(d_k)$は$f(d)=-d_k+d_{\text{th}}$と設計した。ゆえにAttackablityは
%$$ \mathrm{Attackablity} = \sum_{k=0}^{n-1} \mathrm{score}_{R_k} \cdot f(d_k)$$
%と定義される。Attackablityはspoofing特性を考慮し、攻撃の成功しやすさに応じて報酬とペナルティを与えることで脆弱な場所を表せると考えられる。

\begin{figure}[tb]
  \includegraphics[trim=0mm 40mm 0mm 50mm, clip,scale=0.3]{Fig17.pdf}
  \vspace{-0.5in}
  \caption{(Left) Example of low SMVS. The point cloud is distributed across a wide range of directions, allowing accurate pose estimation from the point cloud outside the spoofing range. (Right) Example of high SMVS. The directional distribution of the point cloud is biased, making it vulnerable as almost all points can be tampered with by spoofing. }
  \label{example_smvs}
\end{figure}
%\begin{figure}[tb]
%  \includegraphics[scale=0.3]{Fig17.pdf}
%  \caption{(左)Attackablityが低い例.幅広い方向に点群が分布し、spoofing範囲外の点群かによって正しく姿勢推定がなされる.(右)Attackablityが高い例.点群の方向分布に偏りがあり、spoofingによってほぼすべての点が改ざんされてしまうため脆弱である.}
%  \label{example_attackablity}
%\end{figure}
%\Nagata{beginfigure[tb]に変更}

In scenes where features are distributed across a wide range of horizontal angles, such as areas surrounded by buildings (Fig. \ref{example_smvs}, left), the frame-wise SMVS becomes small. In these cases, critical data for scan matching is dispersed, allowing correct matching from unaltered points outside the attack range, even if a portion is tampered with.
Conversely, in open areas (Fig. \ref{example_smvs}, right), important matching points tend to cluster, resulting in a higher frame-wise SMVS.
This frame-wise SMVS metric can be utilized to preemptively identify vulnerable regions along a robot's trajectory that are susceptible to spoofing attacks.
%\ref{example_attackablity}のように建造物に囲まれた場所など特徴量が豊富な場所ではAttackablityの値が小さくなる。このようなとき、点群マッチングにおける重要度が分散しているので攻撃範囲外の点群から正しく姿勢推定がなされる。一方、上図右のように開けた場所では特徴量に乏しく、Attackablityが高くなりやすい。ターゲットが走行するであろう経路のAttackablity分布を求めることでspoofingに対して脆弱な領域を発見することができる。攻撃者は脆弱な領域を通過するときに攻撃レーザが当たるような位置に攻撃装置を設置することで効果的にターゲットの姿勢推定プロセスを誤らせることができるようになる。

\subsection{Attack Location Selection}\label{sec:attack_loc_sel}

% The attacker aims to place a spoofer in the real world to attack autonomous vehicles and cause a displacement in their positions. However, there is currently no established theory for placing a spoofer that can maximize the threat. 

Finally, we propose a method to determine the effective spoofer placement by leveraging the frame-wise SMVS analysis. We first identify frames with high frame-wise SMVS scores and estimate the direction of critical point clusters that strongly influence scan matching in these frames. By drawing half-lines in these estimated directions and calculating their intersection points across all frames, we can pinpoint locations of objects crucial for scan matching. These intersections indicate optimal positions for spoofer placement, as tampering with these critical objects is likely to maximize the impact of the attack on position estimation.

% We assume that an object in the map on which pose estimation heavily depends can be found around the intersection. Therefore, it is believed that by placing the spoofer in a way that can tamper with the area where the intersections are concentrated, the position displacement due to spoofing can be maximized.
% In the process of calculating Attackability, the direction of points that pose estimation heavily depends on is identified. Additionally, vulnerable areas are identified based on the SVMS scores. Therefore, by extracting frames with high Attackability and determining the vector direction that contributes most to matching in those frames, we can estimate the direction of the important point cloud for matching. 

% A half-line is drawn in the estimated direction of the important point cloud, and the intersection with half-lines drawn from all other frames is calculated. 

\begin{comment}
For simplification in actual calculations, a sufficiently long line segment is used instead of a half-line. When the endpoint's position is the most contributing direction to matching $\theta$, and the start point is $(x_{\text{start}}, y_{\text{start}})$, the endpoint $(x_{\text{end}}, y_{\text{end}})$ is calculated as
\begin{align}
  x_{\text{end}} &= x_{\text{start}}+r \cos{\theta}, & y_{\text{end}} &= y_{\text{start}}+r \sin{\theta},
\end{align}
where $r$ is the length of the line segment and is considered sufficiently long.
\end{comment}

To determine spoofer placement from line intersections, we first remove outliers from intersection coordinates, considering points within $\pm2\sigma$ of the mean as potential critical object candidates. After outlier removal, we calculate the intersection points' bounding box and approximate the trajectory with a linear function.
We derive this approximation using least squares method on the top SMVS points, resulting in a function $f(x) = mx + n$. The perpendicular line passing through the bounding box center $(C_x, C_y)$ is represented as: 
\begin{align}
g(x)=-\frac{1}{m}x+(-mC_x + C_y),
\end{align}
This line serves as the candidate for spoofer placement. We use the bounding box center to ensure a wide central area is tampered with, allowing flexibility in placement due to real-world constraints. In practical applications, the spoofer should be placed 10-15m away from the trajectory intersection point, balancing spoofing duration and effectiveness.

 %Therefore, the approximate line can be represented as a function of the $x$ coordinate as $f(x)=mx+n$. The line that is perpendicular to the approximate line of the path and passes through the center coordinates of the bounding box has a slope of $-\frac{1}{m}$ and an intercept of $-mC_x + C_y$, where the center coordinates of the bounding box are $(x, y) = (C_x, C_y)$. Therefore, this line is given by
%\begin{align}
%g(x)=-\frac{1}{m}x+(-mC_x + C_y),
%\end{align}
%and this line becomes a candidate location for placing the spoofer. The reason for using the center point of the bounding box is that it is necessary to tamper with a wide central point. The reason for providing flexibility in the candidate location is that, in actual attacks, the spoofer cannot be placed just anywhere due to obstacles like walls. In practice, the spoofer should be placed 10 to 15 meters away from the intersection with the trajectory, as placing it too close to the target's path will shorten the spoofing duration, and placing it too far will make spoofing ineffective.
%\textbf{3. spoofer設置位置の決定} 

%攻撃者は実世界にspooferを設置し、自動運転車両を攻撃して危険水域の位置ずれを引き起こすことを目的としている。しかし、今現在は脅威を最大化できるようなspooferの設置理論は確立されていない。そこで、私たちは姿勢推定が強く依存している点群を狙って改ざんできる位置にspooferを設置する手法を提案する。

%Attackablityを求めるプロセスで姿勢推定が強く依存している点群の方向が分かっている。また、1方向への依存度が大きく脆弱な場所がAttackablityの大小によって評価されている。ゆえにAttackablityの高いフレームを抽出し、それらのフレームにおいて最もマッチングに寄与している方向のベクトルを求めることでマッチングにおける重要点群の方向を推定する。重要点群の推定方向に半直線を引き、他の全フレームから引かれた半直線と交点を求める。すると、交点に姿勢推定が強く依存する物体があると考えられる。したがって、交点が集中する場所を改ざんできるように攻撃者はspooferを設置すればspoofingによる位置ずれを最大化できると考えられる。

%なお、実際の計算では簡略化のため、半直線の代わりに十分に長い線分を用いている。終点の位置は最もマッチングに寄与している方向が$\theta$、始点が$(x_{\text{start}}, y_{\text{start}})$であるとき、終点$(x_{\text{end}}, y_{\text{end}})$は
%$$\begin{array}{ll}
%x_{\text{end}} = x_{\text{start}}+rt_{x} \\
%y_{\text{end}} = y_{\text{start}}+rt_{y}
%\end{array}
%t_x = \cos{\theta}, t_y = \sin{\theta}
%$$
%と求められる。なお、$r$は線分の長さで十分長いものとする。

%次に、得られた線分の交点からspooferの設置位置を決定する手法について説明する。位置決定の前処理として、交点座標から外れ値を取り除く。本研究では交点座標の平均値から±2$\sigma$に収まる点のみを重要物体候補として扱うこととした。外れ値を取り除いたのち、交点座標のbounding boxと移動経路の近似直線を求める。抽出されたAttackablity上位の点から最小二乗法を用いて係数を決定し、近似直線とした。ゆえに近似直線は$x$座標の関数$f(x)=mx+n$として表せる。そして、経路の近似直線と直交し、bounding boxの中心座標を通る直線は傾き$-\frac{1}{m}$、切片はbounding boxの中心座標を$(x, y) = (C_x, C_y)$としたとき、$-mC_x + C_y$である。ゆえにこの直線は$$g(x)=-\frac{1}{m}+(-mC_x + C_y)$$であり、直線上がspooferの設置位置候補となる。なお、bounding boxの中心点を用いたのは幅広い中央点を改ざんしなくてはならないからである。設置位置候補に自由度を設けたのは実際の攻撃では壁などにより、任意の場所にspooferを設置できないためである。実際の運用ではターゲットの移動コースとspooferが近すぎるとspoofingできる時間が短く、遠すぎるとspoofingすることができないので現実的にはTrajectoryとの交点から10から15mほど離して設置するとよいだろう。

\section{Evaluation}

We first evaluate whether our SMVS can effectively prioritize locations based on their vulnerability level, and then evaluate 
the actual attack effectiveness of the SLAMSpoof on the three major LiDAR-based localization methods in autonomous driving scenarios not only with a simulator but also with physical-world experiments.

% In this section, we evaluate the validity of our SMVS to effectively prioritize attack locations potentially vulnerable to attack and 

\subsection{SMVS Validity Evaluation}
\vspace{-0.09in}

\noindent \textbf{Experimental Settings:}
To verify if our SMVS can effectively prioritize attack locations based on their vulnerability, we simulated 16 different attack device placement patterns along the course similar to Fig.~\ref{header_fig}, each designed to yield various SMVS. For each placement position, we ran 10 simulations to measure the trajectory deviation caused by LiDAR spoofing. Here, we evaluate via RMSE of the Absolute Pose Error (APE)~\cite{zhang2018tutorial} and the maximum value of Relative Pose Error (RPE)~\cite{zhang2018tutorial}.

%\Nagata{APE (Absolute Pose Error) is the average error between the true pose and the estimated pose at each time step. It is an effective metric for evaluating the overall accuracy of the estimated trajectory. On the other hand, RPE (Relative Pose Error) extracts a portion of the trajectory and compares the relative error between frames by comparing the true trajectory with the estimated one.}

In our experiments, we modeled three patterns of LiDAR spoofing attacks based on the results in~\cite{sato2024lidar}.
The High-Frequency Removal (HFR) Attack, which uses high-frequency pulses to alter existing point clouds, replaces points within an 80$\tcdegree$ range with salt-and-pepper noise. Since such noise is easily removed by preprocessing pipelines, we also simulated a version where points within this angular range were simply removed.
For the injection attack, we simulated the appearance of a wall at a fixed distance, spanning 80$\tcdegree$ degrees, as in Fig.\ref{example_spoofing}.

To verify the generality of our proposed method, we conducted simulations on three major localization techniques: KISS-ICP~\cite{vizzo2023ral}, A-LOAM~\cite{zhang2014loam}, hdl-localization~\cite{koide2019portable}. Since hdl-localization requires a prior environment map, we first ran A-LOAM without any attacks to generate the necessary map data, which was then used during the execution.
%また、複数手法\cite{vizzo2023ral, zhang2014loam, koide2019portable}に対してシミュレーションを行い提案手法の一般性を検証した。本研究ではKISS-ICP \cite{vizzo2023ral}、A-LOAM\cite{zhang2014loam}、hdl-localization \cite{koide2019portable}に対して攻撃を行った。hdl-localizationは事前地図を必要としているので攻撃を行わずにA-LOAMを実行して地図データを取得し、自己位置推定に用いた。

We assumed that localization methods considering local plane shapes behave similarly. The importance of each point was calculated based on point cloud matching using GICP~\cite{segal2009generalized}. To calculate SMVS, we divided the point cloud into 72 small regions, thus each $R_k$ has an angular range of 5\textdegree.
We model that the removal attack's range is approximately $\pm40\tcdegree$ from the central angle, corresponding to 8 adjacent regions from the center. Consequently, we set $d_{th} = 8$ to impose penalties outside this attack range.

%spooferの設置位置はコース上16か所に配置し、それぞれの設置位置について10回ずつシミュレーションを行った。

%まず、HFR攻撃のシミュレーションを行った。HFRでは正常な点群データを上書きする際に高周波パルスを照射する。そのため、図\ref{example_spoofing}上図のようにランダムなノイズが注入される。そこで単純に点群を消失させた場合とノイズが入り込んだ場合の2パターンについてシミュレーションを行った。また、偽物体注入攻撃では図\ref{example_spoofing}の下図のように攻撃者は照射するレーザの遅延量を制御して存在しない物体を出現させることができる。本研究ではToFが一定になるように遅延量を調整し、LiDARからの一定距離離れた場所に偽の壁が出現するようにした。

%We assume that the pose estimation method, which considers local plane shapes, behaves similarly, and the importance of each point is calculated based on point cloud matching using GICP~\cite{segal2009generalized}. For the calculation of Attackability, the point cloud was divided into 72 small regions as shown in 3-D. Thus, $R_k (k=0, 5, \cdots 355(\mathrm{deg})) $ is defined. Moreover, since the HFR attack range is approximately 80°, the realistically attackable range is within 40° from the central angle. Because the regions up to eight away from the central angle can be sufficiently tampered with, $d_{th} = 8$ was set to impose a penalty outside the attack range. Therefore, the function $f(d)$ is defined as $f(d) = -d + 8$.

%Regarding the performance of the spoofer, it is assumed that tracking is possible within a 150° forward range and a distance of 30m from the target. However, the occlusion of the laser by obstacles is not considered. The evaluation metrics used were the RMSE of APE and the maximum value of RPE.
%The map used for validation is shown in \ref{fig5}, which is a point cloud obtained by Livox Mid360.
%局所的平面形状を考慮した姿勢推定手法は似た挙動をすると仮定し、各点の重要度はGICP\cite{segal2009generalized}による点群マッチングを元に算出した。Attackablityの計算のため、3-Dのようにして点群を72個の小領域に分割した。ゆえ$R_k
%(k=0, 5, \cdots 355(\mathrm{deg})) $である。また、HFRの攻撃範囲はおよそ80°なので中心角から40°の範囲が現実的に攻撃可能な範囲である。中心角から8つ離れた領域までは十分改ざん可能であるから、攻撃範囲外にペナルティを与えるため$d_{th} = 8$ とした。したがって、関数$f(d)$は
%$f(d) = -d + 8$である。

%また、spooferの性能に関してはターゲットとの距離30m、前方150°の範囲内で追尾が可能であると想定した。ただし、障害物によるレーザの遮蔽は考慮していない。なお、評価指標はAPEのRMSEおよびRPEの最大値を用いた。

%検証を行ったmapは\ref{fig5}であり、Livox mid360によって取得された点群である。

%\begin{figure}[tb]
%  \centering
%  \includegraphics[scale=0.28]{sim_map_attackablity.pdf}
%  \caption{(左)シミュレーションに用いた地図. コース左側でAtackablityが高くなり、右側で低くなることを見込んだ.(右)シミュレーション環境におけるAttackablityの分布
%  \ken{Fig.6とFig.7は何を言いたいかよくわからないかも。なくてもOK?}
%  }
  %\Nagata{実験1はAttackablityと位置ずれの関係を確認するための実験です。そのため、Attackablityが低い場所にもspooferを設置して実験しています。設置位置を確定させて大きな位置ずれを与えられるかは実験2で示します。}}
%  \label{fig5}
%\end{figure}


\begin{table*}[tb]
\centering
\caption{SMVS-APE relation. $\mathcal{S}$ means SMVS value.}
\vspace{-0.1in}
%\koide{A だと分かりづらいので $\mathcal{A} \mathfrak{A} \mathbb{A}$ のどれかにしましょう} }
\label{tab_APE}
\begin{tabular}{c|cccccc}
\toprule
\multicolumn{7}{c}{KISS-ICP\cite{vizzo2023ral}} \\ \midrule
SMVS                                             & HFR (no noise) [m]    & HFR (noise) [m]       & Injection [m]        & HFR (no noise) [deg]  & HFR (noise) [deg]     & Injection [deg]      \\ \midrule
-10000 \textless $\mathcal{S}$ \textless -6000 & 0.164±0.057          & 0.191±0.065          & 0.623±0.182          & 0.755±0.815          & 0.905±0.771          & 1.801±0.617          \\
-6000 \textless $\mathcal{S}$ \textless -3000  & 0.179±0.075          & 0.212±0.135          & 0.946±0.069          & 1.223±1.743          & 1.495±1.547          & 4.609±1.945          \\
-3000 \textless $\mathcal{S}$ \textless -1000  & 0.167±0.057          & 0.264±0.074          & 7.732±0.947          & 0.911±0.964          & 1.996±1.321          & 13.04±3.087          \\
-1000 \textless $\mathcal{S}$                 & \textbf{1.323±1.492} & \textbf{8.800±3.963} & \textbf{16.74±2.071} & \textbf{4.191±3.642} & \textbf{39.28±38.59} & \textbf{30.80±10.57} \\ \midrule \addlinespace \midrule
\multicolumn{7}{c}{A-LOAM\cite{zhang2014loam}} \\ \midrule
SMVS                                             & HFR( no noise ) [m]    & HFR (noise)  [m]       & injection [m]        & HFR( no noise ) [deg]  & HFR (noise)  [deg]     & injection [deg]      \\ \midrule
-10000 \textless $\mathcal{S}$ \textless -6000 & 0.155±0.095          & 0.152±0.058          & 0.124±0.069          & 1.206±0.552          & 1.247±0.779          & 1.021±0.631          \\
-6000 \textless $\mathcal{S}$ \textless -3000  & 0.130±0.037          & 0.154±0.026          & 0.089±0.014          & 1.525±0.321          & 3.635±0.997          & 1.229±0.050          \\
-3000 \textless $\mathcal{S}$ \textless -1000  & 0.344±0.072          & 0.319±0.083          & 0.357±0.080          & 4.830±0.775          & 6.542±2.507          & 4.551±0.666          \\
-1000 \textless $\mathcal{S}$                  & \textbf{3.135±4.545} & \textbf{2.608±3.206} & \textbf{4.094±3.595} & \textbf{55.08±53.45} & \textbf{53.86±48.93} & \textbf{48.91±41.97} \\ \midrule \addlinespace \midrule
\multicolumn{7}{c}{hdl\_localization\cite{koide2019portable} } \\ \midrule 
SMVS                                             & HFR( no noise ) [m]    & HFR (noise)  [m]       & injection [m]        & HFR( no noise ) [deg]  & HFR (noise)  [deg]     & injection [deg]      \\ \midrule
-10000 \textless $\mathcal{S}$ \textless -6000 & 0.386±0.221          & 0.282±0.114          & 0.328±0.159          & 2.407±0.932          & 1.807±0.594          & 2.188±1.173          \\
-6000 \textless $\mathcal{S}$ \textless -3000  & 0.482±0.165          & 0.305±0.107          & 0.394±0.017          & 2.738±1.043          & 1.794±0.514          & 1.962±0.501          \\
-3000 \textless $\mathcal{S}$ \textless -1000  & 0.296±0.136          & 0.311±0.160          & 0.346±0.115          & 1.768±0.588          & 2.013±0.875          & 2.054±0.588          \\
-1000 \textless $\mathcal{S}$                  & \textbf{1.848±3.805} & \textbf{3.761±5.985} & \textbf{2.954±5.132} & \textbf{22.38±51.23} & \textbf{22.18±42.82} & \textbf{31.86±58.50} \\ \bottomrule
\end{tabular}
\end{table*}


%In the simulation, the spoofer was placed at 16 different locations along the course where the target passes, ensuring that it did not physically obstruct the path. The placement locations were chosen to avoid clustering in one area, allowing for variation in Attackability during the attacks.
%シミュレーションではターゲットが通過するコースの脇、通行を物理的に妨害することないようにspooferを16点場所を変えて設置し、評価を行った。設置位置は1か所に偏らないようにし、攻撃時のAttackablityがばらつくようにした。

\noindent \textbf{Results:}
Table \ref{tab_APE} summarizes the relationship between the SMVS and the resulting position deviation. For all localization algorithms and attack methods, the APE remained low when the SMVS was below -1000. However, when the score exceeded -1000, the APE increased dramatically in both translational and rotational directions. This suggests that the SMVS is an effective indicator of SLAM vulnerability.

For the KISS-ICP and hdl-localization, retaining salt-and-pepper noise during HFR attacks yielded greater attack effectiveness. This indicates that large-scale salt-and-pepper noise can confuse localization algorithms, suggesting that incorporating sufficient denoising in preprocessing steps enhance security.
Across all algorithms, injection attacks had higher impacts than removal attacks, revealing that localization systems are more vulnerable to injection attacks than removal attacks. We attribute this to the fact that partial disappearance of geometric features often occur due to occlusion and is covered by SLAM's robustness. In contrast, the injection of structured point clouds is more challenging to mitigate.

%表\ref{tab_APE}は並進方向および回転方向のAPEとAttackablityの関係をまとめた表である。どのアルゴリズム、攻撃手法であってもAttackablityが-1000よりも小さいときにはAPEは抑えられていたが、-1000よりも大きいときには並進方向、回転方向ともに急激にAPEが大きくなった。HFRに注目したとき、KISS-ICPとhdl\_localizationではノイズの影響を加味した場合のほうが推定される軌跡に大きな影響を与えられた。また、偽装壁を注入する攻撃ではノイズなしHFRと比較するとどのアルゴリズムでも壁注入攻撃のほうが大きな影響が与えられた。したがって、SLAMは消失攻撃よりも注入攻撃に対して脆弱であった。これは幾何特徴量の消失は遮蔽により現実的に起こりうるので実装上で対策されていたためと考えられる。

%結果、Attackablityが大きくなる場所で攻撃を行うとSLAMの能力を超えた誤差の蓄積が生じAPEが増大した。したがって提案した指標Attackablityはspoofingに対するSLAMの脆弱性を表す指標として有効であると言える。

\subsection{Evaluation of Spoofer Placement Optimization}

\noindent \textbf{Experimental Settings:}
To evaluate the effectiveness of our spoofer placement optimization method, we conducted simulations comparing three spoofer deployment strategies for vehicles traveling on a course similar to Fig.\ref{header_fig}. The strategies compared were: randomly placing a single spoofer 
%\ken{N=100}
200 times within the range of X,Y=90m and 50m 
%\ken{X,Y=90m and 50m} 
%\Nagata{このX, Yは乱数の範囲ですか?}\ken{そうです}
, targeted placement at the top-10 highest SMVS locations along the trajectory, and using the method proposed in \S \ref{sec:attack_loc_sel}. 

%\begin{figure}[H]
%  \centering
%  \includegraphics[scale=0.3]{Fig13.pdf}
%  \caption{Ⅲ-B 3.の手法によってspooferの設置位置候補を計算した場合の結果. 
%  \ken{Fig.6とFig.7は何を言いたいかよくわからないかも。なくてもOK?}
%  }
%  \label{spooferlocation}
%\end{figure}

\noindent \textbf{Results:}
As shown in Table \ref{decided_trans}, our proposed method resulted in greater positioning errors compared to the other methods. When compared to the random placement strategy, our method produced 12.6 times larger errors for HFR-based attacks and 15.4 times larger errors for injection attacks. We believe this significant increase in positioning errors is due to our proposed spoofer placement technique, which quantifies vulnerabilities in scan matching using SMVS and determines optimal placement locations to maximize the consequences.\vspace{-0.07in}
%表\ref{decided_trans}に示したように、提案手法は(spoofing手法に関わらず)比較手法よりも大きな位置ずれを与えられた。
%\Nagata{完全ランダムの場合と比較するとHFRによる攻撃の場合12.6倍、injectionの場合XX倍誤差が大きくなった。また、最大Attackablityの場合と比較するとHFRの場合37.7倍、injectionの場合XX倍大きな位置推定誤差を与えることができた。} \ken{なぜ優れていたか考察を入れる。}

%\Nagata{優れていた理由:我々が提案するspooferの設置手法では自己位置推定にクリティカルな影響を与える物体を探索し、それらの点群を最大限改ざんできるような設置位置を与えた。SLAMSpoof scoreは単フレームでの脆弱性指標なので連続してspoofingした場合には推定誤差を最大化できなかったと考えられる。injection:randomでも位置ずれが大きい.実際には広範囲に壁を注入するのが難しく、制約がより厳しいためこのような設置位置最適化方策が必要になってくると考えられる}

%\Nagata{Eng:In the proposed spoofer deciding method , we search for objects that critically impact self-position estimation and determine the placement that maximizes the modification of their point clouds. Since the SLAMSpoof score is a vulnerability indicator for a single frame, it is considered that the estimation error could not be maximized when spoofing continuously.}
%The result showed that there were bushes in the area where the intersection points were concentrated. Since the candidate spoofer placement locations are on the blue line shown in \ref{spooferlocation}, a spoofing simulation was conducted by placing the spoofer at $(x, y) = (6, 9.38)$ to include the Trajectory in the attack range as much as possible.

%To demonstrate the effectiveness of the proposed spoofer placement method, a comparison was made with other placement methods. First, a point on the Trajectory was randomly selected, and the spoofer was placed on a line perpendicular to the approximate straight line of the Trajectory passing through the selected point, followed by 100 attacks. Second, the spoofer was placed on a line perpendicular to the approximate straight line of the Trajectory passing through the point where Attackability is maximized, and 10 attacks were conducted. Finally, 10 attacks were performed from the location determined by the proposed method, and the estimated movement trajectories were recorded.

%For the spoofing conditions, the attack methods used were HFR and wall injection, as in Experiment 1. The self-position estimation was performed using KISS-ICP and evaluated by the RMSE of the translational APE. The map data and Attackability distribution used were the same as those in Experiment 1.
%結果、交点が集中する場所には茂みが存在していた。また、spooferの設置位置候補は\ref{spooferlocation}に示した青線上であるから、可能な限りTrajectoryが攻撃範囲に含まれるように$(x, y) = (6, 9.38)$にspooferを設置してspoofingシミュレーションを行った。

%提案するspoofer設置位置決定手法が有効であるか示すため、他の設置位置決定手法との比較を行った。第一に、ランダムにTrajectoryの1点を選択し、選択した点を通りTrajectoryの近似直線と直交する直線上にspooferを設置して100回攻撃を行った。第二に、Attackablityが最大となる点を通りTrajectoryの近似直線と直交する直線上にspooferを設置し、10回攻撃を行った。提案手法により決定した設置位置から10回攻撃を行い、推定された移動軌跡を記録した。

%spoofingの条件として、攻撃方法は実験1でも用いたHFRと壁注入を用いた。また、KISS-ICPにより自己位置推定を行い並進方向APEのRMSEによって評価した。マップデータおよびAttackablityの分布は実験1と同様のものを用いた。

\begin{table}[tb]
\centering
\setlength{\tabcolsep}{1.5pt}
\caption{Comparison of estimated position error by spoofer position determination methods (Metrics:APE, Unit:m)}
\label{decided_trans}
\begin{tabular}{c|ccc} % 列の数に応じて修正
\toprule
               & Random [m]    & Top-10 SMVS [m] & \textbf{SLAMSpoof (ours)} [m] \\ \midrule
HFR (noise)    & 1.122±3.029         & 0.375±0.190                    & \textbf{14.13±3.684}      \\
Injection      & 0.714±1.759         & 0.401±0.164                    & \textbf{11.00±3.684}       \\ \bottomrule
\end{tabular}
\end{table}

\subsection{Physical-World Evaluation}
\vspace{-0.05in}
Finally, we evaluate the effectivness of the SLAMSpoof on a ground vehicle with LiDAR.
%LiDAR搭載ロボットを攻撃し、シミュレーション上で示された特性が実世界で再現できるか検証する。この実験により、現実世界でのSLAMシステムの脆弱性を明らかにする。

\noindent \textbf{Experimental Settings:}
We equipped the target robot with a VLP-32c LiDAR sensor as shown in Fig.\ref{physical exp}. To vary the SMVS within the course, the maximum range was limited to 50 meters. We utilized the attack device proposed in \cite{suzukiwip}, which is capable of pursuing and attacking moving objects outdoors at distances of up to 50
%\ken{50}
meters. This setup allowed us to test the effectiveness of our method against a mobile target in a realistic outdoor environment.
%For the experiment, the target vehicle was equipped with a VLP-32c LiDAR. This LiDAR is capable of performing arbitrary shape injection attacks, allowing for the same types of attacks as in the simulation. The attack device used was the one proposed by Sato~\cite{sato2024lidar}.
%実験の条件として、ターゲットとなる移動体にはVLP-32cというLiDARを搭載した。
%このLiDARは任意形状のinjection攻撃が可能であり、シミュレーションと同様の攻撃が可能である。また、攻撃装置は佐藤\cite{sato2024lidar}が提案したものを用いた。
The attack was conducted as outlined in Sec.III. We employed the same localization algorithm and attack methodology as in our previous simulation experiments.
To evaluate the induced position displacements, we calculated the RMSE between the expected path under attack conditions and the nearest points on the benign path. Due to the varying coordinate systems used by different methods, we converted the predicted trajectories of other techniques to the coordinate system of hdl-localization for a standardized evaluation.

%攻撃は\ref{proposed_method_concept}に示した手順で行った。攻撃位置の探索に用いる地図データもターゲットと同様にVLP32cで取得し、地図の作成にはA-LOAM~\cite{zhang2014loam}を使用した。ただし、IMUやGNSSといったセンサとのフュージョンは行っていない。

%攻撃手法はHFR~\cite{sato2024lidar}および虚偽壁点群の注入である。また、ターゲットとする姿勢推定アルゴリズムはシミュレーション実験と同様にKISS-ICP、A-LOAM、hdl\_localizationの3種類とした。

\begin{figure}[tb]
  \centering
  \includegraphics[trim=0mm 0mm 0mm 40mm, clip, scale=0.3]{sashikae6.pdf}
  \vspace{-0.5in}
  \caption{
  (a) Outdoor experiment setup: The spoofer targets the victim's vehicle $\leq$50 meters away; (b) Spoofer configurations: laser beam with the same wavelength as the target LiDAR is generated by the diode and collimated by the lens; (c) Tracker device~\cite{suzukiwip}: it can detect the victim LiDAR with an IR camera and track it with the rotating table on the servo motor; (d) The victim ground vehicle with LiDAR.
  % (a) Outdoor experiment setup: The spoofer was placed to track and attack the LiDAR attached to the moving victim. (b) Structure of the spoofer. Light with the same wavelength as the LiDAR is generated by a laser diode, collimated by a lens, and emitted as a parallel beam. (c) Tracker by Suzuki \cite{suzukiwip}. The LiDAR is detected by a camera, and the spoofer's direction is adjusted and tracked using a motor. (d) The victim used in the experiment. The LiDAR was mounted on an electric wheelchair, which was manually operated.
  }%\ken{屋外のセットアップの写真にする。また右は鈴木のspooferと可動ユニットの写真が必要にする。LiDARはいらない。}}
  \label{physical exp}
\end{figure}
%\begin{figure}[tb]
%  \centering
%  \includegraphics[scale=0.26]{Fig22.pdf}
%  \caption{(a)屋外実験の様子.spooferを設置し,移動するvictimに取り付けられたLiDARを追尾して攻撃を行った.(b)spooferの構造.LiDARと同じ波長の光をレーザダイオードで発生させ,レンズでコリメートして平行光として出射する.(c)鈴木\cite{suzukiwip}のTracker.カメラでLiDARを検出し,モータでspooferの方向を変えて追尾する.(d)実験に用いたvictim.LiDARを電動車いすに取り付け,手動で操縦した.}%\ken{屋外のセットアップの写真にする。また右は鈴木のspooferと可動ユニットの写真が必要にする。LiDARはいらない。}}
%  \label{fig7}
%\end{figure}
%The attack was carried out following the procedure shown in \ref{proposed_method_concept}. The map data used for exploring attack locations was also obtained using the VLP-32c, similar to the target, and A-LOAM~\cite{zhang2014loam} was used for map creation. However, sensor fusion with IMU or GNSS was not performed.

%The attack methods included HFR~\cite{sato2024lidar} and the injection of false wall point clouds. The target pose estimation algorithms were the same as those in the simulation experiments: KISS-ICP, A-LOAM, and hdl\_localization.
%攻撃は\ref{proposed_method_concept}に示した手順で行った。攻撃位置の探索に用いる地図データもターゲットと同様にVLP32cで取得し、地図の作成にはA-LOAM~\cite{zhang2014loam}を使用した。ただし、IMUやGNSSといったセンサとのフュージョンは行っていない。

%攻撃手法はHFR~\cite{sato2024lidar}および虚偽壁点群の注入である。また、ターゲットとする姿勢推定アルゴリズムはシミュレーション実験と同様にKISS-ICP、A-LOAM、hdl\_localizationの3種類とした。

\noindent \textbf{Results:}
As shown in Fig.\ref{header_fig}, our experiments resulted in critical position displacements across all localization methods adopted in this study. Table \ref{e2e_result} presents the quantitative results, demonstrating that regardless of the attack technique or SLAM method employed, we were able to induce estimation errors on the order of several meters. This finding conclusively proves that LiDAR spoofing can significantly impact estimated positions in real-world scenarios.

While simulation results favored injection attacks, real-world experiments proved removal attacks more effective. This discrepancy likely stems from synchronization challenges in the receiving system, limiting the amount of injectable point cloud data. In practice, only 8-10 layers could be injected, despite the target LiDAR performing 32-layer vertical scans.

%It was possible to inject HFR and walls in the real world as well. However, since the distance to the target was closer than in the spoofer studies and the angle changes were larger, the spoofer could not keep up, resulting in cases where the point cloud was not tampered with even within the expected attack range. As a result, critical position shifts occurred in all the SLAM methods adopted in this study, as shown in Figure \ref{header_fig}(c).

%The induced position shift was evaluated by measuring the RMSE of the distance between the predicted trajectory under attack and the nearest benign point. Additionally, since the coordinate systems differ among SLAM methods, the predicted trajectories of other methods were transformed into the coordinate system of hdl\_localization. The results of the evaluation are shown in Table \ref{e2e_result}. Estimation errors on the order of several meters were observed, regardless of the attack method or SLAM method. This means that LiDAR spoofing can significantly affect the estimated position in the real world as well.
%実世界においてもHFRおよび壁の注入は可能であった。しかし、ターゲットとの距離がspooferの研究よりも近く、角度変化が大きかったため追従が間に合わなかったので想定される攻撃範囲内でも点群が改ざんされないことがあった。結果として、図\ref{header_fig}(c)のように本研究で採用したすべてのSLAM手法においてクリティカルな位置ずれが生じた。

%生じた位置ずれに関し、攻撃下での予想経路とbenignの最近傍点の距離のRMSEを測ることで評価した。また、SLAM手法によって座標系が異なるのでhdl\_localizationの座標系に他手法の予測軌跡を変換した。評価した結果は表\ref{e2e_result}に示した。攻撃手法、SLAM手法に関わらず数mオーダーの推定誤差を与えられた。つまり、実世界でもLiDAR spoofingによって推定位置に重大な影響を及ぼすことができる。

\begin{table}[tb]
\centering
\caption{Physical-world evaluation results}
\label{e2e_result}
\begin{tabular}{c|ccc} % 列の数に応じて修正
\toprule
               & KISS-ICP\cite{vizzo2023ral}    & A-LOAM\cite{zhang2014loam} & hdl\_locatization\cite{koide2019portable} \\ \midrule
HFR     & 8.379 m       & 6.127 m                   & 4.217 m     \\
Injection      & 4.781 m & 4.075 m           & 2.777 m      \\ \bottomrule
\end{tabular}
\end{table}

\section{DISCUSSIONS}
\vspace{-0.05in}
\noindent \textbf{Countermeasures:}
Our study on SLAMSpoof attacks highlights the need for robust defense mechanisms in LiDAR-based localization systems. Potential countermeasures include LiDAR systems with pulse signature technology and sensor fusion with IMUs. Pulse signatures have shown promise against HFR and injection attacks~\cite{sato2024lidar}. Thus, implementing such features in the LiDAR could also provide effective defense against SLAMSpoof. IMU fusion, as implemented in algorithms like LIO-SAM\cite{shan2020lio}, may offer resilience against spoofing by maintaining accurate positional data even when LiDAR is compromised.

\noindent \textbf{Social Impact:} We strongly recommend that the robot developer should not solely rely on LiDAR-based localization in the safety and security critical scenarios although LiDAR-based localization is so far the only choice to achieve centimeter-level accuracy required for autonomous driving. Our SMVS can be used to assess the vulnerability of the operational sites.  Robot developers may consider additional sensors for localization (e.g., IMU and GNSS) If SMVS is so high (e.g., $\geq$-1,000) in their application sites while it may need additional costs.\vspace{-0.1in}

%ken: 繰り返しなので省略
%\noindent \textbf{Future Work:}
%Future research could focus on evaluating SLAMSpoof's impact on multi-sensor systems, particularly those integrating IMU data. Such investigations would be valuable in assessing the effectiveness of these potential countermeasures and informing the development of more secure autonomous navigation systems.

\section{CONCLUSION}
\vspace{-0.05in}
In this study, we developed SLAMSpoof, the first practical LiDAR spoofing attack on localization systems for self-driving vehicles. By manipulating geometric features in point cloud data, our method induces severe localization errors that can force vehicles off-road or bypass critical traffic information. We introduced the Scan Matching Vulnerability Score (SMVS), a novel metric that quantifies vulnerabilities in scan matching algorithms and demonstrates strong correlation with attack outcomes across diverse localization methods.
Our real-world experiments validate the attack effectiveness, resulting in dangerous localization errors of $\geq$4.2 m—exceeding typical lane widths—across three popular LiDAR-based SLAM algorithms. 
Our work provides valuable insights for developing more resilient localization technologies, essential for ensuring the safety and reliability of autonomous vehicles in real-world environments.

%This study revealed the threats posed to SLAM in the real world. To demonstrate threats to more realistic SLAM systems, we are considering attacks on SLAM systems equipped with IMUs.
%\Nagata{文章を短くした}
%In this study, we propose an optimized LiDAR spoofing technique to target pose estimation in SLAM systems. Our experimental results demonstrate that the proposed attacks can induce significant positional deviations on the order of several meters, potentially leading to catastrophic failures of the robot. To further investigate the robustness of SLAM systems, we plan to extend our research to systems equipped with IMUs to evaluate the impact of sensor fusion on the proposed attacks.
%本研究ではLiDAR spoofing技術をSLAMによる姿勢推定に対して最適化する手法を確立した。結果、攻撃によって3つの手法で数mオーダーの位置ずれが生じロボットは致命的な影響を受けた。今後の方針として、この研究でLiDAR onlyのSLAMに対して十分大きなエラーを引き起こす手法が明らかとなった。そこで、センサフュージョンによる影響を評価するためにIMUを搭載したSLAMシステムに対する攻撃を検討している。

\bibliographystyle{IEEEtran}
\bibliography{citation.bib}
\end{document}
