\section{Related work and Background}
%\Nagata{佐藤さんが追加した引用を.bibに追加}
\subsection{LiDAR Spoofing Attacks} \label{sec:lidar_spoofing}
LiDAR is fundamentally vulnerable to other malicious laser sources as the nature of the LiDAR mechanism that senses the laser reflection. This type of attacks is known as LiDAR spoofing attacks____, where Fig.\ref{example_spoofing} shows the point cloud under attack. Based on the attack effects, there are two major types of LiDAR spoofing attacks: injection and removal attacks. For injection attacks, once initial works____ demonstrated that their attacks can inject a few hundred points on the point cloud, the state-of-the-art attack____ can compromise 99\% of points within 83$^\circ$ ($\sim$7k points). 
On the other hand, removal attacks____ try to erase the legitimate points from the detected point clouds. The state-of-the-art removal attack, HFR attack____, can remove almost all points within 80$^\circ$ range. 

\begin{figure}[tb]
  \centering
  \includegraphics[trim=25mm 40mm 0mm 30mm,scale=0.36]{Fig19.pdf}
  \caption{The effects of spoofing attacks on LiDAR. In an injection attack (top-right), false point cloud data representing a non-existent wall is inserted into the LiDAR scan. In a removal attack (bottom-right), injected noise obscures the pedestrians' point cloud, effectively erasing them from the LiDAR scan.}
  \label{example_spoofing}
\end{figure}

%LiDARは自律走行において最も重要なセンサの1つである。自律走行分野では光を照射し、その反射光を受光するまでの時間から物体との距離を測定するdirect Time of Flight (dToF)方式のLiDARが広く用いられている。この測定により、周囲の環境が3次元的な点群として得られ、この点群データは障害物検知による衝突回避やSLAMによる地図生成に利用される。しかしdToF LiDARはその測定原理上、悪意あるレーザ光によって点群を操作される脆弱性を持つ。

%この脆弱性を突いた攻撃として、先行研究では大きく2つの攻撃が提案されている。連続的なレーザパルス波形を照射することで偽の物体を注入する攻撃____と、ジャミングによって広範囲の点群を消失させる攻撃____である。
%提案されているLiDAR spoofing攻撃の中で、最も広範囲かつ多様なLiDARセンサにおいて有効性を確認しているのが佐藤ら____の手法である。____では高周波レーザパルスをLiDARに照射し、測距データを偽のノイズで上書きする手法HFRを提案し、障害物の点群を消失させ、衝突事故の発生を実世界で実証している。加えて、____では偽の車や歩行者といった任意形状の偽物体注入攻撃も実証している。
%この論文でも佐藤ら____が提案したセットアップを用いて注入、消失攻撃を行う。

\subsection{LiDAR-based Localization} \label{sec:lidar_slam}
Localization is a methodology to identify the ego location on a map with sensor inputs. While the multi-sensor fusion (e.g., LiDAR-IMU fusion) approach has been gaining popularity owing to its accuracy and robustness____, the localization solely with LiDAR is still popular due to its simplicity and applicability____. In this study, we focus on the three popular LiDAR-based localization algorithms, LiDAR-based localization methods, namely A-LOAM____, KISS-ICP____, and hdl\_localization____, covering three major point distance calculations (plane and edge feature matching____, ICP____, NDT____) and two major scan matching schemes (online local map and using prior-map).
This study is the first to explore the vulnerability of these major LiDAR-based localization methods against LiDAR spoofing attacks in driving vehicles.
% Localization methods based on multi-sensor fusion (e.g., LiDAR-IMU fusion) have been gaining popularity owing to their accuracy and robustness ____. However, methods that rely solely on LiDAR still offer advantages in terms of system simplicity and applicability, and they have been actively studied ____ and used in many robotic applications. As these LiDAR-only methods share the basic principles of point cloud registration with the aforementioned sensor-fusion-based methods, in this work, we focus on analyzing the vulnerability of three popular LiDAR-based localization methods, namely A-LOAM ____, KISS-ICP ____, and hdl\_localization ____. These selected algorithms employ distinct point distance metrics (plane and edge feature matching, ICP, NDT) and scan matching schemes (creating a local map vs. using a prior-map). We consider demonstrating the vulnerability of these methods highlights the generality of our proposed attack method and provides insights into the vulnerability of a wide range of LiDAR-based localization systems.

% LiDAR-based localization methods can be categorized by whether using a prior map or not (i.e., SLAM). 
% スペースもないので各手法の説明は外しました
% A-LOAM is a re-implementation of LOAM ____, which is one of the most influential LiDAR SLAM algorithms based on the matching of edge and plane feature points. KISS-ICP is the most popular algorithm among recent LiDAR-based SLAM methods that is based on point-to-point ICP combined with adaptive thresholding and robust estimation. hdl\_localization ____ is a prior-map-based localization algorithm that uses NDT scan matching to estimate the sensor pose on the map. 

\subsection{Prior Attack Attempts on LiDAR-based Localization} \label{sec:prior_attack}
As also discussed in____, no successful LiDAR spoofing attack was demonstrated on LiDAR-based SLAM for driving autonomous vehicles. Yoshida et al.____ showed that LiDAR-based SLAM could be vulnerable to a malicious point injection. However, this work only evaluates their attack on 2D-LiDAR, which cannot support autonomous driving, in a lab-level scenario. %Furthermore, the injection attack does not work on recent lines of LiDAR as discussed in~\S\ref{sec:lidar_spoofing}.
Fukunaga et al.____ demonstrate random LiDAR spoofing attacks to compromise the prior environment map generation. This work does not target to attack the online localization systems in the driving vehicle and admits that their random attack cannot cause major attack impacts on the x (longitudinal) and y (lateral) axes. %which is a 5-meter lateral deviation per 160-meter longitudinal driving. While attack on such a long distance could be possible in the map generation process, but should not be easy to keep attacking fastly driving vehicles. At least, no prior works have demonstrated such attacks.

%Previous research on spoofing attacks against LiDAR has mainly focused on deceiving object detectors, with only a few studies evaluating the impact of spoofing attacks on SLAM systems. Fukunaga et al.____ evaluated the impact on SLAM systems by conducting a spoofing attack that causes the disappearance of point clouds____. However, this study attacked during the pre-map creation phase, causing position deviations by performing route planning with the contaminated map. Since pre-map creation is infrequent and involves human inspection, this approach does not represent a realistic attack on autonomous driving systems. Additionally, the reported results are primarily based on simulations, and attacks on long-distance autonomous driving systems in real-world scenarios have not been demonstrated.
%従来のLiDARへのspoofing攻撃研究は主に物体検出器を騙すことに主眼が置かれており、SLAMに関してspoofing攻撃の影響を評価した研究は少ない。福永ら____は点群を消失させるspoofing攻撃____を行ってSLAMシステムへの影響を評価している。しかしながら、この研究では事前地図作成中に攻撃し、汚染された地図を用いて経路計画を行うことで位置ずれを生じさせている。事前地図作成の頻度は低く、人によるチェックも入るため、現実的な自動運転システムへの攻撃ではない。また報告された結果はシミュレーションが主であり、長距離走行する実世界での自動運転システムに対する攻撃は実証されていない。

\subsection{Threat Model and Attack Goal}
We assume that the attacker can place a LiDAR spoofing device on a roadside where the victim robot will pass through, which is known as ``Spoofer placed in environment'' threat model____. We assume that the victim uses a 3D LiDAR such as Velodyne VLP-16____ and Livox Horizon____ and solely relies on LiDAR-based localization without sensor fusion with other sensors such as IMU and GNSS.  
We assume robots that follow a fixed route (e.g., autonomous bus) as the attack target, and the attacker can know the route where the victim will pass by.
The attacker can know which LiDAR is used in the victim vehicle based on the appearance or some documents. However, we assume that the attacker cannot know which localization algorithm is used by the victim. The attacker aims to deviate the victim from their driving lane. According to____, the deviation needs 0.29 meters on local roads and 0.74 meters on highways to touch the lane line.

% 吉岡:こちらはexperiment sectionに移動
%\noindent \textbf{Attacker Capabilities:}
%The attacker places an attack device on the roadside and irradiates the target vehicle with a laser when it passes by to tamper with the target's LiDAR point cloud. In this study, we consider two types of spoofing attacks that have been physically demonstrated in current LiDAR spoofing research: injecting wall-like point clouds and deleting point clouds within a certain range ____. Examples of affected point clouds are shown in Figure \ref{example_spoofing}.
%With a setup similar to ____, it is possible to inject or delete point clouds within a range of approximately 80 degrees for the VLP-16, allowing for the tampering of approximately 5000 point clouds. \koide{何点中の5000点?}\Nagata{28800点中の5000点です。spoofing範囲内に限れば6800点中の5000点です。}
%To realize attacks on robots moving outdoors, we use a spoofer with tracking using an infrared camera, as proposed by Suzuki ____. This attack device can track a vehicle traveling at 35 km/h from a distance of 45 meters and continuously perform spoofing attacks on the LiDAR sensor.