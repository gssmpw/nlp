\section{Introduction}


Few-shot Continual Relation Extraction - FCRE \citep{qin-joty-2022-continual, chen-etal-2023-consistent} has emerged as a challenging problem, where models must continuously adapt to identify new relations in each sentence with a limited amount of data while preserving all the previous information accumulated over time. Current approaches have demonstrated researchers' efforts to simultaneously maintain important characteristics of FCRE models, including the (i) flexibility to adapt to new relations, (ii) preserve generalization, and (iii) effectively avoid forgetting. These requirements have been recently addressed through a special regularization strategy \citep{tran-etal-2024-preserving, DBLP:conf/acl/WangWH23}. Besides, they can also be achieved through memory enrichment strategies \citep{ma-etal-2024-making} or by designing special prompt inputs \citep{DBLP:conf/acl/ChenWS23} to guide the learning of models.

However, the \textit{existing work are somewhat impractical} when most of them focus solely on optimizing models in {the scenario with a predefined set of entities and relations}. Specifically, the models are only trained and tested on a given set of entities in each sentence, with the corresponding targets also being predefined, making the questions of the applicability of current approaches. Nevertheless, in the real open-world scenarios \citep{DBLP:conf/www/XuLSY19, Mazumder2024}, more potential pairs of entities can appear in a testing sample, which is often uncovered in the training dataset. To deal with this drawback, several methods \citep{WANG2023151, zhao-etal-2025-dynamic, zhao-etal-2023-open, meng-etal-2023-rapl} have considered unknown labels, but their training only relies on available information, including provided entities and relations from the training set, and poorly considers a NOTA (None Of The Above) label for all possible relations that are uncovered. 
% In addition, there are the facts that there are also pairs of entities for which the corresponding relations cannot be determined (i.e.,  unknown or NA labels) \citep{liu-etal-2023-novel,xie-etal-2021-revisiting}, 
% Therefore, in this work, we introduce more realistic testing paradigms and conduct intensive experiments demonstrating the behaviors and performance of the existing FCRE methods in these scenarios. The results offer fresh insights, revealing existing limitations and promising potentials that can be further explored to develop efficient and highly applicable FCRE models.
% Through the experimental results, we found the necessity of FCRE models in dealing with undetermined relations. Particularly, these models should be able to \textcolor{red}{[Fix this - overclaimed]} 
Therefore, it is essential to develop FCRE models, which are able to
\textit{(a) identify relations between possible pairs of entities in a sentence by holistically taking advantage of the available information from training dataset, and, most importantly, (b) indicate whether the corresponding relations are known or not, and whether they are reasonable}. 

Related to these requirements, Open Information Extraction - OIE \citep{liu2022open, zhou2022survey,li2023evaluating} has been known as a solution for covering all possible pairs of entities and relations. Most recently, EDC \citep{zhang-soh-2024-extract} has proposed an efficient OIE strategy to extract any triples from given texts to construct relational graphs. 
However, this work and other related ones in the field of Knowledge Graph Construction (KGC) are only limited to exploiting information from a fixed dataset, while their ability in Fewshot Continual Learning scenarios, where there are always emerging relations, has not been explored. In this challenging scenario, the models need to adapt to information arriving sparsely and continuously over time. Besides, adapting the model to new information also poses challenges in reproducing relations from the previous knowledge that has been integrated into the graph.

To address the challenges of FCRE in practical settings and to explore the potential of KGC models in this complex scenario, we propose a novel solution inspired by the concept of Open Information Extraction (OIE). More specifically, our approach involves leveraging Named Entity Recognition (NER) \citep{zaratiana2023glinergeneralistmodelnamed} to extract and analyze all possible pairs of entities, focusing on identifying both determined and undetermined relations between them. We then utilize OIE for open extraction and generating corresponding descriptions that help effectively align sample representations. In addition, our sample-description matching schema is also a more effective solution for building KGC models in the latest SOTA \citep{zhang-soh-2024-extract}, which was based on description-description constraints. For testing, we dedicatedly employ OIE to filter out non-relational samples before they enter the FCRE modules, thereby specifically determining whether a tested sample has a known or unknown relation, or whether it is reasonable, thus improving overall testing performance. 

In summary, our key contributions include:
\begin{itemize}
\item We propose a novel solution for FCRE via Open Information Extraction, which 
% not only helps to avoid forgetting, overfitting but also 
effectively deals with unknown labels by considering all possible pairs of entities and corresponding relations in each sentence.
% by fully exploiting the information from the training set and building a determined and undetermined labeling system. Based on this foundation, 
Based on this, we can determine whether a pair has a known or reasonable relationship, which existing methods have not considered.

% \item We introduce a novel approach to relation description, combining perspectives from Open Information Extraction and original definitions to enhance knowledge retention in continual learning. Our method augments descriptions for greater diversity, improving performance in few-shot scenarios by reducing overfitting and increasing model robustness. This enables better generalization, even with many undetermined relations.


\item For the first time, we consider the potential of Knowledge Graph Construction (KGC) models in the setting of Continuous Learning, where there are always emerging relations. In addition, our novel approach effectively elevates schema matching in KGC as minimizing errors from LLM-based schema extraction.

\item Experimental results indicate our superior performance over FCRE baselines in all cases with N/A relation or not. In addition, the superiority over KGC models in this extreme scenario is also demonstrated.


% \item Finally, we tackle few-shot relation extraction in a more realistic setting with numerous undetermined relations. Our model excels in identifying relation types while preserving prior knowledge, outperforming existing methods.

% \item We conduct extensive experiments in realistic setting, where many undetermined relations exist among all possible entities. Our model demonstrates superior performance in detecting correct relation types while retaining knowledge of previously learned relations. 

% \item We also compare our approach with a method inherent used for constructing knowledge graphs in context of continual relation extraction,then reversely highlighting potential of our method for addressing number of relation increasing when constructing knoweldge graph more efficient.

% \item We propose using OIE to filter out non-relational samples and generate candidate relation descriptions. These descriptions support our alignment process, which stabilizes the learning representation of samples and relation descriptions. This approach effectively handles the large and imbalanced number of undetermined relations encountered during both training and testing in real-world scenarios.
\end{itemize}
% Therefore, to deal with FCRE in a real open-world setting, as well as explore the potentials of OIE in this challenging setting, we have proposed a novel solution to address all of the above requirements. Specifically, we explore and develop FCRE in context a lot of pair of entities extracted by leveraging the power of NER \citep{zaratiana2023glinergeneralistmodelnamed} that need to extracted relation that comprehensively identify learned relations or undetermined. Then, OIE is leveraged to enhance the sentence, entities, relation to generate candidate description to support triplet to align sample representation to learned relation type more stable and balancing in large number of undetermined relation. Also, consider OIE in scenarios that is utilized for filering no relation sample before FCRE modules to get overall increasing performance in extracting relation type exactly continual comeing while perserving previous knowledge

% In summary, our contributions including
% We propose to leverage OIE in possible used to filter no relation and mainly is used for generating candidate relation description that support to our alignment for triple (3) losses in memorize , stablzing learn representation of sample and relation description while deal with large, imbalance number undetermined relation take in training and testing as real world

% Extensive experiments for traditional method in our more realistic context that has many undetermined relation of all possible entities, and show out my best performance of our model in detect correct relation type while preserving knowledge about previously learned relations, and also compare to method that used for contruct knowlegde graph to show potiential used for this problem in continual scenarios
% Additionally, We apply OIE with FCRE in inference relation extraction pineline to enhance unknown relation extraction. It is as the first explore, compare potential of FCRE in open relation to construct graph problem when 

% pairs of entities and correspondingly expand pairs of relations, we have , from which the training set is continuously expanded and enriched, with information about entities, relations, and corresponding descriptions being extracted and utilized as thoroughly as possible. In addition, due to the inadvertent mismatch between the available training set and the information collected from NER as well as LLMs, we propose special matching loss functions, effectively aligning information between the "triplets", helping models capture new knowledge more comprehensively, robustly, and with better generalization.





% <draft>
% \begin{enumerate}
%     \item \textbf{From Relation Extraction to Few-Shot Relation Extraction:} 
%     Traditional Relation Extraction (RE) has evolved into Few-Shot Relation Extraction (FSRE), which aims to generalize to unseen relations with limited training examples.

%     \item \textbf{Handling Unknown and No-Relation Cases:} 
%     While traditional Relation Extraction considers Undetermined Relation (UR) cases (unknown or "no relation"), Few-Shot Relation Extraction often overlooks these, leading to limitations in real-world applications.

%     \item \textbf{Our Contributions:}
%     \begin{itemize}
%         \item We introduce a novel approach for generating and incorporating Undetermined Relation  data into training, making FCRE more realistic and applicable while providing insights into recent FCRE approaches.
%         \item We propose a triplet alignment strategy among Sample, Relation Description and Candidate Relation Description, combined with data augmentation techniques to improve performance.
%         \item We leverage an LLM-based Open Information Extraction (OIE) system to filter out nonexistent relations, enhancing Few-Shot Continual Relation Extraction (FCRE) in handling contexts with a high prevalence of Undetermined Relation as in the real world, then gain SOTA performance
%     \end{itemize}
% \end{enumerate}
