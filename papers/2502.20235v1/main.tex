% CVPR 2025 Paper Template; see https://github.com/cvpr-org/author-kit

\documentclass[10pt,twocolumn,letterpaper]{article}

%%%%%%%%% PAPER TYPE  - PLEASE UPDATE FOR FINAL VERSION
\usepackage{cvpr}              % To produce the CAMERA-READY version
%\usepackage[review]{cvpr}      % To produce the REVIEW version
\usepackage{algorithm}
\usepackage{algpseudocode}
\usepackage{float}

\usepackage{multicol}
\usepackage{multirow}
\usepackage{makecell}
% \usepackage[pagenumbers]{cvpr} % To force page numbers, e.g. for an arXiv version

% Import additional packages in the preamble file, before hyperref
\newcommand{\CG}{\mathcal{G}\xspace}
\newcommand{\CV}{\mathcal{V}\xspace}
\newcommand{\CE}{\mathcal{E}\xspace}
\newcommand{\CA}{\mathcal{A}\xspace}
\newcommand{\CF}{\mathcal{F}\xspace}
\newcommand{\CR}{\mathcal{R}\xspace}
\newcommand{\CB}{\mathcal{B}\xspace}
\newcommand{\CX}{\mathcal{X}\xspace}
\newcommand{\CK}{\mathcal{K}\xspace}
\newcommand{\CM}{\mathcal{M}\xspace}
\newcommand{\CC}{\mathcal{C}\xspace}
\newcommand{\CL}{\mathcal{L}\xspace}
\newcommand{\CI}{\mathcal{I}\xspace}
\newcommand{\CQ}{\mathcal{Q}\xspace}
\newcommand{\CO}{\mathcal{O}\xspace}
\newcommand{\CP}{\mathcal{P}\xspace}
\newcommand{\CS}{\mathcal{S}\xspace}
\newcommand{\CT}{\mathcal{T}\xspace}
\newcommand{\CJ}{\mathcal{J}\xspace}
\usepackage[para]{footmisc}
\usepackage{subfig}
% \usepackage{subcaption}
% \usepackage{array}
% \usepackage{colortbl}



% It is strongly recommended to use hyperref, especially for the review version.
% hyperref with option pagebackref eases the reviewers' job.
% Please disable hyperref *only* if you encounter grave issues, 
% e.g. with the file validation for the camera-ready version.
%
% If you comment hyperref and then uncomment it, you should delete *.aux before re-running LaTeX.
% (Or just hit 'q' on the first LaTeX run, let it finish, and you should be clear).
\definecolor{cvprblue}{rgb}{0.21,0.49,0.74}
\usepackage[pagebackref,breaklinks,colorlinks,allcolors=cvprblue]{hyperref}

%\usepackage[table]{xcolor}
\usepackage{multirow} 
%%%%%%%%% PAPER ID  - PLEASE UPDATE
\def\paperID{4620} % *** Enter the Paper ID here
\def\confName{CVPR}
\def\confYear{2025}

\newcommand{\yang}[1]{{\color{Red}#1}}
\newcommand{\yangc}[1]{{\color{Red}[ZY: #1]}}

\newcommand{\zichong}[1]{{\color{Blue}#1}}
\newcommand{\gx}[1]{{\color{Green}#1}}

%%%%%%%%% TITLE - PLEASE UPDATE
\title{Attention Distillation: A Unified Approach to Visual Characteristics Transfer}
%\title{Attention Distillation: A Unified Approach to Visual Transfer and Synthesis}
%\title{Attention Distillation for Style-Texture Specific Image Generation}
%\title{Attention Distillation for Neural Style Transfer and Texture Synthesis}
%\title{Neural Style Transfer and Texture Synthesis via Attention Distillation}

%%%%%%%%% AUTHORS - PLEASE UPDATE
% \author{First Author\\
% Institution1\\
% Institution1 address\\
% {\tt\small firstauthor@i1.org}
% % For a paper whose authors are all at the same institution,
% % omit the following lines up until the closing ``}''.
% % Additional authors and addresses can be added with ``\and'',
% % just like the second author.
% % To save space, use either the email address or home page, not both
% \and
% Second Author\\
% Institution2\\
% First line of institution2 address\\
% {\tt\small secondauthor@i2.org}
% }

\author{Yang Zhou \qquad Xu Gao \qquad Zichong Chen  \qquad Hui Huang\thanks{Corresponding author.}\\
	Visual Computing Research Center, CSSE, Shenzhen University\\
	%\tt\small \{zhouyangvcc, szchenkaijian, xiaorongjun000, hhzhiyan\}@gmail.com 
}


\begin{document}
 % \maketitle
\centering

\includegraphics[width=0.85\textwidth]{images/teaser.pdf}
\caption{
IP-Composer enables compositional generation from a set of visual concepts. These are portrayed through a set of input images, along with a prompt describing the desired concept to be extracted from each.
}


\begin{abstract}

% Recent works to jointly reconstruct 3D human and object from a single RGB image, are mostly model-based, that fail to capture the fine details of the clothed human body and object surface. In this paper, we introduce ReCHOR, a novel, model-free, first-method to produce realistic clothed human-object reconstructions from a monocular view. This is extremely challenging due to human-object occlusions, diverse interactions and depth ambiguity, as it needs to infer both 3D spatial awareness and high resolution details. Our core idea is based on estimating neural implicit representations for human and object respectively by an attention-based neural implicit model that attends to pixel-aligned features from both the global human-object image for spatial awareness and  the local separate view of human and object images for high quality details. Additionally, the network is conditioned on semantic features from an initial estimated human-object pose prior and a generative diffusion model that inpaints occluded regions, thus enabling the retrieval of details from them.
% We also propose a synthetic dataset with rendered scenes of diverse, inter-occluded 3D human and object scans, to train our network. We evaluate our method on the synthetic and real world BEHAVE dataset. Our experiments show that our method outperforms the SOTA in achieving realistic clothed human-object reconstructions.
Recent approaches to jointly reconstruct 3D humans and objects from a single RGB image represent 3D shapes with template-based or coarse models, which fail to capture details of loose clothing on human bodies. In this paper, we introduce a novel implicit approach for jointly reconstructing realistic 3D clothed humans and objects from a monocular view. For the first time, we model both the human and the object with an implicit representation, allowing to capture more realistic details such as clothing. This task is extremely challenging due to human-object occlusions and the lack of 3D information in 2D images, often leading to poor detail reconstruction and depth ambiguity. To address these problems, we propose a novel attention-based neural implicit model that leverages image pixel alignment from both the input human-object image for a global understanding of the human-object scene and from local separate views of the human and object images to improve realism with, for example, clothing details. Additionally, the network is conditioned on semantic features derived from an estimated human-object pose prior, which provides 3D spatial information about the shared space of humans and objects. To handle human occlusion caused by objects, we use a generative diffusion model that inpaints the occluded regions, recovering otherwise lost details. For training and evaluation, we introduce a synthetic dataset featuring rendered scenes of inter-occluded 3D human scans and diverse objects. Extensive evaluation on both synthetic and real-world datasets demonstrates the superior quality of the proposed human-object reconstructions over competitive methods.
\end{abstract}
\section{Introduction}
\label{sec:intro}
% Image editing methods in diffusion models depend on user-defined control directions - users can unlock their creativity using these methods by specifying the desired manipulation through prompts~\cite{gandikota2023concept}, reference images~\cite{ruiz2022dreambooth, kumari2022customdiffusion, gal2022image, chen2024trainingfreeregionalpromptingdiffusion}, or attribute vectors~\cite{parmar2023zero,hertz2022prompt}. In this work, we ask a fundamentally different question: \emph{Can we automatically discover the underlying visual structure of a concept within diffusion model's knowledge?} %Rather than requiring user-specified controls, we aim to decompose the model's internal knowledge into meaningful directions.

% This question touches on a fundamental limitation in how we interact with diffusion models. Current control methods ~\cite{zhang2023addingconditionalcontroltexttoimage, gandikota2023concept, ye2023ipadaptertextcompatibleimage,ye2023ipadaptertextcompatibleimage, hertz2024stylealignedimagegeneration, li2023photomaker, shi2024instantbooth, chen2024trainingfreeregionalpromptingdiffusion} require users to specify their desired manipulations in advance, limiting interactive creativity. This contrasts with natural human artistic workflows, where creators dynamically explore creative ideas while jointly refining them toward meaningful artistic outcomes~\cite{hoffmann2016modeling}. This synergy between specification and exploration is not new to generative models. Early GAN architectures naturally developed disentangled latent spaces that enabled continuous\cite{harkonen2020ganspace,radford2015unsupervised, wu2021stylespace, shen2020interfacegan}, compositional control over generated images. Users could explore these spaces to discover interesting variations that would be difficult to describe in words~\cite{wu2021stylespace}, then combine them to achieve their creative goals~\cite{grabe2022towards}. 


% While diffusion models have largely superseded GANs in conditional image synthesis~\cite{dhariwal2021diffusion},  their underlying structure remains less understood. Diffusion models achieve remarkable diversity through high-dimensional latents, unlike GANs' compact latent spaces.  With a single prompt, diffusion models can generate radically different variations through different random initializations of input noise. We ask - Is it possible to discover interpretable structure within this vast space of variations?

Text-to-image diffusion models are capable of generating remarkable visual variations from a single prompt through different random initializations. However, this vast creative potential remains largely opaque to users---while we can generate diverse images, we lack understanding of the underlying structure of these variations. This presents a fundamental challenge: how can we discover and expose the latent visual capabilities encoded within these models?

\let\thefootnote\relax \footnote{$^{*}$Correspondence to \texttt{gandikota.ro@northeastern.edu}}

The challenge touches on a key limitation in how we interact with diffusion models today. Current control methods require users to explicitly specify their desired edits in advance through prompts~\cite{gandikota2023concept}, reference images~\cite{zhang2023addingconditionalcontroltexttoimage, chen2024trainingfreeregionalpromptingdiffusion, ruiz2022dreambooth,kumari2022customdiffusion, Ryu_lora, hu2021lora}, or attribute vectors~\cite{ye2023ipadaptertextcompatibleimage, hertz2024stylealignedimagegeneration, li2023photomaker, shi2024instantbooth,parmar2023zero,hertz2022prompt}. That contrasts sharply with natural human creative workflows, where artists dynamically explore creative ideas and jointly refine them toward meaningful artistic outcomes~\cite{hoffmann2016modeling}. The need for pre-specified controls creates a barrier between users and the full creative potential of these models.

Interestingly, earlier generative models like GANs~\cite{gans,karras2019style,brock2018large} naturally developed more interpretable internal structures. Their compact latent spaces often exhibited emergent disentanglement~\cite{harkonen2020ganspace,radford2015unsupervised, wu2021stylespace, shen2020interfacegan}, enabling continuous and compositional control over generated images. Users could explore these spaces to discover interesting variations that would be difficult to describe in words~\cite{wu2021stylespace}, then combine them to achieve their creative goals~\cite{grabe2022towards}.

Diffusion models have largely superseded GANs in conditional image synthesis~\cite{dhariwal2021diffusion}, achieving greater diversity through much higher-dimensional latents. And yet an understanding of the underlying structure of these larger latent spaces has remained elusive. In this work, we ask a fundamental question: \emph{Can we automatically discover the visual structure within a diffusion model's knowledge of a concept?} Rather than requiring user-specified controls, we aim to decompose the model's internal representations into expressive directions that users can explore and combine.

To address these needs, we present \textbf{SliderSpace}, a framework that brings systematic explorability to diffusion models. Given just a text prompt, SliderSpace discovers a canonical set of meaningful, diverse, and controllable directions within the model's knowledge of that concept. Each direction is implemented as a low-rank adapter~\cite{hu2021lora} that can be scaled and composed with others, allowing users to explore and smoothly combine different aspects of variation, as shown in Figure~\ref{fig:intro}.

We ground SliderSpace discovery in three key requirements for meaningful decomposition of a diffusion model's visual manifold: 
\begin{enumerate}
    \item \textbf{Unsupervised Discovery:} The decomposition process should emerge from the intrinsic structure of the model's learned representation, rather than being guided by predefined attributes. This ensures we capture the true topology of the model's knowledge space rather than projecting our assumptions onto it.
    
    \item \textbf{Semantic Orthogonality:} Each discovered control must represent a distinct semantic direction. This is enforced in a semantic feature space, like CLIP, where every slider has an orthogonal effect in embeddings. This prevents discovering multiple controls that create similar semantic effects, making the system more efficient and easier.
    
    \item \textbf{Distribution Consistency:} Directions must induce consistent transformations across both random seeds and prompt variations. 
\end{enumerate}

These requirements naturally lead to our proposed framework, which we formalize in Section~\ref{sec:method}. As we show in our experiments, SliderSpace is architecture-agnostic, working with both conventional U-Net based models like Stable Diffusion~\cite{rombach2022high, rombach2022sd20, podell2023sdxl, turbo, dmd} and recent transformer-based architectures like Flux~\cite{flux}.

We demonstrate the expressiveness of SliderSpace through three applications: First, we show how SliderSpace can decompose high-level concepts into diverse and expressive components, revealing the natural axes of variation in the model's understanding. Second, we explore artistic style variation, where SliderSpace discovers directions that match or exceed the diversity of manually curated artist lists while being judged more useful by human evaluators. Finally, we show how SliderSpace can help reverse the mode collapse commonly observed in distilled diffusion models, restoring diversity while maintaining generation speed.

Beyond providing practical creative control, SliderSpace opens new avenues for understanding and utilizing the latent capabilities of diffusion models. By mapping these models' visual potential into intuitive, composable directions, we take a step toward making their creative possibilities more accessible and interpretable to users.

% Image editing methods in diffusion models unlock the creativity of users. In this work we ask an alternate question: \emph{Can we organize and expose what of the diffusion model is already capable of?}.
% Existing methods for controlling image generation typically require users to manually specify edit directions for desired changes. This process is time-consuming, requires technical expertise, and limits the spontaneity of the creative process. For instance, if a user wants to adjust the smile of a generated person, they must explicitly request this edit, often through imprecise prompt engineering or model fine-tuning. This approach of predefined controls or manual specifications restricts users from fully exploring the latent capabilities of the model. There may be interesting stylistic variations or attributes that the model can generate, but users have no easy way to discover or utilize these.

% Natural visual disentanglement was an emergent property in the latent space of Generative Adversarial Models (GANs) \cite{harkonen2020ganspace,radford2015unsupervised, wu2021stylespace, shen2020interfacegan}. In particular, it has been observed that StyleGAN~\cite{karras2019style} stylespace neurons offer detailed control over many meaningful aspects of images that would be difficult to describe in words~\cite{wu2021stylespace}. However, diffusion models do not share such a compact latent space~\cite{park2023unsupervised}; and efforts to uncover such a space in the semantic embeddings of the text conditioning have met with limited success \nik{Nick - is there a specific citation you were thinking about?}.

% In this work we introduce \textbf{SliderSpace}, which takes a step towards uncovering an analogous low dimensional representation of diffusion models' visual breadth; in essence treating the diffusion model as many generators sharing parameters, where a particular generator is defined by a specific prompt. For a given prompt we sample many random seeds (and optionally prompt expansions using an LLM), generate the corresponding images, and apply an off the shelf feature extractor (in this work CLIP, but our method can be applied to any differentiable feature extractor). We use PCA to analyze these features, and for each of the leading $k$ principal components we train a LoRA \cite{} which causes the diffusion model to produces images which increase the feature magnitude along that component when passed back through the same feature extractor. This leads to a 'Slider' for each principal component, because each LoRA can be scaled and applied to the original diffusion model, continuously varying those visual features in the generated results (as measured, in our case, by CLIP).

% There are many other works that enhance the controllability of diffusion models. One common approach is enabling users to add spatial constraints to a generation either manually, or via a reference image \cite{zhang2023addingconditionalcontroltexttoimage, chen2024trainingfreeregionalpromptingdiffusion}, a second is leveraging more abstract embeddings (e.g. identity, style) extracted from a reference image \cite{ye2023ipadaptertextcompatibleimage, hertz2024stylealignedimagegeneration, li2023photomaker, shi2024instantbooth}, a third is finetuning a foundation model to better generate a concept important to the user \cite{ruiz2022dreambooth, kumari2022customdiffusion, Ryu_lora, hu2021lora}, and a fourth (most relevant to this work) is finding low-rank adaptors of the model based on a prompt or small training set which can be scaled to provide continous control over one aspect of generated image (e.g. night vs day, basic vs luxury, etc.) \cite{gandikota2023concept}. SliderSpace is complementary to all of these methods and offers something distinct. All of the other methods we are aware require the user (and / or model designer) to know in advance what type of control they want. In contrast SliderSpace assists users in discovering and controlling hidden capabilities present in the diffusion model's distribution of possible generations.

%We propose that truly intuitive creative control in a text-to-image model should meet three key criteria: \emph{discoverability}, \emph{intuitiveness}, and \emph{specificity}. The model should reveal controllable attributes that may not be immediately obvious, offer controls that are easy to understand and manipulate, and ensure each control affects a distinct attribute of the generated image.

% We demonstrate the utility and power of SliderSpace using three applications built on top of SDXL-DMD \cite{dmd}, because its fast generation speed lends itself well to the continuous control offered by SliderSpace.

% First, we study concept decomposition (Section \ref{sec:concept_exp}), where we learn sliders for a specific concept (e.g. 'monster', 'waterfall', 'car'). Through quantitative metrics of diversity and text alignment we demonstrate that the learned sliders dramatically boost the diversity of generations when randomly applied without harming text alignment; we also ask humans to qualitatively judge these results in a user study where they find the SliderSpace results to be more 'Diverse', 'Useful', and 'Creative' than our baselines.

% Second, we attempt to compare the automatic discoveries of SliderSpace to a large scale manual study of artistic styles (Section \ref{sec:art_exp}), open-sourced by ParrotZone \cite{parrotzone}. In this study SDXL was prompted with over 4300 artist names,  and based on visual inspection the cases of successful stylistic mimicry recorded. Quantitatively SliderSpace more closely matches the distribution of artistic variation discovered by ParrotZone than other baselines, and in our user studies was judged to be significantly more 'Diverse' and 'Useful' than the baselines. To our surprise humans even judged SliderSpace results to be slightly more 'Diverse' than the results generated by the manually discovered artist names of \cite{parrotzone}.

% Third, we attempt to use SliderSpace to reverse the mode collapse commonly observed in distilled few-step diffusion models relative to the original teacher model (Section \ref{sec:diverse_exp}). We quantitatively demonstrate that applying SliderSpace to SDXL-DMD leads to more closely matching the distribution of images by the original teacher, SDXL.

%Through extensive experiments on various state-of-the-art text-to-image models, we demonstrate that SliderSpace significantly enhances user control and creative expression in AI-assisted image generation tasks. Our method enables a range of applications, including concept decomposition and control, diversity improvement in generated images, customization dissection and edits, and the exploration of artistic styles inherent in the model.

% SliderSpace goes beyond providing a practical tool for enhanced creative control. By mapping the visual potential of diffusion models it can open new avenues for generative creativity and deepens our understanding of each model's hidden potential.
\section{Related Work}

\paragraph{LLMs for Agent tasks.}

Our research is related to deploying large language models (LLMs) as agents for decision-making tasks in interactive environments~\citep{liu2023agentbench,zhou2023webarena,shridhar2020alfred,toyama2021androidenv}. Earlier works, such as~\citep{yao2023webshopscalablerealworldweb}, fine-tuned models like BERT~\citep{devlin2019bertpretrainingdeepbidirectional} for decision-making in simplified environments, such as online shopping or mobile phone manipulation. With the advent of large language models~\citep{brown2020languagemodelsfewshotlearners,openai2024gpt4technicalreport}, it became feasible to perform decision-making tasks through zero-shot or few-shot in-context learning. To better assess the capabilities of LLMs as agents, several models have been developed~\citep{deng2024mind2web,xiong2024watch,hong2023cogagent,yan2023gpt}. Most approaches~\citep{zheng2024seeact,deng2024mind2web} provide the agent with observation and action history, and the language model predicts the next action via in-context learning. Additionally, some methods~\citep{zhang2023building,li2023camel,song2024trial} attempt to distill trajectories from state-of-the-art language models to train more effective policy models. In contrast, our paper introduces a novel framework that automatically learns a reward model from LLM agent navigation, using it to guide the agents in making more effective plans.

\textbf{LLM Planning.} Our paper is also related to planning with large language models. Early researchers~\citep{brown2020languagemodelsfewshotlearners} often prompted large language models to directly perform agent tasks. Later, \citet{yao2022react} proposed ReAct, which combined LLMs for action prediction with chain-of-thought prompting~\citep{wei2022chain}. Several other works~\citep{yao2023treethoughtsdeliberateproblem,hao2023reasoning,zhao2023large,qiao2024agentplanningworldknowledge} have focused on enhancing multi-step reasoning capabilities by integrating LLMs with tree search methods. Our model differs from these previous studies in several significant ways. First, rather than solely focusing on text generation tasks, our pipeline addresses multi-step action planning tasks in interactive environments, where we must consider not only historical input but also multimodal feedback from the environment. Additionally, our pipeline involves automatic learning of the reward model from the environment without relying on human-annotated data, whereas previous works rely on prompting-based frameworks that require large commercial LLMs like GPT-4~\citep{openai2024gpt4technicalreport} to learn action prediction. Furthermore, \Model supports a variety of planning algorithms beyond tree search.

\textbf{Learning from AI Feedback.} In contrast to prior work on LLM planning, our approach also draws on recent advances in learning from AI feedback~\citep{bai2022constitutional,lee2023rlaif,yuan2024self,sharma2024critical,pan2024autonomous,koh2024tree}. These studies initially prompt state-of-the-art large language models to generate text responses that adhere to predefined principles and then potentially fine-tune the LLMs with reinforcement learning. Like previous studies, we also prompt large language models to generate synthetic data. However, unlike them, we focus not on fine-tuning a better generative model but on developing a classification model that evaluates how well action trajectories fulfill the intended instructions. This approach is simpler, requires no reliance on state-of-the-art LLMs, and is more efficient. We also demonstrate that our learned reward model can integrate with various LLMs and planning algorithms, consistently improving their performance.

\textbf{Inference-Time Scaling.} ~\citet{snell2024scaling} validates the efficacy of inference-time scaling for language models. Based on inference-time scaling, various methods have been proposed, such as random sampling~\citep{wang2022self} and tree-search methods~\citep{hao2023reasoning, zhang2024accessing, guan2025rstar}. Concurrently, several works have also leveraged inference-time scaling to improve the performance of agentic tasks. ~\citet{koh2024tree} adopts a training-free approach, employing MCTS to enhance policy model performance during inference and prompting the LLM to return the reward. ~\citet{gu2024your} introduces a novel speculative reasoning approach to bypass irreversible actions by leveraging LLMs or VLMs. It also employs tree search to improve performance and prompts an LLM to output rewards. ~\citet{yu2024exact} proposes Reflective-MCTS to perform tree search and fine-tune the GPT model, leading to improvements in ~\citet{koh2024visualwebarena}. ~\citet{putta2024agent} also utilizes MCTS to enhance performance on web-based tasks such as ~\citet{yao2023webshopscalablerealworldweb} and real-world booking environments. ~\cite{lin2025qlass} utilizes the stepwise reward to give effective intermediate guidance across different agentic tasks. Our work differs from previous efforts in two key aspects: (1) Broader Application Domain. Unlike prior studies that primarily focus on tasks from a single domain, our method demonstrates strong generalizability across web agents, mathematical reasoning, and scientific discovery domains, further proving its effectiveness. (2) Flexible and Effective Reward Modeling. Instead of simply prompting an LLM as a reward model, we finetune a small scale VLM~\citep{lin2023vila} to evaluate input trajectories. %Our reward scores range continuously between 0 and 1, in contrast to existing methods that rely on discrete scoring (e.g., 0 and 1, or 0, 0.5, and 1) through direct LLM prompting.

% Concurrently, several works have also leveraged inference-time scaling to improve the performance of agentic tasks. ~\citet{pan2024autonomous} demonstrates that LLMs and VLMs, such as the GPT series, can function as evaluators or reward models to provide guidance for fine-tuning or reflection, thereby enhancing digital agents. This lays the groundwork for subsequent studies that directly prompt LLMs as reward models. ~\citet{koh2024tree} adopts a training-free approach, employing MCTS to enhance policy model performance during inference. However, it is limited to web environments~\citep{koh2024visualwebarena}. Moreover, its value function relies on prompting an LLM, which is less effective than our proposed method. We validate our approach through ablation studies, demonstrating that our fine-tuned reward model is more effective. ~\citet{gu2024your} introduces a novel speculative reasoning approach to bypass irreversible actions, such as purchasing a product, by leveraging LLMs or VLMs. It also employs tree search to improve performance, but it remains restricted to the web domain~\citep{koh2024visualwebarena, deng2024mind2web}. Additionally, it lacks reward modeling and instead prompts an LLM to output rewards. ~\citet{yu2024exact} proposes Reflective-MCTS to perform tree search and fine-tune the GPT model, leading to improvements in ~\citep{koh2024visualwebarena}. However, this work focuses solely on a single web agent task, and its reward modeling is derived from multi-agent debate, differing from our more effective and efficient reward modeling approach. ~\citet{putta2024agent} also utilizes MCTS to enhance performance, but it is limited to web-based tasks such as ~\citep{yao2023webshopscalablerealworldweb} and real-world booking environments.


\section{Methodology}
\paragraph{Preliminaries.}
We primarily focus on the homologous model merging, in which $\boldsymbol{\theta}_i$ all come from the same base model $\boldsymbol{\theta}_{\rm{base}}$. Given $K$ tasks $\{T_1,T_2,\cdots,T_K\}$ and $K$ corresponding fine-tuned models with parameters $\{\boldsymbol{\theta}_1,\boldsymbol{\theta}_2,\cdots,\boldsymbol{\theta}_K\}$, model merging aims to combine $K$ fine-tuned models into one single model simultaneously performing on $\{T_1,T_2,\cdots,T_K\}$ without post-training~\cite{method_p1_1,method_p1_2}.
Task vector~\cite{ilharco2023editing,yang2024adamerging} is a key element in merging method which could enhances the base model‘s ability or enable the model to handle other tasks. Specifically, for task $T_i$, the task vector $\boldsymbol\tau_i\in \mathbb{R}^D$ is defined as the vector obtained by subtracting the SFT weights $\boldsymbol{\theta}_i$ from the base model weight
$\boldsymbol{\theta}_{\rm{base}}$, \emph{i.e.}, $\boldsymbol\tau_i=\boldsymbol{\theta}_i-\boldsymbol{\theta}_{\rm{base}}$. The merged model could be denoted as $\boldsymbol{\theta}_m=\boldsymbol{\theta}_{\rm{base}}+\sum_i \lambda_i\boldsymbol{\tau}_i$, which $\lambda_i$ is the scaling factor measuring the importance of task vector. For clarification, we also denote the neuron set in $\boldsymbol{\theta}_i$ as $\mathcal{N}_i$, the neuron set in $\boldsymbol{\tau}_i$ as $\mathcal{T}_i$.



\begin{algorithm}[!ht]
    \caption{LED-Merging}
    \label{alg1}
    \begin{algorithmic}[1]
        \REQUIRE  base model $\boldsymbol{\theta}_{\rm{base}}$, SFT models $\{\boldsymbol{\theta}_{i}\mid i\in [K]\}$, mask ratios \{$r_{i} \mid i\in [K]\}$, scaling factors $\{\lambda_i\mid i\in[K]\}$, location datasets $\{\mathcal{X}_{i}\mid i\in[K]\}$
        \ENSURE merged parameter $\boldsymbol{\theta}_{m}$
        \STATE $\mathcal{M}\leftarrow\phi$
        \STATE $\boldsymbol{\theta}_{m}\leftarrow \boldsymbol{\theta}_{\rm{base}}$
        \FOR{$i\in [K]$}
        \STATE $I(\boldsymbol{\theta}_i)=\mathbb{E}_{x\sim \mathcal{X}_i}|\boldsymbol{\theta}_{i}\odot \nabla_{\boldsymbol{\theta}_i}\mathcal{L}(x)|$
        \STATE $I(\boldsymbol{\theta}_{\rm{base}})=\mathbb{E}_{x\sim \mathcal{X}_i}|\boldsymbol{\theta}_{\rm{base}}\odot \nabla_{\boldsymbol{\theta}_{\rm{base}}}\mathcal{L}(x)|$
        
        \STATE calculate $\mathcal{T}^{r_i}_{i}$ following Equation \ref{vote}
        \STATE  $\mathcal{M}\leftarrow \mathcal{M}\cup\{\mathcal{T}^{r_i}_i\}$
       
        
   
        
        
        \ENDFOR  
        \FOR{$i\in [K]$}
        
        \STATE calculate $\text{Disjoint}(\mathcal{T}_i^{r_i})$ use Equation~\ref{disjoint_safety}
        \STATE $\boldsymbol{m}_i \leftarrow \boldsymbol{0}$
        \FOR{$d\in \mathcal{T}_i^{r_i}$}
        \STATE $\boldsymbol{m}_{i,d}=1$
        \ENDFOR
        \STATE $\boldsymbol{\theta}_{m}\leftarrow \boldsymbol{\theta}_{m}+\lambda_i \boldsymbol{\tau}_i\odot \boldsymbol{m}_{i}$
        \ENDFOR
    \end{algorithmic}
\end{algorithm}
    %\vspace{-5pt}
\begin{figure*}[h!]
    \centering
    \includegraphics[width=\linewidth]{figs/pipeline_v2.pdf}
    \vspace{-40mm}
    \caption{Overview of our two-stage training pipeline {\ours}.}
    \label{fig:pipeline}
\end{figure*}


\paragraph{LED-Merging: Location, Election, and Disjoint Merging}
To address the neuron misidentification and interference issues in existing model merging methods, we propose LED-Merging (Location, Election, and Disjoint Merging). Specifically, previous studies \cite{modelstock, ilharco2023editing, tiesmerging} fail to accurately identify safety-related neurons in task vectors with a single magnitude score, namely \textit{neuron misidentification}. Meanwhile, there exists an interference between safety-related and utility-related task vector neurons during the merging process, namely \textit{neuron interference}. To address neuron misidentification, we first locate important neurons both in the base and fine-tuned models and then elect neurons from the task vector considering these two scores together. Subsequently, to mitigate the interference, we introduce a disjoint step, isolating these important neurons so that they influence different base neurons. The whole process is illustrated in Figure~\ref{fig:method}. 




In the location and election step, we consider the importance score from base and fine-tuned models simultaneously to locate task-specific neurons. In this way, it is more accurate than relying on the magnitude score alone because task-specific neurons with high importance score in the fine-tuned model may not necessarily score high in the base model, and vice versa.

{\textbf{Location}}.  We first calculate importance scores for each neuron in a base/fine-tuned model. Given a location dataset $\mathcal{X}_i=\{(x,y)_k\}$, where $x$ is the question and $y$ is the answer, we calculate the importance scores for the weight $\boldsymbol{\theta}_i\in\mathbb{R}^D$ in any  layer as follows~\cite{snip,spareseGPT,sun2024a}:
\begin{equation}
    I(\boldsymbol{\theta}_i)=\mathbb{E}_{x\sim \mathcal{X}_i}[\boldsymbol{\theta}_i\odot \nabla _{\boldsymbol{\theta}_i}\mathcal{L}(x)],
    \label{location}
\end{equation}
which $\mathcal{L}(x)=-\log p(y\mid x)$ is the conditional negative log-likelihood loss. We choose the SNIP score~\cite{snip} because it balances computational efficiency and performance~\cite{cq}. Please refer to Sec.~\ref{sec:ablation} for the comparison between different location methods. After computing importance scores, we choose top-$r_i$ neurons as the important neuron subset $\mathcal{N}_{i}^{r_i}$ from $I(\boldsymbol{\theta}_i)$.
 
 % After computing locating scores, we select the neurons scoring both high in base and fine-tuned models as important neurons in task vectors. Then in the disjoint step,  with preventing  polysemantic neurons  from receiving gradient updates towards different directions,
 % we use set difference to isolate the safety   and utility-related neurons  and construct corresponding masks for merging process,

{\textbf{Election}}. A natural question is how to select important neurons in the task vector $\boldsymbol{\tau}_i$ based on $I(\boldsymbol{\theta}_{\rm{base}})$ and $I(\boldsymbol{\theta}_{i})$. The important neurons in the base model may be different from neurons in the fine-tuned model. Therefore, we introduce the following election strategy to select neurons with high scores in both base and fine-tuned models:
\begin{equation}
    \mathcal{T}_i^{r_i}=\mathcal{N}_i^{r_i}\cap \mathcal{N}_{\rm{base}}^{r_i}.
    \label{vote}
\end{equation}
\emph{Remark}. We compare different choosing methods, including scoring low or high in base or fine-tuned model in Section~\ref{sec:ablation} and find that Equation \ref{vote} achieves the best performance.





{\textbf{Disjoint}}. As important neurons from different task vectors may conflict with each other at the same position, we use the set difference to disjoint the neurons from others to prevent interference:
\begin{equation}
    \text{Disjoint}(\mathcal{T}^{r_i}_{i})=\mathcal{T}^{r_i}_{i}-\mathop{\cup}\limits_{{J}\subsetneqq [K],|J|\geq 2}\mathop{\cap}\limits_{j\in {J}}\mathcal{T}^{r_j}_{j}.
    \label{disjoint_safety}
\end{equation}

Next, we construct a mask $\boldsymbol{m}_i\in\mathbb{R}^D$ to implement disjoint in the merging process. Specifically, this mask $\boldsymbol{m}_i$ is used to select neurons from $\mathcal{T}_i$. The mask ratio is $r_i$, where $r\in(0,1]$. The mask $\boldsymbol{m}_i$ can be derived from:
\begin{equation}
    \boldsymbol{m}_{i,d}=\begin{aligned} &\left\{ \begin{array}{ll} 1, & \text{if } d\in \text{Disjoint}(\mathcal{T}_{i}^{r_i}), \\ 0, & \text{otherwise}. \end{array} \right. \end{aligned}
    \label{mask_safety}
\end{equation}


% \subsection{Merging Models with Masks}
{\textbf{Merging}}. The final
merged task vector $\boldsymbol{\tau}_m$ is as follows:
\begin{equation}
    \boldsymbol{\tau}_m= \sum_i \lambda_i\boldsymbol{\tau}_{i}\odot\boldsymbol{m}_i.
    \label{merged_task_vector}
\end{equation}
We summarize the workflow in Algorithm \ref{alg1}.



\section{Experiments}
\label{sec:experiments}

\begin{figure*}[t]
\vspace{-6mm}
    \centering
    \includegraphics[width=0.8\linewidth]{figs/compare.pdf}
    \vspace{-4mm}
    \caption{\textbf{Qualitative comparison} with the baseline for generating a sequence of novel view images.  
    The results demonstrate that our method synthesizes more consistent multi-view images compared to our baseline model (Zero123). In addition, compared to SyncDreamer, our method visually maintains better similarity to the conditioned image and appears more natural.}
    \label{fig:sota_compare}
\vspace{-5mm}
\end{figure*}

\subsection{Experimental Setups}
\textbf{Dataset.}
Following previous work~\cite{zero123, SyncDreamer}, we evaluate our work on the Google Scanned Object (GSO)~\cite{GSO} dataset to verify the zero-shot novel view image synthesis capability. 
We also provide results for additional datasets in the Supplementary Material.
Specifically, we randomly select 30 objects from the GSO dataset with various object categories. 
Unlike recent approaches~\cite{mvdream, SyncDreamer} that aim to enhance the consistency of novel view synthesis models by generating multiple fixed-view images, our method can generate images from any camera pose and any number of views. Therefore, we conduct experiments under different camera pose settings to validate our approach:
specifically, 
1) \textit{16-views with free camera pose}: for each object, we circularly render 16 views with the elevation angles ranging in $[-10\degree, 40\degree]$ and the azimuth angles are evenly distributed in $[0\degree, 360\degree]$. 
2) \textit{16-views with fixed camera pose}: We maintain a constant elevation angle of $30\degree$ and uniformly sample azimuth angles (same as SyncDreamer~\cite{SyncDreamer}).
3) \textit{32-views with free camera pose}: Similar to the first setting, but we sample 32 views.
It's important to note that our method does not require additional training or fine-tuning on any datasets.

\noindent\textbf{Metrics.}
To validate the effectiveness of our method, we mainly evaluate it based on three criteria:
1) \textit{Quality Score}. We evaluate the image quality of synthesized multi-view images by measuring their similarity with ground truth images. Following prior research~\cite{zero123, sparsefusion}, we report the similarity between the synthesized images and the ground truth images with standard metrics: PSNR, SSIM~\cite{ssim}, and LPIPS~\cite{lpips}.
2) \textit{Multi-view Consistency Score}. As the primary goal of our work is to improve the consistency of generated images, we also employ the 3D consistency score~\cite{3dim} to verify the consistency among the synthesized images. Specifically, we train an Instant-NGP~\cite{instant_ngp} with the input image and part of the synthesized novel view images of our model and evaluate the similarity between the remaining synthesized images and the rendered images of Instant-NGP. For the synthesized multi-view images of each object, we allocate $3/4$ for training and reserve the remaining $1/4$ for validation.
Intuitively, if the consistency of synthesized images is improved, the NeRF-like model will train a better object representation, and the re-rendered images will agree more with the validation images.
3) \textit{Input Consistency Score}. To assess the faithfulness of synthesized images in preserving the identity of the input condition image, we introduce the input consistency score. This score calculates the similarity of each synthesized image with the input condition image, utilizing the LPIPS metric.

In addition, we use synthesized multi-view images to train a neural 3D reconstruction model (NeuS~\cite{neus}) and report commonly used Chamfer Distances (CD) and Volume IoUs between the trained 3D model and the ground truth.

\noindent\textbf{Baselines.}
Given that our main goal is to improve the consistency of the trained baseline model without further fine-tuning, we mainly compare our approach with the used baseline model Zero123~\cite{zero123}. Additionally, we compare our method to the SOTA approaches such as PGD~\cite{tseng2023consistent} and SyncDreamer~\cite{SyncDreamer} using the same Zero123 base model.

\noindent\textbf{Implementation Details.}
We use the official checkpoint provided by Zero123~\cite{zero123}, which is trained on objaverse~\cite{objaverse} for 165,000 steps. We inject our epipolar attention layer after step $T=4$ and layer $L=10$ by default. We find that feature fusion weight $\alpha=0.5$, and the number of context views $M=2$ work better.

\begin{table}[t]
\centering
\caption{Comparison of multi-view consistency, image quality, and input consistency of synthesized multi-view images at the 16-view setting with free camera pose.}
\label{tab:view16_free_compare}
\vspace{-2mm}
\scalebox{0.6}{
\begin{tabular}{c ccc ccc c}
\toprule
              & \multicolumn{3}{c}{Multi-view Consistency} & \multicolumn{3}{c}{Quality Score} & \multicolumn{1}{c}{Input Consis.} \\
              \cmidrule(lr){2-4} \cmidrule(lr){5-7} \cmidrule(lr){8-8}
              & PSNR$\uparrow$  & SSIM$\uparrow$ & LPIPS$\downarrow$ 
              & PSNR$\uparrow$  & SSIM$\uparrow$ & LPIPS$\downarrow$ 
              & LPIPS$\downarrow$ 
              \\ \midrule

Zero123
& 15.225        & 0.645       & 0.408
& 14.255        & 0.747       &	0.208
& 0.303         
\\
SyncDreamer
& 14.830        & 0.626       & 0.434
& 12.650        & 0.713       &	0.254
& 0.317         
\\
Ours 
& \best{18.300}	& \best{0.734}	& \best{0.355}
& \best{14.947}	& \best{0.763}	& \best{0.191}
& \best{0.282}
\\

\bottomrule
\end{tabular}
}
\end{table}

\begin{table}[t]
\vspace{-1mm}
\centering
\caption{Comparison of multi-view consistency, image quality, and input consistency at the 16-view setting with fixed camera pose as SyncDreamer~\cite{SyncDreamer}.}
\label{tab:view16_fxied_compare}
\vspace{-3mm}
\scalebox{0.6}{
\begin{tabular}{c ccc ccc c}
\toprule
              & \multicolumn{3}{c}{Multi-view Consistency} & \multicolumn{3}{c}{Quality Score} & \multicolumn{1}{c}{Input Consis.} \\
              \cmidrule(lr){2-4} \cmidrule(lr){5-7} \cmidrule(lr){8-8}
              & PSNR$\uparrow$  & SSIM$\uparrow$ & LPIPS$\downarrow$ 
              & PSNR$\uparrow$  & SSIM$\uparrow$ & LPIPS$\downarrow$ 
              & LPIPS$\downarrow$ 
              \\ \midrule

Zero123
& 16.556        & 0.682       & 0.378
& 14.592        & 0.750       &	0.207
& 0.305         
\\
SyncDreamer
& \best{22.424}        & \best{0.812}       & \best{0.268}
& 15.269        & 0.749       &	0.196
& 0.300         
\\
Ours 
& 21.151	& 0.780	& 0.302
& \best{15.293}	& \best{0.764}	& \best{0.184}
& \best{0.287}
\\

\bottomrule
\end{tabular}
}
\vspace{-4mm}
\end{table}


\subsection{Comparison With Baseline Models}
The quantitative comparison on three settings are shown in Tab.~\ref{tab:view16_free_compare}, Tab.~\ref{tab:view16_fxied_compare}, and Tab.~\ref{tab:view32_free_compare}. The qualitative comparison is shown in Fig.~\ref{fig:sota_compare}.

\begin{table}[t]
\centering
\caption{Comparison of multi-view consistency and image quality scores of synthesized multi-view images at the 32-view setting with free camera pose.}
\vspace{-3mm}
\label{tab:view32_free_compare}
\scalebox{0.7}{
\begin{tabular}{c ccc ccc}
\toprule
              & \multicolumn{3}{c}{Multi-view Consistency} & \multicolumn{3}{c}{Quality Score} \\
              \cmidrule(lr){2-4} \cmidrule(lr){5-7}
              & PSNR$\uparrow$  & SSIM$\uparrow$ & LPIPS$\downarrow$ 
              & PSNR$\uparrow$  & SSIM$\uparrow$ & LPIPS$\downarrow$ 
              \\ \midrule

Zero123
& 16.515        & 0.694       & 0.378
& 15.142        & 0.733       &	0.211
\\
PGD~\cite{tseng2023consistent}
& 18.481        & 0.720       & 0.343
& 15.281        & 0.739       &	0.205
\\
Ours 
& \best{20.655}	& \best{0.792}	& \best{0.305}
& \best{15.268}	& \best{0.742}	& \best{0.203}
\\

\bottomrule
\end{tabular}
}
\vspace{-3mm}
\end{table}

\begin{table*}
  [t]
  \centering
  \resizebox{\textwidth}{!}{%
  \begin{tabular}{cccccccccccc}
    \toprule \multicolumn{2}{c}{Components}                                                             & \multicolumn{5}{c}{Re-executability Rate (\%)} & \multicolumn{5}{c}{Readability (\#)} \\
    \cmidrule(lr){1-2} \cmidrule(lr){3-7} \cmidrule(lr){8-12}        \hspace{8pt}\labelemoji\hspace{8pt}                                                                & \hspace{8pt}\toolemoji\hspace{8pt}                                      & O0                                 & O1             & O2             & O3             & AVG            & O0             & O1             & O2             & O3             & AVG            \\
    \hline
    \rowcolor[rgb]{0.93,0.93,0.93}\multicolumn{12}{c}{\textbf{Initialize with LLM4Decompile-End-6.7B~\citep{llm4decompile}}}   \\
    \xmark                                                                                              & \xmark                                    & 69.51                              & 46.95          & 50.61          & 46.34          & 53.35          & 3.98 & 3.41 & 3.44 & 3.38 & 3.55 \\
    \cmark                                                                                              & \xmark                                    & 75.61                              & 50.61          & 50.00          & 50.00          & 56.55          & 4.01 & 3.44 & 3.39 & \textbf{3.49} & 3.58 \\
    \xmark                                                                                              & \cmark                                    & 83.54                     & \textbf{56.10}          & 51.22          & 50.61 & 60.37 & 4.05 & 3.51 & 3.51 & 3.42 & 3.62 \\
    \cmark                                                                                              & \cmark                                    & \textbf{85.37}                            & \textbf{56.10}                     & \textbf{51.83} & \textbf{52.43}          & \textbf{61.43} & \textbf{4.13} & \textbf{3.60} & \textbf{3.54} & \textbf{3.49} & \textbf{3.69} \\

    \rowcolor[rgb]{0.93,0.93,0.93}\multicolumn{12}{c}{\textbf{Initialize with Deepseek-Coder-6.7B-base~\citep{deepseekcoder}}} \\
    \xmark                                                                                              & \xmark                                    & 59.15                              & 35.98          & 39.02          & 37.80          & 42.99          & 3.71 & 3.05 & 3.16 & 3.05 & 3.24 \\
    \cmark                                                                                              & \xmark                                    & 66.46                              & 41.46          & 38.41          & 36.59          & 45.73          & 3.76 & 3.17 & \textbf{3.21} & 3.08 & 3.31 \\
    \xmark                                                                                              & \cmark                                    & 70.73                              & 39.63          & 39.02          & 40.24          & 47.41          & 3.90 & 3.17 & 3.08 & 3.11 & 3.31 \\
    \cmark                                                                                              & \cmark                                    & \textbf{79.88}                     & \textbf{45.73} & \textbf{43.90} & \textbf{42.68} & \textbf{53.05} & \textbf{3.96} & \textbf{3.21} & 3.18 & \textbf{3.19} & \textbf{3.38} \\
    \bottomrule
  \end{tabular}%
  }
  \caption{The ablation study of different methods across four optimization levels
  (O0, O1, O2, O3), as well as their average scores (AVG). The results in bold represent the optimal performance. The ~\labelemoji~ and ~\toolemoji~ means Relabedling and Function Call. \textbf{Bold} denotes the best performance.}
  \label{tab:ablation}
\end{table*}



\begin{figure*}[ht]
    \centering
    \begin{minipage}{0.65\textwidth}
        \centering
        \includegraphics[width=0.95\linewidth]{figs/ablation.pdf}
        \vspace{-2mm}
        \captionof{figure}{Qualitative Comparison for different design choices. Our method, employing multi-view epipolar attention, demonstrates the best consistency.}
        \label{fig:ablation}
    \end{minipage}\hfill
    \begin{minipage}{0.33\textwidth}
        \centering
        \includegraphics[width=0.8\linewidth]{figs/neus_ver.pdf}
        \vspace{-3mm}
        \caption{Our method shows better direct 3D reconstruction~\cite{neus}.}
        \label{fig:neus}
    \end{minipage}
    \vspace{-5mm}
\end{figure*}

\noindent\textbf{Multi-view Consistency.}
Tab.~\ref{tab:view16_fxied_compare} presents the 3D consistency scores compared to our baseline model (Zero123) and SyncDreamer. The results indicate a significant improvement across all three metrics achieved by our method when compared with Zero123.
While our method exhibits a marginally lower numerical consistency score compared to SyncDreamer, it enables the synthesis of images with arbitrary camera poses.	
This capability is illustrated in Tab.~\ref{tab:view16_free_compare}, where our method consistently enhances consistency with changes in camera pose settings, whereas SyncDreamer fails to do so and exhibits inferior results compared to Zero123.
Furthermore, our method facilitates the synthesis of multi-view images with any number of camera views. This versatility is demonstrated in Tab.~\ref{tab:view32_free_compare}, where our method continues to achieve significant improvements in consistency scores, while SyncDreamer is unable to operate under such conditions.	

Meanwhile, Fig.~\ref{fig:sota_compare} provides a qualitative comparison with the baseline. While both our method and SyncDreamer enhance consistency, our method visually preserves better similarity to the input image, including color and texture details. The input consistency score further corroborates this.

\noindent\textbf{Image Quality.}
While our primary goal centers around enhancing the consistency of synthesized multi-view images, we also evaluate the image quality by comparing the similarity with the ground truth images. The results shown in Tab.~\ref{tab:view16_free_compare}, Tab.~\ref{tab:view16_fxied_compare}, and Tab.~\ref{tab:view32_free_compare} indicate that our method also enhances the image quality under different settings besides improving the consistency.
Moreover, our method shows better image quality compared with SyncDreamer even in the 16-view setting with fixed camera pose.

\noindent\textbf{Input Consistency.}
Input consistency terms whether the results align with the input image.
Fig.~\ref{fig:sota_compare} illustrates that both our method and SyncDreamer enhance multi-view consistency. However, the color and texture details of SyncDreamer's results diverge from the input image and appear visually unnatural.
This discrepancy is evident in the input consistency score presented in Tab.~\ref{tab:view16_fxied_compare}, indicating lower similarity with the condition image in the SyncDreamer results.	

\subsection{Ablation Study}
The overall quantitative results are shown in Tab.~\ref{tab:ablation}, and the qualitative comparisons are shown in Fig.~\ref{fig:ablation}.

\noindent \textbf{Full Attention \vs Epipolar Attention.}
The results presented in Tab.\ref{tab:ablation} and Fig.\ref{fig:ablation} demonstrate that our epipolar attention mechanism can synthesize more consistent multi-view images compared with full attention. Furthermore, our epipolar attention achieves a greater performance improvement compared to full attention when using multiple reference images. This could be attributed to the fact that our epipolar attention more effectively localizes target information, as depicted in Fig.~\ref{fig:full_attn_compare}, thereby reducing noise from the reference images. In the multi-view setting, where multiple reference images are utilized, this noise reduction becomes particularly crucial.
Moreover, it is noteworthy that the epipolar attention mechanism consumes less GPU memory compared to our baseline, as discussed in Sec.~\ref{sec:attn_analysis}.

\noindent \textbf{Attending Single-View \vs Multi-View.}
Applying the epipolar attention significantly improves the consistency between the input and target views. However, the consistency between different views in the unobserved regions of the input view is not well preserved.
After implementing our epipolar attention in the multi-view setting, the consistency across the generated multi-view images is further improved. The last row in Tab.~\ref{tab:ablation} shows that after applying our multi-view epipolar attention, the consistency score is further improved compared with the single-view setting. Besides, the qualitative result in Fig.~\ref{fig:ablation} also shows better consistency among different target views.



\begin{table}[t]
\centering
\vspace{-1mm}
\caption{Comparison of 3D reconstruction results. Our method significantly improves the reconstruction quality.}
\vspace{-3mm}
\label{tab:neus}
\scalebox{0.7}{
\begin{tabular}{c cc}
\toprule
              &  Chamfer Dist.$\downarrow$  & Volume IoU$\uparrow$
\\ \midrule

            Zero123         & 0.017         & 0.819    \\
            SyncDreamer     & \best{0.013}         & \best{0.847}    \\
            Ours            & 0.014	& 0.842 \\

\bottomrule
\end{tabular}
}
\vspace{-5mm}
\end{table}


\vspace{-2mm}
\subsection{Downstream Application}
\vspace{-2mm}
To demonstrate the effectiveness of our method, we also applied it to the downstream 3D reconstruction task. Specifically, we trained the NeuS model~\cite{neus} directly using images synthesized by our method, Zero123, and SyncDreamer, respectively.
The quantitative results in Tab.~\ref{tab:neus} show that the consistent multi-view images synthesized by our method can significantly improve the 3D reconstruction quality.
Additionally, our method exhibits similar performance to SyncDreamer which requires time-consuming re-training.
The qualitative results in Fig.~\ref{fig:neus} show that it is challenging to train the NeuS model directly due to the lack of consistency in the images generated by Zero123. In contrast, our method generates more consistent multi-view images and, therefore, better reconstructs the geometry and texture details.
We show improvements on other downstream applications such as image-to-3D in the Supplementary Material.


\section{Conclusion}

%In this paper, w
We propose a new PEFT method called DiffoRA, which enables efficient and adaptive LLM fine-tuning based on LoRA. 
Instead of adjusting every interior rank, 
%of the decomposition matrices 
%of all modules, 
we argue that adopting LoRA module-wisely is sufficient. 
To achieve this, we construct a DAM to select the modules that are most suitable and essential to fine-tune. We theoretically analyze how the DAM impacts the convergence rate and generalization capability.
%of the pre-trained model. 
Furthermore, we adopt continuous relaxation and discretization to establish DAM.
%for each task. 
To alleviate the issue of discretization discrepancy, we utilize the weight-sharing strategy for optimization. 
%We fully implement our method and t
The experimental results demonstrate that our DiffoRA works consistently better than the baselines across all benchmarks. 

% \clearpage
% {
%     \small
%     \bibliographystyle{ieeenat_fullname}
%     \bibliography{main.bib}
% }
\documentclass{MITstyle}

%\usepackage[table]{xcolor}
\usepackage{chngcntr}
\usepackage{hyperref}
\usepackage{microtype}

\title{A Lightweight and Extensible Cell Segmentation and Classification Model for Whole Slide Images}

\author{Nikita Shvetsov~$^{1, }$\footnote{Correspondence e-mail: nikita.shvetsov@uit.no}, Thomas K. Kilvaer~$^{2, 3}$, Masoud Tafavvoghi~$^{4}$, Anders Sildnes~$^{1}$, \\ Kajsa Møllersen~$^{4}$, Lill-Tove Rasmussen Busund~$^{5, 6}$, Lars Ailo Bongo~$^{1}$ \\
%
\vspace{1em} % Space between authors and afilliations
%
\normalfont{\small $^{1}$Department of Computer Science, UiT The Arctic University of Norway}\\
\normalfont{\small $^{2}$Department of Oncology, University Hospital of North Norway}\\
\normalfont{\small $^{3}$Department of Clinical Medicine, UiT The Arctic University of Norway}\\
\normalfont{\small $^{4}$Department of Community Medicine, UiT The Arctic University of Norway}\\
\normalfont{\small $^{5}$Department of Medical Biology, UiT The Arctic University of Norway} \\
\normalfont{\small $^{6}$Department of Clinical Pathology, University Hospital of North Norway} %\vspace{2em}
}

\begin{document}
\maketitle

\section*{Abstract}

% \begin{abstract}
% Developing clinically useful cell-level analysis tools in digital pathology remains challenging due to limitations in dataset granularity, inconsistent annotations, computational demands of advanced models, and difficulties in integrating new technologies into clinical workflows. To address these challenges, we propose a multi-faceted solution that enhances data quality, model performance, and usability to create a lightweight and extensible cell segmentation and classification model.

% First, we update data labels by employing a cross-relabeling process that refines the labels of two existing datasets, PanNuke and MoNuSAC, to create a new unified dataset with enhanced granularity, encompassing seven distinct cell types. Second, we leverage the H-Optimus foundation model as a fixed encoder to improve feature representation for simultaneous cell segmentation and classification tasks. Third, to address the computational demands of foundation models, we employ knowledge distillation to reduce model size and complexity while maintaining comparable performance. Finally, to facilitate integration into clinical workflows, we integrate the distilled model into the QuPath software, a widely used open-source platform in digital pathology.

% Our results demonstrate improvements in cell segmentation and classification performance using the H‑Optimus-based model compared to a CNN-based model. Specifically, the average $R^2$ improved from 0.575 to 0.871, and the average $PQ$ score improved from 0.450 to 0.492, indicating better alignment with actual cell counts and enhanced segmentation and classification quality. Furthermore, the distilled student model maintains performance comparable to the larger foundation model while reducing the parameter count by a factor of 48.
% Overall, by reducing computational complexity and integrating it into existing workflows, the proposed approach may significantly impact diagnostic processes, reduce the workload of pathologists, and contribute to improved patient outcomes. Though our approach shows potential enhancements in efficiency and usability of cell segmentation and classification models in digital pathology, extensive validation is needed to deploy these models in clinical practice.
% \end{abstract}

%%% shortened abstract
\begin{abstract}
Developing clinically useful cell-level analysis tools in digital pathology remains challenging due to limitations in dataset granularity, inconsistent annotations, high computational demands, and difficulties integrating new technologies into workflows. To address these issues, we propose a solution that enhances data quality, model performance, and usability by creating a lightweight, extensible cell segmentation and classification model. 

First, we update data labels through cross-relabeling to refine annotations of PanNuke and MoNuSAC, producing a unified dataset with seven distinct cell types. Second, we leverage the H-Optimus foundation model as a fixed encoder to improve feature representation for simultaneous segmentation and classification tasks. Third, to address foundation models' computational demands, we distill knowledge to reduce model size and complexity while maintaining comparable performance. Finally, we integrate the distilled model into QuPath, a widely used open-source digital pathology platform. 

Results demonstrate improved segmentation and classification performance using the H-Optimus-based model compared to a CNN-based model. Specifically, average $R^2$ improved from 0.575 to 0.871, and average $PQ$ score improved from 0.450 to 0.492, indicating better alignment with actual cell counts and enhanced segmentation quality. The distilled model maintains comparable performance while reducing parameter count by a factor of 48. By reducing computational complexity and integrating into workflows, this approach may significantly impact diagnostics, reduce pathologist workload, and improve outcomes. Although the method shows promise, extensive validation is necessary prior to clinical deployment.
\end{abstract}
\clearpage

\section{Introduction}
In digital pathology, accurate segmentation and classification of cells are crucial for many diagnostic, prognostic, and predictive analyses \cite{Jaber_Beziaeva_etal._2019,Lin_Pan_etal._2022,Park_Ock_etal._2022,Shen_Choi_etal._2024}. Nowadays, developments in computational pathology offer multiple solutions \cite{H._Qu_P._Wu_etal._2020,Javed_Mahmood_etal._2020} to utilize cell-level datasets to train machine learning models that solve these problems. The quality and specificity of training datasets are critical for robust and accurate models. Adhering to the principle of "garbage in, garbage out", it is essential to ensure that these datasets are extensively and accurately labeled with distinct classes that reflect the diverse biological characteristics of different cell types. Unfortunately, the number of open-source datasets comprising such high-quality annotations is limited. Existing cell segmentation datasets \cite{Gamper_Koohbanani_etal._2019,Graham_Vu_etal._2019,Verma_Kumar_etal._2021} may offer extensive annotations for certain cell types while providing more general labels for others. For example, in PanNuke, which is one of the largest open-source datasets comprising labeled cells, various types of morphologically and functionally different inflammatory cells like macrophages and lymphocytes are clustered in a broad "inflammatory" class. Consequently, these classes are frequently omitted from analyses or aggregated into broader meta-classes \cite{Gamper_Koohbanani_etal._2020} and likely interfere with other cell classes included in the dataset. This and similar inconsistencies in annotation granularity limit the ability of machine learning models to learn the comprehensive and nuanced features necessary for accurate cell segmentation and classification. To address these challenges, methods for refining and standardizing dataset annotations are essential to enhance the quality of training data.

A complementary approach to mitigate the absence of high-quality training data is the use of foundation models. Foundation models as encoders are defined as large-scale, versatile networks pre-trained on vast, diverse datasets using self-supervised learning, contrasting with convolutional neural network (CNN) pre-trained encoders that rely on supervised learning with labeled data. In practice, foundation models leverage enormous amounts of weakly or unlabeled data from millions of whole slide images (WSIs) and employ self-attention mechanisms to capture long-range dependencies and global context \cite{Chen_Ding_etal._2024,Saillard_Jenatton_etal._2024,Vorontsov_Bozkurt_etal._2024,Xu_Usuyama_etal._2024}. As a consequence, foundation models are able to produce transferable feature representations across different cell types and tissue environments. The feature representations can be leveraged by decoder networks to produce segmentation masks and pixel-level classifications. Because foundation models have comprehensive feature representations, they can be effectively fine-tuned using much smaller amounts of cell-level data compared to the large datasets needed to train models from scratch. Furthermore, foundation models incorporate adversarial training elements or contrastive learning \cite{Chen_Ding_etal._2024,Xu_Usuyama_etal._2024}, enhancing their resilience and adaptability by exposing them to challenging and varied scenarios during training. This may result in more generalizable models, often making them well-suited for diverse and complex tasks in digital pathology.

Despite the inherent advantages of foundation models, their deployment for practical use faces its own obstacles. In particular, they require substantial computational power, financial investments and rigorous testing to ensure reliability and efficacy for a given task \cite{Akkus_Dangott_etal._2022,Dragomir_Cocuz_etal._2022,Go_2022,Jafri_Farooqui_etal._2024}. Moreover, while foundation models enhance feature representation and performance, they depend on the quality of available annotations for decoder fine-tuning and, like any other model, cannot resolve existing inconsistencies or ambiguities in data labels. Therefore, there remains a critical need for solutions that address both data quality and practical deployment considerations.
Further, integrating new technologies into existing clinical workflows often encounters resistance, as it necessitates adjustments to established diagnostic processes. So, there is a need to develop solutions that could be integrated into current practices, minimizing the burden on medical professionals to adopt new tools \cite{King_Williams_etal._2023}.

Existing solutions \cite{Goldsborough_Philps_etal._2024,Hörst_Rempe_etal._2024}, while addressing some aspects of these challenges, fall short in providing a comprehensive approach. To address the data quality and clinical deployment issues, we propose a multi-faceted solution that encompasses data refinement, model optimization, and integration with existing pathology tools (\hyperref[fig:fig1]{Figure 1}). The outcome is a lightweight cell segmentation and classification model that can be integrated into digital pathology workflows for practical clinical use.

\begin{figure}[h!]
    \centering
    \includegraphics[width=\textwidth, height=0.82\textheight, keepaspectratio]{images/Figure_1.pdf}
    \caption{Overview of the proposed solution, including 1) Data refinement using cross-relabeling, 2) Teacher model development and fine tuning, 3) Student model optimization with knowledge distillation and 4) Student model and QuPath integration}
    \label{fig:fig1}
\end{figure}
\clearpage

Our approach begins with preparing the data for the fine-tuning and training of the machine learning models. We create a refined dataset, acquired via cross-relabeling two cell-level datasets, enhancing annotation specificity and consistency of the labeled data. Subsequently, we create a cell segmentation and classification model based on the foundation model. We leverage the foundation model as a fixed encoder and fine-tune a decoder using the refined dataset to improve generalization across diverse tissue- and cell types.
To ensure that the model remains lightweight and deployable in a possibly resource-constrained environment, we employ knowledge distillation to approximate the functionality of the foundation model. Finally, to facilitate the practical application of our model in digital pathology workflows, we integrate it with the QuPath \cite{Bankhead_Loughrey_etal._2017} application. Each methodological component contributes to the overarching goal of enhancing model performance, generalizability, and usability in clinical settings.

The primary contributions of this paper are:
\begin{enumerate}
    \item \textit{Data labels refinement through cross-relabeling:}
    
    We propose a new method for refining labels of cell-level datasets through cross-relabeling. This method employs classification models to re-label broad and ambiguous instances, resulting in a more diverse dataset. Our evaluation demonstrates that these classification models achieve high accuracy on test subsets, indicating the reliability of the method for label refinement.

    \item \textit{Enhanced model performance via foundation models:}
    
    We employ a foundation model as a feature extractor for the cell segmentation and classification task. In comparison with training a CNN model from scratch, the foundation model backbone only needs fine-tuning, which significantly reduces training time, computational resources and data requirements. We show that using a foundation model encoder leads to better performance in cell segmentation and classification networks than using a CNN-based encoder. This improvement may enable the model to generalize more effectively across various tissue types and imaging methods.
    
    \item \textit{Model optimization through knowledge distillation:}
    
    We show that a smaller student model trained using knowledge distillation on the refined dataset obtained via our cross-relabeling approach from a foundation model achieves comparable performance in cell segmentation and quantification tasks. As a result, this model is more suitable for deployment in environments without high-performance computing resources.
    
    \item \textit{Integration with QuPath:}
    
    We integrate the distilled cell segmentation and classification model into QuPath, a widely used open-source digital pathology platform, to accelerate clinical adaptation by enabling pathologists to more easily incorporate advanced computational tools into their existing workflows.
\end{enumerate}

Through these methodological steps, we aim to bridge the gap between advanced machine learning techniques and practical clinical applications, making accurate and efficient digital pathology accessible in a broader range of healthcare settings.

\section{Refining Existing Datasets Using Cross-Relabeling}
To address the limitations of sparse and ambiguous labeling of cell-level datasets, we propose a generalizable cross-relabeling strategy that can be applied to any dataset containing broadly categorized or imprecisely labeled cell types. This approach involves training and subsequently leveraging classification models to refine broad categories into more specific or biologically relevant classes.
When applied to cell-level data, the methodology includes extracting individual cell images from the dataset patches, preprocessing these images to standardize the size and accommodate partial cells, and then training deep learning classifiers capable of distinguishing between the finer cell subtypes within the coarser categories. 
To illustrate our approach, we focus on the PanNuke \cite{Gamper_Koohbanani_etal._2020, Gamper_Koohbanani_etal._2019} and MoNuSAC \cite{Verma_Kumar_etal._2021} datasets that we have used to train models for cell quantification in our previous works \cite{Shvetsov_Grønnesby_etal._2022,Shvetsov_Sildnes_etal._2024}. We find that for better cell differentiation we have to introduce more granular labels. PanNuke includes a broad classification of "inflammatory" cells, encompassing lymphocytes, macrophages, and neutrophils. Each cell type differs significantly in structure, function, and clinical relevance. Conversely, MoNuSAC uses the label "epithelial" for a class that comprises both benign epithelial cells and malignant neoplastic cells. This practice makes it challenging to differentiate between benign and malignant epithelial cells in the dataset, which is a critical distinction when identifying tumor areas within tissue samples. To address these issues, we implement a cross-relabeling strategy as shown in \hyperref[fig:fig2]{Figure 2}. The key components are two classification models: one is trained on singular cell images from PanNuke data to classify the epithelial meta-class into epithelial and neoplastic classes. The other is trained on MoNuSAC to refine the inflammatory class into lymphocytes, neutrophils, and macrophages.

\begin{figure}[h!]
    \centering
    \includegraphics[width=\textwidth]{images/Figure_2.pdf}
    \caption{Refined dataset generation via cross relabeling}
    \label{fig:fig2}
\end{figure}

The refining approach consists of three consecutive steps. The first is the preprocessing step, in which we extract individual cells from both datasets (\hyperref[fig:fig3]{Figure 3}). The specifics of PanNuke and MoNuSAC patch preparation before cell preprocessing are provided in \hyperref[chap:S1]{Appendix S1}.

\begin{figure}[h!]
    \centering
    \includegraphics[width=\textwidth]{images/Figure_3.pdf}
    \caption{Cell instances preprocessing including (1) cell map extraction, (2) bounding box delineation, (3) adjusting cell boxes and (4) cropping and resizing of cell images}
    \label{fig:fig3}
\end{figure}

During preprocessing, we extract cell type maps from the ground truth label mask and calculate bounding boxes around each cell instance. To accommodate partial cells at patch borders, a common issue in cropped patch images, we employ mirror padding and extend the field of view of the cell label by 15 pixels to capture adjacent cells. We then crop and resize the identified regions to $64 \times 64$ pixels using bicubic interpolation.

The preprocessed PanNuke dataset comprises 68,031 neoplastic and 23,207 epithelial cell images, while MoNuSAC comprises  33,104 lymphocytes, 1,252 neutrophils, and 1,695 macrophages, which we subsequently use in training cell classification models and classifying the cell image data \hyperref[fig:S2]{Appendix Figure S2 (1)}. 

The next step is to train two distinct ResNet50-based classifiers tailored to address the specific labeling challenges inherent in each dataset. We use ResNet50 for classification models due to its proven effectiveness for image classification tasks in histopathology \cite{pan2022reviewmachinelearningapproaches}, and its compatibility with small images. For the PanNuke dataset, we design the classifier, trained on MoNuSAC data, to disaggregate the heterogeneous "inflammatory" cell category into distinct subtypes: lymphocytes, macrophages, and neutrophils. Similarly, for the MoNuSAC dataset, the classifier is trained on PanNuke data and distinguishes between benign and malignant epithelial cells within the overarching "epithelial" label. By applying these targeted classifiers to their respective datasets, we assign more specific labels to individual cell instances, thus enabling us to create a unified dataset.
To ensure a balanced representation of classes, we train both models on datasets that had been equalized to match the size of the least represented class. Thus, we obtain datasets comprising 23,207 samples per class for PanNuke and 1,252 samples per class for MoNuSAC data. Next, we partition both of them into training (70\%), validation (20\%), and testing (10\%) subsets. To mitigate the risk of overfitting, we use a single dropout layer with a rate of p=0.5 in both models and data augmentation using randomized color perturbations, rotation, and horizontal and vertical flipping. We employ AdamW optimizer and the cross-entropy loss function for the training criterion.

To evaluate the two trained models, we measure the classification accuracy on the respective test subsets. The accuracies on the test subset for both classifiers are presented in \hyperref[tab:1]{Table 1}. The PanNuke model achieves an average accuracy of 93.57\%, with higher accuracy for neoplastic cells (96.06\%) compared to epithelial cells (86.26\%). The confusion matrix in Figure A3.1 shows that the model predominantly distinguishes accurately between epithelial and neoplastic tissues, with a substantial number of correct classifications and relatively few misclassifications. The MoNuSAC model demonstrates an average accuracy of 98.92\%, excelling in classifying lymphocytes (99.67\%) and macrophages (94.12\%), with lower performance for neutrophils (85.71\%). The confusion matrix in Figure A3.2 shows that the model identifies lymphocytes and performs reasonably well with macrophages and neutrophils.

\begin{table}[h!]
\renewcommand{\arraystretch}{1.5}
  \centering
  \caption{Cell classification results for PanNuke and MoNuSAC trained models (CI 95\%).}
  \label{tab:1}
  \begin{tabular}{|l|c|c|}
   \hline
   %\rowcolor{gray!30}
    Accuracy               & PanNuke model              & MoNuSAC model              \\
    \hline
    Average      & 0.936 (0.931--0.941)         & 0.989 (0.986--0.993)        \\
    \hline
    Neoplastic   & 0.961 (0.956--0.965)         & -                          \\
    \hline
    Epithelial   & 0.863 (0.849--0.877)         & -                          \\
    \hline
    Lymphocytes  & -                          & 0.997 (0.995--0.999)        \\
    \hline
    Neutrophils  & -                          & 0.857 (0.796--0.918)        \\
    \hline
    Macrophages  & -                          & 0.941 (0.906--0.976)        \\
    \hline
  \end{tabular}
\end{table}

Finally, during the last step, we use the model trained on PanNuke data for epithelial cells in MoNuSAC and the model trained on MoNuSAC for the inflammatory cells class in PanNuke. Specifically, we use classifier models to relabel epithelial cells in MoNuSAC and inflammatory cells in PanNuke data. Then we combine cells with refined labels and the rest of the cells in both datasets to create a refined dataset (\hyperref[fig:S2]{Appendix Figure S2 (2)}). The process of relabeling cells and visualizing them on a patch is shown in \hyperref[fig:fig4]{Figure 4}. The cell counts in the refined dataset are provided in \hyperref[tab:S4]{Appendix Table S4}.

\begin{figure}[h!]
    \centering
    \includegraphics[width=\textwidth, height=0.42\textheight, keepaspectratio]{images/Figure_4.pdf}
    \caption{Cell relabeling procedure for epithelial and inflammatory cell classes}
    \label{fig:fig4}
\end{figure}

%\hfill

Relabeling and combining datasets have been explored in a prior study \cite{Parulekar_Kanwat_etal._2023}, where consecutive fine-tuning on multiple datasets was employed to account for hierarchical class label structures. While the method presented in \cite{Parulekar_Kanwat_etal._2023} is intuitive, it often lacks consistency and requires multiple fine-tuning runs, which can be cumbersome and time-consuming. 
In contrast, cross-relabeling simplifies this process by using specialized classification models tailored to each dataset's specific labeling challenges. This approach provides better transparency and produces a unified dataset encompassing seven distinct cell types across multiple tissue samples, enhancing data diversity for further model training or fine-tuning.

Despite these improvements, cross-relabeling does not entirely resolve issues related to poor labeling quality or the amount of labeled data. Specifically, our results show lower accuracies persist for underrepresented classes, such as macrophages, which may stem from a limited sample availability and intrinsic challenges in distinguishing these cells based solely on H\&E staining. Furthermore, while our method enhances label specificity, it relies on the initial quality of the broad labels; thus, any fundamental inaccuracies in the original annotations can propagate through the relabeling process. Addressing the overall problem of limited data labels may require integrating additional data sources or utilizing complementary immunohistochemical staining methods.
Although the reported performance metrics are obtained from evaluations on the native test sets of each dataset, it is important to note that the primary application of these classifiers is to perform cross-relabeling, where a model trained on one dataset (e.g., PanNuke) is applied to another (e.g., MoNuSAC) and vice versa. We acknowledge that a more systematic evaluation of cross-dataset generalization is needed and could be performed in future work.

Overall, the refined dataset produced by our approach can enhance the supervised training or fine-tuning of cell segmentation and classification models, especially those that utilize pre-trained foundation models to improve feature extraction robustness. In addition, these models can detect nuanced classes that enable researchers to conduct more detailed analyses of biological processes in computational pathology.

\section{Foundation models for robust cell segmentation and classification}

Accurate cell segmentation and classification in digital pathology are hindered by limited labeled data and the fact that conventional CNNs are unable to capture global contextual information due to their local receptive field constraints \cite{Gheflati_Rivaz_2022,Yang_Marcus_etal.}. Traditional approaches in cell quantification have predominantly relied on CNN encoders, such as ResNet50, given their proven effectiveness in semantic segmentation tasks \cite{Deshmane_2023,Graham_Vu_etal._2019,Mukasheva_Koishiyeva_etal._2024,Stringer_Wang_etal._2021}. However, approaches that include fine-tuning of pretrained CNNs, data augmentation, and stain normalization to partially increase data variability and address staining differences often fail to achieve the necessary generalization and robustness across diverse tissue types and staining conditions \cite{G._Wang_W._Li_etal._2018,Gao_Bagci_etal._2018,Karim_El_Khoury_Martin_Fockedey_etal._2021}.

To overcome these challenges, we leverage an encoder-decoder network that uses a foundation model as the encoder and a CNN upsampling decoder (\hyperref[fig:fig5]{Figure 5}) for simultaneous cell segmentation and classification in 2D patches extracted from WSIs. Foundation models with transformer-based architectures are viable alternatives to CNN-based encoders \cite{Shamshad_Khan_etal._2023,Sourget_2023}. They enable the creation of more advanced architectures that can decode or transform learned features more effectively \cite{Chen_Duan_etal._2023,Cheng_Misra_etal._2022,Xie_Wang_etal._2021}.

\begin{figure}[h!]
    \centering
    \includegraphics[width=\textwidth]{images/Figure_5.pdf}
    \caption{UNETR-like model with foundational model as backbone}
    \label{fig:fig5}
\end{figure}

By utilizing a transformer-based encoder, we incorporate global contextual information into the feature extraction process, which is a key advantage of such architectures \cite{Chen_Lu_etal._2021}. This foundation model integration facilitates accurate pixel-wise segmentation and classification without the need for extensive encoder training, thereby potentially improving generalization across varied cellular structures and tissue types.
In our implementation, we employ a modified UNETR \cite{Hatamizadeh_Tang_etal._2021} architecture that combines a vision transformer (ViT) \cite{Dosovitskiy_Beyer_etal._2021} encoder with a CNN-based decoder. The encoder utilizes the pretrained H-Optimus foundation model, which contains 1.1 billion parameters and is trained on over 500,000 H\&E stained WSIs \cite{Saillard_Jenatton_etal._2024}. We extract outputs from four evenly spaced transformer blocks $Z_i$, where $i \in [1, 14, 26, 38]$, to serve as residual connections for the CNN decoder. We select these blocks based on our observation that features from non-adjacent levels of the encoder lead to better overall performance on the test subset.

The CNN decoder upsamples the feature representations, acquired from the transformer blocks, to generate an intermediate vector that is handled by two task-specific layers that generate cell segmentation and classification masks. The first task-specific layer is the ‘Cellpose head’,  which is used to delineate cell instances. The layer generates horizontal and vertical gradient maps to form vector fields that are refined through gradient tracking in a post-processing step using the Cellpose algorithm \cite{Stringer_Wang_etal._2021}, known for its efficacy in cell segmentation tasks and generalizability across multiple domains \cite{Pachitariu_Stringer_2022,Stringer_Pachitariu_2024}. The second task-specific layer is the "Cell type head", which assigns labels to individual pixels. In the post-processing step, we determine the output classification label of each segmented cell instance by majority voting over the labeled pixels that comprise the cell in the segmentation map.

To evaluate model performance and measure the impact of adding a foundation model as backbone, we compare it to a ResNet50-based model. ResNet50 is a widely used solution for encoders in segmentation architectures in the medical domain \cite{Deshmane_2023,Graham_Vu_etal._2019,Mukasheva_Koishiyeva_etal._2024,Stringer_Wang_etal._2021}. For the H-Optimus-based model, we utilize frozen weights for the encoder and only fine-tune the decoder to take advantage of the extensive pre-training of the foundation model. For the ResNet50-based model we start with ImageNet \cite{Deng_Dong_etal.} weights and train both encoder and decoder parts. Hyperparameters for the training step are set to be identical, where possible, for comparable evaluation. 
For this evaluation, we deliberately use the PanNuke dataset to provide a standardized and controlled comparison between the H‑Optimus and ResNet50-based models (\hyperref[fig:S2]{Appendix Figure S2 (3)}). Specifically, we use two of the default PanNuke dataset splits (66\%) for training and validation, and reserve the third split (33\%) for testing.

To address the challenge of cell class imbalance in the PanNuke dataset, which is a common characteristic in most cell-level H\&E patch datasets, both models’ training processes employ a weighted loss function comprising cross-entropy and focal loss \cite{Lin_Goyal_etal._2018}. The focal loss component is adjusted with coefficients derived from each cell class' instance frequency, emphasizing learning from underrepresented classes and enhancing the model's sensitivity to rare but significant cellular patterns. The cross-entropy loss is augmented with spectral decoupling regularization \cite{Pezeshki_Kaba_etal._2021,Pohjonen_Stürenberg_etal._2022} and spatially varying label smoothing \cite{Islam_Glocker_2021}, which potentially stabilizes training and improves generalization in case of complex tissue morphologies. For optimization, we employ the \textit{AdamW} \cite{Loshchilov_Hutter_2019} to counter unbalanced class scenarios, with cosine annealing learning rate scheduler.

We utilize the scikit-learn library \cite{Van_der_Walt_Schönberger_etal._2014} and HoVer-Net \cite{Graham_Vu_etal._2019} implementations of $R^2$ (the coefficient of determination) and $PQ$ (panoptic quality) to evaluate our experiments. Complete mathematical formulations and detailed explanations of these metrics are provided in \hyperref[chap:S5]{Appendix S5}. To compute confidence intervals, we use nonparametric bootstrapping, where after calculating the metric on the full sample, we generated 1000 bootstrap replicates by resampling with replacement and then determined the 95\% confidence intervals as the 2.5th and 97.5th percentiles of the resulting empirical distribution.

%\hfill

The model comparisons are summarized in \hyperref[tab:2]{Table 2}. The H‑Optimus-based model achieves higher $R^2$ across all cell classes compared to the ResNet50-based model, which means that its predictions are more closely aligned with the PanNuke cell counts, indicating a stronger correlation with the observed data. Notably, the improvement of $R^2_{dead}$ may be an indicator of better global contextual representations provided by the foundation model backbone. In terms of segmentation and classification quality combined, measured by the PQ score, the H‑Optimus-based model demonstrates notable improvements across most cell classes. Overall, the average $R^2$ improved from 0.575 to 0.871, while the average $PQ$ score improved from 0.450 to 0.492, demonstrating better performance of the H-Optimus-based model.

\begin{table}[h!]
\renewcommand{\arraystretch}{1.5}
  \centering
  \caption{Cell quantification metrics for baseline and proposed models (CI 95\%).}
  \label{tab:2}
  \begin{tabular}{|l|c|c|}
    \hline
    %\rowcolor{gray!30}
    Metric             & Resnet50-based            & H-optimus-based              \\
    \hline
    $R^2_{neoplastic}$    & 0.681 (0.576--0.769)       & \textbf{0.941 (0.917--0.960)} \\
    \hline
    $R^2_{inflammatory}$  & 0.863 (0.778--0.903)       & \textbf{0.949 (0.918--0.966)} \\
    \hline
    $R^2_{connective}$    & 0.600 (0.488--0.698)       & 0.609 (0.436--0.772)          \\
    \hline
    $R^2_{dead}$          & 0.097 (-11.389--0.669)     & 0.925 (0.404--0.982)          \\
    \hline
    $R^2_{epithelial}$    & 0.635 (0.490--0.747)       & \textbf{0.930 (0.886--0.964)} \\
    \hline
    $PQ_{neoplastic}$       & 0.517 (0.499--0.535)       & \textbf{0.589 (0.575--0.604)} \\
    \hline
    $PQ_{inflammatory}$     & 0.455 (0.429--0.482)       & \textbf{0.528 (0.507--0.549)} \\
    \hline
    $PQ_{connective}$       & 0.416 (0.400--0.431)       & \textbf{0.451 (0.436--0.465)} \\
    \hline
    $PQ_{dead}$             & 0.374 (0.342--0.408)       & 0.292 (0.209--0.365)          \\
    \hline
    $PQ_{epithelial}$       & 0.488 (0.460--0.519)       & \textbf{0.599 (0.579--0.618)} \\
    \hline
  \end{tabular}
\end{table}

Our results  show that integrating the H‑Optimus foundation model within the UNETR architecture enhances the model's ability to segment and classify cells across diverse tissues from PanNuke data. The pretrained transformer encoder provides robust feature representations, resulting in higher average $R^2$ and $PQ$ scores compared to the CNN-based model. This leads to more reliable cell quantification and more accurate downstream analysis. Additionally, the streamlined fine-tuning process reduces computational overhead and training time, making the model more adaptable for new data.

Despite these advancements, the foundation model-based approach does not fully resolve all challenges related to cell segmentation and classification. We observe lower metric scores for underrepresented classes in the training data. Furthermore, foundation models typically encompass billions of parameters, resulting in substantial computational and memory requirements. It therefore poses challenges for deployment in resource-constrained environments, limiting their practical applicability in certain clinical settings.

\section{Model optimization via Knowledge Distillation}

To address the limitations posed by the extensive size of foundation models, we implement knowledge distillation — a model compression technique that leverages the teacher-student paradigm \cite{Hinton_Vinyals_etal._2015}. By training a smaller, more efficient student model to replicate the output of a larger, pre-trained teacher model, we retain performance while significantly reducing the model's complexity and resource requirements (\hyperref[fig:fig6]{Figure 6}).

\begin{figure}[h!]
    \centering
    \includegraphics[width=\textwidth, height=0.45\textheight, keepaspectratio]{images/Figure_6.pdf}
    \caption{Knowledge distillation framework for training a student model using a pre-trained teacher}
    \label{fig:fig6}
\end{figure}

We employ knowledge distillation to compress the H‑Optimus-based teacher model into a more efficient student model. The teacher model is the modified UNETR architecture with the H‑Optimus foundation model described in the previous chapter. The student model is based on a UNet architecture augmented with residual connections and incorporates a smaller ViT encoder with 9 million parameters \cite{Steiner_Kolesnikov_etal._2022,Wightman_2019}. 

First, we fine-tune the teacher model using the refined dataset from the cross-relabeling procedure (Section 2). Initially we train the decoder of the teacher model while keeping the encoder weights frozen. We split the refined dataset into train (70\%), validation (20\%) and test (10\%) subsets (\hyperref[fig:S2]{Appendix Figure S2 (4)}). During fine-tuning, we use the train and validation subsets, while leaving the test subset for model evaluation. We set the training procedure and model hyperparameters to be identical to those that were used to demonstrate the utility of foundation models for the simultaneous cell segmentation and classification task.

Next, we perform knowledge distillation from teacher to student using the refined dataset used to fine-tune the teacher model. The student model is trained to replicate the teacher model's outputs. We utilize a specialized loss function that aligns the student's predicted probability distribution with the teacher's, incorporating the teacher's class probability distribution derived from the output. Following the methodology of Hinton et al. \cite{Hinton_Vinyals_etal._2015}, we experiment with various hyperparameter settings for the temperature ($T$) and the balancing coefficients ($\alpha$ and $\beta$) in the loss function. We vary $T$ from 1 to 20 and adjust $\alpha$ and $\beta$ to balance the distillation and student losses. Through iterative tuning and evaluation, we identify that setting $T=14$, $\alpha=0.3$, and $\beta=0.7$ yields a configuration that converges and closely approximates the teacher model's performance during training.

Finally, we assess the performance of both models using the $R^2$ and $PQ$ (defined in \hyperref[chap:S5]{Appendix S5}) on the test set of the refined dataset (\hyperref[tab:3]{Table 3}). We observe that the 95\% confidence intervals overlap for most cell types, so we cannot claim statistically significant performance differences between the teacher and student models. One exception appears in the neoplastic class. The teacher model produces an $R^2$ of 0.919, while the student model shows an $R^2$ of 0.852. In addition, the student model achieves higher $PQ$ values for the neoplastic and connective classes, though the confidence intervals show overlap.

\begin{table}[h!]
\renewcommand{\arraystretch}{1.5}
  \centering
  \caption{Cell quantification metrics for teacher and distilled student models (CI 95\%).}
  \label{tab:3}
  \begin{tabular}{|l|c|c|}
    \hline
    %\rowcolor{gray!30}
    Metric & Teacher & Student \\
    \hline
    $R^2_{neoplastic}$    & \textbf{0.919} (0.898--0.939) & 0.852 (0.800--0.891) \\
    \hline
    $R^2_{lymphocyte}$    & 0.969 (0.956--0.977)         & 0.969 (0.956--0.978) \\
    \hline
    $R^2_{connective}$    & 0.694 (0.548--0.809)         & 0.618 (0.469--0.741) \\
    \hline
    $R^2_{dead}$          & 0.755 (0.400--0.908)         & 0.424 (0.100--0.731) \\
    \hline
    $R^2_{epithelial}$    & 0.922 (0.870--0.958)         & 0.843 (0.738--0.917) \\
    \hline
    $R^2_{macrophage}$    & 0.384 (-0.369--0.724)        & 0.704 (0.352--0.859) \\
    \hline
    $R^2_{neutrofil}$     & 0.854 (0.578--0.929)         & 0.833 (0.502--0.925) \\
    \hline
    $PQ_{neoplastic}$       & 0.581 (0.569--0.593)         & 0.601 (0.588--0.613) \\
    \hline
    $PQ_{lymphocyte}$       & 0.536 (0.520--0.553)         & 0.563 (0.544--0.579) \\
    \hline
    $PQ_{connective}$       & 0.436 (0.421--0.451)         & 0.457 (0.441--0.474) \\
    \hline
    $PQ_{dead}$             & 0.272 (0.235--0.315)         & 0.279 (0.201--0.369) \\
    \hline
    $PQ_{epithelial}$       & 0.522 (0.500--0.545)         & 0.530 (0.506--0.555) \\
    \hline
    $PQ_{macrophage}$       & 0.524 (0.459--0.588)         & 0.474 (0.405--0.543) \\
    \hline
    $PQ_{neutrofil}$        & 0.541 (0.490--0.592)         & 0.565 (0.522--0.607) \\
    \hline
  \end{tabular}
\end{table}


We further decompose the $PQ$ metric into its $SQ$ and $DQ$ components (\hyperref[tab:S6]{Appendix Table S6}). Both models produce nearly identical $SQ$ values, which indicates that they predict instance boundaries with similar precision. Although the student model shows some improvement in $DQ$ scores for certain classes, the confidence intervals overlap and do not confirm a statistically significant difference.

We observe that the student and teacher models yield comparable detection performance despite the student model using a much smaller and simpler architecture. A model with fewer parameters reduces the risk of overfitting when training data are scarce relative to the model’s complexity \cite{Farias_Ludermir_etal._2022}. The knowledge distillation process also encourages the student model to focus on the most generalizable detection features learned from the teacher. These factors enable the student model to achieve similar detection performance across different cell types.

Additionally, considering the model sizes reported in \hyperref[tab:4]{Table 4}, the distilled model achieves a significant reduction compared to the teacher model, with a 48-fold decrease in parameter count and a 5.5-fold reduction in on-disk size. In inference mode, the teacher model requires 16 GB of VRAM for a batch size of 32, while the distilled model only needs 3 GB of VRAM for the same batch size. These reductions make the distilled model significantly more practical for fine-tuning and deployment in resource-constrained environments.

\begin{table}[h!]
\renewcommand{\arraystretch}{1.5}
  \centering
  \caption{Parameter counts and size of teacher and distilled model}
  \label{tab:4}
  \adjustbox{max width=\textwidth}{%
  \begin{tabular}{|l|c|c|c|}
    \hline
    %\rowcolor{gray!30}
    Metric & H-optimus-based (Teacher) & mobileViT-based (Student) & Magnitude of difference \\
    \hline
    Parameters count       & 1,158,917,906   & \textbf{24,093,393}   & \textbf{48x}  \\
    \hline
    Estimated Total Size (MB) & 87,912       & \textbf{15,935}    & \textbf{5.5x} \\
    \hline
  \end{tabular}%
}
\end{table}

%\hfill

With recent advancements in complex network architectures and the use of pretrained encoders to achieve state-of-the-art performance \cite{Baumann_Dislich_etal._2024,Hörst_Rempe_etal._2024} in cell segmentation and classification tasks, model size, computational complexity, and processing times have increased. This limits the scalability and accessibility of these models. As we demonstrate, this may be mitigated using knowledge distillation. Studies in the field of natural language processing have demonstrated the efficacy of knowledge distillation in retaining the capabilities of the teacher model while achieving significant reductions in size and complexity \cite{Huangpu_Gao_2024,Sun_Yu_etal.}. 

We demonstrate the feasibility of knowledge distillation in digital pathology, specifically for cell segmentation and classification tasks. Moreover, we achieve this performance while also significantly reducing the parameter count. In addressing the challenge of knowledge transfer, we found that distillation from a transformer-based model to a smaller transformer is more straightforward than attempting to map transformer features to CNN blocks. In our experiments, using a CNN-based network as a student results in worse cell quantification performance due to the structural constraints of CNN feature space dimensions. 

Although our primary approach relies on a transformer-based student model that performs well, it can be further optimized to incorporate advantages from CNN architectures. For example, employing alternative techniques such as using ViT adapters \cite{Chen_Duan_etal._2023} or $1 \times 1$ convolutions to adjust feature map sizes may be beneficial for harnessing CNN advantages like enhanced local feature extraction. Moreover, if additional performance improvements are desired, the process can be further enhanced by applying supplementary knowledge distillation techniques, such as self-distillation \cite{Zhang_Song_etal._2019} or online distillation \cite{Houyon_Cioppa_etal._2023}.

Despite these promising results, further validation on independent datasets is necessary to fully understand the model's limitations. Underrepresented classes may pose challenges when addressing complex cases. Pathologists need to validate these models to adopt them in clinical settings. While the distilled models are smaller and more deployable, a technological gap persists because pathologists traditionally rely on established methods for inspecting WSIs and diagnosing diseases. Addressing the complexities involved in deploying models for inference and supporting pathologists in adopting new tools is essential for integrating these models into clinical workflows.

\section{Model integration with QuPath}
Digital pathology tools with graphical user interfaces are essential for visualizing and analyzing WSIs. To make our student model useful in clinical pathology workflows, it needs to be integrated into a tool that enables inspecting regions, creating annotations, and providing quantitative analyses of biomarkers. Therefore, we integrate the trained student model from the previous chapter into the QuPath open‑source platform \cite{Bankhead_Loughrey_etal._2017}. QuPath provides the required annotation, visualization, and analysis tools to interpret complex histological data, including workflows for cell segmentation, classification, and quantification (\hyperref[fig:fig7]{Figure 7}). 

\begin{figure}[h!]
    \centering
    \includegraphics[width=\textwidth]{images/Figure_7.pdf}
    \caption{Visualization of model-generated cell quantification annotations (left) and the corresponding unannotated slide (right) in QuPath}
    \label{fig:fig7}
\end{figure}

To identify the regions in a WSI critical for prognosticating tumor development, such as specific tumor areas or border regions without overlapping healthy tissue, the pathologist uses QuPath to outline these regions. Then, the pathologist initiates a cell segmentation and classification script through the QuPath interface for the selected regions. The resulting annotations and quantified cell information are then directly overlaid onto the WSI in the QuPath interface. Additional design and implementation details are in \hyperref[chap:S7]{Appendix S7}. 

Two common approaches for integrating deep learning models into QuPath are Java‑based native QuPath extensions \cite{Goldsborough_Philps_etal._2024} and the execution of RESTful API requests to a model server coupled with handling the response via an extension, as demonstrated in the application of cell segmentation models applied to immunofluorescence images \cite{Sugawara_2023}. While the community is actively working on these integration strategies, there is currently no universal solution that fully addresses all integration and performance requirements.

Extensions may offer better integration with QuPath, allowing slightly improved performance and more widespread usage of the built-in QuPath models, but they lack the flexibility to customize models and modify their behavior. For example, the newest version of QuPath includes models such as StarDist \cite{Weigert_Schmidt} and InstanSeg \cite{Goldsborough_Philps_etal._2024} that can perform cell segmentation. Both models pose limitations when applied to simultaneous cell segmentation and classification. StarDist performs well only on convex, round shapes by design, whereas some neoplastic, inflammatory, and connective cells exhibit complex and non-convex shapes. InstanSeg provides only semantic segmentation without assigning classes to the segmented cells.

%\hfill

In contrast, our approach offers an alternative integration strategy. It utilizes the paquo library to directly interact with QuPath’s internal application programming interface from within Python. This enables data exchange and processing without the need for intermediate conversion steps and provides greater control over model customization, retraining, and the incorporation of custom processing steps.

The integration of our custom model with QuPath underscores its potential to significantly enhance the diagnostic process by reducing the time burden on pathologists and enabling them to focus on more complex interpretative tasks using familiar software. Leveraging a tool that is already well-established among pathologists increases the likelihood of its adoption into daily clinical workflows. The quantitative data generated through the automated workflow is critical for both clinical decision-making and research, facilitating more accurate biomarker analysis, enabling robust statistical evaluations, and supporting hypothesis generation and testing. Additionally, by streamlining cell segmentation and classification, the tool enhances the scalability and reproducibility of pathological assessments, ultimately contributing to improved diagnostic accuracy and patient outcomes.

\section{Conclusion and future work}

In this study, we address critical challenges in digital pathology and tackle the usability and deployment issues of the developed models in standard computing environments without the need for high-performance computing systems. Our multi-faceted approach encompasses data refinement through cross-relabeling, leveraging foundation models for robust cell segmentation and classification, optimizing model performance via knowledge distillation, and integrating the optimized model into the QuPath software for practical application. This approach is used to construct a capable, versatile, and adjustable model for cell segmentation and classification, with enhanced performance and usability.

\begin{sloppypar}
While our approach shows potential in the field of computational pathology, certain limitations persist. 
For example, our implementation currently exhibits lower performance in detecting macrophages. 
This serves as an instance of the broader challenge of accurately identifying complex cell types. In order to address this issue, extending our approach to incorporate additional data sources, exploring alternative modeling approaches, and integrating other imaging modalities such as immunohistochemical staining may help improve detection accuracy. Moreover, although the distilled model reduces computational demands, integrating advanced deep learning models into clinical practice requires addressing technological gaps and potential resistance to adopting new tools within established diagnostic processes.
\end{sloppypar}

Future work could focus on several key areas to refine the proposed approach and facilitate its adoption in clinical environments. Enhancing the cell-relabeling process with additional datasets \cite{Graham_Jahanifar_etal._2021} could improve the representation of underrepresented cell types and enhance overall model performance. Also, incorporating additional data sources, such as multi-modal imaging or complementary staining methods, may address limitations related to cell type differentiation and class imbalance. Exploring other foundation models \cite{Vorontsov_Bozkurt_etal._2024,Zimmermann_Vorontsov_etal._2024} or introducing additional modalities \cite{Ding_Wagner_etal._2024,Vaidya_Zhang_etal._2025} may provide alternative architectures better suited to specific tasks or offer improved efficiency. Implementing more complex knowledge distillation techniques \cite{Houyon_Cioppa_etal._2023,Zhang_Song_etal._2019} could further optimize the model's performance and adaptability. Additionally, deeper integration with QuPath or other digital pathology software could provide pathologists more control over cell quantification analysis directly within the QuPath interface, thereby increasing accessibility and usability. Such enhancements would not only refine model performance but also ensure greater adaptability and scalability within various clinical environments. Finally, extensive validation of the model by pathologists and benchmarking against independent datasets are essential steps toward establishing the model's reliability and fostering confidence in its clinical utility.

\section*{Acknowledgments} 
This work was funded in part by the Research Council of Norway grant no. 309439 SFI Visual Intelligence, and the North Norwegian Health Authority grant no. HNF1521-20.

\bibliographystyle{IEEEtran}
\begin{sloppypar}
\begin{thebibliography}{99}

\bibitem{chaplot2020neural} Chaplot, Devendra Singh, et al. "Neural topological slam for visual navigation." Proceedings of the IEEE/CVF conference on computer vision and pattern recognition. 2020.

\bibitem{maksymets2021thda} Maksymets, Oleksandr, et al. "Thda: Treasure hunt data augmentation for semantic navigation." Proceedings of the IEEE/CVF International Conference on Computer Vision. 2021.

\bibitem{mezghan2022memory} Mezghan, Lina, et al. "Memory-augmented reinforcement learning for image-goal navigation." 2022 IEEE/RSJ International Conference on Intelligent Robots and Systems (IROS). IEEE, 2022.

\bibitem{al2022zero} Al-Halah, Ziad, Santhosh Kumar Ramakrishnan, and Kristen Grauman. "Zero experience required: Plug \& play modular transfer learning for semantic visual navigation." Proceedings of the IEEE/CVF Conference on Computer Vision and Pattern Recognition. 2022.

\bibitem{ye2021auxiliary} Ye, Joel, et al. "Auxiliary tasks and exploration enable objectgoal navigation." Proceedings of the IEEE/CVF international conference on computer vision. 2021.

\bibitem{chaplot2020object} Chaplot, Devendra Singh, et al. "Object goal navigation using goal-oriented semantic exploration." Advances in Neural Information Processing Systems 33 (2020)

\bibitem{ramakrishnan2022poni} Ramakrishnan, Santhosh Kumar, et al. "Poni: Potential functions for objectgoal navigation with interaction-free learning." Proceedings of the IEEE/CVF Conference on Computer Vision and Pattern Recognition. 2022.

\bibitem{ramrakhya2022habitat} Ramrakhya, Ram, et al. "Habitat-web: Learning embodied object-search strategies from human demonstrations at scale." Proceedings of the IEEE/CVF Conference on Computer Vision and Pattern Recognition. 2022.

\bibitem{mousavian2019visual} Mousavian, Arsalan, et al. "Visual representations for semantic target driven navigation." 2019 International Conference on Robotics and Automation (ICRA). IEEE, 2019.

\bibitem{dhariwal2021diffusion} Dhariwal, Prafulla, and Alexander Nichol. "Diffusion models beat gans on image synthesis." Advances in neural information processing systems 34 (2021)

\bibitem{ho2022classifier} Ho, Jonathan, and Tim Salimans. "Classifier-free diffusion guidance." arXiv preprint arXiv:2207.12598 (2022).

\bibitem{nichol2021glide} Nichol, Alex, et al. "Glide: Towards photorealistic image generation and editing with text-guided diffusion models." arXiv preprint arXiv:2112.10741 (2021)

\bibitem{brooks2023instructpix2pix} Brooks, Tim, Aleksander Holynski, and Alexei A. Efros. "Instructpix2pix: Learning to follow image editing instructions." Proceedings of the IEEE/CVF Conference on Computer Vision and Pattern Recognition. 2023.

\bibitem{fu2023guiding} Fu, Tsu-Jui, et al. "Guiding instruction-based image editing via multimodal large language models." arXiv preprint arXiv:2309.17102 (2023).

\bibitem{geng2024instructdiffusion} Geng, Zigang, et al. "Instructdiffusion: A generalist modeling interface for vision tasks." Proceedings of the IEEE/CVF Conference on Computer Vision and Pattern Recognition. 2024.

\bibitem{zhou2024minedreamer} Zhou, Enshen, et al. "Minedreamer: Learning to follow instructions via chain-of-imagination for simulated-world control." arXiv preprint arXiv:2403.12037 (2024).

\bibitem{zhou2023esc} Zhou, Kaiwen, et al. "Esc: Exploration with soft commonsense constraints for zero-shot object navigation." International Conference on Machine Learning. PMLR, 2023.

\bibitem{yu2023l3mvn} Yu, Bangguo, Hamidreza Kasaei, and Ming Cao. "L3mvn: Leveraging large language models for visual target navigation." 2023 IEEE/RSJ International Conference on Intelligent Robots and Systems (IROS). IEEE, 2023.

\bibitem{gadre2023cows} Gadre, Samir Yitzhak, et al. "Cows on pasture: Baselines and benchmarks for language-driven zero-shot object navigation." Proceedings of the IEEE/CVF Conference on Computer Vision and Pattern Recognition. 2023.

\bibitem{shah2023navigation} Shah, Dhruv, et al. "Navigation with large language models: Semantic guesswork as a heuristic for planning." Conference on Robot Learning. PMLR, 2023.

\bibitem{cai2024bridging} Cai, Wenzhe, et al. "Bridging zero-shot object navigation and foundation models through pixel-guided navigation skill." 2024 IEEE International Conference on Robotics and Automation (ICRA). IEEE, 2024.

\bibitem{yu2023co} Yu, Bangguo, Hamidreza Kasaei, and Ming Cao. "Co-NavGPT: Multi-robot cooperative visual semantic navigation using large language models." arXiv preprint arXiv:2310.07937 (2023).

\bibitem{wu2024voronav} Wu, Pengying, et al. "Voronav: Voronoi-based zero-shot object navigation with large language model." arXiv preprint arXiv:2401.02695 (2024).

\bibitem{qin2023mp5} Qin, Yiran, et al. "Mp5: A multi-modal open-ended embodied system in minecraft via active perception." arXiv preprint arXiv:2312.07472 (2023).

\bibitem{du2024learning} Du, Yilun, et al. "Learning universal policies via text-guided video generation." Advances in Neural Information Processing Systems 36 (2024).

\bibitem{ajay2024compositional} Ajay, Anurag, et al. "Compositional foundation models for hierarchical planning." Advances in Neural Information Processing Systems 36 (2024).

\bibitem{liang2024skilldiffuser} Liang, Zhixuan, et al. "Skilldiffuser: Interpretable hierarchical planning via skill abstractions in diffusion-based task execution." Proceedings of the IEEE/CVF Conference on Computer Vision and Pattern Recognition. 2024.

\bibitem{heusel2017gans} Heusel, Martin, et al. "Gans trained by a two time-scale update rule converge to a local nash equilibrium." Advances in neural information processing systems 30 (2017).

\bibitem{zhang2018unreasonable} Zhang, Richard, et al. "The unreasonable effectiveness of deep features as a perceptual metric." Proceedings of the IEEE conference on computer vision and pattern recognition. 2018.

\bibitem{brown2020language} Brown, Tom B. "Language models are few-shot learners." arXiv preprint arXiv:2005.14165 (2020).

\bibitem{podell2023sdxl} Podell, Dustin, et al. "Sdxl: Improving latent diffusion models for high-resolution image synthesis." arXiv preprint arXiv:2307.01952 (2023).

\bibitem{brohan2022rt} Brohan, Anthony, et al. "Rt-1: Robotics transformer for real-world control at scale." arXiv preprint arXiv:2212.06817 (2022).

\bibitem{brohan2023rt} Brohan, Anthony, et al. "Rt-2: Vision-language-action models transfer web knowledge to robotic control." arXiv preprint arXiv:2307.15818 (2023).

\bibitem{li2024manipllm} Li, Xiaoqi, et al. "Manipllm: Embodied multimodal large language model for object-centric robotic manipulation." Proceedings of the IEEE/CVF Conference on Computer Vision and Pattern Recognition. 2024.

\bibitem{shah2023vint} Shah, Dhruv, et al. "ViNT: A foundation model for visual navigation." arXiv preprint arXiv:2306.14846 (2023).

\bibitem{liu2024visual} Liu, Haotian, et al. "Visual instruction tuning." Advances in neural information processing systems 36 (2024).

\bibitem{hu2021lora} Hu, Edward J., et al. "Lora: Low-rank adaptation of large language models." arXiv preprint arXiv:2106.09685 (2021).

\bibitem{qin2023supfusion} Qin, Yiran, et al. "SupFusion: Supervised LiDAR-camera fusion for 3D object detection." Proceedings of the IEEE/CVF International Conference on Computer Vision. 2023.

\bibitem{qin2024worldsimbench} Qin, Yiran, et al. "Worldsimbench: Towards video generation models as world simulators." arXiv preprint arXiv:2410.18072 (2024).

\bibitem{yu2025gamefactory} Yu, Jiwen, et al. "GameFactory: Creating New Games with Generative Interactive Videos." arXiv preprint arXiv:2501.08325 (2025).

\bibitem{zhou2024code} Zhou, Enshen, et al. "Code-as-Monitor: Constraint-aware Visual Programming for Reactive and Proactive Robotic Failure Detection." arXiv preprint arXiv:2412.04455 (2024).

\bibitem{zhang2024ad} Zhang, Zaibin, et al. "AD-H: Autonomous Driving with Hierarchical Agents." arXiv preprint arXiv:2406.03474 (2024).

\bibitem{wang2024toward} Wang, Chaoqun, et al. "Toward Accurate Camera-based 3D Object Detection via Cascade Depth Estimation and Calibration." arXiv preprint arXiv:2402.04883 (2024).

\bibitem{huang2024story3d} Huang, Yuzhou, et al. "Story3d-agent: Exploring 3d storytelling visualization with large language models." arXiv preprint arXiv:2408.11801 (2024).

\bibitem{savinov2018semi} Savinov, Nikolay, Alexey Dosovitskiy, and Vladlen Koltun. "Semi-parametric topological memory for navigation." arXiv preprint arXiv:1803.00653 (2018).

\bibitem{majumdar2022zson} Majumdar, Arjun, et al. "Zson: Zero-shot object-goal navigation using multimodal goal embeddings." Advances in Neural Information Processing Systems 35 (2022): 32340-32352.

\bibitem{yadav2023offline} Yadav, Karmesh, et al. "Offline visual representation learning for embodied navigation." Workshop on Reincarnating Reinforcement Learning at ICLR 2023. 2023.

\bibitem{yadav2023ovrl} Yadav, Karmesh, et al. "Ovrl-v2: A simple state-of-art baseline for imagenav and objectnav." arXiv preprint arXiv:2303.07798 (2023).

\bibitem{sun2024fgprompt} Sun, Xinyu, et al. "FGPrompt: fine-grained goal prompting for image-goal navigation." Advances in Neural Information Processing Systems 36 (2024).

\bibitem{zhu2017target} Zhu, Yuke, et al. "Target-driven visual navigation in indoor scenes using deep reinforcement learning." 2017 IEEE international conference on robotics and automation (ICRA). IEEE, 2017.

\bibitem{koh2024generating} Koh, Jing Yu, Daniel Fried, and Russ R. Salakhutdinov. "Generating images with multimodal language models." Advances in Neural Information Processing Systems 36 (2024).

\bibitem{krantz2022instance} Krantz, Jacob, et al. "Instance-specific image goal navigation: Training embodied agents to find object instances." arXiv preprint arXiv:2211.15876 (2022).

\bibitem{schulman2017proximal} Schulman, John, et al. "Proximal policy optimization algorithms." arXiv preprint arXiv:1707.06347 (2017).

\bibitem{anderson2018evaluation} Anderson, Peter, et al. "On evaluation of embodied navigation agents." arXiv preprint arXiv:1807.06757 (2018).

\bibitem{lin2024navcot} Lin, Bingqian, et al. "NavCoT: Boosting LLM-Based Vision-and-Language Navigation via Learning Disentangled Reasoning." arXiv preprint arXiv:2403.07376 (2024).

\bibitem{NavGPT} Zhou, Gengze, Yicong Hong, and Qi Wu. "Navgpt: Explicit reasoning in vision-and-language navigation with large language models." Proceedings of the AAAI Conference on Artificial Intelligence.

\bibitem{hahn2021no} Hahn, Meera, et al. "No rl, no simulation: Learning to navigate without navigating." Advances in Neural Information Processing Systems 34 (2021): 26661-26673.

\bibitem{li2025t2isafety} Li, Lijun, et al. "T2ISafety: Benchmark for Assessing Fairness, Toxicity, and Privacy in Image Generation." arXiv preprint arXiv:2501.12612 (2025).

\bibitem{an2024agfsync} An, Jingkun, et al. "AGFSync: Leveraging AI-Generated Feedback for Preference Optimization in Text-to-Image Generation." arXiv preprint arXiv:2403.13352 (2024).


\end{thebibliography}
\end{sloppypar}

\clearpage
\beginsupplement
\section*{Appendix}
\renewcommand{\thesubsection}{S\arabic{subsection}}

\subsection{\label{chap:S1}PanNuke and MoNuSAC preprocessing}
The PanNuke dataset comprises a set of 7,901 RGB patches, each with dimensions of $256 \times 256$ pixels, which we set as the standard patch size for our analysis. In contrast, the MoNuSAC dataset encompasses 294 images of heterogeneous dimensions. To standardize the MoNuSAC images with our experiments, we implement a standardization protocol. Specifically, for images exceeding the dimensions of $256 \times 256$ pixels, we segment them into equal-sized patches and apply mirror padding to the remaining portions to avoid information loss at the peripherals. Patches with dimensions less than $128 \times 128$ pixels are excluded from the dataset due to the insufficient resolution to capture relevant cellular details. For patches where either dimension falls between 128 and 256 pixels, we employ upsampling to achieve the standard patch size. As a result, we obtain a total of 2,823 RGB patches derived from the MoNuSAC dataset for subsequent analysis. For additional details on the MoNuSAC data preparation process, refer to the source code \cite{Shvetsov_2025a}.
\clearpage

\subsection{\label{chap:S2}Data usage for the methodology}

\counterwithin{figure}{subsection}
\renewcommand{\thefigure}{S\arabic{subsection}}

\begin{figure}[h!]
    \centering
    \includegraphics[width=\textwidth, height=0.85\textheight, keepaspectratio]{images/A2.pdf}
    \caption{Overview of the methodology for cross-labeling, dataset refinement, and model comparison. (1) Cross-relabeling - training and testing cell classification models, (2) Cross-relabeling - using cell classification models to create refined dataset, (3) Fine-tuning and training models for comparison, (4) Student knowledge distillation with refined dataset}
    \label{fig:S2}
\end{figure}
\clearpage

\subsection{\label{chap:S3}Confusion matrices for classification models}
\counterwithin{figure}{subsection}
\renewcommand{\thefigure}{S\arabic{subsection}.\arabic{figure}}

\begin{figure}[h!]
    \centering
    \includegraphics[width=\textwidth, height=0.4\textheight, keepaspectratio]{images/A3_1.pdf}
    \caption{Confusion matrix for PanNuke trained model}
    \label{fig:S3.1}
\end{figure}

\begin{figure}[h!]
    \centering
    \includegraphics[width=\textwidth, height=0.4\textheight, keepaspectratio]{images/A3_2.pdf}
    \caption{Confusion matrix for MoNuSAC trained model}
    \label{fig:S3.2}
\end{figure}

\clearpage

\subsection{\label{chap:S4}Datasets cell counts}

\counterwithin{table}{subsection}
\renewcommand{\thetable}{S\arabic{subsection}}

\begin{table}[h!]
\renewcommand{\arraystretch}{2.0}
\centering
\caption{\label{tab:S4}Cell counts for PanNuke, MoNuSAC and refined datasets. Numbers in parentheses indicate preprocessed cell counts for cell classifier models training and testing.}
%\adjustbox{max width=\textwidth}{%
\begin{tabular}{|l|c|c|c|}
\hline
%\rowcolor{gray!30}
Cell type & PanNuke & MoNuSAC & Refined \\
\hline
Neoplastic & 77,403 (68,031) & - & 105,451 \\
\hline
Epithelial & 26,572 (23,207) & - & 29,926 \\
\hline
Epithelial (benign and malignant) & - & 31,402 & - \\
\hline
Inflammatory & 32,276 & - & - \\
\hline
Lymphocytes & - & 37,045 (33,104) & 65,275 \\
\hline
Neutrophils & - & 1,355 (1,252) & 3,833 \\
\hline
Macrophage & - & 1,842 (1,695) & 3,410 \\
\hline
Dead & 2,908 & - & 2,908 \\
\hline
Connective & 50,585 & - & 50,585 \\
\hline
\end{tabular}
%
%}
\end{table}



\clearpage

\subsection{\label{chap:S5}Definition of validation metrics}
\counterwithin{equation}{subsection}
\renewcommand{\theequation}{\arabic{equation}}

\subsubsection{\label{chap:S5.1}R\textsuperscript{2}}
The coefficient of determination, denoted as $R^2$, is a statistical measure that represents the proportion of variance in the dependent variable that is predictable from the independent variables. In the context of cell quantification in pathology, $R^2$ is used to assess how well the predicted quantities of different cell types in a patch align with the actual quantities observed in the ground truth data, with higher values representing more accurate quantification. $R^2$ is defined as
\begin{equation*}
R^2 = 1 - \frac{\sum_{i=1}^n (y_i - \hat{y}_i)^2}{\sum_{i=1}^n (y_i - \bar{y})^2},
\end{equation*}
where $y_i$ represents the actual number of cells of a specific type in the $i$-th image, $\hat{y}_i$ represents the predicted number of cells of that type in the $i$-th image, $\bar{y}$ is the mean of the actual numbers across all images, and $n$ is the total number of images in the dataset.

The $R^2$ metric has a range of $(-\infty, 1]$. An $R^2$ of 1 indicates perfect prediction, where all predicted values exactly match the actual values. An $R^2$ of 0 suggests that the model explains none of the variability of the response data around its mean. If $R^2$ is negative, it indicates that the model performs worse than a model that simply predicts the mean of the actual values for all observations.

\subsubsection{\label{chap:S5.2}PQ}
Panoptic Quality ($PQ$) is a comprehensive metric used to evaluate the performance of segmentation models in tasks that require both instance segmentation and classification. $PQ$ provides a single score that encapsulates both the detection accuracy (i.e., how many objects were correctly identified) and the segmentation quality (i.e., how accurately the objects' boundaries were delineated). This metric is particularly useful in multiclass scenarios where each pixel is classified into distinct categories, such as different cell types in pathology images.

$PQ$ is calculated as the product of two terms: Detection Quality ($DQ$) and Segmentation Quality ($SQ$). It can be expressed as
\begin{equation*}
PQ = DQ \cdot SQ,
\end{equation*}
where
\begin{equation*}
DQ = \frac{TP}{TP + 0.5\, FP + 0.5\, FN},
\end{equation*}
\begin{equation*}
SQ = \frac{\sum_{(p, g) \in \mathcal{M}} IoU(p, g)}{TP}.
\end{equation*}
In these formulas, $TP$ denotes the number of correctly matched instances between ground truth and prediction, $FP$ denotes the predicted instances that have no corresponding ground truth, $FN$ denotes the ground truth instances that were not detected, $IoU(p, g)$ is the Intersection over Union for a pair of matched instances $p$ (prediction) and $g$ (ground truth), and $\mathcal{M}$ is the set of matched pairs.

The $PQ$ metric is calculated for each class and is averaged across classes to provide a global performance measure.

The $PQ$ score has a range of $[0, 1.0]$, where a higher score indicates better performance in both detecting and segmenting the instances correctly. A $PQ$ of 1 signifies perfect identification and segmentation of all instances, whereas a $PQ$ of 0 indicates that no instances were correctly identified and segmented.

\clearpage

\subsection{\label{chap:S6}Segmentation and Detection quality metrics for teacher and student models}

\begin{table}[h!]
\renewcommand{\arraystretch}{2.0}
\centering
\caption{Segmentation and detection quality for student and teacher models (CI 95\%)}
\label{tab:S6}
%\adjustbox{max width=\textwidth}{%
\begin{tabular}{|l|c|c|}
\hline
%\rowcolor{gray!30}
Metric & Teacher & Student \\
\hline
$SQ_{neoplastic}$ & 0.819 (0.815--0.823) & 0.824 (0.819--0.828) \\
\hline
$SQ_{lymphocyte}$ & 0.795 (0.788--0.802) & 0.790 (0.783--0.796) \\
\hline
$SQ_{connective}$ & 0.770 (0.762--0.776) & 0.780 (0.772--0.786) \\
\hline
$SQ_{dead}$ & 0.659 (0.623--0.688) & 0.657 (0.624--0.695) \\
\hline
$SQ_{epithelial}$ & 0.780 (0.770--0.790) & 0.788 (0.779--0.797) \\
\hline
$SQ_{macrophage}$ & 0.788 (0.760--0.810) & 0.757 (0.730--0.783) \\
\hline
$SQ_{neutrofil}$ & 0.782 (0.761--0.801) & 0.775 (0.759--0.792) \\
\hline
$DQ_{neoplastic}$ & 0.706 (0.692--0.719) & 0.727 (0.712--0.741) \\
\hline
$DQ_{lymphocyte}$ & 0.675 (0.656--0.698) & 0.713 (0.691--0.734) \\
\hline
$DQ_{connective}$ & 0.566 (0.546--0.584) & 0.583 (0.565--0.602) \\
\hline
$DQ_{dead}$ & 0.410 (0.361--0.465) & 0.435 (0.306--0.561) \\
\hline
$DQ_{epithelial}$ & 0.668 (0.639--0.694) & 0.673 (0.644--0.702) \\
\hline
$DQ_{macrophage}$ & 0.657 (0.583--0.727) & 0.615 (0.531--0.703) \\
\hline
$DQ_{neutrofil}$ & 0.691 (0.625--0.753) & 0.729 (0.679--0.778) \\
\hline
\end{tabular}
%
%}
\end{table}

\clearpage

\subsection{\label{chap:S7}QuPath integration method}
We adopt an integration strategy leveraging the paquo \cite{Bayer_AG} library, a Python package that enables direct interaction with QuPath’s internal API, thereby facilitating seamless data exchange without intermediate conversion steps. The data processing pipeline (\hyperref[fig:S7]{Appendix Figure S7}) begins with the acquisition of WSIs and their associated annotations from QuPath, which are represented as Shapely \cite{Gillies_Wel_etal._2024} polygons. Utilizing paquo, we directly read, create, and modify these annotations and detections within a QuPath project in the Python environment. Images are then cropped using these polygons and processed by cell segmentation and classification models employing standard vision processing toolkits such as OpenCV, pyvips, and PyTorch. Additionally, QuPath employs Groovy scripts to initiate a Python process that starts the entire pipeline from QuPath graphical interface: fetching polygons, extracting images from them, and running deep learning model inference on the cropped images. 
The results are returned to QuPath, leveraging paquo's Python bindings to manipulate QuPath data while minimizing the computational overhead typically associated with cross-environment communication.

\counterwithin{figure}{subsection}
\renewcommand{\thefigure}{S\arabic{subsection}}

\begin{figure}[h!]
    \centering
    \includegraphics[width=\textwidth]{images/A7.pdf}
    \caption{QuPath integration workflow using Python environment}
    \label{fig:S7}
\end{figure}

Compared to traditional workflows that involve exporting annotations as GeoJSON, classifying them in Python, and reimporting them into QuPath, our approach offers several advantages. We eliminate the need to switch between programming languages, providing a cohesive and streamlined development process entirely within QuPath software and removing the necessity to use other tools. Meanwhile, we avoid storing annotations as intermediate JSON files unless required for external use or archiving. By conducting the entire inference and post-processing workflow within the Python environment, we leverage the power and flexibility of Python libraries for image processing and machine learning. This approach also enables adjustments to any set of labels and models, thereby improving its applicability.

%\hfill

The distilled model and QuPath integration code are packaged into a Docker container, enabling streamlined execution with the Docker engine. Detailed integration code and deployment instructions can be found in the GitHub repository \cite{Shvetsov_2025b}.

Despite these benefits, we acknowledge that the paquo library is a proof‑of‑concept project in its early development stage and has not been tested across all versions of QuPath.

\clearpage

\subsection{\label{chap:S8}Data and code availability statement}
All datasets, models, and code used in this study are publicly available and can be obtained from the repositories listed below. 
The PanNuke \cite{Gamper_Koohbanani_etal._2019} and MoNuSAC \cite{Verma_Kumar_etal._2021} datasets are publicly accessible, and download information along with detailed descriptions can be found in their respective articles. Preprocessing scripts for PanNuke and MoNuSAC data, as well as individual cell extraction scripts, are available on GitHub \cite{Shvetsov_2025a}. The H-Optimus foundation model used in our experiments can be downloaded from the HuggingFace repository \cite{hoptimus2024}, and model information is available on GitHub \cite{Saillard_Jenatton_etal._2024}. In addition, the integration code for QuPath and the distilled model packaged in a Docker container are provided in the repository \cite{Shvetsov_2025b}, and paquo Python library is available from the authors GitHub repository \cite{Bayer_AG}.
\clearpage

\end{document}


 \newpage
 \appendix
 
\clearpage
% \setcounter{page}{1}
% \maketitlesupplementary
\begin{center}
Supplementary Material
\end{center}

% {
%     \onecolumn
%     \centering
%     \Large
%     \textbf{\thetitle}\\
%     \vspace{0.5em}Supplementary Material \\
%     \vspace{1.0em}
% }

\section{Proof of \cref{theorem:dr}}
We require some additional regularity assumptions:
\begin{assumption} 1) The number of classes $C$ is bounded w.r.t the number of samples $N$, 2) the missingness mechanism $P(A=1|Y,\theta)$, as well as its estimated counterpart $P(A=1|Y,\theta)$, are bounded below by some constant $\epsilon > 0$, 3) the quantities $P(Y|X,\theta)$ and $P(A|Y,\theta)$ are estimated using auxiliary samples independent of samples used for the sample averaging.
\label{assumption:extra}
\end{assumption}
Assumptions 1 and 2 are natural. For the missingness mechanism, the ground truth being bounded means that there is a non-vanishing proportion of samples for every class. The boundedness of the estimate can be enforced by clipping the estimate. Assumption 3 is called sample splitting in \cite{kennedy-dr}.

For convenience we use operator $\E_N$ to denote the average of $N$ samples i.e. $\frac{1}{N}\sum_{i=1}^N$. Note that this is by itself a random variable, in contrast to $\E$ which is a fixed number.

\begin{proof}[Proof of \cref{theorem:dr}] Because $C$ is bounded (assumption \ref{assumption:extra}), we can fix a class $c$ and prove the theorem.
Let us define the influence function $\phi$, parameterized by $\theta$, as
\begin{equation}
\phi(O | \theta)(c) = P(Y=c|X,\theta) + \frac{\one(A=1)}{P(A=1|Y,\theta)} (\one(Y=c) - P(Y=c|X,\theta)) - P(Y=c)
\end{equation}
As we have done in the main text, we use $\phi(O)$ to denote the same function but all estimated quantities are replaced with their truths. In other words, we use $\phi(O)$ for $\phi(O|\theta_0)$ where $\theta_0$ is the truth, given that our model contains $\theta_0$ e.g. when the model is consistent.

Recall that:
\begin{equation}
\begin{aligned}
\Psi_{dr}(\theta)(c) &= \frac{1}{N}\sum_{i=1}^N \left\{P(Y=c|X,\theta) + \frac{\one(A=1)}{P(A=1|Y,\theta)} (\one(Y=c) - P(Y=c|X,\theta))\right\}\\
&= \E_N [\phi(O|\theta)(c)] + P(Y=c)
\end{aligned}
\end{equation}

We will show that:
\begin{equation}
\Psi_{dr}(\theta)(c) - P(Y=c) = (\E_N - \E)[\phi(O)(c)] + o_P(N^{-1/2})
\label{eq:proof-linearity}
\end{equation}
To do that, we use the following decomposition
\begin{equation}
\begin{aligned}
\Psi_{dr}(\theta)(c) - P(Y=c) &= \E_N [\phi(O|\theta)(c)] \\
&= (\E_N - \E)[\phi(O)(c)] + (\E_N - \E)[\phi(O|\theta)(c) - \phi(O)(c)] + \E[\phi(O|\theta)(c)]
% &+ (\E_n - \E)[\phi(O;\theta) - \phi(O)]\\
% &+ \E[P(Y=c|X,\theta)] - \E[P(Y=c|X)] + \E[\phi(O,\theta)]
\end{aligned}
\end{equation}
and analyze the second and third term. The third term is:
\begin{equation}
\begin{aligned}
\E[\phi(O|\theta)(c)] &= \E[P(Y=c|X,\theta)] + \E\left[\frac{\one(A=1)}{P(A=1|Y,\theta)}(\one(Y=c) - P(Y=c|X,\theta))\right]- P(Y=c) \\
&= \E\left[P(Y=c|X,\theta) + \frac{P(A=1|Y)}{P(A=1|Y,\theta)}(P(Y=c|X) - P(Y=c|X,\theta))\right] - \E[P(Y=c|X)]\\
&= \E\left[(P(Y=c|X,\theta) - P(Y=c|X)) (P(A=1|Y,\theta) -P(A=1|Y)) \frac{1}{P(A=1|Y,\theta)}\right]\\
\end{aligned}
\end{equation}
by Cauchy-Schwarz inequality:
\begin{equation}
\begin{aligned}
\E[\phi(O|\theta)(c)] &\le \frac{1}{\epsilon} \|P(A=1|Y,\theta) - P(A=1|Y)\|_2 \|P(Y=c|X,\theta) - P(Y=c|X)\|_{L_2(P)}\\
&= \frac{1}{\epsilon} o_P(N^{-1/4} N^{-1/4}) = o_P(N^{-1/2})
\end{aligned}
\end{equation}
by assumption \ref{assumption:4th-root-n} and that $P(A=1|Y,\theta) > \epsilon$ (assumption \ref{assumption:extra}). The second term can be bounded by Chebyshev inequality
% \begin{equation}
% \begin{aligned}
% \E[\E_N[\phi(O|\theta)(c) - \phi(O)(c)]] &= \E[\phi(O|\theta)(c) - \phi(O)(c)]\\
% \var[\E_N[\phi(O|\theta)(c) - \phi(O)(c)]] &= \frac{1}{N}\var[\phi(O|\theta)(c) - \phi(O)(c)] \le 
% \end{aligned}
% \end{equation}
\begin{equation}
P(|(\E_N - \E)[\phi(O|\theta)(c) - \phi(O)(c)]| \ge t) \le \frac{\var[\E_N[\phi(O|\theta)(c) - \phi(O)(c)]]}{t^2} = \frac{\var[\phi(O|\theta)(c) - \phi(O)(c)]}{Nt^2}
\end{equation}
note here that $\theta$ is independent of the samples used for $\E_N$ by assumption \ref{assumption:extra}. For any $\varepsilon > 0$, by picking $t = \frac{1}{\sqrt{N\varepsilon}}$ we get
\begin{equation}
P\left(\left|\frac{(\E_N - \E)[\phi(O|\theta)(c) - \phi(O)(c)]}{N^{-1/2}}\right| \ge \frac{1}{\sqrt{\varepsilon}}\right) \le \varepsilon \var[\phi(O|\theta)(c) - \phi(O)(c)]
\end{equation}
by the definition of $O_P$, we then get
\begin{equation}
(\E_N - \E)[\phi(O|\theta)(c) - \phi(O)(c)] = O_P(N^{-1/2}\var[\phi(O|\theta)(c) - \phi(O)(c)])
\end{equation}
Because $\phi$ is a continuous function of $P(Y|X,\theta)$ and $P(A|Y,\theta)$ (given $P(A|Y,\theta) > \epsilon$, assumption \ref{assumption:extra}), by the continuous mapping theorem and the fact that $P(Y|X,\theta)$ and $P(A|Y,\theta)$ are convergent in probability (assumption \ref{assumption:4th-root-n}), we get $\var[\phi(O|\theta)(c) - \phi(O)(c)] = o_P(1)$. This gives
\begin{equation}
(\E_N - \E)[\phi(O|\theta)(c) - \phi(O)(c)] = o_P(N^{-1/2})
\end{equation}
Therefore, we have shown that the second and third term are both $o_P(N^{-1/2})$, proving \cref{eq:proof-linearity}. As the final step, multiply both sides of this equation by $\sqrt{N}$ we get:
\begin{equation}
\sqrt{N}(\Psi_{dr}(\theta)(c) - P(Y=c)) = \sqrt{N} (\E_N - \E)[\phi(O)(c)] + o_P(1) \rightsquigarrow \mathcal{N}(0, \var[\phi(O)(c)])
\end{equation}
by the central limit theorem, and $\var[\phi(O)(c)] = \E[\phi(O)(c)^2]$ because $\E[\phi(O)(c)] = 0$.
\end{proof}

While we started with the definition of $\phi$, \cref{eq:proof-linearity} shows that $\phi$ is indeed an influence function. Now we show that $\phi$ is also the efficient influence function, by using the characterization of the model's tangent space \cite{tsiatis-missingdata}. Note that the joint probability factorizes as $P(X,A,Y) = P(X)P(Y|X)P(A|Y)$, therefore the tangent space $\mathcal{T}$ factorizes as $\mathcal{T} = \mathcal{T}_{X} \oplus \mathcal{T}_{Y|X} \oplus \mathcal{T}_{A|Y}$ where $\mathcal{T}_X = \{h(X): \E[h] = 0\}$, $\mathcal{T}_{Y|X} = \{h(X,Y): \E[h|X] = 0\}$, $\mathcal{T}_{A|Y} = \{h(A,Y): \E[h|Y] = 0\}$, and the 3 subspaces are pairwise orthogonal. All influence functions are orthogonal to the tangent space, but the influence function that is also in the tangent space has the smallest variance and is called the efficient influence function. As $\phi$ is already an influence function, we need only show that $\phi$ is in $\mathcal{T}$. We write $\phi$ as
\begin{equation}
\phi(O)(c) = (P(Y=c|X) - P(Y=c)) + \left[\frac{\one(A=1)}{P(A=1|Y)} - 1\right](\one(Y=c) - P(Y=c|X)) + (\one(Y=c) - P(Y=c|X))
\end{equation}
and note that the first, second and third term are in $\mathcal{T}_X$, $\mathcal{T}_{A|Y}$ and $\mathcal{T}_{Y|X}$ respectively. Therefore, $\phi$ is indeed in $\mathcal{T}$. The efficient influence function has the smallest variance of all influence function, and therefore our estimator being asymptotically linear in $\phi$ (\cref{eq:proof-linearity}) has the smallest mean squared error in a local asymptotic minimax sense \cite{kennedy-dr, asymptoticstatistics}

\section{Further background and related work}
\paragraph{Discussion on semi-supervised EM.}
It appears that semi-supervised EM was first used for parameter estimation when the missingness mechanism is non-ignorable in \cite{ibrahim1996parameter}, but has not been used for label shift estimation.
Perhaps this is because the semi-supervised situation where additional unlabeled data is available during training is rarer than the test-time adaptation case. EM is well suited to take advantage of the extra unlabeled data to improve the classifier under very scarce and long-tailed labeled data. While the connection between pseudo-labeling and EM has been explored before \cite{entropyminimization}, the situation with label shift has not until recently \cite{simpro}. Here the application of EM is much more interesting, because other than simply giving pseudo-labeling a rigorous formulation, EM also estimates the missingness mechanism (equivalently the label distribution shift), which is important for shift correction and thus high-quality pseudo-labels \cite{acr}. The application of confidence thresholding can be seen as a sparse variant of EM \cite{neal1998view}.

\paragraph{The doubly-robust risk.} 
\label{subsec:dr-risk}
A technique that also derives from the theory of semi-parametric efficiency is orthogonal statistical learning \citep{foster2023orthogonal}. The idea is to minimize the doubly-robust risk:
\label{subsec:method-dr-risk}
\begin{equation}
\label{eq:dr-risk}
\mathcal{R}(\theta_2) = \frac{1}{N} \sum_{i=1}^N \Bigg[ l(x_i, \hat y_i|\theta_2) + \frac{\one(a_i=1)}{P(A=a_i|Y=y_i, \theta_1)} (l(x_i, y_i | \theta_2) - l(x_i, \hat y_i | \theta_2))\Bigg]
\end{equation}
where $l(x,y|\theta) = -\sum_{c=1}^C [y]_c \log P(Y=c|X=x,\theta)$ is the negative cross-entropy. 
The notation $[y]_c$ means that we are using the $c$-entry in a C-dimension probability vector $y$. 
Thus, $y_i$ denotes the one-hot label of observation $i$, while $\hat y_i$ denotes the pseudo-label, which can be one-hot or all-zero. 
Finally, we use $\theta_1$ to denote that $P(a|y,\theta_1)$ is an estimation from a previous stage, but it can be estimated with $\theta_2$ as well. 
The risk $\mathcal{R}(\theta_2)$ can be used as a training loss in a straightforward fashion. 
Similar to the doubly robust estimation of $P(Y)$, the doubly robust risk provides approximately unbiased estimation of the risk. 
This property has been used in \citep{arelabelsinformative, onnonrandommissinglabels, drst} also in the semi-supervised learning setting.
More broadly, it is at the heart of one of the core techniques in heterogenous treatment effect estimation in causal estimation \cite{kennedy2023towards, foster2023orthogonal, wager2018estimation}. 
The focus here is not the estimation of $\mathcal{R}(\theta_2)$ per se, but the quality of the learned model \cite{foster2023orthogonal}.
By using the doubly-robust risk, we can achieve an optimality result similar in spirit to our theorem \cref{theorem:dr}, but for the generalization error.
While this is appealing, in practice there are 2 problems with this approach. First, the inverse probability weight $P(A=a_i|Y=y_i,\theta_1)$ can be very large if the class ratio is highly unlabeled, making training unstable \cite{kallus2020deepmatch, pham2023stable}. 
This problem exists for our estimation as well. However, it is much easier to control for estimation than for training because of the iterative nature of model update. Secondly, we can further write $\mathcal{R}$ as:
\begin{equation}
\mathcal{R}(\theta_2) = \frac{1}{N}\sum_{i=1}^N l\left(x_i, \hat y_i + \frac{\one(a_i=1)}{P(A=a_i|Y=y_i,\theta_1)} (y_i - \hat y_i)\Bigg\vert\theta_2\right)
\end{equation}
which is a cross-entropy loss with new meta-pseudo-labels. However, these labels are not meant to be learned exactly, and furthermore they can be negative. Thus, theoretical works have to put stringent assumptions on the models. In \cref{subsec:ablation-1}, we show that experimentally that the instability problem makes doubly-robust risk performance worse than our 2-stage approach.

\section{Training and hyperparameter settings.}
\label{subsec:training-setting}
For neural network training, we follow the implementation and hyperparameter settings of \cite{simpro}. In particular, we adapt the core code of SimPro for Supervised, MLE and EM. For MLE, we update $P(A|Y)$ using the Adam optimizer with learning rate 1e-3, while for EM we use a momentum update similar to SimPro's update of $P(Y|A)$ because it has a a closed-form solution at each mini-batch. We use Wide ResNet-28-2 on all methods and all datasets in this section, including Imagenet-127, because we are motivated by the fact that stage-1's goal is not classification accuracy but the estimation of a finite-dimensional parameter. When using Wide ResNet-28-2 for Imagenet-127, we use the hyperparameters of CIFAR-100, except we lower the batch size of unlabeled data to 2 times that of labeled data instead of 8 for memory reason. We do not perform additional hyperparameter tuning. All experiments can be performed on 1 A6000 RTX GPU, and are run 3 times. We report the total variation distance between the estimated and the ground truth unlabeled class distribution, similar to its usage in Theorem 3.1 of \cite{lsc}, and the top-1 classification accuracy.

In the second stage of our algorithm, we freeze our estimation and plug it in SimPro and BOAT.
We keep exactly the same hyperparameter settings that SimPro and BOAT use. In particular, for Imagenet-127, we now use ResNet-50 and run each experiment once.
In SimPro, we set the unlabeled class distribution $P(Y|A=0)$ at the E-step;  however, we still keep a running estimate of the class distribution $P(Y)$ in the logit adjustment loss \cref{eq:simpro-la-loss}. While it is possible to use the first stage estimate in the logit adjustment loss, we observe that doing so results in lower accuracy than using the the running average. This is conceptually consistent with the role of the running average - serving not as an accurate estimate of $P(Y)$ but to make the classifier's class distribution uniform through the logit adjustment loss, which is good for the test set. Similarly, in BOAT, we only replace $\Delta_c = \log P(Y|A=1) - \log P(Y|A=0)$ in equation (4) of \cite{boat}, which is adjusting a classifier's predictions from the labeled to the unlabeled class distribution, with our SimPro + DR estimate instead of their on-the-fly estimate. 


% \section{Additional experiments}
% % \begin{table*}[t]
\centering
\caption{Total Variation Distance on CIFAR-10-LT ($N_l = 500$, $M_l = 4000$) with different class imbalance ratios $\gamma_l$ and $\gamma_u$ under five different unlabeled class distributions.}
\label{tab:cifar10-tv}
\resizebox{\textwidth}{!}{
\begin{tabular}{lccccccccccc}
\toprule
& & \multicolumn{2}{c}{consistent} & \multicolumn{2}{c}{uniform} & \multicolumn{2}{c}{reversed} & \multicolumn{2}{c}{middle} & \multicolumn{2}{c}{head-tail} \\
\cmidrule(lr){3-4} \cmidrule(lr){5-6} \cmidrule(lr){7-8} \cmidrule(lr){9-10} \cmidrule(lr){11-12}
& & $\gamma_l = 150$ & $\gamma_l = 100$ & $\gamma_l = 150$ & $\gamma_l = 100$ & $\gamma_l = 150$ & $\gamma_l = 100$ & $\gamma_l = 150$ & $\gamma_l = 100$ & $\gamma_l = 150$ & $\gamma_l = 100$ \\
Model & Estimator & $\gamma_u = 150$ & $\gamma_u = 100$ & $\gamma_u = 1$ & $\gamma_u = 1$ & $\gamma_u = 1/150$ & $\gamma_u = 1/100$ & $\gamma_u = 150$ & $\gamma_u = 100$ & $\gamma_u = 150$ & $\gamma_u = 100$ \\
\midrule
Supervised & MLLS & 0.269 ± 0.252 & 0.038 ± 0.006 & 0.251 ± 0.046 & 0.255 ± 0.060 & 0.429 ± 0.028 & 0.493 ± 0.050 & 0.333 ± 0.042 & 0.320 ± 0.009 & 0.457 ± 0.034 & 0.444 ± 0.043 \\
Supervised & RLLS & 0.043 ± 0.001 & 0.044 ± 0.010 & 0.348 ± 0.034 & 0.305 ± 0.068 & 0.769 ± 0.016 & 0.678 ± 0.028 & 0.430 ± 0.008 & 0.368 ± 0.013 & 0.539 ± 0.018 & 0.503 ± 0.020 \\
\midrule
MLE & IPW & 0.027 ± 0.001 & 0.027 ± 0.000 & 0.319 ± 0.072 & 0.243 ± 0.010 & 0.674 ± 0.020 & 0.646 ± 0.041 & 0.438 ± 0.020 & 0.454 ± 0.026 & 0.547 ± 0.049 & 0.491 ± 0.059 \\
MLE & OR & 0.045 ± 0.004 & 0.042 ± 0.000 & 0.215 ± 0.026 & 0.203 ± 0.032 & 0.433 ± 0.017 & 0.395 ± 0.033 & 0.193 ± 0.006 & 0.209 ± 0.037 & 0.307 ± 0.147 & 0.249 ± 0.130 \\
MLE & DR & 0.090 ± 0.002 & 0.079 ± 0.000 & 0.407 ± 0.027 & 0.360 ± 0.007 & 0.425 ± 0.007 & 0.421 ± 0.029 & 0.256 ± 0.001 & 0.286 ± 0.031 & 0.435 ± 0.136 & 0.362 ± 0.122 \\
\midrule
EM & IPW & 0.035 ± 0.002 & 0.040 ± 0.001 & 0.021 ± 0.001 & 0.029 ± 0.015 & 0.303 ± 0.187 & 0.091 ± 0.010 & 0.119 ± 0.011 & 0.105 ± 0.022 & 0.104 ± 0.026 & 0.104 ± 0.051 \\
EM & OR & 0.037 ± 0.003 & 0.042 ± 0.002 & 0.016 ± 0.001 & 0.024 ± 0.012 & 0.269 ± 0.183 & 0.090 ± 0.008 & 0.122 ± 0.012 & 0.103 ± 0.022 & 0.072 ± 0.012 & 0.073 ± 0.024 \\
EM & DR & 0.034 ± 0.004 & 0.037 ± 0.001 & 0.014 ± 0.001 & 0.027 ± 0.020 & 0.264 ± 0.191 & 0.092 ± 0.005 & 0.111 ± 0.019 & 0.097 ± 0.026 & 0.077 ± 0.016 & 0.073 ± 0.028 \\
\midrule
SimPro & IPW & 0.070 ± 0.011 & 0.058 ± 0.000 & 0.046 ± 0.001 & 0.049 ± 0.005 & 0.254 ± 0.074 & 0.223 ± 0.098 & 0.097 ± 0.025 & 0.067 ± 0.002 & 0.105 ± 0.066 & 0.110 ± 0.079 \\
SimPro & OR & 0.071 ± 0.012 & 0.058 ± 0.000 & 0.045 ± 0.001 & 0.049 ± 0.006 & 0.040 ± 0.003 & 0.059 ± 0.017 & 0.074 ± 0.006 & 0.075 ± 0.002 & 0.033 ± 0.003 & 0.033 ± 0.003 \\
SimPro & DR & 0.017 ± 0.004 & 0.026 ± 0.001 & 0.019 ± 0.002 & 0.018 ± 0.003 & 0.039 ± 0.003 & 0.058 ± 0.025 & 0.091 ± 0.007 & 0.031 ± 0.001 & 0.015 ± 0.003 & 0.019 ± 0.007 \\
\bottomrule
\end{tabular}
}
\end{table*}
% 

\begin{table*}[t]
\centering
\caption{Total Variation Distance on CIFAR-100-LT ($N_l = 50$, $M_l = 400$) with different class imbalance ratios $\gamma_l$ and $\gamma_u$ under five different unlabeled class distributions.}
\label{tab:cifar100-tv}
\resizebox{\textwidth}{!}{
\begin{tabular}{lccccccccccc}
\toprule
& & \multicolumn{2}{c}{consistent} & \multicolumn{2}{c}{uniform} & \multicolumn{2}{c}{reversed} & \multicolumn{2}{c}{middle} & \multicolumn{2}{c}{head-tail} \\
\cmidrule(lr){3-4} \cmidrule(lr){5-6} \cmidrule(lr){7-8} \cmidrule(lr){9-10} \cmidrule(lr){11-12}
& & $\gamma_l = 20$ & $\gamma_l = 10$ & $\gamma_l = 20$ & $\gamma_l = 10$ & $\gamma_l = 20$ & $\gamma_l = 10$ & $\gamma_l = 20$ & $\gamma_l = 10$ & $\gamma_l = 20$ & $\gamma_l = 10$ \\
Model & Estimator & $\gamma_u = 20$ & $\gamma_u = 10$ & $\gamma_u = 1$ & $\gamma_u = 1$ & $\gamma_u = 1/20$ & $\gamma_u = 1/10$ & $\gamma_u = 20$ & $\gamma_u = 10$ & $\gamma_u = 20$ & $\gamma_u = 10$ \\
\midrule
Supervised & MLLS & 0.707 ± 0.016 & 0.313 ± 0.100 & 0.445 ± 0.172 & 0.309 ± 0.119 & 0.383 ± 0.075 & 0.397 ± 0.006 & 0.570 ± 0.001 & 0.373 ± 0.107 & 0.543 ± 0.009 & 0.231 ± 0.057 \\
Supervised & RLLS & 0.520 ± 0.007 & 0.133 ± 0.003 & 0.337 ± 0.125 & 0.253 ± 0.082 & 0.424 ± 0.060 & 0.463 ± 0.003 & 0.454 ± 0.021 & 0.306 ± 0.074 & 0.460 ± 0.028 & 0.241 ± 0.040 \\
\midrule
MLE & IPW & 0.075 ± 0.000 & 0.071 ± 0.001 & 0.229 ± 0.001 & 0.167 ± 0.002 & 0.565 ± 0.005 & 0.443 ± 0.007 & 0.415 ± 0.000 & 0.311 ± 0.005 & 0.343 ± 0.000 & 0.280 ± 0.001 \\
MLE & OR & 0.065 ± 0.002 & 0.061 ± 0.001 & 0.200 ± 0.007 & 0.143 ± 0.001 & 0.526 ± 0.011 & 0.399 ± 0.023 & 0.360 ± 0.003 & 0.256 ± 0.012 & 0.328 ± 0.003 & 0.266 ± 0.005 \\
MLE & DR & 0.149 ± 0.019 & 0.145 ± 0.010 & 0.243 ± 0.004 & 0.214 ± 0.019 & 0.568 ± 0.005 & 0.464 ± 0.014 & 0.403 ± 0.014 & 0.309 ± 0.012 & 0.365 ± 0.007 & 0.320 ± 0.004 \\
\midrule
EM & IPW & 0.097 ± 0.008 & 0.092 ± 0.004 & 0.239 ± 0.007 & 0.179 ± 0.003 & 0.478 ± 0.012 & 0.329 ± 0.020 & 0.262 ± 0.016 & 0.202 ± 0.003 & 0.312 ± 0.002 & 0.227 ± 0.001 \\
EM & OR & 0.121 ± 0.007 & 0.108 ± 0.005 & 0.261 ± 0.007 & 0.189 ± 0.004 & 0.489 ± 0.013 & 0.335 ± 0.020 & 0.274 ± 0.016 & 0.211 ± 0.004 & 0.336 ± 0.003 & 0.235 ± 0.001 \\
EM & DR & 0.125 ± 0.005 & 0.111 ± 0.004 & 0.269 ± 0.007 & 0.194 ± 0.005 & 0.497 ± 0.010 & 0.336 ± 0.024 & 0.281 ± 0.019 & 0.219 ± 0.008 & 0.336 ± 0.007 & 0.233 ± 0.004 \\
\midrule
SimPro & IPW & 0.125 ± 0.001 & 0.100 ± 0.005 & 0.166 ± 0.007 & 0.141 ± 0.009 & 0.353 ± 0.023 & 0.261 ± 0.008 & 0.202 ± 0.003 & 0.158 ± 0.005 & 0.277 ± 0.009 & 0.197 ± 0.003 \\
SimPro & OR & 0.133 ± 0.005 & 0.100 ± 0.004 & 0.160 ± 0.007 & 0.138 ± 0.010 & 0.322 ± 0.014 & 0.253 ± 0.008 & 0.202 ± 0.003 & 0.156 ± 0.005 & 0.269 ± 0.006 & 0.191 ± 0.004 \\
SimPro & DR & 0.122 ± 0.003 & 0.106 ± 0.006 & 0.188 ± 0.009 & 0.149 ± 0.006 & 0.343 ± 0.023 & 0.257 ± 0.007 & 0.219 ± 0.010 & 0.172 ± 0.002 & 0.279 ± 0.007 & 0.198 ± 0.004 \\
\bottomrule
\end{tabular}
}
\end{table*}
%\vfill

\section{Per Scene metrics}
\begin{table*}
\centering
% \hspace{-2em}
\resizebox{\linewidth}{!}{
\setlength{\tabcolsep}{5pt}
\begin{tabular}{l|ccc|cccc}
    \toprule
    & 3DGS~\cite{gsplat} & Mip-Splatting~\cite{mip-splatting} & 3DGS-MCMC~\cite{3dgs-mcmc} & Plenoxels~\cite{plenoxels} & iNGP-Big~\cite{ingp} & MipNerf360~\cite{mipnerf360} & Ours \\
    & \scriptsize{PSNR$\uparrow$ / SSIM$\uparrow$ / LPIPS$\downarrow$} & \scriptsize{PSNR$\uparrow$ / SSIM$\uparrow$ / LPIPS$\downarrow$} & \scriptsize{PSNR$\uparrow$ / SSIM$\uparrow$ / LPIPS$\downarrow$} & \scriptsize{PSNR$\uparrow$ / SSIM$\uparrow$ / LPIPS$\downarrow$} & \scriptsize{PSNR$\uparrow$ / SSIM$\uparrow$ / LPIPS$\downarrow$} & \scriptsize{PSNR$\uparrow$ / SSIM$\uparrow$ / LPIPS$\downarrow$} & \scriptsize{PSNR$\uparrow$ / SSIM$\uparrow$ / LPIPS$\downarrow$} \\
    \midrule
    Room & 30.63 / 0.91 / 0.27 & 31.74 / \textbf{0.93} / 0.27 & \textbf{32.30} / \textbf{0.93} / \textbf{0.25} & 27.59 / 0.84 / 0.42 & 29.69 / 0.87 / 0.26 & \textbf{31.63} / \textbf{0.91} / 0.21 & 30.87 / \textbf{0.91} / \textbf{0.19} \\
    Counter & 28.70 / 0.91 / 0.24 & \textbf{29.16} / \textbf{0.92} / 0.24 & \textbf{29.16} / 0.91 / \textbf{0.23} & 23.63 / 0.76 / 0.44 & 26.69 / 0.82 / 0.31 & \textbf{29.55} / \textbf{0.89} / 0.20 & 28.59 / 0.88 / \textbf{0.19} \\
    Bonsai & 31.98 / 0.94 / 0.24 & 32.31 / \textbf{0.95} / \textbf{0.23} & \textbf{32.67} / \textbf{0.95} / \textbf{0.23} & 24.67 / 0.81 / 0.40 & 30.69 / 0.91 / 0.21 & \textbf{33.46} / \textbf{0.94} / 0.18 & 32.15 / 0.93 / \textbf{0.17} \\
    Kitchen & 30.32 / 0.92 / \textbf{0.14} & 31.55 / 0.93 / 0.15 & \textbf{32.23} / \textbf{0.94} / \textbf{0.14} & 23.42 / 0.65 / 0.45 & 29.48 / 0.86 / 0.20 & \textbf{32.23} / \textbf{0.92} / \textbf{0.13} & 31.40 / 0.91 / \textbf{0.13} \\
    Bicycle & 25.25 / 0.77 / 0.23 & 25.72 / \textbf{0.78} / \textbf{0.19} & \textbf{26.06} / \textbf{0.78} / \textbf{0.19} & 21.92 / 0.50 / 0.51 & 22.17 / 0.51 / 0.45 & \textbf{24.37} / \textbf{0.69} / \textbf{0.30} & 24.19 / 0.68 / 0.31 \\
    Garden & 27.41 / 0.87 / 0.12 & 27.76 / \textbf{0.88} / \textbf{0.11} & \textbf{27.99} / 0.87 / \textbf{0.11} & 23.49 / 0.61 / 0.39 & 25.07 / 0.70 / 0.26 & \textbf{26.98} / 0.81 / 0.17 & 26.58 / \textbf{0.82} / \textbf{0.16} \\
    Stump & 26.55 / 0.78 / 0.24 & 26.94 / \textbf{0.79} / 0.21 & \textbf{27.67} / 0.78 / \textbf{0.20} & 20.66 / 0.52 / 0.50 & 23.47 / 0.59 / 0.42 & \textbf{26.40} / \textbf{0.74} / \textbf{0.26} & 25.48 / 0.71 / 0.29 \\
    \bottomrule
\end{tabular}
}
\caption{
   PSNR, SSIM, and LPIPS scores for Mip-NeRF360~\cite{mipnerf360} scenes.
    }
\label{tab:mipnerf360_all}
\end{table*}













\begin{table*}
\centering
% \hspace{-2em}
\resizebox{\linewidth}{!}{
\setlength{\tabcolsep}{5pt}
\begin{tabular}{l|ccc|cccc}
    \toprule
    & 3DGS~\cite{gsplat} & Mip-Splatting~\cite{mip-splatting} & 3DGS-MCMC~\cite{3dgs-mcmc} & Plenoxels~\cite{plenoxels} & iNGP-Big~\cite{ingp} & MipNerf360~\cite{mipnerf360} & Ours \\
    & \scriptsize{PSNR$\uparrow$ / SSIM$\uparrow$ / LPIPS$\downarrow$} & \scriptsize{PSNR$\uparrow$ / SSIM$\uparrow$ / LPIPS$\downarrow$} & \scriptsize{PSNR$\uparrow$ / SSIM$\uparrow$ / LPIPS$\downarrow$} & \scriptsize{PSNR$\uparrow$ / SSIM$\uparrow$ / LPIPS$\downarrow$} & \scriptsize{PSNR$\uparrow$ / SSIM$\uparrow$ / LPIPS$\downarrow$} & \scriptsize{PSNR$\uparrow$ / SSIM$\uparrow$ / LPIPS$\downarrow$} & \scriptsize{PSNR$\uparrow$ / SSIM$\uparrow$ / LPIPS$\downarrow$} \\
    \midrule
    Dr Johnson & 28.77 / \textbf{0.90} / 0.33 & 28.76 / \textbf{0.90} / \textbf{0.32} & \textbf{29.05} / 0.89 / 0.33 & 23.14 / 0.79 / 0.52 & 28.26 / 0.85 / 0.35 & \textbf{29.14} / \textbf{0.90} / \textbf{0.24} & 28.33 / /0.88 / 0.27 \\
    Playroom & 30.04 / \textbf{0.91} / 0.32 & 30.17 / \textbf{0.91} / 0.33 & \textbf{30.37} / 0.90 / \textbf{0.31} & 22.98 / 0.80 / 0.50 & 21.67 / 0.78 / 0.43 & \textbf{29.66} / \textbf{0.90} / \textbf{0.25} & 29.56 / 0.89 / 0.26 \\
    \bottomrule
\end{tabular}
}
\caption{
   PSNR, SSIM, and LPIPS scores for Deep Blending~\cite{deepblending} scenes.
    }
\label{tab:db_all}
\end{table*}













\Cref{tab:mipnerf360_all,tab:db_all} summarize the error metrics collected for our evaluation of all considered techniques.
These include results for both Mip-NeRF360~\cite{mipnerf360} and Deep Blending~\cite{deepblending} scenes.
However, 3DGRT~\cite{3dgrt} is excluded from per-scene comparisons as these values are not reported in the original paper, and the code is not publicly available.

\vspace{33em}
\hfill

~
\vspace{42em}

% \def\year{2022}\relax
%File: formatting-instructions-latex-2022.tex
%release 2022.1
\documentclass[letterpaper]{article} % DO NOT CHANGE THIS
\usepackage{aaai22}  % DO NOT CHANGE THIS
\usepackage{times}  % DO NOT CHANGE THIS
\usepackage{helvet}  % DO NOT CHANGE THIS
\usepackage{courier}  % DO NOT CHANGE THIS
\usepackage[hyphens]{url}  % DO NOT CHANGE THIS
\usepackage{graphicx} % DO NOT CHANGE THIS
\urlstyle{rm} % DO NOT CHANGE THIS
\def\UrlFont{\rm}  % DO NOT CHANGE THIS
\usepackage{natbib}  % DO NOT CHANGE THIS AND DO NOT ADD ANY OPTIONS TO IT
\usepackage{caption} % DO NOT CHANGE THIS AND DO NOT ADD ANY OPTIONS TO IT
\DeclareCaptionStyle{ruled}{labelfont=normalfont,labelsep=colon,strut=off} % DO NOT CHANGE THIS
\frenchspacing  % DO NOT CHANGE THIS
\setlength{\pdfpagewidth}{8.5in}  % DO NOT CHANGE THIS
\setlength{\pdfpageheight}{11in}  % DO NOT CHANGE THIS
%
% These are recommended to typeset algorithms but not required. See the subsubsection on algorithms. Remove them if you don't have algorithms in your paper.
% \usepackage{algorithm}
% \usepackage{algorithmic}

%
% These are are recommended to typeset listings but not required. See the subsubsection on listing. Remove this block if you don't have listings in your paper.
\usepackage{newfloat}
\usepackage{listings}
\lstset{%
	basicstyle={\footnotesize\ttfamily},% footnotesize acceptable for monospace
	numbers=left,numberstyle=\footnotesize,xleftmargin=2em,% show line numbers, remove this entire line if you don't want the numbers.
	aboveskip=0pt,belowskip=0pt,%
	showstringspaces=false,tabsize=2,breaklines=true}
% \floatstyle{ruled}
% \newfloat{listing}{tb}{lst}{}
% \floatname{listing}{Listing}
%
%\nocopyright
%
% PDF Info Is REQUIRED.
% For /Title, write your title in Mixed Case.
% Don't use accents or commands. Retain the parentheses.
% For /Author, add all authors within the parentheses,
% separated by commas. No accents, special characters
% or commands are allowed.
% Keep the /TemplateVersion tag as is
\pdfinfo{
/Title (Leave Before You Leave: Data-Free Restoration of Pruned Neural Networks Without Fine-Tuning)
/Author ()
/TemplateVersion (2022.1)
}

% DISALLOWED PACKAGES
% \usepackage{authblk} -- This package is specifically forbidden
% \usepackage{balance} -- This package is specifically forbidden
\usepackage{color} %(if used in text)
% \usepackage{CJK} -- This package is specifically forbidden
% \usepackage{float} -- This package is specifically forbidden
% \usepackage{flushend} -- This package is specifically forbidden
% \usepackage{fontenc} -- This package is specifically forbidden
% \usepackage{fullpage} -- This package is specifically forbidden
% \usepackage{geometry} -- This package is specifically forbidden
% \usepackage{grffile} -- This package is specifically forbidden
% \usepackage{hyperref} -- This package is specifically forbidden
% \usepackage{navigator} -- This package is specifically forbidden
% (or any other package that embeds links such as navigator or hyperref)
% \indentfirst} -- This package is specifically forbidden
% \layout} -- This package is specifically forbidden
% \multicol} -- This package is specifically forbidden
% \nameref} -- This package is specifically forbidden
% \usepackage{savetrees} -- This package is specifically forbidden
% \usepackage{setspace} -- This package is specifically forbidden
% \usepackage{stfloats} -- This package is specifically forbidden
% \usepackage{tabu} -- This package is specifically forbidden
% \usepackage{titlesec} -- This package is specifically forbidden
% \usepackage{tocbibind} -- This package is specifically forbidden
% \usepackage{ulem} -- This package is specifically forbidden
% \usepackage{wrapfig} -- This package is specifically forbidden
% DISALLOWED COMMANDS
% \nocopyright -- Your paper will not be published if you use this command
% \addtolength -- This command may not be used
% \balance -- This command may not be used
% \baselinestretch -- Your paper will not be published if you use this command
% \clearpage -- No page breaks of any kind may be used for the final version of your paper
% \columnsep -- This command may not be used
% \newpage -- No page breaks of any kind may be used for the final version of your paper
% \pagebreak -- No page breaks of any kind may be used for the final version of your paperr
% \pagestyle -- This command may not be used
% \tiny -- This is not an acceptable font size.
% \vspace{- -- No negative value may be used in proximity of a caption, figure, table, section, subsection, subsubsection, or reference
% \vskip{- -- No negative value may be used to alter spacing above or below a caption, figure, table, section, subsection, subsubsection, or reference

\setcounter{secnumdepth}{0} %May be changed to 1 or 2 if section numbers are desired.

% The file aaai22.sty is the style file for AAAI Press
% proceedings, working notes, and technical reports.
%

% Title

% Your title must be in mixed case, not sentence case.
% That means all verbs (including short verbs like be, is, using,and go),
% nouns, adverbs, adjectives should be capitalized, including both words in hyphenated terms, while
% articles, conjunctions, and prepositions are lower case unless they
% directly follow a colon or long dash
\title{Leave Before You Leave: Data-Free Restoration of Pruned Neural Networks Without Fine-Tuning}
\author{
    %Authors
    % All authors must be in the same font size and format.
}
\affiliations{
    % %Afiliations
    % \textsuperscript{\rm 1}Association for the Advancement of Artificial Intelligence\\
    % % If you have multiple authors and multiple affiliations
    % % use superscripts in text and roman font to identify them.
    % % For example,

    % % Sunil Issar, \textsuperscript{\rm 2}
    % % J. Scott Penberthy, \textsuperscript{\rm 3}
    % % George Ferguson,\textsuperscript{\rm 4}
    % % Hans Guesgen, \textsuperscript{\rm 5}.
    % % Note that the comma should be placed BEFORE the superscript for optimum readability

    % 2275 East Bayshore Road, Suite 160\\
    % Palo Alto, California 94303\\
    % % email address must be in roman text type, not monospace or sans serif
    % publications22@aaai.org
%
% See more examples next
}

%Example, Single Author, ->> remove \iffalse,\fi and place them surrounding AAAI title to use it
% \iffalse
% \title{My Publication Title --- Single Author}
% \author {
%     Author Name
% }
% \affiliations{
%     Affiliation\\
%     Affiliation Line 2\\
%     name@example.com
% }
% \fi

% \iffalse
% %Example, Multiple Authors, ->> remove \iffalse,\fi and place them surrounding AAAI title to use it
% \title{My Publication Title --- Multiple Authors}
% \author {
%     % Authors
%     First Author Name,\textsuperscript{\rm 1}
%     Second Author Name, \textsuperscript{\rm 2}
%     Third Author Name \textsuperscript{\rm 1}
% }
% \affiliations {
%     % Affiliations
%     \textsuperscript{\rm 1} Affiliation 1\\
%     \textsuperscript{\rm 2} Affiliation 2\\
%     firstAuthor@affiliation1.com, secondAuthor@affilation2.com, thirdAuthor@affiliation1.com
% }
% \fi


% REMOVE THIS: bibentry
% This is only needed to show inline citations in the guidelines document. You should not need it and can safely delete it.
% \usepackage{bibentry}
% END REMOVE bibentry\usepackage{comment}
\usepackage{amsmath,amssymb} % define this before the line numbering.
\usepackage{microtype}
\usepackage{subfigure}
\usepackage{booktabs} % for professional tables
\usepackage{multirow}
\usepackage{diagbox}
\usepackage{appendix}
\usepackage[linesnumbered, lined, boxed, commentsnumbered, ruled, vlined]{algorithm2e}
\usepackage[switch]{lineno} %
\usepackage[utf8]{inputenc} % allow utf-8 input
\usepackage[T1]{fontenc}    % use 8-bit T1 fonts
\usepackage{hyperref}       % hyperlinks
\usepackage{url}            % simple URL typesetting
\usepackage{booktabs}       % professional-quality tables
\usepackage{amsfonts}       % blackboard math symbols
\usepackage{nicefrac}       % compact symbols for 1/2, etc.
\usepackage{microtype}      % microtypography
\usepackage{xcolor}         % colors

\usepackage{graphicx}
\usepackage{comment}
\usepackage{amsmath,amssymb} % define this before the line numbering.
\usepackage{color}
\usepackage{kotex}
\usepackage{microtype}
\usepackage{graphicx}
\usepackage{subfigure}
\usepackage{booktabs} % for professional tables
\usepackage{amssymb}
\usepackage{dsfont}
\usepackage{multirow}
\usepackage{diagbox}
\usepackage{makecell}
% \usepackage{algorithm,algpseudocode}
\usepackage{amsthm}

\newcommand{\eat}[1]{}
\newcommand{\TODO}[1]{{\color{red} \sc \scriptsize ToDo: #1}}
\newcommand{\dchoi}[1]{{\color{blue} \sc \small Choi: #1}}
\newcommand{\lee}[1]{{\color{orange} \sc \small Lee: #1}}
\newcommand{\kim}[1]{{\color{teal} \sc \small Kim: #1}}
\newcommand{\smalltitle}[1]{ \vspace{0.5mm}{\noindent\textbf{#1.}\hspace{0.5mm}}}

\newtheorem{lemma}{Lemma}
\newtheorem{theorem}{Theorem}
\newtheorem{definition}{Definition}


\def\T{\mathcal{T}}
\def\D{\mathcal{D}}
\def\M{\mathcal{M}}
\def\L{\mathcal{L}}
\def\S{\mathcal{S}}
% \def\A{\mathcal{A}}
% \def\Z{\mathcal{Z}}
\def\F{\mathcal{F}}
\def\N{\mathcal{N}}
\def\E{\mathcal{E}}
\def\B{\mathcal{B}}
\def\R{\mathcal{R}}
\DeclareMathOperator*{\argmin}{arg\,min}


\begin{document}

\section{General response}
Thank you for the constructive feedback of all the reviewers. Even though all reviewers find our work is theoretically solid and easy to read, they commonly request us to compare our method with additional related baselines. As a rebuttal summary, we first would like to address this major concern as follows.

\smalltitle{Comparison with more baselines}
Reviewers suggest to compare LBYL with more baselines, but most of them are either weight pruning, the pruning methods using data (synthetic or real), or the methods essentially requiring fine-tuning. There are indeed a number of pruning methods in the literature, but they cannot be fairly compared to ours as most of them need to re-train the pruned network with data as a post-processing step. Please note that our work is not focused on designing a new (data-independent) filter pruning criterion, but intends to devise another post-processing recovery method neither using any single data samples nor fine-tuning the pruned network. Our recovery method can be applicable to any pruned networks instead of fine-tuning as long as we are given their original one.

To best appreciate all the reviewers' comments, we response to each of the detailed questions below.

%%% To our best knowledge, only a couple of works have been proposed (Kim et al. 2020; Srinivas and Babu 2015) in the same problem setting as ours, and we thoroughly tested the performance of our method, compared to the state-of-the-art one (Kim et al. 2020). Furthermore, we also conducted an ablation study to see whether our method can be effective as a preprocessing stage for fast fine-tuning process, as shown in Table 4. Nevertheless, we generally agree with the reviewers that we have to include some missing references in the related work section, which should be done in our final version.

%%% To best appreciate all the reviewers' comments, we response to each of the detailed questions below.

\section{Response to Reviewer 1}
% In my opinion, this work misses some of the data independent neural pruning baselines, which involve coreset based layer-wise approximation in general. I recommend authors add a section in the related work, as in how their work is similar/different to the existing neural pruning works using coreset (for both the filter and fully connected layers). In particular, please have a look at [1/2]. [1] Data-Independent Structured Pruning of Neural Networks via Coresets (IEEE TNNLS) [2] Data-Independent Neural Pruning via Coresets (ICLR'20).
\smalltitle{Q1} Data-Independent Pruning via Coresets (ICLR'20 and TNNLS) is a weight pruning method, not a filter pruning method. As remarked in our introduction, this work deals with only structured pruning, and proposes a recovery method for models pruned by some criterion of structured pruning.

% Where is the closed-form expression for the case when we prune more than 1 filter?
\smalltitle{Q2} We iteratively use the closed-form solution for each single pruned filter to form the delivery matrix in each layer as shown in Algorithm 1. This process would restore the loss caused by multiple pruned filters in each layer.

%How does this scale to pruning and restoring the fully connected layers? I think this should be mentioned, in an appendix (with appropriate pointers to the main paper)
\smalltitle{Q3} We did not prune any neurons in FC layers in this paper, but actually our method would easily be extended to FC layers like NM (Kim et al. 2020).

% Is there any expression for the activation error, in general with an activation function \phi which is just a monotone non-decreasing function with some properties (such as Lipschitz constant?)
\smalltitle{Q4} It is challenging to approximate any non-linear activation function, and therefore we consider only ReLU in this work as shown by AE of Theorem 1. It is our future work to incorporate other activation functions than ReLU.

% Since the final expression doesn't involve the activation error, the paper just uses the L2 norm penalty. While in the case of no batch norm it may seem intuitive that reducing s will reduce the "AE", but with the batch norm, how can one be sure that reducing s is a good choice? Moreover, how does L2 make sure that s are uniform? Two weight vectors, one being uniform and the other being skewed can have equal L2 norms.
\smalltitle{Q5} In Theorem 2, we aim to minimize three error terms (i.e., RE, BE, and AE). Batch normalization (BN) can perfectly be managed by our BE term as BN is just another linear function. What is challenging here is minimizing AE, that is, a data-dependent error term. To this end, we discovered that minimizing the scale of $s$'s is effective for mitigating AE. For example, one abnormally large $s$ is hard to be cancelled out by the other $s$'s in AE. Meanwhile, in the ridge regression, L2 regularization shrinks the values of parameters in a way that larger values would have more severe penalty (i.e., forced to shrink more) than L1 regularization. In this sense, the L2 regularization would make the coefficients be more uniform whereas L1 regularization will make them more sparse.

% and the degree of minimizing the scale (i.e., $\lambda_2$ in equation (10)) is decided with the relationship with other terms (i.e., ``RE'', ``BE'') to minimize the reconstruction error optimally.


% It may be true that considering the L1 absolute error eases the mathematical analysis, to get the final closed-form expression, but have the authors tried using L2 norm square error, and using gradient descent (in case it doesn't lead to a closed-form expression)?
\smalltitle{Q6} As described in Theorem 2, our closed form solution is actually to minimize Eq. (10) consisting of three L2 norm errors, namely $\|\textbf{RE}\|_2^2$, $\|\textbf{BE}\|_2^2$, and a regularization term $\|\mathbf{s}\|_2^2$ instead of AE. Furthermore, our finally derived loss of Theorem 1 is not necessarily L1 norm. Every lemma and theorem still holds even if we use L$P$ norm for any $P \in [1, \infty]$. This is because we did not use any specific properties of L1 norm in our proofs (please refer to Appendix).

% In the definition of residual error, if instead of considering the plain linear combination of remaining filters, one considered using s as a matrix, then can one approximate the error much better (when used say gradient descent to optimize)?
\smalltitle{Q7} This suggestion is indeed interesting for our future work.


% In the tables, does the pruning ratio only consider filters? In the case of VGG-16 most of the parameters are in fully connected layers, so does the pruning ratio only consider the CNN filters? The fact that fully connected layers involve a huge number of parameters further enforces to have an algorithm for linear layers.
\smalltitle{Q8} We prune only convolution filters in CNNs. 
% we prune final convolutional filter therefore we compensate first fully connected layer like NM in VGG-16.


\section{Response to Reviewer 2}
% As far as I know, model restore often leads to distribution variation in output, which demands re-calculation of parameters in batch normalization (mean/variance) thus data inference. Does the proposed method meet similar problems
\smalltitle{Q1} In Theorem 1, the component related to the Batch Normalization (i.e., $\boldsymbol{\B}$) in the reconstruction error is considered as a constant. Therefore, we need not re-calculate the batch normalization parameters which are dependent on data.

%%% We make the reconstruction loss capable of re-scaling batch normalization parameters by the BE term.

% May I consider the proposed method as reassignment of remaining channels (features) to compensate pruned channels? If so, I think [1] should be considered as a comparison method as they share the similar idea that the remaining features are a linear combination of the origin (remaining or pruned) channels.
\smalltitle{Q2} As already discussed in our related work section, [1] (He, Zhang, and Sun 2017) is a data-dependent pruning method with fine-tuning.

% Does further fine-tuning improve the restored pruned model?
\smalltitle{Q3} In Table 4 and Figure 4, our restored model shows the fast convergence speed compared to the others.

% {Detailed Feedback for the Authors}  1. [2] should also be considered as comparison, for they similarly use few (or no) data for pruning restore. [2] compensates the remaining parameters after pruning individual parameter (element-wise pruning). Though [2] focused on pruning while the submission focused on model restore after pruning. Author may compare them from a perspective of model restore.\

\smalltitle{Q4} [2] proposes a layer-wise weight pruning method, which prunes unimportant parameters to minimize the reconstruction error caused by pruning. Although [2] avoids the re-training process after pruning, they need the whole data to compute the reconstruction error, which is therefore orthogonal to our method.

%%% Our LBYL method does not rely on the filter pruning criteria, and therefore even the model pruned by [2] can further be restored by LBYL. We can include [2] as a pruning method with less fine-tuning and less data in the related work section.

%%% (revise,,?) [2] proposes a layer-wise weight pruning method,which prunes unimportant parameters with keeping reconstruction error small using training data. Although [2] avoid the re-training process after pruning, they need whole data set to compute reconstruction error.




% 2. More experiments using fine-tune for filter pruning should be listed as reference
\smalltitle{Q5} We do not suggest a new pruning criterion but propose a data-free restoration of already pruned models. Therefore, it is somewhat pointless to compare various filter pruning methods proposing their own criterion. Nonetheless, it is highly expected for LBYL to improve the final performance even after fine-tuning as reported in Table 4, not to mention that LBYL is superior to its competitors in the other experiments without fine-tuning.


% Nonetheless, it is expected that the overall performance after fine-tuning in different experiments would be similar to what is shown in Table 4 since our restoration method is superior to the other methods across the different experiments when the fine-tuning is not implemented.
%%% (revise,,?) As showed in Table 3. Our proposed method shows better performance than the others in experiment on ImageNet compared to experiment on CIFAR-100. Our method will show better performance after fine-tuning in experiment on ImageNet.




\section{Response to Reviewer 3}
% o	The related works about pruning with less data is also a practical problem setting. It is better to compare the two problem settings (pruning with no data or few data). Besides, a few training data are available, could this method achieves better performance?
\smalltitle{Q1} Our method shows the fastest convergence speed in Table 4 and Figure 4. It means that our restored model can more quickly reach to the best performance with fine-tuning.

\section{Response to Reviewer 4}
% % o	1. Insufficient comparison are done in comparison experiments, which doesn’t meet the requirements for the top meeting to publish.
% \smalltitle{Q1} As you mentioned, it is possible to utilize the synthesized data sets using generative model in order to recovery pruned model. However, our method is a "data-free" restoration method on pruned model,which means we do not use any data like synthesized images.

% o	(1) What is the limitation of this method? Can it be used to various pruned networks?
\smalltitle{Q1} As shown in Appendix, we would say our limitation is that LBYL is more effective in restoring larger CNNs than in already tiny networks like MobileNets of using 1x1 filters.

% Therefore, we will further study in order to apply our method on flop efficient convolutional filter (e.g., 1x1 conv, group conv).

% (2) The proposed method is only compared with NM and pruned, I’m confused that why NM is worse than pruned, as listed in Table 4?
\smalltitle{Q2} In terms of the final accuracy, as mentioned in the introduction, we would say there is no big difference among pruning criteria as long as we can fine-tune the network with lots of data for a very long training time (Liu et al. 2019). Our best guess on why NM becomes eventually worse than just pruned model is that NM can choose a dissimilar filter to recover each pruned one in some layers, which makes hard for fine-tuning to improve the performance. 

% Therefore, NM is hard to fine-tuned than pruned model.

% similarity based restoration like NM ignores that as a layer is deeper, pairwise similarity on filters gets extremely low. That is, 

% Our method proposes a restoration method on pruned model using filter pruning. Therefore we compared with NM,which is the latest data-free compensation method as mentioned in Related Works. 

% (3)The author should provide more comparisons with the related latest methods in the quantitative results, such as
\smalltitle{Q3} Our LBYL method is a data-free restoration method for pruned models without fine-tuning. All the works suggested by the reviewer require a heavy fine-tuning process with synthetic data (not real data though) generated by some kinds of GAN. The term \textit{data-free} we meant in this work is not training any data samples regardless of whether they are real or synthetic for the fast restoration of pruned networks.

% a huge training time and computation resource 

% The latest data-free compression methods that you mentioned is based on generative model like GANs,which requires a huge training time and computation resource while our method use a little computation compared to training generative models,which is quite practical scenarios like edge devices.

\end{document}



\end{document}


