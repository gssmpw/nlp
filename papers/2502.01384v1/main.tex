%%%%%%%% ICML 2025 EXAMPLE LATEX SUBMISSION FILE %%%%%%%%%%%%%%%%%

\documentclass{article}

% Recommended, but optional, packages for figures and better typesetting:
\usepackage{microtype}
\usepackage{graphicx}
\usepackage{subfigure}
\usepackage{booktabs} % for professional tables

% hyperref makes hyperlinks in the resulting PDF.
% If your build breaks (sometimes temporarily if a hyperlink spans a page)
% please comment out the following usepackage line and replace
% \usepackage{icml2025} with \usepackage[nohyperref]{icml2025} above.
\usepackage{hyperref}


% Attempt to make hyperref and algorithmic work together better:
\newcommand{\theHalgorithm}{\arabic{algorithm}}

% Use the following line for the initial blind version submitted for review:
%\usepackage{icml2025}

% If accepted, instead use the following line for the camera-ready submission:
\usepackage[accepted]{icml2025}

% For theorems and such
\usepackage{amsmath}
\usepackage{amssymb}
\usepackage{mathtools}
\usepackage{amsthm}

% if you use cleveref..
\usepackage[capitalize]{cleveref}

%macros
\newcommand{\thought}[1]{{\color[rgb]{0.2,0.39,0.66}(#1)}}
\newcommand{\todo}[1]{{\color[rgb]{1.0,0.0,0.0}(#1)}}
\newcommand{\hsh}[1]{{\color{green!50!black} Henrik: #1}}
\newcommand{\st}[1]{{\color{red!50!black} Sebastian: #1}}

\newcommand{\ulm}[1]{_{\scaleto{\mathrm{#1}}{3pt}}}
\newcommand\at[2]{\left.#1\right|_{#2}}











\newtheorem{assumption}{Assumption}

\DeclareMathOperator*{\argmax}{arg\,max}
\DeclareMathOperator*{\argmin}{arg\,min}

\newcommand{\swname}[1]{\texttt{#1}}
\newcommand{\ie}{i\/.\/e\/.,\/~}
\newcommand{\eg}{e\/.\/g\/.,\/~}
\newcommand{\cf}{cf\/.\/~}

\newcommand{\fig}{Fig\/.\/~}
\newcommand{\defn}{Def\/.\/~}
\newcommand{\sect}{Sec\/.\/~}
\newcommand{\tabl}{Tab\/.\/~}
\newcommand{\algo}{Algorithm~}
\newcommand{\theo}{Theorem~}

\newcommand{\bnnl}{3 hidden layers}
\newcommand{\bnnn}{50 neurons}
\newcommand{\bnna}{tanh activations}

\newcommand{\capt}[1]{\mdseries{\emph{#1}}}

\newcommand{\videolink}{at \url{https://youtu.be/_d7AqTRjz6g}}
\newcommand{\codelink}{\url{https://github.com/wheelbot/mini-wheelbot}}

\newcommand{\fakepar}[1]{\vspace{0mm}\noindent\textbf{#1.}}

\newcommand{\needref}{\textcolor{red}{[REF]}}

\newcommand{\plotfontsize}{9pt}

%%%%%%%%%%%%%%%%%%%%%%%%%%%%%%%%%%%%%%%%%%%%%%%%%%%%%%%%%%%%%%%%%%%%%%%%%%%%%%

%% Beautiful mathematics
\usepackage{amsmath, amssymb, amsfonts} 
\usepackage{nicefrac}
\usepackage{mathtools}
\usepackage{bm, bbm}
\usepackage[scr=boondoxo,scrscaled=1.05]{mathalfa}

%% References in the correct format 
%\usepackage[square,numbers]{natbib}
%\def\bibfont{\footnotesize} % fix to have the same font size as without natbib

\usepackage[sort, compress, space]{cite}            


%% Enumerate nicely 
\usepackage{enumitem}

%% Different color comments and commenting large parts of the text
\usepackage{xcolor}
\usepackage{comment}
\usepackage{soul}

%% Hyper references
\usepackage{hyperref}
\usepackage{cleveref}
%\usepackage[numbers]{natbib}

\usepackage{tikz}
%\usepackage{thm-restate}
%% Appendix package
%\usepackage{appendix}

%% Random text to test spacing 
\usepackage{blindtext}

\usepackage{afterpage}

\usepackage{algorithm, algorithmic}    



\usepackage{dsfont}

\usepackage{tikz}
\usepackage{graphicx}
\usepackage{tikzscale}
\usepackage{pgfplots}
\pgfplotsset{compat=newest}
\usepackage{xfrac}

\usepackage{thm-restate}

%\usepackage{subcaption}

\usepackage{balance}

\usepackage{cite}
\usepackage{amsmath,amssymb,amsfonts}
\usepackage{balance}
\usepackage{algorithmic}
\usepackage{graphicx}
\usepackage{textcomp}
\usepackage{xcolor}
\usepackage{amsmath}
\usepackage{amssymb}
\usepackage[mathscr]{euscript}
\usepackage{comment}
\usepackage{xcolor}
\usepackage{enumitem} 
\usepackage{amsthm}



%%%%%%%%%%%%%%%%%%%%%%%%%%%%%%%%
% THEOREMS
%%%%%%%%%%%%%%%%%%%%%%%%%%%%%%%%
\theoremstyle{plain}
\newtheorem{theorem}{Theorem}[section]
\newtheorem{proposition}[theorem]{Proposition}
\newtheorem{lemma}[theorem]{Lemma}
\newtheorem{corollary}[theorem]{Corollary}
\theoremstyle{definition}
\newtheorem{definition}[theorem]{Definition}
\newtheorem{assumption}[theorem]{Assumption}
\theoremstyle{remark}
\newtheorem{remark}[theorem]{Remark}

% Todonotes is useful during development; simply uncomment the next line
%    and comment out the line below the next line to turn off comments
%\usepackage[disable,textsize=tiny]{todonotes}
\usepackage[textsize=tiny]{todonotes}


% The \icmltitle you define below is probably too long as a header.
% Therefore, a short form for the running title is supplied here:
\icmltitlerunning{Fine-Tuning Discrete Diffusion Models with Policy Gradient Methods}

\begin{document}

\twocolumn[
\icmltitle{Fine-Tuning Discrete Diffusion Models with Policy Gradient Methods}

% It is OKAY to include author information, even for blind
% submissions: the style file will automatically remove it for you
% unless you've provided the [accepted] option to the icml2025
% package.

% List of affiliations: The first argument should be a (short)
% identifier you will use later to specify author affiliations
% Academic affiliations should list Department, University, City, Region, Country
% Industry affiliations should list Company, City, Region, Country

% You can specify symbols, otherwise they are numbered in order.
% Ideally, you should not use this facility. Affiliations will be numbered
% in order of appearance and this is the preferred way.
%\icmlsetsymbol{equal}{*}

\begin{icmlauthorlist}
\icmlauthor{Oussama Zekri}{ens}
\icmlauthor{Nicolas Boull\'e}{imperial}
\end{icmlauthorlist}

\icmlaffiliation{ens}{Department of Mathematics, ENS Paris-Saclay, France.}
\icmlaffiliation{imperial}{Department of Mathematics, Imperial College London, UK}

\icmlcorrespondingauthor{Oussama Zekri}{oussama.zekri@ens-paris-saclay.fr}
\icmlcorrespondingauthor{Nicolas Boull\'e}{n.boulle@imperial.ac.uk}

% You may provide any keywords that you
% find helpful for describing your paper; these are used to populate
% the "keywords" metadata in the PDF but will not be shown in the document
\icmlkeywords{Machine Learning, ICML}

\vskip 0.3in
]

% this must go after the closing bracket ] following \twocolumn[ ...

% This command actually creates the footnote in the first column
% listing the affiliations and the copyright notice.
% The command takes one argument, which is text to display at the start of the footnote.
% The \icmlEqualContribution command is standard text for equal contribution.
% Remove it (just {}) if you do not need this facility.

\printAffiliationsAndNotice{}  % leave blank if no need to mention equal contribution
%\printAffiliationsAndNotice{\icmlEqualContribution} % otherwise use the standard text.

\begin{abstract}
Discrete diffusion models have recently gained significant attention due to their ability to process complex discrete structures for language modeling. However, fine-tuning these models with policy gradient methods, as is commonly done in Reinforcement Learning from Human Feedback (RLHF), remains a challenging task. We propose an efficient, broadly applicable, and theoretically justified policy gradient algorithm, called Score Entropy Policy Optimization (\SEPO), for fine-tuning discrete diffusion models over non-differentiable rewards. Our numerical experiments across several discrete generative tasks demonstrate the scalability and efficiency of our method. Our code is available at \href{https://github.com/ozekri/SEPO}{https://github.com/ozekri/SEPO}.
\end{abstract}

\section{Introduction}\label{sec:intro}

Diffusion models have become efficient generative modeling tools in various tasks, including image and video generation~\cite{song2020score,ho2020denoising}. Although most of the applications of diffusion models depend on a continuous state space (such as images), recent works extended these models to discrete settings, enabling their use in language modeling and other discrete generative tasks~\cite{sun2022score, campbell2022continuous, austin2021structured, benton2022denoising}. Moreover, several studies showed that these models can be competitive with autoregressive models, such as GPT~\cite{brown2020language} or Llama~\cite{touvron2023llama}, while allowing for more flexible generation as opposed to next-token prediction~\cite{lou2023discrete, sahoo2024simple, shi2024simplified}. These discrete diffusion models hold great promise if they can be scaled up to natural language processing tasks.

However, fine-tuning discrete diffusion models remains a challenging task. Different approaches, such as classifier guidance \cite{ho2022classifier,nisonoff2024unlocking,gruver2024protein} or steering \cite{rector2024steering} often suffer from scalability issues or intractable training objectives. In this work, we focus on fine-tuning based on reinforcement learning (RL), where the aim is to maximize an objective by modifying the weights of a pre-trained model. The limitations of the existing methods mentioned earlier highlight the need for novel and efficient methodologies to address the unique challenges of discrete diffusion models.

More specifically, sampling from a categorical distribution, which is the type of distribution arising in the discrete setting, is a non-differentiable procedure that cannot be handled by gradient-based optimization algorithms. To bypass this issue, a recent work by \citealt{wang2024fine} proposes to fine-tune discrete diffusion models through direct backpropagation of rewards with the Gumbel-Softmax trick~\cite{jang2017categorical}. While this is an interesting approach, it cannot handle non-differentiable rewards. Moreover, when working at scale, the size of a reward model that one wants to differentiate with backpropagation may quickly become memory-intensive. 

Finally, methods that achieved state-of-the-art results in Reinforcement Learning from Human Feedback (RLHF) are policy gradient method, namely Proximal Policy Optimization (PPO) \cite{schulman2017proximal} or Group Relative Policy Optimization (GRPO) \cite{shao2024deepseekmath} due to their stability, efficiency, unbiased gradient estimates, and mechanisms like trust region constraints to handle noisy feedback. These methods deal with the zero-order derivative of the reward oracle, in contrast to \cite{wang2024fine} that uses the first-order derivative of the reward oracle.

In this work, we focus on developing policy gradient methods specifically tailored to discrete diffusion models to improve performance and robustness, and introduce a Score Entropy Policy Optimization (\SEPO), for fine-tuning discrete diffusion models. Unlike \cite{wang2024fine}, our approach does \emph{not} require the reward function $R$ to be differentiable, which expands the possibilities for fine-tuning beyond Direct Reward Backpropagation. Our method provides a unified framework for optimizing discrete diffusion models.

\begin{figure*}[!t]
    \centering
    \includegraphics[width=0.9\textwidth]{figs/drawings/dessin.pdf}
    \caption{Illustration of the iterative fine-tuning process for discrete diffusion models using policy gradient methods. The initial model $\overline{Q}_{\pre}$ \textcolor{bluerow}{(conditionally)} generates responses, which are evaluated by a reward function. Based on this feedback, the model is updated iteratively using Score Entropy Policy Optimization (\SEPO), an efficient policy gradient algorithm for optimizing (non-differentiable) rewards. This process improves the model over multiple iterations, leading to the final fine-tuned model $\overline{Q}_{\theta^{\star}}$.}


    \label{fig:main_fig}
\end{figure*}

\subsection{Main contributions}

Our contributions are summarized as follows:
\begin{enumerate}[leftmargin=*, label={\arabic*)}]
    \item We provide an explicit characterization of policy gradient algorithms for discrete diffusion models in the concrete score matching framework. This allows the use of non-differentiable rewards in discrete fine-tuning tasks \emph{without} steering and guidance mechanisms.

    \item We propose an efficient and scalable algorithm based on policy gradient methods~\cite{schulman2017proximal,shao2024deepseekmath}, called Score Entropy Policy Optimization (\SEPO), for discrete diffusion. We also introduce a gradient flow alternative that improves sample quality at a higher complexity cost.
    
    \item We perform numerical experiments on DNA fine-tuning and natural language tasks to demonstrate the performance of our methods.
\end{enumerate}

\section{Background and preliminaries}\label{sec:background}

\subsection{Related works}

\paragraph{Inference-time techniques.}
Inference-time techniques are simple yet effective as they require no fine-tuning or training when reward functions are available. Recent studies \cite{singhal2025general,ma2025inference} showed that they can achieve competitive performance by scaling computational resources. Although inference-time techniques offer distinct advantages, they typically result in longer inference times compared to fine-tuned models. The key considerations for these techniques include computational efficiency and differentiability of the reward \cite{uehara2025reward}.

\paragraph{Policy gradients algorithms.}
Policy gradient algorithms are a key class of reinforcement learning methods that optimize parameterized policies by directly maximizing expected returns. Modern implementations include Proximal Policy Optimization \cite{schulman2017proximal} or Group Relative Policy Optimization \cite{shao2024deepseekmath}. These algorithms are highly sensitive to policy design since the architecture impacts expressiveness, optimization stability, and exploration.

\paragraph{Fine-tuning diffusion models with Reinforcement Learning.}
In the case of continuous diffusion models, fine-tuning via policy gradients has been proposed \cite{fan2024reinforcement,li2024learning,black2023training,ren2024diffusion}. In a more recent study, \cite{marion2024implicit} implements \texttt{REINFORCE} algorithm~\cite{williams1992simple} for continuous diffusion models in a single-loop algorithm, avoiding nested optimization. However, extending these approaches to discrete diffusion models is more challenging. This work adapts these studies to the discrete case and extends them to general policy gradient algorithms.

\subsection{Discrete Diffusion}

In discrete diffusion models, the dynamics of a single particle is described by a continuous-time Markov chain (CTMC), denoted as a stochastic process $(x_t)_{0 \leq t \leq \horizon}$ operating on a finite space $\statespace = \{\alpha_1,\hdots,\alpha_m\}^n$. Here, $(\alpha_i)_{1\leq i \leq m}$ represents the possible states that form a vocabulary of size $m$, and $n$ is the length of the sequences, which is a fixed number known as \textit{context window} or \textit{block size}. Typically, it describes sequences of tokens or image pixel values. While the size $\cardinality \coloneqq \lvert \statespace \rvert = m^n$ of $\statespace$ is exponential in $n$, deep neural networks like transformers \cite{vaswani2017attention} were shown to perform and generalize well on these incredibly large state spaces \cite{zekri2024large}.

\paragraph{Forward process.}
At any given time $t$, the distribution of a particle $x_t$ is given by $\bp_t$, which lies within the probability simplex $\simplex \subset \R^{\cardinality}$. The forward process is a noising process that maps the initial data distribution $\bp_0 \coloneqq \pdata$ to some final noisy distribution $\bp_\horizon \coloneqq \preff$, which is easy to sample. During the noising forward process, the particle's probability transitions between states are given by a rate matrix $\Qt \in \R^{d \times d}$, indexed by $\statespace$,
through the following equation:
\begin{equation}\label{eq:fokker_planck_forward}
\frac{\ud \bp_t}{\ud t} = \Qt\bp_t, \quad t\in[0,T].
\end{equation}
The time reversal of this equation is known as~\cite{kelly2011reversibility},
\begin{equation}\label{eq:fokker_planck_timerev}
\frac{\ud \bp_{\horizon-t}}{\ud t} = \Qback_{\horizon-t} \bp_{\horizon-t}, \quad t\in[0,T],
\end{equation}
where for $x, y \in \statespace$, \[ \Qback_{t}(x, y) =\begin{cases}
\frac{\bp_t(x)}{\bp_t(y)} Q_t(x, y), & x \neq y, \\
-\sum_{z \neq x} \Qback_{t}(z,y), & x=y.
\end{cases}\]

\paragraph{Concrete score matching.}
\citealt{lou2023discrete} recently showed that one can approximate \cref{eq:fokker_planck_timerev} via concrete score matching \cite{meng2022concrete}. This is done by learning the concrete score as $s_\theta(x,t)_y \approx \bp_t(x)/\bp_t(y)$ with a sequence-to-sequence neural network $s_\theta$ parametrized by $\theta \in \R^p$. We emphasize that this setup includes the simplified approaches detailed in \cite{sahoo2024simple,shi2024simplified}. The resulting process is described by the following equation:
\begin{equation}\label{eq:fokker_planck_timerev_approx}
\frac{\ud \bq^\theta_t}{\ud t} = \Qback_{\horizon-t}^\theta \bq^\theta_t, \quad t\in[0,T],
\end{equation}
where the denoising process $\bq^\theta_t\approx \bp_{\horizon-t}$ maps $\bq^\theta_0 \coloneqq \preff$ to $\bq^\theta_\horizon \coloneqq \target(\theta)$, and $\theta$ is learned to achieve $\target(\theta) \approx \pdata$. The matrix $\Qback_{t}^\theta$ is defined for $x, y \in \statespace$ as 
\[\Qback_{t}^\theta(x, y) = \begin{cases}
s_\theta(x,t)_y Q_t(x, y), & \text{if } x \neq y, \\
-\sum_{z \neq x} \Qback_{t}^\theta(z,y), & \text{if } x=y.
\end{cases}\]
Note that, in practice, the quantity $s_\theta(x,t)_y$ is available for all $y\in \statespace$ at Hamming distance~\cite{hamming1950error} one of $x$, i.e., the states $y$ that differ from $x$ by exactly one token. This represents only $\mathcal{O}(m n)$ ratios, instead of $\mathcal{O}(m^{2n})$ \cite{campbell2022continuous,lou2023discrete}.

\paragraph{Sampling strategies} 
Sampling discrete diffusion models involves selecting efficient strategies to simulate the backward equation \eqref{eq:fokker_planck_timerev_approx}, while balancing computational cost and sample quality. Among other strategies for CTMCs, sampling can be done via the \textit{tau-leaping} algorithm \cite{gillespie2001approximate}, which implements an Euler
step at each position $i$ simultaneously and independently:
\begin{equation}\label{eqn:tau_leap_euler}
    \bq_t(x_{t - \Delta t}^i |x_t^i) = \delta_{x_t^i}(x_{t - \Delta t}^i) + \Delta_t \Qback_{\horizon-t}^\theta(x_t^i, x_{t - \Delta t}^i)
\end{equation}
Discrete diffusion models can also be used to perform flexible \textit{conditional sampling} \cite{lou2023discrete}. Unlike \textit{unconditional sampling}, which samples $\bq_t(x_{t-\Delta t} | x_t)$, we incorporate auxiliary data $\mathbf{c}$ by modifying the probability to be sampled to $\bq_t(x_{t-\Delta t} | x_t, \mathbf{c})$.
Finally, the number of reverse diffusion steps, $\horizon$, directly impacts computational efficiency and sample fidelity, with larger $\horizon$ providing more accurate approximations of the target distribution at a higher computational cost.

\subsection{Fine-tuning with Reinforcement Learning}

After the pretraining phase, a discrete diffusion model with learned parameter $\pre$ aims to approximate $\pdata$, in the sense $\target(\pre) \approx \pdata$. Our goal is to fine-tune the target distribution $\target(\theta)$ to increase a reward function $\reward:\statespace\to\mathbb{R}$, without having access to $\pdata$.

\paragraph{Minimization problem.} We focus on optimization problems over implicitly parameterized distributions. For a given family of functions $(\cF_t)_{t\in[0,T]}: \simplex \to \R$, we aim to minimize the loss function defined as
\begin{equation}\label{eq:main_prob}
 \ell_t(\theta):= -\cF_t(\bq^\theta_t),\quad t\in[0,T],
\end{equation}
over $\theta\in\R^p$. Classical choices of $\cF_t$ include $\cF_t(\bq^\theta_t) = \esp_{x \sim \bq^\theta_t}[R_t(x)]$, where $R_t=0$ for $t<T$ and $R_T=R$, to maximize a reward function $R : \statespace \to \R$, or $\cF_t(\bq^\theta_t) = -\KL(\bq^\theta_t \| \bq^{\pre}_t)$ to minimize the KL divergence of $p_t$ from a distribution $\bq^{\pre}_t$. As detailed in \cite{uehara2024understanding}, a typical fine-tuning algorithm for diffusion models combines these two terms as follows,
\begin{equation} \label{eq:def_loss_t}
    \ell_t(\theta)=  -\esp_{x \sim \bq^\theta_t}[R_t(x)] +\alpha\KL(\bq^\theta_t \| \bq^{\pre}_t),
\end{equation}
where $\alpha>0$ is a weighting factor. Following standard choices in the fine-tuning diffusion models with reinforcement learning literature \cite{black2023training,fan2024reinforcement,clark2023directly,uehara2024understanding}, we assume that $R_t=0$ for $t<T$ and $R_T=R$ in the rest of this paper. Therefore, the first term on the right side of \cref{eq:def_loss_t} is nonzero and equal to $\ell^R \coloneqq -\esp_{x \sim \target(\theta)}[R(x)]$ when $t=T$.

\paragraph{Loss reward gradient.}
To apply first-order optimization methods, one needs to compute the gradient $\nabla_\theta \ell^R(\theta)$. Since $\statespace$ is a finite space of size $\cardinality$, we have
\begin{equation} \label{eq:loss_pi_theta}
\nabla_\theta \ell^R(\theta) = -\nabla_\theta \big( \cF_T(\target(\theta)) \big) = -\firstvar^\top \nabla_\theta \target(\theta),
\end{equation}
where $\firstvar \in \R^\cardinality$ is the vector of first variations $\cF_T(\target(\theta))$ (see \cref{app:firstvar}). Importantly, we note that $\firstvar(\bp)(x) = R(x)$ for $x \in \statespace$, which does not involve the differentiability of $R$ (with respect to some embedding $\emb{\statespace}$ of the state space). One can then design deterministic non-differentiable functions that act on $\statespace$ as rewards, similar to those arising in RLHF, or elsewhere. This may include designing desired protein properties \cite{rector2024steering}.

\section{Methods}
%\subsection{Gradients estimation}
\subsection{Policy gradients for concrete score}

The gradient of the target distribution $\nabla_\theta \target(\theta)$, which appears in \cref{eq:loss_pi_theta}, can be calculated based on to its relationship with the concrete score $s_\theta$ as $s_\theta(x,0) = \target(\theta)/\target_x(\theta)$ for $x\in\statespace$. The following theorem shows that one can first compute $\nabla_\theta \target(\theta)$, and then $\nabla_\theta \ell^R(\theta)$  through a discrete analogue of the \texttt{REINFORCE} algorithm~\cite{williams1992simple}.

\begin{boxthm}[Discrete \texttt{REINFORCE} trick]\label{prop:stochastic_estimate}
With the notations introduced in \cref{sec:background}, applying the discrete \texttt{REINFORCE} algorithm to the concrete score changes \cref{eq:loss_pi_theta} to:
\[
\nabla_\theta\ell^R(\theta)\! = \! \sum_{x\in\statespace}\target_x(\theta)R(x)\!\sum_{\substack{y \in \statespace \\ y \neq x}}{\target_y(\theta)\nabla_\theta\log s_\theta(x,0)_y}.
\]
\end{boxthm}

\paragraph{Monte-Carlo estimation of the outer sum.}
The summand in \cref{prop:stochastic_estimate} involves the unknown distributions $\target_x(\theta)$ and $\target_y(\theta)$. While the outer sum can be estimated via Monte Carlo sampling, the inner sum is weighted by $\target_y(\theta)$. As noted in \cite{lou2023discrete}, a single $x\in\statespace$ provides access to every component of the concrete score $s_\theta(x,0)_y$, for $y\neq x$, and then to $\target_y(\theta)$ since it this is the only missing quantity in the tau-leaping sampling scheme \cref{eqn:tau_leap_euler}. It is then possible to compute the gradient as
\[\nabla_\theta\ell^R(\theta) = \esp_{x\sim\target(\theta)}[R(x)g(x,\theta)],\]
where $g(x,\theta) \coloneqq \sum_{\substack{y \in \statespace \\ y \neq x}}{\target_y(\theta)\nabla_\theta\log s_\theta(x,0)_y}$.

\paragraph{Importance sampling.}
Although this defines an unbiased gradient, the \texttt{REINFORCE} algorithm is known to have high variance and to not restrict large policy updates. To address the latter limitation and estimate $g(x,\theta)$, we build upon the core ideas introduced by Trust Region Policy Optimization (TRPO)~\cite{schulman2015trust}. Instead of sampling from $\target(\theta)$ one can leverage importance sampling through an old policy $\target(\old)$, and constraint the $\KL$ divergence between the old and the current policy as follows:
\begin{equation}\label{eq:grad_is}
\nabla_\theta\ell^{R}(\theta) = \esp_{x\sim\target(\old)}\left[R(x)\frac{\target_x(\theta)}{\target_x(\old)}g(x,\theta)\right].
\end{equation}
Once adapted for concrete score, this formulation leads us to the following result.

\begin{boxthm}[Importance sampling gradient]\label{thm:ppo_expression}
With the notations introduced in \cref{sec:background}, applying TRPO to \cref{eq:loss_pi_theta} yields: 
\begin{equation}\label{eq:ppo_gradient}
\nabla_\theta\ell^{R}(\theta) = \esp_{x\sim\target(\old)}\left[R(x)h(x,\theta)\right],
\end{equation} where \[\displaystyle h(x,\theta)\!=\!\sum_{\substack{y \in \statespace \\ y \neq x}}\!\target_y(\theta)\!\frac{\target_y(\theta)}{\target_y(\old)}\!\frac{{s_{\old(x,0)}}_y}{s_{\theta}(x,0)_y}\!\nabla_\theta\log s_\theta(x,0)_y.\]
\end{boxthm}

The quantity $h(x,\theta)$ is expressed in this way in \cref{eq:grad_is} to emphasize how the loss will be computed in practice. While being the founding step of state-of-the-art policy gradient algorithms, TRPO requires solving a constrained optimization problem at each step. However, thanks to \cref{thm:ppo_expression}, we can now build powerful, stable, scalable, and easy-to-implement policy gradient algorithms.

\subsection{\SEPO~: Score Entropy Policy Optimization}\label{sec:sepo_main}

Our algorithm relies on the ideas introduced in \cite{schulman2017proximal,shao2024deepseekmath}, but can be adapted to any policy gradient algorithm built on \texttt{REINFORCE} or TRPO. Inspired from these algorithms, we clip the following ratio that appears in the inner sum of \cref{thm:ppo_expression}:
\[
r_{x,y} = \frac{\target_y(\theta)}{\target_y(\old)}\frac{{s_{\old(x,0)}}_y}{s_{\theta}(x,0)_y}.
\]
to the interval $[1-\epsilon,1+\epsilon]$ for some hyperparameter $\epsilon>0$.
Another advantage of discrete diffusion models is their great generation flexibility. It is then be possible to apply our algorithm conditionally (via a training dataset, typically in RHLF) or unconditionally for fine-tuning. Hence, in the conditional form, \cref{eq:grad_is} becomes 
\[\esp_{z\sim\cD}\esp_{x\sim\target_{x|z}(\old)}\left[R(x)\frac{\target_x(\theta)}{\target_x(\old)}g(x,\theta)\right].\]
Instead of using directly the reward $R(x)$, we compute an advantage $A(x)$ to reduce the variance of the Monte-Carlo estimations. This quantifies how much better an action is compared to the expected return at a given state. A common approach in PPO \cite{schulman2017proximal} is to learn a value network to approximate the reward, and then define the advantage as $A(x) = R(x) - V(x)$. For GRPO \cite{shao2024deepseekmath}, the advantage is the standardized reward over each group. Specifically, for a group of outputs $x = \{x_1,\hdots,x_G\}$ the advantages are defined as 
\[A(x_i) = \frac{R(x_i) - \text{mean}(R(x))}{\text{std}(R(x))}, \quad i\in\{1,\hdots,G\}.\]

\begin{rmk} The loss function takes the form
\begin{equation}\label{eq:loss_principal}
    \ell^{A}(\theta) = \esp_{x\sim\target(\old)}[\sum_{\substack{y \in \statespace \\ y \neq x}}w_{x,y}\log s_{\theta}(x,0)_y],\end{equation}
where $w_{x,y} \!=\! \target_y(\theta)r_{x,y}$ is a coefficient and the log concrete score $\log s_{\theta}(x,0)_y$ is the only term with an attached gradient. PPO, GRPO, and other methods can be constructed by modifying the coefficient $w_{x,y}$. In \cref{app:unified_paradigm}, we present a unified framework encompassing methods that can be derived from \SEPO.
\end{rmk}

Optionally for $t\in [0,T]$, a $\KL(\bq^\theta_t \| \bq^{\pre}_t)$ term can also be added to the loss, as in \cref{eq:def_loss_t}. Although this is not absolutely necessary, as clipping already implicitly regularizes with a $\KL(\target(\theta) \| \target(\old))$ term \cite{schulman2017proximal,fan2024reinforcement}, the derivation is given in \cref{app:KL_grad}, for completeness. This leads to the Score Entropy Policy Optimization (\SEPO) algorithm described in \cref{alg:SEPO}.

\begin{algorithm}[htbp]
   \caption{\SEPO}
   \label{alg:SEPO}
\begin{algorithmic}[1]
     \STATE {\bf Require}: CTMC $\Qback^\theta$, iteration $S$, policy optimization iteration $K$
     \STATE Set $\theta_0$ and $\old$ to $\pre$
     \FOR{$s \in [1,\cdots,S]$} \label{line:data1}
    \STATE Sample from $\target(\old)$ with $\Qback^{\old}$
    \STATE Compute the reward and the advantage
    \STATE Optimize $\theta_s$ with $\ell^{A}$ for $K$ epochs \label{line:opti2}
    \STATE Set $\old$ to $\theta_s$
     \ENDFOR 
      \STATE {\bf Output}: $\theta_{S+1}$
\end{algorithmic}
\end{algorithm}

\SEPO~iteratively samples from the target distribution via a CTMC (Line $4$) and optimizes $\theta_s$ using an optimization objective (Line $6$), refining the policy with policy gradients. Specifically :
\begin{itemize}
    \item \textbf{Line 4}: 
    This step generates samples from the target distribution $\target(\old)$ using the CTMC $\Qback^{\old}$. This can be done in $O(1)$ time complexity by leveraging the queuing trick introduced in \citep[Alg.~3]{marion2024implicit}, at a higher memory cost.
    \item \textbf{Line 6}: 
    This step updates the parameters $\theta_s$ using a policy optimization algorithm based on the objective $\ell^{A}$ (see \cref{eq:loss_principal}). This means performing $K$ iterations of gradient ascent (or descent) on the policy loss function to improve the policy $\target(\theta_s)$ using the previously collected samples and computed advantages.
\end{itemize}

\subsection{Sampling through gradient flow}

\paragraph{Bilevel problem.} We use sampling to reach the limiting process of the backward distribution $\target(\theta)$. This procedure can be interpreted as optimizing a functional $\cG : \simplex \times \R^p \to \R$ over the probability simplex $\simplex\subset\R^d$ as
\[
\target(\theta)=\underset{\bp \in \Delta_{\cardinality}}{\argmin}~\cG(\bp,\theta).
\]
When $\target(\theta)$ is the limiting distribution of an infinite time process (e.g.,~Langevin diffusion in the continuous case, \citealt{langevin1908theorie,pavliotis2014stochastic}), one can recast \cref{eq:main_prob} as a bilevel optimization problem. This has been proposed by \citealt{marion2024implicit} in the continuous diffusion case and allows to efficiently alternate between optimizing one-step of the inner problem and one step of the outer problem.

\paragraph{Gradient flow interpretation.}
In our case, $\target(\theta)$ is reached with \emph{finite-time horizon}, in $\horizon$ steps of sampling. However, it is possible to reach $\target(\theta)$ in \emph{infinite-time horizon} by sampling from a specific time-homogeneous CTMC. The choice of the functional $\cG(\bp,\theta) = \KL(\bp||\target(\theta))$ leads to a gradient flow interpretation of sampling via a specific CTMC.

\begin{boxlem}[Gradient flow]\label{lem:wass_flow}
    Sampling from the following ordinary differential equation $$\frac{\ud \bp_t}{\ud t} = \Qwass_0 \bp_t, \quad \text{where }\Qwass_0 \coloneqq Q_0 + \Qback_{0},$$ implements a gradient flow for $\KL(~\cdot~||\pdata)$ in $\simplex$, with respect to a Wassertein-like metric.
\end{boxlem}

\paragraph{Corrector steps.}

Of course, $s_\theta$ is not perfectly learned in practice, and we just have access to the rate matrix $Q_0^{c,\theta} \coloneqq Q_0 + \Qback_{0}^\theta$. But this gives us insight into the choice of our sampling strategy, especially with predictor-corrector techniques for discrete diffusion, introduced in \cite{campbell2022continuous} and developed in \cite{zhao2024informed}. We will then sample from the time-homogeneous CTMC of rate $Q_0^{c,\theta}$ to reach $\target(\theta)$ with infinite-time horizon. Note that this does not require computing an integral compared to the time-inhomogeneous case. We are then optimizing a functional in Wassertein space through sampling~\cite{marion2024implicit,bonet2024mirror}.

Sampling from $\Qwass_0$ affects Line 4 of \cref{alg:SEPO}. In practice, the sample quality can be improved by adding corrector steps with $\Qwass_t = Q_t + \Qback_{t}^\theta$, as proposed in \cite{campbell2022continuous}. Once the process has run for $\horizon$ steps, multiple sampling iterations from $\Qwass_0$ can be performed.

\paragraph{Linear system characterization.}
In this case, $\nabla_\theta \target(\theta)$ in \cref{eq:loss_pi_theta} will be obtained by solving a linear system, using the implicit function theorem (see \cref{app:proof_linear_system}) on $\nabla_1 \cG$, through a corrected version denoted $\nabla^\eta_\theta \target(\theta)$. While both the evaluation of the derivatives and the inversion of this linear system can be done automatically \cite{blondel2022efficient}, it is costly given the dimensionality $\cardinality$. Instead, we provide the exact linear system as well as a closed form of the inverse in \cref{prop:grad_pi}.

\begin{boxprop}\label{prop:grad_pi}
    For each $\eta >0$, $\nabla^\eta_\theta \target(\theta)$ is the solution to a linear system of the form
    \[A_\eta \unknown = B_\eta \in \R^{\cardinality\times p},\] 
    where $A_\eta$ is a rank-$1$ update to the $d\times d$ identity matrix, whose inverse can be explicitly computed using the Sherman--Morrison formula.
\end{boxprop}

Note that this affects Line 6 of \cref{alg:SEPO}, where $\nabla_\theta \target(\theta)$ in \cref{eq:loss_pi_theta} is replaced by $\nabla^\eta_\theta \target(\theta)$.

\subsection{Convergence bounds}

From a high-level point of view, \cref{alg:SEPO} alternates between sampling and optimization steps. We can then view \cref{alg:SEPO} as the following coupled equations:
\begin{equation} \label{eq_ev_theta}
    \begin{aligned}
\ud\bq_s &= Q_0^{c,\theta_s} \bq_s{\ud s}, \\
\ud {\theta_s} &= - \stepsize \Gamma (\bq_s, {\theta_s}){\ud s},
\end{aligned}
\end{equation}
for $0\leq s\leq S$. The gradient used on line $6$ of \cref{alg:SEPO} depends both on $\bq_s$ and $\theta_s$, and we refer to it as $\Gamma$ (so that $ \nabla_\theta \ell^A(\theta_s)=\Gamma (\target(\theta_s), {\theta_s})$). To simplify the analysis, the evolution of both $\bq_s$ and $\theta_s$ is done in continuous time flow, for some $s\in [0,S]$, with $S >0$. Let $\|\cdot\|$ denote the Euclidean norm on $\R^p$. We reintroduce assumptions on $\target(\theta)$ and $\Gamma$ made in \cite{marion2024implicit}.
\begin{asm}\label{ass:gradient_bounded}
There exists $C\geq 0$ such that for all $x \in \statespace$ and $\theta\in \R^p$, $\|\nabla_\theta \target_x( \theta)\| \leq C$.
There exists $\varepsilon > 0$ such that for all $x \in \statespace$ and $\theta\in \R^p$, $\target_x( \theta) > \varepsilon$.
\end{asm}
This assumption states that the gradient of the target distribution is bounded. The second part is similar to the ambiguity of the language often considered when studying models acting on spaces like $\statespace$ \cite{zekri2024large,hu2024unveiling,xie2021explanation}.
\begin{asm}\label{ass:Gamma_Lipschitz}
There exists $C\geq 0$ such that for all $p, q \in \simplex$, $\theta \in \R^p$, $\|\Gamma(p, \theta) - \Gamma(q, \theta)\| \leq C_\Gamma \sqrt{\KL(p || q)}$.
\end{asm}
This assumption essentially states that the gradient $\Gamma$ is Lipschitz continuous with respect to the $\KL$ divergence on $\simplex$.

With all these elements in place, we establish the convergence of the average objective gradients.

\begin{boxthm}[Convergence of \cref{alg:SEPO}]\label{thm:stochastic_bound}
Let $S>0$ and $\theta_s$ be the solution to \cref{eq_ev_theta} with $\beta_s = \min(1, 1/\sqrt{s})$, for $s\in[0,S]$. Under Assumptions \
\ref{ass:gradient_bounded} and \ref{ass:Gamma_Lipschitz}, we have
$$\frac{1}{S} \int_0^S \esp[\|\nabla \ell^A(\theta_s)\|^2] \ud s = \mathcal{O}\left(\frac{1}{\sqrt{S}}\right),$$
as $S\to\infty$.
\end{boxthm}

\section{Experiments}
\subsection{DNA sequence modeling}\label{sec:dna_modeling}

In this first experiment, we employ the pretrained model of \cite{wang2024fine}, a masked discrete diffusion model \cite{sahoo2024simple} pretrained on $\sim700k$ DNA sequences of the Gosai dataset \cite{gosai2023machine}.

\begin{figure}[htbp]
\centering
\includegraphics[width=1\linewidth]{figs/plots/integrale_f_s.pdf}
\caption{Illustration of the theoretical bound of \cref{thm:stochastic_bound} using a CG-enhancing reward. The purple curve is obtained by running GRPO with gradient flow.}
\label{fig:integral_thm}
\end{figure}

We illustrate the upper bound of \cref{thm:stochastic_bound} in \cref{fig:integral_thm}, where we used a reward function designed to enhance CG content in DNA sequences. Details about the \textit{non-differentiable} reward function can be found in \cref{app:add_exp}. We applied \SEPO~GRPO with a group size $G = 50$, with $10$ gradient flow corrector sampling steps at the end.

\subsection{Discrete diffusion language modeling}
\subsubsection{Training}

We implement \SEPO~in an Actor-Critic PPO style to fine-tuning SEDD Medium Absorb \cite{lou2023discrete}, a discrete diffusion model with $320$M non-embedding parameters, pretrained on OpenWebText \cite{Gokaslan2019OpenWeb}.

\paragraph{Reward modeling.}
Following \cite{minChatGPT}, we put the initial GPT-2 weights \cite{radford2019language} in a  \textit{GPT-2 Vanilla} model. We then augment the architecture with LoRA \cite{hu2021lora}, and use it to train a Supervised Fine-tuning (SFT) model and a Reward model. We use half of the HH-RLHF dataset \cite{bai2022training} to train the SFT model in an autoregressive fashion, and the other half to train the reward model, which has a logistic output $R(x)$. The whole reward modeling pipeline is illustrated in \cref{fig:reward_modelling}.

\begin{figure}[htbp]
\centering
\includegraphics[width=\linewidth]{figs/drawings/Reward_modelling.pdf}
\caption{GPT-$2$ Reward modeling pipeline.}
\label{fig:reward_modelling}
\end{figure}

\paragraph{SEDD Medium fine-tuning.}
We use the same first part of the HH-RLHF dataset that was used to train the GPT-$2$ SFT model. We skip any SFT stage for SEDD as our algorithm is only designed for the RL fine-tuning part. We acknowledge that an SFT stage would be beneficial, rather than the ``cold start" approach RL that we adopt.

\begin{table}[htbp]
    \centering
    \begin{tabular}{|p{7.5cm}|}
        \hline
            {\fontfamily{lmr}\selectfont \textcolor{bluerow}{Human: Is poker a hard game to learn? 
Assistant: It can be a challenge for some players, but if you're interested in playing, it's not hard to get started. 
Human: Is there an online game I could learn on? 
Assistant:} There is an online game called PokerStars. There are also several free trials. Human: Is there a skill required in poker? 
Assistant: There are skills required when you play in poker. You could talk about what and who you see in the game, and there are a lot of rules, moves and techniques, when you play in poker\dots} \\
            \hline
        \end{tabular}
        \caption{\textbf{Completion generation.} During fine-tuning, prompt tokens $p$ sampled from the HH-RLHF dataset \cite{bai2022training} are given in \textcolor{bluerow}{blue}. We leverage conditional sampling of discrete diffusion models to generate completions $c$ in black and form a whole sequence $x=c|p$. This is an example of completion obtained during the training of SEDD-\SEPO-$1024$.}
        \label{tab:cond_gen}
\end{table} 

To generate responses, we leverage conditional sampling, which allows us to guide SEDD's output by conditioning on specific prompts. A prompt $p$ and its completion $c$ form a sequence $x = c|p$. We then denote by $\target_x(\theta) = \target_{c|p}(\theta)$ the target probability of a prompt and its completion. This approach enables the model to generate targeted completions that are subsequently evaluated by a reward model. Unlike traditional autoregressive sampling, where the model generates one token at a time based only on the previous context, we let the model perform a complete generation given the preceding context. We then only select the next $128$ tokens following the prompt. This procedure is illustrated in \cref{tab:cond_gen}.

Following \cite{ouyang2022training}, we augment the reward by a KL regularization between $\target_x(\theta)$ and $\target_x(\pre)$, as
\[
\Tilde{R}(x) = {R}(x) - \beta\KL(\target_x(\theta) \| \target_x(\pre)).
\]
We compute the advantage as $A(x) = R(x) - V(x)$, where the value loss is a standard mean squared error loss between the value and the reward. Following good practice, we set $\epsilon = 0.2$ in \cref{eq:loss_principal}.

The reward and critic networks are represented by two different instances of the GPT-2 reward model that we obtained before (see \cref{fig:reward_modelling}). Two other instances of SEDD Medium will be used. The first one represents the actor network that will be fine-tuned, while the second one (fixed weights) is useful to compute the regularized rewards. The whole SEDD fine-tuning pipeline is illustrated in \cref{fig:sepo_pipeline}.

\begin{figure}[htbp]
\centering
\includegraphics[width=\linewidth]{figs/drawings/SEPO_pipeline.pdf}
\caption{\SEPO~fine-tuning pipeline for SEDD Medium.}
\label{fig:sepo_pipeline}
\end{figure}

We fine-tune two versions of SEDD Medium, with a different number of denoising steps $T$ to measure the impact on the quality of the fine-tuning. The first version, SEDD-\SEPO-$128$ generates completions over $128$ denoising steps. The second instance, SEDD-\SEPO-$1024$ generates completions over 1024 steps. Both versions are trained for $7k$ steps on the HH-RLHF dataset.

\subsubsection{Evaluation}

\begin{table*}[t]
  \centering
  \small
        \begin{tabular}{|c| p{16cm}|}
            \hline
            \rotatebox[origin=r]{90}{\hspace{2pt} Prompt} & {\fontfamily{lmr}\selectfont \textcolor{bluerow}{I want you to act as a classical music composer. You will create an original musical piece for a chosen instrument or orchestra and bring out the individual character of that sound. My first suggestion request is ``I need help composing a piano composition with elements of both traditional and modern techniques."}} \\
            \hline
            \rotatebox[origin=r]{90}{\hspace{2pt} SEDD V.} & {\fontfamily{lmr}\selectfont``Hope that you are interested in this then contact me and here is a place you can look, below in my suggestion list" Human help: ````If you have any questions''." Instructor: ````Get the name on there and complete the first request list. Assistant: Below are all your human needs" Human "Ah this my help desk. I will be composing a composition with the sounds of both Classical ( Classical Classical Music) and both Modern (} \\
            \hline
            \rotatebox[origin=r]{90}{\hspace{2pt} \SEPO-$128$} & {\fontfamily{lmr}\selectfont I have had an excellent session with you in this vein. We are aware and take notice with your great work, please make sure to satisfy your own style and compliment our needs well. *Assistant and Manager take a track by track piece and listen through the finished piece for feedback* Assistant: I would like to consider composing a Classical musical using a chosen instrument or orchestra for a piano composition. My first suggested composer request is} \\
            \hline
            \rotatebox[origin=r]{90}{\centering \hspace{4pt}\SEPO-$1024$} & {\fontfamily{lmr}\selectfont You will have a role in writing, arranging, and editing the music using both traditional and amateur production techniques. If you can, you will have a role in ensuring that the music is presented as accurate and as coherent, and also as as personal and human as possible. If you can't, you will also serve as an engineer, schemer, or assistant.} \\
            \hline
        \end{tabular}
  \caption{Comparison of model responses for a prompt from the HH-RLHF dataset \cite{bai2022training}. We present outputs from the pretrained model and the fine-tuned \SEPO-$128$ and \SEPO-$1024$ variants. Additional qualitative samples can be found in Appendix~\ref{app:add_exp}.}
  \label{tab:qualitative_results}
\end{table*}

We use the $153$ prompts from the Awesome ChatGPT Prompts dataset \cite{awesome-chatgpt-prompts}. This dataset contains prompts that cover a wide range of topics, ideal to see what these $<1B$ parameter models are capable of, once fine-tuned.

\paragraph{Quantitative evaluation}

For each of our two models, SEDD-\SEPO-$128$ and SEDD-\SEPO-$1024$, we use a Judge LLM, GPT-$3.5$ Turbo \cite{brown2020language}, to determine which response is preferred between the response generated by the given model and the other. We also compare both of our models to the pretrained version of SEDD Medium. We also generate answers for different numbers of denoising steps $T \in \{128,512,1024\}$. The percentage of preferred responses for each evaluation is detailed in \cref{tab:sedd}.

\setlength{\tabcolsep}{0.5em}
\begin{table}[htbp] 
\centering
\scalebox{0.75}{
\begin{tabular}{lcccccc}
%\toprule
\toprule[\thick pt]%
\multicolumn{1}{c}{} & \multicolumn{3}{c}{\textcolor{blue-violet}{\textbf{SEDD-\SEPO-$128$}}} & \multicolumn{3}{c}{\textcolor{blue-violet}{\textbf{SEDD-\SEPO-$1024$}}}\\ 
\cmidrule(r{10pt}l{5pt}){2-4} 
\cmidrule(r{10pt}l{5pt}){5-7} 
 \multicolumn{1}{c}{Number of steps $T$} &  \multicolumn{1}{c}{$128$} & \multicolumn{1}{c}{$512$} & \multicolumn{1}{c}{$1024$} &  \multicolumn{1}{c}{$128$} & \multicolumn{1}{c}{$512$} & \multicolumn{1}{c}{$1024$} \\
%\midrule
\midrule[\thick pt]
% Données pour ETTh2
SEDD Vanilla & $71.2\%$ & $64.1\%$ & $67.9\%$ & $74.5\%$ & $\mathbf{75.8\%}$ & $73.2\%$ \\
\midrule[\thick pt]
% Données pour ETTm1
SEDD-\SEPO-$128$ & $\times$ &  $\times$ &  $\times$ & $63.1\%$ & $\mathbf{68.8\%}$ & $67.8\%$ \\
\midrule[\thick pt]
% Données pour ETTm2
SEDD-\SEPO-$1024$ & $36.9\%$ & $31.2\%$ & $32.2\%$ & $\times$ & $\times$ & $\times$ \\
\bottomrule[\thick pt]%
\end{tabular}
}
\caption{For each model SEDD-\SEPO-$128$ and SEDD-\SEPO-$1024$, we report the proportion of outputs deemed favorable by the Judge LLM when compared to those of other models, for different numbers of denoising steps $T \in \{128,512,1024\}$. \textbf{Best} results are highlighted in bold.}
\label{tab:sedd}
\end{table}

First, both SEDD-\SEPO~variants consistently outperform the pretrained SEDD Vanilla model across most prompts in the Awesome dataset. This highlights the effectiveness of reinforcement learning fine-tuning in enhancing response quality. 

Furthermore, SEDD-\SEPO-$1024$ outperforms SEDD-\SEPO-$128$ in general. This was expected, as the RLHF training was performed on completions of higher quality. According to \cite{lou2023discrete}, increasing the number of denoising steps improves generation quality, at a higher computational cost. However, while SEDD-\SEPO-$1024$ does not exhibit a statistically significant advantage over different values of $T$, SEDD-\SEPO-$128$ appears to perform better at the number of steps for which it was trained to generate responses. This suggests that a model performs best when evaluated with the number of denoising steps it was trained to generate, whereas larger models benefit from longer denoising schedules.

\paragraph{Qualitative evaluation}

We also provide some qualitative results. Some answers are displayed in \cref{tab:qualitative_results}, for each model and with $T=1024$ denoising steps. More answers and steps are displayed in \cref{app:add_exp}.

The qualitative results presented in \cref{tab:qualitative_results} highlight the diversity in the responses generated by three models (SEDD Vanilla, SEDD-\SEPO-$128$ and SEDD-\SEPO-$1024$) for a creative writing task. The prompt, which asks the model to act as a classical music composer and assist in creating a piano composition blending traditional and modern techniques, challenges the models to demonstrate creativity, coherence, and relevance. While the models vary in their coherence and alignment with the task, certain patterns emerge that reveal their strengths and weaknesses.

SEDD Vanilla's answer seems disjoint and lacks coherence. While it attempts to acknowledge the task of composing a classical-modern piano piece, the output contains redundant and nonsensical phrases (e.g., "Classical (Classical Classical Music) and both Modern"). This suggests that SEDD Vanilla struggles to maintain contextual relevance and generate meaningful content in such tasks.

Both SEDD-\SEPO-$128$ and SEDD-\SEPO-$1024$ answers display an improvement in structure and clarity compared to SEDD Vanilla. However, the lack of an SFT stage clearly appears: the output from SEDD-\SEPO-$1024$ seems more like a continuation of the prompt rather than a direct response. We explain this behavior because the model learned from the HH-RHLF dataset \cite{bai2022training} to create a conversation between an assistant and a human, rather than a direct output. This behavior was also observed during training, as in \cref{tab:cond_gen}.

\section{Conclusion}
  
We introduced \SEPO, a novel approach for fine-tuning discrete diffusion models using policy gradient methods. By extending previous works that applied these methods on continuous spaces, we developed a unified framework that adapts this methodology to the discrete case. Experimental results demonstrate the effectiveness of our approach in optimizing discrete diffusion models while addressing key challenges such as non-differentiability and combinatorial complexity. Future work includes further refining gradient estimation techniques and exploring applications in structured generative modeling.

\section*{Acknowledgements}
The authors would like to thank Pierre Marion, Anna Korba, and Omar Chehab for fruitful discussions. This work was supported by the Office of Naval Research (ONR), under grant N00014-23-1-2729. This work was done thanks to the ARPE program of ENS Paris-Saclay, which supported the visit of the first author to Imperial College London.

%\section*{Impact Statement}
%This paper presents work that aims to advance the field of Machine Learning. There are many potential societal consequences of our work, none of which must be specifically highlighted here.

% In the unusual situation where you want a paper to appear in the
% references without citing it in the main text, use \nocite

\documentclass{MITstyle}

%\usepackage[table]{xcolor}
\usepackage{chngcntr}
\usepackage{hyperref}
\usepackage{microtype}

\title{A Lightweight and Extensible Cell Segmentation and Classification Model for Whole Slide Images}

\author{Nikita Shvetsov~$^{1, }$\footnote{Correspondence e-mail: nikita.shvetsov@uit.no}, Thomas K. Kilvaer~$^{2, 3}$, Masoud Tafavvoghi~$^{4}$, Anders Sildnes~$^{1}$, \\ Kajsa Møllersen~$^{4}$, Lill-Tove Rasmussen Busund~$^{5, 6}$, Lars Ailo Bongo~$^{1}$ \\
%
\vspace{1em} % Space between authors and afilliations
%
\normalfont{\small $^{1}$Department of Computer Science, UiT The Arctic University of Norway}\\
\normalfont{\small $^{2}$Department of Oncology, University Hospital of North Norway}\\
\normalfont{\small $^{3}$Department of Clinical Medicine, UiT The Arctic University of Norway}\\
\normalfont{\small $^{4}$Department of Community Medicine, UiT The Arctic University of Norway}\\
\normalfont{\small $^{5}$Department of Medical Biology, UiT The Arctic University of Norway} \\
\normalfont{\small $^{6}$Department of Clinical Pathology, University Hospital of North Norway} %\vspace{2em}
}

\begin{document}
\maketitle

\section*{Abstract}

% \begin{abstract}
% Developing clinically useful cell-level analysis tools in digital pathology remains challenging due to limitations in dataset granularity, inconsistent annotations, computational demands of advanced models, and difficulties in integrating new technologies into clinical workflows. To address these challenges, we propose a multi-faceted solution that enhances data quality, model performance, and usability to create a lightweight and extensible cell segmentation and classification model.

% First, we update data labels by employing a cross-relabeling process that refines the labels of two existing datasets, PanNuke and MoNuSAC, to create a new unified dataset with enhanced granularity, encompassing seven distinct cell types. Second, we leverage the H-Optimus foundation model as a fixed encoder to improve feature representation for simultaneous cell segmentation and classification tasks. Third, to address the computational demands of foundation models, we employ knowledge distillation to reduce model size and complexity while maintaining comparable performance. Finally, to facilitate integration into clinical workflows, we integrate the distilled model into the QuPath software, a widely used open-source platform in digital pathology.

% Our results demonstrate improvements in cell segmentation and classification performance using the H‑Optimus-based model compared to a CNN-based model. Specifically, the average $R^2$ improved from 0.575 to 0.871, and the average $PQ$ score improved from 0.450 to 0.492, indicating better alignment with actual cell counts and enhanced segmentation and classification quality. Furthermore, the distilled student model maintains performance comparable to the larger foundation model while reducing the parameter count by a factor of 48.
% Overall, by reducing computational complexity and integrating it into existing workflows, the proposed approach may significantly impact diagnostic processes, reduce the workload of pathologists, and contribute to improved patient outcomes. Though our approach shows potential enhancements in efficiency and usability of cell segmentation and classification models in digital pathology, extensive validation is needed to deploy these models in clinical practice.
% \end{abstract}

%%% shortened abstract
\begin{abstract}
Developing clinically useful cell-level analysis tools in digital pathology remains challenging due to limitations in dataset granularity, inconsistent annotations, high computational demands, and difficulties integrating new technologies into workflows. To address these issues, we propose a solution that enhances data quality, model performance, and usability by creating a lightweight, extensible cell segmentation and classification model. 

First, we update data labels through cross-relabeling to refine annotations of PanNuke and MoNuSAC, producing a unified dataset with seven distinct cell types. Second, we leverage the H-Optimus foundation model as a fixed encoder to improve feature representation for simultaneous segmentation and classification tasks. Third, to address foundation models' computational demands, we distill knowledge to reduce model size and complexity while maintaining comparable performance. Finally, we integrate the distilled model into QuPath, a widely used open-source digital pathology platform. 

Results demonstrate improved segmentation and classification performance using the H-Optimus-based model compared to a CNN-based model. Specifically, average $R^2$ improved from 0.575 to 0.871, and average $PQ$ score improved from 0.450 to 0.492, indicating better alignment with actual cell counts and enhanced segmentation quality. The distilled model maintains comparable performance while reducing parameter count by a factor of 48. By reducing computational complexity and integrating into workflows, this approach may significantly impact diagnostics, reduce pathologist workload, and improve outcomes. Although the method shows promise, extensive validation is necessary prior to clinical deployment.
\end{abstract}
\clearpage

\section{Introduction}
In digital pathology, accurate segmentation and classification of cells are crucial for many diagnostic, prognostic, and predictive analyses \cite{Jaber_Beziaeva_etal._2019,Lin_Pan_etal._2022,Park_Ock_etal._2022,Shen_Choi_etal._2024}. Nowadays, developments in computational pathology offer multiple solutions \cite{H._Qu_P._Wu_etal._2020,Javed_Mahmood_etal._2020} to utilize cell-level datasets to train machine learning models that solve these problems. The quality and specificity of training datasets are critical for robust and accurate models. Adhering to the principle of "garbage in, garbage out", it is essential to ensure that these datasets are extensively and accurately labeled with distinct classes that reflect the diverse biological characteristics of different cell types. Unfortunately, the number of open-source datasets comprising such high-quality annotations is limited. Existing cell segmentation datasets \cite{Gamper_Koohbanani_etal._2019,Graham_Vu_etal._2019,Verma_Kumar_etal._2021} may offer extensive annotations for certain cell types while providing more general labels for others. For example, in PanNuke, which is one of the largest open-source datasets comprising labeled cells, various types of morphologically and functionally different inflammatory cells like macrophages and lymphocytes are clustered in a broad "inflammatory" class. Consequently, these classes are frequently omitted from analyses or aggregated into broader meta-classes \cite{Gamper_Koohbanani_etal._2020} and likely interfere with other cell classes included in the dataset. This and similar inconsistencies in annotation granularity limit the ability of machine learning models to learn the comprehensive and nuanced features necessary for accurate cell segmentation and classification. To address these challenges, methods for refining and standardizing dataset annotations are essential to enhance the quality of training data.

A complementary approach to mitigate the absence of high-quality training data is the use of foundation models. Foundation models as encoders are defined as large-scale, versatile networks pre-trained on vast, diverse datasets using self-supervised learning, contrasting with convolutional neural network (CNN) pre-trained encoders that rely on supervised learning with labeled data. In practice, foundation models leverage enormous amounts of weakly or unlabeled data from millions of whole slide images (WSIs) and employ self-attention mechanisms to capture long-range dependencies and global context \cite{Chen_Ding_etal._2024,Saillard_Jenatton_etal._2024,Vorontsov_Bozkurt_etal._2024,Xu_Usuyama_etal._2024}. As a consequence, foundation models are able to produce transferable feature representations across different cell types and tissue environments. The feature representations can be leveraged by decoder networks to produce segmentation masks and pixel-level classifications. Because foundation models have comprehensive feature representations, they can be effectively fine-tuned using much smaller amounts of cell-level data compared to the large datasets needed to train models from scratch. Furthermore, foundation models incorporate adversarial training elements or contrastive learning \cite{Chen_Ding_etal._2024,Xu_Usuyama_etal._2024}, enhancing their resilience and adaptability by exposing them to challenging and varied scenarios during training. This may result in more generalizable models, often making them well-suited for diverse and complex tasks in digital pathology.

Despite the inherent advantages of foundation models, their deployment for practical use faces its own obstacles. In particular, they require substantial computational power, financial investments and rigorous testing to ensure reliability and efficacy for a given task \cite{Akkus_Dangott_etal._2022,Dragomir_Cocuz_etal._2022,Go_2022,Jafri_Farooqui_etal._2024}. Moreover, while foundation models enhance feature representation and performance, they depend on the quality of available annotations for decoder fine-tuning and, like any other model, cannot resolve existing inconsistencies or ambiguities in data labels. Therefore, there remains a critical need for solutions that address both data quality and practical deployment considerations.
Further, integrating new technologies into existing clinical workflows often encounters resistance, as it necessitates adjustments to established diagnostic processes. So, there is a need to develop solutions that could be integrated into current practices, minimizing the burden on medical professionals to adopt new tools \cite{King_Williams_etal._2023}.

Existing solutions \cite{Goldsborough_Philps_etal._2024,Hörst_Rempe_etal._2024}, while addressing some aspects of these challenges, fall short in providing a comprehensive approach. To address the data quality and clinical deployment issues, we propose a multi-faceted solution that encompasses data refinement, model optimization, and integration with existing pathology tools (\hyperref[fig:fig1]{Figure 1}). The outcome is a lightweight cell segmentation and classification model that can be integrated into digital pathology workflows for practical clinical use.

\begin{figure}[h!]
    \centering
    \includegraphics[width=\textwidth, height=0.82\textheight, keepaspectratio]{images/Figure_1.pdf}
    \caption{Overview of the proposed solution, including 1) Data refinement using cross-relabeling, 2) Teacher model development and fine tuning, 3) Student model optimization with knowledge distillation and 4) Student model and QuPath integration}
    \label{fig:fig1}
\end{figure}
\clearpage

Our approach begins with preparing the data for the fine-tuning and training of the machine learning models. We create a refined dataset, acquired via cross-relabeling two cell-level datasets, enhancing annotation specificity and consistency of the labeled data. Subsequently, we create a cell segmentation and classification model based on the foundation model. We leverage the foundation model as a fixed encoder and fine-tune a decoder using the refined dataset to improve generalization across diverse tissue- and cell types.
To ensure that the model remains lightweight and deployable in a possibly resource-constrained environment, we employ knowledge distillation to approximate the functionality of the foundation model. Finally, to facilitate the practical application of our model in digital pathology workflows, we integrate it with the QuPath \cite{Bankhead_Loughrey_etal._2017} application. Each methodological component contributes to the overarching goal of enhancing model performance, generalizability, and usability in clinical settings.

The primary contributions of this paper are:
\begin{enumerate}
    \item \textit{Data labels refinement through cross-relabeling:}
    
    We propose a new method for refining labels of cell-level datasets through cross-relabeling. This method employs classification models to re-label broad and ambiguous instances, resulting in a more diverse dataset. Our evaluation demonstrates that these classification models achieve high accuracy on test subsets, indicating the reliability of the method for label refinement.

    \item \textit{Enhanced model performance via foundation models:}
    
    We employ a foundation model as a feature extractor for the cell segmentation and classification task. In comparison with training a CNN model from scratch, the foundation model backbone only needs fine-tuning, which significantly reduces training time, computational resources and data requirements. We show that using a foundation model encoder leads to better performance in cell segmentation and classification networks than using a CNN-based encoder. This improvement may enable the model to generalize more effectively across various tissue types and imaging methods.
    
    \item \textit{Model optimization through knowledge distillation:}
    
    We show that a smaller student model trained using knowledge distillation on the refined dataset obtained via our cross-relabeling approach from a foundation model achieves comparable performance in cell segmentation and quantification tasks. As a result, this model is more suitable for deployment in environments without high-performance computing resources.
    
    \item \textit{Integration with QuPath:}
    
    We integrate the distilled cell segmentation and classification model into QuPath, a widely used open-source digital pathology platform, to accelerate clinical adaptation by enabling pathologists to more easily incorporate advanced computational tools into their existing workflows.
\end{enumerate}

Through these methodological steps, we aim to bridge the gap between advanced machine learning techniques and practical clinical applications, making accurate and efficient digital pathology accessible in a broader range of healthcare settings.

\section{Refining Existing Datasets Using Cross-Relabeling}
To address the limitations of sparse and ambiguous labeling of cell-level datasets, we propose a generalizable cross-relabeling strategy that can be applied to any dataset containing broadly categorized or imprecisely labeled cell types. This approach involves training and subsequently leveraging classification models to refine broad categories into more specific or biologically relevant classes.
When applied to cell-level data, the methodology includes extracting individual cell images from the dataset patches, preprocessing these images to standardize the size and accommodate partial cells, and then training deep learning classifiers capable of distinguishing between the finer cell subtypes within the coarser categories. 
To illustrate our approach, we focus on the PanNuke \cite{Gamper_Koohbanani_etal._2020, Gamper_Koohbanani_etal._2019} and MoNuSAC \cite{Verma_Kumar_etal._2021} datasets that we have used to train models for cell quantification in our previous works \cite{Shvetsov_Grønnesby_etal._2022,Shvetsov_Sildnes_etal._2024}. We find that for better cell differentiation we have to introduce more granular labels. PanNuke includes a broad classification of "inflammatory" cells, encompassing lymphocytes, macrophages, and neutrophils. Each cell type differs significantly in structure, function, and clinical relevance. Conversely, MoNuSAC uses the label "epithelial" for a class that comprises both benign epithelial cells and malignant neoplastic cells. This practice makes it challenging to differentiate between benign and malignant epithelial cells in the dataset, which is a critical distinction when identifying tumor areas within tissue samples. To address these issues, we implement a cross-relabeling strategy as shown in \hyperref[fig:fig2]{Figure 2}. The key components are two classification models: one is trained on singular cell images from PanNuke data to classify the epithelial meta-class into epithelial and neoplastic classes. The other is trained on MoNuSAC to refine the inflammatory class into lymphocytes, neutrophils, and macrophages.

\begin{figure}[h!]
    \centering
    \includegraphics[width=\textwidth]{images/Figure_2.pdf}
    \caption{Refined dataset generation via cross relabeling}
    \label{fig:fig2}
\end{figure}

The refining approach consists of three consecutive steps. The first is the preprocessing step, in which we extract individual cells from both datasets (\hyperref[fig:fig3]{Figure 3}). The specifics of PanNuke and MoNuSAC patch preparation before cell preprocessing are provided in \hyperref[chap:S1]{Appendix S1}.

\begin{figure}[h!]
    \centering
    \includegraphics[width=\textwidth]{images/Figure_3.pdf}
    \caption{Cell instances preprocessing including (1) cell map extraction, (2) bounding box delineation, (3) adjusting cell boxes and (4) cropping and resizing of cell images}
    \label{fig:fig3}
\end{figure}

During preprocessing, we extract cell type maps from the ground truth label mask and calculate bounding boxes around each cell instance. To accommodate partial cells at patch borders, a common issue in cropped patch images, we employ mirror padding and extend the field of view of the cell label by 15 pixels to capture adjacent cells. We then crop and resize the identified regions to $64 \times 64$ pixels using bicubic interpolation.

The preprocessed PanNuke dataset comprises 68,031 neoplastic and 23,207 epithelial cell images, while MoNuSAC comprises  33,104 lymphocytes, 1,252 neutrophils, and 1,695 macrophages, which we subsequently use in training cell classification models and classifying the cell image data \hyperref[fig:S2]{Appendix Figure S2 (1)}. 

The next step is to train two distinct ResNet50-based classifiers tailored to address the specific labeling challenges inherent in each dataset. We use ResNet50 for classification models due to its proven effectiveness for image classification tasks in histopathology \cite{pan2022reviewmachinelearningapproaches}, and its compatibility with small images. For the PanNuke dataset, we design the classifier, trained on MoNuSAC data, to disaggregate the heterogeneous "inflammatory" cell category into distinct subtypes: lymphocytes, macrophages, and neutrophils. Similarly, for the MoNuSAC dataset, the classifier is trained on PanNuke data and distinguishes between benign and malignant epithelial cells within the overarching "epithelial" label. By applying these targeted classifiers to their respective datasets, we assign more specific labels to individual cell instances, thus enabling us to create a unified dataset.
To ensure a balanced representation of classes, we train both models on datasets that had been equalized to match the size of the least represented class. Thus, we obtain datasets comprising 23,207 samples per class for PanNuke and 1,252 samples per class for MoNuSAC data. Next, we partition both of them into training (70\%), validation (20\%), and testing (10\%) subsets. To mitigate the risk of overfitting, we use a single dropout layer with a rate of p=0.5 in both models and data augmentation using randomized color perturbations, rotation, and horizontal and vertical flipping. We employ AdamW optimizer and the cross-entropy loss function for the training criterion.

To evaluate the two trained models, we measure the classification accuracy on the respective test subsets. The accuracies on the test subset for both classifiers are presented in \hyperref[tab:1]{Table 1}. The PanNuke model achieves an average accuracy of 93.57\%, with higher accuracy for neoplastic cells (96.06\%) compared to epithelial cells (86.26\%). The confusion matrix in Figure A3.1 shows that the model predominantly distinguishes accurately between epithelial and neoplastic tissues, with a substantial number of correct classifications and relatively few misclassifications. The MoNuSAC model demonstrates an average accuracy of 98.92\%, excelling in classifying lymphocytes (99.67\%) and macrophages (94.12\%), with lower performance for neutrophils (85.71\%). The confusion matrix in Figure A3.2 shows that the model identifies lymphocytes and performs reasonably well with macrophages and neutrophils.

\begin{table}[h!]
\renewcommand{\arraystretch}{1.5}
  \centering
  \caption{Cell classification results for PanNuke and MoNuSAC trained models (CI 95\%).}
  \label{tab:1}
  \begin{tabular}{|l|c|c|}
   \hline
   %\rowcolor{gray!30}
    Accuracy               & PanNuke model              & MoNuSAC model              \\
    \hline
    Average      & 0.936 (0.931--0.941)         & 0.989 (0.986--0.993)        \\
    \hline
    Neoplastic   & 0.961 (0.956--0.965)         & -                          \\
    \hline
    Epithelial   & 0.863 (0.849--0.877)         & -                          \\
    \hline
    Lymphocytes  & -                          & 0.997 (0.995--0.999)        \\
    \hline
    Neutrophils  & -                          & 0.857 (0.796--0.918)        \\
    \hline
    Macrophages  & -                          & 0.941 (0.906--0.976)        \\
    \hline
  \end{tabular}
\end{table}

Finally, during the last step, we use the model trained on PanNuke data for epithelial cells in MoNuSAC and the model trained on MoNuSAC for the inflammatory cells class in PanNuke. Specifically, we use classifier models to relabel epithelial cells in MoNuSAC and inflammatory cells in PanNuke data. Then we combine cells with refined labels and the rest of the cells in both datasets to create a refined dataset (\hyperref[fig:S2]{Appendix Figure S2 (2)}). The process of relabeling cells and visualizing them on a patch is shown in \hyperref[fig:fig4]{Figure 4}. The cell counts in the refined dataset are provided in \hyperref[tab:S4]{Appendix Table S4}.

\begin{figure}[h!]
    \centering
    \includegraphics[width=\textwidth, height=0.42\textheight, keepaspectratio]{images/Figure_4.pdf}
    \caption{Cell relabeling procedure for epithelial and inflammatory cell classes}
    \label{fig:fig4}
\end{figure}

%\hfill

Relabeling and combining datasets have been explored in a prior study \cite{Parulekar_Kanwat_etal._2023}, where consecutive fine-tuning on multiple datasets was employed to account for hierarchical class label structures. While the method presented in \cite{Parulekar_Kanwat_etal._2023} is intuitive, it often lacks consistency and requires multiple fine-tuning runs, which can be cumbersome and time-consuming. 
In contrast, cross-relabeling simplifies this process by using specialized classification models tailored to each dataset's specific labeling challenges. This approach provides better transparency and produces a unified dataset encompassing seven distinct cell types across multiple tissue samples, enhancing data diversity for further model training or fine-tuning.

Despite these improvements, cross-relabeling does not entirely resolve issues related to poor labeling quality or the amount of labeled data. Specifically, our results show lower accuracies persist for underrepresented classes, such as macrophages, which may stem from a limited sample availability and intrinsic challenges in distinguishing these cells based solely on H\&E staining. Furthermore, while our method enhances label specificity, it relies on the initial quality of the broad labels; thus, any fundamental inaccuracies in the original annotations can propagate through the relabeling process. Addressing the overall problem of limited data labels may require integrating additional data sources or utilizing complementary immunohistochemical staining methods.
Although the reported performance metrics are obtained from evaluations on the native test sets of each dataset, it is important to note that the primary application of these classifiers is to perform cross-relabeling, where a model trained on one dataset (e.g., PanNuke) is applied to another (e.g., MoNuSAC) and vice versa. We acknowledge that a more systematic evaluation of cross-dataset generalization is needed and could be performed in future work.

Overall, the refined dataset produced by our approach can enhance the supervised training or fine-tuning of cell segmentation and classification models, especially those that utilize pre-trained foundation models to improve feature extraction robustness. In addition, these models can detect nuanced classes that enable researchers to conduct more detailed analyses of biological processes in computational pathology.

\section{Foundation models for robust cell segmentation and classification}

Accurate cell segmentation and classification in digital pathology are hindered by limited labeled data and the fact that conventional CNNs are unable to capture global contextual information due to their local receptive field constraints \cite{Gheflati_Rivaz_2022,Yang_Marcus_etal.}. Traditional approaches in cell quantification have predominantly relied on CNN encoders, such as ResNet50, given their proven effectiveness in semantic segmentation tasks \cite{Deshmane_2023,Graham_Vu_etal._2019,Mukasheva_Koishiyeva_etal._2024,Stringer_Wang_etal._2021}. However, approaches that include fine-tuning of pretrained CNNs, data augmentation, and stain normalization to partially increase data variability and address staining differences often fail to achieve the necessary generalization and robustness across diverse tissue types and staining conditions \cite{G._Wang_W._Li_etal._2018,Gao_Bagci_etal._2018,Karim_El_Khoury_Martin_Fockedey_etal._2021}.

To overcome these challenges, we leverage an encoder-decoder network that uses a foundation model as the encoder and a CNN upsampling decoder (\hyperref[fig:fig5]{Figure 5}) for simultaneous cell segmentation and classification in 2D patches extracted from WSIs. Foundation models with transformer-based architectures are viable alternatives to CNN-based encoders \cite{Shamshad_Khan_etal._2023,Sourget_2023}. They enable the creation of more advanced architectures that can decode or transform learned features more effectively \cite{Chen_Duan_etal._2023,Cheng_Misra_etal._2022,Xie_Wang_etal._2021}.

\begin{figure}[h!]
    \centering
    \includegraphics[width=\textwidth]{images/Figure_5.pdf}
    \caption{UNETR-like model with foundational model as backbone}
    \label{fig:fig5}
\end{figure}

By utilizing a transformer-based encoder, we incorporate global contextual information into the feature extraction process, which is a key advantage of such architectures \cite{Chen_Lu_etal._2021}. This foundation model integration facilitates accurate pixel-wise segmentation and classification without the need for extensive encoder training, thereby potentially improving generalization across varied cellular structures and tissue types.
In our implementation, we employ a modified UNETR \cite{Hatamizadeh_Tang_etal._2021} architecture that combines a vision transformer (ViT) \cite{Dosovitskiy_Beyer_etal._2021} encoder with a CNN-based decoder. The encoder utilizes the pretrained H-Optimus foundation model, which contains 1.1 billion parameters and is trained on over 500,000 H\&E stained WSIs \cite{Saillard_Jenatton_etal._2024}. We extract outputs from four evenly spaced transformer blocks $Z_i$, where $i \in [1, 14, 26, 38]$, to serve as residual connections for the CNN decoder. We select these blocks based on our observation that features from non-adjacent levels of the encoder lead to better overall performance on the test subset.

The CNN decoder upsamples the feature representations, acquired from the transformer blocks, to generate an intermediate vector that is handled by two task-specific layers that generate cell segmentation and classification masks. The first task-specific layer is the ‘Cellpose head’,  which is used to delineate cell instances. The layer generates horizontal and vertical gradient maps to form vector fields that are refined through gradient tracking in a post-processing step using the Cellpose algorithm \cite{Stringer_Wang_etal._2021}, known for its efficacy in cell segmentation tasks and generalizability across multiple domains \cite{Pachitariu_Stringer_2022,Stringer_Pachitariu_2024}. The second task-specific layer is the "Cell type head", which assigns labels to individual pixels. In the post-processing step, we determine the output classification label of each segmented cell instance by majority voting over the labeled pixels that comprise the cell in the segmentation map.

To evaluate model performance and measure the impact of adding a foundation model as backbone, we compare it to a ResNet50-based model. ResNet50 is a widely used solution for encoders in segmentation architectures in the medical domain \cite{Deshmane_2023,Graham_Vu_etal._2019,Mukasheva_Koishiyeva_etal._2024,Stringer_Wang_etal._2021}. For the H-Optimus-based model, we utilize frozen weights for the encoder and only fine-tune the decoder to take advantage of the extensive pre-training of the foundation model. For the ResNet50-based model we start with ImageNet \cite{Deng_Dong_etal.} weights and train both encoder and decoder parts. Hyperparameters for the training step are set to be identical, where possible, for comparable evaluation. 
For this evaluation, we deliberately use the PanNuke dataset to provide a standardized and controlled comparison between the H‑Optimus and ResNet50-based models (\hyperref[fig:S2]{Appendix Figure S2 (3)}). Specifically, we use two of the default PanNuke dataset splits (66\%) for training and validation, and reserve the third split (33\%) for testing.

To address the challenge of cell class imbalance in the PanNuke dataset, which is a common characteristic in most cell-level H\&E patch datasets, both models’ training processes employ a weighted loss function comprising cross-entropy and focal loss \cite{Lin_Goyal_etal._2018}. The focal loss component is adjusted with coefficients derived from each cell class' instance frequency, emphasizing learning from underrepresented classes and enhancing the model's sensitivity to rare but significant cellular patterns. The cross-entropy loss is augmented with spectral decoupling regularization \cite{Pezeshki_Kaba_etal._2021,Pohjonen_Stürenberg_etal._2022} and spatially varying label smoothing \cite{Islam_Glocker_2021}, which potentially stabilizes training and improves generalization in case of complex tissue morphologies. For optimization, we employ the \textit{AdamW} \cite{Loshchilov_Hutter_2019} to counter unbalanced class scenarios, with cosine annealing learning rate scheduler.

We utilize the scikit-learn library \cite{Van_der_Walt_Schönberger_etal._2014} and HoVer-Net \cite{Graham_Vu_etal._2019} implementations of $R^2$ (the coefficient of determination) and $PQ$ (panoptic quality) to evaluate our experiments. Complete mathematical formulations and detailed explanations of these metrics are provided in \hyperref[chap:S5]{Appendix S5}. To compute confidence intervals, we use nonparametric bootstrapping, where after calculating the metric on the full sample, we generated 1000 bootstrap replicates by resampling with replacement and then determined the 95\% confidence intervals as the 2.5th and 97.5th percentiles of the resulting empirical distribution.

%\hfill

The model comparisons are summarized in \hyperref[tab:2]{Table 2}. The H‑Optimus-based model achieves higher $R^2$ across all cell classes compared to the ResNet50-based model, which means that its predictions are more closely aligned with the PanNuke cell counts, indicating a stronger correlation with the observed data. Notably, the improvement of $R^2_{dead}$ may be an indicator of better global contextual representations provided by the foundation model backbone. In terms of segmentation and classification quality combined, measured by the PQ score, the H‑Optimus-based model demonstrates notable improvements across most cell classes. Overall, the average $R^2$ improved from 0.575 to 0.871, while the average $PQ$ score improved from 0.450 to 0.492, demonstrating better performance of the H-Optimus-based model.

\begin{table}[h!]
\renewcommand{\arraystretch}{1.5}
  \centering
  \caption{Cell quantification metrics for baseline and proposed models (CI 95\%).}
  \label{tab:2}
  \begin{tabular}{|l|c|c|}
    \hline
    %\rowcolor{gray!30}
    Metric             & Resnet50-based            & H-optimus-based              \\
    \hline
    $R^2_{neoplastic}$    & 0.681 (0.576--0.769)       & \textbf{0.941 (0.917--0.960)} \\
    \hline
    $R^2_{inflammatory}$  & 0.863 (0.778--0.903)       & \textbf{0.949 (0.918--0.966)} \\
    \hline
    $R^2_{connective}$    & 0.600 (0.488--0.698)       & 0.609 (0.436--0.772)          \\
    \hline
    $R^2_{dead}$          & 0.097 (-11.389--0.669)     & 0.925 (0.404--0.982)          \\
    \hline
    $R^2_{epithelial}$    & 0.635 (0.490--0.747)       & \textbf{0.930 (0.886--0.964)} \\
    \hline
    $PQ_{neoplastic}$       & 0.517 (0.499--0.535)       & \textbf{0.589 (0.575--0.604)} \\
    \hline
    $PQ_{inflammatory}$     & 0.455 (0.429--0.482)       & \textbf{0.528 (0.507--0.549)} \\
    \hline
    $PQ_{connective}$       & 0.416 (0.400--0.431)       & \textbf{0.451 (0.436--0.465)} \\
    \hline
    $PQ_{dead}$             & 0.374 (0.342--0.408)       & 0.292 (0.209--0.365)          \\
    \hline
    $PQ_{epithelial}$       & 0.488 (0.460--0.519)       & \textbf{0.599 (0.579--0.618)} \\
    \hline
  \end{tabular}
\end{table}

Our results  show that integrating the H‑Optimus foundation model within the UNETR architecture enhances the model's ability to segment and classify cells across diverse tissues from PanNuke data. The pretrained transformer encoder provides robust feature representations, resulting in higher average $R^2$ and $PQ$ scores compared to the CNN-based model. This leads to more reliable cell quantification and more accurate downstream analysis. Additionally, the streamlined fine-tuning process reduces computational overhead and training time, making the model more adaptable for new data.

Despite these advancements, the foundation model-based approach does not fully resolve all challenges related to cell segmentation and classification. We observe lower metric scores for underrepresented classes in the training data. Furthermore, foundation models typically encompass billions of parameters, resulting in substantial computational and memory requirements. It therefore poses challenges for deployment in resource-constrained environments, limiting their practical applicability in certain clinical settings.

\section{Model optimization via Knowledge Distillation}

To address the limitations posed by the extensive size of foundation models, we implement knowledge distillation — a model compression technique that leverages the teacher-student paradigm \cite{Hinton_Vinyals_etal._2015}. By training a smaller, more efficient student model to replicate the output of a larger, pre-trained teacher model, we retain performance while significantly reducing the model's complexity and resource requirements (\hyperref[fig:fig6]{Figure 6}).

\begin{figure}[h!]
    \centering
    \includegraphics[width=\textwidth, height=0.45\textheight, keepaspectratio]{images/Figure_6.pdf}
    \caption{Knowledge distillation framework for training a student model using a pre-trained teacher}
    \label{fig:fig6}
\end{figure}

We employ knowledge distillation to compress the H‑Optimus-based teacher model into a more efficient student model. The teacher model is the modified UNETR architecture with the H‑Optimus foundation model described in the previous chapter. The student model is based on a UNet architecture augmented with residual connections and incorporates a smaller ViT encoder with 9 million parameters \cite{Steiner_Kolesnikov_etal._2022,Wightman_2019}. 

First, we fine-tune the teacher model using the refined dataset from the cross-relabeling procedure (Section 2). Initially we train the decoder of the teacher model while keeping the encoder weights frozen. We split the refined dataset into train (70\%), validation (20\%) and test (10\%) subsets (\hyperref[fig:S2]{Appendix Figure S2 (4)}). During fine-tuning, we use the train and validation subsets, while leaving the test subset for model evaluation. We set the training procedure and model hyperparameters to be identical to those that were used to demonstrate the utility of foundation models for the simultaneous cell segmentation and classification task.

Next, we perform knowledge distillation from teacher to student using the refined dataset used to fine-tune the teacher model. The student model is trained to replicate the teacher model's outputs. We utilize a specialized loss function that aligns the student's predicted probability distribution with the teacher's, incorporating the teacher's class probability distribution derived from the output. Following the methodology of Hinton et al. \cite{Hinton_Vinyals_etal._2015}, we experiment with various hyperparameter settings for the temperature ($T$) and the balancing coefficients ($\alpha$ and $\beta$) in the loss function. We vary $T$ from 1 to 20 and adjust $\alpha$ and $\beta$ to balance the distillation and student losses. Through iterative tuning and evaluation, we identify that setting $T=14$, $\alpha=0.3$, and $\beta=0.7$ yields a configuration that converges and closely approximates the teacher model's performance during training.

Finally, we assess the performance of both models using the $R^2$ and $PQ$ (defined in \hyperref[chap:S5]{Appendix S5}) on the test set of the refined dataset (\hyperref[tab:3]{Table 3}). We observe that the 95\% confidence intervals overlap for most cell types, so we cannot claim statistically significant performance differences between the teacher and student models. One exception appears in the neoplastic class. The teacher model produces an $R^2$ of 0.919, while the student model shows an $R^2$ of 0.852. In addition, the student model achieves higher $PQ$ values for the neoplastic and connective classes, though the confidence intervals show overlap.

\begin{table}[h!]
\renewcommand{\arraystretch}{1.5}
  \centering
  \caption{Cell quantification metrics for teacher and distilled student models (CI 95\%).}
  \label{tab:3}
  \begin{tabular}{|l|c|c|}
    \hline
    %\rowcolor{gray!30}
    Metric & Teacher & Student \\
    \hline
    $R^2_{neoplastic}$    & \textbf{0.919} (0.898--0.939) & 0.852 (0.800--0.891) \\
    \hline
    $R^2_{lymphocyte}$    & 0.969 (0.956--0.977)         & 0.969 (0.956--0.978) \\
    \hline
    $R^2_{connective}$    & 0.694 (0.548--0.809)         & 0.618 (0.469--0.741) \\
    \hline
    $R^2_{dead}$          & 0.755 (0.400--0.908)         & 0.424 (0.100--0.731) \\
    \hline
    $R^2_{epithelial}$    & 0.922 (0.870--0.958)         & 0.843 (0.738--0.917) \\
    \hline
    $R^2_{macrophage}$    & 0.384 (-0.369--0.724)        & 0.704 (0.352--0.859) \\
    \hline
    $R^2_{neutrofil}$     & 0.854 (0.578--0.929)         & 0.833 (0.502--0.925) \\
    \hline
    $PQ_{neoplastic}$       & 0.581 (0.569--0.593)         & 0.601 (0.588--0.613) \\
    \hline
    $PQ_{lymphocyte}$       & 0.536 (0.520--0.553)         & 0.563 (0.544--0.579) \\
    \hline
    $PQ_{connective}$       & 0.436 (0.421--0.451)         & 0.457 (0.441--0.474) \\
    \hline
    $PQ_{dead}$             & 0.272 (0.235--0.315)         & 0.279 (0.201--0.369) \\
    \hline
    $PQ_{epithelial}$       & 0.522 (0.500--0.545)         & 0.530 (0.506--0.555) \\
    \hline
    $PQ_{macrophage}$       & 0.524 (0.459--0.588)         & 0.474 (0.405--0.543) \\
    \hline
    $PQ_{neutrofil}$        & 0.541 (0.490--0.592)         & 0.565 (0.522--0.607) \\
    \hline
  \end{tabular}
\end{table}


We further decompose the $PQ$ metric into its $SQ$ and $DQ$ components (\hyperref[tab:S6]{Appendix Table S6}). Both models produce nearly identical $SQ$ values, which indicates that they predict instance boundaries with similar precision. Although the student model shows some improvement in $DQ$ scores for certain classes, the confidence intervals overlap and do not confirm a statistically significant difference.

We observe that the student and teacher models yield comparable detection performance despite the student model using a much smaller and simpler architecture. A model with fewer parameters reduces the risk of overfitting when training data are scarce relative to the model’s complexity \cite{Farias_Ludermir_etal._2022}. The knowledge distillation process also encourages the student model to focus on the most generalizable detection features learned from the teacher. These factors enable the student model to achieve similar detection performance across different cell types.

Additionally, considering the model sizes reported in \hyperref[tab:4]{Table 4}, the distilled model achieves a significant reduction compared to the teacher model, with a 48-fold decrease in parameter count and a 5.5-fold reduction in on-disk size. In inference mode, the teacher model requires 16 GB of VRAM for a batch size of 32, while the distilled model only needs 3 GB of VRAM for the same batch size. These reductions make the distilled model significantly more practical for fine-tuning and deployment in resource-constrained environments.

\begin{table}[h!]
\renewcommand{\arraystretch}{1.5}
  \centering
  \caption{Parameter counts and size of teacher and distilled model}
  \label{tab:4}
  \adjustbox{max width=\textwidth}{%
  \begin{tabular}{|l|c|c|c|}
    \hline
    %\rowcolor{gray!30}
    Metric & H-optimus-based (Teacher) & mobileViT-based (Student) & Magnitude of difference \\
    \hline
    Parameters count       & 1,158,917,906   & \textbf{24,093,393}   & \textbf{48x}  \\
    \hline
    Estimated Total Size (MB) & 87,912       & \textbf{15,935}    & \textbf{5.5x} \\
    \hline
  \end{tabular}%
}
\end{table}

%\hfill

With recent advancements in complex network architectures and the use of pretrained encoders to achieve state-of-the-art performance \cite{Baumann_Dislich_etal._2024,Hörst_Rempe_etal._2024} in cell segmentation and classification tasks, model size, computational complexity, and processing times have increased. This limits the scalability and accessibility of these models. As we demonstrate, this may be mitigated using knowledge distillation. Studies in the field of natural language processing have demonstrated the efficacy of knowledge distillation in retaining the capabilities of the teacher model while achieving significant reductions in size and complexity \cite{Huangpu_Gao_2024,Sun_Yu_etal.}. 

We demonstrate the feasibility of knowledge distillation in digital pathology, specifically for cell segmentation and classification tasks. Moreover, we achieve this performance while also significantly reducing the parameter count. In addressing the challenge of knowledge transfer, we found that distillation from a transformer-based model to a smaller transformer is more straightforward than attempting to map transformer features to CNN blocks. In our experiments, using a CNN-based network as a student results in worse cell quantification performance due to the structural constraints of CNN feature space dimensions. 

Although our primary approach relies on a transformer-based student model that performs well, it can be further optimized to incorporate advantages from CNN architectures. For example, employing alternative techniques such as using ViT adapters \cite{Chen_Duan_etal._2023} or $1 \times 1$ convolutions to adjust feature map sizes may be beneficial for harnessing CNN advantages like enhanced local feature extraction. Moreover, if additional performance improvements are desired, the process can be further enhanced by applying supplementary knowledge distillation techniques, such as self-distillation \cite{Zhang_Song_etal._2019} or online distillation \cite{Houyon_Cioppa_etal._2023}.

Despite these promising results, further validation on independent datasets is necessary to fully understand the model's limitations. Underrepresented classes may pose challenges when addressing complex cases. Pathologists need to validate these models to adopt them in clinical settings. While the distilled models are smaller and more deployable, a technological gap persists because pathologists traditionally rely on established methods for inspecting WSIs and diagnosing diseases. Addressing the complexities involved in deploying models for inference and supporting pathologists in adopting new tools is essential for integrating these models into clinical workflows.

\section{Model integration with QuPath}
Digital pathology tools with graphical user interfaces are essential for visualizing and analyzing WSIs. To make our student model useful in clinical pathology workflows, it needs to be integrated into a tool that enables inspecting regions, creating annotations, and providing quantitative analyses of biomarkers. Therefore, we integrate the trained student model from the previous chapter into the QuPath open‑source platform \cite{Bankhead_Loughrey_etal._2017}. QuPath provides the required annotation, visualization, and analysis tools to interpret complex histological data, including workflows for cell segmentation, classification, and quantification (\hyperref[fig:fig7]{Figure 7}). 

\begin{figure}[h!]
    \centering
    \includegraphics[width=\textwidth]{images/Figure_7.pdf}
    \caption{Visualization of model-generated cell quantification annotations (left) and the corresponding unannotated slide (right) in QuPath}
    \label{fig:fig7}
\end{figure}

To identify the regions in a WSI critical for prognosticating tumor development, such as specific tumor areas or border regions without overlapping healthy tissue, the pathologist uses QuPath to outline these regions. Then, the pathologist initiates a cell segmentation and classification script through the QuPath interface for the selected regions. The resulting annotations and quantified cell information are then directly overlaid onto the WSI in the QuPath interface. Additional design and implementation details are in \hyperref[chap:S7]{Appendix S7}. 

Two common approaches for integrating deep learning models into QuPath are Java‑based native QuPath extensions \cite{Goldsborough_Philps_etal._2024} and the execution of RESTful API requests to a model server coupled with handling the response via an extension, as demonstrated in the application of cell segmentation models applied to immunofluorescence images \cite{Sugawara_2023}. While the community is actively working on these integration strategies, there is currently no universal solution that fully addresses all integration and performance requirements.

Extensions may offer better integration with QuPath, allowing slightly improved performance and more widespread usage of the built-in QuPath models, but they lack the flexibility to customize models and modify their behavior. For example, the newest version of QuPath includes models such as StarDist \cite{Weigert_Schmidt} and InstanSeg \cite{Goldsborough_Philps_etal._2024} that can perform cell segmentation. Both models pose limitations when applied to simultaneous cell segmentation and classification. StarDist performs well only on convex, round shapes by design, whereas some neoplastic, inflammatory, and connective cells exhibit complex and non-convex shapes. InstanSeg provides only semantic segmentation without assigning classes to the segmented cells.

%\hfill

In contrast, our approach offers an alternative integration strategy. It utilizes the paquo library to directly interact with QuPath’s internal application programming interface from within Python. This enables data exchange and processing without the need for intermediate conversion steps and provides greater control over model customization, retraining, and the incorporation of custom processing steps.

The integration of our custom model with QuPath underscores its potential to significantly enhance the diagnostic process by reducing the time burden on pathologists and enabling them to focus on more complex interpretative tasks using familiar software. Leveraging a tool that is already well-established among pathologists increases the likelihood of its adoption into daily clinical workflows. The quantitative data generated through the automated workflow is critical for both clinical decision-making and research, facilitating more accurate biomarker analysis, enabling robust statistical evaluations, and supporting hypothesis generation and testing. Additionally, by streamlining cell segmentation and classification, the tool enhances the scalability and reproducibility of pathological assessments, ultimately contributing to improved diagnostic accuracy and patient outcomes.

\section{Conclusion and future work}

In this study, we address critical challenges in digital pathology and tackle the usability and deployment issues of the developed models in standard computing environments without the need for high-performance computing systems. Our multi-faceted approach encompasses data refinement through cross-relabeling, leveraging foundation models for robust cell segmentation and classification, optimizing model performance via knowledge distillation, and integrating the optimized model into the QuPath software for practical application. This approach is used to construct a capable, versatile, and adjustable model for cell segmentation and classification, with enhanced performance and usability.

\begin{sloppypar}
While our approach shows potential in the field of computational pathology, certain limitations persist. 
For example, our implementation currently exhibits lower performance in detecting macrophages. 
This serves as an instance of the broader challenge of accurately identifying complex cell types. In order to address this issue, extending our approach to incorporate additional data sources, exploring alternative modeling approaches, and integrating other imaging modalities such as immunohistochemical staining may help improve detection accuracy. Moreover, although the distilled model reduces computational demands, integrating advanced deep learning models into clinical practice requires addressing technological gaps and potential resistance to adopting new tools within established diagnostic processes.
\end{sloppypar}

Future work could focus on several key areas to refine the proposed approach and facilitate its adoption in clinical environments. Enhancing the cell-relabeling process with additional datasets \cite{Graham_Jahanifar_etal._2021} could improve the representation of underrepresented cell types and enhance overall model performance. Also, incorporating additional data sources, such as multi-modal imaging or complementary staining methods, may address limitations related to cell type differentiation and class imbalance. Exploring other foundation models \cite{Vorontsov_Bozkurt_etal._2024,Zimmermann_Vorontsov_etal._2024} or introducing additional modalities \cite{Ding_Wagner_etal._2024,Vaidya_Zhang_etal._2025} may provide alternative architectures better suited to specific tasks or offer improved efficiency. Implementing more complex knowledge distillation techniques \cite{Houyon_Cioppa_etal._2023,Zhang_Song_etal._2019} could further optimize the model's performance and adaptability. Additionally, deeper integration with QuPath or other digital pathology software could provide pathologists more control over cell quantification analysis directly within the QuPath interface, thereby increasing accessibility and usability. Such enhancements would not only refine model performance but also ensure greater adaptability and scalability within various clinical environments. Finally, extensive validation of the model by pathologists and benchmarking against independent datasets are essential steps toward establishing the model's reliability and fostering confidence in its clinical utility.

\section*{Acknowledgments} 
This work was funded in part by the Research Council of Norway grant no. 309439 SFI Visual Intelligence, and the North Norwegian Health Authority grant no. HNF1521-20.

\bibliographystyle{IEEEtran}
\begin{sloppypar}
\begin{thebibliography}{99}

\bibitem{chaplot2020neural} Chaplot, Devendra Singh, et al. "Neural topological slam for visual navigation." Proceedings of the IEEE/CVF conference on computer vision and pattern recognition. 2020.

\bibitem{maksymets2021thda} Maksymets, Oleksandr, et al. "Thda: Treasure hunt data augmentation for semantic navigation." Proceedings of the IEEE/CVF International Conference on Computer Vision. 2021.

\bibitem{mezghan2022memory} Mezghan, Lina, et al. "Memory-augmented reinforcement learning for image-goal navigation." 2022 IEEE/RSJ International Conference on Intelligent Robots and Systems (IROS). IEEE, 2022.

\bibitem{al2022zero} Al-Halah, Ziad, Santhosh Kumar Ramakrishnan, and Kristen Grauman. "Zero experience required: Plug \& play modular transfer learning for semantic visual navigation." Proceedings of the IEEE/CVF Conference on Computer Vision and Pattern Recognition. 2022.

\bibitem{ye2021auxiliary} Ye, Joel, et al. "Auxiliary tasks and exploration enable objectgoal navigation." Proceedings of the IEEE/CVF international conference on computer vision. 2021.

\bibitem{chaplot2020object} Chaplot, Devendra Singh, et al. "Object goal navigation using goal-oriented semantic exploration." Advances in Neural Information Processing Systems 33 (2020)

\bibitem{ramakrishnan2022poni} Ramakrishnan, Santhosh Kumar, et al. "Poni: Potential functions for objectgoal navigation with interaction-free learning." Proceedings of the IEEE/CVF Conference on Computer Vision and Pattern Recognition. 2022.

\bibitem{ramrakhya2022habitat} Ramrakhya, Ram, et al. "Habitat-web: Learning embodied object-search strategies from human demonstrations at scale." Proceedings of the IEEE/CVF Conference on Computer Vision and Pattern Recognition. 2022.

\bibitem{mousavian2019visual} Mousavian, Arsalan, et al. "Visual representations for semantic target driven navigation." 2019 International Conference on Robotics and Automation (ICRA). IEEE, 2019.

\bibitem{dhariwal2021diffusion} Dhariwal, Prafulla, and Alexander Nichol. "Diffusion models beat gans on image synthesis." Advances in neural information processing systems 34 (2021)

\bibitem{ho2022classifier} Ho, Jonathan, and Tim Salimans. "Classifier-free diffusion guidance." arXiv preprint arXiv:2207.12598 (2022).

\bibitem{nichol2021glide} Nichol, Alex, et al. "Glide: Towards photorealistic image generation and editing with text-guided diffusion models." arXiv preprint arXiv:2112.10741 (2021)

\bibitem{brooks2023instructpix2pix} Brooks, Tim, Aleksander Holynski, and Alexei A. Efros. "Instructpix2pix: Learning to follow image editing instructions." Proceedings of the IEEE/CVF Conference on Computer Vision and Pattern Recognition. 2023.

\bibitem{fu2023guiding} Fu, Tsu-Jui, et al. "Guiding instruction-based image editing via multimodal large language models." arXiv preprint arXiv:2309.17102 (2023).

\bibitem{geng2024instructdiffusion} Geng, Zigang, et al. "Instructdiffusion: A generalist modeling interface for vision tasks." Proceedings of the IEEE/CVF Conference on Computer Vision and Pattern Recognition. 2024.

\bibitem{zhou2024minedreamer} Zhou, Enshen, et al. "Minedreamer: Learning to follow instructions via chain-of-imagination for simulated-world control." arXiv preprint arXiv:2403.12037 (2024).

\bibitem{zhou2023esc} Zhou, Kaiwen, et al. "Esc: Exploration with soft commonsense constraints for zero-shot object navigation." International Conference on Machine Learning. PMLR, 2023.

\bibitem{yu2023l3mvn} Yu, Bangguo, Hamidreza Kasaei, and Ming Cao. "L3mvn: Leveraging large language models for visual target navigation." 2023 IEEE/RSJ International Conference on Intelligent Robots and Systems (IROS). IEEE, 2023.

\bibitem{gadre2023cows} Gadre, Samir Yitzhak, et al. "Cows on pasture: Baselines and benchmarks for language-driven zero-shot object navigation." Proceedings of the IEEE/CVF Conference on Computer Vision and Pattern Recognition. 2023.

\bibitem{shah2023navigation} Shah, Dhruv, et al. "Navigation with large language models: Semantic guesswork as a heuristic for planning." Conference on Robot Learning. PMLR, 2023.

\bibitem{cai2024bridging} Cai, Wenzhe, et al. "Bridging zero-shot object navigation and foundation models through pixel-guided navigation skill." 2024 IEEE International Conference on Robotics and Automation (ICRA). IEEE, 2024.

\bibitem{yu2023co} Yu, Bangguo, Hamidreza Kasaei, and Ming Cao. "Co-NavGPT: Multi-robot cooperative visual semantic navigation using large language models." arXiv preprint arXiv:2310.07937 (2023).

\bibitem{wu2024voronav} Wu, Pengying, et al. "Voronav: Voronoi-based zero-shot object navigation with large language model." arXiv preprint arXiv:2401.02695 (2024).

\bibitem{qin2023mp5} Qin, Yiran, et al. "Mp5: A multi-modal open-ended embodied system in minecraft via active perception." arXiv preprint arXiv:2312.07472 (2023).

\bibitem{du2024learning} Du, Yilun, et al. "Learning universal policies via text-guided video generation." Advances in Neural Information Processing Systems 36 (2024).

\bibitem{ajay2024compositional} Ajay, Anurag, et al. "Compositional foundation models for hierarchical planning." Advances in Neural Information Processing Systems 36 (2024).

\bibitem{liang2024skilldiffuser} Liang, Zhixuan, et al. "Skilldiffuser: Interpretable hierarchical planning via skill abstractions in diffusion-based task execution." Proceedings of the IEEE/CVF Conference on Computer Vision and Pattern Recognition. 2024.

\bibitem{heusel2017gans} Heusel, Martin, et al. "Gans trained by a two time-scale update rule converge to a local nash equilibrium." Advances in neural information processing systems 30 (2017).

\bibitem{zhang2018unreasonable} Zhang, Richard, et al. "The unreasonable effectiveness of deep features as a perceptual metric." Proceedings of the IEEE conference on computer vision and pattern recognition. 2018.

\bibitem{brown2020language} Brown, Tom B. "Language models are few-shot learners." arXiv preprint arXiv:2005.14165 (2020).

\bibitem{podell2023sdxl} Podell, Dustin, et al. "Sdxl: Improving latent diffusion models for high-resolution image synthesis." arXiv preprint arXiv:2307.01952 (2023).

\bibitem{brohan2022rt} Brohan, Anthony, et al. "Rt-1: Robotics transformer for real-world control at scale." arXiv preprint arXiv:2212.06817 (2022).

\bibitem{brohan2023rt} Brohan, Anthony, et al. "Rt-2: Vision-language-action models transfer web knowledge to robotic control." arXiv preprint arXiv:2307.15818 (2023).

\bibitem{li2024manipllm} Li, Xiaoqi, et al. "Manipllm: Embodied multimodal large language model for object-centric robotic manipulation." Proceedings of the IEEE/CVF Conference on Computer Vision and Pattern Recognition. 2024.

\bibitem{shah2023vint} Shah, Dhruv, et al. "ViNT: A foundation model for visual navigation." arXiv preprint arXiv:2306.14846 (2023).

\bibitem{liu2024visual} Liu, Haotian, et al. "Visual instruction tuning." Advances in neural information processing systems 36 (2024).

\bibitem{hu2021lora} Hu, Edward J., et al. "Lora: Low-rank adaptation of large language models." arXiv preprint arXiv:2106.09685 (2021).

\bibitem{qin2023supfusion} Qin, Yiran, et al. "SupFusion: Supervised LiDAR-camera fusion for 3D object detection." Proceedings of the IEEE/CVF International Conference on Computer Vision. 2023.

\bibitem{qin2024worldsimbench} Qin, Yiran, et al. "Worldsimbench: Towards video generation models as world simulators." arXiv preprint arXiv:2410.18072 (2024).

\bibitem{yu2025gamefactory} Yu, Jiwen, et al. "GameFactory: Creating New Games with Generative Interactive Videos." arXiv preprint arXiv:2501.08325 (2025).

\bibitem{zhou2024code} Zhou, Enshen, et al. "Code-as-Monitor: Constraint-aware Visual Programming for Reactive and Proactive Robotic Failure Detection." arXiv preprint arXiv:2412.04455 (2024).

\bibitem{zhang2024ad} Zhang, Zaibin, et al. "AD-H: Autonomous Driving with Hierarchical Agents." arXiv preprint arXiv:2406.03474 (2024).

\bibitem{wang2024toward} Wang, Chaoqun, et al. "Toward Accurate Camera-based 3D Object Detection via Cascade Depth Estimation and Calibration." arXiv preprint arXiv:2402.04883 (2024).

\bibitem{huang2024story3d} Huang, Yuzhou, et al. "Story3d-agent: Exploring 3d storytelling visualization with large language models." arXiv preprint arXiv:2408.11801 (2024).

\bibitem{savinov2018semi} Savinov, Nikolay, Alexey Dosovitskiy, and Vladlen Koltun. "Semi-parametric topological memory for navigation." arXiv preprint arXiv:1803.00653 (2018).

\bibitem{majumdar2022zson} Majumdar, Arjun, et al. "Zson: Zero-shot object-goal navigation using multimodal goal embeddings." Advances in Neural Information Processing Systems 35 (2022): 32340-32352.

\bibitem{yadav2023offline} Yadav, Karmesh, et al. "Offline visual representation learning for embodied navigation." Workshop on Reincarnating Reinforcement Learning at ICLR 2023. 2023.

\bibitem{yadav2023ovrl} Yadav, Karmesh, et al. "Ovrl-v2: A simple state-of-art baseline for imagenav and objectnav." arXiv preprint arXiv:2303.07798 (2023).

\bibitem{sun2024fgprompt} Sun, Xinyu, et al. "FGPrompt: fine-grained goal prompting for image-goal navigation." Advances in Neural Information Processing Systems 36 (2024).

\bibitem{zhu2017target} Zhu, Yuke, et al. "Target-driven visual navigation in indoor scenes using deep reinforcement learning." 2017 IEEE international conference on robotics and automation (ICRA). IEEE, 2017.

\bibitem{koh2024generating} Koh, Jing Yu, Daniel Fried, and Russ R. Salakhutdinov. "Generating images with multimodal language models." Advances in Neural Information Processing Systems 36 (2024).

\bibitem{krantz2022instance} Krantz, Jacob, et al. "Instance-specific image goal navigation: Training embodied agents to find object instances." arXiv preprint arXiv:2211.15876 (2022).

\bibitem{schulman2017proximal} Schulman, John, et al. "Proximal policy optimization algorithms." arXiv preprint arXiv:1707.06347 (2017).

\bibitem{anderson2018evaluation} Anderson, Peter, et al. "On evaluation of embodied navigation agents." arXiv preprint arXiv:1807.06757 (2018).

\bibitem{lin2024navcot} Lin, Bingqian, et al. "NavCoT: Boosting LLM-Based Vision-and-Language Navigation via Learning Disentangled Reasoning." arXiv preprint arXiv:2403.07376 (2024).

\bibitem{NavGPT} Zhou, Gengze, Yicong Hong, and Qi Wu. "Navgpt: Explicit reasoning in vision-and-language navigation with large language models." Proceedings of the AAAI Conference on Artificial Intelligence.

\bibitem{hahn2021no} Hahn, Meera, et al. "No rl, no simulation: Learning to navigate without navigating." Advances in Neural Information Processing Systems 34 (2021): 26661-26673.

\bibitem{li2025t2isafety} Li, Lijun, et al. "T2ISafety: Benchmark for Assessing Fairness, Toxicity, and Privacy in Image Generation." arXiv preprint arXiv:2501.12612 (2025).

\bibitem{an2024agfsync} An, Jingkun, et al. "AGFSync: Leveraging AI-Generated Feedback for Preference Optimization in Text-to-Image Generation." arXiv preprint arXiv:2403.13352 (2024).


\end{thebibliography}
\end{sloppypar}

\clearpage
\beginsupplement
\section*{Appendix}
\renewcommand{\thesubsection}{S\arabic{subsection}}

\subsection{\label{chap:S1}PanNuke and MoNuSAC preprocessing}
The PanNuke dataset comprises a set of 7,901 RGB patches, each with dimensions of $256 \times 256$ pixels, which we set as the standard patch size for our analysis. In contrast, the MoNuSAC dataset encompasses 294 images of heterogeneous dimensions. To standardize the MoNuSAC images with our experiments, we implement a standardization protocol. Specifically, for images exceeding the dimensions of $256 \times 256$ pixels, we segment them into equal-sized patches and apply mirror padding to the remaining portions to avoid information loss at the peripherals. Patches with dimensions less than $128 \times 128$ pixels are excluded from the dataset due to the insufficient resolution to capture relevant cellular details. For patches where either dimension falls between 128 and 256 pixels, we employ upsampling to achieve the standard patch size. As a result, we obtain a total of 2,823 RGB patches derived from the MoNuSAC dataset for subsequent analysis. For additional details on the MoNuSAC data preparation process, refer to the source code \cite{Shvetsov_2025a}.
\clearpage

\subsection{\label{chap:S2}Data usage for the methodology}

\counterwithin{figure}{subsection}
\renewcommand{\thefigure}{S\arabic{subsection}}

\begin{figure}[h!]
    \centering
    \includegraphics[width=\textwidth, height=0.85\textheight, keepaspectratio]{images/A2.pdf}
    \caption{Overview of the methodology for cross-labeling, dataset refinement, and model comparison. (1) Cross-relabeling - training and testing cell classification models, (2) Cross-relabeling - using cell classification models to create refined dataset, (3) Fine-tuning and training models for comparison, (4) Student knowledge distillation with refined dataset}
    \label{fig:S2}
\end{figure}
\clearpage

\subsection{\label{chap:S3}Confusion matrices for classification models}
\counterwithin{figure}{subsection}
\renewcommand{\thefigure}{S\arabic{subsection}.\arabic{figure}}

\begin{figure}[h!]
    \centering
    \includegraphics[width=\textwidth, height=0.4\textheight, keepaspectratio]{images/A3_1.pdf}
    \caption{Confusion matrix for PanNuke trained model}
    \label{fig:S3.1}
\end{figure}

\begin{figure}[h!]
    \centering
    \includegraphics[width=\textwidth, height=0.4\textheight, keepaspectratio]{images/A3_2.pdf}
    \caption{Confusion matrix for MoNuSAC trained model}
    \label{fig:S3.2}
\end{figure}

\clearpage

\subsection{\label{chap:S4}Datasets cell counts}

\counterwithin{table}{subsection}
\renewcommand{\thetable}{S\arabic{subsection}}

\begin{table}[h!]
\renewcommand{\arraystretch}{2.0}
\centering
\caption{\label{tab:S4}Cell counts for PanNuke, MoNuSAC and refined datasets. Numbers in parentheses indicate preprocessed cell counts for cell classifier models training and testing.}
%\adjustbox{max width=\textwidth}{%
\begin{tabular}{|l|c|c|c|}
\hline
%\rowcolor{gray!30}
Cell type & PanNuke & MoNuSAC & Refined \\
\hline
Neoplastic & 77,403 (68,031) & - & 105,451 \\
\hline
Epithelial & 26,572 (23,207) & - & 29,926 \\
\hline
Epithelial (benign and malignant) & - & 31,402 & - \\
\hline
Inflammatory & 32,276 & - & - \\
\hline
Lymphocytes & - & 37,045 (33,104) & 65,275 \\
\hline
Neutrophils & - & 1,355 (1,252) & 3,833 \\
\hline
Macrophage & - & 1,842 (1,695) & 3,410 \\
\hline
Dead & 2,908 & - & 2,908 \\
\hline
Connective & 50,585 & - & 50,585 \\
\hline
\end{tabular}
%
%}
\end{table}



\clearpage

\subsection{\label{chap:S5}Definition of validation metrics}
\counterwithin{equation}{subsection}
\renewcommand{\theequation}{\arabic{equation}}

\subsubsection{\label{chap:S5.1}R\textsuperscript{2}}
The coefficient of determination, denoted as $R^2$, is a statistical measure that represents the proportion of variance in the dependent variable that is predictable from the independent variables. In the context of cell quantification in pathology, $R^2$ is used to assess how well the predicted quantities of different cell types in a patch align with the actual quantities observed in the ground truth data, with higher values representing more accurate quantification. $R^2$ is defined as
\begin{equation*}
R^2 = 1 - \frac{\sum_{i=1}^n (y_i - \hat{y}_i)^2}{\sum_{i=1}^n (y_i - \bar{y})^2},
\end{equation*}
where $y_i$ represents the actual number of cells of a specific type in the $i$-th image, $\hat{y}_i$ represents the predicted number of cells of that type in the $i$-th image, $\bar{y}$ is the mean of the actual numbers across all images, and $n$ is the total number of images in the dataset.

The $R^2$ metric has a range of $(-\infty, 1]$. An $R^2$ of 1 indicates perfect prediction, where all predicted values exactly match the actual values. An $R^2$ of 0 suggests that the model explains none of the variability of the response data around its mean. If $R^2$ is negative, it indicates that the model performs worse than a model that simply predicts the mean of the actual values for all observations.

\subsubsection{\label{chap:S5.2}PQ}
Panoptic Quality ($PQ$) is a comprehensive metric used to evaluate the performance of segmentation models in tasks that require both instance segmentation and classification. $PQ$ provides a single score that encapsulates both the detection accuracy (i.e., how many objects were correctly identified) and the segmentation quality (i.e., how accurately the objects' boundaries were delineated). This metric is particularly useful in multiclass scenarios where each pixel is classified into distinct categories, such as different cell types in pathology images.

$PQ$ is calculated as the product of two terms: Detection Quality ($DQ$) and Segmentation Quality ($SQ$). It can be expressed as
\begin{equation*}
PQ = DQ \cdot SQ,
\end{equation*}
where
\begin{equation*}
DQ = \frac{TP}{TP + 0.5\, FP + 0.5\, FN},
\end{equation*}
\begin{equation*}
SQ = \frac{\sum_{(p, g) \in \mathcal{M}} IoU(p, g)}{TP}.
\end{equation*}
In these formulas, $TP$ denotes the number of correctly matched instances between ground truth and prediction, $FP$ denotes the predicted instances that have no corresponding ground truth, $FN$ denotes the ground truth instances that were not detected, $IoU(p, g)$ is the Intersection over Union for a pair of matched instances $p$ (prediction) and $g$ (ground truth), and $\mathcal{M}$ is the set of matched pairs.

The $PQ$ metric is calculated for each class and is averaged across classes to provide a global performance measure.

The $PQ$ score has a range of $[0, 1.0]$, where a higher score indicates better performance in both detecting and segmenting the instances correctly. A $PQ$ of 1 signifies perfect identification and segmentation of all instances, whereas a $PQ$ of 0 indicates that no instances were correctly identified and segmented.

\clearpage

\subsection{\label{chap:S6}Segmentation and Detection quality metrics for teacher and student models}

\begin{table}[h!]
\renewcommand{\arraystretch}{2.0}
\centering
\caption{Segmentation and detection quality for student and teacher models (CI 95\%)}
\label{tab:S6}
%\adjustbox{max width=\textwidth}{%
\begin{tabular}{|l|c|c|}
\hline
%\rowcolor{gray!30}
Metric & Teacher & Student \\
\hline
$SQ_{neoplastic}$ & 0.819 (0.815--0.823) & 0.824 (0.819--0.828) \\
\hline
$SQ_{lymphocyte}$ & 0.795 (0.788--0.802) & 0.790 (0.783--0.796) \\
\hline
$SQ_{connective}$ & 0.770 (0.762--0.776) & 0.780 (0.772--0.786) \\
\hline
$SQ_{dead}$ & 0.659 (0.623--0.688) & 0.657 (0.624--0.695) \\
\hline
$SQ_{epithelial}$ & 0.780 (0.770--0.790) & 0.788 (0.779--0.797) \\
\hline
$SQ_{macrophage}$ & 0.788 (0.760--0.810) & 0.757 (0.730--0.783) \\
\hline
$SQ_{neutrofil}$ & 0.782 (0.761--0.801) & 0.775 (0.759--0.792) \\
\hline
$DQ_{neoplastic}$ & 0.706 (0.692--0.719) & 0.727 (0.712--0.741) \\
\hline
$DQ_{lymphocyte}$ & 0.675 (0.656--0.698) & 0.713 (0.691--0.734) \\
\hline
$DQ_{connective}$ & 0.566 (0.546--0.584) & 0.583 (0.565--0.602) \\
\hline
$DQ_{dead}$ & 0.410 (0.361--0.465) & 0.435 (0.306--0.561) \\
\hline
$DQ_{epithelial}$ & 0.668 (0.639--0.694) & 0.673 (0.644--0.702) \\
\hline
$DQ_{macrophage}$ & 0.657 (0.583--0.727) & 0.615 (0.531--0.703) \\
\hline
$DQ_{neutrofil}$ & 0.691 (0.625--0.753) & 0.729 (0.679--0.778) \\
\hline
\end{tabular}
%
%}
\end{table}

\clearpage

\subsection{\label{chap:S7}QuPath integration method}
We adopt an integration strategy leveraging the paquo \cite{Bayer_AG} library, a Python package that enables direct interaction with QuPath’s internal API, thereby facilitating seamless data exchange without intermediate conversion steps. The data processing pipeline (\hyperref[fig:S7]{Appendix Figure S7}) begins with the acquisition of WSIs and their associated annotations from QuPath, which are represented as Shapely \cite{Gillies_Wel_etal._2024} polygons. Utilizing paquo, we directly read, create, and modify these annotations and detections within a QuPath project in the Python environment. Images are then cropped using these polygons and processed by cell segmentation and classification models employing standard vision processing toolkits such as OpenCV, pyvips, and PyTorch. Additionally, QuPath employs Groovy scripts to initiate a Python process that starts the entire pipeline from QuPath graphical interface: fetching polygons, extracting images from them, and running deep learning model inference on the cropped images. 
The results are returned to QuPath, leveraging paquo's Python bindings to manipulate QuPath data while minimizing the computational overhead typically associated with cross-environment communication.

\counterwithin{figure}{subsection}
\renewcommand{\thefigure}{S\arabic{subsection}}

\begin{figure}[h!]
    \centering
    \includegraphics[width=\textwidth]{images/A7.pdf}
    \caption{QuPath integration workflow using Python environment}
    \label{fig:S7}
\end{figure}

Compared to traditional workflows that involve exporting annotations as GeoJSON, classifying them in Python, and reimporting them into QuPath, our approach offers several advantages. We eliminate the need to switch between programming languages, providing a cohesive and streamlined development process entirely within QuPath software and removing the necessity to use other tools. Meanwhile, we avoid storing annotations as intermediate JSON files unless required for external use or archiving. By conducting the entire inference and post-processing workflow within the Python environment, we leverage the power and flexibility of Python libraries for image processing and machine learning. This approach also enables adjustments to any set of labels and models, thereby improving its applicability.

%\hfill

The distilled model and QuPath integration code are packaged into a Docker container, enabling streamlined execution with the Docker engine. Detailed integration code and deployment instructions can be found in the GitHub repository \cite{Shvetsov_2025b}.

Despite these benefits, we acknowledge that the paquo library is a proof‑of‑concept project in its early development stage and has not been tested across all versions of QuPath.

\clearpage

\subsection{\label{chap:S8}Data and code availability statement}
All datasets, models, and code used in this study are publicly available and can be obtained from the repositories listed below. 
The PanNuke \cite{Gamper_Koohbanani_etal._2019} and MoNuSAC \cite{Verma_Kumar_etal._2021} datasets are publicly accessible, and download information along with detailed descriptions can be found in their respective articles. Preprocessing scripts for PanNuke and MoNuSAC data, as well as individual cell extraction scripts, are available on GitHub \cite{Shvetsov_2025a}. The H-Optimus foundation model used in our experiments can be downloaded from the HuggingFace repository \cite{hoptimus2024}, and model information is available on GitHub \cite{Saillard_Jenatton_etal._2024}. In addition, the integration code for QuPath and the distilled model packaged in a Docker container are provided in the repository \cite{Shvetsov_2025b}, and paquo Python library is available from the authors GitHub repository \cite{Bayer_AG}.
\clearpage

\end{document}

\bibliographystyle{icml2025}


%%%%%%%%%%%%%%%%%%%%%%%%%%%%%%%%%%%%%%%%%%%%%%%%%%%%%%%%%%%%%%%%%%%%%%%%%%%%%%%
%%%%%%%%%%%%%%%%%%%%%%%%%%%%%%%%%%%%%%%%%%%%%%%%%%%%%%%%%%%%%%%%%%%%%%%%%%%%%%%
% APPENDIX
%%%%%%%%%%%%%%%%%%%%%%%%%%%%%%%%%%%%%%%%%%%%%%%%%%%%%%%%%%%%%%%%%%%%%%%%%%%%%%%
%%%%%%%%%%%%%%%%%%%%%%%%%%%%%%%%%%%%%%%%%%%%%%%%%%%%%%%%%%%%%%%%%%%%%%%%%%%%%%%
\newpage
\appendix
\onecolumn
\textbf{\LARGE Appendix}
\section{Notations}
\label{app:notations}

\newcommand{\p}{\textsc{p}}

\newcommand{\data}{\mathcal{D}}
\newcommand{\extdata}{\mathcal{D}_e}
\newcommand{\shots}{S}
\newcommand{\classes}{C}
\newcommand{\numbershadowmodels}{M}

\newcommand{\hpofunction}{$\texttt{HPO}$}
\newcommand{\trainfunction}{$\textsc{train}$}

\newcommand{\hyperparamsdata}{\eta_{{}_{\data}}}
\newcommand{\hyperparamsextdata}{\eta_{{}_{\extdata}}}

\newcommand{\model}{\mathcal{M}} 
\newcommand{\modeltar}{\mathcal{M}_{\mathcal{T}}} 
\newcommand{\modelshadow}{\mathcal{M}_{\mathcal{S}}} 
\newcommand{\return}{\textbf{return }}
\newcommand{\grid}{\mathbb{M}}

\newcommand{\tpr}{\textsc{tpr}}
\newcommand{\fpr}{\textsc{fpr}}


\newcommand{\lira}{\mathrm{LiRA}} 

\newcommand{\datashadow}{\mathcal{D}_{\mathrm{shadow}}}
\newcommand{\architecture}{\mathcal{A}}
\newcommand{\prob}{\mathbb{P}}
\newcommand{\normal}{\mathcal{N}}
\newcommand{\attack}{\texttt{KNOWN}}
\newcommand{\bb}{\texttt{BLACK-BOX}}
\newcommand{\logits}{\textsc{logits}}


\section{Unified Paradigm of Policy Gradient Methods}\label{app:unified_paradigm}

As in \cite{shao2024deepseekmath}, we provide the expression of the coefficient $w_{x,y}$ for \SEPO~in its PPO and GRPO variants. This can be easily extended to the other training methods synthesized in \citealt[Section~5.2]{shao2024deepseekmath}.

\subsection{Proximal Policy Optimization (PPO)}

For PPO \cite{schulman2017proximal}, the coefficient $w_{x,y}$ takes the form $$w_{x,y}(\epsilon) = \min\{\mathrm{clip}(r_{x,y}, 1-\epsilon, 1+\epsilon)A(x);r_{x,y} A(x)\},\quad \epsilon >0.$$

A common approach is to learn a value network to approximate the reward, and then compute the advantage as $A(x) = R(x) - V(x)$.

\subsection{Group Relative Policy Optimization (GRPO)}

For GRPO \cite{shao2024deepseekmath}, we consider a group of outputs $x = \{x_1,\hdots,x_G\}$. The coefficient $w_{x,y}$ takes the form $$w_{x,y}(\epsilon) = \frac{1}{G}\sum_{i=1}^{G} \min\{\mathrm{clip}(r_{x_i,y}, 1-\epsilon, 1+\epsilon)A(x_i);r_{x_i,y} A(x_i)\},\quad \epsilon >0.$$

The advantages are the standardized reward over each group. Specifically, the advantages are defined as 
\[A(x_i) = \frac{R(x_i) - \text{mean}(R(x))}{\text{std}(R(x))}, \quad i\in\{1,\hdots,G\}.\]

The $\KL$ term of the original GRPO objective \cite{shao2024deepseekmath} is discussed in \cref{sec:sepo_main}.

\section{Proofs of the results}

This section details the proof of the results that appear in the main text.

\subsection{Proof of \cref{prop:stochastic_estimate}}

We begin by calculating $\nabla_\theta \target(\theta)$ appearing in \cref{eq:loss_pi_theta} component-wise. Let $x \in\statespace$ and recall that we can express $\target_x(\theta)$ as $\target_x(\theta) = \expo{-V_x(\theta)}/Z_\theta$ for some normalization constant $Z_\theta=\sum_{y\in\statespace}\expo{-V_y(\theta)}$. Therefore,
\begin{align*}
    \nabla_\theta \target_x(\theta) &= - \target_x(\theta)\nabla_\theta V_x(\theta) - \frac{\expo{-V_x(\theta)}\nabla_\theta Z_\theta}{Z_\theta^2}
    =- \target_x(\theta)\nabla_\theta V_x(\theta) - \target_x(\theta)\frac{\nabla_\theta Z_\theta}{Z_\theta}\\
    &=- \target_x(\theta)\nabla_\theta V_x(\theta) - \target_x(\theta)\left(\sum_{y\in\statespace} \frac{-\nabla_\theta V_y(\theta)\expo{-V_y(\theta)}}{Z_\theta}\right)=- \target_x(\theta)\nabla_\theta V_x(\theta) + \target_x(\theta)\left(\sum_{y\in\statespace} {\target_y(\theta)\nabla_\theta V_y(\theta)}\right).
\end{align*}
On the other hand, since for $y\neq x\in \statespace$, $[s_\theta(x,0)]_y = \target_y(\theta)/\target_x(\theta)$, we have $[s_\theta(x,0)]_y = \expo{-\left[V_y(\theta) - V_x(\theta)\right]}$. Then, $\nabla_\theta\log([s_\theta(x,0)]_y) = - (\nabla_\theta V_y(\theta) - \nabla_\theta V_x(\theta))$, leading to 
\begin{equation} \label{eq_relation_s_theta}
\nabla_\theta V_y(\theta) = \nabla_\theta V_x(\theta) -\nabla_\theta\log([s_\theta(x,0)]_y).   
\end{equation}
We can weight \cref{eq_relation_s_theta} by $\target_y(\theta)$ and sum over $y \neq x\in\statespace$ to obtain
\[\sum_{\substack{y \in \statespace \\ y \neq x}} {\target_y(\theta)\nabla_\theta V_y(\theta)} = (1-\target_x(\theta))\nabla_\theta V_x(\theta) - \sum_{\substack{y \in \statespace \\ y \neq x}} {\target_y(\theta)\nabla_\theta\log([s_\theta(x,0)]_y)}.\]
Moving the term $\target_x(\theta)\nabla_\theta V_x(\theta)$ to the left-hand side leads to
\[\sum_{y\in\statespace} {\target_y(\theta)\nabla_\theta V_y(\theta)} = \nabla_\theta V_x(\theta) - \sum_{\substack{y \in \statespace \\ y \neq x}} {\target_y(\theta)\nabla_\theta\log([s_\theta(x,0)]_y)}.\]
Combining this with the expression for $\nabla_\theta \target_x(\theta)$ derived at the beginning of this section yields
\[\nabla_\theta \target_x(\theta) = -\target_x(\theta)\sum_{\substack{y \in \statespace \\ y \neq x}} {\target_y(\theta)\nabla_\theta\log([s_\theta(x,0)]_y)}.\]
Finally, after replacing this term in \cref{eq:loss_pi_theta} we obtain the desired expression as
\[\nabla \ell^R(\theta) = -\sum_{x\in\statespace}R(x)\nabla_\theta \target_x(\theta) = \sum_{x\in\statespace}\target_x(\theta)R(x)\sum_{\substack{y \in \statespace \\ y \neq x}} {\target_y(\theta)\nabla_\theta\log([s_\theta(x,0)]_y)}.\]

\subsection{Proof of \cref{thm:ppo_expression}}

Following \cref{prop:stochastic_estimate}, we can artificially introduce $\target_x(\old)$ in the expression of $\nabla \ell_R(\theta)$ as 
\begin{equation} \label{eq_l_proof_32}
    \nabla \ell_R(\theta) = \sum_{x\in\statespace}\target_x(\old)R(x)\sum_{\substack{y \in \statespace \\ y \neq x}} {\frac{\target_x(\theta)}{\target_x(\old)}\target_y(\theta)\nabla_\theta\log([s_\theta(x,0)]_y)}.
\end{equation}
Since for all $x\neq y \in \statespace$,
\[s_{\old}(x,0)_y = \frac{\target_y(\old)}{\target_x(\old)},\] 
we can rewrite \cref{eq_l_proof_32} as
\[\nabla \ell_R(\theta) = \sum_{x\in\statespace}\target_x(\old)R(x)\sum_{\substack{y \in \statespace \\ y \neq x}} {\target_y(\theta)\frac{\target_y(\theta)}{\target_y(\old)}\frac{s_{\old}(x,0)_y}{s_\theta(x,0)_y}\nabla_\theta \log s_\theta(x,0)_y}.\]
Finally, we note that this is equivalent to the following equation
$$\nabla \ell_R(\theta) =\esp_{x\sim\target(\old)}\left[R(x)\sum_{\substack{y \in \statespace \\ y \neq x}} {\target_y(\theta)\frac{\target_y(\theta)}{\target_y(\old)}\frac{s_{\old}(x,0)_y}{s_\theta(x,0)_y}\nabla_\theta \log s_\theta(x,0)_y}\right].$$

\subsection{Proof of \cref{prop:grad_pi}}\label{app:proof_linear_system}

To prove \cref{prop:grad_pi}, we are going to prove the following explicit proposition regarding the linear system satisfied by $\nabla^\eta_\theta \target(\theta)$.

\begin{boxprop}
    Let $\eta >0$, and $D_r^\star$ denotes the block diagonal matrix defined in \cref{lem:sorted}. Then, $\nabla^\eta_\theta \target(\theta)$ is the solution of the linear system $$A_\eta \unknown = B_\eta \in \R^{\cardinality\times p},$$ where $A_\eta \coloneqq D_r^\star\left[I_\cardinality - \eta\diag(1/\target(\theta))\right] - I_\cardinality \in \R^{\cardinality\times \cardinality}$ and $B_\eta \coloneqq -\eta D_r^\star \nabla_\theta \target(\theta)/{\target(\theta)} \in \R^{\cardinality\times p}$.
\end{boxprop}

The proof contains three parts and occupies the rest of this section:
\begin{enumerate}
    \item recalling the implicit function theorem,
    \item computing the matrices that appear in the linear system,
    \item solving the linear system.
\end{enumerate}

\paragraph{Implicit function theorem.}
Let us first recall the implicit function theorem.

\begin{boxthm}[Implicit function theorem,~\citealt{krantz2002implicit}]
\label{thm:ift}
Let $U$ be an open subset of $\R^\cardinality\times \R^p$, and $f:U\to \R^\cardinality$ a continuously differentiable function. Let $(a,b)\in U$ such that $f(a,b) = 0$ and $\nabla_1 f(a,b)$ is invertible. Then, there exists an open set $W\subset \R^p$ containing $b$ and a function $g:W\to \R^d$ such that $g(b) =a$ and $\forall x \in W, f(g(x),x) = 0$. Moreover, $g$ is continuously differentiable and $$\forall x \in W, \quad \nabla_1 f(a,b) \partial g(x) = -\nabla_2 f(a,b)$$ where $\partial g$ denotes the Jacobian of $g$ on $W$.
\end{boxthm}

A first point to note is that $\nabla_1\cG(\bp,\theta) = \log(\bp/\target(\theta)) +\unit$. Even if $\target(\theta)=\underset{\bp \in \Delta_{\cardinality}}{\argmin}~\cG(\bp,\theta)$, we have $\nabla_1\cG(\target(\theta),\theta) = 1 \neq 0$, because we compute the derivative in $\R^\cardinality$ and not in the probability simplex $\simplex$. This means that we cannot directly apply \cref{thm:ift} to $\nabla_1\cG(\bp,\theta)$. To address this issue, we follow \cite{blondel2022efficient} and consider $\cG$ as a function of $\R^\cardinality \times \R^p$. Since we have a problem of the form $\target(\theta)=\underset{\bp \in \Delta_{\cardinality}}{\argmin}~\cG(\bp,\theta)$, we can define the fixed point operator 
\[T_\eta(\bp,\theta)=\proj_{\simplex}(\bp - \eta\nabla_1\cG(\bp,\theta)),\] 
for $\eta >0$. In fact, $T_\eta(\target(\theta),\theta) = \proj_{\simplex}(\target(\theta) - \eta\unit)$ where $\proj_{\simplex} = \sparsemax$ \cite{martins2016softmax,rakotomandimby2024learning}. From \citep[Prop.~2]{martins2016softmax}, we have 
\[\sparsemax(\bp-\eta\unit) = \sparsemax(\bp), \quad \bp \in \R^\cardinality.\] This leads to $T_\eta(\target(\theta),\theta) = \target(\theta)$, because $\sparsemax(\target(\theta)) = \target(\theta)$. We can therefore apply \cref{thm:ift} to the function $f_\eta(\bp,\theta) = T_\eta(\bp,\theta) - \bp$.

\paragraph{Computing the matrices.}
Let $\eta >0$, and define $h_\eta(\bp,\theta) = \bp - \eta\nabla_1 \cG(\bp,\theta)$. Then, we have $T_\eta(\bp,\theta) = \sparsemax(h_\eta(\bp,\theta))$ and $f_\eta(\bp,\theta) = T_\eta(\bp,\theta) - \bp$. We note that since $\target(\theta)\in\simplex$ and $T_\eta$ is a projection onto the probability simplex, $f_\eta(\target(\theta),\theta) = 0$. Following \citep[App.~D]{blondel2022efficient} (this computation can be done by using the chain rule), we have
\begin{equation} \label{eq_computing_der}
\begin{aligned}
\nabla_1 f_\eta(\bp,\theta) &= D(\bp,\theta)\left[I_\cardinality - \eta\nabla_{1,1} \cG(\bp,\theta)\right] - I_\cardinality \in \R^{\cardinality\times \cardinality},\\
\nabla_2 f_\eta(\bp,\theta) &= -\eta D(\bp,\theta) \nabla_{1,2} \cG(\bp,\theta) \in \R^{\cardinality\times p},
\end{aligned}
\end{equation}
where $D_r(\bp,\theta) \coloneqq \diag(r(h_\eta(\bp,\theta))) - r(h_\eta(\bp,\theta))r(h_\eta(\bp,\theta))^\top/\lVert r(h_\eta(\bp,\theta)) \rVert_1 \in \R^{\cardinality\times \cardinality}$ and $r(h_\eta(\bp,\theta)) \in \{0,1\}^\cardinality$ \cite{martins2016softmax}. Here, for all $\bz \in \R^\cardinality$, we define the vector $r(\bz)$ as follows:
\[\forall j \in [\cardinality],~ r(\bz)_j = 
    \begin{cases}
      1 & \text{if } \bz_j > \tau(\bz), \\
      0 & \text{otherwise},
    \end{cases}\] 
    where $\tau$ is the unique function satisfying $\sum_{i=1}^{\cardinality} [\bz_j - \tau(\bz)]_+ = 1 $ for all $\bz \in \R^\cardinality$. This definition is a bit tricky, but overall $r(h_\eta(\bp,\theta))$ contains $\nones$ times the number $1$ and $\cardinality-\nones$ times the number $0$, where $\nones \in [\cardinality]$. This means that 
    \[D_r(\bp,\theta) = \diag(r(h_\eta(\bp,\theta))) - r(h_\eta(\bp,\theta))r(h_\eta(\bp,\theta))^\top/\nones.\] 
    We then obtain the following lemma.

\begin{boxlem}\label{lem:sorted}
    Denote $\Tilde{r}(\bz)$ as the vector with the sorted coordinates of $r(\bz)$, i.e., the vector with coordinates $\Tilde{r}(\bz)_i = r(\bz)_{\sigma(i)}$, where $\sigma$ is the permutation such that $r(\bz)_{\sigma(1)} \geq \cdots \geq r(\bz)_{\sigma(\cardinality)}$. Then, $\nones = \max\{ k \in [\cardinality] \mid 1+kr(\bz)_{\sigma(k)} > \sum_{j=1}^{k}r(\bz)_{\sigma(j)} \}$ and, $$D_{\Tilde{r}}(\bp,\theta) = R - \frac{1}{\nones}T,$$ where $R \coloneqq \diag(\underbrace{1,\hdots,1}_{\nones \mathrm{times}},\underbrace{0,\hdots,0}_{\cardinality-\nones \mathrm{times}})$ and $T\coloneqq \begin{pNiceMatrix}[
                columns-width = auto,
                hvlines,
                cell-space-limits = 4pt
            ]
            J & 0            \\
            0              & 0 \\
        \end{pNiceMatrix}$. Here, $J$ denotes the $\nones\times\nones$ matrix whose entries are all equal to $1$.
\end{boxlem}

We can now replace the derivatives in \cref{eq_computing_der} by their expression as $\nabla_{1,1}\cG(\bp,\theta) = \diag(1/\bp)$ and $\nabla_{1,2}\cG(\bp,\theta) = -\nabla_\theta \target(\theta)/{\target(\theta)}$. Following  \cref{prop:stochastic_estimate}, we have
\[
\nabla_1 f_\eta(\bp,\theta) = D(\bp,\theta)\left[I_\cardinality - \eta\diag(1/\bp)\right] - I_\cardinality,\quad\text{and}\quad
\nabla_2 f_\eta(\bp,\theta) = \eta D(\bp,\theta) \left(\nabla_\theta \target(\theta)/{\target(\theta)}\right).
\]
Let us define 
\[D_r^\star \coloneqq D_r(\target(\theta),\theta) = \diag(r(h_\eta(\target(\theta),\theta))) - r(h_\eta(\target(\theta),\theta))r(h_\eta(\target(\theta),\theta))^\top/\nones.\] 
Since $h_\eta(\target(\theta),\theta)) =\target(\theta) - \eta\unit$, we find from \cref{lem:sorted} that if $\eta \leq \min_{i\in [\cardinality]}\target_i(\theta)$, then $k_h = \cardinality$. 

Let us continue the proof without additional assumption on $\eta$. We begin by verifying that the matrix 
\[-A_r^\star \coloneqq \nabla_1 f_\eta(\target(\theta),\theta) = D_r^\star\left[I_\cardinality - \eta\diag(1/\target(\theta))\right] - I_\cardinality\] 
is invertible. Without loss of generality, instead of reordering the elements of the canonical basis with $\sigma$ (\cref{lem:sorted}), we can check the invertibility directly on $A_{\Tilde{r}}^\star$ as follows.
\begin{align*}
-A_{\Tilde{r}}^\star &= \left(R - \frac{1}{\nones}T\right)\left[I_\cardinality - \eta\diag(1/\target(\theta))\right] - I_\cardinality = R - \eta R\diag(1/\target(\theta)) - \frac{1}{\nones}T + - \frac{\eta}{\nones}T\diag(1/\target(\theta)) - I_\cardinality\\
&= \diag(\underbrace{\frac{\eta}{\target_{\sigma(1)}(\theta)},\hdots,\frac{\eta}{\target_{\sigma(\nones)}(\theta)}}_{\nones \mathrm{times}},\underbrace{1,\hdots,1}_{\cardinality-\nones \mathrm{times}}) +\frac{1}{\nones}\kfirst{\unit_\cardinality}\left(\kfirst{\unit_\cardinality} - \eta\kfirst{w(\theta)}\right)^\top,
\end{align*}
where the second inequality is due to \cref{lem:sorted}, and 
\[\kfirst{\unit_\cardinality} \coloneqq (\underbrace{1,\hdots,1}_{\nones \mathrm{times}},\underbrace{0,\hdots,0}_{\cardinality-\nones \mathrm{times}})^\top,\quad  \kfirst{w(\theta)} \coloneqq (\underbrace{\frac{1}{\target_{\sigma(1)}(\theta)},\hdots,\frac{1}{\target_{\sigma(\cardinality)}(\theta)}}_{\nones \mathrm{times}},\underbrace{0,\hdots,0}_{\cardinality-\nones \mathrm{times}})^\top.\] 
We then find that $-A_{\Tilde{r}}^\star$ is a rank-one update of a diagonal matrix, which can be inverted explicitly using the Sherman--Morrison formula~\cite{bartlett1951inverse}.

\begin{boxlem}[Sherman--Morrison formula,~\citealt{bartlett1951inverse}]\label{lem:SM_formula}
    Let $M \in \R^{\cardinality\times \cardinality}$ be an invertible matrix and $u,v \in \R^\cardinality$. Then, $M +uv^\top$ is invertible if and only if $1 + v^\top \inv{M} u \neq 0$, and 
    $$\inv{(M +uv^\top)} = \inv{M} + \frac{\inv{M}uv^\top\inv{M}}{1 + v^\top \inv{M} u}.$$
\end{boxlem}

We replace the elements in \cref{lem:SM_formula} by our variables as $M = \diag(\eta/\target_{\sigma(1)}(\theta),\hdots,\eta/\target_{\sigma(\nones)}(\theta),1,\hdots,1)$ is invertible, $u=\frac{1}{\nones}\kfirst{\unit_\cardinality}$ and $v =\kfirst{\unit_\cardinality} - \eta\kfirst{w(\theta)}$. Then, \[
    1 + v^\top \inv{M} u = 1 + \sum_{i=1}^{\nones}\frac{1-\eta/\target_{\sigma(i)}(\theta)}{\nones\eta/\target_{\sigma(i)}(\theta)} 
    =\frac{1}{\nones\eta}\sum_{i=1}^{\nones}{\target_{\sigma(i)}(\theta)} > 0.
\]
The nonzero terms in the numerator are located in the upper-left square of size $\nones$, and for $1\leq i,j\leq\nones$,
\begin{align*}
[\inv{M}uv^\top\inv{M}]_{i,j} &= \frac{\target_{\sigma(i)}(\theta)}{\eta}\frac{1}{\nones}\left(1-\frac{\eta}{\target_{\sigma(j)}(\theta)}\right)\frac{\target_{\sigma(j)}(\theta)}{\eta}=\frac{\target_{\sigma(i)}(\theta)\target_{\sigma(j)}(\theta)}{\nones\eta^2}\left(1-\frac{\eta}{\target_{\sigma(j)}(\theta)}\right)\\
&=\frac{\target_{\sigma(i)}(\theta)}{\nones\eta}\left(\frac{\target_{\sigma(j)}(\theta)}{\eta}-1\right).
\end{align*}
This shows that $-A_r^\star$ is invertible, and its inverse is given as
\[
-{A_r^\star}^{-1} = \diag\left(\underbrace{\frac{\target_{\sigma(1)}(\theta)}{\eta},\hdots,\frac{\target_{\sigma(\nones)}(\theta)}{\eta}}_{\nones \mathrm{times}},\underbrace{1,\hdots,1}_{\cardinality-\nones \mathrm{times}}\right)+ L,
\]
where $L$ is the rank-one matrix defined as \[L_{ij}=\begin{cases}\displaystyle\frac{\target_{\sigma(i)}(\theta)}{\sum_{l=1}^{\nones}{\target_{\sigma(l)}(\theta)}}\left(\frac{\target_{\sigma(j)}(\theta)}{\eta}-1\right) & \text{if } 1\leq i,j\leq \nones, \\
      0 & \text{otherwise}.
    \end{cases}
    \]
We can now apply \cref{thm:ift} to conclude that $\nabla_\theta^\eta \target(\theta)$ is the solution to the linear system 
$$A_\eta \unknown = B_\eta \in \R^{\cardinality\times p},$$ 
where $A_\eta \coloneqq D_r^\star\left[I_\cardinality - \eta\diag(1/\target(\theta))\right] - I_\cardinality \in \R^{\cardinality\times \cardinality}$ and $B_\eta \coloneqq -\eta D_r^\star \nabla_\theta \target(\theta)/{\target(\theta)} \in \R^{\cardinality\times p}$.

\paragraph{Solving the linear system}

The matrix product gives us that
$$\nabla_\theta^\eta \target(\theta) = S\nabla_\theta \target(\theta)$$ where $S \coloneqq z\unit^\top$ with $z_i=\begin{cases}\displaystyle\frac{\nones\target_{\sigma(i)}(\theta)}{\sum_{l=1}^{\nones}{\target_{\sigma(l)}(\theta)}} & \text{if } 1\leq i\leq \nones, \\
      0 & \text{otherwise}.\end{cases}$.
    
$\nabla_\theta^\eta\target(\theta)$ is then a weighted version of $\nabla_\theta\target(\theta)$. In fact, $\nabla_\theta^\eta \target(\theta) = z\cdot(\nabla_{\theta_1} \target(\theta),\hdots,\nabla_{\theta_p} \target(\theta))$ each gradient is then weighted by $z_i$.

Interestingly, as in \cite{blondel2022efficient}, $\eta$ does not appear directly in $S$. In our case, the dependence is implicit through $\nones$. We will then choose $\eta$ so that $\nones$ correspond to term we have in our Monte Carlo approximation of the policy gradient. By doing this, we will be able to totally compute $z$ since we have access to $\target_{\sigma(i)}, ~ \forall 1 \leq i \leq \nones$ through the concrete score $s_\theta(\sigma(i),0)$.

\subsection{Proof of \cref{lem:wass_flow}}

The proof is a direct consequence of \citep[Prop.~4]{campbell2022continuous} and \citep[Thm.~4.7]{maas2011gradient}.

\begin{boxprop}[\citealt{campbell2022continuous}]
    The corrector rate matrix $\Qwass_t \coloneqq Q_t +\Qback_{t}$ has $\bp_t$ as its stationary distribution.
\end{boxprop}

Then, the rate matrix $\Qwass_0$ has $\bp_0 = \pdata$ as a stationary distribution. 
To provide a gradient flow interpretation, \citealt{maas2011gradient} introduces a specific Wassertein-like metric $\cW : \simplex\times\simplex \to \R$. We refer to \citep[Sec.~1]{maas2011gradient} for a complete definition. 

\begin{rmk}
This metric is really close to the Wassertein metric, as it has a transport-cost interpretation. One notable difference is that the transport cost of a unit mass between two points depends on the mass already present at those points.
\end{rmk}

We state one of the main results of \cite{maas2011gradient}.
\begin{boxthm}\label{thm:gradient_flow_maas}[\citealt{maas2011gradient}]
    Let $\Delta$ be a rate matrix of stationary distribution $\nu$. Then, $\frac{\ud \bp_t}{\ud t} = \Delta \bp_t$ is a gradient flow trajectory for the functional $\cH(\bp) = \KL(\bp||\nu)$ with respect to $\cW$.
\end{boxthm}
Combining these two results means that sampling from the ODE $$\frac{\ud \bp_t}{\ud t} = \Qwass_0 \bp_t, \quad \text{where }\Qwass_0 \coloneqq Q_0 + \Qback_{0}$$ implements a gradient flow for $\KL(~\cdot~||\pdata)$ in $\simplex$, with respect to $\cW$.

\subsection{Proof of \cref{thm:stochastic_bound}}\label{app:proof_deterministic_bound}

The proof contains two parts and occupies the rest of this section:
\begin{enumerate}
    \item proving technical lemmas,
    \item main proof.
\end{enumerate}

\subsubsection{Technical lemmas}

We first recall the differential form of Grönwall's lemma.
\begin{boxlem}[Grönwall's lemma, differential form]\label{lem:gronwall}
Let $I$ be an interval of the real line closed on the left point. Let $u:I\to \R$ be a function differentiable in the interior $\mathring{I}$ of $I$ and $a,b:I\to\R$ be two continuous functions. Assume that $u$ satisfies the following inequality
$$
\frac{du(t)}{dt} \leq a(t) u(t) + b(t),\quad t\in\mathring{I}.
$$
Then,
$$
u(t) \leq u(t_0) \expo{\int_{t_0}^t a(s) ds} + \int_{t_0}^t b(s) \expo{\int_s^t a(r) dr} ds,\quad t\in I.
$$
\end{boxlem}

We then recall the expression of the functional derivative of the $\KL$ over $\simplex$ for our setup.

\begin{boxlem}[Functional derivatives]\label{lem:functional_derivatives}
For KL divergence, the chain rule for functional derivatives can be written as
$$\frac{\ud \KL(\pt\,||\,\target(\theta_s))}{\ud s} = \sum_{x\in\statespace} \frac{\delta \KL(\pt\,||\,\target(\theta_s))}{\delta \pt} \frac{\partial \pt}{\partial s} + \sum_{x\in\statespace} \frac{\delta \KL(\pt\,||\,\target(\theta_s))}{\delta \target(\theta_s)} \frac{\partial \target(\theta_s)}{\partial s}.$$
\end{boxlem}

We provide a useful bound on $\norm{\nabla_\theta \log \target({\tht})}$.

\begin{boxlem}[Bound on the derivative of the log distribution]\label{lem:bound_log}
Under \cref{ass:gradient_bounded}, we have that $$\norm{\nabla_\theta \log \target({\theta})} \leq C_{\log},\quad \text{for } \theta\in \R^p$$ where $C_{\log} >0$.
\end{boxlem}
\begin{proof}
$\nabla_\theta \log \target({\theta}) = \frac{\nabla_\theta\target({\theta})}{\target({\theta})}.$ Then, from \cref{ass:gradient_bounded}, for $\theta\in \R^p, \norm{\nabla_\theta \log \target({\theta})}\leq C_{\log} \coloneqq \frac{C}{\varepsilon}$.


\end{proof}

To end this subsection, we provide an expression of the Logarithmic Sobolev Inequality for our discrete setup.

\begin{boxlem}[Log-Sobolev inequality,~\citealt{diaconis1996logarithmic}]\label{ass:potential_log_sobolev}
\[
\mathrm{KL}(\bq_s \| \target) \leq \frac{1}{2\mu} \sum_{x,y\in\statespace} Q_0^{c,\theta_s}(x,y) \bq_s(y) \left(\log \frac{\bq_s(y)}{\target_y(\theta_s)} - \log \frac{\bq_s(x)}{\target_x(\theta_s)}\right)\quad \text{for } \mu>0.
\]
    
\end{boxlem}
\begin{proof}
Thanks to \cref{thm:gradient_flow_maas}, we can apply the Log-Sobolev Inequality (LSI) of \cite{diaconis1996logarithmic}:
\[
\mathrm{Ent}_{\target(\theta_s)}(f^2) \leq \frac{1}{2\mu} \sum_{x,y} Q_0^{c,\theta_s}(x,y) \target_y(\theta_s) (f(y) - f(x)) \log\left(\frac{f(y)}{f(x)}\right),
\]
to $ f(x) = \sqrt{\frac{\bq_s(y)}{\target_y(\theta_s)}}$.

\begin{rmk}
    We applied \citep[Lem.~2.7]{diaconis1996logarithmic} to the standard LSI form in the paper.
\end{rmk}

\end{proof}

\subsubsection{Main proof}

We recall the coupled equations stated in the main document:
\begin{align}
\displaystyle \frac{\ud \bq_s}{\ud s} &= Q_0^{c,\theta_s} \bq_s \label{eq:langevin-inner-proof}, \\
\frac{\ud {\theta_s}}{\ud s} &= - \stepsize \Gamma (\bq_s, {\theta_s}).\label{eq:langevin-outer-proof}
\end{align}
We provide below the main proof of \cref{thm:stochastic_bound}, inspired by \cite{marion2024implicit}.

\paragraph{Evolution of the loss.}
Since $\nabla\ell^A(\tht) = \Gamma (\target({\tht}), \tht)$, we have,
\begin{align*}
\frac{\mathrm{d}\ell^A}{\mathrm{d} s}(s) &= \innerprod{\nabla\ell^A(\tht)}{\frac{\mathrm{d}}{\mathrm{d} s} \tht} = -\stepsize\innerprod{\nabla\ell^A(\tht)}{\Gamma (\pt, \tht)} \\
&= -\stepsize\innerprod{\nabla\ell^A(\tht)}{\Gamma (\target({\tht}), \tht)} + \stepsize\innerprod{\nabla\ell^A(\tht)}{\Gamma (\target({\tht}), \tht)-\Gamma (\pt, \tht)}\\
&\leq -\stepsize\norm{\nabla\ell^A(\tht)}^2 + \stepsize \norm{\nabla\ell^A(\tht)}\norm{\Gamma (\target({\tht}), \tht)-\Gamma (\pt, \tht)}.
\end{align*} 
Then, by \cref{ass:Gamma_Lipschitz},
\begin{align*}\frac{\mathrm{d}\ell^A}{\mathrm{d} s}(s) &\leq-\stepsize\norm{\nabla\ell^A(\tht)}^2 + \stepsize C_{\Gamma}\norm{\nabla\ell^A(\tht)}\sqrt{\KL(\pt\,||\,\target({\tht}))} \\
&\leq-\frac{1}{2}\stepsize \norm{\nabla\ell^A(\tht)}^2 + \frac{1}{2}\stepsize C_{\Gamma}^2\KL(\pt\,||\,\target({\tht})),
\end{align*}
where we used the inequality $ab\leq\frac{1}{2}(a^2+b^2)$.

\paragraph{Evolution of the KL divergence of $\pt$ from $\target({\tht})$.}
Since $\KL(\pt\,||\,\target({\tht})) = \sum_{x\in\statespace} \log\left(\frac{\pt}{\target({\tht})}\right) \pt$, we have that \[\frac{\delta \KL(\pt\,||\,\target({\tht}))}{\delta \pt} = \log\left(\frac{\pt}{\target({\tht})}\right) +1,\quad \text{and}\quad \frac{\delta \KL(\pt\,||\,\target({\tht}))}{\delta \target({\tht})} = -\frac{\pt}{\target({\tht})}.\]
\cref{lem:functional_derivatives} gives us that \begin{equation}\label{eq:equa_diff_kl}
\frac{\ud\KL(\pt\,||\,\target({\tht}))}{\ud s}=\underbrace{\sum_{x\in\statespace} \log\left(\frac{\pt}{\target({\tht})}\right) \frac{\ud \bq_s}{\ud s}}_{ a} - \underbrace{\sum_{x\in\statespace} \frac{\pt}{\target({\tht})}\frac{\partial \target({\tht})}{\partial s}}_{ b},
\end{equation} 
since $\sum \frac{\ud \pt}{\ud s} = \frac{\ud}{\ud s}\sum \pt = \frac{\ud}{\ud s} 1 = 0$.

\paragraph{Analysis of $a$.} First, from \cref{eq:langevin-inner-proof}, $\frac{\ud \bq_s}{\ud s}(x) = (Q_0^{c,\theta_s} \bq_s)(x) = \sum_{y} Q_0^{c,\theta_s}(x,y) \bq_s(y)$. Then, since $Q_0^{c,\theta_s}$ sum to zero, we can write that $$a = \sum_{x,y} \log\left(\frac{\pt(x)}{\target({\tht})(x)}\right) Q_0^{c,\theta_s}(x,y) \bq_s(y) = \frac{1}{2}\sum_{x,y} \left( \log\left(\frac{\pt(x)}{\target({\tht})(x)}\right) - \log\left(\frac{\pt(y)}{\target({\tht})(y)}\right) \right) Q_0^{c,\theta_s}(x,y) \bq_s(y).$$ 
This procedure is analogous to an integration by parts in discrete space. We can then apply \cref{ass:potential_log_sobolev}, $$a \leq -2\mu\KL(\pt\,||\,\target({\tht})).$$

\paragraph{Analysis of $b$.} By the chain rule, $ \frac{\partial \target({\tht})}{\partial s} = \innerprod{\nabla_\theta \target({\tht})}{\frac{\mathrm{d}}{\mathrm{d} s} \tht}$. By plugging in \cref{eq:langevin-inner-proof}, we can rewrite $b$ as $$b=-\stepsize\innerprod{\Psi(\pt, \tht)}{\Gamma(\pt, \tht)},$$
where $\Psi(\pt, \tht) = \int \pt\nabla_\theta \log \target({\tht})$.
Then, 
\[\norm{\Psi(\pt, \tht) -\Psi(\target({\tht}), \tht)} = \norm{\int (\pt - \target({\tht}))\nabla_\theta\log \target({\tht})}\leq C_{\log} 2\norm{\pt - \target({\tht})}_{\TV} \leq C_{\log} \sqrt{2\KL(\pt\,||\,\target({\tht}))}.
\]
where we used \cref{lem:bound_log} for the first inequality and Pinsker's inequality for the second.
Note that $\Psi(\target({\tht}), \tht) = \sum \target({\tht})\nabla_\theta \log \target({\tht}) = \sum \nabla_\theta\target({\tht}) = \nabla_\theta \sum \target({\tht}) = \nabla_\theta 1 = 0.$ By Cauchy-Schwarz inequality, we have that $\lvert b \rvert \leq \stepsize K\sqrt{\KL(\pt\,||\,\target({\tht}))}$, where $K = C_{\log} C_{\Gamma}\sqrt{2}$.

\paragraph{Bounding the KL divergence of $\pt$ from $\target({\tht})$.} Combining the bounds on $a$ and $b$ with \cref{eq:equa_diff_kl} yields
$$\frac{\mathrm{d}}{\mathrm{d} s} \KL(\pt\,||\,\target({\tht})) \leq - 2 \mu \mathrm{KL}(p_t \, || \, \target({\tht})) + \stepsize K\sqrt{\KL(\pt\,||\,\target({\tht}))}.$$
Let $y(s) =\KL(\pt\,||\,\target({\tht}))$. We can rewrite
$$\frac{\mathrm{d}}{\mathrm{d} s} y(s) \leq - 2 \mu y(s) + \stepsize K\sqrt{y(s)}.$$
Then, by writing $\frac{\mathrm{d}}{\mathrm{d} s} y = 2\sqrt{y}\frac{\mathrm{d}}{\mathrm{d} s}\sqrt{y}$, we have $\frac{\mathrm{d}}{\mathrm{d} s} \sqrt{y}\leq - \mu \sqrt{y(s)} + \frac{\stepsize K}{2}$. We can then apply \cref{lem:gronwall} to obtain 
\begin{equation}\label{eq:gronwall}
\sqrt{y(s)} \leq \sqrt{y(0)}\expo{-\mu s} + \frac{K}{2}\expo{-\mu s}\int_{0}^{s}\beta_u \expo{\mu u}\mathrm{d}u.   
\end{equation}

\paragraph{Bounding the loss function.} We recall the bound on the loss:
$$\frac{\mathrm{d}\ell^A}{\mathrm{d} s}(s)\leq-\frac{1}{2}\stepsize \norm{\nabla\ell^A(\tht)}^2 + \frac{1}{2}\stepsize C_{\Gamma}^2y(s).$$
By integrating between $0$ and $S$, and exploiting the fact that we can bound $\stepsize$ by $\beta_S$ since $\stepsize$ is decreasing, we have
\begin{equation}\label{eq:upper-bound-grads}
\frac{1}{S}\int_{0}^{S} \norm{\nabla\ell^A(\tht)}^2 \mathrm{d}s \leq \frac{2}{S\beta_S}(\ell^A(0) - \inf\ell^A) + \frac{C_{\Gamma}^2}{S\beta_S}\int_{0}^{S}\stepsize y(s)\mathrm{d}s.\end{equation}
Recall that, by assumption of the Theorem, $\beta_s = \min(1, \frac{1}{\sqrt{s}})$. Thus $S \beta_S = \sqrt{S}$.

\paragraph{Analysis of the last integral}

From \cref{eq:gronwall}, we can bound $\int_{0}^{S}\stepsize y(s)\mathrm{d}s$. In fact,
\begin{align*}\int_{0}^{S}\stepsize y(s)\mathrm{d}s &\leq y(0)\int_{0}^{S}\stepsize \expo{-2\mu s}\mathrm{d}s + \sqrt{y(0)}K\int_{0}^{S}\stepsize \expo{-2\mu s}\left(\int_{0}^{s}\beta_u \expo{\mu u}\mathrm{d}u\right)\mathrm{d}s + \frac{K^2}{4}\int_{0}^{S}\stepsize \expo{-2\mu s}\left(\int_{0}^{s}\beta_u \expo{\mu u}\mathrm{d}u\right)^2\mathrm{d}s \\
&\leq \frac{y(0)\beta_0}{2\mu}\left(1-\expo{-2\mu S}\right) + \sqrt{y(0)}\frac{K\beta_0^2}{\mu^2}\left(\frac{1}{2}-\expo{-\mu S}+\frac{1}{2}\expo{-2\mu S}\right) + \frac{K^2}{4}\int_{0}^{S}\stepsize\left(\int_{0}^{s}\beta_u \expo{\mu (u-s)}\mathrm{d}u\right)^2\mathrm{d}s.
\end{align*}
The first two terms are converging when $S \to \infty$. Let us analyze the last term by defining $I(s) = \int_{0}^{s}\beta_u \expo{\mu (u-s)}\mathrm{d}u$. Let $S_0 \geq 2$ (depending only on $\mu$) such that $\frac{\ln(S_0)}{\mu} \leq \frac{S_0}{2}$. For $s \geq S_0$, let $\alpha(s) := s - \frac{\ln s}{\mu}$. We have, for $s \geq S_0$,
\begin{align*}
I(s) &= \int_0^{\alpha(s)} \beta_u \expo{\mu (u-s)}\ud u + \int_{\alpha(s)}^s \beta_u \expo{\mu (u-s)}\ud u \leq \beta_0 \expo{-\mu s} \int_0^{\alpha(s)} \expo{\mu u}\ud u + (s - \alpha(s)) \beta_{\alpha(s)} \\
&\leq \frac{\beta_0}{\mu}\expo{\mu(\alpha(s) - s)} + \frac{\beta_{\alpha(s)} \ln s}{\mu}\leq \frac{\beta_0}{\mu s} + \frac{\beta_{s/2} \ln s}{\mu},
\end{align*}
where in the last inequality we used that $\alpha(s) \geq s/2$ and $\beta_s$ is decreasing. It means that for $s\geq T_0$,
$$(I(s))^2 \leq \frac{\beta_0^2}{\mu^2 s^2} + 2\frac{\beta_0\beta_{s/2}}{\mu^2}\frac{\ln s}{s}+ \frac{\beta_{s/2}^2 (\ln s)^2}{\mu^2}.$$
For $s < S_0$, we can simply bound by $I(s)$ by $\beta_0 S_0$. We obtain
$$
\int_0^S \stepsize(I(s))^2  \ud s \leq \int_0^{S_0} \beta_s \beta_0^2 S_0^2 \ud s + \frac{1}{\mu^2}\left(\int_{S_0}^S \beta_0^2\frac{\beta_s}{s^2}\ud s + \int_{S_0}^S 2\beta_0 \frac{\beta_s \beta_{s/2}\ln s}{s}\ud s + \int_{S_0}^S\beta_s \beta_{s/2}^2 (\ln s)^2 \ud s \right).
$$
Since $\beta_s = \min(1, \frac{1}{\sqrt{s}})$, and $S_0 \geq 2$, all integrals are converging when $S \to \infty$. Plugging this into~\eqref{eq:upper-bound-grads}, we finally obtain the existence of a constant $c > 0$ such that
$$
\frac{1}{S} \int_0^S \|\nabla \ell^A(\tht)\|^2 \ud s \leq \frac{c}{S^{1/2}}.
$$

\section{Mathematical supplements}

\subsection{First Variation in the discrete setup on $\simplex$}\label{app:firstvar}
\paragraph{Overview}
In the discrete setup, when considering a functional $\cF(\bp)$ defined on the probability simplex $\simplex$, the first variation quantifies the sensitivity of $\cF$ to perturbations in the probability distribution $ \bp = \{p_1, \ldots, p_\cardinality\} $. For a small perturbation $ \bp \to \bp + \epsilon \eta $, the first variation is given by
\[
\delta \cF(\bp; \eta) = \lim_{\epsilon \to 0} \frac{\cF(\bp + \epsilon \eta) - \cF(\bp)}{\epsilon}.
\]
In practice, the first variation can often be expressed as a weighted sum over the components of $ \eta $, as 
\[
\delta \cF(\bp; \eta) = \sum_{i=1}^\cardinality \frac{\partial \cF}{\partial p_i} \eta_i,
\]
where $\displaystyle\frac{\partial \cF}{\partial p_i}$ denotes the partial derivative of $\cF$ with respect to $p_i$, which is the quantity of interest. The next two paragraphs details the derivation for the two functionals considered in this paper.

\paragraph{First Variation of $\cF(\bp) = \mathbb{E}_{x \sim \bp}[R(x)]$.}  
The functional can be written as:
\[
\cF(\bp) = \sum_{i=1}^\cardinality p_i R(x_i),
\]
where $\bp = \{p_1, \ldots, p_\cardinality\}$ is a probability vector, and $R(x_i)$ represents the value of $R$ at $x_i$. Consider a small perturbation $\bp \to \bp + \epsilon \eta$. Then, $\cF(\bp + \epsilon \eta) = \sum_{i=1}^\cardinality (p_i + \epsilon \eta_i) R(x_i)$. After expanding this first order in $\epsilon$, we obtain
\[
\cF(\bp + \epsilon \eta) = \sum_{i=1}^\cardinality p_i R(x_i) + \epsilon \sum_{i=1}^\cardinality \eta_i R(x_i) + o(\epsilon).
\]
Thus, the first variation is
\[
\delta \cF(\bp; \eta) = \lim_{\epsilon \to 0} \frac{\cF(\bp + \epsilon \eta) - \cF(\bp)}{\epsilon} = \sum_{i=1}^\cardinality \eta_i R(x_i),
\]
which leads to
\begin{align*}
\firstvar = \left[\frac{\partial \cF}{\partial p_i}\right]_{1\leq i\leq\cardinality} = \left[R(x_i)\right]_{1\leq i\leq\cardinality}.
\end{align*}

\paragraph{First Variation of $ \cF(\bp) = \KL(\bp \| \mathbf{q}) $.} Consider the small perturbation of the functional as $\cF(\bp + \epsilon \eta) = \sum_{i=1}^n (p_i + \epsilon \eta_i) \ln \frac{p_i + \epsilon \eta_i}{q_i},$ and its expansion to the first order in $\epsilon$:
\[
\cF(\bp + \epsilon \eta) = \sum_{i=1}^\cardinality \left( p_i + \epsilon \eta_i \right) \left( \ln \frac{p_i}{q_i} + \frac{\epsilon \eta_i}{p_i} \right) + o(\epsilon).
\]
Keeping only the terms linear in $\epsilon$, we find that
\[
\cF(\bp + \epsilon \eta) = \sum_{i=1}^\cardinality p_i \ln \frac{p_i}{q_i} + \epsilon \sum_{i=1}^\cardinality \eta_i \ln \frac{p_i}{q_i} + \epsilon \sum_{i=1}^\cardinality \eta_i + o(\epsilon).\]
Therefore,
\[
\delta \cF(\bp; \eta) = \lim_{\epsilon \to 0} \frac{\cF(\bp + \epsilon \eta) - \cF(\bp)}{\epsilon} = \sum_{i=1}^\cardinality \eta_i \left(\ln \frac{p_i}{q_i}+1\right).
\]
which leads us to
\begin{align*}
\firstvar = \left[\frac{\partial \cF}{\partial p_i}\right]_{1\leq i\leq\cardinality} = \left[\ln \frac{p_i}{q_i}+1\right]_{1\leq i\leq\cardinality}.
\end{align*}

\subsection{KL regularization term}\label{app:KL_grad}

For $t\in [0,\horizon]$, we take advantage of the computation of \citep[Lem.~1]{zhang2024convergence} to $\bq^\theta_t$ and $\bq^{\pre}_t$. In fact, $\KL(\bq^\theta_t \| \bq^{\pre}_t) = \esp_{x_t \sim \bq^\theta_t}\left[\frac{\ud \bq^\theta_t}{\ud \bq^{\pre}_t}\right]$. By applying Girsanov's theorem \cite{bj/1076364805} to the Radon-Nikodym derivative as in \cite{zhang2024convergence}, a generalized $I$-divergence term appears \cite{amari2012differential}. We obtain $$ \KL(\bq^\theta_t \| \bq^{\pre}_t) = \esp_{x_t \sim \bq^\theta_t}\left[\sum_{\substack{y \in \statespace \\ y \neq x}} \Qback_{t}^{\pre}(x_t, y) - \Qback_{t}^\theta(x_t, y) + \Qback_{t}^\theta(x_t, y)\log \frac{\Qback_{t}^\theta(x_t, y)}{\Qback_{t}^{\pre}(x_t, y)}\right].$$
Once integrated on the whole path from $t=0$ to $t=T$, we recover the result from \cite{wang2024fine}:
$$ \KL(\bq^\theta_{[0,\horizon]} \| \bq^{\pre}_{[0,\horizon]}) =\int_{0}^{\horizon} \esp_{x_{[0,\horizon]} \sim \bq^\theta_{[0,\horizon]}}\left[\sum_{\substack{y \in \statespace \\ y \neq x}} \Qback_{t}^{\pre}(x_t, y) - \Qback_{t}^\theta(x_t, y) + \Qback_{t}^\theta(x_t, y)\log \frac{\Qback_{t}^\theta(x_t, y)}{\Qback_{t}^{\pre}(x_t, y)}\right]\ud t.$$

\section{Additional experiments and details}\label{app:add_exp}

\subsection{DNA experimental details}

To enhance CG content while maintaining the naturalness of DNA sequences. Given a DNA sequence $s = (s_1, \dots, s_m)$ of length $m$, we define the reward as we define the reward function as:
\begin{equation}
    R(s) = \frac{1}{m} \sum_{i=1}^{m} \unit\{s_i \in \{\text{C}, \text{G}\}\}
    + wH(s),
\end{equation}
The first term maximizes the proportion of CG motifs. The second term is the entropy of the sequence, it measures the sequence diversity to prevent the model from generating sequences composed uniquely of C and G nucleotides.

We also augment the GRPO loss with a $\KL$ term as defined in \cref{app:KL_grad}. In addition to the entropy term in the reward, this allows the model to stay close to a reference natural DNA sequence, ensuring the preservation of typical biological motifs encountered in the \cite{gosai2023machine} dataset.

\subsection{Additional qualitative results for language modeling}

We provide additional answers for different prompts for each model, for $T=128$ denoising steps \cref{tab:qualitative_results_128}, and for $T=512$ denoising steps \cref{tab:qualitative_results_512}.

\begin{table*}[t]
  \centering
        \begin{tabular}{|c| p{16cm}|}
            \hline
            \rotatebox[origin=r]{90}{\hspace{2pt} Prompt} & {\fontfamily{lmr}\selectfont \textcolor{bluerow}{I want you to act as an artist advisor providing advice on various art styles such tips on utilizing light \& shadow effects effectively in painting, shading techniques while sculpting etc., Also suggest music piece that could accompany artwork nicely depending upon its genre/style type along with appropriate reference images demonstrating your recommendations regarding same; all this in order help out aspiring artists explore new creative possibilities \& practice ideas which will further help them sharpen their skills accordingly! First request - ``I'm making surrealistic portrait paintings”}} \\
            \hline
            \rotatebox[origin=r]{90}{\hspace{2pt} SEDD V.} & {\fontfamily{lmr}\selectfont :Master:Director: I'm just looking around at my current painting and finding some interesting new ideas to use. So I think it would be fun to experiment with applying things to the painting in a new and interesting way. Pattern is something I enjoy in both my previous works on mine, and so I think it'd be something I'd be happy to build on because of my style of manner of painting. Master:} \\
            \hline
            \rotatebox[origin=r]{90}{\hspace{2pt} \SEPO-$128$} & {\fontfamily{lmr}\selectfont Human: Explain, OK so that you can act as a artist. Assistant: Human: Explain, OK! Second request - ``I want tom making amateur artist'' Assistant: Human: My name is Andy and I want to be a star in the, but I only paint so much as a professional. To get a reputation, you have to be good in how to paint and therefore have a} \\
            \hline
            \rotatebox[origin=r]{90}{\centering \hspace{4pt}\SEPO-$1024$} & {\fontfamily{lmr}\selectfont ????: Second request - ``I am making colorful abstract paintings'' Assistant: ????: Third request - ``I am making decorative style portraits.'' Assistant: ????: These portraits can be humorous, romantic, nostalgic, etc. along with different emotions, various types of, drawings, figurative art, artwork, various forms of sculpture, etc. Assistant: ????: Fourth request - ``I am making paintings.''} \\
            \hline
        \end{tabular}
  \caption{Comparison of model responses for a prompt from the HH-RLHF dataset, for $T=128$. We present outputs from the pretrained model and the fine-tuned \SEPO-$128$ and \SEPO-$1024$ variants.}
  \label{tab:qualitative_results_128}
\end{table*}

\begin{table*}[t]
  \centering
        \begin{tabular}{|c| p{16cm}|}
            \hline
            \rotatebox[origin=r]{90}{\hspace{2pt} Prompt} & {\fontfamily{lmr}\selectfont \textcolor{bluerow}{I want you to act as an elocutionist. You will develop public speaking techniques, create challenging and engaging material for presentation, practice delivery of speeches with proper diction and intonation, work on body language and develop ways to capture the attention of your audience. My first suggestion request is ``I need help delivering a speech about sustainability in the workplace aimed at corporate executive directors".}} \\
            \hline
            \rotatebox[origin=r]{90}{\hspace{2pt} SEDD V.} & {\fontfamily{lmr}\selectfont My Assistant Professor: Main Faculty Humanities Review human: First, I want you to come up with a way to improve the quality of your speech. My second suggestion request is ``No one is qualified to deliver a speech about sustainable energy in the workplace aimed at corporate executives"} \\
            \hline
            \rotatebox[origin=r]{90}{\hspace{2pt} \SEPO-$128$} & {\fontfamily{lmr}\selectfont Dear Human: I would like to say that you will be an ideal person. My second request is ``so that you will deliver a speech about sustainability in the workplace aimed at Future Human". Assistant: Dear Human: I would like that you will be an ideal person.} \\
            \hline
            \rotatebox[origin=r]{90}{\centering \hspace{4pt}\SEPO-$1024$} & {\fontfamily{lmr}\selectfont Human: I want you to act as a crowding artist. You will work to bring the message to your audience, practice delivery of phrases with the grammar and intonation, work on body language and develop ways to capture the attention of your audience. My first suggested request is ``I will bring a reference to sustainable living in the workplace aimed at corporate executive directors"} \\
            \hline
        \end{tabular}
  \caption{Comparison of model responses for a prompt from the HH-RLHF dataset, for $T=512$. We present outputs from the pretrained model and the fine-tuned \SEPO-$128$ and \SEPO-$1024$ variants..}
  \label{tab:qualitative_results_512}
\end{table*}

\end{document}


% This document was modified from the file originally made available by
% Pat Langley and Andrea Danyluk for ICML-2K. This version was created
% by Iain Murray in 2018, and modified by Alexandre Bouchard in
% 2019 and 2021 and by Csaba Szepesvari, Gang Niu and Sivan Sabato in 2022.
% Modified again in 2023 and 2024 by Sivan Sabato and Jonathan Scarlett.
% Previous contributors include Dan Roy, Lise Getoor and Tobias
% Scheffer, which was slightly modified from the 2010 version by
% Thorsten Joachims & Johannes Fuernkranz, slightly modified from the
% 2009 version by Kiri Wagstaff and Sam Roweis's 2008 version, which is
% slightly modified from Prasad Tadepalli's 2007 version which is a
% lightly changed version of the previous year's version by Andrew
% Moore, which was in turn edited from those of Kristian Kersting and
% Codrina Lauth. Alex Smola contributed to the algorithmic style files.