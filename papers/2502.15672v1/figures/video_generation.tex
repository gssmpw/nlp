\begin{figure}[h]
\centering
\begin{subfigure}{\linewidth} 
\includegraphics[width=\linewidth]{figures/images/video_generation/2757_context.png}
\caption{Real context frames.}
\label{fig:context_frames}
\end{subfigure}
\begin{subfigure}{\linewidth} 
\includegraphics[width=\linewidth]{figures/images/video_generation/2757_small.png}
\caption{Generated frames with \vm-S (fine-tuned).}
\label{fig:video_generation_vms}
\end{subfigure}
\begin{subfigure}{\linewidth} 
\includegraphics[width=\linewidth]{figures/images/video_generation/2757_big.png}
\caption{Generated frames with \vm-L (fine-tuned).}
\label{fig:video_generation_vml}
\end{subfigure}
\caption{\textbf{Video generation with \vm}. Given 4 context frames (\subref{fig:context_frames}), we generate the next 4 frames with \vm-S (\subref{fig:video_generation_vms}) and \vm-L (\subref{fig:video_generation_vml}). Both models correctly predict the overall structure, but only \vm-L generates the expected motion.}
%\caption{\textbf{Video generation with \vm}. Given 4 context frames (\subref{fig:context_frames}), we generate 4 frames with \vm-S (\subref{fig:video_generation_vms}) and \vm-L (\subref{fig:video_generation_vml}). Both models generate corretly the overall structure, \vm-L generates an expected motion.}
\label{fig:video_generation}
\end{figure}