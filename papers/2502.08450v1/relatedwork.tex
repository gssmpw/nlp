\section{Related Works}
%Auxiliary training the AES model with various other tasks has proven to be effective \cite{gec_aes1, gec_aes2, gec_aes3}.   
%To improve the performance of AES, several studies have auxiliary trained the model with various other tasks such as morpho-syntactic labeling, type and quality prediction, and sentiment analysis \cite{gec_aes1, gec_aes2, gec_aes3}. Suggesting that detecting the number and type of grammatical errors in the essay is a beneficial indicator for the quality of the essay, \citet{gec_aes5} jointly trained grammatical error detection task with the scoring model via multi-task learning. Instead of jointly training auxiliary tasks that require resources, our direct use of corrected text output significantly reduces the training burden.   

%Previously, GEC has been used as a feature extraction tool for conventional feature engineering approaches to extract grammatical features \cite{gec_aes6} or measure the number of grammar corrections \cite{gec_aes7}. Unlike the existing studies, we directly utilize the text output generated by the GEC (without training) as input for the scoring model.

%As in traditional feature engineering methods,
% 제안사항
To improve the performance of AES, several studies have auxiliary trained the model with various other tasks such as morpho-syntactic labeling, type and quality prediction, and sentiment analysis \cite{gec_aes1, gec_aes2, gec_aes3}. Instead of jointly training auxiliary tasks, our direct use of corrected text output significantly reduces the training burden.

There have been attempts to apply grammatical error information. Suggesting that detecting grammatical errors is a beneficial indicator for the quality of the essay, \citet{gec_aes5} jointly trained grammatical error detection task with the scoring model. Also, \citet{gec_aes6} utilize grammatical features proposed by \citet{criterial_feature} and \citet{gec_aes7} use GEC to measure the number of grammar corrections. Unlike the existing studies, we directly utilize the text output generated by the GEC without additional training as input for the scoring model.
%


% 태희 작성 버전
%There have been numerous(?) studies that used auxiliary tasks to improve AES performance(\citet{gec_aes1}; \citet{gec_aes2}; \citet{gec_aes3}). Among these studies(?), several have focused on incorporating grammatical elements in AES system. For instance, \citet{gec_aes5} enhanced AES performance through multi-task learning by incorporating grammatical error detection task. \citet{gec_aes6} has focused on extracting grammatical features by using CEFR and explicitly using them as input of AES. \citet{gec_aes7} explored the use of grammar error correction(GEC) model, but they utilized GEC model only to measure the number of corrections.
%In this paper, we use GEC model to obtain corrected sentences from the essay and applied ERRANT(\citet{bryant-etal-2017-automatic}) to tag the modified parts.
%To the best of our knowledge, this is the first try to explicitly(?) incorporate the output of GEC model as input of AES model.