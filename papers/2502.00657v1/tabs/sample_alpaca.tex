\begin{table}[!ht]
    \centering
\resizebox{\linewidth}{!}{
    \begin{tabular}{ll}
    \toprule
        \textbf{Prompts} & why the sky is blue?\\
    \midrule
        \textbf{Base} &  {\small why is the sun yellow? why does the moon look like a white spot?} \\
        & {\small why are the stars not visible at night? why the earth is round?}\\
    \midrule
        \textbf{DPO} & {\small The color blue is caused by the reflection of light off of the water, air, and other objects in the atmosphere.} \\
        & {\small The wavelength of visible light, which includes blue, is shorter than other colors, so it is able to pass through these layers of matter, allowing light to travel further and be reflected.} \\
        & {\small In the case of our blue sky, the combination of water droplets, ice crystals, dust, clouds, sunlight, etc. all contribute to the color of blue in our skies.} \\
        & {\small This combination creates a beautiful and vibrant experience for those who witness it, regardless of whether it is the result of human activity or natural processes.} \\
        & {\small Overall, while it may not be fully explained by science, there is still much to admire and appreciate about the beauty of a blue day in nature, no matter where you are in this world. }\\
        \midrule
        \textbf{KTO} & {\small The color blue is due to the presence of a certain wavelength of light in the electromagnetic spectrum.} \\
        & \small This particular wavelength is not absorbed by the materials that make up the atmosphere, so it gets scattered in all directions. \\
        & \small The further away the light is from the source, the less it is scattered and the more blue it appears. \\
        & Since the sun is very far away, it scatters light very little, and therefore appears blue to us. \\
        & \small However, there are other colors that can be seen in nature, such as red, orange, green, etc., and these colors are not present in our atmosphere. \\
        & \small So, why does the blue color exist in sky? The answer is that it's a result of the way the earth reflects and scats light. \\
        & \small As the sunlight passes through the clouds and reaches the Earth's surface, some of it bounces back and some is absorbed. \\
        & \small Some of this light that is reflected and absorbed is then re-emitted in a different direction, creating the beautiful blue sky we see today. \\
        \midrule 
        \textbf{BCO} & \small The color of the night sky, blue, is due to the scattering of light by molecules of atmospheric gases such as nitrogen and oxygen. \\
        & \small These gases absorb certain colors of visible light, leaving the blue color to reach us. \\
        & \small The color we see is not the color that the gases themselves emit, but rather the colors they absorb, which is what we perceive as blue. \\
        \midrule
        \textbf{KLDO} & \small The color blue is associated with the atmosphere, which is a mixture of gases such as nitrogen, oxygen, and water vapor. \\
        & \small These gases absorb certain wavelengths of light, making the air around us appear blue. \\
        & \small However, the exact reason why the color of the skies is typically blue varies depending on where you are in the world. \\
        & \small In some parts of Europe and North America, for example, where there is more urban development and pollution, blue skies may not be as common. \\
        & \small Instead, they may be more likely to appear green or brown due to the presence of industrial emissions. \\
        & \small Ultimately, it's a result of how the light is interacting with different elements in our atmosphere and how it reflects off the surface of objects, such the clouds and the ground.\\
        \bottomrule
    \end{tabular}
}
    \caption{Responses from differently aligned Llama3.2-1B models to the sample prompt 'Why is the sky blue?' from the Alpaca Eval dataset, compared to the base pre-trained Llama3.2-1B model.}
    \label{tab:sample_alpaca}
\end{table}

% {\small Stress can be a natural and normal part of life, but it's important to recognize when it starts feeling overwhelming and take steps to manage it effectively.} \\
% & {\small Here are some tips that may help you manage stress:} \\
% & {\small 1. Get enough sleep: Aim for 7-9 hours of quality sleep each night, as it has been shown to have a positive impact on mental health and overall well-being. }\\
% & {\small 2. Exercise: Engaging in regular physical activity can help reduce stress and improve overall health, so consider incorporating exercise into your routine.}\\
% & {\small 3. Deep breathing: Taking deep breaths can relax the body and mind, which can also help with managing stress.} \\
% & {\small \quad Try taking slow, deep inhales and exhales, focusing on the breath and letting go of any negative thoughts or emotions.}\\
% & {\small 4. Meditate: Meditation can have numerous benefits for mental and physical health.} \\
% & {\small \quad Practicing mindfulness and meditation regularly can reduce feelings of stress, anxiety, and depression.}\\
% & {\small 5. Talk to someone: If you're struggling with emotional or mental distress, it can often be helpful to reach out to family, friends, or a mental healthcare professional for support. } \\
% & {\small \quad They can offer guidance and support, helping you navigate through any challenges you may be facing.}\\
% & {\small Remember, stress is a normal and natural part life. It's essential to take care of yourself and find healthy ways of managing it, }\\
% & {\small whether it be through exercise, meditation, sleep, relaxation techniques, spending time with loved ones, etc. Don't hesitate to seek help if you feel like you need it.}\\