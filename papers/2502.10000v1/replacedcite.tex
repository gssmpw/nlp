\section{Related Work}
\label{sec:rw}

%\work{other work on heterogeneous architectures (performance and energy)}{Adrien}

Our main interest lies in the problem of throughput optimization for partially-replicable task chains.
We focus on solutions using pipeline and replicated parallelism, and interval mapping____.
As we are unaware of any solutions to our specific research problem in the state of the art, we will focus our discussion here on variations of this problem.

\textbf{Throughput on homogeneous architectures:}
OTAC____ provides an optimal solution for partially-replicable task chains using pipeline and replicated parallelism.
We provide more details about OTAC in Section~\ref{sec:heuristics}, as our two greedy heuristics are based on its main ideas.
OTAC itself is inspired by Nicol’s algorithm____, which is an optimal solution for the Chain-to-chains partitioning (CCP) problem where only pipelining is possible.
Finally, when all tasks are replicable, the optimal solution in homogeneous resources is to build a pipeline with a single stage that is replicated across all resources____.
Nonetheless, this does not apply for heterogeneous architectures.

\textbf{Throughput on heterogeneous architectures:}
Benoit and Robert offered three heuristics for building interval mappings on unrelated heterogeneous architectures____.
Among them, BSL and BSC use a combination of binary search and greedy allocation, which is similar to the general scheme of OTAC and our proposed heuristics.
These heuristics, however, do not consider replicated parallelism.

\textbf{Makespan on heterogeneous architectures:} 
Topcuoglu et al.____ introduced HEFT (one of the most used heuristics for this kind of problem) and the CPOP to schedule directed acyclic graphs (DAGs) over unrelated heterogeneous resources.
Eyraud-Dubois and Krumar____ proposed HeteroPrioDep to schedule DAGs over two types of unrelated resources. %using a spoliation mechanism to stop tasks when their preferred resources become available.
Sadly, none of these strategies applies for throughput optimization, nor for pipeline and replicated parallelism.
Agullo et al.____ studied the performance of dynamic schedulers on two types of unrelated resources through simulation and real-world experiments.
We also employ both kinds of experiments in our evaluation, but dynamic schedulers from current runtime systems are usually inefficient at our task granularity of interest (tens to thousands of $\mu$s)____.
Benavides et al.____ proposed a heuristic for the flow shop scheduling problem on unrelated resources, but their solution is not easily transposable for pipeline and replicated parallelism.

\textbf{SDR on heterogeneous architectures:} Mack et al.____ proposed the use of the CEDR heterogeneous runtime system to encapsulate and enable GNU~Radio’s signal processing blocks (tasks) in FPGA and GPU-based systems on chip.
They use dynamic scheduling heuristics and imitation learning to co-schedule GNU~Radio’s blocks with other applications.
In contrast, our approaches build static pipeline decompositions and schedules for a lower runtime overhead.
We believe our algorithms can be integrated to GNU~Radio in its future version (4.0)____ when it will abandon its thread-per-block schedule, enabling better performance by avoiding its current issues related to locality and OS scheduling policies____.