\section{Related Works}
We have assessed multiple previous research efforts related to machine learning-based supervised, semi supervised, and unsupervised algorithms relevant to time series analysis. We conducted a thorough analysis to identify the shortcomings of the current system. This research improved the classification framework we developed for brain cancers.
\begin{itemize}
\item Pendela Kanchanamala et al. \cite{b9} utilized MRI to developed an optimization-enhanced hybrid deep learning model for classification and detection of brain tumors.

\item Emrah Irmak \cite{b10} attained an accuracy of 92.66\% utilizing a bespoke CNN model for classifying normal, meningioma, pituitary, glioma and metastatic brain tumors.

\item For classification of brain cancer Ayadi et al. \cite{b11} proposed a CNN-based computer assisted diagnosis (CAD) method. Three separate datasets were used to conducted the experiment using 18-weighted layered CNN model. Where they achieved 94.74\% classification accuracy for brain tumor type classification and for tumor grading, they achieved 90.35\%. 

\item Khan et al. \cite{b12} (2020) introduced a deep learning approach for the classification of brain cancers as malignant or benign, utilizing 253 genuine brain MRI scans supplemented with data augmentation techniques. Edge detection was employed to define the region of interest in the MRI image before feature extraction using a basic CNN model. The achieved categorization accuracy was 89\%.

\item The potential of deep learning techniques for glioma classification by MR imaging is examined by Banerjee et al. \cite{b13}. For 2D images the researchers assessed the effectiveness of transfer learning employing VGGNet and ResNEt architectures, attaining accuracies of 84\% and 90\%. 

\item In a distinct study \cite{b14}, researchers proposed two methodologies for glioma grading, which involved segmentation utilizing a customized U-Net model. A regional convolutional neural network (R-CNN) was employed for the classification job in each two-dimensional image slice of the MRIs. Their proposed 2D Mask R-CNN achieved an accuracy of 96\%.Data augmentation enhanced the outcomes, as evidenced by the classification efficacy of the 2D model.

\item In the research conducted by A. M. Dikande Simo, two models were proposed, trained utilizing the Brain Tumor MRI Dataset \cite{b27}. Four optimizers were evaluated across three classification tasks, with Adam demonstrating superior performance in differentiating tumor from non-tumor brains.
Where they got 100\% training accuracy and 98\% validation and test accuracy.

\item In the research of \cite{b28} A. Nag et al. introduces TumorGANet, a sophisticated model that integrates ResNet50 and Generative Adversarial Networks (GANs) for the classification of brain tumors. The model demonstrates exceptional accuracy of 99.53\% and achieves precision and recall rates of 100\%, supported by Explainable AI methodologies such as LIME. Nevertheless, the dependence on a particular dataset and the restricted examination of real-world clinical variability may limit its generalizability.

\item In the research conducted by  A. Rath, B. S. P. Mishra, and D. K. Bagal,\cite{b29} they uesd pretrained ResNet50 model to improve  accuracy and efficency utilizing a balanced dataset of 2,577 MRI images having binary class of tumors and healthy instances of patients.

\end{itemize}
A particular application, certain machine learning models demonstrate superior efficacy compared to others. However, the effectiveness of these models in classifying cardiovascular illnesses has not yet reached parity. Further advancement is required to enhance the existing work. Our motivation is to contribute to the field using deep learning methods to improve the performance and efficiency of detection and classification.