\documentclass[a4wide,11pt,reqno]{article}
\usepackage{bm}
\usepackage{mathrsfs}
%\usepackage[nottoc]{tocbibind} % makes the bibliography apper in the table of contents
\usepackage{fullpage}
\usepackage{enumerate}
\usepackage[utf8]{inputenc}
\textwidth 430pt \textheight 600pt
\textheight=8.9in \textwidth=6.2in \oddsidemargin=0.25cm
\evensidemargin=0.25cm \topmargin=-.5cm
\renewcommand{\baselinestretch}{1}
\usepackage{amssymb, amsmath, amsfonts}
\usepackage{mathrsfs}
\usepackage{epsfig,amsbsy,amsthm} % "amsbsy" produce bold math symbol
\usepackage{graphics}
\usepackage{caption}
%\usepackage{report}
%\usepackage[fleqn]{amsmath}
%\usepackage{color}
\usepackage{subcaption}
\usepackage{float,epsfig}
\usepackage{fixmath}
\usepackage{graphics}
\usepackage{pgfpages}
\usepackage{appendix}
\numberwithin{equation}{section}  %It gives the eq. no in fraction format.
\usepackage{hyperref} % this gives link
\usepackage{graphicx}% the above two gives link
\usepackage{pst-all}
\usepackage{calligra}
%\usepackage{color}
\usepackage{tikz}
\usepackage{calligra}
\usepackage{color}
\usepackage{tikz}
\usepackage{cleveref}
\usepackage[utf8]{inputenc}
\usepackage[autostyle]{csquotes}
%%%%%%%%%%%%%%%%%%%%%%%%%%%%%%%
\usepackage[section]{algorithm}
\usepackage{multirow}
\usepackage{array}
\newcolumntype{H}{>{\setbox0=\hbox\bgroup}c<{\egroup}@{}}
%\pagestyle{empty}
%%%%%%%%%%%%%%%%%%%%%%%%%%%%%%%%%%%%%
\DeclareMathAlphabet{\mathpzc}{OT1}{pzc}{m}{it}
\DeclareMathAlphabet{\mathcalligra}{T1}{calligra}{m}{n}
\DeclareMathAlphabet{\mathpzc}{OT1}{pzc}{m}{it}
\DeclareMathAlphabet{\mathcalligra}{T1}{calligra}{m}{n}
\DeclareMathAlphabet{\mathpzc}{OT1}{pzc}{m}{it}
\DeclareMathAlphabet{\mathcalligra}{T1}{calligra}{m}{n}
%\usepackage[T4]{fontenc}
%%%%%%%%%%%%%%%%%%%%%%%%%%%%%%%%%%%%%%%%%%
\begin{document}
	\newtheorem{theorem}{\bf Theorem}[section]
	\newtheorem{proposition}[theorem]{\bf Proposition}
	\newtheorem{definition}{\bf Definition}[section]
	\newtheorem{corollary}[theorem]{\bf Corollary}
	\newtheorem{exam}[theorem]{\bf exam}
	\newtheorem{remark}[theorem]{\bf Remark}
	%\newtheorem{remark}[theorem]{Remark}  % SIAM format
	\newtheorem{lemma}[theorem]{\bf Lemma}
	\newtheorem{assum}[theorem]{\bf Assumption}
	%%%%%%%%%%%%%%%%%%%%%%%%%%%%%%%%%%%%%%%%%%%%%
	\newcommand{\von}{\vskip 1ex}
	\newcommand{\vone}{\vskip 2ex}
	\newcommand{\vtwo}{\vskip 4ex}
	\newcommand{\ds}{\displaystyle}
	%\def \bmathcal{P}{\bm{\mathcal{P}}}
	%\def \brho{\bm{\rho}}
	\def \noin{\noindent}
	%%%%%%%%%%%%%%%%%%%%%%%%%%%%%%%%%%%%%%%%%%%%%%
	\newcommand{\be}{\begin{equation}}
		\newcommand{\ee}{\end{equation}}
	\newcommand{\beno}{\begin{equation*}}
		\newcommand{\eeno}{\end{equation*}}
	\newcommand{\ba}{\begin{align}}
		\newcommand{\ea}{\end{align}}
	\newcommand{\bano}{\begin{align}}
		\newcommand{\eano}{\end{align}}
	\newcommand{\bea}{\begin{eqnarray}}
		\newcommand{\eea}{\end{eqnarray}}
	\newcommand{\beano}{\begin{eqnarray*}}
		\newcommand{\eeano}{\end{eqnarray*}}
	%\def\bmatrix#1{\left[ \begin{matrix} #1 \end{matrix} \right]}
	\def \noin{\noindent}
	\def\arraystretch{1.3}
	%%%%%%%%%%%%%%%%%%%%%%%%%%% new commands %%%%%%%%%
	%%%%%%%%%%%%%%%%%%%%%%%%%%%%%%%%%%%%%%%%%%%%%%%%%%%
	\def \tcK{{\tilde {\mathcal K}}}    
	\def \O{{\Omega}}
	\def \cT{{\mathcal T}}
	\def \cV{{\mathcal V}}
	\def \cE{{\mathcal E}}
	\def \R{{\mathbb R}}
	\def \V{{\mathbb V}}
	\def \S{{\mathbb S}}
	\def \N{{\mathbb N}}
	\def \Z{{\mathbb Z}}
	\def \Mc{{\mathcal M}}
	\def \Cc{{\mathcal C}}
	\def \Rc{{\mathcal R}}
	\def \Ec{{\mathcal E}}
	\def \Gc{{\mathcal G}}
	\def \Tc{{\mathcal T}}
	\def \Qc{{\mathcal Q}}
	\def \Ic{{\mathcal I}}
	\def \Pc{{\mathcal P}}
	\def \Oc{{\mathcal O}}
	\def \Uc{{\mathcal U}}
	\def \Yc{{\mathcal Y}}
	\def \Ac{{\mathcal A}}
	\def \Bc{{\mathcal B}}
	\def \k{\mathpzc{k}}
	%\def \z{\mathpzc{z}}
	%\def \Zf{\mathfrak{z}}
	\def \Rp{\mathpzc{R}}
	%\def \Jp{\mathpzc{J}}
	\def \Os{\mathscr{O}}
	\def \Js{\mathscr{J}}
	\def \Es{\mathscr{E}}
	\def \Qs{\mathscr{Q}}
	\def \Ss{\mathscr{S}}
	\def \Cs{\mathscr{C}}
	\def \Ds{\mathscr{D}}
	\def \Ms{\mathscr{M}}
	\def \Ts{\mathscr{T}}
	\def \LL{L^{\infty}(L^{2}(\Omega))}
	\def \LH{L^{2}(0,T;H^{1}(\Omega))}
	\def \B {\matheorem{BDF}}
	\def \el {\matheorem{el}}
	\def \re {\matheorem{re}}
	\def \e {\matheorem{e}}
	\def \div {\matheorem{div}}
	\def \CN {\matheorem{CN}}
	\def \Rs   {\mathbf{R}_{{\matheorem es}}}
	\def \Rb {\mathbf{R}}
	\def \Jb {\mathbf{J}}
	\def  \apos {\emph{a posteriori~}}
	
	\def\mean#1{\left\{\hskip -5pt\left\{#1\right\}\hskip -5pt\right\}}
	\def\jump#1{\left[\hskip -3.5pt\left[#1\right]\hskip -3.5pt\right]}
	\def\smean#1{\{\hskip -3pt\{#1\}\hskip -3pt\}}
	\def\sjump#1{[\hskip -1.5pt[#1]\hskip -1.5pt]}
	\def\jumptwo{\jump{\frac{\p^2 u_h}{\p n^2}}}
%%%%%%%%%%%%%%%--BODY--%%%%%%%%%%%%%%%%%%
%%-----------------------------
%%      the top matter
%%-----------------------------
\title{Adaptive SIPG method for approximations of boundary control problems governed by parabolic PDEs}
\author{Ram Manohar\thanks{Department of Mathematics \& Statistics, Indian Institute of Technology Kanpur, Kanpur - 208016, India \tt{(rmanohar267@gmail.com)}.}, ~~Kedarnath Buda\thanks{Department of Mathematics \& Statistics, Indian Institute of Technology Kanpur, Kanpur - 208016, India   \tt{Kedarnath.buda@gmail.com}.}, ~~B. V. Rathish Kumar\thanks{Department of Mathematics \& Statistics, Indian Institute of Technology Kanpur, Kanpur - 208016, India   \tt{(drbvrk11@gmail.com)}.}, 
	~~ and ~~Rajen Kumar Sinha\thanks{Department of Mathematics, Indian Institute of Technology Guwahati, Guwahati - 781039, India \tt{(rajen@iitg.ac.in)}.}}

\date{}

\maketitle
\textbf{ Abstract.}{\small{  This study presents an aposteriori error analysis of adaptive finite element approximations of parabolic boundary control problems with bilateral box constraints that act on a Neumann boundary.  The control problem is discretized using the symmetric interior penalty Galerkin (SIPG) technique. We derive both reliable and efficient type residual-based error estimators coupling with the data oscillations.  The implementation of these error estimators  serves as a guide for the adaptive mesh refinement process, indicating whether or not more refinement is required. Although the control error estimator effectively captured control approximation errors, it had limitations in guiding refinement localization in critical cases. To overcome this, an alternative control indicator was used in numerical tests. The results demonstrated the clear superiority of adaptive refinements over uniform refinements, confirming the proposed approach's effectiveness in achieving accurate solutions while optimizing computational efficiency.  numerical experiment showcases the effectiveness of the derived error estimators.  \\ 

\textbf{Key words.} Aposteriori error analysis, parabolic boundary control problems, adaptive finite element methods, symmetric interior penalty Galerkin technique. 


\vspace{.1in}
\textbf{AMS subject classifications.} $65{N}30$, $65{N}50$, $49J20$, $65K10$.


\section{Introduction}
Let $\Gamma:=\partial \Omega$ be the boundary  of a bounded polygonal domain $\Omega \subset \mathbb{R}^{2}$ such that $\Gamma=\overline{\Gamma_{D}} \cup \overline{\Gamma_{N}}$ with $\Gamma_{D} \cap \Gamma_{N}=\emptyset$. For adaptive finite element approximation,  we take into the account of following control-constrained problem:
\begin{equation}
\underset{q \in \mathpzc{Q}_{Nad}}{\min} \int_0^T\Big\{\frac{1}{2}\int_{\O}(y-y_{d})^2\, dx+\frac{\alpha}{2} \int_{\Gamma_{N}}(q-q_{d})^2\,ds+\int_{\Gamma_{N}} r_{_N}\,y\, ds\Big\}dt \label{contfunc}
\end{equation}
subject to the conditions
\begin{subequations}
	\begin{eqnarray}
		\frac{\partial y}{\partial t}- \Delta y+a_0 y &=&f \quad \;\;\;\;\;\;\;\;\;\text {in } \Omega\times (0,T], \label{contstate}\\
		y &=&g_{_D} \quad \;\;\;\;\;\; \text { on } \Gamma_{_D}\times (0,T], \label{diricon}\\
		\frac{\partial y}{\partial n} &=&q+g_{_N} \quad  \text { on } \Gamma_{_N}\times (0,T], \label{neumanncon}\\
		y(x,0) &=& y_0(x) \quad \;\;\; \text{ in } \Omega, \label{incon}
	\end{eqnarray}
\end{subequations}
where $\mathpzc{Q}_{Nad}$ is a closed convex set with control constraints defined by
\begin{equation}
\mathpzc{Q}_{Nad}:=\big\{q \in L^2(I;L^{2}(\Gamma_{_N})): q_{a} \leq q(x,t) \leq q_{b} \;\; \text{ a.e. } x \in \Gamma_{_N} \;\; \text{and} \;\; \forall t\in I \big\}. \label{contspace}
\end{equation}
Here $I=[0,T]$, $q_{a},\, q_{b} \in L^{\infty}(I;L^{\infty}\left(\Gamma_{_N}\right))$ with $q_{a} \leq q_{b}$ for almost all $(x,t) \in \Gamma_{_N} \times I$. The desired control function $q_d$ serves as a reference for the control (see, \cite{clever2010, hoppeiliash2008}). Keep in mind that the aforementioned method also works for the most prevalent and specific situation, $q_{d}=0$, this means that there is no prior knowledge about the ideal control. Furthermore, the coefficient $r_{_N}$ in \eqref{contfunc} is added explicitly to the cost functional which ensures that the boundary aspect of any given adjoint-state function are achieved (see, Eq. \eqref{2.17nuemanncod-adj}). With similar cost functionals, we are delighted to recommend \cite{casasdhamo2012,hinzematthes2009,kohlssiebert2014}, and indeed the citations thereof. We choose the given functions and parameters in the following spaces to illustrate the well-posedness of the minimization problem \eqref{contfunc}--\eqref{contspace}:
 \begin{equation}
	f,\, y_{d} \in L^2(I;L^{2}(\Omega)),\, g_{_D} \in L^2(I;H^{1 / 2}\left(\Gamma_{_D}\right)),\, q_{d},\, g_{_N},\, r_{_N} \in L^2(I;L^{2}\left(\Gamma_{_N}\right)),\, \alpha>0,\, a_0 \in L^{\infty}(\Omega), \label{assump1}
\end{equation}
and there exists a positive constant $c$ such that
\begin{equation}
	a_0 \geq c > 0 \quad \text { a.e. in } \Omega. \label{assum2}
\end{equation}
Optimization problems governed by partial differential equations (PDEs) arise in various real-life scenarios, including the shape optimization of technological devices \cite{pironneau2001}, parameter identification in environmental processes, and addressing control flow problems \cite{fernando2001, perotto2008,  skiba2006}. These problems are inherently complex, demanding meticulous attention to secure efficient numerical approximations. Among all the methods, the adaptive finite element method (AFEM) stands out as particularly effective. The AFEM operates through the successive iterations of a sequence comprising the following steps: 
$$
\texttt{ Solve} \rightarrow \texttt{ Estimate } \rightarrow \texttt{ Mark } \rightarrow \texttt{ Refine}.$$
In the \texttt {Solve} phase, the optimization problem is numerically tackled within a finite-dimensional space defined by a given mesh. But the crucial component for the procedure can be determined in the \texttt{Estimate} stage. Here, local error indicators are computed based on discrete solutions without knowing for the exact solution. These indications are essential in designing mesh adaption algorithms that optimize computations and convey computational cost equitably. Subsequently, the \texttt{Marking} step utilizes the information gleaned from these indicators to select a subset of elements for refinement. Afterwards, the \texttt{Refine} step executes the refinement process within the adaptive loop.

Although the AFEM, which made significant contributions to the pioneering work of Babuška and Rheinboldt \cite{rheinboldt1978}, has long been recognized as a popular approach for efficiently solving initial and boundary value problems governed by partial differential equations (PDEs). It's application to constrained optimal control problems (OCPs) has only recently gained popularity due to the contribution of Liu and Yan \cite{liuyan2001} and Becker, Kapp, and Rannacher \cite{becker2000}. In their seminal work \cite{liuyan2001}, Liu and Yan proposed a residual-type aposteriori error estimator tailored for OCPs. Concurrently, Becker, Kapp, and Rannacher \cite{becker2000} introduced a dual-weighted goal-oriented adaptivity strategy. For a comprehensive understanding of these methodologies, we refer to \cite{hoppeiliash06, kieweghoppe2010, hoppeiliash2008, zhou2009, matang2002, yanzhou2009, yucilbenner2015, arasozen2015, yanyu2014, rmrk2021, manohar2022local, manohar2024error} for residual-type estimators, and to \cite{benedix2009, gunthetr2008, mhoppe2010, wollner2010} for insights into the dual-weighted goal-oriented approach, along with references therein for further exploration of recent advancements. Moreover, ensuring the theoretical success of aposteriori error estimators has prompted endeavors such as those undertaken in \cite{spomao2011, gaevskaya2007, gongyyan2007, siebertrosch2014} to establish the convergence of the AFEM for OCPs. These theoretical analyses serve to underpin the practical applicability and reliability of the AFEM in this domain.

Adaptive mesh refinement presents an attractive avenue for tackling OCPs, especially those characterized by layers or singularities within certain regions of the mesh. This adaptivity enables local refinement around these layers as necessary, effectively achieving the desired residual bound with minimal degrees of freedom. The aposteriori error analysis of OCPs is covered in a large amount of literature, although distributed OCPs are the main emphasis. It is a domain that is well explored in works like \cite{becker2000,mhoppe2010,kieweghoppe2010,hoppeiliash2008, zhou2009,matang2002, Pratibha2024,shakya17, shakya19, wollner2010,yanzhou2009,yucilbenner2015,arasozen2015,yanyu2014}.  However, numerical solutions for boundary OCPs have not received adequate consideration. Existing studies primarily investigate residual type error estimators \cite{kieweghoppe2006,hoppeiliash06,kohlssiebert2014,liuyan09,liuyan2001}, with some delving into hierarchical type estimators \cite{kohlssiebert2014}, all primarily utilizing continuous finite element discretizations, except for \cite{kohlssiebert2014}.
In the latter, discontinuous finite elements for control discretization are used for the first time by Kohls, Rösch, and Siebert. In particular,cular, research by Leykekhman \cite{leykekhman2012} suggests that discontinuous Galerkin methods offer superior convergence behavior for OCPs featuring boundary layers. Optimal convergence orders are achievable when errors are computed away from the boundary or interior layers. Moreover, employing discontinuous finite elements for control discretization on the Neumann boundary facilitates more efficient projection operator computation as elucidated in \cite[Sec. 4.1]{kohlssiebert2014}.

Discontinuous Galerkin methods possess several inherent advantages over alternative finite element methods. 
They have the ability to handle curved borders and inhomogeneous boundary conditions with convenience, create state and test spaces with simplicity, manage nonmatching grids with simplicity, and develop $hp$-adaptive grid improvements. Although these methods have long been used, interest in them has increased recently due to the availability of low-cost processors. For further insights into discontinuous Galerkin methods, the reader may refer to seminal works such as \cite{arnoldbrezzi2002,warburton2008,kanschat2009,kanschat2007,pascal2007,bevere2014}. Despite their applicability, discontinuous Galerkin methods have mostly been explored in the context of distributed OCPs \cite{leykekhman2012,yucilbenner2015,arasozen2015,yanyu2014}. To the best of our knowledge, their application to boundary OCPs has not yet been studied in any existing study, indicating a promising topic for future research and development.

In this study, we focus on deriving reliable and efficient aposteriori error estimators tailored specifically for boundary OCPs governed by elliptic equations. These problems are discretized using the symmetric interior penalty Galerkin (SIPG) method, chosen for its advantageous symmetric property. This property ensures that discretization and optimization operations can be seamlessly interchanged, enhancing computational efficiency. Our aposteriori error analysis of the boundary control problem (BCP) encompasses a comprehensive assessment of errors in the state, adjoint-state, control, and co-control variables. Additionally, we meticulously account for data oscillations to encompass the problem's data, including coefficients of the equations, right-hand side terms, and boundary conditions, in the most general setting possible.
It is worth noting that while considerations of data oscillations have been addressed in prior works such as \cite{aniswarth2007,mnsiebert} for single state equations and in \cite{hoppeiliash06,kieweghoppe2010} for OCPs, our paper extends this analysis to the context of  boundary OCPs governed by parabolic PDEs. By incorporating these considerations, our aim is to provide a more comprehensive and understanding of the error dynamics inherent in such problems. 

The subsequent sections of this paper unfold as follows. Section \ref{section222}: 
Here, we introduce the weak formulation of the Neumann BCP governed by a  parabolic PDE featuring bilateral constraints on the control variable. We provide the optimality conditions, articulating them in terms of the state, adjoint-state, control, and co-control variables, with reference to the Lagrangian multiplier associated with the control. Section \ref{section333}: This section delineates the Symmetric Interior Penalty Galerkin (SIPG) discretization methodology applied to the boundary OCP. We expound upon the discretization process, elucidating the numerical techniques employed to address the problem effectively.
Section \ref{sec4errest}: In this segment, we provide a comprehensive discussion on the aposteriori error estimators utilized in our analysis. Specifically, we utilize a residual-type error estimator to evaluate global discretization errors across all variables, encompassing edge and element residuals. Additionally, we incorporate considerations of data oscillations into the error analysis. Moreover, we derive local upper and lower aposteriori error estimates tailored to the BCP. Section \ref{section5555} presents numerical findings aimed as illustrating the performance and efficacy of our proposed adaptive mesh refinement strategy. Through empirical results and analyses, we demonstrate the accuracy and efficiency of our methodology in practical scenarios. A concluding remark is presented at the end.



\section{Weak representation of the OCP} \label{section222}
We adopt the standard notation from Lebesgue and Sobolev space theory, as presented in \cite{adams1975}. For $ m \in \mathbb{N}$, we denote the inner product, seminorm and norm on $H^{m}(\Omega)$ by $(\cdot, \cdot)_{m, \Omega}$, $|\cdot|_{m, \Omega}\; \text{and}\; \|\cdot\|_{m, \Omega}$, respectively. At this point, we specify the state and test function spaces by
$$
\mathpzc{Y}=\big\{ v \in L^2(I; H^{1}(\Omega))\cap H^1(I;L^2(\Omega)):\;\; v|_{\Gamma_{_D}}=g_{_D}\big\}, $$
and  $$ \mathpzc{V}=\big\{ \phi \in L^2(I;H^{1}(\Omega)):\;\;\phi|_{\Gamma_{_D}}=0 \big\},
$$
respectively. The bilinear form  $a(\cdot,\cdot):~\mathpzc{Y}\times \mathpzc{V} \mapsto \mathbb{R}$ is given by, for $y\in \mathpzc{Y}$,
$$
a(y, \phi)=\int_{\Omega}(\nabla y \cdot \nabla \phi+a_0 y \phi) dx, \;\;\forall \; \phi \in \mathpzc{V}.
$$
This leads to the following abstract formulation of the problem  \eqref{contstate}--\eqref{incon}. For each $t\in [0,T]$, to find $y \in \mathpzc{Y}$ such that
$$
\big(\frac{\partial y}{\partial t}, \phi \big)+a(y, \phi)=(f, \phi)+\left(q+g_{N}, v\right)_{L^2(\Gamma_{_N})} \quad \forall \phi \in\mathpzc{V}.
$$
 It is well known \cite{fatt1999,lions1971,liuyan2001} that the BCP \eqref{contfunc}--\eqref{contspace} with the assumption \eqref{assump1}--\eqref{assum2} has a unique solution $(y, q) \in \mathpzc{Y} \times \mathpzc{Q}_{Nad}$ iff there is an adjoint-state variable $z \in \mathpzc{V}$, $t\in [0,T]$, satisfying
 \begin{subequations}
\begin{eqnarray}
\big(\frac{\partial y}{\partial t}, \phi \big)+a(y, \phi)&=&(f, \phi)+\left(q+g_{_N}, \phi \right)_{L^2(\Gamma_{_N})}  \quad \forall \phi \in \mathpzc{V},\label{weakformstate} \\
y(x,0)&=&y_0(x),\\
-\big(\frac{\partial z}{\partial t}, \psi \big)+a(\psi, z)&=&\left(y-y_{d}, \psi\right)+\left(r_{_N}, \psi \right)_{L^2(\Gamma_{_N})} \quad \forall \psi \in \mathpzc{V}, \label{weakformadjoint-state}\\
z(x,T)&=&0,\\
\left(\alpha \left(q-q_{d}\right)+z, \varphi-q\right)_{L^2(\Gamma_{_N})} &\geq& 0 \quad \forall \varphi \in \mathpzc{Q}_{Nad}, \label{weakformcontrol}
\end{eqnarray}
\end{subequations}
where the adjoint-state variable $z$ is determined by the following system
\begin{subequations}
\begin{eqnarray}
-\frac{\partial z}{\partial t}-\Delta z+a_0 z &=&y-y_{d} \quad \text { in } \Omega \times [0,T), \label{2.14adjoint-state}\\
z(\cdot, T) &=&0 \quad \text { in } \Omega,\\
z &=&0 \quad \text { on } \Gamma_{_D}\times [0,T), \\
\frac{\partial z}{\partial n} &=&r_{_N} \quad \text { on } \Gamma_{_N}\times [0,T).\label{2.17nuemanncod-adj}
\end{eqnarray}
\end{subequations}
Introducing a Lagrange multiplier $\mu \in L^2(I; L^{2}\left(\Gamma_{_N}\right))$ to enforce the control constraints, for $\gamma>0$, the optimality system \eqref{weakformstate}--\eqref{weakformcontrol} is given by

 \begin{subequations}
\begin{eqnarray}
&&~~~\big(\frac{\partial y}{\partial t}, \phi \big)+a(y, \phi)~=~(f, \phi)+\left(q+g_{_N}, \phi \right)_{L^2(\Gamma_{_N})}  \quad \forall \phi \in \mathpzc{V},\label{weakstate1} \\
&&~~~~~~~~~~~~~~~~~y(x,0)~=~y_0(x),\\
&&-\big(\frac{\partial z}{\partial t}, \psi \big)+a(\psi, z)~=~\left(y-y_{d}, \psi\right)+\left(r_{_N}, \psi \right)_{L^2(\Gamma_{_N})} \quad \forall \psi \in \mathpzc{V}, \label{weakadjoint-state1}\\
&&~~~~~~~~~~~~~~~~~z(x,T)~=~0,\\
&&~~~ \mu+\alpha\left(q-q_{d}\right)+z~=~0 \quad \text { a.e. in }\; \Gamma_{_N},\label{optmallitycon1} \\
&&\mu-\max \left\{0, \mu+\gamma\left(q-q_{b}\right)\right\}+\min \left\{0, \mu-\gamma\left(q_{a}-q\right)\right\}~=~0 \quad \text { a.e. in }\; \Gamma_{_N}.\label{projminmax1}
\end{eqnarray}
\end{subequations}
We now present an equivalent representation of \eqref{projminmax1} \eqref{projminmax1} analogous to the pointwise complementarity system with $\mu=\mu_{b}-\mu_{a}$:
\begin{eqnarray}
 & \mu_{b} \geq 0, \quad q-q_{b} \leq 0, \quad \mu_{b}\left(q-q_{b}\right)=0,\label{compcond1} \\
 & \mu_{a} \geq 0, \quad q_{a}-q \leq 0, \quad \mu_{a}\left(q_{a}-q\right)=0. \label{compcond2}
\end{eqnarray}
It is  found  that \eqref{weakstate1}--\eqref{projminmax1} possesses Newton differentiability   at least for the $\gamma=\alpha$ (cf., \cite{itokunisch2002}). 
To tackle the nonsmoothness in \eqref{projminmax1}, we employ a semismooth Newton iteration, which can be effectively combined with an active set strategy. This approach enables us to identify the active sets at each Newton iteration step by
\begin{eqnarray}
&\mathpzc{A}_{a}=\left\{x \in \Gamma_{_N}: \mu-\gamma\left(q_{a}-q\right)<0\right\}, \label{activeseta}\\
&\mathpzc{A}_{b}=\left\{x \in \Gamma_{_N}: \mu+\gamma\left(q-q_{b}\right)>0\right\}, \label{activesetb}
\end{eqnarray}
and inactive set is given by $\mathcal{I}=\Gamma_{N} \backslash\left\{\mathpzc{A}_{a} \cup \mathpzc{A}_{b}\right\}$. The complementarity conditions in \eqref{compcond1}--\eqref{compcond2} can be equivalently expressed as
\begin{subequations}\label{2.28contcompcond}
\begin{align}
&q=q_{a}, \quad \mu_{b}=0, \quad \mu \leq 0 \quad \text { a.e. on }\; \mathpzc{A}_{a}, \label{compcon11}\\
&q=q_{b}, \quad \mu_{a}=0, \quad \mu \geq 0 \quad \text { a.e. on }\; \mathpzc{A}_{b}, \label{compcon12}\\
&q_{a}<q<q_{b}, \quad \mu_{a}=\mu_{b}=0, \quad \mu=0 \quad \text { a.e. on }\; \mathcal{I},\label{compcon13}
\end{align}
\end{subequations}
%%%%%%%%%%%%%%%%%%%%%%%%%%%
\section{Finite dimensional setting of the PBCP} \label{section333}
We employ the Symmetric Interior Penalty Galerkin (SIPG) method to discretize the minimization problem \eqref{contfunc}-\eqref{contspace}. \\

\noindent
\textbf{(a) \tt  Computational-domain discretization.} Let $\mathscr{T}_{h}$ be a shape-regular simplicial triangulation of $\bar{\Omega}$, where the triangle boundaries align with the boundary
$\Gamma$. The triangulation satisfies, for the triangles $K_i,\, K_j \in \mathscr{T}_{h},\, i \neq j$, then $K_i \cap K_j$ is either empty or a vertex or an edge. Let us introduce the following notions which is crucial for the subsequent analysis.
\begin{enumerate}
\item $\mathcal{E}_{0,h}:$ the set of all interior edges
\item $\mathcal{E}_{B,h}:$ the set of all boundary edges, which is further decomposed into
\begin{itemize}
\item[i.]  $\mathcal{E}_{D,h}:$ the set of Dirichlet boundary edges;
\item[ii.] $\mathcal{E}_{N,h}:$ the set of Neumann boundary edges, respectively.
\end{itemize}
\item $\mathcal{E}_{h}=\mathcal{E}_{0, h} \cup \mathcal{E}_{B,h}:$ the set of all edges,
\item For each element $K\in \mathscr{T}_h$ and each edge $\mathcal{E}_{h}$, we define:
\begin{itemize}
    \item[i.] $h_K=diam(K)$: the diameter of element $K$;
    \item[ii.] $h_E=length(E)$: the length of edge $E$;
    \item[iii.] $h=max\{h_K: K\in \mathscr{T}_h\}$: the maximum element diameter
\end{itemize}
\end{enumerate}
Consider an interior edge $E\in \mathcal{E}_{0, h}$ shared by two elements $K$ and $K^e$ in $\mathscr{T}_{h}$ such that $E=\partial K\cap \partial K^e$. Let $\mathbf{n}_{K}$ and $\mathbf{n}_{K^{e}}$ be the unit outward normals to  $\partial K$ and $\partial K^{e}$, respectively. For a piecewise continuous scalar function $\varphi$, we denote the traces of $\varphi$ along $E$ from inside $K$ and  $K^{e}$  by$\varphi|_{E}$ and $\varphi^{e}|_{E}$, respectively. We define the mean and jump of $\varphi$ across the edge $E$ as follows
\begin{subequations}\label{jumpaveg}
\begin{align}
&\smean{\varphi} :=\frac{1}{2}(\varphi|_{E}+\varphi^{e}|_{E}), \\ 
&\sjump{\varphi} :=\varphi|_{E} \mathbf{n}_{K}+\varphi^{e}|_{E} \mathbf{n}_{K^{e}}.  
\end{align}
\end{subequations}
Analogously, for a piecewise continuous vector field $\nabla \varphi$,  we define the mean and jump across an edge $E$ by
\begin{subequations}\label{gradjumpaveg}
\begin{align}
&\smean{\nabla \varphi }:=\frac{1}{2}(\nabla \varphi|_{E}+\nabla \varphi^{e}|_{E}), \\  &\sjump{\nabla\varphi}:=\nabla \varphi|_{E} \cdot \mathbf{n}_{K}+\nabla \varphi^{e}|_{E} \cdot \mathbf{n}_{K^{e}}.
\end{align}
\end{subequations}
For a boundary edge $E \in \mathcal{E}_{B,h}$, there exists an element $K\in \mathscr{T}_h$ such that $E=\partial K \cap \Gamma$. We define the mean and jump operators on this boundary edge as $\smean{\nabla \varphi }=\nabla \varphi$ and 
$\sjump{\varphi}=\varphi \mathbf{n}$, where $\mathbf{n}$ is the unit normal vector on $E$ outside of $K$.

Next, we introduce the discrete spaces for the state, adjoint-state, and control variables, along with the corresponding test space, by
\begin{eqnarray}
&&\mathpzc{V}_{h}~=~\mathpzc{Y}_{h}=\left\{v(t) \in L^{2}(\Omega):\quad v(t)|_{K} \in \mathbb{P}_{1}(K) \quad \forall K \in \mathscr{T}_{h}, \;\; \forall t \in I \right\}, \label{femspacestate}\\
&&\mathpzc{Q}_{N,h}~=~\left\{q(t) \in L^{2}\left(\Gamma_{_N}\right):\quad q(t)|_{E} \in \mathbb{P}_{1}(E) \quad \forall E \in \mathcal{E}_{N,h}, \;\; \forall t\in I\right\},\label{femspacecontrol}
\end{eqnarray}
respectively. $\mathbb{P}_{1}(K)$ (resp., $\mathbb{P}_{1}(E)$) is the set of linear polynomials in $K$ (resp., on $E$). Note that the space $\mathpzc{Y}_{h}$ of discrete states and the space of test functions $\mathpzc{V}_{h}$ are identical due to the weak treatment of boundary conditions in discontinuous Galerkin methods. For all $(y,z,q) \in L^2(I;\mathpzc{Y}_{h}) \times L^2(I;\mathpzc{Q}_{N,h,}) \times L^2(I; \mathpzc{V}_{h})$, we define the (bi)linear forms as   
\begin{subequations}\label{3.4libili}
\begin{align}
	 a_{h}(y, \phi)&=\sum_{K \in \mathscr{T}_{h}} \int_{K}(\nabla y \cdot \nabla \phi+a_0 y \phi)\,dx \nonumber\\
	&\hspace{0.5cm}-\sum_{E \in \mathcal{E}_{0,h} \cup \mathcal{E}_{D,h}} \int_{E}( \smean{\nabla y} \cdot \sjump{\phi} +\smean{\nabla \phi} \cdot \sjump{y})\,ds\nonumber \\
	&\hspace{0.5cm}+\sum_{E \in \mathcal{E}_{0,h} \cup \mathcal{E}_{D,h}} \frac{\sigma_0}{h_{E}} \int_{E} \sjump{y} \cdot \sjump{\phi}\,ds, \label{weakform}\\
b_{h}(q, \phi)&=\sum_{E \in \mathcal{E}_{N,h}} \int_{E} q \phi\,ds, \label{weakformcon}\\
l_{h}(\phi)&=\sum_{K \in \mathscr{T}_{h}} \int_{K} f \phi d x+\sum_{E \in \mathcal{E}_{D,h}} \int_{E} g_{_D}\left(\frac{\sigma_0}{h_{E}} \mathbf{n}_{E} \cdot \sjump{\phi}-\smean{\nabla \phi}\right)\,ds \nonumber\\
	&\hspace{0.5cm} +\sum_{E \in \mathcal{E}_{N,h}} \int_{E} g_{_N} \phi\, ds,\label{linearfuc} 
\end{align}
\end{subequations}
where the penalty parameter $\sigma_0 \in \mathbb{R}_{0}^{+}$ which is independent of mesh parameter $h$. The penalty parameter $\sigma_0$ should be sufficiently large to guarantee the stability of the discontinuous Galerkin method. The discontinuous Galerkin approximation solution converges to the continuous Galerkin solution as the penalty parameter goes to infinity, for details (see,  \cite{chapman2014}).


Since, the bilinear form $a_{h}(\cdot, \cdot)$ is consistent with the state equation \eqref{contstate}-\eqref{incon} for a fixed given control $q$ in the following sense: If $y$ satisfies \eqref{contstate}-\eqref{incon}, then, for each $ t\in I$, we have
\begin{eqnarray}
\big(\frac{\partial y}{\partial t}, \phi \big)+a_{h}(y, \phi)&=&(f, \phi)_{0, \Omega}+\left(q+g_{_N}, \phi\right)_{L^2(\Gamma_{_N})}+\sum_{E \in \mathcal{E}_{D,h}}\left(y, \frac{\sigma_0}{h_{E}} \mathbf{n}_{E} \cdot \sjump{\phi}-\smean{\nabla \phi}\right)_{L^2(E)}\nonumber\\
&&+\sum_{E \in \mathcal{E}_{N,h}}\left(\mathbf{n}_{E} \cdot \nabla y, \phi\right)_{L^2(E)} \quad \forall \phi \in \mathpzc{V}_{h}, \label{abstractform}\\
y(x,0)&=&y_0(x).
\end{eqnarray}
For a fixed control $q_{h}=q$, we introduce the SIPG approximation $y_{h}$ of the solution $y$ of the state system \eqref{contstate}-\eqref{incon} such that
\begin{eqnarray}
\big(\frac{\partial y_h}{\partial t},\phi \big)+a_{h}\left(y_{h}, \phi\right)&=&l_{h}(\phi)+b_{h}\left(q_{h}, \phi\right) \quad \forall \phi \in \mathpzc{V}_{h},\label{absform}\\
y_h(x,0)&=&y_{h,0}(x),
\end{eqnarray}
where $y_{h,0}(x)$ is a suitable approximation or projection of the initial condition $y_0(x)$. This leads to the following orthogonality relation
\begin{equation}
a_{h}\left(y-y_{h}, \phi\right)=0 \quad \forall \phi \in \mathpzc{V}_{h}.\label{orthogonality}
\end{equation}
The subsequent error analysis will frequently rely on the following inverse and trace inequalities (cf.,, \cite{brenner2002, pascal2007}), which are necessary for the analysis
\begin{lemma}
Let $\mathpzc{D}$ be a bounded domain (or polygonal) with  sufficiently smooth  boundary $\Gamma_{_\mathpzc{D}}$. Then there exists the positive constant $c_{t r}$ such that the following estimates holds, for any $\chi\in H^2(\mathpzc{D})\cap H^1(\mathpzc{D})$, 
\begin{eqnarray}
&&	\|\chi\|_{0, \Gamma_{_\mathpzc{D}}} ~\leq~ c_{t r}\|\chi\|_{1, \mathpzc{D}}, \quad \forall \chi\in H^{1}(\mathpzc{D}), \label{traceinq}\\
&&	\|\chi\|_{0, \Gamma_{_\mathpzc{D}}} ~\leq~ c_{t r}\left(h_{\mathpzc{D}}^{-1}\|\chi\|_{0, \mathpzc{D}}^{2}+h_{\mathpzc{D}}\|\nabla \chi\|_{0, \mathpzc{D}}^{2}\right)^{1 / 2}, \quad \forall \chi \in H^{1}(\mathpzc{D}),\label{tracegeninq}
\end{eqnarray}
and, the inverse estimate
\begin{equation}
|\chi|_{j, \mathpzc{D}} \leq c_{inv} h_{\mathpzc{D}}^{i-j}|\chi|_{i, \mathpzc{D}} \quad \forall \chi \in \mathbb{P}_{k}(\mathpzc{D}), \quad 0 \leq i \leq j \leq 2.\hspace{0.3cm} \label{inversinq}
\end{equation}  
\end{lemma}
\noindent
For convenience, we use the same notation $c_{t r}$ for the constants in trace inequalities \eqref{traceinq} and \eqref{tracegeninq}, although they have distinct values.

The following lemma demonstrates the continuity and coercivity of the bilinear form (see, \cite[Lemma 3.1]{pascal2007}). We omit the detail of the proof. The proof can be easily followed by using the Cauchy-Schwarz inequality, trace inequality \eqref{tracegeninq} and the inverse inequality \eqref{inversinq}.
%%%%%%%%%%%%%%%%%%%%%%%%%%%%%%%%%%%%%%%%%%%%%%%%%%%%%%%%%%%%%%%%%%
\begin{lemma}\label{lemma3.1} Let $a_{h}(\cdot, \cdot)$ be the bilinear form as defined in \eqref{weakform}, Then, the following properties hold
\begin{itemize}
\item[\bf{(i)}]{\tt Continuity:} For all $y, \phi \in \mathpzc{Y}_{h}$,
\begin{equation}
 \left|a_{h}(y, \phi)\right| \leq 2|\|y\|| \|\phi\|. \label{bddness} \hspace{5cm}
	\end{equation} 
\item[\bf{(ii)}]{\tt Coercivity:} There exists a positive constant $c_{a}$ such that	
\begin{equation}
	a_{h}(\phi, \phi) \geq c_{a}\||\phi\||^{2}, \quad \forall \phi \in \mathpzc{V}+\mathpzc{V}_{h}, \label{coercivity}\hspace{5.5cm}
	\end{equation}	
	with the following mesh-dependent energy norm
\begin{align}
|\| \phi |\|&:=\Big[\sum_{K \in \mathscr{T}_{h}}\big(\|\nabla \phi\|_{L^2(K)}^{2}+a_0\|\phi\|_{L^2(K)}^{2}\big)\nonumber \\
&\hspace{0.5cm}+\sum_{E \in \mathcal{E}_{0,h} \cup \mathcal{E}_{D,h}}\big( h_{E} \|\smean{\nabla \phi}\|_{L^2(E)}^{2}+\frac{\sigma_0}{h_{E}}\| \sjump{\phi} \|_{L^2(E)}^{2}\big)\Big]^{1 / 2}.\label{energynorm}
\end{align}
\end{itemize}
\end{lemma}
%%%%%%%%%%%%%%%%%%%%%%%%%%%%%%%%%%%%%%%%%%%
\noindent
To simplify the presentation, we adopt the following notation
\begin{equation}
|\| \phi |\|_{L^2(I)}~=~\Big(\int_{0}^T |\| \phi |\|^2\,dt\Big)^{1/2}.\hspace{4cm} \label{eng2.19}
\end{equation}
Further, we assume that the source function $f$, the desired state $y_{d}$, the reaction term $a_0$, the desired control $q_{d}$, the Neumann boundary conditions $g_{_N}, r_{_N}$, the lower bound $q_{a}$, and the upper bound $q_{b}$, respectively are approximated by the functions $f_{h},\, y_{d, h},\, a_{0,h},\, q_{d,h},\, g_{_N,h},\, r_{_N,h},\, q_{a,h},\,\text{and}\; q_{b,h}$, where
\begin{equation}
f_{h},\, y_{d, h},\, a_{0,h} \in L^2(I;\mathpzc{V}_{h}),\quad \text{and} \quad q_{d,h},\, g_{_N,h},\, r_{_N,h},\, q_{a,h},\, q_{b,h} \in L^2(I;\mathpzc{Q}_{Nad,h}). \label{givendata}
\end{equation}
Moreover, the Dirichlet boundary data $g_{_D}$ is approximated by $g_{_D,h}\in L^2(I;\mathpzc{Q}_{Nad,h})$, where 
$$ \mathpzc{Q}_{Nad,h}=\{ q(t) \in L^{2}(\Gamma_{_D}):\quad q|_{E} \in \mathbb{P}^{1}(E) \quad \forall E \in \mathcal{E}_{D,h},\; t\in I\}.$$
Next, we apply the SIPG method to discretize the boundary control problem \eqref{contfunc}--\eqref{contspace}.
To determine a pair of discrete functions $\left(y_{h}, q_{h}\right) \in L^2(I;\mathpzc{Y}_{h}) \times L^2(I;\mathpzc{Q}_{Nad,h})$ satisfying
\begin{eqnarray}
&&\underset{q_h \in \mathpzc{Q}_{Nad,h}}{\min} \int_0^T\Big\{\frac{1}{2}\int_{\O}(y_h-y_{d,h})^2 dx+\frac{\alpha}{2} \int_{\Gamma_{N}}(q_h-q_{d,h})^2ds+\int_{\Gamma_{N}} r_{_N,h}\, y_h ds\Big\}dt \label{discretfunc}\\
\text{over} && \nonumber\\
&&~~~~~~~~~~~~~ \big(\frac{\partial y_h}{\partial t}, \phi_h)+a_{h}\left(y_{h}, \phi_h \right)=l_{h}\left(\phi_h \right)+b_{h}\left(q_{h}, \phi_h \right), \quad \phi_h \in \mathpzc{V}_{h},\label{discretformstate}\\
&&~~~~~~~~~~~~~~ y_h(x,0)=y_{h,0},\label{disint}
\end{eqnarray}
and the discrete control space
\begin{equation}
\mathpzc{Q}_{Nad,h}=\left\{q_{h} \in \mathpzc{Q}_{N,h}:\quad  q_{a,h} \leq q_{h} \leq q_{b,h}\right\}.\label{discretecontspace}
\end{equation}
This leads to the existence of a discrete ad-joint state $z_{h} \in \mathpzc{V}_{h}$ such that the following optimality conditions for the optimization problem \eqref{discretfunc}--\eqref{discretecontspace} are satisfied:
\begin{subequations}
\begin{eqnarray}
	\big(\frac{\partial y_h}{\partial t}, \phi_h)+a_{h}\left(y_{h}, \phi_h \right)&=& l_{h}\left(\phi_h \right)+b_{h}\left(q_{h}, \phi_h \right), \quad \phi_h \in \mathpzc{V}_{h},\label{disoptstate}\\
	y_h(x,0)&=&y_{h,0},\label{disint1}\\
	-\big(\frac{\partial z_h}{\partial t}, \psi_h \big)+a_h(\psi_h, z_h)&=&\left(y_h-y_{d,h}, \psi\right)+\left(r_{_N,h}, \psi_h \right)_{L^2(\Gamma_{_N})}, \quad \forall \psi_h \in \mathpzc{V}_h, \label{disoptadjoint-state}\\
	z_h(x,T)&=&0,\label{distfinal}\\
	\left(\alpha \left(q_h-q_{d,h}\right)+z_h, \varphi_h-q_h\right)_{L^2(\Gamma_{_N})} &\geq& 0, \quad \forall \varphi_h \in \mathpzc{Q}_{Nad,h}. \label{disfirstoptcond}
\end{eqnarray}
\end{subequations}
Analogous to the continuous case, introducing the discrete co-control $\mu_{h} \in \mathpzc{Q}_{Nad,h}$, allows us to restate the first-order optimality condition \eqref{disfirstoptcond} as follows
\begin{eqnarray}
&&\alpha \left(q_{h}-q_{d,h}\right)+z_{h}+\mu_{h}=0, \label{disopt11}\\
&&\mu_{h}-\max \left\{0, \mu_{h}+\gamma\left(q_{h}-q_{b,h}\right)\right\}+\min \left\{0, \mu_{h}-\gamma\left(q_{a,h}-q_{h}\right)\right\}=0.\label{disopt12}
\end{eqnarray}
By defining $\mu_{h}=\mu_{b,h}-\mu_{a,h}$, we can equivalently express \eqref{disopt12} as the following discrete complementarity system
\begin{eqnarray}
&&\mu_{a,h} \geq 0, \quad q_{a,h}-q_{h} \leq 0, \quad \mu_{a,h}\left(q_{a,h}-q_{h}\right)=0,\label{discompcondi1}\\
&&\mu_{b,h} \geq 0, \quad q_{h}-q_{b,h} \leq 0, \quad \mu_{b,h}\left(q_{h}-q_{b,h}\right)=0.\label{discompcondi2}
\end{eqnarray}
We then define the discrete active and inactive sets as follows
\begin{eqnarray}
&&\mathpzc{A}_{a, h}=\bigcup\left\{x \in E \mid \mu_{h}(x)-\gamma\left(q_{a,h}(x)-q_{h}(x)\right)<0, \quad \forall E \in \mathcal{E}_{N,h}\right\} \label{activeset1}\\
&&\mathpzc{A}_{b, h}=\bigcup\left\{x \in E \mid \mu_{h}(x)+\gamma\left(q_{h}(x)-q_{b,h}(x)\right)>0, \quad \forall E \in \mathcal{E}_{N,h}\right\}\label{activeset2}
\end{eqnarray}
and
\begin{equation}
\mathcal{I}_{h}=\mathcal{E}_{N,h} \backslash\left\{\mathpzc{A}_{a, h} \cup \mathpzc{A}_{b, h}\right\},
\end{equation}
respectively. Analogous to the continuous case, the complementarity conditions \eqref{discompcondi1}--\eqref{discompcondi2} can be restated as
\begin{subequations}
\begin{eqnarray}
&&q_{h}=q_{a,h}, \quad \mu_{b,h}=0, \quad \mu_{h} \leq 0~~ \quad \text { on }~~ \mathpzc{A}_{a, h}, \\
&&q_{h}=q_{b,h}, \quad \mu_{a,h}=0, \quad \mu_{h} \geq 0~~ \quad \text { on }~~ \mathpzc{A}_{b, h}, \\
&&q_{a,h}<q_{h}<q_{b,h}, \quad \mu_{a,h}=\mu_{b,h}=0, \quad \mu_{h}=0~~ \quad \text { on }~~ \mathcal{I}_{h}.
\end{eqnarray}
\end{subequations}
We proceed to develop the space-time approximations of the optimal boundary control problem \eqref{contfunc}-\eqref{contspace} utilizing the combination of backward Euler time-stepping scheme with the discontinuous Galerkin spatial discretization method.\\

\noindent
\textbf{(b) \tt Time-domain discretization.} 
We consider a partition of the time domain $I=[0,T]$ into $N$ subintervals, denoted as $I_i=(t_{i-1},t_i]$ such that $I=\cup_{i=1}^{N_T} \bar{I}_i$, where $0=t_0<t_1<t_2<\ldots<t_N=T$. The time step size is given by  $k_i=t_i-t_{i-1}$, $i=1,\,2,\ldots,\,{N_T}$. 

Furthermore, we define $\mathscr{T}^i_h$, $i=1,\,2,\,\ldots,\,N_T,$ be the triangulation of $\bar{\Omega}$ at the time level $t_i$. At each time level $t_i$, we introduce the set of all edges $\mathcal{E}^i_{h}$, which can be decomposed into interior edges $\mathcal{E}^i_{0, h}$ and boundary edges $\mathcal{E}^i_{B,h}$. The boundary edges are further divided into Dirichlet boundary edges $\mathcal{E}^i_{D,h}$ and Neumann boundary edges $\mathcal{E}^i_{N,h}$, respectively. 

Consistent with the semi-discrete framework, we introduce the finite-dimensional spaces $\mathpzc{Y}_h^i$, $\mathpzc{V}_h^i$, and $\mathpzc{Q}^i_{Nad,h}$ associated with the triangulation $\mathscr{T}^i_h$ at each time level $t_i$, for $i=1,\,2,\,\ldots,\,N_T$, by
\begin{eqnarray}
\mathpzc{V}^i_{h}~=~\mathpzc{Y}^i_{h}=\left\{v \in L^2(I;L^{2}(\Omega)):\quad v|_{K} \in \mathbb{P}_{1}(K) \quad \forall K \in \mathscr{T}_{h}^i \right\}, \label{femspacestatefully}
\end{eqnarray}
and
\begin{eqnarray}
\mathpzc{Q}^i_{N,h}~=~\big\{q \in L^2(I;L^2(\Gamma_{_N})):\quad q|_{E} \in \mathbb{P}_{1}(E) \quad \forall E \in \mathcal{E}^i_{N,h}\big\},\label{femspacecontrolfully}
\end{eqnarray}
respectively. Then, the space-time discretization of the OCP \eqref{discretfunc}--\eqref{discretecontspace} is given by 
\begin{eqnarray}
&&\underset{q^i_h \in \mathpzc{Q}^i_{Nad,h}}{\min}\sum_{i=1}^{N_T} k_i \Big\{\frac{1}{2}\int_\O (y^i_h-y^i_{d,h})^2dx+\frac{\alpha}{2} \int_{\Gamma_{N}}(q^i_h-q^i_{d,h})^{2}ds+\int_{\Gamma_{N}} r^i_{_N,h}\, y^i_h ds\Big\} \label{fullydiscretfunc}\\
\text {over} && \nonumber\\
&&~~~~~~~\big(\frac{y^i_h-y^{i-1}_h}{k_i}, \phi_h)+a_{h}\left(y^i_{h}, \phi_h \right)=l^i_{h}\left(\phi_h \right)+b_{h}\left(q^i_{h}, \phi_h \right), \quad \phi_h \in \mathpzc{V}^i_{h},\; i=1,\,2,\ldots,\,N,\label{fullydiscretformstate}\\
&&~~~~~~~~~y_h(x,0)=y_{h}^0,\label{fullydisint}
\end{eqnarray}
with the discrete set of admissible controls
\begin{equation}
\mathpzc{Q}^i_{Nad,h}=\left\{q^i_{h} \in \mathpzc{Q}^i_{N,h}:\quad  q^i_{a,h} \leq q^i_{h} \leq q^i_{b,h}\right\}.\label{fullydiscretecontspace}
\end{equation}
The fully-discrete problem \eqref{fullydiscretfunc}--\eqref{fullydiscretecontspace} has a unique solution $(y_h^i,q^i_h)\in \mathpzc{Y}_h^i\times \mathpzc{Q}^i_{Nad,h}$, $i=1,\,2,\ldots,\,N_T$ iff there exists a unique $z_h^{i-1}\in \mathpzc{V}_h^i,\, i=1,\,2,\,\ldots,\,N_T$ such that the following optimality conditions satisfies, for $i=[1:N_T]$:
 \begin{subequations}
\begin{eqnarray}
\big(\frac{y^i_h-y^{i-1}_h}{k_i}, \phi_h)+a_{h}\left(y^i_{h}, \phi_h \right)&=&l^i_{h}\left(\phi_h \right)+b_{h}\left(q^i_{h}, \phi_h \right), \quad  { \phi_h \in \mathpzc{V}^i_{h},}\label{disoptstate12}\\ %{\, i=1,\,2,\ldots,\,N,}
y_h(x,0)&=&y_{h,0},\label{disint12}\\
-\big(\frac{z^i_h-z^{i-1}_h}{k_i}, \psi_h \big)+a_h(\psi_h, z^{i-1}_h)&=&\left(y^i_h-y^i_{d,h}, \psi\right)+\left(r^i_{_N,h}, \psi_h \right)_{L^2(\Gamma_{_N})}, \quad \forall \psi_h \in \mathpzc{V}^i_h, \label{disoptadjoint-state12}\\ %\,i=N,\,N-1,\ldots,\,1,
z_h(x,t_N)&=&0,\label{distfinal12}\\
\left(\alpha \left(q^i_h-q^i_{d,h}\right)+z^{i-1}_h, \varphi_h-q^i_h\right)_{L^2(\Gamma_{_N})} &\geq& 0, \quad \forall \varphi_h \in \mathpzc{Q}^i_{Nad,h}. 
 \label{disfirstoptcond12} %\, i=1,\,2,\ldots,\,N.
\end{eqnarray}
\end{subequations}
For each time subinterval $I_i,\; i=1,\,2\,\ldots,\,N_T$,
we define the piecewise linear and continuous time interpolants $Y_h(t)$ and $Z_h(t)$, valid for all   $t\in I_i,$ by
\begin{eqnarray*}
&&Y_h(t)|_{t\in I_i}:=\frac{t-t_{i-1}}{k_i}y_h^i+\frac{t_i-t}{k_i}y_h^{i-1},\\
&&Z_h(t)|_{t\in I_i}:=\frac{t-t_{i-1}}{k_i}z_h^i+\frac{t_i-t}{k_i}z_h^{i-1},\\
\text{and}&&\\
&&Q_h(t)|_{t\in I_i}:=q_h^i.
\end{eqnarray*}
Similarly, for $t\in I_i,\; i=1,\,2\,\ldots,\,N_T$, we set 
\begin{subequations} 
\begin{eqnarray}
&&\mathcal{G}_{D,h}(t)~:=~\frac{t-t_{i-1}}{k_i}g_{_D,h}^i+\frac{t_i-t}{k_i}g_{_D,h}^{i-1},\\
&&\mathcal{G}_{N,h}(t)~:=~\frac{t-t_{i-1}}{k_i}g_{_N,h}^i+\frac{t_i-t}{k_i}g_{_N,h}^{i-1},\\
&&\mathcal{R}_{N,h}(t)~:=~\frac{t-t_{i-1}}{k_i}r_{_N,h}^i+\frac{t_i-t}{k_i}r_{_N,h}^{i-1}.
\end{eqnarray}
\end{subequations}
For $\omega \in C(I,L^2(\Omega))$, let $\omega(x,t_i)=\hat{\omega}(x,t)|_{t\in (t_{i-1},t_i]}$  and  $\omega(x,t_{i-1})=\tilde{\omega}(x,t)|_{t\in (t_{i-1},t_i]}$ denote the restrictions of $\omega(x,\cdot)$ to the interval $(t_{i-1},t_i]$. Then, for each time subinterval $I_i,\; i=1,\,2\,\ldots,\,N_T$, the optimality conditions can be restated as follows
\begin{subequations}
\begin{eqnarray}
\big(\frac{\partial Y_h}{\partial t}, \phi_h)+a_{h}(\hat{Y}_h, \phi_h)&=&\hat{l}_{h}\left(\phi_h \right)+b_{h}\left(Q_h, \phi_h \right), \quad  { \phi_h \in \mathpzc{V}^i_{h},}\label{redisoptstate12}\\ %{\, i=1,\,2,\ldots,\,N,}
Y^0_h&=&y_{h,0},\label{redisint12}\\
-\big(\frac{\partial Z_h}{\partial t}, \psi_h \big)+a_h(\psi_h, \tilde{Z}_h)&=&\big(\hat{Y}_h-\hat{y}_{d,h}, \psi_h\big)+\left(\hat{\mathcal{R}}_{N,h}, \psi_h \right)_{L^2(\Gamma_{_N})}, \, \forall \psi_h \in \mathpzc{V}^i_h,\;\;\; \label{redisoptadjoint-state12}\\ 
Z_h^N&=&0,\label{redistfinal12}\\
\big(\alpha \left(Q_h-\hat{q}_{d,h}\right)+\tilde{Z}_h, \varphi_h-Q_h\big)_{L^2(\Gamma_{_N})} &\geq& 0, \quad \forall \varphi_h \in \mathpzc{Q}^i_{Nad,h}. 
\label{redisfirstoptcond12}
\end{eqnarray}
\end{subequations}
By introducing the discrete co-control $\hat{\mu}_{h} \in \mathpzc{Q}^i_{Nad,h}$,  for each $t\in  I$, we can restate the first-order optimality condition \eqref{redisfirstoptcond12}  as follows
\begin{eqnarray}
&&\alpha \left(Q_{h}-\hat{q}_{d,h}\right)+\tilde{Z}_{h}+\hat{\mu}_{h}=0, \label{redisopt11}\\
&&\hat{\mu}_{h}-\max \left\{0, \hat{\mu}_{h}+\gamma\left(Q_{h}-\hat{q}_{b,h}\right)\right\}+\min \left\{0, \mu_{h}-\gamma\left(\hat{q}_{a,h}-Q_h\right)\right\}=0.\label{redisopt12}
\end{eqnarray}
In analogy to the continuous case, an equivalent formulation of \eqref{redisopt12}  with $\hat{\mu}_{h}=\hat{\mu}_{b,h}-\hat{\mu}_{a,h}$ can be expressed as follows
\begin{eqnarray}
&&\hat{\mu}_{a,h} \geq 0, \quad \hat{q}_{a,h}-Q_h \leq 0, \quad \hat{\mu}_{a,h}\left(\hat{q}_{a,h}-Q_{h}\right)=0,\label{discompcondi111}\\
&&\hat{\mu}_{b,h} \geq 0, \quad Q_{h}-\hat{q}_{b,h} \leq 0, \quad \hat{\mu}_{b,h}\left(Q_{h}-\hat{q}_{b,h}\right)=0.\label{discompcondi222}
\end{eqnarray}
For $i\in [1:N_T],$ the active  and inactive sets are given by 
\begin{subequations}
\begin{eqnarray}
&&\mathpzc{A}^i_{a, h}=\bigcup\left\{x \in E \mid \hat{\mu}_{h}(x)-\gamma\left(\hat{q}_{a,h}(x)-Q_{h}(x)\right)<0, \quad \forall E \in \mathcal{E}^i_{N,h}\right\} \label{activeset111}\\
&&\mathpzc{A}^i_{b, h}=\bigcup\left\{x \in E \mid \hat{\mu}_{h}(x)+\gamma\left(Q_{h}(x)-\hat{q}_{b,h}(x)\right)>0, \quad \forall E \in \mathcal{E}^i_{N,h}\right\}\label{activeset222}\\
&&\mathcal{I}^i_{h}=\mathcal{E}^i_{N,h} \backslash\{\mathpzc{A}^i_{a, h} \cup \mathpzc{A}^i_{b, h}\}.
\end{eqnarray}
\end{subequations}
 Furthermore, for $i\in [1:N_T]$, the complementarity conditions \eqref{discompcondi111}--\eqref{discompcondi222} can be compactly expressed as
 \begin{subequations}
\begin{eqnarray}
&&Q_{h}=\hat{q}_{a,h}, \quad \hat{\mu}_{b,h}=0, \quad \hat{\mu}_{h} \leq 0~~ \quad \text { on }~~ \mathpzc{A}^i_{a, h}, \\
&&Q_{h}=\hat{q}_{b,h}, \quad \hat{\mu}_{a,h}=0, \quad \hat{\mu}_{h} \geq 0~~ \quad \text { on }~~ \mathpzc{A}^i_{b, h}, \\
&&\hat{q}_{a,h}<Q_{h}<\hat{q}_{b,h}, \quad \hat{\mu}_{a,h}=\hat{\mu}_{b,h}=0, \quad \hat{\mu}_{h}=0~~ \quad \text { on }~~ \mathcal{I}^i_{h}.
\end{eqnarray}
\end{subequations}
We now proceed to the error analysis of the finite element approximation for the boundary control problem (BCP) governed by equations \eqref{contfunc}-\eqref{contspace}.
%%%%%%%%%%%%%%%%%%%%%%%%%%%%%%%%%%%%%%%%%%%%%%%%%%%%%%%%%%%%%%%%%%%%%%%%%%%%%%%%%
\section{Aposteriori Error Analysis}\label{sec4errest}
This section focuses on the development of residual-based a posteriori error estimators for quantifying errors in the state, adjoint-state, control, and co-control. Specifically, we measure the state and adjoint-state errors in the $|\|\cdot\||_{L^2(I)}$ as defined in \eqref{eng2.19}, while the control $(q)$ and co-control $(\mu)$ errors are measured in the $L^2(I;L^{2}(\Gamma_{_N}))$-norm.

To initiate this analysis, we derive upper bounds for these errors, leveraging essential approximation results from \cite{baker1995}. 
Let  $\psi=e-\phi_h$, where $\phi_h$ is the best piecewise constant approximation of the error $e$. Here, $e$ represents either the state error $e_y(=y(Q_h)-Y_h)$ or the adjoint-state error $e_z(=z(Q_h)-Z_h)$. Thus we have 
\begin{eqnarray}
\|\psi\|_{L^2(K)}~\leq~ C h_K\|\nabla e\|_{L^2(K)},\quad K\in \mathscr{T}_h^i.\nonumber 
\end{eqnarray} 
An application of the trace inequality \eqref{tracegeninq} leads to
\begin{eqnarray}
\sum_{K\in \mathscr{T}_h^i}h_K^{-2}\|\psi\|^2_{L^2(K)}~\leq~ C\sum_{K\in \mathscr{T}_h^i} \|\nabla e\|^2_{L^2(K)}.\label{appinq4.24} %{c_{1,2}}
\end{eqnarray}
Additionally, exploiting the mesh property $h_E^{-1}h_K\leq 1$ to have
\begin{eqnarray}
\sum_{E\in \mathcal{E}_{0,h}^i}h_E^{-1}\big(\|\smean{\psi}\|^2_{L^2(E)}+\|\sjump{\psi}\|^2_{L^2(E)}\big)&\leq & C\sum_{E\in \mathcal{E}_{0,h}^i} \sum_{K;\,E\in \Gamma_K}h_E^{-1}\big(h_K^{-1}\|\psi\|^2_{L^2(K)}+h_K\|\nabla \psi\|^2_{L^2(K)}\big)\nonumber\\&\leq & C \sum_{K\in \mathscr{T}_h^i} \|\nabla e\|^2_{L^2(K)},\label{appinq4.25}%c_{1,3}
\end{eqnarray}
and 
\begin{eqnarray}
\sum_{E\in \mathcal{E}_{\mathscr{B},h}^i}h_E^{-1}\|\psi\|^2_{L^2(E)}&\leq & C\sum_{E\in \mathcal{E}_{\mathscr{B},h}^i} \sum_{K;\,E\in \Gamma_K}h_E^{-1}\big(h_K^{-1}\|\psi\|^2_{L^2(K)}+h_K\|\nabla \psi\|^2_{L^2(K)}\big)\nonumber\\&\leq &C\sum_{K\in \mathscr{T}_h^i} \|\nabla e\|^2_{L^2(K)}.\label{appinq4.26} %{c_{1,4}}
\end{eqnarray}
We proceed to introduce the residual-type error estimators tailored to the PBCP governed by equations \eqref{contfunc}-\eqref{contspace}, as
\begin{enumerate}
\item {\tt Element-wise residuals:} 
We begin by introducing element-wise residuals for the state and co-state variables, denoted by 
$\eta_{y,K}^i,\, \text{and}\; \eta_{z,K}^i$, respectively. For $ K\in \mathscr{T}_h^i,\;i\in [1:N_T],$ these residuals are defined by 
\begin{subequations}\label{spacest4.4}
	\begin{align}
		&\eta_{y,K}^i~:=~ h_K\|\hat{f}_h-Y_{h,t}+\Delta\hat{Y}_h-a_{0,h}\hat{Y}_h\|_{L^2(K)},\\
		&\eta_{z,K}^i~:=~h_K\|\hat{Y}_h-\hat{y}_{d,h}+Z_{h,t}+\Delta \tilde{Z}_h-a_{0,h}\tilde{Z}_h\|_{L^2(K)}.
	\end{align}
\end{subequations}
\item {\tt Interior edge residuals:} 
Let $\eta^i_{y,E_0} $ and $\eta^i_{z,E_0}$ be the interior edge residuals corresponding to state and adjoint-state variables, respectively, associated with each interior edge $E_0\in \mathcal{E}_{0,h}^i$, are defined as 
\begin{eqnarray}
	\eta^i_{y,E_0}&:=&h_{E_0}^{1/2}\|\sjump{\nabla \hat{Y}_h}\|_{L^2(E_0)}+\sigma_0 \,h_{E_0}^{-1/2}\|\sjump{\hat{Y}_h}\|_{L^2(E_0)},\\
	\eta^i_{z,E_0}&:=&h_{E_0}^{1/2} \|\sjump{\nabla \tilde{Z}_h}\|_{L^2(E_0)}+\sigma_0 \,h_{E_0}^{-1/2} \| \sjump{\tilde{Z}_h}\|_{L^2(E_0)},\\
	\tilde{\eta}_{y,E_0}^i&:=&\sigma_0\, h_{E_0}^{-1/2}\|\sjump{\hat{Y}_h}\|_{L^2(E_0)}+\sigma_0 h_{E_0}^{-1/2}\|\sjump{Y_h}\|_{L^2(E_0)}, \\ %\sqrt{(\sigma_0+1)} 
	\tilde{\eta}_{z,E_0}^i&:=&\sigma_0\, h_{E_0}^{-1/2}\| \sjump{\tilde{Z}_h}\|_{L^2(E_0)}+\sigma_0\,h_{E_0}^{-1/2}\|\sjump{Z_h}\|_{L^2(E_0)}. % \sqrt{(\sigma_0+1)}
\end{eqnarray}
\item {\tt Dirichlet and Neumann boundary edge residuals:} We define the boundary edge residuals, which are associated with $E_D\in \mathcal{E}_{_D,h}^i$ and $E_N\in \mathcal{E}_{_N,h}^i $, respectively, and given as
\begin{itemize}
\item[i.] Dirichlet edges residuals $\eta^i_{y,g_D}$, $\eta^i_{y,E_D} $ and $\eta^i_{z,E_D}$ by 
\begin{eqnarray}
&&\eta^i_{y,g_D}:=\sigma_0\, h^{-1/2}_{E_D}\|\hat{\mathcal{G}}_{D,h}-\hat{Y}_h\|_{L^2(E_D)},\quad E_D\in \mathcal{E}_{_D,h}^i,\\
&&\eta^i_{y,E_D}:=\sigma_0 \,h^{-1/2}_{E_D}\|\hat{Y}_h\|_{L^2(E_D)}+\sigma_0 \,h^{-1/2}_{E_D}\|Y_h\|_{L^2(E_D)}, \\
&&\eta^i_{z,E_D}:= \sigma_0 \, h^{-1/2}_{E_D}\| \tilde{Z}_h\|_{L^2({E_D})}+ \sigma_0\,h^{-1/2}_{E_D}\|Z_h\|_{L^2({E_D})}, \quad E_D\in \mathcal{E}_{_D,h}^i, %\sqrt{(\sigma_0+1)}
\end{eqnarray}
\item[ii.] and, the Neumann boundary edge residuals $\eta^i_{y,E_N} $,  $\eta^i_{z,E_N}$,  and $\eta^i_{q,E_N}$  by 
\begin{eqnarray}
&&\eta^i_{y,E_N}:=h_{E_N}^{1/2}\|Q_h+\hat{\mathcal{G}}_{N,h}-\mathbf{n}_E\cdot \nabla \hat{Y}_h\|_{L^2(E_N)},\quad E_N\in \mathcal{E}_{_N,h}^i,\label{4.12yen}\\
&&\eta^i_{z,E_N}:= h_{E_N}^{1/2} \|\hat{\mathcal{R}}_{N,h}- \mathbf{n}_{E_N}\cdot \nabla \tilde{Z}_h\|_{L^2(E_N)},\quad E_N\in \mathcal{E}_{_N,h}^i,\label{4.13en}\\
&&\eta^i_{q,E_N}:= h_{E_N} \|\mathbf{n}_{E_N}\cdot \nabla(\alpha \left(Q_h-\hat{q}_{d,h}\right)+\tilde{Z}_h)\|_{L^2(E_N)}, \quad E_N\in \mathcal{E}_{_N,h}^i,\label{4.14444qen}
\end{eqnarray}
\end{itemize}
\item Moreover, data oscillations in the error analysis are represented by
\begin{subequations}\label{dataest4.17}
	\begin{align}
		&\varTheta_y^2~:=~\sum_{i=1}^{N_T}\int_{t_{i-1}}^{t_i}\Big[\sum_{K\in \mathscr{T}_h^i}\underbrace{h_K^2\big(\|(a_{0,h}-a_0)\hat{Y}_h\|^2_{L^2(K)}+\|f-\hat{f}_h\|^2_{L^2(K)}\big)}_{(\varTheta_{y,K}^i)^2}\nonumber\\
        &\hspace{1cm}+\sum_{E\in \mathcal{E}_{N,h}}\underbrace{h_E\|g_N-\hat{\mathcal{G}}_{N,h}\|^2_{L^2(E)}}_{(\varTheta_{y,E_N}^i)^2}+\sum_{E\in \mathcal{E}^i_{D,h}}\underbrace{\frac{\sigma_0}{h_E}\|g_D-\mathcal{G}_{D,h}\|^2_{L^2(E)}}_{(\varTheta_{y,E_D}^i)^2}\Big]dt,\\
		&\varTheta_z^2~:=~\sum_{i=1}^{N_T}\int_{t_{i-1}}^{t_i}\Big[\sum_{K\in \mathscr{T}_h^i}\underbrace{h_K^2\big(\|(a_{0,h}-a_0)\tilde{Z}_h\|^2_{L^2(K)}+\|\hat{y}_{d,h}-y_d\|^2_{L^2(K)}\big)}_{(\varTheta_{z,K}^i)^2}\nonumber\\
        &\hspace{1cm}+\sum_{E\in \mathcal{E}_{_N,h}^i}\underbrace{h_E\|r_{_N}-\hat{\mathcal{R}}_{N,h}\|^2_{L^2(E)}}_{(\varTheta_{z,E_N}^i)^2}\Big]dt,\\
		&\varTheta_q^2 ~:=~ \sum_{i=1}^{N_T}\int_{t_{i-1}}^{t_i}\sum_{E\in \mathcal{E}_{N,h}^i}\Big[\underbrace{\|\alpha (q_d-\hat{q}_{d,h})\|^2_{L^2(E)}+\|q_a-\hat{q}_{a,h}\|^2_{L^2(E)}+\|q_b-\hat{q}_{b,h}\|^2_{L^2(E)}}_{(\varTheta^i_{q,E_N})^2} \Big]dt.\label{4.17qen}
	\end{align}
\end{subequations}
\item We define the temporal error estimators for the state and adjoint-state variables as $\varTheta_{y,T}$ and $\varTheta_{z,T}$ respectively, which are expressed as 
\begin{subequations}\label{tempest4.18}
	\begin{align}
		\varTheta_{y,T}^2 ~:=~\sum_{i=1}^{N_T}\int_{t_{i-1}}^{t_i}\||\hat{Y_h}-Y_h\||^2dt, \quad \text{and} \quad
		\varTheta_{z,T}^2~:=~\sum_{i=1}^{N_T}\int_{t_{i-1}}^{t_i}\||Z_h-\tilde{Z_h}\||^2dt,
	\end{align}
\end{subequations}
respectively. Further, we define $$\Xi_{T_{yz}}:=\big(\varTheta_{y,T}^2+\varTheta_{z,T}^2 \big)^{1/2}.$$
\item Ultimately, we introduce the global error indicators for the state, adjoint-state, and control variables, defined as
\begin{subequations}\label{sumofest4.19}
	\begin{align}
		&\eta_y^2~:=~\sum_{K\in \mathscr{T}_h^i}(\eta_{y_0,K})^2+ \sum_{i=1}^{N_T}\int_{t_{i-1}}^{t_i}\Big[\sum_{K\in\mathscr{T}_h^i} (\eta_{y,K}^i)^2
		+\sum_{E\in \mathcal{E}^i_{N,h}} (\eta^i_{y,E_N})^2+\sum_{E\in \mathcal{E}^i_{D,h}}(\eta^i_{y,g_D})^2\nonumber\\
        &\hspace{1cm}+\sum_{E\in \mathcal{E}^i_{0,h}}(\eta^i_{y,E_0})^2+\sum_{E\in \mathcal{E}^i_{0,h}}(\tilde{\eta}^i_{y,E_0})^2+\sum_{E\in \mathcal{E}^i_{D,h}}(\eta^i_{y,E_D})^2\Big]dt,\\ %+(\eta^i_{y,mix})^2
		&\eta^2_z~:=~\sum_{i=1}^{N_T}\int_{t_{i-1}}^{t_i}\Big[\sum_{K\in \mathscr{T}_h^i} (\eta^i_{z,K})^2+\sum_{E\in \mathcal{E}_{N,h}^i }(\eta^i_{z,E})^2+\sum_{E\in \mathcal{E}_{0,h}^i}(\eta^i_{z,E})^2+\sum_{E\in \mathcal{E}_{D,h}^i }(\eta^i_{z,E})^2
		\nonumber\\
        &\hspace{1cm}+\sum_{E\in \mathcal{E}_{0,h}^i }(\tilde{\eta}_{z,E}^i)^2\Big]dt,\\ %+(\eta^i_{z,mix})^2
		&\eta^2_q~:=~ \sum_{i=1}^{N_T}\int_{t_{i-1}}^{t_i}\sum_{E_N\in \mathcal{E}_{N,h}^i}(\eta^i_{q,E_N})^2dt,\label{4.19cq}
	\end{align}
\end{subequations}
were $\eta_{y_0,K}$ denotes the initial data error estimator and defined as
\begin{eqnarray}
	\eta_{y_0,K}&:=&\|y_0-Y_{0,h}\|_{L^2(K)}.
\end{eqnarray}
\end{enumerate}
Moreover, the sum of the residual error estimators for the SIPG discretization of the PBCP is given by
\begin{eqnarray}
\Upsilon_{yzq}~:=~ (\eta_{y}^2+\eta_z^2+\eta_q^2)^{1/2}, \label{upsilon}
\end{eqnarray}
and $\varTheta$ related to data oscillations is defined by
\begin{eqnarray}
\varTheta_{yzq}~:=~ (\varTheta^2_y+\varTheta^2_z+\varTheta^2_q)^{1/2}.\label{vartheta}
\end{eqnarray}
\subsection{Reliable-type Aposteriori Error Estimation}\label{section4444}
This section focuses on deriving reliability-type a posteriori error estimates for the optimal control problem (OCP) governed by equations \eqref{contfunc}--\eqref{contspace}. To achieve this, we begin by introducing intermediate problems for the state and adjoint-state variables.
Given control variable $q^* \in \mathpzc{Q}_{Nad}$, to determine a pair $(y(q^*), z(q^*))\in \mathpzc{Y}\times \mathpzc{V}$ that satisfies the following control system
\begin{eqnarray}
\big(\frac{\partial y(q^*)}{\partial t}, \phi \big)+a(y(q^*), \phi)&=&(f,  \phi)+\left(q^*+g_{_N}, \phi \right)_{L^2(\Gamma_{_N})}  \quad \forall \phi \in \mathpzc{V},\label{intstate} \\
y(q^*)(x,0)&=&y_0(x),\label{intinitial}\\
-\big(\frac{\partial z(q^*)}{\partial t}, \psi \big)+a(\psi, z(q^*))&=&\left(y(q^*)-y_{d}, \psi\right)+\left(r_{_N}, \psi \right)_{L^2(\Gamma_{_N})} \quad \forall \psi \in \mathpzc{V}, \label{intadjoint-state}\\
z(q^*)(x,T)&=&0.\label{intfinal}
\end{eqnarray}
To derive the error equations for the state and adjoint-state variables, we subtract the intermediate state and adjoint-state equations \eqref{intstate} and \eqref{intadjoint-state} with $q^*=Q_h$ from the continuous state and adjoint-state \eqref{weakformstate} and  \eqref{weakformadjoint-state}, respectively. This yields
\begin{eqnarray}
\Big(\frac{\partial (y-y(Q_h))}{\partial t}, \phi \Big)+a(y-y(Q_h), \phi)&=&\left(q-Q_h, \phi \right)_{L^2(\Gamma_{_N})}  \quad \forall \phi \in \mathpzc{V}, \label{erroreqstate}\\
-\Big(\frac{\partial (z-z(Q_h))}{\partial t}, \psi \Big)+a(\psi, z-z(Q_h))&=&(y-y(Q_h), \psi) \quad \forall \psi \in \mathpzc{V}.\label{erroreqadjoint-state} 
\end{eqnarray}
We do the integration over $[0,T]$ by setting $\phi= y-y(Q_h)$ and $\psi=z-z(Q_h)$ in \eqref{erroreqstate} and \eqref{erroreqadjoint-state}. Then we use Lemma \ref{lemma3.1} and the trace inequality \eqref{traceinq} to get
\begin{align}
\frac{1}{2}\|(y-y(Q_h))(T)\|^2+c_a\int_0^T\||y-y(Q_h)\||^2dt&\leq\|q-Q_h\|_{L^2(I;L^2(\Gamma_{_N}))} \|y-y(Q_h)\|_{L^2(I;L^2(\Gamma_{_N}))}\nonumber\\ &\leq c_0c_{tr}\|q-Q_h\|_{L^2(I;L^2(\Gamma_{_N}))} \|y-y(Q_h)\|_{L^2(I;H^1(\Omega))}\nonumber\\\||y-y(Q_h)\||_{L^2(I)}&\leq \frac{c_0c_{tr}}{c_a}\|q-Q_h\|_{L^2(I;L^2(\Gamma_{_N}))}, \label{statebound}
\end{align}
where $c_0$ is defined as the minimum of $a_0$ and its reciprocal $a_0^{-1}$. Likewise,
\begin{align}
\frac{1}{2}\|(z-z(Q_h))(0)\|^2+c_a\int_0^T\||z-z(Q_h)\||^2dt&\leq\|y-y(Q_h)\|_{L^2(I;L^2(\Omega))}\|z-z(Q_h)\|_{L^2(I;L^2(\Omega))}\nonumber\\
\||z-z(Q_h)\||_{L^2(I)}&\leq\||y-y(Q_h)\||_{L^2(I)}.
\label{adjoint-statebound} 
\end{align} 
%%%%%%%%%%%%%%%%%%%%%%%%%%%%%%%%%%%%%%%%%%%%%%%%%%%%%%%%%%%%%%%%%%%%%%%%%%
The subsequent lemma establishes a reliable aposteriori error estimate.
\begin{lemma}\label{lm4.1int}
Let $(y,z,q)\in \mathpzc{Y} \times \mathpzc{V} \times \mathpzc{Q}_{Nad}$ and $(Y_h, Z_h,Q_h)\in \mathpzc{Y}^i_h \times \mathpzc{V}^i_h \times \mathpzc{Q}^i_{Nad,h},\;i\in[1:N_T],$ be the solutions of \eqref{weakformstate}--\eqref{weakformcontrol} and \eqref{redisoptstate12}--\eqref{redisfirstoptcond12}, respectively, and the co-control  variable $\mu$ and the discrete co-control variable $\mu_h$ as defined in \eqref{optmallitycon1} and \eqref{redisopt11}, respectively.  Assume that $\mathpzc{Q}^i_{Nad,h} \subset \mathpzc{Q}_{Nad,h}$, $\big(\alpha \left(Q_h-\hat{q}_{d,h}\right)+\tilde{Z}_h\big)|_{E\in \mathcal{E}^i_{N,h}} \in H^1(E)$ and that there is a positive constant $C_{4,1}$ and $\varphi_h \in \mathpzc{Q}^i_{Nad,h}$ such that %[42]
\begin{eqnarray}
&&\int_0^T\|\big(\alpha \left(Q_h-\hat{q}_{d,h}\right)+\tilde{Z}_h, \varphi_h-Q_h\big)\|_{L^2(\Gamma_{_N})}dt\nonumber\\&&\hspace{1cm}\leq~C_{4,1} \int_0^T \sum_{E \in \mathcal{E}^i_{N,h}} h_E \|\mathbf{n}_E\cdot \nabla(\alpha \left(Q_h-\hat{q}_{d,h}\right)+\tilde{Z}_h)\|_{L^2(E)}\|q-Q_h\|_{L^2(E)}dt.\label{assmption11}
\end{eqnarray}
Then, we have
\begin{align}
&\||y-Y_h\||_{L^2(I)}+\||z-Z_h\||_{L^2(I)}+\|q-Q_h\|_{L^2(I;L^2(\Gamma_{_N}))}+\|\mu-\hat{\mu}_h\|_{L^2(I;L^2(\Gamma_{_N}))}\nonumber\\ & \hspace{0.5cm} \leq C_{4,4}\eta_q+C_{4,5} \|\alpha(q_d-\hat{q}_{d,h})\|_{L^2(I;L^2(\Gamma_{_N}))}+C_{4,6}\|Z_h-\tilde{Z}_h\|_{L^2(I;H^1(\Omega))}\nonumber\\
&\hspace{0.5cm}+C_{4,7}\||z(Q_h)-Z_h\||_{L^2(I)}
+C_{4,8}\||y(Q_h)-Y_h\||_{L^2(I)}.
\end{align}
where $C_{4,i},\, i=4,\,5,\ldots,\,8$ are positive constants, and the pair $(y(Q_h),z(Q_h))\in \mathpzc{Y}\times \mathpzc{V}$ is a solution of the problem \eqref{intstate}--\eqref{intfinal} with $q^*=Q_h$.
\end{lemma}
\begin{proof} The proof of this lemma proceeds in the following steps.	We first deduce the bounds for the state and the co-state errors.
\noindent
\begin{enumerate}
\item[(i)]{\tt Bounds for the state and adjoint-state errors.}
Applying the energy norm definition \eqref{energynorm} and the triangle inequality, together with the bound \eqref{statebound}, yields
\begin{eqnarray}
\||y-Y_h\||_{L^2(I)}&\leq &\||y-y(Q_h)\||_{L^2(I)} +\||y(Q_h)-Y_h\||_{L^2(I)},\nonumber\\
&\leq &c_0c_{tr}c_a^{-1}\|q-Q_h\|_{L^2(I;L^2(\Gamma_{_N}))}+\||y(Q_h)-Y_h\||_{L^2(I)}, \label{stbound}
\end{eqnarray}
Employing the triangle inequality and the bound in Eq. \eqref{adjoint-statebound} with similar argument leads to the following adjoint-state error bound
\begin{eqnarray}
\||z-Z_h\||_{L^2(I)}&\leq&\||z-z(Q_h)\||_{L^2(I)}+\||z(Q_h)-Z_h\||_{L^2(I)},\nonumber\\
&\leq&c_0c_{tr}c_a^{-1}\|q-Q_h\|_{L^2(I;L^2(\Gamma_{_N}))}+\||z(Q_h)-Z_h\||_{L^2(I)}.\label{adjoint-statebound2}
\end{eqnarray}
Next we bound the co-control error. 
\item[(ii)]{\tt Bound for the co-control error.}  From the inequalities \eqref{optmallitycon1} and \eqref{redisopt11} with \eqref{adjoint-statebound2}, we find that
\begin{eqnarray}
\|\mu -\hat{\mu}_h\|_{L^2(I;L^2{(\Gamma_{N})})}&\leq & \alpha \|q-Q_h\|_{L^2(I;L^2{(\Gamma_{N})})}+ \alpha \|q_d-\hat{q}_{d,h}\|_{L^2(I;L^2{(\Gamma_{N})})}\nonumber\\&&+\|z-\tilde{Z}_h\|_{L^2(I;L^2(\Gamma_{_N}))}\nonumber \\
&\leq & (\alpha+c_0c_{tr}c_a^{-1}) \|q-Q_h\|_{L^2(I;L^2{(\Gamma_{N})})}+ \alpha \|q_d-\hat{q}_{d,h}\|_{L^2(I;L^2{(\Gamma_{N})})}\nonumber\\&&+\||z(Q_h)-Z_h\||_{L^2(I)}+c_0c_{tr}\||Z_h-\tilde{Z}_h\||_{L^2(I)}.\label{cocontbdd}
\end{eqnarray}
Next, we derive a bound for the control error.
\item[(iii)]{\tt Bound for the control error.} We observe that
\begin{eqnarray}
\alpha\int_0^T\|q-Q_h\|^2_{L^2(\Gamma_{_N})}dt&=&\int^T_0(\alpha (q-Q_h), q-Q_h)_{L^2(\Gamma_{_N})}dt\nonumber\\&=&\int^T_0(\alpha q, q-Q_h)_{L^2(\Gamma_{_N})}dt-\int^T_0(\alpha Q_h, q-Q_h)_{L^2(\Gamma_{_N})}dt.\nonumber
\end{eqnarray}
With the choice $\varphi=q-Q_h$, the first-order optimality condition \eqref{weakformcontrol} becomes
\begin{align}
&\alpha\int_0^T\|q-Q_h\|^2_{L^2(\Gamma_{_N})}dt~\leq ~\int^T_0(\alpha q_d-z, q-Q_h)_{L^2(\Gamma_{_N})}dt-\int^T_0(\alpha Q_h, q-Q_h)_{L^2(\Gamma_{_N})}dt\nonumber\\
&\hspace{0.5cm}=\int_0^T(\alpha(q_d-\hat{q}_{d,h}),q-Q_h)_{L^2(\Gamma_{_N})}dt-\int_0^T(\alpha(Q_h-\hat{q}_{d,h})+\tilde{Z}_h,q-Q_h)_{L^2(\Gamma_{_N})}dt\nonumber\\&\hspace{0.5cm}-\int_0^T(z-\tilde{Z}_h,q-Q_h)_{L^2(\Gamma_{_N})}dt\nonumber\\
&\hspace{0.5cm}\leq\int_0^T(\alpha(q_d-\hat{q}_{d,h}),q-Q_h)_{L^2(\Gamma_{_N})}dt+\int_0^T(\alpha(Q_h-\hat{q}_{d,h})+\tilde{Z}_h,\varphi_h-q)_{L^2(\Gamma_{_N})}dt\nonumber\\
&\hspace{0.5cm}-\int_0^T(\alpha(Q_h-\hat{q}_{d,h})+\tilde{Z}_h,\varphi_h-Q_h)_{L^2(\Gamma_{_N})}dt-\int_0^T(z-\tilde{Z}_h,q-Q_h)_{L^2(\Gamma_{_N})}dt.\label{4.14}
\end{align}
By virtue of inequality \eqref{redisfirstoptcond12}, we have $-\int_0^T(\alpha(Q_h-\hat{q}_{d,h})+\tilde{Z}_h,\varphi_h-Q_h)_{L^2(\Gamma_{_N})}dt\leq 0$. Consequently, the Eq. \eqref{4.14} becomes
\begin{align}
&\alpha\int_0^T\|q-Q_h\|^2_{L^2(\Gamma_{_N})}dt~\leq~ \int_0^T(\alpha(q_d-\hat{q}_{d,h}),q-Q_h)_{L^2(\Gamma_{_N})}dt\nonumber\\ &\hspace{0.5cm}+\int_0^T(\alpha(Q_h-\hat{q}_{d,h})+\tilde{Z}_h,\varphi_h-q)_{L^2(\Gamma_{_N})}dt+\int_0^T(\tilde{Z}_h-z,q-Q_h)_{L^2(\Gamma_{_N})}dt\nonumber\\
&:=~\mathcal{I}_1+\mathcal{I}_2+\mathcal{I}_3.\label{4.15}
\end{align}
Our next step is to estimate the terms $\mathcal{I}_1,\, \mathcal{I}_2$ and $\mathcal{I}_3$. Invoking the well-known inequality $ab\leq \frac{1}{2\epsilon}a^2+\frac{\epsilon}{2}b^2$ for all $a,\,b\in \mathbb{R}^{+}$, we can estimate $\mathcal{I}_1$ as follows

\begin{eqnarray}
&&\mathcal{I}_1 =\int_0^T(\alpha(q_d-\hat{q}_{d,h}),q-Q_h)_{L^2(\Gamma_{_N})}dt \nonumber\\
&&\hspace{0.5cm}\leq \frac{1}{2\epsilon}\|\alpha(q_d-\hat{q}_{d,h})\|^2_{L^2(I;L^2(\Gamma_{_N}))}+\frac{\epsilon}{2}\|q-Q_{h}\|^2_{L^2(I;L^2(\Gamma_{_N}))}.\label{4.16}
\end{eqnarray}
The term $\mathcal{I}_2$ is bounded by utilizing the inequality \eqref{assmption11}
\begin{eqnarray}
&&\mathcal{I}_2~=~\int_0^T(\alpha(Q_h-\hat{q}_{d,h})+\tilde{Z}_h,\varphi_h-q)_{L^2(\Gamma_{_N})}dt\nonumber\\
&&\hspace{0.5cm}\leq \frac{1}{2 \epsilon}C^2_{4,1}\int_0^T \sum_{E\in \mathcal{E}_{N,h}^i}h_E^2 \|\mathbf{n}_E\cdot \nabla(\alpha \left(Q_h-\hat{q}_{d,h}\right)+\tilde{Z}_h)\|^2_{L^2(E)}dt+\frac{\epsilon }{2}\|q-Q_{h}\|^2_{L^2(I;L^2(\Gamma_{_N}))}.\label{4.17}
\end{eqnarray}
Lastly, we estimate the final term
\begin{eqnarray}
&&\mathcal{I}_3~=~\int_0^T(\tilde{Z}_h-z,q-Q_h)_{L^2(\Gamma_{_N})}dt =\int_0^T(\tilde{Z}_h-z(Q_h),q-Q_h)_{L^2(\Gamma_{_N})}dt\nonumber\\
&&\hspace{1.0cm}+\int_0^T(z(Q_h)-z,q-Q_h)_{L^2(\Gamma_{_N})}dt~:=~\mathpzc{J}_1+\mathpzc{J}_2. \label{4.18}
\end{eqnarray}
We proceed to estimate the terms $\mathpzc{J}_1$ and $\mathpzc{J}_2$. To bound the term $\mathpzc{J}_1$, we employ the $\epsilon-$form of Young's inequality, which yields

\begin{eqnarray}
&&\mathpzc{J}_1~=~ \int_0^T(z(Q_h)-\tilde{Z}_h,q-Q_h)_{L^2(\Gamma_{_N})}dt \nonumber\\
&&\hspace{0.5cm}\leq \frac{1}{2 \epsilon}\|z(Q_h)-\tilde{Z}_h\|^2_{L^2(I;L^2(\Gamma_{_N}))}+ \frac{\epsilon}{2}\|q-Q_h\|^2_{L^2(I;L^2(\Gamma_{_N}))}\nonumber\\
&&\hspace{0.5cm}\leq \frac{c_{tr}^2c_0^2}{2\epsilon}\||z(Q_h)-Z_h\||^2_{L^2(I)}+\frac{c_{tr}^2c_0^2}{2\epsilon}\||Z_h-\tilde{Z}_h\||^2_{L^2(I)}+ \frac{\epsilon}{2}\|q-Q_h\|^2_{L^2(I;L^2(\Gamma_{_N}))}.\label{J_24.22}
\end{eqnarray}
Setting $\phi=z(Q_h)-z \in \mathpzc{V}$ in \eqref{erroreqstate}, and integrate  from $0$ to $T$, then integration by parts formula with $(y-y(Q_h))(0)=0=(z-z(Q_h))(T)$  yields the bound of the term $\mathpzc{J}_2$
\begin{eqnarray}
\mathpzc{J}_2&=&\int_0^T(z(Q_h)-z,q-Q_h)_{L^2(\Gamma_{_N})}dt\nonumber\\
&=&\int_0^T\Big(-\frac{\partial(z(Q_h)-z)}{\partial t}, y-y(Q_h) \Big)dt+\int_0^Ta( z(Q_h)-z,y-y(Q_h))dt\nonumber\\&=&\int_0^T\left(y(Q_h)-y, y-y(Q_h)\right)dt~=~-\int_0^T\|y-y(Q_h)\|^2_{L^2(\Omega)}dt~\leq~0.\label{J1bdd}
\end{eqnarray}
Combine \eqref{4.15}--\eqref{J1bdd}, and set $C^2_{4,2}=\max\{1, C_{4,1}^2, c_{tr}^2 c_0^2, c_{tr}^2\}$ and $\epsilon=\alpha/3$, we find that
\begin{eqnarray}
&&\int_0^T\|q-Q_h\|^2_{L^2(\Gamma_{_N})}dt~\leq~  \frac{9C^2_{4,2}}{2\alpha^2}\Big[\int_0^T \sum_{E\in \mathcal{E}_{N,h}^i}h_E^2 \|\mathbf{n}_E\cdot \nabla(\alpha \left(Q_h-\hat{q}_{d,h}\right)+\tilde{Z}_h)\|^2_{L^2(E)}dt\nonumber\\
&&~~~~~~~+\|\alpha(q_d-\hat{q}_{d,h})\|^2_{L^2(I;L^2(\Gamma_{_N}))}+\||Z_h-\tilde{Z}_h\||^2_{L^2(I)}+\||z(Q_h)-Z_h\||^2_{L^2(I)}\Big],\nonumber
\end{eqnarray}
and hence
\begin{eqnarray}
&&\|q-Q_h\|_{L^2(I;L^2(\Gamma_{_N}))}~\leq~  \frac{3C_{4,2}}{\sqrt{2} \alpha}\Big[\Big(\sum_{i=1}^{N_T}\int_{t_{i-1}}^{t_i}\sum_{E_N\in \mathcal{E}_{N,h}^i}(\eta^i_{q,E_N})^2dt\Big)^{1/2}\nonumber\\&&~~~~~+\|\alpha(q_d-\hat{q}_{d,h})\|_{L^2(I;L^2(\Gamma_{_N}))}+\||Z_h-\tilde{Z}_h\||_{L^2(I)}+\||z(Q_h)-Z_h\||_{L^2(I)}\Big],\label{4.23bdd}
\end{eqnarray}
Combining the bounds \eqref{stbound}--\eqref{cocontbdd} with \eqref{4.23bdd}, and set $C_{4,3}=3c_0c_{tr}c^{-1}_a+\alpha+1,\,C_{4,4}=\frac{3C_{4,2}}{\sqrt{2} \alpha}C_{4,3},\;C_{4,5}=C_{4,4}+1,\;C_{4,6}=c_0c_{tr}+C_{4,4},\;C_{4,7}=C_{4,5}+1,\;C_{4,8}=1$,  this completes the rest of the proof.
\end{enumerate}
\end{proof}
\begin{remark}
 A crucial assumption underlying our reliable error estimation is that $\mathpzc{Q}^i_{Nad,h} \subset \mathpzc{Q}_{Nad}$ and  \eqref{assmption11} hold, (cf.,, \cite[Lemma 3.1]{liuyan2001}).  If this inclusion property is not satisfied, additional work is needed to establish the validity of our error estimate (cf., \cite[Remark 3.2]{liuyan2001}).
\end{remark}    
We now proceed to establish the reliable error bounds for the discrepancies between the continuous and discrete solutions of the state and adjoint-state variables, expressed in terms of estimators and data oscillations.
\begin{lemma}[\bf{Intermediate error bounds}] \label{interrbdlm4.3}
Let $(Y_h,Z_h,Q_h)\in \mathpzc{Y}^i_h\times \mathpzc{V}^i_h\times \mathpzc{Q}_{Nad,h}^i,\, i=[1:N_T]$ and $(y(Q_h),z(Q_h))\in \mathpzc{Y}\times \mathpzc{V}$ be the solutions of  \eqref{redisoptstate12}--\eqref{redisfirstoptcond12} and \eqref{intstate}--\eqref{intfinal} with $q^*=Q_h$, respectively. Then there exists two positive constants $c_{1,10}$ and $c_{1,12}$ such that the following estimates hold:
\begin{eqnarray}
\||z(Q_h)-Z_h\||^2_{L^2(I)}&\leq& c_{1,10}\big[\varTheta_z^2+\varTheta_{z,T}^2+\eta_{z}^2 +\|y(Q_h)-\hat{Y}_h\|^2_{L^2(I;L^2(\Omega))}\big], \label{intadbound4.37}\\
\||y(Q_h)-Y_h\||_{L^2(I)}&\leq& c_{1,12}\big[\varTheta_y^2 +\varTheta_{y,T}^2 +\eta_{y}^2+  \|q-Q_h\|^2_{L^2(I;L^2(\Gamma_{N}))}\big].\label{intadstatebound4.49}
\end{eqnarray}
\end{lemma}
\begin{proof} We first proceed with the intermediate adjoint-state error bounds.

\noindent
\textbf{(i)}{ \tt Bound for intermediate adjoint-state error.} 
Letting $e_z=z(Q_h)-Z_h$ and choosing $\phi \in \mathpzc{V}$, we then compute
\begin{eqnarray}
&&-(e_{z,t}, \phi)+a_h(e_z, \phi)\nonumber\\&&~=~-(z_t(Q_h), \phi)+a_h(z(Q_h), \phi)-[-(Z_{h,t},\phi)+a_h(\tilde{Z_h}, \phi)+a_h(Z_h-\tilde{Z_h}, \phi)]\nonumber\\
&&~=~(y(Q_h)-y_d, \phi)+(r_{_N}, \phi)_{L^2(\Gamma_{_N})}+a_h(\tilde{Z_h}-Z_h, \phi)-[-(Z_{h,t},\phi-\phi_h)\nonumber\\&&~~~~~+a_h(\tilde{Z_h}, \phi-\phi_h)-(Z_{h,t},\phi_h)+a_h(\tilde{Z_h}, \phi_h)]\nonumber\\
&&~=~(y(Q_h)-y_d, \phi)+(r_{_N}, \phi)_{L^2(\Gamma_{_N})}+a_h(\tilde{Z_h}-Z_h, \phi)-[(\hat{Y}_h-\hat{y}_{d,h}, \phi_h)+(\hat{\mathcal{R}}_{N,h}, \phi_h)_{L^2(\Gamma_{_N})}]\nonumber\\&&~~~~~-[-(Z_{h,t},\phi-\phi_h)+a_h(\tilde{Z_h}, \phi-\phi_h)]\nonumber\\
&&~=~(y(Q_h)-\hat{Y}_h,\phi)+(\hat{y}_{d,h}-y_d,\phi)+(r_{_N}-\hat{\mathcal{R}}_{N,h}, \phi)+ a_h(\tilde{Z_h}-Z_h, \phi)\nonumber\\&&~~~~~+[(\hat{Y}_h-\hat{y}_{d,h}+Z_{h,t},\phi-\phi_h)-a_h(\tilde{Z_h}, \phi-\phi_h)+(\hat{\mathcal{R}}_{N,h}, \phi-\phi_h)_{L^2(\Gamma_{_N})}].\nonumber
\end{eqnarray}
Setting $\phi=e_z$ and $\psi=\phi-\phi_h$, we obtain
\begin{eqnarray}
&&-(e_{z,t}, e_z)+a_h(e_z, e_z)~=~(y(Q_h)-\hat{Y}_h,e_z)+(\hat{y}_{d,h}-y_d,e_z)+(r_{_N}-\hat{\mathcal{R}}_{N,h}, e_z)_{L^2(\Gamma_{_N})}\nonumber\\&&~~~~+ a_h(\tilde{Z_h}-Z_h, e_z)+[(\hat{Y}_h-\hat{y}_{d,h}+Z_{h,t},\psi)-a_h(\tilde{Z_h}, \psi)+(\hat{\mathcal{R}}_{N,h}, \psi)_{L^2(\Gamma_{_N})}],\nonumber
\end{eqnarray}
and hence 
\begin{eqnarray}
&&-\frac{1}{2}\|e_z\|^2+c_a\||e_z\||^2\nonumber\\&&~\leq~(y(Q_h)-\hat{Y}_h,e_z)+(\hat{y}_{d,h}-y_d,e_z)+(r_{_N}-\hat{\mathcal{R}}_{N,h}, e_z)_{L^2(\Gamma_{_N})}+ a_h(\tilde{Z_h}-Z_h, e_z)\nonumber\\&&~~~~+[(\hat{Y}_h-\hat{y}_{d,h}+Z_{h,t},\psi)+(\hat{\mathcal{R}}_{N,h}, \psi)_{L^2(\Gamma_{_N})}-\sum_{K\in \mathscr{T}_h^i}\{(\nabla \tilde{Z}_h, \nabla \psi)_K+(a_0\tilde{Z}_h,\psi)_K\}\nonumber\\&&~~~~+\sum_{E\in \mathcal{E}_{0,h}^i \cup \mathcal{E}_{D,h}^i }\{(\smean{\nabla \psi},\sjump{\tilde{Z}_h})_E+(\smean{\nabla \tilde{Z}_h}, \sjump{\psi})_E-\frac{\sigma_0}{h_E}(\sjump{\psi},\sjump{\tilde{Z}_h})_E\}].\label{4.26eq}
\end{eqnarray}
We now simplify the term $\sum_{K\in \mathscr{T}_h^i}(\nabla \tilde{Z}_h, \nabla \psi)_K$ using integration by parts formula, which gives
\begin{eqnarray}
\sum_{K\in \mathscr{T}_h^i}(\nabla \tilde{Z}_h, \nabla \psi)_K&=&\sum_{K\in \mathscr{T}_h^i}(-\Delta \tilde{Z}_h, \psi)_K+\sum_{E\in \mathcal{E}_{0,h}^i}\{(\smean{\nabla \tilde{Z}_h},\sjump{\psi})_E+(\sjump{ \nabla \tilde{Z}_h},\smean{\psi})_E\}\nonumber\\&&+\sum_{E\in \mathcal{E}_{D,h}^i}(\mathbf{n}_{E} \cdot \nabla \tilde{Z}_h,\psi)_E+\sum_{E\in \mathcal{E}_{N,h}^i}(\mathbf{n}_{E} \cdot \nabla \tilde{Z}_h,\psi)_E.\label{4.27eq}
\end{eqnarray}
Substituting \eqref{4.27eq} into \eqref{4.26eq} reveals that
\begin{align}
&-\frac{1}{2}\|e_z\|^2+c_a\||e_z\||^2\nonumber\\&\hspace{0.7cm}\leq~(y(Q_h)-\hat{Y}_h,e_z)+(\hat{y}_{d,h}-y_d,e_z)+(r_{_N}-\hat{\mathcal{R}}_{N,h}, e_z)_{L^2(\Gamma_{_N})}+ a_h(\tilde{Z_h}-Z_h, e_z)\nonumber\\&\hspace{1cm}+\sum_{K\in \mathscr{T}_h^i}[(\hat{Y}_h-\hat{y}_{d,h}+Z_{h,t}+\Delta \tilde{Z}_h-a_{0,h}\tilde{Z}_h,\psi)_K+((a_{0,h}-a_0)\tilde{Z}_h,\psi)_K]
\nonumber\\&\hspace{1cm}+\sum_{E\in \mathcal{E}_{N,h}^i }(\hat{\mathcal{R}}_{N,h}-\mathbf{n}_{E} \cdot \nabla \tilde{Z}_h, \psi)_{E}+\sum_{E\in \mathcal{E}_{D,h}^i }(\mathbf{n}_{E} \cdot \nabla \psi-\frac{\sigma_0}{h_E}\psi, \tilde{Z}_h)_E\nonumber\\&\hspace{1cm}+\sum_{E\in \mathcal{E}_{0,h}^i}[(\smean{\nabla \psi},\sjump{\tilde{Z}_h})_E-(\sjump{\nabla \tilde{Z}_h}, \smean{\psi})_E-\frac{\sigma_0}{h_E}(\sjump{\psi},\sjump{\tilde{Z}_h})_E].\label{4.28eq}
\end{align}
Thanks to the orthogonality relation, for any $\upsilon \in \mathpzc{V}_h^i \cap H^1(\Omega)$ that vanishes on $\Gamma_{_D}$,it holds that
\begin{align}
&0=a_h(e_z, \tilde{Z}_h-\upsilon)
~=~\sum_{K\in \mathscr{T}_h^i}[(\nabla e_z,\nabla (\tilde{Z}_h-\upsilon))_K+(a_0e_z,\tilde{Z}_h-\upsilon)_K]\nonumber\\&\hspace{0.5cm}-\sum_{E\in \mathcal{E}_{0,h}^i}[(\smean{\nabla (\tilde{Z}_h-\upsilon)}, \sjump{ e_z})_E+(\smean{\nabla e_z}, \sjump{\tilde{Z}_h})_E-\frac{\sigma_0}{h_E}(\sjump{ e_z},\sjump{\tilde{Z}_h})_E]\nonumber\\&\hspace{0.5cm}-\sum_{E\in \mathcal{E}_{D,h}^i}[(\mathbf{n}_E\cdot\nabla(\tilde{Z}_h-\upsilon), e_z)_E+(\mathbf{n}_E\cdot\nabla e_z, \tilde{Z}_h)_E+\frac{\sigma_0}{h_E}(Z_h, \tilde{Z}_h)_E].\label{4.29eq}
\end{align}
Incorporating \eqref{4.29eq} in \eqref{4.28eq}, and substituting $\psi=e_z-\phi_h$, where $\phi_h$ is constant on $\mathscr{T}_h^i$, we obtain
\begin{align}
&-\frac{1}{2}\|e_z\|^2+c_a\||e_z\||^2~\leq~(y(Q_h)-\hat{Y}_h,e_z)+(\hat{y}_{d,h}-y_d,e_z)+(r_{_N}-\hat{\mathcal{R}}_{N,h}, e_z)_{L^2(\Gamma_{_N})}\nonumber\\&\hspace{0.5cm}+ a_h(\tilde{Z_h}-Z_h, e_z)+\sum_{K\in \mathscr{T}_h^i}[(\hat{Y}_h-\hat{y}_{d,h}+Z_{h,t}+\Delta \tilde{Z}_h-a_{0,h}\tilde{Z}_h,\psi)_K+((a_{0,h}-a_0)\tilde{Z}_h,\psi)_K]
\nonumber\\
&\hspace{0.5cm}+\sum_{E\in \mathcal{E}_{N,h}^i }(\hat{\mathcal{R}}_{N,h}-\mathbf{n}_{E} \cdot \nabla \tilde{Z}_h, \psi)_{E}-\sum_{E\in \mathcal{E}_{D,h}^i }\frac{\sigma_0}{h_E}(\psi, \tilde{Z}_h)_E-\sum_{E\in \mathcal{E}_{0,h}^i}[\frac{\sigma_0}{h_E}(\sjump{\psi},\sjump{\tilde{Z}_h})_E+(\sjump{\nabla \tilde{Z}_h}, \smean{\psi})_E]\nonumber\\
&\hspace{0.5cm}+\sum_{K\in \mathscr{T}_h^i}[(\nabla e_z,\nabla (\tilde{Z}_h-\upsilon))_K+(a_0e_z,\tilde{Z}_h-\upsilon)_K]-\sum_{E\in \mathcal{E}_{0,h}^i}[(\smean{\nabla (\tilde{Z}_h-\upsilon)}, \sjump{ e_z})_E+\frac{\sigma_0}{h_E}(\sjump{Z_h},\sjump{\tilde{Z}_h})_E]\nonumber
\end{align}
\begin{align}
-\sum_{E\in \mathcal{E}_{D,h}^i}[(\mathbf{n}_E\cdot\nabla(\tilde{Z}_h-\upsilon), e_z)_E+\frac{\sigma_0}{h_E}(Z_h, \tilde{Z}_h)_E].\label{4.30eq} \hspace{4cm}
\end{align}
After integrating \eqref{4.30eq} from $t_{i-1}$ to $t_i$ with respect to time, and summing over all time steps from $1$ to $N_T$, with $e_z(t_N)=0$, we arrive at
\begin{eqnarray}
&&\frac{1}{2}\|e_z(0)\|^2+c_a\sum_{i=1}^{N_T} \int_{t_{i-1}}^{t_i}\||e_z\||^2dt~\leq\sum_{i=1}^{N_T}\int_{t_{i-1}}^{t_i}(y(Q_h)-\hat{Y}_h,e_z)dt+\sum_{i=1}^{N_T}\int_{t_{i-1}}^{t_i}(\hat{y}_{d,h}-y_d,e_z)dt\nonumber\\&&~~~~+\sum_{i=1}^{N_T}\int_{t_{i-1}}^{t_i}(r_{_N}-\hat{\mathcal{R}}_{N,h}, e_z)_{L^2(\Gamma_{_N})}dt+ \sum_{i=1}^{N_T}\int_{t_{i-1}}^{t_i}a_h(\tilde{Z_h}-Z_h, e_z)dt\nonumber\\
&&~~~~+\sum_{i=1}^{N_T}\int_{t_{i-1}}^{t_i}\sum_{K\in \mathscr{T}_h^i}[(\hat{Y}_h-\hat{y}_{d,h}+Z_{h,t}+\Delta \tilde{Z}_h-a_{0,h}\tilde{Z}_h,\psi)_K+((a_{0,h}-a_0)\tilde{Z}_h,\psi)_K]dt
\nonumber\\
&&~~~~+\sum_{i=1}^{N_T}\int_{t_{i-1}}^{t_i}\sum_{E\in \mathcal{E}_{N,h}^i }(\hat{\mathcal{R}}_{N,h}-\mathbf{n}_{E} \cdot \nabla \tilde{Z}_h, \psi)_{E}dt-\sum_{i=1}^{N_T}\int_{t_{i-1}}^{t_i}\sum_{E\in \mathcal{E}_{D,h}^i }\frac{\sigma_0}{h_E}(\psi, \tilde{Z}_h)_Edt\nonumber\\&&~~~~-\sum_{i=1}^{N_T}\int_{t_{i-1}}^{t_i}\sum_{E\in \mathcal{E}_{0,h}^i}[\frac{\sigma_0}{h_E}(\sjump{\psi},\sjump{\tilde{Z}_h})_E+(\sjump{\nabla \tilde{Z}_h}, \smean{\psi})_E]dt+\sum_{i=1}^{N_T}\int_{t_{i-1}}^{t_i}\sum_{K\in \mathscr{T}_h^i}[(\nabla e_z,\nabla (\tilde{Z}_h-\upsilon))_K\nonumber\\&&~~~~+(a_0e_z,\tilde{Z}_h-\upsilon)_K]dt-\sum_{i=1}^{N_T}\int_{t_{i-1}}^{t_i}\sum_{E\in \mathcal{E}_{0,h}^i}(\smean{\nabla (\tilde{Z}_h-\upsilon)}, \sjump{ e_z})_Edt
\nonumber\\&&~~~~-\sum_{i=1}^{N_T}\int_{t_{i-1}}^{t_i}\sum_{E\in \mathcal{E}_{D,h}^i}(\mathbf{n}_E\cdot\nabla(\tilde{Z}_h-\upsilon), e_z)_Edt-\sum_{i=1}^{N_T}\int_{t_{i-1}}^{t_i}\sum_{E\in \mathcal{E}_{0,h}^i}\frac{\sigma_0}{h_E}(\sjump{Z_h},\sjump{\tilde{Z}_h})_Edt
\nonumber\\&&~~~~-\sum_{i=1}^{N_T}\int_{t_{i-1}}^{t_i}\sum_{E\in \mathcal{E}_{D,h}^i}\frac{\sigma_0}{h_E}(Z_h, \tilde{Z}_h)_Edt~
:=~\sum_{j=1}^{8}\mathpzc{B}_j.\label{4.31eq}
\end{eqnarray}
We now proceed to estimate each term $\mathpzc{B}_j,\,j=1,\,\ldots,8$, separately. For any $\epsilon_i>0$, then, applying the $\epsilon$-form of Young's inequality, we obtain the bound for $\mathpzc{B}_1$ as
\begin{eqnarray}
\mathpzc{B}_1~\leq~\Big|\sum_{i=1}^{N_T}\int_{t_{i-1}}^{t_i}(y(Q_h)-\hat{Y}_h,e_z)dt\Big| ~\leq~
\frac{1}{2\epsilon_1}\|y(Q_h)-\hat{Y}_h\|^2_{L^2(I;L^2(\Omega))}+\frac{\epsilon_1}{2} \||e_z\||^2_{L^2(I)}. \nonumber%\label{4.355eq}
\end{eqnarray}
Analogously, we can bound the term $\mathpzc{B}_2$ as
\begin{eqnarray}
\mathpzc{B}_2~\leq~ \Big|\sum_{i=1}^{N_T}\int_{t_{i-1}}^{t_i}(\hat{y}_{d,h}-y_d,e_z)dt\Big| ~\leq~ \frac{1}{2\epsilon_2}\sum_{i=1}^{N_T}\int_{t_{i-1}}^{t_i}\|\hat{y}_{d,h}-y_d\|^2_{L^2(\Omega)}dt+\frac{\epsilon_2}{2}\||e_z\||^2_{L^2(I)}.\nonumber
\end{eqnarray}
Next we consider the boundary term $\mathpzc{B}_3$ and use of trace inequality leads to
\begin{eqnarray}
\mathpzc{B}_3=~ \sum_{i=1}^{N_T}\int_{t_{i-1}}^{t_i}(r_{_N}-\hat{\mathcal{R}}_{N,h}, e_z)_{L^2(\Gamma_{_N})}dt
\leq~ \frac{c^2_{tr}c_0^2}{2\epsilon_3}\sum_{i=1}^{N_T}\int_{t_{i-1}}^{t_i}\|r_{_N}-\hat{\mathcal{R}}_{N,h}\|^2_{L^2(\Gamma_{_N})}dt+\frac{\epsilon_3}{2} \||e_z\||^2_{L^2(I)}. \nonumber
\end{eqnarray}
Applying the continuity of the bi-linear form on the term $\mathpzc{B}_4$, we find that
\begin{eqnarray}
\mathpzc{B}_4~=~ \sum_{i=1}^{N_T}\int_{t_{i-1}}^{t_i}a_h(\tilde{Z_h}-Z_h, e_z)dt
~\leq~ \frac{2}{\epsilon_4}  \sum_{i=1}^{N_T}\int_{t_{i-1}}^{t_i}\||\tilde{Z_h}-Z_h\||^2dt+\frac{\epsilon_4}{2} \||e_z\||^2_{L^2(0,T)}.\nonumber
\end{eqnarray}
Next, we focus on estimating the term $\mathpzc{B}_5$ involving $\psi$ Eq. \eqref{4.36eq}. Utilizing the $\epsilon$-form of Young's inequality and rearranging the terms, we derive
\begin{align}
\mathpzc{B}_5&= \sum_{i=1}^{N_T}\int_{t_{i-1}}^{t_i}\sum_{K\in \mathscr{T}_h^i}\Big[(\hat{Y}_h-\hat{y}_{d,h}+Z_{h,t}+\Delta \tilde{Z}_h-a_{0,h}\tilde{Z}_h,\psi)_K+((a_{0,h}-a_0)\tilde{Z}_h,\psi)_K\Big]dt
\nonumber\\
&~~+\sum_{i=1}^{N_T}\int_{t_{i-1}}^{t_i}\sum_{E\in \mathcal{E}_{N,h}^i }(\hat{\mathcal{R}}_{N,h}-\mathbf{n}_{E} \cdot \nabla \tilde{Z}_h, \psi)_{E}dt-\sum_{i=1}^{N_T}\int_{t_{i-1}}^{t_i}\sum_{E\in \mathcal{E}_{D,h}^i }\frac{\sigma_0}{h_E}(\psi, \tilde{Z}_h)_Edt\nonumber\\&~~-\sum_{i=1}^{N_T}\int_{t_{i-1}}^{t_i}\sum_{E\in \mathcal{E}_{0,h}^i}\Big[\frac{\sigma_0}{h_E}(\sjump{\psi},\sjump{\tilde{Z}_h})_E+(\sjump{\nabla \tilde{Z}_h}, \smean{\psi})_E\Big]dt\nonumber\\
\leq &~ \frac{1}{2 \epsilon_5} \sum_{i=1}^{N_T}\int_{t_{i-1}}^{t_i}\Big[\sum_{K\in \mathscr{T}_h^i}h_K^2\big\{\|\hat{Y}_h-\hat{y}_{d,h}+Z_{h,t}+\Delta \tilde{Z}_h-a_{0,h}\tilde{Z}_h\|^2_{L^2(K)}+\|(a_{0,h}-a_0)\tilde{Z}_h\|^2_{L^2(K)}\big\}\Big]dt\nonumber\\
&~~+\frac{1}{2 \epsilon_6}\sum_{i=1}^{N_T}\int_{t_{i-1}}^{t_i}\sum_{E\in \mathcal{E}_{N,h}^i }h_E\|\hat{\mathcal{R}}_{N,h}-\mathbf{n}_{E} \cdot \nabla \tilde{Z}_h\|^2_{L^2(E)}dt+\frac{1}{2 \epsilon_7}\sum_{i=1}^{N_T}\int_{t_{i-1}}^{t_i}\sum_{E\in \mathcal{E}_{7D,h}^i }\frac{\sigma^2_0}{h_E}\| \tilde{Z}_h\|^2_{L^2(E)}dt\nonumber\\&~~+\frac{1}{2 \epsilon_8}\sum_{i=1}^{N_T}\int_{t_{i-1}}^{t_i}\sum_{E\in \mathcal{E}_{0,h}^i}\frac{\sigma_0^2}{h_E}\|\sjump{\tilde{Z}_h}\|^2_{L^2(E)}dt+\frac{1}{2 \epsilon_9}\sum_{i=1}^{N_T}\int_{t_{i-1}}^{t_i}\sum_{E\in \mathcal{E}_{0,h}^i}h_E\|\sjump{\nabla \tilde{Z}_h}\|^2_{L^2(E)}dt\nonumber\\&~~+\frac{\epsilon_5}{2} \sum_{i=1}^{N_T}\int_{t_{i-1}}^{t_i}\sum_{K\in \mathscr{T}_h^i}h_K^{-2}\|\psi\|^2_{L^2(K)} dt+\frac{\epsilon_6}{2} \sum_{i=1}^{N_T}\int_{t_{i-1}}^{t_i} \sum_{E\in \mathcal{E}_{N,h}^i }h_E^{-1}\|\psi\|^2_{L^2(E)}dt\nonumber\\&~~+\frac{\epsilon_7}{2} \sum_{i=1}^{N_T}\int_{t_{i-1}}^{t_i}\sum_{E\in \mathcal{E}_{D,h}^i }h_E^{-1}\|\psi\|^2_{L^2(E)}dt+\frac{\epsilon_8}{2} \sum_{i=1}^{N_T}\int_{t_{i-1}}^{t_i}\sum_{E\in \mathcal{E}_{0,h}^i}h_E^{-1}\|\sjump{\psi}\|^2_{L^2(E)}dt\nonumber\\&~~+ \frac{\epsilon_9}{2} \sum_{i=1}^{N_T}\int_{t_{i-1}}^{t_i}\sum_{E\in \mathcal{E}_{0,h}^i}  h^{-1}_E\|\smean{\psi}\|^2_{L^2(E)}dt.\nonumber \label{4.36eq}
\end{align}
Applying the approximation results \eqref{appinq4.24}--\eqref{appinq4.26} with $e=e_z$, we obtain
\begin{align}
\mathpzc{B}_5&\leq~ \frac{1}{2 \epsilon_5} \sum_{i=1}^{N_T}\int_{t_{i-1}}^{t_i}\Big[\sum_{K\in \mathscr{T}_h^i}h_K^2\big\{\|\hat{Y}_h-\hat{y}_{d,h}+Z_{h,t}+\Delta \tilde{Z}_h-a_{0,h}\tilde{Z}_h\|^2_{L^2(K)}\nonumber\\
&~~+\|(a_{0,h}-a_0)\tilde{Z}_h\|^2_{L^2(K)}\big\}\Big]dt+\frac{1}{2 \epsilon_6}\sum_{i=1}^{N_T}\int_{t_{i-1}}^{t_i}\sum_{E\in \mathcal{E}_{N,h}^i}h_E\|\hat{\mathcal{R}}_{N,h}-\mathbf{n}_{E} \cdot \nabla \tilde{Z}_h\|^2_{L^2(E)}dt\nonumber\\&~~+\frac{1}{2\epsilon_7}\sum_{i=1}^{N_T}\int_{t_{i-1}}^{t_i}\sum_{E\in \mathcal{E}_{D,h}^i }\frac{\sigma_0^2}{h_E}\| \tilde{Z}_h\|^2_{L^2(E)}dt+\frac{1}{2 \epsilon_8}\sum_{i=1}^{N_T}\int_{t_{i-1}}^{t_i}\sum_{E\in \mathcal{E}_{0,h}^i}\frac{\sigma_0^2}{h_E}\|\sjump{\tilde{Z}_h}\|^2_{L^2(E)}dt\nonumber\\&~~+\frac{1}{2 \epsilon_9}\sum_{i=1}^{N_T}\int_{t_{i-1}}^{t_i}\sum_{E\in \mathcal{E}_{0,h}^i}h_E\|\sjump{\nabla \tilde{Z}_h}\|^2_{L^2(E)}dt+\frac{\epsilon}{2}C \sum_{i=1}^{N_T}\int_{t_{i-1}}^{t_i}\sum_{K\in \mathscr{T}_h^i}\|\nabla e_z\|^2_{L^2(K)} dt,
\end{align}
with $\epsilon= \epsilon_5=\epsilon_6=\epsilon_7=\epsilon_8=\epsilon_9$. 
We proceed to examine the terms involving  $\tilde{Z}_h-\upsilon$, denoted as $\mathpzc{B}_6$

\begin{align}
\mathpzc{B}_6&~=~\sum_{i=1}^{N_T}\int_{t_{i-1}}^{t_i}\sum_{K\in \mathscr{T}_h^i}[(\nabla e_z,\nabla (\tilde{Z}_h-\upsilon))_K+(a_0e_z,\tilde{Z}_h-\upsilon)_K]dt
\nonumber\\
&\hspace{0.5cm}-\sum_{i=1}^{N_T}\int_{t_{i-1}}^{t_i}\sum_{E\in \mathcal{E}_{0,h}^i}(\smean{\nabla (\tilde{Z}_h-\upsilon)}, \sjump{ e_z})_Edt-\sum_{i=1}^{N_T}\int_{t_{i-1}}^{t_i}\sum_{E\in \mathcal{E}_{D,h}^i}(\mathbf{n}_E\cdot\nabla(\tilde{Z}_h-\upsilon), e_z)_Edt 
\nonumber\\&\hspace{0.5cm}\leq  \frac{\sigma_0^2}{2 \epsilon_{10}}\sum_{i=1}^{N_T}\int_{t_{i-1}}^{t_i}\sum_{K\in \mathscr{T}_h^i}\|\nabla (\tilde{Z}_h-\upsilon)\|^2_{L^2(K)}dt+\frac{1}{2 \epsilon_{11}} \sum_{i=1}^{N_T}\int_{t_{i-1}}^{t_i}\sum_{K\in \mathscr{T}_h^i}\|\tilde{Z}_h-\upsilon\|^2_{L^2(K)}]dt\nonumber\\&\hspace{0.5cm}+\frac{\sigma_0^2}{2 \epsilon_{12}} \sum_{i=1}^{N_T}\int_{t_{i-1}}^{t_i}\sum_{E\in \mathcal{E}_{0,h}^i} h_E\|\smean{\nabla (\tilde{Z}_h-\upsilon)}\|^2_{L^2(E)}dt\nonumber\\&\hspace{0.5cm}+\frac{\sigma_0^2}{2\epsilon_{13}} \sum_{i=1}^{N_T}\int_{t_{i-1}}^{t_i}\sum_{E\in \mathcal{E}_{D,h}^i}h_E\|\mathbf{n}_E\cdot\nabla(\tilde{Z}_h-\upsilon)\|^2_{L^2(E)}dt\nonumber\\&\hspace{0.5cm}+\frac{\epsilon_{10}}{2\sigma_0^2} \sum_{i=1}^{N_T}\int_{t_{i-1}}^{t_i}\sum_{K\in \mathscr{T}_h^i}\|\nabla e_z\|^2 dt+\frac{\epsilon_{11}}{2}\sum_{i=1}^{N_T}\int_{t_{i-1}}^{t_i}\sum_{K\in \mathscr{T}_h^i}a_0\|e_z\|^2_{L^2(K)}dt\nonumber\\&\hspace{0.5cm}+\frac{\epsilon_{12}}{2\sigma_0^2} \sum_{i=1}^{N_T}\int_{t_{i-1}}^{t_i}\sum_{E\in \mathcal{E}_{0,h}^i}h_E^{-1}\|\sjump{e_z}\|^2_{L^2(E)}dt+\frac{\epsilon_{13}}{2\sigma_0^2}\sum_{i=1}^{N_T}\int_{t_{i-1}}^{t_i}\sum_{E\in \mathcal{E}_{D,h}^i}h_E^{-1}\|e_z\|^2_{L^2(E)}dt.\label{4.3355eq}
\end{align}
To bound the terms involving $\nabla(\tilde{Z}_h-\upsilon)$ in \eqref{4.3355eq}, we utilize the trace inequality and the inverse inequality. Furthermore,  invoking \cite[Theorem 2.1]{pascal2007} we have
\begin{align}
\mathpzc{B}_6 &\leq~ \frac{1}{2}\Big(\frac{1}{\epsilon_{10}}+\frac{1}{\epsilon_{12}}+\frac{1}{\epsilon_{13}}\Big) \tilde{C}\sum_{i=1}^{N_T}\int_{t_{i-1}}^{t_i}\Big[\sum_{E\in \mathcal{E}_{0,h}^i}\frac{\sigma_0^2}{h_E}\|\sjump{\tilde{Z}_h}\|^2_{L^2(E)}+\sum_{E\in \mathcal{E}_{D,h}^i}\frac{\sigma_0^2}{h_E}\|\tilde{Z}_h\|^2_{L^2(E)}\Big]dt\nonumber\\
&\hspace{0.5cm}+ \frac{1}{2\epsilon_{11}}\tilde{C} \sum_{i=1}^{N_T}\int_{t_{i-1}}^{t_i}\Big[\sum_{E\in \mathcal{E}_{0,h}^i}h_E\|\sjump{\tilde{Z}_h}\|^2_{L^2(E)}+\sum_{E\in \mathcal{E}_{D,h}^i}h_E\|\tilde{Z}_h\|^2_{L^2(E)}\Big]dt\nonumber\\
&\hspace{0.5cm}+\frac{\epsilon_{10}}{2} \sum_{i=1}^{N_T}\int_{t_{i-1}}^{t_i}\sum_{K\in \mathscr{T}_h^i}\|\nabla e_z\|^2 dt+\frac{\epsilon_{11}}{2}\sum_{i=1}^{N_T}\int_{t_{i-1}}^{t_i}\sum_{K\in \mathscr{T}_h^i}a_0\|e_z\|^2_{L^2(K)}dt\nonumber\\
%\end{align}
%\begin{align}
	&\hspace{0.5cm}+\frac{\epsilon_{12}}{2\sigma_0^2}C \sum_{i=1}^{N_T}\int_{t_{i-1}}^{t_i}\sum_{K\in \mathscr{T}_h^i}\|\nabla e_z\|^2dt+\frac{\epsilon_{13}}{2\sigma_0^2}C\sum_{i=1}^{N_T}\int_{t_{i-1}}^{t_i}\sum_{K\in \mathscr{T}_h^i}\|\nabla e_z\|^2dt.\nonumber %\label{4.42eq}
\end{align}
An application of the Cauchy-Schwarz inequality yields the estimate for $\mathpzc{B}_7$ as
\begin{align}
\mathpzc{B}_7&\leq~ \Big|\sum_{i=1}^{N_T}\int_{t_{i-1}}^{t_i}\sum_{E\in \mathcal{E}_{0,h}^i}\frac{\sigma_0}{h_E}(\sjump{Z_h},\sjump{\tilde{Z}_h})_Edt
\Big|\nonumber\\
&\leq~ \frac{1}{2\sigma_0} \sum_{i=1}^{N_T}\int_{t_{i-1}}^{t_i}\Big[\sum_{E\in \mathcal{E}_{0,h}^i}\frac{\sigma_0^2}{h_E}\|\sjump{Z_h}\|^2_{L^2(E)}+\sum_{E\in \mathcal{E}_{0,h}^i}\frac{\sigma_0^2}{h_E}\|\sjump{\tilde{Z}_h}\|^2_{L^2(E)}\Big]dt.\nonumber %\label{4.43eq}
\end{align}
Likewise, we bound the term $\mathpzc{B}_8$ by
\begin{eqnarray}
\mathpzc{B}_8\leq \Big|\sum_{i=1}^{N_T}\int_{t_{i-1}}^{t_i}\sum_{E\in \mathcal{E}_{D,h}^i}\frac{\sigma_0}{h_E}(Z_h, \tilde{Z}_h)_Edt\Big|\leq\frac{1}{2\sigma_0} \sum_{i=1}^{N_T}\int_{t_{i-1}}^{t_i}\Big[\sum_{E\in \mathcal{E}_{D,h}^i}\frac{\sigma_0^2}{h_E}\|Z_h\|^2_{L^2(E)}+\sum_{E\in \mathcal{E}_{D,h}^i}\frac{\sigma_0^2}{h_E}\|\tilde{Z}_h\|^2_{L^2(E)}\Big]dt.\label{4.44eq} \nonumber
\end{eqnarray}
By consolidating the estimates for $\mathpzc{B}_1$ through $\mathpzc{B}_8$ with \eqref{4.31eq} and incorporating $\|\nabla e_z\|_{L^2(K)}\leq \||e_z\||$ and $a_0\|e_z\|_{L^2(K)}\leq \||e_z\||$, we arrive at
\begin{align}
&\frac{1}{2}\|e_z(0)\|^2+c_a\sum_{i=1}^{N_T}\int_{t_{i-1}}^{t_i}\||e_z\||^2dt~\leq~ \frac{1}{2\epsilon_1}\|y(Q_h)-\hat{Y}_h\|^2_{L^2(I;L^2(\Omega))}+ \frac{1}{2}\max\{\frac{1}{\epsilon_2}, \frac{c_0^2c_{tr}^2}{\epsilon_3},\frac{1}{\epsilon_5}\} \nonumber\\&~~\times\sum_{i=1}^{N_T}\int_{t_{i-1}}^{t_i}\Big[\sum_{K\in \mathscr{T}_h^i}\big(h_K^2\|(a_{0,h}-a_0)\tilde{Z}_h\|^2_{L^2(K)}+\|\hat{y}_{d,h}-y_d\|^2_{L^2(K)}\big)+\sum_{E\in \mathcal{E}_{N,h}^i}\|r_{_N}-\hat{\mathcal{R}}_{N,h}\|^2_{L^2(E)}\Big]dt\nonumber\\&~~+\frac{2}{ \epsilon_4}\, \sum_{i=1}^{N_T}\int_{t_{i-1}}^{t_i}\||\tilde{Z_h}-Z_h\||^2dt+ \frac{1}{2\epsilon_5} \sum_{i=1}^{N_T}\int_{t_{i-1}}^{t_i}\Big[\sum_{K\in \mathscr{T}_h^i}h_K^2\big\{\|\hat{Y}_h-\hat{y}_{d,h}+Z_{h,t}+\Delta \tilde{Z}_h-a_{0,h}\tilde{Z}_h\|^2_{L^2(K)}\big\}\Big]dt\nonumber\\
&~~+\frac{1}{2\epsilon_6}\sum_{i=1}^{N_T}\int_{t_{i-1}}^{t_i}\sum_{E\in \mathcal{E}_{N,h}^i }h_E\|\hat{\mathcal{R}}_{N,h}-\mathbf{n}_{E} \cdot \nabla \tilde{Z}_h\|^2_{L^2(E)}dt+\frac{1}{2}\max\{\frac{1}{\epsilon_8},\frac{1}{\epsilon_9}\}\sum_{i=1}^{N_T}\int_{t_{i-1}}^{t_i}\sum_{E\in \mathcal{E}_{0,h}^i}\Big[\frac{\sigma_0^2}{h_E}\|\sjump{\tilde{Z}_h}\|^2_{L^2(E)}\nonumber\\&~~+h_E\|\sjump{\nabla \tilde{Z}_h}^2_{L^2(E)}\Big]dt+c_{1,8}\sum_{i=1}^{N_T}\int_{t_{i-1}}^{t_i}\sum_{E\in \mathcal{E}_{D,h}^i }\Big[\frac{(\sigma_0+1)}{h_E}\| \tilde{Z}_h\|^2_{L^2(E)}+\frac{(\sigma_0+1)}{h_E}\|Z_h\|^2_{L^2(E)}\Big]dt\nonumber\\&~~
 +c_{1,8}\sum_{i=1}^{N_T}\int_{t_{i-1}}^{t_i}\sum_{E\in \mathcal{E}_{0,h}^i }\Big[\frac{(\sigma_0+1)}{h_E}\| \sjump{\tilde{Z}_h}\|^2_{L^2(E)}+\frac{(\sigma_0+1)}{h_E}\|\sjump{Z_h}\|^2_{L^2(E)}\Big]dt \nonumber\\&~~+ c_{1,9} \sum_{i=1}^{N_T}\int_{t_{i-1}}^{t_i}\Big[\sum_{E\in \mathcal{E}_{0,h}^i}h_E\|\sjump{\tilde{Z}_h}\|^2_{L^2(E)}+\sum_{E\in \mathcal{E}_{D,h}^i}h_E\|\tilde{Z}_h\|^2_{L^2(E)}\Big]dt
 +\frac{(5C+6)\epsilon}{2}\; \sum_{i=1}^{N_T}\int_{t_{i-1}}^{t_i}\||e_z\||^2dt,
\end{align}
where $c_{1,7}=c(\epsilon)\max\{1,c^2_{tr}c_0^2\},\quad c_{1,8}=\frac{1}{2}\max\{3c(\epsilon ),3c(\epsilon )c_{1,6},1\}, \quad c_{1,9}=c_{1,6}c(\epsilon)$. Setting $c_{1,10}=\frac{2}{c_a}\max\{4c(\epsilon),c_{1,7},c_{1,8},c_{1,9}\}$ with $\epsilon=\frac{c_a}{(5C+6)}$, and then kick-back arguments yields the inequality \eqref{intadbound4.37}. \\

\noindent
\textbf{(ii) \tt  Bound for the intermediate state error}. 
Analogous to the adjoint-state analysis, we define $e_y=y(Q_h)-Y_h$, $\phi \in \mathpzc{Y}$ and $\phi_h\in  \mathpzc{V}_h$. By invoking \eqref{redisoptstate12} and rearranging the terms, we find that  
\begin{align}
&(e_{y,t}, \phi)+a_h(e_y, \phi)~=~(y_t(Q_h), \phi)+a_h(y(Q_h), \phi)-[(Y_{h,t},\phi)+a_h(\hat{Y_h}, \phi)+a_h(Y_h-\hat{Y_h}, \phi)]
\nonumber\\&\hspace{0.6cm}=(f,\phi)+(q+g_N, \phi)_{L^2(\Gamma_{_N})}+a_h(\hat{Y_h}-Y_h, \phi)-[(Y_{h,t},\phi-\phi_h)+a_h(\hat{Y}_h,\phi-\phi_h)]\nonumber\\&\hspace{1.0cm}-[(Y_{h,t},\phi_h)+a_h(\hat{Y}_h,\phi_h)]\nonumber\\
&\hspace{0.6cm}=~(f-\hat{f}_h,\phi)+(g_N-\hat{\mathcal{G}}_{N,h}, \phi)_{L^2(\Gamma_{_N})}+(q-Q_h, \phi)_{L^2(\Gamma_{_N})}+a_h(\hat{Y_h}-Y_h, \phi)\nonumber\\&\hspace{1.0cm}+\big[\sum_{K\in\mathscr{T}_h^i}(\hat{f}_h-Y_{h,t},\phi-\phi_h)_K-a_h(\hat{Y}_h,\phi-\phi_h)+\sum_{E\in \mathcal{E}^i_{N,h}}(Q_h+\hat{\mathcal{G}}_{N,h},\phi-\phi_h)_E\nonumber\\&\hspace{1.0cm}-\sum_{E\in \mathcal{E}^i_{D,h}}(\hat{\mathcal{G}}_{D,h},\frac{\sigma_0}{h_E}\mathbf{n}_E \cdot \sjump{\phi_h}- \smean{\nabla \phi_h})_E\big].
\end{align}
Set $\phi=e_y$, $\psi=e_y-\phi_h$, then use of bilinear form \eqref{weakform} on the term $(\nabla \hat{Y}_h, \nabla \psi)_K$, and similar to the treatment of the adjoint-state in \eqref{4.27eq}. This yields 
\begin{align}
&(e_{y,t}, \phi)+a_h(e_y, \phi)~=~
(f-\hat{f}_h,\phi)+(g_N-\hat{\mathcal{G}}_{N,h}, \phi)_{L^2(\Gamma_{_N})}+(q-Q_h, \phi)_{L^2(\Gamma_{_N})}+a_h(\hat{Y_h}-Y_h, \phi)\nonumber\\&\hspace{0.6cm}+\sum_{K\in\mathscr{T}_h^i}\Big[(\hat{f}_h-Y_{h,t}+\Delta\hat{Y}_h-a_{0,h}\hat{Y}_h,\psi)_K+((a_{0,h}-a_0)\hat{Y}_h,\psi)_K\Big]\nonumber\\&\hspace{0.6cm}+\sum_{E\in \mathcal{E}^i_{N,h}}(Q_h+\hat{\mathcal{G}}_{N,h}-\mathbf{n}_E\cdot \nabla \hat{Y}_h,\psi)_E-\sum_{E\in \mathcal{E}^i_{D,h}}(\hat{\mathcal{G}}_{D,h},\frac{\sigma_0}{h_E}\mathbf{n}_E \cdot \sjump{\phi_h}- \smean{\nabla \phi_h})_E\nonumber
\end{align}
\begin{align}
&\hspace{0.6cm}-\sum_{E\in \mathcal{E}^i_{0,h}}\Big[(\smean{\nabla \hat{Y}_h},\sjump{\psi})_E+(\sjump{\nabla \hat{Y}_h},\smean{\psi})_E\Big]-\sum_{E\in  \mathcal{E}^i_{D,h}}(\mathbf{n}_E\cdot \nabla \hat{Y}_h,\psi)_E+\sum_{E\in \mathcal{E}^i_{0,h}\cup \mathcal{E}^i_{D,h}}\Big[(\smean{\nabla \hat{Y}_h},\sjump{\psi})_E\nonumber\\&\hspace{0.6cm}+(\smean{\nabla \psi},\sjump{\hat{Y}_h})_E\Big]-\sum_{E\in \mathcal{E}^i_{0,h}\cup \mathcal{E}^i_{D,h}}\frac{\sigma_0}{h_E}(\sjump{\hat{Y}_h},\sjump{\psi})_E~=~
(f-\hat{f}_h,\phi)+(g_N-\hat{\mathcal{G}}_{N,h}, \phi)_{L^2(\Gamma_{_N})}\nonumber\\&\hspace{0.6cm}+(q-Q_h, \phi)_{L^2(\Gamma_{_N})}+a_h(\hat{Y_h}-Y_h, \phi)+\sum_{K\in\mathscr{T}_h^i}\Big[(\hat{f}_h-Y_{h,t}+\Delta\hat{Y}_h-a_{0,h}\hat{Y}_h,\psi)_K\nonumber\\&\hspace{0.6cm}+((a_{0,h}-a_0)\hat{Y}_h,\psi)_K\Big]+\sum_{E\in \mathcal{E}^i_{N,h}}(Q_h+\hat{\mathcal{G}}_{N,h}-\mathbf{n}_E\cdot \nabla \hat{Y}_h,\psi)_E
+\sum_{E\in \mathcal{E}^i_{D,h}}\Big[(\smean{\nabla \psi},\sjump{\hat{Y}_h})_E \nonumber\\&\hspace{0.6cm}-\frac{\sigma_0}{h_E}(\sjump{\hat{Y}_h}, \sjump{\psi})_E\Big]+\sum_{E\in \mathcal{E}^i_{0,h}}\Big[(\smean{\nabla \psi}, \sjump{\hat{Y}_h})_E-\frac{\sigma_0}{h_E}(\sjump{\hat{Y}_h}, \sjump{\psi})_E-(\sjump{\nabla \hat{Y}_h},\smean{\psi})_E\Big]
\nonumber\\&\hspace{0.6cm}-\sum_{E\in \mathcal{E}^i_{D,h}}(\hat{\mathcal{G}}_{D,h},\frac{\sigma_0}{h_E}\mathbf{n}_E \cdot \sjump{\phi_h}- \smean{\nabla \phi_h})_E. \label{4.49eqb}
\end{align}
For any $\upsilon \in \mathpzc{V}_h^i \cap H^1(\Omega)$ satisfying the boundary condition $\upsilon|_{\Gamma_{_D}}=g_D$, we have
\begin{align}
&0~=~a_h(e_y, \hat{Y}_h-\upsilon) 
~=~\sum_{K\in \mathscr{T}_h^i}[(\nabla e_y,\nabla (\hat{Y}_h-\upsilon))_K+(a_0e_y,\hat{Y}_h-\upsilon)_K]-\sum_{E\in \mathcal{E}_{0,h}^i}[(\smean{\nabla (\hat{Y}_h-\upsilon)}, \sjump{ e_y})_E\nonumber\\
&\hspace{0.6cm}+(\smean{\nabla e_y}, \sjump{\hat{Y}_h})_E-\frac{\sigma_0}{h_E}(\sjump{ e_y},\sjump{\hat{Y}_h})_E]-\sum_{E\in \mathcal{E}_{D,h}^i}[(\mathbf{n}_E\cdot\nabla(\hat{Y}_h-\upsilon), e_y)_E+(\mathbf{n}_E\cdot\nabla e_y, \hat{Y}_h-g_D)_E\nonumber\\
&\hspace{0.6cm}-\frac{\sigma_0}{h_E}(e_y, \hat{Y}_h-g_D)_E].\label{4.29eqstate}
\end{align}
Following a similar approach as in the adjoint-state analysis \eqref{4.29eq}, we substitute \eqref{4.29eqstate} in \eqref{4.49eqb} and rearrange the terms, to obtain

\begin{align}
&(e_{y,t}, \phi)+a_h(e_y, \phi)~=~ (f-\hat{f}_h,\phi)+(g_N-\hat{\mathcal{G}}_{N,h}, \phi)_{L^2(\Gamma_{_N})}+(q-Q_h, \phi)_{L^2(\Gamma_{_N})}+a_h(\hat{Y_h}-Y_h, \phi)\nonumber\\&\hspace{0.6cm}+\sum_{K\in\mathscr{T}_h^i}\Big[(\hat{f}_h-Y_{h,t}+\Delta\hat{Y}_h-a_{0,h}\hat{Y}_h,\psi)_K+((a_{0,h}-a_0)\hat{Y}_h,\psi)_K\Big]+\sum_{E\in \mathcal{E}^i_{N,h}}(Q_h+\hat{\mathcal{G}}_{N,h}-\mathbf{n}_E\cdot \nabla \hat{Y}_h,\psi)_E%+\sum_{E\in \mathcal{E}^i_{D,h}}(\hat{\mathcal{G}}_{D,h}-\hat{Y}_h,\frac{\sigma_0}{h_E}\mathbf{n}_E \cdot \sjump{\psi}- \smean{\nabla \psi})_E
%+\sum_{E\in \mathcal{E}^i_{D,h}}\Big[(\smean{\nabla \psi},\sjump{\hat{Y}_h})_E-\frac{\sigma_0}{h_E}(\sjump{\hat{Y}_h}, \sjump{\psi})_E\Big]
\nonumber\\&\hspace{0.6cm}-\sum_{E\in \mathcal{E}^i_{0,h}}\Big[\frac{\sigma_0}{h_E}(\sjump{\hat{Y}_h}, \sjump{\psi})_E+(\sjump{\nabla \hat{Y}_h},\smean{\psi})_E\Big]
%\nonumber\\&&-\sum_{E\in \mathcal{E}^i_{D,h}}(\hat{\mathcal{G}}_{D,h},\frac{\sigma_0}{h_E}\mathbf{n}_E \cdot \sjump{\phi_h}- \smean{\nabla \phi_h})_E
+\sum_{K\in\mathscr{T}_h^i}\Big[(\nabla e_y,\nabla (\hat{Y}_h-\upsilon))_K+(a_0e_y,\hat{Y}_h-\upsilon)_K\Big]\nonumber\\&\hspace{0.6cm}-\sum_{E\in \mathcal{E}^i_{0,h}}(\smean{\nabla (\hat{Y}_h-\upsilon)},[[e_y]])_E-\sum_{E\in \mathcal{E}^i_{D,h}}(\mathbf{n}_E \cdot\nabla (\hat{Y}_h-\upsilon),e_y)_E+\sum_{E\in \mathcal{E}^i_{0,h}}\frac{\sigma_0}{h_E}(\sjump{e_y},\sjump{\hat{Y}_h})_E\nonumber\\&\hspace{0.6cm}-\sum_{E\in \mathcal{E}^i_{D,h}}\frac{\sigma_0}{h_E}(g_D-\mathcal{G}_{D,h},\hat{Y}_h)_E-\sum_{E\in \mathcal{E}^i_{D,h}}(\hat{\mathcal{G}}_{D,h}-\hat{Y}_h,\frac{\sigma_0}{h_E}\mathbf{n}_E \cdot \sjump{\phi_h}- \smean{\nabla \phi_h})_E\nonumber\\&\hspace{0.6cm}-\sum_{E\in \mathcal{E}^i_{D,h}}\frac{\sigma_0}{h_E}(\sjump{\hat{Y}_h},\sjump{\phi})_E+\sum_{E\in \mathcal{E}^i_{D,h}}(\smean{\nabla \phi},\sjump{\hat{Y}_h})_E.\nonumber 
\end{align}
Choosing $\phi=e_y$, we leverage the coercivity of the bilinear form. Integrating the resulting expression from $t_{i-1}$ to $t_i$ and summing over all time steps from $1$ to $N_T$, we arrive at
\begin{align}
&\frac{1}{2}\|e_y(T)\|^2_{L^2(\Omega)}+c_a\sum_{i=1}^{N_T}\int_{t_{i-1}}^{t_i}\||e_y\||^2dt~\leq~\frac{1}{2}\|e_y(0)\|^2_{L^2(\Omega)}+\sum_{i=1}^{N_T}\int_{t_{i-1}}^{t_i}(f-\hat{f}_h,e_y)dt\nonumber\\&\hspace{0.6cm}+\sum_{i=1}^{N_T}\int_{t_{i-1}}^{t_i}(g_N-\hat{\mathcal{G}}_{N,h}, e_y)_{L^2(\Gamma_{_N})}dt+\sum_{i=1}^{N_T}\int_{t_{i-1}}^{t_i}(q-Q_h, e_y)_{L^2(\Gamma_{_N})}dt+\sum_{i=1}^{N_T}\int_{t_{i-1}}^{t_i}a_h(\hat{Y_h}-Y_h, e_y)dt
\nonumber
\end{align}
\begin{align}
&\hspace{0.6cm}+\sum_{i=1}^{N_T}\int_{t_{i-1}}^{t_i}\sum_{K\in\mathscr{T}_h^i}\Big[(\hat{f}_h-Y_{h,t}+\Delta\hat{Y}_h-a_{0,h}\hat{Y}_h,\psi)_K+((a_{0,h}-a_0)\hat{Y}_h,\psi)_K\Big]dt\nonumber\\&\hspace{0.6cm}+\sum_{i=1}^{N_T}\int_{t_{i-1}}^{t_i}\sum_{E\in \mathcal{E}^i_{N,h}}(Q_h+\hat{\mathcal{G}}_{N,h}-\mathbf{n}_E\cdot \nabla \hat{Y}_h,\psi)_Edt-\sum_{i=1}^{N_T}\int_{t_{i-1}}^{t_i}\sum_{E\in \mathcal{E}^i_{0,h}}\Big[\frac{\sigma_0}{h_E}(\sjump{\hat{Y}_h},\sjump{\psi})_E\nonumber\\&\hspace{0.6cm}+(\sjump{\nabla \hat{Y}_h},\smean{\psi})_E\Big]dt
+\sum_{i=1}^{N_T}\int_{t_{i-1}}^{t_i}\sum_{K\in\mathscr{T}_h^i}\Big[(\nabla e_y,\nabla (\hat{Y}_h-\upsilon))_K+(a_0e_y,\hat{Y}_h-\upsilon)_K\Big]dt\nonumber\\&\hspace{0.6cm}-\sum_{i=1}^{N_T}\int_{t_{i-1}}^{t_i}\sum_{E\in \mathcal{E}^i_{0,h}}(\smean{\nabla (\hat{Y}_h-\upsilon)},[[e_y]])_Edt-\sum_{i=1}^{N_T}\int_{t_{i-1}}^{t_i}\sum_{E\in \mathcal{E}^i_{D,h}}(\mathbf{n}_E \cdot\nabla (\hat{Y}_h-\upsilon),e_y)_Edt\nonumber\\&\hspace{0.6cm}+\sum_{i=1}^{N_T}\int_{t_{i-1}}^{t_i}\sum_{E\in \mathcal{E}^i_{0,h}}\Big[\frac{\sigma_0}{h_E}(\sjump{Y_h},\sjump{\hat{Y}_h})_E+(\smean{\nabla \psi}, \sjump{\hat{Y}_h})_E\Big]dt\nonumber\\&\hspace{0.6cm}-\sum_{i=1}^{N_T}\int_{t_{i-1}}^{t_i}\sum_{E\in \mathcal{E}^i_{D,h}}\frac{\sigma_0}{h_E}(g_D-g_{D,h},\hat{Y}_h)_Edt-\sum_{i=1}^{N_T}\int_{t_{i-1}}^{t_i}\sum_{E\in \mathcal{E}^i_{D,h}} \frac{\sigma_0}{h_E}(\sjump{\hat{Y}_h},[[e_y]])_Edt\nonumber\\&\hspace{0.6cm}+\sum_{i=1}^{N_T}\int_{t_{i-1}}^{t_i}\sum_{E\in \mathcal{E}^i_{D,h}} (\smean{\nabla e_y},\sjump{\hat{Y}_h})_Edt+\sum_{i=1}^{N_T}\int_{t_{i-1}}^{t_i}\sum_{E\in \mathcal{E}^i_{D,h}}(\hat{\mathcal{G}}_{D,h}-\hat{Y}_h,\frac{\sigma_0}{h_E}\mathbf{n}_E \cdot \sjump{\phi_h}- \smean{\nabla \phi_h})_Edt \nonumber\\&\hspace{0.6cm}:=~\frac{1}{2}\|e_y(0)\|^2_{L^2(\Omega)}+\sum_{j=1}^{10}\mathcal{\bar{I}}_j.\label{4.51bbdb}
\end{align}
We now proceed to estimate each term $\mathcal{\bar{I}}_j,\,j=1,\ldots,10$, individually. Beginning with $\mathcal{\bar{I}}_1$, we apply the Young's inequality, to obtain 
\begin{eqnarray}
\mathcal{\bar{I}}_1~\leq~c(\epsilon)\sum_{i=1}^{N_T}\int_{t_{i-1}}^{t_i}\sum_{K\in \mathscr{T}_h^i}\|f-\hat{f}_h\|^2_{L^2(K)}dt+\epsilon\||e_y\||^2_{L^2(I)}.\nonumber
\end{eqnarray}
Analogously, we derive the bounds for the terms  $\mathcal{\bar{I}}_2$ and $\mathcal{\bar{I}}_3$ as
\begin{eqnarray}
\mathcal{\bar{I}}_2~\leq~c(\epsilon)\sum_{i=1}^{N_T}\int_{t_{i-1}}^{t_i}\sum_{E\in \mathcal{E}_{N,h} }\|g_N-\hat{\mathcal{G}}_{N,h}\|^2_{L^2(E)}dt+\epsilon \||e_y\||^2_{L^2(I)}.\nonumber
\end{eqnarray}
and 
\begin{eqnarray}
\mathcal{\bar{I}}_3~\leq~c(\epsilon)\sum_{i=1}^{N_T}\int_{t_{i-1}}^{t_i}\sum_{E\in \mathcal{E}_{N,h} }\|q-Q_h\|^2_{L^2(E)}dt+\epsilon \||e_y\||^2_{L^2(I)},\nonumber
\end{eqnarray}
respectively. By taking the advantage of continuity of the bilinear form and applying Young's inequality, we establish the bound for $\mathcal{\bar{I}}_4$ as
\begin{eqnarray}
\mathcal{\bar{I}}_4~\leq~c(\epsilon)\sum_{i=1}^{N_T}\int_{t_{i-1}}^{t_i}\||\hat{Y_h}-Y_h\||^2dt+\epsilon\||e_y\||^2_{L^2(I)}.\nonumber
\end{eqnarray}
Moving on to the terms involving $\psi$, we employ inequalities \eqref{appinq4.24}-\eqref{appinq4.26}, followed by the Cauchy-Schwarz inequality and subsequently Young's inequality, to deduce that

\begin{align}
&\mathcal{\bar{I}}_5~=~\sum_{i=1}^{N_T}\int_{t_{i-1}}^{t_i}\sum_{K\in\mathscr{T}_h^i}\Big[(\hat{f}_h-Y_{h,t}+\Delta\hat{Y}_h-a_{0,h}\hat{Y}_h,\psi)_K+((a_{0,h}-a_0)\hat{Y}_h,\psi)_K\Big]dt\nonumber\\&\hspace{0.6cm}+\sum_{i=1}^{N_T}\int_{t_{i-1}}^{t_i}\sum_{E\in \mathcal{E}^i_{N,h}}(Q_h+\hat{\mathcal{G}}_{N,h}-\mathbf{n}_E\cdot \nabla \hat{Y}_h,\psi)_Edt
-\sum_{i=1}^{N_T}\int_{t_{i-1}}^{t_i}\sum_{E\in \mathcal{E}^i_{0,h}}\Big[\frac{\sigma_0}{h_E}(\sjump{\hat{Y}_h},\sjump{\psi})_E\nonumber\\&\hspace{0.6cm}+(\sjump{\nabla \hat{Y}_h},\smean{\psi})_E\Big]dt
~\leq~c(\epsilon)\sum_{i=1}^{N_T}\int_{t_{i-1}}^{t_i}\sum_{K\in\mathscr{T}_h^i}h_K^2\Big[\|\hat{f}_h-Y_{h,t}+\Delta\hat{Y}_h-a_{0,h}\hat{Y}_h\|^2_{L^2(K)}\nonumber\\&\hspace{0.6cm}+\|(a_{0,h}-a_0)\hat{Y}_h\|^2_{L^2(K)}\Big]dt+\epsilon \sum_{i=1}^{N_T}\int_{t_{i-1}}^{t_i}\sum_{K\in\mathscr{T}_h^i}h_K^{-2}\|\psi\|^2_{L^2(K)}dt+\epsilon\sum_{i=1}^{N_T}\int_{t_{i-1}}^{t_i}\sum_{E\in \mathcal{E}^i_{N,h}}h_E^{-1}\|\psi\|^2_{L^2(E)}dt\nonumber\\&\hspace{0.6cm}+c(\epsilon)\sum_{i=1}^{N_T}\int_{t_{i-1}}^{t_i}\sum_{E\in \mathcal{E}^i_{N,h}}h_E\|Q_h+\hat{\mathcal{G}}_{N,h}-\mathbf{n}_E\cdot \nabla \hat{Y}_h\|^2_{L^2(E)}dt
+c(\epsilon)\sum_{i=1}^{N_T}\int_{t_{i-1}}^{t_i}\sum_{E\in \mathcal{E}^i_{0,h}}\Big[\frac{\sigma^2_0}{h_E}\|\sjump{\hat{Y}_h}\|^2_{L^2(E)}\nonumber\\&\hspace{0.6cm}+h_E\|\sjump{\nabla \hat{Y}_h}\|^2_{L^2(E)}\Big]dt+\epsilon\sum_{i=1}^{N_T}\int_{t_{i-1}}^{t_i}\sum_{E\in \mathcal{E}^i_{0,h}}\Big[h^{-1}_E\|\sjump{\psi}\|^2_{L^2(E)}+h^{-1}_E\|\smean{\psi}\|^2_{L^2(E)}\Big]dt.\nonumber
\end{align} 
We proceed to estimate the terms involving $(\hat{Y}_h-\upsilon)$ and $\nabla(\hat{Y}_h-\upsilon)$, employing a similar approach as in the analysis of $\mathscr{A}_6$. By applying the trace inequality, inverse inequality, and leveraging the result from \cite[Theorem 2.1]{pascal2007}, we obtain
\begin{align}
&\mathcal{\bar{I}}_6=\sum_{i=1}^{N_T}\int_{t_{i-1}}^{t_i}\sum_{K\in\mathscr{T}_h^i}\Big[(\nabla e_y,\nabla (\hat{Y}_h-\upsilon))_K+(a_0e_y,\hat{Y}_h-\upsilon)_K\Big]dt\nonumber\\&\hspace{0.6cm}-\sum_{i=1}^{N_T}\int_{t_{i-1}}^{t_i}\sum_{E\in \mathcal{E}^i_{0,h}}(\smean{\nabla (\hat{Y}_h-\upsilon)},[[e_y]])_Edt-\sum_{i=1}^{N_T}\int_{t_{i-1}}^{t_i}\sum_{E\in \mathcal{E}^i_{D,h}}(\mathbf{n}_E \cdot\nabla (\hat{Y}_h-\upsilon),e_y)_Edt\nonumber\\
 &\hspace{0.6cm}\leq \epsilon \sum_{i=1}^{N_T}\int_{t_{i-1}}^{t_i}\sum_{K\in \mathscr{T}_h^i}\|\nabla e_y\|^2 dt+ 3c(\epsilon)c_{1,6}\sum_{i=1}^{N_T}\int_{t_{i-1}}^{t_i}\Big[\sum_{E\in \mathcal{E}_{0,h}^i}h_E^{-1}\|\sjump{\hat{Y}_h}\|^2_{L^2(E)}+\sum_{E\in \mathcal{E}_{D,h}^i}h_E^{-1}\|\hat{Y}_h\|^2_{L^2(E)}\Big]dt\nonumber\\&\hspace{0.6cm}+\epsilon\sum_{i=1}^{N_T}\int_{t_{i-1}}^{t_i}\sum_{K\in \mathscr{T}_h^i}a_0\|e_y\|^2_{L^2(K)}dt+ c_{1,6}c(\epsilon) \sum_{i=1}^{N_T}\int_{t_{i-1}}^{t_i}\Big[\sum_{E\in \mathcal{E}_{0,h}^i}h_E\|\sjump{\hat{Y}_h}\|^2_{L^2(E)}+\sum_{E\in \mathcal{E}_{D,h}^i}h_E\|\hat{Y}_h\|^2_{L^2(E)}\Big]dt\nonumber\\&\hspace{0.6cm}+\epsilon \sum_{i=1}^{N_T}\int_{t_{i-1}}^{t_i}\Big[\sum_{E\in \mathcal{E}_{0,h}^i}h_E^{-1}\|\sjump{Y_h}\|^2_{L^2(E)}+\sum_{E\in \mathcal{E}_{D,h}^i}h_E^{-1} \|Y_h\|^2_{L^2(E)}\Big]dt.\nonumber
\end{align}
Applying Young's inequality, we establish a bound for the term $\mathcal{I}_7$, as
\begin{align}
&\mathcal{\bar{I}}_7~=~\sum_{i=1}^{N_T}\int_{t_{i-1}}^{t_i}\sum_{E\in \mathcal{E}^i_{D,h}}\frac{\sigma_0}{h_E}(g_D-g_{D,h},\hat{Y}_h)_Edt~\leq~ c(\epsilon)
\sum_{i=1}^{N_T}\int_{t_{i-1}}^{t_i}\sum_{E\in \mathcal{E}^i_{D,h}}\frac{\sigma_0}{h_E}\|g_D-g_{D,h}\|^2_{L^2(E)}dt\nonumber\\&\hspace{0.8cm}+\epsilon\sum_{i=1}^{N_T}\int_{t_{i-1}}^{t_i}\sum_{E\in \mathcal{E}^i_{D,h}}\frac{\sigma_0}{h_E}\|\hat{Y}_h\|^2_{L^2(E)}dt.
\end{align}
By invoking Young's inequality, we derive the bound of $\mathcal{\bar{I}}_9$ as 

\begin{align}
&\mathcal{\bar{I}}_8=|\mathcal{\bar{I}}_8|~=~\Big|\sum_{i=1}^{N_T}\int_{t_{i-1}}^{t_i}\sum_{E\in \mathcal{E}^i_{0,h}}\frac{\sigma_0}{h_E}(\sjump{\hat{Y}_h},[[e_y]])_Edt\Big|\nonumber\\&\hspace{0.6cm}\leq c(\epsilon) \sum_{i=1}^{N_T}\int_{t_{i-1}}^{t_i}\sum_{E\in \mathcal{E}^i_{0,h}}\frac{\sigma^2_0}{h_E}\|\sjump{\hat{Y}_h}\|^2_{L^2(E)}dt+\epsilon \sum_{i=1}^{N_T}\int_{t_{i-1}}^{t_i}\sum_{E\in \mathcal{E}^i_{0,h}}h_E^{-1}\|[[e_y]]\|^2_{L^2(E)}dt.\nonumber
\end{align}
Analogously, the bound of $\mathcal{\bar{I}}_{9}$ is given by
\begin{align}
&\mathcal{\bar{I}}_{9}=~ \sum_{i=1}^{N_T}\int_{t_{i-1}}^{t_i}\sum_{E\in \mathcal{E}^i_{D,h}} (\smean{\nabla e_y},\sjump{\hat{Y}_h})_Edt
\leq \frac{1}{2}\sum_{i=1}^{N_T}\int_{t_{i-1}}^{t_i}\Big[\sum_{E\in \mathcal{E}^i_{D,h}}h_E\|\smean{\nabla e_y}\|^2_{L^2(E)}\nonumber\\
&\hspace{0.6cm}+\sum_{E\in \mathcal{E}^i_{D,h}}h_E^{-1}\|\sjump{\hat{Y}_h}\|^2_{L^2(E)}\Big]dt\leq \frac{1}{2}\sum_{i=1}^{N_T}\int_{t_{i-1}}^{t_i}\sum_{E\in \mathcal{E}^i_{D,h}}h_E^{-1}\|\sjump{\hat{Y}_h}\|^2_{L^2(E)}dt+\frac{1}{2}\||e_y\||^2_{L^2(I)}.\nonumber
\end{align}
To conclude, we estimate the final term $\mathcal{\bar{I}}_{10}$. We recall that  $\phi_h$ is the best piecewise constant approximation of $e_p$. Specifically, choosing $\phi_h=T_{x_0}^m[\phi]=\sum_{|\ell | \leq m}\frac{1}{\ell !} \partial^{\ell}\phi(x_0)(x-x_0)^{\ell},$ with $x_0\in K\; \text{and}\; x\in \Omega$ (cf., [5]), we have $\||\phi_h\||= \||T_{x_0}^m[\phi]\||=\||e_y\||$ for $m=0$ and $\phi=e_y$. By employing the Cauchy-Schwarz inequality and Young's inequality, we deduce that

\begin{align}
&\mathcal{\bar{I}}_{10}~=~\sum_{i=1}^{N_T}\int_{t_{i-1}}^{t_i}\sum_{E\in \mathcal{E}^i_{D,h}}(\hat{\mathcal{G}}_{D,h}-\hat{Y}_h,\frac{\sigma_0}{h_E}\mathbf{n}_E \cdot \sjump{\phi_h}- \smean{\nabla \phi_h})_Edt\nonumber\\
&\hspace{0.6cm}\leq c(\epsilon)\sum_{i=1}^{N_T}\int_{t_{i-1}}^{t_i}\sum_{E\in \mathcal{E}^i_{D,h}}\frac{\sigma_0}{h_E}\|\hat{\mathcal{G}}_{D,h}-\hat{Y}_h\|^2_{L^2(E)}dt+\epsilon \||e_y\||^2_{L^2(I)}.\nonumber
\end{align}
Merging the estimates for $(\mathcal{\bar{I}}_{1})$--$(\mathcal{\bar{I}}_{10})$ in $\ref{4.51bbdb}$. Applying the inequalities \eqref{appinq4.24}--\eqref{appinq4.26} with $a_0\|e_p\|_{L^2(K)} \leq \||e_p\||$ and $\|\nabla e_p\|_{L^2(K)} \leq \||e_p\||$, it follows that  
\begin{align}
&\frac{1}{2}\|e_y(T)\|^2_{L^2(\Omega)}+c_a\sum_{i=1}^{N_T}\int_{t_{i-1}}^{t_i}\||e_y\||^2dt~\leq~c_{1,11}\Big\{\sum_{K\in \mathscr{T}_h^i}\|y_0-Y_{0,h}\|^2_{L^2(K)}\nonumber\\&\hspace{0.6cm}+  \sum_{i=1}^{N_T}\int_{t_{i-1}}^{t_i}\Big[\sum_{K\in \mathscr{T}_h^i}\big(h_K^2\|(a_{0,h}-a_0)\hat{Y}_h\|^2_{L^2(K)}+\|f-\hat{f}_h\|^2_{L^2(K)}\big)+\sum_{E\in \mathcal{E}_{N,h} }\|g_N-\hat{\mathcal{G}}_{N,h}\|^2_{L^2(E)}\nonumber\\&\hspace{0.6cm}+\sum_{E\in \mathcal{E}^i_{D,h}}\frac{\sigma_0}{h_E}\|g_D-g_{D,h}\|^2_{L^2(E)}\Big]dt+ \sum_{i=1}^{N_T}\int_{t_{i-1}}^{t_i}\sum_{E\in \mathcal{E}_{N,h} }\|q-Q_h\|^2_{L^2(E)}dt+\sum_{i=1}^{N_T}\int_{t_{i-1}}^{t_i}\||\hat{Y_h}-Y_h\||^2dt\nonumber\\&\hspace{0.6cm}+ \sum_{i=1}^{N_T}\int_{t_{i-1}}^{t_i}\Big[ \sum_{K\in\mathscr{T}_h^i}h_K^2\|\hat{f}_h-Y_{h,t}+\Delta\hat{Y}_h-a_{0,h}\hat{Y}_h\|^2_{L^2(K)}+\sum_{E\in \mathcal{E}^i_{N,h}}h_E\|Q_h+\hat{\mathcal{G}}_{N,h}-\mathbf{n}_E\cdot \nabla \hat{Y}_h\|^2_{L^2(E)}\nonumber\\&\hspace{0.6cm}+\sum_{E\in \mathcal{E}^i_{D,h}}\frac{\sigma_0}{h_E}\|\hat{\mathcal{G}}_{D,h}-\hat{Y}_h\|^2_{L^2(E)} \Big]dt+\sum_{i=1}^{N_T}\int_{t_{i-1}}^{t_i}\sum_{E\in \mathcal{E}^i_{0,h}}\Big[\frac{\sigma^2_0}{h_E}\|\sjump{\hat{Y}_h}\|^2_{L^2(E)}+h_E\|\sjump{\nabla \hat{Y}_h}\|^2_{L^2(E)}\Big]dt\nonumber\\&\hspace{0.6cm}+\sum_{i=1}^{N_T}\int_{t_{i-1}}^{t_i}\sum_{E\in \mathcal{E}^i_{0,h}}\Big[\frac{(\sigma_0+1)}{h_E}\|\sjump{\hat{Y}_h}\|^2_{L^2(E)}+\frac{(\sigma_0+1)}{h_E}\|\sjump{Y_h}\|^2_{L^2(E)} \Big]dt\nonumber\\
&\hspace{0.6cm}+\sum_{i=1}^{N_T}\int_{t_{i-1}}^{t_i}\sum_{E\in \mathcal{E}^i_{D,h}}\Big[\frac{(\sigma_0+1)}{h_E}\|\hat{Y}_h\|^2_{L^2(E)}+\frac{(\sigma_0+1)}{h_E}\|Y_h\|^2_{L^2(E)} \Big]dt+\sum_{i=1}^{N_T}\int_{t_{i-1}}^{t_i}\Big[\sum_{E\in \mathcal{E}_{0,h}^i}h_E\|\sjump{\hat{Y}_h}\|^2_{L^2(E)}\nonumber
\end{align}
\begin{align}
&+\sum_{E\in \mathcal{E}_{D,h}^i}h_E\|\hat{Y}_h\|^2_{L^2(E)}\Big]dt\Big\}+4 \epsilon (c+2)\||e_y\||^2_{L^2(I)}.\hspace{4cm}
\end{align}
By choosing $c_{1,12}$=$\frac{2}{c_a}c_{1,11}$ with $\epsilon=\frac{c_a}{8(c+2)}$, where $c_{1,11}=\max\{3c_{1,6}c(\epsilon),c(\epsilon),2\epsilon,1/2\}$, and applying kick-back arguments, we ultimately establish the desired inequality \eqref{intadstatebound4.49}, thus completing the proof.
\end{proof}
Combining the results from Lemmas \ref{lm4.1int}--\ref{interrbdlm4.3}, we can readily deduce the following result. The proof is straightforward and thus we omitted.
\begin{theorem}\label{4.3thm}
Let $(y,z,q)\in \mathpzc{Y} \times \mathpzc{V} \times \mathpzc{Q}_{Nad}$ and $(Y_h, Z_h,Q_h)\in \mathpzc{Y}^i_h \times \mathpzc{V}^i_h \times \mathpzc{Q}^i_{Nad,h},\;i\in[1:N],$ be the solutions of \eqref{weakformstate}--\eqref{weakformcontrol} and \eqref{redisoptstate12}--\eqref{redisfirstoptcond12}, respectively, and the co-control  variable $\mu$ and the discrete co-control variable $\mu_h$ as defined in \eqref{optmallitycon1} and \eqref{redisopt11}, respectively.  Assume that all the conditions of Lemma \ref{lm4.1int} is fulfilled. Then there exists a positive constant $C_{4,9}$ such that the following reliability type error estimation holds:
\begin{eqnarray}
&&\||y-Y_h\||_{L^2(I)}+\||z-Z_h\||_{L^2(I)}+\|q-Q_h\|_{L^2(I;L^2(\Gamma_{_N}))}+\|\mu-\hat{\mu}_h\|_{L^2(I;L^2(\Gamma_{_N}))}\nonumber\\&&~~\leq~ C_{4,9}\big(\varTheta_{yzq}+\Xi_{yz}+\Upsilon_{yzq} \big).
\end{eqnarray}
\end{theorem}
%%%%%%%%%%%%%%%%%%%%%%%%%%%%%%%%%%%%%%
\subsection{Efficient-type Aposteriori Error Estimation}
%%%%%%%%%%%%%%%%%%%%%%%%%%%%%%%%%%%%%%%%%%%%%%%%55
This section examines the efficiency of error estimators, taking into account data oscillations. We show that local error estimators are bounded above by the corresponding local errors, data oscillations, and active control contributions. To this end, we employ bubble functions, as in [35, 49], and introduce element bubble functions $\mathpzc{b}_K$ based on the barycentric coordinates $\lambda_j, j = 1; 2; 3,$ of each triangle $K$,
%%%%%%%%%%%%%%%%%%%%%%%%%
\begin{equation}
\mathpzc{b}_K=27\lambda_1\,\lambda_2\,\lambda_3. \label{eltbubblefunc}
\end{equation}
In contrast, edge bubble functions, denoted by $\mathpzc{b}_E$, are given by
\begin{equation}
\mathpzc{b}_E|_K=4\lambda_1\,\lambda_2, \quad \mathpzc{b}_E|_{K^E}=4\lambda^e_1\,\lambda^e_2,
\end{equation}
where $\lambda_1,\,\lambda_2 \;(\text{or} \lambda^e_1,\,\lambda^e_2)$ are the barycentric functions of the triangle $K$ (or $K^e$) on the edge $E\in K\cap K^e$. 
Additionally, the bubble functions possess the following properties
\begin{subequations}
\begin{align}
&\|\mathpzc{b}_K\|_{L^\infty(K)}~=~\max_{K} \mathpzc{b}_K~=~1,\quad  \mathpzc{b}_K\in H_0^1(K),\\
\text{and}\nonumber\\
&\|\mathpzc{b}_E\|_{L^\infty(E)}~=~\max_{\varOmega_K} \mathpzc{b}_E~=~1,\quad  \mathpzc{b}_K\in H_0^1(\omega_E).
\end{align}
\end{subequations}
Here $\omega_E$ denotes the patch of two elements sharing the edge $E$. We
recall from [49] that there exist constants $c_{5,i},\,i=1,\,2,\ldots,\,7$, which depend on the shape regularity of the triangulations  $\mathscr{T}^i_h,\,i=1,\,2,\ldots,\,N,$ such that the polynomials $\varPhi$ and $\varPsi$, defined on the elements and the patch $\omega_E$, satisfy for any element $K\in \mathscr{T}^i_h$, edge $E\in \mathcal{E}_h^i$,
\begin{subequations}
\begin{align}
	&\|\varPhi\|^2_{L^2(K)}~\leq~c_{5,1}\,(\varPhi,\varPhi \mathpzc{b}_K)_K, \quad  K\in \mathscr{T}^i_h,\label{4.66al} \vspace*{2cm} \\ 
	& \|\varPhi \mathpzc{b}_K\|_{L^2(K)}~\leq~c_{5,2}\, \|\varPhi\|_{L^2(K)}, \quad  K\in \mathscr{T}^i_h,\label{4.66bl} \vspace*{2cm} \\ 
	& \|\nabla (\varPhi \mathpzc{b}_K)\|_{L^2(K)}~\leq~c_{5,3}\,h_K^{-1} \|\varPhi\|_{L^2(K)}, \quad  K\in \mathscr{T}^i_h,\label{4.66cl} \vspace*{2cm}\\  &\|\varPsi\|^2_{L^2(E)}~\leq~c_{5,4}\,(\varPsi,\varPsi \mathpzc{b}_E)_E, \quad  E\in \mathcal{E}^i_h,\label{4.66dl} \vspace*{2cm}\\ 
	& \|\varPsi \mathpzc{b}_E\|_{L^2(E)}~\leq~c_{5,5}\, \|\varPsi\|_{L^2(E)}, \quad  E\in \mathcal{E}^i_h,\label{4.66el} \vspace*{2cm}\\ 
	& \|\varPsi \mathpzc{b}_E\|_{L^2(\omega_E)}~\leq~c_{5,6}\,h^{1/2}_E \|\varPsi\|_{L^2(E)}, \quad  \omega_E=K\cup K_E,\quad E=K\cap K_E,\label{4.66fl} \vspace*{2cm}\\ 
	& \|\nabla (\varPsi \mathpzc{b}_E)\|_{L^2(\omega_E)}~\leq~c_{5,7}\,h_E^{-1/2} \|\varPsi\|_{L^2(E)}, \quad  \omega_E=K\cup K_E,\quad E=K\cap K_E, \label{4.66gl} 
\end{align}
\end{subequations}
respectively. We now introduce the local energy norm $\||\cdot\||_{\mathcal{S}}$ on a set of elements $\mathcal{S}$ by 
\begin{equation}
|\| \phi |\|_{\mathcal{S}}:=\Big(\sum_{K \in \mathcal{S}}(\|\nabla \phi\|_{L^2(K)}^{2}+a_0\|\phi\|_{L^2(K)}^{2} )
+\sum_{\overset{E \in \mathcal{E}_{0,h} \cup \mathcal{E}_{D,h};}{E\in \partial K,\,K\subset \mathcal{S} }} (h_{E} \|\smean{\nabla \phi}\|_{L^2(E)}^{2}+\frac{\sigma_0}{h_{E}}\| \sjump{\phi} \|_{L^2(E)}^{2})\Big)^{1 / 2},\label{localenergynorm}\nonumber
\end{equation}
and the time-dependent energy-norm is denoted by
$$\||\phi\||_{L^2(I;\mathcal{S})}~=~\Big(\int_0^T |\| \phi |\|_{\mathcal{S}}^2\,dt\Big)^{1/2}.$$
The following lemma provides lower bounds on the residual-type error estimators $\eta_{y,K}^i$ and $\eta_{z,K}^i$.
\begin{lemma}\label{lm44eltestyz}
Let $(y,z,q)\in \mathpzc{Y} \times \mathpzc{V} \times \mathpzc{Q}_{Nad}$ and $(Y_h, Z_h,Q_h)\in \mathpzc{Y}^i_h \times \mathpzc{V}^i_h \times \mathpzc{Q}^i_{Nad,h},\;i\in[1:N],$ be the solutions of \eqref{weakformstate}--\eqref{weakformcontrol} and \eqref{redisoptstate12}--\eqref{redisfirstoptcond12}, respectively. Further, let the error estimators $\eta_{y,K}^i$, $\eta_{z,K}^i$ and the data oscillations $\varTheta_{y,K}$, $\varTheta_{z,K}$ be defined as in \eqref{spacest4.4} and \eqref{dataest4.17}. Then there are  positive constant $c_{5,8}$ such that the following estimates hold:
\begin{subequations}
\begin{align}
&(\eta_{y,K}^i)^2~\leq~c_{5,8}\Big(\||y-Y_h|\|_{L^2(I;K)}^2+\varTheta_{y,K}^2+h_K^2\|y_t-Y_{h,t}\|^2_{L^2(K)}\Big),\label{4.67aelt}\\
&(\eta_{z,K}^i)^2~\leq~c_{5,8}\Big(\||z-Z_h\||_{L^2(I;K)}^2+\varTheta_{z,K}^2+h_K^2\|z_t-Z_{h,t}\|^2_{L^2(K)}+h_K^2\|y-\hat{Y}_h\|_{L^2(I;L^2(K))}^2\Big), \label{4.67belt}
\end{align}
\end{subequations}
where $c_{5,8}=c_{5,1}^2\max\{c_{5,2}^2,c_{5,3}^2\}$.
\end{lemma}
\begin{proof}
Set $\xi= \varPhi \mathpzc{b}_K$ with $\varPhi=\hat{f}_h-Y_{h,t}+\Delta\hat{Y}_h-a_{0,h}\hat{Y}_h$ in \eqref{4.66al}, where $\mathpzc{b}_K$ is the bubble function defined as in \eqref{eltbubblefunc}. Then, using the fact that $(f-y_t+\Delta y-a_0y)|_K=0$ and $\xi|_{\Gamma_K}=0$, where $\Gamma_K$ refer the boundary of $K$, we have 
\begin{align}
h_K^2\|\varPhi\|^2_{L^2(K)} &\leq~c_{5,1}h_K^2(\hat{f}_h-Y_{h,t}+\Delta\hat{Y}_h-a_{0,h}\hat{Y}_h, \xi)_K\nonumber\\
&\leq~c_{5,1}h_K^2\big\{(\hat{f}_h-f, \xi)_K+(y_t-Y_{h,t}, \xi)_K+(\nabla (y-\hat{Y}_h), \nabla \xi)_K+((a_0-a_{0,h})\hat{Y}_h, \xi)_K\nonumber\\&~~~~~+(a_0(y-\hat{Y}_h), \xi)_K\big\}\nonumber\\
&\leq~\frac{c^2_{5,1}}{2}\max\{c_{5,2}^2,c_{5,3}^2\}\big\{h_K^2\|\hat{f}_h-f\|^2_{L^2(K)}+h_K^2\|y_t-Y_{h,t}\|^2_{L^2(K)}+\|\nabla (y-\hat{Y}_h)\|^2_{L^2(K)}\nonumber\\&~~~~~+h_K^2\|(a_0-a_{0,h})\hat{Y}_h\|^2_{L^2(K)}+h_K^2\|a_0(y-\hat{Y}_h)\|^2_{L^2(K)}\big\}+\frac{1}{2}h_K^2\|\varPhi\|^2_{L^2(K)}.\nonumber
\end{align} 
By invoking the kick-back argument, we derive the inequality stated in  \eqref{4.67aelt}.\\

We proceed similarly by setting $\xi= \varPhi \mathpzc{b}_K$ with $\varPhi=\hat{Y}_h-\hat{y}_{d,h}+Z_{h,t}+\Delta \tilde{Z}_h-a_{0,h}\tilde{Z}_h$ in \eqref{4.66al}. Using the facts that $(y-y_d+z_t+\Delta z-a_0z)$ and $\xi$ vanish on $K$ and the boundary $\Gamma_K$, respectively. Following analogous steps, we obtain
\begin{align}
(\eta_{z,K}^i)^2 
	&\leq~c_{5,1}h_K^2\big\{(\hat{Y}_h-y, \xi)_K+(y_d-\hat{y}_{d,h}, \xi)_K+(Z_{h,t}-z_t, \xi)_K+(\nabla (\tilde{z}_h-z), \nabla \xi)_K\nonumber\\&~~~~~+((a_0-a_{0,h})\tilde{Z}_h, \xi)_K+(a_0(z-\tilde{Z}_h), \xi)_K\big\}\nonumber\\
	&\leq~\frac{c^2_{5,1}}{2}\max\{c_{5,2}^2,c_{5,3}^2\}\big\{h_K^2\|\hat{Y}_h-y\|^2_{L^2(K)}+h_K^2\|y_d-\hat{y}_{d,h}\|^2_{L^2(K)}+h_K^2\|z_t-Z_{h,t}\|^2_{L^2(K)}\nonumber\\&~~~~~+\|\nabla (z-\tilde{Z}_h)\|^2_{L^2(K)}+h_K^2\|(a_0-a_{0,h})\tilde{Z}_h\|^2_{L^2(K)}+h_K^2\|a_0(z-\tilde{Z}_h)\|^2_{L^2(K)}\big\}+\frac{1}{2}(\eta_{z,K}^i)^2.\nonumber
\end{align} 
The kick-back argument ultimately yields the desired inequality \eqref{4.67belt}, which concludes the proof of the lemma.
\end{proof}
%%%%%%%%%%%%%%%%%%%%%%%%%%%%%%%%%%%%%%%%%%%%%%
\begin{lemma}
Let $(y,z,q)\in \mathpzc{Y} \times \mathpzc{V} \times \mathpzc{Q}_{Nad}$ and $(Y_h, Z_h,Q_h)\in \mathpzc{Y}^i_h \times \mathpzc{V}^i_h \times \mathpzc{Q}^i_{Nad,h},\;i\in[1:N],$ be the solutions of \eqref{weakformstate}--\eqref{weakformcontrol} and \eqref{redisoptstate12}--\eqref{redisfirstoptcond12}, respectively. Further, let the error estimators $\eta_{y,K}^i$, $\eta_{z,K}^i$ and the data oscillations $\varTheta^i_{y,K}$ $\varTheta^i_{z,K},\;i\in[1:N],$ be defined as in \eqref{spacest4.4} and \eqref{dataest4.17}. In addition, let $\omega_E=K \cup K_E$ be the union of any two elements, $(i.e., K,\;K_E$ such that $E=K\cap K_E)$, and the mesh shape regularity $(i.e., h_E/h_K\leq \beta_0$ with  $\beta_0>1)$.  Then there exist  positive constants $a_{5,1}$ and $a_{5,2}$ such that the following estimates hold, for each $t\in (t_{i-1},t_i],\; i\in[1:N]$,
\begin{subequations}
\begin{align}
&h_E \| \sjump{ \nabla \hat{Y}_h }\|^2_{L^2(E)}~\leq~a_{5,1}\Big(\||y-Y_h\||_{L^2(\omega_E)}^2+\sum_{K \in \omega_E}(\eta_{y,K}^i)^2+\sum_{K \in \omega_E} (\varTheta^i_{y,K})^2+\|\partial_t(y-Y_h)\|^2_{L^2(\omega_E)}\Big), \label{4.68al}\\
&h_E \| \sjump{ \nabla \tilde{Z}_h }\|^2_{L^2(E)}~\leq~a_{5,2} \Big(\||z-Z_h\||_{L^2(\omega_E)}^2+\sum_{K \in \omega_E}(\eta_{z,K}^i)^2+\sum_{K \in \omega_E}(\varTheta^i_{z,K})^2+\|\partial_t(z-Z_h)\|^2_{L^2(\omega_E)}\nonumber\\&~~~~~~~~~~~~~~~~~~~~~~~~~~+~\|y-Y_h\|_{L^2(\omega_E)}^2\Big), \label{4.68b}
\end{align}
\end{subequations}
where $a_{5,1}=a_{5,2}=2c^2_{5,4} \beta_1^2$ with $\beta_1=\max\{\beta_0c_{5,6},c_{5,7}\}$.
\end{lemma}
\begin{proof} Setting  $\xi= \varPsi \mathpzc{b}_E $ with $ \varPsi =\sjump{\nabla \hat{Y}_h}$ in \eqref{4.66dl}, and recalling that  $\sjump{\nabla y}=0$ on the interior edges, we integrate over the two elements comprising $\omega_E$ to obtain
\begin{align}
h_E \| \sjump{ \nabla \hat{Y}_h }\|^2_{L^2(E)}&\leq~	c_{5,4} h_E(\sjump{ \nabla \hat{Y}_h },\xi)_{L^2(E)}=c_{5,4} h_E(\sjump{ \nabla \hat{Y}_h }-\sjump{\nabla y},\xi)_{L^2(E)}\nonumber\\
&=~ c_{5,4} h_E(\Delta (\hat{Y}_h- y),\xi)_{L^2(\omega_E)}+c_{5,4} h_E(\nabla (\hat{Y}_h- y),\nabla \xi)_{L^2(\omega_E)}\nonumber\\
&=~ c_{5,4} h_E(\hat{f}_h-Y_{h,t}+\Delta \hat{Y}_h-a_{0,h}\hat{Y}_{h},\xi)_{L^2(\omega_E)}+c_{5,4} h_E(f-y_t-a_0y,\xi)_{L^2(\omega_E)}\nonumber\\&\hspace{0.5cm}+~c_{5,4} h_E(\nabla (\hat{Y}_h- y),\nabla \xi)_{L^2(\omega_E)}-c_{5,4} h_E(\hat{f}_h-Y_{h,t}-a_{0,h}\hat{Y}_{h},\xi)_{L^2(\omega_E)}\nonumber\\
&=~ c_{5,4} h_E(\hat{f}_h-Y_{h,t}+\Delta \hat{Y}_h-a_{0,h}\hat{Y}_{h},\xi)_{L^2(\omega_E)}+c_{5,4} h_E(f-\hat{f}_h,\xi)_{L^2(\omega_E)}\nonumber\\&\hspace{0.5cm}+~c_{5,4} h_E(Y_{h,t}-y_t,\xi)_{L^2(\omega_E)}+c_{5,4} h_E((a_{0,h}-a_0)\hat{Y}_{h},\xi)_{L^2(\omega_E)}\nonumber\\&\hspace{0.5cm}+~c_{5,4} h_E(a_0(\hat{Y}_{h}-y),\xi)_{L^2(\omega_E)}+c_{5,4} h_E(\nabla (\hat{Y}_h- y),\nabla \xi)_{L^2(\omega_E)}.\nonumber
\end{align}
By invoking the PDE \eqref{contstate} and utilizing the inequalities \eqref{4.66fl}, \eqref{4.66gl} with the mesh shape regularity condition $(h_E/h_K\leq \beta_0$ with $\beta_0>1$), we derive
\begin{align}
h_E \| \sjump{ \nabla \hat{Y}_h }\|^2_{L^2(E)}&\leq~ c_{5,4}  \max\{\beta_0c_{5,6},c_{5,7}\} h^{1/2}_E\|\varPsi\|_{L^2(E)} \times \big[h_K^2 \big( \|\hat{f}_h-Y_{h,t}+\Delta \hat{Y}_h-a_{0,h}\hat{Y}_{h}\|^2_{L^2(\omega_E)}\nonumber\\&\hspace{0.5cm}+ \|f-\hat{f}_h\|^2_{L^2(\omega_E)}+ \|Y_{h,t}-y_t\|^2_{L^2(\omega_E)} + \|(a_{0,h}-a_0)\hat{Y}_{h}\|_{L^2(\omega_E)}\nonumber\\&\hspace{0.5cm}+ \|a_0(\hat{Y}_{h}-y)\|^2_{L^2(\omega_E)} \big)+\| \nabla(\hat{Y}_h- y)\|^2_{L^2(\omega_E)}\big]^{1/2}.\nonumber
\end{align}
The Young's inequality followed by kick-back arguments directly yields the desired estimate \eqref{4.68al}.\\

Following a similar argument as before, we proceed to establish the inequality \eqref{4.68b}. Choosing $\zeta= \sjump{\nabla \tilde{Z}_h} \mathpzc{b}_E$ and utilize \eqref{4.66dl} to obtain
\begin{align}
h_E \| \sjump{ \nabla \tilde{Z}_h }\|^2_{L^2(E)}&\leq~ c_{5,4} h_E(\sjump{ \nabla \tilde{Z}_h },\zeta)_{L^2(E)}=c_{5,4} h_E(\sjump{ \tilde{Z}_h }-\sjump{\nabla z},\zeta)_{L^2(E)}\nonumber\\
&=~ c_{5,4} h_E(\Delta (\tilde{Z}_h- z),\zeta)_{L^2(\omega_E)}+c_{5,4} h_E(\nabla (\tilde{Z}_h- z),\nabla \zeta)_{L^2(\omega_E)}\nonumber\\
%&=~ c_{5,4} h_E(\hat{Y}_h-\hat{y}_{d,h}-Z_{h,t}+\Delta \tilde{Z}_h-a_{0,h}\tilde{Z}_{h},\zeta)_{L^2(\omega_E)}+c_{5,4} h_E(y-y_d+z_t-a_0z,\zeta)_{L^2(\omega_E)}\nonumber\\&~~~~+~c_{5,4} h_E(\nabla (\tilde{Z}_h- z),\nabla \zeta)_{L^2(\omega_E)}-c_{5,4} h_E(\hat{Y}_h-\hat{y}_{d,h}-Z_{h,t}-a_{0,h}\tilde{Z}_{h},\zeta)_{L^2(\omega_E)}\nonumber\\
&=~ c_{5,4} h_E(\hat{Y}_h-\hat{y}_{d,h}+Z_{h,t}+\Delta \tilde{Z}_h-a_{0,h}\tilde{Z}_{h},\zeta)_{L^2(\omega_E)}+c_{5,4} h_E(y-\hat{Y}_h,\zeta)_{L^2(\omega_E)}\nonumber\\&~~~~+~c_{5,4} h_E(\hat{y}_{d,h}-y_d,\zeta)_{L^2(\omega_E)}+c_{5,4} h_E(z_t-Z_{h,t},\zeta)_{L^2(\omega_E)}+c_{5,4} h_E((a_{0,h}-a_0)\tilde{Z}_{h},\zeta)_{L^2(\omega_E)}\nonumber\\&~~~~+~c_{5,4} h_E(a_0(\tilde{Z}_{h}-z),\zeta)_{L^2(\omega_E)}+c_{5,4} h_E(\nabla (\tilde{Z}_h- z),\nabla \zeta)_{L^2(\omega_E)}.\nonumber
\end{align}
Applying the inequalities \eqref{4.66fl} and \eqref{4.66gl}, we arrive at
\begin{align}
&h_E \| \sjump{ \nabla \tilde{Z}_h }\|^2_{L^2(E)}~\leq~ c_{5,4}  \max\{\beta_0c_{5,6},c_{5,7}\} h^{1/2}_E\|\sjump{ \nabla \tilde{Z}_h }\|_{L^2(E)}\nonumber\\&~~~~ \times \big[h_K^2 \big( \|\hat{Y}_h-\hat{y}_{d,h}+Z_{h,t}+\Delta \tilde{Z}_h-a_{0,h}\tilde{Z}_{h}\|^2_{L^2(\omega_E)}\nonumber\\&~~~~+ \|y-\hat{Y}_h\|^2_{L^2(\omega_E)}+ \|\hat{y}_{d,h}-y_d\|^2_{L^2(\omega_E)}+ \|z_t-Z_{h,t}\|^2_{L^2(\omega_E)} \nonumber\\&~~~~+ \|(a_{0,h}-a_0)\tilde{Z}_{h}\|_{L^2(\omega_E)}+ \|a_0(\tilde{Z}_{h}-z)\|^2_{L^2(\omega_E)} \big)+\| \nabla(\tilde{Z}_h- z)\|^2_{L^2(\omega_E)}\big]^{1/2}.\nonumber
\end{align}
An application of the Young's inequality yields the desired result, thus concluding the proof of \eqref{4.68b}.
\end{proof}
%%%%%%%%%%%%%%%%%%%%%%%%%
\begin{lemma}
Let $(y,z,q)\in \mathpzc{Y} \times \mathpzc{V} \times \mathpzc{Q}_{Nad}$ and $(Y_h, Z_h,Q_h)\in \mathpzc{Y}^i_h \times \mathpzc{V}^i_h \times \mathpzc{Q}^i_{Nad,h},\;i\in[1:N],$ be the solutions of \eqref{weakformstate}--\eqref{weakformcontrol} and \eqref{redisoptstate12}--\eqref{redisfirstoptcond12}, respectively. Further, for any $E_N\in \partial K,\, K\in \mathcal{T}^i_h, i\in [1:N]$, let the error estimators $\eta_{y,E_N}^i$, $\eta_{z,E_N}^i$ and the data oscillations $\varTheta^i_{y,K}$, $\varTheta^i_{y,E_N}$, $\varTheta^i_{z,K}, \;\varTheta^i_{z,E_N}$, $\;i\in[1:N],$ be defined as in \eqref{4.12yen}-\eqref{4.13en} and \eqref{dataest4.17}, respectively.  Then there exist  positive constants $a_{5,3}$ and $a_{5,4}$ such that the following estimates hold, for each $t\in (t_{i-1},t_i],\; i\in[1:N]$,
    
\begin{subequations}
\begin{align}
(\eta_{y,E_N}^i)^2&\leq~a_{5,3}\Big(\||y-Y_h\||_{L^2(K)}^2+(\eta^i_{y,K})^2+(\varTheta^i_{y,K})^2+(\varTheta^i_{y,E_N})^2+\|\partial_t(y-Y_h)\|^2_{L^2(K)}\nonumber\\&\hspace{0.5cm}+\|q-Q_h^2\|_{L^2(E)} \Big),\label{4.69ayen}\\
(\eta_{z,E_N}^i)^2&\leq~ a_{5,4}\Big(\||z-Z_{h}\||^2_{L^2(K)}+(\eta^i_{y,K})^2+(\varTheta^i_{z,K})^2+(\varTheta^i_{z,E_N})^2+\|\partial_t(z-Z_h)\|^2_{L^2(K)}\nonumber\\&\hspace{0.5cm}+\|y-\hat{Y}_h\|^2_{L^2(K)}\Big), \label{4.69bzen}
\end{align}
\end{subequations}
where $a_{5,3}=a_{5,4}= 2c^2_{5,4}\beta^2_2$ with $\beta_2=\max\{c_{5,5},\beta_0c_{5,6},c_{5,7} \}$.
\end{lemma}
\begin{proof}
First, we prove the estimate \eqref{4.69ayen}. To begin, we use the inequality \eqref{4.66dl} and  set $\xi=\varPsi  \mathpzc{b}_E$ with $\varPsi=\mathbf{n}_E\cdot \nabla \hat{Y}_h-Q_h-\hat{\mathcal{G}}_{N,h}$, to obtain
 \begin{align}
 &h_{E_N}\|Q_h+\hat{\mathcal{G}}_{N,h}-\mathbf{n}_E\cdot \nabla \hat{Y}_h\|^2_{L^2(E_N)}\leq~ c_{5,4} h_{E_N} (\mathbf{n}_E\cdot \nabla \hat{Y}_h-Q_h-\hat{\mathcal{G}}_{N,h}, \xi)_{L^2(E_N)} \nonumber\\&\hspace{0.6cm}=~ c_{5,4}h_{E_N}\big[ (\mathbf{n}_E\cdot \nabla (\hat{Y}_h-y), \xi)_{L^2(E_N)}+(q-Q_h,\xi)_{L^2(E_N)}\nonumber\\&\hspace{0.6cm}+(g_N-\hat{\mathcal{G}}_{N,h},\xi)_{L^2(E_N)}\big].
 \end{align}
%Since $\mathbf{n}_E\cdot \nabla y=q+g_N$ on the edge of the Neumann boundary.
Invoking Green's inequality on the element  $K$ that contains $E_N\subset K$, and incorporating equation \eqref{contstate}, we find
\begin{align}
&h_{E_N}\|Q_h+\hat{\mathcal{G}}_{N,h}-\mathbf{n}_E\cdot \nabla \hat{Y}_h\|^2_{L^2(E_N)}\leq~ c_{5,4}h_{E_N}\big[ (\hat{f}-Y_{h,t}+\Delta \hat{Y}_h-a_{0,h}\hat{Y}_h, \xi)_{L^2(K)}+(f-\hat{f}_h,\xi)_{L^2(K)}\nonumber\\&\hspace{0.6cm}+(Y_{h,t}-y_t,\xi)_{L^2(K)}+((a_{0,h}-a_0)\hat{Y}_{h},\xi)_{L^2(K)}+(a_0(\hat{Y}_{h}-y),\xi)_{L^2(K)}+c_{5,4} h_E(\nabla (\hat{Y}_h- y),\nabla \xi)_{L^2(K)}\nonumber\\&\hspace{0.6cm}+(q-Q_h,\xi)_{L^2(E_N)}+(g_N-\hat{\mathcal{G}}_{N,h},\xi)_{L^2(E_N)}\big].
\end{align}
By leveraging the bounds provided by inequalities \eqref{4.66el}-\eqref{4.66gl}, we deduce
\begin{align}
&h_{E_N}\|Q_h+\hat{\mathcal{G}}_{N,h}-\mathbf{n}_E\cdot \nabla \hat{Y}_h\|^2_{L^2(E_N)}\leq~ c_{5,4}h^{1/2}_{E_N}\max\{c_{5,5},\beta_0c_{5,6},c_{5,7} \}\|\varPsi\|_{L^2(E_N)}\big[ h_K^2\big(\|\hat{f}-Y_{h,t}\nonumber\\
&\hspace{0.4cm}+\Delta \hat{Y}_h-a_{0,h}\hat{Y}_h \|_{L^2(K)}^2+\|f-\hat{f}_h\|_{L^2(K)}^2\big)+\|\partial_t(Y_h-y)\|^2_{L^2(K)}+h_K^2\|(a_{0,h}-a_0)\hat{Y}_{h}\|_{L^2(K)}\nonumber\\&\hspace{0.4cm}+h_K^2\|a_0(\hat{Y}_{h}-y)\|^2_{L^2(K)}+\|\nabla (\hat{Y}_h- y)\|^2_{L^2(K)}+\|q-Q_h\|^2_{L^2(E_N)}+\|g_N-\hat{\mathcal{G}}_{N,h}\|^2_{L^2(E_N)}\big]^{1/2},	
\end{align}
and hence
\begin{align}
&h_{E_N}\|Q_h+\hat{\mathcal{G}}_{N,h}-\mathbf{n}_E\cdot \nabla \hat{Y}_h\|^2_{L^2(E_N)}\leq~ c_{5,4}h^{1/2}_{E_N}\max\{c_{5,5},\beta_0c_{5,6},c_{5,7} \}\|\varPsi\|_{L^2(E_N)}\times \big[\||y-Y_h\||^2_{L^2(K)}\nonumber\\	&\hspace{0.4cm}+ (\eta^i_{y,K})^2+(\varTheta_{y,K}^i)^2+(\varTheta^i_{y,E_N})^2+\|\partial_t(Y_h-y)\|^2_{L^2(K)}+\|q-Q_h\|^2_{L^2(E_N)}\big].
\end{align}
By invoking Young's inequality and employing a kickback argument, we establish the desired inequality \eqref{4.69ayen}.\\

We now turn our attention to establishing inequality \eqref{4.69bzen}. From the inequality \eqref{4.66dl}, and  by setting $\zeta= (\mathbf{n}_{E_N}\cdot \nabla \tilde{Z}_h-\hat{\mathcal{R}}_{N,h})  \mathpzc{b}_E$, analogously, we have
\begin{align}
&h_{E_N}\|\hat{\mathcal{R}}_{N,h}-\mathbf{n}_{E_N}\cdot \nabla \tilde{Z}_h\|^2_{L^2(E_N)}\leq~ c_{5,4} h_{E_N} (\mathbf{n}_{E_N} \cdot \nabla \tilde{Z}_h-\hat{\mathcal{R}}_{N,h}, \zeta)_{L^2(E_N)} \nonumber\\&\hspace{0.6cm}=~ c_{5,4}h_{E_N}\big[ (\mathbf{n}_E\cdot \nabla (\tilde{Z}_h-z), \zeta)_{L^2(E_N)}+(r_{N}-\hat{\mathcal{R}}_{N,h},\zeta)_{L^2(E_N)}\big].\nonumber
\end{align}  
Since $\mathbf{n}_{E_N} \cdot \nabla \tilde{Z}_h=r_{_N}$. Use of the Green's inequality on element $K$ with $E_N\subset K$, and \eqref{2.14adjoint-state} yields
\begin{align}
&h_{E_N}\|\hat{\mathcal{R}}_{N,h}-\mathbf{n}_{E_N}\cdot \nabla \tilde{Z}_h\|^2_{L^2(E_N)}\leq~ c_{5,4} h_E\big[(\hat{Y}_h-\hat{y}_{d,h}+Z_{h,t}+\Delta \tilde{Z}_h-a_{0,h}\tilde{Z}_{h},\zeta)_{L^2(K)}\nonumber\\&\hspace{0.4cm}+(y-\hat{Y}_h,\zeta)_{L^2(K)}+(\hat{y}_{d,h}-y_d,\zeta)_{L^2(K)}+(z_t-Z_{h,t},\zeta)_{L^2(K)}+((a_{0,h}-a_0)\tilde{Z}_{h},\zeta)_{L^2(K)}\nonumber\\&\hspace{0.4cm}+(a_0(\tilde{Z}_{h}-z),\zeta)_{L^2(K)}+(\nabla (\tilde{Z}_h- z),\nabla \zeta)_{L^2(K)}+(r_{N}-\hat{\mathcal{R}}_{N,h},\zeta)_{L^2(E_N)}\big].\nonumber
\end{align} 
Applying the inequalities \eqref{4.66el}-\eqref{4.66gl}, we get
\begin{align}
&h_{E_N}\|\hat{\mathcal{R}}_{N,h}-\mathbf{n}_{E_N}\cdot \nabla \tilde{Z}_h\|^2_{L^2(E_N)}\leq~ c_{5,4} \max\{c_{5,5},\beta_0c_{5,6},c_{5,7}\}\, h^{1/2}_E\,\|\hat{\mathcal{R}}_{N,h}-\mathbf{n}_{E_N}\cdot \nabla \tilde{Z}_h\|_{L^2(E_N)} \nonumber\\~&\hspace{0.4cm}\times \big[\||z-Z_{h}\||^2_{L^2(K)}+(\eta^i_{z,K})^2+(\varTheta^i_{z,K})^2+(\varTheta^i_{z,E_N})^2+\|\partial_t(z-Z_h)\|^2_{L^2(K)}+\|y-\hat{Y}_h\|^2_{L^2(K)}\big]^{1/2}.\nonumber
\end{align} 
The proof of inequality \eqref{4.69bzen} is finalized through an application of Young's inequality.
\end{proof}
Our next step is to obtain estimates for the control estimator, with a particular emphasis on controlling the discretization error associated with the control variables.
%%%%%%%%%%%%%%%%%%%%%%%%%%%%%%%%%%%%%%%%%%%%%
\begin{lemma}\label{lm47etqelt}
Let $(y,z,q)\in \mathpzc{Y} \times \mathpzc{V} \times \mathpzc{Q}_{Nad}$ and $(Y_h, Z_h,Q_h)\in \mathpzc{Y}^i_h \times \mathpzc{V}^i_h \times \mathpzc{Q}^i_{Nad,h},\;i\in[1:N],$ be the solution of \eqref{weakformstate}--\eqref{weakformcontrol} and \eqref{redisoptstate12}--\eqref{redisfirstoptcond12}, respectively. Further, for any $E_N\in \partial K,\, K\in \mathcal{T}^i_h, i\in [1:N]$, let the error estimators $\eta_{q,E_N}^i$ and  $\varTheta^i_{q}$ be defined as in  \eqref{4.14444qen}  and \eqref{4.17qen}, respectively.  Then there exist a positive constant $a_{5,5}$ such that the following estimates hold, for each $t\in (t_{i-1},t_i],\; i\in[1:N]$,
\begin{subequations}
\begin{align}
&(\eta_{q,E_N}^i)^2~\leq~a_{5,5}\Big(\|q-Q_h\|^2_{L^2(\Gamma_{_N})}+\||z-Z_h\||_{L^2(K)}^2+(\varTheta^i_{q})^2+\nonumber\\&\hspace{1.4cm}+h^2_{E_N} \|(\mathbf{n}_{E_N}\cdot \nabla(\alpha \left(Q_h-\hat{q}_{d,h}\right)+\tilde{Z}_h))\chi_{_{\mathpzc{A}_h}}\|^2_{L^2(E_N)}\Big),\label{4.74qen}
\end{align}
\end{subequations}
where $a_{5,5}=\max\{1,c_{inv}^2\}$, and $\mathpzc{A}_h$ denotes the active set  such that $\mathpzc{A}_h=\mathpzc{A}_{a,h}\cup \mathpzc{A}_{b,h}$.
\end{lemma}
\begin{proof}
From \eqref{4.14444qen}, we have
\begin{align}
(\eta^i_{q,E_N})^2 = h^2_{E_N} \|\mathbf{n}_{E_N}\cdot \nabla(\alpha \left(Q_h-\hat{q}_{d,h}\right)+\tilde{Z}_h)\|^2_{L^2(E_N)}.\nonumber
\end{align}
Using \eqref{2.28contcompcond} with indicator function $\chi$, we have $(\alpha \left(q-q_d\right)+z)\chi_{_{\mathcal{I}_h}}=0$. Combining this with the inverse inequality \eqref{inversinq}, we deduce
\begin{align}
(\eta^i_{q,E_N})^2& ~\leq h^2_{E_N} \|(\mathbf{n}_{E_N}\cdot \nabla(\alpha \left(Q_h-\hat{q}_{d,h}\right)+\tilde{Z}_h))\chi_{_{\mathcal{I}_h}}\|^2_{L^2(E_N)}+h^2_{E_N} \|(\mathbf{n}_{E_N}\cdot \nabla(\alpha \left(Q_h-\hat{q}_{d,h}\right)\nonumber\\& \vspace{0.4cm}+\tilde{Z}_h))\chi_{_{\mathpzc{A}_h}}\|^2_{L^2(E_N)}
\leq c_{inv}^2 \|(\alpha \left(Q_h-\hat{q}_{d,h}\right)+\tilde{Z}_h))\chi_{_{\mathcal{I}_h}}-(\alpha \left(q-q_{d,h}\right)+z)\chi_{_{\mathcal{I}_h}}\|^2_{L^2(E_N)}\nonumber\\&\vspace{0.4cm}+h^2_{E_N} \|(\mathbf{n}_{E_N}\cdot \nabla(\alpha \left(Q_h-\hat{q}_{d,h}\right)+\tilde{Z}_h))\chi_{_{\mathpzc{A}_h}}\|^2_{L^2(E_N)}
	\leq c_{inv}^2 \|(\alpha \left(Q_h-q\right)\|^2_{L^2(E_N)}\nonumber\\&\vspace{0.4cm}+\|z-\tilde{Z}_h\|^2_{L^2(E_N)}+\|(\alpha \left(q_d-\hat{q}_{d,h}\right)\|^2_{L^2(E_N)}+h^2_{E_N} \|(\mathbf{n}_{E_N}\cdot \nabla(\alpha \left(Q_h-\hat{q}_{d,h}\right)+\tilde{Z}_h))\chi_{_{\mathpzc{A}_h}}\|^2_{L^2(E_N)}\nonumber\\&\vspace{0.4cm}\leq \max\{1,c_{inv}^2\} \big[\|q-Q_h|^2_{L^2(E_N)}+\|z-\tilde{Z}_h\|^2_{L^2(E_N)}+(\varTheta^i_{q,E_N})^2\nonumber\\&\vspace{0.4cm}+h^2_{E_N} \|(\mathbf{n}_{E_N}\cdot \nabla(\alpha \left(Q_h-\hat{q}_{d,h}\right)+\tilde{Z}_h))\chi_{_{\mathpzc{A}_h}}\|^2_{L^2(E_N)}\big]. 
\end{align}
Thus, the desired result follows
\end{proof}
%%%%%%%%%%%%%%%%%%%%%%%%%%%%%%%%%
We are now well-positioned to establish a reliable estimate for $\Upsilon_{yzq}$, encompassing local error components, data oscillations, and active control contributions inherent to local error estimators. The forthcoming result underscores the efficiency of these estimators.
%%%%%%%%%%%%%%%%%%%%%%%%%%%%%%%%%%%%%%%%%%%%%%%%%%%%%%%%%%%5
\begin{theorem}\label{4.8thm}
Let $(y,z,q)\in \mathpzc{Y} \times \mathpzc{V} \times \mathpzc{Q}_{Nad}$ and $(Y_h, Z_h,Q_h)\in \mathpzc{Y}^i_h \times \mathpzc{V}^i_h \times \mathpzc{Q}^i_{Nad,h},\;i\in [1: N_T],$ be the solutions of \eqref{weakformstate}--\eqref{weakformcontrol} and \eqref{redisoptstate12}--\eqref{redisfirstoptcond12}, respectively. Further, let the estimator $\Upsilon_{yzq}$ and the data oscillation $\varTheta_{yzq}$ be defined as in \eqref{upsilon} and \eqref{vartheta}, respectively. We assume that all the conditions of  Lemma \ref{lm4.1int} are satisfied. Then there exists a positive constant $a_{5,6}$ such that the following inequality holds,
\begin{align}
\Upsilon_{yzq} &\leq~ a_{5,6}\Big(\|q-Q_h\|_{L^2(I; \Gamma_{_N})}+\||y-Y_h\||_{L^2(I)}+\||z-Z_h\||_{L^2(I)}+ \varTheta_{yzq} +\nonumber\\&~~~~~~~~~~~~~~~+ \Xi_{T_{yz}}+\sum_{i=1}^{N_T}\sum_{E\in \mathcal{E}^i_{N,h}} h_{E_N} \|(\mathbf{n}_{E_N}\cdot \nabla(\alpha \left(Q_h-\hat{q}_{d,h}\right)+\tilde{Z}_h))\chi_{_{\mathpzc{A}_h}}\|_{L^2(I_i;E_N)}\Big). \label{mainine475}
\end{align}
\end{theorem}
%%%%%%%%%%%%%%%%%%%%%%%
\begin{proof}
Availing the regularity of the exact solution $y$, we note that $\sjump{y}=0$. Further, utilizing the fact that $y=g_D$ on $\Gamma_D$ and manipulating the terms, one can easily deduce that  by using the definition of the energy norm
\begin{align}
\sum_{E \in \mathcal{E}^i_{0,h}} \frac{\sigma^2_0}{ h_{E}}\|\hat{Y}_h\|_{L^2(E)} &+ \sum_{E \in \mathcal{E}^i_{D,h}}\frac{\sigma^2_0}{ h_{E}}\|\hat{\mathcal{G}}_{D,h}-\hat{Y}_h\|_{L^2(E)}\leq~ a_{5,7}\Big(\||y-Y_h\||^2 +\varTheta^2_{y,T}\nonumber\\
&+\sum_{E \in \mathcal{E}^i_{D,h}} \frac{\sigma^2_0}{ h_{E}}  \|\hat{\mathcal{G}}_{D,h}-g_D\|^2_{L^2(E)}\Big), \label{yhbd476}
\end{align}
where $a_{5,7}$ denotes the positive constant. Similarly, use of the regularity of $z$ and energy norm leads to the following error estimates 
\begin{align}
\sum_{E \in \mathcal{E}^i_{0,h}} \frac{\sigma^2_0}{ h_{E}}\|\tilde{Z}_h\|_{L^2(E)} &+ \sum_{E \in \mathcal{E}^i_{D,h}}\frac{\sigma^2_0}{ h_{E}}\|\tilde{Z}_h\|_{L^2(E)}\leq~ a_{5,8}\Big(\||z-Z_h\||^2 +\varTheta^2_{z,T}\Big), \label{zhbd477}
\end{align}
where $a_{5,8}$ is the positive constant. Combining all estimates from the Lemmas \ref{lm44eltestyz}-\ref{lm47etqelt} together with the estimates \eqref{yhbd476}-\eqref{zhbd477}, then integrating over the time domain and setting the maximum $a_{5,6}$ over all the constants involved in the estimates, we obtain the desired inequality \eqref{mainine475}.
\end{proof}
%%%%%%%%%%%%%%%%%%%%%%%%%%%
%%%%%%%%%%%%%%%%%%%%%%%%%%%%%%%%%%%%%%%%%%%%%%%%%%%
\section{Numerical assessment} \label{section5555}
This section presents a numerical experiment designed to validate the efficacy of the error estimators developed in the previous section. Note that the results presented in Theorems \eqref{4.3thm} and \eqref{4.8thm} underscore the reliability and efficiency of the estimator $\Upsilon_{yzq}$. However, the estimator $\eta_q$ reflects the approximation error of the control and may not effectively guide the localization of refinement in certain significant scenarios. Consequently, enhancing this significant step numerically becomes imperative. Hence, in our numerical tests, we adopt
\begin{equation}
\bar{\eta}_q=\sum_{i=1}^{N_T}\sum_{E\in \mathcal{E}^i_{N,h}} h_{E_N} \|(\mathbf{n}_{E_N}\cdot \nabla(\alpha \left(Q_h-\hat{q}_{d,h}\right)+\tilde{Z}_h))\chi_{_{\mathcal{I}_h}}\|_{L^2(I_i;E_N)}\nonumber
\end{equation}
as a control indicator instead of employing $\eta_q$ as defined in equation \eqref{4.19cq}. Nevertheless, the numerical approximation of the characteristic function $\chi_{_{\mathcal{I}_h}}$ is given by:
\begin{equation}
\chi_{_{\mathcal{I}_h}} \approx \frac{(Q_h-\hat{q}_{a,h})\times (\hat{q}_{b,h}-Q_h)}{h^\mu+(Q_h-\hat{q}_{a,h})\times (\hat{q}_{b,h}-Q_h)}
\end{equation}
with $\mu>0$ \cite{liliuyan2000comp}. The formula employed for determining the Effectivity Index (Eff. Index) is given by:
\begin{equation}\label{5.3effind}
\text{Eff. Index} := \frac{\eta_y+\eta_z+\eta_q}{ \||y-Y_h\||_{L^2(I)}+\||z-Z_h\||_{L^2(I)}+\|q-Q_h\|_{L^2(I;L^2(\Gamma_{_N}))}+\|\mu-\tilde{\mu}_h\|_{L^2(I;L^2( \Gamma_{_N} ))}}.
\end{equation}
Additionally, the projection of the control is defined as
\begin{equation}\label{5.4proj}
P_{[q_a,q_b]}(\phi):= \max\{q_a,\min\{q_b,\phi\}\}.
\end{equation}
For the purpose of marking strategy, we define the following conditions, for $i\in [1:N]$:
\begin{subequations}\label{5.2Numsbulk}
\begin{eqnarray}
\theta \sum_{K\in \mathscr{T}_h^i} \big\{(\eta^i_{y,K})^2+(\eta^i_{z,K})^2\big\}&\leq& \sum_{K\in \mathcal{M}_K^i} \big\{(\eta^i_{y,K})^2+(\eta^i_{z,K})^2\big\},\\
\theta \sum_{E\in \mathcal{E}_h^i} \big\{(\eta^i_{y,E})^2+(\eta^i_{z,E})^2+(\eta^i_{q,E})^2\big\}&\leq& \sum_{E\in \mathcal{M}_E^i} \big\{(\eta^i_{y,E})^2+(\eta^i_{z,E})^2+(\eta^i_{q,E})^2\big\},
\end{eqnarray}
\end{subequations}
where $\theta$ be a given parameter with $0 < \theta < 1$. Moreover,   $\mathcal{M}^i_K$ and $\mathcal{M}^i_E$ are the initial subsets chosen from $\mathscr{T}^i_h$ and $\mathcal{E}^i_h$, respectively, for refinement.

For the computations, we assume that all associated constants are one. Additionally, we conduct the computations using FreeFem++ software \cite{Hetch}, while the figures are generated by implementing Matlab within the FreeFem++ environment. In the adaptive procedure, we use the following space-time adaptive algorithm for the SIPG discretization of the optimization problem \eqref{contfunc}–\eqref{contspace}:
 Given parameters $\delta_1\in (0,1), \; \delta_2>1$, $\Lambda_1 >0$, $\Lambda_2\in (0, \Lambda_1)$, the space tolerance $\epsilon_{space}$ and the time tolerance $ \epsilon_{time}$. 
%%%%%%%%%%%%%%%%%%%%%%%%%%%%%%%%%%%
\begin{algorithm}[H] 
\caption{\bf{: \tt Space-time Adaptive Finite Element} \vspace{0.15cm}}\label{algo5.1} 
\begin{flushleft}\vspace{.15cm}
\noindent
\textbf{Step ~1.}  {\bf \texttt{Initialize}:} \hspace{.05cm} Choose the initial conforming mesh $\mathscr{T}_h^0$  and set counter $n=0$. \vspace{.2cm}\\

\noindent
\textbf{Step ~2.}~~ {\bf \texttt{Solve}:} \hspace{.05cm} Evaluate the discrete problem on $\mathscr{T}_h^n$. \vspace{.2cm}\\

\noindent
\textbf{Step~3.} {\bf \texttt{Estimate}:} \hspace{.05cm} Compute the error estimators \vspace{.1cm}\\

\hspace{2.5cm} {\bf \texttt{Check:}} \hspace{.05cm} Sum of temporal error estimators $\geq \Lambda_1 \frac{\epsilon_{time}}{T}$ \vspace{.1cm}\\
\hspace{4cm} {\bf \texttt{Do}}\hspace{.05cm}  $ k_n:=\delta_1 k_{n-1},\;\; t_n:=t_{n-1}+k_{n}$ \vspace{.1cm}\\
\hspace{2.5cm} {\bf \texttt{Check:}}\hspace{.05cm}  If \eqref{5.2Numsbulk} not satisfied. Then {\bf Go To} { \bf \texttt{Step} 4}.  \vspace{.2cm}\\

\textbf{Step ~4.} {\bf \texttt{Mark}:}\hspace{.05cm} Mark the subsets  $\mathcal{M}^n_K\subset \mathscr{T}^n_h$ and $\mathcal{M}^n_E\subset \mathcal{E}^n_h$ such that  $\mathcal{M}^n_K$ and $\mathcal{M}^n_E$ contain \vspace{.1cm}\\
\hspace{2.4cm} at least one element $\mathscr{T}^n_h$ and $\mathcal{E}^n_h$, respectively, with the largest error indicators.\vspace{.2cm}\\

\noindent
\textbf{Step ~5.} {\bf \texttt{Refine}:}\hspace{.05cm} Using bisection technique refine each element $K$ of $\mathscr{T}^n_h$ and $E$ of $\mathcal{E}^n_h$ \vspace{.1cm}\\
 \hspace{2.6cm} and get $\mathscr{T}^{n}_{\tilde{h}}$ and  $\mathcal{E}^{n}_{\tilde{h}}$.\vspace{.1cm}\\
\hspace{2.5cm} {\bf \texttt{Check:}}\hspace{.05cm}  If sum of temporal error estimators $\geq \Lambda_1 \frac{\epsilon_{time}}{T}$ \vspace{.1cm}\\
\hspace{4.5cm} {\bf \texttt{Do}}\hspace{.05cm}  $ k'_n:=\delta_1 k_{n-1},\;\; t_n:=t_{n-1}+k'_{n}$ \vspace{.2cm}\\
\hspace{4cm}Else If sum of temporal error estimators $\leq \Lambda_2 \frac{\epsilon_{time}}{T}$ \vspace{0.1cm}\\
\hspace{4.5cm} {\bf \texttt{Do}} \hspace{.05cm}  $ k'_n:=\delta_2 k_{n-1},\;\; t_n:=t_{n-1}+k'_{n}$ \vspace{.2cm}\\

\noindent
\textbf{Step ~6} {\bf Set} $n:=n+1$ and {\bf Go To} { \bf \texttt{Step} 1}. \vspace{.15cm}\\
\end{flushleft}
\end{algorithm}
We first {\bf initialize} the mesh. Then we employ the SIPG discretization to {\bf solve} numerically the OCP with respect to the prescribed triangulation $\mathscr{T}^i_h,\, i=1,\,2,\ldots,\,N$. 
Next, we use the derived estimators $\eta_y,\, \eta_z$ and $\eta_q$ in \eqref{sumofest4.19} to {\bf estimating} the  state, adjoint-state and the control errors. In adaptive method, for {\bf marking} the interior and the edge elements, we use the bulk strategy \cite{dorfler1996}. In marking strategy, the larger value of $\theta$ leads to a large sets $\mathcal{M}^i_K$ and $\mathcal{M}^i_E$ for the refinement, 
while the smaller value of $\theta$ provides the small sets $\mathcal{M}^i_K$ and $\mathcal{M}^i_E$ for the refinement, 
but more refinement loops needed. It is observed that the bulk criteria may take into account data oscillations, as was the case for the estimators $\eta$ in \eqref{5.2Numsbulk}. Next and final step is {\bf refine}. In this step, the marked elements in the previous step are refined by using longest edge bisection method, while the elements of the marked edges are refined by bisection technique \cite{chen2008}. Then the adaptive algorithm is accomplished after a number of mesh enhancements leading to a solution with an estimated error inside of a given tolerance. The adaptive meshes are generated via the derived estimators in Section \ref{sec4errest}. 

Below, we present numerical results to evaluate the precision of the acquired estimators discussed in Section 4, as well as the efficacy of the adaptive algorithm \ref{algo5.1}. We utilize piecewise linear polynomials to approximate the state, adjoint-state, control, and co-control variables. 
The regularization value $\alpha$ is set equal to the parameter $\gamma$ employed in defining the active and inactive sets. Additionally, equal tolerances are applied to both time and space considerations in numerical computations, i.e., $\epsilon_{time}=\epsilon_{space}= \epsilon$ (say).  
%%%%%%%%%%%%%%%%%%%%%%%%%%
The following modified exam is taken from \cite{siebertrosch2014}.
\begin{exam}\label{ex5.1}
We consider a convex domain $\Omega=[0,3]\times[0,3]$ with boundary $\Gamma$, where the Neumann boundary $\Gamma_{_N}=\Gamma$ and the time domain $I=[0,1]$. The control $q$ is acting only on the part of the boundary $\Gamma$, say, $\bar{\Gamma}_{N}=\{0\}\times [1,2]$. However, we assume that the reaction term $a_0=1$ and the regularization parameter $\alpha =1$. Further, we choose the following  analytical solutions
\begin{align}
& y(x, t)~=~ t \times e^{-10(x_1^2+x_2^2)}, \quad \text{where} \quad  x=(x_1,x_2), \\
&z(x, t)~=~ \frac{(1-t) \mathpzc{K} }{2n} \times \Big[(2n+1) \times \big( \frac{2x_2}{3}-1 \big)-\big( \frac{2x_2}{3}-1 \big )^{2n+1} \Big],  \\
&q(x, t)~=~ P_{[q_a, q_b]} \big( z(x, t) \big),  \quad
\mu(x, t)~=~ z(x,t)-P_{[q_a, q_b]}  \big( z(x, t) \big), 
\end{align}
and the given data as follows
\begin{align}
&f(x,t)~=~\big\{[41-400\times (x_1^2+x_2^2)] \times  t +1 \big\} \times e^{-10(x_1^2+x_2^2)},  \quad q_d(x,t)~=~0, \quad  r_{_N}(x,t)~=~0,\\
&y_d(x,t)~=~ t\times e^{-10(x_1^2+x_2^2)}+\frac{(t-2)\mathpzc{K}}{2n}\times\Big[(2n+1)\times\big(\frac{2x_2}{3}-1\big)-\big(\frac{2x_2}{3}-1\big)^{2n+1} \Big]\\
&\hspace{1.7cm}-\big(\frac{8n+4}{9}\big)\times(1-t)\mathpzc{K}\times\big(\frac{2x_2}{3}-1\big)^{2n-1},  \\
% \end{align}
% \begin{align}
&g_{_N}(x,t)~=~
\begin{cases}
-P_{[q_a, q_b]} \big(z(x, t)\big), \quad (x,t)\in\bar{\Gamma}_N \times I,\\
-60\times t \times e^{-10(9+x_2^2)}, \quad x_1=3,\quad \text{and} \quad t \in I,\\
-60\times t \times e^{-10(x_1^2+9)}, \quad x_2=3,\quad \text{and} \quad t \in I,\\
\;\;\;0, \quad \text{Otherwise},
\end{cases}
\end{align}
respectively, with $\mathpzc{K}=10$ and $n=20$.
\end{exam}
%%%%%%%%%%%%%%%%%%%%%%%%%%%%%%%%5
%%%%%%%%%%%%%%%%%%%%%%%%%%%%%%%%%%%%%%%%%%%
We provide a comprehensive analysis of errors concerning the state, adjoint-state, control and co-control variables observed on both uniform and adaptive meshes with fixing $\theta =0.40$ and the tolerances $(\epsilon=)\; 0.001$. The errors for the state and adjoint-state are evaluated in the energy norm, represented as $L^2(I;|\|\cdot|\|)$-norm, while for the control and co-control are assessed in the $L^2(I; L^2(\Gamma))$-norm. 
Adaptive meshes (Figure \ref{meshuniadpt} (right)) are generated by employing error indicators based on an initial uniform mesh (Figure \ref{meshuniadpt} (left)) with mesh size $N_x\times N_y=80\times 80$. Notably, Figure \ref{meshuniadpt} 
indicates a significant reduction in the number of elements and nodes within the adapted meshes compared to the uniform mesh. The distinct characteristics of the state, adjoint-state, and control variables require localized refinements across different regions of the domain, as illustrated in Figure \ref{meshuniadpt} (right). Due to the small exponential peak in the state variable $(y)$, adaptive meshes are particularly concentrated around the origin, as depicted in the right portion of Figure \ref{meshuniadpt}. Figure \ref{meshuniadpt} (left and right) demonstrates that the utilized error estimators effectively adapt the meshes to the pertinent areas. Figure \ref{appmeshuex} displays the profiles of the approximated state (left) and adjoint-state (right) variables on adaptive meshes at penalty parameter $\sigma_0=10^2$, $q_a=-5.0$, $q_b=5.0$ and time $t=0.5$, while the Figure \ref{controlcocontrol} shows the approximated control (left) and the co-control (right).

Further, the active and inactive sets are analyzed for the different choices for $q_a$ and $q_b$, as illustrated in Figures \ref{inactiveset}--\ref{qpp05mup05}. First, we selected $q_a=-8.5$ and $q_b=8.5$, and we saw that the controls are inactive in the set $\mathcal{I}=[1, 2]$ for this case (cf., Figures \ref{inactiveset} (left)) and the co-control $\mu$ is vanishing according to the definition, which is efficiently captured by the indicators (cf., Figures \ref{inactiveset} (right)). Any choice of $q_a$ and $q_b$ between $-8.0$ and $8.0$, the controls are active in some regions and inactive in some of the regions, as 
% %%%%%%%%%%%%%%%%%%%%%%%%%%%%%%%%%%%%%%%%%%%%%%%%%%%%%
%%%%%%%%%%%%%%%%%%%%%%%%%%%%%%%%
\begin{figure}[H]
\begin{center}
\includegraphics[width=0.49\textwidth]{unif80.png}  \hfill
\includegraphics[width=0.49\textwidth]{adpt333804.png}
\caption{Uniform meshes with size $80\times 80$ (NDOF=$91164$) and adaptively generated meshes at $3^{rd}$-iteration (NDOF=$333804$), respectively.}\label{meshuniadpt} \vspace{0.5cm}
\end{center}
\begin{center}
\includegraphics[width=0.49\textwidth]{stateF.png}  \hfill 
\includegraphics[width=0.49\textwidth]{costateF.png} 
\caption{Profile of the approximated DG state ($y_h$) and adjoint-state ($z_h$) on adaptive meshes with   $\sigma_0=100$.}\label{appmeshuex}  \vspace{1cm}
\end{center}
\begin{center}
\includegraphics[width=0.49\textwidth]{controlF.png}   \hfill
 \includegraphics[width=0.49\textwidth]{cocontrolF.png}
\caption{Profile of approximated DG control ($q_h$) and and co-control on adaptive meshes $(\mu_h)$.}\label{controlcocontrol} 
\end{center}
\end{figure}
% %%%%%%%%%%%%%%%%%%%%%% 
% %%%%%%%%%%%%%%%%%%         CONTROL ANALYSIS 
% %%%%%%%%%%%%%%%%%%%%%%%%%%%%%%%%%%%%%%%%%
\begin{figure}[H]
\begin{center}
\includegraphics[width=0.45\textwidth]{qn85p85.png}
\hfill
\includegraphics[width=0.45\textwidth]{mun85tp85.png}
\caption{Computation of control $(q)$ (left) and co-control $(\mu)$ (right) on the inactive set $\mathcal{I}=\{0\}\times [1, 2]$.}\label{inactiveset} 
\includegraphics[width=0.45\textwidth]{qn2t0.png}\hfill %{ControlErrEx1}
\includegraphics[width=0.45\textwidth]{mun2to.png}%{CocontrolErrEx1}
\caption{Computation of control $(q)$ (left) and co-control $(\mu)$ (right) on the active-inactive sets $\mathcal{A}=\{0\}\times (\mathcal{A}_a \cup \mathcal{A}_b)$ and $\mathcal{I}=\{0\}\times [1.33, 1.50]$, respectively.}\label{mixedqmu20} \vspace{1cm}
%%%%%%%%%%%%
\includegraphics[width=0.45\textwidth]{qp0t2.png} \hfill
\includegraphics[width=0.45\textwidth]{mup0t2.png}
\caption{Computation of control $(q)$ (left) and co-control $(\mu)$ (right) on the active-inactive sets $\mathcal{A}=\{0\}\times (\mathcal{A}_a \cup \mathcal{A}_b)$ and $\mathcal{I}=\{0\}\times [1.50, 1.62]$, respectively.} \label{qpp02mup02} 
\end{center}
\end{figure}
%%%%%%%%%%%%%%%%%%%%%%%%%%%%%%
%%%%%%%%%%%%%%%%%%%%%%%%%%%%%%%%%%%%%%%%%
\begin{figure}[H]
\begin{center}
\includegraphics[width=0.45\textwidth]{qn35t0.png}
\hfill
\includegraphics[width=0.45\textwidth]{mun35t0.png}
\caption{Computation of control $(q)$ (left) and co-control $(\mu)$ (right) on the active-inactive sets $\mathcal{A}=\{0\}\times (\mathcal{A}_a \cup \mathcal{A}_b)$ and  $\mathcal{I}=\{0\}\times [1.28, 1.50]$.}\label{inactive3.50set} \vspace{1cm}
\includegraphics[width=0.45\textwidth]{qp0t35.png}\hfill %{ControlErrEx1}
\includegraphics[width=0.45\textwidth]{mup0t35.png}%{CocontrolErrEx1}
\caption{Computation of control $(q)$ (left) and co-control $(\mu)$ (right) on the active-inactive sets $\mathcal{A}=\{0\}\times (\mathcal{A}_a \cup \mathcal{A}_b)$ and $\mathcal{I}=\{0\}\times [1.50, 1.72]$, respectively.}\label{mixedqmu} \vspace{1cm}
%%%%%%%%%%%%
\includegraphics[width=0.45\textwidth]{qn5t0.png} \hfill
\includegraphics[width=0.45\textwidth]{mun5t0.png}
\caption{Computation of control $(q)$ (left) and co-control $(\mu)$ (right) on the active-inactive sets $\mathcal{A}=\{0\}\times (\mathcal{A}_a \cup \mathcal{A}_b)$ and $\mathcal{I}=\{0\}\times [1.2, 1.52]$, respectively.} \label{qpp05mup05} 
\end{center}
\end{figure}
%%%%%%%%%%%%
\begin{figure}[H]
\begin{center}
\includegraphics[width=0.45\textwidth]{qp0t5.png} \hfill
\includegraphics[width=0.45\textwidth]{mup0t5.png}
\caption{Computation of control $(q)$ (left) and co-control $(\mu)$ (right) on the active-inactive sets $\mathcal{A}=\{0\}\times (\mathcal{A}_a \cup \mathcal{A}_b)$ and $\mathcal{I}=\{0\}\times [1.5, 1.8]$, respectively.} \label{qpp05mup05} \vspace{1cm}
\includegraphics[width=0.45\textwidth]{StateErrEx13}
\hfill
\includegraphics[width=0.45\textwidth]{AdjErrEx13}
\caption{The state (left) and adjoint-state (right) error plots on the uniform and adaptive  meshes.}\label{stateadjoint-stateerrors} \vspace{1cm}
%%%%%%%%%%%%%%%%
\includegraphics[width=0.45\textwidth]{ContErrEx13}\hfill %{ControlErrEx1}
\includegraphics[width=0.45\textwidth]{CocontErrEx13}%{CocontrolErrEx1}
\caption{The control (left) and co-control (right) error plots on the uniform and adaptive  mesh.}\label{controlcocontrolerrors} 
\end{center}
\end{figure}
%%%%%%%%%%%%%%%%%%%
\begin{figure}[H]
\begin{center}
\includegraphics[width=0.45\textwidth]{EffIndEx11} \hfill
\includegraphics[width=0.45\textwidth]{EstEx1}
\end{center}
\caption{Plots of the effective index (left) and the components of the error estimator and data oscillations (left).}\label{iffind} 
\end{figure}
\noindent
presented in Figures \ref{mixedqmu20}-\ref{qpp05mup05} and the corresponding co-controls thereof. We observe that the co-controls disappear in the portion in which the control is inactive $\mathcal{I}$; refer to right Figures \ref{mixedqmu20}-\ref{qpp05mup05}. Further, the active sets are denoted by $\mathcal{A}$, which are clearly mentioned in the figures.

Moreover, Figure \ref{stateadjoint-stateerrors} illustrates state (left) and adjoint-state (right) errors in uniform and adaptive meshes by setting tolerances $\epsilon=0.00001$ and
various marking parameters $\theta=0.25$, $\theta=0.50$, and $\theta=0.75$ in the bulk strategy. Similarly, Figure \ref{controlcocontrolerrors} displays the control and co-control errors in uniform and adaptive meshes using $\theta=0.25, 0.50$ and $0.75$, respectively. The effective index (left) and the components of the error estimator and the data oscillations plots (left) with the marking parameter $\theta=0.45$ are presented in Figure \ref{iffind}.
%%%%%%%%%%%%%%%%%% 
%%%%%%%%%%%%%%%%%%%%%%%%%%%%%%%%%%%%%%%%%%%%%%%%%%%%%
\section{Conclusion} \label{section6666}
This study investigated the a posteriori error analysis for the SIPG method for parabolic BCPs with bilateral control constraints. We have employed piecewise-linear polynomials for the discretization of the state, adjoint-state, and control variables. Both lower and upper bounds for the error estimates are established showcasing the efficacy and dependability of the proposed error estimator through consideration of data oscillations. While the control error estimator $\eta_q$ proved effective in capturing the approximation error of the control, it was found to have limitations in providing guidance for refinement localization in certain critical instances. To address these deficiencies, we employ an alternative control indicator, $\bar{\eta}_q$, in numerical calculations. The results of these computations unequivocally highlighted the superiority of adaptive refinements over uniform refinements, underscoring the effectiveness of the proposed approach in achieving accurate solutions while optimizing computational efficiency. Our numerical findings emphasize the superiority of adaptive refinements over uniform refinements.
\bibliographystyle{siam}
\bibliography{RMRKBVR19022025.bib}
\end{document}
