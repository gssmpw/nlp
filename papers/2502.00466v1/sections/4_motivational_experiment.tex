\section{Motivational Experiments}
\vspace{-0.5em}
\begin{figure}[h]
    \centering
    \subfigure[Qualitative results in MemoryS7 for DIAMOND and ours.]{\includegraphics[width=0.45\textwidth]{imgs/minigrid-qualitative.pdf}} \vspace{-1em}
    \subfigure[Quantitative results in MemoryS7/S9: DIAMOND \& ours.]{\includegraphics[width=0.45\textwidth]{imgs/minigrid-curves.pdf}}

    \caption{Motivational examples for both qualitative and quantitative evidences to demonstrate that DIAMOND face difficulties in imagining accurate future under long horizon memorization tasks. }
    \label{fig:motivational_experiment} \vspace{-2em}
\end{figure}

To substantiate the memory limitations of the DIAMOND model, we conducted experiments using the MiniGrid MemoryS7 and MemoryS9 environments~\cite{chevalier-boisvert2023minigrid}. These experiments evaluate memory consistency and temporal prediction capabilities in world models. Fig.~\ref{fig:motivational_experiment} presents qualitative comparisons among the DIAMOND model, our proposed EDELINE method, and the Oracle world model. The world models were trained on MiniGrid MemoryS7, and their state predictions were evaluated. The upper half of Fig.~\ref{fig:motivational_experiment}~(a) shows an initial action sequence of four consecutive left turns followed by four no-op actions. DIAMOND and EDELINE then autoregressively predicted the subsequent states. The lower half of Fig.~\ref{fig:motivational_experiment}~(a) reveals the prediction outcomes. At timestep 9, DIAMOND generated incorrect predictions with an extra key object in the state. The model's predictions at timesteps 10 to 11 exhibited inaccuracies in the outer wall representations. These prediction errors result from the environment's partial observability and DIAMOND's four-frame context limitation. The input of four idle frames at timestep 9 led to loss of earlier state context. In contrast, EDELINE maintained accurate predictions throughout the sequence. This improved performance stems from EDELINE's architectural design, which addresses the memory constraints inherent in DIAMOND. The accurate prediction at timestep 12 by DIAMOND can be attributed to its access to the 8th state, which enabled accurate predictions at timestep 13 through the correct 12th state context. Fig.~\ref{fig:motivational_experiment}~(b) illustrates DIAMOND's performance limitations in both MemoryS7 and MemoryS9 environments due to insufficient memory capacity in partially observable scenarios. In contrast, EDELINE shows near-optimal performance under these conditions. \vspace{-1em}