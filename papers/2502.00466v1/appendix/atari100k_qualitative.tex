\section{Atari 100k Qualitative Analysis}
\label{appendix:atari_100k_qualitative}
To provide deeper insights into EDELINE's superior performance, we conduct qualitative analysis on three Atari games where our approach demonstrates the most significant improvements over DIAMOND: BankHeist, DemonAttack, and Hero. 
Figure~\ref{fig:atari_qualitative} presents temporal sequences comparing ground truth gameplay with predictions from both EDELINE and DIAMOND world models.

In BankHeist, agents maximize scores through repeated map traversal to encounter new enemies. EDELINE's SSM-enhanced world model maintains consistent tracking of the player character position throughout prediction sequences, while DIAMOND's model shows progressive degradation with the character eventually disappearing from predictions.
For DemonAttack, EDELINE successfully captures the relationship between enemy hits and score updates in its predictions. DIAMOND preserves basic visual structure but fails to reflect these crucial state transitions.
The Hero environment showcases EDELINE's long-range prediction capabilities, accurately capturing sequences of the player breaking obstacles and navigating new areas. 
These qualitative results highlight how EDELINE's architectural innovations - SSM-based memory, unified training, and diffusion modeling - enable robust state tracking, action-consequence modeling, and temporal consistency. These capabilities directly contribute to EDELINE's superior performance on the Atari 100k benchmark.
\vspace{-1em}
\begin{figure*}[h!]
\centering
\includegraphics[width=0.96\textwidth]{imgs/atari100k-qualitative.pdf}
\caption{Qualitative comparison of world model predictions on three Atari games. Each panel shows temporal sequences comparing ground truth with EDELINE and DIAMOND predictions. In BankHeist (top), EDELINE successfully tracks the player character while DIAMOND loses this information. DemonAttack (middle) demonstrates EDELINE's accurate modeling of score updates upon successful hits. Hero (bottom) showcases EDELINE's ability to maintain consistent predictions of complex character-environment interactions across extended sequences. Colored boxes highlight successful (\textcolor{green}{green}) and failed (\textcolor{red}{red}) predictions of key game elements.}
\label{fig:atari_qualitative}
\end{figure*}