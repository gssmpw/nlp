\section{Extended Performance Analysis}
\label{appendix:atari_100k_additional}
\begin{figure}[h]
\centering
\subfigure[Performance profiles]{
\includegraphics[width=0.45\textwidth]{imgs/performance_profile.pdf}
\label{fig:perf_profile}
}
\subfigure[Optimality gaps]{
\includegraphics[width=0.5\textwidth]{imgs/optimality_gap.pdf}
\label{fig:opt_gap}
}
\caption{Additional performance analyses. (a) Performance profiles showing fraction of runs achieving scores above different human normalized score thresholds. (b) Optimality gaps demonstrating relative distance to theoretical optimal performance across different methods.}
\label{fig:extended_analysis}
\end{figure}

To provide additional performance insights, we analyze EDELINE using performance profiles and optimality gaps \cite{agarwal2021deep}. Fig.~\ref{fig:perf_profile} shows the empirical cumulative distribution of human-normalized scores (HNS) across all 26 Atari games. The y-axis indicates the fraction of games achieving scores above each HNS threshold $\tau$. EDELINE consistently maintains a higher fraction of games across different thresholds compared to baseline methods, particularly in the middle range ($\tau$ between 1 and 4).
We further analyze model performance through optimality gaps shown in Fig.~\ref{fig:opt_gap}. The optimality gap measures the distance between model performance and theoretical optimal behavior. EDELINE achieves the smallest optimality gap among all compared methods, demonstrating its effectiveness in approaching optimal performance across the Atari 100k benchmark suite.
These detailed analyses complement the aggregate metrics presented in Section~\ref{subsec:atari_100k_experiments} by providing a more granular view of performance distribution and theoretical efficiency.

\begin{figure*}[h]
\centering
\includegraphics[width=\linewidth]{imgs/all_method_compare.pdf}
\label{fig:all_cmp}
\vspace{-2em}
\caption{\textbf{Comparison of state-of-the-art model-based RL methods without using look-ahead search techniques on the Atari 100k benchmark.} EDELINE outperforms all existing model-based approaches on the Atari 100k benchmark. Previous methods can be categorized by their world model architectures: Transformer-based models (TWM~\cite{robine2023TWM}, IRIS~\cite{micheli2023iris}, REM~\cite{cohen2024rem}, STORM~\cite{zhang2023storm}, $\Delta$-IRIS~\cite{alonso2023delta-iris}, TWISTER~\cite{anonymous2025twister}), RNN-based models (DreamerV3~\cite{hafner2024DreamerV3}, PaMoRL~\cite{wang2024pamorl}, HarmonyDream~\cite{ma2024harmonydream}), SSM-based models (Drama~\cite{anonymous2025drama}), and Diffusion-based models (DIAMOND~\cite{alonso2024diamond}). Our proposed EDELINE advances the state-of-the-art by integrating diffusion modeling with state space models, combining their respective strengths in visual generation and temporal modeling.} 
\end{figure*}