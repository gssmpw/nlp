%%
%% This is file `sample-sigconf.tex',
%% generated with the docstrip utility.
%%
%% The original source files were:
%%
%% samples.dtx  (with options: `all,proceedings,bibtex,sigconf')
%% 
%% IMPORTANT NOTICE:
%% 
%% For the copyright see the source file.
%% 
%% Any modified versions of this file must be renamed
%% with new filenames distinct from sample-sigconf.tex.
%% 
%% For distribution of the original source see the terms
%% for copying and modification in the file samples.dtx.
%% 
%% This generated file may be distributed as long as the
%% original source files, as listed above, are part of the
%% same distribution. (The sources need not necessarily be
%% in the same archive or directory.)
%%
%%
%% Commands for TeXCount
%TC:macro \cite [option:text,text]
%TC:macro \citep [option:text,text]
%TC:macro \citet [option:text,text]
%TC:envir table 0 1
%TC:envir table* 0 1
%TC:envir tabular [ignore] word
%TC:envir displaymath 0 word
%TC:envir math 0 word
%TC:envir comment 0 0
%%
%% The first command in your LaTeX source must be the \documentclass
%% command.
%%
%% For submission and review of your manuscript please change the
%% command to \documentclass[manuscript, screen, review]{acmart}.
%%
%% When submitting camera ready or to TAPS, please change the command
%% to \documentclass[sigconf]{acmart} or whichever template is required
%% for your publication.
%%
%%
%\documentclass[sigconf]{acmart}
%%
%\documentclass[sigconf,natbib=true]{acmart}
\documentclass[sigconf,natbib=true,anonymous=false]{acmart}
\settopmatter{printacmref=true}

\AtBeginDocument{%
  \providecommand\BibTeX{{%
    \normalfont B\kern-0.5em{\scshape i\kern-0.25em b}\kern-0.8em\TeX}}}

% \setcopyright{acmlicensed}
% \copyrightyear{2018}
% \acmYear{2018}
% \acmDOI{XXXXXXX.XXXXXXX}
% \acmConference[Conference acronym 'XX]{Make sure to enter the correct
%   conference title from your rights confirmation emai}{June 03--05,
%   2018}{Woodstock, NY}
% \acmISBN{978-1-4503-XXXX-X/18/06}  

\copyrightyear{2024} 
\acmYear{2024} 
% \setcopyright{acmlicensed}\acmConference[SIGIR '24]{Proceedings of the 47th International ACM SIGIR Conference on Research and Development in Information Retrieval}{July 14--18, 2024}{Washington, DC, USA}
% \acmBooktitle{Proceedings of the 47th International ACM SIGIR Conference on Research and Development in Information Retrieval (SIGIR '24), July 14--18, 2024, Washington, DC, USA}
% \acmDOI{10.1145/3626772.3657878}
% \acmISBN{979-8-4007-0431-4/24/07}


\newcounter{BalanceAtReference}
\setcounter{BalanceAtReference}{43}
\newcounter{ReferenceIndexForBalancing}

\makeatletter

% Disable acmart's automatic invocation of \balance from \AtEndDocument,
% which is usually too late.
\global\@ACM@balancefalse

% Invoke command when the \bibitem reaches the specified value
\def\@balancelastpageonce{%
  \ifnum\value{ReferenceIndexForBalancing}=\value{BalanceAtReference}
    \newpage
  \else
    \relax
  \fi
  \stepcounter{ReferenceIndexForBalancing}
}
\pretocmd{\bibitem}{\@balancelastpageonce}
  {} % on success
  {\@latex@error{Patching \bibitem failed}{\@ehd}}

% \usepackage{xeCJK}
\usepackage{amsmath}
\usepackage{amsfonts}
\usepackage[linesnumbered,ruled,vlined]{algorithm2e}
\usepackage{graphicx} % DO NOT CHANGE THIS
\usepackage{booktabs}
\usepackage{array}
\usepackage{enumitem}
\usepackage{amsthm}
\usepackage{algpseudocode}
\usepackage[switch]{lineno}
\usepackage[utf8]{inputenc}
\usepackage{eqparbox}
\usepackage[nopar]{lipsum}
\usepackage{multirow}
\usepackage{makecell}
\usepackage{xcolor}
\usepackage{hhline} 
\usepackage{microtype}
\usepackage{diagbox}
\usepackage{longtable} % For longtable in Appendix (Niklas)
\usepackage{adjustbox}
\usepackage{mdframed}
\usepackage{framed,color} 
\usepackage{balance} 
% \usepackage{algorithm}
% \usepackage{algpseudocode}


\definecolor{shadecolor}{rgb}{.92, .92, .92}
\newenvironment{myframe}{%
\def\FrameCommand{\fboxsep=\FrameSep \colorbox{shadecolor}}%
\MakeFramed {\hsize\linewidth\advance\hsize-\width \FrameRestore}}%
{\endMakeFramed}


\usepackage{listings}  % Required for lstlisting environment
\usepackage{xcolor}     % Optional: Enhances syntax highlighting
\usepackage{relsize}

\usepackage{booktabs,multirow,array}
\newcolumntype{N}{@{}m{0pt}@{}}%a fix for array package

\newcommand\ExtraSep
{\dimexpr\cmidrulewidth+\aboverulesep+\belowrulesep\relax}


\usepackage[many]{tcolorbox}
\newtcolorbox{fancyquotes}{%
    enhanced jigsaw, 
    breakable,      % allow page breaks
    frame hidden,   % hide the default frame
    left=0.5cm,       % left margin
    right=0.1cm,      % right margin
    overlay={%
        \node [scale=8,
            text=black,
            inner sep=0pt,] at ([xshift=-1cm,yshift=-1cm]frame.north west){}; 
        \node [scale=8,
            text=black,
            inner sep=0pt,] at ([xshift=1cm]frame.south east){};  
            },
                parbox=false,
}

% \usepackage{mathtools, nccmath}
% \usepackage{scrextend}
% \deffootnote[.25in]{.25in}{.15in}{\makebox[.25in][r]{\thefootnotemark .\hspace{.15in}}}

\makeatletter




\usepackage{pifont} 
\definecolor{darkgreen}{rgb}{0.0,0.5,0.0}
\newcommand{\cmark}{\textcolor{darkgreen}{\ding{51}}}
\newcommand{\xmark}{\textcolor{red}{\ding{55}}}



\definecolor{c4}{RGB}{255,225,187}
\definecolor{c2}{RGB}{209, 233, 184}
\definecolor{c3}{RGB}{137 170 123}
\definecolor{c1}{RGB}{249, 229, 229}
\definecolor{c5}{RGB}{255, 128, 128}
\definecolor{c6}{RGB}{251, 132, 002}
\newcommand{\binghui}[1]{{\color{c5}{ [\textbf{bh:}} #1]}}
\newcommand{\qinhan}[1]{{\color{c6}{ [\textbf{qh:}} #1]}}
\newcommand{\zhiyou}[1]{{\color{c2}{ [\textbf{zy:}} #1]}}
\newcommand{\wzr}[1]{{\color{c3}{ [\textbf{zr:}} #1]}}

\newtheorem{hypothesis}{Hypothesis}[section]
\newtheorem{theorem}{Theorem}[section]
\newtheorem{lemma}{Lemma}[section]
\newtheorem{prop}{Proposition}[section]
\newtheorem{corollary}{Corollary}[section]
\newtheorem{example}{Example}[section]
\newtheorem*{proof*}{Proof}

%% \BibTeX command to typeset BibTeX logo in the docs
\AtBeginDocument{%
  \providecommand\BibTeX{{%
    Bib\TeX}}}

%% Rights management information.  This information is sent to you
%% when you complete the rights form.  These commands have SAMPLE
%% values in them; it is your responsibility as an author to replace
%% the commands and values with those provided to you when you
%% complete the rights form.
\setcopyright{acmlicensed}
\copyrightyear{2025}
\acmYear{2025}
\acmDOI{XXXXXXX.XXXXXXX}
%% These commands are for a PROCEEDINGS abstract or paper.
% \acmConference[Conference acronym 'XX]{Make sure to enter the correct
%   conference title from your rights confirmation emai}{June 03--05,
%   2018}{Woodstock, NY}

% \acmConference[SIGIR ’25]{the 48th International ACMSIGIR Conference on Research and Development in Information Retrieval}{July 13--18, 2025}{Padua, Italy}
%%
%%  Uncomment \acmBooktitle if the title of the proceedings is different
%%  from ``Proceedings of ...''!
%%
%%\acmBooktitle{Woodstock '18: ACM Symposium on Neural Gaze Detection,
%%  June 03--05, 2018, Woodstock, NY}
%\acmISBN{978-1-4503-XXXX-X/18%/06}


%%
%% Submission ID.
%% Use this when submitting an article to a sponsored event. You'll
%% receive a unique submission ID from the organizers
%% of the event, and this ID should be used as the parameter to this command.
%%\acmSubmissionID{123-A56-BU3}

%%
%% For managing citations, it is recommended to use bibliography
%% files in BibTeX format.
%%
%% You can then either use BibTeX with the ACM-Reference-Format style,
%% or BibLaTeX with the acmnumeric or acmauthoryear sytles, that include
%% support for advanced citation of software artefact from the
%% biblatex-software package, also separately available on CTAN.
%%
%% Look at the sample-*-biblatex.tex files for templates showcasing
%% the biblatex styles.
%%

%%
%% The majority of ACM publications use numbered citations and
%% references.  The command \citestyle{authoryear} switches to the
%% "author year" style.
%%
%% If you are preparing content for an event
%% sponsored by ACM SIGGRAPH, you must use the "author year" style of
%% citations and references.
%% Uncommenting
%% the next command will enable that style.
%%\citestyle{acmauthoryear}


%%
%% end of the preamble, start of the body of the document source.
\begin{document}
\begin{sloppypar}

%%
%% The "title" command has an optional parameter,
%% allowing the author to define a "short title" to be used in page headers.

% arxiv\title{MRAMG-Bench: A BeyondText Benchmark for Multimodal Retrieval-Augmented Multimodal Generation}


\title{MRAMG-Bench: A BeyondText Benchmark for Multimodal Retrieval-Augmented Multimodal Generation}
%%
%% The "author" command and its associated commands are used to define
%% the authors and their affiliations.
%% Of note is the shared affiliation of the first two authors, and the
%% "authornote" and "authornotemark" commands
%% used to denote shared contribution to the research.
\author{Qinhan Yu$^*$}
\email{yuqinhan@stu.pku.edu.cn}
\orcid{0009-0004-0445-0786}
\affiliation{%
  \institution{Peking University}
  \city{Beijing}
  \country{China}
}


\author{Zhiyou Xiao$^*$}
\email{xiaozhiyou@stu.pku.edu.cn}
\affiliation{%
  \institution{Peking University}
  \city{Beijing}
  \country{China}
}

\author{Binghui Li$^*$}
\email{libinghui@pku.edu.cn}
\affiliation{%
  \institution{Peking University}
  \city{Beijing}
  \country{China}
}

\author{Zhengren Wang}
\email{wzr@stu.pku.edu.cn}
\affiliation{%
  \institution{Peking University}
  \city{Beijing}
  \country{China}
}

\author{Chong Chen$^\dagger$}
\email{chenchong55@huawei.com}
\affiliation{%
  \institution{Huawei Cloud BU}
  \city{Beijing}
  \country{China}
}

\author{Wentao Zhang$^\dagger$}
\email{wentao.zhang@pku.edu.cn}
\affiliation{%
  \institution{Peking University}
  \city{Beijing}
  \country{China}
}



%%
%% By default, the full list of authors will be used in the page
%% headers. Often, this list is too long, and will overlap
%% other information printed in the page headers. This command allows
%% the author to define a more concise list
%% of authors' names for this purpose.
\renewcommand{\shortauthors}{}

%%
%% The abstract is a short summary of the work to be presented in the
%% article.

    %Recent advancements in retrieval-augmented generation (RAG) have shown remarkable performance in enhancing response accuracy and relevance by integrating external knowledge into generative models.
\begin{abstract}
Recent advancements in Retrieval-Augmented Generation (RAG) have shown remarkable performance in enhancing response accuracy and relevance by integrating external knowledge into generative models.
However, existing RAG methods primarily focus on providing text-only answers, even in multimodal retrieval-augmented generation scenarios. In this work, we introduce the Multimodal Retrieval-Augmented Multimodal Generation (MRAMG) task, which aims to generate answers that combine both text and images, fully leveraging the multimodal data within a corpus.
Despite the importance of this task, there is a notable absence of a comprehensive benchmark to effectively evaluate MRAMG performance. To bridge this gap, we introduce the MRAMG-Bench, a carefully curated, human-annotated dataset comprising 4,346 documents, 14,190 images, and 4,800 QA pairs, sourced from three categories: Web Data, Academic Papers, and Lifestyle. The dataset incorporates diverse difficulty levels and complex multi-image scenarios, providing a robust foundation for evaluating multimodal generation tasks. To facilitate rigorous evaluation, our MRAMG-Bench incorporates a comprehensive suite of both statistical and LLM-based metrics,  enabling a thorough analysis of the performance of popular generative models in the MRAMG task. Besides, we propose an efficient multimodal answer generation framework that leverages both LLMs and MLLMs to generate multimodal responses. Our datasets are available at: https://huggingface.co/MRAMG.
\end{abstract}

  


%%
%% The code below is generated by the tool at http://dl.acm.org/ccs.cfm.
%% Please copy and paste the code instead of the example below.
%%
% \begin{CCSXML}
% <ccs2012>
%  <concept>
%   <concept_id>00000000.0000000.0000000</concept_id>
%   <concept_desc>Do Not Use This Code, Generate the Correct Terms for Your Paper</concept_desc>
%   <concept_significance>500</concept_significance>
%  </concept>
%  <concept>
%   <concept_id>00000000.00000000.00000000</concept_id>
%   <concept_desc>Do Not Use This Code, Generate the Correct Terms for Your Paper</concept_desc>
%   <concept_significance>300</concept_significance>
%  </concept>
%  <concept>
%   <concept_id>00000000.00000000.00000000</concept_id>
%   <concept_desc>Do Not Use This Code, Generate the Correct Terms for Your Paper</concept_desc>
%   <concept_significance>100</concept_significance>
%  </concept>
%  <concept>
%   <concept_id>00000000.00000000.00000000</concept_id>
%   <concept_desc>Do Not Use This Code, Generate the Correct Terms for Your Paper</concept_desc>
%   <concept_significance>100</concept_significance>
%  </concept>
% </ccs2012>
% \end{CCSXML}

% \ccsdesc[500]{Do Not Use This Code~Generate the Correct Terms for Your Paper}
% \ccsdesc[300]{Do Not Use This Code~Generate the Correct Terms for Your Paper}
% \ccsdesc{Do Not Use This Code~Generate the Correct Terms for Your Paper}
% \ccsdesc[100]{Do Not Use This Code~Generate the Correct Terms for Your Paper}


\begin{CCSXML}
<ccs2012>
   <concept>
       <concept_id>10002951.10003317</concept_id>
       <concept_desc>Information systems~Information retrieval</concept_desc>
       <concept_significance>500</concept_significance>
       </concept>
 </ccs2012>
\end{CCSXML}

\ccsdesc[500]{Information systems~Information retrieval}

%%
%% Keywords. The author(s) should pick words that accurately describe
%% the work being presented. Separate the keywords with commas.
\keywords{Multimodal Retrieval-Augmented Multimodal Generation, Large Language Model, Multimodal
Large Language Model, Evaluation}
%% A "teaser" image appears between the author and affiliation
%% information and the body of the document, and typically spans the
%% page.
% \begin{teaserfigure}
%   \includegraphics[width=\textwidth]{sampleteaser}
%   \caption{Seattle Mariners at Spring Training, 2010.}
%   \Description{Enjoying the baseball game from the third-base
%   seats. Ichiro Suzuki preparing to bat.}
%   \label{fig:teaser}
% \end{teaserfigure}

% \received{20 February 2007}
% \received[revised]{12 March 2009}
% \received[accepted]{5 June 2009}

%%
%% This command processes the author and affiliation and title
%% information and builds the first part of the formatted document.
\begin{teaserfigure}
    \centering
\includegraphics[width=1.0\columnwidth, height=0.5\textheight]{Fig/Task_example_combine.pdf}
    \caption{Illustration of the MRAMG task (above), with scenarios below showing how integrating text and images enhances clarity and understanding.}
    \label{fig:task_example}
\end{teaserfigure}
\maketitle



\section{Introduction}
\label{sec:intro}
Retrieval-Augmented Generation (RAG) addresses the challenges of outdated knowledge and hallucinatory outputs in Large Language Models (LLMs) by incorporating external knowledge into the reasoning process, enhancing response accuracy and relevance  \citep{hallucination, gupta2024rag}. Traditional RAG methods focus on retrieving textual knowledge, limiting their ability to leverage multimodal information such as images and tables.

With the evolution of generative models, the shift from traditional LLMs to multimodal large language models (MLLMs) has facilitated a more comprehensive 
%and nuanced 
understanding by integrating both textual and visual inputs \citep{wang2024comprehensive,wang2024qwen2}. 
Correspondingly, Multimodal Retrieval-Augmented Generation (MRAG) \citep{chen2022murag} builds upon traditional RAG by incorporating both textual and visual information, thereby enhancing the quality of generated responses through the image understanding capabilities of MLLMs.
However, while the retrieval process incorporates multimodal information, current MRAG methods primarily focus on multimodal input and text-based output. 
%Despite MLLMs' powerful image understanding capabilities, they continue to demonstrate a significant propensity for hallucinations in their generated textual outputs. 
Despite their advanced image comprehension capabilities, MLLMs frequently exhibit hallucinations when converting visual inputs into textual outputs \citep{bai2024hallucination}. 
This raises a natural question:``\textit{Why not present the images directly rather than describing their content?}"

Indeed, there is an emerging trend in LLM deployment toward generating  multimodal outputs that seamlessly integrate both text and images.
%In the deployment of LLMs, the ability to generate multimodal outputs—combining text and images—has become a core expectation for users.
%Increasingly, users demand visual content alongside textual information, preferring to ``see" rather than be told. 
Users increasingly favor visual content over purely textual information, preferring to ``see" rather than merely be ``told".
This growing need is particularly evident in critical scenarios, as shown in Figure \ref{fig:task_example}:
%\begin{figure*}[t]
  %  \centering
 %   \includegraphics[width=1.0\linewidth]{Fig/Task_example_combine.pdf}
 %   \caption{
%    Scenarios illustrating that integrating text and images enhances clarity and understanding.
  %  }
   % \label{fig:examples}
%\end{figure*}
\begin{itemize} 
    \item \textbf{Images as answers}: ``\textit{Show, don't tell}." In certain cases, an image provides the most effective response. For instance, when asked about the appearance of a cat, a photograph is the most direct and convincing answer—seeing truly is believing. 
    % \binghui{WebQA++}
    \item \textbf{Text-Image integration for 
    answers}: 
    %``Seeing is believing."
    %``A picture and words go hand in hand." 
    ``\textit{A picture and words are two puzzle pieces—only together do they reveal the full story.}"
    %``The eye is the best painter."  
    Integrating pictures and words,  such as in step-by-step recipe instructions, significantly enhances comprehension which offers clearer insights and fosters a more intuitive understanding of the content.
    \item %Images enhancing generative results
    \textbf{Images as a catalyst for richer answer}: 
    ``\textit{A picture is the icing on the cake.}" 
    %``The eye is the best painter." 
    In applications such as tourist spot descriptions, integrating both textual and visual content can significantly enhance the quality of the generated response.
    \end{itemize}
\vspace{-3pt}
To address this critical demand, we introduce the \textbf{Multimodal Retrieval-Augmented Multimodal Generation (MRAMG)} task, which aims to fully leverage the rich multimodal information within a corpus by generating integrated text-and-image answers (See Section \ref{sec:task_definition} for the definition).
However, current benchmarks for evaluating this critical task suffer from a notable lack of suitable datasets and scientifically rigorous evaluation metrics, as shown in Table \ref{tab:compare}. This significant shortcoming hinders progress in this emerging field.



To bridge this gap, we introduce the \textbf{MRAMG-Bench}, a novel benchmark designed to evaluate the MRAMG task comprehensively.
The MRAMG-Bench consists of six carefully curated datasets, comprising 4,346 documents, 14,190 images, and 4,800 QA pairs, 
%spanning across three diverse domains
sourced from three domains—Web Data, Academic Paper Data, and Lifestyle Data—across seven distinct data sources.
Furthermore, MRAMG-Bench introduces unique challenges, such as hierarchical difficulty levels and order-based reasoning, which rigorously test the reasoning capabilities of both LLMs and MLLMs.  
To the best of our knowledge, MRAMG-Bench is the first benchmark that challenges models to autonomously determine the number, selection, and ordering of images in their responses, effectively simulating the complex scenarios encountered in real-world user interactions.


This benchmark is systematically constructed through a multi-stage workflow with both automatic construction of GPT-4o and meticulous formulation of human annotators, which is illustrated in Figure \ref{fig:data_construction}. 
Traditional evaluations of multimodal answers often rely on subjective metrics, such as image-text consistency, which are assessed through LLM-based judgments or human evaluations \citep{zhang2023internlm,zhu2024murar}.
However, these methods suffer from high subjectivity or evaluator inconsistency.
To mitigate these issues, we introduce a statistically grounded evaluation framework that systematically assesses both retrieval and generation performance. By integrating objective statistical metrics with LLM-based assessments, our framework delivers a comprehensive, multi-dimensional evaluation of multimodal answers, ensuring a fair and rigorous assessment process.

Additionally, we propose a general multimodal answer generation framework that integrates rule-based and model-based approaches. This flexible and scalable solution enables both LLMs and MLLMs to generate interleaved text-image responses.
Our main contributions can be summarized as follows:
\begin{itemize}
    \item 
    We formalize the MRAMG task 
    %(Section \ref{sec:task_definition}) and introduce MRAMG-Bench (Section \ref{sec:dataset})
    , a benchmark consisting of 4,800 human-annotated instances across multiple domains, featuring hierarchical difficulty levels and order-based reasoning challenges. To ensure rigorous evaluation, we propose a robust, statistically grounded set of metrics that assess multimodal responses across various performance dimensions. %(Section \ref{sec:experiments}).
    %We formalize the MRAMG task and present MRAMG-Bench, a benchmark comprising multi-domain data
    %Web-Data, Academic Paper Data, and Lifestyle Data
    %, totaling 4,800 expert-annotated instances with hierarchical difficulty levels and order-based reasoning challenges. 
    \item 
    %We propose a flexible multimodal answer generation framework leveraging generative large models and vision-language models.
    We propose a flexible multimodal answer generation framework that leverages LLMs and MLLMs, combining rule-based and model-based approaches. %(Section \ref{sec:framework}).
    
 %   \item We propose a robust, statistically grounded set of evaluation metrics to rigorously assess multimodal responses across multiple performance dimensions.
    %We introduce a robust, statistically grounded set of comprehensive evaluation metrics designed to rigorously assess multimodal responses across a range of performance dimensions.
    \item 
    We conduct a comprehensive evaluation of 11 advanced generative models on MRAMG-Bench, providing valuable insights into the capabilities and limitations of existing multimodal generation approaches. %(Section \ref{sec:experiments})
\end{itemize}




\section{Related Work}
\label{sec:related_work}

\iffalse
\paragraph{Multimodal RAG methods.}
RAG \citep{2020RAG, zhao2024retrieval} enhances the effectiveness of LLMs by integrating external knowledge into their reasoning process, addressing issues such as outdated training data and hallucinatory responses \citep{hallucination, gupta2024rag}.
As Multimodal Large Language Models (MLLMs) have advanced in their capacity to integrate and generate text from both textual and visual inputs, the development of Multimodal Retrieval-Augmented Generation (mRAG) has naturally followed as an extension, with methods like MuRAG \citep{chen2022murag} and REACT \citep{liu2023learning} retrieving relevant image-text pairs from external memory.
%Similarly, REACT \citep{liu2023learning} introduces a framework for constructing customized visual models by retrieving relevant image-text pairs from web-scale data and training modularized blocks.
To tackle Visual Question Answering (VQA) task, KAT \citep{gui2021kat} leverages the CLIP \citep{radford2021learning} image encoder to retrieve and associate specific image regions with external knowledge bases
%, thereby enhancing the model’s understanding of visual content.
%Meanwhile, RA-VQA \citep{lin2022retrieval} integrates differentiable DPR \citep{karpukhin2020dense} with answer generation, leveraging retrieved knowledge to achieve outstanding performance in VQA.
Unlike existing mRAG methods that convert multimodal information into purely textual outputs, our work addresses a distinct MRAMG task, where the output seamlessly integrates both textual and visual information.
%, preserving the inherent multimodal richness of the data.
A closely related recent work, MuRAR \citep{zhu2024murar}, addresses source attribution by retrieving multimodal elements from attributed documents. Furthermore, M2RAG \citep{ma2024multi} introduces a multi-stage image insertion framework, which involves multiple calls to the generation model during a single answer generation process.
However, these approaches often face challenges of high computational overhead due to repeated model invocations. In this paper, we propose a general framework for multimodal answer generation, leveraging a single invocation of the large generative model.
\paragraph{Classic RAG Datasets}
Well-established benchmarks for RAG, such as MS-MARCO \citep{MSMARCO} (a large-scale QA dataset based on real user queries), TriviaQA \citep{joshi2017triviaqa} (which features trivia questions requiring evidence-based answers), HotpotQA \citep{yang2018hotpotqa} (focused on multi-hop reasoning), Natural Questions (NQ) \citep{Naturalquestions} (based on real Google search queries), and SQuAD \citep{rajpurkar2016squad} (a reading comprehension dataset with span-based answers), are widely used to evaluate RAG performance \citep{petroni2020kilt, QAE}. 
However, these datasets focus on text-based tasks, while real-world applications increasingly require seamless integration of textual and visual information. To address this gap, we introduce a novel benchmark for MRAMG evaluation.
\paragraph{Multimodal RAG Datasets}
Various multimodal RAG datasets have been developed to address tasks requiring multimodal knowledge. 
OK-VQA \citep{marino2019ok} and A-OKVQA \citep{schwenk2022okvqa} that require retrieving external knowledge beyond the image content, whereas MMSearch \citep{jiang2024mmsearch} benchmarks MLLMs as multimodal search engines, focusing on image-to-image retrieval.
In contrast, MultiModalQA \citep{talmor2021multimodalqa} presents a more challenging scenario, where the questions do not include images but require joint reasoning across text and tables to answer complex questions.
While MultiModalQA questions are template-based, WebQA \citep{chang2022webqa} is a multi-hop, manually crafted multimodal QA dataset that involves the retrieval of relevant visual content for questions.
However, WebQA provides purely textual answers, relies solely on MLLMs for reasoning over retrieved images, and lacks textual support, making it unsuitable for language models that require linguistic context for coherent response generation.
M2RAG \citep{ma2024multi}, while constructing a multimodal corpus, is limited to only 200 questions and a corpus restricted to web pages. 
Additionally, the dataset relies on automated generation without manual verification and lacks ground truth for each query, which further complicates accurate evaluation.
To address these challenges, MRAMG-Bench introduces a comprehensive
multimodal dataset specifically designed for MRAMG tasks.  
Each question is meticulously paired with a precise, integrated text-image answer, enabling comprehensive and statistically rigorous evaluation.
\fi

\subsection{Multimodal RAG}
Retrieval-Augmented Generation (RAG) \citep{2020RAG, zhao2024retrieval} enhances the capabilities of Large Language Models (LLMs) by integrating external knowledge, addressing limitations such as outdated training data and hallucinated outputs \citep{hallucination, gupta2024rag}. Recent advancements in Multimodal Large Language Models (MLLMs), which combine textual and visual inputs for generation tasks \citep{wang2024comprehensive}, have spurred the development of Multimodal Retrieval-Augmented Generation (MRAG) as an extension of traditional RAG. Notable approaches, including MuRAG \citep{chen2022murag} and REACT \citep{liu2023learning}, retrieve image-text pairs from external memory to support multimodal generation. In the context of Visual Question Answering (VQA), KAT \citep{gui2021kat} employs the CLIP \citep{radford2021learning} image encoder to associate specific image regions with external knowledge bases. Unlike existing MRAG methods that primarily generate textual outputs, our work focuses on a novel MRAMG task, where the output seamlessly integrates both textual and visual information. A closely related work, MuRAR \citep{zhu2024murar}, addresses source attribution by retrieving multimodal elements from attributed documents. M2RAG \citep{ma2024multi} further extends this by introducing a multi-stage image insertion framework, which involves multiple invocations of the generative model during a single answer generation process. However, these methods are often hindered by high computational costs due to repeated model calls. In contrast, we propose a more efficient framework for multimodal answer generation, leveraging a single invocation of the generative model.
\vspace{-2mm}
\subsection{RAG Benchmarks}
The effective evaluation of Retrieval-Augmented Generation (RAG) models is crucial for advancing their development and optimization. 
Established benchmarks \citep{MSMARCO,joshi2017triviaqa,yang2018hotpotqa,Naturalquestions,rajpurkar2016squad} are commonly used to evaluate RAG models \citep{petroni2020kilt, QAE}. MS-MARCO \citep{MSMARCO} is a large-scale question answering (QA) dataset based on real user queries. TriviaQA \citep{joshi2017triviaqa} consists of trivia questions that require evidence-based answers. HotpotQA \citep{yang2018hotpotqa} focuses on multi-hop reasoning. Natural Questions (NQ) \citep{Naturalquestions} is derived from real Google search queries. SQuAD \citep{rajpurkar2016squad} is a reading comprehension dataset with span-based answers.
While these text-based benchmarks have proven effective in assessing RAG performance, they fall short in evaluating multimodal tasks, which require the integration of both textual and visual information. 
To bridge this gap, we introduce a novel benchmark for evaluating the MRAMG task.
\subsection{Multimodal RAG Benchmarks}
Similar to traditional RAG benchmarks, various multimodal RAG benchmarks \citep{marino2019ok,schwenk2022okvqa,jiang2024mmsearch,talmor2021multimodalqa,chang2022webqa,ma2024multi} have been developed to address tasks requiring multimodal knowledge. OK-VQA \citep{marino2019ok} and A-OKVQA \citep{schwenk2022okvqa} evaluate the multimodal reasoning ability of models using external knowledge beyond image content. MMSearch \citep{jiang2024mmsearch} evaluates MLLMs as multimodal search engines, focusing on image-to-image retrieval. MultiModalQA \citep{talmor2021multimodalqa} presents a more challenging scenario, where questions do not include images but require joint reasoning across text and tables to answer complex questions. While MultiModalQA relies on template-based questions, WebQA \citep{chang2022webqa} is a multi-hop, manually crafted multimodal QA dataset that involves retrieving relevant visual content for questions.
However, WebQA provides only textual answers, relies entirely on MLLMs for reasoning over retrieved images, and lacks textual support, making it unsuitable for language models that depend on linguistic context for coherent response generation. Notably, although M2RAG \citep{ma2024multi} constructs a multimodal corpus, it is limited to just 200 questions and a corpus confined to web pages. Additionally, the dataset relies on automated generation without manual verification and lacks ground truth for each query, further complicating accurate evaluation.
To address these challenges, our MRAMG-Bench introduces a comprehensive multimodal benchmark specifically designed for MRAMG task. Each question is meticulously paired with a precise, integrated text-image answer, enabling comprehensive and statistically rigorous evaluation.





\begin{table*}[tb!]
  \centering
   \caption{
Comparison of MRAMG-Bench with existing RAG benchmarks, where \includegraphics[height=1em]{Fig/text.png} represents text and \includegraphics[height=1em]{Fig/pic.png} represents images.
   }
   \vspace{-2mm}
\resizebox{0.9\textwidth}{!}{
  \begin{tabular}{l|cc|cccc|ccc}
    \toprule
    \multirow{2}[4]{*}{DATASETS} & \multicolumn{2}{c|}{Documents} & \multicolumn{4}{c|}{Questions}        & \multicolumn{3}{c}{Answers} \\
\cmidrule{2-10}          & Type  & Domain & Retrieval Modality & Difficulty Levels? & Type  & Num   & Exist? & Type  & Human-Annotated? \\
    \midrule
    HotpotQA \citep{yang2018hotpotqa} & \includegraphics[height=1em]{Fig/text.png}  & Web     & \includegraphics[height=1em]{Fig/text.png}  & \xmark    & \includegraphics[height=1em]{Fig/text.png}  & 113k   & \cmark    & \includegraphics[height=1em]{Fig/text.png}  & \cmark\\
    OK-VQA \citep{marino2019ok} & \includegraphics[height=1em]{Fig/text.png}  & Open Domain   & \includegraphics[height=1em]{Fig/text.png}  & \xmark   & \includegraphics[height=1em]{Fig/text.png} \includegraphics[height=1em]{Fig/pic.png} & 14k   & \cmark  & \includegraphics[height=1em]{Fig/text.png}  & \cmark\\
    WebQA \citep{chang2022webqa} & \includegraphics[height=1em]{Fig/text.png}  \includegraphics[height=1em]{Fig/pic.png} & Web      & \includegraphics[height=1em]{Fig/text.png} \includegraphics[height=1em]{Fig/pic.png} & \xmark    & \includegraphics[height=1em]{Fig/text.png}  & 56.6k & \cmark  & \includegraphics[height=1em]{Fig/text.png}  & \cmark\\
     MMSearch \citep{jiang2024mmsearch} &   \includegraphics[height=1em]{Fig/pic.png} & Multi-domain     &  \includegraphics[height=1em]{Fig/pic.png} & \xmark    & \includegraphics[height=1em]{Fig/text.png} \includegraphics[height=1em]{Fig/pic.png} & 300 & \cmark  & \includegraphics[height=1em]{Fig/text.png}  & \cmark\\
    M2RAG \citep{ma2024multi} & \includegraphics[height=1em]{Fig/text.png} \includegraphics[height=1em]{Fig/pic.png} & Web     & \includegraphics[height=1em]{Fig/text.png} \includegraphics[height=1em]{Fig/pic.png} & \xmark    & \includegraphics[height=1em]{Fig/text.png}  & 200   & \xmark    & \xmark    & \xmark \\
    \midrule
    MRAMG-Bench & \includegraphics[height=1em]{Fig/text.png} \includegraphics[height=1em]{Fig/pic.png} & Multi-domain    & \includegraphics[height=1em]{Fig/text.png} \includegraphics[height=1em]{Fig/pic.png} & \cmark  & \includegraphics[height=1em]{Fig/text.png}  & 4.8k   & \cmark  & \includegraphics[height=1em]{Fig/text.png} \includegraphics[height=1em]{Fig/pic.png} & \cmark\\
    \bottomrule
    \end{tabular}%
}
  \label{tab:compare}%
\end{table*}%


\section{Task Formulation}
\label{sec:task_definition}

In this section, we formally introduce the definition of the task in \textbf{Multimodal Retrieval-Augmented Multimodal Generation (MRAMG)}, focusing on constructing multimodal answers $\mathcal{A}$ based on a given text-form user query $q$ and its associated multimodal information $\mathcal{D}$. See Figure \ref{fig:task_example} for an illustration.

\textbf{Multi-Document Multimodal Information.} We consider a multi-document multimodal knowledge base as the form $\mathcal{D} = \{d_1, d_2, \ldots, d_n\}$.
For each multimodal document $d\in \mathcal{D}$, it has the interleaved form $d = \{T_1, I_1, T_2, I_2, \dots, T_l, I_l, \dots \}$, where $T_i$ denotes the $i$-th text block (i.e., a semantic text paragraph) and $I_i$ denotes the $i$-th image within the document.

\textbf{Multimodal-Retrieve and Multimodal-Generation.} Specifically, the objective is to generate a multimodal answer $\mathcal{A} = \mathcal{F}(q, \mathcal{D}^*_q, \mathcal{M})$, where $\mathcal{F}$ represents the multimodal-answer generation method (i.e, generation framework), $\mathcal{D}^*_q = \{d_{j_1}, d_{j_2}, \dots, d_{j_k}\}$ 
represents the top-$k$ documents from $\mathcal{D}$ with the highest relevance to the query $q$, and $\mathcal{M}$ denotes a certain foundation generative model, including LLMs and MLLMs.
Indeed, given the user query $q$ and the retrieved documents $\mathcal{D}_q^*$ and the generative model $\mathcal{M}$, the generation framework $\mathcal{F}$ is required to return a multimodal answer $\mathcal{A}$ by selecting appropriate images from $\mathcal{D}_q^*$ and integrating them with the generated text.
% \textbf{Interleaved Multimodal Answer.}
% Furthermore, we hope that the multimodal answer $\mathcal{A}$ is composed of text-answer blocks $\{s_1, s_2, \dots, s_n\}$ and corresponding multimodal information blocks $\{i_1, i_2, \dots, i_n\} $, i.e, $\mathcal{A}=\{s_1, i_1, s_2, i_2, \dots, s_n, i_n\}$, where each $s_j $ represents a text-based answer block (i.e., a semantic sentence), and each $i_j$ represents the corresponding multimodal information block (i.e., an image), which may be inserted or left empty. This structure enables the generation of answers that integrate both textual and multimodal content, enhancing the depth and comprehensiveness of the response.

%\begin{figure}[t]
  %  \centering
 %   \includegraphics[width=1.0\linewidth]{Fig/MRAMG_task.pdf}
 %   \caption{Illustration of MRAMG Task.}
  %  \label{fig:task_formulation}
%\end{figure}


\section{Dataset Construction}
\label{sec:dataset}
\begin{figure*}[t]
    \centering
    \includegraphics[width=1.0\linewidth]{Fig/data_construction.pdf}
    \caption{
    The construction pipeline of MRAMG-Bench. It is divided into three stages: (1) Data Selection and Preprocessing, we collect, clean and preprocess data to ensure the quality of initial datasets. (2) QA Generation and Refinement, we formulate questions and answers and refine the formulated QA pairs. (3) Data Quality Check, we engage annotators and experts in a three-stage data quality check to ensure high-quality of the entire benchmark.
    }
    \label{fig:data_construction}
\end{figure*}
% MRMAG-Bench is a real-world, multimodal benchmark designed to
% address the current gap in comprehensive evaluation frameworks for MRMAG tasks. A key challenge in developing MRMAG tasks lies in crafting questions that do not explicitly reference images but require answers that naturally incorporate them. For instance, the
% question “How do you make scrambled eggs?” should be answered
% with a sequence of images depicting the cooking process. To address this challenge, we introduce this MRMAG-Bench benchmark.
In this section, we detail the construction of MRAMG-Bench. It consists of three domains: Web, Academic Paper and Lifestyle, which respectively contains web-based datasets—MRAMG-Wit, MRAMG-Wiki, MRAMG-Web, and a scientific dataset-MRAMG-Arxiv, and two lifestyle-oriented datasets-MRAMG-Recipe, MRAMG-Manual. The process of creating datasets in different domains can be divided into three stages: (1) Data Selection and Preprocessing, (2) QA Generation and Refinement, (3) Data Quality Check. The overview of this process is shown in Figure \ref{fig:data_construction}.
\subsection{Data Selection and Preprocessing}
We undertake three steps during this stage: (1) Collect, (2) Filter Data and Enhance Context, and (3) Refine.
%This stage is to collect and clean multimodal document information from various sources, resulting in documents where images and text are correctly aligned.

\subsubsection{Collect} The goal of this step is to collect MRAMG task-oriented multimodal document contextual data.
\vspace{-1mm}

\paragraph{Web Data.} This category includes Wikipedia pages and articles, derived from Wit dataset \citep{srinivasan2021wit}, WikiWeb2M dataset \citep{burns2023wikiweb2m}, and WebQA dataset \citep{chang2022webqa}. These datasets typically feature a rich integration of text and images, with the insertion order of images in Wikipedia, which serves as a reliable reference. We select val sets and test sets from Wit dataset \citep{srinivasan2021wit}, and sample 82k entries from the 100k entries in the test set of WikiWeb2M dataset \citep{burns2023wikiweb2m} , and choose img-posFacts set in WebQA dataset \citep{chang2022webqa}, as the initial collection of MRAMG-Wit dataset, MRAMG-Wiki dataset and MRAMG-Web dataset.
These datasets primarily consist of single-image documents with relatively short length. Consequently, we categorize the three Web Data datasets as \textbf{level 1: easy} (difficulty level), where models can easily handle our task in this domain due to simple scenarios.
\vspace{-1mm}
\paragraph{Academic Paper Data.} We collect 150 LaTeX source files and their corresponding PDF documents from the arXiv repository, with publication years spanning 2023 to 2024. These papers exhibit a seamless integration of text and images, providing an optimal foundation for tasks that demand precise image-text alignment. We manually select 110 high-quality papers with rich image-text interleaved content as the initial collection of MRAMG-Arxiv. It is defined as the \textbf{level 2: medium} dataset in our benchmark, where QA instances generally contain 1 to 4 images. Targeting at our task, it presents more challenges for models to retrieve and comprehend information from scientific document.
\vspace{-1mm}
\paragraph{Lifestyle Data.} Datasets in this domain features in fine-integration of a sequence of operation images and instruction texts. These dataset are thoroughly applicable to practical scenarios encountered in users' daily lives, which are collected from RecipeQA \citep{yagcioglu2018recipeqa}, 
ManualsLib\footnote{\url{https://www.manualslib.com}}
%\citep{manual}
and the Technical Dataset\footnote{\url{https://www.kaggle.com/datasets/ahmedhgabr/technical-illustration/data}}.
%\citep{manual}
%\citep{kaggle}.
We select the evaluation set from RecipeQA \citep{yagcioglu2018recipeqa}, and select 40 categories of instruction manuals with rich text-image content for subsequent construction of MRAMG-Recipe dataset and MRAMG-Manual dataset, respectively. We define Lifestyle Data as \textbf{level 3: difficult}, as they exhibit a significantly higher density of images and encompass diverse problem types, including single-image, no-image, and multi-image answers. 
Moreover, these datasets pose a significant challenge for models in determining the order of image insertion.
%More importantly, these datasets introduce a significant challenge for models related to the order of image insertion.
\subsubsection{Filter Data and Enhance Context}
The goal of this step is to clean data and acquire high-quality contexts, with images correctly inserted in the appropriate context positions.

For Web Data, we first filter out data entries with excessively long or short context, and data entries lacking images information or with overly large-sized images. We then implement two stages of deduplication: (1) MinHash-based deduplication \citep{shrivastava2014defense} method to remove data with similar images, and (2) string and semantic similarity measures \citep{TF-IDF,BGE} to further deduplicate data. Subsequently, we enhance the original contexts through crawling entire texts on original Wikipedia pages, and extending contexts leveraging GPT-4o based on the given information provided in the original data. For MRAMG-Web dataset, original collection contains questions where each question generally involves two entities and requires two distinct images for answering. We then extract entities included in the questions and retrieve entity-relevant texts in dataset as context supplements.

For Academic Paper Data, we engage ten graduate
students specializing in computer science, with expertise in fields such as Natural Language Processing (NLP) and Computer Vision (CV), to serve as expert annotators. 
Each annotator is assigned papers within their specific domain of expertise, ensuring precise and contextually relevant annotations. 
%Each annotator is assigned papers within their domain of expertise to ensure accurate and contextually meaningful annotations. 
The annotators meticulously read the selected papers, identify and extract textual content and corresponding figures that are most suitable for creating interleaved text-image contexts. This ensures accurate alignment of text and images in contexts.

For Lifestyle Data, in MRAMG-Recipe dataset, each recipe contains multiple step texts and each step text contains multiple images. We first combine multiple images into one image for each recipe step text and insert them at the end of each step text. We then aggregate all the recipe step texts as an entire image-text interleaved recipe context. In MRAMG-Manual dataset, we employ MinerU \citep{wang2024mineru} to parse the PDF manual documents into markdown format, preserving the original contextual order of images and text within each manual.
%in order to preserve the original contextual order of images and text within each manual. 
A team of annotators is tasked with removing sensitive or irrelevant information, adjusting image placements within the content, and correcting any errors in the text.
%A team of annotators is required to remove texts with sensitive or irrelevant information, adjust image positions in the manual texts, and correct errors existing in the texts.

\subsubsection{Refine}
The final step in this stage is to refine image-text interleaved contexts along with the questions or answers existing in the original data. We leverage both GPT-4o and human verification for quality check. This step aims to achieve two key objectives: correct and consistent image-text context content and appropriate image positions in contexts.


\subsection{QA Generation and Refinement}
This stage can be further divided into three steps: (1) question generation, (2) answer generation, (3) QA pair refine.
%This stage involves generating and refining question-answer pairs from processed multimodal documents across various datasets.

\subsubsection{Question Generation}
For MRAMG-Web dataset, we retain the questions contained in original dataset. As for MRAMG-Wiki, MRAMG-Wit and MRAMG-Recipe datasets, we leverage GPT-4o to generate high-quality, context-specific questions tailored to the provided text and images. Moreover, the questions in MRAMG-Manual are generated manually by human annotators. As for MRAMG-Arxiv dataset, we engage expert annotators to meticulously formulate questions. Overall, We adhere to follow two criteria: (1) The questions are constructed based on the corresponding contexts. (2) The questions must be natural and practical and they present potential to be answered with texts integrated with images.

\subsubsection{Answer Generation} 
For Web Data, given the generated or original questions $Q$ and corresponding context $C$, we generate image-text interleaved answers $A$ using GPT-4o following CoT reasoning strategy \citep{COT}: (1) \textbf{Question Validity Assessment}, we first ensure valid questions $ValQ$ can be directly answered by contexts $C$ and remove invalid questions $InvalQ$. (2) \textbf{Evidence Extraction}, we then extract evidences $E$ from contexts $C$ to substantiate the answer $A$ for later answer construction. (3) \textbf{Answer Construction}, we construct highly reliable, accurate and coherent answer based on valid questions $ValQ$ and evidences $E$. (4) \textbf{Image Integration} we finally enrich the constructed answers $A$ with a series of images integrated in the optimal positions. 

For Academic Paper Data, the expert annotators are responsible for reviewing previous formulated questions, constructing answers where images or tables play an integral role, and seamlessly integrating visual elements into the answer, to provide a cohesive and multimodal understanding of the content. For Lifestyle Data, in MRAMG-Recipe dataset, there are two types of questions: questions related to the entire recipe (recipe-specific) and questions only relevant to a specific step (step-specific). We combine all the step texts as the answers of recipe-specific questions and use answers generated by GPT-4o as the answers related to step-specific questions. We manually insert images at the end of answer texts. In MRAMG-Manual dataset, answers are meticulously formulated by annotators, to ensure accurate selection and order of images in each manual.

\subsubsection{QA Pair Refine}
We further refine the question-answer pair formulated in the previous steps following the optimization approach outlined by \citet{zhu2024rageval}.
We leverage GPT-4o to identify and extract evidences from contexts that directly supports the answer, while simultaneously recognizing keywords in the answer and checking their alignment with the extracted evidences. We refine the contradictions in the answers by cross-referencing the original answer and the extracted evidence, and discard irrelevant information to ensure the answer is contextually grounded.

\subsection{Data Quality Check}

To ensure a high quality and reliability of MRAMG, we further check the consistency and correctness of QA pairs and their alignment with context-related images. We engage annotators in a structured data quality check process.
The process is carried out in three stages:
\begin{itemize}
    \item In the first stage, a group of annotators review all QA pairs, assessing their correctness and consistency with both the context and image descriptions. They identify any problematic entries for further revision. Moreover, we also integrate GPT-4o for QA Evaluation, to further enhance the quality assurance process.
    \item In the second stage, a separate group of annotators correct any identified issues, including those related to image insertions, question formulations, context alignment, image descriptions, and answer correctness. Specifically, expert annotators are further tasked with a meticulous review and correcting over the MRAMG-Arxiv dataset.
    \item In the third stage, a team of 
    %auditors
    reviewers conduct a comprehensive recheck to ensure the overall accuracy of the dataset. This step validates that all corrections have been properly implemented and that the datasets meet the required quality standards.
\end{itemize}

%Overall, from the above procedures, we have formulated a total of 4800 QA pairs in MRAMG-Bench.
\subsection{Data Statistics}
\label{sec:statistics}

The MRAMG-Bench, summarized in Table~\ref{tab:statistics}, comprises 4,800 QA pairs across three distinct domains: Web Data, Academic Paper Data, and Lifestyle Data. 
It integrates six datasets—\textbf{MRAMG-Wit}, \textbf{MRAMG-Wiki}, \textbf{MRAMG-Web}, \textbf{MRAMG-Arxiv}, \textbf{MRAMG-Recipe}, and \textbf{MRAMG-Manual}—containing 4,346 documents and 14,190 images, with tasks categorized into three difficulty levels. 
Datasets like MRAMG-Recipe and MRAMG-Manual feature longer texts and higher image density, posing greater challenges compared to others.
The benchmark's query distribution captures varying levels of complexity:
Single-image QA primarily involves direct image-text associations.
Two-image QA requires comparative analysis or step-wise reasoning.
Three+ image QA focuses on complex multi-step processes.
Text-only QA emphasizes pure textual reasoning and serves to evaluate the model's ability to exclude irrelevant images when they are not required.



\begin{table*}[htbp]
  \centering
  \caption{The statistics of MRAMG-Bench. Multimodal Element Density refers to the proportion of images to text within a document, indicating image density relative to document length.}
  \vspace{-1mm}
    \resizebox{0.9\textwidth}{!}{
    \begin{tabular}{lccccccc}
    \toprule
    \textbf{DATASETS} & \textbf{MRAMG-Wit} & \textbf{MRAMG-Wiki} & \textbf{MRAMG-Web} & \textbf{MRAMG-Arxiv} & \textbf{MRAMG-Recipe} & \textbf{MRAMG-Manual} & \textbf{Overall} \\
    \midrule
    \textbf{Difficulty Level} & 1     & 1     & 1     & 2     & 3     & 3     & 1, 2, 3 \\
    \textbf{Queries} & 600   & 500   & 750   & 200   & 2360  & 390   & 4800 \\
    \textbf{Documents} & 639   & 538   & 1500  & 101   & 1528  & 40    & 4346 \\
    \textbf{Images} & 639   & 538   & 1500  & 337   & 8569  & 2607  & 14190 \\
    \textbf{Avg Images Per Doc} & 1.00  & 1.00  & 1.00  & 3.34  & 5.61  & 65.18 & 3.27 \\
    \textbf{Avg Length Per Doc} & 532.31 & 865.28 & 98.42 & 853.5 & 485.22 & 6365.4 & 468.37 \\
    \textbf{Multimodal Element Density} & 1.88$\times 10^{-3}$ & 1.16$\times 10^{-3}$ & 1.02$\times 10^{-2}$ & 3.91$\times 10^{-3}$ & 1.16$\times 10^{-2}$ & 1.02$\times 10^{-2}$ & 6.98$\times 10^{-3}$ \\
    \textbf{Text-Only QA} & 0     & 0     & 0     & 0     & 92    & 40    & 132 \\
    \textbf{Single-Image QA} & 600   & 500   & 0     & 184   & 1481  & 92    & 2858 \\
    \textbf{Two-Image QA} & 0     & 0     & 750   & 12    & 60    & 126   & 948 \\
    \textbf{Three+ Images QA} & 0     & 0     & 0     & 3     & 727   & 132   & 862 \\
    \bottomrule
    \end{tabular}%
    }
  \label{tab:statistics}%
\end{table*}%



%\section{Data Statistics}
\label{sec:statistics}

The MRAMG-Bench, summarized in Table~\ref{tab:statistics}, comprises 4,800 QA pairs across three distinct domains: Web Data, Academic Paper Data, and Lifestyle Data. 
It integrates six datasets—\textbf{MRAMG-Wit}, \textbf{MRAMG-Wiki}, \textbf{MRAMG-Web}, \textbf{MRAMG-Arxiv}, \textbf{MRAMG-Recipe}, and \textbf{MRAMG-Manual}—containing 4,346 documents and 14,190 images, with tasks categorized into three difficulty levels. 
Datasets like MRAMG-Recipe and MRAMG-Manual feature longer texts and higher image density, posing greater challenges compared to others.
The benchmark's query distribution captures varying levels of complexity:
Single-image QA primarily involves direct image-text associations.
Two-image QA requires comparative analysis or step-wise reasoning.
Three+ image QA focuses on complex multi-step processes.
Text-only QA emphasizes pure textual reasoning and serves to evaluate the model's ability to exclude irrelevant images when they are not required.
%The benchmark's query distribution reflects varying levels of complexity. Single-image QA includes 2,858 queries, primarily involving straightforward image-text associations. Two-image QA, consisting of 948 queries, requires comparative analysis or step-wise reasoning. Three+ image QA encompasses 862 queries, focusing on complex processes. Text-only QA, with 132 queries, emphasizes tasks exclusively based on textual reasoning. A specific subset is designed to assess the model’s ability to avoid inserting irrelevant images when they are not needed.
%The benchmark integrates six datasets: \textbf{MRAMG-Wit}, \textbf{MRAMG-Wiki}, \textbf{Web-MQA+}, \textbf{MRAMG-Arxiv}, \textbf{MRAMG-Recipe}, and \textbf{MRAMG-Manual}. Together, these datasets contain 4,346 documents and 14,190 images, with tasks stratified into three levels of difficulty. For example, MRAMG-Recipe and MRAMG-Manual not only feature longer textual content but also exhibit higher image density, making them more challenging compared to others like Web-MQA+.
%These datasets represent diverse real-world scenarios, including online information retrieval, scientific literature analysis, and daily activity planning, providing a comprehensive foundation for evaluating MRAMG task.


\begin{table*}[htbp]
  \centering
  \caption{The statistics of MRAMG-Bench. Multimodal Element Density refers to the proportion of images to text within a document, indicating image density relative to document length.}
  \vspace{-1mm}
    \resizebox{0.9\textwidth}{!}{
    \begin{tabular}{lccccccc}
    \toprule
    \textbf{DATASETS} & \textbf{MRAMG-Wit} & \textbf{MRAMG-Wiki} & \textbf{MRAMG-Web} & \textbf{MRAMG-Arxiv} & \textbf{MRAMG-Recipe} & \textbf{MRAMG-Manual} & \textbf{Overall} \\
    \midrule
    \textbf{Difficulty Level} & 1     & 1     & 1     & 2     & 3     & 3     & 2.15 \\
    \textbf{Queries} & 600   & 500   & 750   & 200   & 2360  & 390   & 4800 \\
    \textbf{Documents} & 639   & 538   & 1500  & 101   & 1528  & 40    & 4346 \\
    \textbf{Images} & 639   & 538   & 1500  & 337   & 8569  & 2607  & 14190 \\
    \textbf{Avg Images Per Doc} & 1.00  & 1.00  & 1.00  & 3.34  & 5.61  & 65.18 & 3.27 \\
    \textbf{Avg Length Per Doc} & 532.31 & 865.28 & 98.42 & 853.5 & 485.22 & 6365.4 & 468.37 \\
    \textbf{Multimodal Element Density} & 1.88$\times 10^{-3}$ & 1.16$\times 10^{-3}$ & 1.02$\times 10^{-2}$ & 3.91$\times 10^{-3}$ & 1.16$\times 10^{-2}$ & 1.02$\times 10^{-2}$ & 6.98$\times 10^{-3}$ \\
    \textbf{Text-Only QA} & 0     & 0     & 0     & 0     & 92    & 40    & 132 \\
    \textbf{Single-Image QA} & 600   & 500   & 0     & 184   & 1481  & 92    & 2858 \\
    \textbf{Two-Image QA} & 0     & 0     & 750   & 12    & 60    & 126   & 948 \\
    \textbf{Three+ Images QA} & 0     & 0     & 0     & 3     & 727   & 132   & 862 \\
    \bottomrule
    \end{tabular}%
    }
  \label{tab:statistics}%
\end{table*}%




\section{A General Framework for MRAMG}
\label{sec:framework}

In this section, we present our generation framework, which consists of two stages: (1) the retrieval of relevant documents, and (2) the generation of multimodal answers based on the retrieved information, utilizing a foundational generative model. In the second stage, we propose three distinct strategies for answer generation: (a) LLM-based method: directly generating multimodal answers using LLMs, (b) MLLM-based method: directly generating multimodal answers using MLLMs, and (c) rule-based method: first obtaining a pure text answer from generative large models and then applying our matching-based algorithm to insert appropriate images into the text answer, resulting in multimodal answers. All the prompts we used can be found at: https://huggingface.co/MRAMG.

\subsection{Retrieving Relevant Documents}
%In accordance with the standard retrieval-augmented generation (RAG) framework \citep{lewis2020retrieval}, 
We employ the BGE-M3 embedding model \citep{chen2024bge} to select the top-$k$ most relevant documents by calculating the cosine similarity between the user query $q$ and each document $d_i$ in the embedding space.
Then, we sequentially concatenate the text blocks and image blocks from each selected document, resulting in the multimodal information to be used: the text information $\mathcal{T}_q^* = \{t_1, t_2, \dots, t_l\}$ and the image infomation $\mathcal{I}_q^* = \{i_1, i_2, \dots, i_n\}$.

\subsection{Generating Multimodal Answer}

% To generate multimodal answers, we consider three different strategies. 
%For LLMs, we use the first and third strategies, while for MLLMs, we use the second and third strategies. Below, we provide a detailed explanation of the three strategies. The first two methods are similar to concurrent works \citet{zhu2024murar} and \citet{ma2024multi}.


\subsubsection{LLM-Based Method}

We first consider how to generate multimodal answers using LLMs. Since LLMs cannot directly process image formats such as images for understanding, we choose an alternative method. 
In fact, we notice that the captions and surrounding context of images often offer detailed descriptions and supplementary information, which enhance the understanding of the images in the context of the query. Thus, we leverage the contextual information of images within the text context, as well as the captions of the images themselves, as textual substitute for the images when inputting into the language model. Specifically, we insert image placeholders $p_1, p_2, \dots, p_n$ at the corresponding positions of the retrieved text information $\mathcal{T}_q^*$, resulting in a new context $\mathcal{C}_q^* = \{t_1, t_2, \dots, t_{l_1}, p_1, t_{l_1 + 1}, \dots, t_{l_j}, p_j, t_{l_j +1}, \dots\}$. We consider that the context $t_{l_j}$ and $t_{l_j + 1}$ around the placeholder $p_j$ for the $j$-th image $i_j$ provide a suitable textual description for that image. Then, we generate the multimodal answer as $\mathcal{A} = \mathcal{M}(\mathcal{P}_{llm}, q, \mathcal{C}_q^{*})$, where $\mathcal{P}_{llm}$ is the prompt, and the images in the output are represented as placeholders.

\subsubsection{MLLM-Based Method}

For multimodal large language models (MLLMs), we directly input the textual information $\mathcal{T}_q^{*}$ and the image information $\mathcal{I}_q^{*}$. However, current open-source or closed-source MLLMs typically impose limits on the number of input images. Therefore, we are unable to use all the images in $\mathcal{I}_q^{*}$ as input. To address this, we propose leveraging the CLIP-based multimodal embedding model \citep{koukounas2024jina} to compute the embedding similarity between each image and the query, and then selecting the top-N images based on the similarity for input, denoted as $\operatorname{TopN}(q, \mathcal{I}_q^*)$. Formally, for a given MLLM $\mathcal{M}$, we generate the multimodal answer as $\mathcal{A} = \mathcal{M}(\mathcal{P}_{mllm}, \mathcal{T}_q^{*}, \operatorname{TopN}(q, \mathcal{I}_q^*))$, where $\mathcal{P}_{mllm}$ is the prompt, and the images in the output are also represented as placeholders.

\subsubsection{Rule-Based Method}

Here, we consider an alternative strategy for generating multimodal answers. Specifically, we first allow the generative model to produce a pure textual answer based on the provided context. Then, we employ a matching-based insertion algorithm to select appropriate images from the retrieved set and insert them at suitable positions within the textual answer, thereby forming the multimodal answer.

\textbf{Step 1: Generate Text Answer.} We first use large generative model $\mathcal{M}$ to generate a pure text answer $\mathcal{A}_{txt}=\mathcal{M}(\mathcal{P}_{txt},\mathcal{T}_q^*)$, where $\mathcal{P}_{txt}$ denotes the text-answer-generating prompt. Subsequently, we segment the pure textual answer into sentences, i.e., $\mathcal{A}_{txt} = \{s_1, s_2, \dots, s_m\}$. 

\textbf{Step 2: Construct Bipartite Graph.} We then consider constructing the following bipartite graph $\mathcal{B} = (\mathcal{S}, \mathcal{I}, \mathcal{E})$, where $\mathcal{S} = \{s_1, s_2, \dots, s_m\}$ represents the set of representative vertices corresponding to the sentences in all pure text answers, and $\mathcal{I} = \{i_1, i_2, \dots, i_n\}$ represents the set of representative vertices corresponding to each retrieved image. Here, we construct the edge set $\mathcal{E}$ by calculating the string similarity $\alpha_{k,l}$ (using BLEU method \citep{papineni2002bleu}) and semantic similarity $\beta_{k,l}$ (using BGE-M3 model \citep{chen2024bge}) for each sentence-image pair $(s_k, i_l)$. Specifically, to filter out irrelevant images, we define the edge set as $\mathcal{E} = \{(k, l) \mid \alpha_{k,l} > \alpha^* \text{ or } \beta_{k,l} > \beta^*\}$, where $\alpha^*, \beta^*$ are two hyper-parameters as thresholds, and each egde $(k,l)$ has the weight $w_{k,l} = \lambda\alpha_{k,l} + (1-\lambda)\beta_{k,l}$, where $\lambda\in [0,1]$ is a hyper-parameter.

\textbf{Step 3: Insert Images By Matching Algorithms.} Indeed, we aim to solve the following optimization problem:

\begin{equation}
\label{problem:matching}
    \operatornamewithlimits{max}_{M \subset \mathcal{E}} \sum_{(k,l) \in M} w_{k,l} ,
\end{equation}
where $M$ is a subset of the edge set $\mathcal{E}$. Specifically, we consider two types of constraints for $M$:

\begin{itemize}
    \item Each image can be inserted at most once, but multiple images are allowed to be inserted after the same sentence.
    \item Each image can be inserted at most once, and each sentence can have at most one image inserted after it.
\end{itemize}

For the first constraint, we can solve for the optimal solution to problem (\ref{problem:matching}) as $M^* = \left\{(k^*, l) \mid k^* = \operatornamewithlimits{argmax}_{(k,l) \in \mathcal{E}} w_{k,l}, \forall l \in [n] \right\}$. For the second constraint, $M$ forms a matching of the bipartite graph $\mathcal{B}$. We use the standard Edmonds' Blossom algorithm \citep{Edmonds_1965} to solve the weighted maximum bipartite graph matching problem in this case. Finally, we insert the images into the output in the form of placeholders. We test the performance of the two matching-based insertion algorithms on the web and manual datasets, with results shown in Figure~\ref{fig:enter-label}.
Overall, the weighted bipartite graph matching algorithm, which restricts each sentence to have at most one image inserted after it, performed better. Therefore, we adopted this method for subsequent large-scale experimental evaluations.


\begin{figure}
    \centering
\includegraphics[width=0.95\linewidth]{Fig/comparison_graphs.pdf}
    \caption{Comparison of the generative performance of two matching algorithms on different datasets.}
    \label{fig:enter-label}
\end{figure}

\section{Experiments}
\label{sec:experiments}

\begin{table}[tb!]
\caption{Retrieval performance of the BGE-M3 model on the MRAMG-Bench.}
  \centering
  \vspace{-1mm}
\resizebox{0.49\textwidth}{!}{
    \begin{tabular}{ccccccc}
    \toprule
    \textbf{Metrics} & MRAMG-Wit & MRAMG-Wiki & MRAMG-Web & MRAMG-Arxiv & MRAMG-Recipe & MRAMG-Manual \\
    \midrule
    \textbf{Image Recall@10} & 0.99  & 0.99  & 0.99  & 0.94  & 0.87  & 0.72 \\
    \textbf{Context  Recall@10} & 0.9   & 0.89  & 0.95  & 0.83  & 0.91  & 0.87 \\
    \bottomrule
    \end{tabular}%

  \label{tab:retrieve}%
}
  \label{tab:latest1}%
\end{table}%


\begin{table*}[htbp]
  \centering
  \caption{Comprehensive performance results on MRAMG-Bench. Prec., Rec., B.S.,Ord., Comp., Pos., and Avg. represent image precision, image recall, BERTScore,image ordering score, comprehensive score, image position score, and average score, respectively. The highest scores for each group are highlighted in bold.}
  \vspace{-1mm}
  \resizebox{\textwidth}{!}{
    \begin{tabular}{cl|cccccc|cccccc|ccccccc}
    \toprule
    \multirow{2}[4]{*}{\textbf{Framework}} & \multicolumn{1}{c|}{\multirow{2}[4]{*}{\textbf{Model}}} & \multicolumn{6}{c|}{\textbf{ Web Data}}       & \multicolumn{6}{c|}{\textbf{Academic Paper Data}} & \multicolumn{7}{c}{\textbf{ Lifestyle Data}} \\
\cmidrule{3-21}          &       & \textbf{Prec.} & \textbf{Rec.} & \textbf{B.S.} & \textbf{Comp.} & \textbf{Pos.} & \textbf{Avg.} & \textbf{Prec.} & \textbf{Rec.} & \textbf{B.S.} & \textbf{Comp.} & \textbf{Pos.} & \textbf{Avg.} & \textbf{Prec.} & \textbf{Rec.} & \textbf{B.S.} & \textbf{Ord.} & \textbf{Comp.} & \textbf{Pos.} & \textbf{Avg.} \\
    \midrule
    \multicolumn{1}{c}{\multirow{11}[2]{*}{\textbf{Rule-Based}}} & \multicolumn{1}{p{4.125em}|}{GPT-4o } & 43.54 & 37.30 & 92.35 & 77.15 & 43.86 & 50.75 & 55.42 & 63.04 & 94.67 & 84.20 & 75.75 & 68.02 & 47.04 & 63.54 & 92.01 & 43.54 & 79.17 & \textbf{77.13} & \textbf{65.68} \\
          & GPT-4o-mini  & 38.20 & 33.08 & 91.10 & 76.59 & 38.57 & 46.87 & 51.71 & 59.29 & 94.36 & 85.20 & 73.75 & 66.07 & 49.14 & 61.52 & 91.13 & \textbf{43.87} & 79.32 & 75.89 & 64.92 \\
          & Claude-3.5-Sonnet  & 47.65 & 38.43 & 93.42 & \textbf{80.63} & 48.14 & 53.55 & 55.17 & 62.79 & 94.09 & 84.20 & 75.75 & 67.48 & \textbf{50.32} & 61.17 & 92.02 & 43.32 & \textbf{79.50} & 76.39 & 64.91 \\
          & Gemini-1.5-Pro  & 31.27 & 26.62 & 87.37 & 74.83 & 31.76 & 41.21 & 52.43 & 56.29 & 93.85 & 83.20 & 70.28 & 64.15 & 49.18 & 50.57 & 88.32 & 38.73 & 78.14 & 73.14 & 60.58 \\
          & DeepSeek-V3  & \textbf{55.49} & \textbf{46.62} & \textbf{94.12} & 80.18 & \textbf{56.16} & \textbf{59.04} & \textbf{56.12} & \textbf{67.29} & \textbf{94.90} & \textbf{84.30} & \textbf{78.46} & \textbf{70.05} & 27.07 & 57.56 & \textbf{92.42} & 24.07 & 74.19 & 65.19 & 57.22 \\
          & Qwen2-VL-7B-Instruct  & 37.98 & 34.81 & 91.26 & 70.99 & 38.28 & 46.32 & 49.17 & 52.17 & 92.09 & 78.90 & 67.08 & 60.88 & 43.79 & 60.93 & 91.49 & 39.52 & 77.70 & 77.33 & 63.87 \\
          & Qwen2-VL-72B-Instruct  & 34.81 & 31.49 & 88.70 & 71.89 & 35.14 & 43.75 & 45.42 & 48.71 & 92.39 & 79.60 & 65.42 & 59.50 & 31.82 & 49.30 & 89.88 & 25.51 & 72.92 & 70.94 & 56.45 \\
          & InternVL-2.5-8B  & 30.78 & 28.38 & 87.45 & 69.88 & 31.05 & 40.55 & 39.20 & 48.29 & 91.42 & 76.70 & 65.17 & 58.16 & 29.39 & 51.47 & 90.11 & 23.28 & 72.76 & 71.29 & 55.79 \\
          & InternVL-2.5-78B  & 38.79 & 34.49 & 90.97 & 74.82 & 39.06 & 47.04 & 52.21 & 62.00 & 94.51 & 85.40 & 75.38 & 67.57 & 22.61 & \textbf{67.71} & 92.19 & 19.41 & 74.95 & 56.55 & 56.26 \\
          & Llama-3.1-8B-Instruct  & 23.86 & 20.54 & 82.64 & 58.40 & 26.15 & 33.63 & 21.50 & 23.08 & 85.92 & 58.70 & 29.00 & 35.32 & 28.23 & 36.25 & 81.29 & 18.33 & 65.12 & 60.90 & 46.44 \\
          & Llama-3.3-70B-Instruct & 48.84 & 41.73 & 92.87 & 77.60 & 49.59 & 54.31 & 53.00 & 58.17 & 94.39 & 83.60 & 73.42 & 65.55 & 30.27 & 50.38 & 92.91 & 25.33 & 73.08 & 69.53 & 57.30 \\
    \midrule
    \multicolumn{1}{c}{\multirow{8}[2]{*}{\textbf{MLLM-Based}}} & GPT-4o  & 82.75 & 80.57 & \textbf{94.70} & 85.95 & 84.65 & 79.66 & \textbf{60.39} & 74.29 & \textbf{95.15} & 87.50 & \textbf{90.39} & \textbf{76.87} & \textbf{43.77} & 44.68 & \textbf{92.50} & \textbf{32.47} & 81.36 & \textbf{77.35} & 61.01 \\
          & GPT-4o-mini  & 69.98 & 86.08 & 94.59 & 80.58 & 72.93 & 76.29 & 36.17 & 74.79 & 95.08 & 83.20 & 74.66 & 68.62 & 29.34 & 47.71 & 91.71 & 21.95 & 77.19 & 69.95 & 56.32 \\
          & Claude-3.5-Sonnet  & \textbf{91.15} & \textbf{93.68} & 93.73 & \textbf{86.70} & \textbf{93.71} & \textbf{85.51} & 47.12 & \textbf{83.50} & 94.65 & \textbf{87.60} & 86.38 & 74.84 & 29.35 & 52.09 & 90.92 & 21.88 & 79.85 & 74.80 & 57.71 \\
          & Gemini-1.5-Pro  & 89.27 & 90.49 & 92.13 & 83.77 & 91.05 & 83.30 & 58.13 & 80.25 & 94.30 & 85.90 & 83.61 & 75.14 & 38.59 & \textbf{57.40} & 90.05 & 31.99 & \textbf{81.37} & 69.84 & \textbf{61.58} \\
          & Qwen2-VL-7B-Instruct  & 26.48 & 32.65 & 87.51 & 60.98 & 30.24 & 40.19 & 1.63  & 4.00  & 84.62 & 49.80 & 4.46  & 21.11 & 9.66  & 15.15 & 84.84 & 4.26  & 55.92 & 16.22 & 26.69 \\
          & Qwen2-VL-72B-Instruct  & 59.65 & 60.59 & 92.69 & 80.50 & 61.49 & 64.16 & 31.99 & 44.87 & 93.53 & 84.20 & 55.16 & 55.56 & 19.61 & 26.25 & 91.21 & 12.36 & 74.36 & 39.94 & 41.36 \\
          & InternVL-2.5-8B  & 52.71 & 65.86 & 91.89 & 72.78 & 61.17 & 63.30 & 12.22 & 27.87 & 83.72 & 58.10 & 28.49 & 35.34 & 22.19 & 37.94 & 89.41 & 14.54 & 73.99 & 60.11 & 48.22 \\
          & InternVL-2.5-78B & 75.50 & 76.27 & 93.98 & 84.11 & 78.68 & 74.96 & 36.62 & 55.00 & 94.47 & 84.80 & 64.11 & 61.01 & 21.43 & 29.09 & 91.11 & 13.53 & 75.43 & 51.99 & 45.23 \\
    \midrule
    \multicolumn{1}{c}{\multirow{7}[2]{*}{\textbf{LLM-Based}}} & GPT-4o  & 80.86 & 77.92 & \textbf{94.92} & 85.74 & 81.41 & 77.73 & \textbf{65.28} & 76.54 & \textbf{95.23} & 88.90 & 84.84 & 76.91 & 47.48 & 62.40 & 92.29 & 42.55 & 80.60 & 80.43 & 66.11 \\
          & GPT-4o-mini  & 69.58 & 90.95 & 94.35 & 81.29 & 70.96 & 76.91 & 37.69 & 83.33 & 95.01 & 84.50 & 69.07 & 69.91 & 44.36 & 38.27 & \textbf{92.69} & 31.16 & 83.24 & 54.96 & 53.33 \\
          & Claude-3.5-Sonnet  & 92.47 & 93.86 & 94.32 & \textbf{86.17} & 93.43 & 85.88 & 62.17 & \textbf{88.00} & 94.37 & \textbf{89.60} & \textbf{88.17} & \textbf{79.22} & 59.83 & 64.45 & 91.51 & 51.51 & \textbf{84.63} & \textbf{82.58} & 69.04 \\
          & Gemini-1.5-Pro  & \textbf{93.63} & 93.84 & 93.48 & 84.35 & \textbf{94.29} & 85.85 & 59.85 & 78.63 & 94.32 & 87.60 & 80.15 & 75.00 & \textbf{62.23} & \textbf{68.34} & 90.83 & \textbf{54.62} & 83.10 & 79.66 & \textbf{70.80} \\
          & DeepSeek-V3  & 93.08 & \textbf{95.51} & 94.68 & 85.64 & 93.98 & \textbf{86.45} & 46.57 & 81.13 & 94.70 & 87.50 & 70.01 & 72.53 & 45.71 & 67.56 & 91.78 & 40.35 & 82.64 & 77.03 & 66.04 \\
          & Llama-3.1-8B-Instruct  & 28.87 & 31.35 & 82.53 & 52.59 & 31.31 & 38.94 & 1.50  & 2.00  & 80.61 & 43.40 & 4.00  & 18.36 & 11.71 & 12.75 & 75.36 & 6.21  & 40.97 & 17.15 & 23.23 \\
          & Llama-3.3-70B-Instruct & 74.26 & 95.49 & 94.35 & 82.44 & 76.01 & 79.35 & 38.78 & 84.88 & 95.01 & 83.40 & 64.59 & 68.93 & 35.29 & 69.34 & 91.89 & 30.60 & 80.15 & 70.66 & 61.86 \\
    \bottomrule
    \end{tabular}%
    }
  \label{table:main_results}%
\end{table*}%

%Now, we present our evaluation metric, experiment settings and results.

\subsection{Evaluation Metric}

\subsubsection{Retrieve Evaluation}

To evaluate the retrieval performance, 
We consider the following metrics:

\begin{itemize}
    \item 
    \textbf{Context Recall \citep{es2023ragas}} uses LLMs to evaluate whether the retrieved documents contain all the relevant  information required for answer generation.
    \item 
    \textbf{Visual Recall} measures the percentage of retrieved images relative to the total number of images in the ground truth.
    % It is computed as:
    % \[
    % \text{Visual Recall} = \frac{\text{Retrieved Relevant Images}}{\text{Total Relevant Images in Ground Truth}}
    % \]
    % where ``Retrieved Relevant Images" refers to the number of images retrieved that are present in the ground truth, and ``Total Relevant Images in Ground Truth" refers to the total number of relevant images that should have been retrieved.
\end{itemize}

\subsubsection{Generation Evaluation}

To evaluate performance of multimodal answers, we consider the following metrics\footnote{See detailed metrics at: https://huggingface.co/MRAMG
.}, which can be divided into two categories: statistical-based metrics (first six metrics) and LLM-based metrics (last four metrics). %We use the following \textit{statistical-based metrics}:

\begin{itemize}
    \item 
    \textbf{Image Precision} measures the percentage of correct images in the multimodal answer relative to the total number of inserted images, assessing whether irrelevant images were introduced. 
    % It is computed as:
    % \[
    % \text{Image Precision} = \frac{\text{True Positives}}{\text{True Positives} + \text{False Positives}}
    % \]
    % where True Positives are the correctly inserted images, and False Positives are irrelevant images that were included.

    \item 
    \textbf{Image Recall} measures the percentage of correct images in the multimodal answer relative to the total number of images in the ground truth, evaluating whether the answer effectively includes useful image information. 
    % It is computed as:
    % \[
    % \text{Image Recall} = \frac{\text{True Positives}}{\text{True Positives} + \text{False Negatives}}
    % \]
    % where False Negatives are the images in the ground truth that were omitted in the generated multimodal answer.

    \item 
    \textbf{Image F1 Score} is the harmonic mean of Precision and Recall, providing an overall evaluation of the image quality in the multimodal answer. 
    % It is calculated as:
    % \[
    % \text{Image F1 Score} = 2 \times \frac{\text{Image Precision} \times \text{Image Recall}}{\text{Image Precision} + \text{Image Recall}}
    % \]
    \item 
    \textbf{Image Ordering Score} evaluates whether the order of images inserted into the multimodal answer matches the order of images in the ground truth \footnote{Here, we apply the ordering score only to lifestyle data (MRAMG-Recipe and MRAMG-Manual), as instances in these datasets typically contain a large number of images, and the order of the images is crucial.}. Specifically, we compute the weighted edit distance between the two image sequences. % to reflect the difference in their order. %See more details in Appendix \ref{appendix:ordering_score}.
    \item 
    \textbf{Rouge-L \citep{lin-2004-rouge}} is a text generation evaluation metric based on the longest common subsequence, measuring the structural similarity between the answer and the ground truth.
    \item 
    \textbf{BERTScore \citep{zhang2019bertscore}} is a text generation evaluation metric based on BERT \citep{devlin2018bert}, used to assess the semantic similarity between the text in the answer and the ground truth.
    \item 
    \textbf{Image Relevance} measures evaluate the relevance of the inserted image to the query-answer pair, specifically assessing whether the content described by the image is meaningfully related to the content of the QA \citep{zhu2024murar,ma2024multi}. %This metric assigns a score to each image appearing in the answer, with scores ranging from 1 to 5.
    \item 
    \textbf{Image Effectiveness} measures evaluate the effectiveness of the images inserted into the multimodal answer, assessing whether the images align with the QA content and contribute to the understanding of the answer \citep{zhu2024murar,ma2024multi}. 
    %This metric also assigns a score to each image, with scores ranging from 1 to 5.
    \item 
    \textbf{Image Position Score} is used to assess the appropriateness of the image placement in the multimodal answer. 
    %It assigns a score of either 0 or 1 to each image, based on whether its position is deemed correct and suitable.
    \item 
    \textbf{Comprehensive Score} reflects overall quality of the multimodal answer, evaluating whether the answer appropriately addresses the query and maintains overall coherence. %It particularly considers whether the insertion of images enhances the answer, making it visually engaging and more expressive. 
    %This metric assigns a score to the complete answer, with scores ranging from 1 to 5.
\end{itemize}
\vspace{-5pt}
\subsection{Experimental Baselines and Settings}
To further validate our dataset and generation framework, we consider testing the generation performance of multimodal answers across a wide range of LLMs and MLLMs. 

\paragraph{Model Baselines} We evaluate 4 closed-source models and 7 open-source models. Specifically, we select 4 popular closed-source multimodal large models: GPT-4o \citep{gpt4o}, GPT-4o-mini \citep{gpt4o_mini}, Claude-3.5-Sonnet \citep{anthropic_claude_2024}, Gemini-1.5-Pro \citep{team2024gemini}. For the open-source models, we choose 4 multimodal large models: Qwen2-VL-7B-Instruct \citep{wang2024qwen2}, Qwen2-VL-72B-Instruct \citep{wang2024qwen2}, InternVL-2.5-8B \citep{chen2024expanding}, InternVL-2.5-78B \citep{chen2024expanding} and 3 text-based large models: 
DeepSeek-V3 \citep{liu2024deepseek}, Llama-3.1-8B-Instruct \citep{dubey2024llama} and Llama-3.3-70B-Instruct \citep{dubey2024llama}. Among these, for the four most widely used closed-source multimodal models, we test all the llm/mllm/rule-based strategies in our framework. For the four other multimodal models, we test the mllm/rule-based strategies. For the remaining three text-based models, we test the llm/rule-based strategies.

\paragraph{Experiment Details} In the multimodal documents, images are replaced with <PIC> placeholders and each document is chunked using a size of 256 with the SentenceSplitter \citep{Liu_LlamaIndex_2022} to ensure efficient processing. During the retrieval phase, we use the BGE-M3 model \citep{chen2024bge} to retrieve the top-k $= 10$ relevant chunks, which contain the corresponding images. Concretely, images presented within the retrieved chunks are regarded as the corresponding retrieved images. In the evaluation stage, we utilize GPT-4o \citep{gpt4o} as the judging model to assess the performance of LLM-based metrics.
\vspace{-10pt}
% Here, we present the details of our experiment. In the retrieval stage, we first chunk the multimodal document using a chunk size of 256, ensuring that placeholders replace images to prevent them from being split across chunks. The BGE-M3 embedding model \citep{chen2024bge} is then employed to calculate the cosine similarity between each chunk and the query, retrieving the top-10 relevant chunks based on this similarity. During this process, image retrieval is handled by referencing the textual context surrounding the images. In the evaluation stage, we use GPT-4o \citep{gpt4o} as the judging model to assess the performance of LLM-based metrics.
% Here, we will present our experiment details. In the retrieve stage, We use SentenceSplitter \citep{Liu_LlamaIndex_2022} to chunk the multimodal document, and then employ the BGE-M3 embedding model \citep{chen2024bge} to calculate the cosine similarity between each chunk and the query for retrieval. In this process, images in the multimodal document are replaced with placeholders, and image retrieval is conducted based on the textual context surrounding the images. In the evaluation stage, we utilize GPT-4o \citep{gpt4o} as the judging model to assess the performance of LLM-based metrics.
% \qinhan{是介绍清楚chunk size = 256
% 对于doc中的图片替换为占位符,使用chunk size =256进行切块,(为了避免pic占位符被切割,使用SentenceSplitter 可写可不写)
% 对于检索阶段,Embedding model选择bge-m3,top-k =10
% 图片通过检索到到相关chunk被召回

% }
% \binghui{切chunk}

% \binghui{图片如何retrieve}


\subsection{Experiment Results}

% Now, we present our experiment results about retrieval performance and generation performance.

\subsubsection{Retrieval Performance}

As shown in Table \ref{tab:retrieve}, we observe that both image recall and context recall perform well for Web Data %(including MRAMG-Wit, MRAMG-Wiki, and MRAMG-Web)
, with MRAMG-Web achieving particularly high scores of 0.99 and 0.95, respectively. 
In contrast, both metrics progressively decline for the three other datasets
%the MRAMG-Arxiv, MRAMG-Recipe, and MRAMG-Manual datasets
, which can be attributed to their longer text passages and higher image counts. Nevertheless, overall retrieval effectiveness remains strong. 
\vspace{-8pt}
\subsubsection{Generation Performance}
\label{generation}

In this section, we evaluate generation performance, with results presented for three domains (web/academic paper/lifestyle). Due to space limitations, full results are available at: https://huggingface.co/MRAMG.


\textbf{Generation Performance Across Different Datasets.} From the results in Table \ref{table:main_results}, it is evident that the overall generation performance significantly decreases as the dataset complexity increases, which aligns with our expectations regarding the varying difficulty of different datasets. Notably, the performance of rule-based methods on simpler datasets is basically suboptimal, which can be attributed to data characteristics, particularly related to the Avg Images Per Doc metric. According to Table \ref{tab:statistics}, this metric is 1 for all the Web Data datasets 
%(MRAMG-Wit, MRAMG-Wiki, MRAMG-Web)
, whereas in the other datasets, it is substantially higher. With fewer images per document, the rule-based illustration algorithm struggles to effectively associate images with sentences, as multiple sentences often correspond to a single image. This results in ambiguous correlations and complicates threshold selection, leading to a higher error rate. In contrast, as the number of images increases, the gap in correlation between rule-based methods and other approaches becomes more pronounced. Under these conditions, rule-based matching performs better due to improved alignment with sentence-image relationships.

\textbf{Generation Performance Across Different Models.} As shown in Table \ref{table:main_results}, advanced models such as Gemini, Claude, GPT-4o, and Deepseek-V3 consistently outperform smaller open-source models ($\sim7$B parameters) across all domains and methods. These smaller models exhibit subpar performance across different methods and dataset domains, even when utilizing rule-based generation techniques. In contrast, larger open-source models ($\sim70$B parameters) significantly reduce the performance gap with closed-source models, achieving results within approximately a tenth of the margin on simpler datasets, such as Web Data. For example, InternVL-2.5-78B and Llama-3.3-70B-Instruct attain Image Precision scores of 75.5 and 74.26, respectively, with average scores surpassing 70. On more challenging datasets, however, the performance gap becomes more pronounced, highlighting the limitations of open-source models in handling complex MRAMG tasks. Specifically, when employing MLLM-based methods, Qwen2-VL-72B-Instruct achieves an Image Recall score of only 26.25, which is significantly lower than Gemini's 57.4. This result underscores the challenges faced by even 70B-scale models in accurately identifying images in complex scenarios.  
Nevertheless, smaller open-source models remain a cost-effective solution for simpler applications with limited computational resources. In particular, the order metric for the Lifestyle domain demonstrates poor performance across all models and methods, with none achieving a passing score. Notably, GPT-4o performs best under both the Rule-based and MLLM-based methods, scoring 43.54 and 32.47, respectively, while the Gemini model attains the highest score of 54.62 under the LLM-based method. However, even these scores fall short of the passing threshold, underscoring the ambiguity in LLM and MLLM models regarding the concept of image insertion order, which remains an unresolved challenge.




















% \textbf{Generation Performance Across Different Generation Methods.}
% In comparing different methods, an overall performance trend emerges:
% \begin{equation}
%     \text{LLM-based} > \text{MLLM-based} > \text{Rule-based} . \nonumber
% \end{equation}
% For example, on Web Data, the Gemini model achieves average scores of 85.85, 83.3, and 41.21 for LLM-based, MLLM-based, and Rule-based methods, respectively.

% LLM-based method: By integrates the context surrounding images into the generation process, this method achieve natural and precise image insertion and highlighting the critical role of image context in ensuring insertion accuracy. 

% MLLM-based methods: While effective on simpler datasets like Web, performance deteriorates on more challenging datasets due to the increased difficulty of distinguishing between visually similar images, revealing the current limitations of model capabilities. In this method, the Gemini model achieves the highest overall scores on the Web and Lifestyle datasets, scoring 85.51 and 61.58, respectively, while Claude leads on the Academic Paper Data with a score of 76.87.

% Rule-based methods: While these methods exhibit significantly lower performance compared to model-based approaches on simpler datasets, the performance gap diminishes as the dataset complexity increases. In the lifestyle domain, rule-based methods even outperform certain model-based approaches. For instance, GPT-4o achieves an overall score of 65.68 using rule-based methods, surpassing its MLLM-based score of 61.01. Notably, the Deepseek-V3 model demonstrates outstanding performance on Web and Academic Paper data, achieving top scores across almost all metrics. This suggests that for simpler datasets, rule-based methods align well with the Deepseek-V3 model, indicating a tendency of the model to leverage such approaches effectively.

% Despite limitations, rule-based methods offer key advantages:
% \begin{itemize}
%     \item Flexibility: Unconstrained by context window sizes or input image limits.
%     \item Cost Efficiency: Reduce computational costs by up to one-third vs. LLM-based and half vs. MLLM-based methods.
%     \item Stability: Avoid instability issues like erroneous placeholder generation.
% \end{itemize}
% Overall, rule-based methods offer a viable and efficient alternative, particularly in resource-constrained or stability-critical scenarios.

% The suboptimal performance of rule-based methods on simpler datasets can be attributed to data characteristics, particularly the Avg Images Per Doc metric. In web data, this metric is 1, whereas the other two datasets exhibit significantly higher values. With fewer images per document, the rule-based illustration algorithm struggles to effectively associate images with sentences, as multiple sentences often correspond to a single image. This leads to ambiguous correlations and makes threshold selection challenging, resulting in a higher error rate. Conversely, as the number of images increases, the correlation gap between rule-based methods and other approaches becomes more pronounced. Under such conditions, rule-based matching demonstrates better performance due to improved alignment with sentence-image relationships.

% 另一方面,LLM-based方法的表现普遍不亚于(基本是较好于)Rule-based的方法,这体现出目前的大模型的in-context reasoning能力

\textbf{Generation Performance Across Different Generation Methods.}  
In comparing different methods, an overall performance trend emerges: $\text{LLM-based} > \text{MLLM-based} > \text{Rule-based}$.  
% \begin{equation}
%     \text{LLM-based} > \text{MLLM-based} > \text{Rule-based} . \nonumber
% \end{equation}
For example, on Web Data, the Gemini model achieves average scores of 85.85, 83.3, and 41.21 for LLM-based, MLLM-based, and Rule-based methods, respectively.

\textbf{LLM-based methods:} By integrating contextual information surrounding images into the generation process, this method achieves natural and precise image insertion, underscoring the critical role of context in ensuring insertion accuracy.

\textbf{MLLM-based methods:} While effective on simpler datasets such as Web Data, their performance deteriorates on more challenging datasets due to the increased difficulty of distinguishing visually similar images, revealing the current limitations of model capabilities. In this category, the Gemini model achieves the highest average scores on the Web and Lifestyle Data, scoring 85.51 and 61.58, respectively, while Claude outperforms others on the Academic Paper Data with a score of 76.87.

\textbf{Rule-based methods:} Although these methods exhibit significantly lower performance compared to model-based approaches on simpler datasets, the performance gap diminishes as dataset complexity increases. In the Lifestyle Data domain, rule-based methods even outperform certain model-based approaches. For instance, GPT-4o achieves an average score of 65.68 using rule-based methods, surpassing its MLLM-based score of 61.01. Notably, the Deepseek-V3 model demonstrates outstanding performance on Web and Academic Paper Data, achieving top scores across almost all metrics. This suggests that for simpler datasets, rule-based methods align well with the Deepseek-V3 model, indicating the model's tendency to leverage such approaches effectively. Despite their limitations, rule-based methods offer several key advantages:  
\begin{itemize}
    \item \textbf{Flexibility:} Not constrained by context window sizes or input image limits.
    \item \textbf{Cost Efficiency:} Reduces computational costs by up to one-third compared to LLM-based methods and by half compared to MLLM-based methods.
    \item \textbf{Stability:} Avoids instability issues such as erroneous placeholder generation.
\end{itemize}
Overall, rule-based methods provide a viable and efficient alternative, particularly in resource-constrained or stability-critical scenarios. Furthermore, the performance of LLM-based methods is generally on par with or even superior to rule-based methods, showcasing the strong \textit{in-context reasoning} capabilities of modern large models.

% \textbf{Order Metric Analysis.} In particular, the Order metric for the Lifestyle domain demonstrates poor performance across all models and methods, with none achieving a passing score. Notably, GPT-4o performs best under both the Rule-based and MLLM-based methods, scoring 43.54 and 32.47, respectively, while the Gemini model attains the highest score of 54.62 under the LLM-based method. However, even these scores fall short of the passing threshold, underscoring the ambiguity in LLM and MLLM models regarding the concept of image insertion order, which remains an unresolved challenge.




\section{Conclusion}
\label{sec:conclusion}
To adapt the growing importance of generating multimodal answers, MRAMG has emerged as a critical task that aligns with real-world user demands.
To address the lack of evaluation resources for this task, we present MRAMG-Bench, a meticulously curated dataset containing 4,800 questions across diverse domains, varying difficulty levels, and rich image contexts. We further propose a comprehensive evaluation strategy, incorporating statistical and LLM-based metrics to rigorously assess both retrieval and generation performance. In addition, we introduce a general multimodal generation framework to enable models to produce interleaved text-image responses. Our evaluation of eleven generative models highlights notable limitations in handling challenging datasets and selecting the correct image order, emphasizing the necessity for deeper exploration of the MRAMG task.





%%
%% The next two lines define the bibliography style to be used, and
%% the bibliography file.
\bibliographystyle{ACM-Reference-Format}
\bibliography{sample-base}


%%
%% If your work has an appendix, this is the place to put it.
%\appendix

% 
% \section{Related Work}
\label{sec:related_work}

\iffalse
\paragraph{Multimodal RAG methods.}
RAG \citep{2020RAG, zhao2024retrieval} enhances the effectiveness of LLMs by integrating external knowledge into their reasoning process, addressing issues such as outdated training data and hallucinatory responses \citep{hallucination, gupta2024rag}.
As Multimodal Large Language Models (MLLMs) have advanced in their capacity to integrate and generate text from both textual and visual inputs, the development of Multimodal Retrieval-Augmented Generation (mRAG) has naturally followed as an extension, with methods like MuRAG \citep{chen2022murag} and REACT \citep{liu2023learning} retrieving relevant image-text pairs from external memory.
%Similarly, REACT \citep{liu2023learning} introduces a framework for constructing customized visual models by retrieving relevant image-text pairs from web-scale data and training modularized blocks.
To tackle Visual Question Answering (VQA) task, KAT \citep{gui2021kat} leverages the CLIP \citep{radford2021learning} image encoder to retrieve and associate specific image regions with external knowledge bases
%, thereby enhancing the model’s understanding of visual content.
%Meanwhile, RA-VQA \citep{lin2022retrieval} integrates differentiable DPR \citep{karpukhin2020dense} with answer generation, leveraging retrieved knowledge to achieve outstanding performance in VQA.
Unlike existing mRAG methods that convert multimodal information into purely textual outputs, our work addresses a distinct MRAMG task, where the output seamlessly integrates both textual and visual information.
%, preserving the inherent multimodal richness of the data.
A closely related recent work, MuRAR \citep{zhu2024murar}, addresses source attribution by retrieving multimodal elements from attributed documents. Furthermore, M2RAG \citep{ma2024multi} introduces a multi-stage image insertion framework, which involves multiple calls to the generation model during a single answer generation process.
However, these approaches often face challenges of high computational overhead due to repeated model invocations. In this paper, we propose a general framework for multimodal answer generation, leveraging a single invocation of the large generative model.
\paragraph{Classic RAG Datasets}
Well-established benchmarks for RAG, such as MS-MARCO \citep{MSMARCO} (a large-scale QA dataset based on real user queries), TriviaQA \citep{joshi2017triviaqa} (which features trivia questions requiring evidence-based answers), HotpotQA \citep{yang2018hotpotqa} (focused on multi-hop reasoning), Natural Questions (NQ) \citep{Naturalquestions} (based on real Google search queries), and SQuAD \citep{rajpurkar2016squad} (a reading comprehension dataset with span-based answers), are widely used to evaluate RAG performance \citep{petroni2020kilt, QAE}. 
However, these datasets focus on text-based tasks, while real-world applications increasingly require seamless integration of textual and visual information. To address this gap, we introduce a novel benchmark for MRAMG evaluation.
\paragraph{Multimodal RAG Datasets}
Various multimodal RAG datasets have been developed to address tasks requiring multimodal knowledge. 
OK-VQA \citep{marino2019ok} and A-OKVQA \citep{schwenk2022okvqa} that require retrieving external knowledge beyond the image content, whereas MMSearch \citep{jiang2024mmsearch} benchmarks MLLMs as multimodal search engines, focusing on image-to-image retrieval.
In contrast, MultiModalQA \citep{talmor2021multimodalqa} presents a more challenging scenario, where the questions do not include images but require joint reasoning across text and tables to answer complex questions.
While MultiModalQA questions are template-based, WebQA \citep{chang2022webqa} is a multi-hop, manually crafted multimodal QA dataset that involves the retrieval of relevant visual content for questions.
However, WebQA provides purely textual answers, relies solely on MLLMs for reasoning over retrieved images, and lacks textual support, making it unsuitable for language models that require linguistic context for coherent response generation.
M2RAG \citep{ma2024multi}, while constructing a multimodal corpus, is limited to only 200 questions and a corpus restricted to web pages. 
Additionally, the dataset relies on automated generation without manual verification and lacks ground truth for each query, which further complicates accurate evaluation.
To address these challenges, MRAMG-Bench introduces a comprehensive
multimodal dataset specifically designed for MRAMG tasks.  
Each question is meticulously paired with a precise, integrated text-image answer, enabling comprehensive and statistically rigorous evaluation.
\fi

\subsection{Multimodal RAG}
Retrieval-Augmented Generation (RAG) \citep{2020RAG, zhao2024retrieval} enhances the capabilities of Large Language Models (LLMs) by integrating external knowledge, addressing limitations such as outdated training data and hallucinated outputs \citep{hallucination, gupta2024rag}. Recent advancements in Multimodal Large Language Models (MLLMs), which combine textual and visual inputs for generation tasks \citep{wang2024comprehensive}, have spurred the development of Multimodal Retrieval-Augmented Generation (MRAG) as an extension of traditional RAG. Notable approaches, including MuRAG \citep{chen2022murag} and REACT \citep{liu2023learning}, retrieve image-text pairs from external memory to support multimodal generation. In the context of Visual Question Answering (VQA), KAT \citep{gui2021kat} employs the CLIP \citep{radford2021learning} image encoder to associate specific image regions with external knowledge bases. Unlike existing MRAG methods that primarily generate textual outputs, our work focuses on a novel MRAMG task, where the output seamlessly integrates both textual and visual information. A closely related work, MuRAR \citep{zhu2024murar}, addresses source attribution by retrieving multimodal elements from attributed documents. M2RAG \citep{ma2024multi} further extends this by introducing a multi-stage image insertion framework, which involves multiple invocations of the generative model during a single answer generation process. However, these methods are often hindered by high computational costs due to repeated model calls. In contrast, we propose a more efficient framework for multimodal answer generation, leveraging a single invocation of the generative model.
\vspace{-2mm}
\subsection{RAG Benchmarks}
The effective evaluation of Retrieval-Augmented Generation (RAG) models is crucial for advancing their development and optimization. 
Established benchmarks \citep{MSMARCO,joshi2017triviaqa,yang2018hotpotqa,Naturalquestions,rajpurkar2016squad} are commonly used to evaluate RAG models \citep{petroni2020kilt, QAE}. MS-MARCO \citep{MSMARCO} is a large-scale question answering (QA) dataset based on real user queries. TriviaQA \citep{joshi2017triviaqa} consists of trivia questions that require evidence-based answers. HotpotQA \citep{yang2018hotpotqa} focuses on multi-hop reasoning. Natural Questions (NQ) \citep{Naturalquestions} is derived from real Google search queries. SQuAD \citep{rajpurkar2016squad} is a reading comprehension dataset with span-based answers.
While these text-based benchmarks have proven effective in assessing RAG performance, they fall short in evaluating multimodal tasks, which require the integration of both textual and visual information. 
To bridge this gap, we introduce a novel benchmark for evaluating the MRAMG task.
\subsection{Multimodal RAG Benchmarks}
Similar to traditional RAG benchmarks, various multimodal RAG benchmarks \citep{marino2019ok,schwenk2022okvqa,jiang2024mmsearch,talmor2021multimodalqa,chang2022webqa,ma2024multi} have been developed to address tasks requiring multimodal knowledge. OK-VQA \citep{marino2019ok} and A-OKVQA \citep{schwenk2022okvqa} evaluate the multimodal reasoning ability of models using external knowledge beyond image content. MMSearch \citep{jiang2024mmsearch} evaluates MLLMs as multimodal search engines, focusing on image-to-image retrieval. MultiModalQA \citep{talmor2021multimodalqa} presents a more challenging scenario, where questions do not include images but require joint reasoning across text and tables to answer complex questions. While MultiModalQA relies on template-based questions, WebQA \citep{chang2022webqa} is a multi-hop, manually crafted multimodal QA dataset that involves retrieving relevant visual content for questions.
However, WebQA provides only textual answers, relies entirely on MLLMs for reasoning over retrieved images, and lacks textual support, making it unsuitable for language models that depend on linguistic context for coherent response generation. Notably, although M2RAG \citep{ma2024multi} constructs a multimodal corpus, it is limited to just 200 questions and a corpus confined to web pages. Additionally, the dataset relies on automated generation without manual verification and lacks ground truth for each query, further complicating accurate evaluation.
To address these challenges, our MRAMG-Bench introduces a comprehensive multimodal benchmark specifically designed for MRAMG task. Each question is meticulously paired with a precise, integrated text-image answer, enabling comprehensive and statistically rigorous evaluation.





\begin{table*}[tb!]
  \centering
   \caption{
Comparison of MRAMG-Bench with existing RAG benchmarks, where \includegraphics[height=1em]{Fig/text.png} represents text and \includegraphics[height=1em]{Fig/pic.png} represents images.
   }
   \vspace{-2mm}
\resizebox{0.9\textwidth}{!}{
  \begin{tabular}{l|cc|cccc|ccc}
    \toprule
    \multirow{2}[4]{*}{DATASETS} & \multicolumn{2}{c|}{Documents} & \multicolumn{4}{c|}{Questions}        & \multicolumn{3}{c}{Answers} \\
\cmidrule{2-10}          & Type  & Domain & Retrieval Modality & Difficulty Levels? & Type  & Num   & Exist? & Type  & Human-Annotated? \\
    \midrule
    HotpotQA \citep{yang2018hotpotqa} & \includegraphics[height=1em]{Fig/text.png}  & Web     & \includegraphics[height=1em]{Fig/text.png}  & \xmark    & \includegraphics[height=1em]{Fig/text.png}  & 113k   & \cmark    & \includegraphics[height=1em]{Fig/text.png}  & \cmark\\
    OK-VQA \citep{marino2019ok} & \includegraphics[height=1em]{Fig/text.png}  & Open Domain   & \includegraphics[height=1em]{Fig/text.png}  & \xmark   & \includegraphics[height=1em]{Fig/text.png} \includegraphics[height=1em]{Fig/pic.png} & 14k   & \cmark  & \includegraphics[height=1em]{Fig/text.png}  & \cmark\\
    WebQA \citep{chang2022webqa} & \includegraphics[height=1em]{Fig/text.png}  \includegraphics[height=1em]{Fig/pic.png} & Web      & \includegraphics[height=1em]{Fig/text.png} \includegraphics[height=1em]{Fig/pic.png} & \xmark    & \includegraphics[height=1em]{Fig/text.png}  & 56.6k & \cmark  & \includegraphics[height=1em]{Fig/text.png}  & \cmark\\
     MMSearch \citep{jiang2024mmsearch} &   \includegraphics[height=1em]{Fig/pic.png} & Multi-domain     &  \includegraphics[height=1em]{Fig/pic.png} & \xmark    & \includegraphics[height=1em]{Fig/text.png} \includegraphics[height=1em]{Fig/pic.png} & 300 & \cmark  & \includegraphics[height=1em]{Fig/text.png}  & \cmark\\
    M2RAG \citep{ma2024multi} & \includegraphics[height=1em]{Fig/text.png} \includegraphics[height=1em]{Fig/pic.png} & Web     & \includegraphics[height=1em]{Fig/text.png} \includegraphics[height=1em]{Fig/pic.png} & \xmark    & \includegraphics[height=1em]{Fig/text.png}  & 200   & \xmark    & \xmark    & \xmark \\
    \midrule
    MRAMG-Bench & \includegraphics[height=1em]{Fig/text.png} \includegraphics[height=1em]{Fig/pic.png} & Multi-domain    & \includegraphics[height=1em]{Fig/text.png} \includegraphics[height=1em]{Fig/pic.png} & \cmark  & \includegraphics[height=1em]{Fig/text.png}  & 4.8k   & \cmark  & \includegraphics[height=1em]{Fig/text.png} \includegraphics[height=1em]{Fig/pic.png} & \cmark\\
    \bottomrule
    \end{tabular}%
}
  \label{tab:compare}%
\end{table*}%
%\input{SIG_Appendix}
\end{sloppypar}
\end{document}
\endinput
%%
%% End of file `sample-sigconf.tex'.
