\section{Related Work}
\label{sec:related_work}

\iffalse
\paragraph{Multimodal RAG methods.}
RAG ____ enhances the effectiveness of LLMs by integrating external knowledge into their reasoning process, addressing issues such as outdated training data and hallucinatory responses ____.
As Multimodal Large Language Models (MLLMs) have advanced in their capacity to integrate and generate text from both textual and visual inputs, the development of Multimodal Retrieval-Augmented Generation (mRAG) has naturally followed as an extension, with methods like MuRAG ____ and REACT ____ retrieving relevant image-text pairs from external memory.
%Similarly, REACT ____ introduces a framework for constructing customized visual models by retrieving relevant image-text pairs from web-scale data and training modularized blocks.
To tackle Visual Question Answering (VQA) task, KAT ____ leverages the CLIP ____ image encoder to retrieve and associate specific image regions with external knowledge bases
%, thereby enhancing the model’s understanding of visual content.
%Meanwhile, RA-VQA ____ integrates differentiable DPR ____ with answer generation, leveraging retrieved knowledge to achieve outstanding performance in VQA.
Unlike existing mRAG methods that convert multimodal information into purely textual outputs, our work addresses a distinct MRAMG task, where the output seamlessly integrates both textual and visual information.
%, preserving the inherent multimodal richness of the data.
A closely related recent work, MuRAR ____, addresses source attribution by retrieving multimodal elements from attributed documents. Furthermore, M2RAG ____ introduces a multi-stage image insertion framework, which involves multiple calls to the generation model during a single answer generation process.
However, these approaches often face challenges of high computational overhead due to repeated model invocations. In this paper, we propose a general framework for multimodal answer generation, leveraging a single invocation of the large generative model.
\paragraph{Classic RAG Datasets}
Well-established benchmarks for RAG, such as MS-MARCO ____ (a large-scale QA dataset based on real user queries), TriviaQA ____ (which features trivia questions requiring evidence-based answers), HotpotQA ____ (focused on multi-hop reasoning), Natural Questions (NQ) ____ (based on real Google search queries), and SQuAD ____ (a reading comprehension dataset with span-based answers), are widely used to evaluate RAG performance ____. 
However, these datasets focus on text-based tasks, while real-world applications increasingly require seamless integration of textual and visual information. To address this gap, we introduce a novel benchmark for MRAMG evaluation.
\paragraph{Multimodal RAG Datasets}
Various multimodal RAG datasets have been developed to address tasks requiring multimodal knowledge. 
OK-VQA ____ and A-OKVQA ____ that require retrieving external knowledge beyond the image content, whereas MMSearch ____ benchmarks MLLMs as multimodal search engines, focusing on image-to-image retrieval.
In contrast, MultiModalQA ____ presents a more challenging scenario, where the questions do not include images but require joint reasoning across text and tables to answer complex questions.
While MultiModalQA questions are template-based, WebQA ____ is a multi-hop, manually crafted multimodal QA dataset that involves the retrieval of relevant visual content for questions.
However, WebQA provides purely textual answers, relies solely on MLLMs for reasoning over retrieved images, and lacks textual support, making it unsuitable for language models that require linguistic context for coherent response generation.
M2RAG ____, while constructing a multimodal corpus, is limited to only 200 questions and a corpus restricted to web pages. 
Additionally, the dataset relies on automated generation without manual verification and lacks ground truth for each query, which further complicates accurate evaluation.
To address these challenges, MRAMG-Bench introduces a comprehensive
multimodal dataset specifically designed for MRAMG tasks.  
Each question is meticulously paired with a precise, integrated text-image answer, enabling comprehensive and statistically rigorous evaluation.
\fi

\subsection{Multimodal RAG}
Retrieval-Augmented Generation (RAG) ____ enhances the capabilities of Large Language Models (LLMs) by integrating external knowledge, addressing limitations such as outdated training data and hallucinated outputs ____. Recent advancements in Multimodal Large Language Models (MLLMs), which combine textual and visual inputs for generation tasks ____, have spurred the development of Multimodal Retrieval-Augmented Generation (MRAG) as an extension of traditional RAG. Notable approaches, including MuRAG ____ and REACT ____, retrieve image-text pairs from external memory to support multimodal generation. In the context of Visual Question Answering (VQA), KAT ____ employs the CLIP ____ image encoder to associate specific image regions with external knowledge bases. Unlike existing MRAG methods that primarily generate textual outputs, our work focuses on a novel MRAMG task, where the output seamlessly integrates both textual and visual information. A closely related work, MuRAR ____, addresses source attribution by retrieving multimodal elements from attributed documents. M2RAG ____ further extends this by introducing a multi-stage image insertion framework, which involves multiple invocations of the generative model during a single answer generation process. However, these methods are often hindered by high computational costs due to repeated model calls. In contrast, we propose a more efficient framework for multimodal answer generation, leveraging a single invocation of the generative model.
\vspace{-2mm}
\subsection{RAG Benchmarks}
The effective evaluation of Retrieval-Augmented Generation (RAG) models is crucial for advancing their development and optimization. 
Established benchmarks ____ are commonly used to evaluate RAG models ____. MS-MARCO ____ is a large-scale question answering (QA) dataset based on real user queries. TriviaQA ____ consists of trivia questions that require evidence-based answers. HotpotQA ____ focuses on multi-hop reasoning. Natural Questions (NQ) ____ is derived from real Google search queries. SQuAD ____ is a reading comprehension dataset with span-based answers.
While these text-based benchmarks have proven effective in assessing RAG performance, they fall short in evaluating multimodal tasks, which require the integration of both textual and visual information. 
To bridge this gap, we introduce a novel benchmark for evaluating the MRAMG task.
\subsection{Multimodal RAG Benchmarks}
Similar to traditional RAG benchmarks, various multimodal RAG benchmarks ____ have been developed to address tasks requiring multimodal knowledge. OK-VQA ____ and A-OKVQA ____ evaluate the multimodal reasoning ability of models using external knowledge beyond image content. MMSearch ____ evaluates MLLMs as multimodal search engines, focusing on image-to-image retrieval. MultiModalQA ____ presents a more challenging scenario, where questions do not include images but require joint reasoning across text and tables to answer complex questions. While MultiModalQA relies on template-based questions, WebQA ____ is a multi-hop, manually crafted multimodal QA dataset that involves retrieving relevant visual content for questions.
However, WebQA provides only textual answers, relies entirely on MLLMs for reasoning over retrieved images, and lacks textual support, making it unsuitable for language models that depend on linguistic context for coherent response generation. Notably, although M2RAG ____ constructs a multimodal corpus, it is limited to just 200 questions and a corpus confined to web pages. Additionally, the dataset relies on automated generation without manual verification and lacks ground truth for each query, further complicating accurate evaluation.
To address these challenges, our MRAMG-Bench introduces a comprehensive multimodal benchmark specifically designed for MRAMG task. Each question is meticulously paired with a precise, integrated text-image answer, enabling comprehensive and statistically rigorous evaluation.





\begin{table*}[tb!]
  \centering
   \caption{
Comparison of MRAMG-Bench with existing RAG benchmarks, where \includegraphics[height=1em]{Fig/text.png} represents text and \includegraphics[height=1em]{Fig/pic.png} represents images.
   }
   \vspace{-2mm}
\resizebox{0.9\textwidth}{!}{
  \begin{tabular}{l|cc|cccc|ccc}
    \toprule
    \multirow{2}[4]{*}{DATASETS} & \multicolumn{2}{c|}{Documents} & \multicolumn{4}{c|}{Questions}        & \multicolumn{3}{c}{Answers} \\
\cmidrule{2-10}          & Type  & Domain & Retrieval Modality & Difficulty Levels? & Type  & Num   & Exist? & Type  & Human-Annotated? \\
    \midrule
    HotpotQA ____ & \includegraphics[height=1em]{Fig/text.png}  & Web     & \includegraphics[height=1em]{Fig/text.png}  & \xmark    & \includegraphics[height=1em]{Fig/text.png}  & 113k   & \cmark    & \includegraphics[height=1em]{Fig/text.png}  & \cmark\\
    OK-VQA ____ & \includegraphics[height=1em]{Fig/text.png}  & Open Domain   & \includegraphics[height=1em]{Fig/text.png}  & \xmark   & \includegraphics[height=1em]{Fig/text.png} \includegraphics[height=1em]{Fig/pic.png} & 14k   & \cmark  & \includegraphics[height=1em]{Fig/text.png}  & \cmark\\
    WebQA ____ & \includegraphics[height=1em]{Fig/text.png}  \includegraphics[height=1em]{Fig/pic.png} & Web      & \includegraphics[height=1em]{Fig/text.png} \includegraphics[height=1em]{Fig/pic.png} & \xmark    & \includegraphics[height=1em]{Fig/text.png}  & 56.6k & \cmark  & \includegraphics[height=1em]{Fig/text.png}  & \cmark\\
     MMSearch ____ &   \includegraphics[height=1em]{Fig/pic.png} & Multi-domain     &  \includegraphics[height=1em]{Fig/pic.png} & \xmark    & \includegraphics[height=1em]{Fig/text.png} \includegraphics[height=1em]{Fig/pic.png} & 300 & \cmark  & \includegraphics[height=1em]{Fig/text.png}  & \cmark\\
    M2RAG ____ & \includegraphics[height=1em]{Fig/text.png} \includegraphics[height=1em]{Fig/pic.png} & Web     & \includegraphics[height=1em]{Fig/text.png} \includegraphics[height=1em]{Fig/pic.png} & \xmark    & \includegraphics[height=1em]{Fig/text.png}  & 200   & \xmark    & \xmark    & \xmark \\
    \midrule
    MRAMG-Bench & \includegraphics[height=1em]{Fig/text.png} \includegraphics[height=1em]{Fig/pic.png} & Multi-domain    & \includegraphics[height=1em]{Fig/text.png} \includegraphics[height=1em]{Fig/pic.png} & \cmark  & \includegraphics[height=1em]{Fig/text.png}  & 4.8k   & \cmark  & \includegraphics[height=1em]{Fig/text.png} \includegraphics[height=1em]{Fig/pic.png} & \cmark\\
    \bottomrule
    \end{tabular}%
}
  \label{tab:compare}%
\end{table*}%