%%%%%%%% ICML 2025 EXAMPLE LATEX SUBMISSION FILE %%%%%%%%%%%%%%%%%

\documentclass{article}

% Recommended, but optional, packages for figures and better typesetting:
\usepackage{microtype}
\usepackage{graphicx}
\usepackage{subfigure}
\usepackage{multirow}
\usepackage{booktabs} % for professional tables

% hyperref makes hyperlinks in the resulting PDF.
% If your build breaks (sometimes temporarily if a hyperlink spans a page)
% please comment out the following usepackage line and replace
% \usepackage{icml2025} with \usepackage[nohyperref]{icml2025} above.
\usepackage{hyperref}


% Attempt to make hyperref and algorithmic work together better:
\newcommand{\theHalgorithm}{\arabic{algorithm}}

% Use the following line for the initial blind version submitted for review:
% \usepackage{icml2025}

% If accepted, instead use the following line for the camera-ready submission:
\usepackage[accepted]{icml2025}

% For theorems and such
\usepackage{amsmath}
\usepackage{amssymb}
\usepackage{mathtools}
\usepackage{amsthm}

% if you use cleveref..
\usepackage[capitalize,noabbrev]{cleveref}

%%%%%%%%%%%%%%%%%%%%%%%%%%%%%%%%
% THEOREMS
%%%%%%%%%%%%%%%%%%%%%%%%%%%%%%%%
\theoremstyle{plain}
\newtheorem{theorem}{Theorem}[section]
\newtheorem{proposition}[theorem]{Proposition}
\newtheorem{lemma}[theorem]{Lemma}
\newtheorem{corollary}[theorem]{Corollary}
\theoremstyle{definition}
\newtheorem{definition}[theorem]{Definition}
\newtheorem{assumption}[theorem]{Assumption}
\theoremstyle{remark}
\newtheorem{remark}[theorem]{Remark}

% Todonotes is useful during development; simply uncomment the next line
%    and comment out the line below the next line to turn off comments
%\usepackage[disable,textsize=tiny]{todonotes}
\usepackage[textsize=tiny]{todonotes}
\def\method{\text MixMin~}
\def\methodnospace{\text MixMin}
\def\genmethod{$\mathbb{R}$\text Min~}
\def\genmethodnospace{ $\mathbb{R}$\text Min}

\usepackage{placeins}


% The \icmltitle you define below is probably too long as a header.
% Therefore, a short form for the running title is supplied here:
\icmltitlerunning{Submission and Formatting Instructions for ICML 2025}

\begin{document}

\twocolumn[
\icmltitle{Accelerating Anchors via Specialization and Feature Transformation}

% It is OKAY to include author information, even for blind
% submissions: the style file will automatically remove it for you
% unless you've provided the [accepted] option to the icml2025
% package.

% List of affiliations: The first argument should be a (short)
% identifier you will use later to specify author affiliations
% Academic affiliations should list Department, University, City, Region, Country
% Industry affiliations should list Company, City, Region, Country

% You can specify symbols, otherwise they are numbered in order.
% Ideally, you should not use this facility. Affiliations will be numbered
% in order of appearance and this is the preferred way.
\icmlsetsymbol{equal}{*}

\begin{icmlauthorlist}
\icmlauthor{Haonan Yu}{pku,keylab}
\icmlauthor{Junhao Liu}{pku,keylab}
\icmlauthor{Xin Zhang}{pku,keylab}
\end{icmlauthorlist}

\icmlaffiliation{pku}{School of Computer Science, Peking University, Beijing, China}
\icmlaffiliation{keylab}{Key Lab of High Confidence Software Technologies(Peking University), Ministry of Education, Beijing, China}

\icmlcorrespondingauthor{Xin Zhang}{xin@pku.edu.cn}

% You may provide any keywords that you
% find helpful for describing your paper; these are used to populate
% the "keywords" metadata in the PDF but will not be shown in the document
\icmlkeywords{Machine Learning, ICML}

\vskip 0.3in
]

% this must go after the closing bracket ] following \twocolumn[ ...

% This command actually creates the footnote in the first column
% listing the affiliations and the copyright notice.
% The command takes one argument, which is text to display at the start of the footnote.
% The \icmlEqualContribution command is standard text for equal contribution.
% Remove it (just {}) if you do not need this facility.

\printAffiliationsAndNotice{}  % leave blank if no need to mention equal contribution
% \printAffiliationsAndNotice{\icmlEqualContribution} % otherwise use the standard text.

\begin{abstract}


The choice of representation for geographic location significantly impacts the accuracy of models for a broad range of geospatial tasks, including fine-grained species classification, population density estimation, and biome classification. Recent works like SatCLIP and GeoCLIP learn such representations by contrastively aligning geolocation with co-located images. While these methods work exceptionally well, in this paper, we posit that the current training strategies fail to fully capture the important visual features. We provide an information theoretic perspective on why the resulting embeddings from these methods discard crucial visual information that is important for many downstream tasks. To solve this problem, we propose a novel retrieval-augmented strategy called RANGE. We build our method on the intuition that the visual features of a location can be estimated by combining the visual features from multiple similar-looking locations. We evaluate our method across a wide variety of tasks. Our results show that RANGE outperforms the existing state-of-the-art models with significant margins in most tasks. We show gains of up to 13.1\% on classification tasks and 0.145 $R^2$ on regression tasks. All our code and models will be made available at: \href{https://github.com/mvrl/RANGE}{https://github.com/mvrl/RANGE}.

\end{abstract}


\section{Introduction}
Backdoor attacks pose a concealed yet profound security risk to machine learning (ML) models, for which the adversaries can inject a stealth backdoor into the model during training, enabling them to illicitly control the model's output upon encountering predefined inputs. These attacks can even occur without the knowledge of developers or end-users, thereby undermining the trust in ML systems. As ML becomes more deeply embedded in critical sectors like finance, healthcare, and autonomous driving \citep{he2016deep, liu2020computing, tournier2019mrtrix3, adjabi2020past}, the potential damage from backdoor attacks grows, underscoring the emergency for developing robust defense mechanisms against backdoor attacks.

To address the threat of backdoor attacks, researchers have developed a variety of strategies \cite{liu2018fine,wu2021adversarial,wang2019neural,zeng2022adversarial,zhu2023neural,Zhu_2023_ICCV, wei2024shared,wei2024d3}, aimed at purifying backdoors within victim models. These methods are designed to integrate with current deployment workflows seamlessly and have demonstrated significant success in mitigating the effects of backdoor triggers \cite{wubackdoorbench, wu2023defenses, wu2024backdoorbench,dunnett2024countering}.  However, most state-of-the-art (SOTA) backdoor purification methods operate under the assumption that a small clean dataset, often referred to as \textbf{auxiliary dataset}, is available for purification. Such an assumption poses practical challenges, especially in scenarios where data is scarce. To tackle this challenge, efforts have been made to reduce the size of the required auxiliary dataset~\cite{chai2022oneshot,li2023reconstructive, Zhu_2023_ICCV} and even explore dataset-free purification techniques~\cite{zheng2022data,hong2023revisiting,lin2024fusing}. Although these approaches offer some improvements, recent evaluations \cite{dunnett2024countering, wu2024backdoorbench} continue to highlight the importance of sufficient auxiliary data for achieving robust defenses against backdoor attacks.

While significant progress has been made in reducing the size of auxiliary datasets, an equally critical yet underexplored question remains: \emph{how does the nature of the auxiliary dataset affect purification effectiveness?} In  real-world  applications, auxiliary datasets can vary widely, encompassing in-distribution data, synthetic data, or external data from different sources. Understanding how each type of auxiliary dataset influences the purification effectiveness is vital for selecting or constructing the most suitable auxiliary dataset and the corresponding technique. For instance, when multiple datasets are available, understanding how different datasets contribute to purification can guide defenders in selecting or crafting the most appropriate dataset. Conversely, when only limited auxiliary data is accessible, knowing which purification technique works best under those constraints is critical. Therefore, there is an urgent need for a thorough investigation into the impact of auxiliary datasets on purification effectiveness to guide defenders in  enhancing the security of ML systems. 

In this paper, we systematically investigate the critical role of auxiliary datasets in backdoor purification, aiming to bridge the gap between idealized and practical purification scenarios.  Specifically, we first construct a diverse set of auxiliary datasets to emulate real-world conditions, as summarized in Table~\ref{overall}. These datasets include in-distribution data, synthetic data, and external data from other sources. Through an evaluation of SOTA backdoor purification methods across these datasets, we uncover several critical insights: \textbf{1)} In-distribution datasets, particularly those carefully filtered from the original training data of the victim model, effectively preserve the model’s utility for its intended tasks but may fall short in eliminating backdoors. \textbf{2)} Incorporating OOD datasets can help the model forget backdoors but also bring the risk of forgetting critical learned knowledge, significantly degrading its overall performance. Building on these findings, we propose Guided Input Calibration (GIC), a novel technique that enhances backdoor purification by adaptively transforming auxiliary data to better align with the victim model’s learned representations. By leveraging the victim model itself to guide this transformation, GIC optimizes the purification process, striking a balance between preserving model utility and mitigating backdoor threats. Extensive experiments demonstrate that GIC significantly improves the effectiveness of backdoor purification across diverse auxiliary datasets, providing a practical and robust defense solution.

Our main contributions are threefold:
\textbf{1) Impact analysis of auxiliary datasets:} We take the \textbf{first step}  in systematically investigating how different types of auxiliary datasets influence backdoor purification effectiveness. Our findings provide novel insights and serve as a foundation for future research on optimizing dataset selection and construction for enhanced backdoor defense.
%
\textbf{2) Compilation and evaluation of diverse auxiliary datasets:}  We have compiled and rigorously evaluated a diverse set of auxiliary datasets using SOTA purification methods, making our datasets and code publicly available to facilitate and support future research on practical backdoor defense strategies.
%
\textbf{3) Introduction of GIC:} We introduce GIC, the \textbf{first} dedicated solution designed to align auxiliary datasets with the model’s learned representations, significantly enhancing backdoor mitigation across various dataset types. Our approach sets a new benchmark for practical and effective backdoor defense.



\section{Background} \label{section:LLM}

% \subsection{Large Language Model (LLM)}   

Figure~\ref{fig:LLaMA_model}(a) shows that a decoder-only LLM initially processes a user prompt in the “prefill” stage and subsequently generates tokens sequentially during the “decoding” stage.
Both stages contain an input embedding layer, multiple decoder transformer blocks, an output embedding layer, and a sampling layer.
Figure~\ref{fig:LLaMA_model}(b) demonstrates that the decoder transformer blocks consist of a self attention and a feed-forward network (FFN) layer, each paired with residual connection and normalization layers. 

% Differentiate between encoder/decoder, explain why operation intensity is low, explain the different parts of a transformer block. Discuss Table II here. 

% Explain the architecture with Llama2-70B.

% \begin{table}[thb]
% \renewcommand\arraystretch{1.05}
% \centering
% % \vspace{-5mm}
%     \caption{ML Model Parameter Size and Operational Intensity}
%     \vspace{-2mm}
%     \small
%     \label{tab:ML Model Parameter Size and Operational Intensity}    
%     \scalebox{0.95}{
%         \begin{tabular}{|c|c|c|c|c|}
%             \hline
%             & Llama2 & BLOOM & BERT & ResNet \\
%             Model & (70B) & (176B) & & 152 \\
%             \hline
%             Parameter Size (GB) & 140 & 352 & 0.17 & 0.16 \\
%             \hline
%             Op Intensity (Ops/Byte) & 1 & 1 & 282 & 346 \\
%             \hline
%           \end{tabular}
%     }
% \vspace{-3mm}
% \end{table}

% {\fontsize{8pt}{11pt}\selectfont 8pt font size test Memory Requirement}

\begin{figure}[t]
    \centering
    \includegraphics[width=8cm]{Figure/LLaMA_model_new_new.pdf}
    \caption{(a) Prefill stage encodes prompt tokens in parallel. Decoding stage generates output tokens sequentially.
    (b) LLM contains N$\times$ decoder transformer blocks. 
    (c) Llama2 model architecture.}
    \label{fig:LLaMA_model}
\end{figure}

Figure~\ref{fig:LLaMA_model}(c) demonstrates the Llama2~\cite{touvron2023llama} model architecture as a representative LLM.
% The self attention layer requires three GEMVs\footnote{GEMVs in multi-head attention~\cite{attention}, narrow GEMMs in grouped-query attention~\cite{gqa}.} to generate query, key and value vectors.
In the self-attention layer, query, key and value vectors are generated by multiplying input vector to corresponding weight matrices.
These matrices are segmented into multiple heads, representing different semantic dimensions.
The query and key vectors go though Rotary Positional Embedding (RoPE) to encode the relative positional information~\cite{rope-paper}.
Within each head, the generated key and value vectors are appended to their caches.
The query vector is multiplied by the key cache to produce a score vector.
After the Softmax operation, the score vector is multiplied by the value cache to yield the output vector.
The output vectors from all heads are concatenated and multiplied by output weight matrix, resulting in a vector that undergoes residual connection and Root Mean Square layer Normalization (RMSNorm)~\cite{rmsnorm-paper}.
The residual connection adds up the input and output vectors of a layer to avoid vanishing gradient~\cite{he2016deep}.
The FFN layer begins with two parallel fully connections, followed by a Sigmoid Linear Unit (SiLU), and ends with another fully connection.
\section{RoleMRC}
\label{sec:method}

In this section, we build RoleMRC. Figure\,\ref{fig:method} illustrates the overall pipeline of RoleMRC from top to bottom, which is divided into three steps.

\subsection{A Meta-pool of 10k Role Profiles}
\label{sec:meta_pool}
We first collect a meta-pool of 10k role profile using two open-source datasets, with Step 1 and 2.

\paragraph{Step 1: Persona Sampling.} We randomly sample 10.5k one-sentence demographic persona description from PersonaHub\,\cite{ge2024scaling}, such as ``\emph{A local business owner interested in economic trends}'', as shown at the top of Figure\,\ref{fig:method}. 

\paragraph{Step 2: Role Profile Standardization.} Next, we use a well-crafted prompt with gpt-4o\,\cite{gpt4o} to expand each sampled persona into a complete role profile, in reference to the 1-shot standardized example. Illustrated in the middle of Figure\,\ref{fig:method}, we require a standardized role profile consisting of seven components: \emph{Role Name and Brief Description}, \emph{Specific Abilities and Skills}, \emph{Speech Style}, \emph{Personality Characteristics}, \emph{Past Experience and Background}, \emph{Ability and Knowledge Boundaries} and \emph{Speech Examples}. %Setting standard specifications helps convert the generated role profiles into formatted records, which is beneficial for the post quality control. 
Standardizing these profiles ensures structured formatting, simplifying quality control. 
After manual checking and format filtering, we remove 333 invalid responses from gpt-4o, resulting in 10.2k final role profiles. We report complete persona-to-profile standardization prompt and structure tree of final role profiles in Appendix\,\ref{sec:app_prompt_1} and \,\ref{sec:app_tree}, respectively.

Machine Reading Comprehension (MRC) is one of the core tasks for LLMs to interact with human users. Consequently, we choose to synthesize fine-grained role-playing instruction-following data based on MRC. We first generate a retrieval pool containing 808.7k MRC data from the MSMARCO training set\,\cite{bajaj2016ms}. By leveraging SFR-Embedding\,\cite{SFR-embedding-2}, we perform an inner product search to identify the most relevant and least relevant MRC triplets (Passages, Question, Answer) for each role profile. For example, the middle part of Figure\,\ref{fig:method} shows that for the role \emph{Jessica Thompson, a resilient local business owner}, the most relevant question is about \emph{the skill of resiliency}, while the least relevant question is \emph{converting Fahrenheit to Celsius}. After review, we categorise the most relevant MRC triplet as within a role's knowledge boundary, and the least relevant MRC triplet as beyond their expertise.

\begin{figure}[t]
    \centering
    \includegraphics[width=1.0\linewidth]{figures/step3.png}
    \caption{The strategy of gradually synthesizing finer role-playing instructions in step 3 of Figure\,\ref{fig:method}.}
    \vspace{-1.0em}
    \label{fig:step3}
\end{figure}

\subsection{38k Role-playing Instructions}
Based on the role profiles, we then adopt \textbf{Step 3: Multi-stage Dialogue Synthesis} to generate 38k role-playing instructions, progressively increasing granularity across three categories %including three types with gradually finer granularity 
(Figure\,\ref{fig:step3}):
%\begin{itemize}
%[leftmargin=*,noitemsep,topsep=0pt]

\noindent \textbf{\underline{Free Chats.}} The simplest dialogues, free chats, are synthesized at first. Here, we ask gpt-4o to simulate and generate multi-turn open-domain conversations between the role and an imagined user based on the standardized role profile. When synthesizing the conversation, we additionally consider two factors: the \textbf{initial speaker} in the starting round of the conversation, and whether the role's speech has \textbf{a narration wrapped in brackets} at the beginning (e.g., \emph{(Aiden reviews the network logs, his eyes narrowing as he spots unusual activity) I found it!}). The narration refers to a short, vivid description of the role's speaking state from an omniscient perspective, which further strengthens the sense of role's depth and has been adopted in some role-playing datasets\,\cite{tu2024charactereval}. 

As shown on the left side of Figure\,\ref{fig:step3}, based on the aforementioned two factors, we synthesize four variations of Free Chats. In particular, when  narration is omitted, we deleted all the 
narration content in the speech examples from the role profile; %and for the case that 
when narration is allowed, we retain the narration content, and also add instructions to allow appropriate insertion of narration in the task prompt of gpt-4o. It worth to note that, in narration-allowed dialogues, not every response of the role has narration inserted to prevent overfitting. All categories of data in RoleMRC incorporate narration insertion and follow similar control mechanisms. The following sections will omit further details on narration.

\noindent \textbf{\underline{On-scene MRC Dialogues.}} The synthesis of on-scene MRC dialogues can be divided into two parts. The first part is similar to the free chats. As shown by the {\color{lightgreen}{green round rectangle}} in the upper part of Figure\,\ref{fig:step3}, we ask gpt-4o to synthesize a conversation (lower left corner of Figure\,\ref{fig:step3}) between the role and the user focusing on relevant passages. This part of the synthesis and the Free Chats share the entire meta-pool, so each consisting of 5k dialogues.

The remaining part forms eight types of single-turn role-playing Question Answering (QA). In the middle of Figure\,\ref{fig:step3}, we randomly select a group of roles and examined the most relevant MRCs they matched: if the question in the MRC is answerable, then the ground truth answer is stylized to match the role profile; otherwise, a seed script of ``unanswerable'' is randomly selected then stylized. The above process generates four groups of 1k data from type ``[1]'' to type``[4]''. According to the middle right side of Figure\,\ref{fig:step3}, we also select a group of roles and ensure that the least relevant MRCs they matched contain answerable QA pairs. Since the most irrelevant MRCs are outside the knowledge boundary of the roles, the role-playing responses to these questions are ``out-of-mind'' refusal or ``let-me-try'' attempt, thus synthesizing four groups of 1k data, from type ``[5]'' to type ``[8]''.

\noindent \textbf{\underline{Ruled Chats.}} We construct Ruled Chats by extending On-scene MRC Dialogues in categories ``[1]'' to ``[8]'' with incorporated three additional rules, as shown in the right bottom corner of Figure\,\ref{fig:step3}. For the \textbf{multi-turn rules}, we apply them to the four unanswerable scenarios ``[3]'', ``[4]'', ``[5]'', and ``[6]'', adding a user prompt that  forces the role to answer. Among them, data ``[3]'' and ``[4]'' maintain refusal since the questions in MRC are unanswerable; while ``[5]'' and ``[6]'' are transformed into attempts to answer despite knowledge limitations. For the \textbf{nested formatting rules}, we add new formatting instructions to the four categories of data ``[1]'', ``[2]'', ``[3]'', and ``[4]'', such as requiring emojis,  capitalization, specific punctuation marks, and controlling the total number of words, then modify the previous replies accordingly. For the last \textbf{prioritized rules}, we apply them to subsets ``[1]'' and ``[2]'' that contain normal stylized answers, inserting a  global refusal directive from the system, and thus creating a conflict between system instructions and the role's ability boundary.
%\end{itemize}

\begin{table}[t]
\resizebox{\columnwidth}{!}{%
  \begin{tabular}{c|c|c|c|c|c}
    \toprule
    & & \textbf{S*} & \textbf{P*} & \textbf{\#Turns} & \textbf{\#Words} \\ 
    \midrule
    \multirow{13.5}{*}{\textbf{RoleMRC}} 
    & \multicolumn{5}{c|}{\textbf{Free Chats}} \\ 
    \cmidrule(lr){2-6}
    & Chats & 5k & / & 9.47 & 38.62 \\ 
    \cmidrule(lr){2-6}
    & \multicolumn{5}{c|}{\textbf{On-scene MRC Dialogues}} \\ 
    \cmidrule(lr){2-6} 
    & On-scene Chats & 5k & / & 9.2 & 43.18 \\
    & Answer & 2k & 2k & 1 & 39.45 \\ 
    & No Answer & 2k & 2k & 1 & 47.09 \\ 
    & Refusal & 2k & 2k & 1 & 48.41 \\ 
    & Attempt & 2k & 2k & 1 & 47.92 \\ 
    \cmidrule(lr){2-6}
    & \multicolumn{5}{c|}{\textbf{Ruled Chats}} \\ 
    \cmidrule(lr){2-6}
    & Multi-turn & 2k & 2k & 2 & 42.47 \\ 
    & Nested & 1.6k & 1.6k & 1 & 46.17 \\ 
    & Prioritized & 2.4k & 2.4k & 1 & 42.65 \\ 
    \midrule
    & \textbf{Total} & 24k & 14k & 3.5 & 40.6 \\ 
    \midrule
    \multirow{3}{*}{\textbf{-mix}} 
    & RoleBench & 16k & / & 1 & 23.95 \\ 
    & RLHFlow & 40k & / & 1.39 & 111.79 \\ 
    & UltraFeedback & / & 14k & 1 & 199.28 \\ 
    \midrule
    & \textbf{Total} & 80k & 28k & 2 & 67.1 \\ 
    \bottomrule
  \end{tabular}}
  \vspace{-2mm}
  \caption{Statistics of RoleMRC. In particular, the column names S*, P*, \#Turns, and \#Words, stands for size of single-label data, size of pair-label data, average turns, and average number of words per reply, respectively. RoleMRC-mix expands RoleMRC by adding existing role-playing data.}
 \vspace{-3mm}
  \label{tab:roleMRC}
\end{table}

\subsection{Integration and Mix-up}
All the seed scripts and prioritized rules used for constructing On-scene Dialogues and Ruled Chats are reported in Appendix\,\ref{sec:app_scripts}. These raw responses are logically valid manual answers that remain unaffected by the roles' speaking styles, making them suitable as negative labels to contrast with the stylized answers. Thanks to these meticulous seed texts, we obtain high-quality synthetic data with stable output from gpt-4o. After integration, as shown in Table\,\ref{tab:roleMRC}, the final RoleMRC contains 24k single-label data for Supervised Fine-Tuning (SFT) and 14k pair-label data for Human Preference Optimization (HPO)\,\cite{ouyang2022training,rafailov2023direct,sampo,hong2024reference}. Considering that fine-tuning LLMs with relatively fixed data formats may lead to catastrophic forgetting\,\cite{kirkpatrick2017overcoming}, we create RoleMRC-mix as a robust version by incorporating external role-playing data (RoleBench\,\cite{wang2023rolellm}) and general instructions (RLHFlow\,\cite{dong2024rlhf}, UltraFeedback\,\cite{cui2023ultrafeedback}).


\section{Dataset}
\label{sec:dataset}

\subsection{Data Collection}

To analyze political discussions on Discord, we followed the methodology in \cite{singh2024Cross-Platform}, collecting messages from politically-oriented public servers in compliance with Discord's platform policies.

Using Discord's Discovery feature, we employed a web scraper to extract server invitation links, names, and descriptions, focusing on public servers accessible without participation. Invitation links were used to access data via the Discord API. To ensure relevance, we filtered servers using keywords related to the 2024 U.S. elections (e.g., Trump, Kamala, MAGA), as outlined in \cite{balasubramanian2024publicdatasettrackingsocial}. This resulted in 302 server links, further narrowed to 81 English-speaking, politics-focused servers based on their names and descriptions.

Public messages were retrieved from these servers using the Discord API, collecting metadata such as \textit{content}, \textit{user ID}, \textit{username}, \textit{timestamp}, \textit{bot flag}, \textit{mentions}, and \textit{interactions}. Through this process, we gathered \textbf{33,373,229 messages} from \textbf{82,109 users} across \textbf{81 servers}, including \textbf{1,912,750 messages} from \textbf{633 bots}. Data collection occurred between November 13th and 15th, covering messages sent from January 1st to November 12th, just after the 2024 U.S. election.

\subsection{Characterizing the Political Spectrum}
\label{sec:timeline}

A key aspect of our research is distinguishing between Republican- and Democratic-aligned Discord servers. To categorize their political alignment, we relied on server names and self-descriptions, which often include rules, community guidelines, and references to key ideologies or figures. Each server's name and description were manually reviewed based on predefined, objective criteria, focusing on explicit political themes or mentions of prominent figures. This process allowed us to classify servers into three categories, ensuring a systematic and unbiased alignment determination.

\begin{itemize}
    \item \textbf{Republican-aligned}: Servers referencing Republican and right-wing and ideologies, movements, or figures (e.g., MAGA, Conservative, Traditional, Trump).  
    \item \textbf{Democratic-aligned}: Servers mentioning Democratic and left-wing ideologies, movements, or figures (e.g., Progressive, Liberal, Socialist, Biden, Kamala).  
    \item \textbf{Unaligned}: Servers with no defined spectrum and ideologies or opened to general political debate from all orientations.
\end{itemize}

To ensure the reliability and consistency of our classification, three independent reviewers assessed the classification following the specified set of criteria. The inter-rater agreement of their classifications was evaluated using Fleiss' Kappa \cite{fleiss1971measuring}, with a resulting Kappa value of \( 0.8191 \), indicating an almost perfect agreement among the reviewers. Disagreements were resolved by adopting the majority classification, as there were no instances where a server received different classifications from all three reviewers. This process guaranteed the consistency and accuracy of the final categorization.

Through this process, we identified \textbf{7 Republican-aligned servers}, \textbf{9 Democratic-aligned servers}, and \textbf{65 unaligned servers}.

Table \ref{tab:statistics} shows the statistics of the collected data. Notably, while Democratic- and Republican-aligned servers had a comparable number of user messages, users in the latter servers were significantly more active, posting more than double the number of messages per user compared to their Democratic counterparts. 
This suggests that, in our sample, Democratic-aligned servers attract more users, but these users were less engaged in text-based discussions. Additionally, around 10\% of the messages across all server categories were posted by bots. 

\subsection{Temporal Data} 

Throughout this paper, we refer to the election candidates using the names adopted by their respective campaigns: \textit{Kamala}, \textit{Biden}, and \textit{Trump}. To examine how the content of text messages evolves based on the political alignment of servers, we divided the 2024 election year into three periods: \textbf{Biden vs Trump} (January 1 to July 21), \textbf{Kamala vs Trump} (July 21 to September 20), and the \textbf{Voting Period} (after September 20). These periods reflect key phases of the election: the early campaign dominated by Biden and Trump, the shift in dynamics with Kamala Harris replacing Joe Biden as the Democratic candidate, and the final voting stage focused on electoral outcomes and their implications. This segmentation enables an analysis of how discourse responds to pivotal electoral moments.

Figure \ref{fig:line-plot} illustrates the distribution of messages over time, highlighting trends in total messages volume and mentions of each candidate. Prior to Biden's withdrawal on July 21, mentions of Biden and Trump were relatively balanced. However, following Kamala's entry into the race, mentions of Trump surged significantly, a trend further amplified by an assassination attempt on him, solidifying his dominance in the discourse. The only instance where Trump’s mentions were exceeded occurred during the first debate, as concerns about Biden’s age and cognitive abilities temporarily shifted the focus. In the final stages of the election, mentions of all three candidates rose, with Trump’s mentions peaking as he emerged as the victor.
\section{Experimental Methodology}\label{sec:exp}
In this section, we introduce the datasets, evaluation metrics, baselines, and implementation details used in our experiments. More experimental details are shown in Appendix~\ref{app:experiment_detail}.

\textbf{Dataset.}
We utilize various datasets for training and evaluation. Data statistics are shown in Table~\ref{tab:dataset}.

\textit{Training.}
We use the publicly available E5 dataset~\cite{wang2024improving,springer2024repetition} to train both the LLM-QE and dense retrievers. We concentrate on English-based question answering tasks and collect a total of 808,740 queries. From this set, we randomly sample 100,000 queries to construct the DPO training data, while the remaining queries are used for contrastive training. During the DPO preference pair construction, we first prompt LLMs to generate expansion documents, filtering out queries where the expanded documents share low similarity with the query. This results in a final set of 30,000 queries.

\textit{Evaluation.}
We evaluate retrieval effectiveness using two retrieval benchmarks: MS MARCO \cite{bajaj2016ms} and BEIR \cite{thakur2021beir}, in both unsupervised and supervised settings.

\textbf{Evaluation Metrics.}
We use nDCG@10 as the evaluation metric. Statistical significance is tested using a permutation test with $p<0.05$.

\textbf{Baselines.} We compare our LLM-QE model with three unsupervised retrieval models and five query expansion baseline models.
% —

Three unsupervised retrieval models—BM25~\cite{robertson2009probabilistic}, CoCondenser~\cite{gao2022unsupervised}, and Contriever~\cite{izacard2021unsupervised}—are evaluated in the experiments. Among these, Contriever serves as our primary baseline retrieval model, as it is used as the backbone model to assess the query expansion performance of LLM-QE. Additionally, we compare LLM-QE with Contriever in a supervised setting using the same training dataset.

For query expansion, we benchmark against five methods: Pseudo-Relevance Feedback (PRF), Q2Q, Q2E, Q2C, and Q2D. PRF is specifically implemented following the approach in~\citet{yu2021improving}, which enhances query understanding by extracting keywords from query-related documents. The Q2Q, Q2E, Q2C, and Q2D methods~\cite{jagerman2023query,li2024can} expand the original query by prompting LLMs to generate query-related queries, keywords, chains-of-thought~\cite{wei2022chain}, and documents.


\textbf{Implementation Details.} 
For our query expansion model, we deploy the Meta-LLaMA-3-8B-Instruct~\cite{llama3modelcard} as the backbone for the query expansion generator. The batch size is set to 16, and the learning rate is set to $2e-5$. Optimization is performed using the AdamW optimizer. We employ LoRA~\cite{hu2022lora} to efficiently fine-tune the model for 2 epochs. The temperature for the construction of the DPO data varies across $\tau \in \{0.8, 0.9, 1.0, 1.1\}$, with each setting sampled eight times. For the dense retriever, we utilize Contriever~\cite{izacard2021unsupervised} as the backbone. During training, we set the batch size to 1,024 and the learning rate to $3e-5$, with the model trained for 3 epochs.

\section{Discussion}

\subsection{Related Prior Work}

\textbf{Training Energy Based Models}.
A number of recent works aim to improve EBM training. \citet{zhu2023learning} uses a diffusion model to reduce the number of Langevin steps within the recovery likelihood approach of \cite{gao2020learning}. \cite{schroder2024energy} eliminates the need for MCMC and $\nabla_x$-computation of the energy during training by using a contrastive loss with forward noising process instead of MCMC, coined Energy Divergence (ED). ED is a promising alternative to E-DSM and has connections to score-matching, but, as shown in \Cref{tab:unconditional_gen}, it is not yet competitive, and  suffers from a bias by choice of noisy energy function.

\textbf{Composition with MCMC}. Similar to our work, \cite{du2023reduce} also uses energy-parameterized diffusion models but performs controlled generation with MCMC rather than SMC. SMC is known to suffer weight degeneracy high dimension, resulting in lack of diversity across particles, MCMC does not suffer from this, though requires additional non-parallel steps which is time consuming. The approaches are however complementary, and indeed one may perform MCMC after resampling steps to promote diversity.

\textbf{Sequential Monte Carlo in Diffusion Models.}
Many recent works use SMC within diffusion models for conditional generation, we detail the FKM formulations of these works in in \Cref{app:fkm_potentials}.

\cite{wu2024practical} uses twisted SMC with a classifier-guided proposal \citep{dhariwal2021diffusion} and potentials approximated with diffusion posterior sampling \citep{chungdiffusion}, which has been detailed as a FKM by concurrent work \citep{zhao2024conditional}. \cite{cardoso2024monte} and \cite{dou2024diffusion} tackle linear inverse problems where potentials have a closed form using Gaussian conjugacy. \cite{li2024derivative} perform SMC for both discrete and continuous diffusion models whereby potentials consist of a reward function applied to $\mathbb{E}[\bfX_0|x_t]$.

\cite{liu2024correcting} corrects conditional generations using an adversarially trained density ratio potential, and scales this to text-to-image models. 


\textbf{SMC for LLMs}. SMC is not only popular within diffusion models, but has been successful within large language models (LLMs) \citep{lew2023sequential, zhao2024probabilistic}. \cite{lew2023sequential} uses a FKM formulation with indicator based potential functions similar to as detailed in \Cref{sec:bounded}, and \cite{zhao2024probabilistic} discuss using SMC for text using potentials from reward functions or learning such potentials via contrastive twist learning, similar to contrastive learning for EBMs.

\subsection{Concurrent work} Since submission/ acceptance of our work \footnote{Submission October 2024}, there have been a number of relevant concurrent works. 

\textbf{FKM Interpretation}. \cite{singhal2025general} similarly to \cite{zhao2024conditional} and this work, detail sampling diffusion models in terms of KFM. \cite{singhal2025general} follow \cite{li2024derivative} in using reward functions based potentials but focus on text-to-image reward, and explore further heuristics such as or combining rewards via sum or max; and sampling $\bfX_0|x_t$ via nested diffusion \citep{elata2024nested} as input to their reward rather than using $\mathbb{E}[\bfX_0|x_t]$ as done in \cite{li2024derivative}.

\textbf{SMC for discrete diffusion}.  \cite{lee2025debiasing} use SMC for low temperature sampling for discrete diffusion models. \cite{xu2024energy} use a pretrained autoregressive likelihood model applied to samples $\bfX_0|x_t$ for a potential within discrete diffusion sampling.

\textbf{Composition}. \cite{skreta2024superposition} construct a cheap density estimator by simulating from an SDE, which can be computed at sampling time if using reverse diffusion solver, though it is not clear if this can be used in conjunction with resampling and Langevin corrector schemes. \cite{skreta2024superposition} then use this estimator to perform composition-type sampling, however their logical \textsc{AND} appears to differ from other more commonly used logical \textsc{AND} operations, in that it targets samples with equal probability between classes rather than generating both classes, e.g. "a CAT and a DOG" results in a cat/dog hybrid optical illusion rather than a separate cat and separate dog in one image. 

\cite{bradley2025mechanisms} explore composition more formally, establishing types of composition and cases where summing scores is sufficient without need for SMC correction as performed in this work or with MCMC correction from \citep{du2023reduce}.




\paragraph{Summary}
Our findings provide significant insights into the influence of correctness, explanations, and refinement on evaluation accuracy and user trust in AI-based planners. 
In particular, the findings are three-fold: 
(1) The \textbf{correctness} of the generated plans is the most significant factor that impacts the evaluation accuracy and user trust in the planners. As the PDDL solver is more capable of generating correct plans, it achieves the highest evaluation accuracy and trust. 
(2) The \textbf{explanation} component of the LLM planner improves evaluation accuracy, as LLM+Expl achieves higher accuracy than LLM alone. Despite this improvement, LLM+Expl minimally impacts user trust. However, alternative explanation methods may influence user trust differently from the manually generated explanations used in our approach.
% On the other hand, explanations may help refine the trust of the planner to a more appropriate level by indicating planner shortcomings.
(3) The \textbf{refinement} procedure in the LLM planner does not lead to a significant improvement in evaluation accuracy; however, it exhibits a positive influence on user trust that may indicate an overtrust in some situations.
% This finding is aligned with prior works showing that iterative refinements based on user feedback would increase user trust~\cite{kunkel2019let, sebo2019don}.
Finally, the propensity-to-trust analysis identifies correctness as the primary determinant of user trust, whereas explanations provided limited improvement in scenarios where the planner's accuracy is diminished.

% In conclusion, our results indicate that the planner's correctness is the dominant factor for both evaluation accuracy and user trust. Therefore, selecting high-quality training data and optimizing the training procedure of AI-based planners to improve planning correctness is the top priority. Once the AI planner achieves a similar correctness level to traditional graph-search planners, strengthening its capability to explain and refine plans will further improve user trust compared to traditional planners.

\paragraph{Future Research} Future steps in this research include expanding user studies with larger sample sizes to improve generalizability and including additional planning problems per session for a more comprehensive evaluation. Next, we will explore alternative methods for generating plan explanations beyond manual creation to identify approaches that more effectively enhance user trust. 
Additionally, we will examine user trust by employing multiple LLM-based planners with varying levels of planning accuracy to better understand the interplay between planning correctness and user trust. 
Furthermore, we aim to enable real-time user-planner interaction, allowing users to provide feedback and refine plans collaboratively, thereby fostering a more dynamic and user-centric planning process.




% In the unusual situation where you want a paper to appear in the
% references without citing it in the main text, use \nocite
\nocite{langley00}

\FloatBarrier
\newpage

\bibliography{main}
\bibliographystyle{icml2025}


%%%%%%%%%%%%%%%%%%%%%%%%%%%%%%%%%%%%%%%%%%%%%%%%%%%%%%%%%%%%%%%%%%%%%%%%%%%%%%%
%%%%%%%%%%%%%%%%%%%%%%%%%%%%%%%%%%%%%%%%%%%%%%%%%%%%%%%%%%%%%%%%%%%%%%%%%%%%%%%
% APPENDIX
%%%%%%%%%%%%%%%%%%%%%%%%%%%%%%%%%%%%%%%%%%%%%%%%%%%%%%%%%%%%%%%%%%%%%%%%%%%%%%%
%%%%%%%%%%%%%%%%%%%%%%%%%%%%%%%%%%%%%%%%%%%%%%%%%%%%%%%%%%%%%%%%%%%%%%%%%%%%%%%
\newpage
\appendix
\onecolumn
% \section{You \emph{can} have an appendix here.}

% You can have as much text here as you want. The main body must be at most $8$ pages long.
% For the final version, one more page can be added.
% If you want, you can use an appendix like this one.  

% The $\mathtt{\backslash onecolumn}$ command above can be kept in place if you prefer a one-column appendix, or can be removed if you prefer a two-column appendix.  Apart from this possible change, the style (font size, spacing, margins, page numbering, etc.) should be kept the same as the main body.
%%%%%%%%%%%%%%%%%%%%%%%%%%%%%%%%%%%%%%%%%%%%%%%%%%%%%%%%%%%%%%%%%%%%%%%%%%%%%%%
%%%%%%%%%%%%%%%%%%%%%%%%%%%%%%%%%%%%%%%%%%%%%%%%%%%%%%%%%%%%%%%%%%%%%%%%%%%%%%%


\end{document}


% This document was modified from the file originally made available by
% Pat Langley and Andrea Danyluk for ICML-2K. This version was created
% by Iain Murray in 2018, and modified by Alexandre Bouchard in
% 2019 and 2021 and by Csaba Szepesvari, Gang Niu and Sivan Sabato in 2022.
% Modified again in 2023 and 2024 by Sivan Sabato and Jonathan Scarlett.
% Previous contributors include Dan Roy, Lise Getoor and Tobias
% Scheffer, which was slightly modified from the 2010 version by
% Thorsten Joachims & Johannes Fuernkranz, slightly modified from the
% 2009 version by Kiri Wagstaff and Sam Roweis's 2008 version, which is
% slightly modified from Prasad Tadepalli's 2007 version which is a
% lightly changed version of the previous year's version by Andrew
% Moore, which was in turn edited from those of Kristian Kersting and
% Codrina Lauth. Alex Smola contributed to the algorithmic style files.
%%%%%%%% ICML 2025 EXAMPLE LATEX SUBMISSION FILE %%%%%%%%%%%%%%%%%

\documentclass{article}

% Recommended, but optional, packages for figures and better typesetting:
\usepackage{microtype}
\usepackage{graphicx}
\usepackage{subfigure}
\usepackage{booktabs} % for professional tables

% hyperref makes hyperlinks in the resulting PDF.
% If your build breaks (sometimes temporarily if a hyperlink spans a page)
% please comment out the following usepackage line and replace
% \usepackage{icml2025} with \usepackage[nohyperref]{icml2025} above.
\usepackage{hyperref}


% Attempt to make hyperref and algorithmic work together better:
\newcommand{\theHalgorithm}{\arabic{algorithm}}

% Use the following line for the initial blind version submitted for review:
% \usepackage{icml2025}

% If accepted, instead use the following line for the camera-ready submission:
\usepackage[accepted]{icml2025}

% For theorems and such
\usepackage{amsmath}
\usepackage{amssymb}
\usepackage{mathtools}
\usepackage{amsthm}

% if you use cleveref..
\usepackage[capitalize,noabbrev]{cleveref}

%%%%%%%%%%%%%%%%%%%%%%%%%%%%%%%%
% THEOREMS
%%%%%%%%%%%%%%%%%%%%%%%%%%%%%%%%
\theoremstyle{plain}
\newtheorem{theorem}{Theorem}[section]
\newtheorem{proposition}[theorem]{Proposition}
\newtheorem{lemma}[theorem]{Lemma}
\newtheorem{corollary}[theorem]{Corollary}
\theoremstyle{definition}
\newtheorem{definition}[theorem]{Definition}
\newtheorem{assumption}[theorem]{Assumption}
\theoremstyle{remark}
\newtheorem{remark}[theorem]{Remark}

% Todonotes is useful during development; simply uncomment the next line
%    and comment out the line below the next line to turn off comments
%\usepackage[disable,textsize=tiny]{todonotes}
\usepackage[textsize=tiny]{todonotes}


% The \icmltitle you define below is probably too long as a header.
% Therefore, a short form for the running title is supplied here:
\icmltitlerunning{Submission and Formatting Instructions for ICML 2025}

\begin{document}

\twocolumn[
\icmltitle{Submission and Formatting Instructions for \\
           International Conference on Machine Learning (ICML 2025)}

% It is OKAY to include author information, even for blind
% submissions: the style file will automatically remove it for you
% unless you've provided the [accepted] option to the icml2025
% package.

% List of affiliations: The first argument should be a (short)
% identifier you will use later to specify author affiliations
% Academic affiliations should list Department, University, City, Region, Country
% Industry affiliations should list Company, City, Region, Country

% You can specify symbols, otherwise they are numbered in order.
% Ideally, you should not use this facility. Affiliations will be numbered
% in order of appearance and this is the preferred way.
\icmlsetsymbol{equal}{*}

\begin{icmlauthorlist}
\icmlauthor{Firstname1 Lastname1}{equal,yyy}
\icmlauthor{Firstname2 Lastname2}{equal,yyy,comp}
\icmlauthor{Firstname3 Lastname3}{comp}
\icmlauthor{Firstname4 Lastname4}{sch}
\icmlauthor{Firstname5 Lastname5}{yyy}
\icmlauthor{Firstname6 Lastname6}{sch,yyy,comp}
\icmlauthor{Firstname7 Lastname7}{comp}
%\icmlauthor{}{sch}
\icmlauthor{Firstname8 Lastname8}{sch}
\icmlauthor{Firstname8 Lastname8}{yyy,comp}
%\icmlauthor{}{sch}
%\icmlauthor{}{sch}
\end{icmlauthorlist}

\icmlaffiliation{yyy}{Department of XXX, University of YYY, Location, Country}
\icmlaffiliation{comp}{Company Name, Location, Country}
\icmlaffiliation{sch}{School of ZZZ, Institute of WWW, Location, Country}

\icmlcorrespondingauthor{Firstname1 Lastname1}{first1.last1@xxx.edu}
\icmlcorrespondingauthor{Firstname2 Lastname2}{first2.last2@www.uk}

% You may provide any keywords that you
% find helpful for describing your paper; these are used to populate
% the "keywords" metadata in the PDF but will not be shown in the document
\icmlkeywords{Machine Learning, ICML}

\vskip 0.3in
]

% this must go after the closing bracket ] following \twocolumn[ ...

% This command actually creates the footnote in the first column
% listing the affiliations and the copyright notice.
% The command takes one argument, which is text to display at the start of the footnote.
% The \icmlEqualContribution command is standard text for equal contribution.
% Remove it (just {}) if you do not need this facility.

%\printAffiliationsAndNotice{}  % leave blank if no need to mention equal contribution
\printAffiliationsAndNotice{\icmlEqualContribution} % otherwise use the standard text.

\begin{abstract}
This document provides a basic paper template and submission guidelines.
Abstracts must be a single paragraph, ideally between 4--6 sentences long.
Gross violations will trigger corrections at the camera-ready phase.
\end{abstract}

\section{Electronic Submission}
\label{submission}

Submission to ICML 2025 will be entirely electronic, via a web site
(not email). Information about the submission process and \LaTeX\ templates
are available on the conference web site at:
\begin{center}
\textbf{\texttt{http://icml.cc/}}
\end{center}

The guidelines below will be enforced for initial submissions and
camera-ready copies. Here is a brief summary:
\begin{itemize}
\item Submissions must be in PDF\@. 
\item If your paper has appendices, submit the appendix together with the main body and the references \textbf{as a single file}. Reviewers will not look for appendices as a separate PDF file. So if you submit such an extra file, reviewers will very likely miss it.
\item Page limit: The main body of the paper has to be fitted to 8 pages, excluding references and appendices; the space for the latter two is not limited in pages, but the total file size may not exceed 10MB. For the final version of the paper, authors can add one extra page to the main body.
\item \textbf{Do not include author information or acknowledgements} in your
    initial submission.
\item Your paper should be in \textbf{10 point Times font}.
\item Make sure your PDF file only uses Type-1 fonts.
\item Place figure captions \emph{under} the figure (and omit titles from inside
    the graphic file itself). Place table captions \emph{over} the table.
\item References must include page numbers whenever possible and be as complete
    as possible. Place multiple citations in chronological order.
\item Do not alter the style template; in particular, do not compress the paper
    format by reducing the vertical spaces.
\item Keep your abstract brief and self-contained, one paragraph and roughly
    4--6 sentences. Gross violations will require correction at the
    camera-ready phase. The title should have content words capitalized.
\end{itemize}

\subsection{Submitting Papers}

\textbf{Anonymous Submission:} ICML uses double-blind review: no identifying
author information may appear on the title page or in the paper
itself. \cref{author info} gives further details.

\medskip

Authors must provide their manuscripts in \textbf{PDF} format.
Furthermore, please make sure that files contain only embedded Type-1 fonts
(e.g.,~using the program \texttt{pdffonts} in linux or using
File/DocumentProperties/Fonts in Acrobat). Other fonts (like Type-3)
might come from graphics files imported into the document.

Authors using \textbf{Word} must convert their document to PDF\@. Most
of the latest versions of Word have the facility to do this
automatically. Submissions will not be accepted in Word format or any
format other than PDF\@. Really. We're not joking. Don't send Word.

Those who use \textbf{\LaTeX} should avoid including Type-3 fonts.
Those using \texttt{latex} and \texttt{dvips} may need the following
two commands:

{\footnotesize
\begin{verbatim}
dvips -Ppdf -tletter -G0 -o paper.ps paper.dvi
ps2pdf paper.ps
\end{verbatim}}
It is a zero following the ``-G'', which tells dvips to use
the config.pdf file. Newer \TeX\ distributions don't always need this
option.

Using \texttt{pdflatex} rather than \texttt{latex}, often gives better
results. This program avoids the Type-3 font problem, and supports more
advanced features in the \texttt{microtype} package.

\textbf{Graphics files} should be a reasonable size, and included from
an appropriate format. Use vector formats (.eps/.pdf) for plots,
lossless bitmap formats (.png) for raster graphics with sharp lines, and
jpeg for photo-like images.

The style file uses the \texttt{hyperref} package to make clickable
links in documents. If this causes problems for you, add
\texttt{nohyperref} as one of the options to the \texttt{icml2025}
usepackage statement.


\subsection{Submitting Final Camera-Ready Copy}

The final versions of papers accepted for publication should follow the
same format and naming convention as initial submissions, except that
author information (names and affiliations) should be given. See
\cref{final author} for formatting instructions.

The footnote, ``Preliminary work. Under review by the International
Conference on Machine Learning (ICML). Do not distribute.'' must be
modified to ``\textit{Proceedings of the
$\mathit{42}^{nd}$ International Conference on Machine Learning},
Vancouver, Canada, PMLR 267, 2025.
Copyright 2025 by the author(s).''

For those using the \textbf{\LaTeX} style file, this change (and others) is
handled automatically by simply changing
$\mathtt{\backslash usepackage\{icml2025\}}$ to
$$\mathtt{\backslash usepackage[accepted]\{icml2025\}}$$
Authors using \textbf{Word} must edit the
footnote on the first page of the document themselves.

Camera-ready copies should have the title of the paper as running head
on each page except the first one. The running title consists of a
single line centered above a horizontal rule which is $1$~point thick.
The running head should be centered, bold and in $9$~point type. The
rule should be $10$~points above the main text. For those using the
\textbf{\LaTeX} style file, the original title is automatically set as running
head using the \texttt{fancyhdr} package which is included in the ICML
2025 style file package. In case that the original title exceeds the
size restrictions, a shorter form can be supplied by using

\verb|\icmltitlerunning{...}|

just before $\mathtt{\backslash begin\{document\}}$.
Authors using \textbf{Word} must edit the header of the document themselves.

\section{Format of the Paper}

All submissions must follow the specified format.

\subsection{Dimensions}




The text of the paper should be formatted in two columns, with an
overall width of 6.75~inches, height of 9.0~inches, and 0.25~inches
between the columns. The left margin should be 0.75~inches and the top
margin 1.0~inch (2.54~cm). The right and bottom margins will depend on
whether you print on US letter or A4 paper, but all final versions
must be produced for US letter size.
Do not write anything on the margins.

The paper body should be set in 10~point type with a vertical spacing
of 11~points. Please use Times typeface throughout the text.

\subsection{Title}

The paper title should be set in 14~point bold type and centered
between two horizontal rules that are 1~point thick, with 1.0~inch
between the top rule and the top edge of the page. Capitalize the
first letter of content words and put the rest of the title in lower
case.

\subsection{Author Information for Submission}
\label{author info}

ICML uses double-blind review, so author information must not appear. If
you are using \LaTeX\/ and the \texttt{icml2025.sty} file, use
\verb+\icmlauthor{...}+ to specify authors and \verb+\icmlaffiliation{...}+ to specify affiliations. (Read the TeX code used to produce this document for an example usage.) The author information
will not be printed unless \texttt{accepted} is passed as an argument to the
style file.
Submissions that include the author information will not
be reviewed.

\subsubsection{Self-Citations}

If you are citing published papers for which you are an author, refer
to yourself in the third person. In particular, do not use phrases
that reveal your identity (e.g., ``in previous work \cite{langley00}, we
have shown \ldots'').

Do not anonymize citations in the reference section. The only exception are manuscripts that are
not yet published (e.g., under submission). If you choose to refer to
such unpublished manuscripts \cite{anonymous}, anonymized copies have
to be submitted
as Supplementary Material via OpenReview\@. However, keep in mind that an ICML
paper should be self contained and should contain sufficient detail
for the reviewers to evaluate the work. In particular, reviewers are
not required to look at the Supplementary Material when writing their
review (they are not required to look at more than the first $8$ pages of the submitted document).

\subsubsection{Camera-Ready Author Information}
\label{final author}

If a paper is accepted, a final camera-ready copy must be prepared.
%
For camera-ready papers, author information should start 0.3~inches below the
bottom rule surrounding the title. The authors' names should appear in 10~point
bold type, in a row, separated by white space, and centered. Author names should
not be broken across lines. Unbolded superscripted numbers, starting 1, should
be used to refer to affiliations.

Affiliations should be numbered in the order of appearance. A single footnote
block of text should be used to list all the affiliations. (Academic
affiliations should list Department, University, City, State/Region, Country.
Similarly for industrial affiliations.)

Each distinct affiliations should be listed once. If an author has multiple
affiliations, multiple superscripts should be placed after the name, separated
by thin spaces. If the authors would like to highlight equal contribution by
multiple first authors, those authors should have an asterisk placed after their
name in superscript, and the term ``\textsuperscript{*}Equal contribution"
should be placed in the footnote block ahead of the list of affiliations. A
list of corresponding authors and their emails (in the format Full Name
\textless{}email@domain.com\textgreater{}) can follow the list of affiliations.
Ideally only one or two names should be listed.

A sample file with author names is included in the ICML2025 style file
package. Turn on the \texttt{[accepted]} option to the stylefile to
see the names rendered. All of the guidelines above are implemented
by the \LaTeX\ style file.

\subsection{Abstract}

The paper abstract should begin in the left column, 0.4~inches below the final
address. The heading `Abstract' should be centered, bold, and in 11~point type.
The abstract body should use 10~point type, with a vertical spacing of
11~points, and should be indented 0.25~inches more than normal on left-hand and
right-hand margins. Insert 0.4~inches of blank space after the body. Keep your
abstract brief and self-contained, limiting it to one paragraph and roughly 4--6
sentences. Gross violations will require correction at the camera-ready phase.

\subsection{Partitioning the Text}

You should organize your paper into sections and paragraphs to help
readers place a structure on the material and understand its
contributions.

\subsubsection{Sections and Subsections}

Section headings should be numbered, flush left, and set in 11~pt bold
type with the content words capitalized. Leave 0.25~inches of space
before the heading and 0.15~inches after the heading.

Similarly, subsection headings should be numbered, flush left, and set
in 10~pt bold type with the content words capitalized. Leave
0.2~inches of space before the heading and 0.13~inches afterward.

Finally, subsubsection headings should be numbered, flush left, and
set in 10~pt small caps with the content words capitalized. Leave
0.18~inches of space before the heading and 0.1~inches after the
heading.

Please use no more than three levels of headings.

\subsubsection{Paragraphs and Footnotes}

Within each section or subsection, you should further partition the
paper into paragraphs. Do not indent the first line of a given
paragraph, but insert a blank line between succeeding ones.

You can use footnotes\footnote{Footnotes
should be complete sentences.} to provide readers with additional
information about a topic without interrupting the flow of the paper.
Indicate footnotes with a number in the text where the point is most
relevant. Place the footnote in 9~point type at the bottom of the
column in which it appears. Precede the first footnote in a column
with a horizontal rule of 0.8~inches.\footnote{Multiple footnotes can
appear in each column, in the same order as they appear in the text,
but spread them across columns and pages if possible.}

\begin{figure}[ht]
\vskip 0.2in
\begin{center}
\centerline{\includegraphics[width=\columnwidth]{icml_numpapers}}
\caption{Historical locations and number of accepted papers for International
Machine Learning Conferences (ICML 1993 -- ICML 2008) and International
Workshops on Machine Learning (ML 1988 -- ML 1992). At the time this figure was
produced, the number of accepted papers for ICML 2008 was unknown and instead
estimated.}
\label{icml-historical}
\end{center}
\vskip -0.2in
\end{figure}

\subsection{Figures}

You may want to include figures in the paper to illustrate
your approach and results. Such artwork should be centered,
legible, and separated from the text. Lines should be dark and at
least 0.5~points thick for purposes of reproduction, and text should
not appear on a gray background.

Label all distinct components of each figure. If the figure takes the
form of a graph, then give a name for each axis and include a legend
that briefly describes each curve. Do not include a title inside the
figure; instead, the caption should serve this function.

Number figures sequentially, placing the figure number and caption
\emph{after} the graphics, with at least 0.1~inches of space before
the caption and 0.1~inches after it, as in
\cref{icml-historical}. The figure caption should be set in
9~point type and centered unless it runs two or more lines, in which
case it should be flush left. You may float figures to the top or
bottom of a column, and you may set wide figures across both columns
(use the environment \texttt{figure*} in \LaTeX). Always place
two-column figures at the top or bottom of the page.

\subsection{Algorithms}

If you are using \LaTeX, please use the ``algorithm'' and ``algorithmic''
environments to format pseudocode. These require
the corresponding stylefiles, algorithm.sty and
algorithmic.sty, which are supplied with this package.
\cref{alg:example} shows an example.

\begin{algorithm}[tb]
   \caption{Bubble Sort}
   \label{alg:example}
\begin{algorithmic}
   \STATE {\bfseries Input:} data $x_i$, size $m$
   \REPEAT
   \STATE Initialize $noChange = true$.
   \FOR{$i=1$ {\bfseries to} $m-1$}
   \IF{$x_i > x_{i+1}$}
   \STATE Swap $x_i$ and $x_{i+1}$
   \STATE $noChange = false$
   \ENDIF
   \ENDFOR
   \UNTIL{$noChange$ is $true$}
\end{algorithmic}
\end{algorithm}

\subsection{Tables}

You may also want to include tables that summarize material. Like
figures, these should be centered, legible, and numbered consecutively.
However, place the title \emph{above} the table with at least
0.1~inches of space before the title and the same after it, as in
\cref{sample-table}. The table title should be set in 9~point
type and centered unless it runs two or more lines, in which case it
should be flush left.

% Note use of \abovespace and \belowspace to get reasonable spacing
% above and below tabular lines.

\begin{table}[t]
\caption{Classification accuracies for naive Bayes and flexible
Bayes on various data sets.}
\label{sample-table}
\vskip 0.15in
\begin{center}
\begin{small}
\begin{sc}
\begin{tabular}{lcccr}
\toprule
Data set & Naive & Flexible & Better? \\
\midrule
Breast    & 95.9$\pm$ 0.2& 96.7$\pm$ 0.2& $\surd$ \\
Cleveland & 83.3$\pm$ 0.6& 80.0$\pm$ 0.6& $\times$\\
Glass2    & 61.9$\pm$ 1.4& 83.8$\pm$ 0.7& $\surd$ \\
Credit    & 74.8$\pm$ 0.5& 78.3$\pm$ 0.6&         \\
Horse     & 73.3$\pm$ 0.9& 69.7$\pm$ 1.0& $\times$\\
Meta      & 67.1$\pm$ 0.6& 76.5$\pm$ 0.5& $\surd$ \\
Pima      & 75.1$\pm$ 0.6& 73.9$\pm$ 0.5&         \\
Vehicle   & 44.9$\pm$ 0.6& 61.5$\pm$ 0.4& $\surd$ \\
\bottomrule
\end{tabular}
\end{sc}
\end{small}
\end{center}
\vskip -0.1in
\end{table}

Tables contain textual material, whereas figures contain graphical material.
Specify the contents of each row and column in the table's topmost
row. Again, you may float tables to a column's top or bottom, and set
wide tables across both columns. Place two-column tables at the
top or bottom of the page.

\subsection{Theorems and such}
The preferred way is to number definitions, propositions, lemmas, etc. consecutively, within sections, as shown below.
\begin{definition}
\label{def:inj}
A function $f:X \to Y$ is injective if for any $x,y\in X$ different, $f(x)\ne f(y)$.
\end{definition}
Using \cref{def:inj} we immediate get the following result:
\begin{proposition}
If $f$ is injective mapping a set $X$ to another set $Y$, 
the cardinality of $Y$ is at least as large as that of $X$
\end{proposition}
\begin{proof} 
Left as an exercise to the reader. 
\end{proof}
\cref{lem:usefullemma} stated next will prove to be useful.
\begin{lemma}
\label{lem:usefullemma}
For any $f:X \to Y$ and $g:Y\to Z$ injective functions, $f \circ g$ is injective.
\end{lemma}
\begin{theorem}
\label{thm:bigtheorem}
If $f:X\to Y$ is bijective, the cardinality of $X$ and $Y$ are the same.
\end{theorem}
An easy corollary of \cref{thm:bigtheorem} is the following:
\begin{corollary}
If $f:X\to Y$ is bijective, 
the cardinality of $X$ is at least as large as that of $Y$.
\end{corollary}
\begin{assumption}
The set $X$ is finite.
\label{ass:xfinite}
\end{assumption}
\begin{remark}
According to some, it is only the finite case (cf. \cref{ass:xfinite}) that is interesting.
\end{remark}
%restatable

\subsection{Citations and References}

Please use APA reference format regardless of your formatter
or word processor. If you rely on the \LaTeX\/ bibliographic
facility, use \texttt{natbib.sty} and \texttt{icml2025.bst}
included in the style-file package to obtain this format.

Citations within the text should include the authors' last names and
year. If the authors' names are included in the sentence, place only
the year in parentheses, for example when referencing Arthur Samuel's
pioneering work \yrcite{Samuel59}. Otherwise place the entire
reference in parentheses with the authors and year separated by a
comma \cite{Samuel59}. List multiple references separated by
semicolons \cite{kearns89,Samuel59,mitchell80}. Use the `et~al.'
construct only for citations with three or more authors or after
listing all authors to a publication in an earlier reference \cite{MachineLearningI}.

Authors should cite their own work in the third person
in the initial version of their paper submitted for blind review.
Please refer to \cref{author info} for detailed instructions on how to
cite your own papers.

Use an unnumbered first-level section heading for the references, and use a
hanging indent style, with the first line of the reference flush against the
left margin and subsequent lines indented by 10 points. The references at the
end of this document give examples for journal articles \cite{Samuel59},
conference publications \cite{langley00}, book chapters \cite{Newell81}, books
\cite{DudaHart2nd}, edited volumes \cite{MachineLearningI}, technical reports
\cite{mitchell80}, and dissertations \cite{kearns89}.

Alphabetize references by the surnames of the first authors, with
single author entries preceding multiple author entries. Order
references for the same authors by year of publication, with the
earliest first. Make sure that each reference includes all relevant
information (e.g., page numbers).

Please put some effort into making references complete, presentable, and
consistent, e.g. use the actual current name of authors.
If using bibtex, please protect capital letters of names and
abbreviations in titles, for example, use \{B\}ayesian or \{L\}ipschitz
in your .bib file.

\section*{Accessibility}
Authors are kindly asked to make their submissions as accessible as possible for everyone including people with disabilities and sensory or neurological differences.
Tips of how to achieve this and what to pay attention to will be provided on the conference website \url{http://icml.cc/}.

\section*{Software and Data}

If a paper is accepted, we strongly encourage the publication of software and data with the
camera-ready version of the paper whenever appropriate. This can be
done by including a URL in the camera-ready copy. However, \textbf{do not}
include URLs that reveal your institution or identity in your
submission for review. Instead, provide an anonymous URL or upload
the material as ``Supplementary Material'' into the OpenReview reviewing
system. Note that reviewers are not required to look at this material
when writing their review.

% Acknowledgements should only appear in the accepted version.
\section*{Acknowledgements}

\textbf{Do not} include acknowledgements in the initial version of
the paper submitted for blind review.

If a paper is accepted, the final camera-ready version can (and
usually should) include acknowledgements.  Such acknowledgements
should be placed at the end of the section, in an unnumbered section
that does not count towards the paper page limit. Typically, this will 
include thanks to reviewers who gave useful comments, to colleagues 
who contributed to the ideas, and to funding agencies and corporate 
sponsors that provided financial support.

\section*{Impact Statement}

Authors are \textbf{required} to include a statement of the potential 
broader impact of their work, including its ethical aspects and future 
societal consequences. This statement should be in an unnumbered 
section at the end of the paper (co-located with Acknowledgements -- 
the two may appear in either order, but both must be before References), 
and does not count toward the paper page limit. In many cases, where 
the ethical impacts and expected societal implications are those that 
are well established when advancing the field of Machine Learning, 
substantial discussion is not required, and a simple statement such 
as the following will suffice:

``This paper presents work whose goal is to advance the field of 
Machine Learning. There are many potential societal consequences 
of our work, none which we feel must be specifically highlighted here.''

The above statement can be used verbatim in such cases, but we 
encourage authors to think about whether there is content which does 
warrant further discussion, as this statement will be apparent if the 
paper is later flagged for ethics review.


% In the unusual situation where you want a paper to appear in the
% references without citing it in the main text, use \nocite
\nocite{langley00}

\bibliography{example_paper}
\bibliographystyle{icml2025}


%%%%%%%%%%%%%%%%%%%%%%%%%%%%%%%%%%%%%%%%%%%%%%%%%%%%%%%%%%%%%%%%%%%%%%%%%%%%%%%
%%%%%%%%%%%%%%%%%%%%%%%%%%%%%%%%%%%%%%%%%%%%%%%%%%%%%%%%%%%%%%%%%%%%%%%%%%%%%%%
% APPENDIX
%%%%%%%%%%%%%%%%%%%%%%%%%%%%%%%%%%%%%%%%%%%%%%%%%%%%%%%%%%%%%%%%%%%%%%%%%%%%%%%
%%%%%%%%%%%%%%%%%%%%%%%%%%%%%%%%%%%%%%%%%%%%%%%%%%%%%%%%%%%%%%%%%%%%%%%%%%%%%%%
\newpage
\appendix
\onecolumn
\section{You \emph{can} have an appendix here.}

You can have as much text here as you want. The main body must be at most $8$ pages long.
For the final version, one more page can be added.
If you want, you can use an appendix like this one.  

The $\mathtt{\backslash onecolumn}$ command above can be kept in place if you prefer a one-column appendix, or can be removed if you prefer a two-column appendix.  Apart from this possible change, the style (font size, spacing, margins, page numbering, etc.) should be kept the same as the main body.
%%%%%%%%%%%%%%%%%%%%%%%%%%%%%%%%%%%%%%%%%%%%%%%%%%%%%%%%%%%%%%%%%%%%%%%%%%%%%%%
%%%%%%%%%%%%%%%%%%%%%%%%%%%%%%%%%%%%%%%%%%%%%%%%%%%%%%%%%%%%%%%%%%%%%%%%%%%%%%%


\end{document}


% This document was modified from the file originally made available by
% Pat Langley and Andrea Danyluk for ICML-2K. This version was created
% by Iain Murray in 2018, and modified by Alexandre Bouchard in
% 2019 and 2021 and by Csaba Szepesvari, Gang Niu and Sivan Sabato in 2022.
% Modified again in 2023 and 2024 by Sivan Sabato and Jonathan Scarlett.
% Previous contributors include Dan Roy, Lise Getoor and Tobias
% Scheffer, which was slightly modified from the 2010 version by
% Thorsten Joachims & Johannes Fuernkranz, slightly modified from the
% 2009 version by Kiri Wagstaff and Sam Roweis's 2008 version, which is
% slightly modified from Prasad Tadepalli's 2007 version which is a
% lightly changed version of the previous year's version by Andrew
% Moore, which was in turn edited from those of Kristian Kersting and
% Codrina Lauth. Alex Smola contributed to the algorithmic style files.
