\subsection{Example Entry from LiveBench Reasoning}
\vspace{0.4cm}
\begin{tcolorbox}[title=Question, breakable]
In this question, assume each person either always tells the truth or always lies. The person at the campground thinks their friend is lying. Mateo is at the aquarium. The person at the restaurant says the person at the hotel lies. Farid is at the movie theater. The person at the movie theater says the person at the campground lies. Ryan is at the shopping mall. The person at the cafe says the person at the campground lies. The person at the observatory says the person at the museum lies. The person at the museum says the person at the restaurant tells the truth. The person at the campground says the person at the airport tells the truth. The person at the campground tells the truth. Luna is at the restaurant. The person at the movie theater saw a firetruck. Isabella is at the observatory. The person at the amusement park says the person at the museum tells the truth. The person at the restaurant says the person at the shopping mall tells the truth. Liam is at the beach. Ayaan is at the airport. The person at the restaurant says the person at the beach tells the truth. Maya is at the amusement park. The person at the restaurant says the person at the aquarium lies. The person at the restaurant says the person at the movie theater lies. The person at the library says the person at the campground tells the truth. Elowen is at the hotel. Priya is at the library. The person at the gym says the person at the restaurant tells the truth. Kehinde is at the cafe. Anika is at the museum. Quan is at the campground. Soren is at the gym. Does the person at the museum tell the truth? Does the person at the restaurant tell the truth? Does the person at the movie theater tell the truth? Think step by step, and then put your answer in **bold** as a list of three words, yes or no (for example, **yes, no, yes**). If you don't know, guess.
\end{tcolorbox}

\definecolor{mydarkgreen}{rgb}{0.0, 0.5, 0.0} %
\begin{paracol}{2}
\switchcolumn[0]
\begin{tcolorbox}[breakable, colframe=mydarkgreen!70, colback=mydarkgreen!10, title=Correct response]
Here's how we can break down this logic puzzle:

1. **Quan is at the campground and tells the truth.** This is a given fact.

2. **Quan (at the campground) says the person at the airport tells the truth.** Since Quan tells the truth, the person at the airport (Ayaan) tells the truth.

3. **The person at the movie theater (Farid) says the person at the campground (Quan) lies.** Since Quan tells the truth, Farid is lying.

4. **The person at the restaurant (Luna) says the person at the hotel (Elowen) lies.**  We don't know yet if Luna is lying or telling the truth.

5. **The person at the museum (Anika) says the person at the restaurant (Luna) tells the truth.** We still don't know if either of them are truthful.

6. **The person at the restaurant (Luna) says the person at the shopping mall (Ryan) tells the truth.** We still don't know Luna's truthfulness.

7. **The person at the restaurant (Luna) says the person at the beach (Liam) tells the truth.** Luna's truthfulness remains unknown.

8. **The person at the restaurant (Luna) says the person at the aquarium (Mateo) lies.**  We still don't know about Luna.

9. **The person at the restaurant (Luna) says the person at the movie theater (Farid) lies.**  We know Farid lies, so Luna is telling the truth in *this* statement.

10. **Since Luna tells the truth about Farid, Luna *always* tells the truth.**  This unlocks several other pieces of information.

11. **Luna (at the restaurant) says the person at the hotel (Elowen) lies.** Since Luna tells the truth, Elowen lies.

12. **Anika (at the museum) says the person at the restaurant (Luna) tells the truth.**  Since Luna tells the truth, Anika is also telling the truth.

13. **The person at the cafe (Kehinde) says the person at the campground lies.** Since Quan tells the truth, Kehinde is lying.


Therefore:

* **Anika (museum):** Tells the truth.
* **Luna (restaurant):** Tells the truth.
* **Farid (movie theater):** Lies.

**yes, yes, no**
\end{tcolorbox}

\switchcolumn
\begin{tcolorbox}[breakable, colframe=red!70, colback=red!10, title=Incorrect Response]
Here's how we can solve this logic puzzle:

1. **Quan is at the campground and tells the truth.** This is given.

2. **The person at the airport (Ayaan) tells the truth.** Quan (campground) says the person at the airport tells the truth, and Quan tells the truth.

3. **Anika (museum) lies.**  The person at the observatory (Isabella) says the person at the museum lies.  Since we don't know yet whether Isabella tells the truth, this doesn't tell us anything about Anika.  We'll have to come back to this.

4. **Luna (restaurant) lies.**  The person at the restaurant says the person at the hotel (Elowen) lies.  Later, the person at the restaurant says the person at the campground (Quan) lies.  Since Quan tells the truth, Luna (restaurant) must be lying.

5. **Anika (museum) lies.** Because Luna lies, and Luna says that the person at the hotel lies, Elowen tells the truth. Because Anika says the person at the restaurant (Luna) tells the truth, and Luna actually lies, Anika (Museum) must lie.


6. **Farid (movie theater) lies.** The person at the movie theater says the person at the campground lies. Since Quan (campground) tells the truth, Farid (movie theater) must be lying.

Therefore:

* **Anika (museum): lies**
* **Luna (restaurant): lies**
* **Farid (movie theater): lies**

**Answer: no, no, no** 
\end{tcolorbox}

\end{paracol}


\subsection{Example Entry from LiveBench Math}
\vspace{0.4cm}
\begin{tcolorbox}[title=Question, breakable]
How many positive perfect squares less than $2023$ are divisible by $5$? $\textbf{(A) }11\qquad\textbf{(B) }10\qquad\textbf{(C) }9\qquad\textbf{(D) }12\qquad\textbf{(E) }8$ If you cannot determine the correct multiple-choice answer, take your best guess. Once you have your answer, please duplicate that letter five times in a single string. For example, if the answer is F, then write FFFFF.
\end{tcolorbox}
\definecolor{mydarkgreen}{rgb}{0.0, 0.5, 0.0} %


\begin{paracol}{2}

\switchcolumn[0]
\begin{tcolorbox}[breakable, colframe=mydarkgreen!70, colback=mydarkgreen!10, title=Correct response]
Let $n$ be a positive perfect square divisible by $5$.
Then $n = k^2$ for some positive integer $k$, and $5 | n$, which implies that $5|k^2$.
Since $5$ is a prime number, $5|k^2$ means $5|k$.
Thus, $k$ must be a multiple of $5$, i.e., $k=5m$ for some positive integer $m$.
Then $n = k^2 = (5m)^2 = 25m^2$.
Thus $n$ must be a multiple of $25$.
We are looking for the number of positive perfect squares less than $2023$ that are divisible by $5$.
The perfect squares that are divisible by $5$ must be multiples of $25$.
We seek the number of multiples of $25$ that are perfect squares less than $2023$.
Since $n=25m^2 < 2023$, we have $m^2 < \frac{2023}{25} \approx 80.92$.
Thus, we are looking for perfect squares $m^2$ less than $80.92$.
The perfect squares are $1, 4, 9, 16, 25, 36, 49, 64, 81, \dots$.
The perfect squares less than $80.92$ are $m^2=1, 4, 9, 16, 25, 36, 49, 64$. There are $8$ such values.
The corresponding values of $n = 25m^2$ are
$25(1)=25$
$25(4)=100$
$25(9)=225$
$25(16)=400$
$25(25)=625$
$25(36)=900$
$25(49)=1225$
$25(64)=1600$
The next multiple of $25$ that is a perfect square would be $25(81)=2025$, which is not less than $2023$.
So, there are $8$ perfect squares less than $2023$ that are divisible by $5$.

Final Answer: The final answer is $\boxed{EEEEE}$
\end{tcolorbox}

\switchcolumn
\begin{tcolorbox}[breakable, colframe=red!70, colback=red!10, title=Incorrect Response]
Let $n$ be a positive integer such that $n^2$ is divisible by $5$.
Then $n^2$ is divisible by $5$. Since $5$ is a prime number, $n$ must also be divisible by $5$.
Thus $n$ can be written in the form $n = 5k$ for some integer $k$.
We want to find how many positive perfect squares $n^2$ are less than $2023$.
Since $n = 5k$, we have $n^2 = (5k)^2 = 25k^2 < 2023$.
Dividing by $25$ we get
$$k^2 < \frac{2023}{25} \approx 80.92$$
We are looking for integer values of $k$ such that $k^2 < 80.92$.
Since $k$ must be positive, $k$ can take values $1, 2, 3, \dots$. We have
\begin{itemize}
    \item $1^2 = 1 < 80.92$
    \item $2^2 = 4 < 80.92$
    \item $3^2 = 9 < 80.92$
    \item $4^2 = 16 < 80.92$
    \item $5^2 = 25 < 80.92$
    \item $6^2 = 36 < 80.92$
    \item $7^2 = 49 < 80.92$
    \item $8^2 = 64 < 80.92$
    \item $9^2 = 81 \nless 80.92$
\end{itemize}
Thus $k$ can take values from $1$ to $8$. The corresponding $n$ values are $n = 5k = 5, 10, 15, 20, 25, 30, 35, 40$. The perfect squares less than $2023$ and divisible by $5$ are $5^2, 10^2, 15^2, 20^2, 25^2, 30^2, 35^2, 40^2$ or $25, 100, 225, 400, 625, 900, 1225, 1600$.
There are $8$ such numbers.

Final Answer: The final answer is $\boxed{C}$CCCC
\end{tcolorbox}

\end{paracol}


\subsection{Example Entry from MATH}
\vspace{0.4cm}
\begin{tcolorbox}[title=Question, breakable]
If 4 daps = 7 yaps, and 5 yaps = 3 baps, how many daps equal 42 baps?
\end{tcolorbox}
\definecolor{mydarkgreen}{rgb}{0.0, 0.5, 0.0} %


\begin{paracol}{2}

\switchcolumn[0]
\begin{tcolorbox}[breakable, colframe=mydarkgreen!70, colback=mydarkgreen!10, title=Correct response]
Let 'd' represent daps, 'y' represent yaps, and 'b' represent baps.
We are given:
4 daps = 7 yaps, which can be written as:
4d = 7y

and

5 yaps = 3 baps, which can be written as:
5y = 3b

We want to find the number of daps that are equal to 42 baps. Let 'x' be the number of daps. We want to find x such that:
xd = 42b

From 5y = 3b, we can express yaps in terms of baps:
y = (3/5)b

Now substitute this value of y into the first equation, 4d = 7y:
4d = 7 * (3/5)b
4d = (21/5)b

We want to find xd = 42b, so we can set:
xd = 42b

We have 4d = (21/5)b, so we can solve for d:
d = (21/20)b

Now substitute this into xd = 42b:
x * (21/20)b = 42b

Since b is not zero, we can divide both sides by b:
x * (21/20) = 42

Now, solve for x:
x = 42 * (20/21)
x = (42/21) * 20
x = 2 * 20
x = 40

So, 40 daps are equal to 42 baps.

$\boxed{40}$
Final Answer: The final answer is $\boxed{40}$

\end{tcolorbox}

\switchcolumn
\begin{tcolorbox}[breakable, colframe=red!70, colback=red!10, title=Incorrect Response]
Let 'd' represent daps, 'y' represent yaps, and 'b' represent baps.
We are given two relationships:

1. 4 daps = 7 yaps, which can be written as $4d = 7y$
2. 5 yaps = 3 baps, which can be written as $5y = 3b$

We want to find how many daps are equal to 42 baps.  We want to find $d$ in terms of $b$, and then evaluate for $b=42$.

First, we can find the relationship between yaps and baps from the second equation:
$5y = 3b$, so $y = \frac{3}{5}b$

Now, substitute this expression for $y$ in terms of $b$ into the first equation:
$4d = 7y$
$4d = 7\left(\frac{3}{5}b\right)$
$4d = \frac{21}{5}b$

Now, we want to find $d$ in terms of $b$, so divide both sides by 4:
$d = \frac{21}{5 \cdot 4}b$
$d = \frac{21}{20}b$

Finally, we want to find how many daps equal 42 baps, so we substitute $b=42$:
$d = \frac{21}{20} \times 42$
$d = \frac{21 \times 42}{20}$
$d = \frac{882}{20}$
$d = \frac{441}{10}$
$d = 44.1$

So, 44.1 daps equal 42 baps.

Final Answer: The final answer is $\boxed{44.1}$

\end{tcolorbox}

\end{paracol}

\subsection{Example of Response with Invalid Reasoning but Correct Final Answer}

\vspace{0.4cm}
\begin{tcolorbox}[title=Question, breakable]
You have seven bags of gold coins. Each bag has the same number of gold coins. One day, you find a bag of 53 coins. You decide to redistribute the number of coins you have so that all eight bags you hold have the same number of coins. You successfully manage to redistribute all the coins, and you also note that you have more than 200 coins. What is the smallest number of coins you could have had before finding the bag of 53 coins?
\end{tcolorbox}
\definecolor{mydarkgreen}{rgb}{0.0, 0.5, 0.0} %


\begin{tcolorbox}[breakable, colframe=red!70, colback=red!10, title=Incorrect Response Reaching Correct Final Answer]
Let $n$ be the number of coins in each of the seven bags. After finding the bag of 53 coins, you have a total of $7n + 53$ coins.
You redistribute these coins into 8 bags such that each bag contains the same number of coins.
Therefore, the total number of coins is divisible by 8, so $7n+53$ is divisible by 8.
$7n + 53 \equiv 0 \pmod{8}$.
We want to find $n$.
$7n \equiv -53 \pmod{8}$.
Since $-53 \equiv -53 + 8(7) \equiv -53 + 56 \equiv 3 \pmod{8}$, we have
$7n \equiv 3 \pmod{8}$.

Since $7 \equiv -1 \pmod{8}$, we multiply both sides by $-1$ to get
$-n \equiv -7n \equiv -3 \pmod{8}$, so $n \equiv -3 \equiv 5 \pmod{8}$.
Thus $n$ must be of the form $8k + 5$ for some integer $k \geq 0$.
The total number of coins originally was $7n$. Since $n=8k+5$, the total number of coins originally is
$7(8k+5) = 56k + 35$.
We are given that the total number of coins is greater than 200.
So $7n > 200$, or $n > \frac{200}{7} \approx 28.57$.
Since $n = 8k + 5$, we want the smallest integer $k$ such that $8k + 5 > \frac{200}{7}$.
For $k=0$, $n=5$, so $7n=35<200$.
For $k=1$, $n=13$, so $7n=91<200$.
For $k=2$, $n=21$, so $7n=147<200$.
For $k=3$, $n=29$, so $7n=203>200$.

We check that $7n + 53 = 203 + 53 = 256$ which is divisible by $8$ ($256 = 8 \times 32$).
Then $256/8 = 32$ coins in each bag.

Thus the smallest number of coins you could have had before finding the bag is $7n = 7(29) = 203$.

Final Answer: The final answer is $\boxed{203}$.
\end{tcolorbox}
