In this section we present the prompts used to query models to self-verify candidate responses.
In \Cref{tab:prompt_tuning_results}, this corresponds to the ``Main'' prompt style.
We also provide the prompts used for tie-breaking comparisons.
These prompts are broken into multiple parts to encourage longer responses.

\subsection{Verification Scoring Prompts}

\begin{tcolorbox}[title=Verification Prompt 1,breakable]
[Commandments] Over the coming interactions, you must fulfill the following commandments:
1. *Be excruciatingly detailed and exhaustive in your analyses.* This will often mean that your responses will be long. Do not cut your responses short. When you are asked to fulfill a list of tasks, you must fulfill each and every task to completion.

2. *Be structured and systematic in your responses*. Organize your thoughts in a clear hierarchical format. Use neutral, rigorous mathematical language to write things in your own words and avoid subjective descriptions.

3. *Never put the cart before the horse.* Before making a claim or statements, always verbally reason out your chain of thought and convince yourself of it in an exhaustive fashion. Instead of saying "X is Y because Z", say "Consider Z. Therefore ..., meaning that X is Y.". Avoid premature conclusions.

[Background] An examiner has presented a math problem along with a "candidate solution". Your purpose is to assist in evaluating whether the *final answer* in the candidate solution is correct. We will guide you through a series of exercises to fulfill this purpose. The question is a real exam problem, which means there is always one unique correct answer.

Here is the math question.
\{\}
Here is the candidate solution.
\{\}

** Current Task **
Your task is to rewrite the candidate solution into a rigorous, structured format consistent with mathematical convention. You should rewrite the solution in your own words, and work out every step in exhaustive detail, allowing us to more easily check for errors down the line. As a starting point, you will first decompose the candidate solution into a list of self-contained *lemmas*. Later on, we will have you use your lemmas to rewrite the candidate solution in the following format:
\begin{Verbatim}[breaklines,breaksymbolleft=]
\begin{theorem} ... is \boxed{...}. \end{theorem}
\begin{proof}[Proof of Theorem] ... \end{proof}
\begin{lemma}\label{...} ... \end{lemma}
\begin{proof}[Proof of Lemma~\ref{...}] ... \end{proof}
...
\end{Verbatim}

For now, focus only on identifying, structuring, and clearly defining each lemma without proofs. We will write proofs for these lemmas, and a proof for the final answer using these lemmas, at a later point in time.

[Steelmanning]
If you encounter any part of the candidate solution that seems incorrect or unclear, reason about it in your scratchpad. Explain your thought process to clarify your understanding. Consider whether there might be an alternative interpretation or if you could be overlooking or overcomplicating something. If, after reasoning through it, you still find it confusing, note in your scratchpad that this part may need to be revisited and proceed while treating the potentially problematic step as temporarily correct. At the current stage, we want to try to presume the candidate solution as being correct. Try your best to *steel man* the candidate solution.

[Task Specifics]
To ensure accuracy, you will decompose the candidate solution into self-contained *lemmas*. These lemmas will serve as building blocks, allowing us to reconstruct the candidate solution in a more rigorous and conventional mathematical format. 

[Steps to Follow]
1. **Initial Analysis**: Carefully read through the candidate solution and break it down into its primary steps.
2. **Lemma Identification**: Identify each logical segment of the candidate solution that could serve as a lemma. For each lemma:
   - **Scratchpad**: Before writing the lemma formally, record a brief scratchpad of thoughts outlining assumptions, conditions, and any definitions necessary to ensure the lemma is self-contained. This should help you catch potential errors or misapplications that might be subtle. Reason about how to write the lemma in a fashion that allows for easy verification of both the lemma, and how the lemma is used in the final proof.
   - **Lemma Statement**: After organizing your thoughts, write each lemma in a clear and rigorous manner. Ensure each lemma is isolated from others, so it does not depend on assumptions not explicitly stated within it.

[Response Format] Structure your response as follows.
\begin{Verbatim}[breaklines,breaksymbolleft=]
# Analyzing the solution
(Study the language of the question, keeping an eye out for important details.)
(Identify the candidate solution's final answer. Work backwards through the solution to identify the intermediate claims that led to the final answer.)
(Proceed from the beginning of the candidate solution, working forwards to identify the main steps of the solution.)
# Identifying lemmas
Scratchpad: [Reason about what lemma you should write. Reason about how to word the statement of the lemma to make it easy to verify when the lemma is being used correctly. Reason about how to write the lemma in a self-contained fashion, and what definitions and assumptions are needed.]
\begin{lemma}
\label{lem:lemma_name}
[State the lemma formally here]
\end{lemma}
Scratchpad: ...
\begin{lemma}
...
\end{lemma}
...
\end{Verbatim}
\end{tcolorbox}

\begin{tcolorbox}[breakable,title=Verification Prompt 2]
Now that you have structured the candidate solution into lemmas, you will proceed to rewrite the candidate solution in your own words. We will first have you write the *main proof for the final answer*. At the next round, I will ask you to write proofs for each lemma you identified.

[Details on Writing a **Main Proof for Final Answer**]
Write a rigorous "main proof" that directly addresses the candidate solution's final answer. This proof should reference each lemma sequentially to logically reach the final answer. Ensure that the proof is rigorous and that each step explicitly relies on or follows from a lemma. Your goal is to work out every baby step of the proof in exhaustive detail, spelling out every tiny step, including even tiny details about arithmetic and algebraic manipulations.

[Objective] 
Your rewrite of the candidate solution should be comprehensive and precise. Your goal is to write the candidate solution so that it is easy to verify whether it is correct or identify any errors at both the high and low levels.

[Steelmanning] 
Once again, try to "steel man" the candidate solution. If you encounter any part of the candidate solution that seems incorrect or unclear, reason about it in your scratchpad and try to find an explanation; then proceed treating the potentially problematic step as correct. We want to try to presume the candidate solution as being correct for now.

[Response Format] 
Structure your response as follows to ensure clarity.
\begin{Verbatim}[breaklines,breaksymbolleft=]
\begin{theorem}[Main Claim]
(Write the math question and final answer as a statement. For example, "The smallest non-zero integers is \boxed{1}.". This statement should contain all information from the question, and be written in your own words. If it is not obvious how to write the student solution's final answer as a theorem, you can simply let the theorem take the form: "{Question}? The answer is {Answer}.")
\end{theorem}
Scratchpad: (Reason about how you will write the main proof.)
\begin{proof}
(Provide a proof of the final result. Remember to explicitly reference lemmas via \ref{lem:lemma_name}.)
\end{proof}
\end{Verbatim}
\end{tcolorbox}

\begin{tcolorbox}[breakable,title=Verification Prompt 3]
Your task is now to provide proofs for each and every lemma you identified in the candidate solution.

[Details on Writing **Lemma Proofs**]
For each lemma, write a detailed and thorough proof of the lemma. Each proof should methodically use the assumptions, definitions, and conditions in the lemma's statement to verify the conclusion.
This proof should be rigorous and work out every baby step in exhaustive detail, spelling out every tiny step, including even tiny details about arithmetic and algebraic manipulations.
Remember that you may encounter a lemma that "builds" on other lemmas, in that it uses the other lemmas to prove its claim. In these cases, you should have already written the lemma so that it incorporates those other lemmas as "assumptions", so that you do not need to replicate the work. If this is not the case, please revise the lemma statement accordingly.

[Objective] Your rewrite of the candidate solution should be comprehensive and precise. Your goal is to write the candidate solution so that it is easy to verify whether it is correct or identify any errors at both the high and low levels.

[Steelmanning] Once again, try to "steel man" the candidate solution. If you encounter any part of the candidate solution that seems incorrect or unclear, reason about it in your scratchpad and try to find an explanation; then proceed treating the potentially problematic step as correct. We want to try to presume the candidate solution as being correct for now.

[Response Format] Structure your response as follows to ensure clarity and rigor. You must provide proofs for all lemmas. Do not skip any lemmas, defer any writing to the future, or leave any proofs incomplete. You must return full proofs for every lemma; I am an automated system and unable to handle incomplete responses. You will not have a chance to finish your response later. Provide a complete response.
\begin{Verbatim}[breaklines,breaksymbolleft=]
\begin{lemma}
\label{...}
...
\end{lemma}
Scratchpad: (Reason about how you will write the lemma proof.)
\begin{proof}
(Provide a proof of the lemma.)
\end{proof}
... (repeat for EVERY lemma)
\end{Verbatim}
\end{tcolorbox}

\begin{tcolorbox}[breakable,title=Verification Prompt 4]
Now we will proceed to analyzing the correctness of each suspicious claim. For each item in the claim:
1. Check if the candidate solution's final answer indeed depends on the claim. This step is important to rule out false positives where the candidate solution makes a False claim, but the claim does not really affect the final answer so it's Falsity does not result in an actual error in the final answer. If this check is failed, i.e. the final answer does not depend on the claim, then discard this claim and continue to the next suspicious claim.
2. Check that you are unable to prove the suspicious claim is True. Do this by trying to prove the suspicious claim. You can make multiple attempts. If you are able to prove the suspicious claim, meaning that the suspicious claim is correct, then discard this claim and continue to the next suspicious claim.
3. Check that the claim is False by proving a corrected alternative version of the claim. If you are unable to do this, continue to the next suspicious claim.
4. Correct this suspicious claim in the candidate solution and determine the corrected final answer in a step-by-step manner. If the final answer changes or correcting this error invalidates the candidate solution's proof approach, mark this suspicious claim as a "fatal error".
At the end, say **Yes** if you found a fatal error. Otherwise, say **No, final answer is correct**.
\end{tcolorbox}

\begin{tcolorbox}[breakable,title=Verification Prompt 5]
Now we will proceed to analyzing the correctness of the candidate solution. We have two goals: identify if there is an error, and if there is an error, repair the candidate solution and identify whether the final answer has changed. For now, we will focus on methodically combing through the candidate solution and checking each step for a potential error.

[Steps to Follow]
You will proceed through the candidate solution, one baby step at a time. In your exhaustive investigation of the solution, you will first form a short list of potential errors. Each entry in this list should be written as a self-contained, standalone mathematical claim that the candidate solution relies on being correct, but which you find suspicious upon first inspection. After forming this list, you will then proceed to validate each potential error by performing a detailed error validation check.

* Main Proof *
For now, focus on the main proof. Assume that every lemma, as written, is correct. Step through the main proof of the candidate solution in exhaustive detail to find potential errors.
Do this by first proceeding through the main proof one sentence (or other reasonable unit of content) at a time. Quote the part of the proof you are at. Then add a discussion where you try to elaborate on that part and verify every baby step made in that unit of the proof. Then repeat this for the next unit of the proof until you have gone through (quoted and analyzed) each part of the proof. You should structure your output like
\begin{Verbatim}[breaklines,breaksymbolleft=]
> Quoted part of the proof
Your analysis and detailed discussion verifying every small thing, e.g. redoing every baby step, checking assumptions, rephrasing things to make sure they sound correct to you, etc. Make sure you double check the exact language of the question and the exact language of the lemmas used.
> Next quoted part of the proof
...
\end{Verbatim}

Then, write down in a bullet point list any potential errors you find in the main proof. Before each bullet point, write a scratchpad that verbalizes: your thought process and---upon finding an error---your thought process for writing the erroneous step as a self-contained mathematical claim. Try to be selective and only include things that you consider likely to be errors.
You should structure your output like
\begin{Verbatim}[breaklines,breaksymbolleft=]
Scratchpad: ...
* ...
Scratchpad: ...
* ...
...
\end{Verbatim}
\end{tcolorbox}

\begin{tcolorbox}[breakable,title=Verification Prompt 6]
You are now to verify the proof of each and every lemma. You must verify all lemmas in this conversation turn; you will not have another chance.

* Lemma ... Proof * (repeat this for each lemma)
Follow the same procedure as before. First, step through the lemma proof one sentence (or other reasonable unit of content) at a time. Quote the part of the proof you are at. Then add a discussion where you try to elaborate on that part and verify every baby step made in that unit of the proof. Then repeat this for the next unit of the proof until you have gone through (quoted and analyzed) each part of the proof.
Then, write down in a bullet point list any potential errors you find in the lemma proof. Before each bullet point, write a scratchpad that verbalizes: your thought process and---upon finding an error---your thought process for writing the erroneous step as a self-contained mathematical claim. Try to be selective and only include things that you consider likely to be errors.
Structure your response as follows.
\begin{Verbatim}[breaklines,breaksymbolleft=]
# Review of Lemma ...
## Stepping through
> Quoted part of the proof
Your analysis and detailed discussion verifying every small thing, e.g. redoing every baby step, checking assumptions, rephrasing things to make sure they sound correct to you, etc. Make sure you double check the exact language of the lemma's claim.
> Next quoted part of the proof
## Potential errors
Scratchpad: ...
* ...
Scratchpad: ...
* ...
...
\end{Verbatim}
\end{tcolorbox}

\begin{tcolorbox}[breakable,title=Verification Prompt 7]
Now we will proceed to analyzing the correctness of each suspicious claim. For each item in the claim:
1. Check if the candidate solution's final answer indeed depends on the claim. This step is important to rule out false positives where the candidate solution makes a False claim, but the claim does not really affect the final answer so it's Falsity does not result in an actual error in the final answer. If this check is failed, i.e. the final answer does not depend on the claim, then discard this claim and continue to the next suspicious claim.
2. Check that you are unable to prove the suspicious claim is True. Do this by trying to prove the suspicious claim. You can make multiple attempts. If you are able to prove the suspicious claim, meaning that the suspicious claim is correct, then discard this claim and continue to the next suspicious claim.
3. Check that the claim is False by proving a corrected alternative version of the claim. If you are unable to do this, continue to the next suspicious claim.
4. Correct this suspicious claim in the candidate solution and determine the corrected final answer in a step-by-step manner. If the final answer changes or correcting this error invalidates the candidate solution's proof approach, mark this suspicious claim as a "fatal error".
At the end, say **Yes** if you found a fatal error. Otherwise, say **No, final answer is correct**.
\end{tcolorbox}

\begin{tcolorbox}[breakable,title=Verification Prompt 8]
I want you to now, in the style of the *rewritten solution* (i.e., in the theorem-lemma format I had you rewrite the original solution in), write an improved solution that does the following:
1) It corrects any errors you found in the original solution (only if you ended up finding an error). If you do not know how to fix the error, then just write the step you are unsure about as an "Assumption (to be revisited later)", making it clear that the proof is not complete and needs that step to be filled in.
2) It clarifies ambiguities or fills in gaps in the original solution that led you to suspect errors (even if the suspicion was not ultimately substantiated). We want your improved solution to be clear enough that future readers would not have to work through the same worries you did.
3) It adopts the thoroughness, structured format, exhaustive detail, and rigor of your rewritten version of the solution.

I have included again the original question and original solution below for your reference. \{\}\{\}
For now, I want you to brainstorm how to write the revised solution. Do not yet proceed to writing the solution. Structure your response as follows. Do not cut your response short. You have unlimited space.
\begin{Verbatim}[breaklines,breaksymbolleft=]
# What errors need to be corrected?
Discuss in detail, in a step-by-step manner, what errors you found in the original solution and how you plan to correct them.
If you found no errors previously, note so and continue to the next section.
# Which ambiguities need to be clarified or gaps need to be filled?
Discuss in detail, in a step-by-step manner, what ambiguities or gaps you plan to clarify.
Draw on our previous exchanges explicitly. Recall what were potential points of confusion during this verification process.
# Action Plan
Prepare a rough battle plan for how you will write the improved solution. Note that it will probably be longer than you original solution, given all of the revisions you have planned. So this is just to plan out the main parts, and things to keep an eye out for.
\end{Verbatim}
\end{tcolorbox}

\begin{tcolorbox}[breakable,title=Verification Prompt 9]
Now, I want you to write your improvement of the original solution. You must provide the entirety of your solution and nothing else. This message will be copy pasted to an external system, so your response must be complete and self-contained. Do not shorten anything. Provide it to me exactly and in its entirety. Remember to say nothing else and structure your output as follows, wrapped in tripe quotes and saying nothing else.
\end{tcolorbox}

\subsection{Comparison Prompts}

\begin{tcolorbox}[breakable,title=Comparison Prompt 1]
Your job is to answer a difficult math question. I will provide you with two solutions written by my students. These two solutions disagree over what the answer should be. Both solutions may potentially have minor errors, even if one indeed reaches the final answer despite the minor flaws. You will be given a series of instructions, over multiple interactions, that will guide you through discerning the correct final answer. You will be expected to complete each step carefully while obeying the following prime directives.

1. Always provide complete responses. Never shorten your responses. You are allocated 10,000 tokens per-response. Your instructions are provided according to a fixed schedule; you must complete them in the same conversation turn as you will not have later opportunities to do so.

2. Speak carefully. Always reason in a step-by-step chain-of-thought manner. Your responses must always resemble an internal monologue, which means you verbally reason things out before reaching conclusions, rather than pulling answers out of thin air.

3. Be rigorous. Always validate your logic by attempting to mathematically formalize it to avoid silly "common sense" errors. Always work out mathematical steps in small baby steps, even seemingly obvious arithmetic or algebraic manipulations.

4. Backtrack when you have made a mistake. It is not uncommon for when to verbally say something that is false or silly during an internal monologue. Constantly introspect and if you have made an error, identify it and "backtrack" to just before you made the error.

5. Never claim that anything is incorrect or wrong or right or correct. You must always say that something is "potentially incorrect" or "potentially correct" or "seems incorrect" or "seems correct". You must then work it out in baby steps and give your more informed judgement. But you will never say that something is correct or right or wrong.

\# The Question
\{\}

\# The First Solution
\{\}

\# The Second Solution
\{\}

\# Your current Task
Examine the first solution. Focus for now on just reading through the main proof and each lemma's statement. You must read through the main proofs and lemma statements in a meticulous sentence-by-sentence manner. For now, you do not need to read through the lemma proofs in detail. Though later on, if we have questions arise about particular lemmas, you will need to review the lemma proof carefully. In addition, I want you to narrate your process of reading through each proof. This means that, while you read through a solution, you must always quote the part of the solution that you are currently reading. Then, under your quote, provide your mental process. Here are some questions that you should keep top of mind: What is the approach being taken by the solution? What are the main leaps in logic? How does the solution try to be rigorous?
Structure your response as follows:

\begin{Verbatim}[breaklines,breaksymbolleft=]
# Main Proof
> "Quote of the sentence you are reading."
Your thought process. Discuss the quote. Perhaps compare it to a sentence in another solution that is doing the same exact thing. Perhaps note that it does a fairly complex algebraic manipulation, and spend a minute to double-check it by working it out in baby-steps. If it references a lemma, discuss the lemma's statement, how it's being used, and why it's allowed to be used in this way.
> "Quote of next sentence..."
...
# Lemma 1
...
(continue through the ENTIRE REST OF THE SOLUTION)
\end{Verbatim}

Remember you only need to do this for the first solution for now.
You must provide a complete response. You must go through the entire solution line by line.
Ignore character limits and do not cut your response short. Do not cut your response short, you will not get another chance.
\end{tcolorbox}

\begin{tcolorbox}[breakable,title=Comparison Prompt 2]
Do the same for the second solution. Again, be meticulous and adopt the same output format.
\end{tcolorbox}

\begin{tcolorbox}[breakable,title=Comparison Prompt 3]
Identify similarities and disagreements between the first solution and the second solution. Do not yet judge which side is right; merely try to be investigative about: what are the sources of these disagreements? You must continue to "quote" any solution parts that you reference.
Remember your mandates:

1. Always provide complete responses. Never shorten your responses. You are allocated 10,000 tokens per-response. Your instructions are provided according to a fixed schedule; you must complete them in the same conversation turn as you will not have later opportunities to do so.

2. Speak carefully. Always reason in a step-by-step chain-of-thought manner. Your responses must always resemble an internal monologue, which means you verbally reason things out before reaching conclusions, rather than pulling answers out of thin air.

3. Be rigorous. Always validate your logic by attempting to mathematically formalize it to avoid silly "common sense" errors. Always work out mathematical steps in small baby steps, even seemingly obvious arithmetic or algebraic manipulations.

4. Backtrack when you have made a mistake. It is not uncommon for when to verbally say something that is false or silly during an internal monologue. Constantly introspect and if you have made an error, identify it and "backtrack" to just before you made the error.

5. Never claim that anything is incorrect or wrong or right or correct. You must always say that something is "potentially incorrect" or "potentially correct" or "seems incorrect" or "seems correct". You must then work it out in baby steps and give your more informed judgement. But you will Never say that something is correct or right or wrong..

Be extremely careful with respect to the exact wording of the question: \{\}

Remember: you must be humble and careful in your judgements. Just because you think one side is correct doesn't mean that your assessment is sound. Always keep an open mind. Always double check you aren't missing out on any more potential points of disagreement. Work through this in a careful detailed chain-of-thought that irons out every small detail..

Structure your responses into the following sections. Each section must be a lengthy, detailed, well-structured investigation.
\begin{Verbatim}[breaklines,breaksymbolleft=]
# Identify Disagreements
Reading through each solution, identify as many as places as possible where the solutions differ. Be meticulous and form a detailed list.
# Attribute Disagreements
For each disagreement, try to trace the disagreement back to earlier places in the respective proofs. What are the points at which the solutions begin to diverge? Can we guess at the root causes of the disagreements? Could it be because the solutions disagree over how to interpret the question? Could it be because they disagree about what approach to take? Could it be because the solutions reach a different calculation? CAREFULLY pour over solutions from both sides. This detective work must be careful and methodical and detailed and exhaustive.
\end{Verbatim}

You must provide your full response. Ignore character limits. Do not cut your response short. Obey my instructions exactly.
\end{tcolorbox}

\begin{tcolorbox}[breakable,title=Comparison Prompt 4]

Conduct detailed analysis to try and understand why the solutions reach different final answers. Do not yet judge who is right. Do they disagree on how to read the question? Do they just happen to take very different mathematical approaches? Do they diverge at a particular logical step or calculation? You must continue to "quote" any solution parts that you reference.
Remember your mandates:

1. Always provide complete responses. Never shorten your responses. You are allocated 10,000 tokens per-response. Your instructions are provided according to a fixed schedule; you must complete them in the same conversation turn as you will not have later opportunities to do so.

2. Speak carefully. Always reason in a step-by-step chain-of-thought manner. Your responses must always resemble an internal monologue, which means you verbally reason things out before reaching conclusions, rather than pulling answers out of thin air.

3. Be rigorous. Always validate your logic by attempting to mathematically formalize it to avoid silly "common sense" errors. Always work out mathematical steps in small baby steps, even seemingly obvious arithmetic or algebraic manipulations.

4. Backtrack when you have made a mistake. It is not uncommon for when to verbally say something that is false or silly during an internal monologue. Constantly introspect and if you have made an error, identify it and "backtrack" to just before you made the error.

5. Never claim that anything is incorrect or wrong or right or correct. You must always say that something is "potentially incorrect" or "potentially correct" or "seems incorrect" or "seems correct". You must then work it out in baby steps and give your more informed judgement. But you will never say that something is correct or right or wrong.

Be extremely careful with respect to the exact wording of the question:
\{\}

Remember: you must be humble and careful in your judgements. Just because you think one side is correct doesn't mean that your assessment is sound. Always keep an open mind. Always double check you aren't missing out on any more potential points of disagreement. Work through this in a careful detailed chain-of-thought that irons out every small detail.. You are NOT to judge which side is correct, for now.

Structure your responses into the following sections. Each section must be a lengthy, detailed, well-structured investigation.
\begin{Verbatim}[breaklines,breaksymbolleft=]
# Crystallize Approaches
If the two sets of solutions differ in approach, it can be hard to judge which side is correct about the final answer since you cannot easily compare the solutions of each. However, even if you cannot directly compare the steps, you can try to find "intermediate" points of disagreement. That is, as part of their respective arguments, two solutions that use completely different approaches may make contradicting subclaims that you can investigate in more detail. These disagreements may point you to where one approach has a "mistake" in its proof. Your end goal: identify the meaningful differences (if any) in the approaches taken by each side and speculate as to why these differences resulted in them reaching different final answers. Remember: you must be humble and careful in your judgements. Just because you think one side is correct doesn't mean that your assessment is sound. Always keep an open mind. Work through this in a careful detailed chain-of-thought that irons out every small detail..
# Crystallize Disagreements
For each disagreement, you must now collect evidence from both sides. You are a careful mathematically rigorous agent with good common-sense reasoning. Try to use your common sense and rigour to work through each side's evidence. Work through this in a careful detailed chain-of-thought that irons out every small detail..
\end{Verbatim}
You must provide your full response. Ignore character limits. Do not cut your response short. Obey my instructions exactly.
\end{tcolorbox}

\begin{tcolorbox}[breakable,title=Comparison Prompt 5]
You must now determine which of the two solutions reaches the correct final answer. Do not leap to judgement.
First, carefully consider the sources of the solutions disagreement over the final answer.
Then, for each of these, almost axiomatic, root disagreements:
1) Carefully consider the arguments that both sides make and the evidence that both sides present. For each side of each disagreement, try to *steelman* their argument: write it in a clear, detailed convincing fashion that leaves nothing unspoken and uninvestigated.
2) Use your own common sense and mathematical skills to try and rule on which side is right. Do not leap to judgement. Be careful, rigorous, meticulous and don't forget common sense.
Finally, carefully reason through what the correct solution should be and which side reached the correct final answer. Be meticulous.
Structure your response as follows. This must be more detailed and rigorous than all your previous responses, since this is a careful process.
\begin{Verbatim}[breaklines,breaksymbolleft=]
# Recall Key Disagreements
...
# Disagreement 1
## Steelmanning Each Side
...
## Ruling On A Side
...
# Which Side Is Right?
...
\end{Verbatim}

You must provide your full response. Ignore character limits. Do not cut your response short. Obey my instructions exactly.
\end{tcolorbox}
