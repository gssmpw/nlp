\subsection{Temperature Tuning}
\label{app:temp}

After attempting temperature tuning on the training split of the MATH dataset, we found that the choices of temperatures $\sigma_{\mathrm{inf}}$ and $\sigma_{\mathrm{verif}}$ did not significantly affect performance.
In Table~\ref{tab:inference_temp_modified} and Table~\ref{tab:verification_temp_modified}, we compare the post-verification accuracy of our pipeline for various temperature choices.
The Verification@20 figures are obtained without running tie-breaking and with only 20 verification attempts.

\begin{table}[H]
\centering
\begin{tabular}{cccc}
\toprule
\textbf{Inference Temp \(\sigma_{\mathrm{inf}}\)} & \textbf{Pass@20} & \textbf{Consistency@20} & \textbf{Verification@20} \\
\midrule
0.2 & 89/100 & 76/100 & 82/100 \\
1.0 & 94/100 & 73/100 & 80/100 \\
1.5 & 89/100 & 73/100 & 79/100 \\
2.0 & 89/100 & 75/100 & 82/100 \\
\bottomrule
\end{tabular}
\caption{Accuracy rates of the Gemini v1.5 Pro model on the training split of MATH using different methods of selecting from 20 generated responses and varying the temperature used to generate the responses. The temperature used for verification attempts is fixed at \(\sigma_{\mathrm{verif}} = 1.0\). Inference temperature does not significantly affect downstream performance.}
\label{tab:inference_temp_modified}
\end{table}

\begin{table}[H]
\centering
\begin{tabular}{cccc}
\toprule
\textbf{Verification Temp \(\sigma_{\mathrm{verif}}\)} & \textbf{Pass@20} & \textbf{Consistency@20} & \textbf{Verification@20} \\
\midrule
0.2 & 89/100 & 73/100 & 76/100 \\
1.0 & 89/100 & 73/100 & 79/100 \\
2.0 & 89/100 & 73/100 & 79/100 \\
\bottomrule
\end{tabular}
\caption{Accuracy rates of the Gemini v1.5 Pro model on the training split of MATH using different methods of selecting from 20 generated responses and varying the temperature used to for verification attempts. The temperature used for generating responses is fixed at \(\sigma_{\mathrm{inf}} = 1.5\). Verification temperature does not significantly affect downstream performance.}
\label{tab:verification_temp_modified}
\end{table}


\subsection{Olympiad LiveBench Math Subtask}
\label{app:oddity}

The one task for which we saw no lift from verification is the Olympiad questions from LiveBench MATH.
These questions are not formatted as open-ended problems.
Rather, they take a very specific form of asking one to fill in a pre-written proof from a menu of expression options, and to output a specific sequence of indices corresponding to these options.
This is incompatible with our verification pipeline, which asks the verification model to rewrite candidate responses in a theorem-lemma format where the theorem states the final answer.
For example, the final answer to the Olympiad question at the bottom of this section is the following sequence:

\tcbox[on line,  boxrule=0.5pt, top=2pt, bottom=0pt, left=1pt, right=1pt]{
19,32,20,2,14,1,27,21,31,36,3,30,5,16,29,34,7,4,6,18,15,22,9,25,28,35,26,8,13,24,23,17,33,11,10,12}.

\begin{tcolorbox}[breakable, title=Representative example of Olympiad task from LiveBench MATH]
You are given a question and its solution. The solution however has its formulae masked out using the tag <missing X> where X indicates the identifier for the missing tag. You are also given a list of formulae in latex in the format "<expression Y> = $\text{latex code}$" where Y is the identifier for the formula. Your task is to match the formulae to the missing tags in the solution. Think step by step out loud as to what the answer should be. If you are not sure, give your best guess. Your answer should be in the form of a list of numbers, e.g., 5, 22, 3, ..., corresponding to the expression identifiers that fill the missing parts. For example, if your answer starts as 5, 22, 3, ..., then that means expression 5 fills <missing 1>, expression 22 fills <missing 2>, and expression 3 fills <missing 3>.

The question is:
Find all integers $n \geq 3$ such that the following property holds: if we list the divisors of $n !$ in increasing order as $1=d_1<d_2<\cdots<d_k=n!$, then we have 
\[
d_2-d_1 \leq d_3-d_2 \leq \cdots \leq d_k-d_{k-1} .
\]

Find all integers $n \geq 3$ such that the following property holds: if we list the divisors of $n !$ in increasing order as $1=d_1<d_2<\cdots<d_k=n!$, then we have 
\[
d_2-d_1 \leq d_3-d_2 \leq \cdots \leq d_k-d_{k-1} .
\]

The solution is:
We can start by verifying that <missing 1> and $n=4$ work by listing out the factors of <missing 2> and <missing 3>. We can also see that <missing 4> does not work because the terms $15, 20$, and $24$ are consecutive factors of <missing 5>. Also, <missing 6> does not work because the terms <missing 7>, and $9$ appear consecutively in the factors of <missing 8>.

We can start by verifying that <missing 9> and <missing 10> work by listing out the factors of <missing 11> and $4!$. We can also see that <missing 12> does not work because the terms $15, 20$, and $24$ are consecutive factors of $5!$. Also, <missing 13> does not work because the terms <missing 14>, and $9$ appear consecutively in the factors of <missing 15>.

Note that if we have a prime number <missing 16> and an integer <missing 17> such that both $k$ and <missing 18> are factors of <missing 19>, then the condition cannot be satisfied.

If <missing 20> is odd, then <missing 21> is a factor of <missing 22>. Also, <missing 23> is a factor of $n!$. Since <missing 24> for all <missing 25>, we can use Bertrand's Postulate to show that there is at least one prime number $p$ such that <missing 26>. Since we have two consecutive factors of <missing 27> and a prime number between the smaller of these factors and $n$, the condition will not be satisfied for all odd <missing 28>.

If <missing 29> is even, then $(2)(\frac{n-2}{2})(n-2)=n^2-4n+4$ is a factor of <missing 30>. Also, $(n-3)(n-1)=n^2-4n+3$ is a factor of <missing 31>. Since <missing 32> for all $n\geq8$, we can use Bertrand's Postulate again to show that there is at least one prime number $p$ such that <missing 33>. Since we have two consecutive factors of <missing 34> and a prime number between the smaller of these factors and $n$, the condition will not be satisfied for all even <missing 35>.

Therefore, the only numbers that work are $n=3$ and <missing 36>.

The formulae are:
\begin{itemize}
    \item <expression 1> $n=6$ 
    \item <expression 2> $n=5$ 
    \item <expression 3> $3!$ 
    \item <expression 4> $k+1$ 
    
    \textcolor{blue}{... (omitted for brevity)}
    
    \item <expression 33> $n<p<n^2-4n+3$ 
    \item <expression 34> $p>n$ 
    \item <expression 35> $n<p<n^2-2n$ 
    \item <expression 36> $n=4$ 
\end{itemize}

Your final answer should be STRICTLY in the format:

\textbf{Detailed reasoning}

Answer: <comma separated list of numbers representing expression identifiers>

\end{tcolorbox}
