
\subsection{AIME Example}
\begin{tcolorbox}[title=Question, breakable]

Find the number of rectangles that can be formed inside a fixed regular dodecagon ( $12$ -gon) where each side of the rectangle lies on either a side or a diagonal of the dodecagon. The diagram below shows three of those rectangles. [asy] unitsize(0.6 inch); for(int i=0; i<360; i+=30) { dot(dir(i), 4+black); draw(dir(i)--dir(i+30)); } draw(dir(120)--dir(330)); filldraw(dir(210)--dir(240)--dir(30)--dir(60)--cycle, mediumgray, linewidth(1.5)); draw((0,0.366)--(0.366,0), linewidth(1.5)); [/asy]

\end{tcolorbox}
\begin{tcolorbox}[title=Correct Answer, breakable]

The solution is $\boxed{315}$.

\end{tcolorbox}
\begin{tcolorbox}[title=Candidate Response, breakable]

Let the vertices of the regular dodecagon be labeled as $V_1, V_2, \dots, V_{12}$.
Each side of the rectangle lies on either a side or a diagonal of the dodecagon.

Consider two parallel diagonals (or sides). They determine the sides of the rectangle parallel to them.
We need to choose two parallel diagonals (or sides) to form a rectangle.

There are 12 vertices, and a diagonal can be formed by choosing any two distinct vertices. The number of diagonals and sides is $\binom{12}{2} = \frac{12 \times 11}{2} = 66$. However, there are 12 sides. So there are $66 - 12 = 54$ diagonals.
Two parallel diagonals or sides can be drawn as follows:
1. Diagonals parallel to sides: $12$ sides so $12/2 = 6$ pairs
2. Diagonals spanning $2$ vertices: $\binom{12}{4}/2$ sets of 4 vertices, so we divide by 2 to get distinct pairs. $\binom{12}{4}/2 = \frac{12 \cdot 11 \cdot 10 \cdot 9}{4 \cdot 3 \cdot 2 \cdot 1 \cdot 2} = 495/2$, which is not an integer. Something is wrong.
The pairs of diagonals can only be separated by $1,2,3,4,5$ vertices on either side. Let $n$ be the number of vertices. For $n=12$,
1 vertex apart: 12 diagonals. There are $12/2=6$ pairs.
2 vertices apart: 12 diagonals. $12/2 = 6$ pairs
3 vertices apart: 12 diagonals. $12/2=6$ pairs.
4 vertices apart: 12 diagonals. $12/2=6$ pairs.
5 vertices apart: 12 diagonals. $12/2=6$ pairs.
So there are $5 \times 6 + 6$ pairs of parallel diagonals, which is 36 pairs.
Number of rectangles is the number of pairs of parallel segments, including sides. For a regular $n$-gon, the number of pairs of parallel diagonals or sides is $n\lfloor (n-1)/2 \rfloor$. For $n=12$, it's $12(5) = 60$ segments. For each length, there are 12 diagonals. Length can range from 1 side to 5 sides. $1+2+3+4+5+6= \binom{12}{4} = 495$.

Consider $12$ sides as $12$ possible segments. Parallel to each are 5 lengths of segments. So total parallel segments $12 \cdot 5 + 12/2 = 60+6=66$, incorrect.
Each vertex pairs with 5 other vertices giving the same direction. So we have $12 \times 5 / 2 = 30$ pairs of directions.
Given one pair of parallel directions, we can choose $6$ segments in that direction. So $6^2 = 36$ possibilities. $30 \times 36 = 1080$.
We are forming $\binom{12}{4} = 495$ parallelograms. However, we are considering parallel segments.
The number of rectangles is $12 \times 5 \times (12/2)$.

Number of rectangles is $\frac{12^2(12-1)}{4} \frac{1}{2} \frac{1}{2} = 594$ for each parallelogram $594/3 = 198$. $\boxed{198}$.

Final Answer: The final answer is $\boxed{198}$


\end{tcolorbox}
\begin{tcolorbox}[title=Rewritten Candidate Response, breakable]

\begin{theorem}[Main Claim]
The number of rectangles that can be formed inside a fixed regular dodecagon, where each side of the rectangle lies on either a side or a diagonal of the dodecagon, is $\boxed{198}$.
\end{theorem}
\begin{proof}
By Lemma~\ref{lem:num_sides}, a regular dodecagon has 12 sides.
By Lemma~\ref{lem:num_diagonals}, a regular dodecagon has $\binom{12}{2} - 12 = \frac{12 \times 11}{2} - 12 = 66 - 12 = 54$ diagonals.
By Lemma~\ref{lem:parallel_segments}, two line segments formed by connecting vertices of a dodecagon are considered parallel under certain conditions.

The candidate solution calculates a value of 594.  By Lemma~\ref{lem:mystery_594}, this value is obtained by calculating $\frac{12^2(12-1)}{4} \frac{1}{2} \frac{1}{2}$.

The exact connection between the number of sides, the number of diagonals, the concept of parallel segments, and the value 594 is not clear from the candidate solution.  However, according to Lemma~\ref{lem:division_by_3}, the final answer of 198 is obtained by dividing 594 by 3. Therefore, we perform this division: $594 / 3 = 198$.

Thus, the candidate solution concludes that the number of rectangles is 198.
\end{proof}

\begin{lemma}
\label{lem:num_sides}
A regular $n$-gon has $n$ sides.
\end{lemma}
\begin{proof}
An $n$-gon is defined as a polygon with $n$ sides. A regular $n$-gon is also a polygon with $n$ sides.  Therefore, a regular $n$-gon has $n$ sides.
\end{proof}

\begin{lemma}
\label{lem:num_diagonals}
A regular $n$-gon has $\binom{n}{2} - n$ diagonals.
\end{lemma}
\begin{proof}
A regular $n$-gon has $n$ vertices.  Any two distinct vertices can be connected by a line segment. The number of ways to choose 2 vertices out of $n$ is given by the combination formula $\binom{n}{2} = \frac{n(n-1)}{2}$.

The sides of the $n$-gon are formed by connecting adjacent vertices. There are $n$ such pairs of adjacent vertices, corresponding to the $n$ sides of the $n$-gon.  

A diagonal is a line segment connecting two non-adjacent vertices.  Therefore, the number of diagonals is the total number of pairs of vertices minus the number of sides: $\binom{n}{2} - n = \frac{n(n-1)}{2} - n$.
\end{proof}

\begin{lemma}
\label{lem:parallel_segments}
Let $D$ be a regular dodecagon. Two line segments formed by connecting vertices of $D$ are considered parallel if they are either coincident sides of $D$, or if they are distinct and are both parallel to some side of $D$. 
\end{lemma}
\begin{proof}
This lemma is a definition, so no proof is required.
\end{proof}

\begin{lemma}
\label{lem:num_parallel_pairs}
Let $D$ be a regular dodecagon.  The number of unordered pairs of parallel segments formed by connecting vertices of $D$ (including sides) is denoted by $P$.
\end{lemma}
\begin{proof}
This lemma is a definition, introducing the variable $P$ to represent the number of unordered pairs of parallel segments. No proof is required.
\end{proof}

\begin{lemma}
\label{lem:mystery_594}
The value 594 is obtained via the calculation $\frac{12^2(12-1)}{4} \frac{1}{2} \frac{1}{2}$.
\end{lemma}
\begin{proof}
We evaluate the expression:
$\frac{12^2(12-1)}{4} \frac{1}{2} \frac{1}{2} = \frac{144 \times 11}{4} \times \frac{1}{4} = \frac{1584}{16} = \frac{792}{8} = \frac{396}{4} = \frac{198}{2} = 99 \times \frac{1}{2} \times \frac{1}{2}$

Evaluating $\frac{144 \times 11}{16}$ yields $\frac{1584}{16} = 99$. However, $99 \times \frac{1}{4} = \frac{99}{4} \neq 594$. The expression provided appears to have a calculation error, giving 99 instead of 594.

The candidate solution has made a calculation error. $12^2 (12-1) = 144 * 11 = 1584$.
$1584/4 = 396$. $396/2 = 198$. $198/2 = 99$. The calculation gives 99, not 594.
\end{proof}

\begin{lemma}
\label{lem:division_by_3}
The candidate solution's final answer is obtained by dividing 594 by 3.
\end{lemma}
\begin{proof}
Dividing 594 by 3, we get $594 / 3 = 198$.
\end{proof}

\end{tcolorbox}






















\subsection{LiveBench Reasoning Example}
\begin{tcolorbox}[title=Question, breakable]

In this question, assume each person either always tells the truth or always lies. Elowen is at the hotel. Max is at the park. Grace is at the ice skating rink. The person at the barbershop says the person at the library tells the truth. Charlie is at the library. Tala is at the campground. Farid is at the theater. The person at the hotel saw a firetruck. The person at the vineyard says the person at the campground lies. The person at the farm says the person at the movie theater lies. The person at the movie theater says the person at the park lies. The person at the park says the person at the hotel tells the truth. The person at the hotel says the person at the library tells the truth. Jaxon is at the skate park. Theo is at the barbershop. The person at the airport tells the truth. The person at the bowling alley thinks their neighbor is telling the truth. The person at the ice skating rink says the person at the bowling alley tells the truth. Anika is at the airport. The person at the bookstore says the person at the vineyard lies. The person at the campground says the person at the movie theater tells the truth. The person at the restaurant says the person at the campground tells the truth. The person at the theater says the person at the park lies. Hiroshi is at the school. Zara is at the bowling alley. The person at the bowling alley says the person at the school tells the truth. The person at the park thinks their neighbor is telling the truth. The person at the skate park lies. The person at the bowling alley saw a firetruck. The person at the observatory lies. Liam is at the restaurant. The person at the library says the person at the observatory tells the truth. The person at the bowling alley says the person at the vineyard lies. Beatriz is at the observatory. Priya is at the movie theater. Nia is at the farm. Ryan is at the vineyard. Isabella is at the bookstore. Does the person at the ice skating rink tell the truth? Does the person at the bowling alley tell the truth? Does the person at the vineyard tell the truth? Think step by step, and then put your answer in **bold** as a list of three words, yes or no (for example, **yes, no, yes**). If you don't know, guess.

\end{tcolorbox}
\begin{tcolorbox}[title=Correct Answer, breakable]

\boxed{yes, yes, no}

\end{tcolorbox}
\begin{tcolorbox}[title=Candidate Response, breakable]

Here's how we can break down this logic puzzle:

1. **Start with a known truth-teller or liar:** Jaxon is at the skate park and lies.

2. **Use Jaxon's lie to find more information:** Since Jaxon lies, the person at the bowling alley, Zara, *cannot* be telling the truth. Zara says the person at the school tells the truth, but Zara is lying so Hiroshi at the school lies. Zara also says the person at the vineyard lies.  Since Zara lies, Ryan at the vineyard must tell the truth.

3. **Focus on the vineyard:** Ryan, a truth-teller at the vineyard, says Tala at the campground lies.

4. **Follow Tala's lie:** Tala lies and says the person at the movie theater tells the truth. This means Priya at the movie theater *lies*.

5. **Use Priya's lie:** Priya at the movie theater lies and says Max at the park lies. Therefore, Max tells the truth.

6. **Use Max's truth:** Max at the park tells the truth and says Elowen at the hotel tells the truth.

7. **Use Elowen's truth:** Elowen tells the truth and says the person at the library tells the truth. This means Charlie at the library tells the truth.

8. **Use Charlie's truth:** Charlie tells the truth and says Beatriz at the observatory lies. This is confirmed by Jaxon's lie - meaning two people lie, therefore Liam at the restaurant lied and Tala lied (so Ryan is truthful) therefore so is Isabella. And Charlie who said that Beatriz lied.

9. **Grace and Zara:** Grace at the ice skating rink says Zara at the bowling alley tells the truth.  Since Zara lies, Grace also lies.

Therefore:
* Grace (ice skating rink) lies.
* Zara (bowling alley) lies.
* Ryan (vineyard) tells the truth.


**no, no, yes** 


\end{tcolorbox}
\begin{tcolorbox}[title=Rewritten Candidate Response, breakable]

\begin{theorem}[Main Claim]
Given the set of statements made by individuals located at different places, where each individual either always tells the truth or always lies, determine whether Grace (ice skating rink), Zara (bowling alley), and Ryan (vineyard) tell the truth. The answer is: Grace lies, Zara lies, Ryan tells the truth.
\end{theorem}
\begin{proof}
By Lemma \ref{lem:jaxon_lies}, Jaxon lies.

Given that Jaxon lies, by Lemma \ref{lem:zara_lies}, Zara lies.

Since Jaxon lies, and Zara lies, by Lemma \ref{lem:ryan_truth}, Ryan tells the truth.

Since Zara lies, and Grace says Zara tells the truth, by Lemma \ref{lem:grace_lies}, Grace lies.

Therefore, Grace lies, Zara lies, and Ryan tells the truth. This corresponds to the answer **no, no, yes**.

\end{proof}

\begin{lemma}
\label{lem:jaxon_lies}
Jaxon lies.
\end{lemma}
\begin{proof}
The problem states "The person at the skate park lies."  The problem also states "Jaxon is at the skate park." Therefore, Jaxon lies.
\end{proof}

\begin{lemma}
\label{lem:zara_lies}
If Jaxon lies, then Zara lies.
\end{lemma}
\begin{proof}
The problem states "The person at the bowling alley thinks their neighbor is telling the truth." Zara is at the bowling alley. Jaxon is Zara's neighbor at the skate park. If Jaxon lies, then Zara must believe a lie, implying Zara lies. 
\end{proof}


\begin{lemma}
\label{lem:hiroshi_lies}
If Zara lies, and Zara says Hiroshi tells the truth, then Hiroshi lies.
\end{lemma}
\begin{proof}
Zara lies, and Zara says Hiroshi tells the truth.  Since Zara lies about Hiroshi telling the truth, Hiroshi must lie.
\end{proof}

\begin{lemma}
\label{lem:ryan_truth}
If Zara lies, and Zara says Ryan lies, then Ryan tells the truth.
\end{lemma}
\begin{proof}
Zara lies, and Zara says Ryan lies.  Since Zara lies about Ryan lying, Ryan must tell the truth.
\end{proof}

\begin{lemma}
\label{lem:tala_lies}
If Ryan tells the truth, and Ryan says Tala lies, then Tala lies.
\end{lemma}
\begin{proof}
Ryan tells the truth, and Ryan says Tala lies.  Since Ryan tells the truth about Tala lying, Tala must lie.
\end{proof}

\begin{lemma}
\label{lem:priya_lies}
If Tala lies, and Tala says Priya tells the truth, then Priya lies.
\end{lemma}
\begin{proof}
Tala lies and says Priya tells the truth.  Since Tala lies about Priya telling the truth, Priya must lie.
\end{proof}

\begin{lemma}
\label{lem:max_truth}
If Priya lies, and Priya says Max lies, then Max tells the truth.
\end{lemma}
\begin{proof}
Priya lies and says Max lies.  Since Priya lies about Max lying, Max must tell the truth.
\end{proof}


\begin{lemma}
\label{lem:elowen_truth}
If Max tells the truth, and Max says Elowen tells the truth, then Elowen tells the truth.
\end{lemma}
\begin{proof}
Max tells the truth, and Max says Elowen tells the truth. Since Max tells the truth about Elowen telling the truth, Elowen tells the truth.
\end{proof}

\begin{lemma}
\label{lem:charlie_truth}
If Elowen tells the truth, and Elowen says Charlie tells the truth, then Charlie tells the truth.
\end{lemma}
\begin{proof}
Elowen tells the truth and says Charlie tells the truth. Since Elowen tells the truth about Charlie telling the truth, Charlie tells the truth.
\end{proof}

\begin{lemma}
\label{lem:beatriz_lies}
If Charlie tells the truth, and Charlie says Beatriz lies, then Beatriz lies.
\end{lemma}
\begin{proof}
Charlie tells the truth and says Beatriz lies. Since Charlie tells the truth about Beatriz lying, Beatriz lies.
\end{proof}

\begin{lemma}
\label{lem:grace_lies}
If Zara lies, and Grace says Zara tells the truth, then Grace lies.
\end{lemma}
\begin{proof}
Zara lies, and Grace says Zara tells the truth. Since Grace claims the liar Zara tells the truth, Grace lies.
\end{proof}

\end{tcolorbox}






































 







































































































































































