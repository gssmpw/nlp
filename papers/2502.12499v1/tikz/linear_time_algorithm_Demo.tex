\begin{figure}
    \centering
    \begin{subfigure}[t]{.45\textwidth}
        \begin{tikzpicture}[scale=0.5]
    \draw[->, thick] (-0.5, 0) -- (12, 0);
    \draw[->, thick] (0, -2.5) -- (0, 10);

    \draw[thick] (1, -0.2) -- (1, 0.2);
    \draw[thick] (2, -0.2) -- (2, 0.2);
    \draw[thick] (3, -0.2) -- (3, 0.2);
    \draw[thick] (4, -0.2) -- (4, 0.2);
    \draw[thick] (5, -0.2) -- (5, 0.2);
    \draw[thick] (6, -0.2) -- (6, 0.2);
    \draw[thick] (7, -0.2) -- (7, 0.2);
    \draw[thick] (8, -0.2) -- (8, 0.2);
    \draw[thick] (9, -0.2) -- (9, 0.2);
    \draw[thick] (10, -0.2) -- (10, 0.2);
    \draw[thick] (11, -0.2) -- (11, 0.2);

    % Dots for M values
    \fill[black] (10, 1) circle (3pt);
    \fill[black] (9, 2) circle (3pt);
    \fill[black] (8, 3.5) circle (3pt);
    \fill[black] (7, 4.5) circle (3pt);
    \fill[black] (5, 6.5) circle (3pt);
    \fill[black] (4, 7) circle (3pt);
    \fill[black] (3, 8) circle (3pt);

    % Stars for U values
    \node[scale=0.75] at (3, 0.5) {\textbf{*}};
    \node[scale=0.75] at (4, 1.5) {\textbf{*}};
    \node[scale=0.75] at (5, 2) {\textbf{*}};
    \node[scale=0.75] at (7, 3) {\textbf{*}};
    \node[scale=0.75] at (8, 4) {\textbf{*}};
    \node[scale=0.75] at (9, 5) {\textbf{*}};
    \node[scale=0.75] at (10, 5.5) {\textbf{*}};

    \node[scale=0.75] at (7, -1) {$j$};
    \node[scale=0.75] at (8, -2) {$j^*(2)$};
\end{tikzpicture}
\comment{The values of $M(\cdot)$ and $U(i, \cdot)$ at the $i$-th iteration.
The $x$-coordinate represents the index of $M(\cdot)$ and $U(i, \cdot)$, while the $y$-coordinate represents their corresponding values.
Black dots indicate the values of $M(\cdot)$ present in the queue $Q_i$, and stars represent the values of $U(i, \cdot)$.
The $j$ below the $x$-axis marks the value of $j$ after the ``while'' loop in Algorithm~?.
The $j^*$ below the $x$-axis marks the value of updated $j^*$ at the $i$-th iteration.
}
        \caption{The values of $M(\cdot)$ and $U(2, \cdot)$ when we compute $M(2)$.}
        \label{fig:compute_M(2)}
    \end{subfigure}
    \begin{subfigure}[t]{.45\textwidth}
        \begin{tikzpicture}[scale=0.5]
    \draw[->, thick] (-0.5, 0) -- (12, 0);
    \draw[->, thick] (0, -2.5) -- (0, 10);

    \draw[thick] (1, -0.2) -- (1, 0.2);
    \draw[thick] (2, -0.2) -- (2, 0.2);
    \draw[thick] (3, -0.2) -- (3, 0.2);
    \draw[thick] (4, -0.2) -- (4, 0.2);
    \draw[thick] (5, -0.2) -- (5, 0.2);
    \draw[thick] (6, -0.2) -- (6, 0.2);
    \draw[thick] (7, -0.2) -- (7, 0.2);
    \draw[thick] (8, -0.2) -- (8, 0.2);
    \draw[thick] (9, -0.2) -- (9, 0.2);
    \draw[thick] (10, -0.2) -- (10, 0.2);
    \draw[thick] (11, -0.2) -- (11, 0.2);

    % Stars for previous U values
    \node[scale=0.75, color=lightgray] at (3, 0.5) {\textbf{*}};
    \node[scale=0.75, color=lightgray] at (4, 1.5) {\textbf{*}};
    \node[scale=0.75, color=lightgray] at (5, 2) {\textbf{*}};
    \node[scale=0.75, color=lightgray] at (7, 3) {\textbf{*}};
    \node[scale=0.75, color=lightgray] at (8, 4) {\textbf{*}};
    \node[scale=0.75, color=lightgray] at (9, 5) {\textbf{*}};
    \node[scale=0.75, color=lightgray] at (10, 5.5) {\textbf{*}};

    % Dots for M values
    \fill[black] (10, 1) circle (3pt);
    \fill[black] (9, 2) circle (3pt);
    \fill[black] (8, 3.5) circle (3pt);
    \fill[black] (7, 4.5) circle (3pt);
    \fill[black] (5, 6.5) circle (3pt);
    \fill[black] (4, 7) circle (3pt);
    \fill[black] (3, 8) circle (3pt);
    \fill[white] (3, 8) circle (2pt);
    \fill[black] (2, 7.5) circle (3pt);

    % Stars for U values
    \node[scale=0.75] at (2, 3) {\textbf{*}};
    \node[scale=0.75] at (4, 6) {\textbf{*}};
    \node[scale=0.75] at (5, 7.5) {\textbf{*}};
    \node[scale=0.75] at (7, 8) {\textbf{*}};
    \node[scale=0.75] at (8, 8.5) {\textbf{*}};
    \node[scale=0.75] at (9, 9) {\textbf{*}};
    \node[scale=0.75] at (10, 9.25) {\textbf{*}};

    \draw[-{>[scale=1]}, dotted] (4, 1.5) -- (4, 6-0.2);
    \draw[-{>[scale=1]}, dotted] (5, 2) -- (5, 7.5-0.2);
    \draw[-{>[scale=1]}, dotted] (7, 3) -- (7, 8-0.2);
    \draw[-{>[scale=1]}, dotted] (8, 4) -- (8, 8.5-0.2);
    \draw[-{>[scale=1]}, dotted] (9, 5) -- (9, 9-0.2);
    \draw[-{>[scale=1]}, dotted] (10, 5.5) -- (10, 9.25-0.2);

%    \node[scale=1] at (8, -1) {$j$};
%    \node[scale=1] at (4, -2) {$j$};
%    \node[scale=1] at (4, -3) {$j*$};
    \node[scale=0.75] at (4, -1) {$j$};
    \node[scale=0.75] at (4, -2) {$j^*(1)$};
\end{tikzpicture}
\comment{The distribution of $M(\cdot)$ and $U(i, \cdot)$ during the $(i - 1)$-th iteration.
The head of $Q$ is removed from the queue (the hollow black dot) since $M(i)$ (the black dot on the left of the hollow black dot) is smaller than it.
All the $U(i, \cdot)$ (the light gray stars) become the $U(i - 1, \cdot)$ (the black stars).
Note that we don't have to compute every values of $U(i,\cdot)$.
The first $j$ below the $x$-axis marks the value of $j$ before the ``while'' loop at the $(i - 1)$-th iteration.
In other words, it equals the $j^*$ at the $i$-th iteration.
The second $j$ below the $x$-axis marks the value of $j$ after the ``while'' loop at the $(i - 1)$-th iteration.
The $j^*$ below the $x$-axis marks the value of $j^*$ at the $(i - 1)$-th iteration.
} 
        \caption{The values of $M(\cdot)$ and $U(1, \cdot)$ when we compute $M(1)$.
        The gray stars are the values of $U(2,\cdot)$.
        The hollow dot represents the element at the head of $Q$ larger than $M(2)$ and is therefore removed from $Q$ before $M(2)$ is inserted.}
        \label{fig:compute_M(1)}
    \end{subfigure}
    \caption{The values of $M(\cdot)$ and $U(i, \cdot)$ when we compute $M(2)$ and $M(1)$.
    The $x$-coordinate represents the index of $M(\cdot)$ and $U(i, \cdot)$, while the $y$-coordinate represents their corresponding values.
    Black dots indicate the values of $M(\cdot)$ present in the queue $Q_i$, and stars represent the values of $U(i, \cdot)$.
    The $x$-coordinate $i$ without a black dot indicates that $M(i)$ has been removed from the queue $Q$.
    The $j$ below the $x$-axis marks the value of $j$ after the ``while'' loop in Algorithm 6.
    The $j^*(i)$ below the $x$-axis marks the value of updated $j^*_(i)$.
    }
\label{fig:computation_Demo}
\end{figure}