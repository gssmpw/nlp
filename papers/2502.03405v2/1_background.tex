In this section, we succinctly present the relevant elements related to the notion
of graph ratio-cut. The curious reader may refer to \citet{spectralclustering} for
a more detailed account of spectral clustering and other types of graph cuts.

Let $\gG=(\gV,\gE,\gK)$ be an undirected weighted graph where $\gV \defeq \lrcb{v_i
\mid 1 \leq i \leq n} \subset \mathbb{R}^p$ is the set of vertices, $\gE \subset
\gV \times \gV$ is the set of edges $e_{ij}$ linking vertices $v_i$ and $v_j$, and
$\gK: \gV \times \gV \to \R^+$ is a symmetric non-negative kernel. Let $\mW$ be the
symmetric $n \times n$ adjacency matrix where $\emW_{ij} = \gK(v_i, v_j)$. The
degree of the vertex $v_i$ is $d_i=\sum_j \emW_{ij}$, and the degree matrix
$\mD\defeq\diag{(d_1,\ldots,d_n)}$.

% such that $\cup_{i=1}^{k} \sC_i=\gV$
Let $k \geq 2$ and $\gC_k=\lrcb{\sC_\ell|1\leq\ell\leq k}$ a partitioning of the
graph $\gG$ into $k$ disjoint clusters. We shall represent the subset $\sC_\ell
\subset \gV$ using the binary assignment vector $\1{\sC_\ell} \in \lrcb{0,1}^n$
where $\1{\sC_\ell}(i)=1$ if and only if $v_i \in \sC_\ell$. The size of $\sC_\ell$
is measured using its cardinality $|\sC_\ell| = \sum_{i=1}^n \1{\sC_\ell}(i)$, and
we denote by $\vf^{(\ell)} \defeq \by{\sqrt{|\sC_\ell|}} \1{\sC_\ell}$ its
\textit{ratio assignment} when $|\sC_\ell| >0$.

The \textit{ratio-cut} for $\gC_k$ is defined as:
\begin{align} \label{eq:ratiocut}
	\rcut(\gC_k) & \defeq \by{2}\sum_{\ell=1}^{k} \by{|\sC_\ell|}
	\sum_{i,j\in \sC_\ell\times \comp{\sC_\ell}} \emW_{ij}        \\
	             & = \by{2}\sum_{\ell=1}^{k} \by{|\sC_\ell|}
	\1{\sC_\ell}^\top \mW \lrp{\bm{1}_n-\1{\sC_\ell}},
\end{align}
where $\comp{\sA}$ denotes $\gV\backslash \sA$, the complement of $\sA$ in $\gV$.

We define the unnormalized Laplacian matrix as $\mL_{un}\defeq \mD-\mW$, which can
be used to express the ratio-cut of $\gC_k$ as:
\begin{align}
	\label{eq:rawlap}
	\rcut(\gC_k)
	% =\by{2}\sum_{\ell=1}^{k} \vf^{(\ell)\top} L_{un}\vf^{(\ell)}
	=\by{2}\Tr\lrb{\mF_{\gC_k}^\top \mL_{un} \mF_{\gC_k}},
\end{align}
where $\mF_{\gC_k} \defeq \lrb{\vf^{(1)},\ldots \vf^{(k)}}\in \R^{n \times k}$ is the
ratio assignments matrix.

Since the clusters should be disjoint and different from the naive partitioning
$\lrcb{\gV,\emptyset,\ldots,\emptyset}$, we can express these constraints as
$\mF_{:,\ell}^\top\1{V}\neq 1$ for all $1\leq \ell\leq k$.

The optimization of the ratio-cut is then equivalent to minimizing a Rayleigh
quotient on $\{0,1\}^{n\times k}$. Solving~\cref{eq:rayleighquo} on the set
$\{0,1\}^n$ is generally an NP-hard problem. The optimization is hence relaxed so
that the unknown vectors are in the unit sphere $S^{(n)} = \lrcb{x \in \R^n,
\normtwo{x}^2 =1}$. The objective is then:

\begin{align}
	\label{eq:rayleighquo}
	\begin{aligned}
		 & \underset{\mF}{\text{minimize}} &  & \Tr(\mF^\top \mL \mF )                                      \\
		 & \text{subject to}               &  & \mF^\top \mF = \mI_k \text{ and } \mF_{:,j}^\top\1{V}\neq 1
	\end{aligned}
\end{align}

The minimization of the Rayleigh quotient under the outlined constraints yields the
$k$ smoothest eigenvectors of the Laplacian matrix (excluding the trivial first
eigenvector $\1{V}$). Vertices within the same cluster are anticipated to have
similar projections onto the solutions of the relaxed problem. As we ascend the
Laplacian spectrum, the projections onto the eigenvectors encapsulate more specific
(higher frequency) features. Subsequently, the binary assignments are determined
through $k$-means clustering~\citep{standardspectral} of the relaxed problem's
solution.

In this paper, we introduce a novel approach to optimizing~\cref{eq:rayleighquo}
that circumvents the necessity for spectral decomposition of extensive matrices or
kernels, thereby sidestepping the relaxation of the problem into Euclidean space.
Rather, we propose to relax the problem into an optimization over a simplex through
the parameterization of the cluster assignment distribution.
