%%
%% This is file `sample-sigconf-authordraft.tex',
%% generated with the docstrip utility.
%%
%% The original source files were:
%%
%% samples.dtx  (with options: `all,proceedings,bibtex,authordraft')
%% 
%% IMPORTANT NOTICE:
%% 
%% For the copyright see the source file.
%% 
%% Any modified versions of this file must be renamed
%% with new filenames distinct from sample-sigconf-authordraft.tex.
%% 
%% For distribution of the original source see the terms
%% for copying and modification in the file samples.dtx.
%% 
%% This generated file may be distributed as long as the
%% original source files, as listed above, are part of the
%% same distribution. (The sources need not necessarily be
%% in the same archive or directory.)
%%
%%
%% Commands for TeXCount
%TC:macro \cite [option:text,text]
%TC:macro \citep [option:text,text]
%TC:macro \citet [option:text,text]
%TC:envir table 0 1
%TC:envir table* 0 1
%TC:envir tabular [ignore] word
%TC:envir displaymath 0 word
%TC:envir math 0 word
%TC:envir comment 0 0
%%
%%
%% The first command in your LaTeX source must be the \documentclass
%% command.
%%
%% For submission and review of your manuscript please change the
%% command to \documentclass[manuscript, screen, review]{acmart}.
%%
%% When submitting camera ready or to TAPS, please change the command
%% to \documentclass[sigconf]{acmart} or whichever template is required
%% for your publication.
%%
%%
% Define the flag
\newif\ifpreprintversion
\preprintversiontrue % Set to true
% \preprintversionfalse % Set to false

% USE ANONYMOUS FOR SUBMISSION
\ifpreprintversion
    \documentclass[sigconf]{acmart}
\else
    \documentclass[sigconf,anonymous]{acmart}
\fi

%%
%% \BibTeX command to typeset BibTeX logo in the docs
\AtBeginDocument{%
  \providecommand\BibTeX{{%
    Bib\TeX}}}

%% Rights management information.  This information is sent to you
%% when you complete the rights form.  These commands have SAMPLE
%% values in them; it is your responsibility as an author to replace
%% the commands and values with those provided to you when you
%% complete the rights form.
\setcopyright{acmlicensed}
\copyrightyear{2025}
\acmYear{2025}
\acmDOI{XXXXXXX.XXXXXXX}

%% These commands are for a PROCEEDINGS abstract or paper.
\acmConference[Conference acronym 'XX]{Make sure to enter the correct
  conference title from your rights confirmation emai}{June 03--05,
  2018}{Woodstock, NY}
%%
%%  Uncomment \acmBooktitle if the title of the proceedings is different
%%  from ``Proceedings of ...''!
%%
%%\acmBooktitle{Woodstock '18: ACM Symposium on Neural Gaze Detection,
%%  June 03--05, 2018, Woodstock, NY}
\acmISBN{978-1-4503-XXXX-X/18/06}


%%
%% Submission ID.
%% Use this when submitting an article to a sponsored event. You'll
%% receive a unique submission ID from the organizers
%% of the event, and this ID should be used as the parameter to this command.
%%\acmSubmissionID{123-A56-BU3}

%%
%% For managing citations, it is recommended to use bibliography
%% files in BibTeX format.
%%
%% You can then either use BibTeX with the ACM-Reference-Format style,
%% or BibLaTeX with the acmnumeric or acmauthoryear sytles, that include
%% support for advanced citation of software artefact from the
%% biblatex-software package, also separately available on CTAN.
%%
%% Look at the sample-*-biblatex.tex files for templates showcasing
%% the biblatex styles.
%%

%%
%% The majority of ACM publications use numbered citations and
%% references.  The command \citestyle{authoryear} switches to the
%% "author year" style.
%%
%% If you are preparing content for an event
%% sponsored by ACM SIGGRAPH, you must use the "author year" style of
%% citations and references.
%% Uncommenting
%% the next command will enable that style.
%%\citestyle{acmauthoryear}

% Custom ToDO, etc.
\newcommand{\CG}{\mathcal{G}\xspace}
\newcommand{\CV}{\mathcal{V}\xspace}
\newcommand{\CE}{\mathcal{E}\xspace}
\newcommand{\CA}{\mathcal{A}\xspace}
\newcommand{\CF}{\mathcal{F}\xspace}
\newcommand{\CR}{\mathcal{R}\xspace}
\newcommand{\CB}{\mathcal{B}\xspace}
\newcommand{\CX}{\mathcal{X}\xspace}
\newcommand{\CK}{\mathcal{K}\xspace}
\newcommand{\CM}{\mathcal{M}\xspace}
\newcommand{\CC}{\mathcal{C}\xspace}
\newcommand{\CL}{\mathcal{L}\xspace}
\newcommand{\CI}{\mathcal{I}\xspace}
\newcommand{\CQ}{\mathcal{Q}\xspace}
\newcommand{\CO}{\mathcal{O}\xspace}
\newcommand{\CP}{\mathcal{P}\xspace}
\newcommand{\CS}{\mathcal{S}\xspace}
\newcommand{\CT}{\mathcal{T}\xspace}
\newcommand{\CJ}{\mathcal{J}\xspace}
\usepackage[para]{footmisc}
\usepackage{subfig}
% \usepackage{subcaption}
% \usepackage{array}
% \usepackage{colortbl}



\begin{document}


% Our macros
\newcommand{\thought}[1]{{\color[rgb]{0.2,0.39,0.66}(#1)}}
\newcommand{\todo}[1]{{\color[rgb]{1.0,0.0,0.0}(#1)}}
\newcommand{\hsh}[1]{{\color{green!50!black} Henrik: #1}}
\newcommand{\st}[1]{{\color{red!50!black} Sebastian: #1}}

\newcommand{\ulm}[1]{_{\scaleto{\mathrm{#1}}{3pt}}}
\newcommand\at[2]{\left.#1\right|_{#2}}











\newtheorem{assumption}{Assumption}

\DeclareMathOperator*{\argmax}{arg\,max}
\DeclareMathOperator*{\argmin}{arg\,min}

\newcommand{\swname}[1]{\texttt{#1}}
\newcommand{\ie}{i\/.\/e\/.,\/~}
\newcommand{\eg}{e\/.\/g\/.,\/~}
\newcommand{\cf}{cf\/.\/~}

\newcommand{\fig}{Fig\/.\/~}
\newcommand{\defn}{Def\/.\/~}
\newcommand{\sect}{Sec\/.\/~}
\newcommand{\tabl}{Tab\/.\/~}
\newcommand{\algo}{Algorithm~}
\newcommand{\theo}{Theorem~}

\newcommand{\bnnl}{3 hidden layers}
\newcommand{\bnnn}{50 neurons}
\newcommand{\bnna}{tanh activations}

\newcommand{\capt}[1]{\mdseries{\emph{#1}}}

\newcommand{\videolink}{at \url{https://youtu.be/_d7AqTRjz6g}}
\newcommand{\codelink}{\url{https://github.com/wheelbot/mini-wheelbot}}

\newcommand{\fakepar}[1]{\vspace{0mm}\noindent\textbf{#1.}}

\newcommand{\needref}{\textcolor{red}{[REF]}}

\newcommand{\plotfontsize}{9pt}

%%
%% The "title" command has an optional parameter,
%% allowing the author to define a "short title" to be used in page headers.
\title{DROP: Poison Dilution via Knowledge Distillation for Federated Learning}
%\title{'Now you see me, now you don't!' Targeted Backdoor attacks for Federated Learning; quirks and remedies.}
% DROP- Distillation-based Reduction Of Poisoning

%%
%% The "author" command and its associated commands are used to define
%% the authors and their affiliations.
%% Of note is the shared affiliation of the first two authors, and the
%% "authornote" and "authornotemark" commands
%% used to denote shared contribution to the research.

\author{Georgios Syros $^{*\dagger}$}
\affiliation{%
  \institution{Northeastern University}
  \country{}
}
\author{Anshuman Suri$^*$}
\affiliation{%
  \institution{Northeastern University}
  \country{}
}

\author{Farinaz Koushanfar}
\affiliation{%
  \institution{University of California, San Diego}
  \country{}
}
\author{Cristina Nita-Rotaru}
\affiliation{%
  \institution{Northeastern University}
  \country{}
}
\author{Alina Oprea}
\affiliation{%
  \institution{Northeastern University}
  \country{}
}


%%
%% By default, the full list of authors will be used in the page
%% headers. Often, this list is too long, and will overlap
%% other information printed in the page headers. This command allows
%% the author to define a more concise list
%% of authors' names for this purpose.
\renewcommand{\shortauthors}{Syros et al.}

%%
%% The abstract is a short summary of the work to be presented in the
%% article.
\begin{abstract}
  Federated Learning is vulnerable to adversarial manipulation, where malicious clients can inject poisoned updates to influence the global model's behavior. While existing defense mechanisms have made notable progress, they fail to protect against adversaries that aim to induce targeted backdoors under different learning and attack configurations. 
  To address this limitation, we introduce \textit{DROP} (\textbf{D}istillation-based \textbf{R}eduction \textbf{O}f \textbf{P}oisoning), a novel defense mechanism that combines clustering and activity-tracking techniques with extraction of benign behavior from clients via knowledge-distillation to tackle stealthy adversaries that manipulate low data poisoning rates and diverse malicious client ratios within the federation. Through extensive experimentation, our approach demonstrates superior robustness compared to existing defenses across a wide range of learning configurations. Finally, we evaluate existing defenses and our method  under the challenging setting of non-IID client data distribution and highlight the challenges of designing a resilient FL defense in this setting.
  
  %when trying to defend systems against stealthy targeted backdoors when operating under non-i.i.d data distributions.
\end{abstract}

%%
%% The code below is generated by the tool at http://dl.acm.org/ccs.cfm.
%% Please copy and paste the code instead of the example below.
%%
\ifpreprintversion
\else
\begin{CCSXML}
<ccs2012>
   <concept>
       <concept_id>10002978.10003006</concept_id>
       <concept_desc>Security and privacy~Systems security</concept_desc>
       <concept_significance>500</concept_significance>
       </concept>
   <concept>
       <concept_id>10010147.10010257</concept_id>
       <concept_desc>Computing methodologies~Machine learning</concept_desc>
       <concept_significance>500</concept_significance>
       </concept>
 </ccs2012>
\end{CCSXML}
    \ccsdesc[500]{Security and privacy~Systems security}
    \ccsdesc[500]{Computing methodologies~Machine learning}
\fi

%%
%% Keywords. The author(s) should pick words that accurately describe
%% the work being presented. Separate the keywords with commas.
\keywords{Federated Learning Security, Targeted Backdoor Attacks, Poisoning Defenses, Knowledge Distillation}

% \received{20 February 2025}
% \received[revised]{12 March 2009}
% \received[accepted]{5 June 2009}

\ifpreprintversion
    \setcopyright{none}
    \settopmatter{printacmref=false} % Removes citation information below abstract
    \renewcommand\footnotetextcopyrightpermission[1]{} % removes footnote with conference information in first column
    \pagestyle{plain}
\fi

\maketitle

\ifpreprintversion
    \renewcommand{\thefootnote}{}
    \footnotetext{$^*$ Equal Contribution}
    \footnotetext{$^\dagger$ Correspondence to syros.g@northeastern.edu}
\fi

\begin{figure}[ht]
    \centering
    \includegraphics[width=0.8\linewidth]{graphs/greater_than_naive.pdf}
    \vspace{0.5cm}
    \includegraphics[width=0.8\linewidth]{graphs/p1_bottom.png}
    \vspace{-5pt}
    \caption{\textcolor{positional}{Positional} vs.\ \textcolor{nonpositional}{non-positional} circuits. In a \textcolor{nonpositional}{non-positional} circuit, the same edges must be included at all positions. A \textcolor{positional}{positional} circuit can distinguish between the same edge at different positions. This specificity yields better trade-offs between circuit size and faithfulness. It can also increase both precision and recall.}
    \label{fig:p1}
    \vspace{-5pt}
\end{figure}

\section{Introduction}

\looseness=-1
A primary goal of interpretability research is to characterize the internal mechanisms in language models (LMs) and other NLP models. 
A core approach in this area is \textbf{circuit discovery}---identifying the minimal subgraph within the model's computation graph that performs a specific task \citep{olah2021framework,olah-mech}.
Typically, the nodes of a circuit represent model components (e.g., attention heads, neurons, or layers).
While manual circuit discovery methods can yield position-specific insights \citep{wanginterpretability,goldowskydill2023localizingmodelbehaviorpath}, \emph{automatic methods often overlook positional information}, treating components as uniformly relevant across all input token positions \citep{conmytowards,syed2023attribution}. 
For instance, if an attention head is included in a circuit, it is assumed to contribute equally to the computation for every position in the input sequence.
The assumption that circuits are position-invariant ignores the fact that different positions often require distinct computations.
By ignoring positions, current methods limit their ability to capture mechanisms that operate across positions, such as interactions between attention heads across positions.

In this study, we start by demonstrating that positional agnosticism is a significant limitation (\S\ref{sec:motivating}). Then, to address these limitations, we introduce a new approach: position-aware edge attribution patching (PEAP; \S\ref{sec:full_circ_discovery}; Figure~\ref{fig:p1}). Current approaches  assume that if an edge is in a circuit, then the same edge will be in the circuit at all positions, thus leading to low precision. It is also assumed that an edge's importance should be aggregated across positions before deciding whether it should be included in the circuit; this can lead to cancellation effects, and thus low recall. PEAP instead allows us to compute the importance of cross-positional edges, and separately evaluates edge importance at each position. We show that this leads to smaller and more accurate circuits; see Figure~\ref{fig:p1}.

Incorporating positional information into circuit discovery is straightforward when inputs have the same length and structure across examples.

However, realistic datasets are not nearly this templatic.
How, then, can we incorporate positional information into automatic circuit discovery?
To address this challenge, we propose \textbf{schemas} (\S\ref{sec:schema}). 
Schemas assign semantic labels to spans of tokens, enabling information aggregation across examples even when the spans differ in length.

For example, in the input ``The \textcolor{positional}{war} lasted from 1453 to 14\underline{\hspace{1em}},'' the span ``\textcolor{positional}{war}'' could be labeled as ``\emph{Subject}''.
This enables handling spans with varying lengths: the phrase ``\textcolor{positional}{Black Plague}'' in another example can be treated as a single positional span with the same role as ``\textcolor{positional}{war}''.
In experiments with two LMs and three tasks, we find that circuits discovered using schemas achieve a better trade-off between circuit size and faithfulness to the model's behavior than position-agnostic circuits.
Importantly, position-aware circuits offer a more precise representation of the underlying mechanisms, providing a more concise foundation for mechanistic explanations.

We also present a fully automated pipeline for schema generation and application (\S\ref{sec:schema-generation}) using large language models (LLMs). 
We evaluate the quality of the generated schemas and their utility in discovering position-aware circuits (\S\ref{sec:schema-eval}).
Notably, circuits derived using automatically generated and applied schemas achieve comparable faithfulness scores to circuits discovered with human-designed and manually applied schemas.

We summarize our contributions as follows:
\begin{itemize}[noitemsep,leftmargin=*,topsep=1pt,parsep=1pt]
    \item Introduce a position-aware circuit discovery method, which obtains better faithfulness than position-agnostic discovery.  
    \item Introduce dataset schemas,  facilitating positional circuit discovery in more naturalistic settings. 
    \item Develop an automated schema generation and application pipeline with LLMs, yielding schemas that are comparable to manually-annotated ones.
\end{itemize}

\section{Basic Background: Supervised Learning and the PAC Model}
\label{sec:background}

At this point almost everyone has heard of machine learning (ML). Anyone likely to stumble upon this article will have also heard of its most influential special case, supervised learning, and those theoretically inclined will also be familiar with the PAC model. Nonetheless, I will set the stage by  recapping the basics.

\subsection{Basics of Supervised Learning}%Let's set the stage in any case

\emph{Supervised Learning} is the task of ``coming up'' with a function $f: \X \to \Y$ to ``explain'' or ``fit'' a sequence of input/output examples   $(x_1,y_1), \ldots, (x_n,y_n)$, with $x_i \in \X$ and $y_i \in \Y$.  Here $\X$ is a \emph{data domain} consisting of \emph{datapoints} $x \in \X$, $\Y$ is a \emph{label set} consisting of \emph{labels} $y \in \Y$, and the sequence $(x_1,y_1),\ldots,(x_n,y_n)$ is the \emph{training data} consisting of \emph{labeled examples (a.k.a. samples)}~$(x_i,y_i)$.  I~will refer to the chosen function $f$ as a \emph{predictor}, and to $n$ as the \emph{sample size}. A \emph{learning algorithm} takes as input training data, and outputs (some representation of) a predictor $f \in \Y^\X$.\footnote{Note that this describes the usual \emph{batch}, a.k.a.~\emph{offline}, setting of supervised learning. I do not discuss other paradigms such as online or active learning in this article.} 



Success in supervised learning is defined as \emph{generalization} to  future examples: For a typical \emph{test example}  $(x_{\tst},y_{\tst})$, the predicted label $y'_{\tst}=f(x_{\tst})$ should ``equal'' $y_{\tst}$, perhaps approximately. We usually assume the test example is drawn from the same  ``source'' as the training data  --- commonly, i.i.d.~from the same distribution. The quality of the prediction is quantified by $\ell(y'_{\tst},y_{\tst})$, where $\ell:~\Y~\times~\Y \to \RR_{\geq 0}$ is a \emph{loss function} chosen as part of the problem definition. Common loss functions include the 0-1 loss $\ell_{0-1}(y',y) = [y' \neq y]$ for \emph{classification} problems,\footnote{The notation $[P]$ denotes $1$ when predicate $P$ is true, and denotes $0$ when $P$ is false.} as well as the absolute loss $|y'-y|$ or squared loss $(y'-y)^2$ for \emph{regression problems} featuring $\Y  \sse \RR$.

Nontrivial generalization properties are typically only possible if one assumes something about the data.\footnote{The need for such an assumption is formalized by the  \emph{no free lunch theorems} of supervised learning \cite{wolpert_connection_1992,wolpert_lack_1996,schaffer_conservation_1994}.} The Bayesian approach to  machine learning, common in many applications, assumes some parametric form for the distribution generating the data, and postulates a prior on the parameters. This is not the approach I will take in this article. Instead, I will focus on the frequentist --- and some would say ``worst-case'' or ``adversarial'' ---  approach that is common in the computational learning theory community, embodied by the PAC model. Here we assume that the (training and test) data can be explained, perhaps approximately, by a function in some ``simple enough to learn'' class of functions $\H \sse \Y^\X$, often called the \emph{hypotheses}. Equivalently, we  seek a predictor which explains the unseen data roughly  as well as the best hypothesis $h^* \in \H$, whether or not we assume that $h^*$ itself provides a perfect explanation.



 \paragraph{Common Algorithmic Templates.} Perhaps the best known general-purpose supervised learning algorithm is \emph{empirical risk minimization (ERM)}, which chooses as its predictor a hypothesis $f \in \H$ minimizing $\frac{1}{n} \sum_{i=1}^n \ell(f(x_i),y_i)$ --- a quantity called the \emph{training error}, \emph{empirical error}, or \emph{empirical risk} of $f$. %\footnote{When multiple hypotheses minimize the empirical risk, we assume ERM breaks ties arbitrarily.}
A common template for generalizing ERM involves adding a \emph{regularization term} $\psi(f)$ to the  objective function, typically chosen to measure some notion of ``hypothesis complexity.'' An algorithm instantiating this template is known as a \emph{structural risk minimizer (SRM)}, and chooses as its predictor the hypothesis $f \in \H$ minimizing the \emph{structural risk} $\frac{1}{n} \sum_{i=1}^n \ell(f(x_i),y_i) + \psi(f)$. Other well-known algorithms, such as gradient descent and its variations,  can frequently be interpreted as approximate implementations of ERM or SRM.


\paragraph{Proper vs Improper Learning.} A learning algorithm is said to be \emph{proper} if its predictor $f$ is always chosen from the hypothesis class, i.e., $f \in \H$, otherwise it is said to be \emph{improper}. ERM  is an example of a proper learning algorithm, as are SRM algorithms of the form described above.  In the \emph{proper regime} of learning, algorithms are required to be proper. This article will be concerned with the more flexible \emph{improper regime} (a.k.a \emph{representation-independent learning}), where no such constraint is placed on the learner. In other words, all we care about is predictive power at test time, rather than any insights derived from the functional form or representation of the predictor~itself.


\subsection{The PAC Model}
A standard mathematical setup for evaluation of supervised learning algorithms, at least in the theoretical computer science community, is Valiant's \emph{Probably Approximately Correct (PAC) model} of learning (see e.g.~\cite{kearns_introduction_1994,mohri_foundations_2018}). Here, we assume there is an unknown distribution $\D$ on $\X \times \Y$ from which training and test data are  drawn.  Specifically, the labeled datapoints of the training set  $(x_1,y_1), \ldots, (x_n,y_n)$, as well as the test data  $(x_\tst,y_\tst)$, are i.i.d.~from $\D$. Often it is assumed that $\D$ lies in some class of distributions of interest. The \emph{true expected loss}, or simply \emph{loss}, of a predictor $f: \X \to \Y$ is the expected loss it incurs on draws from $\D$, written $L_\D(f) = \Ex_{(x,y) \sim \D} \ell(f(x),y)$.


There are two main ``settings'' in PAC learning. The  \emph{realizable setting} only requires that the data be perfectly explained by some hypothesis in $\H$. More generally, the \emph{agnostic setting} makes no assumption relating the data to the hypotheses, but shifts the goalposts as necessary to allow nontrivial guarantees: the expected loss at test time is evaluated only ``relative'' to that of the best hypothesis $h^* \in \H$. There are other settings which make more nuanced assumptions, such as $\D$ being of a particular parametric form or its support living in some (unknown) lower-dimensional space, etc. I will mostly discuss the realizable and agnostic settings in this article, those being the simplest and most studied from a theoretical perspective. %TODO:We will briefly discuss other settings in Section ??

The PAC model demands high probability guarantees of learners, in the worst case over distributions of interest. Consider first the realizable setting, where $\D$ is such that $\min_{h \in \H} L_{\D}(h) = 0$. A PAC learner has \emph{error} $\epsilon=\epsilon(n)$ and \emph{confidence} $\delta=\delta(n)$ if, when training data consists of $n$ i.i.d~samples from a realizable distribution $\D$, it produces a predictor $f$  satisfying $L_\D(f) \leq \epsilon$ with probability at least $1-\delta$. In the agnostic setting, where $\D$ can be arbitrary, we require $L_\D(f) - \min_{h \in \H} L_\D(h) \leq \epsilon$ with probability $1-\delta$.

In both the realizable and agnostic settings, we look for PAC learners with small $\epsilon$ and $\delta$ as a function of the sample size $n$. An equivalent perspective looks at the sample complexity $m(\epsilon,\delta)$, which is the minimum sample size which guarantees error  at most $\epsilon$ with probability at least $1-\delta$. We say a problem is \emph{PAC learnable} if its PAC sample complexity is finite whenever $\epsilon,\delta > 0$.

For most PAC learning problems, learnability and sample complexity are characterized in terms of a  ``dimension'' of the hypothesis class. Most prominently this is the \emph{VC dimension} for binary classification, the \emph{fat shattering dimension} for agnostic regression, and the \emph{DS dimension} for multiclass classification (see \cite{anthony_neural_1999,daniely_optimal_2014,brukhim_characterization_2022}). Treatment of these is beyond the scope of this article. The unfamiliar reader need not worry, however,  as dimensions will feature only tangentially in our~discussion.




%\paragraph{Learning settings: Realizable, Agnostic, etc.} In learning theory, evaluating a supervised learning algorithm requires specifying a data model and an objective. We will leave the details of the data model flexible for now, to allow for both the PAC model and the adversarial transductive model. Nonetheless we will describe two variations, which we call ``settings'', which cut across different models. The  \emph{realizable setting}  requires only that the data be perfectly explained by some hypothesis $h \in \H$ --- i.e., there exists a hypothesis which is guaranteed to suffer a loss of $0$ on training and test data. The performance of the learning algorithm is its expected loss at test time for some ``worst case'' realizable instance. More generally, the \emph{agnostic setting} makes no assumption relating the data to the hypotheses, but shifts the goalposts as necessary to allow nontrivial guarantees: the expected loss at test time is evaluated only ``relative'' to that of the best hypothesis $h^* \in \H$, again for some ``worst case'' instance. There are other settings which make more nuanced assumptions about the data, such as it is drawn from a distribution of a particular parametric form, or that it lives in some (unknown) lower-dimensional space, etc. We will mostly discuss the realizable and agnostic settings, those being the simplest and most studied from a theoretical perspective.




%%% Local Variables:
%%% mode: latex
%%% TeX-master: "learning_matching"
%%% End:

\section{Background and Related Work}
\label{sec:setup}

In this section, we describe the components of a RAG system (\Cref{sec:RAG_description}), revisit membership inference for machine learning (\Cref{sec:MI_in_ML}), and
discuss recent works on privacy leakage in RAG systems in (\Cref{sec:priv_leakage_RAG}).

\subsection{Retrieval Augmented Generation (RAG)}
\label{sec:RAG_description}

Let $\mathcal{G}$ be some generative LLM, with some retriever model $\mathcal{R}$, and $\mathcal{D}$ denote the set of documents part of the RAG system $\mathcal{S}$.
Most real-world systems that deploy user-facing LLMs rely on guardrails \citep{dong2024building} to detect and avoid potentially malicious queries. One such technique that also happens to benefit RAG systems \citep{ma-etal-2023-query, beck2025raising, mo-etal-2023-convgqr, lin2020conversational, wang2024maferw} is ``query rewriting", where the given query $q$ is transformed before being passed on to the RAG system. Query rewriting is helpful in dealing with ambiguous queries, correcting typographical errors, providing supplementary information, in addition its utility in circumventing some adversarial prompts \citep{jain2023baseline}.
\begin{align}
    \hat{q} = \text{rewrite}(q).
\end{align}
For the transformed query $\hat{q}$, the retriever $R$ begins by producing an embedding for $\hat{q}$ and based on some similarity function (typically cosine similarity), fetching the $k$ most relevant documents
\begin{align}
    D_k = \operatorname*{arg\,top-}k_{d \in \mathcal{D}} \text{sim}(\hat{q}, d),
\end{align}
where $\text{sim}()$ represents the similarity function, and \(\operatorname*{arg\,top-}k\) selects the top-$k$ documents with the highest similarity scores.
The generator $\mathcal{G}$ then generates an output based on the contextual information from the retrieved documents \citep{lewis2020retrieval}:
\begin{align}
    y = \mathcal{G}(\text{ins}(\hat{q}, D_k)),
\end{align}
where $\text{ins}(q, D_k)$ represents the query and context wrapped in a system instruction for the generative model An end user only gets to submit query $q$ to the RAG system $\mathcal{S}$ and observe the response $y$ directly in the form of generated text.

\subsection{\bf Membership Inference in ML}
\label{sec:MI_in_ML}

Membership inference attacks (MIAs) in machine learning seek to determine whether a specific data point \( x^* \) is part of a dataset involved in the ML pipeline, such as training \citep{shokri2017membership, nasr2019comprehensive, sablayrolles2019white, watson2022on, carlini2022membership} or fine-tuning data \cite{fu2024membership, mattern2023membership}. Formally, given access to a model \( \mathcal{M} \), an adversary constructs an inference function \( \mathcal{A} \) that outputs:
\[
\mathcal{A}(x^*, \mathcal{M}) \in \{1, 0\},
\]
where \( 1 \) indicates that \( x^* \) is a member of the dataset, and \( 0 \) indicates otherwise. Such attacks have been explored across a broad spectrum of models—including traditional ML architectures~\citep{shokri2017membership}, LLMs \citep{duan2024membership}, and diffusion models \citep{duan2023diffusion}—by exploiting behavioral discrepancies between data seen during training (members) and unseen data (non-members). For instance, many ML models assign higher confidence scores to member data points \citep{shokri2017membership}.

MIAs have shown varying degrees of success across different domains, including images and tabular data \citep{zarifzadeh2024low, carlini2022membership, suri2024do, shokri2017membership}. However, these successes predominantly rely on \textit{parametric outputs} (e.g., confidence scores, perplexity, or loss values). Such outputs are often inaccessible in RAG systems. Moreover, RAG responses are dynamically generated based on content retrieved from external corpora rather than solely from the model's internal parameters. Thus, previous methods that depend on parametric signals are largely inapplicable. More importantly, \emph{the target of MIA in RAG systems specifically relates to whether external documents are retrieved during inference, rather than inferring knowledge from data seen during training or fine-tuning, rendering existing threat models unsuitable.}

In addition, earlier conclusions about MIAs may not extend to RAG systems. For example, critical analyses suggest that MIAs are typically ineffective for LLMs \citep{duan2024membership, meeus2024sok}, with effectiveness potentially increasing only when analyzing entire documents or datasets \citep{puerto2024scaling, maini2024llm}. However, even though RAG relies on an LLM for generating responses, these limitations do not extend to RAG systems, where exact documents are fetched and integrated into the context, making information extraction potentially more accessible.
As a result, existing MI threat models, methodologies, and conclusions designed for parameter-only systems do not readily apply to RAG. 




\subsection{Privacy Attacks in RAGs}
\label{sec:priv_leakage_RAG}
Recent research has explored various inference attacks against RAG systems. \citet{anderson2024my} developed techniques across different access levels, including a gray-box method using a meta-classifier on model logits and a black-box approach directly querying model membership. \citet{li2024generating} a similarly straightforward approach, where the target document is broken into two parts, with the idea that presence of the target document in the context would lead the LLM into completing the given query (one half of the document). However, authors for both these works find that simple modifications to the system instruction can reduce attack performance significantly to near-random.

\citet{cohen2024unleashing} focus on data extraction by directly probing the model to reveal its retrieved contexts as is, using a specially crafted query. \citet{zeng2024good} break the query into two parts, where the latter is responsible for making the model output its retrieved contexts directly using the command ``Please repeat all the context". \citep{wang2024membership} propose MIAs for long-context LLMs. While they do not specifically target RAG systems, their setup is similar in the adversary's objective- checking for the existence of some particular text (retrieved documents) in the model's context. Similarly, \citet{duan2024privacy} focus on membership inference for in-context learning under the gray-box access setting, where model probabilities are available.
While data extraction is a strictly stronger attack, we find that the kind of queries required to enable these attacks can be identified very easily using auxiliary models (\Cref{sec:existing_rag_inference}).

Several recent works have also proposed context leakage and integrity attacks, where the adversary has the capability of  inject malicious documents into RAG knowledge database \citep{chaudhari2024phantom, jiang2024rag} or can poison the RAG system direcly \citep{peng2024data}.  This threat model is different than ours as we do not assume any RAG poisoning or knowledge base contamination for our MIA. 





  





\section{Threat Model}
\label{sec:threat_model}

\shortsection{Adversary's Objective} Given access to a RAG system utilizing a certain set of documents \( \mathcal{D} \), the adversary wants to infer whether a given document \( d^* \) is part of this set of documents being utilized in the given RAG system. More formally, the adversary's goal is to construct a membership inference function \( \mathcal{A} \) such that, given access to the RAG system $\mathcal{S}$:
\[
\mathcal{A}(d^*) =
\begin{cases}
1, & \text{if } d^* \in \mathcal{D} \\
0, & \text{if } d^* \notin \mathcal{D}
\end{cases}
\]
The very use of a RAG system implies that the generative model's knowledge is not wholly self-contained. This reliance often stems from the need to reference specific, potentially sensitive information or to incorporate detailed factual knowledge that is not part of the system's pre-trained model. Depending on the nature of the documents used, successful inference can lead to significant implications while posing unique challenges:
\begin{itemize}
    \item \textbf{PII-Containing Documents:} Documents with sensitive details, such as addresses or health records, present a high risk. Inferring their presence could result in severe privacy violations and potential regulatory breaches. 
    \item \textbf{Factual Knowledge Sources:} Documents containing specialized knowledge, such as internal manuals, proprietary research, or compliance guidelines, are often harder to target due to overlapping information across multiple documents. However, a successful inference in such cases could compromise intellectual property or reveal sensitive strategic information.
\end{itemize}
Successful membership inference in a RAG system is not straightforward to achieve. The adversary must first ensure that the target document $d^*$, if present, is consistently retrieved by the RAG system during its operation. Additionally, the adversary must craft queries in a manner that not only distinguishes the target document from other potentially related documents in $\mathcal{D}$ but also bypasses any intermediate processes employed by the RAG system (as discussed in \Cref{sec:existing_rag_inference}) that may limit inference success.

\shortsection{Adversary's Capabilities} We operate under a black-box access model where the adversary can query the target RAG system, but possesses no information about its underlying models or components. We assume the adversary has access to an auxiliary LLM, which it leverages to generate queries and interpret answers.
The adversary lacks knowledge of the retriever and generator models used by the victim, including their hyperparameters (\eg $k$ for top-$k$ retrieval, temperature settings for generation, \etc). The adversary also lacks knowledge of system instructions used in the victim RAG system, or query-rewriting strategies (if any) employed. Like in a typical membership inference scenario, the adversary owns a set of non-member documents from the same data distribution, which it uses to establish thresholds for predicting membership. Unlike some prior work \citep{chaudhari2024phantom} that assumes the ability to inject poisoned documents, the adversary in this setup has \textbf{no read or write access to the data used by the victim's RAG system}.




\section{Defenses Falter Against Targeted Backdoors}
\label{sec:existing_finicky}

Several works have proposed defenses against poisoning attacks in FL \citep{zhang2023flip, nguyen2022flame, blanchard2017machine, yin2018byzantine,fung2018mitigating,cao2021fltrust,wang2022flare,pillutla2022robust,shejwalkar2021manipulating}. It is standard for such works to focus on a specific learning configuration, for instance setting a specific learning rate and batch size for clients. While demonstrating superior performance in a particular setup is useful, it provides no guarantees about how well the defense would work in other valid learning configurations. We begin with an exploration of some of these learning configurations (\Cref{sec:fl_setup_matters}) and observe that there exist several equally-valid learning configurations where the model achieves acceptable MTA and the adversary's objective is preserved, thus making them all equally valid FL configurations. However, we find that all existing defenses we evaluated are effective only for a subset of these valid configurations and no defense is resilient across all configurations (\Cref{sec:defenses_fail_across_configs}).

\subsection{Learning Configuration Matters for Attack Success}
\label{sec:fl_setup_matters}

To understand the impact of the learning configuration on model robustness and adversarial susceptibility, we conduct a grid-search analysis over key hyperparameters, varying the client's learning rate, batch size, and number of epochs on the CIFAR-10 dataset.
Our findings in \Cref{fig:fl_setup_impact} reveal that small changes in the learning configuration can significantly alter the model’s vulnerability to backdoor attacks.
\begin{figure}[h!]
    \includegraphics[width=.98\linewidth]{assets/section4/undefended_asr.pdf}
    \caption{Visualizing the impact of the learning configuration (learning rate, batch size, and number of local epochs) on ASR, for 1.25\% DPR and 20\% MCR, for CIFAR-10 with IID data. We only visualize configurations with MTA $\geq80\%$. The attack is successful on multiple configurations (in yellow).}
    \label{fig:fl_setup_impact}
\end{figure}
% Certain configurations produce extreme outcomes, such as low MTA or low ASR, but more interestingly, many configurations exhibit simultaneously high MTA and high ASR. This combination indicates a "\textbf{danger zone}" of configurations that yield high model utility while remaining highly vulnerable to adversarial manipulation.
% \anshuman{Could compress below further to be 3-4 lines each at most.} \georgios{Just did. Check it out.}

%One particularly concerning observation is that low learning rates (e.g., 0.01) and small batch sizes (e.g. 64) create conditions where targeted backdoor attacks become extremely stealthy. In this regime, adversarial updates remain close to benign updates, making it difficult for anomaly-based defenses to detect them. This also allows the global model to maintain high MTA while the attack achieves a high ASR.Importantly, this phenomenon arises even in the absence of adversarial manipulation, as natural variability in client updates creates overlapping patterns between benign and adversarial updates. Given the natural variability, it is difficult to distinguish benign updates from adversarial updates, making the FL setup itself a critical factor in system vulnerability.

\subsubsection{Impact of learning rate}

%In general, learning configurations with a lower learning rate (e.g. 0.01) generally yield local updates that are smoother with respect to the client's local data and also consistently yield results with high MTA and ASR. On the other hand, higher learning rates (e.g. 0.1) lead to local updates that when aggregated cause performance degradation both on the MTA and the ASR. \georgios{A counter for 0.1 lr instability would be adopting a moving average aggregation for server aggregation.}

%Lower learning rates (\eg 0.01) produce local updates that are smoother and more possibly aligned with the client's local data distribution. This characteristic not only enhances the MTA but also enables the adversary to inject backdoors with minimal deviation from benign updates, resulting in consistently high ASR. By contrast, higher learning rates (\eg 0.1) lead to noisier local updates, which, when aggregated at the server, can cause performance degradation on both MTA and ASR. This instability at higher learning rates poses challenges for backdoor attacks but also negatively impacts the utility of the global model.

Lower learning rates (\eg 0.01, 0.025) allow for smaller, more gradual updates to the model parameters, making it easier for adversarial objectives to be embedded into the global model with minimal deviation from benign updates, resulting in a high ASR. In contrast, higher learning rates (\eg 0.1) cause larger, less stable updates that disrupt the optimization process, leading to degraded performance, with large drops in MTA and ASR alike.

\subsubsection{Impact of batch size}

%A larger batch size enables setups with different learning rates to achieve high MTA and ASR. On the other hand, a smaller batch size is making it more difficult to inject a targeted backdoor and only very low learning rates (e.g. 0.01) can successfully inject the backdoor.

% Batch size has a significant impact on both MTA and ASR.

% Larger batch sizes (\eg 256) reduce the variability in local updates, creating conditions where high MTA and ASR are achievable across a range of learning rates. This makes large batch sizes particularly susceptible to backdoor attacks, as adversarial updates blend seamlessly with benign updates. Conversely, smaller batch sizes (\eg 64) increase the variability in client updates, making it more difficult for the adversary to inject a targeted backdoor. In such setups, only very low learning rates (\eg 0.01) succeed in achieving a high ASR, as they produce smoother adversarial updates that can evade detection despite the higher noise introduced by small batch sizes.

Larger batch sizes (\eg 256) reduce variability in local updates, enabling high MTA and ASR across learning rates, making them more vulnerable to backdoor attacks. Smaller batch sizes (\eg 64) increase update variability, requiring very low learning rates (\eg 0.01) for effective backdoor injection.


\subsubsection{Impact of training epochs}

%Configurations with a small number of local training epochs (e.g. 2, 5) seem to be the ones that are more susceptible to backdoor poisoning. In general, the larger the number of local training epochs the clients use on locally training their models, the harder it is for the adversary to inject a targeted backdoor. 

% The number of local training epochs is another critical factor influencing the success of targeted backdoor attacks. Fewer local epochs (\eg 2 or 5) create conditions where clients submit updates that reflect only shallow optimization on their local data. This reduces the distinction between benign and adversarial updates, making the system more susceptible to backdoor poisoning. In contrast, larger numbers of local epochs (\eg 10 or 20) allow clients to perform more thorough local optimization, amplifying the differences between benign and adversarial updates. This makes it harder for adversarial updates to successfully inject backdoors while maintaining high MTA.

Fewer local epochs (\eg 2 or 5) lead to shallow optimization, reducing distinctions between benign and adversarial updates, making backdoor attacks more effective. Training with more local epochs (\eg 10 or 20) induces a \textit{polarization} effect, where clients' local models become more tightly aligned with their respective datasets. This stronger alignment reduces the influence of malicious updates during aggregation, as the backdoor signal becomes diluted and less effective at propagating into the global model. This polarization is even stronger when high LRs are used in conjunction with more training epochs, as model's MTA also suffers.

\Cref{tab:fl_setup_exps} highlights a range of configurations that we term the "\textbf{danger zones}"—settings where the FL system achieves both high MTA (above 80\%) and adequately high ASR (above 85\%). These configurations are of great interest because they strike a balance between utility and vulnerability; accurate for legitimate tasks while remaining highly susceptible to targeted backdoor attacks. This highlights the critical need for a defense to be resilient across multiple learning configurations, ensuring robustness regardless of the setup and achieving true \textit{learning configuration independence}. This is particularly desirable because relying on a single configuration, even if effective, is not always feasible in practice. Various constraints, such as the batch size supported by a specific device, the number of epochs a client is willing to commit to, or other resource limitations, can dictate configurations in practice. Therefore, a defense mechanism that can adapt to different setups without compromising its efficacy is essential for practical and widespread deployment.

%To demonstrate the impact of FL setup, we conducted a series of grid-search experiments using the CIFAR-10 dataset in an iid client data setting, varying the learning rate, batch size, and other hyperparameters. In the absence of any defense, we found that certain configurations led to significantly higher ASR and lower MTA, while others resulted in low ASR but at the cost of degraded MTA. Crucially, some configurations produced both high ASR and high MTA, highlighting the existence of "danger zones" where targeted backdoor attacks are especially effective. A key finding is that low learning rates (e.g., 0.01) and small batch sizes create conditions where the attack becomes exceptionally stealthy. In this regime, the attack remains covert, preserving high MTA while maintaining a high ASR, which poses a significant challenge for defense strategies. This observation reveals that even without explicit adversarial intervention, some FL configurations are inherently vulnerable. \georgios{"inherently" vulnerable is a bold statement, maybe we should opt for a different term?}

% \todo{Start with a grid-search style result for experiments without any defense to show how some configurations lead to very bad MTA and/or ASR but more importantly, there are several successful candidates. Give general rule-of-thumb for configs to use in evaluations}

\subsection{Limitations of Existing Defenses}
\label{sec:defenses_fail_across_configs}

Based on our analysis, we identify 10 learning configurations where the attack is stealthy (minimal impact on MTA) and highly successful (has ASR higher than 85\%), as given in \Cref{tab:fl_setup_exps}. For our evaluation we consider key, prominent defense methods which offer diverse strategies to combat backdoor attacks in FL. These defense strategies include coordinate-based approaches like Median \citep{yin2018byzantine} and Multi-Krum \citep{blanchard2017machine}, trust-based methods such as FLTrust \citep{cao2021fltrust}, reputation-based schemes like FoolsGold \citep{fung2018mitigating} and FLARE \citep{wang2022flare}, anomaly detection frameworks like FLAME \citep{nguyen2022flame} and local adversarial training methods like FLIP \citep{zhang2023flip}. For more details on these defenses and other related works, see \Cref{sec:experiments} and \Cref{sec:related_work}.

\begin{table}[h]
    \centering
    \small
    \begin{tabular}{llcc|cc}
    \toprule
    \textbf{Config} & \textbf{LR} & \textbf{BS} & \textbf{Epochs} & \textbf{MTA (\%)} & \textbf{ASR (\%)} \\
    \midrule
    C1 & 0.05 & 128 & 2 & 85.08 & 96.5 \\
    C2 & 0.05 & 256 & 2 & 86.29 & 95.8 \\
    C3 & 0.025 & 256 & 5 & 88.33 & 94.8 \\
    C4 & 0.01 & 64 & 2 & 87.03 & 93.9  \\
    C5 & 0.025 & 128 & 2 & 86.65 & 92.9 \\
    C6 & 0.025 & 256 & 2 & 82.97 & 91.3 \\
    C7 & 0.1 & 256 & 2 & 85.55 & 91.1 \\
    C8 & 0.01 & 128 & 2 & 84.47 & 89.7 \\ 
    C9 & 0.01 & 128 & 5 & 89.04 & 86.6\\
    C10 & 0.01 & 256 & 10 & 87.97 & 85.6 \\
    \bottomrule
    \end{tabular}
    \caption{Client FL configurations for successful stealthy attacks on CIFAR-10 \ie cases with MTA $\geq 80\%$ and ASR $\geq 85\%$. }
    \label{tab:fl_setup_exps}
\end{table}
%Each of these methods aims to limit the influence of malicious updates during aggregation. For example, Median evaluate their defense on a single, simpler learning task --MNIST \citep{} and only mention the total number of clients that participate in the federation. The authors do not mention anything related to the client learning setup (lr, batch size or number of local training epochs). Multi-Krum attempt to reduce the impact of outliers by using robust statistics. In their paper the authors again only mention how the data is partitioned amongst clients without any significant mention of the client local learning setup. They do however evaluate their method on various different batch sizes. FLTrust relies on a trusted server-side reference model to filter out anomalous updates. The authors report the use of a 'combined' (i.e. the product of the server and local lr) learning rate of 0.002, a batch size of 64 and 1 as the total number of local training epochs. FoolsGold tracks client updates to identify and penalize suspiciously similar contributions. They do not explicitly report a learning rate and the number of local training epochs for their evaluation setup. They report a batch size of 10 and 50, depending on the dataset. FLAME leverages anomaly detection to flag potentially malicious gradients. They do not mention anything about the local learning setup apart from how the federation is structured. 
% \anshuman{Too long; it's nice to have these details mentioned for sure, but should not overlead reader before we get to our main results. Can give a 1-2 sentence summary here} \georgios{What do you mean by too long?? Are you talking about the following section in which I describe the differences per setup?}\anshuman{I mean the defense-wise discussion that follows this comment, about exactly which settings previous defenses considered. We can give an example or two and give a blanket statement about a lack of consistency in evaluation setup (and explain them in detail somewhere in the appendix)} \georgios{Understood.}
% MOVE TO APPENDIX.
%Each of these methods aims to limit the influence of malicious updates during aggregation. However, most works provide limited or inconsistent details about their evaluation setups, particularly concerning client learning configurations such as learning rate, batch size, and the number of local training epochs. For instance, Median is evaluated on a simpler learning task (MNIST) and specifies only the total number of participating clients. The authors provide no details about the client learning setup, including learning rate, batch size, or number of local training epochs. Similarly, Multi-Krum reduces the impact of outliers using robust statistics but focuses primarily on how the data is partitioned among clients. While it does evaluate the method across different batch sizes, it lacks significant discussion of the broader local training setup. FLTrust adopts a trusted server-side reference model to filter anomalous updates. The authors report using a "combined" learning rate of 0.002, a batch size of 64, and a single local training epoch. In contrast, FoolsGold, which identifies and penalizes suspiciously similar client contributions, does not explicitly report the learning rate or number of local training epochs in its evaluation. Instead, it mentions using batch sizes of 10 or 50 depending on the dataset. FLAME, which leverages anomaly detection to flag potentially malicious gradients, describes the structure of the federation but provides no information about the local learning setup, such as learning rate, batch size, or training epochs.
% Each of these methods aims to limit the influence of malicious updates during aggregation. However, %
Evaluation setups for these defenses are often inconsistent, with significant variation in client learning configurations such as learning rate, batch size, and the number of local training epochs. For instance, some works (e.g., FLTrust) specify fixed learning rates and batch sizes, while others (e.g., Multi-Krum, FLAME) provide little to no details about the client training setup, focusing instead on aggregation logic. This lack of standardization in evaluation protocols raises concerns about the reproducibility and generalizability of reported results. A detailed comparison of these evaluation setups is provided in \Cref{app:baseline_details}.

% Shifted to discussion section
% This lack of clarity and standardization in reporting client learning configurations limits the reproducibility of these methods and raises concerns about their robustness across diverse FL setups.

% Under certain learning configurations, these defenses fail to detect or mitigate backdoor attacks, allowing adversaries to achieve high ASR.

\begin{figure}[h]
    \includegraphics[width=.9\linewidth]{assets/section3/defenses_heatmap.pdf}
    \caption{ASR (\%) for our defense (DROP) and various existing defenses for 10 FL configurations (1.25\% DPR, 20\% MCR) where stealthy attacks are possible. No existing defense provides consistent protection across all configurations.}
    \label{fig:baselines_heatmap}
\end{figure}

% When applying existing defenses under these same conditions, we observe that none of them consistently prevent the backdoor attack
We find that these approaches often struggle against targeted backdoor attacks due to the attack's stealthy nature (\Cref{fig:baselines_heatmap}). Median and Multi-Krum fail to exclude the adversary’s updates since the poisoned gradients remain close to the statistical norm, while FLTrust struggles to identify the malicious contributions due to the minimal deviation from expected update patterns. Similarly, FoolsGold proves largely ineffective because the adversary’s contributions exhibit insufficient variability across rounds, making them difficult to detect. FLAME performs reasonably well in certain configurations but fails against attacks that exploit low learning rates. Among the defensive baselines, FLIP is the only one that slightly reduces the ASR, though its mitigation is insufficient to provide robust protection.
Existing defenses thus have a severe limitation when it comes to stealthy backdoors: \textbf{their resilience is sensitive to the specific FL learning configuration.}

\section{Our Defense: DROP}
\label{sec:proposed_defense}

\begin{figure*}[htbp]
    \centering
    \includegraphics[width=\textwidth]{assets/system_final.png}
    \caption{Overview of the proposed DROP defense. Each round \( t \) begins with the server broadcasting the global model to all clients and selecting a subset for local training, which may include both benign (green) and malicious (red) clients. After updates are submitted, DROP employs: (1) \textbf{Agglomerative Clustering} to detect anomalous updates, (2) \textbf{Activity Monitoring \& Penalization} to track and penalize suspicious clients, and (3) \textbf{Knowledge Distillation}, where a GAN-generated synthetic dataset and client logits guide the distillation of the global model. The final model \( \mathbf{w}_{t+1} \) serves as the global model for round \( t+1 \).}
    \label{fig:drops} 
\end{figure*}

From our analysis, it is clear that the learning configuration of a FL system plays a pivotal role in the success of targeted backdoor attacks. Certain configurations exhibit inherent vulnerabilities, allowing adversarial updates to bypass detection and achieve high ASR. Moreover, the performance of existing defenses shows significant variability across different configurations, revealing their lack of robustness across learning setups. These findings emphasize the urgent need for a \textit{universal} defense mechanism that is resilient against a wide array of attack strategies and remains agnostic to the underlying learning configuration. 

To address this challenge, we introduce \textbf{DROP} (\textit{\textbf{D}istillation-based \textbf{R}eduction \textbf{O}f \textbf{P}oisoning}), a federated framework designed to counter three critical adversarial scenarios:
\begin{itemize}
    \item aggressive adversaries that may resort to high-DPR attacks,
    \item diverse MCRs  within the federation, and
    \item stealthy, low-DPR attacks that exploit specific learning configurations.
\end{itemize}
These scenarios each present unique challenges that require a cascade of countermeasures, as summarized in the following sections. An overview of our approach is given in \Cref{fig:drops}.
%
% DROP operates within the FedAvg \citep{mcmahan2017communication} paradigm (described in \Cref{sec:background}). \anshuman{FedAvg is presumably the default choice for FL; do we need to state it here explicitly while describing DROP?}
% At each training round $t$, the local updates from participating clients $\mathcal{C}_t$ undergo a series of three countermeasures to ensure robustness against targeted backdoor attacks. 

The first countermeasure, \textit{Agglomerative Clustering} (\Cref{subsec:agglo_clustering}), identifies and isolates malicious updates by clustering submitted models based on their pairwise Euclidean distances.
% The updates are divided into two clusters: a benign cluster \(C_b\) and a suspicious cluster \(C_s\). The smaller of the two clusters, \(C_s\), is pruned, and the updates within it are deemed \textit{suspicious}.
This process is particularly effective against high-DPR attacks, where malicious updates deviate significantly from benign updates.  
%
The second countermeasure, \textit{Activity Monitoring} (\Cref{subsec:activity_monitoring}), tracks client behavior across training rounds. By maintaining a reputation score based on prior clustering results, this mechanism penalizes clients frequently flagged as suspicious and reduces their influence in future aggregations. This reputation-based approach ensures robustness against per round benign-to-malicious client ratio fluctuations, where the proportion of malicious clients can vary widely across rounds due to random client selection.  

The third and final countermeasure, \textit{Knowledge Distillation} (\Cref{subsec:kd}), addresses the most challenging class of attacks: stealthy, low-DPR strategies. These attacks encode adversarial objectives subtly into model updates, making them indistinguishable from benign updates and enabling them to bypass clustering schemes. To neutralize these residual adversarial signals, we use synthetic data generated by a GAN trained via logit-driven distillation to produce a \textit{cleansed} version of the global model. This process ensures robustness against low-DPR attacks and renders the defense agnostic to learning configurations.

Through this cascade of countermeasures, DROP achieves the following three goals:  
\begin{enumerate}
    \item Resilience against aggressive, high data poisoning rates.
    \item Adaptability to diverse malicious client ratios.
    \item Robustness against stealthy, low data poisoning rates.  
\end{enumerate}
% \anshuman{Just mentioned these 3 at the start of the section- maybe refer to those instead of repeating here?}
Next, we provide a detailed explanation of the design and motivation behind each component. The entire DROP algorithm is presented in Algorithm \ref{alg:drop}. 

% DROP ALGORITHM
\begin{algorithm}
\caption{DROP}
\label{alg:drop}
\SetKwInput{kwGlobals}{Globals input}
\SetKwFunction{FMain}{DROP}
\SetKwFunction{FClustering}{AgglomerativeClustering}
\SetKwFunction{FActivity}{UpdateActivity}
\SetKwFunction{FScore}{score}
\SetKwFunction{FDistillation}{KD}
\SetKwProg{Fn}{Function}{:}{}
\SetKwFor{ForEach}{for each}{do}{end}

\kwGlobals{initial model parameters $w_0$, total training rounds $T$, total client set $\mathcal{N}$, ban threshold $\tau_b$, generator $G$, clean seed set $\mathcal{D}_{\text{clean}}$, knowledge-distillation $KD$}

\For{each round $t \in \{1, \dots, T\}$}{
    \tcp{\textcolor{blue}{Step 1: Training and Update Collection}}
    \ForEach{client $c \in \mathcal{C}_t \subset \mathcal{N}$}{
        $w_c^t \leftarrow \text{ClientLocalTraining}(w_{t-1})$ \tcp*[r]{\textcolor{blue}{Client $c$ trains locally and returns update}}
    }
    
    \tcp{\textcolor{blue}{Step 2: Agglomerative Clustering}}
    $\mathcal{C}_{\text{b}}, \mathcal{C}_{\text{s}} \leftarrow \FClustering(\{w_c^t : c \in \mathcal{C}\})$ \tcp*[r]{\textcolor{blue}{Cluster client updates into \textbf{b}enign and \textbf{s}uspect groups}}
    
    \tcp{\textcolor{blue}{Step 3: Activity Monitoring}}
    \ForEach{client $c \in \mathcal{C}$}{
        \FActivity($c$) % \tcp*[r]{\textcolor{blue}{Update the activity score for client $c$}}
        \If{$c \in \mathcal{C}_{\text{b}} \land \FScore(c) \geq \tau_b$}{
            \tcp{\textcolor{blue}{Exclude $c$ if it exceeds penalty threshold}}
            $\mathcal{C}_{\text{b}} \leftarrow \mathcal{C}_{\text{b}} \setminus \{c\}$ %\tcp*[r]{\textcolor{blue}{Exclude client $c$ if it has exceeded penalty threshold}}
        }
    }
    
    \tcp{\textcolor{blue}{Aggregate model updates from benign clients}}
    $w_t \leftarrow \text{Aggregate}(\{w_c^t : c \in \mathcal{C}_{\text{b}}\})$ % \tcp*[r]{\textcolor{blue}{Aggregate model from benign client updates only}}

    \tcp{\textcolor{blue}{Step 4: Logit-Driven Model Distillation}}
    $w_t \leftarrow \FDistillation(w_t, \{w_c^t : c \in \mathcal{C}_{\text{b}}\}, G, \mathcal{D}_{\text{clean}})$ %\tcp*[r]{Use collective logits to cleanse the global model}

    \tcp{\textcolor{blue}{Step 5: Broadcast the Cleansed Global Model}}
        \ForEach{client $c \in \mathcal{C}$}{
            \text{Send}($c, w_t$);
        }
}


% \Fn{\FClustering{$\{w_c^t : c \in \mathcal{C}\}, \tau_c$}}{
%     Form clusters of client updates using agglomerative clustering with Ward linkage\;
%     Identify cluster(s) containing outlier updates using threshold $\tau_c$\;
%     \KwRet{$\mathcal{C}_{\text{benign}}, \mathcal{C}_{\text{suspect}}$} \tcp*[r]{Return benign and suspect client sets}
% }

% \Fn{\FActivity{$c$, isBenign}}{
%     \If{isBenign}{
%         Increase client $c$'s "benign score" by 1\;
%     }
%     \Else{
%         Increase client $c$'s "penalty score" by 1\;
%     }
%     \If{Client $c$'s "benign score" $\geq \tau_a \times$ "penalty score"}{
%         Allow client $c$ to rejoin $\mathcal{C}$ if it was previously banned\;
%     }
% }

% \Fn{\FDistillation{$w_t, \{w_c^t : c \in \mathcal{C}_{\text{benign}}\}, G$}}{
%     \tcp{** Logit Aggregation from Client Updates **}
%     Aggregate logits $\mathcal{L} \leftarrow \text{AverageLogits}(\{w_c^t : c \in \mathcal{C}_{\text{benign}}\})$ \tcp*[r]{Compute ensemble logits from benign clients}

%     \tcp{** Generator-Guided Knowledge Distillation **}
%     \For{$i = 1, \dots, N_{\text{distill}}$}{
%         Generate synthetic data $\mathcal{D}_{\text{synthetic}} \leftarrow G(\mathcal{L})$\;
%         Update $w_t$ using knowledge distillation loss on $\mathcal{D}_{\text{synthetic}}$\;
%     }
%     \KwRet{$w_t$} \tcp*[r]{Return the cleaned model}
% }
\end{algorithm}

% \subsection{Agglomerative Clustering}
% \label{subsec:agglo_clustering}

% The first line of defense in our system is \textit{Agglomerative Clustering} (AC) \citep{müllner2011modernhierarchicalagglomerativeclustering, auror}, chosen for its suitability in detecting subtle deviations caused by targeted backdoor attacks. Clustering is a widely adopted countermeasure in filtering-based FL frameworks \citep{nguyen2022flame, fung2018mitigating, baybfed, mesas} due to its effectiveness in identifying dissimilarities within a population of model updates. These dissimilarities are often the result of adversarial strategies employing a high DPR that aggressively embed backdoors into updates. We choose AC for clustering due to its flexibility and hierarchical structure; unlike K-means, which is based on ranking distances that are too small to convey useful information \anshuman{What does that mean?}\alina{k-means has a number of limitations: sensitivity to initial centroid selection, susceptibility to outliers, etc.}, or HDBSCAN \citep{campello2013density}, which relies on density-based assumptions \anshuman{Why would that not work in our scenario?}\alina{That is not the issue, but need to set some hyper-params}, AC dynamically merges clusters until a stopping criterion is met. \alina{The use of AC needs better motivation, e.g., less hyper-parameters to tune}

% The use of \textit{Ward linkage} \citep{Ward01031963} is they key to AC's effectiveness \anshuman{Is ward-linkage a feature of AC, or one of several ways to perform part of the clustering? If former, can shorten this part a bit} \georgios{Yes it is exclusive to agglomerative clustering as far as I know. I have not seen it being used in other clustering methods.}. \alina{AC can be done with multiple linkage methods, and Ward is one of them, you can motivate why it's the right choice, as it min the variance of clusters.} By minimizing intra-cluster variance during the merging process, Ward linkage is particularly adept at detecting fine-grained and stealthy backdoors that can be embedded within benign-looking updates. This property ensures that subtle differences in high-dimensional weight spaces—indicative of adversarially perturbed updates—are captured and separated.

\subsection{Agglomerative Clustering}  
\label{subsec:agglo_clustering}  

The first line of defense in our system is \textit{Agglomerative Clustering} (AC) \citep{müllner2011modernhierarchicalagglomerativeclustering}, selected for its flexibility and hierarchical structure, which make it well-suited for detecting subtle deviations caused by targeted backdoor attacks. Clustering is a widely adopted countermeasure in filtering-based FL frameworks \citep{nguyen2022flame, fung2018mitigating, baybfed, mesas, auror} due to its ability to identify dissimilarities within a population of model updates, often resulting from adversarial strategies that embed backdoors aggressively with a high DPR. We chose AC over methods like K-means and HDBSCAN due to its lower sensitivity to hyper-parameter selection and its dynamic merging process, which eliminates the need to pre-specify the number of clusters. Unlike K-means, which can be sensitive to initial centroid selection and outliers, AC provides more consistent results by iteratively merging clusters based on similarity. Similarly, while HDBSCAN \citep{campello2013density} relies on density-based assumptions that require careful tuning of hyper-parameters, AC operates without such assumptions, making it more robust in diverse FL scenarios.

The use of \textit{Ward linkage} \citep{Ward01031963} further enhances AC's effectiveness by ensuring that intra-cluster variance is minimized during the merging process. Among the various linkage methods available for AC, Ward linkage is particularly effective for FL tasks, as it prioritizes clusters with low internal variance, which is crucial for distinguishing fine-grained and stealthy backdoors embedded within benign-looking updates. By capturing and separating subtle differences in high-dimensional weight spaces, Ward linkage ensures that adversarial updates are reliably isolated from legitimate ones.  

Let \(\mathcal{W} = \{\mathbf{w}_1, \mathbf{w}_2, \dots, \mathbf{w}_n\}\) denote the set of model updates from \(n\) participating clients in a given training round \(t\). Let $d(\mathbf{w}_i, \mathbf{w}_j) = \| \mathbf{w}_i - \mathbf{w}_j \|_2$ be the Euclidean distance between two model updates \(\mathbf{w}_i\) and \(\mathbf{w}_j\).
%
The clustering process starts with each client update \(\mathbf{w}_i\) being treated as a singleton cluster. At each step, the two clusters \(\mathcal{A}\) and \(\mathcal{B}\) with the smallest inter-cluster distance are merged. Using \textit{Ward linkage}, the distance between two clusters \(\mathcal{A}\) and \(\mathcal{B}\) is defined as the increase in total intra-cluster variance caused by merging the two clusters:
\begin{equation}
    d_{\text{Ward}}(\mathcal{A}, \mathcal{B}) = \frac{|\mathcal{A}| \, |\mathcal{B}|}{|\mathcal{A}| + |\mathcal{B}|} \, \| \boldsymbol{\mu}_{\mathcal{A}} - \boldsymbol{\mu}_{\mathcal{B}} \|_2^2,
\end{equation}
where \(|\mathcal{A}|\) and \(|\mathcal{B}|\) are the sizes (number of points) of clusters \(\mathcal{A}\) and \(\mathcal{B}\), and \(\boldsymbol{\mu}_{\mathcal{A}}\) and \(\boldsymbol{\mu}_{\mathcal{B}}\) are their respective centroids.
% \begin{equation}
%     \boldsymbol{\mu}_{\mathcal{A}} = \frac{1}{|\mathcal{A}|} \sum_{\mathbf{w} \in \mathcal{A}} \mathbf{w}, \quad 
%     \boldsymbol{\mu}_{\mathcal{B}} = \frac{1}{|\mathcal{B}|} \sum_{\mathbf{w} \in \mathcal{B}} \mathbf{w}.
% \end{equation}

The hierarchical clustering process proceeds iteratively until a stopping criterion is met. In our case, the process halts upon forming exactly two predefined clusters, \(\{\mathcal{C}_b, \mathcal{C}_s\}\), which effectively capture natural groupings of benign and malicious model updates.
In the context of FL, it is assumed that benign client updates will cluster together due to the shared training objective, while adversarial updates will form separate, smaller clusters due to their deviation from normal update patterns.

Minimizing intra-cluster variance ensures that the clustering process captures subtle distinctions in high-dimensional weight spaces, where even minor variations can signal meaningful differences, such as those between adversarially perturbed weights and legitimate ones. The hierarchical structure produced by Ward linkage allows flexibility in examining update patterns at various levels of similarity, enabling dynamic control over how clusters are merged. This approach is particularly valuable in FL, where client updates are naturally noisy due to heterogeneous data distributions but can also be adversarially manipulated. By clustering updates, this method enables the system to separate and potentially eliminate outliers or manipulated model updates.
% \newline
% \framedtext{\underline{\textbf{Goal 1}}: The clustering component aims to eliminate malicious updates that deviate significantly from their benign counterparts in each training round, offering resilience against high data poisoning rate attacks.}\label{goal1}
\goalbox{\underline{\textbf{Goal 1}}: The clustering component aims to eliminate malicious updates that deviate significantly from their benign counterparts in each training round, offering resilience against high data poisoning rate attacks.}\label{goal1}

\subsection{Activity Monitoring}
\label{subsec:activity_monitoring}

The second line of defense, complementing the clustering component, is an \textit{activity monitoring mechanism}, which tracks the behavior of participating clients over the course of training. 

This mechanism addresses a limitation of clustering-based defenses in federated learning, which assume that the number of malicious updates in a round is smaller than the number of benign ones. In real-world scenarios, however, client selection is random, and the defender has no prior knowledge of the number of malicious clients in any given round. This randomness can lead to rounds where adversarial clients outnumber benign ones, causing misidentification of the smaller cluster.
%To mitigate this, prior works commonly assume an upper bound on the proportion of malicious clients, often constraining the number of malicious participants in each round to less than 50\% of the selected clients. While this assumption allows clustering-based defenses to reliably identify and prune smaller clusters, enforcing it typically requires operating under low MCRs or artificially controlling client sampling to maintain a fixed benign-to-malicious client ratio across rounds \citep{zhang2023flip}, both of which limit the practical applicability of such methods.
Prior works often assume an upper bound on the proportion of malicious clients (typically less than 50\%), but enforcing this requires low MCRs or artificially controlled client sampling  to maintain a fixed benign-to-malicious client ratio across rounds \citep{zhang2023flip}, limiting practical applicability.
It is important to note that although the total proportion of adversarial clients in the federation (MCR) remains fixed, the actual ratio of malicious clients in any given round can fluctuate due to the random selection of clients.

The maximum tolerable MCR of a defense—the fraction of clients in the federation that can be adversarial while still allowing the defense to mitigate an attack—depends on the total number of clients in the federation, \(N\), and the number of clients sampled in each round, \(C\). 
%\anshuman{What do we mean by 'supported' threshold?} \georgios{It is the fraction of clients being malicious in the federation that a defense can tolerate to mitigate an attack. I changed the text a bit to make it more clear.} 
Under a random sampling strategy, \(C\) clients are drawn uniformly from the total \(N\) clients. Let $\rho$ be the MCR \ie proportion of malicious clients in the entire federation, so the expected number of malicious clients ($\mathcal{M})$ in a round is \(\mathbb{E}[\mathcal{M}] = \rho \cdot C\). By modeling the selection of clients as a Binomial distribution, we can compute a lower bound on the probability of malicious clients outnumbering benign ones in any given round as:
\begin{align}
    P\left(\mathcal{M} \geq \frac{C}{2}\right) \geq  1 - \left(4\rho(1-\rho)\right)^{\frac{C}{2}}.
\end{align}
For a derivation of the above, please see \Cref{app:bound_analysis}.
% By applying the Chernoff bound \citep{chernoff}, we can bound the probability that the number of malicious clients \(M\) significantly deviates from its expectation. Specifically, for any \(\delta > 0\):
% \begin{equation}
    % \mathbb{P}\left[\mathcal{M} \geq (1 + \delta) \cdot \rho \cdot C\right] \leq \exp\left(-\frac{\delta^2 \cdot \rho \cdot C}{2 + \delta}\right).
% \end{equation}
This bound implies that as \(C\) increases, the likelihood of selecting a disproportionately large number of malicious clients diminishes exponentially. Conversely, when \(C\) is small,
% the variability in the number of malicious clients per round increases, raising
the probability that malicious clients outnumber benign clients within the selected subset increases. 
%
For example, consider an FL system with \(N = 100\) total clients and a MCR of 40\% (\(\rho = 0.4\)). If \(C = 20\) clients are randomly selected in a given round, the expected number of malicious clients in the subset is \(\mathbb{E}[\mathcal{M}] = \rho \cdot C = 0.4 \cdot 20 = 8\).
However, with the inequality above we can see the probability of selecting more than 10 malicious clients can be \underline{non-trivial} ($\approx 0.34$), potentially resulting in a subset where malicious clients constitute more than 50\% of the selected participants (\(M > C/2\)). In fact, the bound suggests that in about a third of the FL training rounds, the malicious clients will be the majority. 
% However, due to the randomness of client selection, the actual number of malicious clients can deviate significantly from this expectation.
% Using the Chernoff bound and setting \( (1 + \delta) \cdot \rho \cdot C = C/2\), the probability of selecting more than 10 malicious clients\footnote{The detailed steps for the calculation of the bound can be found in \cref{sec:chernoff_appendix}.} (\(M > 10\)) can be \underline{non-trivial} (\(\mathbb{P}\left[\mathcal{M} \geq 10\right] \lesssim 0.8\)), potentially resulting in a subset where malicious clients constitute more than 50\% of the selected participants (\(M > C/2\)).
% This scenario undermines clustering-based defenses, as the smaller cluster would incorrectly represent benign clients.

To address scenarios where the number of malicious clients exceeds the benign ones in certain rounds, we propose a reputation-based mechanism. This approach tracks client behavior across rounds, penalizing suspicious clients identified by the clustering component (\Cref{subsec:agglo_clustering}) and gradually reducing their influence over the global model. By doing so, our method dynamically adjusts to varying levels of malicious participation without relying on rigid assumptions about MCR or artificially controlling client sampling. For instance, FLIP~\cite{zhang2023flip} enforces a fixed per-round MCR by '\textit{randomly}' selecting 10 clients per round, ensuring exactly 4 adversaries and 6 benign clients, thus artificially tampering with the randomness of client selection.
% \anshuman{Clarify that reputation helps account for misclassifications in rounds where adversary outnumbers benign clients; maybe can use the bound to show that on average as long as MCR is less than 50\%, in the long run there will be more benign clients identified successfully than incorrectly classified as adversaries} \georgios{Would the following paragraph that I have added work?}
By incorporating a reputation system, our defense mitigates errors caused by transient imbalances where adversarial clients may temporarily outnumber benign ones. Over multiple rounds, as long as the overall MCR remains below 50\%, the mechanism ensures that benign clients are correctly identified and retained more frequently than adversarial clients. This design helps prevent long-term accumulation of adversarial influence, even in rounds where clustering misclassifications may occur.

Our reputation-based mechanism employs a penalty and reward system that essentially tracks the trustworthiness of each client, enabling the server to mitigate persistent malicious activity while accounting for potential false positives. Specifically, by maintaining a cumulative penalty score \(\pi(c)\) for each client \(c\), the server can distinguish between clients with occasional false-positive detections and those consistently submitting suspicious updates. This ensures that benign clients incorrectly flagged as suspicious in a few rounds do not face permanent exclusion from the system.

\shortsection{Calculating Penalty Scores}
We begin by initializing the penalty score \(\pi(c) = 0\) for every client. As training proceeds, this score is dynamically updated based on the clustering results from each round.
% \newline
% \newline
% \underline{\textbf{Rule 1}}:
During each training round \( t \), the clustering component (\Cref{subsec:agglo_clustering}) determines whether a client's update is benign or suspicious by assigning the client with a penalty score:
\begin{align}
    \pi(c) =
    \begin{cases}
    \pi(c) + p, & \text{if } c \in \mathcal{C}_s\\
    \max(0, \pi(c) - r),              & \text{otherwise}
    \end{cases}
\end{align}
\newline
Essentially, if a client's update is flagged as \textit{suspicious} (\( c \in \mathcal{C}_s \)), its penalty score is increased by a constant penalty value \( p \). Otherwise, it is reduced by a constant reward value \( r \), ensuring the score remains non-negative. Capping the penalty score at zero prevents potentially malicious clients from gaining undue advantage in cases where they are repeatedly flagged as benign due to false positives. $p$ and $r$ are hyper-parameters determined by the server.

The server uses these scores to regulate client participation. Clients with a clean history \ie consistently flagged as benign (\(\pi(c) = 0\)) are deemed trustworthy and allowed to contribute to the global model, whereas clients with a record of suspicious update submissions (\(\pi(c) > 0\)) are restricted from participating in the aggregation process. However, depending on the server administration policy, the penalty system could be even stricter. Clients that accumulate excessive penalty points due to repeated suspicious updates are permanently \textit{blacklisted} from contributing in subsequent rounds:
\begin{comment}
\newline
\newline
\underline{\textbf{Rule 2}}: \textbf{\textit{(Optional)}} Specifically, the server permanently bans any client whose penalty score exceeds a predefined threshold \( \tau \):
\begin{equation}
    \pi(c) \geq \tau \implies \text{ban}(c).
\end{equation}
\georgios{Another option would be to rename Rule 2 to shortsection “Calculating Penalty/Trust Scores” and omit the rule formulation but still keep the text around it}
\end{comment}
Optionally, the server could permanently ban any client whose penalty score exceeds a predefined threshold.
% \newline
% \framedtext{\underline{\textbf{Goal 2}}: The activity monitoring component helps prevent malicious actors (based on their accumulated penalty points) from poisoning the global model during rounds in which the malicious outnumber the benign participants.}
\goalbox{\underline{\textbf{Goal 2}}: The activity monitoring component helps prevent malicious actors (based on their accumulated penalty points) from poisoning the global model during rounds in which the malicious outnumber the benign participants.}\label{goal2}



\subsection{Knowledge Distillation}
\label{subsec:kd}

Having tackled high-DPR and dynamic-MCR attacks, the most challenging category to defend against is low-DPR attacks.
These attacks are difficult to detect due to their stealthy nature, where adversarial objectives are subtly embedded into model weights, making the updates nearly indistinguishable from benign ones. This smooth encoding allows them to bypass traditional clustering-based defenses that rely on statistical or geometric properties of the updates. To counter this, DROP introduces a knowledge distillation-based cleansing mechanism that neutralizes residual adversarial signals in the global model using clean, synthetic data, effectively mitigating the impact of stealthy backdoor updates while preserving the model’s utility for legitimate tasks.

% Traditional clustering-based defenses rely on identifying significant deviations in update patterns to separate malicious updates from benign ones. However, low-DPR attacks are specifically designed to minimize these deviations, allowing adversarial updates to bypass clustering filters entirely.  To address this challenge, DROP introduces a knowledge distillation-based cleansing component that removes the residual adversarial signals embedded in the global model. By distilling the knowledge from the global model onto a surrogate model using clean, synthetic data, this component isolates and neutralizes the effects of stealthy backdoor updates while preserving the utility of the global model for legitimate tasks.
% \anshuman{Above two paragraphs can be shortened by a lot}

This approach leverages model stealing attacks, such as the \textit{MAZE} model stealing framework \citep{kariyappa2021maze}, which achieves state-of-the-art performance in extracting machine learning models under black-box conditions. The goal here is not to engage in model stealing as traditionally intended, but to clone the model without introducing any malicious behavior from compromised clients. \textit{MAZE} combines two key components: query synthesis and knowledge distillation. Knowledge Distillation \cite{hinton2015distillingknowledgeneuralnetwork} is a technique where a "student" model learns to approximate the output logits of a larger "teacher" model, enabling the transfer of knowledge while preserving essential decision boundaries. A generative adversarial network (GAN) \cite{radford2016unsupervisedrepresentationlearningdeep} synthesizes inputs to probe a target model's decision boundaries, and knowledge distillation transfers the target model's behavior to a surrogate model trained on these synthetic queries. By minimizing reliance on labeled data, \textit{MAZE} efficiently approximates the target model's functionality.

In our approach, we adapt and extend the \textit{MAZE} framework to take advantage of the white-box access available to the server in FL, as the server owns the global model. This adjustment eliminates the need for zeroth-order gradient estimation used in black-box settings and allows for direct backpropagation, improving both efficiency and accuracy. To ensure high-quality synthetic data, we initialize the GAN with a set of \( n \) clean samples, denoted as \( \mathcal{D}_\text{clean} = \{\mathbf{x}_i \}_{i=1}^n \) in step 4 of Algorithm \ref{alg:drop}, which helps align the generated queries with the original data distribution. Our approach requires only a small set of clean samples—typically less or equal than the size of a single client's dataset—making it practical and feasible for real-world FL scenarios.

In a training round $t$, instead of relying solely on the global model’s logits, we aggregate logits from client updates that pass the filtering process (\ie $\mathcal{C}_b$). Let the logit output from the global model be \( \mathbf{z}_t = f(\mathbf{x}; \mathbf{w}_t) \), where \( \mathbf{w}_t \) represents the global model parameters at round \( t \). Similarly, let the logit outputs from benign client models \( \mathcal{C}_b \subseteq \mathcal{C}_t \) be:
\begin{equation}
    \mathbf{z}_{t}^c = f(\mathbf{x}; \mathbf{w}_t^c), \quad \forall c \in \mathcal{C}_b.
\end{equation}
We compute the ensemble logits \( \mathbf{\bar{z}}_t \) as the average of the logits from benign clients:
\begin{equation}
    \mathbf{\bar{z}}_t = \frac{1}{|\mathcal{C}_b|} \sum_{c \in \mathcal{C}_b} \mathbf{z}_{t}^c.
\end{equation}

The generator \( G \) uses the ensemble logits \( \mathbf{\bar{z}}_t \) as feedback to synthesize new queries \( \mathcal{D}_{\text{synthetic}} \). These synthetic queries, denoted as \( \mathbf{x}_{\text{gen}} \), are used to guide a \textit{clone network}, which serves as a cleansed version of the global model. The clone network, parameterized by \( \mathbf{w}^{\text{clone}}_t \), is trained to align its predictions with the ensemble logits. Instead of the KL divergence used in the original \textit{MAZE} framework, we employ the \(\ell_1\)-loss, which provides more stable performance \citep{truong2021data}. The training objective for the clone network is defined as:
\begin{equation}
    \mathcal{L}_{\text{distill}} = \mathbb{E}_{\mathbf{x} \sim \mathcal{D}_{\text{synthetic}}} \left[ \| \mathbf{\bar{z}}_t - \mathbf{z}_t^{\text{clone}} \|_1 \right],
\end{equation}
where \( \mathbf{z}_t^{\text{clone}} = f(\mathbf{x}; \mathbf{w}^{\text{clone}}_t) \) represents the logit predictions of the clone network on synthetic data \( \mathbf{x} \).

The key difference between this approach and the original MAZE framework is that, instead of distilling knowledge from a single victim model, we distill knowledge from the aggregated logits of benign clients. The intuition behind this approach is that the ensemble logits \( \mathbf{\bar{z}}_t \), derived from benign client models, act as a \textit{consensus signal} to overwrite any adversarial influence introduced by poisoned updates. By aligning the clone network’s behavior with the aggregated benign logits, the distillation process effectively cleanses the model of any stealthy backdoor or poisoned behavior that might have evaded detection in previous defense layers. 
% \newline
% \framedtext{\underline{\textbf{Goal 3}}: The collective logit-driven knowledge distillation framework ensures that any subtle adversarial updates which have made their way to the global model are neutralized, restoring the global model’s robustness and reliability.}
\goalbox{\underline{\textbf{Goal 3}}: The collective logit-driven knowledge distillation framework ensures that any subtle adversarial updates which have made their way to the global model are neutralized, restoring the global model’s robustness and reliability.}\label{goal3}

% \subsection{DROPlet: a lightweight version of DROP}
% \label{subsec:droplet}

% While DROP provides comprehensive protection against targeted backdoor attacks through its cascading components, the knowledge distillation component introduces a non-trivial computational overheads. To offer a faster and more streamlined alternative \anshuman{are we saying that DROP is not streamlined? Focus more on lightweight part and less computationally heavy}, we propose \textit{DROPlet}, a lightweight, server-side plugin designed for ease of deployment and compatibility with various types of data. Unlike DROP, DROPlet omits the knowledge distillation component, focusing solely on agglomerative clustering and activity monitoring to detect and mitigate adversarial influences.

% The primary motivation behind DROPlet is to provide an efficient and domain-agnostic defense mechanism that can be easily integrated into existing FL systems without requiring significant computational resources. Since DROPlet does not depend on task-specific features or data modality assumptions, it is applicable across a wide range of FL tasks, including image, text, and tabular data. This versatility ensures that DROPlet can serve as a general-purpose defense framework in diverse real-world FL applications.

% DROPlet also excels in terms of speed and scalability. By excluding the knowledge distillation step, it achieves faster round completion times and reduces the server-side processing load, making it particularly suitable for large-scale FL deployments where computational efficiency is a critical concern. Despite its lightweight nature, DROPlet maintains strong defenses by relying on robust clustering to detect anomalies and activity monitoring to penalize persistently malicious clients. \anshuman{Forward-reference that it does not work in all settings. In fact, this could even be a new "baseline", replacing the likes of Multi-KRUM and Median} This combination ensures reliable performance against a wide range of adversarial strategies, making DROPlet another practical choice for defending FL systems in environments with limited computational resources.

\subsection{DROPlet: a Lightweight, Scalable Defense Mechanism}  
\label{subsec:droplet}  

To provide a faster, more resource-efficient alternative to DROP, we introduce \textit{DROPlet}, a lightweight, server-side plugin designed for easy deployment and high scalability.
Unlike DROP, DROPlet omits the knowledge distillation component and focuses solely on agglomerative clustering and activity monitoring to mitigate adversarial influences with minimal overhead.  

DROPlet is task-agnostic, with no assumptions on the underlying data modality, making it applicable to various FL tasks. This versatility allows DROPlet to function as a general-purpose defense mechanism in diverse FL deployments. By eliminating the computationally intensive knowledge distillation step, it achieves faster round completion times, making it particularly suitable for large-scale FL systems where computational efficiency is crucial.  

Despite its lightweight design, DROPlet offers robust protection by leveraging clustering to detect anomalies and activity monitoring to penalize malicious clients. However, it may not be effective against stealthy, low-DPR attacks (\cref{shortsec:droplet_results}).
% \anshuman{Forward-reference that it does not work in all settings. In fact, this could even be a new "baseline", replacing the likes of Multi-KRUM and Median} \georgios{Good idea, refined the text.}
Nevertheless, DROPlet remains a competitive baseline for FL defense evaluations, offering significant protection while being faster and easier to deploy compared to classical methods like Multi-KRUM and Median. 

\begin{table}
        \renewcommand{\arraystretch}{0.8}
        \begin{threeparttable}
        \begin{tabular}{@{}lcc@{}}
        \toprule
        Parameter & Floating Allegro Hand & Bimanual Robot Arms \\
        \midrule
        % \makecell{Initial object translational \\ perturbation (cm) }& [$\pm1.5$, $\pm1.5$, 0] & [$\pm 5$, $\pm 5$, 0]\\
        % \makecell{Initial object rotational \\ perturbation (rad) }& [0, 0, $\pm 0.3$] & [0, 0, $\pm 0.3$]\\
        Init. obj. trans.  pert. (cm) & [$\pm1.5$, $\pm1.5$, 0] & [$\pm 5$, $\pm 5$, 0]\\
        Init. obj. rot. pert. (rad) & [0, 0, $\pm 0.3$] & [0, 0, $\pm 0.3$]\\
        Object side length (cm) & [5.8, 6.2]  & [28, 32] \\
        Object mass (kg) & [0.1, 0.3]  & [0.25, 0.75]  \\
        Friction coefficients & [0.7, 1.3] &  [0.2, 0.4]  \\
        Task horizon (s) & 25 & 50 / 260  (Panda / iiwa) \\
        \bottomrule
        \end{tabular}
        \end{threeparttable}
        \caption{Ranges of different physical parameters $\theta$. The initial object pose is only perturbed in yaw, x, and y to ensure the object sits stably on the table. }
        \label{tab:domain_randomization}
        \vspace{0.5em}
\end{table}

\begin{figure*}[t]
\centering
\includegraphics[width=1.0\textwidth]{figures/trajopt_unittest.png}
	\caption{\textbf{Trajectory optimization is crucial for generating dynamically feasible trajectories}. (Top) Before trajectory optimization, the kinematically retargeted demos easily lose contact and drive the object out of reach with different physical parameters or slight deviations in object states. (Bottom) Trajectory optimization encourages robots to establish contact with and maintain good manipulability of the object. The tricolor axis indicates the object orientation.}
	\label{fig:trajopt_unittest}
\end{figure*}

\section{Trajectory Optimization Experiments}

While kinematic retargeting of demonstrations might suffice to generate data for simpler manipulation tasks such as pick and place, it often falls short for the more challenging contact-rich tasks requiring frequent contact mode switches and fine-grained actions. In this section, we demonstrate that trajectory optimization is crucial for generating diverse, dynamically feasible contact-rich trajectories on three high-dimensional dexterous manipulation systems: a floating Allegro hand, bimanual iiwa arms, and bimanual Panda arms.

Our data generation framework is agnostic to the choice of the trajectory optimizer. We implement 
% a contact-implicit model predictive controller based on smoothed contact dynamics \cite{suh2024dexterous} and
the cross-entropy method (CEM) \cite{de2005tutorial} to solve \eqref{eq:predictive_control} over a distribution of physical parameters and initial conditions, as specified in Table \ref{tab:domain_randomization}. 
%\russtcomment{Right... the SQP discussion tricked me, but I guess that's only for the retargeting. I thought you had replaced this. In this case, your approach is almost doing RL, but on a policy parameterized as a trajectory... right? why is that better than doing PPO on a small neural net policy, and generating data from that? If you stick with CEM, than this will be your burden of proof, i think?}

\underline{\textbf{Task}} Manipulating the object to a target pose on the table (Fig. \ref{fig:policy_rollouts}). The object is initially placed randomly on the table with an arbitrary face upward. Task success is defined as the object reaching within 3 cm and 0.2 rad of the target pose for the Allegro hand, and within 10 cm and 0.2 rad for the bimanual robot arms.  This task requires long-horizon reasoning of complex multi-contact interactions between the robot and the object. The necessary frequent contact mode switches and high-dimensional action space pose great challenges for traditional model-based planners, while the precise contact interactions require fine-grained control actions. 

\begin{table}
\centering
        \renewcommand{\arraystretch}{0.8}
        \begin{threeparttable}
        \begin{tabular}{@{}lcccc@{}}
        \toprule
        Perturbation & Allegro Hand & iiwa Arms & Panda Arms \\
        \midrule
        Original demo &4 / 24 & 5 / 24 & 6 / 24\\
        Object size & 2 / 24 & 1 / 24 & 4 / 24\\
        % Object mass & 1 / 24& 1 / 24 & \\
        % Friction coefficients & 3 / 24 & 2 / 24 & \\
        Initial object translation & 1 / 24 & 3 / 24 & 2 / 24\\
        Initial object orientation & 2 / 24 & 3 / 24& 3 / 24\\
        \midrule
        Trajectory optimization & 2164 / 3000 & 2252 / 3000 & 2462 / 3000 \\
        \bottomrule
        \end{tabular}
        \end{threeparttable}
        \caption{Success rates of replaying kinematically retargeted trajectories of the 24 original human demos, and trajectory optimization under random perturbations in physical parameters and object initial conditions. }
        \label{tab:kin_success_rate}
\end{table}

% \begin{table}
% \centering
%         \renewcommand{\arraystretch}{0.8}
%         \begin{threeparttable}
%         \begin{tabular}{@{}ccc@{}}
%         \toprule
%         Allegro Hand & iiwa Arms & Panda Arms \\
%         \midrule
%         0.721 & 0.65 & 0.803\\
%         % Task Horizon (s) & 25 & 280 & 50 \\
%         \bottomrule
%         \end{tabular}
%         % \begin{tablenotes}
%         % \itme{*} 
%         % \end{tablenotes}
%         \end{threeparttable}
%         \caption{Success rates of trajectory optimization under random perturbations in physical parameters and object initial conditions. }
%         \label{tab:trajopt_success_rate}
%         \vspace{0.5em}
% \end{table}



% \begin{figure*}[t]
% \centering
% \includegraphics[width=0.7\textwidth]{figures/aug_traj_den.png}
% 	\caption{\textbf{Distribution of object trajectories generated from a single demonstration}. The original demonstration (orange) is locally perturbed and augmented to about 100 dynamically feasible contact-rich trajectories (blue) for each system. The density map represents the object pose distribution of the generated trajectories in the specific 2-dimensional slices.}
%     \label{fig:aug_data_distribution}
% \end{figure*}

% \begin{figure*}[t]
% \centering
% \includegraphics[width=0.9\textwidth]{figures/aug_traj_snapshots.png}
% 	\caption{\textbf{Snapshots of trajectories generated from a single demonstration}. The original demonstration (orange) is locally perturbed and augmented to about 100 dynamically feasible contact-rich trajectories (blue) for each system. The density map represents the object pose distribution of the generated trajectories in the specific 2-dimensional slices.}
%     \label{fig:aug_data_distribution}
% \end{figure*}
\begin{figure*}[t]
\centering
\includegraphics[width=1.0\textwidth]{figures/density_snapshots_aug_traj.png}
	\caption{\textbf{Distribution and snapshots of trajectories generated from a single demonstration.} (a) The original demonstration (orange) is locally perturbed and augmented to about 100 dynamically feasible contact-rich trajectories (blue) for each system. The density map represents the object pose distribution of the generated trajectories in the specific 2-dimensional slices. (b) Snapshots of 30 dynamically feasible trajectories under random physical parameters and object initial poses for bimanual iiwa arms are visualized.}
    \label{fig:aug_data_distribution}
\end{figure*}


\underline{\textbf{Dynamic Feasibility}}
While kinematic motion retargeting can generate visually plausible robot and object trajectories, these trajectories often lack dynamical consistency due to the differences in physical parameters and embodiment between the human demonstrator and the target robot. To illustrate this, we replay the kinematically retargeted trajectories of the original 24 human demos and record the success rates for each system in Table \ref{tab:kin_success_rate}. Furthermore, we randomly sample object sizes and perturbations of initial object poses according to Table \ref{tab:domain_randomization} and roll out the nominal kinematically retargeted trajectories. Some trajectories still succeed under certain perturbations thanks to caging grasps or other strategies that encourage robustness during the human demonstration. For all the systems, the successful rollouts are relatively short, manipulating the object to the goal pose within only 1 or 2 rotations. 
% Notably, the successful trajectories for the iiwa and Panda arms vary significantly, despite being generated from the same initial set of demonstrations.

The low success rate of purely kinematically retargeted trajectories highlights the importance of trajectory optimization for locally refining the demos for the particular embodiments and physical parameters. Before trajectory optimization, the floating Allegro hand lightly touches the cube and easily loses contact when rotating it clockwise (demonstrated in Fig. \ref{fig:trajopt_unittest}a). After trajectory optimization, the hand increases the contact area, establishing a stable grip for rotation. In Fig. \ref{fig:trajopt_unittest}b, similar behavior that encourages contact can be observed for the bimanual iiwa arms: the demo trajectory tries to rotate the box clockwise only using a single arm, while trajectory optimization encourages the other arm to help hold the box and reorient the box more stably. These refinements that encourage contact are particularly helpful when the object is heavier or smaller, or when the friction coefficients are lower than expected. In addition, replaying the kinematically retargeted trajectory often fails when the object pose deviates slightly from the demonstration, driving the object out of reach (visualized in Fig. \ref{fig:trajopt_unittest}c). In contrast, trajectory optimization 
%stabilizes the system in a vicinity around the demonstration, ensuring higher success rates even when the object is perturbed
accounts for the system’s true dynamics and can adjust the robot’s actions accordingly. The success rates of trajectory optimization under random perturbations in physical parameters and object initial conditions for each system are recorded in Table \ref{tab:kin_success_rate}.

\begin{figure*}[t]
    \centering
    \includegraphics[width=0.9\linewidth]{figures/policy_rollouts.png}
    \caption{\textbf{Policy rollouts for different embodiments.} The object manipulation task requires the robots to frequently make and break contact with the object. It also requires precise control of the robot since small deviations in positions can result in missing contact interactions and lead to task failure. } 
    \label{fig:policy_rollouts}
\end{figure*}

\underline{\textbf{Cross-Embodiment Generalization}} We demonstrate that a single set of human demonstrations can be effectively repurposed to generate dynamically consistent, contact-rich trajectories across different robotic embodiments with varying task horizons. Specifically, human demonstrations involving two index fingers manipulating a small cube are retargeted to fixed-base bimanual Kuka LBR iiwa and Franka Emika Panda arms manipulating a larger box (visualized in Fig. \ref{fig:kinematic_retargeting}). This approach addresses key challenges in data collection for contact-rich tasks: directly teleoperating two real robot arms to flip a large box would be both physically demanding and cost-prohibitive due to hardware latency, limited feedback, and the embodiment gap--differences in kinematic structure, degrees of freedom, and workspace between human and robotic arms. In contrast, performing the same task on a smaller scale using human fingers is more intuitive, reduces physical effort, and enables faster, more consistent demonstration collection.

The iiwa and Panda arms differ in contact geometry, velocity limits, and joint constraints, all of which are explicitly modeled within the trajectory optimization framework described in \eqref{eq:predictive_control}. For safe hardware deployment, we enforce conservative velocity limits on the iiwa arms, while only applying soft velocity regularization on the Panda arms in simulation to allow for more aggressive motions.


\underline{\textbf{Data Diversity}} 
Trajectory optimization efficiently augments a single demonstration to a wide distribution of trajectories with locally perturbed physical parameters and initial conditions as visualized in Fig. \ref{fig:aug_data_distribution}. The diverse states in the generated dataset cover a larger training distribution and encourage smoother learned policies, as will be discussed in the next section.
\begin{figure*}[t]
    \centering
    \includegraphics[width=0.9\linewidth]{figures/policy_failure.png}
    \caption{\textbf{Failure cases of baselines.} (a) The baseline policy trained on the original 24 demonstrations for the floating Allegro hand frequently misses contact or gets stuck on the cube. (b-c) The baseline policies for the bimanual robot arms often exhibit jittery motion, resulting in loss of contact, the box being kicked out of reach, or the robot arms running into and getting stuck on the box surface. } 
    \label{fig:policy_failure}
\end{figure*}

\begin{figure}
\centering
\includegraphics[width=0.42\textwidth]{figures/success_rate.png}
	\caption{Success rates of policy evaluation in simulation and hardware. }
	\label{fig:success_rate}
    \vspace*{-0.4cm}
\end{figure}

% \begin{figure}
% \centering
% \includegraphics[width=0.48\textwidth]{figures/jitteriness.png}
% 	\caption{\textbf{Joint angles of bimanual iiwa arms over time. } Each line represents the trajectory of a different joint of the iiwa arms. The policy trained on augmented datasets (b) demonstrates significantly smoother motion compared to the baseline policy (a). }
% 	\label{fig:jitteriness}
% \end{figure}

\begin{figure*}[t]
    \centering
    \includegraphics[width=0.9\linewidth]{figures/hardware_rollout.png}
    \caption{\textbf{Policy rollouts on hardware.} The fixed-base bimanual iiwa arms perform a sequence of coordinated rolling, pitching, and yawing actions to reorient the box to the goal pose. } 
    \label{fig:hardware_rollout}
\end{figure*}

\section{Behavior Cloning Experiments}
We illustrate our framework's capability to efficiently produce diverse, high-quality contact-rich datasets for training behavior cloning policies across multiple robotic platforms, including the floating Allegro hand and the bimanual Panda arms in simulation as well as bimanual iiwa arms on hardware. We show that policies trained on the generated data generalize to a wide distribution of physical parameters and initial conditions, and are much more robust and performant than the ones trained only on the original demonstrations. 
\subsection{Policy Evaluation in Simulation}
\label{subsec:policy_eval_sim}
From only 24 human demonstrations, our data generation pipeline can efficiently generate thousands of dynamically feasible contact-rich trajectories using trajectory optimization. We train state-based diffusion policies \cite{chi2023diffusion} on the 24 original demo trajectories, as well as 500 and 1000 generated trajectories. While our method is compatible with any Behavior Cloning algorithm, we adopt diffusion policies due to its recent success in contact-rich tasks \cite{chi2024universal, zhu2024should, li2024planning}. Fig. \ref{fig:policy_rollouts} visualizes the policy rollouts. We evaluate the performance by conducting 48 policy rollouts for each embodiment in simulation and record the success rates in Fig. \ref{fig:success_rate}. The success criteria are the same as specified in the trajectory optimization experiments.
%For policy evaluation, we visualize the initial states for all evaluation episodes, typical failure cases of baseline policies, and final object pose errors in Fig. \ref{fig:policy_eval}.  
% and validate that the generated data help improve the policy's robustness and generalizability.

\subsubsection{Floating Allegro Hand} 
While the human demonstrator completes the task in approximately 5 seconds on average in the virtual reality environment, the demonstration trajectories are temporally scaled by a factor of 2.5 to ensure smoother, dynamically feasible motions on the floating Allegro hand, which is subject to velocity limits. We define the task horizon as 25 seconds to allow the policy sufficient time to recover from missed contacts and other errors during the execution. The task complexity arises from the 22-dimensional action space of the Allegro hand and the long-horizon nature of the task, which requires a sequence of coordinated rolling, pitching, and yawing actions to reorient the cube to an upright position. These factors together present significant challenges for traditional model-based planners without guidance.

The baseline behavior cloning policy trained on the original set of 24 demonstrations achieves a success rate of $10 / 48 = 21\%$ and exhibits significant jittery behavior when encountering out-of-distribution states. The workspace, characterized by diverse object orientations and translations, is sufficiently large that minor deviations during policy rollouts often drive the trajectory out of the demonstrated distribution. Common failure modes include the Allegro hand repeatedly missing contact with the cube or becoming stuck on its surface while attempting reorientation (visualized in Fig. \ref{fig:policy_failure}a), which often result in the object being trapped in intermediate orientations. In contrast, policies trained on the expanded dataset generated by our pipeline demonstrate a higher likelihood of re-establishing contact with the object after initial misses, resulting in significantly improved success rates up to $39 / 48 = 81\%$.

\begin{figure*}[t]
    \centering
    \includegraphics[width=0.9\linewidth]{figures/hardware_eval.png}
    \caption{\textbf{Policy failure and recovery on hardware.} The baseline policy frequently (a) gets stuck on the box surface when small deviations from the demonstration trajectories occur, and (b) struggles to recover from out-of-distribution states, where the object is never intentionally lifted for accomplishing the task in the generated dataset. Policies trained on augmented datasets (c) sometimes fail due to unmodeled collision geometry, but (d) can recover from undesired sliding by employing firmer grasps found by trajectory optimization. } 
    \label{fig:hardware_eval}
\end{figure*}
\subsubsection{Bimanual Robot Arms}
The baseline policy trained on the original set of 24 human demonstrations achieves a success rate of $27 / 48 = 56\%$ on the bimanual iiwa system. We hypothesize that the restrictive velocity limits encourage more quasi-static behavior, leading to longer trajectories with a higher density of state-action pairs in the training data. In contrast, the baseline policy yields a success rate of $14/48=29\%$ on the bimanual Panda system, likely due to the more dynamic nature of the learned behavior under its looser velocity constraints. Both baseline policies exhibit remarkably jittery motion, frequently kicking the box out of reach, losing contact, or running into and getting stuck on the box surface during reorientation (visualized in Fig. \ref{fig:policy_failure}b and c). Policies trained on the augmented dataset, however, generate significantly smoother trajectories and are capable of re-establishing contact with the object after initial misses, resulting in as high as $44 / 48 = 92\%$ success rates for bimanual iiwa arms and $42 / 48 = 87.5\%$ for bimanual Panda arms. Additionally, the learned policies capture multimodal behaviors observed in the original human demonstrations, such as rotating the box either clockwise or counterclockwise for similar object poses. 


\subsection{Policy Evaluation on Hardware}
We zero-shot deploy the trained policies on hardware for bimanual iiwa arms to flip a 30 cm cubic box on a table (Fig. \ref{fig:hardware_rollout}). An OptiTrack motion capture system is employed to estimate the object pose. The baseline behavior cloning policy only achieves $6/23=26\%$ success rate, with most successful rollouts being relatively short-horizon, involving only 1 or 2 rotations. Common failure modes of the baseline policy include: 1) deviation from the demonstration trajectory, causing the arms to collide with the box surface (Fig. \ref{fig:hardware_eval}a), and 2) significant box sliding during rolling, resulting in the policy encountering out-of-distribution states and failing to recover (Fig. \ref{fig:hardware_eval}b). In contrast, as shown in Fig. \ref{fig:success_rate}b, the policy trained on 500 generated trajectories achieves $17 / 23 = 74\%$ success rate, while the policy trained on 1000 generated trajectories achieves $16/23=70\%$ success rate. Despite occasional box sliding during rolling, these policies demonstrate an improved ability to stabilize the box by using one arm to hold the opposite side more firmly to prevent further sliding (Fig \ref{fig:hardware_eval}d). However, as visualized in Fig \ref{fig:hardware_eval}c, both policies trained on the augmented datasets exhibit failure modes originating from unmodeled collision geometries on iiwa arms, which lead to significant undesired yaw motions of the box during pitch actions.\looseness=-1
\section{Conclusion and future work}
In this study, we examined the ability of LLMs to produce self-generated counterfactual explanations (SCEs).
We design a prompt-based setup for evaluating the efficacy of \SCEs.
Our results show that LLMs consistently struggle with generating valid \SCEs. In many cases model prediction on a \SCE does not yield the same target prediction for which the model crafted the \SCE.
Surprisingly, we find that LLMs put significant emphasis on the context---the prediction on \SCE is significantly impacted by the presence of original prediction and instructions for generating the \SCE.
Based on this empirical evidence, we argue that LLMs are still far from being able to explain their own predictions counterfactually.
Our findings add to similar insights from recent studies on other forms of self-explanations~\cite{lanham2023measuring,tanneru2024quantifying}.



Our work opens several avenues for future work. Inspired by counterfactual data augmentation~\cite{sachdeva2023catfood}, one could include the counterfactual explanation capabilities a part of the LLM training process. This inclusion may enhance the counterfactual reasoning capabilities of the LLM. Follow ups should also explore the effect of prompt tuning, specifically, model-tailored prompts for generating \SCEs. These approaches might lead to better quality \SCEs.


We limited our investigation to open source models of upto 70B parameters. Extending our analysis to larger and more recent models, \eg, DeepSeek R1 671B, and closed source models like OpenAI o3 would be an interesting avenue for future work.

Finally, our experiments were limited to relatively simple tasks: classification and mathematics problems where the solution is an integer. This limitation was mainly due to the fact that it is difficult to automatically judge validity of answers for more open-ended language generation tasks like search and information retrieval. Scaling our analysis to such tasks would require significant human-annotation resources, and is an important direction for future investigations.

% Merged into conclusion
% This work identifies signal collapse as a critical bottleneck in one-shot neural network pruning. Performance loss in pruned networks is due to \textbf{signal collapse} in addition to the removal of critical parameters. We propose \textbf{REFLOW} (\textbf{Re}storing \textbf{F}low of \textbf{Low}-variance signals), a simple yet effective method that mitigates signal collapse without computationally expensive weight updates. By focusing on signal preservation, REFLOW highlights the importance of mitigating signal collapse in sparse networks and enables magnitude pruning to match or surpass state-of-the-art one-shot pruning methods such as CHITA, CBS, and WF.

REFLOW consistently achieves state-of-the-art accuracy across diverse architectures, restoring ResNeXt-101 from under 4.1\% to 78.9\% top-1 accuracy at 80\% sparsity on ImageNet. Its lightweight design makes it a practical solution for both research and deployment, delivering high-quality sparse models without the overhead of traditional approaches. These findings challenge the traditional emphasis on weight selection strategies and underscore the critical role of signal propagation for achieving high-quality sparse networks in the context of one-shot pruning.



% \section*{Conclusion}
This paper aims to enhance our understanding of the computational complexity of computing various Shapley value variants. We found that for various ML models --- including decision trees, regression tree ensembles, weighted automata, and linear regression --- both local and global interventional and baseline SHAP can be computed in polynomial time under HMM modeled distributions. This extends popular algorithms, such as TreeSHAP, beyond their empirical distributional scope. We also establish strict complexity gaps between the various SHAP variants (baseline, interventional, and conditional) and prove the intractability of computing SHAP for tree ensembles and neural networks in simplified scenarios. Overall, we present SHAP as a versatile framework whose complexity depends on four key factors: \begin{inparaenum}[(i)] \item model type, \item SHAP variant, \item distribution modeling approach, \item and local vs. global explanations\end{inparaenum}. We believe this perspective provides deeper insight into the computational complexity of SHAP, paving the way for future work.




%We believe that our framework provides a more intricate understanding of SHAP computation complexity across different models, distributions, and variants, paving the way for further research.

Our work opens promising directions for future research. First, expanding our computational analysis to other SHAP-related metrics, such as asymmetric SHAP~\citep{frye20} and SAGE~\citep{covert2020understanding}, would be valuable. Additionally, we aim to explore more expressive distribution classes and relaxed assumptions beyond those in Section \ref{sec:tractable} while maintaining tractable SHAP computation. Finally, when exact computation is intractable (Section \ref{sec:intractable}), investigating the approximability of SHAP metrics through approximation and parameterized complexity theory~\citep{downey2012parameterized} is an important direction.

%Our work opens several promising avenues for future research on the computational properties of explainable AI methods, with a particular focus on SHAP. First, it would be interesting to broaden the computational analysis conducted in this work to include other popular SHAP-related metrics in the literature, such as asymmetric SHAP \cite{frye20} and SAGE \cite{covert2020understanding}. Also, in the future, we aim to explore more expressive distribution classes and relaxed distributional assumptions—extending beyond those examined in Section \ref{sec:tractable} —that still yield tractable SHAP computation. Finally, when exact computation proves intractable (Section \ref{sec:intractable}), it is worthwhile to theoretically investigate the question of the approximability of computing the SHAP metrics across various configurations, through the lens of approximation and parametrized complexity theory \cite{arora2009computational}.

%This paper aims to deepen our understanding of the computational complexity involved in obtaining different Shapley value variants. We found that for a variety of ML models, including decision trees, tree ensembles for regression, weighted automata, and linear regression models — computing both local and global interventional and baseline SHAP can be done in polynomial time when distributions are modeled by HMMs. This extends the distributional scope of popular algorithms like TreeSHAP, which is limited to empirical distributions. Additionally, we demonstrate a strict complexity gap between SHAP variants, showing that interventional and baseline SHAP can be strictly easier to compute than conditional SHAP. Despite these positive results, we uncovered intractability for various SHAP variants in neural networks and tree ensembles. Finally, we provided generalized complexity relations across SHAP variants. We believe that our framework offers a deeper understanding of the complexity involved in computing SHAP across various variants, models, distributions, as well as in both local and global computations, laying the groundwork for future research.

\ifpreprintversion
\section*{Acknowledgments}
This research was supported by the Department of Defense
Multidisciplinary Research Program of the University Research
Initiative (MURI) under contract W911NF-21-1-0322, and by NSF under grants CNS-2312875 and CNS-2331081.

\fi

\bibliographystyle{ACM-Reference-Format}
\bibliography{main}

\appendix
\section{Bound Analysis on Number of Malicious Clients per Round}
\label{app:bound_analysis}

Let \( N \) denote the total number of clients and \( M \) the number of malicious clients. The ratio of malicious clients is given by the Malicious Client Ratio (MCR), \( \rho = \frac{M}{N} \). In each round of Federated Learning (FL), we sample \( C \) clients randomly from the \( N \) total clients.
%
Let $\mathcal{M}$ be the number of malicious clients (out of $C$) sampled in some round. We are interested in finding a lower bound on the probability that there are more malicious clients than benign ones in a round of FL, \ie \( P\left(\mathcal{M} \geq \frac{C}{2}\right) \).

The selection of malicious clients can be modeled as a Binomial distribution where probability of success is the MCR ($\rho$). Thus,  our desired probability can be written in terms of the CDF $F$ of this distribution:
\begin{align}
    P\left(\mathcal{M} \geq \frac{C}{2}\right) = 1 - F\left(\frac{C}{2}, C, \rho\right).
\end{align}
Using a Chernoff bound \citep{arratia1989tutorial}, we can thus obtain a lower bound on the probability that malicious clients outnumber benign ones in a given round as:
\begin{align}
    P\left(\mathcal{M} \geq \frac{C}{2}\right) &\geq 1 - \exp\left(-C \cdot \left(\frac{1}{2}\left(\ln\frac{1}{2\rho} + \ln\frac{1}{2(1-\rho)}\right)\right)\right) \\
    & = 1 - \exp\left(C \cdot \frac{1}{2} \ln\left(4\rho(1-\rho)\right)\right) \\
    & = 1 - \left(4\rho(1-\rho)\right)^{\frac{C}{2}}.
\end{align}

% \section{Calculations}
\label{sec:calculations}

\subsection{Chernoff Bound Analysis}
\label{sec:chernoff_appendix}

In this section, we provide the detailed steps used to compute the probability bound for the number of malicious clients exceeding a certain threshold in any given round. Specifically, we apply the Chernoff bound to analyze the probability that the number of malicious clients \( M \) in a randomly selected subset of clients exceeds half the total number of selected clients \( C \).

The Chernoff bound provides the following inequality for the probability that \( M \) exceeds \((1 + \delta)\) times its expected value:
\begin{equation}
    \mathbb{P}\left(M \geq (1 + \delta) \cdot \rho \cdot C \right) \leq \exp\left(-\frac{\delta^2 \cdot \rho \cdot C}{2 + \delta}\right),
\end{equation}
where:
\begin{itemize}
    \item \( \rho \) is the malicious client ratio (MCR), i.e., the fraction of total clients that are adversarial.
    \item \( C \) is the total number of clients selected in each round.
    \item \( \delta \) controls how much larger \( M \) is compared to its expected value \( \mathbb{E}[M] = \rho \cdot C \).
\end{itemize}

% \noindent \textbf{Step 1: Setting the threshold}
\shortsection{Step 1: Setting the threshold}
We aim to compute the probability that the number of malicious clients \( M \) exceeds 50\% of the selected clients, i.e., \( M \geq \frac{C}{2} \). Given \( C = 20 \) and \( \rho = 0.4 \), we want:
\[
(1 + \delta) \cdot \rho \cdot C = 10.
\]
Substituting the values of \( \rho \) and \( C \):
\begin{align}
    (1 + \delta) \cdot 0.4 \cdot 20 & = 10,     \nonumber
\end{align}
which gives us $\delta = 0.25$.
% \[
% (1 + \delta) \cdot 8 = 10,
% \]
% \[
% 1 + \delta = \frac{10}{8} = 1.25,
% \]
% \[
% \delta = 1.25 - 1 = 0.25.
% \]

\shortsection{Step 2: Applying the bound}
% \noindent \textbf{Step 2: Applying the bound}
Now, we plug in \( \delta = 0.25 \), \( \rho = 0.4 \), and \( C = 20 \) into the Chernoff bound:
\begin{align}
    \mathbb{P}(M \geq 10) & \leq \exp\left(-\frac{(0.25)^2 \cdot 0.4 \cdot 20}{2 + 0.25}\right) \\
    & = \approx 0.8007374
\end{align}
% \[
% \mathbb{P}(M \geq 10) \leq \exp\left(-\frac{(0.25)^2 \cdot 0.4 \cdot 20}{2 + 0.25}\right),
% \]
% \[
% = \exp\left(-\frac{0.5}{2.25}\right),
% \]
% \[
% = \exp\left(-0.22222\ldots\right),
% \]
% \[
% \approx 0.8007374.
% \]
Thus, the probability that the number of malicious clients exceeds half of the selected clients in a given round is approximately 0.80, indicating that such an event is relatively likely under this configuration. This highlights the need for robust mechanisms that can handle varying malicious client ratios across rounds.

\section{Normal Approximation for Probability of Malicious Majority (Using M and B)}
\label{appendix:normal-approx-MB}

Assume the following setup:
\begin{itemize}
  \item $N$ = total number of clients.
  \item $M_{\text{tot}}$ = total number of malicious clients, where
        $
            M_{\text{tot}} 
            = (\text{MCR}) \times N.
        $
  \item $B_{\text{tot}} = N - M_{\text{tot}}$ = total number of benign (non-malicious) clients.
  \item Each client is selected (independently) with probability 
        $
          p = \frac{C}{N}.
        $
        Thus, the expected number of clients selected in each round is $C$.
\end{itemize}

Let
\[
    M \;\sim\; \mathrm{Binomial}\bigl(M_{\text{tot}},\, p\bigr),
    \quad
    B \;\sim\; \mathrm{Binomial}\bigl(B_{\text{tot}},\, p\bigr),
\]
where 
\[
   B_{\text{tot}} \;=\; N - M_{\text{tot}}.
\]
The random variables $M$ and $B$ are assumed independent, since each client (malicious or benign) is selected via its own independent Bernoulli$(p)$ trial.

\subsection*{Goal}
We want to find the probability that malicious clients \emph{form a strict majority} among the selected clients, that is:
\[
   \Pr\bigl(M > B\bigr).
\]
For large $N$, we can apply the Central Limit Theorem to approximate these binomial distributions by normal distributions:
\[
    M 
    \;\approx\; 
    \mathcal{N}\bigl(\mu_{M}, \;\sigma_{M}^2\bigr),
    \qquad
    B 
    \;\approx\;
    \mathcal{N}\bigl(\mu_{B}, \;\sigma_{B}^2\bigr),
\]
where
\[
    \mu_{M} 
    \;=\; 
    M_{\text{tot}}\,p 
    \;=\; 
    \bigl(\text{MCR}\times N\bigr) 
    \,\frac{C}{N}
    \;=\;
    (\text{MCR}) \times C,
    \quad
    \sigma_{M}^2 
    \;=\;
    M_{\text{tot}}\, p\,(1-p),
\]
\[
    \mu_{B} 
    \;=\;
    B_{\text{tot}}\,p
    \;=\;
    \bigl(N - M_{\text{tot}}\bigr)\,\frac{C}{N}
    \;=\;
    \bigl(1 - \text{MCR}\bigr)\,C,
    \quad
    \sigma_{B}^2 
    \;=\;
    B_{\text{tot}}\,p\,(1-p).
\]
Since $M$ and $B$ are independent, the difference $(M - B)$ is approximately normally distributed with
\[
    M - B 
    \;\approx\;
    \mathcal{N}\bigl(\mu_{M} - \mu_{B},\;
                     \sigma_{M}^2 + \sigma_{B}^2 \bigr).
\]
Observe that
\[
    \mu_{M} - \mu_{B}
    \;=\;
    (\text{MCR}\cdot C) - \bigl((1 - \text{MCR})\cdot C\bigr)
    \;=\;
    C\,\bigl(2\,\text{MCR} - 1\bigr),
\]
and
\[
  \sigma_{M}^2 + \sigma_{B}^2
  \;=\;
  M_{\text{tot}}\,p\,(1-p)
  \;+\;
  B_{\text{tot}}\,p\,(1-p)
\]
\[
  =\;
  \bigl(M_{\text{tot}} + B_{\text{tot}}\bigr)\,p\,(1-p)
  \;=\;
  N\,\frac{C}{N}\,\Bigl(1 - \tfrac{C}{N}\Bigr)
\]
\[
  \;\approx\;
  C.
\]
assuming $C \ll N$. Thus,
\[
    M - B
    \;\approx\;
    \mathcal{N}\Bigl( C\,\bigl(2\,\text{MCR} - 1\bigr), \; C \Bigr).
\]
We want $\Pr(M > B)$, i.e.\ $\Pr(M - B \ge 0)$. We standardize the variable:
\[
  \Pr\bigl(M - B \;\ge\; 0\bigr)
  \;=\;
  \Pr\!\Bigl(\,
    \frac{M - B \;-\; \bigl[C\,(2\,\text{MCR} - 1)\bigr]}{\sqrt{C}}
    \;\ge\;
    \frac{-\,C\,(2\,\text{MCR} - 1)}{\sqrt{C}}
  \Bigr).
\]

If $Z$ is a standard normal random variable, then
\[
  \Pr\bigl(M - B \;\ge\; 0\bigr)
  \;\approx\;
  \Pr\!\Bigl(
    Z \;\ge\; -\,\frac{C\,(2\,\text{MCR} - 1)}{\sqrt{C}}
  \Bigr)
\]
\[
  =
  \Pr\!\Bigl(
    Z \;\le\; \frac{C\,(2\,\text{MCR} - 1)}{\sqrt{C}}
  \Bigr).
\]


Because $1 - \Phi(-x) = \Phi(x)$ for the standard normal CDF $\Phi$, it follows that
\[
    \Pr\bigl(M > B\bigr)
    \;\approx\;
    \Phi\!\Bigl(\sqrt{C}\,\bigl[2\,\text{MCR} - 1\bigr]\Bigr).
\]

\paragraph{Interpretation.}
\begin{itemize}
    \item If $\text{MCR} < 0.5$, then $2\,\text{MCR} - 1 < 0$, and the argument of $\Phi(\,\cdot\,)$ is negative and becomes more negative as $C$ increases. Hence, $\Pr(M > B)$ goes to $0$.
    \item If $\text{MCR} = 0.5$, then the argument of $\Phi$ is zero, so the probability is $0.5$.
    \item If $\text{MCR} > 0.5$, then $2\,\text{MCR} - 1 > 0$, and as $C$ increases, the argument of $\Phi$ becomes large positive. Hence, $\Pr(M > B)$ goes to $1$.
\end{itemize}

This derivation shows that, for large $N$ and not-too-large $C$ (so $p = C/N \ll 1$), the probability of a \emph{malicious majority} in a binomially sampled subset is well approximated by
\[
    \Pr\bigl(\text{Malicious majority}\bigr)
    \;\approx\;
    \Phi\Bigl(\sqrt{C}\,\bigl[2\,\text{MCR} - 1\bigr]\Bigr),
\]
where $\Phi(\,\cdot\,)$ is the standard normal cumulative distribution function.

\section{Related Work}
% \jianfei{have updated.}


\subsection{Cinemagraphs and Looped Video Generation}
Our task is similar to cinemagraphs, which aim to produce looping videos by manipulating an input video manually. However, the manual creation of cinemagraphs is a time-consuming process, even for professional artists.
Previously, learning-based methods faced difficulties in generating or editing an entire video. As a result, prior techniques only applied to specific patterns to create cinemagraphs, for example, water~\cite{holynski2021animating, mahapatra2023synthesizing, liao2013automated}, periodic pattern~\cite{endless_loops}, portrait~\cite{sadtalker, 10.1111/cgf.12147, bertiche2023blowing}, panoramic~\cite{agarwala2005panoramic, 10.1145/3144455}. As for representative work,
Endless Loops~\cite{endless_loops} utilizes CRF to compute loop shifts, and it can only work on the repeated pattern.  \cite{holynski2021animating} presents an image animation method to generate the moving water from a single image utilizing Eulerian motion fields. Text-to-cinemagraphs~\cite{mahapatra2023synthesizing} further extend it by the pre-trained text-to-image stable diffusion model. Several methods ~\cite{generative_image_dynamic, niu2024mofa, shi2024motion} present a two-stage framework to generate the video with trajectory control. However, they only work on certain object types or need manual trajectory design.
While LoopAnimate~\cite{loopanimate} employs multi-stage training and symmetric guidance to achieve looped generation, their generated results are too still.
Besides, we can naively utilize the generative frame interpolation methods~\cite{wang2024generative,xing2024tooncrafter} based on the video diffusion model for a generation by setting the same start and end keyframes. However, since the original frame interpolation model is not trained for cinemagraph, the generated results might also be still. Besides, these methods involve additional larger-scale training for generation, which might cause forgetting problems.
Differently, we directly generate the looped video from the text description, yet with better visual effects, such as the whole movement of the camera and object motion.

\begin{figure*}[t]
\centering
\includegraphics[width=\textwidth]{figures/PLS/latentshift.pdf}
\vspace{-2em}
\caption{\textbf{Latent Shift for looping video generation}. Taking 4 latent toys pre-trained Video Diffusion Models~(VDM) as an example, we build a latent cycle and shift the start point in each denoising step in inference for text-guided looping video generation. Notice that, the shifting is conducted in the latent space, we emit the latent encoder and decoder for easy understanding.}
\vspace{-1em}
\label{fig:PLS}
\end{figure*}

\subsection{Video Generation in Diffusion Model Era}
Due to the stabilizing training process of the Diffusion Model~\cite{ddim, ddpm}, video generation has had a big breakthrough in recent years. Eary works~\cite{imagen-video, make-a-video} directly generate high-resolution videos from cascade models of spatial and temporal layers directly in pixel space.
On the other hand, utilizing the pre-trained text-to-image models~\cite{ldm}, \ie, Stable Diffusion, as the base model, many works try to add additional layers to keep temporal consistency. \cite{videocrafter1, modelscope, magic-video} add temporal attention modules to the base model and train in an end-to-end fashion. Besides, \cite{animatediff} finds that training the models by temporal layers only has a better visual quality. \cite{videocrafter2} proposes a method to increase the visual qualities by a two-stage image and video joint training process. However, these methods only create a short video with limited motions, which restricts its applications in real-world cases. 
Besides the text-to-video diffusion model, new works also train image-to-video models for generation, which is also related to our task. For example, Stable Video Diffusion~\cite{svd} fine-tunes the text-to-video diffusion model with a high-quality data pipeline. DynamicCrafter~\cite{xing2025dynamicrafter} shares a similar idea and trains on the video diffusion model. ToonCrafter \cite{xing2024tooncrafter} and Generative image in-between \cite{wang2024generative} are further finetuning the image-to-video models for the generative frame interpolation. However, as we discussed before, directing utilizing the frame interpolation methods for our task might have issues with the too-still motion.
Recently, Sora \cite{sora} has made a big step in video generation via denoising transformers~(DiT~\cite{DiT}), showing the scalability and advantages. 
Thus, the more recent video generation methods~\cite{mochi, cogvideox,hunyuanvideo} are based on the DiT structure, which has better motion and temporal consistency than previous methods.  

Besides, since these pre-trained large diffusion models are trained from larger-scale datasets, we can repurpose these models for the new task without training. For instance, in the field of image/video editing, works such as Prompt to Prompt~\cite{prompt-to-prompt}, FateZero \cite{fatezero}, and MasaCtrl~\cite{masactrl} have achieved zero-shot editing through attention control. Meanwhile, there also contains some methods that have provided foundational discoveries for zero-shot editing~\cite{yu2023animatezero, freedom} and improving the performance without additional training~\cite{freeinit, freeu}. In this paper, we utilize the most popular open-sourced DiT-based video generation model, \ie, CogVideoX~\cite{cogvideox}, as the base model for looped video generation in a training-free manner. 

% Supported by larger-scale datasets and 3D VAE~\cite{zhao2024cvvae}, this model can generate results with higher image quality and slightly longer frames than previous methods.
% However, the videos generated by these methods still do not meet the demand for longer frames. In this paper, we aim to utilize the pre-trained T2V model for longer video generation in a training-free manner.
% More recently, the denoising transformer-based DiT~\cite{DiT} architecture has shown superior performance on this task. Powered by the larger-scale dataset and 3D VAE~\cite{zhao2024cvvae}, this model can generate longer results than previous methods. However, these methods only can generate short video clips with a single action. In this paper, we aim to utilize the pre-trained single prompt T2V model for the multi-prompt video generation in a training-free manner.


\subsection{Longer Video Generation in Diffusion Model}
\label{longer_video_generation}
Our looping videos can be considered an infinitely longer video generation. In current methods, due to the limited latent length in training the pre-trained text-to-video generation models, several methods are proposed to modify the denoising process of the original diffusion model for new purposes. For the longer video generation, Gen-L-Video~\cite{gen-l-video} uses the weighted sum of different short latent segments in the overlapping area to alleviate the inter-frame continuity issue. However, this method significantly increases the inference time and can lead to smooth transitions between frames. 
FreeNoise~\cite{freenoise} introduces a shuffled latent sequence design and uses attention-based weighting to maintain visual consistency in long videos. However, since the latent changes only occur in the shuffling, the resulting video motion may appear too static and is prone to out-of-memory~(OOM) errors. FIFO~\cite{fifo} uses diagonal denoising for long video generation, maintaining the consistency and coherence of the video. However, there is a training inference gap for reasoning at different noise levels, and it lacks global information modeling. 
Video-Infinity~\cite{video-infinity} uses distributed inference to facilitate global and local information interaction, achieving video consistency while accelerating inference. However, an important limitation is the need for multiple GPUs to run simultaneously, and the quality of generating longer videos is not very good. 
DiTCtrl~\cite{ditctrl} utilizes a mask-based attention-sharing mechanism to maintain semantics, as well as a latent mixing strategy to achieve smooth transitions between video frames. However, this also brings about a significant amount of additional computational costs. 
These longer video generation methods change the combination of latent in the test time to control the generated content in the diffusion process, which inspired our looping video generation from text directly. 

% Due to the limited length of the pre-trained text-to-video generation, there are also methods for generating multi-prompt long videos.
% Gen-L-Video~\cite{gen-l-video} employs weighted summation of latents from different short video clips in overlapping regions to mitigate inter-frame continuity issues. However, this method significantly increases inference time and encounters abrupt transitions between frames. 
% FreeNoise~\cite{freenoise} introduces a shuffled latent sequence design and utilizes attention-based weighting to maintain visual consistency in long videos. Yet, due to latent variations only occurring through shuffling, the resulting video motions may appear overly static and are prone to out-of-memory~(OOM) errors. 
% Video-Infinity~\cite{video-infinity} uses distributed inference to facilitate global and local information interaction, achieving video coherence while speeding up inference. However, a significant limitation is its requirement for multi-GPUs to run, and the quality of the generated videos will also be influenced. FIFO~\cite{fifo} uses diagonal denoising for long video generation, maintaining video consistency and coherence. However, the delay in text guidance affects its application in multi-prompt video generation. 
% %\jianfei{Different from the above methods, our approach xxx}%

% \xiaodong{combine loop video generation to the long video generation}


% \subsection{Zero-shot Model Modification}
% With the vigorous development of diffusion models in the fields of image and video generation, pre-trained text-to-image/video models~\cite{ldm,videocrafter2,animatediff} have shown advanced performance as foundational models, enabling the use of these pre-trained models for open-domain related tasks. For instance, in the field of image/video editing, works such as Prompt to Prompt~\cite{prompt-to-prompt}, FateZero~\cite{fatezero}, and MasaCtrl~\cite{masactrl} have achieved zero-shot editing through attention control. Meanwhile, zero-shot editing methods~\cite{yu2023animatezero, freedom, freeinit, freeu} have provided foundational discoveries for zero-shot editing and offered new perspectives for zero-shot long video generation. 
% Inspired by these editing methods, some approaches~\cite{gen-l-video, freenoise, fifo, video-infinity, ditctrl} have also utilized zero-shot methods to achieve long video generation. This paper also follows suit, designing a zero-shot long video generation method.
% \break

% \clearpage
\section{Experimental Details}
\label{app:exp_details}

\subsection{Baselines}
\label{app:baseline_details}

% \begin{table*}[h]
%     \centering
%     \begin{tabular}{l|cccccc}
%     \toprule
%     Method & Data Distribution & Poisoning Start & Attack-Agnostic & Client Selection & Malicious Clients & DPR\\
%     \midrule
%          FLIP \citep{zhang2023flip} & iid, d(0.5) & After Convergence & No & Enforced & 40\% & ...\\
%          FL-Trust \citep{cao2021fltrust} &  iid, weird &  & & & 20\%\\
%          FoolsGold \citep{fung2018mitigating} & \\
%          MultiKrum \citep{blanchard2017machine} & \\
%          FLAME \citep{nguyen2022flame} & \\
%          DROP (Ours) & iid, d(1.0) & Beginning & Yes & Random & 20\% & ...\\
%     \bottomrule
%     \end{tabular}
%     \caption{\anshuman{In hindsight, this table can go to the Appendix.}}
%     \label{tab:my_label}
% \end{table*}

% Each of these methods aims to limit the influence of malicious updates during aggregation. However, most works provide limited or inconsistent details about their evaluation setups, particularly concerning client learning configurations such as learning rate, batch size, and the number of local training epochs. For instance, Median is evaluated on a simpler learning task (MNIST) and specifies only the total number of participating clients. The authors provide no details about the client learning setup, including learning rate, batch size, or number of local training epochs. Similarly, Multi-Krum reduces the impact of outliers using robust statistics but focuses primarily on how the data is partitioned among clients. While it does evaluate the method across different batch sizes, it lacks significant discussion of the broader local training setup. FLTrust adopts a trusted server-side reference model to filter anomalous updates. The authors report using a "combined" learning rate of 0.002, a batch size of 64, and a single local training epoch. In contrast, FoolsGold, which identifies and penalizes suspiciously similar client contributions, does not explicitly report the learning rate or number of local training epochs in its evaluation. Instead, it mentions using batch sizes of 10 or 50 depending on the dataset. FLAME, which leverages anomaly detection to flag potentially malicious gradients, describes the structure of the federation but provides no information about the local learning setup, such as learning rate, batch size, or training epochs.

% \begin{itemize}
%     \item \textbf{FLIP \citep{zhang2023flip}}: We adapt the original implementation and hyper-parameters. The original defense assumes that both the defense and malicious client poisoning are triggered after model convergence, citing interference in convergence if the malicious activity begins earlier. However, our observations show that targeted backdoors do not disrupt model convergence even if initiated at the beginning of FL training. Starting the defense at convergence is thus ineffective as the poisoning has already occurred. On the other hand, starting too early results in suboptimal performance. Therefore, we activate the defense after the first 10/3 rounds for CIFAR-10/EMNIST respectively.
% \end{itemize}

Each of the defense methods which were presented aims to limit the influence of malicious updates during aggregation. However, most works provide limited or inconsistent details about their evaluation setups, particularly concerning client learning configurations such as learning rate, batch size, and the number of local training epochs. For instance, Median \citep{yin2018byzantine} is evaluated on a simpler learning task (MNIST \cite{mnist}) and specifies only the total number of participating clients, without providing key details about the client learning setup, such as the learning rate, batch size, or the number of local epochs. Similarly, Multi-Krum \citep{blanchard2017machine} reduces the impact of outliers using robust statistics but primarily focuses on how the data is partitioned among clients. While it does evaluate the method across different batch sizes, it lacks a detailed discussion of the broader local training setup. FLTrust \citep{cao2021fltrust} adopts a trusted server-side reference model to filter anomalous updates. The authors report using a "combined" learning rate of 0.002, a batch size of 64, and a single local training epoch. In contrast, FoolsGold \citep{fung2018mitigating}, which identifies and penalizes suspiciously similar client contributions, does not explicitly report the learning rate or the number of local epochs in its evaluation, only mentioning batch sizes of 10 or 50 depending on the dataset. FLAME \citep{nguyen2022flame}, which employs anomaly detection to flag potentially malicious gradients, describes the structure of the federation but omits critical information about the local learning setup, such as learning rate, batch size, or the number of epochs. FLIP \citep{zhang2023flip}, on the other hand, presents a more detailed setup. We adapted the original implementation and hyperparameters for our evaluation. The original defense assumes that both the defense and malicious client poisoning are triggered after model convergence, citing interference in convergence if malicious activity begins earlier. However, our observations show that targeted backdoors do not disrupt model convergence even when initiated at the beginning of FL training. Thus, starting the defense only after convergence is ineffective since poisoning has already occurred by that point. On the other hand, initiating the defense too early leads to suboptimal performance. Therefore, we activate the defense after the first 10 rounds for CIFAR-10 and after the first 3 rounds for EMNIST to balance effectiveness and performance.


\subsection{DROP Parameters}
\label{sec:drop_params}

% Knowledge distillation, particularly in the context of model stealing attacks, is inherently imperfect and cannot replicate the target model exactly. As a result, a minor decrease in MTA is expected. To mitigate this and ensure convergence in the FL setting, the knowledge distillation component of DROP is applied every \(K\) rounds instead of every round, allowing the system to recover any lost MTA in intermediate rounds. For CIFAR-10, \(K = 5\) with a MAZE query budget of 5M per round. For EMNIST, \(K = 40\) with a MAZE query budget of 4M per round.

Knowledge distillation, particularly in the context of model stealing attacks, is inherently imperfect and cannot replicate the target model exactly, resulting in a minor decrease in MTA. To address this and ensure convergence in the FL setting, the knowledge distillation component is applied every \(K\) rounds instead of every round, allowing the system to recover lost MTA during intermediate rounds. The budget parameter in model stealing attacks and knowledge distillation determines the number of queries used to generate synthetic samples, which are then employed to guide the distillation process. A sufficient query budget ensures the generation of high-quality synthetic data that aligns closely with the target model’s decision boundaries, thereby enhancing the effectiveness of knowledge distillation. For CIFAR-10, we set \(K = 5\) with a query budget of 5M queries. For EMNIST, \(K = 40\) with a query budget of 4M queries. These values strike a balance between computational efficiency and the quality of the distilled global model.


\subsection{EMNIST Grid-Search}

\begin{figure}[ht]
    \includegraphics[width=.98\linewidth]{assets/emnist_undefended_asr.png}
    \caption{Visualizing the impact of the FL setup (particularly the learning-rate, batch-size, and number of epochs used by clients) on main-task accuracy attack success rate when poisoned clients aim to inject a targeted backdoor for the EMNIST dataset.}
    \label{fig:fl_setup_impact_emnist}
\end{figure}


In the same fashion as \cref{sec:fl_setup_matters} for CIFAR-10, we conduct a grid-search analysis over key hyperparameters, varying the client’s learning rate, batch size, and number of epochs on the EMNIST \citep{cohen2017emnistextensionmnisthandwritten} dataset.
Our findings in \cref{fig:fl_setup_impact_emnist} indicate that targeted backdoor attacks are more likely to succeed across a wider range of learning parameter combinations, with numerous setups yielding an ASR greater than 80\%. In \cref{tab:fl_setup_exps_emnist}, we highlight ten specific learning configurations where the attack achieves high ASR while maintaining a high MTA, underscoring the vulnerability of these setups to adversarial manipulation.

\begin{table}[ht]
    \centering
    \begin{tabular}{llcc|cc}
    \toprule
    \textbf{Config} & \textbf{LR} & \textbf{BS} & \textbf{Epochs} & \textbf{MTA (\%)} & \textbf{ASR (\%)} \\
    \midrule
    C1 & 0.1 & 32 & 2 & 89.23 & 99.00 \\
    C2 & 0.1 & 64 & 5 & 88.22 & 99.00 \\
    C3 & 0.05 & 32 & 5 & 88.48 & 98.75 \\
    C4 & 0.1 & 32 & 1 & 89.59 & 98.75 \\
    C5 & 0.1 & 64 & 1 & 89.20 & 98.50 \\
    C6 & 0.1 & 32 & 5 & 88.73 & 98.50 \\
    C7 & 0.1 & 64 & 2 & 88.87 & 98.25 \\
    C8 & 0.05 & 32 & 1 & 89.21 & 98.25 \\ 
    C9 & 0.01 & 32 & 2 & 87.73 & 97.25 \\
    C10 & 0.025 & 128 & 5 & 86.61 & 96.00 \\
    \bottomrule
    \end{tabular}
    \caption{Client FL configurations for successful stealthy attacks on EMNIST \ie cases where the MTA $\geq 80\%$ and ASR $\geq 95\%$.}
    \label{tab:fl_setup_exps_emnist}
\end{table}
\clearpage


\end{document}
