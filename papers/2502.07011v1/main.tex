%%
%% This is file `sample-sigconf-authordraft.tex',
%% generated with the docstrip utility.
%%
%% The original source files were:
%%
%% samples.dtx  (with options: `all,proceedings,bibtex,authordraft')
%% 
%% IMPORTANT NOTICE:
%% 
%% For the copyright see the source file.
%% 
%% Any modified versions of this file must be renamed
%% with new filenames distinct from sample-sigconf-authordraft.tex.
%% 
%% For distribution of the original source see the terms
%% for copying and modification in the file samples.dtx.
%% 
%% This generated file may be distributed as long as the
%% original source files, as listed above, are part of the
%% same distribution. (The sources need not necessarily be
%% in the same archive or directory.)
%%
%%
%% Commands for TeXCount
%TC:macro \cite [option:text,text]
%TC:macro \citep [option:text,text]
%TC:macro \citet [option:text,text]
%TC:envir table 0 1
%TC:envir table* 0 1
%TC:envir tabular [ignore] word
%TC:envir displaymath 0 word
%TC:envir math 0 word
%TC:envir comment 0 0
%%
%%
%% The first command in your LaTeX source must be the \documentclass
%% command.
%%
%% For submission and review of your manuscript please change the
%% command to \documentclass[manuscript, screen, review]{acmart}.
%%
%% When submitting camera ready or to TAPS, please change the command
%% to \documentclass[sigconf]{acmart} or whichever template is required
%% for your publication.
%%
%%
% Define the flag
\newif\ifpreprintversion
\preprintversiontrue % Set to true
% \preprintversionfalse % Set to false

% USE ANONYMOUS FOR SUBMISSION
\ifpreprintversion
    \documentclass[sigconf]{acmart}
\else
    \documentclass[sigconf,anonymous]{acmart}
\fi

%%
%% \BibTeX command to typeset BibTeX logo in the docs
\AtBeginDocument{%
  \providecommand\BibTeX{{%
    Bib\TeX}}}

%% Rights management information.  This information is sent to you
%% when you complete the rights form.  These commands have SAMPLE
%% values in them; it is your responsibility as an author to replace
%% the commands and values with those provided to you when you
%% complete the rights form.
\setcopyright{acmlicensed}
\copyrightyear{2025}
\acmYear{2025}
\acmDOI{XXXXXXX.XXXXXXX}

%% These commands are for a PROCEEDINGS abstract or paper.
\acmConference[Conference acronym 'XX]{Make sure to enter the correct
  conference title from your rights confirmation emai}{June 03--05,
  2018}{Woodstock, NY}
%%
%%  Uncomment \acmBooktitle if the title of the proceedings is different
%%  from ``Proceedings of ...''!
%%
%%\acmBooktitle{Woodstock '18: ACM Symposium on Neural Gaze Detection,
%%  June 03--05, 2018, Woodstock, NY}
\acmISBN{978-1-4503-XXXX-X/18/06}


%%
%% Submission ID.
%% Use this when submitting an article to a sponsored event. You'll
%% receive a unique submission ID from the organizers
%% of the event, and this ID should be used as the parameter to this command.
%%\acmSubmissionID{123-A56-BU3}

%%
%% For managing citations, it is recommended to use bibliography
%% files in BibTeX format.
%%
%% You can then either use BibTeX with the ACM-Reference-Format style,
%% or BibLaTeX with the acmnumeric or acmauthoryear sytles, that include
%% support for advanced citation of software artefact from the
%% biblatex-software package, also separately available on CTAN.
%%
%% Look at the sample-*-biblatex.tex files for templates showcasing
%% the biblatex styles.
%%

%%
%% The majority of ACM publications use numbered citations and
%% references.  The command \citestyle{authoryear} switches to the
%% "author year" style.
%%
%% If you are preparing content for an event
%% sponsored by ACM SIGGRAPH, you must use the "author year" style of
%% citations and references.
%% Uncommenting
%% the next command will enable that style.
%%\citestyle{acmauthoryear}

% Custom ToDO, etc.
%
% --- inline annotations
%
\newcommand{\red}[1]{{\color{red}#1}}
\newcommand{\todo}[1]{{\color{red}#1}}
\newcommand{\TODO}[1]{\textbf{\color{red}[TODO: #1]}}
% --- disable by uncommenting  
% \renewcommand{\TODO}[1]{}
% \renewcommand{\todo}[1]{#1}



\newcommand{\VLM}{LVLM\xspace} 
\newcommand{\ours}{PeKit\xspace}
\newcommand{\yollava}{Yo’LLaVA\xspace}

\newcommand{\thisismy}{This-Is-My-Img\xspace}
\newcommand{\myparagraph}[1]{\noindent\textbf{#1}}
\newcommand{\vdoro}[1]{{\color[rgb]{0.4, 0.18, 0.78} {[V] #1}}}
% --- disable by uncommenting  
% \renewcommand{\TODO}[1]{}
% \renewcommand{\todo}[1]{#1}
\usepackage{slashbox}
% Vectors
\newcommand{\bB}{\mathcal{B}}
\newcommand{\bw}{\mathbf{w}}
\newcommand{\bs}{\mathbf{s}}
\newcommand{\bo}{\mathbf{o}}
\newcommand{\bn}{\mathbf{n}}
\newcommand{\bc}{\mathbf{c}}
\newcommand{\bp}{\mathbf{p}}
\newcommand{\bS}{\mathbf{S}}
\newcommand{\bk}{\mathbf{k}}
\newcommand{\bmu}{\boldsymbol{\mu}}
\newcommand{\bx}{\mathbf{x}}
\newcommand{\bg}{\mathbf{g}}
\newcommand{\be}{\mathbf{e}}
\newcommand{\bX}{\mathbf{X}}
\newcommand{\by}{\mathbf{y}}
\newcommand{\bv}{\mathbf{v}}
\newcommand{\bz}{\mathbf{z}}
\newcommand{\bq}{\mathbf{q}}
\newcommand{\bff}{\mathbf{f}}
\newcommand{\bu}{\mathbf{u}}
\newcommand{\bh}{\mathbf{h}}
\newcommand{\bb}{\mathbf{b}}

\newcommand{\rone}{\textcolor{green}{R1}}
\newcommand{\rtwo}{\textcolor{orange}{R2}}
\newcommand{\rthree}{\textcolor{red}{R3}}
\usepackage{amsmath}
%\usepackage{arydshln}
\DeclareMathOperator{\similarity}{sim}
\DeclareMathOperator{\AvgPool}{AvgPool}

\newcommand{\argmax}{\mathop{\mathrm{argmax}}}     



\begin{document}


% Our macros

%
\setlength\unitlength{1mm}
\newcommand{\twodots}{\mathinner {\ldotp \ldotp}}
% bb font symbols
\newcommand{\Rho}{\mathrm{P}}
\newcommand{\Tau}{\mathrm{T}}

\newfont{\bbb}{msbm10 scaled 700}
\newcommand{\CCC}{\mbox{\bbb C}}

\newfont{\bb}{msbm10 scaled 1100}
\newcommand{\CC}{\mbox{\bb C}}
\newcommand{\PP}{\mbox{\bb P}}
\newcommand{\RR}{\mbox{\bb R}}
\newcommand{\QQ}{\mbox{\bb Q}}
\newcommand{\ZZ}{\mbox{\bb Z}}
\newcommand{\FF}{\mbox{\bb F}}
\newcommand{\GG}{\mbox{\bb G}}
\newcommand{\EE}{\mbox{\bb E}}
\newcommand{\NN}{\mbox{\bb N}}
\newcommand{\KK}{\mbox{\bb K}}
\newcommand{\HH}{\mbox{\bb H}}
\newcommand{\SSS}{\mbox{\bb S}}
\newcommand{\UU}{\mbox{\bb U}}
\newcommand{\VV}{\mbox{\bb V}}


\newcommand{\yy}{\mathbbm{y}}
\newcommand{\xx}{\mathbbm{x}}
\newcommand{\zz}{\mathbbm{z}}
\newcommand{\sss}{\mathbbm{s}}
\newcommand{\rr}{\mathbbm{r}}
\newcommand{\pp}{\mathbbm{p}}
\newcommand{\qq}{\mathbbm{q}}
\newcommand{\ww}{\mathbbm{w}}
\newcommand{\hh}{\mathbbm{h}}
\newcommand{\vvv}{\mathbbm{v}}

% Vectors

\newcommand{\av}{{\bf a}}
\newcommand{\bv}{{\bf b}}
\newcommand{\cv}{{\bf c}}
\newcommand{\dv}{{\bf d}}
\newcommand{\ev}{{\bf e}}
\newcommand{\fv}{{\bf f}}
\newcommand{\gv}{{\bf g}}
\newcommand{\hv}{{\bf h}}
\newcommand{\iv}{{\bf i}}
\newcommand{\jv}{{\bf j}}
\newcommand{\kv}{{\bf k}}
\newcommand{\lv}{{\bf l}}
\newcommand{\mv}{{\bf m}}
\newcommand{\nv}{{\bf n}}
\newcommand{\ov}{{\bf o}}
\newcommand{\pv}{{\bf p}}
\newcommand{\qv}{{\bf q}}
\newcommand{\rv}{{\bf r}}
\newcommand{\sv}{{\bf s}}
\newcommand{\tv}{{\bf t}}
\newcommand{\uv}{{\bf u}}
\newcommand{\wv}{{\bf w}}
\newcommand{\vv}{{\bf v}}
\newcommand{\xv}{{\bf x}}
\newcommand{\yv}{{\bf y}}
\newcommand{\zv}{{\bf z}}
\newcommand{\zerov}{{\bf 0}}
\newcommand{\onev}{{\bf 1}}

% Matrices

\newcommand{\Am}{{\bf A}}
\newcommand{\Bm}{{\bf B}}
\newcommand{\Cm}{{\bf C}}
\newcommand{\Dm}{{\bf D}}
\newcommand{\Em}{{\bf E}}
\newcommand{\Fm}{{\bf F}}
\newcommand{\Gm}{{\bf G}}
\newcommand{\Hm}{{\bf H}}
\newcommand{\Id}{{\bf I}}
\newcommand{\Jm}{{\bf J}}
\newcommand{\Km}{{\bf K}}
\newcommand{\Lm}{{\bf L}}
\newcommand{\Mm}{{\bf M}}
\newcommand{\Nm}{{\bf N}}
\newcommand{\Om}{{\bf O}}
\newcommand{\Pm}{{\bf P}}
\newcommand{\Qm}{{\bf Q}}
\newcommand{\Rm}{{\bf R}}
\newcommand{\Sm}{{\bf S}}
\newcommand{\Tm}{{\bf T}}
\newcommand{\Um}{{\bf U}}
\newcommand{\Wm}{{\bf W}}
\newcommand{\Vm}{{\bf V}}
\newcommand{\Xm}{{\bf X}}
\newcommand{\Ym}{{\bf Y}}
\newcommand{\Zm}{{\bf Z}}

% Calligraphic

\newcommand{\Ac}{{\cal A}}
\newcommand{\Bc}{{\cal B}}
\newcommand{\Cc}{{\cal C}}
\newcommand{\Dc}{{\cal D}}
\newcommand{\Ec}{{\cal E}}
\newcommand{\Fc}{{\cal F}}
\newcommand{\Gc}{{\cal G}}
\newcommand{\Hc}{{\cal H}}
\newcommand{\Ic}{{\cal I}}
\newcommand{\Jc}{{\cal J}}
\newcommand{\Kc}{{\cal K}}
\newcommand{\Lc}{{\cal L}}
\newcommand{\Mc}{{\cal M}}
\newcommand{\Nc}{{\cal N}}
\newcommand{\nc}{{\cal n}}
\newcommand{\Oc}{{\cal O}}
\newcommand{\Pc}{{\cal P}}
\newcommand{\Qc}{{\cal Q}}
\newcommand{\Rc}{{\cal R}}
\newcommand{\Sc}{{\cal S}}
\newcommand{\Tc}{{\cal T}}
\newcommand{\Uc}{{\cal U}}
\newcommand{\Wc}{{\cal W}}
\newcommand{\Vc}{{\cal V}}
\newcommand{\Xc}{{\cal X}}
\newcommand{\Yc}{{\cal Y}}
\newcommand{\Zc}{{\cal Z}}

% Bold greek letters

\newcommand{\alphav}{\hbox{\boldmath$\alpha$}}
\newcommand{\betav}{\hbox{\boldmath$\beta$}}
\newcommand{\gammav}{\hbox{\boldmath$\gamma$}}
\newcommand{\deltav}{\hbox{\boldmath$\delta$}}
\newcommand{\etav}{\hbox{\boldmath$\eta$}}
\newcommand{\lambdav}{\hbox{\boldmath$\lambda$}}
\newcommand{\epsilonv}{\hbox{\boldmath$\epsilon$}}
\newcommand{\nuv}{\hbox{\boldmath$\nu$}}
\newcommand{\muv}{\hbox{\boldmath$\mu$}}
\newcommand{\zetav}{\hbox{\boldmath$\zeta$}}
\newcommand{\phiv}{\hbox{\boldmath$\phi$}}
\newcommand{\psiv}{\hbox{\boldmath$\psi$}}
\newcommand{\thetav}{\hbox{\boldmath$\theta$}}
\newcommand{\tauv}{\hbox{\boldmath$\tau$}}
\newcommand{\omegav}{\hbox{\boldmath$\omega$}}
\newcommand{\xiv}{\hbox{\boldmath$\xi$}}
\newcommand{\sigmav}{\hbox{\boldmath$\sigma$}}
\newcommand{\piv}{\hbox{\boldmath$\pi$}}
\newcommand{\rhov}{\hbox{\boldmath$\rho$}}
\newcommand{\upsilonv}{\hbox{\boldmath$\upsilon$}}

\newcommand{\Gammam}{\hbox{\boldmath$\Gamma$}}
\newcommand{\Lambdam}{\hbox{\boldmath$\Lambda$}}
\newcommand{\Deltam}{\hbox{\boldmath$\Delta$}}
\newcommand{\Sigmam}{\hbox{\boldmath$\Sigma$}}
\newcommand{\Phim}{\hbox{\boldmath$\Phi$}}
\newcommand{\Pim}{\hbox{\boldmath$\Pi$}}
\newcommand{\Psim}{\hbox{\boldmath$\Psi$}}
\newcommand{\Thetam}{\hbox{\boldmath$\Theta$}}
\newcommand{\Omegam}{\hbox{\boldmath$\Omega$}}
\newcommand{\Xim}{\hbox{\boldmath$\Xi$}}


% Sans Serif small case

\newcommand{\Gsf}{{\sf G}}

\newcommand{\asf}{{\sf a}}
\newcommand{\bsf}{{\sf b}}
\newcommand{\csf}{{\sf c}}
\newcommand{\dsf}{{\sf d}}
\newcommand{\esf}{{\sf e}}
\newcommand{\fsf}{{\sf f}}
\newcommand{\gsf}{{\sf g}}
\newcommand{\hsf}{{\sf h}}
\newcommand{\isf}{{\sf i}}
\newcommand{\jsf}{{\sf j}}
\newcommand{\ksf}{{\sf k}}
\newcommand{\lsf}{{\sf l}}
\newcommand{\msf}{{\sf m}}
\newcommand{\nsf}{{\sf n}}
\newcommand{\osf}{{\sf o}}
\newcommand{\psf}{{\sf p}}
\newcommand{\qsf}{{\sf q}}
\newcommand{\rsf}{{\sf r}}
\newcommand{\ssf}{{\sf s}}
\newcommand{\tsf}{{\sf t}}
\newcommand{\usf}{{\sf u}}
\newcommand{\wsf}{{\sf w}}
\newcommand{\vsf}{{\sf v}}
\newcommand{\xsf}{{\sf x}}
\newcommand{\ysf}{{\sf y}}
\newcommand{\zsf}{{\sf z}}


% mixed symbols

\newcommand{\sinc}{{\hbox{sinc}}}
\newcommand{\diag}{{\hbox{diag}}}
\renewcommand{\det}{{\hbox{det}}}
\newcommand{\trace}{{\hbox{tr}}}
\newcommand{\sign}{{\hbox{sign}}}
\renewcommand{\arg}{{\hbox{arg}}}
\newcommand{\var}{{\hbox{var}}}
\newcommand{\cov}{{\hbox{cov}}}
\newcommand{\Ei}{{\rm E}_{\rm i}}
\renewcommand{\Re}{{\rm Re}}
\renewcommand{\Im}{{\rm Im}}
\newcommand{\eqdef}{\stackrel{\Delta}{=}}
\newcommand{\defines}{{\,\,\stackrel{\scriptscriptstyle \bigtriangleup}{=}\,\,}}
\newcommand{\<}{\left\langle}
\renewcommand{\>}{\right\rangle}
\newcommand{\herm}{{\sf H}}
\newcommand{\trasp}{{\sf T}}
\newcommand{\transp}{{\sf T}}
\renewcommand{\vec}{{\rm vec}}
\newcommand{\Psf}{{\sf P}}
\newcommand{\SINR}{{\sf SINR}}
\newcommand{\SNR}{{\sf SNR}}
\newcommand{\MMSE}{{\sf MMSE}}
\newcommand{\REF}{{\RED [REF]}}

% Markov chain
\usepackage{stmaryrd} % for \mkv 
\newcommand{\mkv}{-\!\!\!\!\minuso\!\!\!\!-}

% Colors

\newcommand{\RED}{\color[rgb]{1.00,0.10,0.10}}
\newcommand{\BLUE}{\color[rgb]{0,0,0.90}}
\newcommand{\GREEN}{\color[rgb]{0,0.80,0.20}}

%%%%%%%%%%%%%%%%%%%%%%%%%%%%%%%%%%%%%%%%%%
\usepackage{hyperref}
\hypersetup{
    bookmarks=true,         % show bookmarks bar?
    unicode=false,          % non-Latin characters in AcrobatÕs bookmarks
    pdftoolbar=true,        % show AcrobatÕs toolbar?
    pdfmenubar=true,        % show AcrobatÕs menu?
    pdffitwindow=false,     % window fit to page when opened
    pdfstartview={FitH},    % fits the width of the page to the window
%    pdftitle={My title},    % title
%    pdfauthor={Author},     % author
%    pdfsubject={Subject},   % subject of the document
%    pdfcreator={Creator},   % creator of the document
%    pdfproducer={Producer}, % producer of the document
%    pdfkeywords={keyword1} {key2} {key3}, % list of keywords
    pdfnewwindow=true,      % links in new window
    colorlinks=true,       % false: boxed links; true: colored links
    linkcolor=red,          % color of internal links (change box color with linkbordercolor)
    citecolor=green,        % color of links to bibliography
    filecolor=blue,      % color of file links
    urlcolor=blue           % color of external links
}
%%%%%%%%%%%%%%%%%%%%%%%%%%%%%%%%%%%%%%%%%%%


%%
%% The "title" command has an optional parameter,
%% allowing the author to define a "short title" to be used in page headers.
\title{DROP: Poison Dilution via Knowledge Distillation for Federated Learning}
%\title{'Now you see me, now you don't!' Targeted Backdoor attacks for Federated Learning; quirks and remedies.}
% DROP- Distillation-based Reduction Of Poisoning

%%
%% The "author" command and its associated commands are used to define
%% the authors and their affiliations.
%% Of note is the shared affiliation of the first two authors, and the
%% "authornote" and "authornotemark" commands
%% used to denote shared contribution to the research.

\author{Georgios Syros $^{*\dagger}$}
\affiliation{%
  \institution{Northeastern University}
  \country{}
}
\author{Anshuman Suri$^*$}
\affiliation{%
  \institution{Northeastern University}
  \country{}
}

\author{Farinaz Koushanfar}
\affiliation{%
  \institution{University of California, San Diego}
  \country{}
}
\author{Cristina Nita-Rotaru}
\affiliation{%
  \institution{Northeastern University}
  \country{}
}
\author{Alina Oprea}
\affiliation{%
  \institution{Northeastern University}
  \country{}
}


%%
%% By default, the full list of authors will be used in the page
%% headers. Often, this list is too long, and will overlap
%% other information printed in the page headers. This command allows
%% the author to define a more concise list
%% of authors' names for this purpose.
\renewcommand{\shortauthors}{Syros et al.}

%%
%% The abstract is a short summary of the work to be presented in the
%% article.
\begin{abstract}
  Federated Learning is vulnerable to adversarial manipulation, where malicious clients can inject poisoned updates to influence the global model's behavior. While existing defense mechanisms have made notable progress, they fail to protect against adversaries that aim to induce targeted backdoors under different learning and attack configurations. 
  To address this limitation, we introduce \textit{DROP} (\textbf{D}istillation-based \textbf{R}eduction \textbf{O}f \textbf{P}oisoning), a novel defense mechanism that combines clustering and activity-tracking techniques with extraction of benign behavior from clients via knowledge-distillation to tackle stealthy adversaries that manipulate low data poisoning rates and diverse malicious client ratios within the federation. Through extensive experimentation, our approach demonstrates superior robustness compared to existing defenses across a wide range of learning configurations. Finally, we evaluate existing defenses and our method  under the challenging setting of non-IID client data distribution and highlight the challenges of designing a resilient FL defense in this setting.
  
  %when trying to defend systems against stealthy targeted backdoors when operating under non-i.i.d data distributions.
\end{abstract}

%%
%% The code below is generated by the tool at http://dl.acm.org/ccs.cfm.
%% Please copy and paste the code instead of the example below.
%%
\ifpreprintversion
\else
\begin{CCSXML}
<ccs2012>
   <concept>
       <concept_id>10002978.10003006</concept_id>
       <concept_desc>Security and privacy~Systems security</concept_desc>
       <concept_significance>500</concept_significance>
       </concept>
   <concept>
       <concept_id>10010147.10010257</concept_id>
       <concept_desc>Computing methodologies~Machine learning</concept_desc>
       <concept_significance>500</concept_significance>
       </concept>
 </ccs2012>
\end{CCSXML}
    \ccsdesc[500]{Security and privacy~Systems security}
    \ccsdesc[500]{Computing methodologies~Machine learning}
\fi

%%
%% Keywords. The author(s) should pick words that accurately describe
%% the work being presented. Separate the keywords with commas.
\keywords{Federated Learning Security, Targeted Backdoor Attacks, Poisoning Defenses, Knowledge Distillation}

% \received{20 February 2025}
% \received[revised]{12 March 2009}
% \received[accepted]{5 June 2009}

\ifpreprintversion
    \setcopyright{none}
    \settopmatter{printacmref=false} % Removes citation information below abstract
    \renewcommand\footnotetextcopyrightpermission[1]{} % removes footnote with conference information in first column
    \pagestyle{plain}
\fi

\maketitle

\ifpreprintversion
    \renewcommand{\thefootnote}{}
    \footnotetext{$^*$ Equal Contribution}
    \footnotetext{$^\dagger$ Correspondence to syros.g@northeastern.edu}
\fi

\section{Introduction}
\label{section:introduction}

% redirection is unique and important in VR
Virtual Reality (VR) systems enable users to embody virtual avatars by mirroring their physical movements and aligning their perspective with virtual avatars' in real time. 
As the head-mounted displays (HMDs) block direct visual access to the physical world, users primarily rely on visual feedback from the virtual environment and integrate it with proprioceptive cues to control the avatar’s movements and interact within the VR space.
Since human perception is heavily influenced by visual input~\cite{gibson1933adaptation}, 
VR systems have the unique capability to control users' perception of the virtual environment and avatars by manipulating the visual information presented to them.
Leveraging this, various redirection techniques have been proposed to enable novel VR interactions, 
such as redirecting users' walking paths~\cite{razzaque2005redirected, suma2012impossible, steinicke2009estimation},
modifying reaching movements~\cite{gonzalez2022model, azmandian2016haptic, cheng2017sparse, feick2021visuo},
and conveying haptic information through visual feedback to create pseudo-haptic effects~\cite{samad2019pseudo, dominjon2005influence, lecuyer2009simulating}.
Such redirection techniques enable these interactions by manipulating the alignment between users' physical movements and their virtual avatar's actions.

% % what is hand/arm redirection, motivation of study arm-offset
% \change{\yj{i don't understand the purpose of this paragraph}
% These illusion-based techniques provide users with unique experiences in virtual environments that differ from the physical world yet maintain an immersive experience. 
% A key example is hand redirection, which shifts the virtual hand’s position away from the real hand as the user moves to enhance ergonomics during interaction~\cite{feuchtner2018ownershift, wentzel2020improving} and improve interaction performance~\cite{montano2017erg, poupyrev1996go}. 
% To increase the realism of virtual movements and strengthen the user’s sense of embodiment, hand redirection techniques often incorporate a complete virtual arm or full body alongside the redirected virtual hand, using inverse kinematics~\cite{hartfill2021analysis, ponton2024stretch} or adjustments to the virtual arm's movement as well~\cite{li2022modeling, feick2024impact}.
% }

% noticeability, motivation of predicting a probability, not a classification
However, these redirection techniques are most effective when the manipulation remains undetected~\cite{gonzalez2017model, li2022modeling}. 
If the redirection becomes too large, the user may not mitigate the conflict between the visual sensory input (redirected virtual movement) and their proprioception (actual physical movement), potentially leading to a loss of embodiment with the virtual avatar and making it difficult for the user to accurately control virtual movements to complete interaction tasks~\cite{li2022modeling, wentzel2020improving, feuchtner2018ownershift}. 
While proprioception is not absolute, users only have a general sense of their physical movements and the likelihood that they notice the redirection is probabilistic. 
This probability of detecting the redirection is referred to as \textbf{noticeability}~\cite{li2022modeling, zenner2024beyond, zenner2023detectability} and is typically estimated based on the frequency with which users detect the manipulation across multiple trials.

% version B
% Prior research has explored factors influencing the noticeability of redirected motion, including the redirection's magnitude~\cite{wentzel2020improving, poupyrev1996go}, direction~\cite{li2022modeling, feuchtner2018ownershift}, and the visual characteristics of the virtual avatar~\cite{ogawa2020effect, feick2024impact}.
% While these factors focus on the avatars, the surrounding virtual environment can also influence the users' behavior and in turn affect the noticeability of redirection.
% One such prominent external influence is through the visual channel - the users' visual attention is constantly distracted by complex visual effects and events in practical VR scenarios.
% Although some prior studies have explored how to leverage user blindness caused by visual distractions to redirect users' virtual hand~\cite{zenner2023detectability}, there remains a gap in understanding how to quantify the noticeability of redirection under visual distractions.

% visual stimuli and gaze behavior
Prior research has explored factors influencing the noticeability of redirected motion, including the redirection's magnitude~\cite{wentzel2020improving, poupyrev1996go}, direction~\cite{li2022modeling, feuchtner2018ownershift}, and the visual characteristics of the virtual avatar~\cite{ogawa2020effect, feick2024impact}.
While these factors focus on the avatars, the surrounding virtual environment can also influence the users' behavior and in turn affect the noticeability of redirection.
This, however, remains underexplored.
One such prominent external influence is through the visual channel - the users' visual attention is constantly distracted by complex visual effects and events in practical VR scenarios.
We thus want to investigate how \textbf{visual stimuli in the virtual environment} affect the noticeability of redirection.
With this, we hope to complement existing works that focus on avatars by incorporating environmental visual influences to enable more accurate control over the noticeability of redirected motions in practical VR scenarios.
% However, in realistic VR applications, the virtual environment often contains complex visual effects beyond the virtual avatar itself. 
% We argue that these visual effects can \textbf{distract users’ visual attention and thus affect the noticeability of redirection offsets}, while current research has yet taken into account.
% For instance, in a VR boxing scenario, a user’s visual attention is likely focused on their opponent rather than on their virtual body, leading to a lower noticeability of redirection offsets on their virtual movements. 
% Conversely, when reaching for an object in the center of their field of view, the user’s attention is more concentrated on the virtual hand’s movement and position to ensure successful interaction, resulting in a higher noticeability of offsets.

Since each visual event is a complex choreography of many underlying factors (type of visual effect, location, duration, etc.), it is extremely difficult to quantify or parameterize visual stimuli.
Furthermore, individuals respond differently to even the same visual events.
Prior neuroscience studies revealed that factors like age, gender, and personality can influence how quickly someone reacts to visual events~\cite{gillon2024responses, gale1997human}. 
Therefore, aiming to model visual stimuli in a way that is generalizable and applicable to different stimuli and users, we propose to use users' \textbf{gaze behavior} as an indicator of how they respond to visual stimuli.
In this paper, we used various gaze behaviors, including gaze location, saccades~\cite{krejtz2018eye}, fixations~\cite{perkhofer2019using}, and the Index of Pupil Activity (IPA)~\cite{duchowski2018index}.
These behaviors indicate both where users are looking and their cognitive activity, as looking at something does not necessarily mean they are attending to it.
Our goal is to investigate how these gaze behaviors stimulated by various visual stimuli relate to the noticeability of redirection.
With this, we contribute a model that allows designers and content creators to adjust the redirection in real-time responding to dynamic visual events in VR.

To achieve this, we conducted user studies to collect users' noticeability of redirection under various visual stimuli.
To simulate realistic VR scenarios, we adopted a dual-task design in which the participants performed redirected movements while monitoring the visual stimuli.
Specifically, participants' primary task was to report if they noticed an offset between the avatar's movement and their own, while their secondary task was to monitor and report the visual stimuli.
As realistic virtual environments often contain complex visual effects, we started with simple and controlled visual stimulus to manage the influencing factors.

% first user study, confirmation study
% collect data under no visual stimuli, different basic visual stimuli
We first conducted a confirmation study (N=16) to test whether applying visual stimuli (opacity-based) actually affects their noticeability of redirection. 
The results showed that participants were significantly less likely to detect the redirection when visual stimuli was presented $(F_{(1,15)}=5.90,~p=0.03)$.
Furthermore, by analyzing the collected gaze data, results revealed a correlation between the proposed gaze behaviors and the noticeability results $(r=-0.43)$, confirming that the gaze behaviors could be leveraged to compute the noticeability.

% data collection study
We then conducted a data collection study to obtain more accurate noticeability results through repeated measurements to better model the relationship between visual stimuli-triggered gaze behaviors and noticeability of redirection.
With the collected data, we analyzed various numerical features from the gaze behaviors to identify the most effective ones. 
We tested combinations of these features to determine the most effective one for predicting noticeability under visual stimuli.
Using the selected features, our regression model achieved a mean squared error (MSE) of 0.011 through leave-one-user-out cross-validation. 
Furthermore, we developed both a binary and a three-class classification model to categorize noticeability, which achieved an accuracy of 91.74\% and 85.62\%, respectively.

% evaluation study
To evaluate the generalizability of the regression model, we conducted an evaluation study (N=24) to test whether the model could accurately predict noticeability with new visual stimuli (color- and scale-based animations).
Specifically, we evaluated whether the model's predictions aligned with participants' responses under these unseen stimuli.
The results showed that our model accurately estimated the noticeability, achieving mean squared errors (MSE) of 0.014 and 0.012 for the color- and scale-based visual stimili, respectively, compared to participants' responses.
Since the tested visual stimuli data were not included in the training, the results suggested that the extracted gaze behavior features capture a generalizable pattern and can effectively indicate the corresponding impact on the noticeability of redirection.

% application
Based on our model, we implemented an adaptive redirection technique and demonstrated it through two applications: adaptive VR action game and opportunistic rendering.
We conducted a proof-of-concept user study (N=8) to compare our adaptive redirection technique with a static redirection, evaluating the usability and benefits of our adaptive redirection technique.
The results indicated that participants experienced less physical demand and stronger sense of embodiment and agency when using the adaptive redirection technique. 
These results demonstrated the effectiveness and usability of our model.

In summary, we make the following contributions.
% 
\begin{itemize}
    \item 
    We propose to use users' gaze behavior as a medium to quantify how visual stimuli influences the noticebility of redirection. 
    Through two user studies, we confirm that visual stimuli significantly influences noticeability and identify key gaze behavior features that are closely related to this impact.
    \item 
    We build a regression model that takes the user's gaze behavioral data as input, then computes the noticeability of redirection.
    Through an evaluation study, we verify that our model can estimate the noticeability with new participants under unseen visual stimuli.
    These findings suggest that the extracted gaze behavior features effectively capture the influence of visual stimuli on noticeability and can generalize across different users and visual stimuli.
    \item 
    We develop an adaptive redirection technique based on our regression model and implement two applications with it.
    With a proof-of-concept study, we demonstrate the effectiveness and potential usability of our regression model on real-world use cases.

\end{itemize}

% \delete{
% Virtual Reality (VR) allows the user to embody a virtual avatar by mirroring their physical movements through the avatar.
% As the user's visual access to the physical world is blocked in tasks involving motion control, they heavily rely on the visual representation of the avatar's motions to guide their proprioception.
% Similar to real-world experiences, the user is able to resolve conflicts between different sensory inputs (e.g., vision and motor control) through multisensory integration, which is essential for mitigating the sensory noise that commonly arises.
% However, it also enables unique manipulations in VR, as the system can intentionally modify the avatar's movements in relation to the user's motions to achieve specific functional outcomes,
% for example, 
% % the manipulations on the avatar's movements can 
% enabling novel interaction techniques of redirected walking~\cite{razzaque2005redirected}, redirected reaching~\cite{gonzalez2022model}, and pseudo haptics~\cite{samad2019pseudo}.
% With small adjustments to the avatar's movements, the user can maintain their sense of embodiment, due to their ability to resolve the perceptual differences.
% % However, a large mismatch between the user and avatar's movements can result in the user losing their sense of embodiment, due to an inability to resolve the perceptual differences.
% }

% \delete{
% However, multisensory integration can break when the manipulation is so intense that the user is aware of the existence of the motion offset and no longer maintains the sense of embodiment.
% Prior research studied the intensity threshold of the offset applied on the avatar's hand, beyond which the embodiment will break~\cite{li2022modeling}. 
% Studies also investigated the user's sensitivity to the offsets over time~\cite{kohm2022sensitivity}.
% Based on the findings, we argue that one crucial factor that affects to what extent the user notices the offset (i.e., \textit{noticeability}) that remains under-explored is whether the user directs their visual attention towards or away from the virtual avatar.
% Related work (e.g., Mise-unseen~\cite{marwecki2019mise}) has showcased applications where adjustments in the environment can be made in an unnoticeable manner when they happen in the area out of the user's visual field.
% We hypothesize that directing the user's visual attention away from the avatar's body, while still partially keeping the avatar within the user's field-of-view, can reduce the noticeability of the offset.
% Therefore, we conduct two user studies and implement a regression model to systematically investigate this effect.
% }

% \delete{
% In the first user study (N = 16), we test whether drawing the user's visual attention away from their body impacts the possibility of them noticing an offset that we apply to their arm motion in VR.
% We adopt a dual-task design to enable the alteration of the user's visual attention and a yes/no paradigm to measure the noticeability of motion offset. 
% The primary task for the user is to perform an arm motion and report when they perceive an offset between the avatar's virtual arm and their real arm.
% In the secondary task, we randomly render a visual animation of a ball turning from transparent to red and becoming transparent again and ask them to monitor and report when it appears.
% We control the strength of the visual stimuli by changing the duration and location of the animation.
% % By changing the time duration and location of the visual animation, we control the strengths of attraction to the users.
% As a result, we found significant differences in the noticeability of the offsets $(F_{(1,15)}=5.90,~p=0.03)$ between conditions with and without visual stimuli.
% Based on further analysis, we also identified the behavioral patterns of the user's gaze (including pupil dilation, fixations, and saccades) to be correlated with the noticeability results $(r=-0.43)$ and they may potentially serve as indicators of noticeability.
% }

% \delete{
% To further investigate how visual attention influences the noticeability, we conduct a data collection study (N = 12) and build a regression model based on the data.
% The regression model is able to calculate the noticeability of the offset applied on the user's arm under various visual stimuli based on their gaze behaviors.
% Our leave-one-out cross-validation results show that the proposed method was able to achieve a mean-squared error (MSE) of 0.012 in the probability regression task.
% }

% \delete{
% To verify the feasibility and extendability of the regression model, we conduct an evaluation study where we test new visual animations based on adjustments on scale and color and invite 24 new participants to attend the study.
% Results show that the proposed method can accurately estimate the noticeability with an MSE of 0.014 and 0.012 in the conditions of the color- and scale-based visual effects.
% Since these animations were not included in the dataset that the regression model was built on, the study demonstrates that the gaze behavioral features we extracted from the data capture a generalizable pattern of the user's visual attention and can indicate the corresponding impact on the noticeability of the offset.
% }

% \delete{
% Finally, we demonstrate applications that can benefit from the noticeability prediction model, including adaptive motion offsets and opportunistic rendering, considering the user's visual attention. 
% We conclude with discussions of our work's limitations and future research directions.
% }

% \delete{
% In summary, we make the following contributions.
% }
% % 
% \begin{itemize}
%     \item 
%     \delete{
%     We quantify the effects of the user's visual attention directed away by stimuli on their noticeability of an offset applied to the avatar's arm motion with respect to the user's physical arm. 
%     Through two user studies, we identified gaze behavioral features that are indicative of the changes in noticeability.
%     }
%     \item 
%     \delete{We build a regression model that takes the user's gaze behavioral data and the offset applied to the arm motion as input, then computes the probability of the user noticing the offset.
%     Through an evaluation study, we verified that the model needs no information about the source attracting the user's visual attention and can be generalizable in different scenarios.
%     }
%     \item 
%     \delete{We demonstrate two applications that potentially benefit from the regression model, including adaptive motion offsets and opportunistic rendering.
%     }

% \end{itemize}

\begin{comment}
However, users will lose the sense of embodiment to the virtual avatars if they notice the offset between the virtual and physical movements.
To address this, researchers have been exploring the noticing threshold of offsets with various magnitudes and proposing various redirection techniques that maintain the sense of embodiment~\cite{}.

However, when users embody virtual avatars to explore virtual environments, they encounter various visual effects and content that can attract their attention~\cite{}.
During this, the user may notice an offset when he observes the virtual movement carefully while ignoring it when the virtual contents attract his attention from the movements.
Therefore, static offset thresholds are not appropriate in dynamic scenarios.

Past research has proposed dynamic mapping techniques that adapted to users' state, such as hand moving speed~\cite{frees2007prism} or ergonomically comfortable poses~\cite{montano2017erg}, but not considering the influence of virtual content.
More specifically, PRISM~\cite{frees2007prism} proposed adjusting the C/D ratio with a non-linear mapping according to users' hand moving speed, but it might not be optimal for various virtual scenarios.
While Erg-O~\cite{montano2017erg} redirected users' virtual hands according to the virtual target's relative position to reduce physical fatigue, neglecting the change of virtual environments. 

Therefore, how to design redirection techniques in various scenarios with different visual attractions remains unknown.
To address this, we investigate how visual attention affects the noticing probability of movement offsets.
Based on our experiments, we implement a computational model that automatically computes the noticing probability of offsets under certain visual attractions.
VR application designers and developers can easily leverage our model to design redirection techniques maintaining the sense of embodiment adapt to the user's visual attention.
We implement a dynamic redirection technique with our model and demonstrate that it effectively reduces the target reaching time without reducing the sense of embodiment compared to static redirection techniques.

% Need to be refined
This paper offers the following contributions.
\begin{itemize}
    \item We investigate how visual attractions affect the noticing probability of redirection offsets.
    \item We construct a computational model to predict the noticing probability of an offset with a given visual background.
    \item We implement a dynamic redirection technique adapting to the visual background. We evaluate the technique and develop three applications to demonstrate the benefits. 
\end{itemize}



First, we conducted a controlled experiment to understand how users perceived the movement offset while subjected to various distractions.
Since hand redirection is one of the most frequently used redirections in VR interactions, we focused on the dynamic arm movements and manually added angular offsets to the' elbow joint~\cite{li2022modeling, gonzalez2022model, zenner2019estimating}. 
We employed flashing spheres in the user's field of view as distractions to attract users' visual attention.
Participants were instructed to report the appearing location of the spheres while simultaneously performing the arm movements and reporting if they perceived an offset during the movement. 
(\zhipeng{Add the results of data collection. Analyze the influence of the distance between the gaze map and the offset.}
We measured the visual attraction's magnitude with the gaze distribution on it.
Results showed that stronger distractions made it harder for users to notice the offset.)
\zhipeng{Need to rewrite. Not sure to use gaze distribution or a metric obtained from the visual content.}
Secondly, we constructed a computational model to predict the noticing probability of offsets with given visual content.
We analyzed the data from the user studies to measure the influence of visual attractions on the noticing probability of offsets.
We built a statistical model to predict the offset's noticing probability with a given visual content.
Based on the model, we implement a dynamic redirection technique to adjust the redirection offset adapted to the user's current field of view.
We evaluated the technique in a target selection task compared to no hand redirection and static hand redirection.
\zhipeng{Add the results of the evaluation.}
Results showed that the dynamic hand redirection technique significantly reduced the target selection time with similar accuracy and a comparable sense of embodiment.
Finally, we implemented three applications to demonstrate the potential benefits of the visual attention adapted dynamic redirection technique.
\end{comment}

% This one modifies arm length, not redirection
% \citeauthor{mcintosh2020iteratively} proposed an adaptation method to iteratively change the virtual avatar arm's length based on the primary tasks' performance~\cite{mcintosh2020iteratively}.



% \zhipeng{TO ADD: what is redirection}
% Redirection enables novel interactions in Virtual Reality, including redirected walking, haptic redirection, and pseudo haptics by introducing an offset to users' movement.
% \zhipeng{TO ADD: extend this sentence}
% The price of this is that users' immersiveness and embodiment in VR can be compromised when they notice the offset and perceive the virtual movement not as theirs~\cite{}.
% \zhipeng{TO ADD: extend this sentence, elaborate how the virtual environment attracts users' attention}
% Meanwhile, the visual content in the virtual environment is abundant and consistently captures users' attention, making it harder to notice the offset~\cite{}.
% While previous studies explored the noticing threshold of the offsets and optimized the redirection techniques to maintain the sense of embodiment~\cite{}, the influence of visual content on the probability of perceiving offsets remains unknown.  
% Therefore, we propose to investigate how users perceive the redirection offset when they are facing various visual attractions.


% We conducted a user study to understand how users notice the shift with visual attractions.
% We used a color-changing ball to attract the user's attention while instructing users to perform different poses with their arms and observe it meanwhile.
% \zhipeng{(Which one should be the primary task? Observe the ball should be the primary one, but if the primary task is too simple, users might allocate more attention on the secondary task and this makes the secondary task primary.)}
% \zhipeng{(We need a good and reasonable dual-task design in which users care about both their pose and the visual content, at least in the evaluation study. And we need to be able to control the visual content's magnitude and saliency maybe?)}
% We controlled the shift magnitude and direction, the user's pose, the ball's size, and the color range.
% We set the ball's color-changing interval as the independent factor.
% We collect the user's response to each shift and the color-changing times.
% Based on the collected data, we constructed a statistical model to describe the influence of visual attraction on the noticing probability.
% \zhipeng{(Are we actually controlling the attention allocation? How do we measure the attracting effect? We need uniform metrics, otherwise it is also hard for others to use our knowledge.)}
% \zhipeng{(Try to use eye gaze? The eye gaze distribution in the last five seconds to decide the attention allocation? Basically constructing a model with eye gaze distribution and noticing probability. But the user's head is moving, so the eye gaze distribution is not aligned well with the current view.)}

% \zhipeng{Saliency and EMD}
% \zhipeng{Gaze is more than just a point: Rethinking visual attention
% analysis using peripheral vision-based gaze mapping}

% Evaluation study(ideal case): based on the visual content, adjusting the redirection magnitude dynamically.

% \zhipeng{(The risk is our model's effect is trivial.)}

% Applications:
% Playing Lego while watching demo videos, we can accelerate the reaching process of bricks, and forbid the redirection during the manipulation.

% Beat saber again: but not make a lot of sense? Difficult game has complicated visual effects, while allows larger shift, but do not need large shift with high difficulty



\section{Background}\label{sec:backgrnd}

\subsection{Cold Start Latency and Mitigation Techniques}

Traditional FaaS platforms mitigate cold starts through snapshotting, lightweight virtualization, and warm-state management. Snapshot-based methods like \textbf{REAP} and \textbf{Catalyzer} reduce initialization time by preloading or restoring container states but require significant memory and I/O resources, limiting scalability~\cite{dong_catalyzer_2020, ustiugov_benchmarking_2021}. Lightweight virtualization solutions, such as \textbf{Firecracker} microVMs, achieve fast startup times with strong isolation but depend on robust infrastructure, making them less adaptable to fluctuating workloads~\cite{agache_firecracker_2020}. Warm-state management techniques like \textbf{Faa\$T}~\cite{romero_faa_2021} and \textbf{Kraken}~\cite{vivek_kraken_2021} keep frequently invoked containers ready, balancing readiness and cost efficiency under predictable workloads but incurring overhead when demand is erratic~\cite{romero_faa_2021, vivek_kraken_2021}. While these methods perform well in resource-rich cloud environments, their resource intensity challenges applicability in edge settings.

\subsubsection{Edge FaaS Perspective}

In edge environments, cold start mitigation emphasizes lightweight designs, resource sharing, and hybrid task distribution. Lightweight execution environments like unikernels~\cite{edward_sock_2018} and \textbf{Firecracker}~\cite{agache_firecracker_2020}, as used by \textbf{TinyFaaS}~\cite{pfandzelter_tinyfaas_2020}, minimize resource usage and initialization delays but require careful orchestration to avoid resource contention. Function co-location, demonstrated by \textbf{Photons}~\cite{v_dukic_photons_2020}, reduces redundant initializations by sharing runtime resources among related functions, though this complicates isolation in multi-tenant setups~\cite{v_dukic_photons_2020}. Hybrid offloading frameworks like \textbf{GeoFaaS}~\cite{malekabbasi_geofaas_2024} balance edge-cloud workloads by offloading latency-tolerant tasks to the cloud and reserving edge resources for real-time operations, requiring reliable connectivity and efficient task management. These edge-specific strategies address cold starts effectively but introduce challenges in scalability and orchestration.

\subsection{Predictive Scaling and Caching Techniques}

Efficient resource allocation is vital for maintaining low latency and high availability in serverless platforms. Predictive scaling and caching techniques dynamically provision resources and reduce cold start latency by leveraging workload prediction and state retention.
Traditional FaaS platforms use predictive scaling and caching to optimize resources, employing techniques (OFC, FaasCache) to reduce cold starts. However, these methods rely on centralized orchestration and workload predictability, limiting their effectiveness in dynamic, resource-constrained edge environments.



\subsubsection{Edge FaaS Perspective}

Edge FaaS platforms adapt predictive scaling and caching techniques to constrain resources and heterogeneous environments. \textbf{EDGE-Cache}~\cite{kim_delay-aware_2022} uses traffic profiling to selectively retain high-priority functions, reducing memory overhead while maintaining readiness for frequent requests. Hybrid frameworks like \textbf{GeoFaaS}~\cite{malekabbasi_geofaas_2024} implement distributed caching to balance resources between edge and cloud nodes, enabling low-latency processing for critical tasks while offloading less critical workloads. Machine learning methods, such as clustering-based workload predictors~\cite{gao_machine_2020} and GRU-based models~\cite{guo_applying_2018}, enhance resource provisioning in edge systems by efficiently forecasting workload spikes. These innovations effectively address cold start challenges in edge environments, though their dependency on accurate predictions and robust orchestration poses scalability challenges.

\subsection{Decentralized Orchestration, Function Placement, and Scheduling}

Efficient orchestration in serverless platforms involves workload distribution, resource optimization, and performance assurance. While traditional FaaS platforms rely on centralized control, edge environments require decentralized and adaptive strategies to address unique challenges such as resource constraints and heterogeneous hardware.



\subsubsection{Edge FaaS Perspective}

Edge FaaS platforms adopt decentralized and adaptive orchestration frameworks to meet the demands of resource-constrained environments. Systems like \textbf{Wukong} distribute scheduling across edge nodes, enhancing data locality and scalability while reducing network latency. Lightweight frameworks such as \textbf{OpenWhisk Lite}~\cite{kravchenko_kpavelopenwhisk-light_2024} optimize resource allocation by decentralizing scheduling policies, minimizing cold starts and latency in edge setups~\cite{benjamin_wukong_2020}. Hybrid solutions like \textbf{OpenFaaS}~\cite{noauthor_openfaasfaas_2024} and \textbf{EdgeMatrix}~\cite{shen_edgematrix_2023} combine edge-cloud orchestration to balance resource utilization, retaining latency-sensitive functions at the edge while offloading non-critical workloads to the cloud. While these approaches improve flexibility, they face challenges in maintaining coordination and ensuring consistent performance across distributed nodes.


% \section{Setup}
% \label{sec:setup}

% Here we describe the basic setup for federated-learning, along with the threat model we consider in this work (\Cref{sec:threat_model}).

% \todo{Describe slightly-more formally (or maybe not? we probably don't need to describe a lot)}

% \subsection{Threat Model}
% \label{sec:threat_model}

% In this work, we consider a robust and comprehensive adversarial threat model that reflects realistic attack scenarios in FL systems. 

% \shortsection{Adversary's Goals} The adversary's primary objective is to embed a backdoor into the global model \( \mathbf{w}_T \) over the course of \( T \) training rounds. A backdoor attack is considered \emph{successful} if it meets two key criteria:  
% (1) it is \emph{stealthy}, meaning it evades detection by deployed defensive mechanisms with high confidence, and  
% (2) it is \emph{consistent}, ensuring that during inference, the global model \( \mathbf{w}_T \) reliably misclassifies inputs containing the backdoor trigger into the \emph{target class} chosen by the adversary.

% \shortsection{Adversary's Capabilities} The adversary is assumed to have the capability to compromise a certain fraction of clients in the system, while the central server remains trusted and secure. Once a client is compromised, the adversary gains full control over that client’s local resources, including its training data and the ability to dictate its attack strategy. This allows the adversary to manipulate training data (\ie by injecting poisoned examples), alter the training process (\eg training set vs batch-level poisoning), or directly modify the model updates sent to the server. These capabilities enable the adversary to execute sophisticated attacks aimed at undermining the global model’s integrity, including both targeted and untargeted poisoning strategies.

% % \shortsection{Adversary's Objective} As summarized by \citet{shejwalkar2022back}, the adversary's objectives for FL poisoning can be broadly categorized into the following:
% % \begin{enumerate}
% %     \item \textbf{Targeted Attacks}: Aim to misclassify only a specific set/classes of inputs.
% %     \item \textbf{Untargeted Attacks}: Aim to reduce model accuracy on arbitrary inputs.
% %     \item \textbf{Backdoor Attacks}: Aim to trigger misclassification for only certain inputs, \eg input with specific properties (semantic backdoors), inputs with specific trigger patterns (artificial backdoors).
% % \end{enumerate}
% % \todo{Describe our realistic threat model and why it is accurate. From how the clients are selected to the data conditions that are followed.}

% % \shortsection{Adversary's Capabilities}

% The client-sampling strategy in the FL system is assumed to be random, without any mechanisms to explicitly include or exclude specific clients. This randomness introduces significant variability in the number of adversary-controlled clients selected in each round. Unlike prior works that assume a fixed number of malicious clients in every round \citep{blanchard2017machine}, our threat model considers the possibility of variation in adversarial influence across rounds. In some rounds, the number of malicious clients may exceed the number of benign clients in the selected pool, enabling the adversary to exert substantial control over the aggregated update. Conversely, there may also be rounds where no adversarial clients are selected, reflecting the dynamic and unpredictable nature of real-world client participation. \anshuman{I think we should list both and include results for both, but default to no-ability-to-select as the default (can say ``we also explore strategies that allow forcing a certain number of adv clients per round in Appendix ..."}

% %The adversary’s objectives can be broadly categorized into two types: untargeted poisoning and targeted poisoning. In untargeted poisoning attacks, the adversary’s goal is to degrade the overall performance of the global model, reducing its accuracy across tasks and data distributions. On the other hand, targeted poisoning attacks aim to introduce specific biases into the model, such as causing misclassifications for certain classes, inputs, or tasks, while leaving the model’s overall performance largely unaffected. The adversary’s ability to control both the data and the update process on compromised clients provides flexibility to pursue these objectives under various scenarios.

% \shortsection{Data Distribution}
% Our threat model considers both independent and identically distributed (IID) and non-IID data conditions among the clients, ensuring a comprehensive evaluation of adversarial scenarios.
% % While we include IID distributions for completeness and to provide a baseline for comparison, the primary focus of this work is on non-IID settings.
% Unlike prior works that often rely on artificially constructed or overly simplified non-IID setups, our framework strives to replicate the complexity and variability of real-world data distributions. In our model, the data assigned to each client follows no predefined distribution, resulting in heterogeneous and highly imbalanced datasets. Each client can possess a completely random number of training examples per class, with certain classes potentially being overrepresented or absent altogether. This setup mirrors the chaotic and unpredictable data characteristics encountered in practical deployments, making it a more realistic and rigorous testing ground for adversarial defenses.
% \anshuman{Dirichlet for non-iid seems pretty prevelant. We should mention that we evaluate across various "levels" of non-iid, and also comment on how the non-iid nature of data impacts targeted backdoors attacks even more}

% This dynamic threat model captures the inherent unpredictability of FL systems, where the influence of adversarial clients may fluctuate over time. It highlights the significant challenges in designing robust defenses, particularly when malicious clients can dominate certain rounds or evade detection altogether in others. By considering such a stochastic and realistic adversarial setup, this work aims to address the complexities of defending FL systems against highly adaptive and resourceful adversaries.

% Here we describe the basic setup for federated learning, along with the threat model considered in this work (\Cref{sec:threat_model}).

% ========================================

\subsection{Threat Model}
\label{sec:threat_model}

\shortsection{Adversary's Goals} 
The adversary's primary objective is to embed a backdoor into the global model in a way that satisfies two key properties:
\newline
\newline
\underline{\textbf{Property 1}}: \textit{Stealthiness}. The backdoor should avoid arousing suspicion by not degrading performance on data without any triggers, whilst circumventing any defense measures the server may deploy.
\newline
\newline
% \underline{\textbf{Property 2}}: \textit{Consistency}. The backdoor must consistently misclassify inputs with the adversary's trigger \(\mathbf{t}\) into a specific \textit{target class} \(y_{\text{target}}\). Let \(\text{ASR}_t\) denote the attack success rate at round \(t\), defined as the fraction of inputs \(\mathbf{x} \in \mathcal{D}_{\text{victim}}\) misclassified as \(y_{\text{target}}\) by the global model \(\mathbf{w}_t\).
% %
% \anshuman{How about we do something like this:}
% \begin{align}
%     K = \operatorname*{arg\,top-}k_{t} MTA_t \\
%     \arg\max_{k\in K} ASR_k \leq \tau
% \end{align}
% \anshuman{Basically 'ASR for any round where the MTA is decent should be low' so that irrespective of which one the rounds is picked (maybe the MTA was 89\% v/s 89.1\%), the attack should not get through. Could instead look at models with MTA above some threshold $\lambda$ but the gist of it remains the same}
% To ensure a loose upward trend, the attack success rate should not show a persistent decline. This is formalized by ensuring the cumulative sum of increases dominates decreases over a window \(k\):
% \begin{equation}
%     \frac{1}{k} \sum_{i=t-k+1}^{t} \text{ASR}_i \leq \text{ASR}_{t+1}, \quad \forall t \in \{k, \dots, T-1\},
% \end{equation}
% where \(k\) controls trend smoothness. Additionally, the attack must achieve a minimum success rate \(\tau\) by the end of training:
% \begin{equation}
%     \lim_{t \to T} \text{ASR}_t \geq \tau,
% \end{equation}
% where \(\tau\) is a predefined threshold (e.g., \(\tau = 0.8\)). This ensures the backdoor remains effective despite fluctuations.
\underline{\textbf{Property 2}}: \textit{Consistency}. The backdoor attack must maintain a consistently high \textit{Attack Success Rate} (ASR) when the model achieves acceptable MTA, ensuring the adversary's success is not reliant on opportunistic fluctuations in ASR across rounds. Specifically, let \(\text{MTA}_t\) and \(\text{ASR}_t\) denote the main task accuracy and attack success rate at round \(t\), respectively. We define the set of rounds \(K\) where the MTA is above some given threshold \(\lambda\) as:  
\begin{equation}
    K = \{t \mid \text{MTA}_t \geq \lambda\},
\end{equation}  
and require that the ASR across these rounds remains above a specified threshold \(\tau\):  
\begin{equation}
    \min_{t \in K} \text{ASR}_t \geq \tau.
\end{equation}  
This formulation ensures that the backdoor remains effective throughout training, preventing its success from being undermined by transient fluctuations in ASR.
% \anshuman{This section is about the adversary's goals, but the second property is written from the defender's perspective. Should make sure both properties are from the same perspective.} \georgios{Understood what you mean and refined text to use the attacker perspective for property 2.}

\shortsection{Adversary's Capabilities} 
The adversary is successful in compromising and gaining control of a fraction $\rho$ out of all clients \(\mathcal{N}\) in the system, which is referred to as the \textit{Malicious Client Ratio} (MCR). This set of compromised clients $\mathcal{N}_\text{adv}$ is fixed throughout the federation.
% Once a client \(c \in \mathcal{N}_\text{adv}\) is compromised, the adversary gains full control over the client's local training process. This control extends to the client's training data \(\mathcal{D}_c\), local training procedure, and the updates \(\mathbf{w}_c^t\) sent back to the server at each round \(t\). The adversary can modify the local dataset \(\mathcal{D}_c\) by injecting \textit{poisoned} samples of the form \((\mathbf{x}_i + \mathbf{t}, y_{\text{target}})\), where \(\mathbf{t}\) is the adversarial trigger. 
Once a client \(c \in \mathcal{N}_\text{adv}\) is compromised, the adversary gains full control over the client’s local training process, including its dataset \(\mathcal{D}_c\), training procedure, and model updates \(\mathbf{w}_c^t\) submitted to the server at each round \(t\). 

In a targeted backdoor attack, the adversary focuses on a specific class, known as the \textit{victim} class, and poisons only samples from this class. Specifically, the adversary injects \textit{poisoned} samples into the client’s local dataset by applying a trigger \(\mathbf{t}\) to inputs belonging to the victim class and flipping their labels to a designated \textit{target} class \(y_{\text{target}}\). Formally, for a subset \(\mathcal{D}_{\text{poisoned}} \subset \mathcal{D}_{\text{victim}}\), each input is modified as:
\begin{equation}
    \mathcal{D}_{\text{poisoned}} = \{(\mathbf{x}_i + \mathbf{t}, y_{\text{target}}) \mid (\mathbf{x}_i, y_i) \in \mathcal{D}_{\text{victim}}\},
\end{equation}
where \(\mathcal{D}_{\text{victim}}\) denotes the set of samples from the victim class.
%
The proportion of poisoned samples relative to the total number of clean samples is referred to as the \textit{Data Poisoning Rate} (DPR). Additionally, the adversary may alter the client's update \(\mathbf{w}_c^t\) before submission to the server, embedding adversarial changes that facilitate the backdoor's propagation to the global model. For convenience, definitions of all acronyms used throughout the paper can be found in \Cref{tab:glossary}.

% Beyond data manipulation, the adversary has complete control over the local training process, allowing modifications to batch composition, epoch count, and the loss function. This level of control enables sophisticated attack strategies that can be adjusted adaptively over the course of training. 

% We formalize the adversary’s capabilities to control local data, manipulate updates, and adjust training strategies, thereby capturing a broad spectrum of potential attack strategies.

\begin{table}[ht]
    \footnotesize
    \centering
    \begin{tabular}{ll|ll}
    \toprule
    \textbf{Acronym} & \textbf{Definition} & \textbf{Acronym} & \textbf{Definition} \\
    \midrule
    DPR & Data Poisoning Rate & MTA & Main Task Accuracy \\
    MCR & Malicious Client Ratio & ASR & Attack Success Rate \\
    \bottomrule
    \end{tabular}
    \caption{Glossary of acronyms used throughout the paper.}
    \label{tab:glossary}
\end{table}

\section{Defenses Falter Against Targeted Backdoors}
\label{sec:existing_finicky}

Several works have proposed defenses against poisoning attacks in FL \citep{zhang2023flip, nguyen2022flame, blanchard2017machine, yin2018byzantine,fung2018mitigating,cao2021fltrust,wang2022flare,pillutla2022robust,shejwalkar2021manipulating}. It is standard for such works to focus on a specific learning configuration, for instance setting a specific learning rate and batch size for clients. While demonstrating superior performance in a particular setup is useful, it provides no guarantees about how well the defense would work in other valid learning configurations. We begin with an exploration of some of these learning configurations (\Cref{sec:fl_setup_matters}) and observe that there exist several equally-valid learning configurations where the model achieves acceptable MTA and the adversary's objective is preserved, thus making them all equally valid FL configurations. However, we find that all existing defenses we evaluated are effective only for a subset of these valid configurations and no defense is resilient across all configurations (\Cref{sec:defenses_fail_across_configs}).

\subsection{Learning Configuration Matters for Attack Success}
\label{sec:fl_setup_matters}

To understand the impact of the learning configuration on model robustness and adversarial susceptibility, we conduct a grid-search analysis over key hyperparameters, varying the client's learning rate, batch size, and number of epochs on the CIFAR-10 dataset.
Our findings in \Cref{fig:fl_setup_impact} reveal that small changes in the learning configuration can significantly alter the model’s vulnerability to backdoor attacks.
\begin{figure}[h!]
    \includegraphics[width=.98\linewidth]{assets/section4/undefended_asr.pdf}
    \caption{Visualizing the impact of the learning configuration (learning rate, batch size, and number of local epochs) on ASR, for 1.25\% DPR and 20\% MCR, for CIFAR-10 with IID data. We only visualize configurations with MTA $\geq80\%$. The attack is successful on multiple configurations (in yellow).}
    \label{fig:fl_setup_impact}
\end{figure}
% Certain configurations produce extreme outcomes, such as low MTA or low ASR, but more interestingly, many configurations exhibit simultaneously high MTA and high ASR. This combination indicates a "\textbf{danger zone}" of configurations that yield high model utility while remaining highly vulnerable to adversarial manipulation.
% \anshuman{Could compress below further to be 3-4 lines each at most.} \georgios{Just did. Check it out.}

%One particularly concerning observation is that low learning rates (e.g., 0.01) and small batch sizes (e.g. 64) create conditions where targeted backdoor attacks become extremely stealthy. In this regime, adversarial updates remain close to benign updates, making it difficult for anomaly-based defenses to detect them. This also allows the global model to maintain high MTA while the attack achieves a high ASR.Importantly, this phenomenon arises even in the absence of adversarial manipulation, as natural variability in client updates creates overlapping patterns between benign and adversarial updates. Given the natural variability, it is difficult to distinguish benign updates from adversarial updates, making the FL setup itself a critical factor in system vulnerability.

\subsubsection{Impact of learning rate}

%In general, learning configurations with a lower learning rate (e.g. 0.01) generally yield local updates that are smoother with respect to the client's local data and also consistently yield results with high MTA and ASR. On the other hand, higher learning rates (e.g. 0.1) lead to local updates that when aggregated cause performance degradation both on the MTA and the ASR. \georgios{A counter for 0.1 lr instability would be adopting a moving average aggregation for server aggregation.}

%Lower learning rates (\eg 0.01) produce local updates that are smoother and more possibly aligned with the client's local data distribution. This characteristic not only enhances the MTA but also enables the adversary to inject backdoors with minimal deviation from benign updates, resulting in consistently high ASR. By contrast, higher learning rates (\eg 0.1) lead to noisier local updates, which, when aggregated at the server, can cause performance degradation on both MTA and ASR. This instability at higher learning rates poses challenges for backdoor attacks but also negatively impacts the utility of the global model.

Lower learning rates (\eg 0.01, 0.025) allow for smaller, more gradual updates to the model parameters, making it easier for adversarial objectives to be embedded into the global model with minimal deviation from benign updates, resulting in a high ASR. In contrast, higher learning rates (\eg 0.1) cause larger, less stable updates that disrupt the optimization process, leading to degraded performance, with large drops in MTA and ASR alike.

\subsubsection{Impact of batch size}

%A larger batch size enables setups with different learning rates to achieve high MTA and ASR. On the other hand, a smaller batch size is making it more difficult to inject a targeted backdoor and only very low learning rates (e.g. 0.01) can successfully inject the backdoor.

% Batch size has a significant impact on both MTA and ASR.

% Larger batch sizes (\eg 256) reduce the variability in local updates, creating conditions where high MTA and ASR are achievable across a range of learning rates. This makes large batch sizes particularly susceptible to backdoor attacks, as adversarial updates blend seamlessly with benign updates. Conversely, smaller batch sizes (\eg 64) increase the variability in client updates, making it more difficult for the adversary to inject a targeted backdoor. In such setups, only very low learning rates (\eg 0.01) succeed in achieving a high ASR, as they produce smoother adversarial updates that can evade detection despite the higher noise introduced by small batch sizes.

Larger batch sizes (\eg 256) reduce variability in local updates, enabling high MTA and ASR across learning rates, making them more vulnerable to backdoor attacks. Smaller batch sizes (\eg 64) increase update variability, requiring very low learning rates (\eg 0.01) for effective backdoor injection.


\subsubsection{Impact of training epochs}

%Configurations with a small number of local training epochs (e.g. 2, 5) seem to be the ones that are more susceptible to backdoor poisoning. In general, the larger the number of local training epochs the clients use on locally training their models, the harder it is for the adversary to inject a targeted backdoor. 

% The number of local training epochs is another critical factor influencing the success of targeted backdoor attacks. Fewer local epochs (\eg 2 or 5) create conditions where clients submit updates that reflect only shallow optimization on their local data. This reduces the distinction between benign and adversarial updates, making the system more susceptible to backdoor poisoning. In contrast, larger numbers of local epochs (\eg 10 or 20) allow clients to perform more thorough local optimization, amplifying the differences between benign and adversarial updates. This makes it harder for adversarial updates to successfully inject backdoors while maintaining high MTA.

Fewer local epochs (\eg 2 or 5) lead to shallow optimization, reducing distinctions between benign and adversarial updates, making backdoor attacks more effective. Training with more local epochs (\eg 10 or 20) induces a \textit{polarization} effect, where clients' local models become more tightly aligned with their respective datasets. This stronger alignment reduces the influence of malicious updates during aggregation, as the backdoor signal becomes diluted and less effective at propagating into the global model. This polarization is even stronger when high LRs are used in conjunction with more training epochs, as model's MTA also suffers.

\Cref{tab:fl_setup_exps} highlights a range of configurations that we term the "\textbf{danger zones}"—settings where the FL system achieves both high MTA (above 80\%) and adequately high ASR (above 85\%). These configurations are of great interest because they strike a balance between utility and vulnerability; accurate for legitimate tasks while remaining highly susceptible to targeted backdoor attacks. This highlights the critical need for a defense to be resilient across multiple learning configurations, ensuring robustness regardless of the setup and achieving true \textit{learning configuration independence}. This is particularly desirable because relying on a single configuration, even if effective, is not always feasible in practice. Various constraints, such as the batch size supported by a specific device, the number of epochs a client is willing to commit to, or other resource limitations, can dictate configurations in practice. Therefore, a defense mechanism that can adapt to different setups without compromising its efficacy is essential for practical and widespread deployment.

%To demonstrate the impact of FL setup, we conducted a series of grid-search experiments using the CIFAR-10 dataset in an iid client data setting, varying the learning rate, batch size, and other hyperparameters. In the absence of any defense, we found that certain configurations led to significantly higher ASR and lower MTA, while others resulted in low ASR but at the cost of degraded MTA. Crucially, some configurations produced both high ASR and high MTA, highlighting the existence of "danger zones" where targeted backdoor attacks are especially effective. A key finding is that low learning rates (e.g., 0.01) and small batch sizes create conditions where the attack becomes exceptionally stealthy. In this regime, the attack remains covert, preserving high MTA while maintaining a high ASR, which poses a significant challenge for defense strategies. This observation reveals that even without explicit adversarial intervention, some FL configurations are inherently vulnerable. \georgios{"inherently" vulnerable is a bold statement, maybe we should opt for a different term?}

% \todo{Start with a grid-search style result for experiments without any defense to show how some configurations lead to very bad MTA and/or ASR but more importantly, there are several successful candidates. Give general rule-of-thumb for configs to use in evaluations}

\subsection{Limitations of Existing Defenses}
\label{sec:defenses_fail_across_configs}

Based on our analysis, we identify 10 learning configurations where the attack is stealthy (minimal impact on MTA) and highly successful (has ASR higher than 85\%), as given in \Cref{tab:fl_setup_exps}. For our evaluation we consider key, prominent defense methods which offer diverse strategies to combat backdoor attacks in FL. These defense strategies include coordinate-based approaches like Median \citep{yin2018byzantine} and Multi-Krum \citep{blanchard2017machine}, trust-based methods such as FLTrust \citep{cao2021fltrust}, reputation-based schemes like FoolsGold \citep{fung2018mitigating} and FLARE \citep{wang2022flare}, anomaly detection frameworks like FLAME \citep{nguyen2022flame} and local adversarial training methods like FLIP \citep{zhang2023flip}. For more details on these defenses and other related works, see \Cref{sec:experiments} and \Cref{sec:related_work}.

\begin{table}[h]
    \centering
    \small
    \begin{tabular}{llcc|cc}
    \toprule
    \textbf{Config} & \textbf{LR} & \textbf{BS} & \textbf{Epochs} & \textbf{MTA (\%)} & \textbf{ASR (\%)} \\
    \midrule
    C1 & 0.05 & 128 & 2 & 85.08 & 96.5 \\
    C2 & 0.05 & 256 & 2 & 86.29 & 95.8 \\
    C3 & 0.025 & 256 & 5 & 88.33 & 94.8 \\
    C4 & 0.01 & 64 & 2 & 87.03 & 93.9  \\
    C5 & 0.025 & 128 & 2 & 86.65 & 92.9 \\
    C6 & 0.025 & 256 & 2 & 82.97 & 91.3 \\
    C7 & 0.1 & 256 & 2 & 85.55 & 91.1 \\
    C8 & 0.01 & 128 & 2 & 84.47 & 89.7 \\ 
    C9 & 0.01 & 128 & 5 & 89.04 & 86.6\\
    C10 & 0.01 & 256 & 10 & 87.97 & 85.6 \\
    \bottomrule
    \end{tabular}
    \caption{Client FL configurations for successful stealthy attacks on CIFAR-10 \ie cases with MTA $\geq 80\%$ and ASR $\geq 85\%$. }
    \label{tab:fl_setup_exps}
\end{table}
%Each of these methods aims to limit the influence of malicious updates during aggregation. For example, Median evaluate their defense on a single, simpler learning task --MNIST \citep{} and only mention the total number of clients that participate in the federation. The authors do not mention anything related to the client learning setup (lr, batch size or number of local training epochs). Multi-Krum attempt to reduce the impact of outliers by using robust statistics. In their paper the authors again only mention how the data is partitioned amongst clients without any significant mention of the client local learning setup. They do however evaluate their method on various different batch sizes. FLTrust relies on a trusted server-side reference model to filter out anomalous updates. The authors report the use of a 'combined' (i.e. the product of the server and local lr) learning rate of 0.002, a batch size of 64 and 1 as the total number of local training epochs. FoolsGold tracks client updates to identify and penalize suspiciously similar contributions. They do not explicitly report a learning rate and the number of local training epochs for their evaluation setup. They report a batch size of 10 and 50, depending on the dataset. FLAME leverages anomaly detection to flag potentially malicious gradients. They do not mention anything about the local learning setup apart from how the federation is structured. 
% \anshuman{Too long; it's nice to have these details mentioned for sure, but should not overlead reader before we get to our main results. Can give a 1-2 sentence summary here} \georgios{What do you mean by too long?? Are you talking about the following section in which I describe the differences per setup?}\anshuman{I mean the defense-wise discussion that follows this comment, about exactly which settings previous defenses considered. We can give an example or two and give a blanket statement about a lack of consistency in evaluation setup (and explain them in detail somewhere in the appendix)} \georgios{Understood.}
% MOVE TO APPENDIX.
%Each of these methods aims to limit the influence of malicious updates during aggregation. However, most works provide limited or inconsistent details about their evaluation setups, particularly concerning client learning configurations such as learning rate, batch size, and the number of local training epochs. For instance, Median is evaluated on a simpler learning task (MNIST) and specifies only the total number of participating clients. The authors provide no details about the client learning setup, including learning rate, batch size, or number of local training epochs. Similarly, Multi-Krum reduces the impact of outliers using robust statistics but focuses primarily on how the data is partitioned among clients. While it does evaluate the method across different batch sizes, it lacks significant discussion of the broader local training setup. FLTrust adopts a trusted server-side reference model to filter anomalous updates. The authors report using a "combined" learning rate of 0.002, a batch size of 64, and a single local training epoch. In contrast, FoolsGold, which identifies and penalizes suspiciously similar client contributions, does not explicitly report the learning rate or number of local training epochs in its evaluation. Instead, it mentions using batch sizes of 10 or 50 depending on the dataset. FLAME, which leverages anomaly detection to flag potentially malicious gradients, describes the structure of the federation but provides no information about the local learning setup, such as learning rate, batch size, or training epochs.
% Each of these methods aims to limit the influence of malicious updates during aggregation. However, %
Evaluation setups for these defenses are often inconsistent, with significant variation in client learning configurations such as learning rate, batch size, and the number of local training epochs. For instance, some works (e.g., FLTrust) specify fixed learning rates and batch sizes, while others (e.g., Multi-Krum, FLAME) provide little to no details about the client training setup, focusing instead on aggregation logic. This lack of standardization in evaluation protocols raises concerns about the reproducibility and generalizability of reported results. A detailed comparison of these evaluation setups is provided in \Cref{app:baseline_details}.

% Shifted to discussion section
% This lack of clarity and standardization in reporting client learning configurations limits the reproducibility of these methods and raises concerns about their robustness across diverse FL setups.

% Under certain learning configurations, these defenses fail to detect or mitigate backdoor attacks, allowing adversaries to achieve high ASR.

\begin{figure}[h]
    \includegraphics[width=.9\linewidth]{assets/section3/defenses_heatmap.pdf}
    \caption{ASR (\%) for our defense (DROP) and various existing defenses for 10 FL configurations (1.25\% DPR, 20\% MCR) where stealthy attacks are possible. No existing defense provides consistent protection across all configurations.}
    \label{fig:baselines_heatmap}
\end{figure}

% When applying existing defenses under these same conditions, we observe that none of them consistently prevent the backdoor attack
We find that these approaches often struggle against targeted backdoor attacks due to the attack's stealthy nature (\Cref{fig:baselines_heatmap}). Median and Multi-Krum fail to exclude the adversary’s updates since the poisoned gradients remain close to the statistical norm, while FLTrust struggles to identify the malicious contributions due to the minimal deviation from expected update patterns. Similarly, FoolsGold proves largely ineffective because the adversary’s contributions exhibit insufficient variability across rounds, making them difficult to detect. FLAME performs reasonably well in certain configurations but fails against attacks that exploit low learning rates. Among the defensive baselines, FLIP is the only one that slightly reduces the ASR, though its mitigation is insufficient to provide robust protection.
Existing defenses thus have a severe limitation when it comes to stealthy backdoors: \textbf{their resilience is sensitive to the specific FL learning configuration.}

\section{Our Defense: DROP}
\label{sec:proposed_defense}

\begin{figure*}[htbp]
    \centering
    \includegraphics[width=\textwidth]{assets/system_final.png}
    \caption{Overview of the proposed DROP defense. Each round \( t \) begins with the server broadcasting the global model to all clients and selecting a subset for local training, which may include both benign (green) and malicious (red) clients. After updates are submitted, DROP employs: (1) \textbf{Agglomerative Clustering} to detect anomalous updates, (2) \textbf{Activity Monitoring \& Penalization} to track and penalize suspicious clients, and (3) \textbf{Knowledge Distillation}, where a GAN-generated synthetic dataset and client logits guide the distillation of the global model. The final model \( \mathbf{w}_{t+1} \) serves as the global model for round \( t+1 \).}
    \label{fig:drops} 
\end{figure*}

From our analysis, it is clear that the learning configuration of a FL system plays a pivotal role in the success of targeted backdoor attacks. Certain configurations exhibit inherent vulnerabilities, allowing adversarial updates to bypass detection and achieve high ASR. Moreover, the performance of existing defenses shows significant variability across different configurations, revealing their lack of robustness across learning setups. These findings emphasize the urgent need for a \textit{universal} defense mechanism that is resilient against a wide array of attack strategies and remains agnostic to the underlying learning configuration. 

To address this challenge, we introduce \textbf{DROP} (\textit{\textbf{D}istillation-based \textbf{R}eduction \textbf{O}f \textbf{P}oisoning}), a federated framework designed to counter three critical adversarial scenarios:
\begin{itemize}
    \item aggressive adversaries that may resort to high-DPR attacks,
    \item diverse MCRs  within the federation, and
    \item stealthy, low-DPR attacks that exploit specific learning configurations.
\end{itemize}
These scenarios each present unique challenges that require a cascade of countermeasures, as summarized in the following sections. An overview of our approach is given in \Cref{fig:drops}.
%
% DROP operates within the FedAvg \citep{mcmahan2017communication} paradigm (described in \Cref{sec:background}). \anshuman{FedAvg is presumably the default choice for FL; do we need to state it here explicitly while describing DROP?}
% At each training round $t$, the local updates from participating clients $\mathcal{C}_t$ undergo a series of three countermeasures to ensure robustness against targeted backdoor attacks. 

The first countermeasure, \textit{Agglomerative Clustering} (\Cref{subsec:agglo_clustering}), identifies and isolates malicious updates by clustering submitted models based on their pairwise Euclidean distances.
% The updates are divided into two clusters: a benign cluster \(C_b\) and a suspicious cluster \(C_s\). The smaller of the two clusters, \(C_s\), is pruned, and the updates within it are deemed \textit{suspicious}.
This process is particularly effective against high-DPR attacks, where malicious updates deviate significantly from benign updates.  
%
The second countermeasure, \textit{Activity Monitoring} (\Cref{subsec:activity_monitoring}), tracks client behavior across training rounds. By maintaining a reputation score based on prior clustering results, this mechanism penalizes clients frequently flagged as suspicious and reduces their influence in future aggregations. This reputation-based approach ensures robustness against per round benign-to-malicious client ratio fluctuations, where the proportion of malicious clients can vary widely across rounds due to random client selection.  

The third and final countermeasure, \textit{Knowledge Distillation} (\Cref{subsec:kd}), addresses the most challenging class of attacks: stealthy, low-DPR strategies. These attacks encode adversarial objectives subtly into model updates, making them indistinguishable from benign updates and enabling them to bypass clustering schemes. To neutralize these residual adversarial signals, we use synthetic data generated by a GAN trained via logit-driven distillation to produce a \textit{cleansed} version of the global model. This process ensures robustness against low-DPR attacks and renders the defense agnostic to learning configurations.

Through this cascade of countermeasures, DROP achieves the following three goals:  
\begin{enumerate}
    \item Resilience against aggressive, high data poisoning rates.
    \item Adaptability to diverse malicious client ratios.
    \item Robustness against stealthy, low data poisoning rates.  
\end{enumerate}
% \anshuman{Just mentioned these 3 at the start of the section- maybe refer to those instead of repeating here?}
Next, we provide a detailed explanation of the design and motivation behind each component. The entire DROP algorithm is presented in Algorithm \ref{alg:drop}. 

% DROP ALGORITHM
\begin{algorithm}
\caption{DROP}
\label{alg:drop}
\SetKwInput{kwGlobals}{Globals input}
\SetKwFunction{FMain}{DROP}
\SetKwFunction{FClustering}{AgglomerativeClustering}
\SetKwFunction{FActivity}{UpdateActivity}
\SetKwFunction{FScore}{score}
\SetKwFunction{FDistillation}{KD}
\SetKwProg{Fn}{Function}{:}{}
\SetKwFor{ForEach}{for each}{do}{end}

\kwGlobals{initial model parameters $w_0$, total training rounds $T$, total client set $\mathcal{N}$, ban threshold $\tau_b$, generator $G$, clean seed set $\mathcal{D}_{\text{clean}}$, knowledge-distillation $KD$}

\For{each round $t \in \{1, \dots, T\}$}{
    \tcp{\textcolor{blue}{Step 1: Training and Update Collection}}
    \ForEach{client $c \in \mathcal{C}_t \subset \mathcal{N}$}{
        $w_c^t \leftarrow \text{ClientLocalTraining}(w_{t-1})$ \tcp*[r]{\textcolor{blue}{Client $c$ trains locally and returns update}}
    }
    
    \tcp{\textcolor{blue}{Step 2: Agglomerative Clustering}}
    $\mathcal{C}_{\text{b}}, \mathcal{C}_{\text{s}} \leftarrow \FClustering(\{w_c^t : c \in \mathcal{C}\})$ \tcp*[r]{\textcolor{blue}{Cluster client updates into \textbf{b}enign and \textbf{s}uspect groups}}
    
    \tcp{\textcolor{blue}{Step 3: Activity Monitoring}}
    \ForEach{client $c \in \mathcal{C}$}{
        \FActivity($c$) % \tcp*[r]{\textcolor{blue}{Update the activity score for client $c$}}
        \If{$c \in \mathcal{C}_{\text{b}} \land \FScore(c) \geq \tau_b$}{
            \tcp{\textcolor{blue}{Exclude $c$ if it exceeds penalty threshold}}
            $\mathcal{C}_{\text{b}} \leftarrow \mathcal{C}_{\text{b}} \setminus \{c\}$ %\tcp*[r]{\textcolor{blue}{Exclude client $c$ if it has exceeded penalty threshold}}
        }
    }
    
    \tcp{\textcolor{blue}{Aggregate model updates from benign clients}}
    $w_t \leftarrow \text{Aggregate}(\{w_c^t : c \in \mathcal{C}_{\text{b}}\})$ % \tcp*[r]{\textcolor{blue}{Aggregate model from benign client updates only}}

    \tcp{\textcolor{blue}{Step 4: Logit-Driven Model Distillation}}
    $w_t \leftarrow \FDistillation(w_t, \{w_c^t : c \in \mathcal{C}_{\text{b}}\}, G, \mathcal{D}_{\text{clean}})$ %\tcp*[r]{Use collective logits to cleanse the global model}

    \tcp{\textcolor{blue}{Step 5: Broadcast the Cleansed Global Model}}
        \ForEach{client $c \in \mathcal{C}$}{
            \text{Send}($c, w_t$);
        }
}


% \Fn{\FClustering{$\{w_c^t : c \in \mathcal{C}\}, \tau_c$}}{
%     Form clusters of client updates using agglomerative clustering with Ward linkage\;
%     Identify cluster(s) containing outlier updates using threshold $\tau_c$\;
%     \KwRet{$\mathcal{C}_{\text{benign}}, \mathcal{C}_{\text{suspect}}$} \tcp*[r]{Return benign and suspect client sets}
% }

% \Fn{\FActivity{$c$, isBenign}}{
%     \If{isBenign}{
%         Increase client $c$'s "benign score" by 1\;
%     }
%     \Else{
%         Increase client $c$'s "penalty score" by 1\;
%     }
%     \If{Client $c$'s "benign score" $\geq \tau_a \times$ "penalty score"}{
%         Allow client $c$ to rejoin $\mathcal{C}$ if it was previously banned\;
%     }
% }

% \Fn{\FDistillation{$w_t, \{w_c^t : c \in \mathcal{C}_{\text{benign}}\}, G$}}{
%     \tcp{** Logit Aggregation from Client Updates **}
%     Aggregate logits $\mathcal{L} \leftarrow \text{AverageLogits}(\{w_c^t : c \in \mathcal{C}_{\text{benign}}\})$ \tcp*[r]{Compute ensemble logits from benign clients}

%     \tcp{** Generator-Guided Knowledge Distillation **}
%     \For{$i = 1, \dots, N_{\text{distill}}$}{
%         Generate synthetic data $\mathcal{D}_{\text{synthetic}} \leftarrow G(\mathcal{L})$\;
%         Update $w_t$ using knowledge distillation loss on $\mathcal{D}_{\text{synthetic}}$\;
%     }
%     \KwRet{$w_t$} \tcp*[r]{Return the cleaned model}
% }
\end{algorithm}

% \subsection{Agglomerative Clustering}
% \label{subsec:agglo_clustering}

% The first line of defense in our system is \textit{Agglomerative Clustering} (AC) \citep{müllner2011modernhierarchicalagglomerativeclustering, auror}, chosen for its suitability in detecting subtle deviations caused by targeted backdoor attacks. Clustering is a widely adopted countermeasure in filtering-based FL frameworks \citep{nguyen2022flame, fung2018mitigating, baybfed, mesas} due to its effectiveness in identifying dissimilarities within a population of model updates. These dissimilarities are often the result of adversarial strategies employing a high DPR that aggressively embed backdoors into updates. We choose AC for clustering due to its flexibility and hierarchical structure; unlike K-means, which is based on ranking distances that are too small to convey useful information \anshuman{What does that mean?}\alina{k-means has a number of limitations: sensitivity to initial centroid selection, susceptibility to outliers, etc.}, or HDBSCAN \citep{campello2013density}, which relies on density-based assumptions \anshuman{Why would that not work in our scenario?}\alina{That is not the issue, but need to set some hyper-params}, AC dynamically merges clusters until a stopping criterion is met. \alina{The use of AC needs better motivation, e.g., less hyper-parameters to tune}

% The use of \textit{Ward linkage} \citep{Ward01031963} is they key to AC's effectiveness \anshuman{Is ward-linkage a feature of AC, or one of several ways to perform part of the clustering? If former, can shorten this part a bit} \georgios{Yes it is exclusive to agglomerative clustering as far as I know. I have not seen it being used in other clustering methods.}. \alina{AC can be done with multiple linkage methods, and Ward is one of them, you can motivate why it's the right choice, as it min the variance of clusters.} By minimizing intra-cluster variance during the merging process, Ward linkage is particularly adept at detecting fine-grained and stealthy backdoors that can be embedded within benign-looking updates. This property ensures that subtle differences in high-dimensional weight spaces—indicative of adversarially perturbed updates—are captured and separated.

\subsection{Agglomerative Clustering}  
\label{subsec:agglo_clustering}  

The first line of defense in our system is \textit{Agglomerative Clustering} (AC) \citep{müllner2011modernhierarchicalagglomerativeclustering}, selected for its flexibility and hierarchical structure, which make it well-suited for detecting subtle deviations caused by targeted backdoor attacks. Clustering is a widely adopted countermeasure in filtering-based FL frameworks \citep{nguyen2022flame, fung2018mitigating, baybfed, mesas, auror} due to its ability to identify dissimilarities within a population of model updates, often resulting from adversarial strategies that embed backdoors aggressively with a high DPR. We chose AC over methods like K-means and HDBSCAN due to its lower sensitivity to hyper-parameter selection and its dynamic merging process, which eliminates the need to pre-specify the number of clusters. Unlike K-means, which can be sensitive to initial centroid selection and outliers, AC provides more consistent results by iteratively merging clusters based on similarity. Similarly, while HDBSCAN \citep{campello2013density} relies on density-based assumptions that require careful tuning of hyper-parameters, AC operates without such assumptions, making it more robust in diverse FL scenarios.

The use of \textit{Ward linkage} \citep{Ward01031963} further enhances AC's effectiveness by ensuring that intra-cluster variance is minimized during the merging process. Among the various linkage methods available for AC, Ward linkage is particularly effective for FL tasks, as it prioritizes clusters with low internal variance, which is crucial for distinguishing fine-grained and stealthy backdoors embedded within benign-looking updates. By capturing and separating subtle differences in high-dimensional weight spaces, Ward linkage ensures that adversarial updates are reliably isolated from legitimate ones.  

Let \(\mathcal{W} = \{\mathbf{w}_1, \mathbf{w}_2, \dots, \mathbf{w}_n\}\) denote the set of model updates from \(n\) participating clients in a given training round \(t\). Let $d(\mathbf{w}_i, \mathbf{w}_j) = \| \mathbf{w}_i - \mathbf{w}_j \|_2$ be the Euclidean distance between two model updates \(\mathbf{w}_i\) and \(\mathbf{w}_j\).
%
The clustering process starts with each client update \(\mathbf{w}_i\) being treated as a singleton cluster. At each step, the two clusters \(\mathcal{A}\) and \(\mathcal{B}\) with the smallest inter-cluster distance are merged. Using \textit{Ward linkage}, the distance between two clusters \(\mathcal{A}\) and \(\mathcal{B}\) is defined as the increase in total intra-cluster variance caused by merging the two clusters:
\begin{equation}
    d_{\text{Ward}}(\mathcal{A}, \mathcal{B}) = \frac{|\mathcal{A}| \, |\mathcal{B}|}{|\mathcal{A}| + |\mathcal{B}|} \, \| \boldsymbol{\mu}_{\mathcal{A}} - \boldsymbol{\mu}_{\mathcal{B}} \|_2^2,
\end{equation}
where \(|\mathcal{A}|\) and \(|\mathcal{B}|\) are the sizes (number of points) of clusters \(\mathcal{A}\) and \(\mathcal{B}\), and \(\boldsymbol{\mu}_{\mathcal{A}}\) and \(\boldsymbol{\mu}_{\mathcal{B}}\) are their respective centroids.
% \begin{equation}
%     \boldsymbol{\mu}_{\mathcal{A}} = \frac{1}{|\mathcal{A}|} \sum_{\mathbf{w} \in \mathcal{A}} \mathbf{w}, \quad 
%     \boldsymbol{\mu}_{\mathcal{B}} = \frac{1}{|\mathcal{B}|} \sum_{\mathbf{w} \in \mathcal{B}} \mathbf{w}.
% \end{equation}

The hierarchical clustering process proceeds iteratively until a stopping criterion is met. In our case, the process halts upon forming exactly two predefined clusters, \(\{\mathcal{C}_b, \mathcal{C}_s\}\), which effectively capture natural groupings of benign and malicious model updates.
In the context of FL, it is assumed that benign client updates will cluster together due to the shared training objective, while adversarial updates will form separate, smaller clusters due to their deviation from normal update patterns.

Minimizing intra-cluster variance ensures that the clustering process captures subtle distinctions in high-dimensional weight spaces, where even minor variations can signal meaningful differences, such as those between adversarially perturbed weights and legitimate ones. The hierarchical structure produced by Ward linkage allows flexibility in examining update patterns at various levels of similarity, enabling dynamic control over how clusters are merged. This approach is particularly valuable in FL, where client updates are naturally noisy due to heterogeneous data distributions but can also be adversarially manipulated. By clustering updates, this method enables the system to separate and potentially eliminate outliers or manipulated model updates.
% \newline
% \framedtext{\underline{\textbf{Goal 1}}: The clustering component aims to eliminate malicious updates that deviate significantly from their benign counterparts in each training round, offering resilience against high data poisoning rate attacks.}\label{goal1}
\goalbox{\underline{\textbf{Goal 1}}: The clustering component aims to eliminate malicious updates that deviate significantly from their benign counterparts in each training round, offering resilience against high data poisoning rate attacks.}\label{goal1}

\subsection{Activity Monitoring}
\label{subsec:activity_monitoring}

The second line of defense, complementing the clustering component, is an \textit{activity monitoring mechanism}, which tracks the behavior of participating clients over the course of training. 

This mechanism addresses a limitation of clustering-based defenses in federated learning, which assume that the number of malicious updates in a round is smaller than the number of benign ones. In real-world scenarios, however, client selection is random, and the defender has no prior knowledge of the number of malicious clients in any given round. This randomness can lead to rounds where adversarial clients outnumber benign ones, causing misidentification of the smaller cluster.
%To mitigate this, prior works commonly assume an upper bound on the proportion of malicious clients, often constraining the number of malicious participants in each round to less than 50\% of the selected clients. While this assumption allows clustering-based defenses to reliably identify and prune smaller clusters, enforcing it typically requires operating under low MCRs or artificially controlling client sampling to maintain a fixed benign-to-malicious client ratio across rounds \citep{zhang2023flip}, both of which limit the practical applicability of such methods.
Prior works often assume an upper bound on the proportion of malicious clients (typically less than 50\%), but enforcing this requires low MCRs or artificially controlled client sampling  to maintain a fixed benign-to-malicious client ratio across rounds \citep{zhang2023flip}, limiting practical applicability.
It is important to note that although the total proportion of adversarial clients in the federation (MCR) remains fixed, the actual ratio of malicious clients in any given round can fluctuate due to the random selection of clients.

The maximum tolerable MCR of a defense—the fraction of clients in the federation that can be adversarial while still allowing the defense to mitigate an attack—depends on the total number of clients in the federation, \(N\), and the number of clients sampled in each round, \(C\). 
%\anshuman{What do we mean by 'supported' threshold?} \georgios{It is the fraction of clients being malicious in the federation that a defense can tolerate to mitigate an attack. I changed the text a bit to make it more clear.} 
Under a random sampling strategy, \(C\) clients are drawn uniformly from the total \(N\) clients. Let $\rho$ be the MCR \ie proportion of malicious clients in the entire federation, so the expected number of malicious clients ($\mathcal{M})$ in a round is \(\mathbb{E}[\mathcal{M}] = \rho \cdot C\). By modeling the selection of clients as a Binomial distribution, we can compute a lower bound on the probability of malicious clients outnumbering benign ones in any given round as:
\begin{align}
    P\left(\mathcal{M} \geq \frac{C}{2}\right) \geq  1 - \left(4\rho(1-\rho)\right)^{\frac{C}{2}}.
\end{align}
For a derivation of the above, please see \Cref{app:bound_analysis}.
% By applying the Chernoff bound \citep{chernoff}, we can bound the probability that the number of malicious clients \(M\) significantly deviates from its expectation. Specifically, for any \(\delta > 0\):
% \begin{equation}
    % \mathbb{P}\left[\mathcal{M} \geq (1 + \delta) \cdot \rho \cdot C\right] \leq \exp\left(-\frac{\delta^2 \cdot \rho \cdot C}{2 + \delta}\right).
% \end{equation}
This bound implies that as \(C\) increases, the likelihood of selecting a disproportionately large number of malicious clients diminishes exponentially. Conversely, when \(C\) is small,
% the variability in the number of malicious clients per round increases, raising
the probability that malicious clients outnumber benign clients within the selected subset increases. 
%
For example, consider an FL system with \(N = 100\) total clients and a MCR of 40\% (\(\rho = 0.4\)). If \(C = 20\) clients are randomly selected in a given round, the expected number of malicious clients in the subset is \(\mathbb{E}[\mathcal{M}] = \rho \cdot C = 0.4 \cdot 20 = 8\).
However, with the inequality above we can see the probability of selecting more than 10 malicious clients can be \underline{non-trivial} ($\approx 0.34$), potentially resulting in a subset where malicious clients constitute more than 50\% of the selected participants (\(M > C/2\)). In fact, the bound suggests that in about a third of the FL training rounds, the malicious clients will be the majority. 
% However, due to the randomness of client selection, the actual number of malicious clients can deviate significantly from this expectation.
% Using the Chernoff bound and setting \( (1 + \delta) \cdot \rho \cdot C = C/2\), the probability of selecting more than 10 malicious clients\footnote{The detailed steps for the calculation of the bound can be found in \cref{sec:chernoff_appendix}.} (\(M > 10\)) can be \underline{non-trivial} (\(\mathbb{P}\left[\mathcal{M} \geq 10\right] \lesssim 0.8\)), potentially resulting in a subset where malicious clients constitute more than 50\% of the selected participants (\(M > C/2\)).
% This scenario undermines clustering-based defenses, as the smaller cluster would incorrectly represent benign clients.

To address scenarios where the number of malicious clients exceeds the benign ones in certain rounds, we propose a reputation-based mechanism. This approach tracks client behavior across rounds, penalizing suspicious clients identified by the clustering component (\Cref{subsec:agglo_clustering}) and gradually reducing their influence over the global model. By doing so, our method dynamically adjusts to varying levels of malicious participation without relying on rigid assumptions about MCR or artificially controlling client sampling. For instance, FLIP~\cite{zhang2023flip} enforces a fixed per-round MCR by '\textit{randomly}' selecting 10 clients per round, ensuring exactly 4 adversaries and 6 benign clients, thus artificially tampering with the randomness of client selection.
% \anshuman{Clarify that reputation helps account for misclassifications in rounds where adversary outnumbers benign clients; maybe can use the bound to show that on average as long as MCR is less than 50\%, in the long run there will be more benign clients identified successfully than incorrectly classified as adversaries} \georgios{Would the following paragraph that I have added work?}
By incorporating a reputation system, our defense mitigates errors caused by transient imbalances where adversarial clients may temporarily outnumber benign ones. Over multiple rounds, as long as the overall MCR remains below 50\%, the mechanism ensures that benign clients are correctly identified and retained more frequently than adversarial clients. This design helps prevent long-term accumulation of adversarial influence, even in rounds where clustering misclassifications may occur.

Our reputation-based mechanism employs a penalty and reward system that essentially tracks the trustworthiness of each client, enabling the server to mitigate persistent malicious activity while accounting for potential false positives. Specifically, by maintaining a cumulative penalty score \(\pi(c)\) for each client \(c\), the server can distinguish between clients with occasional false-positive detections and those consistently submitting suspicious updates. This ensures that benign clients incorrectly flagged as suspicious in a few rounds do not face permanent exclusion from the system.

\shortsection{Calculating Penalty Scores}
We begin by initializing the penalty score \(\pi(c) = 0\) for every client. As training proceeds, this score is dynamically updated based on the clustering results from each round.
% \newline
% \newline
% \underline{\textbf{Rule 1}}:
During each training round \( t \), the clustering component (\Cref{subsec:agglo_clustering}) determines whether a client's update is benign or suspicious by assigning the client with a penalty score:
\begin{align}
    \pi(c) =
    \begin{cases}
    \pi(c) + p, & \text{if } c \in \mathcal{C}_s\\
    \max(0, \pi(c) - r),              & \text{otherwise}
    \end{cases}
\end{align}
\newline
Essentially, if a client's update is flagged as \textit{suspicious} (\( c \in \mathcal{C}_s \)), its penalty score is increased by a constant penalty value \( p \). Otherwise, it is reduced by a constant reward value \( r \), ensuring the score remains non-negative. Capping the penalty score at zero prevents potentially malicious clients from gaining undue advantage in cases where they are repeatedly flagged as benign due to false positives. $p$ and $r$ are hyper-parameters determined by the server.

The server uses these scores to regulate client participation. Clients with a clean history \ie consistently flagged as benign (\(\pi(c) = 0\)) are deemed trustworthy and allowed to contribute to the global model, whereas clients with a record of suspicious update submissions (\(\pi(c) > 0\)) are restricted from participating in the aggregation process. However, depending on the server administration policy, the penalty system could be even stricter. Clients that accumulate excessive penalty points due to repeated suspicious updates are permanently \textit{blacklisted} from contributing in subsequent rounds:
\begin{comment}
\newline
\newline
\underline{\textbf{Rule 2}}: \textbf{\textit{(Optional)}} Specifically, the server permanently bans any client whose penalty score exceeds a predefined threshold \( \tau \):
\begin{equation}
    \pi(c) \geq \tau \implies \text{ban}(c).
\end{equation}
\georgios{Another option would be to rename Rule 2 to shortsection “Calculating Penalty/Trust Scores” and omit the rule formulation but still keep the text around it}
\end{comment}
Optionally, the server could permanently ban any client whose penalty score exceeds a predefined threshold.
% \newline
% \framedtext{\underline{\textbf{Goal 2}}: The activity monitoring component helps prevent malicious actors (based on their accumulated penalty points) from poisoning the global model during rounds in which the malicious outnumber the benign participants.}
\goalbox{\underline{\textbf{Goal 2}}: The activity monitoring component helps prevent malicious actors (based on their accumulated penalty points) from poisoning the global model during rounds in which the malicious outnumber the benign participants.}\label{goal2}



\subsection{Knowledge Distillation}
\label{subsec:kd}

Having tackled high-DPR and dynamic-MCR attacks, the most challenging category to defend against is low-DPR attacks.
These attacks are difficult to detect due to their stealthy nature, where adversarial objectives are subtly embedded into model weights, making the updates nearly indistinguishable from benign ones. This smooth encoding allows them to bypass traditional clustering-based defenses that rely on statistical or geometric properties of the updates. To counter this, DROP introduces a knowledge distillation-based cleansing mechanism that neutralizes residual adversarial signals in the global model using clean, synthetic data, effectively mitigating the impact of stealthy backdoor updates while preserving the model’s utility for legitimate tasks.

% Traditional clustering-based defenses rely on identifying significant deviations in update patterns to separate malicious updates from benign ones. However, low-DPR attacks are specifically designed to minimize these deviations, allowing adversarial updates to bypass clustering filters entirely.  To address this challenge, DROP introduces a knowledge distillation-based cleansing component that removes the residual adversarial signals embedded in the global model. By distilling the knowledge from the global model onto a surrogate model using clean, synthetic data, this component isolates and neutralizes the effects of stealthy backdoor updates while preserving the utility of the global model for legitimate tasks.
% \anshuman{Above two paragraphs can be shortened by a lot}

This approach leverages model stealing attacks, such as the \textit{MAZE} model stealing framework \citep{kariyappa2021maze}, which achieves state-of-the-art performance in extracting machine learning models under black-box conditions. The goal here is not to engage in model stealing as traditionally intended, but to clone the model without introducing any malicious behavior from compromised clients. \textit{MAZE} combines two key components: query synthesis and knowledge distillation. Knowledge Distillation \cite{hinton2015distillingknowledgeneuralnetwork} is a technique where a "student" model learns to approximate the output logits of a larger "teacher" model, enabling the transfer of knowledge while preserving essential decision boundaries. A generative adversarial network (GAN) \cite{radford2016unsupervisedrepresentationlearningdeep} synthesizes inputs to probe a target model's decision boundaries, and knowledge distillation transfers the target model's behavior to a surrogate model trained on these synthetic queries. By minimizing reliance on labeled data, \textit{MAZE} efficiently approximates the target model's functionality.

In our approach, we adapt and extend the \textit{MAZE} framework to take advantage of the white-box access available to the server in FL, as the server owns the global model. This adjustment eliminates the need for zeroth-order gradient estimation used in black-box settings and allows for direct backpropagation, improving both efficiency and accuracy. To ensure high-quality synthetic data, we initialize the GAN with a set of \( n \) clean samples, denoted as \( \mathcal{D}_\text{clean} = \{\mathbf{x}_i \}_{i=1}^n \) in step 4 of Algorithm \ref{alg:drop}, which helps align the generated queries with the original data distribution. Our approach requires only a small set of clean samples—typically less or equal than the size of a single client's dataset—making it practical and feasible for real-world FL scenarios.

In a training round $t$, instead of relying solely on the global model’s logits, we aggregate logits from client updates that pass the filtering process (\ie $\mathcal{C}_b$). Let the logit output from the global model be \( \mathbf{z}_t = f(\mathbf{x}; \mathbf{w}_t) \), where \( \mathbf{w}_t \) represents the global model parameters at round \( t \). Similarly, let the logit outputs from benign client models \( \mathcal{C}_b \subseteq \mathcal{C}_t \) be:
\begin{equation}
    \mathbf{z}_{t}^c = f(\mathbf{x}; \mathbf{w}_t^c), \quad \forall c \in \mathcal{C}_b.
\end{equation}
We compute the ensemble logits \( \mathbf{\bar{z}}_t \) as the average of the logits from benign clients:
\begin{equation}
    \mathbf{\bar{z}}_t = \frac{1}{|\mathcal{C}_b|} \sum_{c \in \mathcal{C}_b} \mathbf{z}_{t}^c.
\end{equation}

The generator \( G \) uses the ensemble logits \( \mathbf{\bar{z}}_t \) as feedback to synthesize new queries \( \mathcal{D}_{\text{synthetic}} \). These synthetic queries, denoted as \( \mathbf{x}_{\text{gen}} \), are used to guide a \textit{clone network}, which serves as a cleansed version of the global model. The clone network, parameterized by \( \mathbf{w}^{\text{clone}}_t \), is trained to align its predictions with the ensemble logits. Instead of the KL divergence used in the original \textit{MAZE} framework, we employ the \(\ell_1\)-loss, which provides more stable performance \citep{truong2021data}. The training objective for the clone network is defined as:
\begin{equation}
    \mathcal{L}_{\text{distill}} = \mathbb{E}_{\mathbf{x} \sim \mathcal{D}_{\text{synthetic}}} \left[ \| \mathbf{\bar{z}}_t - \mathbf{z}_t^{\text{clone}} \|_1 \right],
\end{equation}
where \( \mathbf{z}_t^{\text{clone}} = f(\mathbf{x}; \mathbf{w}^{\text{clone}}_t) \) represents the logit predictions of the clone network on synthetic data \( \mathbf{x} \).

The key difference between this approach and the original MAZE framework is that, instead of distilling knowledge from a single victim model, we distill knowledge from the aggregated logits of benign clients. The intuition behind this approach is that the ensemble logits \( \mathbf{\bar{z}}_t \), derived from benign client models, act as a \textit{consensus signal} to overwrite any adversarial influence introduced by poisoned updates. By aligning the clone network’s behavior with the aggregated benign logits, the distillation process effectively cleanses the model of any stealthy backdoor or poisoned behavior that might have evaded detection in previous defense layers. 
% \newline
% \framedtext{\underline{\textbf{Goal 3}}: The collective logit-driven knowledge distillation framework ensures that any subtle adversarial updates which have made their way to the global model are neutralized, restoring the global model’s robustness and reliability.}
\goalbox{\underline{\textbf{Goal 3}}: The collective logit-driven knowledge distillation framework ensures that any subtle adversarial updates which have made their way to the global model are neutralized, restoring the global model’s robustness and reliability.}\label{goal3}

% \subsection{DROPlet: a lightweight version of DROP}
% \label{subsec:droplet}

% While DROP provides comprehensive protection against targeted backdoor attacks through its cascading components, the knowledge distillation component introduces a non-trivial computational overheads. To offer a faster and more streamlined alternative \anshuman{are we saying that DROP is not streamlined? Focus more on lightweight part and less computationally heavy}, we propose \textit{DROPlet}, a lightweight, server-side plugin designed for ease of deployment and compatibility with various types of data. Unlike DROP, DROPlet omits the knowledge distillation component, focusing solely on agglomerative clustering and activity monitoring to detect and mitigate adversarial influences.

% The primary motivation behind DROPlet is to provide an efficient and domain-agnostic defense mechanism that can be easily integrated into existing FL systems without requiring significant computational resources. Since DROPlet does not depend on task-specific features or data modality assumptions, it is applicable across a wide range of FL tasks, including image, text, and tabular data. This versatility ensures that DROPlet can serve as a general-purpose defense framework in diverse real-world FL applications.

% DROPlet also excels in terms of speed and scalability. By excluding the knowledge distillation step, it achieves faster round completion times and reduces the server-side processing load, making it particularly suitable for large-scale FL deployments where computational efficiency is a critical concern. Despite its lightweight nature, DROPlet maintains strong defenses by relying on robust clustering to detect anomalies and activity monitoring to penalize persistently malicious clients. \anshuman{Forward-reference that it does not work in all settings. In fact, this could even be a new "baseline", replacing the likes of Multi-KRUM and Median} This combination ensures reliable performance against a wide range of adversarial strategies, making DROPlet another practical choice for defending FL systems in environments with limited computational resources.

\subsection{DROPlet: a Lightweight, Scalable Defense Mechanism}  
\label{subsec:droplet}  

To provide a faster, more resource-efficient alternative to DROP, we introduce \textit{DROPlet}, a lightweight, server-side plugin designed for easy deployment and high scalability.
Unlike DROP, DROPlet omits the knowledge distillation component and focuses solely on agglomerative clustering and activity monitoring to mitigate adversarial influences with minimal overhead.  

DROPlet is task-agnostic, with no assumptions on the underlying data modality, making it applicable to various FL tasks. This versatility allows DROPlet to function as a general-purpose defense mechanism in diverse FL deployments. By eliminating the computationally intensive knowledge distillation step, it achieves faster round completion times, making it particularly suitable for large-scale FL systems where computational efficiency is crucial.  

Despite its lightweight design, DROPlet offers robust protection by leveraging clustering to detect anomalies and activity monitoring to penalize malicious clients. However, it may not be effective against stealthy, low-DPR attacks (\cref{shortsec:droplet_results}).
% \anshuman{Forward-reference that it does not work in all settings. In fact, this could even be a new "baseline", replacing the likes of Multi-KRUM and Median} \georgios{Good idea, refined the text.}
Nevertheless, DROPlet remains a competitive baseline for FL defense evaluations, offering significant protection while being faster and easier to deploy compared to classical methods like Multi-KRUM and Median. 

\section{Experimental Evaluation}\label{section:experiments}
We already achieved our primary objective of deriving time-series-specific subsampling guarantees for DP-SGD adapted to forecasting.
Our main goal for this section is to investigate the trade-offs we discovered in discussing these guarantees.
In addition, we train common probabilistic forecasting architectures on standard datasets to verify the feasibility of training deep differentially private forecasting models while retaining meaningful utility.
The full experimental setup  is described in~\cref{appendix:experimental_setup}.
%An implementation will be made available upon publication.

\subsection{Trade-Offs in Structured Subsampling}

\begin{figure}
    \vskip 0.2in
    \centering
        \includegraphics[width=0.99\linewidth]{figures/experiments/eval_pld_deterministic_vs_random_top_level/daily_20_32_main.pdf}
        \vskip -0.3cm
        \caption{Top-level deterministic iteration (\cref{theorem:deterministic_top_level_wr}) vs top-level WOR sampling (\cref{theorem:wor_top_level_wr}) for $\numinstances=1$.
        Sampling is more private despite requiring more compositions.}
        \label{fig:deterministic_vs_random_top_level_daily_main}
    \vskip -0.2in
\end{figure}




For the following experiments, we assume that we have $N=320$ sequences, batch size $\batchsize = 32$, and noise scale $\sigma = 1$.
We further assume $L=10  (L_F + L_C) + L_F - 1$, so that 
the chance of bottom-level sampling a subsequence containing any specific element is 
$r=0.1$ when choosing $\numinstances = 1$ as the number of subsequences.
In~\cref{appendix:extra_experiments_eval_pld}, we repeat all experiments with a wider range of parameters.
All results are consistent with the ones shown here.

\textbf{Number of Subsequences $\bm{\numinstances}$.}
Let us begin with a trade-off inherent to bi-level subsampling:
We can achieve the same batch size $\batchsize$ with different $\numinstances$, each
leading to different top- and bottom-level amplification.
We claim that $\numinstances = 1$ (i.e., maximum bottom-level amplification) is preferable.
For a fair comparison, we compare our provably tight guarantee for $\numinstances=1$ (\cref{theorem:wor_top_level_wr})
with optimistic lower bounds for $\numinstances > 1$ (\cref{theorem:wor_top_wr_bottom_upper})
instead of our sound upper bounds (\cref{theorem:wor_top_level_wr_general}), i.e.,
we make the competitors stronger.
As shown in~\cref{fig:monotonicity_daily_main}, $\numinstances = 1$ only has smaller $\delta(\epsilon)$ for $\epsilon \geq 10^{-1}$ when considering a single training step.
However, after $100$-fold composition, $\numinstances = 1$ achieves smaller $\delta(\epsilon)$ even in $[10^{-3}, 10^{-1}]$ (see~\cref{fig:monotonicity_composed_daily_main}).
Our explanation is that $\numinstances > 1$ results in larger $\delta(\epsilon)$ for large $\epsilon$, i.e., is more likely to have a large privacy loss.
Because the privacy loss of a composed mechanism is the sum of component privacy losses~\cite{sommer2018privacy}, this is problematic when performing multiple training steps.
We shall thus later use $\numinstances=1$ for training.

%Intuitively, $\delta(\epsilon)$ can be interpreted as the probability that the log-likelihood ratio of $M_x$ and $M_{x'}$ (``privacy loss'') exceeds $\epsilon$.\footnote{For the formal relation between privay loss and privacy profiles, see~\cref{lemma:profile_from_pld} taken from~\cite{balle2018improving}}


\textbf{Step- vs Epoch-Level Accounting.}
Next, we show the benefit of top-level sampling sequences (\cref{theorem:wor_top_level_wr}) instead of deterministically iterating over them (\cref{theorem:deterministic_top_level_wr}), even though we risk privacy leakage at every training step.
For our parameterization and $\numinstances=1$, top-level sampling with replacement requires $10$ compositions per epoch.
As shown in~\cref{fig:deterministic_vs_random_top_level_daily_main}, the resultant epoch-level profile is nevertheless smaller, and remains so after $10$ epochs.
This is consistent with any work on DP-SGD (e.g., \cite{abadi2016deep}) that uses subsampling instead of deterministic iteration.

\textbf{Epoch Privacy vs Length.} In~\cref{appendix:extra_experiments_epoch_length} we additionally explore the fact that, if we wanted to use deterministic top-level iteration, 
the number of subsequences 
$\numinstances$ would affect epoch length.
As expected, we observe that composing many private mechanisms ($\numinstances=1$) is preferable to composing few much less private mechanisms ($\numinstances > 1$) 
when considering a fixed number of training steps.

\begin{figure}
    \vskip 0.2in
    \centering
        \includegraphics[width=0.99\linewidth]{figures/experiments/eval_pld_label_noise/daily_30_32_main.pdf}
        \vskip -0.3cm
        \caption{Varying label noise $\sigma_F$ for top-level WOR and bottom-level WR  (\cref{theorem:data_augmentation_general}) with $\sigma_C = 0, \numinstances=1$.
        Increasing $\sigma_F$ is equivalent to decreasing forecast length.
        }
        \label{fig:label_noise_daily_main}
    \vskip -0.2in
\end{figure}

\textbf{Amplification by Label Perturbation.}
Finally, because the way in which adding Gaussian noise to the context and/or forecast window 
amplifies privacy (\cref{theorem:data_augmentation_general}) 
may be somewhat opaque, let us consider top-level sampling without replacement, bottom-level sampling with replacement,
$\numinstances=1$, $\sigma_C=0$, and varying label noise standard deviations $\sigma_F$. 
As shown in~\cref{fig:label_noise_daily_main}, increasing $\sigma_F$ has the same effect as letting the forecast length $L_C$ go to zero, i.e., eliminates the risk of leaking private information if it appears in the forecast window.
Of course, this data augmentation 
will have an effect on model utility, which we investigate shortly.

\begin{figure*}
\centering
\vskip 0.2in
    \begin{subfigure}{0.49\textwidth}
        \includegraphics[]{figures/experiments/eval_pld_monotonicity_composed/daily_20_32_1_main.pdf}
        \caption{Training step $1$}\label{fig:monotonicity_daily_main}
    \end{subfigure}
    \hfill
    \begin{subfigure}{0.49\textwidth}
        \includegraphics[]{figures/experiments/eval_pld_monotonicity_composed/daily_20_32_100_main.pdf}
        \caption{Training step $100$}\label{fig:monotonicity_composed_daily_main}
    \end{subfigure}\caption{
    Top-level WOR and bottom-level WR sampling under varying number of subsequences.
    Under composition, even optimistic lower bounds (\cref{theorem:wor_top_wr_bottom_upper}) 
    indicate worse privacy for $\numinstances > 1$ than 
    our tight upper bound for $\numinstances=1$ (\cref{theorem:wor_top_level_wr}).}
    \label{fig:monotonicity_daily_main_container}
\vskip -0.2in
\end{figure*}


\subsection{Application to Probabilistic Forecasting}
While the contribution of our work lies in formally analyzing the privacy of DP-SGD adapted to forecasting, 
training models with this algorithm can serve as a sanity-check to verify that the guarantees are sufficiently strong to retain meaningful utility under non-trivial privacy budgets.


\begin{table}[b]
\vskip -0.38cm
\caption{Average CRPS on \texttt{traffic} for $\delta=10^{-7}$. Seasonal, AutoETS, and models with $\epsilon=\infty$ are without noise.}
\label{table:1_event_training_traffic_main}
\vskip 0.18cm
\begin{center}
\begin{small}
\begin{sc}
\begin{tabular}{lcccc}
\toprule
Model & $\epsilon = 0.5$ & $\epsilon = 1$ & $\epsilon = 2$ &  $\epsilon = \infty$ \\
\midrule
SimpleFF & $0.207$ & $0.195$ & $0.193$ & $0.136$ \\ 
DeepAR & $\mathbf{0.157}$ & $\mathbf{0.145}$ & $\mathbf{0.142}$ & $\mathbf{0.124}$ \\
iTransf. & $0.211$ & $0.193$ & $0.188$ & $0.135$ \\
DLinear & $0.204$ & $0.192$ & $0.188$ & $0.140$ \\
\midrule
Seasonal   & $0.251$ & $0.251$ & $0.251$ & $0.251$\\
AutoETS   & $0.407$ & $0.407$ & $0.407$ & $0.407$\\
\bottomrule
\end{tabular}
\end{sc}
\end{small}
\end{center}
\vskip -0.1in
\end{table}

\textbf{Datasets, Models, and Metrics.}
We consider three standard benchmarks: \texttt{traffic}, \texttt{electricity}, and \texttt{solar\_10\_minutes} as used in~\cite{Lai2018modeling}.
We further consider four common architectures: 
A two-layer feed-forward neural network (``SimpleFeedForward''), a recurrent neural network (``DeepAR''~\cite{salinas2020deepar}),
an encoder-only transformer (``iTransformer''~\cite{liu2024itransformer}), and a refined feed-forward network proposed to compete with attention-based models (``DLinear''~\cite{zeng2023transformers}).
We let these architectures parameterize elementwise $t$-distributions to obtain probabilistic forecasts.
We measure the quality of these probabilistic forecasts using continuous ranked probability scores (CRPS), which we approximate via mean weighted quantile losses (details in~\cref{appendix:metrics}).
As a reference for what constitutes ``meaningful utility'', we compare against seasonal na\"{i}ve forecasting and exponential smoothing (``AutoETS'') without introducing any noise.
All experiments are repeated with $5$ random seeds.


\textbf{Event-Level Privacy.} \cref{table:1_event_training_traffic_main} shows CRPS of all models on the \texttt{traffic} test set 
when setting $\delta=10^{-7}$, and training on the training set until reaching a pre-specified $\epsilon$
with $1$-event-level privacy. For the other datasets and standard deviations, see~\cref{appendix:privacy_utility_tradeoff_event_level_privacy}.
The column $\epsilon=\infty$ indicates non-DP training.
As can be seen, models can retain much of their utility and outperform the baselines, even for $\epsilon \leq 1$ which is generally considered a small privacy budget~\cite{ponomareva2023dp}.
For instance, the average CRPS of DeepAR on the traffic dataset is $0.124$ with non-DP training and $0.157$ for $\epsilon=0.5$.
Note that, since all models are trained using  our tight privacy analysis,
which specific model performs best  on which specific dataset is orthogonal to our contribution. 

\textbf{Other results.}
In~\cref{appendix:privacy_utility_tradeoff_user_level_privacy} we additionally train with $w$-event and $w$-user privacy.
In~\cref{appendix:privacy_utility_tradeoff_label_privacy}, we demonstrate that label perturbations can offer an improved privacy--utility trade-off. 
All results confirm that our guarantees for DP-SGD adapted to forecasting are strong enough to enable provably private training while retaining utility.


\section{Conclusion}
\label{sec:Conclusion}
In this paper, we proposed a complete real-time planning and control approach for continuous, reliable, and fast online generation of dynamically feasible Bernstein trajectories and control for FW aircrafts. The generated trajectories span kilometers, navigating through multiple waypoints. By leveraging differential flatness equations for coordinated flight, we ensure precise trajectory tracking. Our approach guarantees smooth transitions from simulation to real-world applications, enabling timely field deployment. The system also features a user-friendly mission planning interface. Continuous replanning  maintains the rajectory curvature 
$\kappa$ within limits, preventing abrupt roll changes.

Future works will include the ability to add  a higher-level kinodynamic path planner to optimize waypoint spatial allocation and improve replanning success, and enhancing the trajectory-tracking algorithm by refining the aerodynamic coefficient estimation. 

% Merged into conclusion
% \section{Discussion of Assumptions}\label{sec:discussion}
In this paper, we have made several assumptions for the sake of clarity and simplicity. In this section, we discuss the rationale behind these assumptions, the extent to which these assumptions hold in practice, and the consequences for our protocol when these assumptions hold.

\subsection{Assumptions on the Demand}

There are two simplifying assumptions we make about the demand. First, we assume the demand at any time is relatively small compared to the channel capacities. Second, we take the demand to be constant over time. We elaborate upon both these points below.

\paragraph{Small demands} The assumption that demands are small relative to channel capacities is made precise in \eqref{eq:large_capacity_assumption}. This assumption simplifies two major aspects of our protocol. First, it largely removes congestion from consideration. In \eqref{eq:primal_problem}, there is no constraint ensuring that total flow in both directions stays below capacity--this is always met. Consequently, there is no Lagrange multiplier for congestion and no congestion pricing; only imbalance penalties apply. In contrast, protocols in \cite{sivaraman2020high, varma2021throughput, wang2024fence} include congestion fees due to explicit congestion constraints. Second, the bound \eqref{eq:large_capacity_assumption} ensures that as long as channels remain balanced, the network can always meet demand, no matter how the demand is routed. Since channels can rebalance when necessary, they never drop transactions. This allows prices and flows to adjust as per the equations in \eqref{eq:algorithm}, which makes it easier to prove the protocol's convergence guarantees. This also preserves the key property that a channel's price remains proportional to net money flow through it.

In practice, payment channel networks are used most often for micro-payments, for which on-chain transactions are prohibitively expensive; large transactions typically take place directly on the blockchain. For example, according to \cite{river2023lightning}, the average channel capacity is roughly $0.1$ BTC ($5,000$ BTC distributed over $50,000$ channels), while the average transaction amount is less than $0.0004$ BTC ($44.7k$ satoshis). Thus, the small demand assumption is not too unrealistic. Additionally, the occasional large transaction can be treated as a sequence of smaller transactions by breaking it into packets and executing each packet serially (as done by \cite{sivaraman2020high}).
Lastly, a good path discovery process that favors large capacity channels over small capacity ones can help ensure that the bound in \eqref{eq:large_capacity_assumption} holds.

\paragraph{Constant demands} 
In this work, we assume that any transacting pair of nodes have a steady transaction demand between them (see Section \ref{sec:transaction_requests}). Making this assumption is necessary to obtain the kind of guarantees that we have presented in this paper. Unless the demand is steady, it is unreasonable to expect that the flows converge to a steady value. Weaker assumptions on the demand lead to weaker guarantees. For example, with the more general setting of stochastic, but i.i.d. demand between any two nodes, \cite{varma2021throughput} shows that the channel queue lengths are bounded in expectation. If the demand can be arbitrary, then it is very hard to get any meaningful performance guarantees; \cite{wang2024fence} shows that even for a single bidirectional channel, the competitive ratio is infinite. Indeed, because a PCN is a decentralized system and decisions must be made based on local information alone, it is difficult for the network to find the optimal detailed balance flow at every time step with a time-varying demand.  With a steady demand, the network can discover the optimal flows in a reasonably short time, as our work shows.

We view the constant demand assumption as an approximation for a more general demand process that could be piece-wise constant, stochastic, or both (see simulations in Figure \ref{fig:five_nodes_variable_demand}).
We believe it should be possible to merge ideas from our work and \cite{varma2021throughput} to provide guarantees in a setting with random demands with arbitrary means. We leave this for future work. In addition, our work suggests that a reasonable method of handling stochastic demands is to queue the transaction requests \textit{at the source node} itself. This queuing action should be viewed in conjunction with flow-control. Indeed, a temporarily high unidirectional demand would raise prices for the sender, incentivizing the sender to stop sending the transactions. If the sender queues the transactions, they can send them later when prices drop. This form of queuing does not require any overhaul of the basic PCN infrastructure and is therefore simpler to implement than per-channel queues as suggested by \cite{sivaraman2020high} and \cite{varma2021throughput}.

\subsection{The Incentive of Channels}
The actions of the channels as prescribed by the DEBT control protocol can be summarized as follows. Channels adjust their prices in proportion to the net flow through them. They rebalance themselves whenever necessary and execute any transaction request that has been made of them. We discuss both these aspects below.

\paragraph{On Prices}
In this work, the exclusive role of channel prices is to ensure that the flows through each channel remains balanced. In practice, it would be important to include other components in a channel's price/fee as well: a congestion price  and an incentive price. The congestion price, as suggested by \cite{varma2021throughput}, would depend on the total flow of transactions through the channel, and would incentivize nodes to balance the load over different paths. The incentive price, which is commonly used in practice \cite{river2023lightning}, is necessary to provide channels with an incentive to serve as an intermediary for different channels. In practice, we expect both these components to be smaller than the imbalance price. Consequently, we expect the behavior of our protocol to be similar to our theoretical results even with these additional prices.

A key aspect of our protocol is that channel fees are allowed to be negative. Although the original Lightning network whitepaper \cite{poon2016bitcoin} suggests that negative channel prices may be a good solution to promote rebalancing, the idea of negative prices in not very popular in the literature. To our knowledge, the only prior work with this feature is \cite{varma2021throughput}. Indeed, in papers such as \cite{van2021merchant} and \cite{wang2024fence}, the price function is explicitly modified such that the channel price is never negative. The results of our paper show the benefits of negative prices. For one, in steady state, equal flows in both directions ensure that a channel doesn't loose any money (the other price components mentioned above ensure that the channel will only gain money). More importantly, negative prices are important to ensure that the protocol selectively stifles acyclic flows while allowing circulations to flow. Indeed, in the example of Section \ref{sec:flow_control_example}, the flows between nodes $A$ and $C$ are left on only because the large positive price over one channel is canceled by the corresponding negative price over the other channel, leading to a net zero price.

Lastly, observe that in the DEBT control protocol, the price charged by a channel does not depend on its capacity. This is a natural consequence of the price being the Lagrange multiplier for the net-zero flow constraint, which also does not depend on the channel capacity. In contrast, in many other works, the imbalance price is normalized by the channel capacity \cite{ren2018optimal, lin2020funds, wang2024fence}; this is shown to work well in practice. The rationale for such a price structure is explained well in \cite{wang2024fence}, where this fee is derived with the aim of always maintaining some balance (liquidity) at each end of every channel. This is a reasonable aim if a channel is to never rebalance itself; the experiments of the aforementioned papers are conducted in such a regime. In this work, however, we allow the channels to rebalance themselves a few times in order to settle on a detailed balance flow. This is because our focus is on the long-term steady state performance of the protocol. This difference in perspective also shows up in how the price depends on the channel imbalance. \cite{lin2020funds} and \cite{wang2024fence} advocate for strictly convex prices whereas this work and \cite{varma2021throughput} propose linear prices.

\paragraph{On Rebalancing} 
Recall that the DEBT control protocol ensures that the flows in the network converge to a detailed balance flow, which can be sustained perpetually without any rebalancing. However, during the transient phase (before convergence), channels may have to perform on-chain rebalancing a few times. Since rebalancing is an expensive operation, it is worthwhile discussing methods by which channels can reduce the extent of rebalancing. One option for the channels to reduce the extent of rebalancing is to increase their capacity; however, this comes at the cost of locking in more capital. Each channel can decide for itself the optimum amount of capital to lock in. Another option, which we discuss in Section \ref{sec:five_node}, is for channels to increase the rate $\gamma$ at which they adjust prices. 

Ultimately, whether or not it is beneficial for a channel to rebalance depends on the time-horizon under consideration. Our protocol is based on the assumption that the demand remains steady for a long period of time. If this is indeed the case, it would be worthwhile for a channel to rebalance itself as it can make up this cost through the incentive fees gained from the flow of transactions through it in steady state. If a channel chooses not to rebalance itself, however, there is a risk of being trapped in a deadlock, which is suboptimal for not only the nodes but also the channel.

\section{Conclusion}
This work presents DEBT control: a protocol for payment channel networks that uses source routing and flow control based on channel prices. The protocol is derived by posing a network utility maximization problem and analyzing its dual minimization. It is shown that under steady demands, the protocol guides the network to an optimal, sustainable point. Simulations show its robustness to demand variations. The work demonstrates that simple protocols with strong theoretical guarantees are possible for PCNs and we hope it inspires further theoretical research in this direction.
% \section{Conclusion}
In this work, we propose a simple yet effective approach, called SMILE, for graph few-shot learning with fewer tasks. Specifically, we introduce a novel dual-level mixup strategy, including within-task and across-task mixup, for enriching the diversity of nodes within each task and the diversity of tasks. Also, we incorporate the degree-based prior information to learn expressive node embeddings. Theoretically, we prove that SMILE effectively enhances the model's generalization performance. Empirically, we conduct extensive experiments on multiple benchmarks and the results suggest that SMILE significantly outperforms other baselines, including both in-domain and cross-domain few-shot settings.

\ifpreprintversion
\section*{Acknowledgments}
This research was supported by the Department of Defense
Multidisciplinary Research Program of the University Research
Initiative (MURI) under contract W911NF-21-1-0322, and by NSF under grants CNS-2312875 and CNS-2331081.

\fi

\bibliographystyle{ACM-Reference-Format}
\bibliography{main}

\appendix
\section{Bound Analysis on Number of Malicious Clients per Round}
\label{app:bound_analysis}

Let \( N \) denote the total number of clients and \( M \) the number of malicious clients. The ratio of malicious clients is given by the Malicious Client Ratio (MCR), \( \rho = \frac{M}{N} \). In each round of Federated Learning (FL), we sample \( C \) clients randomly from the \( N \) total clients.
%
Let $\mathcal{M}$ be the number of malicious clients (out of $C$) sampled in some round. We are interested in finding a lower bound on the probability that there are more malicious clients than benign ones in a round of FL, \ie \( P\left(\mathcal{M} \geq \frac{C}{2}\right) \).

The selection of malicious clients can be modeled as a Binomial distribution where probability of success is the MCR ($\rho$). Thus,  our desired probability can be written in terms of the CDF $F$ of this distribution:
\begin{align}
    P\left(\mathcal{M} \geq \frac{C}{2}\right) = 1 - F\left(\frac{C}{2}, C, \rho\right).
\end{align}
Using a Chernoff bound \citep{arratia1989tutorial}, we can thus obtain a lower bound on the probability that malicious clients outnumber benign ones in a given round as:
\begin{align}
    P\left(\mathcal{M} \geq \frac{C}{2}\right) &\geq 1 - \exp\left(-C \cdot \left(\frac{1}{2}\left(\ln\frac{1}{2\rho} + \ln\frac{1}{2(1-\rho)}\right)\right)\right) \\
    & = 1 - \exp\left(C \cdot \frac{1}{2} \ln\left(4\rho(1-\rho)\right)\right) \\
    & = 1 - \left(4\rho(1-\rho)\right)^{\frac{C}{2}}.
\end{align}

% \section{Calculations}
\label{sec:calculations}

\subsection{Chernoff Bound Analysis}
\label{sec:chernoff_appendix}

In this section, we provide the detailed steps used to compute the probability bound for the number of malicious clients exceeding a certain threshold in any given round. Specifically, we apply the Chernoff bound to analyze the probability that the number of malicious clients \( M \) in a randomly selected subset of clients exceeds half the total number of selected clients \( C \).

The Chernoff bound provides the following inequality for the probability that \( M \) exceeds \((1 + \delta)\) times its expected value:
\begin{equation}
    \mathbb{P}\left(M \geq (1 + \delta) \cdot \rho \cdot C \right) \leq \exp\left(-\frac{\delta^2 \cdot \rho \cdot C}{2 + \delta}\right),
\end{equation}
where:
\begin{itemize}
    \item \( \rho \) is the malicious client ratio (MCR), i.e., the fraction of total clients that are adversarial.
    \item \( C \) is the total number of clients selected in each round.
    \item \( \delta \) controls how much larger \( M \) is compared to its expected value \( \mathbb{E}[M] = \rho \cdot C \).
\end{itemize}

% \noindent \textbf{Step 1: Setting the threshold}
\shortsection{Step 1: Setting the threshold}
We aim to compute the probability that the number of malicious clients \( M \) exceeds 50\% of the selected clients, i.e., \( M \geq \frac{C}{2} \). Given \( C = 20 \) and \( \rho = 0.4 \), we want:
\[
(1 + \delta) \cdot \rho \cdot C = 10.
\]
Substituting the values of \( \rho \) and \( C \):
\begin{align}
    (1 + \delta) \cdot 0.4 \cdot 20 & = 10,     \nonumber
\end{align}
which gives us $\delta = 0.25$.
% \[
% (1 + \delta) \cdot 8 = 10,
% \]
% \[
% 1 + \delta = \frac{10}{8} = 1.25,
% \]
% \[
% \delta = 1.25 - 1 = 0.25.
% \]

\shortsection{Step 2: Applying the bound}
% \noindent \textbf{Step 2: Applying the bound}
Now, we plug in \( \delta = 0.25 \), \( \rho = 0.4 \), and \( C = 20 \) into the Chernoff bound:
\begin{align}
    \mathbb{P}(M \geq 10) & \leq \exp\left(-\frac{(0.25)^2 \cdot 0.4 \cdot 20}{2 + 0.25}\right) \\
    & = \approx 0.8007374
\end{align}
% \[
% \mathbb{P}(M \geq 10) \leq \exp\left(-\frac{(0.25)^2 \cdot 0.4 \cdot 20}{2 + 0.25}\right),
% \]
% \[
% = \exp\left(-\frac{0.5}{2.25}\right),
% \]
% \[
% = \exp\left(-0.22222\ldots\right),
% \]
% \[
% \approx 0.8007374.
% \]
Thus, the probability that the number of malicious clients exceeds half of the selected clients in a given round is approximately 0.80, indicating that such an event is relatively likely under this configuration. This highlights the need for robust mechanisms that can handle varying malicious client ratios across rounds.

\section{Normal Approximation for Probability of Malicious Majority (Using M and B)}
\label{appendix:normal-approx-MB}

Assume the following setup:
\begin{itemize}
  \item $N$ = total number of clients.
  \item $M_{\text{tot}}$ = total number of malicious clients, where
        $
            M_{\text{tot}} 
            = (\text{MCR}) \times N.
        $
  \item $B_{\text{tot}} = N - M_{\text{tot}}$ = total number of benign (non-malicious) clients.
  \item Each client is selected (independently) with probability 
        $
          p = \frac{C}{N}.
        $
        Thus, the expected number of clients selected in each round is $C$.
\end{itemize}

Let
\[
    M \;\sim\; \mathrm{Binomial}\bigl(M_{\text{tot}},\, p\bigr),
    \quad
    B \;\sim\; \mathrm{Binomial}\bigl(B_{\text{tot}},\, p\bigr),
\]
where 
\[
   B_{\text{tot}} \;=\; N - M_{\text{tot}}.
\]
The random variables $M$ and $B$ are assumed independent, since each client (malicious or benign) is selected via its own independent Bernoulli$(p)$ trial.

\subsection*{Goal}
We want to find the probability that malicious clients \emph{form a strict majority} among the selected clients, that is:
\[
   \Pr\bigl(M > B\bigr).
\]
For large $N$, we can apply the Central Limit Theorem to approximate these binomial distributions by normal distributions:
\[
    M 
    \;\approx\; 
    \mathcal{N}\bigl(\mu_{M}, \;\sigma_{M}^2\bigr),
    \qquad
    B 
    \;\approx\;
    \mathcal{N}\bigl(\mu_{B}, \;\sigma_{B}^2\bigr),
\]
where
\[
    \mu_{M} 
    \;=\; 
    M_{\text{tot}}\,p 
    \;=\; 
    \bigl(\text{MCR}\times N\bigr) 
    \,\frac{C}{N}
    \;=\;
    (\text{MCR}) \times C,
    \quad
    \sigma_{M}^2 
    \;=\;
    M_{\text{tot}}\, p\,(1-p),
\]
\[
    \mu_{B} 
    \;=\;
    B_{\text{tot}}\,p
    \;=\;
    \bigl(N - M_{\text{tot}}\bigr)\,\frac{C}{N}
    \;=\;
    \bigl(1 - \text{MCR}\bigr)\,C,
    \quad
    \sigma_{B}^2 
    \;=\;
    B_{\text{tot}}\,p\,(1-p).
\]
Since $M$ and $B$ are independent, the difference $(M - B)$ is approximately normally distributed with
\[
    M - B 
    \;\approx\;
    \mathcal{N}\bigl(\mu_{M} - \mu_{B},\;
                     \sigma_{M}^2 + \sigma_{B}^2 \bigr).
\]
Observe that
\[
    \mu_{M} - \mu_{B}
    \;=\;
    (\text{MCR}\cdot C) - \bigl((1 - \text{MCR})\cdot C\bigr)
    \;=\;
    C\,\bigl(2\,\text{MCR} - 1\bigr),
\]
and
\[
  \sigma_{M}^2 + \sigma_{B}^2
  \;=\;
  M_{\text{tot}}\,p\,(1-p)
  \;+\;
  B_{\text{tot}}\,p\,(1-p)
\]
\[
  =\;
  \bigl(M_{\text{tot}} + B_{\text{tot}}\bigr)\,p\,(1-p)
  \;=\;
  N\,\frac{C}{N}\,\Bigl(1 - \tfrac{C}{N}\Bigr)
\]
\[
  \;\approx\;
  C.
\]
assuming $C \ll N$. Thus,
\[
    M - B
    \;\approx\;
    \mathcal{N}\Bigl( C\,\bigl(2\,\text{MCR} - 1\bigr), \; C \Bigr).
\]
We want $\Pr(M > B)$, i.e.\ $\Pr(M - B \ge 0)$. We standardize the variable:
\[
  \Pr\bigl(M - B \;\ge\; 0\bigr)
  \;=\;
  \Pr\!\Bigl(\,
    \frac{M - B \;-\; \bigl[C\,(2\,\text{MCR} - 1)\bigr]}{\sqrt{C}}
    \;\ge\;
    \frac{-\,C\,(2\,\text{MCR} - 1)}{\sqrt{C}}
  \Bigr).
\]

If $Z$ is a standard normal random variable, then
\[
  \Pr\bigl(M - B \;\ge\; 0\bigr)
  \;\approx\;
  \Pr\!\Bigl(
    Z \;\ge\; -\,\frac{C\,(2\,\text{MCR} - 1)}{\sqrt{C}}
  \Bigr)
\]
\[
  =
  \Pr\!\Bigl(
    Z \;\le\; \frac{C\,(2\,\text{MCR} - 1)}{\sqrt{C}}
  \Bigr).
\]


Because $1 - \Phi(-x) = \Phi(x)$ for the standard normal CDF $\Phi$, it follows that
\[
    \Pr\bigl(M > B\bigr)
    \;\approx\;
    \Phi\!\Bigl(\sqrt{C}\,\bigl[2\,\text{MCR} - 1\bigr]\Bigr).
\]

\paragraph{Interpretation.}
\begin{itemize}
    \item If $\text{MCR} < 0.5$, then $2\,\text{MCR} - 1 < 0$, and the argument of $\Phi(\,\cdot\,)$ is negative and becomes more negative as $C$ increases. Hence, $\Pr(M > B)$ goes to $0$.
    \item If $\text{MCR} = 0.5$, then the argument of $\Phi$ is zero, so the probability is $0.5$.
    \item If $\text{MCR} > 0.5$, then $2\,\text{MCR} - 1 > 0$, and as $C$ increases, the argument of $\Phi$ becomes large positive. Hence, $\Pr(M > B)$ goes to $1$.
\end{itemize}

This derivation shows that, for large $N$ and not-too-large $C$ (so $p = C/N \ll 1$), the probability of a \emph{malicious majority} in a binomially sampled subset is well approximated by
\[
    \Pr\bigl(\text{Malicious majority}\bigr)
    \;\approx\;
    \Phi\Bigl(\sqrt{C}\,\bigl[2\,\text{MCR} - 1\bigr]\Bigr),
\]
where $\Phi(\,\cdot\,)$ is the standard normal cumulative distribution function.

\section{Related Work}
\label{sec:related_work}

% \todo{Introduce related works and highlight gaps- no good defense that works across multiple DPR assumptions, iid and non-iid, etc. also highlight requirement of clean-data which defenses use, which is even more so unlikely to be present for FL. if we want, could even showcase a simple ``finetune at end with clean data" baseline and demonstrate that it beats (I'd think it would?) most existing ``defenses"}

% \anshuman{Mention how FLIP slightly changes the defender's capabilities by allowing sample rejection, but as we find, this also ends up rejecting a good percentage (I believe it was 10 or so but can check later) of clean samples that do not have any triggers (also need to calculate what percentage of these were classified correctly) Metrics under this can be slightly deceptive, as our defense with the same rejection capacity can claim MTA of 82\% but that ignores benign data that is 'rejected'. In fact iirc (we can add concrete numbers later) our conditional clean accuracy is not too far from theirs (not for Table 2 at least)}  

The increasing adoption of FL has spurred extensive research on defending against poisoning attacks, particularly backdoor attacks, which pose a serious threat due to their ability to embed hidden malicious behaviors without degrading the model’s overall performance. Several defense mechanisms have been proposed to mitigate these threats, including robust aggregation methods, trust-based filtering, and anomaly detection techniques. Despite these efforts, existing defenses often exhibit limitations when evaluated across diverse attack configurations, especially under varying data poisoning rates, malicious client ratios within the federation and non-IID client data distributions.

\textit{Robust aggregation-based defenses.} Coordinate-wise median aggregation \citep{yin2018byzantine} and Multi-Krum \citep{blanchard2017machine} are two classical approaches designed to mitigate Byzantine failures by robustly aggregating model updates. The coordinate-wise median computes the median value for each model parameter across client updates, neutralizing the impact of outliers. Multi-Krum, on the other hand, iteratively selects and aggregates updates that are closest to the majority based on pairwise distances, making it resilient to adversarial updates. While these methods are effective against simple poisoning attacks, they struggle to defend against stealthy backdoor attacks, particularly when the data poisoning rate is low or the client data distribution is highly skewed. Additionally, both methods assume IID client data, limiting their robustness in realistic non-IID FL settings.

\textit{Trust-based defenses.} FLTrust \citep{cao2021fltrust} introduces a server-side reference model trained on a small trusted dataset to measure the trustworthiness of client updates. Only updates that align closely with the reference model are aggregated, providing a strong baseline against various adversarial strategies. However, FLTrust’s performance can degrade when the trusted dataset is not fully representative of the overall data distribution and if the attack is very stealthy.  

\textit{Similarity and anomaly detection-based defenses.} Fool’s Gold \citep{fung2018mitigating} uses similarity-based clustering to detect and penalize clients that contribute updates with similar gradients across multiple rounds, under the assumption that adversaries tend to behave similarly. While this approach reduces the contribution of malicious clients, its reliance on similarity metrics makes it susceptible to adaptive adversaries that can evade detection by introducing slight randomness in their updates. FLAME \citep{nguyen2022flame}, which employs adaptive clipping to detect and filter anomalous gradients, also shows promise in defending against targeted attacks. However, as we observe in our experiments, FLAME’s adaptive clipping can become overly aggressive, preventing convergence in some scenarios.  

\textit{Penultimate layer representation-based defenses.} FLARE \citep{wang2022flare} leverages discrepancies in penultimate layer representations (PLR) of model updates to assign trust scores and filter out potentially malicious updates. This approach works well for overt poisoning attempts but, as we demonstrate, struggles to mitigate stealthy attacks with low DPR, where poisoned updates closely resemble benign ones.  

\textit{Adversarial training-based and sample rejection defenses.} FLIP \citep{zhang2023flip} represents a different class of defenses by introducing the ability to reject individual samples during aggregation. By leveraging adversarial training and low-confidence refusals, FLIP aims to reconstruct client-side triggers and reject poisoned updates. While effective to some extent, FLIP’s rejection mechanism often discards a non-negligible fraction of clean samples, which can negatively impact the main task accuracy (MTA). As observed in our experiments, this trade-off in rejection-based defenses can lead to deceptively high MTA metrics, as benign samples that are incorrectly rejected are excluded from evaluation. In contrast, our proposed defense does not require such sample-level rejection, thereby preserving the overall integrity of the clean data and achieving a better balance between MTA and ASR across varying configurations. A crucial consideration when designing FL frameworks is that participating nodes often have limited computational resources, which can constrain the feasibility of complex operations. \eg FLIP relies on client-side adversarial training—a computationally expensive process. While this approach is viable in their setup, where only 10 clients are sampled per round, increasing the number of selected clients significantly extends the duration of each round, making it impractical for large-scale deployments.

Despite the progress made by these defenses, a significant gap remains: no existing method consistently performs well across different DPR levels, IID and non-IID data distributions, and varying MCR. Our proposed defense, DROP, addresses these issues by providing a robust, configuration-agnostic solution that is effective across a broad range of attack and learning configurations. 

% By releasing our codebase, we aim to provide a consistent testbed that future researchers can use to evaluate attacks and defenses under diverse scenarios, fostering more reproducible and standardized research in FL security.

\section{Experimental Details}
\label{app:exp_details}

\subsection{Baselines}
\label{app:baseline_details}

% \begin{table*}[h]
%     \centering
%     \begin{tabular}{l|cccccc}
%     \toprule
%     Method & Data Distribution & Poisoning Start & Attack-Agnostic & Client Selection & Malicious Clients & DPR\\
%     \midrule
%          FLIP \citep{zhang2023flip} & iid, d(0.5) & After Convergence & No & Enforced & 40\% & ...\\
%          FL-Trust \citep{cao2021fltrust} &  iid, weird &  & & & 20\%\\
%          FoolsGold \citep{fung2018mitigating} & \\
%          MultiKrum \citep{blanchard2017machine} & \\
%          FLAME \citep{nguyen2022flame} & \\
%          DROP (Ours) & iid, d(1.0) & Beginning & Yes & Random & 20\% & ...\\
%     \bottomrule
%     \end{tabular}
%     \caption{\anshuman{In hindsight, this table can go to the Appendix.}}
%     \label{tab:my_label}
% \end{table*}

% Each of these methods aims to limit the influence of malicious updates during aggregation. However, most works provide limited or inconsistent details about their evaluation setups, particularly concerning client learning configurations such as learning rate, batch size, and the number of local training epochs. For instance, Median is evaluated on a simpler learning task (MNIST) and specifies only the total number of participating clients. The authors provide no details about the client learning setup, including learning rate, batch size, or number of local training epochs. Similarly, Multi-Krum reduces the impact of outliers using robust statistics but focuses primarily on how the data is partitioned among clients. While it does evaluate the method across different batch sizes, it lacks significant discussion of the broader local training setup. FLTrust adopts a trusted server-side reference model to filter anomalous updates. The authors report using a "combined" learning rate of 0.002, a batch size of 64, and a single local training epoch. In contrast, FoolsGold, which identifies and penalizes suspiciously similar client contributions, does not explicitly report the learning rate or number of local training epochs in its evaluation. Instead, it mentions using batch sizes of 10 or 50 depending on the dataset. FLAME, which leverages anomaly detection to flag potentially malicious gradients, describes the structure of the federation but provides no information about the local learning setup, such as learning rate, batch size, or training epochs.

% \begin{itemize}
%     \item \textbf{FLIP \citep{zhang2023flip}}: We adapt the original implementation and hyper-parameters. The original defense assumes that both the defense and malicious client poisoning are triggered after model convergence, citing interference in convergence if the malicious activity begins earlier. However, our observations show that targeted backdoors do not disrupt model convergence even if initiated at the beginning of FL training. Starting the defense at convergence is thus ineffective as the poisoning has already occurred. On the other hand, starting too early results in suboptimal performance. Therefore, we activate the defense after the first 10/3 rounds for CIFAR-10/EMNIST respectively.
% \end{itemize}

Each of the defense methods which were presented aims to limit the influence of malicious updates during aggregation. However, most works provide limited or inconsistent details about their evaluation setups, particularly concerning client learning configurations such as learning rate, batch size, and the number of local training epochs. For instance, Median \citep{yin2018byzantine} is evaluated on a simpler learning task (MNIST \cite{mnist}) and specifies only the total number of participating clients, without providing key details about the client learning setup, such as the learning rate, batch size, or the number of local epochs. Similarly, Multi-Krum \citep{blanchard2017machine} reduces the impact of outliers using robust statistics but primarily focuses on how the data is partitioned among clients. While it does evaluate the method across different batch sizes, it lacks a detailed discussion of the broader local training setup. FLTrust \citep{cao2021fltrust} adopts a trusted server-side reference model to filter anomalous updates. The authors report using a "combined" learning rate of 0.002, a batch size of 64, and a single local training epoch. In contrast, FoolsGold \citep{fung2018mitigating}, which identifies and penalizes suspiciously similar client contributions, does not explicitly report the learning rate or the number of local epochs in its evaluation, only mentioning batch sizes of 10 or 50 depending on the dataset. FLAME \citep{nguyen2022flame}, which employs anomaly detection to flag potentially malicious gradients, describes the structure of the federation but omits critical information about the local learning setup, such as learning rate, batch size, or the number of epochs. FLIP \citep{zhang2023flip}, on the other hand, presents a more detailed setup. We adapted the original implementation and hyperparameters for our evaluation. The original defense assumes that both the defense and malicious client poisoning are triggered after model convergence, citing interference in convergence if malicious activity begins earlier. However, our observations show that targeted backdoors do not disrupt model convergence even when initiated at the beginning of FL training. Thus, starting the defense only after convergence is ineffective since poisoning has already occurred by that point. On the other hand, initiating the defense too early leads to suboptimal performance. Therefore, we activate the defense after the first 10 rounds for CIFAR-10 and after the first 3 rounds for EMNIST to balance effectiveness and performance.


\subsection{DROP Parameters}
\label{sec:drop_params}

% Knowledge distillation, particularly in the context of model stealing attacks, is inherently imperfect and cannot replicate the target model exactly. As a result, a minor decrease in MTA is expected. To mitigate this and ensure convergence in the FL setting, the knowledge distillation component of DROP is applied every \(K\) rounds instead of every round, allowing the system to recover any lost MTA in intermediate rounds. For CIFAR-10, \(K = 5\) with a MAZE query budget of 5M per round. For EMNIST, \(K = 40\) with a MAZE query budget of 4M per round.

Knowledge distillation, particularly in the context of model stealing attacks, is inherently imperfect and cannot replicate the target model exactly, resulting in a minor decrease in MTA. To address this and ensure convergence in the FL setting, the knowledge distillation component is applied every \(K\) rounds instead of every round, allowing the system to recover lost MTA during intermediate rounds. The budget parameter in model stealing attacks and knowledge distillation determines the number of queries used to generate synthetic samples, which are then employed to guide the distillation process. A sufficient query budget ensures the generation of high-quality synthetic data that aligns closely with the target model’s decision boundaries, thereby enhancing the effectiveness of knowledge distillation. For CIFAR-10, we set \(K = 5\) with a query budget of 5M queries. For EMNIST, \(K = 40\) with a query budget of 4M queries. These values strike a balance between computational efficiency and the quality of the distilled global model.


\subsection{EMNIST Grid-Search}

\begin{figure}[ht]
    \includegraphics[width=.98\linewidth]{assets/emnist_undefended_asr.png}
    \caption{Visualizing the impact of the FL setup (particularly the learning-rate, batch-size, and number of epochs used by clients) on main-task accuracy attack success rate when poisoned clients aim to inject a targeted backdoor for the EMNIST dataset.}
    \label{fig:fl_setup_impact_emnist}
\end{figure}


In the same fashion as \cref{sec:fl_setup_matters} for CIFAR-10, we conduct a grid-search analysis over key hyperparameters, varying the client’s learning rate, batch size, and number of epochs on the EMNIST \citep{cohen2017emnistextensionmnisthandwritten} dataset.
Our findings in \cref{fig:fl_setup_impact_emnist} indicate that targeted backdoor attacks are more likely to succeed across a wider range of learning parameter combinations, with numerous setups yielding an ASR greater than 80\%. In \cref{tab:fl_setup_exps_emnist}, we highlight ten specific learning configurations where the attack achieves high ASR while maintaining a high MTA, underscoring the vulnerability of these setups to adversarial manipulation.

\begin{table}[ht]
    \centering
    \begin{tabular}{llcc|cc}
    \toprule
    \textbf{Config} & \textbf{LR} & \textbf{BS} & \textbf{Epochs} & \textbf{MTA (\%)} & \textbf{ASR (\%)} \\
    \midrule
    C1 & 0.1 & 32 & 2 & 89.23 & 99.00 \\
    C2 & 0.1 & 64 & 5 & 88.22 & 99.00 \\
    C3 & 0.05 & 32 & 5 & 88.48 & 98.75 \\
    C4 & 0.1 & 32 & 1 & 89.59 & 98.75 \\
    C5 & 0.1 & 64 & 1 & 89.20 & 98.50 \\
    C6 & 0.1 & 32 & 5 & 88.73 & 98.50 \\
    C7 & 0.1 & 64 & 2 & 88.87 & 98.25 \\
    C8 & 0.05 & 32 & 1 & 89.21 & 98.25 \\ 
    C9 & 0.01 & 32 & 2 & 87.73 & 97.25 \\
    C10 & 0.025 & 128 & 5 & 86.61 & 96.00 \\
    \bottomrule
    \end{tabular}
    \caption{Client FL configurations for successful stealthy attacks on EMNIST \ie cases where the MTA $\geq 80\%$ and ASR $\geq 95\%$.}
    \label{tab:fl_setup_exps_emnist}
\end{table}
\clearpage


\end{document}
