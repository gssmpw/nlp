\section{Bound Analysis on Number of Malicious Clients per Round}
\label{app:bound_analysis}

Let \( N \) denote the total number of clients and \( M \) the number of malicious clients. The ratio of malicious clients is given by the Malicious Client Ratio (MCR), \( \rho = \frac{M}{N} \). In each round of Federated Learning (FL), we sample \( C \) clients randomly from the \( N \) total clients.
%
Let $\mathcal{M}$ be the number of malicious clients (out of $C$) sampled in some round. We are interested in finding a lower bound on the probability that there are more malicious clients than benign ones in a round of FL, \ie \( P\left(\mathcal{M} \geq \frac{C}{2}\right) \).

The selection of malicious clients can be modeled as a Binomial distribution where probability of success is the MCR ($\rho$). Thus,  our desired probability can be written in terms of the CDF $F$ of this distribution:
\begin{align}
    P\left(\mathcal{M} \geq \frac{C}{2}\right) = 1 - F\left(\frac{C}{2}, C, \rho\right).
\end{align}
Using a Chernoff bound \citep{arratia1989tutorial}, we can thus obtain a lower bound on the probability that malicious clients outnumber benign ones in a given round as:
\begin{align}
    P\left(\mathcal{M} \geq \frac{C}{2}\right) &\geq 1 - \exp\left(-C \cdot \left(\frac{1}{2}\left(\ln\frac{1}{2\rho} + \ln\frac{1}{2(1-\rho)}\right)\right)\right) \\
    & = 1 - \exp\left(C \cdot \frac{1}{2} \ln\left(4\rho(1-\rho)\right)\right) \\
    & = 1 - \left(4\rho(1-\rho)\right)^{\frac{C}{2}}.
\end{align}
