\section{Related Work}
\paragraph{Existing Lattice-Based Hardness Assumptions}  
Lattice-based cryptography has gained widespread attention as a post-quantum alternative due to its worst-case hardness guarantees and efficiency in cryptographic applications. The Learning with Errors (LWE) problem~\cite{oregev_jacm09} was first introduced as a computationally hard assumption with worst-case reductions to GapSVP, forming the basis for lattice-based cryptographic schemes. However, LWE suffers from large ciphertext sizes and computational inefficiencies. To address these limitations, Ring-LWE (RLWE)~\cite{LyubashevskyPR13} was introduced, leveraging polynomial rings to improve efficiency while preserving security reductions to worst-case ideal lattice problems. Subsequent extensions such as Module-LWE (MLWE)~\cite{Albrecht15} and NTRU-based cryptosystems~\cite{Hoffstein1998} have further explored structured variants of LWE to balance security and performance. Recent standardization efforts, such as Kyber~\cite{Bos2018kyber} and Dilithium~\cite{Ducas2018dilithium},have demonstrated the practicality of RLWE-based schemes for key exchange and digital signatures. More recent studies~\cite{May2023,Dottling2023,Pouly2024,Jain2024,Albrecht2023,Bai2023,Chen2023,Ducas2023,Eldar2023} have investigated the concrete security of LWE and its variants, refining worst-case reductions and assessing attack resistance under various parameter settings. A recent work by Peikert and Pepin~\cite{peikert2024} introduces a unified framework that systematically connects various algebraically structured LWE variants, including Ring-LWE, Module-LWE, Polynomial-LWE, Order-LWE, and Middle-Product LWE. The proposed Variety-LWE (V-LWE) departs from the above unified paradigm by introducing a security assumption that does not solely depend on lattice structure: we will show that attacks relying on ideal-lattice reductions, such as hybrid Gröbner basis and lattice-reduction methods, do not trivially extend to V-LWE.

\paragraph{Multivariate Polynomial Rings in Cryptography}  
The use of multivariate polynomial rings in cryptography has primarily been explored in the context of multivariate public key cryptosystems (MPKCs)~\cite{Patarin1996}. However, these approaches do not incorporate LWE-style noise distributions, which are crucial for worst-case to average-case reductions in lattice-based cryptography. Module-LWE~\cite{Albrecht15} extends RLWE by introducing module structures, but it remains within the single-variable polynomial framework and does not fully exploit the algebraic richness of multivariate polynomial rings. Newer works~\cite{jdey_acs23} examine alternative structured lattice assumptions, providing additional perspectives on the interplay between multivariate polynomials and lattice-based security. Our work introduces \textit{algebraic varieties} as a component of an LWE-based cryptographic assumption, extending structured lattice problems into a multivariate setting.

\paragraph{Security Reductions and Structured Lattice Problems}  
The security of structured lattice problems relies on worst-case hardness assumptions. The Ideal Shortest Vector Problem (Ideal-SVP)~\cite{Peikert16} has been extensively studied in the context of RLWE and Module-LWE, providing a foundation for their security reductions. Hybrid algebraic-lattice attacks, combining Gröbner basis techniques with lattice reduction algorithms, have been explored in cryptanalysis~\cite{hzhu_iwsec23}, particularly in the context of structured ring-based schemes. While Gröbner basis methods have shown effectiveness in attacking certain algebraic cryptosystems~\cite{Faugere2002}, their direct application to lattice-based cryptosystems remains constrained due to high computational costs~\cite{pravi_tecs24}. Our work explores the impact of hybrid algebraic-lattice attacks on Variety-LWE, demonstrating that existing attack methodologies do not directly compromise its security.

\paragraph{Homomorphic Encryption and Vector Encryption}  
Fully homomorphic encryption (FHE) schemes, such as BFV~\cite{bfv}, BGV~\cite{Brakerski2014}, and CKKS~\cite{ckks}, enable secure computation on encrypted data but primarily operate within single-variable RLWE-based constructions. Batching methods using the Chinese Remainder Theorem (CRT) allow vectorized computations~\cite{GentryHaleviSmart2012,Chillotti2020tfhe}, but they are constrained by modulus structure and require additional encoding layers. Our work introduces a direct multivariate polynomial structure that inherently supports vector encryption, providing an alternative approach to efficiently processing encrypted high-dimensional data. This approach aligns with research in privacy-preserving machine learning~\cite{Dowlin2017cryptonets}, encrypted search~\cite{danb_crypto04}, and secure federated learning~\cite{Phong2017}. 


\paragraph{Attacks at Multivariate Polynomial LWE}
Bootland et al.~\cite{cbootland_ants20} proposed an attack on the multivariate Ring Learning with Errors (m-RLWE) problem in their 2020 paper \cite{cbootland_ants20}. This attack exploits algebraic relationships between the roots of the defining polynomials in the multivariate ring to reduce the m-RLWE problem to multiple lower-dimensional RLWE problems. However, this attack is not applicable to the proposed Variety-LWE due to key distinctions between the two problems.
Firstly, Variety-LWE is defined over a multivariate polynomial ring constructed from an algebraic variety, which possesses a more complex structure than the multivariate rings considered in \cite{cbootland_ants20}. Secondly, Variety-LWE enforces a ``no-mixed-terms'' constraint on the defining polynomials of the algebraic variety, ensuring that each polynomial depends on only one variable. This constraint guarantees coordinate-wise independence in Variety-LWE, preventing attackers from exploiting relationships between variables.