\section{Related works}
%\hl{updated}

The security of machine learning systems has been extensively studied in classical computing, with attacks ranging from side-channel exploits on hardware accelerators to black-box model extraction via adversarial queries \cite{hua2018reverse}\cite{oh2019towards}. Recent work has extended reverse engineering (RE) to quantum circuits, using lookup tables (LUTs) to map transpiled gate sequences to original QML architectures \cite{ghosh2024quantum}. By analyzing rotation gate ordering and entanglement patterns, adversaries can infer circuit parameters, exposing proprietary model designs. Building on this, \cite{upadhyay2024quantumdatabreachreusing} demonstrates how adversaries can extract state preparation circuits and training labels from QML models, directly stealing training data by reverse-engineering the encoding process. This underscores the criticality of encoding schemes as attack surfaces, as they bridge raw data to quantum computation. Quantum homomorphic encryption (QHE) \cite{fisher2014quantum}, while theoretically viable, imposes prohibitive overheads incompatible with near-term devices. Recent quantum-specific defenses address model theft through strategies like distributed execution (QuMoS \cite{wang2023qumos}) and output obfuscation (STIQ \cite{kundu2024stiq}), but these focus on protecting trained parameters rather than preventing encoding detection. Furthermore, while strategies like circuit partitioning \cite{upadhyay2022robust,upadhyay2023trustworthy} aim to distribute trust across providers, they fail to protect QNNs and related IPs because any untrusted provider with access to transpiled circuits can recover encoding schemes. Our work diverges by addressing encoding-specific transpilation artifacts. Unlike \cite{upadhyay2024quantumdatabreachreusing}, which focuses on training data extraction, we demonstrate that adversaries can preemptively identify encoding methods to streamline subsequent attacks.