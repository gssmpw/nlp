\section{Related Work}
\label{sec:related}

%\todo[inline]{@Antoine: Streamline related work. Add references connected to IUIPC, in WWW proceedings, about our population, and other relatable refs.}
% \todo[inline]{Antoine: privacy in location contexts/IoT}
%\todo[inline]{Antoine: double-check consistency and organization of the section please}

Prior research widely 
% \textcolor{red}{widely} 
explored location privacy (%presented in 
Section~\ref{sec:relatedPrivacy}) and users' perceptions of privacy (%presented in 
Section~\ref{sec:relatedPerception}).
% \textcolor{red}{However, little attention has been devoted to self-reported behaviors towards location data sharing. Moreover, to the best of our knowledge, no studies have used a demonstration platform to improve risk understanding in a way that would change users’ perceptions about the balance between the benefits of invasive technologies and potential risks.}
However, little attention has been devoted to self-reported behaviors towards location data sharing. Moreover, to the best of our knowledge, no studies have used a demonstration platform to improve risk understanding in a way that would change users’ perceptions about the balance between the benefits of invasive technologies and potential risks.

\subsection{Privacy and location}
\label{sec:relatedPrivacy}

The privacy issues raised by location data gained a lot of traction in the last decade~\cite{8482357}. 
In particular, user location traces extracted from various data have been shown to be highly unique~\cite{de2013unique,Zang:2011:ALD:2030613.2030630}. %10.1145/3465481.3470474,
This high uniqueness may act as a digital fingerprint and lead to the re-identification of users if their traces are associated with external knowledge.
This uniqueness does not only concern location traces but characterizes all traces generated by a human ~\cite{journals/popets/Kurtz16, eckersley2010unique, journals/popets/Overdorf16}, and can generate a risk of re-identification.
%% This uniqueness does not only concern location traces but characterizes all traces generated by a human ranging from the personalized settings of mobile devices~\cite{journals/popets/Kurtz16} or Web browsers~\cite{eckersley2010unique}, to the logs of in-car sensors~\cite{journals/popets/EnevTKK16}, phone calls~\cite{Zang:2011:ALD:2030613.2030630}, or the writing style of users on the Web~\cite{journals/popets/Overdorf16}.
Several cases of re-identification have been documented, for instance the re-identification of individuals from web search queries~\cite{article2},  %movie ratings~\cite{Narayanan:2008:RDL:1397759.1398064}, 
taxi logs~\cite{zhang2016inferring}, or the notorious case of Governor William Weld using medical information~\cite{article}.

%\subsection{Re-identification}
The uniqueness and risk of re-identification is not the unique threat related to location traces.
Recent works have demonstrated that location is a very rich contextual information and leads to a strong inferential potential in terms of information that can be learned about individuals~\cite{boutet:hal-02421828}. 
For instance, location traces can reveal Points Of Interest (POI) of users such as their home and workplaces~\cite{10.1145/1868470.1868479}, their race and gender~\cite{10.1145/2684822.2685287}, their social network~\cite{10.1145/2665943.2665960}, it can be used to predict their location patterns~\cite{sadilek2012far}, to link accounts of the same user across different datasets~\cite{10.1145/2872427.2883002}, and infer even more sensitive information such as their religion or personality traits\footnote{Note that the processing of such sensitive information is prohibited by Article 9(1) of the General Data Protection Regulation (GDPR) on the processing of special categories of personal data.}.

Due to the large adoption of mobile devices, the location data of users is extensively collected and shared~\cite{almuhimedi2015your,10.1145/2976749.2978313}.
% with and without the consent of individuals~\cite{10.1145/2976749.2978313}.
The uncontrolled usage of this information can have an important impact on users such as unfair price discrimination~\cite{10.1145/2390231.2390245}.
The increasing use of connected devices collecting personal data and the lack of transparency on how the data is actually exploited raise privacy concerns.
Only a handful of tools have been proposed to improve user awareness about the potential risk of revealing their location.
For instance,~\cite{Riederer2016FindYouAP} propose a tool to inspect the potential of location data, while~\cite{boutet:hal-02421828} show the impact of protection mechanisms of the inference capabilities using a demonstration platform.

%\subsection{Perceptions of location privacy and user studies}
%\subsection{Perceptions of privacy and user studies}
\subsection{Perceptions of privacy and privacy controls}
\label{sec:relatedPerception}

Giving users the benefits of location services on their mobile devices while preserving their privacy is an ongoing challenge, as evidenced by mobile OSs iterating on the user interfaces for location notices and control every few years.
For example, iOS allows users to select whether apps get fine- or coarse-grained location data\footnote{https://support.apple.com/guide/iphone/control-the-location-information-you-share-iph3dd5f9be/ios}, and will
present pop-ups when apps continually access location services in the background.
Similarly, Android apps now have to request background location access separately from general location usage\footnote{https://developer.android.com/about/versions/11/privacy/location},
and the OS
%\footnote{https://developer.android.com/about/versions/11/privacy/permissions\#auto-reset} 
 will automatically revoke unused permissions from apps\footnote{https://developer.android.com/about/versions/11/privacy/permissions}. 
Even though mobile operating systems regularly improve user interfaces, the opaque privacy controls of location services still face criticism\footnote{https://www.attorneygeneral.gov/taking-action/attorney-general-josh-shapiro-announces-391-million-settlement-with-google-over-location-tracking-practices/}.
% Balash et al.~
\cite{277130} explore the users' perceptions regarding access to Google accounts by mobile applications and formulate design recommendations to improve the current third-party management tools offered by Google, such as tracking recent access, automatically revoking access due to app disuse, and providing permission controls.
% Wijesekera et al.~
\cite{190982} analyze authorization preferences in different usage contexts and suggest determining the situations in which users would like to be confronted with security decisions.


%TO ADD:
%\cite{EuroUSEC16, gamarrapercepcion} explores user perceptions of smartphone location privacy via qualitative methods, identifying user motivation and awareness of location-capable apps. Our work is a strong quantitative contribution that builds upon that very closely-aligned prior work. 


%\cite{gamarrapercepcion} measures the perception of location privacy of users with mobile devices and show that users do not have a real concern regarding the privacy of their geolocation data.
%These questions were designed to know the users’ perceptions of privacy concerns in LBS and any actions they take to preserve it. Results: The results show that, in general, the participants do not have a real concern regarding the privacy of their geolocation data, and the majority is not willing to pay to protect their privacy. Conclusions: This type of surveys can generate awareness among participants about the use of their private information. The results expose in this paper can be used to create government policies and 




%Regarding awareness of data collection,
%privacy controls, and transparency-enhancing technologies, several prior works can be cited: ~\cite{274582} about Google location and web history dashboards, 

%~\cite{277130} about forgotten permissions granted for private Google data, 
%~\cite{10.1145/3491102.3502136} about the effectiveness of location privacy zones, 
%cite{190982} about permission preferences under different usage contexts, 
%and \cite{274435} about the possibility of "data export" dumps as TETs (e.g., Google Takeout).
%regulations by technology companies about the privacy management.

Users' perceptions of the risks and benefits of technologies can determine their willingness to adopt them~\cite{EuroUSEC16}. 
More specifically, people are more likely to accept potentially invasive technology if they think its benefits will outweigh its potential risks~\cite{1203752}.
Due to the massive adoption of location-enabled mobile applications, this fact suggests that users' perception of this trade-off is more in favor of the benefits than the potential risks.
This attitude however depends on the perception of the said privacy risks.
Studies have shown that users of mobile phones are often unaware of the data collected by apps running on their devices, and that a majority of users restrict some of their permissions~\cite{almuhimedi2015your} following a better awareness of data collection.
Other studies explored the privacy paradox~\cite{BARTH201955, kang2021smart} where self-reported concerns about privacy appear to be in contradiction with often careless online behaviors.
However, another study focuses on university community~\cite{gamarrapercepcion} and show that this population of users does not have a genuine concern regarding the privacy of their geolocation data.
Note that this paper did not study location data in mobility contexts associated with smartphones as we do.

%measures the perception of location privacy of users with mobile devices and

%\todo[inline]{Next paragraph is out of place}
%The current paper studies the actual users' behavior about data-sharing of location traces.
%While our results comfort the privacy paradox widely observed~\cite{BARTH201955, kang2021smart} -- where self-reported concerns about privacy appear to be in contradiction with often careless online behaviors --, we also show that users tend to underestimate the risks, and with a better understanding about them, they want to reduce the data-sharing.

%\todo[inline]{Next paragraph appears disconnected from the rest}
% \todo[inline]{Victor: user studies about privacy perceptions}
%Users studies about privacy perceptions recently gained a lot of traction in the literature.

Analyzing behavior and understanding users' perceptions are also important notions to grasp in order to further design Privacy and Transparency Enhancing Technologies (PETs and TETs).
For instance, 
% Kaushik et al.~
\cite{274429} conducted an online survey to understand people’s perspectives on solely automated decision-making. They then formulate recommendations on how to design such systems.
In a similar line of work, 
% Islami et al.~
\cite{islami_capturing_2022} conducted in-depth semi-structured interviews with 17 Swedish drivers to analyse their privacy perceptions and preferences for intelligent transportation systems, then to provide recommendations for suitable predefined privacy options.
%For instance, \cite{10.1145/3366423.3380273} compared intention and perception in online discussion and showed that reducing misperception is an important factor to promote healthler conversations.
Several other works~\cite{DBLP:journals/popets/ZuffereyNHH23,DBLP:journals/imwut/VelykoivanenkoN21, 10.1145/3491102.3502136} studied the perceptions of privacy and utility of users related to fitness-trackers and demonstrated a high potential for data minimization (i.e., reducing the volume of data sent to service provider).
% Debatin et al.~
\cite{10.1111/j.1083-6101.2009.01494.x}, in turn, investigated Facebook users' awareness of privacy issues and perceived benefits and risks of utilizing Facebook and recommended better privacy protection, higher transparency and more education about the risks of posting personal information to reduce risky behavior. 
% Bielova et al.~
\cite{bielova:hal-04235032} analysed the behaviors of websites' visitors through a study of the impact of dark patterns on consent decisions. 
% and provided recommendations.
% Veys et al.~
\cite{274435} explored whether current data downloads (such as Google Takeout) actually achieve the transparency goals embodied by the right of access. Most participants indicated that current offerings need improvement to be useful, emphasizing the need for better filtration, visualization, and summarizing to help them hone in on key information.
However, none of these related work specifically targets users' behavior and perception about data-sharing of location data.
Although 
% Martin and Nissenbaum~
\cite{martin_what_2019} clearly addressed users' perceptions of location data, it is to be noted that it does so from an information science standpoint, %-- that is, 
closer to sociology and not from a UX/usability one.
It nonetheless offers a relevant account for the interested reader.


Related to location data, 
% Farke et al.~
\cite{274582} specifically analysed the user perceptions and reactions to Google's My Activity and how this web history dashboard increases or decreases end-users' concerns and benefits regarding data collection. Their results show that participants were surprised by the volume and detail of the collected data, but most of them were significantly more likely to be both 1) less concerned about data collection and 2) to view data collection more beneficially. However, this dashboard does not present any risks such as possible sensitive inferences associated with location traces or places visited. By also presenting the risks, our study shows that users are more concerned about privacy issues.


%Data privacy regulations like GDPR and CCPA define a right of access empowering consumers to view the data companies store about them. Companies satisfy these requirements in part via data downloads, or downloadable archives containing this information. Data downloads vary in format, organization, comprehensiveness, and content. It is unknown, however, whether current data downloads actually achieve the transparency goals embodied by the right of access. In this paper, we report on the first exploration of the design of data downloads. Through 12 focus groups involving 42 participants, we gathered reactions to six companies' data downloads. Using co-design techniques, we solicited ideas for future data download designs, formats, and tools. Most participants indicated that current offerings need improvement to be useful, emphasizing the need for better filtration, visualization, and summarization to help them hone in on key information.

%Privacy dashboards and transparency tools help users review and manage the data collected about them online. Since 2016, Google has offered such a tool, My Activity, which allows users to review and delete their activity data from Google services. We conducted an online survey with n = 153 participants to understand if Google's My Activity, as an example of a privacy transparency tool, increases or decreases end-users' concerns and benefits regarding data collection. While most participants were aware of Google's data collection, the volume and detail was surprising, but after exposure to My Activity, participants were significantly more likely to be both less concerned about data collection and to view data collection more beneficially. Only 25% indicated that they would change any settings in the My Activity service or change any behaviors. This suggests that privacy transparency tools are quite beneficial for online services as they garner trust with their users and improve their perceptions without necessarily changing users' behaviors. At the same time, though, it remains unclear if such transparency tools actually improve end user privacy by sufficiently assisting or motivating users to change or review data collection settings.


%However, none of the related work specifically targets users' behavior and perception about data-sharing of location data. Although~\cite{martin_what_2019} clearly addressed users' perceptions of location data, it is to be noted that it does so from an information science standpoint -- that is, closer to sociology -- and not from a UX/usability one. It nonetheless offers a relevant account for the interested reader.

%Another body of work provides insights into end-users' perceptions of technical systems, and often discusses design recommendations to address concerns raised by participants.
%For instance, \cite{kaushik_how_2021} conducted an online survey to understand people’s perspectives on solely automated decision-making.
%They then formulate recommendations on how to design such systems.
%In a similar line of work, \cite{islami_capturing_2022} conducted in-depth semi-structured interviews with 17 Swedish drivers to analyse their privacy perceptions and preferences for intelligent transportation systems, then to provide recommendations for suitable predefined privacy options.
%\cite{romare2023tapping} recently examined privacy concerns and preferences in IoT Trigger-Action Platforms using focus groups, in which they notably called for more usable privacy settings.
%\cite{bombik_multi-region_2022} compared perceptions and practices of people in three geographic regions regarding privacy and security matters related to Smart Home Devices.
%Also related to IoT devices, \cite{marky_all_2020} investigated through a pre-study of 15 participants the usability of centralized privacy settings.
%The most comprehensive -- and therefore general -- study conducted on privacy expectations in the IoT is~\cite{naeini_privacy_2017}.
%The authors analysed the expectations and preferences of over 1000 participants in 380 IoT vignette scenarios.
%While this work provides a necessary insight into IoT users' privacy preferences, it does not account for location contexts nor for the impact of TETs.
%With regard to age differences in privacy attitudes, \cite{kezer_age_2016} informed us that, when it comes to Social Networks (i.e., Facebook), young adults (18-45, the typical age range for our study) tend to be less concerned about their own privacy and about others' privacy than other age ranges.

%However, none of the related work specifically targets users' behavior and perception about data-sharing of location data.
%Although~\cite{martin_what_2019} clearly addressed users' perceptions of location data, it is to be noted that it does so from an information science standpoint -- that is, sociology -- and not from a UX/usability one.
%It nonetheless offers a relevant account for the interested reader.

%\todo[inline]{Next paragraph is out of place}
%The current paper studies the actual users' behavior about data-sharing of location traces.
%While our results comfort the privacy paradox widely observed~\cite{BARTH201955, kang2021smart} -- where self-reported concerns about privacy appear to be in contradiction with often careless online behaviors --, we also show that users tend to underestimate the risks, and with a better understanding about them, they want to reduce the data-sharing.



%Analyzing behavior and understanding users' perceptions are important notions to grasp in order to further design platforms and PETs.
%For instance, \cite{10.1145/3366423.3380273} compared intention and perception in online discussion and showed that reducing misperception is an important factor to promote healthler conversations.
%\cite{DBLP:journals/popets/ZuffereyNHH23,DBLP:journals/imwut/VelykoivanenkoN21}, studied the perceptions of privacy and utility of users related to fitness-trackers and demonstrated a high potential for data minimization (i.e., reducing the volume of data sent to service provider). \cite{10.1111/j.1083-6101.2009.01494.x}, in turn, investigated Facebook users' awareness of privacy issues and perceived benefits and risks of utilizing Facebook and recommended better privacy protection, higher transparency and more education about the risks of posting personal information to reduce risky behavior. 
%\cite{bielova:hal-04235032} analysed the impact of dark patterns on consent decisions and provided recommendations.