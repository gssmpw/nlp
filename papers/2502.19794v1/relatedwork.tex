\section{Related Work}
\label{sec_related_work}
\subsection{Molecule Generation}
One important task in molecular design is goal-oriented \emph{de novo} molecule generation~\cite{zunger2018inverse}, which aims to design novel compounds with specific biochemical properties. 
Our approach falls within this category, focusing on the generation of molecules under predefined constraints through a non-iterative process. 
In contrast, methods like LIMO~\cite{eckmann2022limo} adopt an iterative approach to achieve similar goals.
Another key task is goal-oriented molecule optimization, exemplified by methods such as MolMIM~\cite{reidenbach2022improving}, which iteratively refines molecules starting from a reference compounds. MOLGEN~\cite{fang2023domain}, on the other hand, is a non-iterative approach that also relies on a reference molecule as the starting point for optimization.
Finally, unconditional de novo molecule generation~\cite{flam2022language,jin2018junction,liu2019hyperbolic} involves the generation of novel molecules without specific property constraints, where the model generates molecules that conform to the statistical distribution of the training data.
A detailed comparison of these generative tasks is presented in \cref{tab_generative_tasks}.

\begin{table*}[ht] 
\centering 
\normalsize
\begin{tabular}{lccc}
\toprule 
Generative Task & Reference Molecule & Model & Optimization Method \\
\midrule 
\multirow{2}{*}{Goal-oriented \emph{de novo} generation} & w/o & LIMO & Iterative \\
& w/o & \textbf{Ours} & Non-Iterative \\
\midrule
\multirow{2}{*}{Goal-oriented optimization} & w/ & MolMIM & Iterative \\
& w/ & MOLGEN & Non-Iterative \\
\midrule
Goal-free \emph{de novo} generation & w/o & SF-RNN & - \\
\bottomrule 
\end{tabular} 
\caption{Comparison of different generative tasks in molecule design. ``w/'' indicates that the input includes a reference molecule as the starting point, while ``w/o'' means that no reference molecule is provided in the input. Goal-free generation involves no constraints on molecule properties or substructures, whereas goal-oriented generation incorporates specific constraints.}
\label{tab_generative_tasks} 
\end{table*}

\begin{figure*}[ht]
\centering
\includegraphics[width=0.85\textwidth]{figures/pic13.png}
\caption{Illustration of numerical enhancement in the training process. Property values are transformed through $\mathcal{N}(\cdot)$ to obtain a holistic numerical encoding, which is then added to each numerical token's word embedding to produce the final encoding of the constraint conditions. $\mathcal{E}(\cdot)$ represents the embedding layer.}
\label{fig_dataset_preprocess}
\end{figure*}

\subsection{Molecule Representations}
From the perspective of generation objectives, inverse molecule design includes 3D molecule generation~\cite{hoogeboom2022equivariant,bao2022equivariant}, 2D graph-based molecule generation~\cite{mahmood2021masked,seff2019discrete}, and 1D sequence-based molecule generation~\cite{flam2022language,segler2018generating}.
There are several inverse molecule design methods focus on 3D molecule generation, including conditional autoregressive model~\cite{gebauer2022inverse}, diffusion models~\cite{bao2022equivariant,hoogeboom2022equivariant}. 
However, we focus on 1D string-based representation that uses simplified text encodings to describe the structure of a chemical species, which is highly interpretable compared to 3D coordinates. 
Sequence representation also offers computational efficiency and facilitates the exploration of longer and more diverse molecules~\cite{fang2023domain}.