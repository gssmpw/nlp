%% The Appendices part is started with the command \appendix;
%% appendix sections are then done as normal sections

\setcounter{figure}{0}
\appendix

\section{Layout detection evaluation}
\label{appendix:layout}
An example of the layout detection tool \textit{surya} from the \textit{marker} OCR model is depicted in Fig.~\ref{fig:PDFComparison}.
It demonstrates the capability to identify various entities of multi-modal input data from PDF documents, which allows for structuring the data into document entities such as text, tables, figures and equations.
%
\begin{figure*}[h]
    \centering
    \includegraphics[width=0.8\linewidth]{fig_tmp/pdf_page_merge.png}
    \caption{
    An example of the layout detection tool \textit{surya} from the \textit{marker} OCR model structuring the data into text, sections, tables, figures and equations (here, \cite{Xu_2016} reproduced with permission from IOP Publishing under a Creative Commons License).
    }
    \label{fig:PDFComparison}
\end{figure*}

\section{Role and prompt of vision transformer model}
\label{appendix:VT}
The following role and prompt are applied to the vision transformer model to transform the visual input data into an accurate textual description:\\
\textit{\small
You are a materials science expert with expertise in interpreting scientific data, graphs, and visualizations, and your task is to provide precise and detailed descriptions of the image content, including any graphs, charts, diagrams, or illustrations.
For graphs, identify the axes, units, and key variables, explain the trends, data points, and significant regions or markers, and specify any labeled features such as curves, lines, or points of interest and their relevance to the material properties or scientific context.
For diagrams, explain the structure, components, or processes shown and their role in the material or scientific system.
For scientific context, relate the visual information to the material science concepts it represents, such as stress-strain behavior, plastic deformation mechanisms, or crystal structure, and ensure that the explanation is precise, avoiding vague language, while making it accessible to someone familiar with material science terminology.
Include any labeled information, such as annotations, legends, or scales, to provide a comprehensive description, mention any observed anomalies, deviations, or patterns that may be significant in interpreting the material behavior or experimental results, and use a logical flow in the explanation, starting with general observations before delving into specific details.
}


\section{Role and prompt of large language model}
\label{appendix:LLM}
The following role and prompt are applied to the large language model to analyze and process textual input data from the considered materials science database:\\
\textit{\small
You are a highly specialized assistant with expertise in material science natural language processing, specifically designed to assist users with their queries.
Your core principles are precision, honesty, and adherence to the provided context.
You do not speculate or mention your role.
Greet users with: "Welcome to the MINDQUEST chatbot! I’m here to assist you with your material science-related questions. Upload your documents, and I’ll do my best to provide precise and helpful insights based on their content. Let’s get started!"
When responding to prompts, use only the provided information and conversation history.
If no data is provided, give the most appropriate answer based on your knowledge but explicitly state that this information is not retrieved from the provided database or uploaded documents.
If the data is incomplete or insufficient, recommend uploading relevant Markdown files for additional context.
Always stay concise, focused, and factual, ensuring your responses are strictly within the scope of the available data or clearly distinguish when external knowledge is applied.
}

\section{Example of \textit{MINDQUEST} chat bot}
\label{appendix:MINDQUEST}
\noindent
\textbf{Question 1:}
What is the difference between molecular dynamics (MD) and discrete dislocation dynamics (DDD) simulations?\\
\textbf{Answer 1:}
\textit{\small Molecular Dynamics (MD) and Discrete Dislocation Dynamics (DDD) are two distinct computational methods used to simulate the behavior of materials at different scales.
MD simulations are a type of molecular mechanics simulation that models the motion of atoms or molecules in a system. It is based on Newton's laws of motion and uses classical mechanics to describe the interactions between particles. MD simulations are typically used to study the dynamics of small systems, such as molecules, nanoparticles, or bulk materials at the atomic scale.  In an MD simulation, the positions and velocities of individual atoms are updated at each time step using numerical integration methods, such as Verlet integration or leapfrog algorithms. The interactions between atoms are described by potential energy functions, such as Lennard-Jones or Morse potentials, which capture the chemical bonding and interactions between atoms. 
DDD simulations, on the other hand, are a type of computational model that describes the behavior of dislocations in materials. Dislocations are line defects in the crystal structure of a material that can move under stress, leading to plastic deformation or strain hardening. DDD simulations aim to capture the dynamics of dislocation motion and interactions with the surrounding material. In a DDD simulation, the material is discretized into a lattice of nodes, each representing a point in space. Dislocations are modeled as line defects that can move through the lattice, interacting with other dislocations and the surrounding material. The motion of dislocations is described using a set of equations that balance the forces acting on them, such as stress, friction, and elastic interactions. \\
**Key differences**:\\
1. **Scale**: MD simulations are typically used to study small systems at the atomic scale, while DDD simulations focus on larger systems, such as bulk materials or engineering structures.\\
2. **Time scale**: MD simulations can capture short-time dynamics (e.g., picosecond timescales), while DDD simulations often require longer simulation times (e.g., milliseconds or seconds) to capture the long-term behavior of dislocations.\\
3. **Mechanisms**: MD simulations rely on classical mechanics and potential energy functions, whereas DDD simulations use a set of equations that balance forces acting on dislocations, including stress, friction, and elastic interactions.\\
4. **Complexity**: DDD simulations are generally more complex than MD simulations due to the need to model the interactions between dislocations and the surrounding material.  In summary, while both MD and DDD simulations aim to understand the behavior of materials at different scales, they employ distinct approaches and mathematical frameworks to capture the underlying physics.}

\noindent\\
\textbf{Question 2:}
What is the key concept of dislocation densities?\\
\textbf{Answer 2:}
\textit{\small The key concepts are:\\
1. **Dislocation Density:**  The number of dislocations per unit volume of material.\\
2. **Geometrically Necessary Dislocation (GND) Density:** The minimum dislocation density required to span the two endpoints of a dislocation line segment.\\
3. **Statistically Stored Dislocation (SSD) Density:** A measure of the dislocation density that is independent of the deformation state, but is randomly introduced and distributed in the material.\\
4. **Total Dislocation Density:** The sum of GND and SSD densities.\\
The key points are: \\
1. Dislocation density can be affected by various factors, including grain size, strain rate, and temperature.\\
2. Geometrically necessary dislocation density increases with increasing strain and decreasing grain size.\\
3. Statistically stored dislocation density is independent of the deformation state, but is randomly introduced and distributed in the material.\\
4. The total dislocation density is a sum of GND and SSD densities. 
}