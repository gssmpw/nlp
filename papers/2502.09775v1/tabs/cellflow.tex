\begin{figure*}[!tb]
    \centering
    \includegraphics[width=\linewidth]{imgs/cellflow_3.png}
    \vspace{-2em}
    \caption{
\textbf{\emph{CellFlow} algorithm.}
\textit{(a) Training.} The neural network \(v_\theta\) learns a velocity field by fitting trajectories between control cell images (\(x_0 \sim p_0\)) and perturbed cell images (\(x_1 \sim p_1\)). At each training step, intermediate states \(x_t\) are sampled along the linear interpolation between \(x_0\) and \(x_1\), with \(t \sim U[0, 1]\). The network minimizes the loss \(L\), which measures the difference between the predicted velocity \(v_\theta(x_t, t, c)\) and the true velocity \((x_1 - x_0)\).
\textit{(b) Inference.} The trained velocity field \(v_\theta\) guides the transformation of a control cell state \(x_0\) into a perturbed cell state \(x_1\). This is achieved by solving an ordinary differential equation iteratively, using numerical integration steps over time \(t\) (e.g., \(t = 0.0, 0.1, 0.2, \ldots, 0.9, 1.0\)). Each step updates the cell state using the learned velocity field.
    }
    \vspace{-1em}
    \label{fig:cellflow}
\end{figure*}