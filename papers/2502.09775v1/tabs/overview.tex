\begin{figure*}[!tb]
    \centering
    \includegraphics[width=\linewidth]{imgs/overview_7.png}
    \vspace{-2em}
    \caption{
\textbf{Overview of \emph{CellFlow}.}
\textit{(a) Objective.} \emph{CellFlow} aims to predict changes in cell morphology induced by chemical or gene perturbations \textit{in silico}. In this example, the perturbation effect reduces the nuclear size.
\textit{(b) Data.} The dataset includes images from high-content screening experiments, where chemical or genetic perturbations are applied to target wells, alongside control wells without perturbations. 
Control wells provide prior information to contrast with target images, enabling the identification of true perturbation effects (e.g., reduced nucleus size) while calibrating non-perturbation artifacts such as batch effects—systematic biases unrelated to the perturbation (e.g., variations in color intensity).
\textit{(c) Problem formulation.} We formulate the task as a distribution-to-distribution problem (many-to-many mapping), where the source distribution consists of control images, and the target distribution contains perturbed images within the same batch.
\textit{(d) Flow matching.} \emph{CellFlow} employs flow matching, a state-of-the-art generative approach for distribution-to-distribution problems. It learns a neural network to approximate a velocity field, continuously transforming the source distribution into the target by solving an ordinary differential equation (ODE).
\textit{(e) Results.} \emph{CellFlow} significantly outperforms baselines in image generation quality, achieving lower Fréchet Inception Distance (FID) and higher classification accuracy for mode-of-action (MoA) predictions.
    }
    \vspace{-1em}
    \label{fig:overview}
\end{figure*}
