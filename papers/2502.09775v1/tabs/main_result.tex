\begin{table*}[!tb]
    \small
    \centering
    \rowcolors{2}{white}{light-light-gray}
    \setlength\tabcolsep{6pt}
    \renewcommand{\arraystretch}{1.1}
    
    \textbf{(a) Main Results} \\
    \begin{tabular}{lcccccccccccc}
    \toprule
    & \multicolumn{4}{c}{BBBC021 (Chemical)} & \multicolumn{4}{c}{RxRx1 (Genetic)} & \multicolumn{4}{c}{JUMP (Combined)} \\
    Method & FID$_o$ & FID$_c$ & KID$_o$ & KID$_c$ & FID$_o$ & FID$_c$ & KID$_o$ & KID$_c$ & FID$_o$ & FID$_c$ & KID$_o$ & KID$_c$ \\
    \midrule
    PhenDiff (MICCAI'24) & 49.5 & 109.2 & 3.10 & 3.18 & 65.9 & 174.4 & 5.19 & 5.29 & 49.3 & 127.3 & 5.09 & 5.17\\
    IMPA (Nature Comm'25) & 33.7 & 76.5 & 2.60 & 2.70 & 41.6 & 164.8 & 2.91 & 2.94 & 14.6 & 99.9 & 1.08 & 1.06\\
    CellFlow & \textbf{18.7} & \textbf{56.8} & \textbf{1.62} & \textbf{1.59} & \textbf{33.0} & \textbf{163.5} & \textbf{2.38} & \textbf{2.40} & \textbf{9.0} & \textbf{84.4} & \textbf{0.63} & \textbf{0.65}\\
    \bottomrule
    \end{tabular}
    \vspace{0.3em}

    \setlength\tabcolsep{4pt}
    \renewcommand{\arraystretch}{1.1}
    
    \textbf{(b) Per Perturbation Results} \\
    \begin{tabular}{lccccccccc}
    \toprule
    & \multicolumn{6}{c}{Chemical Perturbations} & \multicolumn{3}{c}{Genetic Perturbations} \\
    Method & Alsterpaullone & AZ138 & Bryostatin & Colchicine & Mitomycin C & PP-2 & ACSS1 & CRISP3 & RASD1 \\
    \midrule    
    PhenDiff (MICCAI'24) & 106.6 & 120.0 & 106.9 & 111.2 & 110.0 & 121.7 & 157.5 & 144.6 & 180.4\\
    IMPA (Nature Comm'25) & 69.6 & 59.9 & 104.3 & 84.4 & 57.0 & 77.3 & 152.6 & 142.7 & 147.1\\
    CellFlow & \textbf{41.6} & \textbf{44.4} & \textbf{47.0} & \textbf{72.3} & \textbf{42.3} & \textbf{64.3} & \textbf{140.9} & \textbf{125.1} & \textbf{140.1}\\
    \bottomrule
    \end{tabular}
    \vspace{-1em}
    
    \caption{\textbf{Evaluation of \emph{CellFlow}.} 
    \textit{(a) Main results.} \emph{CellFlow} outperforms GAN- and diffusion-based baselines, achieving state-of-the-art performance in cellular morphology prediction across three chemical, genetic, and combined perturbations datasets. Metrics measure the distance between generated and ground-truth samples, with lower values indicating better performance. FID$_o$ (overall FID) evaluates all images, while FID$_c$ (conditional FID) averages results per perturbation $c$. KID values are scaled by 100 for visualization.
    \textit{(b) Per perturbation results.} For six representative chemical perturbations and three genetic perturbations, \emph{CellFlow} generates significantly more accurate images that better capture the perturbation effects than other methods, as measured by the FID score.
    }
    \label{tab:results}
    \vspace{-1em}
\end{table*}