\documentclass{article}

\usepackage{microtype}
\usepackage{graphicx}
\usepackage{subfigure}
\usepackage{booktabs}
\usepackage{hyperref}
\usepackage[accepted]{icml2025}
\usepackage{multirow}
\usepackage{amsmath}
\usepackage{amssymb}
\usepackage{mathtools}
\usepackage{amsthm}
\usepackage[capitalize,noabbrev]{cleveref}
\usepackage[textsize=tiny]{todonotes}
\usepackage{colortbl}
\definecolor{light-light-gray}{gray}{0.92} 
\usepackage{listings}
\usepackage{tcolorbox}
\usepackage{enumitem}

\theoremstyle{plain}
\newtheorem{theorem}{Theorem}[section]
\newtheorem{lemma}[theorem]{Lemma}
\newtheorem{prop}{Proposition}
\newcommand{\indep}{\perp\!\!\!\!\perp} 
\newcommand{\notindep}{\not\!\perp\!\!\!\perp } 

\icmltitlerunning{\emph{CellFlow}: Simulating Cellular Morphology Changes via Flow Matching}

\begin{document}

\twocolumn[
\icmltitle{\emph{CellFlow}: Simulating Cellular Morphology Changes via Flow Matching}
\icmlsetsymbol{equal}{*}
\begin{icmlauthorlist}
\icmlauthor{Yuhui Zhang}{a,equal}
\icmlauthor{Yuchang Su}{b,equal}
\icmlauthor{Chenyu Wang}{c}
\icmlauthor{Tianhong Li}{c}
\icmlauthor{Zoe Wefers}{a}
\icmlauthor{Jeffrey Nirschl}{a}
\icmlauthor{James Burgess}{a}
\icmlauthor{Daisy Ding}{a}
\icmlauthor{Alejandro Lozano}{a}
\icmlauthor{Emma Lundberg}{a}
\icmlauthor{Serena Yeung-Levy}{a}
\end{icmlauthorlist}
\icmlaffiliation{a}{Stanford University}
\icmlaffiliation{b}{Tsinghua University}
\icmlaffiliation{c}{MIT}
\icmlcorrespondingauthor{Yuhui Zhang}{yuhuiz@stanford.edu}
\icmlcorrespondingauthor{Serena Yeung-Levy}{syyeung@stanford.edu}
\icmlkeywords{Machine Learning, ICML}
\vskip 0.3in
]
\printAffiliationsAndNotice{\icmlEqualContribution}

\begin{abstract}

In this work, we tackle the challenge of disambiguating queries in retrieval-augmented generation (RAG) to diverse yet answerable interpretations.
State-of-the-arts follow a Diversify-then-Verify (DtV) pipeline, where diverse interpretations are generated by an LLM,
later used as search queries to retrieve supporting passages.
Such a process
may introduce noise in either interpretations or retrieval,
particularly in enterprise settings, where LLMs---trained on static data---may struggle with domain-specific disambiguations.
Thus, a post-hoc verification phase is introduced to prune noises.
Our distinction is \textbf{to unify diversification with verification} by incorporating feedback from retriever and generator early on.
This joint approach improves both efficiency and robustness by reducing reliance on multiple retrieval and inference steps, which are susceptible to cascading errors.
We validate the efficiency and effectiveness of our method, \ourslong (\ours), on the widely adopted ASQA benchmark
to achieve diverse yet verifiable interpretations.
Empirical results show that \ours improves grounding-aware $\textrm{F}_1$ score by an average of 23\% over the strongest baseline across different backbone LLMs.
\end{abstract}

\section{Introduction}
\label{sec:intro}
% Image editing methods in diffusion models depend on user-defined control directions - users can unlock their creativity using these methods by specifying the desired manipulation through prompts~\cite{gandikota2023concept}, reference images~\cite{ruiz2022dreambooth, kumari2022customdiffusion, gal2022image, chen2024trainingfreeregionalpromptingdiffusion}, or attribute vectors~\cite{parmar2023zero,hertz2022prompt}. In this work, we ask a fundamentally different question: \emph{Can we automatically discover the underlying visual structure of a concept within diffusion model's knowledge?} %Rather than requiring user-specified controls, we aim to decompose the model's internal knowledge into meaningful directions.

% This question touches on a fundamental limitation in how we interact with diffusion models. Current control methods ~\cite{zhang2023addingconditionalcontroltexttoimage, gandikota2023concept, ye2023ipadaptertextcompatibleimage,ye2023ipadaptertextcompatibleimage, hertz2024stylealignedimagegeneration, li2023photomaker, shi2024instantbooth, chen2024trainingfreeregionalpromptingdiffusion} require users to specify their desired manipulations in advance, limiting interactive creativity. This contrasts with natural human artistic workflows, where creators dynamically explore creative ideas while jointly refining them toward meaningful artistic outcomes~\cite{hoffmann2016modeling}. This synergy between specification and exploration is not new to generative models. Early GAN architectures naturally developed disentangled latent spaces that enabled continuous\cite{harkonen2020ganspace,radford2015unsupervised, wu2021stylespace, shen2020interfacegan}, compositional control over generated images. Users could explore these spaces to discover interesting variations that would be difficult to describe in words~\cite{wu2021stylespace}, then combine them to achieve their creative goals~\cite{grabe2022towards}. 


% While diffusion models have largely superseded GANs in conditional image synthesis~\cite{dhariwal2021diffusion},  their underlying structure remains less understood. Diffusion models achieve remarkable diversity through high-dimensional latents, unlike GANs' compact latent spaces.  With a single prompt, diffusion models can generate radically different variations through different random initializations of input noise. We ask - Is it possible to discover interpretable structure within this vast space of variations?

Text-to-image diffusion models are capable of generating remarkable visual variations from a single prompt through different random initializations. However, this vast creative potential remains largely opaque to users---while we can generate diverse images, we lack understanding of the underlying structure of these variations. This presents a fundamental challenge: how can we discover and expose the latent visual capabilities encoded within these models?

\let\thefootnote\relax \footnote{$^{*}$Correspondence to \texttt{gandikota.ro@northeastern.edu}}

The challenge touches on a key limitation in how we interact with diffusion models today. Current control methods require users to explicitly specify their desired edits in advance through prompts~\cite{gandikota2023concept}, reference images~\cite{zhang2023addingconditionalcontroltexttoimage, chen2024trainingfreeregionalpromptingdiffusion, ruiz2022dreambooth,kumari2022customdiffusion, Ryu_lora, hu2021lora}, or attribute vectors~\cite{ye2023ipadaptertextcompatibleimage, hertz2024stylealignedimagegeneration, li2023photomaker, shi2024instantbooth,parmar2023zero,hertz2022prompt}. That contrasts sharply with natural human creative workflows, where artists dynamically explore creative ideas and jointly refine them toward meaningful artistic outcomes~\cite{hoffmann2016modeling}. The need for pre-specified controls creates a barrier between users and the full creative potential of these models.

Interestingly, earlier generative models like GANs~\cite{gans,karras2019style,brock2018large} naturally developed more interpretable internal structures. Their compact latent spaces often exhibited emergent disentanglement~\cite{harkonen2020ganspace,radford2015unsupervised, wu2021stylespace, shen2020interfacegan}, enabling continuous and compositional control over generated images. Users could explore these spaces to discover interesting variations that would be difficult to describe in words~\cite{wu2021stylespace}, then combine them to achieve their creative goals~\cite{grabe2022towards}.

Diffusion models have largely superseded GANs in conditional image synthesis~\cite{dhariwal2021diffusion}, achieving greater diversity through much higher-dimensional latents. And yet an understanding of the underlying structure of these larger latent spaces has remained elusive. In this work, we ask a fundamental question: \emph{Can we automatically discover the visual structure within a diffusion model's knowledge of a concept?} Rather than requiring user-specified controls, we aim to decompose the model's internal representations into expressive directions that users can explore and combine.

To address these needs, we present \textbf{SliderSpace}, a framework that brings systematic explorability to diffusion models. Given just a text prompt, SliderSpace discovers a canonical set of meaningful, diverse, and controllable directions within the model's knowledge of that concept. Each direction is implemented as a low-rank adapter~\cite{hu2021lora} that can be scaled and composed with others, allowing users to explore and smoothly combine different aspects of variation, as shown in Figure~\ref{fig:intro}.

We ground SliderSpace discovery in three key requirements for meaningful decomposition of a diffusion model's visual manifold: 
\begin{enumerate}
    \item \textbf{Unsupervised Discovery:} The decomposition process should emerge from the intrinsic structure of the model's learned representation, rather than being guided by predefined attributes. This ensures we capture the true topology of the model's knowledge space rather than projecting our assumptions onto it.
    
    \item \textbf{Semantic Orthogonality:} Each discovered control must represent a distinct semantic direction. This is enforced in a semantic feature space, like CLIP, where every slider has an orthogonal effect in embeddings. This prevents discovering multiple controls that create similar semantic effects, making the system more efficient and easier.
    
    \item \textbf{Distribution Consistency:} Directions must induce consistent transformations across both random seeds and prompt variations. 
\end{enumerate}

These requirements naturally lead to our proposed framework, which we formalize in Section~\ref{sec:method}. As we show in our experiments, SliderSpace is architecture-agnostic, working with both conventional U-Net based models like Stable Diffusion~\cite{rombach2022high, rombach2022sd20, podell2023sdxl, turbo, dmd} and recent transformer-based architectures like Flux~\cite{flux}.

We demonstrate the expressiveness of SliderSpace through three applications: First, we show how SliderSpace can decompose high-level concepts into diverse and expressive components, revealing the natural axes of variation in the model's understanding. Second, we explore artistic style variation, where SliderSpace discovers directions that match or exceed the diversity of manually curated artist lists while being judged more useful by human evaluators. Finally, we show how SliderSpace can help reverse the mode collapse commonly observed in distilled diffusion models, restoring diversity while maintaining generation speed.

Beyond providing practical creative control, SliderSpace opens new avenues for understanding and utilizing the latent capabilities of diffusion models. By mapping these models' visual potential into intuitive, composable directions, we take a step toward making their creative possibilities more accessible and interpretable to users.

% Image editing methods in diffusion models unlock the creativity of users. In this work we ask an alternate question: \emph{Can we organize and expose what of the diffusion model is already capable of?}.
% Existing methods for controlling image generation typically require users to manually specify edit directions for desired changes. This process is time-consuming, requires technical expertise, and limits the spontaneity of the creative process. For instance, if a user wants to adjust the smile of a generated person, they must explicitly request this edit, often through imprecise prompt engineering or model fine-tuning. This approach of predefined controls or manual specifications restricts users from fully exploring the latent capabilities of the model. There may be interesting stylistic variations or attributes that the model can generate, but users have no easy way to discover or utilize these.

% Natural visual disentanglement was an emergent property in the latent space of Generative Adversarial Models (GANs) \cite{harkonen2020ganspace,radford2015unsupervised, wu2021stylespace, shen2020interfacegan}. In particular, it has been observed that StyleGAN~\cite{karras2019style} stylespace neurons offer detailed control over many meaningful aspects of images that would be difficult to describe in words~\cite{wu2021stylespace}. However, diffusion models do not share such a compact latent space~\cite{park2023unsupervised}; and efforts to uncover such a space in the semantic embeddings of the text conditioning have met with limited success \nik{Nick - is there a specific citation you were thinking about?}.

% In this work we introduce \textbf{SliderSpace}, which takes a step towards uncovering an analogous low dimensional representation of diffusion models' visual breadth; in essence treating the diffusion model as many generators sharing parameters, where a particular generator is defined by a specific prompt. For a given prompt we sample many random seeds (and optionally prompt expansions using an LLM), generate the corresponding images, and apply an off the shelf feature extractor (in this work CLIP, but our method can be applied to any differentiable feature extractor). We use PCA to analyze these features, and for each of the leading $k$ principal components we train a LoRA \cite{} which causes the diffusion model to produces images which increase the feature magnitude along that component when passed back through the same feature extractor. This leads to a 'Slider' for each principal component, because each LoRA can be scaled and applied to the original diffusion model, continuously varying those visual features in the generated results (as measured, in our case, by CLIP).

% There are many other works that enhance the controllability of diffusion models. One common approach is enabling users to add spatial constraints to a generation either manually, or via a reference image \cite{zhang2023addingconditionalcontroltexttoimage, chen2024trainingfreeregionalpromptingdiffusion}, a second is leveraging more abstract embeddings (e.g. identity, style) extracted from a reference image \cite{ye2023ipadaptertextcompatibleimage, hertz2024stylealignedimagegeneration, li2023photomaker, shi2024instantbooth}, a third is finetuning a foundation model to better generate a concept important to the user \cite{ruiz2022dreambooth, kumari2022customdiffusion, Ryu_lora, hu2021lora}, and a fourth (most relevant to this work) is finding low-rank adaptors of the model based on a prompt or small training set which can be scaled to provide continous control over one aspect of generated image (e.g. night vs day, basic vs luxury, etc.) \cite{gandikota2023concept}. SliderSpace is complementary to all of these methods and offers something distinct. All of the other methods we are aware require the user (and / or model designer) to know in advance what type of control they want. In contrast SliderSpace assists users in discovering and controlling hidden capabilities present in the diffusion model's distribution of possible generations.

%We propose that truly intuitive creative control in a text-to-image model should meet three key criteria: \emph{discoverability}, \emph{intuitiveness}, and \emph{specificity}. The model should reveal controllable attributes that may not be immediately obvious, offer controls that are easy to understand and manipulate, and ensure each control affects a distinct attribute of the generated image.

% We demonstrate the utility and power of SliderSpace using three applications built on top of SDXL-DMD \cite{dmd}, because its fast generation speed lends itself well to the continuous control offered by SliderSpace.

% First, we study concept decomposition (Section \ref{sec:concept_exp}), where we learn sliders for a specific concept (e.g. 'monster', 'waterfall', 'car'). Through quantitative metrics of diversity and text alignment we demonstrate that the learned sliders dramatically boost the diversity of generations when randomly applied without harming text alignment; we also ask humans to qualitatively judge these results in a user study where they find the SliderSpace results to be more 'Diverse', 'Useful', and 'Creative' than our baselines.

% Second, we attempt to compare the automatic discoveries of SliderSpace to a large scale manual study of artistic styles (Section \ref{sec:art_exp}), open-sourced by ParrotZone \cite{parrotzone}. In this study SDXL was prompted with over 4300 artist names,  and based on visual inspection the cases of successful stylistic mimicry recorded. Quantitatively SliderSpace more closely matches the distribution of artistic variation discovered by ParrotZone than other baselines, and in our user studies was judged to be significantly more 'Diverse' and 'Useful' than the baselines. To our surprise humans even judged SliderSpace results to be slightly more 'Diverse' than the results generated by the manually discovered artist names of \cite{parrotzone}.

% Third, we attempt to use SliderSpace to reverse the mode collapse commonly observed in distilled few-step diffusion models relative to the original teacher model (Section \ref{sec:diverse_exp}). We quantitatively demonstrate that applying SliderSpace to SDXL-DMD leads to more closely matching the distribution of images by the original teacher, SDXL.

%Through extensive experiments on various state-of-the-art text-to-image models, we demonstrate that SliderSpace significantly enhances user control and creative expression in AI-assisted image generation tasks. Our method enables a range of applications, including concept decomposition and control, diversity improvement in generated images, customization dissection and edits, and the exploration of artistic styles inherent in the model.

% SliderSpace goes beyond providing a practical tool for enhanced creative control. By mapping the visual potential of diffusion models it can open new avenues for generative creativity and deepens our understanding of each model's hidden potential.
This paper addresses a human-robot cooperative navigation task under incomplete information. The remote human operator possesses an outdated map of the environment, while the robot can acquire accurate local observations. The human provides navigation guidance, and the robot communicates environmental updates. Together, they aim to reach a set of goal locations as efficiently as possible.

We design a simulated maze environment, \emph{CoNav-Maze}, adapted from MemoryMaze~\cite{pasukonis2022evaluating} to study this setting. In CoNav-Maze, the robot has perfect knowledge of its position and uses motion primitives to navigate between adjacent grid cells. This setup abstracts away low-level control and estimation errors, focusing on high-level human-robot coordination.

Formally, the environment is modeled as a Markov Decision Process (MDP) defined by the tuple $(\Scal, \Acal, T, R_\mathrm{env}, \gamma)$. \( \mathcal{S} \) is a product space comprising the robot’s discrete finite state and the set of remaining goal locations, capturing both its position and task progress. $\Acal$ is a finite set of actions, including movement to adjacent grids and transmitting a first-person image from one of eight evenly spaced camera angles. $T: \mathcal{S} \times \mathcal{A} \to \mathcal{S}$ is a deterministic transition function. $R_\mathrm{env}: \mathcal{S} \to \mathbb{R}$: is a real-valued reward function. $\gamma \in [0, 1)$ is a discount factor.

At each step $t$, the robot collects a local observation of nearby traversable and blocked cells within a radius $r$. It may also receive a human-provided trajectory $\zeta_t$. The robot then selects an action $a_t$ to either move or transmit an image.

The human operator starts with an inaccurate global map $x \in \mathcal{X}$, representing traversable and blocked cells. By analyzing the robot’s trajectory and image transmissions, the human refines their map to provide more accurate guidance.
\section{Causal IL as CMRs}\label{sec:method}

In this section, we demonstrate that performing causal IL in our framework is possible using trajectory histories as instruments. In the next step, we show that the problem can be described as CMRs and propose an effective algorithm to solve it.

The typical target for IL would be the expert policy $\pi_E$ itself. However, since the expert has access to information, namely $u^o_t$, which the imitator does not, the best thing an imitator can do is to learn a history-dependent policy $\pi_h$ that is the closest to the expert. A natural choice is the conditional expectation of $\pi_E(s_t,u^o_t)$ on the history $h_t$:
\begin{align}
\pi_h(h_t)\coloneqq \expectE_{\probP(u^o_t\mid h_t)}[\pi_E(s_t,u^o_t)]=\expectE[\pi_E(s_t,u^o_t)\mid h_t],\nonumber
\end{align}
% where $p(u^o_t\mid h_t)$ is a distribution over expert-observable confounders and captures the information about $u^o_t$ can be inferred from the trajectory history. 
because the conditional expectation minimizes the least squares criterion~\citep{hastie01statisticallearning} and $\pi_h$ is the best predictor of $\pi_E$ given $h_t$. In $\pi_h$, the distribution $\probP(u^o_t\mid h_t)$ captures the information about $u^o_t$ that can be inferred from trajectory histories.
\begin{remark}
\emph{Learning $\pi_h$ is not trivial. Policies learnt naively using behaviour cloning (i.e., $\expectE[a_t\mid h_t]$) fail to match $\pi_E$. In view of~\cref{eq:action}, we have that
\begin{align} 
\expectE[a_t\mid h_t]&=\expectE[\pi_E(s_t,u^o_t) \mid h_{t}]+\expectE[u^\epsilon_t\mid h_{t}]\nonumber\\
&=\pi_h(h_t)+\expectE[u^\epsilon_t\mid h_{t}],\label{eq:history_policy}
\end{align}
where $\expectE[u^\epsilon_t\mid h_{t}]\neq 0$ due to the spurious correlation between $u^\epsilon_t$ and the trajectory history $h_t$. As a result, $\expectE[a_t\mid h_t]$ becomes biased, which can lead to arbitrarily worse performance compared to $\pi_E$.   }
\end{remark}

\vspace{-5pt}
\paragraph{Derivation of CMRs.} 
Leveraging the confounding horizon from Assumption~\ref{assump:horizon}, it becomes possible to break the spurious correlation using the independence of $u^\epsilon_t$ and $u^\epsilon_{t-k}$. We propose to use the $k$-step trajectory history $h_{t-k}=(s_{1},a_{1},...,s_{t-k})$ as an instrument for the current state $s_t$. Taking the expectation conditional on $h_{t-k}$ in~\cref{eq:history_policy} yields
\begin{align*}
    \expectE[a_t\mid h_{t-k}] & = \expectE\left[\expectE[a_t\mid h_{t}]\mid h_{t-k}\right] \\ & = \expectE[\pi_h(h_t)\mid h_{t-k}]+\expectE[\expectE[u^\epsilon_t\mid h_{t}]\mid h_{t-k}] \\
    & = \expectE[\pi_h(h_t) \mid h_{t-k}]+\expectE[u^\epsilon_t\mid h_{t-k}]
\end{align*}
where we use the fact that $h_{t-k}$ is $\sigma(h_t)$-measurable because $h_{t-k}\subseteq h_t$. Next, recall that $u^\epsilon_t\indep u^\epsilon_{t-k}$ by Assumption~\ref{assump:horizon}, which implies $u^\epsilon_t\indep h_{t-k}$, so that % Hence, since $\expectE[u^\epsilon_t] = 0$, we obtain
\begin{align}
    \expectE[a_t\mid h_{t-k}] &= \expectE[\pi_h(h_t) \mid h_{t-k}]+\expectE[u^\epsilon_t]\nonumber\\
    &=\expectE[\pi_h(h_t) \mid h_{t-k}].
\end{align}

As a result, the problem of learning $\pi_h$ reduces to solving for $\pi_h$ that satisfies the following identity
\begin{align}
    \expectE[a_t-\pi_h(h_t)\mid h_{t-k}]=0,\label{eq:CMR}
\end{align}
which is a CMR problem as defined in~\cref{sec:cmr}. In this case, both $a_t$ and $h_t$ are observed in the confounded expert demonstrations, and $h_{t-k}$ acts as the instrument. 

To make sure the instrument $h_{t-k}$ is valid, we check that it satisfies the conditions of~\cref{assump:iv}. Firstly, we have checked that $u^\epsilon_t\indep h_{t-k}$. Secondly, the environment and the expert policy are non-trivial, which means $\probP(h_t\mid h_{t-k})$ is not constant in $h_{t-k}$. Finally, $h_{t-k}$ indeed only affects $a_t$ through $s_t$ by the Markovian property. However, the strength of the instrument, which informally represents the correlation between the instrument $h_{t-k}$ and $h_t$, plays an important role in how well we can identify $\pi_h(h_t)$ by solving the CMRs in~\cref{eq:CMR}. In particular, we see that, as the confounding horizon $k$ increases, the correlation between $h_{t-k}$ and $h_t$ weakens and $h_{t-k}$ becomes a weaker instrument. This means that it is less able to identify $\pi_h$ via the CMR in~\cref{eq:CMR} and the final learnt imitator will have poorer performance. This is confirmed theoretically in Proposition~\ref{prop:ill-posed} and experimentally in~\cref{sec:exps}, and we will formalise this notion of instrument strength in~\cref{sec:theory}.


% Note this problem is equivalent to solving an IV regression on~\cref{eq:history_policy}, where $Y=\expectE[a_t\lvert h_t]$, $f(x)=\pi_h(h_t)$, $\epsilon=\expectE[u^\epsilon_t$ and the instrument $Z=h_{t-k}$.




\subsection{Practical Algorithms for Solving the CMRs}

\begin{algorithm}[tb]
   \caption{DML-IL}
   \label{alg:DML-IL}
\begin{algorithmic}[1]
   \STATE {\bfseries input} Dataset $\dataset_E$ of expert demonstrations, Confounding noise horizon $k$
   \STATE Initialize the roll-out model $\hat{M}$ as a Gaussian mixture model\label{algo:roll_out_1}
    \REPEAT
   \STATE Sample $(h_{t},a_t)$ from data $\dataset_E$
   \STATE Fit the roll-out model $(h_t,a_t)\sim\hat{M}(h_{t-k})$ to maximize the log likelihood 
\UNTIL{convergence}\label{algo:roll_out_2}
   \STATE Initialize the expert model $\hat \pi_h$ as a neural network
   \REPEAT
   % \FOR{$k=1$ {\bfseries to} $K$}
   \STATE Sample $h_{t-k}$ from $\dataset_E$
   \STATE Generate $\hat{h}_t$ and $\hat{a}_t$ using the roll-out model $\hat{M}$
   \STATE Update $\hat \pi_h$ to minimise the loss $\ell:= \norm{\hat{a}_t - \hat{\pi}_h (\hat h_t)}_2$
   % \ENDFOR
    \UNTIL{convergence}
    \STATE {\bfseries return} A history-dependent imitator policy $\hat{\pi}_h$
\end{algorithmic}
\end{algorithm}

There are various techniques~\citep{Shao2024,Bennett2019,Xu2020,Dikkala2020} for solving the CMRs $\expectE[a_t\lvert h_{t-k}]=\expectE[\pi_h(h_t) \lvert h_{t-k}]$. Here, the \textit{CMR error} that we aim to minimise is given by 
\begin{align*}
\sqrt{\expectE\big[\expectE[a_t-\hat{\pi}_h(h_t)\lvert h_{t-k}]^2\big]}=\norm{\expectE[a_t-\hat{\pi}_h(h_t)\lvert h_{t-k}]}_{2}.    
\end{align*}
In~\cref{alg:DML-IL}, we introduce DML-IL, an algorithm adapted from the IV regression algorithm DML-IV~\citep{Shao2024}\footnote{DML stands for double machine learning~\citep{Chernozhukov2018Double}, which is a statistical technique to ensure fast convergence rate for two-step regression, as is the case in~\cref{alg:DML-IL}.}, which solves our CMRs by minimising the CMR error. The first part of the algorithm (line 3-7) learns a roll-out model $\hat{M}$ that generates a trajectory $k$ steps ahead given $h_{t-k}$. Then, the roll-out model $\hat{M}$ is used to train the policy model $\hat{\pi}_h$ (line 8-13). $\hat{\pi}_h$ takes the generated trajectory $\hat{h}_t$ from $\hat{M}(h_{t-k})$ as inputs, and minimises the mean squared error to the next action. Using generated trajectories is crucial in breaking the spurious correlation caused by $u^\epsilon_t$ between past states and actions, and using the trajectory history before $h_{t-k}$ allows the imitator to infer information about $u^o_t$.

DML-IL can also be implemented with $K$-fold cross-fitting, where the dataset is partitioned into $K$ folds, with each fold alternately used to train $\hat{\pi}_h$ and the remaining folds to train $\hat{M}$. This ensures unbiased estimation and improves the stability of training. The base IV algorithm DML-IV with $K$-fold cross-fitting is theoretically shown to converge at the rate of $O(N^{-1/2})$~\citep{Shao2024}, where $N$ is the sample size, under regularity conditions. DML-IL with $K$-fold cross-fitting (see~\cref{appendix:dmlil} for details) will thus inherit this convergence rate guarantee. 

Note that~\cref{alg:DML-IL} requires the confounding noise horizon $k$ as input. While the exact value of $k$ can be difficult to obtain in reality, any upper bound $\bar{k}$ of $k$ is sufficient to guarantee the correctness of ~\cref{alg:DML-IL}, since $h_{t-\bar{k}}$ is also a valid instrument. Ideally, we would like a data-driven approach to determine $k$. Unfortunately, it is generally intractable to empirically verify whether $h_{t-k}$ is a valid instrument from a static dataset, especially the unconfounded instrument condition (i.e., $h_{t-k}\indep u^\epsilon_t$). Therefore, we rely on the user to provide a sensible choice of $\bar{k}$ based on the environment that does not substantially overestimate $k$.


\subsection{Theoretical Analysis}\label{sec:theory}

% \begin{align}
% p(u_t\lvert do(a_{t-k+1}),...,do(a_{t-1}),s_{t-k+1},...,s_{t-1})&\propto p(h_t)p_{\mu_0}(s_{t-k+1})\prod_{i=t-k+1}^{t-1} \transitions(s_{i+1}\lvert s_i,a_i,u_i)
% \end{align}

% since $$(u_t\indep a_{(t-k+1)...(t-1)} \lvert s_{(t-k+1)...(t_1)})_{\mathcal{G}_{\underline{a{(t-k+1)...(t-1)}}}}$$
% on the causal graph $\mathcal{G}_{\underline{a{(t-k+1)...(t-1)}}}$ where the arrows going into $a_{(t-k+1)...(t-1)}$ are removed.



In this section, we derive theoretical guarantees for our algorithm, focusing on the imitation gap and its relationship with existing work.


On a high level, in order to bound the imitation gap of the learnt policy $\hat{\pi}_h$, i.e., $J(\pi_E)-J(\hat{\pi}_h)$, we need to control:
\begin{enumerate}
    \item[($i$)] The amount of information about the hidden confounders that can be inferred from trajectory histories;
    \item[($ii$)] The ill-posedness (or identifiability) of the set of CMRs, which intuitively measures the strength of the instrument $h_{t-k}$;
    \item[($iii$)] The disturbance of the confounding noise to the states and actions at test time.
\end{enumerate}
These factors are all determined by the environment and the expert policy. To control ($i$), we measure how much information about $u^o_t$ is captured by the trajectory history $h_t$ by analysing the Total Variation (TV) distance between the distribution of $u^o_t$ and $\expectE[u^o_t\lvert h_t]$ along the trajectories of $\pi_E$. To control ($ii$) and ($iii$), we need to introduce the following two key concepts.

\begin{definition}[The ill-posedness of CMRs~\citep{Dikkala2020,Chen2012}]

Given the derived CMRs in~\cref{eq:CMR}, for a policy $\pi\in\Pi$, $\norm{\pi_E-\pi}_2$ is the root mean squared error to the expert and $\norm{\expectE[a_t-\pi(s_t)\lvert s_{t-k}]}_2$ is the CMR error we aim to minimise. Then, the \emph{ill-posedness} $\ill(\Pi,k)$ of the policy space with confounding noise horizon $k$ is given by
\begin{align*}
    \ill(\Pi,k)=\sup_{\pi\in\Pi} \frac{\norm{\pi_E-\pi}_{2}}{\norm{\expectE[a_t-\pi(h_t)\lvert h_{t-k}]}_{2}}.
\end{align*}
\end{definition}
The ill-posedness $\ill(\Pi,k)$ measures the strength of the instrument where a higher $\ill(\Pi,k)$ indicates a weaker instrument. It bounds the ratio between the learning error of the imitator following our CMR objective and its $L_2$ error to the expert policy. 

As discussed previously, intuitively, the strength of the instrument would decrease as the confounding horizon $k$ increases. This is in fact true and is confirmed by the following proposition. The proof is deferred to~\cref{appendix:prop}. 
\begin{proposition}\label{prop:ill-posed}
The ill-posedness $\ill(\Pi,k)$ is monotonically increasing as the confounded horizon $k$ increases.
\end{proposition}

Next, we introduce the notion of c-TV stability.
\begin{definition}[c-total variation stability~\citep{Bassily2021,Swamy2022_temporal}]
Let $P(X)$ be the distribution of a random variable $X:\Omega\rightarrow \mathcal{X}$. $P(X)$ is c-TV stable if for $a_1,a_2\in \mathcal{X}$ and $\Delta>0$,
\begin{align*}
\norm{a_1-a_2}\leq\Delta \implies \delta_{TV}(a_1+X,a_2+X)\leq c\Delta.
\end{align*}
where $\norm{\cdot}$ is some norm defined on $\mathcal{X}$ and $\delta_{TV}$ is the total variation distance.
\end{definition}
A wide range of distributions are c-TV stable. For example, standard normal distributions are $\frac{1}{2}$-TV stable. We apply this notion to the distribution over $u^\epsilon_t$ to bound the disturbance it induces in the trajectory and the expected return.

With the notion of ill-posedness and c-TV stability, we can now analyse and upper bound the imitation gap $J(\pi_E)-J(\hat{\pi}_h)$ by controlling the three components $(i)-(iii)$ discussed above. 
% We present the main result for this paper, where t
The full proof is deferred to~\cref{appendix:gap}.

\begin{theorem}[Imitation Gap Bound]\label{thm:gap}
Let $\hat{\pi}_h$ be the learnt policy with CMR error $\epsilon$ and let $\ill(\Pi,k)$ be the ill-posedness of the problem. Assume that $\delta_{TV}(u^o_t,\expectE_{\pi_E}[u^o_t\lvert h_t])\leq\delta$ for $\delta\in\realNumber^+$, $P(u^\epsilon_t)$ is c-TV stable and $\pi_E$ is deterministic. Then, the imitation gap is upper bounded by 
\begin{align*}
    J(\pi_E)-J(\hat{\pi}_h)\leq T^2\big(c\epsilon\ill(\Pi,k)+2\delta\big)=\mathcal{O}\big(T^2(\delta+\epsilon)\big).
\end{align*}
\end{theorem}
This upper bound scales at the rate of $T^2$, which aligns with the expected behaviour of imitation learning without an interactive expert~\citep{Ross2010}.
Next, we show that the upper bounds on the imitation gap from prior work~\citep{Swamy2022_temporal, Swamy2022} are special cases of
% of  subsumed by the unifying causal IL framework introduced in Section~\ref{sec:setting} are special cases of 
Theorem~\ref{thm:gap}. The proofs are deferred to~\cref{appendix:corollaries}.
\begin{corollary}\label{corollary:noUo}
In the special case that $u^o_t = 0$, i.e., there are no expert-observable confounders, or $u^o_t=\expectE_{\pi_E}[u^o_t\lvert h_t]$, i.e., $u^o_t$ is $\sigma(h_t)$ measurable (all information about $u^o_t$ is contained in the history), the imitation gap is upper bounded by
\begin{align*}
    J(\pi_E)-J(\hat{\pi}_h)\leq T^2\big(c\epsilon\ill(\Pi,k)\big)=\mathcal{O}\big(T^2\epsilon\big),
\end{align*}
which coincides with Theorem 5.1 of~\citet{Swamy2022_temporal}.
\end{corollary}

When there are no hidden confounders, i.e, $u^\epsilon_t=0$, our framework is reduced to that of~\citet{Swamy2022}. However, \citet{Swamy2022} provided an abstract bound that directly uses the supremum of key components in the imitation gap over all possible Q functions to bound the imitation gap. We further extend and concretise the bound using the learning error $\epsilon$ and the TV distance bound $\delta$ instead of relying on the suprema.


\begin{corollary}\label{corollary:unconfounded}
In the special case that $u^\epsilon_t=0$, if the learnt policy has optimisation error $\epsilon$,  the imitation gap is upper bounded by
\begin{align*}
    J(\pi_E)-J(\hat{\pi}_h)\leq T^2\left(\frac{2}{\sqrt{\dim(A)}}\epsilon+2\delta \right),
\end{align*}
which is a concrete bound that extends the abstract bound in Theorem 5.4 of~\cite{Swamy2022}.
\end{corollary}

\begin{remark}
\emph{If both $u^\epsilon_t$ and $u^o_t$ are zero, we then recover the classic setting of IL without confounders~\citep{Ross2010}, and the imitation gap bound is $T^2\epsilon$, where $\epsilon$ is the optimisation error of the algorithm.}
\end{remark}
% \section{Simulation Evaluation \& Results}\label{sec:results}

\subsection{Baseline Planners}

To evaluate the performance of \PlannerName, we compare it against several baseline methods. In the following section, we describe these baselines, their implementation details, and their respective advantages and limitations, particularly in the context of information gathering in large, high-dimensional search spaces. The simulation framework and vehicle parameters remain consistent across all planners, and each method is allowed to replan during testing.

\subsubsection{Monte-Carlo Tree Search}

Monte Carlo Tree Search (MCTS) can be a powerful technique for finding feasible and optimal paths in complex environments. It is a heuristic search algorithm that builds a search tree incrementally through repeated simulations. At each iteration, it selects a node to explore based on a selection policy (often the Upper Confidence Bound or UCB1 algorithm), expands the tree by adding possible actions from that node, runs a simulation from the newly added node, and updates the statistics of nodes along the path traversed during the simulation. 

The UCB1 (Upper Confidence Bound) algorithm is a technique commonly used in the context of multi-armed bandit problems and Monte Carlo Tree Search (MCTS) for balancing exploration and exploitation. It helps in selecting actions or nodes that are likely to yield high rewards while also exploring less-frequented options to gather more information about their potential rewards. 

We formulate our UCB score in the following manner, \\
\begin{equation*}
    UCB_\text{node} = \frac{I(X_{\text{node}})}{\alpha} + C \times \sqrt{\frac{\ln(N_\text{tree})}{N_\text{node}}}
\end{equation*}
%  $
% UCB_\text{node} = \frac{\overline{X_\text{node}}}{\alpha} + C \times \sqrt{\frac{\ln(N_\text{tree})}{N_\text{node}}}
% $ \\
Here $I(X_{\text{node}})$ denotes the estimated information gain from the node, $\alpha$ denotes the normalization factor which is given by $\frac{B}{v_\text{desired}}$, $B$ being the maximum planning budget and $v_\text{desired}$ being the desired speed of our UAV. $C$ denotes the exploration weight, and $N_\text{tree}$ denotes the number of visits to the tree root node while $N_\text{node}$ denotes the number of times the present node has been visited.

After selecting a candidate node, if it has been visited before, it is expanded by applying motion primitives to generate child nodes, growing the tree. Unvisited nodes skip this step. Following expansion, either the unvisited candidate node or one of its children is selected for the simulation phase, where the future values of nodes along the path are estimated to update the total potential information gain. This informs the selection policy in subsequent iterations. Once planning time is exhausted, the path with the highest information gain is returned.

% with authors goes here
\begin{figure}[t]
\centering
\includegraphics[trim={.7cm 0cm .5cm 1.4cm},clip,width=\columnwidth]{figs/5_/Results1v3.pdf}
\caption{The Monte Carlo simulation results for the planners. The plots show the average percent reduction in entropy over the course of the simulations, and the shading shows the 95\% confidence intervals. IA-TIGRIS outperforms all of the baselines.}
\label{fig:mc_results}
\end{figure}

While MCTS is probabilistically guaranteed to converge to the optimal path \cite{mcts_ref_1}, it is constrained to actions within a predefined set of motion primitives. Its reliance on random sampling to estimate the future value of nodes can result in poor approximations, particularly in environments with sparse, localized pockets of high information gain. This limitation is especially pronounced in large search areas or scenarios with large budgets constraints, where estimating future node values becomes increasingly expensive. As a result, in such scenarios, MCTS is often implemented with a finite planning horizon, which can restrict its ability to account for long-term consequences or dependencies in the environment.

% This property of MCTS, which causes unguided exploration of the environment, leads to increased convergence times on the optimal path, as a result of a lot of budget being spent in exploring information sparse areas of the map. 
% Also, the computation time of MCTS increases exponentially with the depth of the search tree. The time complexity of MCTS is given by $\mathcal{O}(\frac{T}{t_\text{iter}} \cdot |A|^d)$. Here, $T$ is the total planning time and $t_\text{iter}$ is the time taken per iteration of the planning loop. $|A|$ is the number of actions and $d$ represents the average depth of the search tree. 

% The above limitations are not inconsequential in the context of performing informative path planning in large high-dimensional search spaces. We compare MCTS with \PlannerName, in \ref{}, and empirically demonstrate its drawbacks and how \PlannerName, is able to outperform MCTS in the context of the mission parameters we examine in this work.  

\subsubsection{Greedy}

For the greedy planner, we iterated through each cell within the search bounds and calculated the reward for a given cell $i$ as $g_i = R(X_i) / d_i$ where $R(X_i)$ is given through \eqref{equ:reward} and $d_i$ represents the Euclidean distance between the current position the robot at the current time $t$ and the closest viewpoint to the cell. To compute this viewpoint, the yaw between the current pose of the robot and the intersected cell is first calculated. Using the robot's sensor configuration and this yaw, $x$ and $y$ coordinates are calculated that view the cell at the desired flight altitude. With this formulation, the planner prioritizes regions with a high ratio of entropy to distance. This can lead to locally optimal choices that contradict with paths that lead to higher information gain over the entire trajectory. 

% without authors goes here
% \begin{figure}[t]
% \centering
% \includegraphics[trim={.7cm 0cm .5cm 1.4cm},clip,width=\columnwidth]{figs/5_/Results1v3.pdf}
% \caption{The Monte Carlo simulation results for the planners. The plots show the average percent reduction in entropy over the course of the simulations, and the shading shows the 95\% confidence intervals. IA-TIGRIS outperforms all of the baselines.}
% \label{fig:mc_results}
% \end{figure}


\begin{figure*}[t]
    \centering
    \begin{subfigure}[b]{0.99\textwidth}
        \centering
        \includegraphics[trim={0cm 0.3cm 0cm 0cm},clip,width=\textwidth]{figs/5_/Fig2v1_target.png}
        % \caption{Slice by targets}
        % \vspace{.1cm}
    \end{subfigure}
    
    \begin{subfigure}[b]{0.99\textwidth}
        \centering
        \includegraphics[trim={0cm 0cm 0cm 0cm},clip,width=\textwidth]{figs/5_/Fig2v1_sigma.png}
        % \caption{Slice by sigma }
    \end{subfigure}
    \caption{A comparison of the methods based on the number of sampled prior clusters and the standard deviation of sampled prior clusters. IA-TIGRIS is most effective compared to the baselines when there is high variation in the search space. As the search space prior information becomes more evenly spread out, the performance gap between the methods tends to decrease.}
    \label{fig:targets_sigmas}
\end{figure*}

\subsubsection{Random}

The random planner operates by iteratively sampling points within the defined search bounds and calculating the minimum-cost path to observe each sampled point. This process is repeated until the available budget is fully expended. The random planner does not utilize any prior information about the environment or target distribution. Additionally, it does not optimize the sequence of actions, instead treating each sampled point independently without considering the global structure of the search problem. This simplicity allows the random planner to highlight the performance benefits of more sophisticated methods by providing a lower-bound comparison for evaluation.

\subsubsection{Coverage}

The coverage planner generates a plan that systematically covers the entire search space using a straightforward lawn-mower pattern. The spacing between each pass is set to match the width of the projected observation footprint at 20\% from the bottom, ensuring that no grid cells are missed. This spacing also maintains a distance that enables high-quality sensor measurements. However, due to the size of the search spaces considered, the coverage planner spends significant time surveying empty regions. This approach results in inefficient use of the budget, as it prioritizes full coverage with safe sensor overlap, even in areas with little or no valuable information. While simple and robust, this method highlights the tradeoff between exhaustive coverage and efficient, targeted exploration.

% \subsubsection{Branch and Bound}
% The branch and bound baseline is based on motion primitive planning. In each future step the drone has a set of motion primitives with future states and each of these future states also has a set of motion primitives. In this way, a tree can be built with multiple path candidates. The path candidate with the highest information gain will be selected and form the output. 

% By adding branch and bound, there will be an estimation of a node's upper bound information reward, using the node's current information reward, updated information map and the remaining budget. If this upper bound is already lower than the information reward of any other node in the tree, the corresponding node will be closed and not expanded in the future to accelerate the expansion of the tree. 



\subsection{Tests and Analysis}
% To evaluate the efficacy of IA-TIGRIS compared to the baseline methods, we conduct Monte Carlo testing as well as analyze how the prior and budget affect the performance of each method. In all of these test cases, there are no time-based or priority rewards and have horizon lengths set to the full budget. All tests were performed using an Intel Xeon CPU E5-2620 v4 @ 2.10GHz.
To evaluate the efficacy of IA-TIGRIS against baseline methods, we perform Monte Carlo testing and analyze the impact of the prior and budget on the performance of each method. In all test cases, rewards are calculated using \eqref{equ:reward}, and horizon lengths are set to match the full budget. The tests are conducted on an Intel Xeon CPU E5-2620 v4 @ 2.10GHz, ensuring consistent computational conditions across all evaluations.

% Random sample across which parameters.

% Quantitative ideas. Look into number and std of prior (metric for this? std of grid cell values, mediuan, mean,). 
% Uniform prior? 
% Split distinct regions, not smooth. 
% Compare to coverage and amount of time to reach specific amount. 
% Compare with different budgets. 
% Repeatability test. 
% Graph size vs time. 
% Look at coverage with different altitudes or widths. Something that shows long horizon vs not nature of things?
% Shape of search space?
% Time/budget to get x\% of all info gain. Have to do moving horizon. 
% Targets detected? 

% Key thought for results where I show time, our optimization does not optimize for time, only final value. Key thing to show across the different budgets. 

% \BM{Qualitative. Nayana idea of plot with example sampled case. Should add one here.} 



\subsubsection{Monte Carlo Testing}
Our simulated testing environment is a $5000\times5000$ m square with Gaussian-distributed prior information randomly placed throughout the search space. The number of prior clusters was sampled uniformly between $[4,20]$, with standard deviations between $[60,450]$, and maximum value between $[0.05,0.5]$. 

The results of $100$ Monte Carlo tests are shown in Fig.~\ref{fig:mc_results}. IA-TIGRIS clearly outperforms the other methods, achieving nearly a $40\%$ greater reduction in entropy than the next best method. Early in the simulation, the greedy method initially gains information more quickly, as expected, but this does not translate to better long-term performance. Since our method optimizes for total information gain, it generates paths that maximize information collection over the entire budget. MCTS performed slightly worse than the greedy approach.

The random paths slightly outperformed the coverage paths. This is likely because the lawnmower strategy requires sufficient overlap between passes to avoid missing areas, and its long straight paths often lead to redundant observations due to the UAV’s forward-facing camera. Changing the heading of the UAV is beneficial to viewing more of the search space, which may explain why random paths performed better.

We also conducted Monte Carlo tests where either the number of prior clusters or their standard deviation was held constant to analyze how variations in the information map affect planner performance. The results, shown in Fig.~\ref{fig:targets_sigmas}, include two cases: the upper figure fixes the number of priors, while the lower figure fixes their standard deviation. All other agent and simulation parameters remained unchanged.


% The first thing to note from these results is that for all tests the proportional performance gap between IA-TIGRIS and the baselines increases as the number and standard deviation of the Gaussian priors decreases. As the search space becomes more uniformly filled with entropy in the information map, the need for longer-horizon planning decreases and other simple or random approaches can perform satisfactorily given the testing budget. As the information becomes more sparsely distribution in the space, such as when the information is contained in separated pockets of areas, there is a greater need to plan longer-horizon paths that reason about the given budget.
% \BM{Could have figures here or refer to others}

Across these tests, the performance gap between IA-TIGRIS and the baselines widens as the number and standard deviation of the Gaussian priors decrease. When entropy is more uniformly distributed across the search space, simpler methods perform reasonably well within the given budget. However, when information is concentrated in sparse, distinct regions, longer-horizon planning becomes essential. In such cases, IA-TIGRIS demonstrates a significant advantage by effectively reasoning about the budget and prioritizing high-value regions.

% Show plot of first plans expected info gain versus planning time. (plans not executed)


\subsubsection{Budget Analysis}
To evaluate the impact of budget constraints on performance, we conducted additional tests beyond our initial Monte Carlo experiments, evaluating budgets of $5000$ m, $10000$ m, $30000$ m, and $60000$ m. Table~\ref{tab:budgets} summarizes the average entropy reduction across these budgets.

\definecolor{tabfirst}{rgb}{1, 0.7, 0.7} % red
\definecolor{tabsecond}{rgb}{1, 0.85, 0.7} % orange
\definecolor{tabthird}{rgb}{1, 1, 0.7} % yellow
\begin{table}[t]
    \centering
    \resizebox{\linewidth}{!}{
    \begin{tabular}{l|ccccc}
    & $5000$ m & 10000 m  & 15000 m& 30000 m& 60000 m\\ \hline

    % \hline
    IA-TIGRIS  &  \cellcolor{tabfirst}$9.41\pm1.0$ &  \cellcolor{tabfirst}$18.28\pm1.8$ & \cellcolor{tabfirst}$25.36\pm2.3$ & \cellcolor{tabfirst}$41.08\pm2.9$ & \cellcolor{tabfirst}$58.85\pm2.9$ \\
    Greedy  &  \cellcolor{tabsecond}$6.99\pm0.8$ &  \cellcolor{tabsecond}$13.10\pm1.5$ & \cellcolor{tabsecond}$17.97\pm2.0$ & \cellcolor{tabthird}$30.00\pm2.3$ & \cellcolor{tabsecond}$49.38\pm3.5$ \\
    MCTS  &  \cellcolor{tabthird}$6.06\pm0.7$ &  \cellcolor{tabthird}$11.80\pm1.1$ & \cellcolor{tabthird}$17.11\pm1.4$ & \cellcolor{tabsecond}$30.21\pm2.2$ & \cellcolor{tabthird}$48.68\pm2.7$ \\
    Random  &  $2.19\pm0.3$ & $4.29\pm0.7$ & $6.61\pm0.6$ & $17.50\pm1.2$ & $22.47\pm1.4$ \\
    Coverage  &  $1.58\pm0.3$ &  $2.82\pm0.4$ & $4.09\pm0.7$ & $12.04\pm1.9$ & $16.77\pm2.4$ \\

    \end{tabular}
    }
    \caption{Monte Carlo testing results given different budgets. The values are the average percent reduction in entropy and the 95\% confidence bounds. \mbox{IA-TIGRIS} had the best performance for all budgets.}
    \label{tab:budgets}
\end{table}
%$\uparrow$ 

IA-TIGRIS consistently achieved the highest entropy reduction across all budget constraints, with a statistically significant margin over alternative methods. Greedy generally ranked second but was slightly outperformed by MCTS at the $30000$ m budget level. Greedy and MCTS exhibited comparable performance throughout the tests, with their results closely tracking each other. Consistent with our previous findings, Random and Coverage methods yielded the lowest results.


Among the tested methods, only IA-TIGRIS and MCTS explicitly incorporate budget constraints into their planning algorithms. Notably, at lower budgets ($5000$ m and $10000$ m), these methods achieved higher entropy reduction compared to the equivalent time steps ($200$ s and $400$ s) in the $15000$ m budget scenario shown in Fig.~\ref{fig:mc_results}. This improved performance stems from IA-TIGRIS's optimization of total path reward under budget constraints, contrasting with the myopic next-best-action approach of the greedy method. The remaining methods---Greedy, Random, and Coverage---maintain consistent behavior regardless of budget constraints, as their planning strategies do not account for resource limitations.


The performance gap between IA-TIGRIS and the next-best method varied with budget size, showing margins of $34.6\%$, $39.5\%$, $41.1\%$, $36.0\%$, and $19.2\%$ in ascending budget order. This gap widened through the first three budget levels as problem complexity increased, before declining significantly at higher budgets. This performance pattern suggests that implementing a planning horizon could enhance efficiency by limiting tree search depth, enabling the planner to prioritize path quality optimization over exhaustive space exploration.


% percent improved from next best
% 34.6, 39.5, 41.1, 36.0, 19.2
% reasons, too long horizon is a larger search space, so less quality paths closer. Or larger horizon, more packing in


% with authors goes here
\begin{figure}[t] 
    \centering
    \renewcommand\arraystretch{0} % Adjust the height between rows here
    \setlength{\tabcolsep}{1pt} % Adjust the column separation here
    \begin{tabular}{c}
        \begin{tikzpicture}
            \node[anchor=south west, inner sep=0] (image) at (0,0) {
                \includegraphics[width=0.9\linewidth]{figs/5_/google_earth_prior.png}
            };
            \begin{scope}[x={(image.south east)},y={(image.north west)}]
                % \fill[OrangeRed] (0.02, 0.03) circle (2pt); 
                % \fill[OrangeRed] (0.51, 0.04) circle (2pt); 
                % \fill[OrangeRed] (0.61, 0.04) arc (0:90:2pt); 
                \fill[Orange, opacity=0.8] (0.74, 0.45) circle (3pt); % Adjust 
                \fill[Orange, opacity=0.8] (0.27, 0.42) circle (3pt); % Adjust 
                \fill[Orange, opacity=0.8] (0.39, 0.63) circle (3pt); % Adjust 
            \end{scope}
        \end{tikzpicture} \\
        % \includegraphics[width=0.9\linewidth]{figs/5_/google_earth_prior.png} \\
        \\
        \includegraphics[width=0.9\linewidth]{figs/5_/google_earth_path.png} 
    \end{tabular}
    \caption{Google Earth screenshots illustrating the mission planning process and execution. Top: Areas of high entropy targeted for search are highlighted in red, representing regions with a binary occupied/unoccupied probability of 0.2. Three points of particular interest, each assigned a 0.5 probability, are marked in orange. Bottom: The executed drone flight path (yellow) shows the optimized path for maximum information gain across the search space.} 
    \label{fig:google_earth}
\end{figure}
\begin{figure}[t]
\centering
% https://docs.google.com/presentation/d/1RjI-QqHpBRLHN60UAxzmQYs4EaWaVCOoSBkEkA39kk0/edit?usp=sharing
\includegraphics[width=\columnwidth]{figs/5_/m600_labeled.jpg}
\caption{Hexarotor system (DJI M600 Pro) with onboard compute and camera. Left image shows drone on the ground, right image shows drone in flight.}
\label{fig:m600}
\end{figure}


\section{Field Deployments}\label{sec:field}


\subsection{Hexarotor Deployment}
The first field experiment that we present uses a hexarotor drone to cover an urban area shown in Fig.~\ref{fig:fig1}.
We designed this field experiment to simulate classifying where cars are within a search area.  
Hence, we set the plan request to focus on parking lots at the field test site (Fig.~\ref{fig:google_earth}, top), with the addition of three chosen grid cells within the parking lots being marked as having a higher uncertainty. The plan request boundaries and priors were created with GPS coordinates in Google Earth, exported as kml files, and then converted into our plan request message format. 

The following sections details the hardware, autonomy, and experimental results for our hexarotor deployments.

% without the authors goes here
% \begin{figure}[t] 
%     \centering
%     \renewcommand\arraystretch{0} % Adjust the height between rows here
%     \setlength{\tabcolsep}{1pt} % Adjust the column separation here
%     \begin{tabular}{c}
%         \begin{tikzpicture}
%             \node[anchor=south west, inner sep=0] (image) at (0,0) {
%                 \includegraphics[width=0.9\linewidth]{figs/5_/google_earth_prior.png}
%             };
%             \begin{scope}[x={(image.south east)},y={(image.north west)}]
%                 % \fill[OrangeRed] (0.02, 0.03) circle (2pt); 
%                 % \fill[OrangeRed] (0.51, 0.04) circle (2pt); 
%                 % \fill[OrangeRed] (0.61, 0.04) arc (0:90:2pt); 
%                 \fill[Orange, opacity=0.8] (0.74, 0.45) circle (3pt); % Adjust 
%                 \fill[Orange, opacity=0.8] (0.27, 0.42) circle (3pt); % Adjust 
%                 \fill[Orange, opacity=0.8] (0.39, 0.63) circle (3pt); % Adjust 
%             \end{scope}
%         \end{tikzpicture} \\
%         % \includegraphics[width=0.9\linewidth]{figs/5_/google_earth_prior.png} \\
%         \\
%         \includegraphics[width=0.9\linewidth]{figs/5_/google_earth_path.png} 
%     \end{tabular}
%     \caption{Google Earth screenshots illustrating the mission planning process and execution. Top: Areas of high entropy targeted for search are highlighted in red, representing regions with a binary occupied/unoccupied probability of 0.2. Three points of particular interest, each assigned a 0.5 probability, are marked in orange. Bottom: The executed drone flight path (yellow) shows the optimized path for maximum information gain across the search space.} 
%     \label{fig:google_earth}
% \end{figure}
% \begin{figure}[t]
% \centering
% % https://docs.google.com/presentation/d/1RjI-QqHpBRLHN60UAxzmQYs4EaWaVCOoSBkEkA39kk0/edit?usp=sharing
% \includegraphics[width=\columnwidth]{figs/5_/m600_labeled.jpg}
% \caption{Hexarotor system (DJI M600 Pro) with onboard compute and camera. Left image shows drone on the ground, right image shows drone in flight.}
% \label{fig:m600}
% \end{figure}

\subsubsection{Hardware System}
The hardware consists of the DJI M600 Pro, shown in Fig.~\ref{fig:m600}, along with the physical sensing and onboard computer payload. The DJI M600 Pro contains a flight controller that handles pose estimation and position-based control. The DJI M600 Pro’s flight controller also handles teleloperation if human intervention is necessary. Beneath the drone's base, we mount a custom hardware payload.
That payload consists of an onboard computer, a Jetson Xavier, to run the autonomy software shown in Fig.~\ref{fig:functional_diagram}.
The payload also contains a downward-facing a camera for sensing the environment. The camera is a Seek S304SP thermal camera.
The camera intrinsics are used to calculate the frustum's intersection with the search map's cells in IA-TIGRIS.

% without authors goes here
\begin{figure}[t]
\centering
% https://lucid.app/lucidchart/f750ddb4-2809-4773-8361-d5fbb1ba49eb/edit?viewport_loc=-257%2C-116%2C2219%2C1140%2C0_0&invitationId=inv_56e8a3a9-e8cf-4cad-a280-48bd967ff651
\includegraphics[trim={0cm 0cm 0cm 0cm},clip,width=\columnwidth]{figs/5_/functional_diagram.jpeg}
\caption{Functional diagram of the DJI M600 Pro autonomy software.}
\label{fig:functional_diagram}
\end{figure}
\begin{figure}[b]
    \centering
    \begin{subfigure}[b]{0.48\columnwidth}
        \centering
        \includegraphics[width=1.0\linewidth]{figs/5_/field_test_altitude_over_time.png}
        \caption{}
        \label{fig:m600_altitude_over_time}
    \end{subfigure}
    \begin{subfigure}[b]{0.48\columnwidth}
        \centering
        \includegraphics[width=1.0\linewidth]{figs/5_/field_test_entropy_over_time.png}
        \caption{}
        \label{fig:m600_entropy_over_time}
    \end{subfigure}
    \caption{The results for our hexarotor field deployment. (a) Plot of flown altitude over time, showing large variation throughout the experiment. (b) Reduction in entropy percentage over time of field experiment.}
\end{figure}

\subsubsection{Autonomy System}
Fig.~\ref{fig:functional_diagram} illustrates the functional system diagram for the real world field test on the DJI M600. The user specifies the initial plan request prior to takeoff. The TIGRIS planner makes an initial plan on that plan request and sends a global path to the waypoint manager. The waypoint manager tracks the current waypoint within the plan and sends the next waypoint to the DJI software development kit, which then sends actuation commands to the motors. The position of the drone is used to calculate the distance from the drone to the ground and sends that distance parameter to the sensor model. The sensor model's true positive and false positive rate is used to calculate the per-cell entropy updates in the search map manager. The search map manager publishes the current information map, and the replanning node sends an updated plan request to the IA-TIGRIS planner every ten seconds.

The drone started at an altitude of $50$ m above the origin of the reference frame. The informed sampler in IA-TIGRIS was set to add states at altitudes of either $30$ m or $60$ m, creating a trade-off between observation area and detector accuracy. The budget was $2000$ m, the planning horizon was $600$ m, and the planning time was $10$ seconds. 

% % without authors goes here
% \begin{figure}[t]
% \centering
% % https://lucid.app/lucidchart/f750ddb4-2809-4773-8361-d5fbb1ba49eb/edit?viewport_loc=-257%2C-116%2C2219%2C1140%2C0_0&invitationId=inv_56e8a3a9-e8cf-4cad-a280-48bd967ff651
% \includegraphics[trim={0cm 0cm 0cm 0cm},clip,width=\columnwidth]{figs/5_/functional_diagram.jpeg}
% \caption{Functional diagram of the DJI M600 Pro autonomy software.}
% \label{fig:functional_diagram}
% \end{figure}
% \begin{figure}[b]
%     \centering
%     \begin{subfigure}[b]{0.48\columnwidth}
%         \centering
%         \includegraphics[width=1.0\linewidth]{figs/5_/field_test_altitude_over_time.png}
%         \caption{}
%         \label{fig:m600_altitude_over_time}
%     \end{subfigure}
%     \begin{subfigure}[b]{0.48\columnwidth}
%         \centering
%         \includegraphics[width=1.0\linewidth]{figs/5_/field_test_entropy_over_time.png}
%         \caption{}
%         \label{fig:m600_entropy_over_time}
%     \end{subfigure}
%     \caption{The results for our hexarotor field deployment. (a) Plot of flown altitude over time, showing large variation throughout the experiment. (b) Reduction in entropy percentage over time of field experiment.}
% \end{figure}

\subsubsection{Experimental Results}


The bottom image of Fig.~\ref{fig:google_earth} shows the path selected by IA-TIGRIS in the search area. The figure highlights how the planner dynamically adjusts altitudes over time to balance coverage and sensing resolution, maximizing information gain. Higher altitudes allow for broader area coverage, while lower altitudes provide more detailed observations where needed. Additionally, the planner prioritizes revisiting the three regions of higher uncertainty, recognizing the need for repeated observations reduce entropy. This adaptive strategy ensures that uncertain areas receive sufficient attention to improve the belief map. As a result, the entropy of the information map decreases to near zero by the end of the mission, as shown in Fig.~\ref{fig:m600_entropy_over_time}, indicating that the planner has effectively gathered the necessary information. This behavior demonstrates the planner’s ability to optimize sensing actions, balancing altitude selection, revisit frequency, and exploration to maximize mission success.

\begin{figure}[t]
\centering
% \includegraphics[width=2.5in]{fig1}
\includegraphics[trim={4cm 4cm 0cm 4cm},clip,width=\columnwidth]{figs/5_/TL1.jpg}
\caption{Fixed-wing platform used for autonomous flights with an onboard camera pitched at 10 degrees\cite{alarewebsite}}
\label{fig:tl1}
\end{figure}






\subsection{Fixed-wing Deployments}

Our proposed approach was extensively tested on the fixed-wing AlareTech TL-1 UAV, shown in Fig.~\ref{fig:tl1}. The UAV is equipped with an onboard camera pitched at 10 degrees, which introduces a more challenging planning problem due to the non-holonomic motion model and the camera's field of view. Over more than 20 flight hours and 100 flights running IA-TIGRIS, we validated our approach with the objective to search for objects of interest in a large search space across a variety of test scenarios, including different terrain types, varying environmental conditions, and diverse target distributions. An example mission from these tests is shown in Fig.~\ref{fig:fwd}. In this scenario, the planner was given the search bounds and a designated high-priority region. The resulting flight path prioritized revisiting the high-priority area twice, optimizing sensor use and ensuring maximum information gain. This strategy led to the successful detection of the object of interest, with its estimated position marked by the red dot in the figure. 

The map on the upper right in Fig.~\ref{fig:fwd} shows the information map after plan execution was complete. Due to the UAV's limited budget, the upper right and lower left corners of the map are not searched by the agent. The budget is instead utilized to search over the area of higher priority two times. Compared to the paths in Fig.~\ref{fig:google_earth}, we observe that the paths for the fixed wing are smoother and have a larger turning radius, demonstrating how IA-TIGRIS respects the motion constraints of the vehicle. We can also see the effect of wind on the path execution, where the flown path shown in green deviates from the planned path shown in yellow. This illustrates the importance of online planning in the cases where this deviation is large or would accumulate over the course of a longer mission and cause the expected observed area to be much different than actual observed area. 

\begin{figure}[t]
\centering
% \includegraphics[width=2.5in]{fig1}
% [trim={left bottom right top},clip]
\includegraphics[trim={3.0cm, 1.0cm, 3.0cm, 1.0cm},clip,width=\columnwidth]{figs/5_/ONRFig_v3.pdf}
\caption{An example path generated for the fixed-wing platform conducting a large-area search for an object of interest. The larger black rectangle denotes the search bounds, while the smaller black rectangle highlights a region of higher uncertainty. The red dot marks the estimated position of the detected object based on image detections. The upper-right map displays the information state after planning is complete, while the middle plot shows the percent change in entropy over mission time. The flown path illustrates a balance between allocating resources to the high-priority region and exploring other areas within the search space.}
\label{fig:fwd}
\end{figure}

% Also tested extensively on the AlareTech TL-1 (citation?) tube launched UAV seen in Fig.~\ref{fig:tl1}.

% Talk about amount of flights, hours. Platform. Compute. Show visualization fo example flight. Talk about objects of interest in a broad sense (no mention of water/ocean/land for targets). Follow similar figure format as previous section. Main thing we want to highlight is the differences introduced in plans by having a fixed-wing platform compared to a drone. Include image of Alare TL-1 somewhere.

% One big figure showing all the info we want to convey. 

% \BM{Pitch 10 degrees, onboard computer type, etc}


% \subsection{VTOL?}
% what would it bring?


\section{Related Work}

\paragraph{LLMs for Agent tasks.}

Our research is related to deploying large language models (LLMs) as agents for decision-making tasks in interactive environments~\citep{liu2023agentbench,zhou2023webarena,shridhar2020alfred,toyama2021androidenv}. Earlier works, such as~\citep{yao2023webshopscalablerealworldweb}, fine-tuned models like BERT~\citep{devlin2019bertpretrainingdeepbidirectional} for decision-making in simplified environments, such as online shopping or mobile phone manipulation. With the advent of large language models~\citep{brown2020languagemodelsfewshotlearners,openai2024gpt4technicalreport}, it became feasible to perform decision-making tasks through zero-shot or few-shot in-context learning. To better assess the capabilities of LLMs as agents, several models have been developed~\citep{deng2024mind2web,xiong2024watch,hong2023cogagent,yan2023gpt}. Most approaches~\citep{zheng2024seeact,deng2024mind2web} provide the agent with observation and action history, and the language model predicts the next action via in-context learning. Additionally, some methods~\citep{zhang2023building,li2023camel,song2024trial} attempt to distill trajectories from state-of-the-art language models to train more effective policy models. In contrast, our paper introduces a novel framework that automatically learns a reward model from LLM agent navigation, using it to guide the agents in making more effective plans.

\textbf{LLM Planning.} Our paper is also related to planning with large language models. Early researchers~\citep{brown2020languagemodelsfewshotlearners} often prompted large language models to directly perform agent tasks. Later, \citet{yao2022react} proposed ReAct, which combined LLMs for action prediction with chain-of-thought prompting~\citep{wei2022chain}. Several other works~\citep{yao2023treethoughtsdeliberateproblem,hao2023reasoning,zhao2023large,qiao2024agentplanningworldknowledge} have focused on enhancing multi-step reasoning capabilities by integrating LLMs with tree search methods. Our model differs from these previous studies in several significant ways. First, rather than solely focusing on text generation tasks, our pipeline addresses multi-step action planning tasks in interactive environments, where we must consider not only historical input but also multimodal feedback from the environment. Additionally, our pipeline involves automatic learning of the reward model from the environment without relying on human-annotated data, whereas previous works rely on prompting-based frameworks that require large commercial LLMs like GPT-4~\citep{openai2024gpt4technicalreport} to learn action prediction. Furthermore, \Model supports a variety of planning algorithms beyond tree search.

\textbf{Learning from AI Feedback.} In contrast to prior work on LLM planning, our approach also draws on recent advances in learning from AI feedback~\citep{bai2022constitutional,lee2023rlaif,yuan2024self,sharma2024critical,pan2024autonomous,koh2024tree}. These studies initially prompt state-of-the-art large language models to generate text responses that adhere to predefined principles and then potentially fine-tune the LLMs with reinforcement learning. Like previous studies, we also prompt large language models to generate synthetic data. However, unlike them, we focus not on fine-tuning a better generative model but on developing a classification model that evaluates how well action trajectories fulfill the intended instructions. This approach is simpler, requires no reliance on state-of-the-art LLMs, and is more efficient. We also demonstrate that our learned reward model can integrate with various LLMs and planning algorithms, consistently improving their performance.

\textbf{Inference-Time Scaling.} ~\citet{snell2024scaling} validates the efficacy of inference-time scaling for language models. Based on inference-time scaling, various methods have been proposed, such as random sampling~\citep{wang2022self} and tree-search methods~\citep{hao2023reasoning, zhang2024accessing, guan2025rstar}. Concurrently, several works have also leveraged inference-time scaling to improve the performance of agentic tasks. ~\citet{koh2024tree} adopts a training-free approach, employing MCTS to enhance policy model performance during inference and prompting the LLM to return the reward. ~\citet{gu2024your} introduces a novel speculative reasoning approach to bypass irreversible actions by leveraging LLMs or VLMs. It also employs tree search to improve performance and prompts an LLM to output rewards. ~\citet{yu2024exact} proposes Reflective-MCTS to perform tree search and fine-tune the GPT model, leading to improvements in ~\citet{koh2024visualwebarena}. ~\citet{putta2024agent} also utilizes MCTS to enhance performance on web-based tasks such as ~\citet{yao2023webshopscalablerealworldweb} and real-world booking environments. ~\cite{lin2025qlass} utilizes the stepwise reward to give effective intermediate guidance across different agentic tasks. Our work differs from previous efforts in two key aspects: (1) Broader Application Domain. Unlike prior studies that primarily focus on tasks from a single domain, our method demonstrates strong generalizability across web agents, mathematical reasoning, and scientific discovery domains, further proving its effectiveness. (2) Flexible and Effective Reward Modeling. Instead of simply prompting an LLM as a reward model, we finetune a small scale VLM~\citep{lin2023vila} to evaluate input trajectories. %Our reward scores range continuously between 0 and 1, in contrast to existing methods that rely on discrete scoring (e.g., 0 and 1, or 0, 0.5, and 1) through direct LLM prompting.

% Concurrently, several works have also leveraged inference-time scaling to improve the performance of agentic tasks. ~\citet{pan2024autonomous} demonstrates that LLMs and VLMs, such as the GPT series, can function as evaluators or reward models to provide guidance for fine-tuning or reflection, thereby enhancing digital agents. This lays the groundwork for subsequent studies that directly prompt LLMs as reward models. ~\citet{koh2024tree} adopts a training-free approach, employing MCTS to enhance policy model performance during inference. However, it is limited to web environments~\citep{koh2024visualwebarena}. Moreover, its value function relies on prompting an LLM, which is less effective than our proposed method. We validate our approach through ablation studies, demonstrating that our fine-tuned reward model is more effective. ~\citet{gu2024your} introduces a novel speculative reasoning approach to bypass irreversible actions, such as purchasing a product, by leveraging LLMs or VLMs. It also employs tree search to improve performance, but it remains restricted to the web domain~\citep{koh2024visualwebarena, deng2024mind2web}. Additionally, it lacks reward modeling and instead prompts an LLM to output rewards. ~\citet{yu2024exact} proposes Reflective-MCTS to perform tree search and fine-tune the GPT model, leading to improvements in ~\citep{koh2024visualwebarena}. However, this work focuses solely on a single web agent task, and its reward modeling is derived from multi-agent debate, differing from our more effective and efficient reward modeling approach. ~\citet{putta2024agent} also utilizes MCTS to enhance performance, but it is limited to web-based tasks such as ~\citep{yao2023webshopscalablerealworldweb} and real-world booking environments.
\section{Concluding Remarks}
In this paper, we proposed a novel approach utilizing multimodal LLMs to generate gesture-aware speech recognition transcripts for patients with language disorders. Our framework integrates verbal speech and iconic gestures, enabling the generation of enriched transcripts that capture the latent meaning conveyed through both modalities. Through extensive experimentation, we demonstrated that the proposed method effectively contextualizes incomplete or disfluent speech by incorporating gesture information, leading to more accurate and meaningful representations of the speaker's intent. These findings highlight the potential of our approach to significantly contribute to the field of speech and language therapy, offering innovative tools that can enhance the quality of life for individuals with language disorders by facilitating better communication and assessment methods.

\subsection{Ethical Statement} 
Our dataset was obtained from AphasiaBank with the approval of the Institutional Review Board (IRB) and adheres to the data sharing guidelines set by TalkBank\footnote{https://talkbank.org/share/ethics.html}. This includes complying with the Ground Rules for all TalkBank databases, which are based on the American Psychological Association Code of Ethics~\cite{american2002ethical}.

\subsection{Limitation \& Future Work} 
%This study represents a preliminary investigation into using multimodal LLMs to generate gesture-aware speech recognition transcripts. 
While the results are promising, we recognize several limitations and outline our plans to extend this work further.

One primary limitation is the absence of a definitive ground truth for quantitative evaluation. Since our model generates transcripts by synthesizing speech and gesture data from scratch, traditional benchmarks, such as comparisons with standard speech recognition outputs, are insufficient. Moreover, existing original transcripts lack gesture annotations, making direct comparisons challenging. In future work, we aim to address this gap by collaborating with certified pathologists to conduct qualitative assessments, such as A-B preference tests, to evaluate the effectiveness of gesture-enriched transcripts in accurately conveying the speaker's intentions.

To support quantitative evaluations, we plan to develop novel metrics that assess transcript quality, including grammar accuracy, semantic consistency, and the integration of multimodal information. Such metrics will provide a more objective basis for assessing our model's performance and facilitate comparisons with other multimodal and unimodal approaches.

Another limitation of this study is its focus on structured gestures from a specific task, the Peanut Butter Sandwich Task. While this task offers a controlled context for testing our approach, it does not encompass the diversity of gestures and communication patterns seen in everyday scenarios. As part of our future work, we plan to expand the scope of our model to include tasks such as the Cinderella Story Recall Task~\cite{bird1996cinderella}, which involves unstructured and complex narrative gestures. This expansion will allow us to evaluate the adaptability and robustness of our model in handling varied linguistic and gestural contexts.

In summary, while this study establishes a strong foundation for gesture-aware speech recognition, we aim to refine and extend our methods through collaborative qualitative evaluations, the development of robust quantitative metrics, and broader task applications. These efforts will ensure that our approach continues to evolve, ultimately contributing to more effective communication tools and interventions for individuals with language disorders.





\newpage
\section*{Impact Statement}
\label{sec:impact}
\emph{CellFlow} introduces a novel framework for modeling cellular behavior under genetic and chemical perturbations, which uses flow-matching to generate high-fidelity cellular images and predict phenotypic trajectories. This tool addresses critical challenges in experimental biology by providing scalable and interpretable computational models for studying perturbations at both single-cell and population levels. \emph{CellFlow} has the potential to accelerate therapeutic discovery and drug repurposing by rapidly screening compounds through in-silico simulations of cell responses to perturbations. Follow-up biology experiments would be directed toward the most promising candidates based on computational experiments. By modeling the space of genetic and chemical perturbations, \emph{CellFlow} can facilitate the identification of novel therapeutic targets and compounds, streamlining biomedical research. In addition to medical applications, \emph{CellFlow} can accelerate basic research into cell biology processes by modeling responses to genetic or chemical perturbations. However, we acknowledge that these are early attempts to model complex and dynamic biological systems, and future research with larger and more diverse datasets will improve performance. Furthermore, we are limited by current datasets that focus on a few cancer cell lines, which could introduce bias and may not fully represent normal physiology. We are committed to ensuring robustness in our models and mitigating biases to the extent possible, given the constraints of the dataset and the available training data. In summary, \emph{CellFlow} bridges machine learning and cellular biology, enabling new frontiers in virtual cell modeling, drug discovery, and systems biology research with broad implications for science and medicine.
\bibliography{example_paper}
\bibliographystyle{icml2025}

\newpage
\appendix
\onecolumn
% \newpage
\clearpage
% Appendix
\appendix
\label{sec:appendix}

\subsection{Real World Setup}

\noindent\textbf{Deployment hardware}
% DOF of the robot, workspace, etc.
The humanoid robot on which we deploy \our is Unitree H1~\cite{H1-page}.
Following \citet{cheng2024tv}, we assembled two DYNAMIXEL XL330-M288-T motors~\citep{dynamixel-page} with 3D printed gimble parts and a ZED Mini stereo camera ~\cite{zed-page} for two-DoF (yaw and pitch) active sensing.
Each arm of H1 has 5 DoFs and a 6-DoF end-effector from~\cite{dexterous-page}, and other DoFs on the robot are not used.


\noindent\textbf{Motion capture system}
\label{app:mocap}
We use ArUco markers and two cameras to build a simple motion capture system.
We put four ArUco markers on the four corners of the workspace table to locate the cameras.
Each human in the workspace wears two 3D-printed wristbands with four ArUco markers on each of them, illustrated in \Cref{fig:mocap_sys}. 
The cameras are calibrated and located using the OpenCV library and capture the human wristband's position in real time.
Each wristband has an additional IMU sensor to capture the orientation of the human's wrist.
To reduce the noise in the collected data, we use a Kalman filter to smooth the data.

\begin{figure}[b]
    \centering
    % \includegraphics[width=0.8\linewidth]{example-image}
    \includegraphics[width=\linewidth]{figs/src/mocap.pdf}
    \caption{\small \textbf{Setup of motion capture system.} a) The two Motion Capture Cameras are used to detect the ArUco markers. The video recorded by the RGB-D Camera is used to process human motion and human hand details of the Leader. b) Follower and Leader both wear wristbands with ArUco markers for hand position detection. c) We 3D-print our wristbands with 4 ArUco markers on 4 surfaces and embed a IMU beneath the upper surface.}
    % \label{fig:intent_corner}
    \label{fig:mocap_sys}
\end{figure}

\begin{figure}
    \centering
    \includegraphics[width=0.85\linewidth]{figs/src/body_hand_obj.pdf}
    \caption{\small \textbf{Visualization results of human body motion detection, object detection, and human hand detection.} The upper image demonstrates the human motion detection (only upper body is used) and object detection (in Dining scenario). The lower image demonstrate the human hand detection.}
    \label{fig:zed_detect}
\end{figure}

\noindent\textbf{Motion Detection and Object Detection}
As is illustrated in \Cref{method:real-robot}, we use the Body Tracking feature in ZED API to detect the body motion of the human and a fine-tuned YoloV11~\cite{khanam2024yolov11} model to detect objects on the table. For hand detection, we use HaMeR\cite{pavlakos2024reconstructing} to obtain the human hand motion and then retarget\cite{qin2024anyteleopgeneralvisionbaseddexterous} it to the 6-DoF robot hand. The visualization results are shown in \Cref{fig:zed_detect}.


\subsection{Skill Descriptions}
\label{app:skills}
%  task description, succ. condition, reverse task(if exist), and instruction
In this section, we describe the skills that we deploy on the humanoid robot. The details include the description, success condition, reverse skill (if exists), and the human intention related to the skill.
Note that the intention is inferred mainly from human behavior, hand positions, and the relative location of objects.
The latter two can be concluded trivially to the start condition and end transition, and are shown by skill in \Cref{tab:skills}.
We concern the intention mainly on the human body motion in the narration as follows:


\noindent\textbf{1) Scenario 1: Humanoid as a Dining Waiter}

% Description of this scenario.
In this scenario, the leader human and the robot sit face-to-face at the side of a dining table. 
There are plates with food, a Coke can, a tissue box, and a sponge on the table.
In the following skill descriptions, the humanoid robot takes the role of a helpful waiter and serves the leader human a meal with these objects.
$10$ skills related to $4$ objects are listed below:

\begin{itemize}[leftmargin=*]
    \item \textit{Pick Can}: The robot picks up a can with its right arm from the table. 
    The success condition is that the can is lifted off the table.
    The interruption data includes the human taking the can away or the human putting his hand on the can. 
    \textit{Place Can} is the corresponding reverse skill.
    The leader human shows the intention by pointing to the can when the right robot arm is empty.
    
    \item \textit{Place Can}: The robot places the can back on the table with its right arm. 
    The success condition is that the can is placed on the table and the robot's hand is lifted off the can.
    The intention of this skill is shown by the leader human pointing to the place on the table where the can was placed before.
    
    \item \textit{Get Plate from Human}: The robot fetches a plate from the hand of the human with its left arm.
    The success condition is the plate in the dexterous hand when the human loosens the grip of the plate.
    The leader human shows the intention by handing a plate forward.
    
    \item \textit{Place Plate to Stack}: The robot stacks the plate in its right hand onto a pile of plates on the table.
    The success condition is the plate settled on the top of the plate pile without slipping.
    The intention is given by the leader human pointing to the stack.
    Interruption data, in which the human touches the plate to stick the motion, is added to the collected dataset.
    
    \item \textit{Pick Place from Table}: The robot lifts a plate on the table by both arms and holds it in the left hand.
    Application of this skill succeeds if there is no slippage in the motion until the plate is held.
    The leader human points to the plate to show the intention.
    
    \item \textit{Handover Plate}: The robot protracts its left arm to give the leader human the plate on it.
    The showcase of this skill ends when the plate is put into the human hand.
    The leader human simply stretches the right hand out to show the intention.
    It is the reverse skill of \textit{Get Plate from Human}.
    
    \item \textit{Pick Sponge}: The robot picks up the sponge with its right arm. 
    The sponge is placed beside the can.
    When it is lifted off the table the skill showcase succeeds.
    % \textit{Place Sponge} is the reverse skill.
    The leader human shows intention by mimicking washing.
    Data where the human snatches the sponge before the robot reaches it adds to the dataset. 
    In case this happens during deployment, the robot withdraws its hand to the idle state.

    \item \textit{Brush with Sponge}: It is a complex skill using both arms.
    The start condition is a plate in the left hand and a sponge in the right one when the leader human makes the washing gesture (same as that in \textit{Pick Sponge}) again.
    To apply this skill, the robot moves the sponge close to the plate and rubs the sponge on the plate to brush it.
    The success condition is the robot keeps the periodic brushing motion for over $10$ seconds.

    \item \textit{Place Sponge}: The reverse skill of \textit{Pick Sponge}.
    The robot puts the sponge in the right hand back onto the table to complete the skill demonstration.
    The intention is shown by the leader human pointing to the place on the table where the sponge was placed before (similar to skill \textit{Place Can}).

    \item \textit{Pick a Piece of Tissue}: The leader human points to the tissue box to express the intention. 
    Then the robot uses its left hand to pull a tissue from the tissue box placed on the table corner and gives it to the leader human.
    The skill showcase succeeds when the leader human receives the tissue.
\end{itemize}

\begin{table*}
    \begin{center}
        \caption{Description of the skills. Notes: The \textbf{Start Condition} or \textbf{End Transition} \texttt{[A, B]} means that object \texttt{A} is in the left hand of the robot and \texttt{B} is in the right hand. \texttt{empty} for this hand must be empty, \texttt{any} for this hand could hold any object or be empty, and \texttt{-} for this object remains unchanged after the skill is completed.}
        \label{tab:skills}
        % \begin{tabular}{c|c|c|c|c|c|c}
        \begin{tabular}{ccccccc}
        \toprule
        \textbf{Scenarios} & \textbf{Object} & \textbf{Skill Name} & \textbf{Start Condition} & \textbf{End Transition} & \textbf{Num. of Data} & \textbf{Arm}\\
            \midrule
            % can
            \multirow{11}{*}{\begin{tabular}{c}
                 Scenario 1 \\
                 Dining Waiter
            \end{tabular}}  & \multirow{2}{*}{can} & Pick Can & \texttt{[any, empty]} & \texttt{[-, can]} & $107$ & Right\\
            & & Place Can & \texttt{[any, can]} & \texttt{[-, empty]} & $100$ & Right\\
            % plate
            \cmidrule{2-7}
            & \multirow{4}{*}{plate} & Get Plate from Human  & \texttt{[empty, any]} & \texttt{[plate, -]} & $100$ & Left\\
            & & Place Plate to Stack & \texttt{[plate, any]} & \texttt{[empty, -]} &  $98$ & Left\\
            & & Pick Plate from Table & \texttt{[empty, empty]} & \texttt{[empty, plate]} & $115$ & Dual-Arm\\
            & & Handover Plate & \texttt{[plate, any]} & \texttt{[empty, -]} & $115$ & Left\\
            % sponge
            \cmidrule{2-7}
            & \multirow{3}{*}{sponge} & Pick Sponge  & \texttt{[any, empty]} & \texttt{[-, sponge]} & $89$ & Right\\
            & & Brush with Sponge & \texttt{[plate, sponge]} & \texttt{[-, -]} & $81$ & Dual-Arm\\
            & & Place Sponge & \texttt{[any, sponge]} & \texttt{[-, empty]} & $82$ & Right\\
            % tissue
            \cmidrule{2-7}
            & tissue & Pick a Piece of Tissue  & \texttt{[empty, any]} & \texttt{[-, -]} & $105$ & Left\\
            \midrule
            % cap
        \multirow{8}{*}{\begin{tabular}{c}
             Scenario 2  \\
             Office Assistant
        \end{tabular}} & \multirow{2}{*}{cap} & Settle Cap & \texttt{[any, empty]} & \texttt{[-, -]} & $111$ & Right\\
            & & Handover Cap & \texttt{[any, empty]} & \texttt{[-, -]} & $110$ & Right\\
            % book
            \cmidrule{2-7}
            & book & Pick Book & \texttt{[empty, any]} & \texttt{[-, -]} & $115$ & Left\\
            % stamp
            \cmidrule{2-7}
            & \multirow{3}{*}{stamp} & Pick Stamp & \texttt{[any, empty]} & \texttt{[-, stamp]} & $92$ & Right\\
            & & Stamp the Paper & \texttt{[any, stamp]} & \texttt{[-, -]} & $87$ & Right\\
            & & Place Stamp & \texttt{[any, stamp]} & \texttt{[-, empty]} & $89$ & Right\\
            % lamp
            \cmidrule{2-7}
            & lamp & Turn off/on the Lamp & \texttt{[empty, any]} & \texttt{[-, -]} & $85$ & Left\\
            \midrule 
            \multirow{6}{*}{\begin{tabular}{c}
                 Expressive Motions
            \end{tabular}
            } & \multirow{6}{*}{None} & Cheers & \texttt{[any, can]} & \texttt{[-, -]} & $66$ & Dual-Arm\\
            & & Wave & \texttt{[any, empty]} & \texttt{[-, -]} & $39$ & Dual-Arm\\
            & & Shake Hands & \texttt{[any, empty]} & \texttt{[-, -]} & $51$ & Dual-Arm\\
            & & Take Photo & \texttt{[any, empty]} & \texttt{[-, -]}  & $31$ & Dual-Arm\\
            & & Thumb Up & \texttt{[empty, empty]} & \texttt{[-, -]}  & $22$ & Dual-Arm\\
            & & Spread out Hands & \texttt{[empty, empty]} & \texttt{[-, -]} & $26$ & Dual-Arm\\
            \bottomrule
        \end{tabular}
    \end{center}
\end{table*}

\noindent\textbf{2) Scenario 2: Humanoid as an Office Assistant}

In this scenario, the leader human sits across the humanoid robot at an office table.
This time the robot transforms into an office assistant and deals with complicated cases such as stamping paper for approval, settling a baseball cap on the rack, picking and handing a book over, and reacting properly if the human takes a snap in working.
There are $7$ skills related to $4$ objects in this scenario.

\begin{itemize}[leftmargin=*]

\item \textit{Settle Cap}: The robot gets a cap from the leader human's hand and settles it on a hat rack with its right arm.
The skill showcase begins with the human holding the cap with both hands and ends with the robot pulling its hand back from the hat rack.

\item \textit{Handover Cap}: The robot takes the cap off the hat rack and sends it to the leader human.
It is the reverse skill of \textit{Settle Cap}.
The related intention is inferred when seeing the human pointing to the rack.
The success condition is that the human has received the cap.

\item \textit{Pick Book}: The robot picks a book from the shelf and hands it over.
The skill begins with the human gesturing toward the book.
When the human takes the book, this skill is completed successfully.

\item \textit{Pick Stamp}: The robot picks up the stamp on the table with its right hand. 
The skill succeeds when the stamp is lifted near the hand in an idle posture.
The leader human instructs the execution of this skill by passing along the paper.

\item \textit{Stamp the Paper}: It is a delicate operation to make an issue for approval.
The robot presses the stamp down onto the paper to mark a sign.
This skill is considered successful only if one mark is imprinted.
It is noteworthy that printing more than one mark in a single execution means that the model fails to predict the ending, thus being treated as a failure case.
The sign of the related intention is the leader human pointing at the paper.
To make an in-skill interruption, the human covers the paper with a hand to make the robot withdraw its hand if the pressing is not done.

\item \textit{Place Stamp}: The robot places the stamp back with its right hand.
It is the reverse skill of \textit{Pick Stamp} and is triggered by withdrawing the paper.

\item \textit{Turn off/on the Lamp}: Turning on and off the lamp share the same motion, and thus are trained as one skill.
When the human slumps over the office desk to take a nap, the robot taps the switch of the lamp to turn off it.
And when the human wakes up and lifts the head, the robot operates the same motion to turn on the lamp.
    
\end{itemize}

\noindent\textbf{3) Interactive Motion Skills}

Some skills are not involved with object operation and, thus, are not trained as manipulation skills. 
They are noted as motion skills.
These skills are considered successful when the robot performs the motion properly as the human shows the intention and recovers to the idle posture when the intention no longer sustains.

\begin{itemize}[leftmargin=*]
\item \textit{Cheers}: The robot reaches out the right hand to touch the bottle held by the right hand of the human.
Though holding a Coke can in the right hand during deployment, the robot does not manipulate the object.
For this reason, this skill is not trained in a manipulation demonstration model.

\item \textit{Wave}: The robot lifts up the right hand and waves the right hand when the leader human is waving also.

\item \textit{Shake Hands}: The robot stretches its right hand out to touch the hand of the leader human with a handshaking posture.

\item \textit{Take Photo}: The robot lifts up the right hand and makes a V-sign when the human raises the phone to take a photo, and puts the hand done as the human puts away the phone.

\item \textit{Thumb Up}: The robot reaches both hands out with the thumbs up as the human gives it a thumb-up.
Human intention with the left hand, right hand, or both is approved.

\item \textit{Spread out Hands}: The robot stretches its arms out to the sides with palms up when the leader human spreads its hands out.
    
\end{itemize}

\subsection{Prompt for VLMs}
Here are the prompts we give to Qwen and GPT-4o-mini in evaluation of the intention prediction module.

% For the \textbf{Dining} scenario, our prompt is:
\noindent\begin{tcolorbox}[
    colframe=darkgray, % Dark grey frame color
    boxrule=0.5pt, % Frame thickness
    colback=lightgray!20, %
    arc=3pt, % Rounded corners
    fontupper=\small,
    % width=.475\textwidth,
    breakable, title={Prompt for the \textbf{Dining} scenario},
    % height=9cm,
    ]
    
\textit{You are a humanoid robot sitting in front of a human and equipped with a camera slightly tilted downward on your head, providing a first-person perspective. I am assigning you a new task to recognize to human gestures in front of you. Remember, the person is sitting facing you, so be mindful of their gestures. If the person is holding a cup to you and trying to cheer with you, answer `Cheers'. If the person is giving you a thumbs-up, answer `Thumbup'. If the person extends their right hand to shake hands with you, answer `ShakingHand'.If the person is waving to you with the right hand, answer `Waving`. If the person is taking a photo of you with a cellphone, answer `Taking Photo`. If the person is spreading out both hands in a gesture of resignation, answer `Spreading Hands`. If the person is pointing to a Coke can in the middle of the table (on your right side), answer `Pointing Can'. If the person is pointing to an empty spot on the table with no objects (on your right side), answer `Pointing Table`. If the person is pointing to a tissue box at the far left of the table, answer `Pointing TissueBox'. If the person is pointing to a plate in the middle of the table (just in front of you), answer `Pointing Plate'. If the person is holding out the right hand with the palm open toward you, answer `Palmup'. If the person is handing you a plate, answer `Handing Plate'. If the person is clenching their right fist, holding their left hand open and upward, and placing their right hand above the left as if pretending to wash a plate, answer `Washing'. If the person is pointing at a stack of plates on the left side of the table, answer `Pointing Plates'. If the person is pointing at a sponge on the right side of the table, answer `Pointing Sponge'. If the person is crossing his arms to form an X shape, answer `Cancel'. If no significant gestures are made, answer `Idle'. 
Respond directly with the corresponding options [Cheers, Thumbup, ShakingHand, Pointing Can, Pointing TissueBox, Pointing Plate, Palmup, Handing Plate, Washing, Pointing Plates, Pointing Sponge, Cancel, Idle] based on the current image and observed gestures. Directly reply with the chosen answer only, without any additional characters.}
\end{tcolorbox}

% For the \textbf{Office} scenario, our prompt is:
\noindent\begin{tcolorbox}[
    colframe=darkgray, % Dark grey frame color
    boxrule=0.5pt, % Frame thickness
    colback=lightgray!20, %
    arc=3pt, % Rounded corners
    fontupper=\small,
    % width=.475\textwidth,
    breakable, title={Prompt for the \textbf{Office} scenario},
    % height=9cm,
    ]
\textit{You are a humanoid robot sitting in front of a human and equipped with a camera slightly tilted downward on your head, providing a first-person perspective. I am assigning you a new task to recognize human gestures in front of you. Remember, the person is sitting facing you, so be mindful of their gestures. If the person is giving you a thumbs-up, answer `Thumbup`. If the person extends their right hand to shake hands with you, answer `ShakingHand`. If the person is waving to you with the right hand, answer `Waving`. If the person is taking a photo of you with a cellphone, answer `Taking Photo`. If the person is spreading out both hands in a gesture of resignation, answer `Spreading Hands`. If the person is handing you a cap, answer `Handing Cap`. If the person is pointing at a cap place on the right of the table, answer `Pointing Cap`. If the person is handing a document to you with both hands and you are NOT holding a stamp, answer `Handing File`. If a document is placed in the center of the table in front of you, and the person is pointing to it with the right hand, answer `Pointing File`. If the person retrieves the document from your side of the table to the other side, directly across from you, and you are still holding the stamp, answer `Retrieve File`. If the person is lying down on the table and the lamp is ON, answer `Lie Down`. If the person is sitting up from the table and the lamp is OFF, answer `Sit up`. If the person is pointing at the books standing in the top left corner of the table, answer `Pointing Book`. If the person is crossing the arms to form an X shape, answer `Cancel`. If no significant gestures are made, answer `Idle`. You are NOT holding a stamp right now and the lamp is now ON, observe the image and gestures carefully. Respond directly with the corresponding options [Pointing Book, Handing Cap, Pointing Cap, Handing File, Pointing File, Retrieve File, Lie Down, Sit up, ShakingHand, Thumbup, Cancel, Idle]. Directly reply with the chosen answer ONLY, without any additional characters.}
\end{tcolorbox}

The sentence `\textit{You are NOT holding a stamp right
now and the lamp is now ON}' is modified at each query according to the current situation (whether the robot is holding a stamp and whether the lamp is on).

\subsection{Implementation of RHINO Modules}

\label{app:implementation}

\noindent\textbf{1) Reactive Planner}
% Initialize skillState ← IDLE  
% Initialize occupancyState ← EMPTY  
% Initialize lastIntention ← NONE  

% function reactivePlanner(currentInput):  
%     predictedIntent ← intentionModel.predict(currentInput)  
%     if predictedIntent is stable for K steps and predictedIntent ≠ lastIntention:  
%         if skillState is a manipulation skill and interruptionAllowed:  
%             skillState ← reverseOf(skillState)  
%             occupancyState ← updateOccupancyState()  

%         if startConditionFor(predictedIntent) is satisfied by occupancyState:  
%             skillState ← correspondingSkill(predictedIntent)  
%         else:  
%             path ← findPath(occupancyState, startConditionFor(predictedIntent))  
%             skillState ← executePath(path)  

%         lastIntention ← predictedIntent  

%     if currentSkillSucceeded(skillState) or currentSkillFailed(skillState):  
%         skillState ← IDLE  

%     return skillState  

% \begin{algorithm}[t]
%     \caption{Reactive Planner Pseudo-code}
%     \label{alg:reactive_planner}
%     \SetAlgoLined
%     \KwIn{
%       \\
%       \quad $M_t, H_t$: human body and hand posture at time $t$\\
%       \quad $B^i_t$: 3D bounding boxes of objects/hands at time $t$\\
%       \quad $P_t$: human head pose at time $t$\\
%       \quad $\text{IntentionPredictor}(\cdot)$: Transformer-based model for human intention\\
%       \quad $k$: number of consecutive frames required to confirm an intention\\
%       \quad $\mathcal{S}$: set of all possible skills\\
%       \quad $\text{startCondition}(s)$: the hand-occupancy requirement for skill $s$\\
%       \quad $\text{endTransition}(s)$: the change of hand occupancy after $s$ completes\\
%       \quad $\mathcal{G}$: directed graph of hand-occupancy transitions
%     }
%     \KwOut{Next robot action (skill execution)}
    
%     \textbf{Initialize:}\\
%     \quad $\text{currentSkill} \leftarrow \text{idle}$ \\
%     \quad $\text{currentOccupancy} \leftarrow [~]$ \ \tcp{Empty if no object in either hand}
%     \quad $\text{consecutiveCount}[i] \leftarrow 0, \forall i \in \text{AllIntentions}$ \\
%     \quad $\text{pendingSkill} \leftarrow \text{None}$ \\
    
%     \While{robot is running}{
%         \textbf{Step 1: Capture human behavior.}\\
%         \quad Retrieve sensor data $(M_t, H_t, B^i_t, P_t)$ from the RGB-D camera and detection models.\\
        
%         \textbf{Step 2: Predict human intention.}\\
%         \quad $I_t \leftarrow \text{IntentionPredictor}(M_t, H_t, B^i_t, P_t)$\\
%         \quad \ForEach{possible intention $i$}{
%             \uIf{$i = I_t$}{
%                 $\text{consecutiveCount}[i] \leftarrow \text{consecutiveCount}[i] + 1$\\
%             }
%             \Else{
%                 $\text{consecutiveCount}[i] \leftarrow 0$\\
%             }
%         }
        
%         \textbf{Step 3: Confirm intention for skill switching.}\\
%         \quad $I^* \leftarrow \arg\max_{i \in \text{AllIntentions}} \text{consecutiveCount}[i]$\\
%         \quad \If{$\text{consecutiveCount}[I^*] \geq k$}{
%             \textcolor{blue}{\tcp{We have a stable intention $I^*$}}
%             \quad $\text{pendingSkill} \leftarrow \text{mapIntentionToSkill}(I^*)$\;
%         }
        
%         \textbf{Step 4: Check if a skill switch is needed.}\\
%         \quad \uIf{$\text{pendingSkill} \neq \text{None}$}{
%             \quad \textcolor{blue}{\tcp{Interrupt current skill if needed}}
%             \quad \If{$\text{currentSkill} \neq \text{idle}$ \textbf{and} $\text{pendingSkill} \neq \text{currentSkill}$}{
%                 \quad \textcolor{blue}{\tcp{Handle interruption of manipulation skill if active}}
%                 \quad \text{reverseSkill}(\text{currentSkill})\;
%             }
%             \quad \textcolor{blue}{\tcp{Attempt to start the pending skill}}
%             \quad $s \leftarrow \text{pendingSkill}$\;
%             \quad \If(\tcp*[f]{Check start condition}){$\text{startCondition}(s)$ \textbf{is not satisfied by} $\text{currentOccupancy}$}{
%                 \quad \textcolor{blue}{\tcp{Find shortest path in occupancy graph $\mathcal{G}$ to fulfill start condition}}
%                 \quad $path \leftarrow \text{findOccupancyPath}(\mathcal{G}, \text{currentOccupancy}, \text{startCondition}(s))$\;
%                 \quad \ForEach{skill $p$ \textbf{in} $path$}{
%                     \quad Execute $p$ to modify $\text{currentOccupancy}$\;
%                 }
%             }
%             \quad \textcolor{blue}{\tcp{Now that occupancy requirements are met, start skill $s$}}
%             \quad $\text{currentSkill} \leftarrow s$\;
%             \quad $\text{pendingSkill} \leftarrow \text{None}$\;
%         }
        
%         \textbf{Step 5: Execute the current skill.}\\
%         \quad \uIf(\tcp*[f]{For manipulation skill}){$\text{currentSkill}$ \text{is manipulation}}{
%             \quad \If{\text{skill succeeds}}{
%                 \quad $\text{endTransition}(\text{currentSkill}) \rightarrow \text{currentOccupancy}$\;
%                 \quad $\text{currentSkill} \leftarrow \text{idle}$\;
%             }
%             \quad \ElseIf{\text{skill fails or times out}}{
%                 \quad \text{handleFailure}(\text{currentSkill})\;
%                 \quad $\text{currentSkill} \leftarrow \text{idle}$\;
%             }
%         }
%         \quad \ElseIf(\tcp*[f]{For motion skill}){$\text{currentSkill}$ \text{is motion}}{
%             \quad \If(\tcp*[f]{Stop if new intention or canceled}){a new stable intention emerges}{
%                 \quad $\text{currentSkill} \leftarrow \text{idle}$\;
%             }
%             \quad \Else{
%                 \quad Continue motion skill execution\;
%             }
%         }
%         \quad \Else{
%             \quad \textcolor{blue}{\tcp{Idle state: do nothing unless new skill arrives}}
%             \quad \text{currentSkill} \leftarrow \text{idle}\;
%         }
        
%         \textbf{Step 6: Repeat at 30Hz.}\\
%         \quad \textcolor{blue}{\tcp{Proceed to the next time step}}
%     }
%     \end{algorithm}
    
    % \begin{algorithm}[t]
    %     \caption{Reactive Planner Simplified Pseudo-code}
    %     \label{alg:reactive_planner_simplified}
    %     \SetAlgoLined
    %     \KwIn{
    %       \quad $M_t, H_t$: human body and hand posture at time $t$ \\
    %       \quad $B^i_t$: 3D bounding boxes of objects/hands at time $t$ \\
    %       \quad $P_t$: human head pose at time $t$ \\
    %       \quad $\mathcal{S}$: set of skills \\
    %       \quad $k$: number of frames for stable intention
    %     }
    %     \KwOut{Next robot action (skill execution)}
        
    %     \textbf{Initialize:} \\
    %     \quad $\text{currentSkill} \leftarrow \text{idle}$, $\text{currentOccupancy} \leftarrow [~]$ \\
    %     \quad $\text{pendingSkill} \leftarrow \text{None}$, $\text{consecutiveCount}[i] \leftarrow 0, \forall i$ \\
    %     \While{robot is running}{
    %         \textbf{Capture data:} \\
    %         \quad Retrieve $(M_t, H_t, B^i_t, P_t)$ from sensors.\\
            
    %         \textbf{Predict intention:} \\
    %         \quad $I_t \leftarrow \text{IntentionPredictor}(M_t, H_t, B^i_t, P_t)$. \\
    %         \quad \ForEach{intention $i$}{
    %             \quad $\text{consecutiveCount}[i] \leftarrow \text{consecutiveCount}[i] + 1$ if $i = I_t$ else 0.\\
    %         }
            
    %         \textbf{Confirm intention:} \\
    %         \quad $I^* \leftarrow \arg\max \text{consecutiveCount}[i]$ \\
    %         \quad \If{$\text{consecutiveCount}[I^*] \geq k$}{
    %             \quad $\text{pendingSkill} \leftarrow \text{mapIntentionToSkill}(I^*)$. \\
    %         }
            
    %         \textbf{Skill switch:} \\
    %         \quad \If{$\text{pendingSkill} \neq \text{None}$}{
    %             \quad \If{$\text{currentSkill} \neq \text{idle}$ \textbf{and} $\text{pendingSkill} \neq \text{currentSkill}$}{
    %                 \quad $\text{reverseSkill}(\text{currentSkill})$. \\
    %             }
    %             \quad \If{$\text{startCondition}(\text{pendingSkill})$ is met by $\text{currentOccupancy}$}{
    %                 \quad $\text{currentSkill} \leftarrow \text{pendingSkill}$. \\
    %                 \quad $\text{pendingSkill} \leftarrow \text{None}$. \\
    %             }
    %             \quad \Else{
    %                 \quad \text{Find and execute preparatory skills using the occupancy transition graph}. \\
    %             }
    %         }
            
    %         \textbf{Execute skill:} \\
    %         \quad \If{$\text{currentSkill}$ is manipulation}{
    %             \quad \If{skill succeeds}{ 
    %                 \quad $\text{endTransition}(\text{currentSkill})$. \\
    %                 \quad $\text{currentSkill} \leftarrow \text{idle}$. \\
    %             }
    %             \quad \ElseIf{skill fails or times out}{
    %                 \quad \text{handle failure}. \\
    %                 \quad $\text{currentSkill} \leftarrow \text{idle}$. \\
    %             }
    %         }
    %         \quad \ElseIf{$\text{currentSkill}$ is motion}{
    %             \quad \If{new intention emerges}{
    %                 \quad $\text{currentSkill} \leftarrow \text{idle}$. \\
    %             }
    %         }
    %         \quad \Else{
    %             \quad \text{Stay idle until next skill request}. \\
    %         }
    %     }
    %     \end{algorithm}

\begin{algorithm}
\caption{Pseudo-code for Skill Transitions of Reactive Planner.}
\begin{algorithmic}[1]
\label{alg:planner}
\STATE $Skill \gets \text{Idle}$
\WHILE{$true$}
    \STATE $human\_intention \gets \text{Recognize\_Human\_Intention()}$
    \IF{human intention is stable for $k$ frames and human intention != current intention}
        \IF{human intention $=$ Idle and Skill $=$ Manipulation}
            % continue the loop
            \STATE Continue
        \ENDIF
        \IF{Skill $=$ Manipulation and interruptionAllowed}
            \STATE $Skill \gets \text{Reverse\_Skill}(Skill)$
        \ENDIF
        \IF{$\text{Start\_Condition}(human\_intention)$ is not satisfied by hand occupancy}
             \STATE $path \gets$  FindPath(occupancy, StartCondition $(human\_intention))$
            % \STATE $\begin{aligned}
            %     path \gets & \text{FindPath}(\text{occupancy}, \\
            %     & \text{StartCondition}(human\_intention))
            % \end{aligned}$
            \STATE $Skill \gets \text{Execute\_Path}(path)$
        \ELSE
            \STATE $Skill \gets \text{Corresponding Skill} (human\_intention)$
        \ENDIF
    \ELSIF{SkillSucceeded(Skill) or SkillTimeout(Skill)}
        \STATE $Skill \gets \text{Idle}$
        \IF{SkillSucceeded(Skill)}
            \STATE $\text{Hand Occupancy} \gets \text{End\_Transition}(Skill)$
        \ENDIF
    \ENDIF
\ENDWHILE
\end{algorithmic}
\end{algorithm}


% \begin{figure}[b]
%     \centering
%     % \includegraphics[width=0.8\linewidth]{example-image}
%     \includegraphics[width=\linewidth]{figs/src/grp1.pdf}
%     \caption{\small \textbf{Setup of motion capture system.} a) The two Motion Capture Cameras are used to detect the ArUco markers. The video recorded by the RGB-D Camera is used to process human motion and human hand details of the Leader. b) Follower and Leader both wear wristbands with ArUco markers for hand position detection. c) We 3D-print our wristbands with 4 ArUco markers on 4 surfaces and embed a IMU beneath the upper surface.}
    
%     % \label{fig:intent_corner}
%     \label{fig:mocap_sys}
% \end{figure}
\begin{figure}[t!]
    \centering
    \begin{subfigure}[t]{\linewidth}
        \centering
        \includegraphics[width=\linewidth]{figs/src/grp1.pdf}
        \caption{Dining scenario.}
    \end{subfigure}%
    \\
    \begin{subfigure}[t]{\linewidth}
        \centering
        \includegraphics[width=0.8\linewidth]{figs/src/grp2.pdf}
        \caption{Office scenario.}
    \end{subfigure}
    \caption{Occupancy graph of two different scenarios.}\label{fig:occupancy}
\end{figure}

% \begin{table}
    \begin{center}
     \caption{Hyper-parameters of the Transformer backbone for Reactive Planner.}
        \begin{tabular}{cc}
            \toprule
            hyper-parameter & value \\
            \midrule
            latent dimension  & $128$ \\
            num head &$8$ \\
            num layers & $3$ \\
            batch size & $256$ \\
            feed-forward dimension & $128$ \\
            maximum epoch & $300$ \\
            learning rate & $0.0001$ \\
            \bottomrule
        \end{tabular}
    \end{center}
    \label{tab:planner-hyper}
\end{table}
The switching logic of the skill planner is listed in ~\Cref{alg:planner}.
The directed graphs of occupancy are shown in~\Cref{fig:occupancy}.
Here we further explain the \texttt{Recognize\_Human\_Intention()} function in detail, which is implemented as a transformer-based classifier. The model input includes:
\begin{itemize}[leftmargin=*]
    \item \textbf{Upper Body Human Posture}: a 36-dim human upper body skeleton, namely the 6D rotation of the wrist, elbow and shoulder joints for each arm.
    \item \textbf{Human Hand Pose}: a 12-dim human hand pose vector. For each hand, we retarget the detected human hand pose to our robot hand with IK, and take the 6 joint pos as human hand pose vector.
    \item \textbf{Robot Hand Occupancy}: a 10-dim robot hand occupancy label. Since we have at most 5 objects in total (Can, Cup, Plate, Sponge, Tissue), we use a 5-dim one-hot label for each hand to represent the object held in the robot's hand. If the robot is not holding anything, the label will be all-zeros.
    \item \textbf{Human Details}: a 19-dim vector, including the x and y-axis of each human hand position, the z-axis (height) of the human head position, and a 7-dim label for the nearest object to each hand. The nearest object label is concatenated by a 5-dim one-hot label of the object type, the distance from the object to the human hand, and the average of IOU and IOFs of the object bounding box and the human hand bounding box.
\end{itemize}
We use an MLP encoder to encode the concatenated vector of  \textbf{Upper Body Human Posture}, \textbf{Human Hand Pose} and \textbf{Robot Hand Occupancy}, and another MLP to encode \textbf{Human Details} to latent dimension. The concatenated latent vector is processed by a Transformer backbone, followed by a final MLP layer to predict the human intention class. The 
hyper-parameters of the Transformer backbone are listed in ~\Cref{tab:planner-hyper}.

% \begin{table}[t]
%     \begin{center}
%      \caption{Hyper-parameters of the ACT model for manipulation skills.}
%      \label{tab:act-hyper}
%         \begin{tabular}{cc}
%             \toprule
%             hyper-parameter & value \\
%             \midrule
%             KL weight  & 10 \\
%             Cross-entropy weight &1 \\
%             chunk size & 30 \\
%             hidden dimension & 512 \\
%             batch size & 45 \\
%             feed-forward dimension & 3200 \\
%             maximum epoch & 50000 \\
%             learning rate & 0.00005 \\
%             \bottomrule
%         \end{tabular}
%     \end{center}
% \end{table}

\begin{table*}[t]
    \begin{minipage}[t]{0.3\linewidth}
        \begin{center}
            \caption{Hyper-parameters of the Reactive Planner.}
            \label{tab:planner-hyper}
            \resizebox{0.9\linewidth}{!}{
            \begin{tabular}{cc}
                \toprule
                hyper-parameter&value\\
                \midrule
                latent dimension&128\\
                num head&8\\
                num layers&3\\
                batch size&256\\
                feed-forward dimension&128\\
                maximum epoch&300\\
                learning rate&0.0001\\
                \bottomrule
            \end{tabular}
            }
        \end{center}
    \end{minipage}
    \hfill
    \begin{minipage}[t]{0.3\linewidth}
        \begin{center}
            \caption{Hyper-parameters of the Motion Generation Model.}
            \label{tab:motion-hyper}
            \resizebox{0.9\linewidth}{!}{
            \begin{tabular}{cc}
                \toprule
                hyper-parameter&value\\
                \midrule
                latent dimension&256\\
                num head&8\\
                num layers&4\\
                feed-forward dimension&256\\
                diffusion steps&300\\
                sampling steps&30\\
                batch size&512\\
                maximum epoch&4000\\
                learning rate&0.0001\\
                \bottomrule
            \end{tabular}
            }
        \end{center}
    \end{minipage}
    \hfill
    \begin{minipage}[t]{0.3\linewidth}
        \begin{center}
            \caption{Hyper-parameters of the ACT model for manipulation skills.}
            \label{tab:act-hyper}
            \resizebox{0.9\linewidth}{!}{
            \begin{tabular}{cc}
                \toprule
                hyper-parameter&value\\
                \midrule
                KL weight&10\\
                Cross-entropy weight&1\\
                chunk size&30\\
                hidden dimension&512\\
                batch size&45\\
                feed-forward dimension&3200\\
                maximum epoch&50000\\
                learning rate&0.00005\\
                \bottomrule
            \end{tabular}
            }
        \end{center}
    \end{minipage}
\end{table*}



\noindent\textbf{2) Interactive Motion Generation}

% \begin{table}
    \begin{center}
     \caption{Hyper-parameters of the Interactive Motion Generation Model.}
        \begin{tabular}{cc}
            \toprule
            hyper-parameter & value \\
            \midrule
            latent dimension  & $256$ \\
            num head &$8$ \\
            num layers & $4$ \\
            feed-forward dimension & $256$ \\
            diffusion steps & $300$ \\
            sampling steps & $30$ \\
            batch size & $512$ \\
            maximum epoch & $4000$ \\
            learning rate & $0.0001$ \\
            \bottomrule
        \end{tabular}
    \end{center}
    \label{tab:motion-hyper}
\end{table}
For interactive motion generation, we use a transformer-based diffusion model, which denoises the past 30 frames of human and robot motions and future 5 frames of robot motions. Both human motion and robot motion consist of upper-body motion (36-dim for humans and 10-dim for humanoid), hand motion (6-dim for each hand,) and hand occupancy label (5-dim one-hot label for each hand). Besides, the predicted human intention label is also conditioned during the diffusion process. The hyper-parameters of our model are listed in \Cref{tab:motion-hyper}.


\noindent\textbf{3) Manipulation Skills}

Thanks to the stability of model training, most of the hyper-parameters are basically consistent across all skills.
The volume of data for training each skill is shown as a column in \Cref{tab:skills}.
The hyper-parameters in training ACT models~\cite{zhao2023learning} are shown as \Cref{tab:act-hyper}.
% We mainly adopted those from ~\cite{cheng2024tv} and adjusted some important hyper-parameters.

\begin{table*}[htb]
    \begin{center}
        \caption{Performance of manipulation module across manipulation skills. }
        \label{tab:manipulation_detailed}
        % \begin{tabular}{c|c|c|c|c|c|c}
        \begin{tabular}{ccccccc}
        \toprule
        \textbf{Scenarios} & \textbf{Object} & \textbf{Skill Name} & \textbf{Success Rate} & \textbf{Average Time} & \begin{tabular}{c}
             \textbf{Success Rate}  \\
             \textbf{(Human)}
        \end{tabular} & \begin{tabular}{c}
             \textbf{Average Time}  \\
             \textbf{(Human)}
        \end{tabular} \\
            \midrule
            % can
            \multirow{11}{*}{\begin{tabular}{c}
                 Scenario 1 \\
                 Dining Waiter
            \end{tabular}}  & \multirow{2}{*}{can} & Pick Can & 1.00 & 5.31 & 1.00 & 5.77\\
            & & Place Can & 1.00 & 4.10 & 0.93 & 4.65\\
            % plate
            \cmidrule{2-7}
            & \multirow{4}{*}{plate} & Get Plate from Human  & 1.00 & 4.86 & 0.98 & 5.12 \\
            & & Place Plate to Stack & 0.95 & 8.19 & 0.97 & 6.91 \\
            & & Pick Plate from Table & 0.90 & 10.75 & 0.96 & 8.60 \\
            & & Handover Plate & 1.00 & 5.79 & 1.00 & 5.14 \\
            % sponge
            \cmidrule{2-7}
            & \multirow{3}{*}{sponge} & Pick Sponge  & 0.95 & 8.19 & 1.00 & 7.45 \\
            & & Brush with Sponge & 0.90 & 10.02 & 1.00 & 4.18 \\
            & & Place Sponge & 0.85 & 5.57 & 0.98 & 5.41 \\
            % tissue
            \cmidrule{2-7}
            & tissue & Pick a Piece of Tissue  & 0.95 & 9.43 & 0.91 & 9.54\\
            \midrule
            % cap
            \multirow{8}{*}{\begin{tabular}{c}
             Scenario 2  \\
             Office Assistant
            \end{tabular}} & \multirow{2}{*}{cap} & Settle Cap & 1.00 & 7.50 & 0.91 & 8.50\\
            & & Handover Cap & 0.85 & 8.64 & 0.90 & 10.48 \\
            % book
            \cmidrule{2-7}
            & book & Pick Book & 0.95 & 10.81 & 0.93 & 10.21 \\
            % stamp
            \cmidrule{2-7}
            & \multirow{3}{*}{stamp} & Pick Stamp & 1.00 & 4.80 & 0.92 & 3.91 \\
            & & Stamp the Paper & 0.80 & 5.64 & 0.92 & 3.11 \\
            & & Place Stamp & 1.00 & 4.74 & 0.93 & 4.83 \\
            % lamp
            \cmidrule{2-7}
            & lamp & Turn off/on the Lamp & 1.00 & 5.06 & 0.96 & 3.53 \\
            % \midrule 
            % \multirow{6}{*}{\begin{tabular}{c}
            %      Expressive Motions
            % \end{tabular}
            % } & \multirow{6}{*}{None} & Cheers & & & $0$ & Dual-Arm\\
            % & & Wave & & & $0$ & Dual-Arm\\
            % & & Shake Hands & & & $0$ & Dual-Arm\\
            % & & Take Photo & & & $0$ & Dual-Arm\\
            % & & Thumb Up & & & $0$ & Dual-Arm\\
            % & & Spread out Hands & & & $0$ & Dual-Arm\\
            \bottomrule
        \end{tabular}
    \end{center}
\end{table*}

For the prediction of the success signal, we marked the last $n_{s}$ frames of the recorded data as $1$ (completed) and other frames as $0$ to generate a 0/1 label.
$n_{s}$ is set to $25$ in most of the skills and shifted to $10$ in three of them of which the ending frames changed sharply in motion.
The special skills are \textit{Pick Stamp}, \textit{Stamp the Paper}, and \textit{Place Stamp}.

% Jingxiao: Ignore some details
% \begin{table*}
    \begin{center}
        \caption{Hyper-parameters in real-world deployment across the manipulation tasks. Notes: \textbf{threshold} indicated the \textbf{progress threshold}, \textbf{initial} for \textbf{initial joint position}, \textbf{warm-up} and \textbf{time-out} means the two timings in termination prediction respectively.}
        \label{tab:skill_hyper}
        % \begin{tabular}{c|c|c|c|c|c|c}
        \begin{tabular}{ccccccc}
        \toprule
        \textbf{Scenarios} & \textbf{Object} & \textbf{Task Name} & \begin{tabular}{c}
             \textbf{threshold}  % \\
             % \textbf{threshold}
        \end{tabular} & \begin{tabular}{c}
             \textbf{initial}  % \\
             % \textbf{(Human)}
        \end{tabular} & \begin{tabular}{c}
             \textbf{warm-up}  % \\
             % \textbf{(Human)}
        \end{tabular} & \begin{tabular}{c}
             \textbf{time-out} % \\
             % \textbf{(Human)}
        \end{tabular} \\
            \midrule
            % can
            \multirow{11}{*}{\begin{tabular}{c}
                 Scenario 1 \\
                 Dining Waiter
            \end{tabular}}  & \multirow{2}{*}{can} & Pick Can & $0.85$ & \texttt{predicted} & $2$ & $15$\\
            & & Place Can & $0.85$ & \texttt{fixed} & $2$ & $15$\\
            % plate
            \cmidrule{2-7}
            & \multirow{4}{*}{plate} & Get Plate from Human  & $0.90$ & \texttt{fixed} & $2$ & $15$ \\
            & & Place Plate to Stack & $0.85$ & \texttt{fixed} & $2$ & $15$ \\
            & & Pick Plate from Table & $0.85$ & \texttt{fixed} & $2$ & $15$ \\
            & & Handover Plate & $0.85$ & \texttt{fixed} & $2$ & $15$ \\
            % sponge
            \cmidrule{2-7}
            & \multirow{3}{*}{sponge} & Pick Sponge  & $0.85$ & \texttt{fixed} & $2$ & $15$ \\
            & & Brush with Sponge & \texttt{unused} & \texttt{predicted} & $2$ & $10$ \\
            & & Place Sponge & $0.85$ & \texttt{fixed} & $2$ & $15$ \\
            % tissue
            \cmidrule{2-7}
            & tissue & Pick a Piece of Tissue  & $0.85$ & \texttt{fixed} & $2$ & $15$\\
            \midrule
            % cap
        \multirow{8}{*}{\begin{tabular}{c}
             Scenario 2  \\
             Office Assistant
        \end{tabular}} & \multirow{2}{*}{cap} & Settle Cap & $0.85$ & \texttt{fixed} & $2$ & $15$\\
            & & Handover Cap & $0.85$ & \texttt{predicted} & $2$ & $15$ \\
            % book
            \cmidrule{2-7}
            & book & Pick Book & $0.85$ & \texttt{predicted} & $2$ & $15$ \\
            % stamp
            \cmidrule{2-7}
            & \multirow{3}{*}{stamp} & Pick Stamp & $0.95$ & \texttt{fixed} & $2$ & $15$ \\
            & & Stamp the Paper & $0.85$ & \texttt{predicted} & $2$ & $10$ \\
            & & Place Stamp & $0.85$ & \texttt{predicted} & $2$ & $15$ \\
            % lamp
            \cmidrule{2-7}
            & lamp & Turn off/on the Lamp & $0.85$ & \texttt{fixed} & $2$ & $15$ \\
            % \midrule 
            % \multirow{6}{*}{\begin{tabular}{c}
            %      Expressive Motions
            % \end{tabular}
            % } & \multirow{6}{*}{None} & Cheers & & & $0$ & Dual-Arm\\
            % & & Wave & & & $0$ & Dual-Arm\\
            % & & Shake Hands & & & $0$ & Dual-Arm\\
            % & & Take Photo & & & $0$ & Dual-Arm\\
            % & & Thumb Up & & & $0$ & Dual-Arm\\
            % & & Spread out Hands & & & $0$ & Dual-Arm\\
            \bottomrule
        \end{tabular}
    \end{center}
\end{table*}

% Some simple but effective tricks are utilized to make skills perform better when deployed on the robot. 
% Firstly, we set a \textbf{progress threshold} to filter the noise of terminal condition prediction. 
% During deployment, the skill execution is terminated when the model predicts a success signal larger than the threshold.
% The threshold is typically $0.85$ and varies in certain skills.
% Moreover, to avoid verbose discussion that the robot should start from any state to finish the skill, we set the control target to an \textbf{initial joint position} before performing a skill.
% The initial position is typically selected as the first action of one slice in the collected dataset (noted as \texttt{fixed}) and is replaced by the model prediction of the first state (noted as \texttt{predicted}) in certain skills in consideration of robustness.
% However, there are some cases where the start and end state of the same skill are similar, leading to termination soon or dragging on in skill execution.
% We assign a pair of \textbf{warm-up time} and \textbf{time-out period} to avoid this phenomenon.
% A skill can not be predicted to be completed before the former and is regarded as terminated after the latter.
% They are typically $2$ and $15$ seconds respectively, and adjusted in only a few skills.
% In some skills with periodic features in the motion, the \textbf{time-out period} takes the place of \textbf{progress threshold}.
% In those skills, the robot executes a series of periodic actions until time out.

% The detailed selection of hyper-parameters across the manipulation skills is shown as \Cref{tab:skill_hyper}. 
% It shows that the manipulation module needs only a few simple adjustments to work on various skills.




\noindent\textbf{4) Safety Supervisor}


\begin{figure}[t]
    \centering
    \includegraphics[width=0.9\linewidth]{figs/src/safe.pdf}
    \caption{The interface displays the work of the safety supervisor, with sphere markers representing the collision boxes of the human hands and robot arms. These markers move in sync with the interaction. When an unsafe collision is detected, the human hand markers change color from green to red.}
    \label{fig:safe}
\end{figure}

The collision box is calculated using $14$ key points across each arm. The key points at specific joints and their midpoints are identified as follows:

\begin{itemize}[leftmargin=*]
    \item The origins of the shoulder pitch, shoulder yaw, elbow, and wrist joints are defined as key points.
    \item Additional key points include the midpoints between the shoulder yaw and elbow joints, and between the elbow and wrist joints.
    \item A further key point is defined at one-third the distance beyond the elbow towards the wrist, extending from the segment between these two joints.
\end{itemize}
This structured delineation allows for precise calculations pertinent to robotic arm movements within a predefined spatial configuration.

The human hands are shaped by the detected key points from body detection model of ZED API, from which each hand is reconstructed as $5$ points.
Once one of the points is close to any robot key point in $0.1$ meters, an unsafe signal is broadcast to pause the robot control.

We also provide the visualization of the safety supervisor, of which the interface shown in \Cref{fig:safe}. When the human hand key-points collide with any collision box, the supervisor will send an unsafe signal to halt the robot.
Our safety supervisor runs at $30$Hz.


\subsection{Detailed Experiment Results}

\label{app:result}

\noindent\textbf{1) Planner}
\begin{figure*}[t]
    \vspace{-3.0em}

    \centering
    \begin{minipage}{0.48\textwidth}
        \centering
        \includegraphics[height=8.5cm, keepaspectratio]{figs/src/confusion_matrix_ours_dining.pdf}
        \label{fig:image1}
    \end{minipage}
    \begin{minipage}{0.48\linewidth}
        \centering
        \includegraphics[height=8.5cm, keepaspectratio]{figs/src/confusion_matrix_ours_office.pdf}
        \label{fig:image2}
    \end{minipage}

    \vspace{-1.5em}

    \begin{minipage}{0.48\textwidth}
        \centering
        \includegraphics[height=8.5cm, keepaspectratio]{figs/src/confusion_matrix_wopos_dining.pdf}
        \label{fig:image3}
    \end{minipage}
    \begin{minipage}{0.48\linewidth}
        \centering
        \includegraphics[height=8.5cm, keepaspectratio]{figs/src/confusion_matrix_wopos_office.pdf}
        \label{fig:image4}
    \end{minipage}

    \vspace{-1.5em}

    \begin{minipage}{0.48\textwidth}
        \centering
        \includegraphics[height=8.5cm, keepaspectratio]{figs/src/confusion_matrix_dining_IDFalse_finetuneFalse_GPTTrue.pdf}
        \label{fig:image5}
    \end{minipage}
    \begin{minipage}{0.48\linewidth}
        \centering
        \includegraphics[height=8.5cm, keepaspectratio]{figs/src/confusion_matrix_office_IDFalse_finetuneFalse_GPTTrue.pdf}
        \label{fig:image6}
    \end{minipage}
    
    \caption{\textbf{Confusion Matrices of Our Model (with and without Human Details) and GPT-4o-mini.} To show the results more clearly, we did not color the cell in the top left corner since ”idle” accounts for a significant proportion in the data.}
    \label{fig:Confusion Matrix}
\end{figure*}


We use confusion matrices to show the classification performance of our planner on the test dataset. The confusion matrices for our model, our model without human details and GPT-4o-mini on the test datasets of the dining and office scenarios are shown in \Cref{fig:Confusion Matrix}. 
% To show the results more clearly, We did not color the cell in the top left corner (the cell enclosed by a dashed line) since "idle" accounts for a significant proportion in the data. 

As is shown in the confusion matrices, although the model mainly relies on human body motion and human hand motion input for classification, human details can help the model better deal with certain situations, such as avoiding mis-classification into Idle.

\noindent\textbf{2) Objects Manipulation}

The detailed success rates and average execution times across skills are presented in \Cref{tab:manipulation_detailed}, from which the statistics in \Cref{tab:manipulation} are derived.

In most skills, the manipulation module of \our autonomously executes motions following the patterns of teleoperation data within a comparable time frame.
Trained exclusively on successful human teleoperation cases, the module demonstrates both effectiveness and robustness to slight scene variations during deployment. 
As a result, it achieves higher success rates in skills involving simple motions with abundant training data, such as \textit{Pick Can}, \textit{Handover Plate}, and \textit{Place Stamp}.

However, certain skills pose challenges for the manipulation module. In \textit{Place Plate to Stack} and \textit{Stamp the Paper}, the robot hesitates to drop the plate or press the stamp due to prediction noise. 
In \textit{Pick Plate from Table}, it must overcome increased friction against the table when joint positions deviate from those in the collected data.
Another challenge arises in \textit{Brush with Sponge}, where the success signal predictor struggles to assess the progress of the periodic motion accurately. 
As a result, termination is constrained by a $10$-second timeout. 
These various factors contribute to a longer average execution time for these four skills compared to human performance.

Referring to the experiment on in-skill interruption data presented in \Cref{tab:safety_manipulation}, we select \textit{Pick Can}, \textit{Stamp the Paper}, and \textit{Place Plate to Stack} as representative skills for interruptions occurring during the stages of fetching an object, operating with the object, and returning the object, respectively.

To ensure a controlled data volume across different interruption ratios, we assign a fixed data amount of $M$ to each skill. 
In a full data collection for any given skill, the total data amount is $N$, with $N_{in}$ representing the portion containing in-skill interruptions. 
The ratio of interrupted data in a selected subset is denoted as $\alpha$, meaning that $\lceil\alpha M\rceil$ slices contain interruptions. 
To maintain this proportion, we set $M=N-N_{in}+1$. 
Specifically, $M$ is set to $69$, $66$, and $76$ for the three skills, respectively.

It is worth noting that the assigned data amount is smaller than that used in full data collection models (see \Cref{tab:skills}), which results in a decrease in skill success rate and interrupt success rate compared to the final model.

Each ACT model in this experiment uses the same hyper-parameters as those employed for the corresponding skill in both training and deployment with the full data collection.


\noindent\textbf{3) End2End Policy}


\begin{table}[htb]
    \begin{center}
        \caption{Detailed success rate of \our framework and end-to-end model on different skills.}
        \label{tab:framework_detailed}
        \begin{tabular}{cc|ccccc}
            \toprule
            \multicolumn{2}{c|}{\textbf{Method}} & \textbf{cheers} & \textbf{pick} & \textbf{place} & \textbf{handshake} & \textbf{wave} \\
            \midrule
\multicolumn{2}{c|}{Ours} & 1.00 & 1.00 & 0.85 & 1.00 & 1.00 \\
            \midrule
\multirow{2}{*}{E2E/1} & I.D. & 1.00 & -  & -  & -  & -  \\ & O.O.D.  & 0.95 & -  & -  & -  & -  \\
            \midrule
\multirow{2}{*}{E2E/3} & I.D. & 1.00 & 0.75 & 0.35 & -  & -  \\ & O.O.D.  & 0.95 & 0.30 & 0.45 & -  & -  \\
            \midrule
\multirow{2}{*}{E2E/5} & I.D. & 0.95 & 0.70 & 0.80 & 0.85 & 0.90 \\ & O.O.D.  & 1.00 & 0.55 & 0.35 & 0.60 & 0.85 \\
            \bottomrule
        \end{tabular}
    \end{center}
\end{table}

The detailed result to derive \Cref{tab:framework} is shown in \Cref{tab:framework_detailed}.
For each single skill, we collected $100$ slices of motion in the dataset, which is close to the average volume of all the manipulation skills.
As the small model is not capable of the skills which are totally unseen, we only examine each E2E model with the success rate of skills in the training data.




\end{document}
