\section{Introduction}
\label{sec:intro}

\section{Overview}

\revision{In this section, we first explain the foundational concept of Hausdorff distance-based penetration depth algorithms, which are essential for understanding our method (Sec.~\ref{sec:preliminary}).
We then provide a brief overview of our proposed RT-based penetration depth algorithm (Sec.~\ref{subsec:algo_overview}).}



\section{Preliminaries }
\label{sec:Preliminaries}

% Before we introduce our method, we first overview the important basics of 3D dynamic human modeling with Gaussian splatting. Then, we discuss the diffusion-based 3d generation techniques, and how they can be applied to human modeling.
% \ZY{I stopp here. TBC.}
% \subsection{Dynamic human modeling with Gaussian splatting}
\subsection{3D Gaussian Splatting}
3D Gaussian splatting~\cite{kerbl3Dgaussians} is an explicit scene representation that allows high-quality real-time rendering. The given scene is represented by a set of static 3D Gaussians, which are parameterized as follows: Gaussian center $x\in {\mathbb{R}^3}$, color $c\in {\mathbb{R}^3}$, opacity $\alpha\in {\mathbb{R}}$, spatial rotation in the form of quaternion $q\in {\mathbb{R}^4}$, and scaling factor $s\in {\mathbb{R}^3}$. Given these properties, the rendering process is represented as:
\begin{equation}
  I = Splatting(x, c, s, \alpha, q, r),
  \label{eq:splattingGA}
\end{equation}
where $I$ is the rendered image, $r$ is a set of query rays crossing the scene, and $Splatting(\cdot)$ is a differentiable rendering process. We refer readers to Kerbl et al.'s paper~\cite{kerbl3Dgaussians} for the details of Gaussian splatting. 



% \ZY{I would suggest move this part to the method part.}
% GaissianAvatar is a dynamic human generation model based on Gaussian splitting. Given a sequence of RGB images, this method utilizes fitted SMPLs and sampled points on its surface to obtain a pose-dependent feature map by a pose encoder. The pose-dependent features and a geometry feature are fed in a Gaussian decoder, which is employed to establish a functional mapping from the underlying geometry of the human form to diverse attributes of 3D Gaussians on the canonical surfaces. The parameter prediction process is articulated as follows:
% \begin{equation}
%   (\Delta x,c,s)=G_{\theta}(S+P),
%   \label{eq:gaussiandecoder}
% \end{equation}
%  where $G_{\theta}$ represents the Gaussian decoder, and $(S+P)$ is the multiplication of geometry feature S and pose feature P. Instead of optimizing all attributes of Gaussian, this decoder predicts 3D positional offset $\Delta{x} \in {\mathbb{R}^3}$, color $c\in\mathbb{R}^3$, and 3D scaling factor $ s\in\mathbb{R}^3$. To enhance geometry reconstruction accuracy, the opacity $\alpha$ and 3D rotation $q$ are set to fixed values of $1$ and $(1,0,0,0)$ respectively.
 
%  To render the canonical avatar in observation space, we seamlessly combine the Linear Blend Skinning function with the Gaussian Splatting~\cite{kerbl3Dgaussians} rendering process: 
% \begin{equation}
%   I_{\theta}=Splatting(x_o,Q,d),
%   \label{eq:splatting}
% \end{equation}
% \begin{equation}
%   x_o = T_{lbs}(x_c,p,w),
%   \label{eq:LBS}
% \end{equation}
% where $I_{\theta}$ represents the final rendered image, and the canonical Gaussian position $x_c$ is the sum of the initial position $x$ and the predicted offset $\Delta x$. The LBS function $T_{lbs}$ applies the SMPL skeleton pose $p$ and blending weights $w$ to deform $x_c$ into observation space as $x_o$. $Q$ denotes the remaining attributes of the Gaussians. With the rendering process, they can now reposition these canonical 3D Gaussians into the observation space.



\subsection{Score Distillation Sampling}
Score Distillation Sampling (SDS)~\cite{poole2022dreamfusion} builds a bridge between diffusion models and 3D representations. In SDS, the noised input is denoised in one time-step, and the difference between added noise and predicted noise is considered SDS loss, expressed as:

% \begin{equation}
%   \mathcal{L}_{SDS}(I_{\Phi}) \triangleq E_{t,\epsilon}[w(t)(\epsilon_{\phi}(z_t,y,t)-\epsilon)\frac{\partial I_{\Phi}}{\partial\Phi}],
%   \label{eq:SDSObserv}
% \end{equation}
\begin{equation}
    \mathcal{L}_{\text{SDS}}(I_{\Phi}) \triangleq \mathbb{E}_{t,\epsilon} \left[ w(t) \left( \epsilon_{\phi}(z_t, y, t) - \epsilon \right) \frac{\partial I_{\Phi}}{\partial \Phi} \right],
  \label{eq:SDSObservGA}
\end{equation}
where the input $I_{\Phi}$ represents a rendered image from a 3D representation, such as 3D Gaussians, with optimizable parameters $\Phi$. $\epsilon_{\phi}$ corresponds to the predicted noise of diffusion networks, which is produced by incorporating the noise image $z_t$ as input and conditioning it with a text or image $y$ at timestep $t$. The noise image $z_t$ is derived by introducing noise $\epsilon$ into $I_{\Phi}$ at timestep $t$. The loss is weighted by the diffusion scheduler $w(t)$. 
% \vspace{-3mm}

\subsection{Overview of the RTPD Algorithm}\label{subsec:algo_overview}
Fig.~\ref{fig:Overview} presents an overview of our RTPD algorithm.
It is grounded in the Hausdorff distance-based penetration depth calculation method (Sec.~\ref{sec:preliminary}).
%, similar to that of Tang et al.~\shortcite{SIG09HIST}.
The process consists of two primary phases: penetration surface extraction and Hausdorff distance calculation.
We leverage the RTX platform's capabilities to accelerate both of these steps.

\begin{figure*}[t]
    \centering
    \includegraphics[width=0.8\textwidth]{Image/overview.pdf}
    \caption{The overview of RT-based penetration depth calculation algorithm overview}
    \label{fig:Overview}
\end{figure*}

The penetration surface extraction phase focuses on identifying the overlapped region between two objects.
\revision{The penetration surface is defined as a set of polygons from one object, where at least one of its vertices lies within the other object. 
Note that in our work, we focus on triangles rather than general polygons, as they are processed most efficiently on the RTX platform.}
To facilitate this extraction, we introduce a ray-tracing-based \revision{Point-in-Polyhedron} test (RT-PIP), significantly accelerated through the use of RT cores (Sec.~\ref{sec:RT-PIP}).
This test capitalizes on the ray-surface intersection capabilities of the RTX platform.
%
Initially, a Geometry Acceleration Structure (GAS) is generated for each object, as required by the RTX platform.
The RT-PIP module takes the GAS of one object (e.g., $GAS_{A}$) and the point set of the other object (e.g., $P_{B}$).
It outputs a set of points (e.g., $P_{\partial B}$) representing the penetration region, indicating their location inside the opposing object.
Subsequently, a penetration surface (e.g., $\partial B$) is constructed using this point set (e.g., $P_{\partial B}$) (Sec.~\ref{subsec:surfaceGen}).
%
The generated penetration surfaces (e.g., $\partial A$ and $\partial B$) are then forwarded to the next step. 

The Hausdorff distance calculation phase utilizes the ray-surface intersection test of the RTX platform (Sec.~\ref{sec:RT-Hausdorff}) to compute the Hausdorff distance between two objects.
We introduce a novel Ray-Tracing-based Hausdorff DISTance algorithm, RT-HDIST.
It begins by generating GAS for the two penetration surfaces, $P_{\partial A}$ and $P_{\partial B}$, derived from the preceding step.
RT-HDIST processes the GAS of a penetration surface (e.g., $GAS_{\partial A}$) alongside the point set of the other penetration surface (e.g., $P_{\partial B}$) to compute the penetration depth between them.
The algorithm operates bidirectionally, considering both directions ($\partial A \to \partial B$ and $\partial B \to \partial A$).
The final penetration depth between the two objects, A and B, is determined by selecting the larger value from these two directional computations.

%In the Hausdorff distance calculation step, we compute the Hausdorff distance between given two objects using a ray-surface-intersection test. (Sec.~\ref{sec:RT-Hausdorff}) Initially, we construct the GAS for both $\partial A$ and $\partial B$ to utilize the RT-core effectively. The RT-based Hausdorff distance algorithms then determine the Hausdorff distance by processing the GAS of one object (e.g. $GAS_{\partial A}$) and set of the vertices of the other (e.g. $P_{\partial B}$). Following the Hausdorff distance definition (Eq.~\ref{equation:hausdorff_definition}), we compute the Hausdorff distance to both directions ($\partial A \to \partial B$) and ($\partial B \to \partial A$). As a result, the bigger one is the final Hausdorff distance, and also it is the penetration depth between input object $A$ and $B$.


%the proposed RT-based penetration depth calculation pipeline.
%Our proposed methods adopt Tang's Hausdorff-based penetration depth methods~\cite{SIG09HIST}. The pipeline is divided into the penetration surface extraction step and the Hausdorff distance calculation between the penetration surface steps. However, since Tang's approach is not suitable for the RT platform in detail, we modified and applied it with appropriate methods.

%The penetration surface extraction step is extracting overlapped surfaces on other objects. To utilize the RT core, we use the ray-intersection-based PIP(Point-In-Polygon) algorithms instead of collision detection between two objects which Tang et al.~\cite{SIG09HIST} used. (Sec.~\ref{sec:RT-PIP})
%RT core-based PIP test uses a ray-surface intersection test. For purpose this, we generate the GAS(Geometry Acceleration Structure) for each object. RT core-based PIP test takes the GAS of one object (e.g. $GAS_{A}$) and a set of vertex of another one (e.g. $P_{B}$). Then this computes the penetrated vertex set of another one (e.g. $P_{\partial B}$). To calculate the Hausdorff distance, these vertex sets change to objects constructed by penetrated surface (e.g. $\partial B$). Finally, the two generated overlapped surface objects $\partial A$ and $\partial B$ are used in the Hausdorff distance calculation step.

Building a virtual cell that simulates cellular behaviors in silico has been a longstanding dream in computational biology~\cite{slepchenko2003quantitative, johnson2023building, bunne2024build}. Such a system would revolutionize drug discovery by rapidly predicting how cells respond to new compounds or genetic modifications, significantly reducing the cost and time of biomedical research by prioritizing the experiments most likely to succeed based on the virtual cell simulation~\cite{carpenter2007image}. Moreover, this could unlock personalized therapeutic development by building digital twins of cells from patients to simulate patient-specific responses~\cite{katsoulakis2024digital}.

Two recent advances have made creating a generative virtual cell model possible. On the computational side, generative models now excel at modeling and sampling from complex data distributions, demonstrating remarkable success in synthesizing texts, images, videos, and biological sequences~\cite{openai2024gpt4technicalreport, esser2024scaling,pmlr-v235-kondratyuk24a,hayes2025simulating}. Concurrently, on the biotechnology side, automated high-content screening has generated massive imaging datasets --- reaching terabytes or petabytes --- that capture how cells respond to hundreds of thousands of chemical compounds and genetic modifications~\cite{chandrasekaran2023jump, fay2023rxrx3}.

In this work, we introduce \emph{CellFlow}, an image-generative model that simulates how cellular morphology changes in response to chemical or genetic perturbations (Figure~\ref{fig:overview}a). \emph{CellFlow}’s key innovation is formulating cellular morphology prediction as a distribution-to-distribution learning problem, and leveraging flow matching~\cite{lipman2022flow}, a state-of-the-art generative modeling technique designed for distribution-wise transformation, to solve this problem.

Specifically, cell morphology data are collected through high-content microscopy screening, where images of control and perturbed cells are captured from experimental wells across different batches (Figure~\ref{fig:overview}b). Control wells, which receive no drug treatment or genetic modifications, play a crucial role in providing prior information and serving as a reference to distinguish true perturbation effects from other sources of variation. They help calibrate non-perturbation factors, such as batch effects—systematic biases unrelated to perturbations, including variations in color or intensity, akin to distribution shifts in machine learning. Properly incorporating control wells is essential for capturing actual perturbation effects rather than artifacts, yet many existing methods overlook this aspect~\cite{yang2021mol2image,navidi2024morphodiff,cook2024diffusion}. To address this, we frame cellular morphology prediction as a distribution-to-distribution mapping problem (Figure~\ref{fig:overview}c), where the source distribution consists of control cell images, and the target distribution comprises perturbed cell images from the same batch.

To address this distribution-to-distribution problem, \emph{CellFlow} employs flow matching, a state-of-the-art generative modeling approach designed for distribution-wise transformations (Figure~\ref{fig:overview}d). The framework continuously transforms the source distribution into the target using an ordinary differential equation (ODE) by learning a neural network to approximate a velocity field. This direct and native distribution transformation enabled by flow matching is intuitively more effective than previous methods, which rely on adding extra components to GANs, incorporating the source as a condition, or mapping between distributions and noise using diffusion models~\cite{palma2023predicting,hung2024lumic,bourou2024phendiff}.

We demonstrate the effectiveness of \emph{CellFlow} on three datasets: BBBC021 (chemical perturbations)~\cite{caie2010high}, RxRx1 (genetic modifications via CRISPR or ORF)~\cite{sypetkowski2023rxrx1}, and JUMP (combined chemical and genetic perturbations)~\cite{chandrasekaran2023jump}. \emph{CellFlow} generates high-fidelity images of cellular changes in response to perturbations across all datasets, improving FID scores by 35\% over previous approaches. The generated images capture meaningful biological patterns, demonstrated by a 12\% improvement in predicting mode-of-action compared to existing methods (Figure~\ref{fig:overview}e). Importantly, \emph{CellFlow} maintains consistent performance across diverse experimental conditions and generalizes to held-out perturbations never seen during training, showing its broad applicability. 

Moreover, \emph{CellFlow} introduces two key capabilities with significant potential for biological research (Figure~\ref{fig:interpolation}). First, it effectively corrects batch effects by conditioning on control cells from different batches. By comparing control images with generated images, it can disentangle true perturbation-induced morphological changes from experimental batch artifacts. Second, \emph{CellFlow} enables bidirectional interpolation between cellular states due to the continuous and reversible nature of the velocity field in flow matching. This interpolation provides a means to explore intermediate cellular morphologies and potentially gain deeper insights into dynamic perturbation responses.

In summary, by formulating cellular morphology prediction as a distribution-to-distribution problem and using flow matching as a solution, \emph{CellFlow} enables accurate prediction of perturbation responses (Figure~\ref{fig:overview}). \emph{CellFlow} not only achieves state-of-the-art performance but unlocks new capabilities such as handling batch effects or visualizing cellular state transitions, significantly advancing the field towards a virtual cell for drug discovery and personalized therapy.
