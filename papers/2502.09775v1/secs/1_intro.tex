\section{Introduction}
\label{sec:intro}

\begin{figure*}[t]
\begin{center}
\includegraphics[width=.85\linewidth]{fig_overview_v3.pdf}
\end{center}
\caption{
FastAtlas Overview: In each frame, we compute charts spanning fully or partially visible triangles (a), determine texture space bounding boxes for the visible portions of the view-space projections of each chart, and tightly pack these boxes into atlases (b, here $2K \times 2K$). We simultaneously bijectively parameterize and shade the charts into their atlas boxes, obtaining high quality texture space shading (c), and use this shading to render the shaded frames (d).}
\label{fig:overview}
\label{fig:alg_overview}
\end{figure*}

\section{Overview}
\label{sec:overview}
Our work has two core contributions: a real-time, GPU-based algorithm for tight packing of general parameterized charts into compact atlases; and a real-time TSS method that
utilizes this packing.  

\paragraph*{FastAtlas Packing.}
FastAtlas runs entirely on the GPU as a series of compute shaders. It takes the bounding boxes of parameterized charts as input, and packs them into an atlas (Fig~\ref{fig:overview}b, Sec.~\ref{sec:pack}). As such, the only input it requires are the dimensions of the bounding boxes.
Its outputs are deterministic; identical input charts are packed into identical atlases. This is critical for TSS and similar applications, as it ensures that consecutive frames taken from the same camera view have the same shading. Even minute shading differences across such frames can cause sampling jitter, leading to undesirable flicker \cite{baker2012rock}. 
While prior methods such as \cite{mueller2018shading,hladky2019tessellated,hladky2021snakebinning,Neff2022MSA} cap the dimensions of the charts that can be packed as-is for a given atlas size, and scale down all charts that exceed these dimensions, we scale all charts by the same factor, and do so only when strictly necessary to achieve packing success (Figs~\ref{fig:atlas},~\ref{fig:sas_issues}). 

\paragraph*{TSS using FastAtlas.}
Our end-to-end TSS atlas generation method combines the packing method above with a novel approach for computing seamless per-frame charts. 
We define our charts as the connected components of the visible surfaces in each frame (Fig.~\ref{fig:overview}a), and efficiently compute them using a parallel union-find algorithm (Sec.~\ref{sec:visible}). Since the boundaries of these charts coincide with the contours of the rendered surface, they are {\em invisible} to the viewer. This approach 
eliminates the artifacts caused by shading discontinuities along visible seams (Fig.~\ref{fig:seams}). 

\begin{parWithWrapFigure}
\begin{wrapfigure}{l}{.27\columnwidth}%
\includegraphics[width=\linewidth]{fig_inset_view_plane.pdf}%
\end{wrapfigure}
We bijectively parametrize the {\em visible portions} of our charts by projecting them to view space (inset). This maps a constant number of texels to each pixel in the final rendered output, evenly distributing residual undersampling error across all image pixels. While conceptually straightforward, efficiently parameterizing charts containing partially visible triangles using viewspace projection is non-trivial, as the visible portions may no longer be triangular (e.g. green triangle in the inset); applying naive projection to triangles with vertices behind the camera may produce ill-posed results. Clipping triangles before projection is both computationally expensive and significantly complicates downstream operations. We avoid explicit clipping by observing that all that is required for atlas packing is the dimensions of, potentially conservative, bounding boxes of these projected visible portions. We compute such bounding boxes without explicit chart clipping by adapting a conservative screen coverage estimator \shortcite{Blinn:CalculatingScreenCoverage} (Sec.~\ref{sec:box}). We then pack the computed boxes using FastAtlas. 
\end{parWithWrapFigure}

Finally, we shade the visible portion of each chart into its corresponding atlas bounding box (Fig~\ref{fig:overview}c). 
The resulting texture is then used during rasterization as a standard texture map (Fig. ~\ref{fig:overview}d). 
Our framework is compatible with all existing approaches for texture space shading, including forward shading, raytraced illumination, or deferred shading in texture space \cite{baker:2016}. In the examples shown, we use the standard forward shading based rendering pipeline included in the G3D Innovation Engine \cite{G3D17}, a commercial grade renderer.


Building a virtual cell that simulates cellular behaviors in silico has been a longstanding dream in computational biology~\cite{slepchenko2003quantitative, johnson2023building, bunne2024build}. Such a system would revolutionize drug discovery by rapidly predicting how cells respond to new compounds or genetic modifications, significantly reducing the cost and time of biomedical research by prioritizing the experiments most likely to succeed based on the virtual cell simulation~\cite{carpenter2007image}. Moreover, this could unlock personalized therapeutic development by building digital twins of cells from patients to simulate patient-specific responses~\cite{katsoulakis2024digital}.

Two recent advances have made creating a generative virtual cell model possible. On the computational side, generative models now excel at modeling and sampling from complex data distributions, demonstrating remarkable success in synthesizing texts, images, videos, and biological sequences~\cite{openai2024gpt4technicalreport, esser2024scaling,pmlr-v235-kondratyuk24a,hayes2025simulating}. Concurrently, on the biotechnology side, automated high-content screening has generated massive imaging datasets --- reaching terabytes or petabytes --- that capture how cells respond to hundreds of thousands of chemical compounds and genetic modifications~\cite{chandrasekaran2023jump, fay2023rxrx3}.

In this work, we introduce \emph{CellFlow}, an image-generative model that simulates how cellular morphology changes in response to chemical or genetic perturbations (Figure~\ref{fig:overview}a). \emph{CellFlow}’s key innovation is formulating cellular morphology prediction as a distribution-to-distribution learning problem, and leveraging flow matching~\cite{lipman2022flow}, a state-of-the-art generative modeling technique designed for distribution-wise transformation, to solve this problem.

Specifically, cell morphology data are collected through high-content microscopy screening, where images of control and perturbed cells are captured from experimental wells across different batches (Figure~\ref{fig:overview}b). Control wells, which receive no drug treatment or genetic modifications, play a crucial role in providing prior information and serving as a reference to distinguish true perturbation effects from other sources of variation. They help calibrate non-perturbation factors, such as batch effects—systematic biases unrelated to perturbations, including variations in color or intensity, akin to distribution shifts in machine learning. Properly incorporating control wells is essential for capturing actual perturbation effects rather than artifacts, yet many existing methods overlook this aspect~\cite{yang2021mol2image,navidi2024morphodiff,cook2024diffusion}. To address this, we frame cellular morphology prediction as a distribution-to-distribution mapping problem (Figure~\ref{fig:overview}c), where the source distribution consists of control cell images, and the target distribution comprises perturbed cell images from the same batch.

To address this distribution-to-distribution problem, \emph{CellFlow} employs flow matching, a state-of-the-art generative modeling approach designed for distribution-wise transformations (Figure~\ref{fig:overview}d). The framework continuously transforms the source distribution into the target using an ordinary differential equation (ODE) by learning a neural network to approximate a velocity field. This direct and native distribution transformation enabled by flow matching is intuitively more effective than previous methods, which rely on adding extra components to GANs, incorporating the source as a condition, or mapping between distributions and noise using diffusion models~\cite{palma2023predicting,hung2024lumic,bourou2024phendiff}.

We demonstrate the effectiveness of \emph{CellFlow} on three datasets: BBBC021 (chemical perturbations)~\cite{caie2010high}, RxRx1 (genetic modifications via CRISPR or ORF)~\cite{sypetkowski2023rxrx1}, and JUMP (combined chemical and genetic perturbations)~\cite{chandrasekaran2023jump}. \emph{CellFlow} generates high-fidelity images of cellular changes in response to perturbations across all datasets, improving FID scores by 35\% over previous approaches. The generated images capture meaningful biological patterns, demonstrated by a 12\% improvement in predicting mode-of-action compared to existing methods (Figure~\ref{fig:overview}e). Importantly, \emph{CellFlow} maintains consistent performance across diverse experimental conditions and generalizes to held-out perturbations never seen during training, showing its broad applicability. 

Moreover, \emph{CellFlow} introduces two key capabilities with significant potential for biological research (Figure~\ref{fig:interpolation}). First, it effectively corrects batch effects by conditioning on control cells from different batches. By comparing control images with generated images, it can disentangle true perturbation-induced morphological changes from experimental batch artifacts. Second, \emph{CellFlow} enables bidirectional interpolation between cellular states due to the continuous and reversible nature of the velocity field in flow matching. This interpolation provides a means to explore intermediate cellular morphologies and potentially gain deeper insights into dynamic perturbation responses.

In summary, by formulating cellular morphology prediction as a distribution-to-distribution problem and using flow matching as a solution, \emph{CellFlow} enables accurate prediction of perturbation responses (Figure~\ref{fig:overview}). \emph{CellFlow} not only achieves state-of-the-art performance but unlocks new capabilities such as handling batch effects or visualizing cellular state transitions, significantly advancing the field towards a virtual cell for drug discovery and personalized therapy.
