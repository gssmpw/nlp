\section*{Summary of Appendix}

\begin{itemize}
\item \S\ref{sec:theory} presents a formal proof supporting our mathematical formulation.
\item \S\ref{sec:algorithm} details the \emph{CellFlow} algorithm.
\item \S\ref{sec:experimental} provides additional experimental details.
\item \S\ref{sec:batch_effect} offers a more in-depth discussion of batch effects.
\item \S\ref{sec:data} describes the datasets used in our study.
\item \S\ref{sec:comparison} includes qualitative comparisons of \emph{CellFlow} against baselines.
\item \S\ref{sec:trajectory} presents additional visualization of bidirectional trajectories between control images and perturbed images.
\item \S\ref{sec:results_appendix} provides additional tables comparing our method to baselines in out-of-distribution scenarios.
\item \S\ref{sec:related_appendix} provides a table to compare our work with related works.
\end{itemize}


\newpage
\section{Theory Proof}
\label{sec:theory}

\begin{center}
\begin{tikzpicture}[->, >=stealth, node distance=2cm, font=\sffamily]
    \node[draw, circle, fill=gray!30] (b) {$B$};
    \node[draw, circle, fill=gray!30, right of=b] (c) {$C$};
    \node[draw, circle, fill=gray!30, below left=of b] (x0tilde) {$\tilde{X_0}$};
    \node[draw, circle, below of=b] (x0) {$X_0$};
    \node[draw, circle, fill=gray!30, below of=c] (x1) {$X_1$};
    \draw (b) -- (x0);
    \draw (c) -- (x1);
    \draw (x0) -- (x1);
    \draw (b) -- (x0tilde);
\end{tikzpicture}
\end{center}

\setcounter{prop}{0}
\begin{prop}
    \label{prop:cond2}
    Given random variables $B$, $C$, $X_0$, $\tilde{X_0}$, and $X_1$ following the graphical model above with joint distribution $p(b, c, x_0, \tilde{x_0}, x_1)$, the distribution $p(x_1|x_0,c)$ can be better approximated by the conditional distribution $p(x_1|\tilde{x_0},c)$ than $p(x_1|c)$ in expectation. Formally,
    $$\mathbb{E}_{p(x_0, \tilde{x_0}, c)}\left[ D_{\text{KL}}(p(x_1|x_0,c)||p(x_1|\tilde{x_0},c)) \right] \leq \mathbb{E}_{p(x_0, c)}\left[ D_{\text{KL}}(p(x_1|x_0,c)||p(x_1|c)) \right]$$
\end{prop}

\begin{proof}
    According to the definition of conditional mutual information, the term on the right-hand side can be expressed as:
    \begin{equation}
    \begin{split}
        \mathbb{E}_{p(x_0, c)}\left[ D_{\text{KL}}(p(x_1|x_0,c)||p(x_1|c)) \right] &= \int p(x_0, c) p(x_1|x_0,c) \log \frac{p(x_1|x_0,c)}{p(x_1|c)} dx_1dx_0dc \nonumber\\
        &= \int p(c) \mathbb{E}_{p(x_0,x_1|c)} \left[ \log \frac{p(x_1,x_0|c)}{p(x_1|c) p(x_0|c)}\right] dc \nonumber\\
        &= \mathbb{E}_{p(c)} \left[ D_{\text{KL}}(p(x_1,x_0|c)||p(x_1|c)p(x_0|c)) \right] = I(X_1;X_0|C)
    \end{split}
    \end{equation}

    Based on the graphical model, we have the conditional independence $X_1 \indep \tilde{X_0}|X_0,C$. Thus, we have $p(x_1|x_0,c)=p(x_1|x_0, \tilde{x_0},c)$. Similarly, we can express the term on the right-hand side as conditional mutual information:
    \begin{equation}
        \begin{split}
            \mathbb{E}_{p(x_0, \tilde{x_0}, c)}\left[ D_{\text{KL}}(p(x_1|x_0,c)||p(x_1|\tilde{x_0},c)) \right] &= \mathbb{E}_{p(x_0, \tilde{x_0}, c)}\left[ D_{\text{KL}}(p(x_1|x_0, \tilde{x_0},c)||p(x_1|\tilde{x_0},c)) \right]\nonumber\\
            &= \int p(x_0, \tilde{x_0}, c) p(x_1|x_0,\tilde{x_0},c) \log \frac{p(x_1|x_0, \tilde{x_0},c)}{p(x_1|\tilde{x_0},c)} dx_1d\tilde{x_0}dx_0dc \nonumber\\
            &= I(X_1;X_0|\tilde{X_0},C)
        \end{split}
    \end{equation}

Further, based on the property of conditional mutual information, we have
\begin{equation}
    \begin{split}
    I(X_1;X_0|C)&=I(X_1;X_0|\tilde{X_0},C) + I(X_1;\tilde{X_0}|C) - I(X_1;\tilde{X_0}|X_0,C)\nonumber\\
    &=I(X_1;X_0|\tilde{X_0},C) + I(X_1;\tilde{X_0}|C)
    \end{split}
\end{equation}
where the second equality is due to the conditional independence relationship $X_1 \indep \tilde{X_0}|X_0,C$, and $I(X_1;\tilde{X_0}|X_0,C)=0$

Therefore, 
\begin{equation}
    \begin{split}
    \mathbb{E}_{p(x_0, c)}\left[ D_{\text{KL}}(p(x_1|x_0,c)||p(x_1|c)) \right] &= \mathbb{E}_{p(x_0, \tilde{x_0}, c)}\left[ D_{\text{KL}}(p(x_1|x_0,c)||p(x_1|\tilde{x_0},c)) \right] + I(X_1;\tilde{X_0}|C) \nonumber\\
    &\geq \mathbb{E}_{p(x_0, \tilde{x_0}, c)}\left[ D_{\text{KL}}(p(x_1|x_0,c)||p(x_1|\tilde{x_0},c)) \right] 
    \end{split}
\end{equation}

The inequality holds strictly when $I(X_1;\tilde{X_0}|C)>0$, i.e., $X_1\notindep \tilde{X_0}|C$, which generally holds true when batch effect exists and variables $X_0$ and $\tilde{X_0}$ are associated by $B$ according to the graphical model.

\end{proof}


\newpage
\section{\emph{CellFlow} Algorithm}
\label{sec:algorithm}

\section{The general case: Proof of \texorpdfstring{\Cref{thm:main-decomp}}{Theorem 1.6}}\label{sec:algo}

First, we show that data structure of \Cref{l:max_min_query} can be used to compute distances witnessed by shortest paths that pass through a constant-size separator.

\begin{lemma}\label{l:single_adhesion}
Fix a constant $k \in \mathbb{N}$. There exists an algorithm which as the input receives an edge-weighted graph $G$ on $n$ vertices and $m$ edges together with a partition of its vertices into three sets $A, B, C$ such that $|B| \leq k$ and there are no edges between $A$ and $C$, and as the output computes $\max_{c \in C} \dist(a, c)$ for every $a \in A$. The running time is $\Oh(m \log n + n \log^{k - 1} n)$.
\end{lemma}

\begin{proof}
Let $B = \{b_1, \ldots, b_k\}$. For any $a \in A, c \in C$, we have $\dist(a, c) = \min_{i \in [k]} \dist(a, b_i) + \dist(c, b_i)$. First, we run Dijkstra's algorithm from every vertex in $B$ to find $\dist(v, b_i)$ for every $v \in V(G)$ and $i \in [k]$. Next, we use \Cref{l:max_min_query} to construct a data structure $\mathbb{D}$ for the point set $\{(\dist(c, b_1), \dots, \dist(c, b_k))\colon c\in C\}\subseteq \mathbb{R}^k$. Now, the value $\max_{c \in C} \dist(a, c)$ for any given $a$ is equal to the answer of $\mathbb{D}$ to the query with argument $(\dist(a, b_1), \dots, \dist(a, b_k))$.
\end{proof}

After computing the distances over a constant-size separator, we will use the following observation to simplify one of the sides of the separation.

\begin{lemma}\label{l:inserting_paths}
Let $G$ be a edge-weighted connected graph and let $A, B, C$ be a partition of its vertices such that there are no edges between $A$ and $C$. For every pair of vertices $u, v \in B$, let $P_{u, v}$ be any shortest path from $u$ to $v$ with all internal vertices in $C$ (assuming such a path exists).

Let $G'$ denote a graph obtained from $G[A \cup B]$ by adding an edge from $u$ to $v$ of weight equal to the length of $P_{u, v}$, for all $u, v \in B$ for which $P_{u, v}$ exists. Then,  $$\dist_G(s, t) = \dist_{G'}(s, t)\qquad\textrm{for all }s,t\in A\cup B.$$
\end{lemma}
\begin{proof}
Let $G''$ be the graph obtained by adding new edges of $G'$ to $G$.
Fix any $s, t \in A \cup B$ and let $P$ denote the shortest path from $s$ to $t$ in $G''$ which minimizes the number of vertices from $C$ visited. Naturally, the weight of $P$ is equal $\dist_G(s, t)$. Assume that such path visits at least one vertex of $C$. Then, the path $P$ is of the form $s \xrightarrow{P_1} x \xrightarrow{P_2} y \xrightarrow{P_3} t$, where $x, y \in B$ and all the internal vertices of $P_2$ are in $C$. By the construction of $G'$, $P_2$ can be replaced with a direct edge from $x$ to $y$ of the same weight. We obtain a same weight path with a smaller number of vertices of $C$ visited, which is a contradiction. Therefore, $P$ is entirely contained in $A \cup B$, hence it exists in $G'$. This shows that $\dist_G(s, t) = \dist_{G'}(s, t)$.
\end{proof}


The next lemma encapsulates the main algorithmic content of the proof of \Cref{thm:main-decomp}. The algorithm will split the tree decomposition provided on input into smaller parts for which the eccentricities are easier to calculate. We use the following lemma to handle a single such part.
\begin{lemma}\label{l:star}
Fix constants $k, g \in \mathbb{N}, 0 < \delta < \frac{1}{54}$. Assume we are given $n \in \mathbb{N}$, an edge-weighted graph $G$ on at most $n$ vertices with a weight function $w \colon E(G) \to \mathbb{N}$, a vertex subset $A$ and a collection of non-empty vertex subsets $V_0, V_1, \dots, V_\ell$ satisfying the following conditions:
\begin{itemize}[nosep]
	\item The sum of weights of all the edges in $G$ is bounded by $\Oh(n)$.
	\item $V(G) \setminus A = V_0 \cup V_1 \cup \dots \cup V_\ell$.
	\item $|A| \leq k$.
	\item For every $i \in [\ell]$, $G[V_i \setminus V_0]$ is connected, $N_G(V_i \setminus V_0) = V_i \cap V_0$, $|V_i| = \Oh(n^\delta)$, and $|V_0 \cap V_i| \leq 4$.
	\item For all $i, j \in [\ell], i \neq j$, $V_i \setminus V_0$ and $V_j \setminus V_0$ are disjoint and non-adjacent in $G$.
	\item Every edge $uv \in E(G)$ with $u, v \not\in A$ is contained in $G[V_i]$ for some $i\in \{0,1,\ldots,\ell\}$.
	\item The graph obtained by taking $G[V_0]$ and adding a clique on $V_0 \cap V_i$ for every $i \in [\ell]$ has Euler genus bounded by $g$.
\end{itemize}
Then, we can compute the eccentricity of every vertex of $G$ in time $\Oh \left( n^{1 + \frac{150 + 54 \delta}{151}} \log^k n \right)$.
\end{lemma}

\begin{proof}
Fix $\delta' = \frac{1 + 97 \delta}{151}$; we have $\delta' - \delta = \frac{1 - 54\delta}{151} > 0$.
Let $E_i$ denote the set of edges with one endpoint in $V_i$ and the other endpoint in $V_i \setminus V_0$. For $i \in [\ell]$, we shall say that $V_i$ is {\em{heavy}} if the sum of weights of $E_i$ is larger than $n^{\delta'}$. Since the sets $E_i$ are pairwise disjoint and the total sum of weights of all the edges is bounded by $\Oh(n)$, the number of heavy subsets is bounded by $\Oh(n^{1 - \delta'})$. Without loss of generality, we may assume that $V_{\ell' + 1}, \dots, V_\ell$ are heavy and $V_1, \dots, V_{\ell'}$ are not, for some $\ell'\in \{0,\ldots,\ell\}$.


For any source vertex $s$, we can calculate distances from $s$ to every vertex of $G$  using breadth first search in time $\Oh(\sum_{e \in E(G)} w(e)) = \Oh(n)$.
In particular, for every $\ell' < i \leq \ell$, we can compute the distances from every vertex of $V_i$ to every vertex of $G$ in total time $\Oh(n^{2 - \delta' + \delta})$, because $$|V_{\ell'+1}\cup \ldots\cup V_{\ell}|\leq n^{1-\delta'}\cdot \Oh(n^\delta)=\Oh(n^{1-\delta'+
\delta}).$$
Additionally, we calculate distances $\dist_G(a, v)$ for every $a \in A, v \in V(G)$ in time $O(n)$.

For every $i \in [\ell]$ and $u,v \in V_0 \cap V_i$, there exists a shortest path $P_{i,u,v}$ from $u$ to $v$ with all internal vertices belonging to $V_i - V_0$ due to the assumption that $G[V_i - V_0]$ is connected and $N_G(V_i - V_0) = V_i \cap V_0$. Therefore, the distance from $u$ to $v$ is bounded by the sum of weights of edges in $E_i$. In particular, for $i \in [\ell']$, $\dist_G(u, v) \leq n^{\delta'}$.

We define $\widetilde{G}$ to be the graph obtained by taking $G[A \cup V_0 \cup \dots \cup V_{\ell'}]$ and applying the following operation for every $i \in \{\ell' + 1, \dots, \ell\}$:
for each pair of vertices $u, v \in A \cup (V_0 \cap V_i)$, add an edge in $\widetilde{G}$ between $u$ and $v$ with weight equal to the total weight of $P_{i,u,v}$. For a fixed $i, u$, we can find $P_{i, u, v}$ for all $v$ using breadth first search in time $\Oh(n)$. Taking a sum over all $i, u$, we get that $\tilde{G}$ can be computed in total time $\Oh(n^{2 - \delta'})$.


\begin{claim}\label{cl:wG}
The sum of the edge weights in $\widetilde{G}$ is $\Oh(n)$. Moreover, for all $u, v \in V(\widetilde{G})$, we have $\dist_{\widetilde{G}}(u, v) = \dist_{G}(u, v)$.
\end{claim}

\begin{proof}
Consider $i \in \{\ell' + 1, \dots, \ell\}$ and any $u, v \in A \cup (V_0 \cap V_i)$ for which we added an edge. Its weight is bounded by the sum of weights of edges in $E_i$. Therefore, the total weight of all edges added is at most
$$
\sum_{i \in \{\ell' + 1, \dots, \ell\}} \left( |A \cup (V_0 \cap V_i)|^2 \sum_{e \in E_i} w(e) \right) \leq (4 + k)^2 \sum_{e \in E(G)} w(e) = \Oh(n).
$$
This proves the first part of the claim.

For the second part of the claim, consider any $i \in \{\ell' + 1, \dots, \ell \}$ and observe that by our assumptions, $A \cup (V_0 \cap V_i)$ separates $(V_0 \cup \dots \cup V_{\ell'} \cup V_{i + 1} \cup \dots \cup V_\ell) \setminus V_i$ from $V_i \setminus V_0$. Hence it suffices to repeatedly apply \Cref{l:inserting_paths}.
\end{proof}

For every $u \in V(\widetilde{G})$, we have $\ecc_G(u) = \max(\ecc_{\widetilde{G}}(v), \max_{v \in V(G) \setminus V(\widetilde{G})} \dist_G(u, v))$. Note, that we already know all the distances $\dist_G(u, v)$ for $v \in V(G) \setminus V(\widetilde{G})$. Similarly, we can already compute $\ecc_G(u)$ for every $u \in V(G) \setminus V(\widetilde{G})$. Therefore, it remains to compute $\ecc_{\widetilde{G}}(v)$ for each $v \in V(\widetilde{G})$. Our goal is to show that this can be done efficiently using \Cref{l:main_ecc}.

Now, let $G'$ be the graph obtained from $\tilde{G}$ by replacing every edge $e$ non-indicent to $A$ with $w(e)\geq 2$ with a path of length $w(e)$ consisting of unit-weight edges. This operation again preserves the distances. Since the sum of edge weights in $\tilde{G}$ is of $\Oh(n)$, the total number of vertices in $G'$ is of $\Oh(n)$. For $0 \leq i \leq \ell'$, we write $V'_i$ to denote the set $V_i$ together with all the vertices added as a part of a path between two endpoints in $V_i$.
As $V_i$ is not heavy for each $i\in [\ell']$, we have
$$
|V'_i \setminus V'_0| \leq |V_i| + \sum_{e \in E_i} w(e) = \Oh(n^{\delta'})\qquad \textrm{for all }i\in [\ell'].
$$

Let $G_0$ denote the graph $G'[V'_0]$ and let $G_0^*$ denote the graph $G'- A$ with $V'_i - V'_0$ contracted to a single vertex $v_i^*$, for each $i \in [\ell']$; note that, all edges of $G_0$ and $G_0^*$ have unit weight.

\begin{claim}
	The graph $G_0^*$ is does not contain $K_{t}$ as a minor, where $t = \Oh(\sqrt{g})$.
\end{claim}

\begin{proof}
Let $\bar{G}_0$ denote the graph obtained by taking $G_0$ and adding a clique on $V_0 \cap V_i$ for every $i \in [\ell']$.
By lemma assumptions and the fact that subdividing edges does not increase the Euler genus, $\bar{G}_0$ has Euler genus at most $g$. In particular, $\bar{G}_0$ is $K_{t'}$-minor-free for some $t' = \Oh(\sqrt{g})$, because the Euler genus of $K_{t'}$ is $\Omega({t'}^2)$.

Similarly, let $\bar{G}_0^*$ be the graph obtained by taking $G_0^*$ and adding a clique on each $V_0 \cap V_i$.
Note, that $\bar{G}_0^* - \{v_1^*, \dots, v_{\ell'}^*\}$ is precisely $\bar{G}_0$. Let $t = \max(t', 6)$.
Recall that a minor model of a clique $K_t$ consists of $t$ pairwise vertex-disjoint connected subgraphs, called
branch sets, such that there is at least one edge between each pair of the branch sets.
Consider a minor model $\varphi$ of $K_{t}$ in $\bar{G}^*_0$.
Note that $\varphi$ cannot contain any singleton branch set of the form $\{v^*_i\}$, for the degree of $v^*_i$ in $\bar{G}^*_0$ is at most $4 < t - 1$. Furthermore, since $N_{\bar{G}^*_0}(v^*_i) = V_0 \cap V_i$, any branch set containing $v^*_i$ and at least one other vertex contains some $u \in V_0 \cap V_i$, and $N_{\bar{G}^*_0}(v^*_i)\subseteq N_{\bar{G}^*_0}(u)$, hence removing $v^*_i$ from this branch set preserves the model. Therefore, we can assume without loss of generality that all branch sets of $\varphi$ are disjoint from $\{v^*_1, \dots, v^*_{\ell'}\}$, hence $\varphi$ is a minor model of $K_{t}$ in $\bar{G}_0$. This is a contradiction, as $t \geq t'$ and $\bar{G}_0$ is $K_{t'}$-minor-free. Therefore, $\bar{G}_0^*$ is $K_t$-minor-free, hence $G_0^*$ also.
\end{proof}

Let $\rho' = \frac{2 - 108 \delta}{151} > 0$. The graph $G^*_0$ is a unit-weight graph and is $K_{t}$-minor-free.
Hence, by applying \Cref{t:r_division} to $G^*_0$ (with $\varepsilon = \rho'/2$)
we obtain an $n^{\rho'}$-division $\mathcal{R}_0$ in time $\Oh(n^{1 + \rho'})$.
We extend it to $G' - A$ by mapping every contracted vertex $v^*_i$ to $N_{G' - A}[V'_i - V'_0] = (V'_i - V'_0) \cup (V_0 \cap V_i)$. Formally, we put $V''_i \coloneqq N_{G' - A}[V'_i - V'_0]$ and 
$$
\mathcal{R} \coloneqq \left\{ (R_0 \cap V'_0) \cup \bigcup_{i \colon v^*_i \in R_0} V''_i \colon R_0 \in \mathcal{R}_0 \right\}.
$$

Now, we argue that $\mathcal{R}$ is a reasonable division of $G' - A$. Clearly, all sets $R \in \mathcal{R}$ are connected in $G' - A$. Pick any $R \in \mathcal{R}$ and let $R_0$ be its corresponding set in $\mathcal{R}_0$.
Every vertex $v^*_i$ is mapped to a set of size $\Oh(n^{\delta'})$, therefore
$$|R| \leq |R_0| \cdot \Oh(n^{\delta'}) = \Oh(n^{\rho' + \delta'}).$$

By our construction, for every $i \in [\ell']$, $R$ is either disjoint from $V'_i - V'_0$ or contains whole $N_{G' - A}[V'_i - V'_0]$. This means that no vertex belonging to any $V'_i - V'_0$ can be in $\partial R$, hence $\partial R \subseteq V'_0$.

Pick any $u \in \partial R \cap R_0$. Assume that $u \not\in \partial R_0$. Then every vertex of $N_{G_0^*}(u)$ must be in $R_0$, hence $N_{G - A'}(u) \subseteq R$, which is a contradiction. This means that $\partial R \cap R_0 \subseteq \partial R_0$.

Pick any $u \in \partial R - R_0$. Then, $u \in V_0 \cap V_i$ for some $i \in [\ell']$ such that $v_i^* \in R_0$. Moreover, $v_i^* \in \partial R_0$ and is adjacent to $u$ in $G_0^*$. The number of such $u$ is bounded by $4 |\partial R_0 \cap \{ v_1^*, \dots, v_{\ell'}^* \}|$.

Putting two cases together, we obtain:
$$
\sum_{R \in \mathcal{R}} |\partial R| = \sum_{R \in \mathcal{R}} \left(|\partial R \cap R_0| + |\partial R - R_0|\right) \leq \sum_{R_0 \in \mathcal{R}_0} \left(|\partial R_0| + 4 |\partial R_0 \cap \{ v_1^*, \dots, v_{\ell'}^* \}|\right) = \Oh(n^{1 - \frac{1}{2}\rho'}).
$$

It remains to show the following claim.

\begin{claim}
Pick any $R \in \mathcal{R}, s_R \in R$. The number of different distance profiles on $R$ relative to $s_R$ in $G' - A$ is of $\Oh(n^{48\rho' + 54\delta'})$.
\end{claim}
\begin{proof}
We look at every vertex $v \in V(G') \setminus A$ and consider three cases: $v \in R$, $v \in V'_0$, and $v \in V'_i \setminus (V'_0 \cup R)$ for some $i \in [\ell']$. By our construction, $R \cap V'_0$ is non-empty, hence w.l.o.g. we can assume that $s_R \in V'_0$ as whether two vertices have the same profile on $R$ is independent of the choice of the pivot vertex.

In the first case, there are at most $|R| = \Oh(n^{\rho' + \delta'})$ such vertices, hence they realise at most that many profiles.

In the second case, we want to observe that profile of any vertex $v \in V'_0$ on $R$ depends only on its profile on $R \cap V'_0$ (relative to $s_R$). Pick any $t \in R - V'_0$. Then $t \in V'_i - V'_0$ for some $i \in [\ell']$, $V_i \cap V_0 \subseteq R \cap V'_0$, and every path from $v$ to $t$ intersects $V_i \cap V_0$. In particular, distances from $v$ to vertices of $V_i \cap V_0$ determine its distance to $t$, which proves the observation.

Let $\tilde{G}_0$ denote the graph obtained by taking $G'[V'_0]$ and for every $i \in [\ell'], u, v \in V_0 \cap V_i$ adding a disjoint path from $u$ to $v$ of length $\dist(u, v)$. Let $P_i$ denote the vertex set of paths added between $V_0 \cap V_i$. For every $t \in V'_0$ we have $\dist_{G' - A}(v, t) = \dist_{\tilde{G}_0}(v, t)$, so it suffices to bound the number of profiles on $R \cap V'_0$ in $\tilde{G}_0$. By our assumptions, $\tilde{G}_0$ has Euler genus bounded by $g$ and all $P_i$ are of size $\Oh(n^{\delta'})$.

Let $R_0$ be the set of $\mathcal{R}_0$ corresponding to $R$. Let $\tilde{R}_0$ denote the set $(R \cap V'_0) \cup \bigcup_{i : v^*_i \in R_0} P_i$. Such set is connected in $\tilde{G}_0$. Moreover, similarly to $R$, its size is $\Oh(n^{\rho' + \delta'})$. Applying \Cref{thm:distprofiles}, we get that the number of distance profiles on $\tilde{R}_0$ in $\tilde{G}_0$ is $\Oh(n^{12(\rho' + \delta')})$, which also bounds the number of profiles on $R$ in $G' - A$ realised by $V'_0$.

For the third case, assume $v \in V'_i \setminus (V'_0 \cup R)$ for some $i\in [\ell']$. Every path from $v$ to any vertex of $R$ in $G' - A$ intersects $V_i \cap V_0$. Let $v_1, \dots v_p$ be the vertices of $V_i \cap V_0$, where $p \leq 4$. The profile of $v$ on $R$ is then determined by the following:
\begin{itemize}[nosep]
 \item[(a)] the profile of each $v_j$ on $R$,
 \item[(b)] $\dist_{G' - A}(v, v_j) - \dist_{G' - A}(v, v_1)$ for each $2 \leq j \leq p$, and
 \item[(c)] $\dist_{G' - A}(s_R, v_j) - \dist_{G' - A}(s_R, v_1)$ for each $2 \leq j \leq p$ where $s_R$ is some pivot vertex of $R$.
\end{itemize}
By the previous case, the number of distance profiles of each $v_j$ is $\Oh(n^{12(\rho' + \delta')})$. The distances between $v$ and $v_j$ are bounded by $|V'_i|$, hence each quantity described in (b) can take $\Oh(n^{\delta'})$ different possible values. Similarly, since $v_1$ and $v_j$ are connected via $V'_i$, $|\dist_{G' - A}(s_R, v_j) - \dist_{G' - A}(s_R, v_1)| \leq \Oh(n^{\delta'})$. The number of different possible profiles of such $v$ is therefore bounded by $\Oh(n^{48(\rho' + \delta') + 6\delta'}) = \Oh(n^{48\rho' + 54\delta'})$. This finishes the proof of the claim.
\end{proof}

Now we can apply \Cref{l:main_ecc} to graph $G'$ with apex set $A$, $X = V(\widetilde{G})$, and the following constants: $$\rho = \rho' + \delta',\qquad \gamma = 1 - \frac{1}{2}\rho',\quad \textrm{and}\quad \alpha = 48\rho' + 54 \delta'.$$ This allows us to calculate all $V(\widetilde{G})$-eccentricities in $G'$ in time
$$
\Oh \left( \left(
	n^{ 2 - \frac{1}{2} \rho' } +
	n^{ 1 + 48\rho' + 54 \delta' }
\right) \log^k n \right) =
\Oh \left( n^{1 + \frac{150 + 54 \delta}{151}} \log^k n \right).
$$
Since for each $v\in V(\widetilde{G})$ we have $\ecc_{\widetilde{G}}(v) = \max_{u \in V(\widetilde{G})} \dist_{\widetilde{G}}(v, u) = \max_{u \in V(\widetilde{G})} \dist_{G'}(v, u)$, this means that we have successfully computed all the eccentricities in $\widetilde{G}$; and as we argued, this is enough to compute all the eccentricities in $G$ as well.

Finally, the total running time of the algorithm is
$$
\Oh \left( n^{1 + \frac{150 + 54 \delta}{151}} \log^k n + n^{2 - \delta' + \delta} \right) =
\Oh \left( n^{1 + \frac{150 + 54 \delta}{151}} \log^k n \right).
$$\qedhere
\end{proof}


\begin{lemma}\label{l:star2}
Fix constants $k, g \in \mathbb{N}, 0 < \delta < \frac{1}{54}$. Assume we are given $n \in \mathbb{N}$, an edge-weighted graph $G$ on at most $n$ vertices with a weight function $w \colon E(G) \to \mathbb{N}$, a vertex subset $A$ and a collection of non-empty vertex subsets $V_0, V_1, \dots, V_\ell$ satisfying the same conditions as in \Cref{l:star} with the following differences:
\begin{itemize}
	\item we don't require $G[V_i - V_0]$ to be connected and $V_i - V_0$ to be adjacent to whole $V_i \cap V_0$;
	\item instead of $|V_0 \cap V_i| \leq 4$, we require $|V_0 \cap V_i| \leq k$.
\end{itemize}
Then, we can compute the eccentricity of every vertex of $G$ in time $\Oh \left( n^{1 + \frac{150 + 54 \delta}{151}} \log^{k + 5g} n \right)$.
\end{lemma}

\begin{proof}
We will reduce our input to one which will satisfy the conditions of \Cref{l:star}. We start by addressing the adhesions $V_0 \cap V_i$ containing too many vertices.

Let $G_0$ denote the graph $G[V_0]$ with cliques placed at $V_0 \cap V_i$ for every $i \in [\ell]$.
For every $i \in [\ell]$ we repeat the following procedure: while $|V_0 \cap V_i| > 4$,
remove arbitrary $5$ vertices from $V_0 \cap V_i$. Since $|V_0 \cap V_i| \leq k$ for each $i\in [\ell]$,
this procedure can be implemented in total time $\Oh(n)$. As a result, at the end we have $|V_0 \cap V_i| \leq 4$ for all $i \in [\ell]$. Let $M$ be the set of all the removed vertices. By our assumptions, $G_0$ has Euler genus bounded by $g$, hence it cannot contain $g + 1$ pairwise disjoint copies of $K_5$
(as the Euler genus of a graph is the sum of the Euler genera of its 2-connected components~\cite{StahlB77} and $K_5$ is not planar). Each removed quintiple of vertices induces a $K_5$ in $G_0$, hence we have $|M| \leq 5g$. We set $A' = A \cup M$ and may thus assume that $V_i$ is disjoint from $A'$ for all $0 \leq i \leq \ell$.

Now, fix $i \in [\ell]$. Let $C^i_1, \dots, C^i_{r_i}$ denote the connected components of $V_i - V_0$ in $G - A'$. We define $W^i_j := N_{G - A'}[C^i_j]$ for every $j \in [r_i]$. Clearly, all $W^i_j$ induce a connected subgraph of $G$ and satisfy $N_{G - A'}(W^i_j - V_0) = W^i_j \cap V_0$. We put $V'_0 := V_0$ and enumerate
$$
\{V'_1, V'_2, \dots V'_{\ell'}\} := \{ W^i_j \colon i \in [\ell], j \in [r_i] \}.
$$
It is easy to verify that the sets $A'$ and $V'_0, V'_1, \dots, V'_{\ell'}$ satisfy the conditions of \Cref{l:star}. We apply said lemma to calculate the eccentricity of every vertex of $G$ in the desired time.
\end{proof}



The next statement is a reformulation of \Cref{thm:main-decomp}.

\begin{theorem}
Fix constants $k, g \in \mathbb{N}$. Assume we are given a graph $G$ on $n$ vertices together with its tree decomposition $(T, \beta)$ and a set of private apices $A_t \subseteq \beta(t)$ for each node $t\in V(T)$ such that the following conditions hold:
\begin{itemize}[nosep]
 \item For every node $t \in V(T)$, we have $|A_t| \leq k$.
 \item For every edge $st \in E(T)$,  we have $|\beta(v) \cap \beta(u)|\leq k$.
 \item For every node $t \in V(T)$, graph obtained by taking $G[\beta(t)] - A_t$ and turning  $(\beta(t) \cap \beta(s))\setminus A_t$ into a clique for every edge $st \in E(T)$ has Euler genus bounded by $g$.
\end{itemize}
Then, we can compute the eccentricity of every vertex of $G$ in time $\Oh \left( n^{1 + \frac{355}{356}} \log^{k + 5g} n \right)$.
\end{theorem}

\begin{proof}
We may assume that $|V(T)|\leq n$, for every tree decomposition with no two bags comparable by inclusion has this property; and adjacent comparable bags can be merged by contracting the edge between them.

For a node $t\in V(T)$, by the {\em{weight}} of $t$ we mean the size of the corresponding bag, that is, $|\beta(t)|$. For any subset of nodes $S \subseteq V(T)$, we define $\beta(S) \coloneqq \bigcup_{t \in S} \beta(t)$ By the {\em{weight}} of $S$, we mean the total weight of the elements of $S$, that is, $\sum_{t\in S} |\beta(t)|$. 

\begin{claim}\label{cl:weight-T}
The weight of $V(T)$ is of $\Oh(n)$.
\end{claim}

\begin{proof}
The sets $\beta'(t) := \beta(t) - \bigcup_{s \in N_T(t)} \beta(s)$ are pairwise disjoint. We have
$$
\sum_{t \in V(T)} |\beta(t)| =
\sum_{t \in V(T)} |\beta'(t)| + 2 \cdot \sum_{st \in E(T)} |\beta(s) \cap \beta(t)| \leq
|V(T)| + 2k|E(T)| = \Oh(n).
$$
\end{proof}

Since every bag induces a graph of bounded Euler genus, the number of edges contained in a bag is linear in its size. In particular, this implies that the total number of edges of $G$ is also bounded by $\Oh(n)$.

We set $$\delta \coloneqq \frac{1}{356}\qquad\textrm{and}\qquad \Delta \coloneqq \frac{355}{356}.$$ Root the tree $T$ in an arbitrarily chosen node; this naturally imposes an ancestor-descendant relation in $T$ (for convenience, every node is considered its own ancestor and descendant).

We start by partitioning $T$ into connected subtrees using the following procedure.
We proceed bottom-up over $T$, processing nodes in any order so that a node is processed after all its strict descendants have been processed. Along the way, we mark some nodes and split the edges of $T$ into heavy and light. Let $t \in V(T)$ be the currently processed non-root node of $T$ and let $e \in E(T)$ be the edge connecting $t$ with its parent. If the total weight of all the unmarked nodes that are descendants of $t$ is at least $n^\delta$ (recall that this includes $t$ itself as well), then we declare $e$ heavy and mark all the descendants of $t$ that were unmarked so far. Otherwise, the edge $e$ is declared light and the procedure proceeds to further nodes of $T$.

Observe that
removing all heavy edges splits $T$ into connected subtrees, say $T'_1, \cdots T'_m$. All of the subtrees, except for possibly the subtree containing the root node, are of weight at least $n^\delta$. In particular, the number of subtrees $m$, and therefore the number of heavy edges, is  bounded by $\Oh(n^{1 - \delta})$. Moreover, in every subtree $T'_i$, removing the node closest to the root splits $T'_i$ into smaller components, each of weight less than $n^\delta$.

Fix a heavy edge $e$ and let $T^e_1$ and $T^e_2$ be the two subtrees into which $T$ splits after removing~$e$. Let $X^e_i = \beta(T^e_i)$ for $i \in \{1, 2\}$. Put $A_e = X^e_1 \setminus X^e_2$, $C_e = X^e_2 \setminus X^e_1$, and $B_e = X^e_1 \cap X^e_2$. By the properties of tree decompositions, such choice of $A_e, B_e, C_e$ satisfies the conditions of \Cref{l:single_adhesion}, hence in time $\Oh(n \log^{k - 1} n)$ we can compute $\max_{v \in X^e_2} \dist_G(u,v)$ for every $u \in X^e_1$, and $\max_{u \in X^e_1} \dist_G(u,v)$ for every $v \in X^e_2$. Computing this for every heavy edge $e$ takes total time $\Oh(n^{2 - \delta} \log^{k - 1} n)$.

Fix any subtree $T'=T'_j$. Let $e_1 = t^{e_1}_1t^{e_1}_2, e_2 = t^{e_2}_1 t^{e_2}_2, \dots, e_\ell = t^{e_\ell}_1 t^{e_\ell}_2$ denote the heavy edges incident to $T'$, where $t^{e_i}_1 \in V(T')$ and $V(T') \subseteq V(T_1^{e_i})$ for every $i \in [\ell]$.
For a vertex $v \in \beta(T')$, let
$$d_0(v) = \max_{u \in \beta(T')} \dist_G(v, u)\qquad\textrm{and}\qquad d_i(v) = \max_{u \in X_2^{e_i}}\dist_G(v,u),\quad\textrm{for } i \in [\ell].$$ We have $\ecc(v) = \max \{ d_i(v)\colon i\in \{0,1,\ldots,\ell\}\}$.The values of $d_i(v)$ are already calculated for all $i\in [\ell]$, hence it remains to compute $d_0(v)$.

For every $i \in [\ell]$ and every pair of vertices $u, v \in \beta(t^{e_i}_1) \cap \beta(t^{e_i}_2)$ we find a shortest path between $u$ and $v$ with all internal vertices inside $X^{e_i}_2$ (or determine that it doesn't exist). For a fixed $u, v$ this can be done in time $\Oh(n)$. Since in total we perform this step at most $2k^2$ times per heavy edge, it takes $\Oh(n^{2 - \delta})$ time in total. Let $P_{i, u, v}$ denote such path, assuming it exists.

Let $G'$ denote the graph obtained from $G[\beta(T')]$ by taking every $i, u, v$ for which $P_{i, u, v}$ exists and adding an edge between $u$ and $v$ of weight equal to the total weight of $P_{i, u, v}$.
The weight of every edge inserted in $\beta(t^{e_i}_1) \cap \beta(t^{e_i}_2)$ is bounded by $|X^{e_i}_2|+1$. The total weight of all edges inserted is therefore at most
$$
\sum_{i \in [\ell]} |\beta(t^{e_i}_1) \cap \beta(t^{e_i}_2)|^2 \cdot (|X^{e_i}_2|+1) \leq
k^2 \sum_{i \in [\ell]} (|X^{e_i}_2|+1) = \Oh(n),
$$
where the last equality follows from the fact that all the trees $T^{e_i}_2$ are pairwise disjoint.
By \Cref{l:inserting_paths}, we have $\dist_{G'}(u, v) = \dist_G(u, v)$ for each $u, v \in \beta(T')$. Hence, computing $d_0(v)$ for every $v \in \beta(T')$ is equivalent to computing the eccentricity of every vertex in $G'$.

If the size of $\beta(T')$ is smaller than $n^\Delta$, we compute the eccentricities naively in time $\Oh(|\beta(T')|^2)$, 
noting that $G'$ has $\Oh(|\beta(T')|)$ edges (thanks to Claim~\ref{cl:weight-T} and bounded genus assumption 
of the last bullet of the theorem statement). Otherwise, we argue that we can use the algorithm in \Cref{l:star} as follows.

Let $t$ be the node of $T'$ closest to the root. Let $s_1, \dots, s_p$ be the children of $t$ in $T$ and let $T''_i$ denote the connected component of $T' - \{t\}$ containing $s_i$. Set $V_0 = \beta(t)$ and $V_i = \beta(T''_i)$ for $i \in [p]$.

It is now easy to verify that $G'$ and sets $A, \{V_i\colon 0\leq i\leq p\}$ selected this way satisfy the assumptions of \Cref{l:star2}. This allows us to use it to compute the eccentricities in $G'$ in time
$$
\Oh \left( n^{1 + \frac{150 + 54\delta}{151}} \log^{k + 5g} n \right) =
\Oh \left( n^{1 + \frac{354}{356}} \log^{k + 5g} n \right).
$$
As we argued, from these eccentricities, we may easily compute all the eccentricities in $G$.

Now, let us analyse the total running time of the whole algorithm. We invoke \Cref{l:star} $\Oh(n^{1 - \Delta})$ times, since we apply it only to subtrees $T'_i$ of size at least $n^\Delta$. The total running time of those applications is hence
$$
\Oh \left( n^{2 + \frac{354}{356} - \Delta} \log^{k + 5g} n \right) =
\Oh \left( n^{1 + \frac{355}{356}} \log^{k + 5g} n \right).
$$
We compute the eccentricities naively for subtrees smaller than $n^\Delta$, hence the total running time of this computation is
$$
\sum_{i \in [m] \colon |\beta(T'_i)| \leq n^\Delta} |\beta(T'_i)|^2 \leq
n^\Delta \cdot \sum_{i \in m} |\beta(T'_i)| = \Oh(n^{1 + \Delta})=\Oh\left(n^{1+\frac{355}{356}}\right).
$$
The rest of computation can be done in $\Oh(n^{2 - \delta} \log^k n)$. Therefore, the whole algorithm runs in time $\Oh \left( n^{1 + \frac{355}{356}} \log^{k + 5g} n \right)$.
\end{proof}

Algorithm \ref{alg:celldiff} provides a detailed overview of the \emph{CellFlow} algorithm, covering both training and inference. During training, the model learns to predict velocity between an initial and target distribution by interpolating between samples, applying noise and condition dropout, and optimizing an L2 loss between predicted and true velocities. In inference, the trained model iteratively updates a sample from the initial distribution toward the target distribution using classifier-free guidance, ultimately generating a new sample that aligns with the target distribution.

\section{Experimental Details}
\label{sec:experimental}

\textbf{Model architecture.} \emph{CellFlow} employs a UNet-based velocity field parameterization with input and output channels matching the dataset. It features four stages for downsampling and upsampling, with each stage halving or doubling the resolution and using a hidden size of 128. This hierarchical UNet design focuses on efficient 2D spatial learning for pixel-level flow matching.

\textbf{Perturbation encoding.} We encode perturbations following IMPA’s approach~\cite{palma2023predicting}. For chemical embeddings, we use 1024-dimensional Morgan Fingerprints generated with RDKit. For gene embeddings, CRISPR and ORF embeddings combine Gene2Vec with HyenaDNA-derived sequence representations, resulting in final dimensions of 328 and 456, respectively.

\textbf{Training details.} Models are trained for 100 epochs on 4 A100 GPUs using the Adam optimizer with a learning rate of 1e-4 and a batch size of 128, requiring 8, 16, and 36 hours for BBBC021, RxRx1, and JUMP, respectively. The noise injection probability, condition drop probability, and classifier-free guidance strength are set to 0.5, 0.2, and 1.2, respectively. Models are selected based on the lowest FID scores on the validation set.


\newpage
\section{Batch Effects}
\label{sec:batch_effect}

\textbf{1. What Are Batch Effects in Microscopy Experiments?}  

Batch effects refer to a form of \textit{distribution shift} in microscopy experiments, where non-biological variations arise due to differences in experimental conditions across imaging sessions or batches. These effects can be classified as a type of \textit{covariate shift}, where technical factors alter the distribution of input features, including:

\begin{itemize}[itemsep=0pt, parsep=0pt, topsep=0pt]
    \item \textbf{Microscope or camera settings} – Variations in sensor sensitivity, illumination, resolution, or imaging modalities.
    \item \textbf{Experimental procedures} – Differences in sample preparation, staining protocols, reagent batches, or handling by different researchers.
    \item \textbf{Environmental conditions} – Changes in temperature, humidity, or laboratory-specific conditions that may be difficult to control.
\end{itemize}

As a result, images of biologically identical cells may appear different solely due to variations in imaging conditions rather than biological differences.

\textbf{2. Why Do Batch Effects Matter?}  

Batch effects pose a major challenge to reproducible biomedical research by obscuring true biological effects of perturbations. Additionally, machine learning models may inadvertently learn batch-specific artifacts instead of meaningful biological patterns. Key issues caused by batch effects include:

\begin{itemize}[itemsep=0pt, parsep=0pt, topsep=0pt]
    \item \textbf{Poor generalization} – Models trained on batch-affected images may fail to classify new samples from a different experimental setup.
    \item \textbf{False discoveries} – Uncorrected batch effects can confound biological signals, leading to misleading conclusions.
    \item \textbf{Reduced reproducibility} – Results may not replicate across laboratories or imaging systems due to unaccounted technical biases.
\end{itemize}

\textbf{3. Visualization of Batch Effects}  

Figure~\ref{fig:batch_effect} visualizes three batches of BBBC021 images using PCA, showing that each batch forms a distinct cluster. Notably, control (ctrl) and perturbed (trt) images from the same batch cluster together, rather than forming separate control and target clusters. This illustrates the \textbf{batch effect}—a systematic bias within each batch that is unrelated to the perturbation itself.

\textbf{4. How \emph{CellFlow} Addresses Batch Effects?}  

\emph{CellFlow} mitigates batch effects by using control images as initialization during both training and inference, transporting them to target images within the same batch. This ensures that the model learns only the \textbf{relative difference} between control and perturbed images. By conditioning on control images from different batches, \emph{CellFlow} effectively captures the \textbf{true perturbation effect} while preserving batch-specific artifacts. Figure~\ref{fig:batch_effect} demonstrates this, showing that predicted images remain within the same batch cluster when given a control image from that cluster.

\vspace{-1em}
\begin{figure*}[htbp]
    \centering
    \includegraphics[width=0.49\linewidth]{imgs/batch_effect.png}
    \caption{\textbf{Visualization of batch effects in BBBC021 and how \emph{CellFlow} addresses batch effects.}  }
    \vspace{-1em}
    
    \label{fig:batch_effect}
\end{figure*}


\newpage
\section{Datasets}
\label{sec:data}
As described below, all data used in this study are publicly available and utilized under their respective licenses. No new data were generated for this study.

\textbf{BBBC021 dataset.}  
We utilized the BBBC021v1 image set~\cite{caie2010high}, available from the \href{https://bbbc.broadinstitute.org/BBBC021}{Broad Bioimage Benchmark Collection}~\cite{ljosa2012annotated}. The BBBC021 dataset focuses on chemical perturbations in MCF-7 breast cancer cells, serving as a robust benchmark for image-based phenotypic profiling. It comprises 97,504 fluorescent microscopy images of cells treated with 113 small molecules across eight concentrations, targeting diverse cellular mechanisms such as actin disruption, Aurora kinase inhibition, and microtubule stabilization. Each image includes multi-channel labels for DNA, F-actin, and beta-tubulin, facilitating detailed morphological analysis. Metadata provides mechanism-of-action (MOA) annotations for compounds and experimental conditions, enabling applications in mechanistic prediction and phenotypic similarity analysis. Table \ref{tab:moa} shows MoA classes for all BBBC021 perturbations. Images were processed at a resolution suitable for segmentation and deep learning tasks.

\textbf{RxRx1 dataset.}  
The RxRx1 dataset~\cite{sypetkowski2023rxrx1}, available under a \href{https://creativecommons.org/licenses/by-nc-sa/4.0/}{CC-BY-NC-SA-4.0 license} from Recursion Pharmaceuticals at \href{https://www.rxrx.ai/rxrx1\#Download}{rxrx.ai}, focuses on genetic perturbations using CRISPR-mediated gene knockouts. It contains 170,943 images representing 1,042 genetic perturbations in HUVEC cells, with control conditions to address experimental variability. Images were captured across six channels, including nuclear and cytoskeletal markers, enabling high-dimensional phenotypic analysis. Preprocessing steps included segmentation, cropping, and resizing to standardize the data for computational analysis. This dataset supports tasks such as feature extraction, phenotypic clustering, and representation learning.

\textbf{JUMP dataset (CPJUMP1).}  
The JUMP dataset~\cite{chandrasekaran2023jump}, available under a \href{https://creativecommons.org/publicdomain/zero/1.0/deed.en}{CC0 1.0 license}, integrates both genetic and chemical perturbations, offering the most comprehensive image-based profiling resource to date. It includes approximately 3 million images capturing the phenotypic responses of 75 million single cells to genetic knockouts (CRISPR/ORF) and chemical perturbations. Key features include:  

\begin{itemize}
    \item \textbf{Chemical-genetic pairing:} Perturbations targeting the same genes are tested in parallel to assess phenotypic convergence or divergence.  
    \item \textbf{Controlled conditions:} Imaging was standardized across cell types (U2OS and A549), time points (short and extended durations), and experimental setups.  
    \item \textbf{Primary group:} Forty plates profiling CRISPR knockouts and ORF overexpression.  
    \item \textbf{Secondary group:} Additional plates exploring extended experimental conditions.  
\end{itemize}

The JUMP dataset uniquely enables the study of phenotypic relationships between genetic and chemical perturbations and supports the development of predictive models for multi-modal cellular responses. Public access to the dataset and associated analysis pipelines is available via \href{https://broad.io/cpjump1}{Broad's JUMP repository}.

\begin{table}[htbp]
    \centering
    \small
    \rowcolors{2}{white}{light-light-gray}
    \setlength\tabcolsep{6pt}
    \renewcommand{\arraystretch}{1.1}
    \begin{tabular}{ll}
        \toprule
        \textbf{Compound} & \textbf{MoA} \\
        \midrule
        Cytochalasin B & Actin disruptors \\
        Cytochalasin D & Actin disruptors \\
        Latrunculin B & Actin disruptors \\
        AZ258 & Aurora kinase inhibitors \\
        AZ841 & Aurora kinase inhibitors \\
        Mevinolin/Lovastatin & Cholesterol-lowering \\
        Simvastatin & Cholesterol-lowering \\
        Chlorambucil & DNA damage \\
        Cisplatin & DNA damage \\
        Etoposide & DNA damage \\
        Mitomycin C & DNA damage \\
        Camptothecin & DNA replication \\
        Floxuridine & DNA replication \\
        Methotrexate & DNA replication \\
        Mitoxantrone & DNA replication \\
        AZ138 & Eg5 inhibitors \\
        PP-2 & Epithelial \\
        Alsterpaullone & Kinase inhibitors \\
        Bryostatin & Kinase inhibitors \\
        PD-169316 & Kinase inhibitors \\
        Colchicine & Microtubule destabilizers \\
        Demecolcine & Microtubule destabilizers \\
        Nocodazole & Microtubule destabilizers \\
        Vincristine & Microtubule destabilizers \\
        Docetaxel & Microtubule stabilizers \\
        Epothilone B & Microtubule stabilizers \\
        Taxol & Microtubule stabilizers \\
        ALLN & Protein degradation \\
        Lactacystin & Protein degradation \\
        MG-132 & Protein degradation \\
        Proteasome inhibitor I & Protein degradation \\
        Anisomycin & Protein synthesis \\
        Cyclohexamide & Protein synthesis \\
        Emetine & Protein synthesis \\
        DMSO & DMSO \\
        \bottomrule
    \end{tabular}
    \caption{\textbf{Modes of action (MoA) for compounds in BBBC021.}}
    \label{tab:moa}
\end{table}


\newpage
\section{Qualitative Comparison}
\label{sec:comparison}

In this section, we present additional generated samples to further demonstrate the effectiveness of our method. Figures \ref{fig:qualitative_bbbc}, \ref{fig:qualitative_rxrx1}, and \ref{fig:qualitative_jump} show qualitative comparisons on the BBBC021, RxRx1, and JUMP datasets, respectively. Our approach more accurately captures key biological effects, whereas images generated by IMPA fail to reflect real biological responses, and those from PhenDiff appear blurry with significant detail loss.

\begin{figure*}[htbp]
    \centering
    \includegraphics[width=0.8\linewidth]{imgs/qualitative_bbbc.pdf}
    \caption{\textbf{More qualitative comparisons of generated samples on BBBC021.}  }
    \label{fig:qualitative_bbbc}
\end{figure*}
\begin{figure*}[htbp]
    \centering
    \includegraphics[width=0.8\linewidth]{imgs/qualitative_rxrx1.pdf}
    \caption{\textbf{More qualitative comparisons of generated samples on RxRx1.}  }
    \label{fig:qualitative_rxrx1}
\end{figure*}
\begin{figure*}[htbp]
    \centering
    \includegraphics[width=0.8\linewidth]{imgs/qualitative_cpg.pdf}
    \caption{\textbf{More qualitative comparisons of generated samples on JUMP.}  }
    \label{fig:qualitative_jump}
\end{figure*}


\section{Trajectory}
\label{sec:trajectory}

\textbf{Forward interpolation.} Our generation process aims to transform a control image into its corresponding perturbed image using our flow matching model. This is achieved by iteratively solving an ODE, where the velocity field predicted by the model guides the transformation at each timestep. As iterations progress, the image gradually evolves towards its final state at $t = 1$, representing the fully perturbed cell morphology.

\textbf{Backward interpolation.} Due to the bidirectional nature of our model, we can also perform a reversible generation process by inverting the velocity direction. This allows us to start from the perturbed image and gradually recover the original control image, demonstrating the reversible capabilities of our method.

\textbf{Trajectory examples.} Figures \ref{fig:bbbc_trajectory1} and \ref{fig:bbbc_trajectory2} illustrate these bidirectional transformations. The top section of each figure depicts the forward trajectory, where the control image is progressively updated based on the learned velocity field, ultimately generating the perturbed image at $t = 1$. The bottom section shows the reverse trajectory, where the process is reversed, progressively reconstructing the original control image. This capability, which is absent in diffusion-based methods, offers a promising approach for simulating morphological trajectories during perturbation responses. Moreover, \emph{CellFlow}’s reversible distribution transformation enables modeling of backward transitions in cell states, with potential applications in studying recovery dynamics and predicting treatment outcomes.

To further demonstrate our approach, we present trajectory examples for two drugs. The first, PP-2, reduces cell adhesion and disrupts actin reorganization, leading to a more dispersed cell distribution. In Figure \ref{fig:bbbc_trajectory1}, the forward trajectory shows cells transitioning from a clustered to a more diffuse state, while the reverse trajectory restores the original aggregation. The second, Chlorambucil, induces pyknosis (nuclear shrinkage). In Figure \ref{fig:bbbc_trajectory2}, the forward process shows one of the three nuclei undergoing cell death or division, leaving only two nuclei in the final state, while the reverse trajectory reconstructs the original three-nucleus configuration. These results highlight our method’s ability to capture biologically meaningful morphological transitions in both directions.

\begin{figure*}[htbp]
    \centering
    \includegraphics[width=\linewidth]{imgs/bbbc_trajectory.jpg}
    \caption{\textbf{(1/2) Bidirectional interpolation trajectory in BBBC021.}  }
    \label{fig:bbbc_trajectory1}
\end{figure*}
\begin{figure*}[htbp]
    \centering
    \includegraphics[width=\linewidth]{imgs/bbbc_trajectory2.jpg}
    \caption{\textbf{(2/2) Bidirectional interpolation trajectory in BBBC021.}  }
    \label{fig:bbbc_trajectory2}
\end{figure*}


\newpage
\section{More Results}
\label{sec:results_appendix}

\textbf{Out-of-distribution generalization.} Table \ref{tab:ood_per_class} reports results on the out-of-distribution (OOD) set in BBBC021, evaluating performance on perturbations absent from the training set. This highlights our method’s strong generalization ability to novel chemical perturbations. The FID score measures the similarity between generated and real distributions, with lower values indicating a closer match. As shown in the table, our method effectively captures the biological effects of each perturbation, generating images that closely resemble real cellular responses. Robust OOD generalization is essential for biological research, enabling the exploration of untested interventions, analysis of unknown cellular responses, and the design of new drugs by simulating effects before experimental validation.

\begin{table}[htbp]
    \rowcolors{2}{white}{light-light-gray}
    \centering
    \setlength\tabcolsep{3pt}
    \small
    \begin{tabular}{lcccccccc}
        \toprule
        Method
        & AZ841 & Cyclohexamide & Cytochalasin D & Docetaxel & Epothilone B & Lactacystin & Latrunculin B & Simvastatin \\
        \midrule
        PhenDiff & 136.5 & 224.0 & 180.3 & 160.1 & 131.5 & 139.7 & 132.5 & 108.9 \\
        IMPA & 131.7 & 189.9 & 180.6 & 130.6 & 120.7 & 133.7 & 128.5 & \textbf{79.7} \\
        CellFlow & \textbf{84.1} & \textbf{99.5} & \textbf{129.4} & \textbf{81.9} & \textbf{93.7} & \textbf{106.9} & \textbf{97.9} & 90.9 \\
        \bottomrule
    \end{tabular}
    \caption{\textbf{Out-of-distribution generalization results per perturbation.}}
    \label{tab:ood_per_class}
\end{table}

\section{Related Works}
\label{sec:related_appendix}

Table~\ref{tab:related} presents a brief comparison of our work with existing methods for cellular morphology generation.

\begin{table*}[htbp]
    \small
    \centering
    \rowcolors{2}{white}{light-light-gray}
    \setlength\tabcolsep{6pt}
    \renewcommand{\arraystretch}{1.1}
    \begin{tabular}{lccc}
    \toprule
    Paper & Generative Model & Use Control Image & How to Use Control Image \\
    \midrule
    Mol2Image~\cite{yang2021mol2image} & Normalizing Flows & No & - \\
    IMPA~\cite{palma2023predicting} & GAN & Yes & Add AdaIn layers to GAN \\
    MorphoDiff~\cite{navidi2024morphodiff} & Diffusion & No & - \\
    PhenDiff~\cite{bourou2024phendiff} & Diffusion & Yes & Map control to noise then to target \\
    LUMIC~\cite{hung2024lumic} & Diffusion & Yes & Add DINO embedding as condition \\
    pDIFF~\cite{cook2024diffusion} & Diffusion & No & - \\
    CellFlow (Ours) & Flow Matching & Yes & Initialization as source distribution \\
    \bottomrule
    \end{tabular}
    \caption{\textbf{Related works on cell morphology generation.}}
    \label{tab:related}
\end{table*}
