\section{Related Works}
\label{sec:related}

\textbf{Generative models.}
Generative models are a fundamental class of machine learning approaches that learn to model and sample from probability distributions. Traditional methods such as autoregressive models, normalizing flows, and GANs face limitations in generation speed, expressiveness, or training stability~\cite{van2016pixel,papamakarios2021normalizing,goodfellow2014generative}. Recent score-based approaches, particularly diffusion models~\cite{sohl2015deep,song2019generative,ho2020denoising} and flow matching~\cite{lipman2022flow,lipman2024flow,liu2022flow,liu2024flowing}, address these challenges by learning continuous-time transformations between distributions, achieving state-of-the-art performance in generating images, videos, and biological sequences~\cite{openai2024gpt4technicalreport,esser2024scaling,pmlr-v235-kondratyuk24a,hayes2025simulating}. Unlike diffusion models, which map from Gaussian noise, flow matching directly transforms between arbitrary distributions. This property remains underexplored in machine learning due to limited application scenarios~\cite{liu2024flowing}, yet it is particularly well-suited for cellular morphology prediction, where accurately modeling the transition from unperturbed to perturbed cell states is crucial.

\textbf{Cellular morphology prediction.} Cellular morphology serves as a powerful phenotypic readout in biological research, offering critical insights into cellular states~\cite{perlman2004multidimensional,loo2007image}. Predicting morphological changes \textit{in silico} enables rapid virtual drug screening and the development of personalized therapeutic strategies, significantly accelerating biomedical discoveries~\cite{carpenter2007image,bunne2024build}. While initial progress has been made in this direction, existing approaches face three major limitations. Some neglect control cell images, failing to capture true perturbation changes and making predictions vulnerable to batch effects~\cite{yang2021mol2image,navidi2024morphodiff,cook2024diffusion}. Others rely on outdated generative techniques such as normalizing flows and GANs, which suffer from training instability and limited image fidelity~\cite{lamiable2023revealing,palma2023predicting}. Additionally, some methods use suboptimal approaches to model distribution transformation, such as a two-step diffusion process~\cite{bourou2024phendiff,hung2024lumic}. Our work addresses these challenges by reframing morphology prediction as a distribution-to-distribution translation problem and leveraging flow matching, which naturally models cellular state transformations while ensuring high image quality and stable training, paving the way for constructing virtual cells for biomedical research.  
