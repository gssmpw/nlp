In this work, we have presented \emph{MonoForce}, an explainable, physics-informed,
and end-to-end differentiable model that predicts the robot's trajectory from monocular camera images.
As a \emph{grey-box} model (physics-based and data-driven), it benefits from the end-to-end trainability
in different domains while it still retains the determinism of its physics engine and explainability
of its estimates of terrain properties.
Thanks to the method's end-to-end differentiability, it is possible to incorporate
external sensor measurements in the training pipeline.
Additionally, due to the inclusion of the physics information, our model is able not only to
generalize better (predict more accurate trajectories than data-driven baselines) but also provide interpretable intermediate outputs, for example
\emph{friction} and \emph{robot-terrain interaction forces}.
The training process is self-supervised; it only requires monocular camera images,
lidar scans, and SLAM-reconstructed trajectories.
The model learns to recognize non-rigid obstacles similar to those that have been driven over in the reference trajectories
while keeping the understanding and prediction capabilities for obstacles it has not encountered yet.
It treats them as rigid obstacles until future data prove otherwise.

In the experiments, we have shown that MonoForce generates accurate terrain heightmaps
that in turn serve as a basis for robot-terrain interaction force and trajectory estimates.
These estimates are valid both on rigid and non-rigid terrains.
For terrain prediction, our results are comparable to the lidar-based methods (VoxelNet).
We also study the benefits of sensor fusion on terrain estimation and robot-terrain interaction accuracy.
It was shown that for data-driven approaches, the inclusion of other sensor modalities helps
to predict the robot's trajectories more precisely.
In addition, it was demonstrated that our physics-based approach
($\nabla$Physics) outperforms  its data-driven baseline (TrajLSTM~\cite{pang2019aircraft})
in terms of trajectory estimation accuracy.
