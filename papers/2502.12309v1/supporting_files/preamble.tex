\usepackage[multiple]{footmisc} 
\usepackage[normalem]{ulem}
\usepackage{setspace}
\usepackage{amsmath}
\usepackage{amsfonts}
\usepackage{amssymb}
\usepackage{enumerate}
\usepackage{accents}
\usepackage{amsthm}
\usepackage{breakcites}
\usepackage{graphics,epsfig,verbatim,bm,latexsym,url,amsbsy}
\usepackage{rotating}
\usepackage{mathrsfs}
\usepackage{textgreek}
\usepackage{multirow}
\usepackage{graphicx}
\usepackage{subcaption}
\usepackage{array}
\usepackage{url}


\setlength{\columnsep}{0.2in}

\newcommand{\marginnote}[1]{\marginpar{\tiny{\begin{flushleft} {\color{RoyalPurple} #1} \end{flushleft}}}} 
\usepackage{graphicx}



\usepackage{float}
\usepackage{tikz}
\usepackage{caption}


\usepackage{bm}
\usepackage{tikz-cd}

\newcommand{\tr}{\mathsf{T}}
\newcommand{\concavePlanner}{\mathcal{V}}

\usepackage{pstricks, enumerate, pst-node, pst-text, pst-plot}
\usepackage{xparse}
\newcommand{\ind}[1]{{\mathbbm{1}_{\left\{{#1}\right\}}}}
\newcommand{\dd}[2]{{\left\langle {#1},{#2}\right\rangle}}

\DeclareMathOperator{\supp}{supp}
\DeclareMathOperator{\sgn}{sgn}

\def\cc{\curvearrowright}
\def\rd{\mathrm{d}}
\def\re{\mathrm{e}}

\usepackage{graphicx}
\graphicspath{{screenshots/}{images/}} %


\theoremstyle{definition} \newtheorem{remark}{Remark}
\theoremstyle{definition} \newtheorem{claim}{Claim}


\DeclareMathOperator{\interior}{int}
\DeclareMathOperator{\Cov}{Cov}


\long\def\symbolfootnote[#1]#2{\begingroup%
\def\thefootnote{\fnsymbol{footnote}}\footnote[#1]{#2}\endgroup}

\usepackage{footnote}
\makesavenoteenv{tabular}
\usepackage{fancyhdr}
\lhead{\textsc{\documenttitle}} \chead{} \rhead{\today\ / Page
\thepage\ of \pageref{lastpage}}
\newcommand{\documenttitle}{Thesis}

\newcommand{\argmin}{\operatornamewithlimits{argmin}}
\newcommand{\argmax}{\operatornamewithlimits{arg \, max}}
\renewcommand{\max}{\operatornamewithlimits{max}}
\renewcommand{\min}{\operatornamewithlimits{min}}


\newcommand{\be}{\begin{equation}}
\newcommand{\ee}{\end{equation}}
\newcommand{\bes}{\begin{equation*}}
\newcommand{\ees}{\end{equation*}}
\newcommand{\mbf}{}



\usepackage{amsmath}
\usepackage{epsfig,graphics}
\usepackage{etoolbox}

\usepackage{accents}



\newcommand{\change}[1]{\widehat{#1}}

\newcommand{\Q}{\mathbb Q}
\newcommand{\Z}{\mathbb Z}
\newcommand{\R}{\mathbb R}
\renewcommand{\Pr}{\mathbb P}
\renewcommand{\Im}{\text{Im}}


\newcommand{\rdots}{\mathinner{%
  \mkern1mu\raise1pt\hbox{.}%
  \mkern2mu\raise4pt\hbox{.}%
  \mkern2mu\raise7pt\vbox{\kern7pt\hbox{.}}\mkern1mu}}


\DeclareMathOperator{\trace}{trace}
\DeclareMathOperator{\Det}{Det}
\DeclareMathOperator{\FLP}{FLP}
\DeclareMathOperator{\DWH}{DWH}
\DeclareMathOperator{\Mat}{Mat}
\DeclareMathOperator{\Var}{Var}
\DeclareMathOperator{\plimPreliminary}{plim}
\DeclareMathOperator{\var}{var}
\DeclareMathOperator{\straightE}{E}

\usepackage{etex}
\DeclareFontFamily{U}{mathx}{\hyphenchar\font45}
\DeclareFontShape{U}{mathx}{m}{n}{
      <5> <6> <7> <8> <9> <10>
      <10.95> <12> <14.4> <17.28> <20.74> <24.88>
      mathx10
      }{}
\DeclareSymbolFont{mathx}{U}{mathx}{m}{n}
\DeclareFontSubstitution{U}{mathx}{m}{n}
\DeclareMathAccent{\widecheck}{0}{mathx}{"71}
\DeclareMathAccent{\wideparen}{0}{mathx}{"75}



\newcommand{\Ex}{ E}
\newcommand{\N}{ N}
\renewcommand{\Pr}{ P}
\newcommand{\cent}{c}
\newcommand{\centstat}{c^*}
\newcommand{\vc}{v}
\newcommand{\eye}{{I}}
\newcommand{\plim}{\plimPreliminary \displaylimits}                                               %

\usepackage{palatino}




\makeatletter
\def\@footnotecolor{purple!30!blue}
\define@key{Hyp}{footnotecolor}{%
\HyColor@HyperrefColor{#1}\@footnotecolor%
}
\patchcmd{\@footnotemark}{\hyper@linkstart{link}}{\hyper@linkstart{footnote}}{}{}
\makeatother
\hypersetup{footnotecolor=black}


\usepackage{thmtools}
\usepackage{nameref}
\usepackage[nameinlink]{cleveref}
\declaretheorem[name=Theorem,
  refname={theorem,theorems},
  Refname={Theorem,Theorems}]{callmeal}


\newcommand\nc{\newcommand}
\nc\on{\operatorname}
\theoremstyle{definition} \newtheorem{thm}{Theorem}
\theoremstyle{definition} \newtheorem*{thmNoNum}{Theorem}
\theoremstyle{definition} \newtheorem{definition}{Definition}
\theoremstyle{definition} \theoremstyle{remark}
\theoremstyle{definition} \newtheorem{note}{Note}
\theoremstyle{definition} \newtheorem{notation}{Notation}
\theoremstyle{definition} \theoremstyle{plain}
\theoremstyle{definition} \newtheorem{condition}{Condition}
\theoremstyle{definition} \newtheorem{lem}{Lemma}
\theoremstyle{definition} \newtheorem{corollary}{Corollary}
\theoremstyle{definition} \newtheorem{prop}{Proposition}
\theoremstyle{definition} \newtheorem{ax}{Axiom}
\theoremstyle{definition} \newtheorem{assumption}{Assumption}





\theoremstyle{definition} \newtheorem{example}{Example}
\theoremstyle{definition} \newtheorem{rmk}{Remark}
\newtheorem*{PropertyLIN}{Property LIN}
\newtheorem*{PropertyA}{Property NSD}
\theoremstyle{definition} \newtheorem{res}{Result}
\theoremstyle{definition} \newtheorem{que}{Question}
\theoremstyle{definition} \newtheorem{conjecture}{Conjecture}
\newtheorem{fact}{Fact}

\theoremstyle{definition}
\newtheorem{exmp}{Example}[subsection]



