\documentclass{article}



\usepackage{arxiv}

\usepackage[utf8]{inputenc} % allow utf-8 input
\usepackage[T1]{fontenc}    % use 8-bit T1 fonts

\usepackage{url}            % simple URL typesetting
\usepackage{booktabs}       % professional-quality tables
\usepackage{amsfonts}       % blackboard math symbols
\usepackage{nicefrac}       % compact symbols for 1/2, etc.
\usepackage{microtype}      % microtypography
\usepackage{lipsum}		% Can be removed after putting your text content
\usepackage{graphicx}
\usepackage[numbers,square]{natbib}
\usepackage{hyperref}       % hyperlinks
\usepackage{doi}
\usepackage{amsmath} 
\usepackage{float}


\title{Emotion Recognition and Generation: A Comprehensive Review of Face, Speech, and Text Modalities}


%\date{September 9, 1985}	% Here you can change the date presented in the paper title
%\date{} 					% Or removing it

\author{ \href{https://orcid.org/0009-0005-9159-6528}{\includegraphics[scale=0.06]{orcid.pdf}\hspace{1mm}Rebecca Mobbs} \\
	School of Computer Science and Mathematics\\
	Kingston University\\
	London \\
	\texttt{k2369889@kingston.ac.uk} \\
	%% examples of more authors
	\And
	\href{https://orcid.org/0000-0001-6170-0236}{\includegraphics[scale=0.06]{orcid.pdf}\hspace{1mm}Dimitrios Makris} \\
	School of Computer Science and Mathematics\\
	Kingston University\\
	London \\
	\texttt{d.makris@kingston.ac.uk} \\
    	\And
	\href{https://orcid.org/0000-0003-4679-8049}{\includegraphics[scale=0.06]{orcid.pdf}\hspace{1mm}Vasileios Argyriou} \\
	School of Computer Science and Mathematics\\
	Kingston University\\
	London \\
	\texttt{vasileios.argyriou@kingston.ac.uk} \\
	%% \AND
	%% Coauthor \\
	%% Affiliation \\
	%% Address \\
	%% \texttt{email} \\
	%% \And
	%% Coauthor \\
	%% Affiliation \\
	%% Address \\
	%% \texttt{email} \\
	%% \And
	%% Coauthor \\
	%% Affiliation \\
	%% Address \\
	%% \texttt{email} \\
}

% Uncomment to remove the date
%\date{}

% Uncomment to override  the `A preprint' in the header
%\renewcommand{\headeright}{Technical Report}
%\renewcommand{\undertitle}{Technical Report}


%%% Add PDF metadata to help others organize their library
%%% Once the PDF is generated, you can check the metadata with
%%% $ pdfinfo template.pdf
\hypersetup{
pdftitle={A template for the arxiv style},
pdfsubject={q-bio.NC, q-bio.QM},
pdfauthor={David S.~Hippocampus, Elias D.~Striatum},
pdfkeywords={First keyword, Second keyword, More},
}

\begin{document}
\maketitle

\begin{abstract}
Emotion recognition and generation have emerged as crucial topics in Artificial Intelligence research, playing a significant role in enhancing human-computer interaction within healthcare, customer service, and other fields. Although several reviews have been conducted on emotion recognition and generation as separate entities, many of these works are either fragmented or limited to specific methodologies, lacking a comprehensive overview of recent developments and trends across different modalities. In this survey, we provide a holistic review aimed at researchers beginning their exploration in emotion recognition and generation. We introduce the fundamental principles underlying emotion recognition and generation across facial, vocal, and textual modalities. This work categorises recent state-of-the-art research into distinct technical approaches and explains the theoretical foundations and motivations behind these methodologies, offering a clearer understanding of their application. Moreover, we discuss evaluation metrics, comparative analyses, and current limitations, shedding light on the challenges faced by researchers in the field. Finally, we propose future research directions to address these challenges and encourage further exploration into developing robust, effective, and ethically responsible emotion recognition and generation systems. 
\end{abstract}


% keywords can be removed
\keywords{Artificial Intelligence \and AI \and Generative AI \and Emotion Recognition \and Sentiment Recognition \and Face Emotion Recognition \and Facial Expression Recognition \and Speech Emotion Recognition \and Text Emotion Recognition \and Text Sentiment Recognition \and Survey \and Speech to Animation \and Speech to Speech \and Text Generation \and Large Language Models \and Facial Expression Generation \and Speech Emotion Generation \and Text Emotion Generation \and Survey \and Review}


\section{Introduction}

Emotions are central to human communication, shaping interactions through body language, facial expressions, vocal intonations, and textual cues \cite{tian2001recognizing}. Psychological research suggests recognition of emotions is innate in humans, with newborns able to replicate facial expressions and vocal tones as early as two days old \cite{johnson2021face}. Understanding emotions aids in teamwork and cooperation, a concept recognised by Darwin's theories on survival mechanisms \cite{darwin1998expression}. This significance has led to the development of emotion models like Ekman and Friesen’s Facial Action Coding System (FACS), which categorises emotions such as anger, disgust, fear, happiness, sadness, surprise, and contempt \cite{ekman1971constants,ekman1994strong}, forming the basis for many contemporary emotion recognition systems.

As interest in artificial intelligence (AI) grows, emotion recognition and generation technologies have gained traction in fields such as healthcare, customer service, education, and entertainment \cite{einfochips2024website,elevateai2024website,brandt2024facial,huang2023mental,ramis2020using,tang2020classroom}. AI systems can now analyse and simulate emotional responses, allowing machines to engage in more meaningful human-computer interactions. Emotion recognition is used in applications such as driver fatigue detection \cite{liu2020driver} and lie detection \cite{rizwan2020accurate}, while generative models create realistic emotional content in apps like FaceApp \cite{sansar2023societies}, HeadSpace \cite{wasil2022there}, and Wysa \cite{beatty2022evaluating}.

This survey provides a comprehensive review of State-of-the-Art (SOTA) methodologies in AI for emotion recognition and emotion generation, addressing the gap in the literature regarding the integration of these two domains and their applications across multiple modalities. The generation of emotions on faces, Facial Expression Generation (FEG) systems, are termed in the literature as Talking Face or Speech/Text-to-Animation models, while Speech Emotion Generation (SEG) involves Speech-to-Speech or Speech Reenactment methods, and Text Sentiment Generation (TSG) relies on Large Language Models (LLMs). Existing reviews have typically focused on either emotion recognition \cite{Adyapady2023,Deng2023,al2023speech} or emotion generation \cite{cao2023comprehensive,harshvardhan2020comprehensive}, without addressing their intersection. Additionally, Facial Expression Recognition (FER) and FEG have not yet been discussed alongside Speech Emotion Recognition (SER), SEG, or Text Sentiment Recognition (TSR). Research tends to prioritise facial systems due to heightened public interest and the relative ease with which facial expressions are interpreted by both humans and machines \cite{li_2020}. These systems also benefit from extensive pretrained models and datasets derived from computer vision research \cite{toisoul2021estimation}. By exploring both emotion recognition and generation across modalities, this survey aims to offer insights into current techniques, highlight areas for improvement, and guide future research directions.

This survey is structured to provide a holistic examination of the field. Section 1.1 explores various applications of emotion recognition and generation models. Section 2 discusses preprocessing techniques to improve model accuracy and efficiency. Section 3 reviews the datasets commonly used, detailing their characteristics. Sections 4 and 5 present state-of-the-art methods for emotion recognition and generation, respectively, across faces, speech, and text. Section 5.4 discusses emotion control methods accross modalities. Section 6 provides a comparative analysis of evaluation metrics to assess SOTA performance. Section 7 outlines current challenges and future research directions. Finally, Section 9 concludes with a synthesis of key findings and contributions to the development of emotion recognition and generation technologies.

\subsection{Applications} %mention these systems would be good for use with people with hearing or speech impairments
Emotion recognition systems are used across various fields. In customer service, they are utilised to discern customers' emotions and evaluate the effectiveness of sales assistants' communication strategies through assessment of transcripts \cite{elevateai2024website}. Similarly, at self-service checkouts, FER is used to gauge customer satisfaction based on their facial cues \cite{einfochips2024website}. In healthcare, these systems assist in tracking the progression of Alzheimer’s disease \cite{brandt2024facial}, facilitating therapy sessions \cite{huang2023mental}, and supporting individuals with Asperger’s Syndrome in recognising emotions \cite{banerjee2023training}. They are also used in robotics to interpret human emotions during interactions with machines \cite{ramis2020using}, and in educational settings to evaluate students' engagement and learning \cite{tang2020classroom}. Other applications include lie detection \cite{rizwan2020accurate} and monitoring driver fatigue levels \cite{liu2020driver}.

Emotion recognition systems can also serve as foundational tools for training models capable of generating realistic emotional content \cite{li_2020}. These models can be used to create visual virtual assistants and avatars for virtual calls \cite{yasuoka2022effects}. As reliance on chatbots for social interactions and advice increases \cite{van2021m}, there is a growing opportunity for the development of talking head chatbots. Such chatbots would use speech or text input—whether from a customer service representative, therapist, or a text generation model—to produce animated faces with lifelike emotions in real-time. These animated avatars could integrate with AI models such as Character.AI \cite{characterAI}, ChatGPT \cite{achiam2023gpt}, Llama \cite{llama}, or Gemini \cite{team2023gemini} to function as therapeutic or customer service bots. This technology has the potential to provide users with a highly immersive and personalised experience, enhancing or even replacing current customer service chatbots. 

\section{Preprocessing for ER and EG Systems}
Preprocessing is an important stage in deep learning pipelines, particularly when handling data obtained from uncontrolled or 'in-the-wild' environments, such as facial and speech data extracted from movies or textual data from social media. Such data often exhibit significant variability compared to controlled laboratory settings, with variations in background, lighting, noise, and other artefacts. To address these challenges, preprocessing typically involves standard steps like data normalisation, noise reduction, and feature extraction to ensure data consistency and optimise model performance. Below, we explore the specific preprocessing techniques used for processing face, speech, and textual data.

\subsection{Preprocessing for Face Systems}
Preprocessing for facial emotion recognition systems  aims to enhance image quality, standardise data, and extract critical features for accurate model predictions. The initial step involves resizing and cropping facial images to create uniform input dimensions, ensuring consistency across the dataset. By eliminating background elements and focusing on the region of interest, these techniques enable models to concentrate on key facial features. Normalisation, through scaling pixel values to a common range (e.g., 0 to 1 or -1 to 1), ensures uniform pixel intensity across different samples, thereby enhancing the model’s capacity to learn relevant patterns. Common methods such as mean subtraction \cite{Krizhevsky2012} and standard deviation normalisation \cite{LeCun1998} are frequently used. Noise reduction techniques, like Gaussian blurring \cite{Gonzalez2002} and median filtering \cite{Tukey1977}, are used to minimise the impact of noise introduced during image acquisition or transmission.

Techniques such as histogram equalisation \cite{Pizer1987} improve contrast by redistributing pixel intensities, enhancing visibility in images captured under challenging conditions. Data augmentation, involving transformations like rotation, scaling, and flipping, increases training data diversity and mitigates overfitting \cite{Shorten2019}. Furthermore, advanced algorithms such as Haar cascades \cite{Viola2001} and deep learning-based facial landmark detection methods \cite{Zhang2014} are applied to extract and align facial regions, standardising poses and reducing variability. Feature extraction models, such as VGG \cite{Simonyan2014}, ResNet \cite{He2016}, and MobileNet \cite{Howard2017}, are widely used for extracting high-level features. Colour space transformations and quality control measures help streamline data preparation, ensuring only high-quality data is fed into the models \cite{Gonzalez2002, Wang2004}.

\subsection{Preprocessing for Speech Systems}
The primary goals of preprocessing in speech systems are noise reduction, normalisation, segmentation, and feature extraction from raw audio signals. Noise reduction methods like spectral subtraction \cite{Boll1979}, Wiener filtering \cite{Lim1979}, and adaptive filtering \cite{Widrow1985} are used to eliminate background noise which can degrade speech signal quality. Normalisation adjusts amplitude and dynamic range to maintain consistency across recordings \cite{Rabiner1993}. Speech segmentation techniques, such as endpoint detection \cite{Rabiner1975} and silence removal \cite{Sadjadi2013}, isolate speech segments within continuous audio streams, enabling more targeted analysis.

Feature extraction captures the salient characteristics of speech, using Mel-Frequency Cepstral Coefficients (MFCCs) \cite{Davis1980}, which represent spectral properties in a compact form, and Linear Predictive Coding (LPC) \cite{Makhoul1975}, which models the spectral envelope. Other methods like pitch estimation \cite{Boersma1993} and anti-aliasing filtering \cite{Crochiere1983} help preserve signal integrity. Techniques such as de-reverberation \cite{naylor2010speech} and pre-emphasis \cite{o1988speech} further refine the signal quality. For segmentation, windowing techniques like frame blocking divide speech signals into shorter frames, facilitating computational efficiency \cite{hamid2018frame}. Mean and variance normalisation standardises feature scales, improving model robustness to variability in input data \cite{viikki1998recursive}.

\subsection{Preprocessing for Text Systems}
Text preprocessing begins with tokenisation, which breaks down text into smaller units, such as words or characters. This is followed by lowercasing, which standardises the text by treating uppercase and lowercase versions of words identically, thereby reducing vocabulary size and simplifying the learning process \cite{Mikolov2013}. Punctuation and special character removal further eliminate noise which could interfere with learning. Stopwords—such as “and” or “the”—are often removed, as they carry little semantic value \cite{Manning2008}. Stemming and lemmatisation techniques group words with similar meanings, helping models understand linguistic variations \cite{Porter1980, Bird2009}.

% \clearpage % Ensures everything before this point is completed on previous pages
% 
\begin{table}[H]
\small
\centering
\begin{tabular}{|p{3cm}|p{5cm}|p{2cm}|p{2cm}|p{3cm}|}
\hline
\textbf{Name} & \textbf{Description} & \textbf{Type} & \textbf{Size} & \textbf{Emotions} \\ \hline

AffectNet & Extensive facial imagery dataset annotated with discrete and continuous emotion labels. & Image & 450,000 images & Surprise, fear, disgust, happiness, sadness, anger, neutral, contempt \\ \hline

RAF-DB & Diverse facial expression dataset featuring multiple genders, ages, and ethnicities. & Image & 29,672 images & Surprise, fear, disgust, happiness, sadness, anger, neutral \\ \hline

FERPlus & Derived from the FER2013 dataset, enhancing expression annotations through crowdsourcing. & Image & Unlimited & Surprise, fear, disgust, happiness, sadness, anger, neutral, contempt \\ \hline

AFEW & High-resolution videos from YouTube with over 300 subjects and 10,000 sentences. & Video & 16 hours & Surprise, fear, disgust, happiness, sadness, anger, neutral \\ \hline

HDTF & Video clips gathered from TV shows and movies, including various head poses and occlusions. & Video & 1,809 clips & Surprise, fear, disgust, happiness, sadness, anger, neutral \\ \hline

AFEW-VA & Video clips annotated for valence and arousal levels, with 68 facial landmarks per frame. & Video & 600 clips & Surprise, fear, disgust, happiness, sadness, anger, neutral \\ \hline

DFEW & Facial expression dataset created from more than 1,500 movies. & Video & 12,059 clips & Happiness, anger, sadness, fear, disgust, surprise, neutral \\ \hline

CK+ & Laboratory-controlled video data capturing transitions from neutral to peak expression. & Video & 593 sequences & Surprise, fear, disgust, happiness, sadness, anger, contempt \\ \hline

MEAD & High-resolution emotional audiovisual dataset with 60 actors. & Video \& audio & 16,800 hours & Surprise, fear, disgust, happiness, sadness, anger, contempt \\ \hline

LRW & Video sequences of people speaking words in uncontrolled conditions. & Video \& audio & 1,000 utterances & Unlabeled \\ \hline

LibriTTS & Multi-speaker English corpus of read speech at 24kHz for TTS research. & Audio & 585 hours & Unlabeled \\ \hline

VCC2018 & Dataset for speech-to-speech systems, consisting of male and female speakers. & Audio & 464 sentences & Unlabeled \\ \hline

ESD & Collection of audio recordings for studying emotions expressed through speech. & Audio & 7,000 utterances & Neutral, happy, angry, sad, surprise \\ \hline

Empathetic Dialogues & Open-domain conversations between speakers and listeners for empathic responses. & Audio & 24,850 conversations & 32 emotion labels \\ \hline

EMO-DB & German emotional speech recorded by ten professional speakers. & Audio & 535 utterances & 7 emotions \\ \hline

CASIA & Mandarin emotional speech dataset. & Audio & 1,200 snippets & 6 emotions \\ \hline

Amazon Reviews & Large dataset of product reviews provided by Amazon. & Text & Unlimited & - \\ \hline

Twitter & Collection of tweets for social media text analysis. & Text & Unlimited & - \\ \hline

Reddit & Comments and posts from Reddit for understanding informal language. & Text & Unlimited & - \\ \hline

\end{tabular}
\caption{Datasets for ER and EG Systems}
\label{tab:datasets}
\end{table}

% \clearpage % Ensures that the content after this point starts on a new page


Numerical values are encoded or replaced with placeholders to maintain the semantic integrity of the text \cite{Ghosh1994}. Out-of-vocabulary words are managed through tokenisation or character-level representations \cite{Pennington2014}, while padding and truncation ensure uniform sequence lengths, which is crucial for text classification \cite{Goldberg2017}. Pretrained word embeddings, such as Word2Vec \cite{Mikolov2013_2}, can be used to initialise the embedding layers of deep learning models or be fine-tuned during training. Encoding methods like one-hot or integer encoding convert textual data into numerical representations, while pretrained tokenisers accelerate this conversion \cite{Johnson2017}. Text augmentation techniques, such as synonym replacement and paraphrasing, diversify training data and reduce overfitting, improving generalisation \cite{Wei2019}.

\section{Datasets for Face, Text, and Speech ER and EG Systems}
High-quality, diverse datasets are essential for training emotion recognition and generation models. These datasets provide labelled examples from facial expressions, speech, and text, enabling models to learn emotional cues in varied contexts. Some datasets are captured in controlled environments, while others are collected in the wild, offering more complex real-world variations. This section highlights the most widely used datasets across facial, speech, and text systems, focusing on those with comprehensive emotional labelling and diversity (see \ref{tab:datasets}). 

\begin{table}[h]
\small
\centering
\begin{tabular}{|p{3cm}|p{5cm}|p{2cm}|p{2cm}|p{3cm}|}
\hline
\textbf{Name} & \textbf{Description} & \textbf{Type} & \textbf{Size} & \textbf{Emotions} \\ \hline

AffectNet & Extensive facial imagery dataset annotated with discrete and continuous emotion labels. & Image & 450,000 images & Surprise, fear, disgust, happiness, sadness, anger, neutral, contempt \\ \hline

RAF-DB & Diverse facial expression dataset featuring multiple genders, ages, and ethnicities. & Image & 29,672 images & Surprise, fear, disgust, happiness, sadness, anger, neutral \\ \hline

FERPlus & Derived from the FER2013 dataset, enhancing expression annotations through crowdsourcing. & Image & Unlimited & Surprise, fear, disgust, happiness, sadness, anger, neutral, contempt \\ \hline

AFEW & High-resolution videos from YouTube with over 300 subjects and 10,000 sentences. & Video & 16 hours & Surprise, fear, disgust, happiness, sadness, anger, neutral \\ \hline

HDTF & Video clips gathered from TV shows and movies, including various head poses and occlusions. & Video & 1,809 clips & Surprise, fear, disgust, happiness, sadness, anger, neutral \\ \hline

AFEW-VA & Video clips annotated for valence and arousal levels, with 68 facial landmarks per frame. & Video & 600 clips & Surprise, fear, disgust, happiness, sadness, anger, neutral \\ \hline

DFEW & Facial expression dataset created from more than 1,500 movies. & Video & 12,059 clips & Happiness, anger, sadness, fear, disgust, surprise, neutral \\ \hline

CK+ & Laboratory-controlled video data capturing transitions from neutral to peak expression. & Video & 593 sequences & Surprise, fear, disgust, happiness, sadness, anger, contempt \\ \hline

MEAD & High-resolution emotional audiovisual dataset with 60 actors. & Video \& audio & 16,800 hours & Surprise, fear, disgust, happiness, sadness, anger, contempt \\ \hline

LRW & Video sequences of people speaking words in uncontrolled conditions. & Video \& audio & 1,000 utterances & Unlabeled \\ \hline

LibriTTS & Multi-speaker English corpus of read speech at 24kHz for TTS research. & Audio & 585 hours & Unlabeled \\ \hline

VCC2018 & Dataset for speech-to-speech systems, consisting of male and female speakers. & Audio & 464 sentences & Unlabeled \\ \hline

ESD & Collection of audio recordings for studying emotions expressed through speech. & Audio & 7,000 utterances & Neutral, happy, angry, sad, surprise \\ \hline

Empathetic Dialogues & Open-domain conversations between speakers and listeners for empathic responses. & Audio & 24,850 conversations & 32 emotion labels \\ \hline

EMO-DB & German emotional speech recorded by ten professional speakers. & Audio & 535 utterances & 7 emotions \\ \hline

CASIA & Mandarin emotional speech dataset. & Audio & 1,200 snippets & 6 emotions \\ \hline

Amazon Reviews & Large dataset of product reviews provided by Amazon. & Text & Unlimited & - \\ \hline

Twitter & Collection of tweets for social media text analysis. & Text & Unlimited & - \\ \hline

Reddit & Comments and posts from Reddit for understanding informal language. & Text & Unlimited & - \\ \hline

\end{tabular}
\caption{Datasets for ER and EG Systems}
\label{tab:datasets}
\end{table}

\section{Emotion Recognition for Faces, Speech, and Text}
This section will discuss deep learning methodologies for emotion recognition for faces, speech, and text. We will discuss the strengths and limitations of current literature. Most emotion recognition systems use the 8 primary emotions anger, disgust, fear, happiness, sadness, surprise, contempt, and neutral \cite{ekman1994strong}. Unlike traditional methodologies where feature extraction and classification are treated as distinct stages \cite{Schuller2010}, deep learning frameworks for emotion detection enable end-to-end pipelines. A key component in classification is the use of a loss layer, which regulates the back-propagation error, for estimating prediction probabilities for each sample. For example, in CNNs the softmax loss function is typically used to minimise the difference between the predicted class probabilities and the ground-truth. Some models simultaneously predict both discrete emotions and continuous affect dimensions, such as arousal, valence, and strength of emotion \cite{toisoul2021estimation} (see Fig.\ref{emofan}). This aims to minimise data mislabelling and improve overall prediction accuracy. 


 \begin{figure}[hbt!] 

     \centering 

   %  \includegraphics[width=\textwidth]{emofan.jpg} 
     \includegraphics[width=\textwidth]{emofan2.jpg} 

     \caption{The EmoFAN pipeline integrates facial landmark detection, discrete emotion classification, and continuous valence-arousal estimation in a single neural network. This unified model performs all tasks in one pass, using a face-alignment network and an attention mechanism to focus on key facial regions, enhancing accuracy. Joint prediction of both emotion types, combined with knowledge distillation, improves robustness.\cite{toisoul2021estimation}} 

     \label{emofan} 

 \end{figure} 


 

 \subsection{Facial Expression Recognition}
FER systems begin with facial feature detection, whereby the face is identified and isolated. Methods such as the Viola-Jones algorithm, Histogram of Oriented Gradients (HOG), and Convolutional Neural Networks (CNNs) are used. Facial landmark detection identifies key points on the face, then feature extraction focuses on geometric features and appearance features. Traditional machine learning algorithms and deep learning models, especially CNNs, classify these features into emotional categories. CNNs are effective as they automatically learn and extract hierarchical features from raw pixel data \cite{LeCun2015}. The following section will discuss state-of-the-art research in FER with an emphasis on novelty, recurring themes, strengths, and limitations of current research. 

FER systems are classified into two categories: static image and dynamic sequence. While static methods encode spatial information from individual images, dynamic techniques use temporal relationships across frames within sequences \cite{Adyapady2023}. Historically, FER heavily relied on handcrafted features or shallow learning techniques such as Decision Trees \cite{Littlewort2011}, K-Nearest Neighbors (K-NN) \cite{Zhang2011}, and Support Vector Machines (SVM) \cite{Bartlett2006}. However, with the rise in emotion detection competitions such as FER2013 \cite{goodfellow2013challenges}, EMOCA \cite{Danecek2022}, and ABAW 2023 \cite{kollias_2023} a shift towards the use of deep learning techniques occurred. This has coincided with improvements in processing capabilities and network architectures, enabling the widespread adoption of deep learning methodologies. 

Models using pretrained Contrastive Language-Image Pretrained (CLIP) \cite{Radford2021clip} achieve remarkable results in FER. Using the joint embedding space of text and images, CLIP models can understand contextual information across modalities. By training on large datasets containing images paired with descriptions of emotions, CLIP learns to associate visual patterns with their emotional description. One such model which uses CLIP is DFER-CLIP \cite{zhao2023prompting}. This method combines both modalities, using a temporal model atop the CLIP image encoder. Temporal facial features are captured while using descriptions of facial behaviour instead of class names for the text encoder. It uses learnable prompts as context for descriptors of each facial expression class, enabling automatic learning of relevant context information during training. The model's pipeline involves extracting features from facial images or frames, and predicting facial expression descriptions. Furthermore, DFER-CLIP automates the generation of textual descriptors by prompting a language model with queries about useful visual features for each expression, culminating in comprehensive descriptions for classification. 

Attention is a key topic in FER with approaches such as self-attention, patch attention, and cross attention being utilised. EmoFan (see Fig.\ref{emofan}) uses attention mechanisms on facial landmarks and facial heat maps and achieves SOTA results. \cite{Liu_2023} uses patch attention and a pretrained ResNet-18 to extract the facial feature maps to overcome issues caused by occlusion for improved performance. \cite{Mao_2023} uses a similar approach by making use of window-based cross-attention mechanisms in conjunction with landmark detection, and multi-scale feature extraction. In comparison, \cite{gong2024enhanced} uses self-attention and a transformer to identify facial expressions in images or videos where the face is difficult to see. \cite{Danecek2022} addresses a shortfall in labelled datasets by incorporating an emotion recognition model into the 3D face reconstruction framework DECA\cite{tewari2020} This enables improved emotion reconstruction and classification, along with the use of their Emotion Consistency Loss. 

%The use of pretrained facial alignment models for facial landmark detection \cite{toisoul2021estimation}, or the use of 3D modelling to help with reenacting facial expressions \cite{danecek_2022} are two such approaches. Liu et al (2023) on the other hand utilized a modified pretrained ResNet-18 for the backbone of their model \cite{Liu_2023}. Poster++ \cite{Mao_2023} modified the architecture of Poster \cite{Zheng2023}, to address inter-class similarity and intra-class discrepancy in FER, and making a faster and simpler model. 

\subsection{Speech Emotion Recognition}

   \begin{figure}[hbt!] 

     \centering 

     \includegraphics[width=\textwidth]{xie_ser.jpg} 

     \caption{The SER model processes frame-level speech features as input, using a 2-layer LSTM to generate outputs aligned with each frame's corresponding time. The LSTM's internal forget gate has been replaced by an attention gate. To differentiate emotional nuances across time and feature dimensions, the model applies a weighting operation separately on the LSTM's output along both the time and feature dimensions. These two weighted outputs are then fed into fully connected layers, and the final output from the softmax layer provides the classification result.\cite{Xie2023}} 

     \label{xie} 

 \end{figure} 
Recognising emotions in speech involves a multidisciplinary approach, integrating linguistics, psychology, and computer science \cite{Schuller2003}. Acoustic feature analysis, focusing on prosody and voice quality, plays a key role. Prosodic features, such as pitch, intensity, and speech rate, effectively indicate emotions. For example, happiness or excitement use higher pitch and greater variability, while sad voices use lower pitch and slower speech. Voice quality, including elements such as breathiness and tension, can also signal different emotions. Word choices and sentence structures, provide additional clues. Short, abrupt sentences can indicate anger, while longer, complex sentences might suggest calmness. Contextual analysis, considering the situational context and dialog history, is vital, as the same utterance can convey different emotions depending on the context \cite{Scherer2003}. 

Transformer based model ESCM \cite{Yang2023}, achieved state-of-the-art results in SER by adjusting emotions and semantics based on context. They achieve this by using Graph Convolutional Network (GCN) to find correlations between words in spoken coversations. In contrast, \cite{Xie2023} (see Fig.\ref{xie}) introduces a novel approach to speech emotion recognition by integrating attention mechanisms into Long Short Term Memory (LSTM) models. By prioritising relevant information across both time and feature dimensions, the attention-based LSTM architecture improves performance in SER. The use of frame-level features provide a comprehensive representation of emotional content, contributing to the model's accuracy. \cite{gong2024enhanced} use Large Language Models (LLMs) and weakly-supervised learning to label the emotions in speech data, which contributes to the effectiveness of their SER model.

Further innovations in time-frequency analysis have also improved SER. For instance, the fast Continuous Wavelet Transform (fCWT) enables high-resolution analysis of non-stationary speech signals, balancing temporal and spectral features. When combined with Deep Convolutional Neural Networks (DCNNs), this approach enhances the extraction of paralinguistic information, offering robust real-time performance while overcoming limitations of traditional methods like the Short-Term Fourier Transform (STFT) \cite{van2023speech}.

\subsection{Text Sentiment Recognition}
  \begin{figure}[hbt!] 

     \centering 

     \includegraphics[width=\textwidth]{TER_raman.jpg} 

     \caption{The TER system by\cite{Kumar2022} uses a BERT-based dual-channel pipeline for text emotion recognition. First, input sentences are converted into contextual embeddings with a pre-trained BERT model. These embeddings are then processed through two parallel channels: one uses CNN for feature extraction followed by BiLSTM for capturing sequence information, while the other uses BiLSTM first, followed by CNN. The outputs from both channels are concatenated and passed through dense layers for emotion classification. An explainability module further interprets the model's predictions by analysing emotion embedding clusters.} 

     \label{kumar} 

 \end{figure} 
TER focuses on the identification and classification of emotions expressed in textual data using Natural Language Processing models (NLP). NLP models enable machines to understand, interpret, and generate text \cite{Jurafsky2009}. Bidirectional Encoder Representations from Transformers (BERT) \cite{Devlin2019} are used in most modern NLP models \cite{Vaswani2017}. These models are useful for TER due to their ability to capture contextual data and decipher emotions in text, enabling SOTA performance. Campagnano et al. \cite{Campagnano2022semantic} combines BERT encodings with bidirectional LSTM layers to achieve robust emotion classification, particularly in semantic role labelling tasks. \cite{Koptyra2023} use a modified BERT-based architecture to classify emotions for individual sentences and entire texts. \cite{Bianchi2022} use a BERT model trained on data from 100 languages as well as X (formerly Twitter), to detect emotions on social media platforms. In contrast, \cite{Kumar2022} (see Fig.\ref{kumar}) use LSTM and a CNN based model for TER. The use of CNN-LSTM channels extracts both local and global contextual information from input text, working for diverse text inputs. \cite{Koptyra2023, Bianchi2022} address multilingual emotion recognition, developing models and datasets capable of working across languages. As seen in this analysis there is a distinct lack of recent research into TER, highlighting the need for updated studies to address current challenges and advancements in the field.

\section{Emotion Generation for Faces, Speech, and Text}


This section will discuss generated content for faces - which will focus on animated face generation, speech - taking the nuances of audio from one speaker and converting to another voice, and text - the generation of realistic text. Emotion recognition models are sometimes used for training \cite{Li2020dual}, and evaluating \cite{Schuller2004} these models to generate accurate emotional content. Emotion recognition datasets are also utilised for emotion generation models \cite{Zeng2020}. A recent challenge with creating emotionally realistic generated content comes from negativity in public's perception due to media hype surrounding stealing of identities \cite{Matton2019}, deepfakes \cite{Dolhansky2020}, and the rapid rate in which models are being released \cite{Radford2021clip}. This consideration has the capacity to hinder research in these fields due to restrictions on the availability of models for researchers \cite{Goodfellow2019}, due to the fear they will fall into the wrong hands. This section will discuss SOTA methods for these modalities, and will discuss the strengths and limitations of current research.

\subsection{Facial Expression Generation}
%overall
 \begin{figure}[hbt!] 

     \centering 

     \includegraphics[width=\textwidth]{emote_portrait.jpg} 

     \caption{
 In the place of 3D modelling, EMO utilising Stable Diffusion for generating new frames. The  pipeline consists of a Backbone Network paired with a ReferenceNet to maintain identity consistency, audio-attention layers to synchronise facial expressions with audio tonalities, and temporal modules to ensure smooth transitions across frames. Weak control signals, such as a Face Locator and Speed Layers, provide loose guidance for facial positioning and movement velocity, achieving natural and stable head motions across clips.\cite{Tian2024emo}} 

     \label{Tian} 

 \end{figure} 



FEG and face manipulation techniques have been around for years, present on mobile phone apps such as Instagram \cite{instagram}, SnapChat \cite{snapchat}, and AI photo editors such as FaceApp \cite{faceapp} and others. The release of visually appealing talking-head models such as VASA \cite{xu2024vasa} and EMO-Live \cite{Tian2024emo}, have further bolstered public interest in this research area. Talking-head animation refers to models which take as input an image of a person, and generates new frames using audio \cite{Tian2024emo} (see Fig.\ref{Tian}), video \cite{paraperas2021emo}, or text \cite{li_2020} to guide the facial expressions. The manipulation of facial expressions through prompts is another new area of research \cite{Gan2023,Chun2021egsdfa,chen2023expressive}. FEG models often focus on prioritising the manipulation of the mouth, eyes, or poses \cite{zhou2021pose,prajwal2020lip,wang2023seeing,Park2022}, while others focus on overall realism \cite{Tian2024emo,xu2024vasa, zhang2023sadtalker}. With mouth movements now achieving realism pretrained SOTA models such as Wav2Lip \cite{prajwal2020lip} are incorporated into larger models to guide the lip movements, while the model focuses on poses and facial expressions \cite{zhang2023sadtalker}. \cite{Tian2024emo,paraperas2021emo} use 3D face modelling techniques and reconstruction methods to capture detailed facial geometry. This allows for accurate expression synthesis and emotion manipulation. Similarly, other methods use 3D registration and mesh-based representations to achieve realistic Face Expression generation \cite{Niinuma2022,Li2021starganv2}.

Generative Adversarial Networks (GAN) are used for generating animations \cite{Gan2023,Vougioukas2020} due to their ability to create realistic synthetic content. GANs are trained by generating content through a generator network, then using a discriminator network to predict if the generated content is real or not. For Face Expression generation, GANs are combined with other models to generate realistic facial expressions in talking-head animation generation \cite{Kong2021dualpath,Ling2020,Yan2023gan,Chun2021egsdfa,Niinuma2022}. For example, \cite{Chun2021egsdfa} employ LSTM networks and a GAN for speech-driven animation. \cite{Kong2021dualpath} use a GAN to guide the generation process of emotional animations, and preserve the identity of the target face. \cite{Ling2020} focuses on facial expression manipulation using a modified U-Net structure with GANs and achieves precise emotion manipulation. \cite{Wang2021} use GANs and attention mechanisms as the backbone of their text to talking-head generation framework. Meanwhile, \cite{Pikoulis2023} and \cite{Gan2023} utilise GANs in their methodologies for efficient emotional manipulation. Additionally, \cite{Niinuma2022} use a GAN for personalised facial expression manipulation. 
\cite{Tian2024emo} use Diffusion models for generative power and extensive control over the generation of animations. Diffusion models iteratively refine a noisy image into a high-quality sample. This refinement allows for the generation of highly realistic facial expressions, while maintaining control over intensity, duration, and subtle movements. By conditioning the diffusion process on desired expression labels or latent codes, these models produce specific facial expressions with remarkable realism. As diffusion models capture uncertainty during generation, this enables the synthesis of realistic variations. \\\\
Attention's ability to focus on important facial regions and generate realistic facial expressions has enabled them to become a key part of face generation architectures. In \cite{chen2023expressive}, attention mechanisms ensure the generated facial animations accurately capture the speaker's gestures and facial expressions. \cite{Tian2024emo} (see Fig.\ref{Tian}) integrates attention mechanisms into the pipeline to improve the quality and synchronisation of talking portrait videos, attention mechanisms are utilised to refine motion dynamics and speed adjustments. This method achieves realistic talking portrait videos which closely align with the input audio content. \cite{Wang2021} use attention gate and self-attention mechanisms in their text-based talking-head generation framework. By incorporating these mechanisms their model manipulates Action Unit-related embeddings, leading for accurate and expressive facial animations synchronised with input text.\\
CLIP with its multimodal capabilities is useful for facial animation generation tasks. By inputting textual prompts to describing desired emotional states along with images associated with those emotions, CLIP can generate images reflecting the specified emotions. This allows the model to learn associations between text and images which improves its ability to generate content with realistic emotions. TalkCLIP by\cite{ma2023dreamtalk} generates realistic talking head videos of a target speaker with specific speaking styles. Their model utilises CLIP embeddings and an adaptor network to map text descriptions, to speaking style codes. \\

Furthermore, researchers have explored the ability to control the generation of emotions on the faces through various inputs such as speech, video, facial reenactment, and text. Speech data is the most common input medium whereby an animated face video is generated using the emotions in the speech \cite{Chun2021egsdfa,Gan2023,Tian2024emo} (see Fig.\ref{Tian}). Video is used as an input in architectures where the face is changed to a target face using facial reenactment methods \cite{paraperas2021emo}, or the emotions are manipulated via facial reenactment from a static image \cite{Kong2021dualpath}. However, the synchronisation of speech and facial animations rely on robust phoneme processing within the architectures \cite{Wang2021}. Using text as a input is a relatively unexplored field which enables the generation  of Face Expression generation based on the emotion content of textual dialogue \cite{Li2021starganv2}. Other researchers have explored methods to directly control the emotions on the output videos using CLIP text prompts \cite{Gan2023,ma2023dreamtalk}.

\subsection{Speech Emotion Generation}
 \begin{figure}
    \centering
    \includegraphics[width=\linewidth]{prompt_vc_seg.jpg}
    \caption{The PromptVC pipeline uses a latent diffusion model for voice style conversion using natural language prompts. During training, a style encoder extracts a global style vector from the input mel-spectrogram, while HuBERT-based discrete tokens capture linguistic content, refined by a differentiable duration predictor for accurate timing. A prosody encoder models phoneme-level prosody to enhance expressiveness. The latent diffusion model, conditioned on text embeddings, generates the style vector from noise, enabling flexible and precise style control.\cite{yao2024promptvc}}
    \label{promptvc}
\end{figure}
One element of SEG, known as voice conversion, speech-to-speech synthesis, or speech reenactment, involves the transformation of speech signals to modify the vocal characteristics of one speaker to resemble another or to produce entirely synthetic voices. These methods form the basis of SEG, whereby the emotions in a target voice can be changed through prompts \cite{yao2024promptvc}(see Fig.\ref{promptvc}), or by the emotions in a target voice through using a emotional reference voice \cite{Park2023}.

Recent advancements in AI have led to the development of synthetic voices that are almost indistinguishable from human speech. Achieving realism in generated speech involves capturing natural intonation, rhythm, and emotion. Advanced systems, such as those by ElevenLabs \cite{elevenlabs}, use SOTA deep learning techniques to produce high-quality, realistic speech. These systems generate voices that sound authentic and carry unique characteristics associated with individual speakers. This section reviews recent advancements in SEG methodologies.

SEG models use phonetic content, including emotional cues, from a source voice to synthesize audio in a target voice while retaining desired stylistic characteristics \cite{Park2023}. A common approach in SEG involves using language models like BERT \cite{Devlin2019} for extracting contextualized representations of linguistic content, thereby enabling precise alignment between source and target voices. BERT embeddings contribute to the controllability and realism of SEG systems, facilitating accurate transformations in speech style and characteristics, which allows synthetic speech to be tailored to specific emotions \cite{Xu2023, Park2023, Lin2023, Maimon2022}. Traditional approaches often rely on text-based conditioning using transcripts \cite{taylor2009text}; however, recent methods, such as that by \cite{Maimon2022}, employ discrete representations for phonetic content. This enables the capture of non-textual cues, such as laughter, and supports diverse linguistic applications. Additionally, \cite{Lin2023} propose an architecture that integrates source and target encoders with a decoder, preserving critical linguistic and speaker features throughout the conversion process to ensure the synthesized speech remains natural and true to the source.\\\
SEG also benefits from adversarial training techniques inspired by GANs \cite{yang2021ganspeech}. In these frameworks, a discriminator differentiates between target voice samples and synthesized speech, prompting the model to generate speech that convincingly reconstructs the source content while mimicking the target speaker's characteristics.The DDDM-VC model \cite{choi2024dddm} introduces a novel approach for SEG, enhancing controllability by decoupling and independently processing attributes such as content, pitch, and timbre. Through attribute-specific denoising, DDDM-VC achieves high-precision voice style transformations, while the inclusion of prior mixup techniques strengthens robustness in voice adaptation, especially in zero-shot scenarios. This disentangled structure enables DDDM-VC to maintain speaker fidelity and naturalness in synthesized voices across a variety of speaker styles . Similarly, PromptVC by \cite{yao2024promptvc} (see Fig.\ref{promptvc}), uses a latent diffusion model for voice style conversion using natural language prompts. This enables precise control over the attributes in the generated speech. Another method uses Contrastive Predictive Coding (CPC) features to enhance the quality of synthesised speech \cite{Park2023}, which is a self-supervised learning technique for predicting future utterances in latent space. Similarly, \cite{Bous2021} preserves time-synchronisation and fundamental frequency information to maintain the naturalness of converted speech. Finally, two-stage training schemes are frequently used to align hidden representations between source and target speech. The initial stage focuses on reconstructing single utterances to establish alignment, followed by a second stage where multiple utterances refine the conversion process \cite{Lin2021}. This progressive refinement enhances the model's adaptability, improving performance in scenarios with significant divergence between source and target speech characteristics. 

\subsection{Text Sentiment Generation} \label{EmotionalText}% mention social media such as blogs and tweets are used, BERT based models, ChatGPT
\begin{figure}
    \centering
    \includegraphics[width=0.90\linewidth]{llama2.jpg}
    \caption{The LLaMA pipeline involves pre-training transformer-based models on large textual datasets, followed by task-specific fine-tuning through supervised learning, and reinforcement learning with human feedback (RLHF). Efficient fine-tuning is achieved using methods such as QLoRA, which significantly reduce computational requirements. The model is iteratively optimised and evaluated to attain state-of-the-art performance across various applications. \cite{llama}}
    \label{llama_chat}
\end{figure}

TSG models work from a user interface by taking input text, and generating a response (see Fig. \ref{llama_chat}). TSG have the ability to alter the emotional content in existing text. Large Language Models (LLMs) such as ChatGPT \cite{achiam2023gpt}, Llama \cite{llama} (see Fig.\ref{llama_chat}), Gemini \cite{reid2024gemini} can create text with emotions and personality which can pass for human writing. Ensuring accurate grammar and syntax, a diverse and contextually appropriate vocabulary, and consistency in style, tone, and information are all important for TSG. Additionally, typographical errors, realistic mistakes, smooth transitions between ideas and a deep understanding of context also contribute to the text's realism. 

%Advanced natural language generation techniques and feedback loops for refinement further improve this. Field-specific knowledge and contextual understanding are also essential for generating text that aligns with the conversation or topic at hand.
Until recently Recurrent Neural Networks (RNNs) \cite{rumelhart1986learning} were used extensively in text generation due to their ability to handle sequential data by maintaining an internal memory. However, traditional RNNs suffer from the vanishing gradient problem, which impedes long-range dependencies. They also struggled to work on long sentences \cite{hochreiter1997long}. Researchers attempted to combat this by running the RNNs both forward and backward over the textual data \cite{schuster1997bidirectional}, which did not rectify the problem. These limitations led to the development of Long Short-Term Memory (LSTM) networks, a variant of RNNs. LSTMs employ architectures with gated mechanisms, including input, output, and forget gates, enabling them to learn and retain long-term dependencies in sequential data \cite{hochreiter1997long}. This feature makes LSTMs particularly ideal for tasks requiring memory over extended sequences, such as text generation. Another architecture used for text generation are Sequence-to-Sequence (Seq2Seq) models \cite{sutskever2014sequence}, which consist of an encoder and a decoder. Seq2Seq models have shown proficiency in generating coherent and contextually relevant text, making them valuable for emotional text generation tasks. Generative Adversarial Networks (GANs) \cite{goodfellow2014generative}, used mostly in computer vision, have also emerged as useful for text generation tasks. The generator produces synthetic text data, while the discriminator evaluates the authenticity of the generated text. Used in conjunction with the above algorithms, attention mechanisms enable models to focus on relevant parts of the input text sequence when generating a response. Attention mechanisms allow models to weigh the importance of each word in the input sequence dynamically as they generate each word in the output sequence \cite{bahdanau2014neural}. For example, in the Seq2Seq model, attention mechanisms help align the encoder hidden states with the decoder hidden states at each time point, ensuring the model attends to the most relevant parts of the input sequence when generating each word in the output sequence \cite{bahdanau2014neural}.

%However, despite these advancements, emotional text generation remains a challenging field within NLP. Emotions are nuanced and subjective, posing difficulties for humans and machines alike. Often, these systems rely on surface-level features such as punctuation or word choice to infer emotional states, which may lead to imprecise results. Moreover, human interpretation of emotional text varies widely based on individual experiences, cultural background, and personal writing style, adding complexity to the task. This means creating datasets with accurate human annotations is challenging. 

To address these challenges researchers are exploring various approaches. One approach involves fine-tuning pretrained language models such as ChatGPT \cite{achiam2023gpt} for emotion-specific tasks \cite{Goswamy2020human}. This approach uses datasets annotated with emotional labels to train the model to associate linguistic patterns with emotional states. During fine-tuning, adjustments are made to the model's parameters through additional training iterations on emotional text datasets. Developing models with an understanding of contextual cues is essential for accurate emotional text generation. This involves considering factors such as the broader narrative, speaker intent, and audience context to generate realistic text. 

\subsection{Generative Models with Emotion Control}
This section will examine methodologies for implementing emotion control within FEG, SEG, and TSG. Emotion control, in this context, pertains to the systematic generation of content—spanning animations, speech, and textual outputs—characterised by realistic and contextually appropriate emotional expressions. These emotions are elicited or guided through specific prompts or control mechanisms, ensuring that the generated outputs align with intended affective states. The discussion will encompass techniques used to encode, manipulate, and render emotions, as well as the underlying computational models that enable nuanced emotional dynamics across various modalities.

%\begin{figure}[hbt!] 

  %  \centering 

    %\includegraphics[width=0.75\textwidth]{tian_fig.PNG} 

   % \caption{ The EMO-Live method by \cite{Tian2024emo} comprises two main stages: Frames Encoding and Diffusion Process. In Frames Encoding, ReferenceNet extracts features from the reference image and motion frames. In the Diffusion Process, a pretrained audio encoder processes audio embedding. A facial region mask guides facial imagery generation with multi-frame noise. The Backbone Network facilitates denoising, incorporating Reference-Attention and Audio-Attention mechanisms to maintain identity and modulate movements. Temporal Modules adjust motion velocity.} 

    %\label{Tian} 

%\end{figure} 

\subsubsection{Audio Driven Face Expression Generation} 

Fig. \ref{Tian} shows audio driven Face Expression generation by \cite{Tian2024emo}. This method for Face Expression generation takes a reference image as input which is put through a frames encoder. Next, a feature extraction network, called ReferenceNet extracts detailed features from the reference image and after the first iteration, the motion frames, to preserve the identity from the reference image. The architecture then progresses to the diffusion stage where a pretrained audio encoder processes the input voice audio clip, extracting voice features which influence the facial movements and expressions. The Backbone Network, using reference-attention and audio-attention mechanisms, denoises the input data and generating realistic video frames. This comprehensive network architecture ensures the generated video frames sync with the provided audio content. Speed layers fine-tune temporal modules and control head motion across clips, improving consistency and stability in the generated videos. 
%The user interface is initiated with a reference image and an audio clip containing the desired voice content. Users can next configure model settings according to their specific requirements and preferences. This step involves adjusting parameters such as the desired length or duration of the generated video, the level of detail in facial expressions, and the synchronisation between audio and video elements. By customising these settings, users can tailor the generated videos to suit their intended artistic or communicative objectives. This iterative approach allows users to experiment with different configurations and refinements until they achieve a satisfactory outcome that meets their expectations and requirements.

%\begin{figure}[hbt!] 

  %  \centering 

    %\includegraphics[width=0.5\textwidth]{Li_fig.PNG} 

  %  \includegraphics[width=0.5\textwidth]{Li_fig2.PNG} 
    
 %   \caption{This diagram illustrates the pipeline for generating photo-realistic talking-head videos from text inputs, a method by \cite{Li2021starganv2}. Beginning with time-aligned text, specialised modules translate it into animation parameters for mouth, upper face, and head pose. Using a motion capture dataset, the system ensures realistic as facial expressions and head movements. Neural networks are trained to optimise parameter generation. A style-preserving landmark generator aligns facial features with speaker-specific characteristics. Finally, video synthesis renders textures for a realistic talking-head video output, embodying the nuances of the input text.} 

  %  \label{li} 

%\end{figure} 

\subsubsection{Text Driven Face Expression Generation}
The text-based talking-head generation framework by \cite{Li2021starganv2} uses neural networks tailored to different aspects of generating Face Expression animations from textual inputs. Gmou, dedicated to animating mouth movements from phonemes, uses a structure based on CNNs for efficient parallel computation and is trained using a combination of L1 loss and Least Squares Generative Adversarial Network (LSGAN) loss. Similarly, Gupp and Ghed utilise encoder-decoder network structures to synthesize upper face parameters and head pose, respectively, from input words, training with analogous loss functions to ensure realistic outputs. The Style-Preserving Landmark Generator, Gldmk, uses a multi-linear 3D Morphable Model (3DMM) and a fully-connected network to ensure consistency and accuracy in facial expressions, incorporating a unique mapping technique to preserve speaker-specific styles. %Together, these networks work within the framework to generate cohesive and expressive talking-head videos, seamlessly inTSGrating textual inputs with realistic facial animations.

%Users provide time-aligned text inputs of dialogue or speech for the generated talking-head video. The system then processes these inputs through its pipeline, using the trained neural networks to generate animation parameters and synthesise the final video output. Users can then seamlessly view and analyse the generated videos, which accurately depict the subtleties of the speaker's facial expressions and head movements as conveyed by the input text.

%\begin{figure}[hbt!] 

  %  \centering 

    %\includegraphics[width=\textwidth]{para_fig.PNG} 
    
    %\caption{\cite{paraperas2021emoemo}} 

  %  \label{para} 

%\end{figure} 

\subsubsection{Video Driven Face Expression Generation}
%The method uses a sophisticated framework called the Neural Emotion Director (NED) to 

NED by \cite{paraperas2021emo} allows manipulation of Face Expressions in in-the-wild videos while preserving natural speech-related mouth motion. The Face Analysis module incorporates preprocessing steps such as face landmark detection, segmentation, and resizing, alongside 3D Morphable Models (3DMMs) for accurate estimation of 3D face geometry. The Expressions Translator, a GAN, utilises a recurrent network with LSTM units to convert sequences of facial expressions into desired emotions, while maintaining the original mouth motion. A encoder extracts emotion-related style vectors from the input sequences, while the Mapping Network generates style vectors associated with target emotions. A neural face renderer generates realistic frames, incorporating techniques such as multi-band blending for seamless integration of generated faces into the original backgrounds. This ensures the manipulated facial expressions seamlessly blend into real-world scenarios. During testing, N-length sliding windows are applied frame by frame, with the sequences processed through the Expressions Translator. The conditional style vector is either generated by the Mapping Network or extracted from a reference video, allowing for flexible manipulation of emotions in facial videos. 


%\begin{figure}[hbt!] 
  %  \centering 

    %\includegraphics[width=0.5\textwidth]{Gan_fig.PNG} 
    %\includegraphics[width=0.5\textwidth]{Gan_fig2.PNG} 
    
  %  \caption{\cite{Gan2023}} 

 %   \label{ganfig} 

%\end{figure} 

\subsubsection{Emotion Prompted Face Expression Generation}
EAT by \cite{Gan2023} takes in an image of a target face, speech, and an emotion prompt such as happy, sad, or angry, to generate animated videos. The model first trains the CLIP model on emotion labelled datasets to learn audio-visual correlations. This pre-training phase uses enhanced latent representations and a transformer model. Enhanced latent representations capture intricate facial expressions, incorporating identity-specific canonical keypoints, rotation, translation, and expression deformation components. The transformer model predicts synchronised expression deformations from audio inputs and predicts head pose features, and latent source image representations. Next, three primary modules—Deep Emotional Prompts, Emotional Deformation Network (EDN), and Emotional Adaptation Module (EAM)—play integral roles in the emotional adaptation. Deep Emotional Prompts inject emotion-guided expression generation into the model, using latent codes sampled from a Gaussian distribution to provide crucial emotional guidance. EDN complements this by predicting emotion-related expression deformations. EAM further refines the visual quality of generated videos by generating emotion-conditioned features. The architecture also accommodates zero-shot expression editing, which allows text-guided manipulation of talking-head videos without the need for extensive emotional training data. Using the CLIP model, the system aligns generated expressions with textual descriptions, offering users control over the emotional content of the videos. 

\subsubsection{Speech Emotion Generation Model}
The architecture in \cite{Park2023} comprises three main components: source encoder, target encoder, and a decoder. The source encoder uses Wav2Vec 2.0 \cite{baevski2020wav2vec}, a pretrained feature extractor, to capture speech representations from the source utterance. The target encoder processes log mel-spectrograms of utterances from the target speaker, and the decoder consists of transformer layers using both self-attention and cross-attention. A linear projection layer contributes to the final prediction of the desired output voice, following a non-autoregressive approach. The model is trained using a two-stage approach. In the first stage, single utterances from both the source and target speakers are used to reconstruct the log mel-spectrogram of the utterance. In the second stage, multiple utterances, typically 10, from the target speaker are concatenated and fed into the target encoder. Simultaneously, a single utterance from the source speaker is fed into the source encoder.

\subsubsection{Text Sentiment Generation Model}
A model \cite{Goswamy2020human} built upon ChatGPT2 \cite{radford2019language}, has been trained to generate text with specific emotions. The ChatGPT2 model is fine-tuned with text samples annotated with affective labels or sentiment scores. The Plug and Play Language Model (PPLM) framework is integrated into the ChatGPT2 architecture to enable attribute-controlled text generation. PPLM incorporates perturbation and optimisation mechanisms during training, enhancing the model's ability to generate text with specific affective attributes. The model's loss functions include terms which encourage the generation of text with desired emotional attributes and intensity levels. Users specify the desired emotional tone or topic, and the intensity of the emotion desired. The model uses specified attributes and intensity levels to control the content and tone during the text generation process.

\subsubsection{Text Sentiment Generation Chatbot}
The Empathetic Semantic Correlation Model (ESCM) by \cite{Yang2023} generates empathetic responses in dialogues by understanding emotions and semantics. It includes three components: a context encoder, a dynamic correlation encoding module, and an emotion and response predicting module. The dynamic correlation encoding module features dynamic emotion-semantic vectors and a correlation Graph Convolutional Network, adjusting emotions and semantics based on contextual cues. The emotion and response predicting module uses context semantics and correlations to predict emotions and generate empathetic responses. During training, ESCM optimises parameters using multiple loss functions and supervised learning on annotated datasets. In use, ESCM processes dialogue context, adjusts to contextual cues, and continuously learns to provide accurate, empathetic responses.

\section{Evaluation}

This section provides an overview of the metrics used to evaluate ER and EG models across facial, speech, and textual modalities. It explores various evaluation techniques to determine their effectiveness in measuring model performance and accuracy. Furthermore, the comparative analysis within this section examines state-of-the-art methods to identify the most effective approaches. By synthesising findings from recent studies, this evaluation aims to uncover the strengths and limitations of current evaluation frameworks, thereby highlighting which models are most proficient at recognising and generating emotional expressions across different modalities.


\subsection{Evaluation Metrics}

Evaluation metrics are essential for assessing the performance of emotion recognition and generation models across different modalities. This section highlights the most widely used metrics in facial, speech, and text, emotion recognition and generation.

\subsubsection{Common Metrics}

\begin{itemize}
    \item \textbf{Accuracy}: This metric measures the proportion of correctly classified instances among the total instances. It provides a basic overview of model performance but does not account for class imbalances, which can lead to misleading results.
    \begin{equation}
    \text{Accuracy} = \frac{\text{Number of Correct Predictions}}{\text{Total Number of Predictions}}
    \end{equation}

    \item \textbf{F1 Score}: The harmonic mean of precision and recall, providing a balanced measure of a classifier’s performance, particularly in cases with imbalanced datasets. The F1 score is crucial for understanding the trade-off between precision and recall.
    \begin{equation}
    F1 = \frac{2 \times \text{Precision} \times \text{Recall}}{\text{Precision} + \text{Recall}}
    \end{equation}

    \item \textbf{Precision}: Measures the proportion of true positive predictions out of all positive predictions, indicating the accuracy of positive predictions in identifying emotional expressions.
    \begin{equation}
    \text{Precision} = \frac{\text{True Positive}}{\text{True Positive} + \text{False Positive}}
    \end{equation}

    \item \textbf{Recall}: Measures the proportion of true positive predictions out of all actual positives, reflecting the model's ability to identify relevant instances. High recall is essential in applications where missing a positive instance can have significant consequences.
    \begin{equation}
    \text{Recall} = \frac{\text{True Positive}}{\text{True Positive} + \text{False Negative}}
    \end{equation}

    \item \textbf{Mean Opinion Score (MOS)}: Often used in evaluating generated speech and facial expressions, this metric assesses perceived quality by averaging ratings given by human evaluators on a numerical scale, providing a subjective measure of output quality.\\
\end{itemize}
%     \item \textbf{Confusion Matrix}: This tool provides a detailed breakdown of the performance of a classification model, showing the numbers of true positives, true negatives, false positives, and false negatives, allowing for a comprehensive evaluation of model performance.\\
% \end{itemize}

\subsubsection{Metrics for face systems}
\begin{itemize}
\item \textbf{Structural Similarity Index (SSIM}, \ref{Eq:ssim}, is used to assess the similarity between two images. It takes into account luminance, contrast, and structure of the images. SSIM is defined as:
\begin{equation}
    \text{SSIM}(x, y) = \frac{{(2\mu_x\mu_y + C_1)(2\sigma_{xy} + C_2)}}{{(\mu_x^2 + \mu_y^2 + C_1)(\sigma_x^2 + \sigma_y^2 + C_2)}}
    \label{Eq:ssim}
\end{equation}
%Where $\mu_x$, $\mu_y$, $\sigma_x^2$, $\sigma_y^2$, and $\sigma_{xy}$ are the mean, variance, and covariance of the pixel intensities of images $x$ and $y$, respectively, and $C_1$ and $C_2$ are constants to stabilize the division with weak denominator.

\item \textbf{ Fréchet Inception Distance score (FID)}, \ref{Eq:fid}, evaluates the quality of generated images in generative adversarial networks (GANs). It measures the similarity between the distribution of real images and generated images in a feature space learned by a pretrained deep convolutional neural network. FID is defined as:
\begin{equation}
    \text{FID} = \|\mu_x - \mu_y\|^2 + \text{Tr}(\Sigma_x + \Sigma_y - 2(\Sigma_x\Sigma_y)^{1/2})
        \label{Eq:fid}
\end{equation}
%Where $\mu_x$, $\mu_y$ are the mean feature vectors and $\Sigma_x$, $\Sigma_y$ are the covariance matrices of the feature representations of real and generated images, respectively. Tr denotes the trace of a matrix, and $\|^2$ represents the squared Frobenius norm of a matrix.

  \item \textbf{Cumulative Probability Blur Detection (CPBD)} \\
    CPBD quantifies image blur by analysing edge sharpness and comparing edge gradient profiles to perceptual thresholds. A higher CPBD score indicates a clearer image with less blur.
    \[
    \text{CPBD} = \frac{1}{N} \sum_{i=1}^N \mathcal{P}(e_i)
    \]

    \item \textbf{Cosine Similarity (CSIM)} \\
    CSIM measures the similarity between two vectors, such as feature embeddings of source and generated faces. Values range from \(-1\) to \(1\), where \(1\) indicates identical direction and maximum similarity.
    \[
    \text{CSIM} = \frac{\mathbf{A} \cdot \mathbf{B}}{\|\mathbf{A}\| \|\mathbf{B}\|}
    \]

    \item \textbf{Mouth Landmark Distance (M-LMD)} \\
    M-LMD evaluates the average difference in lip keypoint positions between reference and generated videos. It reflects the overall accuracy of lip synchronisation in generated content.
    \[
    M\text{-LMD} = \frac{1}{T} \sum_{t=1}^T \frac{1}{K} \sum_{k=1}^K \| \mathbf{p}_{t,k}^{\text{ref}} - \mathbf{p}_{t,k}^{\text{gen}} \|
    \]

    \item \textbf{Face Landmark Distance (F-LMD)} \\
    F-LMD calculates the keypoint difference between reference and generated faces. It provides insights into face synchronisation.
    \[
    F\text{-LMD}(t) = \frac{1}{K} \sum_{k=1}^K \| \mathbf{p}_{t,k}^{\text{ref}} - \mathbf{p}_{t,k}^{\text{gen}} \|
    \]
\end{itemize}
 
\subsubsection{Metrics for speech and text systems}

\begin{itemize}
    \item \textbf{Word Error Rate (WER)}: Commonly used in speech and text systems, WER quantifies the rate of incorrect words generated by the system compared to a reference transcript. Lower WER scores indicate better system performance in speech and text generation tasks.
    \begin{equation}
    \text{WER} = \frac{\text{Number of Word Errors}}{\text{Total Number of Words in Reference Transcript}}
    \end{equation}

    \item \textbf{Character Error Rate (CER)}: Similar to WER, this metric measures the rate of incorrect characters generated in speech and text systems compared to the reference transcript. It provides a more fine-grained evaluation of textual accuracy, particularly useful in text-based emotion recognition systems.
    \begin{equation}
    \text{CER} = \frac{\text{Number of Character Errors}}{\text{Total Number of Characters in Reference Transcript}}
    \end{equation}

    \item \textbf{Equal Error Rate (EER)}, \ref{eq:eer}, is a point where the false acceptance rate (FAR) and false rejection rate (FRR) are equal in a speaker systems. It represents the operating point where the system's performance is balanced. 
\begin{equation}
    \text{EER} = \text{FAR} = \text{FRR}
    \label{eq:eer}
\end{equation}

 \item \textbf{Mel-cepstral distortion (MCD)}, \ref{eq:mcd}, quantifies the difference between two sets of mel-frequency cepstral coefficients (MFCCs) for speech tasks. 
\begin{equation}
    \text{MCD} = \frac{1}{N} \sum_{i=1}^{N} \| X_i - Y_i \|
    \label{eq:mcd}
\end{equation}

    \item \textbf{Perplexity}: A key metric in text generation, perplexity measures how well a language model predicts a sample of text. It reflects the average branching factor of the model, with lower perplexity indicating better performance.
    \begin{equation}
    \text{Perplexity} = 2^{-\frac{1}{N} \sum_{i=1}^{N} \log P(x_i)}
    \end{equation}

    \item \textbf{Sentiment Accuracy}: For text-based emotion recognition, sentiment accuracy measures how accurately a model classifies the overall emotional tone or sentiment of a text (e.g., positive, negative, neutral). This metric is widely used in applications such as sentiment analysis and Text Sentiment generation.\\
    
    \item \textbf{BLEU (Bilingual Evaluation Understudy Score)}: Commonly used in text generation systems, BLEU compares the generated text to a reference by measuring how many n-grams in the generated text appear in the reference. It is particularly useful for evaluating the fluency and relevance of generated text.
    \begin{equation}
    \text{BLEU} = \exp\left( \min\left(1 - \frac{l_r}{l_c}, 0\right) + \sum_{n=1}^{N} w_n \log p_n \right)
    \end{equation}
\end{itemize}


Evaluation metrics for assessing LLMs include: Massive Multitask Language Understanding (MMLU), Generalized Question-Answering Performance (GPQA), MATH, HumanEval, Multi-Genre Social Media (MGSM), and Discrete Reasoning Over Paragraphs (DROP). MMLU evaluates the models ability to understand and generate text across 57 subjects using multiple choice questions. GPQU evaluates text generation in question answering tasks. MATH tests the models ability to understand mathematical concepts, problem-solving skills, and ability to generate accurate solutions to mathematical queries. HumanEval assesses performance on tasks which require a high level of language comprehension and expression, such as essay writing, and summarisation. MGSM assesses the generation of text for social media across various formats, including tweets, posts, and comments. DROP is used to assess the models ability to extract information from longer texts such as performing logical reasoning and answering questions regarding the text. The F1 score is the measure of models precision and recall in these tasks. All of these metrics are obtained from user studies. Additional metrics include, Recall-Oriented Understudy for Gisting Evaluation (ROUGE) used for evaluating the quality of summaries produced by text systems. The ROUGE score is typically calculated as the F1 score between the generated and reference summaries using the respective metric. 

\subsection{Comparative Analysis for Emotion Recognition and Generation Models}

This section presents a comparative analysis of SOTA methods in ER and EG for faces, speech, and text. We will discuss the most effective methods based on their performance in recognising and generating emotions across these modalities. The performance of these models will be evaluated through experiments and the corresponding results. However, comparing these methods poses challenges due to a lack of uniformity in evaluation metrics, complicating the assessment process. By conducting this comparative analysis of SOTA models, we aim to highlight the most effective methods for emotion recognition and generation.


\subsubsection{Facial Expression Recognition Comparative Analysis}
\begin{table}[htbp]
    \centering
    \caption{FER Comparative Analysis. *Results derived from cited papers.}
    \label{table2}
    %\begin{adjustbox}
    \begin{tabular}{llc}
        \toprule
        \textbf{Model} & \textbf{Dataset} & \textbf{ACC \% $\uparrow$}  \\
        \midrule
        *ESTLNet \cite{gong2024enhanced}         & AFEW             & 0.54 \\
        *EmoFAN \cite{toisoul2021estimation}   & AffectNet & 0.75 \\
        *EMOCA  \cite{Danecek2022}           & AffectNet        & 0.69 \\
        *Poster++ \cite{Mao_2023}          & AffectNet        & 0.63 \\
        *LibreFace \cite{Chang_2024}        & AffectNet        & 0.49 \\
        *Dresvyanskiy 2022 \cite{Dresvyanskiy2022} & AffWild2         & 0.48 \\
        *ESTLNet \cite{gong2024enhanced}         & CK+              & 0.99 \\
        *Sun 2023  \cite{sun_2023}         & CK+              & 0.98 \\
        *Zhao 2023  \cite{zhao2023prompting}          & DFEW             & 0.71 \\
        *ESTLNet \cite{gong2024enhanced}         & DFEW             & 0.69 \\
        *Zhao 2023  \cite{zhao2023prompting}       & FERV39K          & 0.52 \\
        *Hossain 2023 \cite{hossain_2023}     & IMFDB            & 0.64 \\
        *Sun 2023   \cite{sun_2023}      & JAFFE            & 0.98 \\
        *Sun 2023   \cite{sun_2023}       & KDEF             & 0.98 \\
        *Zhao 2023  \cite{zhao2023prompting}       & MAFW             & 0.53 \\
        *ESTLNet \cite{gong2024enhanced}          & Oulu-CASIA       & 0.89 \\
       *Poster++ \cite{Mao_2023}            & RAF-DB           & 0.92 \\
        *PACVT  \cite{Liu_2023}           & RAF-DB           & 0.88 \\
        *LibreFace \cite{Chang_2024}         & RAF-DB           & 0.82 \\
        *Hossain 2023 \cite{hossain_2023}     & SFEW 2.0         & 0.80 \\
        \bottomrule
    \end{tabular}
    %\end{adjustbox}
\end{table}

% \nocite{gong2024enhanced, toisoul2021estimation, Danecek2022, Mao_2023, Chang_2024, Dresvyanskiy2022, sun_2023, zhao2023prompting, hossain_2023, Liu_2023}


% \begin{figure}[ht]
%     \centering
%     \subfigure[FER Comparative Analysis]{
%         \includegraphics[width=0.3\textwidth]{Picture1.png}
%         \label{FER}
%     }
%     \subfigure[FEG Comparative Analysis]{
%         \includegraphics[width=0.6\textwidth]{Picture2.png}
%         \label{FEG}
%     }
%     \caption{Comparative Analysis of Face Systems}
%     \label{fig:combined_images}
% \end{figure}



\ref{table2} summarises the evaluation of FER models, showcasing their performance across various datasets, with accuracy (ACC) as the primary metric. EmoFAN \cite{toisoul2021estimation} achieves the highest accuracy of 75\% on the AffectNet dataset, demonstrating exceptional capabilities in recognising Facial Expressions. Likewise, models such as Poster++ \cite{Mao_2023} display impressive performance with an accuracy of 92\% on the RAF-DB dataset. The variability in performance across different datasets highlights the unique challenges each dataset presents. For example, ESTLNet \cite{gong2024enhanced} exhibits lower performance on the FERV39K dataset, attaining an accuracy of 58.70\%, yet it achieves a remarkable 99\% accuracy on the CK+ dataset. The Sun 2023 \cite{sun_2023} model obtains SOTA scores across the JAFFE, CK+, and KDEF datasets, with an accuracy of 98.00\% in each case. 



\subsubsection{Facial Expression Generation Comparative Analysis}

\begin{table}[htbp]
\centering
\caption{FEG Comparative Analysis.*Results derived from  \cite{ma2023talkclip} and \cite{Gan2023}, ** Results derived from  \cite{Tian2024emo}, *** Results derived from  \cite{ma2023talkclip}, **** Results derived from  \cite{Gan2023}}
\label{table3}
\begin{tabular}{lcccccccccc}

\hline
\textbf{Method} & \textbf{ACC ↑} & \textbf{FID ↓} & \textbf{SyncNet ↑} & \textbf{SSIM ↑} & \textbf{CPBD ↑} & \textbf{M-LMD ↓} & \textbf{F-LMD ↓} & \textbf{Dataset} \\
\hline
*StyleTalk \cite{ma2023styletalk} &  &  &  & 0.8 & 0.26 & 2.49 & 2.04 & HDTF \\
*TalkCLIP \cite{ma2023talkclip} &  &  &  & 0.78 & 0.25 & 2.8 & 2.54 & HDTF \\
*AVCT \cite{wang2022one} &  &  &  & 0.74 & 0.18 & 3.83 & 3.06 & HDTF \\
*Wav2Lip \cite{prajwal2020lip} &  &  &  & 0.59 & 0.26 & 3.84 & 5.12 & HDTF \\
*MakeItTalk \cite{zhou2020makelttalk} &  &  &  & 0.57 & 0.2 & 4.61 & 5.65 & HDTF \\
*PC-AVS \cite{zhou2021pose} &  &  &  & 0.42 & 0.12 & 4.26 & 10.68 & HDTF \\
*EAMM \cite{ji2022eamm}  &  &  &  & 0.36 & 0.13 & 7.67 & 7.74 & HDTF \\
*GC-AVT \cite{liang2022expressive} &  &  &  & 0.33 & 0.24 & 6.34 & 10.7 & HDTF \\
**Wav2Lip \cite{prajwal2020lip} &  & 9.38 & 5.76 &  & 0.36 &  &  & HDTF \\
**SadTalker \cite{zhang2023sadtalker} &  & 10.31 & 4.82 &  & 0.34 &  &  & HDTF \\
**DreamTalk \cite{ma2023dreamtalk} &  & 58.8 & 3.43 &  &  &  &  & HDTF \\
**EMO \cite{Tian2024emo} &  & 8.76 & 3.89 &  &  &  &  & HDTF \\
***Audio2Head \cite{wang2021audio2head} &  &  &  &  & 0.28 &  &  & HDTF \\
***Wang et al \cite{Wang2021} &  &  &  &  & 0.29 &  &  & HDTF \\
****EAT \cite{Gan2023} & 75.43 & 3.52 & 6.22 & 0.77 &  & 1.79 & 2.08 & LRW \\
****Wav2Lip \cite{prajwal2020lip} & 17.87 & 7.56 & 7.89 & 0.73 &  & 1.53 & 2.47 & LRW \\
****PC-AVS \cite{zhou2021pose} & 11.88 & 4.64 & 7.36 & 0.72 & 0.07 & 1.54 & 2.11 & LRW \\
****EAMM \cite{ji2022eamm} & 49.85 & 6.44 & 4.67 & 0.71 & 0.08 & 1.81 & 2.37 & LRW \\
****MakeItTalk \cite{zhou2020makelttalk} & 15.23 & 3.37 & 3.28 & 0.69 &  & 2.16 & 2.99 & LRW \\
****AVCT \cite{wang2022one} & 15.64 & 2.01 & 4.68 & 0.68 &  & 2.55 & 3.23 & LRW \\
****ATVG \cite{chen2019hierarchical} & 17.36 & 51.56 & 2.73 & 0.64 &  & 2.69 & 3.31 & LRW \\
****StyleTalk \cite{ma2023styletalk}  &  &  &  & 0.84 & 0.16 & 3.36 & 2.1 & MEAD \\
****AVCT \cite{wang2022one} & 15.64 & 39.18 & 6.02 & 0.83 & 0.14 & 5.64 & 2.95 & MEAD \\
****TalkCLIP \cite{ma2023talkclip} &  &  &  & 0.83 & 0.16 & 3.6 & 2.4 & MEAD \\
****Wav2Lip \cite{prajwal2020lip} &  &  &  & 0.81 & 0.16 & 3.85 & 2.73 & MEAD \\
****MakeItTalk \cite{zhou2020makelttalk} &  &  &  & 0.73 & 0.1 & 5.3 & 3.9 & MEAD \\
****EAT \cite{Gan2023} & 75.43 & 19.69 & 8.28 & 0.68 &  & 2.25 & 2.47 & MEAD \\
****EAMM \cite{ji2022eamm}  & 49.85 & 22.38 & 6.62 & 0.66 &  & 2.19 & 2.55 & MEAD \\
****PC-AVS \cite{zhou2021pose} & 11.88 & 53.04 & 8.6 & 0.61 &  & 2.66 & 2.7 & MEAD \\
****Wav2Lip \cite{prajwal2020lip} & 17.87 & 67.49 & 8.97 & 0.57 &  & 3.11 & 3.71 & MEAD \\
****MakeItTalk \cite{zhou2020makelttalk} & 15.23 & 51.88 & 5.28 & 0.55 &  & 3.61 & 4 & MEAD \\
****GC-AVT \cite{liang2022expressive} &  &  &  & 0.34 & 0.14 & 8.4 & 8.1 & MEAD \\
\hline
\end{tabular}

\end{table}

% \nocite{ma2023styletalk, ma2023talkclip, wang2022one, prajwal2020lip, zhou2020makelttalk, zhou2021pose, ji2022eamm, liang2022expressive, zhang2023sadtalker, ma2023dreamtalk, Tian2024emo, wang2021audio2head, Wang2021, Gan2023, chen2019hierarchical}


Both quantitative and qualitative methods are used to evaluate FEG models. However, the absence of a universal evaluation framework complicates comparisons across different studies. Most researchers omit estimating the accuracy of the emotions generated by their models; with the exception of \cite{Gan2023,Tian2024emo}, as shown in table \ref{table3}, which includes the metrics ACC and E-FID. The accuracy of emotions in FEG models is evaluated by utilising pretrained FER models or by user studies. Wav2Lip \cite{prajwal2020lip} model demonstrates a high SyncNet accuracy (9.38) and a relatively low FID (5.76) on the HDTF dataset, highlighting its strong synchronisation capabilities. In constrast, the SadTalker \cite{zhang2023sadtalker} model achieves a lower ACC (10.31) and a higher FID (4.82), suggesting potential limitations in generating accurate Facial Expressions. DreamTalk \cite{ma2023dreamtalk} shows promising results with a high ACC (58.8) and moderate FID (3.63), although E-FID (2.25) indicates room for improvement in the fidelity of the generated emotions.

EMO \cite{Tian2024emo} shows moderate performance with an ACC of 8.76 and an E-FID of 0.116, indicating balanced capabilities. MakeItTalk \cite{zhou2020makelttalk} displays poor performance across several metrics, with a low ACC (3.37) and high FID (3.28), suggesting significant challenges in generating accurate emotions. Models evaluated on the LRW dataset, such as AVCT \cite{wang2022one} and PC-AVS \cite{zhou2021pose}, demonstrate considerable performance differences in SSIM and CSIM. The diversity in performance metrics across models and datasets emphasises the necessity for optimisation to enhance the robustness and accuracy of FEG systems.


\subsubsection{Speech Emotion Recognition Comparative Analysis}
\begin{table}[htbp]
\centering
\caption{SER Comparative Analysis: *Results derived from cited papers.}
\label{table4}
\begin{tabular}{lrl} % Adjusted column specification: l (left-aligned), r (right-aligned), l (left-aligned)
\hline
\textbf{Model} & \textbf{ACC} & \textbf{Datasets} \\ % Changed the order of columns
\hline
*Kwon 2020 \cite{kwon2020clstm} & 90.01 & Berlin EMO \\
*Meng 2019 \cite{meng2019speech} & 88.99 & Berlin EMO \\
*Sun 2019 \cite{sun2019decision} & 86.86 & Berlin EMO \\
*Issa 2020 \cite{issa2020speech} & 86.10 & Berlin EMO \\
*Mustageem 2020 \cite{mustaqeem2021optimal} & 85.57 & Berlin EMO \\
*Xie 2023 \cite{Xie2023} & 92.80 & CASIA \\
*Liu 2018 \cite{liu2018speech} & 86.58 & CASIA \\
*Sun 2019 \cite{sun2019decision} & 83.75 & CASIA \\
*Gong 2023 \cite{Gong2023} & 58.70 & CREMA-D \\
%Xie 2023 \cite{Xie2023} & 89.60 & ENTERFACE \\
*Kwon 2020 \cite{kwon2020clstm} & 75.00 & IEMOCAP \\
*Lu 2020 \cite{lu2020iterative} & 72.60 & IEMOCAP \\
*Shamsi 2023 \cite{Shamsi2023} & 70.80 & IEMOCAP \\
*Pepino 2021 \cite{pepino2021emotion} & 67.20 & IEMOCAP \\
*Gong 2023 \cite{Gong2023} & 54.50 & IEMOCAP \\
*Sharma 2021 \cite{sharma2021emotion} & 92.88 & RAVDESS \\
*Pepino 2021 \cite{pepino2021emotion} & 84.30 & RAVDESS \\
*Kwon 2020 \cite{kwon2020clstm} & 80.00 & RAVDESS \\
\hline
\end{tabular}
\end{table}

% \nocite{kwon2020clstm, meng2019speech, sun2019decision, issa2020speech, mustaqeem2021optimal, Xie2023, liu2018speech, Gong2023, lu2020iterative, Shamsi2023, pepino2021emotion, sharma2021emotion}


% \begin{figure}[ht]
%     \centering
%     \subfigure[SER Comparative Analysis]{
%         \includegraphics[width=0.45\textwidth]{Picture3.png}
%         \label{FER}
%     }
%     \subfigure[SEG Comparative Analysis]{
%         \includegraphics[width=0.75\textwidth]{Picture4.png}
%         \label{FEG}
%     }
%     \caption{Comparative Analysis of Speech Systems}
%     \label{fig:combined_images}
% \end{figure}

A comparison of SER models is presented in table \ref{table4}, using accuracy (ACC) as the principal metric. An analysis of the results indicates that Kwon 2020 \cite{kwon2020clstm} achieves the highest accuracy (90.01\%) on the Berlin EMO dataset. Similarly, Xie 2023 \cite{Xie2023} demonstrates outstanding performance on the CASIA dataset, attaining an accuracy of 92.80\%. Conversely, Gong 2023 \cite{Gong2023} reports a low accuracy (58.70\%) on the CREMA-D dataset, indicating potential challenges in recognising emotions within this specific dataset. Lu 2020 \cite{lu2020iterative} and Pepino 2021 \cite{pepino2021emotion}, evaluated on the IEMOCAP dataset, achieve lower accuracies of 72.60\% and 67.20\%, respectively, compared to those tested on other datasets. Models such as Sharma 2021 \cite{sharma2021emotion} on the RAVDESS dataset attain a high accuracy of 92.88\%. This comparison underscores the importance of developing versatile models capable of maintaining high performance across diverse datasets. The results also highlight the ongoing challenges and the necessity for further research to enhance the generalisability and robustness of SER on models across varying emotional contexts.




\subsubsection{Speech Emotion Generation Comparative Analysis}
\begin{table}[ht]
\centering
\caption{SEG Comparative Analysis:*Results derived from cited papers.}
\label{table5}
\begin{tabular}{lrrrrl}
\hline
\textbf{Model} & \textbf{WER (↓)} & \textbf{CER (↓)} & \textbf{EER (↓)} & \textbf{Dataset} \\
\hline
*DISSC \cite{Maimon2022} & 19.1 & 7.9 & 2.6 & ESD \\
*Seq2seq-VC \cite{liu2021any} & 14.9 & 6 & 2.9 & ESD \\
*AutoVC \cite{qian2019autovc}  & 87 & 59.9 & 6.6 & ESD \\
*AutoPST \cite{qian2021global}  & 50.3 & 31.8 & 15.7 & ESD \\
*VQMIVC \cite{wang2021vqmivc}  & 41.5 & 23.66 & 11.84 & LibriSpeech \\
*kNN-VC \cite{baas2023voice}  & 45.92 & 27.55 & 19.19 & LibriSpeech \\
*FreeVC \cite{li2023freevc}  & 5.4 & 2.27 & 35.63 & LibriSpeech \\
*YourTTS \cite{casanova2022yourtts}  & 8.65 & 3.36 & 38.23 & LibriSpeech \\
*Phoneme Hallucinator \cite{shan2024phoneme}  & 5.1 & 2.02 & 44.62 & LibriSpeech \\
*DISSC \cite{Maimon2022}  & 13 & 6.9 & 1.7 & VCTK \\
*DDDM-VC \cite{choi2024dddm}  & 3.49 & 1 & 6.25 & VCTK \\
*AutoVC \cite{qian2019autovc}  & 71.3 & 47.1 & 7.5 & VCTK \\
*Seq2seq-VC \cite{liu2021any} & 2.9 & 1.2 & 1.0 & VCTK \\
*VoiceMixer \cite{lee2021voicemixer}  & 4.2 & 2.39 & 20.75 & VCTK \\
*AutoPST \cite{qian2021global}  & 40.6 & 26.7 & 24.1 & VCTK \\
*AutoVC \cite{qian2019autovc}   & 8.53 & 3.54 & 37.32 & VCTK \\
\hline
\end{tabular}
\end{table}

\nocite{Maimon2022, liu2021any, qian2019autovc, qian2021global, wang2021vqmivc, baas2023voice, li2023freevc, casanova2022yourtts, shan2024phoneme, choi2024dddm, lee2021voicemixer}


Comparative analyses of SEG methods remain limited, as many researchers choose not to compare their approaches against competitors, SEG techniques are evaluated based on their ability to reconstruct and generate voices. Table \ref{table5} provides a comparative analysis of SEG models based on WER, CER, and EER across different datasets. The FreeVC \cite{li2023freevc} model on the LibriSpeech dataset demonstrates the lowest WER (5.4\%) and EER (11.28\%), showcasing superior performance in speech generation tasks. In contrast, the kNN-VC \cite{baas2023voice} model reveals significantly higher error rates, with a WER of 45.92\% and EER of 19.19\%, indicating challenges in generating accurate speech.

The analysis also highlights variability in model performance across different datasets, underscoring the complexity of the task. For example, the AutoVC \cite{qian2019autovc} model on the ESD dataset exhibits a high WER (87.0\%) and CER (31.8\%), reflecting difficulties in maintaining accuracy. 

\subsubsection{Text Sentiment Recognition Comparative Analysis}

\begin{table}[ht]
\centering
\caption{TSR Comparative Analysis: *Results derived from cited papers.}
\label{table6}
\begin{tabular}{lrrl}
\hline
\textbf{Model} & \textbf{F1 score ↑} & \textbf{ACC ↑} & \textbf{Datasets} \\
\hline
*XLM- EMO \cite{Bianchi2022} & 0.85 & 0.85 & Affect in Tweets \\
*Kumar 2022 \cite{Kumar2022} & 0.81 & 0.8 & AffectiveText \\
*Supervised learning \cite{oberlander2018analysis} & 0.71 & -- & AffectiveText \\
*Kumar 2022 \cite{Kumar2022} & 0.83 & 0.81 & Aman \\
*Kumar 2022 \cite{Kumar2022} & 0.72 & 0.73 & EmotionLines \\
*Emotion BERT \cite{huang2019emotionx} & -- & 0.71 & EmotionLines \\
*Multi-level multi-head fusion \cite{ho2020multimodal} & -- & 0.61 & EmotionLines \\
*Context \& Speaker modeling \cite{zhang2019modeling} & 0.59 & -- & EmotionLines \\
*Multi-turn dialogue analysis \cite{kao2019model} & 0.70 & -- & EmotionLines \\
*Kumar 2022 \cite{Kumar2022} & 0.81 & 0.79 & ISEAR \\
*Feature selection \cite{singh2018two} & -- & 0.73 & ISEAR \\
*Emotion distribution learning \cite{zhang2018text} & 0.67 & 0.67 & ISEAR \\
*XLM-T Barbieri 2021 \cite{Barbieri2021} & 0.67 & 0.79 & Sem-EVAL 17 \\
Ohman 2020 \cite{ohman2020xed} & 0.83 & 0.84 & XED \\
\hline
\end{tabular}
\end{table}

% \nocite{Bianchi2022, Kumar2022, oberlander2018analysis, huang2019emotionx, ho2020multimodal, zhang2019modeling, kao2019model, singh2018two, zhang2018text, Barbieri2021, ohman2020_ed}


% \begin{figure}[ht]
%     \centering
%     \subfigure[TSR Comparative Analysis]{
%         \includegraphics[width=0.50\textwidth]{Picture5.png}
%         \label{FER}
%     }
%     \subfigure[TSG Comparative Analysis]{
%         \includegraphics[width=0.75\textwidth]{teg_results.jpg}
%         \label{FEG}
%     }
%     \caption{Comparative Analysis of Text Systems}
%     \label{fig:combined_images}
% \end{figure}

Accuracy and F1 score are the most commonly used metrics for TSR. Table \ref{table6} presents a comparison of SOTA methods evaluated on these metrics across different datasets. Notably, the Emotion BERT \cite{huang2019emotionx} model achieves the highest F1 score of 0.88 on the EmotionLines dataset, indicating its effectiveness in accurately recognising emotions from text. Similarly, the Ohman 2020 \cite{ohman2020xed} model demonstrates high F1 score (0.83) and accuracy (0.84) on the XED dataset, reflecting its robustness in TSR. In contrast, models such as AutoVC \cite{qian2019autovc}" on the ESD dataset show significantly lower performance, with an F1 score of 0.47 and accuracy of 0.5, suggesting potential limitations in effectively recognising Text Sentiments. Models such as Kumar 2022 \cite{Kumar2022} and XLM-EMO \cite{Bianchi2022} demonstrate robust performance with F1 scores and accuracies around 0.85 across multiple datasets, showcasing their adaptability and effectiveness. Conversely, models evaluated on more complex datasets, such as the FERV39K dataset, exhibit lower performance. This comparative analysis emphasises the advancements achieved in TSR while also highlighting the need to enhance model accuracy and generalisability across text datasets.

\subsubsection{Text Sentiment Generation Comparative Analysis}
\begin{table}[ht]
\centering
\caption{TSG Comparative Analysis: *Results derived from \cite{achiam2023gpt}.}
\label{table7}
\begin{tabular}{lrrrrrr}
\hline
\textbf{Model} & \textbf{MMLU (\%)} & \textbf{GQPA (\%)} & \textbf{MATH (\%)} & \textbf{HumanEval (\%)} & \textbf{MGSM (\%)} & \textbf{DROP (f1)} \\
\hline
*GPT-4o \cite{achiam2023gpt} & 88.7 & 53.6 & 60.1 & 90.2 & 67.0 & 90.5 \\
*GPT-4T \cite{achiam2023gpt} & 86.5 & 48.0 & 56.5 & 87.1 & 71.9 & 84.1 \\
*GPT-4 \cite{achiam2023gpt} & 86.4 & 35.7 & 53.2 & 84.9 & 74.4 & 84.1 \\
*Claude3 Opus \cite{Claude3} & 86.8 & 50.4 & 57.8 & 86.7 & 74.4 & 86.0 \\
*Gemini Pro 1.5 \cite{reid2024gemini} & 81.9 & N/A & 42.5 & 74.4 & 67.0 & 83.1 \\
*Gemini Ultra 1.0  \cite{team2023gemini} & 83.7 & N/A & 58.5 & 90.7 & N/A & 84.1 \\
*Llama3 400b  \cite{llama} & 86.1 & 48.0 & 53.2 & 88.7 & 67.0 & 83.5 \\
\hline
\end{tabular}
\end{table}

% \nocite{achiam2023gpt, Claude3, reid2024gemini, team2023gemini, llama}

Qualitative methods, such as user studies, primarily assess TSG performance, utilising metrics such as MMLU, GQPA, MATH, HumanEval, MGSM, and DROP (F1) across various tasks. Due to potential biases inherent in user studies, there is considerable variability in the performance of TSG models across different experiments. This variability may stem from the nature of the questions posed, the diversity in answers generated by the TSG models, and the subjective opinions of the respondents. For consistency, we have selected the results from \cite{achiam2023gpt}. Table \ref{table7} evaluates the performance of large language models (LLMs) in text generation across multiple tasks, using metrics such as MMLU, GQPA, MATH, HumanEval, MGSM, and DROP (F1).

The GPT-4 model \cite{achiam2023gpt} achieves the highest MMLU score of 88.7\%, demonstrating its strong performance in multi-task learning. This model also secures the highest HumanEval score of 90.2\%, indicating its capability to generate realistic text. In contrast, models such as Gemini Ultra 1.0 \cite{team2023gemini} display significantly lower performance, with an MMLU score of 83.7\% and low scores across several other metrics. The table illustrates the varying performance across different tasks, reflecting the strengths and weaknesses of each model. For instance, the Claude 3 Opus model \cite{Claude3} achieves high scores in MMLU (86.8\%) and HumanEval (86.7\%), indicating its balanced proficiency in both multi-task learning and text generation.

% Qualitative methods, such as user studies, primarily assess TSG performance, utilising metrics such as MMLU, GQPA, MATH, HumanEval, MGSM, and DROP (F1) across various tasks. Due to potential biases inherent in user studies, there is considerable variability in the performance of TSG models across different experiments. This variability may stem from the nature of the questions posed, the diversity in answers generated by the TSG models, and the subjective opinions of the respondents. For consistency, we have selected the results from \cite{achiam2023gpt}, as shown in table \ref{table6}. The table evaluates the performance of large language models (LLMs) in text generation across multiple tasks, using metrics such as MMLU, GQPA, MATH, HumanEval, MGSM, and DROP (F1). The GPT-4o model \cite{achiam2023gpt} achieves the highest MMLU score of 88.7\%, demonstrating its strong performance in multi-task learning. This model also secures the highest HumanEval score of 90.2\%, indicating its capability to generate realistic text. In contrast, models such as "Gemini Ultra 1.0" display significantly lower performance, with an MMLU score of 83.7\% and low scores across several other metrics. The table further illustrates the varying performance across different tasks, reflecting the strengths and weaknesses of each model. For instance, the Claude3 Opus model \cite{Claude3} achieves high scores in MMLU (86.8\%) and HumanEval (86.7\%), indicating its balanced proficiency in both multi-task learning and text generation.

\section{Challenges and Future Directions}

Despite significant advances in ER and EG across faces, speech, and text, several key challenges remain. The inherent complexity of emotions—often difficult for humans to interpret reliably—creates challenges for machines, especially in speech and text ER and EG, where non-verbal cues are absent. Subtle expressions of emotions, such as micro-expressions, in FEG add further complexity to emotion recognition and generation processes. A promising direction is to integrate multiple modalities, such as facial cues, speech, text, and body language, to create more robust systems. Advancements in natural language processing (NLP), particularly through transformer models like GPT and BERT, are also essential for capturing linguistic nuances and cultural differences in emotional expression. Generating subtle and dynamic emotions in real time is another challenge, especially for interactive applications like virtual reality. Improved real-time emotion tracking is essential to make ER and EG systems more responsive and functional in dynamic environments.

A shortage of large, diverse datasets limits progress in ER and EG. Current datasets often contain biases or labelling errors and lack generalisability, which hampers model performance. Efforts to collect "in-the-wild" datasets that reflect real-world emotional dynamics and include multiple languages would improve model effectiveness and fairness. Standardised evaluation metrics are also needed to enable consistent assessment and comparison of models. Open-access benchmarks would provide clear standards for evaluating models, measuring both accuracy and emotional appropriateness, and fostering progress across the field. Ethical concerns, such as the misuse of deepfake technology, indicate the need for ethical guidelines and detection mechanisms without hindering technological progress. Finally, techniques like model compression and the use of pretrained models as a foundation for new applications can reduce computational costs.

\section{Conclusion}
This survey explored state-of-the-art methods in emotion recognition and generation across facial, vocal, and textual modalities. With advances in AI, deep learning techniques have enhanced both the accuracy of emotional analysis and the realism of generated content. In particular, deep learning models, such as CNNs and attention-based architectures, have improved FER by learning features directly from raw data. Likewise, SER has advanced through models that integrate linguistic and acoustic features, enhancing classification accuracy through prosodic and contextual analysis. Despite progress, challenges remain in FEG, SEG and TSG. In FEG, accurately capturing the nuances of facial muscle movements and micro-expressions presents substantial difficulty, while ensuring emotional coherence across frames adds further complexity. Similarly, generating realistic emotions in speech and text requires addressing the intricate subtleties of tone, intonation, context, and emotional consistency. Limited labelled data, especially for in-the-wild systems, also impedes model robustness and generalizability. Future research should focus on expanding dataset diversity and improving models for under-explored modalities like speech and text. Multimodal approaches, enabling emotion analysis and generation across faces, speech, and text, hold promise. Ethical considerations, such as preventing misuse in deepfakes, should also guide future developments, paving the way for more empathetic and context-aware AI applications.


%\bibliographystyle{unsrt}
%\bibliography{template}  %%% Uncomment this line and 
\documentclass{MITstyle}

%\usepackage[table]{xcolor}
\usepackage{chngcntr}
\usepackage{hyperref}
\usepackage{microtype}

\title{A Lightweight and Extensible Cell Segmentation and Classification Model for Whole Slide Images}

\author{Nikita Shvetsov~$^{1, }$\footnote{Correspondence e-mail: nikita.shvetsov@uit.no}, Thomas K. Kilvaer~$^{2, 3}$, Masoud Tafavvoghi~$^{4}$, Anders Sildnes~$^{1}$, \\ Kajsa Møllersen~$^{4}$, Lill-Tove Rasmussen Busund~$^{5, 6}$, Lars Ailo Bongo~$^{1}$ \\
%
\vspace{1em} % Space between authors and afilliations
%
\normalfont{\small $^{1}$Department of Computer Science, UiT The Arctic University of Norway}\\
\normalfont{\small $^{2}$Department of Oncology, University Hospital of North Norway}\\
\normalfont{\small $^{3}$Department of Clinical Medicine, UiT The Arctic University of Norway}\\
\normalfont{\small $^{4}$Department of Community Medicine, UiT The Arctic University of Norway}\\
\normalfont{\small $^{5}$Department of Medical Biology, UiT The Arctic University of Norway} \\
\normalfont{\small $^{6}$Department of Clinical Pathology, University Hospital of North Norway} %\vspace{2em}
}

\begin{document}
\maketitle

\section*{Abstract}

% \begin{abstract}
% Developing clinically useful cell-level analysis tools in digital pathology remains challenging due to limitations in dataset granularity, inconsistent annotations, computational demands of advanced models, and difficulties in integrating new technologies into clinical workflows. To address these challenges, we propose a multi-faceted solution that enhances data quality, model performance, and usability to create a lightweight and extensible cell segmentation and classification model.

% First, we update data labels by employing a cross-relabeling process that refines the labels of two existing datasets, PanNuke and MoNuSAC, to create a new unified dataset with enhanced granularity, encompassing seven distinct cell types. Second, we leverage the H-Optimus foundation model as a fixed encoder to improve feature representation for simultaneous cell segmentation and classification tasks. Third, to address the computational demands of foundation models, we employ knowledge distillation to reduce model size and complexity while maintaining comparable performance. Finally, to facilitate integration into clinical workflows, we integrate the distilled model into the QuPath software, a widely used open-source platform in digital pathology.

% Our results demonstrate improvements in cell segmentation and classification performance using the H‑Optimus-based model compared to a CNN-based model. Specifically, the average $R^2$ improved from 0.575 to 0.871, and the average $PQ$ score improved from 0.450 to 0.492, indicating better alignment with actual cell counts and enhanced segmentation and classification quality. Furthermore, the distilled student model maintains performance comparable to the larger foundation model while reducing the parameter count by a factor of 48.
% Overall, by reducing computational complexity and integrating it into existing workflows, the proposed approach may significantly impact diagnostic processes, reduce the workload of pathologists, and contribute to improved patient outcomes. Though our approach shows potential enhancements in efficiency and usability of cell segmentation and classification models in digital pathology, extensive validation is needed to deploy these models in clinical practice.
% \end{abstract}

%%% shortened abstract
\begin{abstract}
Developing clinically useful cell-level analysis tools in digital pathology remains challenging due to limitations in dataset granularity, inconsistent annotations, high computational demands, and difficulties integrating new technologies into workflows. To address these issues, we propose a solution that enhances data quality, model performance, and usability by creating a lightweight, extensible cell segmentation and classification model. 

First, we update data labels through cross-relabeling to refine annotations of PanNuke and MoNuSAC, producing a unified dataset with seven distinct cell types. Second, we leverage the H-Optimus foundation model as a fixed encoder to improve feature representation for simultaneous segmentation and classification tasks. Third, to address foundation models' computational demands, we distill knowledge to reduce model size and complexity while maintaining comparable performance. Finally, we integrate the distilled model into QuPath, a widely used open-source digital pathology platform. 

Results demonstrate improved segmentation and classification performance using the H-Optimus-based model compared to a CNN-based model. Specifically, average $R^2$ improved from 0.575 to 0.871, and average $PQ$ score improved from 0.450 to 0.492, indicating better alignment with actual cell counts and enhanced segmentation quality. The distilled model maintains comparable performance while reducing parameter count by a factor of 48. By reducing computational complexity and integrating into workflows, this approach may significantly impact diagnostics, reduce pathologist workload, and improve outcomes. Although the method shows promise, extensive validation is necessary prior to clinical deployment.
\end{abstract}
\clearpage

\section{Introduction}
In digital pathology, accurate segmentation and classification of cells are crucial for many diagnostic, prognostic, and predictive analyses \cite{Jaber_Beziaeva_etal._2019,Lin_Pan_etal._2022,Park_Ock_etal._2022,Shen_Choi_etal._2024}. Nowadays, developments in computational pathology offer multiple solutions \cite{H._Qu_P._Wu_etal._2020,Javed_Mahmood_etal._2020} to utilize cell-level datasets to train machine learning models that solve these problems. The quality and specificity of training datasets are critical for robust and accurate models. Adhering to the principle of "garbage in, garbage out", it is essential to ensure that these datasets are extensively and accurately labeled with distinct classes that reflect the diverse biological characteristics of different cell types. Unfortunately, the number of open-source datasets comprising such high-quality annotations is limited. Existing cell segmentation datasets \cite{Gamper_Koohbanani_etal._2019,Graham_Vu_etal._2019,Verma_Kumar_etal._2021} may offer extensive annotations for certain cell types while providing more general labels for others. For example, in PanNuke, which is one of the largest open-source datasets comprising labeled cells, various types of morphologically and functionally different inflammatory cells like macrophages and lymphocytes are clustered in a broad "inflammatory" class. Consequently, these classes are frequently omitted from analyses or aggregated into broader meta-classes \cite{Gamper_Koohbanani_etal._2020} and likely interfere with other cell classes included in the dataset. This and similar inconsistencies in annotation granularity limit the ability of machine learning models to learn the comprehensive and nuanced features necessary for accurate cell segmentation and classification. To address these challenges, methods for refining and standardizing dataset annotations are essential to enhance the quality of training data.

A complementary approach to mitigate the absence of high-quality training data is the use of foundation models. Foundation models as encoders are defined as large-scale, versatile networks pre-trained on vast, diverse datasets using self-supervised learning, contrasting with convolutional neural network (CNN) pre-trained encoders that rely on supervised learning with labeled data. In practice, foundation models leverage enormous amounts of weakly or unlabeled data from millions of whole slide images (WSIs) and employ self-attention mechanisms to capture long-range dependencies and global context \cite{Chen_Ding_etal._2024,Saillard_Jenatton_etal._2024,Vorontsov_Bozkurt_etal._2024,Xu_Usuyama_etal._2024}. As a consequence, foundation models are able to produce transferable feature representations across different cell types and tissue environments. The feature representations can be leveraged by decoder networks to produce segmentation masks and pixel-level classifications. Because foundation models have comprehensive feature representations, they can be effectively fine-tuned using much smaller amounts of cell-level data compared to the large datasets needed to train models from scratch. Furthermore, foundation models incorporate adversarial training elements or contrastive learning \cite{Chen_Ding_etal._2024,Xu_Usuyama_etal._2024}, enhancing their resilience and adaptability by exposing them to challenging and varied scenarios during training. This may result in more generalizable models, often making them well-suited for diverse and complex tasks in digital pathology.

Despite the inherent advantages of foundation models, their deployment for practical use faces its own obstacles. In particular, they require substantial computational power, financial investments and rigorous testing to ensure reliability and efficacy for a given task \cite{Akkus_Dangott_etal._2022,Dragomir_Cocuz_etal._2022,Go_2022,Jafri_Farooqui_etal._2024}. Moreover, while foundation models enhance feature representation and performance, they depend on the quality of available annotations for decoder fine-tuning and, like any other model, cannot resolve existing inconsistencies or ambiguities in data labels. Therefore, there remains a critical need for solutions that address both data quality and practical deployment considerations.
Further, integrating new technologies into existing clinical workflows often encounters resistance, as it necessitates adjustments to established diagnostic processes. So, there is a need to develop solutions that could be integrated into current practices, minimizing the burden on medical professionals to adopt new tools \cite{King_Williams_etal._2023}.

Existing solutions \cite{Goldsborough_Philps_etal._2024,Hörst_Rempe_etal._2024}, while addressing some aspects of these challenges, fall short in providing a comprehensive approach. To address the data quality and clinical deployment issues, we propose a multi-faceted solution that encompasses data refinement, model optimization, and integration with existing pathology tools (\hyperref[fig:fig1]{Figure 1}). The outcome is a lightweight cell segmentation and classification model that can be integrated into digital pathology workflows for practical clinical use.

\begin{figure}[h!]
    \centering
    \includegraphics[width=\textwidth, height=0.82\textheight, keepaspectratio]{images/Figure_1.pdf}
    \caption{Overview of the proposed solution, including 1) Data refinement using cross-relabeling, 2) Teacher model development and fine tuning, 3) Student model optimization with knowledge distillation and 4) Student model and QuPath integration}
    \label{fig:fig1}
\end{figure}
\clearpage

Our approach begins with preparing the data for the fine-tuning and training of the machine learning models. We create a refined dataset, acquired via cross-relabeling two cell-level datasets, enhancing annotation specificity and consistency of the labeled data. Subsequently, we create a cell segmentation and classification model based on the foundation model. We leverage the foundation model as a fixed encoder and fine-tune a decoder using the refined dataset to improve generalization across diverse tissue- and cell types.
To ensure that the model remains lightweight and deployable in a possibly resource-constrained environment, we employ knowledge distillation to approximate the functionality of the foundation model. Finally, to facilitate the practical application of our model in digital pathology workflows, we integrate it with the QuPath \cite{Bankhead_Loughrey_etal._2017} application. Each methodological component contributes to the overarching goal of enhancing model performance, generalizability, and usability in clinical settings.

The primary contributions of this paper are:
\begin{enumerate}
    \item \textit{Data labels refinement through cross-relabeling:}
    
    We propose a new method for refining labels of cell-level datasets through cross-relabeling. This method employs classification models to re-label broad and ambiguous instances, resulting in a more diverse dataset. Our evaluation demonstrates that these classification models achieve high accuracy on test subsets, indicating the reliability of the method for label refinement.

    \item \textit{Enhanced model performance via foundation models:}
    
    We employ a foundation model as a feature extractor for the cell segmentation and classification task. In comparison with training a CNN model from scratch, the foundation model backbone only needs fine-tuning, which significantly reduces training time, computational resources and data requirements. We show that using a foundation model encoder leads to better performance in cell segmentation and classification networks than using a CNN-based encoder. This improvement may enable the model to generalize more effectively across various tissue types and imaging methods.
    
    \item \textit{Model optimization through knowledge distillation:}
    
    We show that a smaller student model trained using knowledge distillation on the refined dataset obtained via our cross-relabeling approach from a foundation model achieves comparable performance in cell segmentation and quantification tasks. As a result, this model is more suitable for deployment in environments without high-performance computing resources.
    
    \item \textit{Integration with QuPath:}
    
    We integrate the distilled cell segmentation and classification model into QuPath, a widely used open-source digital pathology platform, to accelerate clinical adaptation by enabling pathologists to more easily incorporate advanced computational tools into their existing workflows.
\end{enumerate}

Through these methodological steps, we aim to bridge the gap between advanced machine learning techniques and practical clinical applications, making accurate and efficient digital pathology accessible in a broader range of healthcare settings.

\section{Refining Existing Datasets Using Cross-Relabeling}
To address the limitations of sparse and ambiguous labeling of cell-level datasets, we propose a generalizable cross-relabeling strategy that can be applied to any dataset containing broadly categorized or imprecisely labeled cell types. This approach involves training and subsequently leveraging classification models to refine broad categories into more specific or biologically relevant classes.
When applied to cell-level data, the methodology includes extracting individual cell images from the dataset patches, preprocessing these images to standardize the size and accommodate partial cells, and then training deep learning classifiers capable of distinguishing between the finer cell subtypes within the coarser categories. 
To illustrate our approach, we focus on the PanNuke \cite{Gamper_Koohbanani_etal._2020, Gamper_Koohbanani_etal._2019} and MoNuSAC \cite{Verma_Kumar_etal._2021} datasets that we have used to train models for cell quantification in our previous works \cite{Shvetsov_Grønnesby_etal._2022,Shvetsov_Sildnes_etal._2024}. We find that for better cell differentiation we have to introduce more granular labels. PanNuke includes a broad classification of "inflammatory" cells, encompassing lymphocytes, macrophages, and neutrophils. Each cell type differs significantly in structure, function, and clinical relevance. Conversely, MoNuSAC uses the label "epithelial" for a class that comprises both benign epithelial cells and malignant neoplastic cells. This practice makes it challenging to differentiate between benign and malignant epithelial cells in the dataset, which is a critical distinction when identifying tumor areas within tissue samples. To address these issues, we implement a cross-relabeling strategy as shown in \hyperref[fig:fig2]{Figure 2}. The key components are two classification models: one is trained on singular cell images from PanNuke data to classify the epithelial meta-class into epithelial and neoplastic classes. The other is trained on MoNuSAC to refine the inflammatory class into lymphocytes, neutrophils, and macrophages.

\begin{figure}[h!]
    \centering
    \includegraphics[width=\textwidth]{images/Figure_2.pdf}
    \caption{Refined dataset generation via cross relabeling}
    \label{fig:fig2}
\end{figure}

The refining approach consists of three consecutive steps. The first is the preprocessing step, in which we extract individual cells from both datasets (\hyperref[fig:fig3]{Figure 3}). The specifics of PanNuke and MoNuSAC patch preparation before cell preprocessing are provided in \hyperref[chap:S1]{Appendix S1}.

\begin{figure}[h!]
    \centering
    \includegraphics[width=\textwidth]{images/Figure_3.pdf}
    \caption{Cell instances preprocessing including (1) cell map extraction, (2) bounding box delineation, (3) adjusting cell boxes and (4) cropping and resizing of cell images}
    \label{fig:fig3}
\end{figure}

During preprocessing, we extract cell type maps from the ground truth label mask and calculate bounding boxes around each cell instance. To accommodate partial cells at patch borders, a common issue in cropped patch images, we employ mirror padding and extend the field of view of the cell label by 15 pixels to capture adjacent cells. We then crop and resize the identified regions to $64 \times 64$ pixels using bicubic interpolation.

The preprocessed PanNuke dataset comprises 68,031 neoplastic and 23,207 epithelial cell images, while MoNuSAC comprises  33,104 lymphocytes, 1,252 neutrophils, and 1,695 macrophages, which we subsequently use in training cell classification models and classifying the cell image data \hyperref[fig:S2]{Appendix Figure S2 (1)}. 

The next step is to train two distinct ResNet50-based classifiers tailored to address the specific labeling challenges inherent in each dataset. We use ResNet50 for classification models due to its proven effectiveness for image classification tasks in histopathology \cite{pan2022reviewmachinelearningapproaches}, and its compatibility with small images. For the PanNuke dataset, we design the classifier, trained on MoNuSAC data, to disaggregate the heterogeneous "inflammatory" cell category into distinct subtypes: lymphocytes, macrophages, and neutrophils. Similarly, for the MoNuSAC dataset, the classifier is trained on PanNuke data and distinguishes between benign and malignant epithelial cells within the overarching "epithelial" label. By applying these targeted classifiers to their respective datasets, we assign more specific labels to individual cell instances, thus enabling us to create a unified dataset.
To ensure a balanced representation of classes, we train both models on datasets that had been equalized to match the size of the least represented class. Thus, we obtain datasets comprising 23,207 samples per class for PanNuke and 1,252 samples per class for MoNuSAC data. Next, we partition both of them into training (70\%), validation (20\%), and testing (10\%) subsets. To mitigate the risk of overfitting, we use a single dropout layer with a rate of p=0.5 in both models and data augmentation using randomized color perturbations, rotation, and horizontal and vertical flipping. We employ AdamW optimizer and the cross-entropy loss function for the training criterion.

To evaluate the two trained models, we measure the classification accuracy on the respective test subsets. The accuracies on the test subset for both classifiers are presented in \hyperref[tab:1]{Table 1}. The PanNuke model achieves an average accuracy of 93.57\%, with higher accuracy for neoplastic cells (96.06\%) compared to epithelial cells (86.26\%). The confusion matrix in Figure A3.1 shows that the model predominantly distinguishes accurately between epithelial and neoplastic tissues, with a substantial number of correct classifications and relatively few misclassifications. The MoNuSAC model demonstrates an average accuracy of 98.92\%, excelling in classifying lymphocytes (99.67\%) and macrophages (94.12\%), with lower performance for neutrophils (85.71\%). The confusion matrix in Figure A3.2 shows that the model identifies lymphocytes and performs reasonably well with macrophages and neutrophils.

\begin{table}[h!]
\renewcommand{\arraystretch}{1.5}
  \centering
  \caption{Cell classification results for PanNuke and MoNuSAC trained models (CI 95\%).}
  \label{tab:1}
  \begin{tabular}{|l|c|c|}
   \hline
   %\rowcolor{gray!30}
    Accuracy               & PanNuke model              & MoNuSAC model              \\
    \hline
    Average      & 0.936 (0.931--0.941)         & 0.989 (0.986--0.993)        \\
    \hline
    Neoplastic   & 0.961 (0.956--0.965)         & -                          \\
    \hline
    Epithelial   & 0.863 (0.849--0.877)         & -                          \\
    \hline
    Lymphocytes  & -                          & 0.997 (0.995--0.999)        \\
    \hline
    Neutrophils  & -                          & 0.857 (0.796--0.918)        \\
    \hline
    Macrophages  & -                          & 0.941 (0.906--0.976)        \\
    \hline
  \end{tabular}
\end{table}

Finally, during the last step, we use the model trained on PanNuke data for epithelial cells in MoNuSAC and the model trained on MoNuSAC for the inflammatory cells class in PanNuke. Specifically, we use classifier models to relabel epithelial cells in MoNuSAC and inflammatory cells in PanNuke data. Then we combine cells with refined labels and the rest of the cells in both datasets to create a refined dataset (\hyperref[fig:S2]{Appendix Figure S2 (2)}). The process of relabeling cells and visualizing them on a patch is shown in \hyperref[fig:fig4]{Figure 4}. The cell counts in the refined dataset are provided in \hyperref[tab:S4]{Appendix Table S4}.

\begin{figure}[h!]
    \centering
    \includegraphics[width=\textwidth, height=0.42\textheight, keepaspectratio]{images/Figure_4.pdf}
    \caption{Cell relabeling procedure for epithelial and inflammatory cell classes}
    \label{fig:fig4}
\end{figure}

%\hfill

Relabeling and combining datasets have been explored in a prior study \cite{Parulekar_Kanwat_etal._2023}, where consecutive fine-tuning on multiple datasets was employed to account for hierarchical class label structures. While the method presented in \cite{Parulekar_Kanwat_etal._2023} is intuitive, it often lacks consistency and requires multiple fine-tuning runs, which can be cumbersome and time-consuming. 
In contrast, cross-relabeling simplifies this process by using specialized classification models tailored to each dataset's specific labeling challenges. This approach provides better transparency and produces a unified dataset encompassing seven distinct cell types across multiple tissue samples, enhancing data diversity for further model training or fine-tuning.

Despite these improvements, cross-relabeling does not entirely resolve issues related to poor labeling quality or the amount of labeled data. Specifically, our results show lower accuracies persist for underrepresented classes, such as macrophages, which may stem from a limited sample availability and intrinsic challenges in distinguishing these cells based solely on H\&E staining. Furthermore, while our method enhances label specificity, it relies on the initial quality of the broad labels; thus, any fundamental inaccuracies in the original annotations can propagate through the relabeling process. Addressing the overall problem of limited data labels may require integrating additional data sources or utilizing complementary immunohistochemical staining methods.
Although the reported performance metrics are obtained from evaluations on the native test sets of each dataset, it is important to note that the primary application of these classifiers is to perform cross-relabeling, where a model trained on one dataset (e.g., PanNuke) is applied to another (e.g., MoNuSAC) and vice versa. We acknowledge that a more systematic evaluation of cross-dataset generalization is needed and could be performed in future work.

Overall, the refined dataset produced by our approach can enhance the supervised training or fine-tuning of cell segmentation and classification models, especially those that utilize pre-trained foundation models to improve feature extraction robustness. In addition, these models can detect nuanced classes that enable researchers to conduct more detailed analyses of biological processes in computational pathology.

\section{Foundation models for robust cell segmentation and classification}

Accurate cell segmentation and classification in digital pathology are hindered by limited labeled data and the fact that conventional CNNs are unable to capture global contextual information due to their local receptive field constraints \cite{Gheflati_Rivaz_2022,Yang_Marcus_etal.}. Traditional approaches in cell quantification have predominantly relied on CNN encoders, such as ResNet50, given their proven effectiveness in semantic segmentation tasks \cite{Deshmane_2023,Graham_Vu_etal._2019,Mukasheva_Koishiyeva_etal._2024,Stringer_Wang_etal._2021}. However, approaches that include fine-tuning of pretrained CNNs, data augmentation, and stain normalization to partially increase data variability and address staining differences often fail to achieve the necessary generalization and robustness across diverse tissue types and staining conditions \cite{G._Wang_W._Li_etal._2018,Gao_Bagci_etal._2018,Karim_El_Khoury_Martin_Fockedey_etal._2021}.

To overcome these challenges, we leverage an encoder-decoder network that uses a foundation model as the encoder and a CNN upsampling decoder (\hyperref[fig:fig5]{Figure 5}) for simultaneous cell segmentation and classification in 2D patches extracted from WSIs. Foundation models with transformer-based architectures are viable alternatives to CNN-based encoders \cite{Shamshad_Khan_etal._2023,Sourget_2023}. They enable the creation of more advanced architectures that can decode or transform learned features more effectively \cite{Chen_Duan_etal._2023,Cheng_Misra_etal._2022,Xie_Wang_etal._2021}.

\begin{figure}[h!]
    \centering
    \includegraphics[width=\textwidth]{images/Figure_5.pdf}
    \caption{UNETR-like model with foundational model as backbone}
    \label{fig:fig5}
\end{figure}

By utilizing a transformer-based encoder, we incorporate global contextual information into the feature extraction process, which is a key advantage of such architectures \cite{Chen_Lu_etal._2021}. This foundation model integration facilitates accurate pixel-wise segmentation and classification without the need for extensive encoder training, thereby potentially improving generalization across varied cellular structures and tissue types.
In our implementation, we employ a modified UNETR \cite{Hatamizadeh_Tang_etal._2021} architecture that combines a vision transformer (ViT) \cite{Dosovitskiy_Beyer_etal._2021} encoder with a CNN-based decoder. The encoder utilizes the pretrained H-Optimus foundation model, which contains 1.1 billion parameters and is trained on over 500,000 H\&E stained WSIs \cite{Saillard_Jenatton_etal._2024}. We extract outputs from four evenly spaced transformer blocks $Z_i$, where $i \in [1, 14, 26, 38]$, to serve as residual connections for the CNN decoder. We select these blocks based on our observation that features from non-adjacent levels of the encoder lead to better overall performance on the test subset.

The CNN decoder upsamples the feature representations, acquired from the transformer blocks, to generate an intermediate vector that is handled by two task-specific layers that generate cell segmentation and classification masks. The first task-specific layer is the ‘Cellpose head’,  which is used to delineate cell instances. The layer generates horizontal and vertical gradient maps to form vector fields that are refined through gradient tracking in a post-processing step using the Cellpose algorithm \cite{Stringer_Wang_etal._2021}, known for its efficacy in cell segmentation tasks and generalizability across multiple domains \cite{Pachitariu_Stringer_2022,Stringer_Pachitariu_2024}. The second task-specific layer is the "Cell type head", which assigns labels to individual pixels. In the post-processing step, we determine the output classification label of each segmented cell instance by majority voting over the labeled pixels that comprise the cell in the segmentation map.

To evaluate model performance and measure the impact of adding a foundation model as backbone, we compare it to a ResNet50-based model. ResNet50 is a widely used solution for encoders in segmentation architectures in the medical domain \cite{Deshmane_2023,Graham_Vu_etal._2019,Mukasheva_Koishiyeva_etal._2024,Stringer_Wang_etal._2021}. For the H-Optimus-based model, we utilize frozen weights for the encoder and only fine-tune the decoder to take advantage of the extensive pre-training of the foundation model. For the ResNet50-based model we start with ImageNet \cite{Deng_Dong_etal.} weights and train both encoder and decoder parts. Hyperparameters for the training step are set to be identical, where possible, for comparable evaluation. 
For this evaluation, we deliberately use the PanNuke dataset to provide a standardized and controlled comparison between the H‑Optimus and ResNet50-based models (\hyperref[fig:S2]{Appendix Figure S2 (3)}). Specifically, we use two of the default PanNuke dataset splits (66\%) for training and validation, and reserve the third split (33\%) for testing.

To address the challenge of cell class imbalance in the PanNuke dataset, which is a common characteristic in most cell-level H\&E patch datasets, both models’ training processes employ a weighted loss function comprising cross-entropy and focal loss \cite{Lin_Goyal_etal._2018}. The focal loss component is adjusted with coefficients derived from each cell class' instance frequency, emphasizing learning from underrepresented classes and enhancing the model's sensitivity to rare but significant cellular patterns. The cross-entropy loss is augmented with spectral decoupling regularization \cite{Pezeshki_Kaba_etal._2021,Pohjonen_Stürenberg_etal._2022} and spatially varying label smoothing \cite{Islam_Glocker_2021}, which potentially stabilizes training and improves generalization in case of complex tissue morphologies. For optimization, we employ the \textit{AdamW} \cite{Loshchilov_Hutter_2019} to counter unbalanced class scenarios, with cosine annealing learning rate scheduler.

We utilize the scikit-learn library \cite{Van_der_Walt_Schönberger_etal._2014} and HoVer-Net \cite{Graham_Vu_etal._2019} implementations of $R^2$ (the coefficient of determination) and $PQ$ (panoptic quality) to evaluate our experiments. Complete mathematical formulations and detailed explanations of these metrics are provided in \hyperref[chap:S5]{Appendix S5}. To compute confidence intervals, we use nonparametric bootstrapping, where after calculating the metric on the full sample, we generated 1000 bootstrap replicates by resampling with replacement and then determined the 95\% confidence intervals as the 2.5th and 97.5th percentiles of the resulting empirical distribution.

%\hfill

The model comparisons are summarized in \hyperref[tab:2]{Table 2}. The H‑Optimus-based model achieves higher $R^2$ across all cell classes compared to the ResNet50-based model, which means that its predictions are more closely aligned with the PanNuke cell counts, indicating a stronger correlation with the observed data. Notably, the improvement of $R^2_{dead}$ may be an indicator of better global contextual representations provided by the foundation model backbone. In terms of segmentation and classification quality combined, measured by the PQ score, the H‑Optimus-based model demonstrates notable improvements across most cell classes. Overall, the average $R^2$ improved from 0.575 to 0.871, while the average $PQ$ score improved from 0.450 to 0.492, demonstrating better performance of the H-Optimus-based model.

\begin{table}[h!]
\renewcommand{\arraystretch}{1.5}
  \centering
  \caption{Cell quantification metrics for baseline and proposed models (CI 95\%).}
  \label{tab:2}
  \begin{tabular}{|l|c|c|}
    \hline
    %\rowcolor{gray!30}
    Metric             & Resnet50-based            & H-optimus-based              \\
    \hline
    $R^2_{neoplastic}$    & 0.681 (0.576--0.769)       & \textbf{0.941 (0.917--0.960)} \\
    \hline
    $R^2_{inflammatory}$  & 0.863 (0.778--0.903)       & \textbf{0.949 (0.918--0.966)} \\
    \hline
    $R^2_{connective}$    & 0.600 (0.488--0.698)       & 0.609 (0.436--0.772)          \\
    \hline
    $R^2_{dead}$          & 0.097 (-11.389--0.669)     & 0.925 (0.404--0.982)          \\
    \hline
    $R^2_{epithelial}$    & 0.635 (0.490--0.747)       & \textbf{0.930 (0.886--0.964)} \\
    \hline
    $PQ_{neoplastic}$       & 0.517 (0.499--0.535)       & \textbf{0.589 (0.575--0.604)} \\
    \hline
    $PQ_{inflammatory}$     & 0.455 (0.429--0.482)       & \textbf{0.528 (0.507--0.549)} \\
    \hline
    $PQ_{connective}$       & 0.416 (0.400--0.431)       & \textbf{0.451 (0.436--0.465)} \\
    \hline
    $PQ_{dead}$             & 0.374 (0.342--0.408)       & 0.292 (0.209--0.365)          \\
    \hline
    $PQ_{epithelial}$       & 0.488 (0.460--0.519)       & \textbf{0.599 (0.579--0.618)} \\
    \hline
  \end{tabular}
\end{table}

Our results  show that integrating the H‑Optimus foundation model within the UNETR architecture enhances the model's ability to segment and classify cells across diverse tissues from PanNuke data. The pretrained transformer encoder provides robust feature representations, resulting in higher average $R^2$ and $PQ$ scores compared to the CNN-based model. This leads to more reliable cell quantification and more accurate downstream analysis. Additionally, the streamlined fine-tuning process reduces computational overhead and training time, making the model more adaptable for new data.

Despite these advancements, the foundation model-based approach does not fully resolve all challenges related to cell segmentation and classification. We observe lower metric scores for underrepresented classes in the training data. Furthermore, foundation models typically encompass billions of parameters, resulting in substantial computational and memory requirements. It therefore poses challenges for deployment in resource-constrained environments, limiting their practical applicability in certain clinical settings.

\section{Model optimization via Knowledge Distillation}

To address the limitations posed by the extensive size of foundation models, we implement knowledge distillation — a model compression technique that leverages the teacher-student paradigm \cite{Hinton_Vinyals_etal._2015}. By training a smaller, more efficient student model to replicate the output of a larger, pre-trained teacher model, we retain performance while significantly reducing the model's complexity and resource requirements (\hyperref[fig:fig6]{Figure 6}).

\begin{figure}[h!]
    \centering
    \includegraphics[width=\textwidth, height=0.45\textheight, keepaspectratio]{images/Figure_6.pdf}
    \caption{Knowledge distillation framework for training a student model using a pre-trained teacher}
    \label{fig:fig6}
\end{figure}

We employ knowledge distillation to compress the H‑Optimus-based teacher model into a more efficient student model. The teacher model is the modified UNETR architecture with the H‑Optimus foundation model described in the previous chapter. The student model is based on a UNet architecture augmented with residual connections and incorporates a smaller ViT encoder with 9 million parameters \cite{Steiner_Kolesnikov_etal._2022,Wightman_2019}. 

First, we fine-tune the teacher model using the refined dataset from the cross-relabeling procedure (Section 2). Initially we train the decoder of the teacher model while keeping the encoder weights frozen. We split the refined dataset into train (70\%), validation (20\%) and test (10\%) subsets (\hyperref[fig:S2]{Appendix Figure S2 (4)}). During fine-tuning, we use the train and validation subsets, while leaving the test subset for model evaluation. We set the training procedure and model hyperparameters to be identical to those that were used to demonstrate the utility of foundation models for the simultaneous cell segmentation and classification task.

Next, we perform knowledge distillation from teacher to student using the refined dataset used to fine-tune the teacher model. The student model is trained to replicate the teacher model's outputs. We utilize a specialized loss function that aligns the student's predicted probability distribution with the teacher's, incorporating the teacher's class probability distribution derived from the output. Following the methodology of Hinton et al. \cite{Hinton_Vinyals_etal._2015}, we experiment with various hyperparameter settings for the temperature ($T$) and the balancing coefficients ($\alpha$ and $\beta$) in the loss function. We vary $T$ from 1 to 20 and adjust $\alpha$ and $\beta$ to balance the distillation and student losses. Through iterative tuning and evaluation, we identify that setting $T=14$, $\alpha=0.3$, and $\beta=0.7$ yields a configuration that converges and closely approximates the teacher model's performance during training.

Finally, we assess the performance of both models using the $R^2$ and $PQ$ (defined in \hyperref[chap:S5]{Appendix S5}) on the test set of the refined dataset (\hyperref[tab:3]{Table 3}). We observe that the 95\% confidence intervals overlap for most cell types, so we cannot claim statistically significant performance differences between the teacher and student models. One exception appears in the neoplastic class. The teacher model produces an $R^2$ of 0.919, while the student model shows an $R^2$ of 0.852. In addition, the student model achieves higher $PQ$ values for the neoplastic and connective classes, though the confidence intervals show overlap.

\begin{table}[h!]
\renewcommand{\arraystretch}{1.5}
  \centering
  \caption{Cell quantification metrics for teacher and distilled student models (CI 95\%).}
  \label{tab:3}
  \begin{tabular}{|l|c|c|}
    \hline
    %\rowcolor{gray!30}
    Metric & Teacher & Student \\
    \hline
    $R^2_{neoplastic}$    & \textbf{0.919} (0.898--0.939) & 0.852 (0.800--0.891) \\
    \hline
    $R^2_{lymphocyte}$    & 0.969 (0.956--0.977)         & 0.969 (0.956--0.978) \\
    \hline
    $R^2_{connective}$    & 0.694 (0.548--0.809)         & 0.618 (0.469--0.741) \\
    \hline
    $R^2_{dead}$          & 0.755 (0.400--0.908)         & 0.424 (0.100--0.731) \\
    \hline
    $R^2_{epithelial}$    & 0.922 (0.870--0.958)         & 0.843 (0.738--0.917) \\
    \hline
    $R^2_{macrophage}$    & 0.384 (-0.369--0.724)        & 0.704 (0.352--0.859) \\
    \hline
    $R^2_{neutrofil}$     & 0.854 (0.578--0.929)         & 0.833 (0.502--0.925) \\
    \hline
    $PQ_{neoplastic}$       & 0.581 (0.569--0.593)         & 0.601 (0.588--0.613) \\
    \hline
    $PQ_{lymphocyte}$       & 0.536 (0.520--0.553)         & 0.563 (0.544--0.579) \\
    \hline
    $PQ_{connective}$       & 0.436 (0.421--0.451)         & 0.457 (0.441--0.474) \\
    \hline
    $PQ_{dead}$             & 0.272 (0.235--0.315)         & 0.279 (0.201--0.369) \\
    \hline
    $PQ_{epithelial}$       & 0.522 (0.500--0.545)         & 0.530 (0.506--0.555) \\
    \hline
    $PQ_{macrophage}$       & 0.524 (0.459--0.588)         & 0.474 (0.405--0.543) \\
    \hline
    $PQ_{neutrofil}$        & 0.541 (0.490--0.592)         & 0.565 (0.522--0.607) \\
    \hline
  \end{tabular}
\end{table}


We further decompose the $PQ$ metric into its $SQ$ and $DQ$ components (\hyperref[tab:S6]{Appendix Table S6}). Both models produce nearly identical $SQ$ values, which indicates that they predict instance boundaries with similar precision. Although the student model shows some improvement in $DQ$ scores for certain classes, the confidence intervals overlap and do not confirm a statistically significant difference.

We observe that the student and teacher models yield comparable detection performance despite the student model using a much smaller and simpler architecture. A model with fewer parameters reduces the risk of overfitting when training data are scarce relative to the model’s complexity \cite{Farias_Ludermir_etal._2022}. The knowledge distillation process also encourages the student model to focus on the most generalizable detection features learned from the teacher. These factors enable the student model to achieve similar detection performance across different cell types.

Additionally, considering the model sizes reported in \hyperref[tab:4]{Table 4}, the distilled model achieves a significant reduction compared to the teacher model, with a 48-fold decrease in parameter count and a 5.5-fold reduction in on-disk size. In inference mode, the teacher model requires 16 GB of VRAM for a batch size of 32, while the distilled model only needs 3 GB of VRAM for the same batch size. These reductions make the distilled model significantly more practical for fine-tuning and deployment in resource-constrained environments.

\begin{table}[h!]
\renewcommand{\arraystretch}{1.5}
  \centering
  \caption{Parameter counts and size of teacher and distilled model}
  \label{tab:4}
  \adjustbox{max width=\textwidth}{%
  \begin{tabular}{|l|c|c|c|}
    \hline
    %\rowcolor{gray!30}
    Metric & H-optimus-based (Teacher) & mobileViT-based (Student) & Magnitude of difference \\
    \hline
    Parameters count       & 1,158,917,906   & \textbf{24,093,393}   & \textbf{48x}  \\
    \hline
    Estimated Total Size (MB) & 87,912       & \textbf{15,935}    & \textbf{5.5x} \\
    \hline
  \end{tabular}%
}
\end{table}

%\hfill

With recent advancements in complex network architectures and the use of pretrained encoders to achieve state-of-the-art performance \cite{Baumann_Dislich_etal._2024,Hörst_Rempe_etal._2024} in cell segmentation and classification tasks, model size, computational complexity, and processing times have increased. This limits the scalability and accessibility of these models. As we demonstrate, this may be mitigated using knowledge distillation. Studies in the field of natural language processing have demonstrated the efficacy of knowledge distillation in retaining the capabilities of the teacher model while achieving significant reductions in size and complexity \cite{Huangpu_Gao_2024,Sun_Yu_etal.}. 

We demonstrate the feasibility of knowledge distillation in digital pathology, specifically for cell segmentation and classification tasks. Moreover, we achieve this performance while also significantly reducing the parameter count. In addressing the challenge of knowledge transfer, we found that distillation from a transformer-based model to a smaller transformer is more straightforward than attempting to map transformer features to CNN blocks. In our experiments, using a CNN-based network as a student results in worse cell quantification performance due to the structural constraints of CNN feature space dimensions. 

Although our primary approach relies on a transformer-based student model that performs well, it can be further optimized to incorporate advantages from CNN architectures. For example, employing alternative techniques such as using ViT adapters \cite{Chen_Duan_etal._2023} or $1 \times 1$ convolutions to adjust feature map sizes may be beneficial for harnessing CNN advantages like enhanced local feature extraction. Moreover, if additional performance improvements are desired, the process can be further enhanced by applying supplementary knowledge distillation techniques, such as self-distillation \cite{Zhang_Song_etal._2019} or online distillation \cite{Houyon_Cioppa_etal._2023}.

Despite these promising results, further validation on independent datasets is necessary to fully understand the model's limitations. Underrepresented classes may pose challenges when addressing complex cases. Pathologists need to validate these models to adopt them in clinical settings. While the distilled models are smaller and more deployable, a technological gap persists because pathologists traditionally rely on established methods for inspecting WSIs and diagnosing diseases. Addressing the complexities involved in deploying models for inference and supporting pathologists in adopting new tools is essential for integrating these models into clinical workflows.

\section{Model integration with QuPath}
Digital pathology tools with graphical user interfaces are essential for visualizing and analyzing WSIs. To make our student model useful in clinical pathology workflows, it needs to be integrated into a tool that enables inspecting regions, creating annotations, and providing quantitative analyses of biomarkers. Therefore, we integrate the trained student model from the previous chapter into the QuPath open‑source platform \cite{Bankhead_Loughrey_etal._2017}. QuPath provides the required annotation, visualization, and analysis tools to interpret complex histological data, including workflows for cell segmentation, classification, and quantification (\hyperref[fig:fig7]{Figure 7}). 

\begin{figure}[h!]
    \centering
    \includegraphics[width=\textwidth]{images/Figure_7.pdf}
    \caption{Visualization of model-generated cell quantification annotations (left) and the corresponding unannotated slide (right) in QuPath}
    \label{fig:fig7}
\end{figure}

To identify the regions in a WSI critical for prognosticating tumor development, such as specific tumor areas or border regions without overlapping healthy tissue, the pathologist uses QuPath to outline these regions. Then, the pathologist initiates a cell segmentation and classification script through the QuPath interface for the selected regions. The resulting annotations and quantified cell information are then directly overlaid onto the WSI in the QuPath interface. Additional design and implementation details are in \hyperref[chap:S7]{Appendix S7}. 

Two common approaches for integrating deep learning models into QuPath are Java‑based native QuPath extensions \cite{Goldsborough_Philps_etal._2024} and the execution of RESTful API requests to a model server coupled with handling the response via an extension, as demonstrated in the application of cell segmentation models applied to immunofluorescence images \cite{Sugawara_2023}. While the community is actively working on these integration strategies, there is currently no universal solution that fully addresses all integration and performance requirements.

Extensions may offer better integration with QuPath, allowing slightly improved performance and more widespread usage of the built-in QuPath models, but they lack the flexibility to customize models and modify their behavior. For example, the newest version of QuPath includes models such as StarDist \cite{Weigert_Schmidt} and InstanSeg \cite{Goldsborough_Philps_etal._2024} that can perform cell segmentation. Both models pose limitations when applied to simultaneous cell segmentation and classification. StarDist performs well only on convex, round shapes by design, whereas some neoplastic, inflammatory, and connective cells exhibit complex and non-convex shapes. InstanSeg provides only semantic segmentation without assigning classes to the segmented cells.

%\hfill

In contrast, our approach offers an alternative integration strategy. It utilizes the paquo library to directly interact with QuPath’s internal application programming interface from within Python. This enables data exchange and processing without the need for intermediate conversion steps and provides greater control over model customization, retraining, and the incorporation of custom processing steps.

The integration of our custom model with QuPath underscores its potential to significantly enhance the diagnostic process by reducing the time burden on pathologists and enabling them to focus on more complex interpretative tasks using familiar software. Leveraging a tool that is already well-established among pathologists increases the likelihood of its adoption into daily clinical workflows. The quantitative data generated through the automated workflow is critical for both clinical decision-making and research, facilitating more accurate biomarker analysis, enabling robust statistical evaluations, and supporting hypothesis generation and testing. Additionally, by streamlining cell segmentation and classification, the tool enhances the scalability and reproducibility of pathological assessments, ultimately contributing to improved diagnostic accuracy and patient outcomes.

\section{Conclusion and future work}

In this study, we address critical challenges in digital pathology and tackle the usability and deployment issues of the developed models in standard computing environments without the need for high-performance computing systems. Our multi-faceted approach encompasses data refinement through cross-relabeling, leveraging foundation models for robust cell segmentation and classification, optimizing model performance via knowledge distillation, and integrating the optimized model into the QuPath software for practical application. This approach is used to construct a capable, versatile, and adjustable model for cell segmentation and classification, with enhanced performance and usability.

\begin{sloppypar}
While our approach shows potential in the field of computational pathology, certain limitations persist. 
For example, our implementation currently exhibits lower performance in detecting macrophages. 
This serves as an instance of the broader challenge of accurately identifying complex cell types. In order to address this issue, extending our approach to incorporate additional data sources, exploring alternative modeling approaches, and integrating other imaging modalities such as immunohistochemical staining may help improve detection accuracy. Moreover, although the distilled model reduces computational demands, integrating advanced deep learning models into clinical practice requires addressing technological gaps and potential resistance to adopting new tools within established diagnostic processes.
\end{sloppypar}

Future work could focus on several key areas to refine the proposed approach and facilitate its adoption in clinical environments. Enhancing the cell-relabeling process with additional datasets \cite{Graham_Jahanifar_etal._2021} could improve the representation of underrepresented cell types and enhance overall model performance. Also, incorporating additional data sources, such as multi-modal imaging or complementary staining methods, may address limitations related to cell type differentiation and class imbalance. Exploring other foundation models \cite{Vorontsov_Bozkurt_etal._2024,Zimmermann_Vorontsov_etal._2024} or introducing additional modalities \cite{Ding_Wagner_etal._2024,Vaidya_Zhang_etal._2025} may provide alternative architectures better suited to specific tasks or offer improved efficiency. Implementing more complex knowledge distillation techniques \cite{Houyon_Cioppa_etal._2023,Zhang_Song_etal._2019} could further optimize the model's performance and adaptability. Additionally, deeper integration with QuPath or other digital pathology software could provide pathologists more control over cell quantification analysis directly within the QuPath interface, thereby increasing accessibility and usability. Such enhancements would not only refine model performance but also ensure greater adaptability and scalability within various clinical environments. Finally, extensive validation of the model by pathologists and benchmarking against independent datasets are essential steps toward establishing the model's reliability and fostering confidence in its clinical utility.

\section*{Acknowledgments} 
This work was funded in part by the Research Council of Norway grant no. 309439 SFI Visual Intelligence, and the North Norwegian Health Authority grant no. HNF1521-20.

\bibliographystyle{IEEEtran}
\begin{sloppypar}
\begin{thebibliography}{99}

\bibitem{chaplot2020neural} Chaplot, Devendra Singh, et al. "Neural topological slam for visual navigation." Proceedings of the IEEE/CVF conference on computer vision and pattern recognition. 2020.

\bibitem{maksymets2021thda} Maksymets, Oleksandr, et al. "Thda: Treasure hunt data augmentation for semantic navigation." Proceedings of the IEEE/CVF International Conference on Computer Vision. 2021.

\bibitem{mezghan2022memory} Mezghan, Lina, et al. "Memory-augmented reinforcement learning for image-goal navigation." 2022 IEEE/RSJ International Conference on Intelligent Robots and Systems (IROS). IEEE, 2022.

\bibitem{al2022zero} Al-Halah, Ziad, Santhosh Kumar Ramakrishnan, and Kristen Grauman. "Zero experience required: Plug \& play modular transfer learning for semantic visual navigation." Proceedings of the IEEE/CVF Conference on Computer Vision and Pattern Recognition. 2022.

\bibitem{ye2021auxiliary} Ye, Joel, et al. "Auxiliary tasks and exploration enable objectgoal navigation." Proceedings of the IEEE/CVF international conference on computer vision. 2021.

\bibitem{chaplot2020object} Chaplot, Devendra Singh, et al. "Object goal navigation using goal-oriented semantic exploration." Advances in Neural Information Processing Systems 33 (2020)

\bibitem{ramakrishnan2022poni} Ramakrishnan, Santhosh Kumar, et al. "Poni: Potential functions for objectgoal navigation with interaction-free learning." Proceedings of the IEEE/CVF Conference on Computer Vision and Pattern Recognition. 2022.

\bibitem{ramrakhya2022habitat} Ramrakhya, Ram, et al. "Habitat-web: Learning embodied object-search strategies from human demonstrations at scale." Proceedings of the IEEE/CVF Conference on Computer Vision and Pattern Recognition. 2022.

\bibitem{mousavian2019visual} Mousavian, Arsalan, et al. "Visual representations for semantic target driven navigation." 2019 International Conference on Robotics and Automation (ICRA). IEEE, 2019.

\bibitem{dhariwal2021diffusion} Dhariwal, Prafulla, and Alexander Nichol. "Diffusion models beat gans on image synthesis." Advances in neural information processing systems 34 (2021)

\bibitem{ho2022classifier} Ho, Jonathan, and Tim Salimans. "Classifier-free diffusion guidance." arXiv preprint arXiv:2207.12598 (2022).

\bibitem{nichol2021glide} Nichol, Alex, et al. "Glide: Towards photorealistic image generation and editing with text-guided diffusion models." arXiv preprint arXiv:2112.10741 (2021)

\bibitem{brooks2023instructpix2pix} Brooks, Tim, Aleksander Holynski, and Alexei A. Efros. "Instructpix2pix: Learning to follow image editing instructions." Proceedings of the IEEE/CVF Conference on Computer Vision and Pattern Recognition. 2023.

\bibitem{fu2023guiding} Fu, Tsu-Jui, et al. "Guiding instruction-based image editing via multimodal large language models." arXiv preprint arXiv:2309.17102 (2023).

\bibitem{geng2024instructdiffusion} Geng, Zigang, et al. "Instructdiffusion: A generalist modeling interface for vision tasks." Proceedings of the IEEE/CVF Conference on Computer Vision and Pattern Recognition. 2024.

\bibitem{zhou2024minedreamer} Zhou, Enshen, et al. "Minedreamer: Learning to follow instructions via chain-of-imagination for simulated-world control." arXiv preprint arXiv:2403.12037 (2024).

\bibitem{zhou2023esc} Zhou, Kaiwen, et al. "Esc: Exploration with soft commonsense constraints for zero-shot object navigation." International Conference on Machine Learning. PMLR, 2023.

\bibitem{yu2023l3mvn} Yu, Bangguo, Hamidreza Kasaei, and Ming Cao. "L3mvn: Leveraging large language models for visual target navigation." 2023 IEEE/RSJ International Conference on Intelligent Robots and Systems (IROS). IEEE, 2023.

\bibitem{gadre2023cows} Gadre, Samir Yitzhak, et al. "Cows on pasture: Baselines and benchmarks for language-driven zero-shot object navigation." Proceedings of the IEEE/CVF Conference on Computer Vision and Pattern Recognition. 2023.

\bibitem{shah2023navigation} Shah, Dhruv, et al. "Navigation with large language models: Semantic guesswork as a heuristic for planning." Conference on Robot Learning. PMLR, 2023.

\bibitem{cai2024bridging} Cai, Wenzhe, et al. "Bridging zero-shot object navigation and foundation models through pixel-guided navigation skill." 2024 IEEE International Conference on Robotics and Automation (ICRA). IEEE, 2024.

\bibitem{yu2023co} Yu, Bangguo, Hamidreza Kasaei, and Ming Cao. "Co-NavGPT: Multi-robot cooperative visual semantic navigation using large language models." arXiv preprint arXiv:2310.07937 (2023).

\bibitem{wu2024voronav} Wu, Pengying, et al. "Voronav: Voronoi-based zero-shot object navigation with large language model." arXiv preprint arXiv:2401.02695 (2024).

\bibitem{qin2023mp5} Qin, Yiran, et al. "Mp5: A multi-modal open-ended embodied system in minecraft via active perception." arXiv preprint arXiv:2312.07472 (2023).

\bibitem{du2024learning} Du, Yilun, et al. "Learning universal policies via text-guided video generation." Advances in Neural Information Processing Systems 36 (2024).

\bibitem{ajay2024compositional} Ajay, Anurag, et al. "Compositional foundation models for hierarchical planning." Advances in Neural Information Processing Systems 36 (2024).

\bibitem{liang2024skilldiffuser} Liang, Zhixuan, et al. "Skilldiffuser: Interpretable hierarchical planning via skill abstractions in diffusion-based task execution." Proceedings of the IEEE/CVF Conference on Computer Vision and Pattern Recognition. 2024.

\bibitem{heusel2017gans} Heusel, Martin, et al. "Gans trained by a two time-scale update rule converge to a local nash equilibrium." Advances in neural information processing systems 30 (2017).

\bibitem{zhang2018unreasonable} Zhang, Richard, et al. "The unreasonable effectiveness of deep features as a perceptual metric." Proceedings of the IEEE conference on computer vision and pattern recognition. 2018.

\bibitem{brown2020language} Brown, Tom B. "Language models are few-shot learners." arXiv preprint arXiv:2005.14165 (2020).

\bibitem{podell2023sdxl} Podell, Dustin, et al. "Sdxl: Improving latent diffusion models for high-resolution image synthesis." arXiv preprint arXiv:2307.01952 (2023).

\bibitem{brohan2022rt} Brohan, Anthony, et al. "Rt-1: Robotics transformer for real-world control at scale." arXiv preprint arXiv:2212.06817 (2022).

\bibitem{brohan2023rt} Brohan, Anthony, et al. "Rt-2: Vision-language-action models transfer web knowledge to robotic control." arXiv preprint arXiv:2307.15818 (2023).

\bibitem{li2024manipllm} Li, Xiaoqi, et al. "Manipllm: Embodied multimodal large language model for object-centric robotic manipulation." Proceedings of the IEEE/CVF Conference on Computer Vision and Pattern Recognition. 2024.

\bibitem{shah2023vint} Shah, Dhruv, et al. "ViNT: A foundation model for visual navigation." arXiv preprint arXiv:2306.14846 (2023).

\bibitem{liu2024visual} Liu, Haotian, et al. "Visual instruction tuning." Advances in neural information processing systems 36 (2024).

\bibitem{hu2021lora} Hu, Edward J., et al. "Lora: Low-rank adaptation of large language models." arXiv preprint arXiv:2106.09685 (2021).

\bibitem{qin2023supfusion} Qin, Yiran, et al. "SupFusion: Supervised LiDAR-camera fusion for 3D object detection." Proceedings of the IEEE/CVF International Conference on Computer Vision. 2023.

\bibitem{qin2024worldsimbench} Qin, Yiran, et al. "Worldsimbench: Towards video generation models as world simulators." arXiv preprint arXiv:2410.18072 (2024).

\bibitem{yu2025gamefactory} Yu, Jiwen, et al. "GameFactory: Creating New Games with Generative Interactive Videos." arXiv preprint arXiv:2501.08325 (2025).

\bibitem{zhou2024code} Zhou, Enshen, et al. "Code-as-Monitor: Constraint-aware Visual Programming for Reactive and Proactive Robotic Failure Detection." arXiv preprint arXiv:2412.04455 (2024).

\bibitem{zhang2024ad} Zhang, Zaibin, et al. "AD-H: Autonomous Driving with Hierarchical Agents." arXiv preprint arXiv:2406.03474 (2024).

\bibitem{wang2024toward} Wang, Chaoqun, et al. "Toward Accurate Camera-based 3D Object Detection via Cascade Depth Estimation and Calibration." arXiv preprint arXiv:2402.04883 (2024).

\bibitem{huang2024story3d} Huang, Yuzhou, et al. "Story3d-agent: Exploring 3d storytelling visualization with large language models." arXiv preprint arXiv:2408.11801 (2024).

\bibitem{savinov2018semi} Savinov, Nikolay, Alexey Dosovitskiy, and Vladlen Koltun. "Semi-parametric topological memory for navigation." arXiv preprint arXiv:1803.00653 (2018).

\bibitem{majumdar2022zson} Majumdar, Arjun, et al. "Zson: Zero-shot object-goal navigation using multimodal goal embeddings." Advances in Neural Information Processing Systems 35 (2022): 32340-32352.

\bibitem{yadav2023offline} Yadav, Karmesh, et al. "Offline visual representation learning for embodied navigation." Workshop on Reincarnating Reinforcement Learning at ICLR 2023. 2023.

\bibitem{yadav2023ovrl} Yadav, Karmesh, et al. "Ovrl-v2: A simple state-of-art baseline for imagenav and objectnav." arXiv preprint arXiv:2303.07798 (2023).

\bibitem{sun2024fgprompt} Sun, Xinyu, et al. "FGPrompt: fine-grained goal prompting for image-goal navigation." Advances in Neural Information Processing Systems 36 (2024).

\bibitem{zhu2017target} Zhu, Yuke, et al. "Target-driven visual navigation in indoor scenes using deep reinforcement learning." 2017 IEEE international conference on robotics and automation (ICRA). IEEE, 2017.

\bibitem{koh2024generating} Koh, Jing Yu, Daniel Fried, and Russ R. Salakhutdinov. "Generating images with multimodal language models." Advances in Neural Information Processing Systems 36 (2024).

\bibitem{krantz2022instance} Krantz, Jacob, et al. "Instance-specific image goal navigation: Training embodied agents to find object instances." arXiv preprint arXiv:2211.15876 (2022).

\bibitem{schulman2017proximal} Schulman, John, et al. "Proximal policy optimization algorithms." arXiv preprint arXiv:1707.06347 (2017).

\bibitem{anderson2018evaluation} Anderson, Peter, et al. "On evaluation of embodied navigation agents." arXiv preprint arXiv:1807.06757 (2018).

\bibitem{lin2024navcot} Lin, Bingqian, et al. "NavCoT: Boosting LLM-Based Vision-and-Language Navigation via Learning Disentangled Reasoning." arXiv preprint arXiv:2403.07376 (2024).

\bibitem{NavGPT} Zhou, Gengze, Yicong Hong, and Qi Wu. "Navgpt: Explicit reasoning in vision-and-language navigation with large language models." Proceedings of the AAAI Conference on Artificial Intelligence.

\bibitem{hahn2021no} Hahn, Meera, et al. "No rl, no simulation: Learning to navigate without navigating." Advances in Neural Information Processing Systems 34 (2021): 26661-26673.

\bibitem{li2025t2isafety} Li, Lijun, et al. "T2ISafety: Benchmark for Assessing Fairness, Toxicity, and Privacy in Image Generation." arXiv preprint arXiv:2501.12612 (2025).

\bibitem{an2024agfsync} An, Jingkun, et al. "AGFSync: Leveraging AI-Generated Feedback for Preference Optimization in Text-to-Image Generation." arXiv preprint arXiv:2403.13352 (2024).


\end{thebibliography}
\end{sloppypar}

\clearpage
\beginsupplement
\section*{Appendix}
\renewcommand{\thesubsection}{S\arabic{subsection}}

\subsection{\label{chap:S1}PanNuke and MoNuSAC preprocessing}
The PanNuke dataset comprises a set of 7,901 RGB patches, each with dimensions of $256 \times 256$ pixels, which we set as the standard patch size for our analysis. In contrast, the MoNuSAC dataset encompasses 294 images of heterogeneous dimensions. To standardize the MoNuSAC images with our experiments, we implement a standardization protocol. Specifically, for images exceeding the dimensions of $256 \times 256$ pixels, we segment them into equal-sized patches and apply mirror padding to the remaining portions to avoid information loss at the peripherals. Patches with dimensions less than $128 \times 128$ pixels are excluded from the dataset due to the insufficient resolution to capture relevant cellular details. For patches where either dimension falls between 128 and 256 pixels, we employ upsampling to achieve the standard patch size. As a result, we obtain a total of 2,823 RGB patches derived from the MoNuSAC dataset for subsequent analysis. For additional details on the MoNuSAC data preparation process, refer to the source code \cite{Shvetsov_2025a}.
\clearpage

\subsection{\label{chap:S2}Data usage for the methodology}

\counterwithin{figure}{subsection}
\renewcommand{\thefigure}{S\arabic{subsection}}

\begin{figure}[h!]
    \centering
    \includegraphics[width=\textwidth, height=0.85\textheight, keepaspectratio]{images/A2.pdf}
    \caption{Overview of the methodology for cross-labeling, dataset refinement, and model comparison. (1) Cross-relabeling - training and testing cell classification models, (2) Cross-relabeling - using cell classification models to create refined dataset, (3) Fine-tuning and training models for comparison, (4) Student knowledge distillation with refined dataset}
    \label{fig:S2}
\end{figure}
\clearpage

\subsection{\label{chap:S3}Confusion matrices for classification models}
\counterwithin{figure}{subsection}
\renewcommand{\thefigure}{S\arabic{subsection}.\arabic{figure}}

\begin{figure}[h!]
    \centering
    \includegraphics[width=\textwidth, height=0.4\textheight, keepaspectratio]{images/A3_1.pdf}
    \caption{Confusion matrix for PanNuke trained model}
    \label{fig:S3.1}
\end{figure}

\begin{figure}[h!]
    \centering
    \includegraphics[width=\textwidth, height=0.4\textheight, keepaspectratio]{images/A3_2.pdf}
    \caption{Confusion matrix for MoNuSAC trained model}
    \label{fig:S3.2}
\end{figure}

\clearpage

\subsection{\label{chap:S4}Datasets cell counts}

\counterwithin{table}{subsection}
\renewcommand{\thetable}{S\arabic{subsection}}

\begin{table}[h!]
\renewcommand{\arraystretch}{2.0}
\centering
\caption{\label{tab:S4}Cell counts for PanNuke, MoNuSAC and refined datasets. Numbers in parentheses indicate preprocessed cell counts for cell classifier models training and testing.}
%\adjustbox{max width=\textwidth}{%
\begin{tabular}{|l|c|c|c|}
\hline
%\rowcolor{gray!30}
Cell type & PanNuke & MoNuSAC & Refined \\
\hline
Neoplastic & 77,403 (68,031) & - & 105,451 \\
\hline
Epithelial & 26,572 (23,207) & - & 29,926 \\
\hline
Epithelial (benign and malignant) & - & 31,402 & - \\
\hline
Inflammatory & 32,276 & - & - \\
\hline
Lymphocytes & - & 37,045 (33,104) & 65,275 \\
\hline
Neutrophils & - & 1,355 (1,252) & 3,833 \\
\hline
Macrophage & - & 1,842 (1,695) & 3,410 \\
\hline
Dead & 2,908 & - & 2,908 \\
\hline
Connective & 50,585 & - & 50,585 \\
\hline
\end{tabular}
%
%}
\end{table}



\clearpage

\subsection{\label{chap:S5}Definition of validation metrics}
\counterwithin{equation}{subsection}
\renewcommand{\theequation}{\arabic{equation}}

\subsubsection{\label{chap:S5.1}R\textsuperscript{2}}
The coefficient of determination, denoted as $R^2$, is a statistical measure that represents the proportion of variance in the dependent variable that is predictable from the independent variables. In the context of cell quantification in pathology, $R^2$ is used to assess how well the predicted quantities of different cell types in a patch align with the actual quantities observed in the ground truth data, with higher values representing more accurate quantification. $R^2$ is defined as
\begin{equation*}
R^2 = 1 - \frac{\sum_{i=1}^n (y_i - \hat{y}_i)^2}{\sum_{i=1}^n (y_i - \bar{y})^2},
\end{equation*}
where $y_i$ represents the actual number of cells of a specific type in the $i$-th image, $\hat{y}_i$ represents the predicted number of cells of that type in the $i$-th image, $\bar{y}$ is the mean of the actual numbers across all images, and $n$ is the total number of images in the dataset.

The $R^2$ metric has a range of $(-\infty, 1]$. An $R^2$ of 1 indicates perfect prediction, where all predicted values exactly match the actual values. An $R^2$ of 0 suggests that the model explains none of the variability of the response data around its mean. If $R^2$ is negative, it indicates that the model performs worse than a model that simply predicts the mean of the actual values for all observations.

\subsubsection{\label{chap:S5.2}PQ}
Panoptic Quality ($PQ$) is a comprehensive metric used to evaluate the performance of segmentation models in tasks that require both instance segmentation and classification. $PQ$ provides a single score that encapsulates both the detection accuracy (i.e., how many objects were correctly identified) and the segmentation quality (i.e., how accurately the objects' boundaries were delineated). This metric is particularly useful in multiclass scenarios where each pixel is classified into distinct categories, such as different cell types in pathology images.

$PQ$ is calculated as the product of two terms: Detection Quality ($DQ$) and Segmentation Quality ($SQ$). It can be expressed as
\begin{equation*}
PQ = DQ \cdot SQ,
\end{equation*}
where
\begin{equation*}
DQ = \frac{TP}{TP + 0.5\, FP + 0.5\, FN},
\end{equation*}
\begin{equation*}
SQ = \frac{\sum_{(p, g) \in \mathcal{M}} IoU(p, g)}{TP}.
\end{equation*}
In these formulas, $TP$ denotes the number of correctly matched instances between ground truth and prediction, $FP$ denotes the predicted instances that have no corresponding ground truth, $FN$ denotes the ground truth instances that were not detected, $IoU(p, g)$ is the Intersection over Union for a pair of matched instances $p$ (prediction) and $g$ (ground truth), and $\mathcal{M}$ is the set of matched pairs.

The $PQ$ metric is calculated for each class and is averaged across classes to provide a global performance measure.

The $PQ$ score has a range of $[0, 1.0]$, where a higher score indicates better performance in both detecting and segmenting the instances correctly. A $PQ$ of 1 signifies perfect identification and segmentation of all instances, whereas a $PQ$ of 0 indicates that no instances were correctly identified and segmented.

\clearpage

\subsection{\label{chap:S6}Segmentation and Detection quality metrics for teacher and student models}

\begin{table}[h!]
\renewcommand{\arraystretch}{2.0}
\centering
\caption{Segmentation and detection quality for student and teacher models (CI 95\%)}
\label{tab:S6}
%\adjustbox{max width=\textwidth}{%
\begin{tabular}{|l|c|c|}
\hline
%\rowcolor{gray!30}
Metric & Teacher & Student \\
\hline
$SQ_{neoplastic}$ & 0.819 (0.815--0.823) & 0.824 (0.819--0.828) \\
\hline
$SQ_{lymphocyte}$ & 0.795 (0.788--0.802) & 0.790 (0.783--0.796) \\
\hline
$SQ_{connective}$ & 0.770 (0.762--0.776) & 0.780 (0.772--0.786) \\
\hline
$SQ_{dead}$ & 0.659 (0.623--0.688) & 0.657 (0.624--0.695) \\
\hline
$SQ_{epithelial}$ & 0.780 (0.770--0.790) & 0.788 (0.779--0.797) \\
\hline
$SQ_{macrophage}$ & 0.788 (0.760--0.810) & 0.757 (0.730--0.783) \\
\hline
$SQ_{neutrofil}$ & 0.782 (0.761--0.801) & 0.775 (0.759--0.792) \\
\hline
$DQ_{neoplastic}$ & 0.706 (0.692--0.719) & 0.727 (0.712--0.741) \\
\hline
$DQ_{lymphocyte}$ & 0.675 (0.656--0.698) & 0.713 (0.691--0.734) \\
\hline
$DQ_{connective}$ & 0.566 (0.546--0.584) & 0.583 (0.565--0.602) \\
\hline
$DQ_{dead}$ & 0.410 (0.361--0.465) & 0.435 (0.306--0.561) \\
\hline
$DQ_{epithelial}$ & 0.668 (0.639--0.694) & 0.673 (0.644--0.702) \\
\hline
$DQ_{macrophage}$ & 0.657 (0.583--0.727) & 0.615 (0.531--0.703) \\
\hline
$DQ_{neutrofil}$ & 0.691 (0.625--0.753) & 0.729 (0.679--0.778) \\
\hline
\end{tabular}
%
%}
\end{table}

\clearpage

\subsection{\label{chap:S7}QuPath integration method}
We adopt an integration strategy leveraging the paquo \cite{Bayer_AG} library, a Python package that enables direct interaction with QuPath’s internal API, thereby facilitating seamless data exchange without intermediate conversion steps. The data processing pipeline (\hyperref[fig:S7]{Appendix Figure S7}) begins with the acquisition of WSIs and their associated annotations from QuPath, which are represented as Shapely \cite{Gillies_Wel_etal._2024} polygons. Utilizing paquo, we directly read, create, and modify these annotations and detections within a QuPath project in the Python environment. Images are then cropped using these polygons and processed by cell segmentation and classification models employing standard vision processing toolkits such as OpenCV, pyvips, and PyTorch. Additionally, QuPath employs Groovy scripts to initiate a Python process that starts the entire pipeline from QuPath graphical interface: fetching polygons, extracting images from them, and running deep learning model inference on the cropped images. 
The results are returned to QuPath, leveraging paquo's Python bindings to manipulate QuPath data while minimizing the computational overhead typically associated with cross-environment communication.

\counterwithin{figure}{subsection}
\renewcommand{\thefigure}{S\arabic{subsection}}

\begin{figure}[h!]
    \centering
    \includegraphics[width=\textwidth]{images/A7.pdf}
    \caption{QuPath integration workflow using Python environment}
    \label{fig:S7}
\end{figure}

Compared to traditional workflows that involve exporting annotations as GeoJSON, classifying them in Python, and reimporting them into QuPath, our approach offers several advantages. We eliminate the need to switch between programming languages, providing a cohesive and streamlined development process entirely within QuPath software and removing the necessity to use other tools. Meanwhile, we avoid storing annotations as intermediate JSON files unless required for external use or archiving. By conducting the entire inference and post-processing workflow within the Python environment, we leverage the power and flexibility of Python libraries for image processing and machine learning. This approach also enables adjustments to any set of labels and models, thereby improving its applicability.

%\hfill

The distilled model and QuPath integration code are packaged into a Docker container, enabling streamlined execution with the Docker engine. Detailed integration code and deployment instructions can be found in the GitHub repository \cite{Shvetsov_2025b}.

Despite these benefits, we acknowledge that the paquo library is a proof‑of‑concept project in its early development stage and has not been tested across all versions of QuPath.

\clearpage

\subsection{\label{chap:S8}Data and code availability statement}
All datasets, models, and code used in this study are publicly available and can be obtained from the repositories listed below. 
The PanNuke \cite{Gamper_Koohbanani_etal._2019} and MoNuSAC \cite{Verma_Kumar_etal._2021} datasets are publicly accessible, and download information along with detailed descriptions can be found in their respective articles. Preprocessing scripts for PanNuke and MoNuSAC data, as well as individual cell extraction scripts, are available on GitHub \cite{Shvetsov_2025a}. The H-Optimus foundation model used in our experiments can be downloaded from the HuggingFace repository \cite{hoptimus2024}, and model information is available on GitHub \cite{Saillard_Jenatton_etal._2024}. In addition, the integration code for QuPath and the distilled model packaged in a Docker container are provided in the repository \cite{Shvetsov_2025b}, and paquo Python library is available from the authors GitHub repository \cite{Bayer_AG}.
\clearpage

\end{document}


\end{document}

