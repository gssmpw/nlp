
\documentclass[10pt,twocolumn,letterpaper]{article}
\usepackage[pagenumbers]{cvpr}
%
% --- inline annotations
%
\newcommand{\red}[1]{{\color{red}#1}}
\newcommand{\todo}[1]{{\color{red}#1}}
\newcommand{\TODO}[1]{\textbf{\color{red}[TODO: #1]}}
% --- disable by uncommenting  
% \renewcommand{\TODO}[1]{}
% \renewcommand{\todo}[1]{#1}



\newcommand{\VLM}{LVLM\xspace} 
\newcommand{\ours}{PeKit\xspace}
\newcommand{\yollava}{Yo’LLaVA\xspace}

\newcommand{\thisismy}{This-Is-My-Img\xspace}
\newcommand{\myparagraph}[1]{\noindent\textbf{#1}}
\newcommand{\vdoro}[1]{{\color[rgb]{0.4, 0.18, 0.78} {[V] #1}}}
% --- disable by uncommenting  
% \renewcommand{\TODO}[1]{}
% \renewcommand{\todo}[1]{#1}
\usepackage{slashbox}
% Vectors
\newcommand{\bB}{\mathcal{B}}
\newcommand{\bw}{\mathbf{w}}
\newcommand{\bs}{\mathbf{s}}
\newcommand{\bo}{\mathbf{o}}
\newcommand{\bn}{\mathbf{n}}
\newcommand{\bc}{\mathbf{c}}
\newcommand{\bp}{\mathbf{p}}
\newcommand{\bS}{\mathbf{S}}
\newcommand{\bk}{\mathbf{k}}
\newcommand{\bmu}{\boldsymbol{\mu}}
\newcommand{\bx}{\mathbf{x}}
\newcommand{\bg}{\mathbf{g}}
\newcommand{\be}{\mathbf{e}}
\newcommand{\bX}{\mathbf{X}}
\newcommand{\by}{\mathbf{y}}
\newcommand{\bv}{\mathbf{v}}
\newcommand{\bz}{\mathbf{z}}
\newcommand{\bq}{\mathbf{q}}
\newcommand{\bff}{\mathbf{f}}
\newcommand{\bu}{\mathbf{u}}
\newcommand{\bh}{\mathbf{h}}
\newcommand{\bb}{\mathbf{b}}

\newcommand{\rone}{\textcolor{green}{R1}}
\newcommand{\rtwo}{\textcolor{orange}{R2}}
\newcommand{\rthree}{\textcolor{red}{R3}}
\usepackage{amsmath}
%\usepackage{arydshln}
\DeclareMathOperator{\similarity}{sim}
\DeclareMathOperator{\AvgPool}{AvgPool}

\newcommand{\argmax}{\mathop{\mathrm{argmax}}}     



%%%%% NEW MATH DEFINITIONS %%%%%

\usepackage{amsmath,amsfonts,bm}
\usepackage{derivative}
% Mark sections of captions for referring to divisions of figures
\newcommand{\figleft}{{\em (Left)}}
\newcommand{\figcenter}{{\em (Center)}}
\newcommand{\figright}{{\em (Right)}}
\newcommand{\figtop}{{\em (Top)}}
\newcommand{\figbottom}{{\em (Bottom)}}
\newcommand{\captiona}{{\em (a)}}
\newcommand{\captionb}{{\em (b)}}
\newcommand{\captionc}{{\em (c)}}
\newcommand{\captiond}{{\em (d)}}

% Highlight a newly defined term
\newcommand{\newterm}[1]{{\bf #1}}

% Derivative d 
\newcommand{\deriv}{{\mathrm{d}}}

% Figure reference, lower-case.
\def\figref#1{figure~\ref{#1}}
% Figure reference, capital. For start of sentence
\def\Figref#1{Figure~\ref{#1}}
\def\twofigref#1#2{figures \ref{#1} and \ref{#2}}
\def\quadfigref#1#2#3#4{figures \ref{#1}, \ref{#2}, \ref{#3} and \ref{#4}}
% Section reference, lower-case.
\def\secref#1{section~\ref{#1}}
% Section reference, capital.
\def\Secref#1{Section~\ref{#1}}
% Reference to two sections.
\def\twosecrefs#1#2{sections \ref{#1} and \ref{#2}}
% Reference to three sections.
\def\secrefs#1#2#3{sections \ref{#1}, \ref{#2} and \ref{#3}}
% Reference to an equation, lower-case.
\def\eqref#1{equation~\ref{#1}}
% Reference to an equation, upper case
\def\Eqref#1{Equation~\ref{#1}}
% A raw reference to an equation---avoid using if possible
\def\plaineqref#1{\ref{#1}}
% Reference to a chapter, lower-case.
\def\chapref#1{chapter~\ref{#1}}
% Reference to an equation, upper case.
\def\Chapref#1{Chapter~\ref{#1}}
% Reference to a range of chapters
\def\rangechapref#1#2{chapters\ref{#1}--\ref{#2}}
% Reference to an algorithm, lower-case.
\def\algref#1{algorithm~\ref{#1}}
% Reference to an algorithm, upper case.
\def\Algref#1{Algorithm~\ref{#1}}
\def\twoalgref#1#2{algorithms \ref{#1} and \ref{#2}}
\def\Twoalgref#1#2{Algorithms \ref{#1} and \ref{#2}}
% Reference to a part, lower case
\def\partref#1{part~\ref{#1}}
% Reference to a part, upper case
\def\Partref#1{Part~\ref{#1}}
\def\twopartref#1#2{parts \ref{#1} and \ref{#2}}

\def\ceil#1{\lceil #1 \rceil}
\def\floor#1{\lfloor #1 \rfloor}
\def\1{\bm{1}}
\newcommand{\train}{\mathcal{D}}
\newcommand{\valid}{\mathcal{D_{\mathrm{valid}}}}
\newcommand{\test}{\mathcal{D_{\mathrm{test}}}}

\def\eps{{\epsilon}}


% Random variables
\def\reta{{\textnormal{$\eta$}}}
\def\ra{{\textnormal{a}}}
\def\rb{{\textnormal{b}}}
\def\rc{{\textnormal{c}}}
\def\rd{{\textnormal{d}}}
\def\re{{\textnormal{e}}}
\def\rf{{\textnormal{f}}}
\def\rg{{\textnormal{g}}}
\def\rh{{\textnormal{h}}}
\def\ri{{\textnormal{i}}}
\def\rj{{\textnormal{j}}}
\def\rk{{\textnormal{k}}}
\def\rl{{\textnormal{l}}}
% rm is already a command, just don't name any random variables m
\def\rn{{\textnormal{n}}}
\def\ro{{\textnormal{o}}}
\def\rp{{\textnormal{p}}}
\def\rq{{\textnormal{q}}}
\def\rr{{\textnormal{r}}}
\def\rs{{\textnormal{s}}}
\def\rt{{\textnormal{t}}}
\def\ru{{\textnormal{u}}}
\def\rv{{\textnormal{v}}}
\def\rw{{\textnormal{w}}}
\def\rx{{\textnormal{x}}}
\def\ry{{\textnormal{y}}}
\def\rz{{\textnormal{z}}}

% Random vectors
\def\rvepsilon{{\mathbf{\epsilon}}}
\def\rvphi{{\mathbf{\phi}}}
\def\rvtheta{{\mathbf{\theta}}}
\def\rva{{\mathbf{a}}}
\def\rvb{{\mathbf{b}}}
\def\rvc{{\mathbf{c}}}
\def\rvd{{\mathbf{d}}}
\def\rve{{\mathbf{e}}}
\def\rvf{{\mathbf{f}}}
\def\rvg{{\mathbf{g}}}
\def\rvh{{\mathbf{h}}}
\def\rvu{{\mathbf{i}}}
\def\rvj{{\mathbf{j}}}
\def\rvk{{\mathbf{k}}}
\def\rvl{{\mathbf{l}}}
\def\rvm{{\mathbf{m}}}
\def\rvn{{\mathbf{n}}}
\def\rvo{{\mathbf{o}}}
\def\rvp{{\mathbf{p}}}
\def\rvq{{\mathbf{q}}}
\def\rvr{{\mathbf{r}}}
\def\rvs{{\mathbf{s}}}
\def\rvt{{\mathbf{t}}}
\def\rvu{{\mathbf{u}}}
\def\rvv{{\mathbf{v}}}
\def\rvw{{\mathbf{w}}}
\def\rvx{{\mathbf{x}}}
\def\rvy{{\mathbf{y}}}
\def\rvz{{\mathbf{z}}}

% Elements of random vectors
\def\erva{{\textnormal{a}}}
\def\ervb{{\textnormal{b}}}
\def\ervc{{\textnormal{c}}}
\def\ervd{{\textnormal{d}}}
\def\erve{{\textnormal{e}}}
\def\ervf{{\textnormal{f}}}
\def\ervg{{\textnormal{g}}}
\def\ervh{{\textnormal{h}}}
\def\ervi{{\textnormal{i}}}
\def\ervj{{\textnormal{j}}}
\def\ervk{{\textnormal{k}}}
\def\ervl{{\textnormal{l}}}
\def\ervm{{\textnormal{m}}}
\def\ervn{{\textnormal{n}}}
\def\ervo{{\textnormal{o}}}
\def\ervp{{\textnormal{p}}}
\def\ervq{{\textnormal{q}}}
\def\ervr{{\textnormal{r}}}
\def\ervs{{\textnormal{s}}}
\def\ervt{{\textnormal{t}}}
\def\ervu{{\textnormal{u}}}
\def\ervv{{\textnormal{v}}}
\def\ervw{{\textnormal{w}}}
\def\ervx{{\textnormal{x}}}
\def\ervy{{\textnormal{y}}}
\def\ervz{{\textnormal{z}}}

% Random matrices
\def\rmA{{\mathbf{A}}}
\def\rmB{{\mathbf{B}}}
\def\rmC{{\mathbf{C}}}
\def\rmD{{\mathbf{D}}}
\def\rmE{{\mathbf{E}}}
\def\rmF{{\mathbf{F}}}
\def\rmG{{\mathbf{G}}}
\def\rmH{{\mathbf{H}}}
\def\rmI{{\mathbf{I}}}
\def\rmJ{{\mathbf{J}}}
\def\rmK{{\mathbf{K}}}
\def\rmL{{\mathbf{L}}}
\def\rmM{{\mathbf{M}}}
\def\rmN{{\mathbf{N}}}
\def\rmO{{\mathbf{O}}}
\def\rmP{{\mathbf{P}}}
\def\rmQ{{\mathbf{Q}}}
\def\rmR{{\mathbf{R}}}
\def\rmS{{\mathbf{S}}}
\def\rmT{{\mathbf{T}}}
\def\rmU{{\mathbf{U}}}
\def\rmV{{\mathbf{V}}}
\def\rmW{{\mathbf{W}}}
\def\rmX{{\mathbf{X}}}
\def\rmY{{\mathbf{Y}}}
\def\rmZ{{\mathbf{Z}}}

% Elements of random matrices
\def\ermA{{\textnormal{A}}}
\def\ermB{{\textnormal{B}}}
\def\ermC{{\textnormal{C}}}
\def\ermD{{\textnormal{D}}}
\def\ermE{{\textnormal{E}}}
\def\ermF{{\textnormal{F}}}
\def\ermG{{\textnormal{G}}}
\def\ermH{{\textnormal{H}}}
\def\ermI{{\textnormal{I}}}
\def\ermJ{{\textnormal{J}}}
\def\ermK{{\textnormal{K}}}
\def\ermL{{\textnormal{L}}}
\def\ermM{{\textnormal{M}}}
\def\ermN{{\textnormal{N}}}
\def\ermO{{\textnormal{O}}}
\def\ermP{{\textnormal{P}}}
\def\ermQ{{\textnormal{Q}}}
\def\ermR{{\textnormal{R}}}
\def\ermS{{\textnormal{S}}}
\def\ermT{{\textnormal{T}}}
\def\ermU{{\textnormal{U}}}
\def\ermV{{\textnormal{V}}}
\def\ermW{{\textnormal{W}}}
\def\ermX{{\textnormal{X}}}
\def\ermY{{\textnormal{Y}}}
\def\ermZ{{\textnormal{Z}}}

% Vectors
\def\vzero{{\bm{0}}}
\def\vone{{\bm{1}}}
\def\vmu{{\bm{\mu}}}
\def\vtheta{{\bm{\theta}}}
\def\vphi{{\bm{\phi}}}
\def\va{{\bm{a}}}
\def\vb{{\bm{b}}}
\def\vc{{\bm{c}}}
\def\vd{{\bm{d}}}
\def\ve{{\bm{e}}}
\def\vf{{\bm{f}}}
\def\vg{{\bm{g}}}
\def\vh{{\bm{h}}}
\def\vi{{\bm{i}}}
\def\vj{{\bm{j}}}
\def\vk{{\bm{k}}}
\def\vl{{\bm{l}}}
\def\vm{{\bm{m}}}
\def\vn{{\bm{n}}}
\def\vo{{\bm{o}}}
\def\vp{{\bm{p}}}
\def\vq{{\bm{q}}}
\def\vr{{\bm{r}}}
\def\vs{{\bm{s}}}
\def\vt{{\bm{t}}}
\def\vu{{\bm{u}}}
\def\vv{{\bm{v}}}
\def\vw{{\bm{w}}}
\def\vx{{\bm{x}}}
\def\vy{{\bm{y}}}
\def\vz{{\bm{z}}}

% Elements of vectors
\def\evalpha{{\alpha}}
\def\evbeta{{\beta}}
\def\evepsilon{{\epsilon}}
\def\evlambda{{\lambda}}
\def\evomega{{\omega}}
\def\evmu{{\mu}}
\def\evpsi{{\psi}}
\def\evsigma{{\sigma}}
\def\evtheta{{\theta}}
\def\eva{{a}}
\def\evb{{b}}
\def\evc{{c}}
\def\evd{{d}}
\def\eve{{e}}
\def\evf{{f}}
\def\evg{{g}}
\def\evh{{h}}
\def\evi{{i}}
\def\evj{{j}}
\def\evk{{k}}
\def\evl{{l}}
\def\evm{{m}}
\def\evn{{n}}
\def\evo{{o}}
\def\evp{{p}}
\def\evq{{q}}
\def\evr{{r}}
\def\evs{{s}}
\def\evt{{t}}
\def\evu{{u}}
\def\evv{{v}}
\def\evw{{w}}
\def\evx{{x}}
\def\evy{{y}}
\def\evz{{z}}

% Matrix
\def\mA{{\bm{A}}}
\def\mB{{\bm{B}}}
\def\mC{{\bm{C}}}
\def\mD{{\bm{D}}}
\def\mE{{\bm{E}}}
\def\mF{{\bm{F}}}
\def\mG{{\bm{G}}}
\def\mH{{\bm{H}}}
\def\mI{{\bm{I}}}
\def\mJ{{\bm{J}}}
\def\mK{{\bm{K}}}
\def\mL{{\bm{L}}}
\def\mM{{\bm{M}}}
\def\mN{{\bm{N}}}
\def\mO{{\bm{O}}}
\def\mP{{\bm{P}}}
\def\mQ{{\bm{Q}}}
\def\mR{{\bm{R}}}
\def\mS{{\bm{S}}}
\def\mT{{\bm{T}}}
\def\mU{{\bm{U}}}
\def\mV{{\bm{V}}}
\def\mW{{\bm{W}}}
\def\mX{{\bm{X}}}
\def\mY{{\bm{Y}}}
\def\mZ{{\bm{Z}}}
\def\mBeta{{\bm{\beta}}}
\def\mPhi{{\bm{\Phi}}}
\def\mLambda{{\bm{\Lambda}}}
\def\mSigma{{\bm{\Sigma}}}

% Tensor
\DeclareMathAlphabet{\mathsfit}{\encodingdefault}{\sfdefault}{m}{sl}
\SetMathAlphabet{\mathsfit}{bold}{\encodingdefault}{\sfdefault}{bx}{n}
\newcommand{\tens}[1]{\bm{\mathsfit{#1}}}
\def\tA{{\tens{A}}}
\def\tB{{\tens{B}}}
\def\tC{{\tens{C}}}
\def\tD{{\tens{D}}}
\def\tE{{\tens{E}}}
\def\tF{{\tens{F}}}
\def\tG{{\tens{G}}}
\def\tH{{\tens{H}}}
\def\tI{{\tens{I}}}
\def\tJ{{\tens{J}}}
\def\tK{{\tens{K}}}
\def\tL{{\tens{L}}}
\def\tM{{\tens{M}}}
\def\tN{{\tens{N}}}
\def\tO{{\tens{O}}}
\def\tP{{\tens{P}}}
\def\tQ{{\tens{Q}}}
\def\tR{{\tens{R}}}
\def\tS{{\tens{S}}}
\def\tT{{\tens{T}}}
\def\tU{{\tens{U}}}
\def\tV{{\tens{V}}}
\def\tW{{\tens{W}}}
\def\tX{{\tens{X}}}
\def\tY{{\tens{Y}}}
\def\tZ{{\tens{Z}}}


% Graph
\def\gA{{\mathcal{A}}}
\def\gB{{\mathcal{B}}}
\def\gC{{\mathcal{C}}}
\def\gD{{\mathcal{D}}}
\def\gE{{\mathcal{E}}}
\def\gF{{\mathcal{F}}}
\def\gG{{\mathcal{G}}}
\def\gH{{\mathcal{H}}}
\def\gI{{\mathcal{I}}}
\def\gJ{{\mathcal{J}}}
\def\gK{{\mathcal{K}}}
\def\gL{{\mathcal{L}}}
\def\gM{{\mathcal{M}}}
\def\gN{{\mathcal{N}}}
\def\gO{{\mathcal{O}}}
\def\gP{{\mathcal{P}}}
\def\gQ{{\mathcal{Q}}}
\def\gR{{\mathcal{R}}}
\def\gS{{\mathcal{S}}}
\def\gT{{\mathcal{T}}}
\def\gU{{\mathcal{U}}}
\def\gV{{\mathcal{V}}}
\def\gW{{\mathcal{W}}}
\def\gX{{\mathcal{X}}}
\def\gY{{\mathcal{Y}}}
\def\gZ{{\mathcal{Z}}}

% Sets
\def\sA{{\mathbb{A}}}
\def\sB{{\mathbb{B}}}
\def\sC{{\mathbb{C}}}
\def\sD{{\mathbb{D}}}
% Don't use a set called E, because this would be the same as our symbol
% for expectation.
\def\sF{{\mathbb{F}}}
\def\sG{{\mathbb{G}}}
\def\sH{{\mathbb{H}}}
\def\sI{{\mathbb{I}}}
\def\sJ{{\mathbb{J}}}
\def\sK{{\mathbb{K}}}
\def\sL{{\mathbb{L}}}
\def\sM{{\mathbb{M}}}
\def\sN{{\mathbb{N}}}
\def\sO{{\mathbb{O}}}
\def\sP{{\mathbb{P}}}
\def\sQ{{\mathbb{Q}}}
\def\sR{{\mathbb{R}}}
\def\sS{{\mathbb{S}}}
\def\sT{{\mathbb{T}}}
\def\sU{{\mathbb{U}}}
\def\sV{{\mathbb{V}}}
\def\sW{{\mathbb{W}}}
\def\sX{{\mathbb{X}}}
\def\sY{{\mathbb{Y}}}
\def\sZ{{\mathbb{Z}}}

% Entries of a matrix
\def\emLambda{{\Lambda}}
\def\emA{{A}}
\def\emB{{B}}
\def\emC{{C}}
\def\emD{{D}}
\def\emE{{E}}
\def\emF{{F}}
\def\emG{{G}}
\def\emH{{H}}
\def\emI{{I}}
\def\emJ{{J}}
\def\emK{{K}}
\def\emL{{L}}
\def\emM{{M}}
\def\emN{{N}}
\def\emO{{O}}
\def\emP{{P}}
\def\emQ{{Q}}
\def\emR{{R}}
\def\emS{{S}}
\def\emT{{T}}
\def\emU{{U}}
\def\emV{{V}}
\def\emW{{W}}
\def\emX{{X}}
\def\emY{{Y}}
\def\emZ{{Z}}
\def\emSigma{{\Sigma}}

% entries of a tensor
% Same font as tensor, without \bm wrapper
\newcommand{\etens}[1]{\mathsfit{#1}}
\def\etLambda{{\etens{\Lambda}}}
\def\etA{{\etens{A}}}
\def\etB{{\etens{B}}}
\def\etC{{\etens{C}}}
\def\etD{{\etens{D}}}
\def\etE{{\etens{E}}}
\def\etF{{\etens{F}}}
\def\etG{{\etens{G}}}
\def\etH{{\etens{H}}}
\def\etI{{\etens{I}}}
\def\etJ{{\etens{J}}}
\def\etK{{\etens{K}}}
\def\etL{{\etens{L}}}
\def\etM{{\etens{M}}}
\def\etN{{\etens{N}}}
\def\etO{{\etens{O}}}
\def\etP{{\etens{P}}}
\def\etQ{{\etens{Q}}}
\def\etR{{\etens{R}}}
\def\etS{{\etens{S}}}
\def\etT{{\etens{T}}}
\def\etU{{\etens{U}}}
\def\etV{{\etens{V}}}
\def\etW{{\etens{W}}}
\def\etX{{\etens{X}}}
\def\etY{{\etens{Y}}}
\def\etZ{{\etens{Z}}}

% The true underlying data generating distribution
\newcommand{\pdata}{p_{\rm{data}}}
\newcommand{\ptarget}{p_{\rm{target}}}
\newcommand{\pprior}{p_{\rm{prior}}}
\newcommand{\pbase}{p_{\rm{base}}}
\newcommand{\pref}{p_{\rm{ref}}}

% The empirical distribution defined by the training set
\newcommand{\ptrain}{\hat{p}_{\rm{data}}}
\newcommand{\Ptrain}{\hat{P}_{\rm{data}}}
% The model distribution
\newcommand{\pmodel}{p_{\rm{model}}}
\newcommand{\Pmodel}{P_{\rm{model}}}
\newcommand{\ptildemodel}{\tilde{p}_{\rm{model}}}
% Stochastic autoencoder distributions
\newcommand{\pencode}{p_{\rm{encoder}}}
\newcommand{\pdecode}{p_{\rm{decoder}}}
\newcommand{\precons}{p_{\rm{reconstruct}}}

\newcommand{\laplace}{\mathrm{Laplace}} % Laplace distribution

\newcommand{\E}{\mathbb{E}}
\newcommand{\Ls}{\mathcal{L}}
\newcommand{\R}{\mathbb{R}}
\newcommand{\emp}{\tilde{p}}
\newcommand{\lr}{\alpha}
\newcommand{\reg}{\lambda}
\newcommand{\rect}{\mathrm{rectifier}}
\newcommand{\softmax}{\mathrm{softmax}}
\newcommand{\sigmoid}{\sigma}
\newcommand{\softplus}{\zeta}
\newcommand{\KL}{D_{\mathrm{KL}}}
\newcommand{\Var}{\mathrm{Var}}
\newcommand{\standarderror}{\mathrm{SE}}
\newcommand{\Cov}{\mathrm{Cov}}
% Wolfram Mathworld says $L^2$ is for function spaces and $\ell^2$ is for vectors
% But then they seem to use $L^2$ for vectors throughout the site, and so does
% wikipedia.
\newcommand{\normlzero}{L^0}
\newcommand{\normlone}{L^1}
\newcommand{\normltwo}{L^2}
\newcommand{\normlp}{L^p}
\newcommand{\normmax}{L^\infty}

\newcommand{\parents}{Pa} % See usage in notation.tex. Chosen to match Daphne's book.

\DeclareMathOperator*{\argmax}{arg\,max}
\DeclareMathOperator*{\argmin}{arg\,min}

\DeclareMathOperator{\sign}{sign}
\DeclareMathOperator{\Tr}{Tr}
\let\ab\allowbreak


\definecolor{cvprblue}{rgb}{0.21,0.49,0.74}
\usepackage[pagebackref,breaklinks,colorlinks,allcolors=cvprblue]{hyperref}


\usepackage{hyperref}
\usepackage{graphicx}
\usepackage{caption}
\usepackage{booktabs} 
\usepackage{multirow}
\usepackage{pifont}%
\newcommand{\cmark}{\ding{51}}%
\newcommand{\xmark}{\ding{55}}%
% \usepackage{minted}
\usepackage{sidecap}

\usepackage[linesnumbered,ruled,vlined]{algorithm2e}
\usepackage{xcolor}

\def\paperID{2239} %
\def\confName{CVPR}
\def\confYear{2025}

\definecolor{mydarkblue}{RGB}{0,0,160} %
\newcommand\mycommfont[1]{\footnotesize\ttfamily\textcolor{mydarkblue}{#1}}
\SetCommentSty{mycommfont}
\SetKwInOut{Parameter}{Hyperparams}

\title{DiSciPLE: Learning Interpretable Programs for Scientific Visual Discovery}


\author{
Utkarsh Mall$^{1}$ \hspace{0.05cm} Cheng Perng Phoo$^{2}$  \hspace{0.05cm} Mia Chiquier$^{1}$ \hspace{0.05cm}
Bharath Hariharan$^{2}$ \hspace{0.05cm}  Kavita Bala$^{2}$ \hspace{0.05cm} Carl Vondrick$^{1}$\\
         $^{1}$Columbia University \qquad
        $^{2}$ Cornell University\\
        \small{\enspace Correspondence: \tt{um2171@columbia.edu}
}\\
\href{https://disciple.cs.columbia.edu/}{disciple.cs.columbia.edu}
}


\newcommand{\fix}{\marginpar{FIX}}
\newcommand{\new}{\marginpar{NEW}}

\newcommand{\best}[1]{\textbf{\textcolor{red}{#1}}}
\newcommand{\sota}[1]{\emph{\textcolor{blue}{#1}}}
\newcommand{\second}[1]{\emph{\textcolor{blue}{#1}}}

\def\disciple{DiSciPLE}

\begin{document}

\maketitle
\begin{abstract}


Visual data is used in numerous different scientific workflows ranging from remote sensing to ecology. As the amount of observation data increases, the challenge is not just to make accurate predictions but also to understand the underlying mechanisms for those predictions. 
Good interpretation is important in scientific workflows, as it allows for better decision-making by providing insights into the data. 
This paper introduces an automatic way of obtaining such interpretable-by-design models, by learning programs that interleave neural networks. We propose \disciple~(Discovering Scientific Programs using LLMs and Evolution) an evolutionary algorithm that leverages common sense and prior knowledge of large language models (LLMs) to create Python programs explaining visual data. Additionally, we propose two improvements: a program critic and a program simplifier to improve our method further to synthesize good programs. On three different real-world problems, \disciple~learns state-of-the-art programs on novel tasks with no prior literature. For example, we can learn programs with 35\% lower error than the closest non-interpretable baseline for population density estimation. The supplementary material can be found at: \href{https://disciple.cs.columbia.edu/pdf/supplementary.pdf}{https://disciple.cs.columbia.edu/pdf/supplementary.pdf}




\end{abstract}


\section{Introduction}\label{sec:introduction}
% We introduce a new question: the properties of the quotient code, and its connection to the original code
% Contribution: introduce a new definition
Let $\basefield$ be a finite field, $\blocklength \in \naturalnumbersset$, and let $\variety \subseteq \field$ be a subset
\footnote{As a convention, we use $\tilde{\square}$ to denote properties of the subset, and thus also the subset itself.}
.
We begin by introducing a new definition applicable to any linear code over $\basefield$: the \emph{$\variety$-quotient code}.
We then illustrate this novel definition using Reed-Muller codes, and present a property of $\variety$ which we use to show that $\variety$-quotient Reed-Muller code \emph{inherits its distance and list decoding radius} from the original Reed-Muller code.
Finally, leveraging known results from additive combinatorics and algebraic geometry, we establish as a corollary that this inheritance holds when $\variety$ is a \emph{high-rank variety}.

\paragraph{The Quotient Code}
Let $\gencode$ be a linear code over $\basefield$.
Each codeword of $\gencode$ can be described as a function $\funcdef{\genfunc}{\field}{\basefield}$ that is in the span of the columns of the code's \emph{generator matrix}.
An equivalent way to describe $\gencode$ is using a \emph{parity check matrix}, where a function $\genfunc$ is a codeword if and only if it satisfies the constraints represented by parity-check matrix.
Each such constraint can be thought of as a requirement over a few inputs of $\genfunc$ from $\field$: the requirement that their weighted sum will equal $0$.

The first novel definition we introduce is the definition of the \emph{$\variety$-induced} code:
\begin{definition}[The $\variety$-Induced Code]
    We define the \emph{$\variety$-induced code $\quotientcode$} to be
    the set of all functions $\funcdef{\onvarfunc}{\variety}{\basefield}$
    \footnote{By convention, we use uppercase letters to denote functions with domain $\field$ and lowercase letters to denote functions with domain $\variety$.}
    that satisfy all the constraints \emph{that lie in $\variety$}.
\end{definition}

Let us briefly describe the connection between codewords in $\field$ and $\variety$-induced codewords.
One can easily verify that each original codeword \emph{restricted} to $\variety$ is a valid codeword in the induced code.
\newline
We call an extension of an $\variety$-induced codeword $\funcdef{\onvarfunc}{\variety}{\basefield}$ to valid codeword in the original code (extending its domain to $\field$), a \emph{lift} of $\onvarfunc$.
When each induced codeword has a unique lift, there is a natural 1-to-1 correspondence between the original and induced codeword.
This becomes substantially more interesting for subsets $\variety$ in which induced codewords have \emph{multiple} lifts.
This non-uniqueness weakens the connection between the original codewords and induced codewords, and leads to a richer range of phenomena (and interesting new challenges).

We also note that the other direction is not always true: For a general subset $\variety$, there might be an induced codeword (a valid codeword in the induced code) that \emph{cannot be lifted} to a valid codeword in $\field$.
We are interested to better understand $\quotientcode$ using $\gencode$ and vice-versa, and therefore we introduce a new notion, which is the notion of the \emph{$\variety$-quotient code}:
\begin{definition}[The $\variety$-Quotient Code]
    Let $\gencode$ be a linear code, and let $\quotientcode$ be the $\variety$-induced code of $\gencode$.
    We say $\quotientcode$ is a \emph{$\variety$-quotient code}
    if every quotient codeword $\onvarfunc \in \quotientcode$ has a lift to $\field$.
\end{definition}
In the case described above, we also say that $\variety$ is a \emph{lift-enabler} for $\gencode$ and that the code $\gencode$ is a \emph{covering code} for the code $\quotientcode$.
\newline
The novelty of this definition is that it captures subsets in which \emph{there is} a correspondence between codewords in $\variety$ and in $\field$,
and the correspondence may be \emph{1-to-many}.

\paragraph{Importance of Definition}
This timely definition extends a fundamental and useful concept previously introduced for graphs and complexes—namely, the notion of a \emph{covering graph} or alternatively, the \emph{quotient graph}.
This concept gained an increasing prominence in theoretical computer science, where it was recently employed to construct \emph{high dimensional expanders}~\cite{dikstein2022newhighdimensionalexpanders, yaacov2024sparsehighdimensionalexpanders}
and achieve improved \emph{local testing} results~\cite{gotlib2022listagreementexpansioncoboundary, dikstein2024agreementtheoremshighdimensional, bafna2024characterizingdirectproducttesting},
where the latter also played a crucial role in constructions of PCPs.
Consequently, the study of covering spaces for graphs has found usages in theoretical computer science and specifically in development of PCPs with enhanced properties.
We believe our question, which explores the analogous question for codes, will similarly lead to meaningful applications in theoretical computer science.

In addition to that, the question of \emph{puncturing} of codes has caught much attention recently, in a line of work~\cite{brakensiek2024genericreedsolomoncodesachieve, alrabiah2024randomlypuncturedreedsolomoncodes, brakensiek2024generalizedgmmdspolynomialcodes, brakensiek2024agcodesachievelistdecoding},
followed by the resolution of the GM-MDS conjecture~\cite{DBLP:journals/corr/abs-1803-02523, DBLP:journals/corr/abs-1803-03752}.
Where the question of puncturing is focused exclusively on the case where the lift is \emph{unique},
the study of quotient codes also tackles subsets $\variety \subseteq \field$ where the lift is \emph{not unique}.
Notably, in the unique-lift case there are well-established lower-bounds for the size of $\variety$ such as~\cite[Theorem 1.1]{DBLP:journals/cc/DoronTT22}.
In contrast, the size of $\variety$ in quotient codes may be \emph{much smaller} than its lower-bound in punctured code (for example in Reed-Muller codes), suggesting the potential for new insights and improved results.

\paragraph{Our Question}
Our goal is to answer the following question:
what properties of $\variety$ will imply that the quotient code inherits its distance and list-decoding radius from the original code?

This question is analogous to the study of quotients of expander graphs—just as not all quotients of an expander necessarily preserve expansion,
not all subsets $\variety$ necessarily yield a well-behaved quotient code.
Understanding the conditions under which expansion is preserved has been a fundamental problem in the study of expanders,
and similarly, identifying the conditions under which a quotient code retains key properties of the original code is a central challenge in our work.
Given this parallel, we believe our question may have broader implications for future research in both coding theory and theoretical computer science.

We answer this question in the context of \emph{Reed-Muller codes}.
Notably, our approach does \emph{not only} address the case of where there are multiple lifts,
but also introduces a novel framework for analyzing unique-lift (puncturing) setting when the field size is constant-a scenario that is typically considered more challenging.

\paragraph{Reed-Muller Codes}
Let $\basefield$ be a finite field, and let $\blocklength, \degree$ be integers.
Each codeword in Reed-Muller code $\reedmullercodeex{\basefield}{\field}{\degree}$,
is defined by a polynomial over $\basefield$ in $\blocklength$ variables with total degree $\leq \degree$
\footnote{We focus on the regime where $\degree, \abs{\basefield}$ are considered constants and $\blocklength$ is considered very large.}
.
The message that one wishes to encode is represented in the code as a polynomial $\funcdef{\genpoly}{\field}{\basefield}$, whose coefficients are the different message characters.
The encoding of the message is a vector of the different evaluation of $\genpoly$ over \emph{all} possible points in $\field$.

Alternatively, one can describe Reed-Muller codes using a set of local constraints.
A function $\funcdef{\genfunc}{\field}{\basefield}$ is a polynomial of degree $\leq \degree$
if and only if the (alternating) sum of each possible \emph{cube}, which is a set of points of the form $\set{x + \sum_{i \in S} y_i}_{S \subseteq \sparens{\degree + 1}}$ for $x, y_1,...,y_{\degree+1} \in \field$, equals $0$.
The set of all cubes is \emph{the set of constraints of degree-$\degree$ polynomials}.

Next, we present our notations for the induced Reed-Muller code:
\begin{notation}[The $\variety$-Induced Reed-Muller Code]
    We say a function $\funcdef{\genfunc}{\variety}{\basefield}$ is a \emph{polynomial of degree $\leq \degree$ \emph{in $\variety$}}
    if it satisfies all the constraints of degree-$\degree$ polynomials \emph{that lie in $\variety$}.
    \newline
    We denote the $\variety$-induced Reed-Muller code:
    \[
        \reedmullercodeex{\basefield}{\variety}{\degree} = \set{\funcdef{\onvarpoly}{\variety}{\basefield} \suchthat \onvarpoly \text{ is a polynomial of degree } \leq \degree \text{ in } \variety}
    \]
\end{notation}

\paragraph{Properties of Induced Reed-Muller Codes}
A study of Ziegler and Kazhdan~\cite{kazhdan2018polynomial, kazhdan2019extendingweaklypolynomialfunctions, kazhdan2020propertieshighranksubvarieties}
shows that if $\variety$ is a \emph{high rank variety}
\footnote{Under some conditions we describe later.}
, then $\variety$ is a \emph{lift-enabler} for $\reedmullercodeex{\basefield}{\field}{\degree}$.
In other words, the authors showed that the $\variety$-induced Reed-Muller code is in fact a \emph{$\variety$-quotient Reed-Muller code}.
%More accurately, Ziegler and Kazhdan showed that every polynomial in a high-rank variety
%can be lifted to a polynomial in $\field$ with the same degree
%\footnote{This is a stronger variant of being a quotient code: every polynomial of degree $\degree^\prime \leq \degree$ in $\variety$ is liftable to a polynomial of \emph{the same degree} $\degree^\prime$ in $\field$.}
%.
%\newline
%We call a subset $\variety \subseteq \field$ that has this property for polynomials of degree $\leq \degree$ a \emph{$\degree$-lift-enabler},
We rely on this property of $\variety$ as a black-box.
See Section~\ref{sec:polynomials_in_X} for more details in this regard.

An additional property of $\variety \subseteq \field$ we rely on is the connection between \emph{algebraic structure} and \emph{random behavior (equidistribution)} of polynomials in $\variety$.
\newline
For $\field$, this connection is a well-studied result~\cite{green2007distribution, kaufman2008worst, DBLP:journals/corr/0001L15}.
%which is formally described by the relation of \emph{rank} and \emph{bias} accordingly.
%We clarify that rank is a measure of algebraic structure, where low rank implies being structured,
%and bias is a measure of lack of random behavior, where low bias implies being equidistributed.
It lies in the heart of many results in higher-order Fourier analysis,
and specifically was used in~\cite{bhowmick2014list} to analyze the list decoding radius of Reed-Muller code in $\field$.
\newline
The equivalent of this relation for subsets $\variety \subseteq \field$ was studied in~\cite{lampert2021relative, gowers2022equidistributionhighrankpolynomialsvariables}.
These works captured the measure of algebraic-structure in $\variety$ by a definition called \emph{relative rank},
and captured the lack of random behavior in $\variety$ by a definition called \emph{relative bias}.
We note that for subsets, the definition of algebraic structure of a polynomial in $\variety$ considers the algebraic structure of \emph{all its possible} lifts.
It was shown in~\cite{lampert2021relative} that when $\variety$ is a high-rank variety, high relative rank implies low relative bias
\footnote{
    Note that even though Gowers and Karam~\cite{gowers2022equidistributionhighrankpolynomialsvariables} also acheived a similar relation for a type of subsets,
    the definition of rank they used is slightly different than the standard definition of rank.
    While this difference may seem unharmful at first, it is, to our knowledge, does not allow to do a \emph{regularization} process
    (note that a generalization of this process is the heart of our proof).
}
.
\newline
We use this property as a black box as well.
When a subset $\variety \subseteq \field$ has such property for polynomials of degree $\leq \degree$, we say that it has the \emph{$\degree$-relative rank-bias property}.
See Section~\ref{sec:relative-rank-bias-property} for more details.

\paragraph{Our Results}
Next, let us present our main theorem more concretely.
Our work focuses on the regime where $\degree < \abs{\basefield}$ for prime finite fields $\basefield = \basefield_p$.
Throughout this paper, we always assume these two assumptions.
Denote the \emph{minimum normalized distance of $\reedmullercodeex{\basefield}{\field}{\degree}$} by $\normalizedcodedistanceex{\basefield}{\field}{\degree}$,
shorthand by $\normalizedcodedistance{\basefield}{\degree}$.
We have:
\[
    \normalizedcodedistance{\basefield}{\degree} = 1 - \degree/\abs{\basefield}
\]
Moreover, we define the \emph{list decoding count} of $\reedmullercodeex{\basefield}{\field}{\degree}$ by:
\[
    \listpolycount{\basefield}{\field}{\degree}{\tau} \definedas
    \max_{\funcdef{\genfunc}{\field}{\basefield}}
        {\abs{\set{\genpoly \in \allpolyset{\leq \degree}{\field}{\basefield} \suchthat {\dist{\genpoly, \genfunc} \leq \tau}}}}
\]
Let $\listdecodingradiusex{\basefield}{\field}{\degree}$ be the \emph{list decoding radius} of $\reedmullercodeex{\basefield}{\field}{\degree}$,
which is the maximum $\tau$ for which $\listpolycount{\basefield}{\field}{\degree}{\tau - \epsilon}$ is bounded by a \emph{constant} depending only on $\epsilon, \abs{\basefield}, \degree$.
\newline
In the paper~\cite{bhowmick2014list} it was shown that for constant field size and degree, the list decoding radius \emph{reaches the distance of the code}, as conjectured earlier by~\cite{10.1145/1374376.1374417}
\footnote{Note that it is known that $\listdecodingradiusex{\basefield}{\field}{\degree} \leq \normalizedcodedistance{\basefield}{\degree}$,
    and therefore, in a sense, their result is \emph{optimal in $\field$} assuming $\degree, \abs{\basefield}$ are considered as constants.}
.
We denote the corresponding distance parameter of $\variety \subseteq \field$ by $\normalizedcodedistanceex{\basefield}{\variety}{\degree}$ and $\listdecodingradiusex{\basefield}{\variety}{\degree}$ respectively.

We next present our main theorem, which establishes that the \emph{list decoding radius} of the quotient Reed-Muller code is \emph{at least as good} as the that of the original code:
\begin{theorem*}[List Decoding Quotient Reed-Muller Code]
\footnote{Informal, for formal see Theorem~\ref{thm:list-decoding-RM-in-X}.}
Let $\basefield$ be a finite (prime) field of constant size, let $\degree \in \naturalnumbersset$ be a constant such that $\degree < \abs{\basefield}$,
and let $\blocklength \in \naturalnumbersset$ be an integer.
\newline
Let $\variety \subseteq \field$ be a subset that is a lift-enabler for $\reedmullercodeex{\basefield}{\field}{\degree}$ and has the $\degree$-relative rank-bias property.
\newline
Then, $\reedmullercodeex{\basefield}{\variety}{\degree}$ inherits its \emph{list decoding radius} from $\reedmullercodeex{\basefield}{\field}{\degree}$, i.e:
\[
    \listdecodingradiusex{\basefield}{\variety}{\degree} \geq \listdecodingradiusex{\basefield}{\field}{\degree}
\]
\end{theorem*}

In addition, we also achieve a (simpler) result regarding the \emph{distance} of the quotient Reed-Muller code (Theorem~\ref{thm:distance-of-RM-in-X}):
Under the conditions described above,
$\reedmullercodeex{\basefield}{\variety}{\degree}$ also inherits its \emph{distance} from $\reedmullercodeex{\basefield}{\field}{\degree}$, i.e
$\normalizedcodedistanceex{\basefield}{\variety}{\degree} \geq \normalizedcodedistanceex{\basefield}{\field}{\degree}$
\footnote{Our techniques also show that also the other direction is true, which yields an \emph{equality} in the distance of the two codes.}
.

As a corollary, using results studied in~\cite{kazhdan2018polynomial, kazhdan2019extendingweaklypolynomialfunctions, lampert2021relative} regarding high-rank varieties, we obtain the following:
%TODO: Add this theorem formally in the end.
\begin{corollary*}[List Decoding Quotient Reed-Muller Code: High Rank Variety]
%TODO: [Informal. For formal see...]
    Let $\variety \subseteq \field$ be a \emph{high rank variety},
    that is, $\variety$ is the set of common zeros of a collection of polynomials $\varpolyset = (\varpoly_1,...,\varpoly_{\varietypolycount})$
    that is of \emph{high rank}
    \footnote{We note that the higher the rank of the collection is, the more accurate the greater or equal in the theorem is.}
    \footnote{We also note that for this result some assumptions are needed regarding the field size or the degree of the polynomials in the collection.}
    , i.e. $\variety = \zerofunc{\varpolyset} = \set{x \suchthat \forall i: \varpoly_i(x) = 0}$.
%    \newline
%    Assume that either $\abs{\basefield} > \degree \cdot \varietydeg$ or all polynomials of the collection $\varpolyset$ are of degree $> \degree$.
    \newline
    Then, $\reedmullercodeex{\basefield}{\variety}{\degree}$ inherits its distance parameters from $\reedmullercodeex{\basefield}{\field}{\degree}$, i.e:
    \begin{enumerate}
        \item $\normalizedcodedistanceex{\basefield}{\variety}{\degree} \geq \normalizedcodedistanceex{\basefield}{\field}{\degree}$.
        \item $\listdecodingradiusex{\basefield}{\variety}{\degree} \geq \listdecodingradiusex{\basefield}{\field}{\degree}$.
    \end{enumerate}
\end{corollary*}

\paragraph{Main Technical Challenge}
We achieve these results by combining the two black-box properties of subsets $\variety \subseteq \field$ we presented.
Analysis of the polynomials in $\variety$ raises a new challenge, as previous techniques that were used to analyze low-degree polynomials,
both regarding $\field$~\cite{green2007distribution} and regarding subsets $\variety$~\cite{lampert2021relative},
were focused on maintaining the behavior of polynomials \emph{in the set they work on} ($\field$ and $\variety$ accordingly).
\newline
The novelty of our new technique is that it uses a similar approach to analyze polynomials $\variety$ as commonly used in $\field$,
\emph{while simultaneously maintaining a connection} between polynomials in $\variety$ to polynomials in $\field$.
This connection allows us to deduce that polynomials in $\variety$ behave similarly to polynomials in $\field$.
Informally, given a question regarding a polynomial in $\variety$, our new technique allows us to
associate it with a ``correct'' lift of it, and answer the question \emph{using properties of its lift}.
We emphasize that the correct lift (the one we later choose to use) \emph{depend} on the question,
thus we cannot pick a single canonical lift to \emph{generally} describe each polynomial in $\variety$.

Next we describe this challenge in more detail.
\newline
Analyses of polynomials in $\field$ were commonly based on the structure-randomness connection of polynomials in $\field$.
To use this connection, a procedure introduced by~\cite{green2007distribution}, which is called the \emph{regularization process}, is often used~\cite{kaufman2008worst, tao2011inverse, hatami2011higher, bhattacharyya2013locally, bhattacharyya2013algorithmic, DBLP:journals/corr/0001L15}.
This procedure takes any collection of polynomials, and constructs from it another collection of polynomials that has \emph{equidistriubtion in $\field$}
and ``captures'' all functions ``captured'' by the previous collection.
This notion of ``capturing'' is formulated by a definition called \emph{measurable},
and thus it is required that every function measurable by the old collection will be measurable by the new collection.

We note that the regularization procedure achieves random behavior in $\field$ by requiring the collection to have an \emph{extremely low algebraic structure}:
This implies the new collection has random behavior (equidistributed) as it is a property of $\field$.
The notion of structure is captured by a definition called \emph{rank}, where a polynomial with high rank has extremely low structure.
Additionally, the notion of lack of random behavior is captured by a definition called \emph{bias}, where a polynomial with low bias behaves randomly (equidistributed).
Therefore, the equidistribution is achieved in the regularization process by constructing a collection with high rank, as \emph{in $\field$ high rank implies low bias}.

To generalize these ideas to $\variety$, one must achieve a similar result in $\variety \subseteq \field$:
Given any collection of polynomials, construct a new collection of polynomials that is both equidistributed in $\variety$
and captures every function in $\variety$ that was previously captured.
In our case, however, we must also ensure that the new collection also captures all functions that were previously-captured \emph{in $\field$},
as in our case we wish to use the connection of polynomials in $\variety$ to polynomials in $\field$.
This can be summarized by 3 requirements:
\begin{enumerate}
    \item The polynomials in the new collection will behave random in $\variety$.
    \item Every function that was measurable in $\variety$ by the old collection will be measurable by the new collection in $\variety$.
    \item Every function that was measurable \emph{in $\field$} by the old collection will be measurable by the new collection \emph{in $\field$}.
\end{enumerate}
Alas, this third-requirement is incompatible with the way we achieve the first requirement.
Achieving the first requirement, which is the random behavior in $\variety$, is done by requiring an extremely low algebraic structure \emph{according to relative rank}.
This requires one to consider all possible lifts of polynomials in the collection to avoid any structure.
\newline
More accurately
\footnote{As the polynomials we have here are polynomials in $\field$ we can not discuss their lift.}
, for a polynomial $\funcdef{\genpoly}{\field}{\basefield}$,
we define an \emph{$\variety$-equivalent polynomial for $\genpoly$} to be
a polynomial in $\field$ that coincides with $\genpoly$ on $\variety$ and has the same degree bound as $\genpoly$
\footnote{This is the same as considering all lifts of the polynomial $\restrictfunc{\genpoly}{\variety}$, assuming such lift exist.}
.
Using this definition, the definition of relative rank requires examining all possible \emph{$\variety$-equivalent} polynomials,
and ensuring non of them exhibit structure.
\newline
Typically (in $\field$ for example), avoiding structure is achieved by replacing every structured polynomial by a \emph{small}
set of less-structured polynomials that capture it.
We note that it is \emph{crucial} that the set is small, and from reasons we did not explain here (see definition~\ref{definition:rank}), it is promised because the polynomial we wish to replace is structured.
\newline
For $\variety$, we aim to avoid \emph{all} $\variety$-equivalent polynomials of a polynomial from being structured.
Achieving this, while keeping the collection small,
requires one to replace the polynomial by a set of less-structured polynomials that capture a \emph{structured-$\variety$-equivalent} of it.
Therefore, this process creates a new collection that captures this $\variety$-equivalent polynomial,
but does not necessarily capture the original polynomial!
\newline
In summary, the challenge is that avoiding the structure of \emph{all} the lifts of a polynomial to achieve equidistribution in $\variety$,
without adding too many polynomials, may harm the functions we capture in $\field$.

\paragraph{Introducing New Tools}
We overcome this challenge by presenting a new definition that relaxes the notion of \emph{measurable} we required for functions in $\field$,
which we call \emph{$\variety$-measurable}.
This enables us to describe a relaxed version of the regularization process,
in which we require that every function in $\field$ that was $\variety$-measurable by the old collection will still be $\variety$-measurable by the new collection.
In contrast to the original regularization process, which mandated that functions that were measurable by the old collection will be measurable by the collection,
this relaxed definition only requires such functions to be \emph{$\variety$-measurable} by the new collection.

Even though we no longer need to capture all previously captured functions in $\field$,
it is important that the new relaxed-definition is strict enough to keep the connection between polynomials in $\variety$ and in $\field$.
Therefore, maintaining the $\variety$-measurable functions throughout the regularization process cannot be done trivially,
and this is handled in a procedure we call \emph{the $\variety$-relative regularization process} which is a stronger-version of the regularization process that is used in $\field$.
This new definition and procedure are thoroughly described in Section~\ref{sec:regularization-relative-to-X}.

We note that these new definition and procedure are a novel contribution of this work, and we believe they can
be useful in future research of the quotient Reed-Muller code.


\subsection{Comparison to Related Work}\label{subsec:previous-work}
In~\cite{bhowmick2014list} the authors studied the list decoding radius of Reed Muller codes $\field$.
They proved that, for prime fields, the list decoding radius \emph{reaches the distance of the code}, as conjectured earlier by~\cite{10.1145/1374376.1374417}
\footnote{Note that it is known that $\listdecodingradiusex{\basefield}{\field}{\degree} \leq \normalizedcodedistance{\basefield}{\degree}$,
    and therefore, in a sense, their result is \emph{optimal in $\field$} assuming $\degree, \abs{\basefield}$ are considered as constants.}
\footnote{We also note that their work also apply to the regime $\degree \geq \abs{\basefield}$. }
.
Formally, they showed the following theorem:
\begin{theorem}~\cite[Theorem 1]{bhowmick2014list}
Let $\basefield$ be a prime field.
Let $\epsilon > 0$ and $\degree, \blocklength \in \naturalnumbersset$.
There exists a constant
\footnote{It is important to note that $c$ is \emph{independent of $\blocklength$}.}
$c \definedas c(\abs{\basefield}, \degree, \epsilon)$ such that:
\[
    \listpolycount{\basefield}{\field}{\degree}{\normalizedcodedistance{\basefield}{\degree}- \epsilon} \leq c
\]
\end{theorem}
Our work gives new tools for analyzing polynomials in $\variety \subseteq \field$,
which we later use to follow their line of proof and show equivalent result \emph{in $\variety$}.

We next present related work regarding the study of polynomial codes in subsets $\variety \subseteq \field$.
Before presenting them specifically, we note that our work has a \emph{fundamental difference} than that of the previous study of polynomials in subsets.
%The essence of the difference lies in the way properties are deduced regarding polynomials in $\variety$.
Most works which studied polynomials over subsets $\variety \subseteq \field$ were focused on subsets in which every polynomial has a \emph{unique} lift.
This ensures that there is a 1-to-1 correspondence between polynomials in $\variety$ and in $\field$
and therefore allows easier connection between polynomials in $\variety$ and in $\field$.
%This connection is the most common approach to deduce properties regarding polynomials in $\variety$:
%such properties are inferred from the characteristics of their lifts,
%which are more comprehensively understood, as each lift is a polynomial in $\field$
%they are deduced from the properties of their lifts, which we better understand (as each lift is a polynomial in $\field$).
%\footnote{A possible reason for that focus is their focus on the \emph{extrinsic} definition of a polynomial,
%    that that is more focused on the behaviour of the origianl codewords \emph{restricted} to $\variety$
%    rather than the intrinsic definition, which captures the code \emph{quotient} by $\variety$.}
%.
\newline
We note that our work is non-trivial even in this case:
it extracts the properties of $\field$ that were used in~\cite{bhowmick2014list}, in a way they can be used to analyze quotient Reed-Muller codes.
However, as described earlier, our work addresses an additional substantial challenge which arise when the lift is \emph{not} unique.
Thus our work is only comparable to other works in the unique-lift case, which is the less-challenging case we address.

The first line of work we mention is this regard is the study of hitting sets for low degree polynomials~\cite{6243404, 10.1145/2554797.2554828, 6875485},
and a stronger variant of it which is the study of pseudorandom-generators against low degree polynomials~\
    \cite{10.1145/1060590.1060594, 4389478, 10.1145/1374376.1374455, 4558816,  Cohen2013PseudorandomGF, derksen2022fooling, dwivedi2024optimalpseudorandomgeneratorslowdegree}
Both definitions capture subsets
\footnote{Sometimes this subset is allowed to be a \emph{multiset}.}
$\variety \subseteq \field$ such that every polynomial over $\field$ has a non-negligible distance from $0$ \emph{when restricted to $\variety$}.
This requirement implicitly implies that every low degree polynomial over $\variety$ has at most a \emph{single} lift.

Another line of work worth mentioning in this regard is~\cite{4558818, guruswami2017efficientlylistdecodablepuncturedreedmuller},
which studied \emph{puncturing of Reed-Muller codes}.
This line of work studied the construction of sets $\variety \subseteq \field$,
such that puncturing Reed-Muller codes over $\variety$, that is, taking every original codeword and \emph{restricting} it to $\variety$, will yield a good error-correction code.
To perform their analysis, it was important that every polynomial in $\variety$ has at most a single lift,
and therefore it was an assumption in their work.
%We also note that our work answers similar questions regarding the distance parameter of the constructed code,
%but instead of focusing on a specific construction,
%it describes general properties for subsets that achieve the desired distance parameters (with an explicit construction that has those properties).

%TODO: change the order of references to be all increasing
The papers~\cite{brakensiek2024genericreedsolomoncodesachieve, alrabiah2024randomlypuncturedreedsolomoncodes, brakensiek2024generalizedgmmdspolynomialcodes}
also studied similar questions.
This line of work is followed by the resolution of the \emph{GM-MDS conjecture}, which was proved by~\cite{DBLP:journals/corr/abs-1803-02523, DBLP:journals/corr/abs-1803-03752}.
%After its resolution, it was proved by~\cite{brakensiek2024genericreedsolomoncodesachieve} that over \emph{a field of size $2^{O(\blocklength)}$}, randomly punctured Reed-Solomon codes are combinatorially list-decodable all the way to the list decoding capacity.
%The paper~\cite{brakensiek2024generalizedgmmdspolynomialcodes} generalizes this result and achieves similar results for \emph{all} polynomial codes over exponentially large fields.
%Following this line of work~\cite{alrabiah2024randomlypuncturedreedsolomoncodes} showed that for Reed-Solomon codes, a similar result can be attained to a field of a smaller size.
%Specifically, \emph{linear in $\blocklength$}.
\newline
We note that these works
were focused on the regime where the field is \emph{large}.
More specifically,
they require that the field is \emph{large in respect of $\blocklength$}, i.e $\Omega(\blocklength)$.
We emphasize that our work is focused on \emph{constant fields}.
Moreover, their results were regarding \emph{random} puncturing, while our result makes an \emph{explicit} puncturing.

%Another line of work in this regard studies puncturing of AG codes~\cite{DBLP:journals/corr/abs-1708-01070, guo2021efficientlistdecodingconstantalphabet, brakensiek2024agcodesachievelistdecoding}.
%The newest paper in this line of work~\cite{brakensiek2024agcodesachievelistdecoding} showed that random puncturing of AG codes achieve list decoding capacity over constant fields,
%and as a corollary they showed that AG codes (without puncturing) achieve list decoding capacity over constant fields.
%\newline
%We note that AG codes are a generalization of Reed-Solomon codes,
%where our work is focused on Reed-Muller codes in subsets $\variety \subseteq \field$, which can be thought of as a generalization of AG codes to multiple transcendental variables.
%
We also note that most studies presented above also achieved results regarding the \emph{rate} of the punctured code.
This property of the code can be analyzed naturally when each polynomial over $\variety$ has a \emph{unique} lift, as such assumption implies that the number of polynomials remains the same in $\variety$ as of in $\field$.
As our work does \emph{not} assume such uniqueness, the rate of the code we consider is not analyzed in our work, and thus remained \emph{an open problem}
\footnote{Note that is highly dependent on $\variety$, as additional assumptions are needed to acheive good results in this regard.}
.

\subsection{Proof Overview}\label{subsec:our-work}
In this subsection we present our main technical contribution, which is how we address the challenge of \emph{non-unique lift}.
This is done by introducing the definition of being \emph{$\variety$-measurable}, and by presenting a new tool which is the \emph{relative regularization process}.

To describe them clearly, we first elaborate more on two definitions we described briefly.
\paragraph{Measurable}
Suppose we have a collection of polynomials
\footnote{In this context we think of $c$ as a small (constant for example).}
$\genpolyset[1] = \parens{\genpoly_1,...,\genpoly_c}$ where $\funcdef{\genpoly_i}{\field}{\basefield}$ is a polynomial of degree $\leq \degree$.
We say a function $\funcdef{\genfunc}{\field}{\basefield}$ is \emph{measurable in respect of $\genpolyset[1]$} if it can be determined by the values of $\genpoly_1,...,\genpoly_c$:
if one knows the values of $\genpoly_1(x),...,\genpoly_c(x)$, then she also knows the value of $\genfunc(x)$.
This mathematical-analysis notion, which was first used in a similar context in~\cite{green2007primescontainarbitrarilylong}, is formally defined as follows:
\begin{definition}[Measurable]
    We say a function $\funcdef{\genfunc}{\field}{\basefield}$ is \emph{measurable in respect of $\genpolyset[1] = \parens{\genpoly_1,...,\genpoly_c}$} if
    there exists $\funcdef{\Gamma_{\genfunc}}{\basefield^c}{\basefield}$ such that:
    \[
        \genfunc(x) = \Gamma_{\genfunc}(\genpoly_1(x),...,\genpoly_c(x))
    \]
\end{definition}
This definition can be thought of as the collection $\genpolyset$ ``captures'' the function $\genfunc$
\footnote{Note that this definition also generalizes to every collection of functions.
For now, one can think of the collcetion as a collection of bounded degree polynomials.}
\footnote{One can think of this definition as a generalization of linear span: the collection \emph{spans} the function, where $\Gamma$ is some notion of a span.}
.

Moreover, it would have been useful had this collection of polynomials been ``pseudo-random'', i.e the vector $\parens{\genpoly_1(x),...,\genpoly_c(x)}$ would be equidistributed over a random input $x \in \field$.
This equidistribution would allow us to better understand functions $\genfunc$ that are measurable in respect of $\genpolyset$.

As $\field$ has the rank-bias property, this equidistribution can be achieved by requiring $\genpolyset$ to be a collection of high-rank.
This is a fundamental idea behind the regularization process, first presented in~\cite{green2007distribution}.
Given a collection of polynomials $\genpolyset$, the regularization process constructs another collection $\genpolyset[2]$ of polynomials (with the same degree bound),
such that $\genpolyset[2]$ is a collection of high-rank (and therefore equidistributed) that \emph{refines} $\genpolyset$.
By refine, we mean that every function that was measurable by the first collection $\genpolyset$ is also measurable by the new collection $\genpolyset[2]$ (See definition~\ref{def:semantic-refinement}).

\paragraph{Relative Rank}
We remind the reader that $\variety$-relative rank is a notion that measures the algebraic structure of a polynomial in a subset $\variety \subseteq \field$, by considering the structure of all of its $\variety$-equivalent polynomials.
This notion was presented by~\cite{gowers2022equidistributionhighrankpolynomialsvariables, lampert2021relative}, and is used to achieve equidistribution in $\variety$ assuming $\variety$ has relative rank-bias property.
It is defined as follows:
\begin{definition}[Relative Rank, informal.
See definition~\ref{def:relative-rank-of-polynomial}]
    Let $\variety \subseteq \field$ be a subset,
    let $\degree \in \naturalnumbersset$, and let $\funcdef{\genpoly}{\field}{\basefield}$ be a polynomial of degree $= \degree$.
    The $\variety$-relative rank of $\genpoly$ is defined as follows:
    \[
        \relrank{\variety}{\genpoly} \definedas \min \set{\rank{\genpoly - \relativeremainder{\genpoly}} \suchthat
        \relativeremainder{\genpoly} \in \allpolyset{\leq \degree}{\field}{\basefield}, \restrictfunc{\relativeremainder{\genpoly}}{\variety} \equiv 0}
    \]
\end{definition}
%\paragraph{Concrete Sets Examples}
%In recent years, a few works have studied subsets such that have the relative rank-bias property, specifically in the regime where $\degree < \abs{\basefield}$.
%\newline
%One of them is a paper~\cite{lampert2021relative}, in the line of work that studies subsets $\variety \subseteq \field$ that are \emph{high rank varieties}~\cite{kazhdan2020propertieshighranksubvarieties, kazhdan2018polynomial, kazhdan2019extendingweaklypolynomialfunctions, kazhdan2017extendinglinearquadraticfunctions}.
%An algebraic variety is defined as the zero set of a collection of polynomials $\set{\varpoly_1,...,\varpoly_{\varietypolycount}}$, i.e. $\variety = \set{x \suchthat \varpoly_1(x)=...=\varpoly_{\varietypolycount}(x)=0}$.
%When the collection of polynomials that generate the algebraic variety has a high rank (as a collection),
%we say that the variety is a \emph{high rank variety}
%\footnote{
%    Note that the rank of a collcetion of polynomials is a rather different from the rank of a single polynomial,
%    but as it captures a similar idea, we leave the exact definitions to later (see definition~\ref{definition:factor-rank}).
%}
%.
%TODO: Always put the names of the authors after an cite. Instead saying "they", always call them by name of say "the authors".
%In~\cite[Theorem 1.8]{lampert2021relative}, Lampert and Ziegler have shown that \emph{high rank varieties have the relative rank-bias property}.
%We translate their statement to our point of view in Corollary~\ref{high-rank-variety-has-limited-rank-relative-bias-property}.

%Additionally, in~\cite{gowers2022equidistributionhighrankpolynomialsvariables}, Gowers and Karam studied subsets $\variety \subseteq \field$ of \emph{restricted alphabets}.
%Formally, let $S \subset \basefield$ be a set that restricts the alphabet, and denote $\variety \definedas S^{\blocklength}$.
%In other words, the codewords of the code are polynomials over $\field$, restricted only to inputs from $S^\blocklength$.
%In~\cite[Theorem 1.4]{gowers2022equidistributionhighrankpolynomialsvariables} it was shown that \emph{restricted alphabet subsets have the relative rank-bias property}.
%\newline
%Note that the definition of rank presented by Gowers and Karam is slightly definition than the standard definition of rank.
%While this difference may seem unharmful at first, it is, to our knowledge, incompatible to the \emph{relative regularization} process (which is the heart of our proof).
%A more detailed explanation regarding this difference is given in Note~\ref{note:comparison-to-gowers-rank}.
%
%\paragraph{Formulating the Notion}
%This property is parametrized by the following parameters:
%The field $\basefield$ which we work on, an integer representing the degree of evaluated polynomials $\degree$,
%and a function $\funcdef{\rankbiasfunc}{\realnumbersset^{+}}{\naturalnumbersset}$ that for every $\epsilon$ returns the relative rank needed so that the bias in $\variety$ will be $< \epsilon$.
%In this paper, we think about the first two parameters as constants.
%We think of the last parameter as \emph{how well the subset is regarding the relative rank-bias property}.
%Under this parameterization, we may say a subset $\variety \subseteq \field$ has $(\rankbiasfunc, \basefield, \degree)$-relative rank-bias property.
%For formal definition, see Subsection~\ref{sec:relative-rank-bias-property}.
%\newline
%Next, we note that sometimes, the requirement that a subset $\variety \subseteq \field$ will have the relative rank-bias property to \emph{all} extent is \emph{impossible} to achieve.
%Here, by ``extent'' we mean how small $\epsilon > 0$ can be, such that high relative rank will imply the bias in $\variety$ is $< \epsilon$.
%We see this challenge raised with subsets that are not described by a ``boolean'' property (where a subset either has it or not, such as restricted alphabets),
%but by subsets that are described by a more ``continuous'' property (properties that subsets can have to some extent).
%An example of the latter is the case where $\variety$ is a high rank variety: the rank of the variety is a natural number, thus being a subset of ``high rank'' is not a boolean property; it ``gets better'' the higher the rank of variety is.
%Subsets that have such properties but still have the relative rank-bias property to some extent, are captured by a definition we call \emph{limited relative rank-bias property}.
%We formulate this notion in Subsection~\ref{subsec:limited-relative-rank-bias-property}.

%\subsubsection{Measurable and The Regularization Process}

\subsubsection{\titlevariety-measurable and The \titlevariety-Relative Regularization Process}
In this subsection we discuss the generalization of the regularization process to subsets $\variety \subseteq \field$ using the equivalent of rank-bias relation in $\variety$.
We name this tool \emph{the relative regularization process}.
%This is a tool that enables us to achieve similar equidistribution properties of $\genpolyset$ in $\variety$ as the regularization process achieved in $\field$.
%Note that, as in $\field$ it was achieved using the rank-bias property of $\field$,
%we now require that $\variety$ has the \emph{relative} rank-bias property.
%Additionally, in this case, instead of refining the collection $\genpolyset$ to a collection of high rank, we refine the collection $\genpolyset$ to be a collection of high \emph{$\variety$-relative rank}.
%This requirement adds an extra challenge, thus requiring us to relax our definition of what is a \emph{refinement}, or what is being ``measurable in respect of $\genpolyset$''.

%We note that this is our main tool that enables us to use tools from high-order Fourier analysis to also analyze polynomials in $\variety$.
Practically, we use this tool to show that given a specific question in mind, every $\funcdef{\onvarpoly}{\variety}{\basefield}$ has some polynomial $\funcdef{\genpoly}{\field}{\basefield}$ that behave ``similarly'' in respect to this question.
This allows us to pull properties of $\genpoly$ to better understand $\onvarpoly$.
The perfect candidate for such $\genpoly$ is a \emph{lift} of $\onvarpoly$.
\newline
In order to use $\genpoly$ to deduce properties of $\onvarpoly$, we use the well-studied properties of polynomials in $\field$ to acheive properties of $\genpoly$, and relate these to properties of $\onvarpoly$.
More specifically, assume that $\onvarpoly$ and $\genpoly$ are measurable in respect of a collection of polynomials $\genpolyset$ (each in its domain).
Our strategy is to use $\genpoly$ to deduce properties of $\Gamma_{\genpoly}$, and then use the properties of $\Gamma_{\genpoly}$ to deduce properties of $\onvarpoly$.

Now let us describe the extra challenge.
We start by following the ideas of the regularization process we described for $\field$.
Assuming the collection is not a collection of $\variety$-relative high rank, then there must exist a polynomial in the collection that has low \emph{relative} rank, which we denote by $\genpoly^\star$
\footnote{More precisely, some linear combination of polynomials has low relative rank.}
.
Note that in relative rank, this does not necessarily mean that $\genpoly^\star$ is of low rank, but that there exists another $\variety$-equivalent polynomial that has a low rank.
Thus, even if we remove the low-rank $\variety$-equivalent polynomial and add to the collection all the polynomials that decomposed it,
we cannot require that every function that was measurable by the old collection will still be measurable by the new collection:
even the polynomial we removed is not necessarily measurable by the new collection!
\newline
To allow such regularization process to still apply, we note that while $\genpoly$ might not be measurable in respect of the new collection, a $\variety$-equivalent polynomial of $\genpoly$ \emph{is} measurable with respect of it.
Therefore, we relax the notion of being measurable to being \emph{$\variety$-measurable}.
\newline
We say a function $\genfunc$ is $\variety$-measurable in respect of $\genpolyset$ if it can be determined by the polynomials of $\genpolyset$
\emph{up to a valid $\variety$-remainder}.
We first describe an incomplete definition, then present the challenge that rises with it, and finally present its resolution.
\begin{definition}[$\variety$-measurable, Incomplete Definition]
\footnote{This incomplete definition lacks the requirement of the \emph{validity} of the $\variety$-remainder}
We say a function $\genfunc$ is $\variety$-measurable
if there exists a function $\funcdef{\Gamma}{\basefield^c}{\basefield}$
and a $\variety$-remainder, i.e a function $\funcdef{\relativeremainder{\genfunc}}{\field}{\basefield}$ with $\restrictfunc{\relativeremainder{\genfunc}}{\variety} \equiv 0$
such that:
\[
    \forall a \in \field: \genfunc(a) = \Gamma(\genpoly_1(a),...,\genpoly_c(a)) + \relativeremainder{\genfunc}(a)
\]
\end{definition}

Previous works analyzing polynomials in $\field$ were able to deduce two things from $\genfunc$ being measurable by $\genpolyset$:
that the structure of $\Gamma$ is similar to the structure of $\genfunc$, and that a random input of $\Gamma$ behave similarly to a random input of $\genfunc$.
\newline
To study polynomials in $\variety$, we wish to connect $\onvarpoly$ to $\genpoly$ (which is a lift of $\onvarpoly$).
Thus, we think of $\genfunc = \genpoly$, and require two similar things.
Firstly, we want the structure of $\Gamma$ to be similar to the structure of $\genfunc$ (in this case, $\genpoly$), which we understand as $\genfunc$ is a polynomial in $\field$.
Secondly, we want a random input of $\Gamma$ to behave similarly to a random input of $\onvarpoly$, as $\onvarpoly$ is the polynomial we wish to understand.
The latter is easily achieved using the fact high $\variety$-relative rank implies equidistribution in $\variety$.
The former, however, might be damaged by the remainder as we defined it: we can only learn the structure of $\Gamma$ using the structure of $\genfunc - \relativeremainder{\genfunc} = \Gamma(\genpoly_1,...,\genpoly_c)$.
However, the structure of $\genfunc - \relativeremainder{\genfunc}$ can be very different from the structure of $\genfunc$,
as we did not require any structure of the $\variety$-remainder $\relativeremainder{\genfunc}$.
Thus, we can not deduce the structure of $\Gamma$ via the structure of $\genfunc$ using the incomplete definition described above.

To handle this issue, we add one more requirement regarding the $\variety$-remainder,
which ensures that the structure of $\genfunc$ can be understood via the structure of $\Gamma$:
\[
    \deg(\genfunc - \relativeremainder{\genfunc}) \leq \deg(\genfunc)
\]
If the $\variety$-remainder also has this property, we say it is a \emph{valid} $\variety$-remainder for $\genfunc$.
This can be summarized by the following (complete) definition:
\begin{definition}[$\variety$-measurable]
    We say a function $\genfunc$ is $\variety$-measurable
    if there exists a function $\funcdef{\Gamma}{\basefield^c}{\basefield}$
    and a \emph{valid} $\variety$-remainder, i.e a function $\funcdef{\relativeremainder{\genfunc}}{\field}{\basefield}$
    with $\restrictfunc{\relativeremainder{\genfunc}}{\variety} \equiv 0$ and $\deg(\genfunc - \relativeremainder{\genfunc}) \leq \deg(\genfunc)$
    such that:
    \[
        \forall a \in \field: \genfunc(a) = \Gamma(\genpoly_1(a),...,\genpoly_c(a)) + \relativeremainder{\genfunc}(a)
    \]
\end{definition}
We use this new definition the following way:
Instead of using $\genfunc$ to understand $\Gamma$, we use $\genfunc - \relativeremainder{\genfunc}$ to do so.
We choose $\genfunc - \relativeremainder{\genfunc}$ as it has the same structure as $\genfunc$, but it is ``closer'' to the function $\Gamma$ as $\genfunc - \relativeremainder{\genfunc} = \Gamma(\genpoly_1,...,\genpoly_c)$
\footnote{One can think of this step as "taking the right $\variety$-equivalent" in respect of $\genpolyset$.}
.
Finally, as $\Gamma$ behaves similarly to $\onvarpoly$ for random inputs, we can use $\Gamma$ to deduce properties regarding $\onvarpoly$.
\newline
With this in hand, let us finish describing the relative-regularization process.
The requirement on the validity of the $\variety$-remainder raises a new challenge in the $\variety$-relative regularization process:
we need to somehow control the structure of the $\variety$-remainder, even though this ``error''
is substituted in $\Gamma$ each time we wish to replace a polynomial in our collection.
We address this challenge using a Lemma proved in~\cite{DBLP:journals/corr/0001L15} called the ``faithful composition lemma'',
which allows us to deduce strong properties regarding the structure of $\Gamma$ given the collection was of a high (regular) rank in the first place.
Therefore, we add to each step of the relative-regularization process a (regular) regularization, which ensures $\Gamma$ is very structured.
This strong structure of $\Gamma$ is later used to control the error and deduce it is in the form of a valid $\variety$-remainder.
For the exact details, see Theorem~\ref{theorem:regularization-in-X}.
We conclude this by informally stating our main technical theorem, which is the relative regularization process we just described:
\begin{theorem}[Relative Regularization Process, Informal, See Theorem~\ref{theorem:regularization-in-X}]
Let $\rankval, \degree \in \naturalnumbersset$ be integers that represents a requested rank and degree respectively,
and let $\genpoly_1,...,\genpoly_c$ be a collection of polynomials of degree $\leq \degree$.
Then, there is another collection $\genpoly^{\prime}_1,...,\genpoly^{\prime}_{c^\prime}$ of polynomials of degree $\leq \degree$,
such that:
\begin{enumerate}
    \item Every function that is $\variety$-measurable in respect to the first collection is also $\variety$-measurable in respect to the new collection.
    \item The new collection is of $\variety$-relative rank $\geq \rankval$.
    \item The new collection is of bounded size, i.e $c^\prime \leq C_{\rankfunc, \degree, c}$.
\end{enumerate}
\end{theorem}

\subsubsection{List Decoding in \titlevariety via \titlevariety-Relative Regularization}
In this subsection, we demonstrate how to use the relative regularization process to achieve our main theorem: analysis of the list decoding radius of $\reedmullercodeex{\basefield}{\variety}{\degree}$.

We follow the line of proof of~\cite{bhowmick2014list}, but this time, we are interested in bounding the amount of polynomials \emph{in $\variety$} around every function \emph{in $\variety$}.
More specifically, we wish to show that there is a constant number of words that are $(\normalizedcodedistance{\basefield}{\degree} - \epsilon)$-close to any fixed function in $\variety$.


%
%Let $\funcdef{\onvarfunc}{\variety}{\basefield}$ be a function.
%We apply Lemma~\ref{every-function-can-be-approximated-by-a-few-functions} with $A = \variety$, $B = \basefield$, $F = {\allpolyset{\leq \degree}{\variety}{\basefield}}$.
%We obtain a collection of a few polynomials $\onvarpolyset[3] \subset \allpolyset{\leq \degree}{\variety}{\basefield}$ defined by $\onvarpolyset[3] = (\onvarpoly[3]_1,...,\onvarpoly[3]_c)$
%that approxiamtes every polynomial of degree $\leq \degree$ in $\variety$.
%This reduces the question to only count polynomials in the radius of functions $\funcdef{\onvarfunc}{\variety}{\basefield}$ of the form:
%\[
%    \onvarfunc(x) = \Gamma (\onvarpoly[3]_1(x),...,\onvarpoly[3]_c(x))
%\]
%for some $\funcdef{\Gamma}{\basefield^c}{\basefield}$.
%
%Next, we lift every polynomial in the collection $\onvarpolyset[3]$ and get a collection of polynomials in $\field$, denoted by $\genpolyset[3]$.
%We note that $\genpolyset[3]$, when restricting each of its functions to $\variety$, measures exactly the same functions that were measurable by $\onvarpolyset[3]$.
%
%Now, we apply the $\variety$-relative regularization process to $\genpolyset[3]$.
%This yields a new collection of polynomials $\genpolyset[3]^\prime = \set{\genpoly[3]_1^{\prime},...,\genpoly[3]_{c^\prime}^{\prime}}$ that is equidistributed in $\variety$
%and captures in $\variety$ the same functions that were captured in $\variety$.
%More formally, the latter states that every function that was measurable by $\genpolyset[3]$ restricted to $\variety$, is still measurable by the new collection restricted to $\variety$.
%We note that here, we do not state the ``validity'' (the structure) of the remainder, which is an important promise given by the definition of $\variety$-measurable.
%This will play a crucial role in a second relative regularization process which we will do during the analysis.

%The strategy of the proof is to show that every polynomial in $\variety$ that is $\normalizedcodedistance{\basefield}{\degree}$-close to $\onvarfunc$ is \emph{measurable} by the collection $\genpolyset[3]^\prime$ in $\variety$.
%This is similar to the strategy of the proof of~\cite{bhowmick2014list}, with difference in the domain of the functions.
%This will bound the number of polynomials in the radius of $\onvarfunc$ by the amount of possible functions measurable by a $c^\prime$-sized collection, that is $\abs{\basefield}^{\abs{\basefield}^{c^\prime}}$.

Let $\funcdef{\onvarfunc}{\variety}{\basefield}$ be a received word.
First, we apply a lemma proved in~\cite[Corollary 3.3]{bhowmick2014list}.
The lemma shows that there is a constant-sized (depending on $\epsilon$) collection of polynomials in $\variety$, denoted by $\onvarpolyset[3]$,
such that the distance of $\onvarfunc$ to \emph{any} polynomial can be approximated by the distance of $\onvarfunc$ to some function that is measurable by $\onvarpolyset[3]$ in $\variety$.
This means that instead of bounding the number of polynomials in the radius of $\onvarfunc$, one can bound the number of polynomials in the radius of some function measurable by $\onvarpolyset[3]$.
Thereby, every polynomial-specific measurable function can be thought of as a \emph{low complexity proxy} for $\onvarfunc$ in respect to the polynomial.

Next, we lift each polynomial from $\onvarpolyset[3]$ and apply the \emph{relative regularization process}.
This yields a new collection of polynomials in $\field$ that is constant sized and randomly-behaving (in $\field$).
Denote this new collection by $\genpolyset[3]^{\prime}$
\footnote{We use the same notations as the original proof for clearannce.}
.
Thereby, the question of list decoding is reduced to the following question:
We have a specific constant-sized randomly-behaving collection of polynomials $\genpolyset[3]^\prime = \set{\genpoly[3]_1^{\prime},...,\genpoly[3]^{\prime}_{c^\prime}}$
that was constructed using the function $\onvarfunc$.
We need to bound the amount of polynomials in $\variety$ that are $(\normalizedcodedistance{\degree}{\basefield} - \epsilon / 2)$-close to be measurable by this collection in $\variety$.
Note that the randomly-behaving property was achieved using the \emph{relative rank-bias property} of $\variety$.
Additionally, we note the collection $\genpolyset[3]^\prime$ is a collection of polynomials in $\field$ which we obtained by using the \emph{lift-enabler property} of $\variety$.

From there (and similarly to the analysis in $\field$),
the strategy is to show that polynomials that are that close to being measurable by the randomly-behaving collection $\genpolyset[3]^\prime$, are in fact \emph{measurable} by it.
This will bound the number of such polynomials by the amount of possible functions that are measurable by $\genpolyset[3]^\prime$, which is constant as the collection is of constant size.


Let $\funcdef{\onvarpoly}{\variety}{\basefield}$ be a polynomial of degree $\leq \degree$, and consider a lift of it $\funcdef{\genpoly}{\field}{\basefield}$.
Consider the collection $\genpolyset[3]^\prime \cup \set{\genpoly}$.
Surely, $\genpoly$ is measurable by this collection in $\field$.
Applying $\variety$-relative-regularization to this collection yields a new collection $\genpolyset[3]^{\prime\prime}$ that is equidistributed in $\variety$, such that every $\variety$-measurable function by the old collection is $\variety$-measurable by the new collection.
By a reason we have not explained in this brief explanation, we can ensure this collection is of the form $\genpolyset[3]^{\prime\prime} = \genpolyset[3]^\prime \cup \set{\genpoly[3]_{1}^{\prime\prime},...,\genpoly[3]_{c^{\prime\prime}}^{\prime\prime}}$.

As $\genpoly$ was $\variety$-measurable by $\genpolyset[3]^\prime \cup \set{\genpoly}$ (it was even measurable), $\genpoly$ is $\variety$-measurable by the new collection $\genpolyset[3]^{\prime\prime}$:
That is, $\genpoly$ is measurable by $\genpolyset[3]^{\prime\prime}$ up to a \emph{valid} remainder, denoted by $\relativeremainder{\genpoly}$.
\newline
This means there exists $\funcdef{\Phi}{\basefield^{c^\prime + c^{\prime\prime}}}{\basefield}$ such that:
\[
    \forall a \in \field: \genpoly(a) = \Phi(\genpoly[3]^\prime_1(a),...,\genpoly[3]^\prime_{c^\prime}(a), \genpoly[3]^{\prime\prime}_1(a),...,\genpoly[3]^{\prime\prime}_{c^{\prime\prime}}(a))) + \relativeremainder{\genpoly}(a)
\]

In $\field$, the proof would follow by studying the structure of the function $\Phi$ and use it to induce that $\Phi$ does not depend on its last $c^{\prime\prime}$ variables.
This implies that $\genpoly$ is measurable by the original collection $\genpolyset[3]^{\prime}$ which concludes the proof
\footnote{Note that in $\field$ there is no remainder, so the equation above (with the last $c^{\prime\prime}$ variables as constants) implies measurability by $\genpolyset[3]^{\prime}$.}
.

More accurately, the analysis in $\field$ used the fact that substituting \emph{randomly behaving} polynomials in $\Phi$ yields a structured function
\footnote{In our notations, this structured function is $\genpoly$, which is a polynomial of degree $\leq \degree$ and thus structured}
.
This is used to show that $\Phi$ as a function by itself, with inputs from $\basefield^{c^\prime + c^{\prime\prime}}$, is a very structured function.
The strong structure of $\Phi$, with the fact that $\Phi$ (with inputs substitued to be the functions of $\genpolyset[3]^{\prime\prime}$) is close to the function $\onvarfunc$,
are then combined to deduce that $\Phi$ does not depend on its last $c^{\prime\prime}$ variables.

This paradigm can not be extended effortlessly to our case.
In $\variety$, deducing that $\Phi$ is very structured requires a one-more major step.
This is because we do \emph{not} know any correspondence in the behavior of $\Phi$ (which we want to understand) with the behavior of $\genpoly$ (which we know is structured).
We only know there is a correspondence between $\Phi$ to another function $\genpoly - \relativeremainder{\genpoly}$, which apriori we do not know is structured!

Fortunately, the relative regularization process (Theorem~\ref{theorem:regularization-in-X}) mandates that the remainder of the measurement is \emph{valid}.
That is, if $\genpoly$ was structured (a polynomial of degree $\leq \degree$), then so does $\genpoly - \relativeremainder{\genpoly}$.
This is \emph{crucial}, as it allows us to use the relation between $\Phi$ and $\genpoly - \relativeremainder{\genpoly}$ to deduce that $\Phi$ is structured,
and continue the original outline of the proof of~\cite{bhowmick2014list}.
For more details in this regard, see Theorem~\ref{thm:list-decoding-RM-in-X}.


\subsection{Organization}\label{subsec:organization}
In Section~\ref{sec:preliminaries} we present some basic notations and conventions,
and define the preliminaries we have regarding high-order Fourier analysis in $\field$: polynomials, rank and regularization.
We later generalize each component we presented in Section~\ref{sec:preliminaries} to study polynomials in $\field$ to also study polynomials in $\variety$:
in Section~\ref{sec:polynomials_in_X} we present the set of polynomials in $\variety$ and present the \emph{lift-enabler property};
in Section~\ref{sec:relative-rank-bias-property} we present the \emph{$\variety$-relative rank-bias property};
and in Section~\ref{sec:regularization-relative-to-X} we present the $\variety$-measurable notion, and our main tool, which is the \emph{$\variety$-relative regularization process}.
Next, we present two applications regarding the distance parameters of Reed-Muller codes in $\variety$:
In Section~\ref{sec:radius-of-RM-over-X} we prove the inheritance of the \emph{distance} of the code;
and in Section~\ref{sec:list-decoding-reed-muller-over-X} we prove the inheritance of the \emph{list decoding distance} of it (which is much more involved).

%\subsection{Acknowledgments}\label{subsec:acknowledgments}
%We express our gratitude to Schahar Lovett for his invaluable consultation on this work and for engaging in several insightful discussions.
%Furthermore, the first author recommends the survey~\cite{book} as an excellent resource for readers seeking an introduction to higher-order Fourier analysis.

\section{Related Works}
\label{sec:related_works}


\noindent\textbf{Diffusion-based Video Generation. }
The advancement of diffusion models \cite{rombach2022high, ramesh2022hierarchical, zheng2022entropy} has led to significant progress in video generation. Due to the scarcity of high-quality video-text datasets \cite{Blattmann2023, Blattmann2023a}, researchers have adapted existing text-to-image (T2I) models to facilitate text-to-video (T2V) generation. Notable examples include AnimateDiff \cite{Guo2023}, Align your Latents \cite{Blattmann2023a}, PYoCo \cite{ge2023preserve}, and Emu Video \cite{girdhar2023emu}. Further advancements, such as LVDM \cite{he2022latent}, VideoCrafter \cite{chen2023videocrafter1, chen2024videocrafter2}, ModelScope \cite{wang2023modelscope}, LAVIE \cite{wang2023lavie}, and VideoFactory \cite{wang2024videofactory}, have refined these approaches by fine-tuning both spatial and temporal blocks, leveraging T2I models for initialization to improve video quality.
Recently, Sora \cite{brooks2024video} and CogVideoX \cite{yang2024cogvideox} enhance video generation by introducing Transformer-based diffusion backbones \cite{Peebles2023, Ma2024, Yu2024} and utilizing 3D-VAE, unlocking the potential for realistic world simulators. Additionally, SVD \cite{Blattmann2023}, SEINE \cite{chen2023seine}, PixelDance \cite{zeng2024make} and PIA \cite{zhang2024pia} have made significant strides in image-to-video generation, achieving notable improvements in quality and flexibility.
Further, I2VGen-XL \cite{zhang2023i2vgen}, DynamicCrafter \cite{Xing2023}, and Moonshot \cite{zhang2024moonshot} incorporate additional cross-attention layers to strengthen conditional signals during generation.



\noindent\textbf{Controllable Generation.}
Controllable generation has become a central focus in both image \citep{Zhang2023,jiang2024survey, Mou2024, Zheng2023, peng2024controlnext, ye2023ip, wu2024spherediffusion, song2024moma, wu2024ifadapter} and video \citep{gong2024atomovideo, zhang2024moonshot, guo2025sparsectrl, jiang2024videobooth} generation, enabling users to direct the output through various types of control. A wide range of controllable inputs has been explored, including text descriptions, pose \citep{ma2024follow,wang2023disco,hu2024animate,xu2024magicanimate}, audio \citep{tang2023anytoany,tian2024emo,he2024co}, identity representations \citep{chefer2024still,wang2024customvideo,wu2024customcrafter}, trajectory \citep{yin2023dragnuwa,chen2024motion,li2024generative,wu2024motionbooth, namekata2024sg}.


\noindent\textbf{Text-based Camera Control.}
Text-based camera control methods use natural language descriptions to guide camera motion in video generation. AnimateDiff \cite{Guo2023} and SVD \cite{Blattmann2023} fine-tune LoRAs \cite{hu2021lora} for specific camera movements based on text input. 
Image conductor\cite{li2024image} proposed to separate different camera and object motions through camera LoRA weight and object LoRA weight to achieve more precise motion control.
In contrast, MotionMaster \cite{hu2024motionmaster} and Peekaboo \cite{jain2024peekaboo} offer training-free approaches for generating coarse-grained camera motions, though with limited precision. VideoComposer \cite{wang2024videocomposer} adjusts pixel-level motion vectors to provide finer control, but challenges remain in achieving precise camera control.

\noindent\textbf{Trajectory-based Camera Control.}
MotionCtrl \cite{Wang2024Motionctrl}, CameraCtrl \cite{He2024Cameractrl}, and Direct-a-Video \cite{yang2024direct} use camera pose as input to enhance control, while CVD \cite{kuang2024collaborative} extends CameraCtrl for multi-view generation, though still limited by motion complexity. To improve geometric consistency, Pose-guided diffusion \cite{tseng2023consistent}, CamCo \cite{Xu2024}, and CamI2V \cite{zheng2024cami2v} apply epipolar constraints for consistent viewpoints. VD3D \cite{bahmani2024vd3d} introduces a ControlNet\cite{Zhang2023}-like conditioning mechanism with spatiotemporal camera embeddings, enabling more precise control.
CamTrol \cite{hou2024training} offers a training-free approach that renders static point clouds into multi-view frames for video generation. Cavia \cite{xu2024cavia} introduces view-integrated attention mechanisms to improve viewpoint and temporal consistency, while I2VControl-Camera \cite{feng2024i2vcontrol} refines camera movement by employing point trajectories in the camera coordinate system. Despite these advancements, challenges in maintaining camera control and scene-scale consistency remain, which our method seeks to address. It is noted that 4Dim~\cite{watson2024controlling} introduces absolute scale but in  4D novel view synthesis (NVS) of scenes.



\section{Method}\label{sec:method}
\begin{figure}
    \centering
    \includegraphics[width=0.85\textwidth]{imgs/heatmap_acc.pdf}
    \caption{\textbf{Visualization of the proposed periodic Bayesian flow with mean parameter $\mu$ and accumulated accuracy parameter $c$ which corresponds to the entropy/uncertainty}. For $x = 0.3, \beta(1) = 1000$ and $\alpha_i$ defined in \cref{appd:bfn_cir}, this figure plots three colored stochastic parameter trajectories for receiver mean parameter $m$ and accumulated accuracy parameter $c$, superimposed on a log-scale heatmap of the Bayesian flow distribution $p_F(m|x,\senderacc)$ and $p_F(c|x,\senderacc)$. Note the \emph{non-monotonicity} and \emph{non-additive} property of $c$ which could inform the network the entropy of the mean parameter $m$ as a condition and the \emph{periodicity} of $m$. %\jj{Shrink the figures to save space}\hanlin{Do we need to make this figure one-column?}
    }
    \label{fig:vmbf_vis}
    \vskip -0.1in
\end{figure}
% \begin{wrapfigure}{r}{0.5\textwidth}
%     \centering
%     \includegraphics[width=0.49\textwidth]{imgs/heatmap_acc.pdf}
%     \caption{\textbf{Visualization of hyper-torus Bayesian flow based on von Mises Distribution}. For $x = 0.3, \beta(1) = 1000$ and $\alpha_i$ defined in \cref{appd:bfn_cir}, this figure plots three colored stochastic parameter trajectories for receiver mean parameter $m$ and accumulated accuracy parameter $c$, superimposed on a log-scale heatmap of the Bayesian flow distribution $p_F(m|x,\senderacc)$ and $p_F(c|x,\senderacc)$. Note the \emph{non-monotonicity} and \emph{non-additive} property of $c$. \jj{Shrink the figures to save space}}
%     \label{fig:vmbf_vis}
%     \vspace{-30pt}
% \end{wrapfigure}


In this section, we explain the detailed design of CrysBFN tackling theoretical and practical challenges. First, we describe how to derive our new formulation of Bayesian Flow Networks over hyper-torus $\mathbb{T}^{D}$ from scratch. Next, we illustrate the two key differences between \modelname and the original form of BFN: $1)$ a meticulously designed novel base distribution with different Bayesian update rules; and $2)$ different properties over the accuracy scheduling resulted from the periodicity and the new Bayesian update rules. Then, we present in detail the overall framework of \modelname over each manifold of the crystal space (\textit{i.e.} fractional coordinates, lattice vectors, atom types) respecting \textit{periodic E(3) invariance}. 

% In this section, we first demonstrate how to build Bayesian flow on hyper-torus $\mathbb{T}^{D}$ by overcoming theoretical and practical problems to provide a low-noise parameter-space approach to fractional atom coordinate generation. Next, we present how \modelname models each manifold of crystal space respecting \textit{periodic E(3) invariance}. 

\subsection{Periodic Bayesian Flow on Hyper-torus \texorpdfstring{$\mathbb{T}^{D}$}{}} 
For generative modeling of fractional coordinates in crystal, we first construct a periodic Bayesian flow on \texorpdfstring{$\mathbb{T}^{D}$}{} by designing every component of the totally new Bayesian update process which we demonstrate to be distinct from the original Bayesian flow (please see \cref{fig:non_add}). 
 %:) 
 
 The fractional atom coordinate system \citep{jiao2023crystal} inherently distributes over a hyper-torus support $\mathbb{T}^{3\times N}$. Hence, the normal distribution support on $\R$ used in the original \citep{bfn} is not suitable for this scenario. 
% The key problem of generative modeling for crystal is the periodicity of Cartesian atom coordinates $\vX$ requiring:
% \begin{equation}\label{eq:periodcity}
% p(\vA,\vL,\vX)=p(\vA,\vL,\vX+\vec{LK}),\text{where}~\vec{K}=\vec{k}\vec{1}_{1\times N},\forall\vec{k}\in\mathbb{Z}^{3\times1}
% \end{equation}
% However, there does not exist such a distribution supporting on $\R$ to model such property because the integration of such distribution over $\R$ will not be finite and equal to 1. Therefore, the normal distribution used in \citet{bfn} can not meet this condition.

To tackle this problem, the circular distribution~\citep{mardia2009directional} over the finite interval $[-\pi,\pi)$ is a natural choice as the base distribution for deriving the BFN on $\mathbb{T}^D$. 
% one natural choice is to 
% we would like to consider the circular distribution over the finite interval as the base 
% we find that circular distributions \citep{mardia2009directional} defined on a finite interval with lengths of $2\pi$ can be used as the instantiation of input distribution for the BFN on $\mathbb{T}^D$.
Specifically, circular distributions enjoy desirable periodic properties: $1)$ the integration over any interval length of $2\pi$ equals 1; $2)$ the probability distribution function is periodic with period $2\pi$.  Sharing the same intrinsic with fractional coordinates, such periodic property of circular distribution makes it suitable for the instantiation of BFN's input distribution, in parameterizing the belief towards ground truth $\x$ on $\mathbb{T}^D$. 
% \yuxuan{this is very complicated from my perspective.} \hanlin{But this property is exactly beautiful and perfectly fit into the BFN.}

\textbf{von Mises Distribution and its Bayesian Update} We choose von Mises distribution \citep{mardia2009directional} from various circular distributions as the form of input distribution, based on the appealing conjugacy property required in the derivation of the BFN framework.
% to leverage the Bayesian conjugacy property of von Mises distribution which is required by the BFN framework. 
That is, the posterior of a von Mises distribution parameterized likelihood is still in the family of von Mises distributions. The probability density function of von Mises distribution with mean direction parameter $m$ and concentration parameter $c$ (describing the entropy/uncertainty of $m$) is defined as: 
\begin{equation}
f(x|m,c)=vM(x|m,c)=\frac{\exp(c\cos(x-m))}{2\pi I_0(c)}
\end{equation}
where $I_0(c)$ is zeroth order modified Bessel function of the first kind as the normalizing constant. Given the last univariate belief parameterized by von Mises distribution with parameter $\theta_{i-1}=\{m_{i-1},\ c_{i-1}\}$ and the sample $y$ from sender distribution with unknown data sample $x$ and known accuracy $\alpha$ describing the entropy/uncertainty of $y$,  Bayesian update for the receiver is deducted as:
\begin{equation}
 h(\{m_{i-1},c_{i-1}\},y,\alpha)=\{m_i,c_i \}, \text{where}
\end{equation}
\begin{equation}\label{eq:h_m}
m_i=\text{atan2}(\alpha\sin y+c_{i-1}\sin m_{i-1}, {\alpha\cos y+c_{i-1}\cos m_{i-1}})
\end{equation}
\begin{equation}\label{eq:h_c}
c_i =\sqrt{\alpha^2+c_{i-1}^2+2\alpha c_{i-1}\cos(y-m_{i-1})}
\end{equation}
The proof of the above equations can be found in \cref{apdx:bayesian_update_function}. The atan2 function refers to  2-argument arctangent. Independently conducting  Bayesian update for each dimension, we can obtain the Bayesian update distribution by marginalizing $\y$:
\begin{equation}
p_U(\vtheta'|\vtheta,\bold{x};\alpha)=\mathbb{E}_{p_S(\bold{y}|\bold{x};\alpha)}\delta(\vtheta'-h(\vtheta,\bold{y},\alpha))=\mathbb{E}_{vM(\bold{y}|\bold{x},\alpha)}\delta(\vtheta'-h(\vtheta,\bold{y},\alpha))
\end{equation} 
\begin{figure}
    \centering
    \vskip -0.15in
    \includegraphics[width=0.95\linewidth]{imgs/non_add.pdf}
    \caption{An intuitive illustration of non-additive accuracy Bayesian update on the torus. The lengths of arrows represent the uncertainty/entropy of the belief (\emph{e.g.}~$1/\sigma^2$ for Gaussian and $c$ for von Mises). The directions of the arrows represent the believed location (\emph{e.g.}~ $\mu$ for Gaussian and $m$ for von Mises).}
    \label{fig:non_add}
    \vskip -0.15in
\end{figure}
\textbf{Non-additive Accuracy} 
The additive accuracy is a nice property held with the Gaussian-formed sender distribution of the original BFN expressed as:
\begin{align}
\label{eq:standard_id}
    \update(\parsn{}'' \mid \parsn{}, \x; \alpha_a+\alpha_b) = \E_{\update(\parsn{}' \mid \parsn{}, \x; \alpha_a)} \update(\parsn{}'' \mid \parsn{}', \x; \alpha_b)
\end{align}
Such property is mainly derived based on the standard identity of Gaussian variable:
\begin{equation}
X \sim \mathcal{N}\left(\mu_X, \sigma_X^2\right), Y \sim \mathcal{N}\left(\mu_Y, \sigma_Y^2\right) \Longrightarrow X+Y \sim \mathcal{N}\left(\mu_X+\mu_Y, \sigma_X^2+\sigma_Y^2\right)
\end{equation}
The additive accuracy property makes it feasible to derive the Bayesian flow distribution $
p_F(\boldsymbol{\theta} \mid \mathbf{x} ; i)=p_U\left(\boldsymbol{\theta} \mid \boldsymbol{\theta}_0, \mathbf{x}, \sum_{k=1}^{i} \alpha_i \right)
$ for the simulation-free training of \cref{eq:loss_n}.
It should be noted that the standard identity in \cref{eq:standard_id} does not hold in the von Mises distribution. Hence there exists an important difference between the original Bayesian flow defined on Euclidean space and the Bayesian flow of circular data on $\mathbb{T}^D$ based on von Mises distribution. With prior $\btheta = \{\bold{0},\bold{0}\}$, we could formally represent the non-additive accuracy issue as:
% The additive accuracy property implies the fact that the "confidence" for the data sample after observing a series of the noisy samples with accuracy ${\alpha_1, \cdots, \alpha_i}$ could be  as the accuracy sum  which could be  
% Here we 
% Here we emphasize the specific property of BFN based on von Mises distribution.
% Note that 
% \begin{equation}
% \update(\parsn'' \mid \parsn, \x; \alpha_a+\alpha_b) \ne \E_{\update(\parsn' \mid \parsn, \x; \alpha_a)} \update(\parsn'' \mid \parsn', \x; \alpha_b)
% \end{equation}
% \oyyw{please check whether the below equation is better}
% \yuxuan{I fill somehow confusing on what is the update distribution with $\alpha$. }
% \begin{equation}
% \update(\parsn{}'' \mid \parsn{}, \x; \alpha_a+\alpha_b) \ne \E_{\update(\parsn{}' \mid \parsn{}, \x; \alpha_a)} \update(\parsn{}'' \mid \parsn{}', \x; \alpha_b)
% \end{equation}
% We give an intuitive visualization of such difference in \cref{fig:non_add}. The untenability of this property can materialize by considering the following case: with prior $\btheta = \{\bold{0},\bold{0}\}$, check the two-step Bayesian update distribution with $\alpha_a,\alpha_b$ and one-step Bayesian update with $\alpha=\alpha_a+\alpha_b$:
\begin{align}
\label{eq:nonadd}
     &\update(c'' \mid \parsn, \x; \alpha_a+\alpha_b)  = \delta(c-\alpha_a-\alpha_b)
     \ne  \mathbb{E}_{p_U(\parsn' \mid \parsn, \x; \alpha_a)}\update(c'' \mid \parsn', \x; \alpha_b) \nonumber \\&= \mathbb{E}_{vM(\bold{y}_b|\bold{x},\alpha_a)}\mathbb{E}_{vM(\bold{y}_a|\bold{x},\alpha_b)}\delta(c-||[\alpha_a \cos\y_a+\alpha_b\cos \y_b,\alpha_a \sin\y_a+\alpha_b\sin \y_b]^T||_2)
\end{align}
A more intuitive visualization could be found in \cref{fig:non_add}. This fundamental difference between periodic Bayesian flow and that of \citet{bfn} presents both theoretical and practical challenges, which we will explain and address in the following contents.

% This makes constructing Bayesian flow based on von Mises distribution intrinsically different from previous Bayesian flows (\citet{bfn}).

% Thus, we must reformulate the framework of Bayesian flow networks  accordingly. % and do necessary reformulations of BFN. 

% \yuxuan{overall I feel this part is complicated by using the language of update distribution. I would like to suggest simply use bayesian update, to provide intuitive explantion.}\hanlin{See the illustration in \cref{fig:non_add}}

% That introduces a cascade of problems, and we investigate the following issues: $(1)$ Accuracies between sender and receiver are not synchronized and need to be differentiated. $(2)$ There is no tractable Bayesian flow distribution for a one-step sample conditioned on a given time step $i$, and naively simulating the Bayesian flow results in computational overhead. $(3)$ It is difficult to control the entropy of the Bayesian flow. $(4)$ Accuracy is no longer a function of $t$ and becomes a distribution conditioned on $t$, which can be different across dimensions.
%\jj{Edited till here}

\textbf{Entropy Conditioning} As a common practice in generative models~\citep{ddpm,flowmatching,bfn}, timestep $t$ is widely used to distinguish among generation states by feeding the timestep information into the networks. However, this paper shows that for periodic Bayesian flow, the accumulated accuracy $\vc_i$ is more effective than time-based conditioning by informing the network about the entropy and certainty of the states $\parsnt{i}$. This stems from the intrinsic non-additive accuracy which makes the receiver's accumulated accuracy $c$ not bijective function of $t$, but a distribution conditioned on accumulated accuracies $\vc_i$ instead. Therefore, the entropy parameter $\vc$ is taken logarithm and fed into the network to describe the entropy of the input corrupted structure. We verify this consideration in \cref{sec:exp_ablation}. 
% \yuxuan{implement variant. traditionally, the timestep is widely used to distinguish the different states by putting the timestep embedding into the networks. citation of FM, diffusion, BFN. However, we find that conditioned on time in periodic flow could not provide extra benefits. To further boost the performance, we introduce a simple yet effective modification term entropy conditional. This is based on that the accumulated accuracy which represents the current uncertainty or entropy could be a better indicator to distinguish different states. + Describe how you do this. }



\textbf{Reformulations of BFN}. Recall the original update function with Gaussian sender distribution, after receiving noisy samples $\y_1,\y_2,\dots,\y_i$ with accuracies $\senderacc$, the accumulated accuracies of the receiver side could be analytically obtained by the additive property and it is consistent with the sender side.
% Since observing sample $\y$ with $\alpha_i$ can not result in exact accuracy increment $\alpha_i$ for receiver, the accuracies between sender and receiver are not synchronized which need to be differentiated. 
However, as previously mentioned, this does not apply to periodic Bayesian flow, and some of the notations in original BFN~\citep{bfn} need to be adjusted accordingly. We maintain the notations of sender side's one-step accuracy $\alpha$ and added accuracy $\beta$, and alter the notation of receiver's accuracy parameter as $c$, which is needed to be simulated by cascade of Bayesian updates. We emphasize that the receiver's accumulated accuracy $c$ is no longer a function of $t$ (differently from the Gaussian case), and it becomes a distribution conditioned on received accuracies $\senderacc$ from the sender. Therefore, we represent the Bayesian flow distribution of von Mises distribution as $p_F(\btheta|\x;\alpha_1,\alpha_2,\dots,\alpha_i)$. And the original simulation-free training with Bayesian flow distribution is no longer applicable in this scenario.
% Different from previous BFNs where the accumulated accuracy $\rho$ is not explicitly modeled, the accumulated accuracy parameter $c$ (visualized in \cref{fig:vmbf_vis}) needs to be explicitly modeled by feeding it to the network to avoid information loss.
% the randomaccuracy parameter $c$ (visualized in \cref{fig:vmbf_vis}) implies that there exists information in $c$ from the sender just like $m$, meaning that $c$ also should be fed into the network to avoid information loss. 
% We ablate this consideration in  \cref{sec:exp_ablation}. 

\textbf{Fast Sampling from Equivalent Bayesian Flow Distribution} Based on the above reformulations, the Bayesian flow distribution of von Mises distribution is reframed as: 
\begin{equation}\label{eq:flow_frac}
p_F(\btheta_i|\x;\alpha_1,\alpha_2,\dots,\alpha_i)=\E_{\update(\parsnt{1} \mid \parsnt{0}, \x ; \alphat{1})}\dots\E_{\update(\parsn_{i-1} \mid \parsnt{i-2}, \x; \alphat{i-1})} \update(\parsnt{i} | \parsnt{i-1},\x;\alphat{i} )
\end{equation}
Naively sampling from \cref{eq:flow_frac} requires slow auto-regressive iterated simulation, making training unaffordable. Noticing the mathematical properties of \cref{eq:h_m,eq:h_c}, we  transform \cref{eq:flow_frac} to the equivalent form:
\begin{equation}\label{eq:cirflow_equiv}
p_F(\vec{m}_i|\x;\alpha_1,\alpha_2,\dots,\alpha_i)=\E_{vM(\y_1|\x,\alpha_1)\dots vM(\y_i|\x,\alpha_i)} \delta(\vec{m}_i-\text{atan2}(\sum_{j=1}^i \alpha_j \cos \y_j,\sum_{j=1}^i \alpha_j \sin \y_j))
\end{equation}
\begin{equation}\label{eq:cirflow_equiv2}
p_F(\vec{c}_i|\x;\alpha_1,\alpha_2,\dots,\alpha_i)=\E_{vM(\y_1|\x,\alpha_1)\dots vM(\y_i|\x,\alpha_i)}  \delta(\vec{c}_i-||[\sum_{j=1}^i \alpha_j \cos \y_j,\sum_{j=1}^i \alpha_j \sin \y_j]^T||_2)
\end{equation}
which bypasses the computation of intermediate variables and allows pure tensor operations, with negligible computational overhead.
\begin{restatable}{proposition}{cirflowequiv}
The probability density function of Bayesian flow distribution defined by \cref{eq:cirflow_equiv,eq:cirflow_equiv2} is equivalent to the original definition in \cref{eq:flow_frac}. 
\end{restatable}
\textbf{Numerical Determination of Linear Entropy Sender Accuracy Schedule} ~Original BFN designs the accuracy schedule $\beta(t)$ to make the entropy of input distribution linearly decrease. As for crystal generation task, to ensure information coherence between modalities, we choose a sender accuracy schedule $\senderacc$ that makes the receiver's belief entropy $H(t_i)=H(p_I(\cdot|\vtheta_i))=H(p_I(\cdot|\vc_i))$ linearly decrease \emph{w.r.t.} time $t_i$, given the initial and final accuracy parameter $c(0)$ and $c(1)$. Due to the intractability of \cref{eq:vm_entropy}, we first use numerical binary search in $[0,c(1)]$ to determine the receiver's $c(t_i)$ for $i=1,\dots, n$ by solving the equation $H(c(t_i))=(1-t_i)H(c(0))+tH(c(1))$. Next, with $c(t_i)$, we conduct numerical binary search for each $\alpha_i$ in $[0,c(1)]$ by solving the equations $\E_{y\sim vM(x,\alpha_i)}[\sqrt{\alpha_i^2+c_{i-1}^2+2\alpha_i c_{i-1}\cos(y-m_{i-1})}]=c(t_i)$ from $i=1$ to $i=n$ for arbitrarily selected $x\in[-\pi,\pi)$.

After tackling all those issues, we have now arrived at a new BFN architecture for effectively modeling crystals. Such BFN can also be adapted to other type of data located in hyper-torus $\mathbb{T}^{D}$.

\subsection{Equivariant Bayesian Flow for Crystal}
With the above Bayesian flow designed for generative modeling of fractional coordinate $\vF$, we are able to build equivariant Bayesian flow for each modality of crystal. In this section, we first give an overview of the general training and sampling algorithm of \modelname (visualized in \cref{fig:framework}). Then, we describe the details of the Bayesian flow of every modality. The training and sampling algorithm can be found in \cref{alg:train} and \cref{alg:sampling}.

\textbf{Overview} Operating in the parameter space $\bthetaM=\{\bthetaA,\bthetaL,\bthetaF\}$, \modelname generates high-fidelity crystals through a joint BFN sampling process on the parameter of  atom type $\bthetaA$, lattice parameter $\vec{\theta}^L=\{\bmuL,\brhoL\}$, and the parameter of fractional coordinate matrix $\bthetaF=\{\bmF,\bcF\}$. We index the $n$-steps of the generation process in a discrete manner $i$, and denote the corresponding continuous notation $t_i=i/n$ from prior parameter $\thetaM_0$ to a considerably low variance parameter $\thetaM_n$ (\emph{i.e.} large $\vrho^L,\bmF$, and centered $\bthetaA$).

At training time, \modelname samples time $i\sim U\{1,n\}$ and $\bthetaM_{i-1}$ from the Bayesian flow distribution of each modality, serving as the input to the network. The network $\net$ outputs $\net(\parsnt{i-1}^\mathcal{M},t_{i-1})=\net(\parsnt{i-1}^A,\parsnt{i-1}^F,\parsnt{i-1}^L,t_{i-1})$ and conducts gradient descents on loss function \cref{eq:loss_n} for each modality. After proper training, the sender distribution $p_S$ can be approximated by the receiver distribution $p_R$. 

At inference time, from predefined $\thetaM_0$, we conduct transitions from $\thetaM_{i-1}$ to $\thetaM_{i}$ by: $(1)$ sampling $\y_i\sim p_R(\bold{y}|\thetaM_{i-1};t_i,\alpha_i)$ according to network prediction $\predM{i-1}$; and $(2)$ performing Bayesian update $h(\thetaM_{i-1},\y^\calM_{i-1},\alpha_i)$ for each dimension. 

% Alternatively, we complete this transition using the flow-back technique by sampling 
% $\thetaM_{i}$ from Bayesian flow distribution $\flow(\btheta^M_{i}|\predM{i-1};t_{i-1})$. 

% The training objective of $\net$ is to minimize the KL divergence between sender distribution and receiver distribution for every modality as defined in \cref{eq:loss_n} which is equivalent to optimizing the negative variational lower bound $\calL^{VLB}$ as discussed in \cref{sec:preliminaries}. 

%In the following part, we will present the Bayesian flow of each modality in detail.

\textbf{Bayesian Flow of Fractional Coordinate $\vF$}~The distribution of the prior parameter $\bthetaF_0$ is defined as:
\begin{equation}\label{eq:prior_frac}
    p(\bthetaF_0) \defeq \{vM(\vm_0^F|\vec{0}_{3\times N},\vec{0}_{3\times N}),\delta(\vc_0^F-\vec{0}_{3\times N})\} = \{U(\vec{0},\vec{1}),\delta(\vc_0^F-\vec{0}_{3\times N})\}
\end{equation}
Note that this prior distribution of $\vm_0^F$ is uniform over $[\vec{0},\vec{1})$, ensuring the periodic translation invariance property in \cref{De:pi}. The training objective is minimizing the KL divergence between sender and receiver distribution (deduction can be found in \cref{appd:cir_loss}): 
%\oyyw{replace $\vF$ with $\x$?} \hanlin{notations follow Preliminary?}
\begin{align}\label{loss_frac}
\calL_F = n \E_{i \sim \ui{n}, \flow(\parsn{}^F \mid \vF ; \senderacc)} \alpha_i\frac{I_1(\alpha_i)}{I_0(\alpha_i)}(1-\cos(\vF-\predF{i-1}))
\end{align}
where $I_0(x)$ and $I_1(x)$ are the zeroth and the first order of modified Bessel functions. The transition from $\bthetaF_{i-1}$ to $\bthetaF_{i}$ is the Bayesian update distribution based on network prediction:
\begin{equation}\label{eq:transi_frac}
    p(\btheta^F_{i}|\parsnt{i-1}^\calM)=\mathbb{E}_{vM(\bold{y}|\predF{i-1},\alpha_i)}\delta(\btheta^F_{i}-h(\btheta^F_{i-1},\bold{y},\alpha_i))
\end{equation}
\begin{restatable}{proposition}{fracinv}
With $\net_{F}$ as a periodic translation equivariant function namely $\net_F(\parsnt{}^A,w(\parsnt{}^F+\vt),\parsnt{}^L,t)=w(\net_F(\parsnt{}^A,\parsnt{}^F,\parsnt{}^L,t)+\vt), \forall\vt\in\R^3$, the marginal distribution of $p(\vF_n)$ defined by \cref{eq:prior_frac,eq:transi_frac} is periodic translation invariant. 
\end{restatable}
\textbf{Bayesian Flow of Lattice Parameter \texorpdfstring{$\boldsymbol{L}$}{}}   
Noting the lattice parameter $\bm{L}$ located in Euclidean space, we set prior as the parameter of a isotropic multivariate normal distribution $\btheta^L_0\defeq\{\vmu_0^L,\vrho_0^L\}=\{\bm{0}_{3\times3},\bm{1}_{3\times3}\}$
% \begin{equation}\label{eq:lattice_prior}
% \btheta^L_0\defeq\{\vmu_0^L,\vrho_0^L\}=\{\bm{0}_{3\times3},\bm{1}_{3\times3}\}
% \end{equation}
such that the prior distribution of the Markov process on $\vmu^L$ is the Dirac distribution $\delta(\vec{\mu_0}-\vec{0})$ and $\delta(\vec{\rho_0}-\vec{1})$, 
% \begin{equation}
%     p_I^L(\boldsymbol{L}|\btheta_0^L)=\mathcal{N}(\bm{L}|\bm{0},\bm{I})
% \end{equation}
which ensures O(3)-invariance of prior distribution of $\vL$. By Eq. 77 from \citet{bfn}, the Bayesian flow distribution of the lattice parameter $\bm{L}$ is: 
\begin{align}% =p_U(\bmuL|\btheta_0^L,\bm{L},\beta(t))
p_F^L(\bmuL|\bm{L};t) &=\mathcal{N}(\bmuL|\gamma(t)\bm{L},\gamma(t)(1-\gamma(t))\bm{I}) 
\end{align}
where $\gamma(t) = 1 - \sigma_1^{2t}$ and $\sigma_1$ is the predefined hyper-parameter controlling the variance of input distribution at $t=1$ under linear entropy accuracy schedule. The variance parameter $\vrho$ does not need to be modeled and fed to the network, since it is deterministic given the accuracy schedule. After sampling $\bmuL_i$ from $p_F^L$, the training objective is defined as minimizing KL divergence between sender and receiver distribution (based on Eq. 96 in \citet{bfn}):
\begin{align}
\mathcal{L}_{L} = \frac{n}{2}\left(1-\sigma_1^{2/n}\right)\E_{i \sim \ui{n}}\E_{\flow(\bmuL_{i-1} |\vL ; t_{i-1})}  \frac{\left\|\vL -\predL{i-1}\right\|^2}{\sigma_1^{2i/n}},\label{eq:lattice_loss}
\end{align}
where the prediction term $\predL{i-1}$ is the lattice parameter part of network output. After training, the generation process is defined as the Bayesian update distribution given network prediction:
\begin{equation}\label{eq:lattice_sampling}
    p(\bmuL_{i}|\parsnt{i-1}^\calM)=\update^L(\bmuL_{i}|\predL{i-1},\bmuL_{i-1};t_{i-1})
\end{equation}
    

% The final prediction of the lattice parameter is given by $\bmuL_n = \predL{n-1}$.
% \begin{equation}\label{eq:final_lattice}
%     \bmuL_n = \predL{n-1}
% \end{equation}

\begin{restatable}{proposition}{latticeinv}\label{prop:latticeinv}
With $\net_{L}$ as  O(3)-equivariant function namely $\net_L(\parsnt{}^A,\parsnt{}^F,\vQ\parsnt{}^L,t)=\vQ\net_L(\parsnt{}^A,\parsnt{}^F,\parsnt{}^L,t),\forall\vQ^T\vQ=\vI$, the marginal distribution of $p(\bmuL_n)$ defined by \cref{eq:lattice_sampling} is O(3)-invariant. 
\end{restatable}


\textbf{Bayesian Flow of Atom Types \texorpdfstring{$\boldsymbol{A}$}{}} 
Given that atom types are discrete random variables located in a simplex $\calS^K$, the prior parameter of $\boldsymbol{A}$ is the discrete uniform distribution over the vocabulary $\parsnt{0}^A \defeq \frac{1}{K}\vec{1}_{1\times N}$. 
% \begin{align}\label{eq:disc_input_prior}
% \parsnt{0}^A \defeq \frac{1}{K}\vec{1}_{1\times N}
% \end{align}
% \begin{align}
%     (\oh{j}{K})_k \defeq \delta_{j k}, \text{where }\oh{j}{K}\in \R^{K},\oh{\vA}{KD} \defeq \left(\oh{a_1}{K},\dots,\oh{a_N}{K}\right) \in \R^{K\times N}
% \end{align}
With the notation of the projection from the class index $j$ to the length $K$ one-hot vector $ (\oh{j}{K})_k \defeq \delta_{j k}, \text{where }\oh{j}{K}\in \R^{K},\oh{\vA}{KD} \defeq \left(\oh{a_1}{K},\dots,\oh{a_N}{K}\right) \in \R^{K\times N}$, the Bayesian flow distribution of atom types $\vA$ is derived in \citet{bfn}:
\begin{align}
\flow^{A}(\parsn^A \mid \vA; t) &= \E_{\N{\y \mid \beta^A(t)\left(K \oh{\vA}{K\times N} - \vec{1}_{K\times N}\right)}{\beta^A(t) K \vec{I}_{K\times N \times N}}} \delta\left(\parsn^A - \frac{e^{\y}\parsnt{0}^A}{\sum_{k=1}^K e^{\y_k}(\parsnt{0})_{k}^A}\right).
\end{align}
where $\beta^A(t)$ is the predefined accuracy schedule for atom types. Sampling $\btheta_i^A$ from $p_F^A$ as the training signal, the training objective is the $n$-step discrete-time loss for discrete variable \citep{bfn}: 
% \oyyw{can we simplify the next equation? Such as remove $K \times N, K \times N \times N$}
% \begin{align}
% &\calL_A = n\E_{i \sim U\{1,n\},\flow^A(\parsn^A \mid \vA ; t_{i-1}),\N{\y \mid \alphat{i}\left(K \oh{\vA}{KD} - \vec{1}_{K\times N}\right)}{\alphat{i} K \vec{I}_{K\times N \times N}}} \ln \N{\y \mid \alphat{i}\left(K \oh{\vA}{K\times N} - \vec{1}_{K\times N}\right)}{\alphat{i} K \vec{I}_{K\times N \times N}}\nonumber\\
% &\qquad\qquad\qquad-\sum_{d=1}^N \ln \left(\sum_{k=1}^K \out^{(d)}(k \mid \parsn^A; t_{i-1}) \N{\ydd{d} \mid \alphat{i}\left(K\oh{k}{K}- \vec{1}_{K\times N}\right)}{\alphat{i} K \vec{I}_{K\times N \times N}}\right)\label{discdisc_t_loss_exp}
% \end{align}
\begin{align}
&\calL_A = n\E_{i \sim U\{1,n\},\flow^A(\parsn^A \mid \vA ; t_{i-1}),\N{\y \mid \alphat{i}\left(K \oh{\vA}{KD} - \vec{1}\right)}{\alphat{i} K \vec{I}}} \ln \N{\y \mid \alphat{i}\left(K \oh{\vA}{K\times N} - \vec{1}\right)}{\alphat{i} K \vec{I}}\nonumber\\
&\qquad\qquad\qquad-\sum_{d=1}^N \ln \left(\sum_{k=1}^K \out^{(d)}(k \mid \parsn^A; t_{i-1}) \N{\ydd{d} \mid \alphat{i}\left(K\oh{k}{K}- \vec{1}\right)}{\alphat{i} K \vec{I}}\right)\label{discdisc_t_loss_exp}
\end{align}
where $\vec{I}\in \R^{K\times N \times N}$ and $\vec{1}\in\R^{K\times D}$. When sampling, the transition from $\bthetaA_{i-1}$ to $\bthetaA_{i}$ is derived as:
\begin{equation}
    p(\btheta^A_{i}|\parsnt{i-1}^\calM)=\update^A(\btheta^A_{i}|\btheta^A_{i-1},\predA{i-1};t_{i-1})
\end{equation}

The detailed training and sampling algorithm could be found in \cref{alg:train} and \cref{alg:sampling}.




\section{Results}
\label{sec:results}

\begin{figure}[ht]
    \centering
\includegraphics[width=\linewidth]{figures/progs.png}
\caption{The best performing programs for each of the 3 benchmark problems as Python programs (left in each card) and the corresponding DAG representation on the right.
The DAG representation allows better visualization of the importance of different components.
The thickness of the red edges determine how important that component is. 
A black edge represents computation; when removed it is either the same as one of its subsequent edges or removing it could result in a bug.
}
\label{fig:mainresults}

\end{figure}

\subsection{Implementation details}
For all our main experiments we used an open-source LLM \emph{llama-3-8b-instruct}~\citep{dubey2024llama} served using the vLLM library~\citep{kwon2023vllm}.
However, in the supplementary, we also explore other open-source language models.
All the visual data in our benchmarks comes from satellite images, so to allow inferring semantic information from it, we use a black-box open-world foundational model for satellite images, GRAFT~\citep{mall2023graft}. 
Some experiments use ground-truth annotations from OpenStreetMaps~\citep{vargas2020osm} as an alternative to disentangle the effect of segmentation from discovery. 

We run our evolutionary method for $T=15$ generations with a population size of $M=100$. 
For all the problems, the input observation data comes from different geographical locations around the world. 
We split this data into three parts. Two-thirds of the easternmost observations are used to create a training-testing split. The remaining one-third of the data is use to evaluate reliability (out-of-distribution generalization). 
We also release this benchmark for future research in this area. 

\subsection{Benchmark for Visual Program Discovery for Scientific Applications}
\label{ssec:setup}
Given the novelty of the visual program discovery task, there exists no pre-existing benchmark.
We define a new benchmark for this task, drawing on scientifically relevant geospatial problems.
Concretely, we choose two different problems in \emph{Demography}: population density and poverty indicators, 
and to a problem in \emph{Climate Science}: for above ground biomass (AGB) estimation. 


It is important to note that for these problems, \emph{true relationships between variables of interest are actually unknown}.
As such, an LLM cannot be expected to produce a good program in a zero-shot manner, because it has never seen these relationships before.
This is in contrast to problems like VQA~\cite{suris-23} where the reasoning required to answer a question is well known and we can simply rely on the LLM's world knowledge.
In the case of scientific discovery, actual data is needed to discover the right reasoning.

In the following, we present the observation datasets, metrics, and overview of primitives. 

\subsubsection{Population Density}

\textbf{Observation Dataset}: 
The problem seeks to predict the population density by observing the satellite images of a region~\citep{metzger-22,metzger-24}. 
We obtain the population density values ($y_i$) for various locations in the USA by using ACS Community Surveys 5-year estimates~\citep{acs2024}.
Input observations ($x_i$) are sentinel-2 satellite images at a resolution of 10m~\citep{drusch2012sentinel}. For this experiment, we also use OpenStreetMaps masks~\citep{vargas2020osm} for 42 different land-use concepts (see supplementary) as part of the input.

\textbf{Metric and Primitives:} 
Population density values are aggregated at the county block group level. The predicted population densities are therefore also aggregated at the county block group level. The metric is the per-block group level average L2 error after applying a log transformation.
Along with the arithmetic, and logical primitives (see supplementary)
, we use open-vocabulary segmentation as a primitive.
The segmentation function returns a binary mask for an input concept. 



\subsubsection{Poverty Indicator}
\textbf{Observation Dataset:} For poverty estimation, we use data from SustainBench~\citep{yeh2021sustainbench}. 
The dataset contains coordinate location as input and wealth asset index as output.

\textbf{Metric and Primitives:} We use L2 error for each location as the evaluation metric. To obtain semantic land use information about a location, we first define a \emph{get\_satellite\_image} function, that returns a sentinel-2 satellite image for any location. This can be used in conjunction with the open-world satellite image recognition model to obtain semantic information about the world. 
Other than this we also include as primitives functions that return average annual temperature, precipitation, nightlight intensity, and elevation at the input location. 

\subsubsection{Aboveground Biomass}

\textbf{Observation Dataset:}  
Similar to poverty estimation, the observation variables are an input location and the output AGB estimate. We use NASA's GEDI ~\citep{dubayah2020gedi} to obtain the observation value for three US states. 
We use data from Massachusetts and Maine (North-East) as the train/test set and Washington (NorthWest) as the out-of-distibution set.

\textbf{Metric and Primitives:} We use L2 error as the metric and the same primitives as poverty estimation.



\subsection{Experimental Setup}
For the same set of training data we compare our best generated program with a set of baselines. 
\begin{enumerate}
    \item \textbf{Mean:} A naive baseline that use the mean of the training observation as the prediction.
    \item \textbf{Concept Bottleneck (CB):} Similar to \cite{koh20concept,yang2023language,oikarinen2023label}, we first extract a list of relevant features and train a linear classifier on it. 
    This method is interpretable due to the bottleneck, however it is not very expressive (see supplementary).
     

    \item \textbf{Deep models:} We use deep models such as ResNets~\cite{he2016resnet} as baseline (see supplementary for details). We use a small and large variant for each.
    
    \item \textbf{Zero-shot:} This baseline tests how good would LLMs be on their own in generating programs solely relying on prior knowledge without any observation. Since the generated programs can vary drastically, we report an average of 5 different zero-shot programs.

    \item \textbf{Random Search:} Instead of evolutionary search, this baseline relies on the stochasticity of LLMs to perform a random search. If ~\disciple~ is better at searching, it should do better than random searching for the same number of calls to an LLM.
\end{enumerate}

\subsection{Results and Discussion}



\begin{table*}
\small
\centering
      \caption{Performance of our programs on in-distribution (left) and out-of-distribution (right) observations across various problems in the proposed benchmark. This shows the reliability of programs produced by \disciple~(\best{red} is best and \sota{blue} is second best).
      } \label{tab:performance}      
      \begin{tabular}{l c c c c c c | c c c c c c} 
        \specialrule{.12em}{.1em}{.1em}       
        & \multicolumn{6}{c}{In distribution}
        & \multicolumn{6}{|c}{OOD}\\
        & \multicolumn{2}{c}{Population Density} & \multicolumn{2}{c}{Poverty} & \multicolumn{2}{c}{AGB}
        & \multicolumn{2}{|c}{Population Density} & \multicolumn{2}{c}{Poverty} & \multicolumn{2}{c}{AGB}
        \\
        & L2-Log & L1-Log & L1 & RMSE & L1 & RMSE
        & L2-Log & L1-Log & L1 & RMSE & L1 & RMSE\\
        \specialrule{.12em}{.1em}{.1em}
        Mean 
        & 0.6696 & 0.6540 & 1.613 & 1.836 & 42.15 & 50.65
        & 0.6734 & 0.6561 & 1.591 & 1.844 & 74.15 & 83.02
        \\
        
        CB 
        & 0.8298 & 0.7279 & 1.229 & \sota{1.476} & 26.33 & 33.49
        & 0.7951 & 0.7112 & 1.257 & 1.504 & 44.19 & 63.52
        \\
        
        Deep - Small  
        & 0.4431 & 0.5006 & 1.238 & 1.637 & 30.72 & 37.03
        & 0.6623 & 0.5967 & \sota{1.284} & \sota{1.654} & \sota{35.27} & \sota{53.06} 
        \\
        
        Deep - Large  
        & \sota{0.3974} & \sota{0.4843} & \sota{1.170} & 1.478 & \best{21.15} & \best{27.86}
        & \sota{0.4460} & \sota{0.5115}  & 1.344 & 1.741 & 35.41 & 70.30  
        \\
        
        Zero-shot 
        & 0.4702 & 0.5371 & 1.525 & 1.754 & 38.80	& 46.41
        & 0.7020 & 0.6412 & 1.510 & 1.773 & 55.11 &	64.32  
        \\

        Random Search 
        & 0.4353 & 0.5118 & 1.277 & 1.679 & 29.40	& 36.70
        & 0.6763 & 0.6298 & 1.418 & 1.840 & 42.32 & 52.53  
        \\
        
        \textbf{Ours} 
        & \best{0.2607} & \best{0.3778} & \best{1.077} & \best{1.314} & \sota{24.79} & \sota{32.99}
        & \best{0.3807} & \best{0.4426} & \best{1.134} & \best{1.420} & \best{31.10} & \best{42.93}
        \\
        \specialrule{.12em}{.1em}{.1em}
      \end{tabular} 
\end{table*}




We first test our programs on unseen \emph{in-domain} observations close to the regions used for training (\cref{tab:performance} (left)).
We observe that ~\disciple~outperforms all interpretable baselines.
It can even outperform a deep model in many cases, specifically on population density estimation, while being significantly more interpretable.
\disciple~ also outperforms zero-shot program inference from LLMs.
As discussed before, this is in line with the fact that \disciple is uncovering new relationships that may not be known to us, and by extension, to the LLM.
The performance of random search while better than zero-shot is significantly worse than ~\disciple. 
This shows that~\disciple~ is able to perform a significantly faster search, by reducing the meaningful search space.
Our evolutionary process effectively leverages data to perform this novel discovery.

\paragraph{Are our programs reliable?} 
If a program is reliable it should be able to generalize to other regions. 
\cref{tab:performance} (right) shows DiSciPLE to these baselines on such an out-of-distribution set.
Here our approach outperforms all baselines \emph{including deep networks}, suggesting that due to its interpretable-by-design representation, our method learns a model that can generalize better and overfit less to the in-distribution training data.

\begin{figure*}[ht]
    \centering
\includegraphics[width=\linewidth]{figures/qualitative_main.png}
\caption{Qualitative comparison of ~\disciple~ with other baselines on the tasks of population density. ~\disciple~ Can map to the true population density maps much more accurately than the baselines (Refer to the supplementary for more comparisons). The maps display population density as the base-10 log of people per square mile.
}
\label{fig:qualitative}
\end{figure*}

We also show these results qualitatively in \cref{fig:qualitative}, by comparing population density predictions of ~\disciple~ and the baselines to the true population density.
It is very clearly evident that ~\disciple~ can model the fine-grained changes in population in unseen regions significantly better than the baselines (refer to supplementary for more visualizations). 


\begin{figure}[ht]
    \centering
\includegraphics[width=\linewidth]{figures/scale_vert.png}         
\caption{Performance of \disciple~compared to deep baselines as we reduce the amount of training observation (in terms of L2 error). The Oracle (blue) uses a program learned from all observations but uses only partial observation for parameter training. ~\disciple~(orange) uses partial observation during evolution as well. While the errors get worse as we reduce the observation data, the drop is significantly less severe for ~\disciple~ compared to deep models, which tend to overfit.}
    \label{fig:scale}
\end{figure}


\paragraph{Are our programs data-efficient?} 
Our methods are only trained on a maximum of 4000 observations.
\cref{fig:scale} further shows that even when the amount of training data is reduced, our approach shows minimal degradation in performance compared to deep networks.
This suggests that while deep models can learn to generalize with a lot more data, our model does not need as much data to begin with, making it data-efficient.


\paragraph{Are our programs interpretable?} Our programs are interpretable-by-design as we can visualize the factors contributing to performance. 
Fig.~\ref{fig:mainresults} shows such programs (left in each card) for all the problems in our benchmark. 
An expert who is working with our method to figure out such programs can add/edit parts of the formula and figure out which/how much do each of these components matters. 

We perform this step of understanding the influence of individual operations by removing each operation in our program and measuring its effects on the final score. 
The DAGs on the right of each program show the program structure and the red edges show the influence of each component proportional to the width. 
This visualization can allow experts to understand which operations are important for the model.
For example, in the program for population density \cref{fig:mainresults}, we can see that semantic concepts such as ``highway'' and ``residential building'' are very important.






\paragraph{Can our method perform better than expert humans?}
Our method would only be useful in real-world scenarios if it can come up with stronger or comparable programs to human experts.  
We test this on the task of AGB, by providing an expert (a PhD student actively working on AGB) with a user interface with the same information as our method. 
The experts took about 1.5 hours to use their domain knowledge and iterate over their program for AGB estimation. However, the best program they could come up with had an L1 error of \textbf{37.65} on the in-distribution set and \textbf{53.20} on the OOD set (compared to \textbf{24.79} and \textbf{31.10} for ~\disciple). We figure this is primarily because experts need to spend more time on the problem. 
In general, experts would spend numerous days to come up with a good program, 
while our method can come up with a better program faster.

\paragraph{Extension to more indicators}
We also test ~\disciple~ on a larger suite of demographic indicators. Using SocialExplorer, we build a suite of 34 demography indicators. Refer to the supplementary for a list of these indicators. This includes demography information such as age group, education status, etc. In \cref{tab:moreindicators}, we report the average performance of our method compared to baselines on this data.
Since different indicators can have different scales, we first normalize all of them to have zero mean and unit standard deviation. 
These indicators are challenging to predict directly from satellite images, as evidenced by the deep model failing to perform significantly better than CB and mean baselines. 
As a result while ~\disciple~ performs better than all the baselines the improvements are not huge.
Nonetheless, ~\disciple~ performs better than every baseline.
This large-scale experiment shows the potential of applying ~\disciple~ to a wider range of problems. More details about these demographic indicators and individual performance on these is shown in the supplementary.

\begin{table}
\small
\centering
      \caption{Performance of ~\disciple~ compared to baselines on a larger suite of challenging 34 demographic indicators. Since the dataset is very challenging, the deep baseline regresses to mean, however with  ~\disciple~ we can still see some improvements.
      } \label{tab:moreindicators}      
      \begin{tabular}{l c c c c} 
      & \multicolumn{2}{c}{Test} & \multicolumn{2}{c}{OOD} \\
     & L1  & RMSE & L1 & RMSE \\
        \specialrule{.12em}{.1em}{.1em}    
     Mean & 0.8578 & 1.1519 & 0.8939 & 1.1948\\
     CB & 0.8249 & 1.1159 & 0.8771 & 1.1767\\
     Deep & 0.8527 & 1.1556 & 0.8942 & 1.1990\\
     \textbf{Ours} & \textbf{0.8159} & \textbf{1.1065} & \textbf{0.8750} & \textbf{1.1719}\\
        \specialrule{.12em}{.1em}{.1em}       
      \end{tabular} 
\end{table}

\subsection{Ablations}

\paragraph{How important is the role of feature-set prediction, critic, and simplification?}
Table~\ref{tab:ablationparts} measures the performance of our model on the task of population density as we successively add these components to the evolutionary algorithm. 
The addition of feature set prediction instead of a single feature helps, as it allows our method to learn expressive linear regression parameters instead of letting the LLM come up with them.
Further adding critic results in further improvement as the programs start covering nicher concepts resulting in better unseen and OOD generalization. 
Finally adding in simplification also improves the program. 
We posit that simplification removes irrelevant features preventing the LLM from focusing on them when performing crossovers.

\begin{table}
\small
\centering
      \caption{Performance of our method as we successively remove the components. Both critic and simplification lead to performance improvement for our method.
      } \label{tab:ablationparts}      
      \begin{tabular}{c c c c c c c} 
      & & & \multicolumn{2}{c}{Test} & \multicolumn{2}{c}{OOD} \\
      Set & Critic & Simpli. & L2 log & L1 log & L2 log & L1 log \\
        \specialrule{.12em}{.1em}{.1em}       
        \xmark & \xmark & \xmark & 0.3159 & 0.4296 & 0.4835 & 0.5178\\    
        \cmark & \xmark & \xmark & 0.2906 & 0.4049 & 0.4258 & 0.4826\\    
        \cmark & \cmark & \xmark & 0.2873 & 0.3984 & 0.4184 & 0.4684 \\    
        \cmark & \cmark & \cmark & \textbf{0.2607} & \textbf{0.3778} & \textbf{0.3807} & \textbf{0.4426}\\    
        \specialrule{.12em}{.1em}{.1em}       
      \end{tabular} 
\end{table}


\paragraph{How important are common sense and prior knowledge of LLMs?}
The two major advantages an LLM provides over traditional tree-search are: 1) better crossover and mutation as LLMs can understand the meaning of the primitives. 2) use of prior knowledge for better-guided search. 
Therefore we remove these two sources of information and test how well can our method perform. 
To remove the understanding of functions we rename them with meaningless terms and remove the descriptions.
To remove the context of the problem we remove the objective prompt.
\cref{tab:ablationcommonsense}, show the performance of our method on density estimation after removing each of these prompts. 
Without common sense, the search cannot even progress away from the initial random programs, resulting in worse-than-mean results (L1 error of 0.84 vs 0.26 for ~\disciple).
This suggests that symbolic regression models, that have no understanding of open-world primitives, would struggle to search.
If we just remove the context of the problem, the model does slightly better and can obtain results better than the mean and zero-shot programs (L1 error of 0.45). This suggests that while the search is moving in the objective's direction, it is slow. 


\begin{table}
\small
\centering
      \caption{Perfomance of our method when removing the context of the problem (objective prompt from the evolution, and when renaming and not describing the primitive functions to the LLM.
      We see significant drops in performance in both cases, suggesting that both common sense and prior knowledge of LLM are important to perform efficient evolutionary search.
      )} \label{tab:ablationcommonsense}      
      \begin{tabular}{c c c} 
      Method & L1 log & L2 log \\
        \specialrule{.12em}{.1em}{.1em}       
        No common-sense & 0.8401 & 0.7186\\    
        No problem context & 0.4498 & 0.5140\\    
        \disciple~full & \textbf{0.2607} & \textbf{0.3778}\\    
        \specialrule{.12em}{.1em}{.1em}       
      \end{tabular} 
\end{table}




\section{Conclusion}
In this work, we propose a simple yet effective approach, called SMILE, for graph few-shot learning with fewer tasks. Specifically, we introduce a novel dual-level mixup strategy, including within-task and across-task mixup, for enriching the diversity of nodes within each task and the diversity of tasks. Also, we incorporate the degree-based prior information to learn expressive node embeddings. Theoretically, we prove that SMILE effectively enhances the model's generalization performance. Empirically, we conduct extensive experiments on multiple benchmarks and the results suggest that SMILE significantly outperforms other baselines, including both in-domain and cross-domain few-shot settings.
\section{Acknowledgements}

% \clearpage
{
\small
\bibliographystyle{ieeenat_fullname}
\bibliography{indest}
}
\clearpage
\appendix


\end{document}
