
\documentclass[10pt,twocolumn,letterpaper]{article}
\usepackage[pagenumbers]{cvpr}
\newcommand{\CG}{\mathcal{G}\xspace}
\newcommand{\CV}{\mathcal{V}\xspace}
\newcommand{\CE}{\mathcal{E}\xspace}
\newcommand{\CA}{\mathcal{A}\xspace}
\newcommand{\CF}{\mathcal{F}\xspace}
\newcommand{\CR}{\mathcal{R}\xspace}
\newcommand{\CB}{\mathcal{B}\xspace}
\newcommand{\CX}{\mathcal{X}\xspace}
\newcommand{\CK}{\mathcal{K}\xspace}
\newcommand{\CM}{\mathcal{M}\xspace}
\newcommand{\CC}{\mathcal{C}\xspace}
\newcommand{\CL}{\mathcal{L}\xspace}
\newcommand{\CI}{\mathcal{I}\xspace}
\newcommand{\CQ}{\mathcal{Q}\xspace}
\newcommand{\CO}{\mathcal{O}\xspace}
\newcommand{\CP}{\mathcal{P}\xspace}
\newcommand{\CS}{\mathcal{S}\xspace}
\newcommand{\CT}{\mathcal{T}\xspace}
\newcommand{\CJ}{\mathcal{J}\xspace}
\usepackage[para]{footmisc}
\usepackage{subfig}
% \usepackage{subcaption}
% \usepackage{array}
% \usepackage{colortbl}



%%%%% NEW MATH DEFINITIONS %%%%%

% \usepackage{amsmath,amsfonts,bm}
\usepackage{amsmath,amsfonts}

\usepackage{pifont}


\newcommand{\R}{\mathbb{R}}


\def\va{{\mathbf{a}}}
\def\vg{{\mathbf{g}}}

% Sets
\def\sR{\mathbb{R}}
\def\sC{\mathbb{C}}
\def\sZ{\mathbb{Z}}
\def\sN{\mathbb{N}}
\def\sQ{\mathbb{Q}}

\def\sS{\mathcal{S}}



% Vectors
\def\vzero{{\mathbf{0}}}
\def\vone{{\mathbf{1}}}
\def\vmu{{\mathbf{\mu}}}
\def\vtheta{{\mathbf{\theta}}}
\def\va{{\mathbf{a}}}
\def\vb{{\mathbf{b}}}
\def\vc{{\mathbf{c}}}
\def\vd{{\mathbf{d}}}
\def\ve{{\mathbf{e}}}
\def\vf{{\mathbf{f}}}
\def\vg{{\mathbf{g}}}
\def\vh{{\mathbf{h}}}
\def\vi{{\mathbf{i}}}
\def\vj{{\mathbf{j}}}
\def\vk{{\mathbf{k}}}
\def\vl{{\mathbf{l}}}
\def\vm{{\mathbf{m}}}
\def\vn{{\mathbf{n}}}
\def\vo{{\mathbf{o}}}
\def\vp{{\mathbf{p}}}
\def\vq{{\mathbf{q}}}
\def\vr{{\mathbf{r}}}
\def\vs{{\mathbf{s}}}
\def\vt{{\mathbf{t}}}
\def\vu{{\mathbf{u}}}
\def\vv{{\mathbf{v}}}
\def\vw{{\mathbf{w}}}
\def\vx{{\mathbf{x}}}
\def\vy{{\mathbf{y}}}
\def\vz{{\mathbf{z}}}
\def\vzeta{{\mathbf{\zeta}}}

% Matrix
\def\mA{{\mathbf{A}}}
\def\mB{{\mathbf{B}}}
\def\mC{{\mathbf{C}}}
\def\mD{{\mathbf{D}}}
\def\mE{{\mathbf{E}}}
\def\mF{{\mathbf{F}}}
\def\mG{{\mathbf{G}}}
\def\mH{{\mathbf{H}}}
\def\mI{{\mathbf{I}}}
\def\mJ{{\mathbf{J}}}
\def\mK{{\mathbf{K}}}
\def\mL{{\mathbf{L}}}
\def\mM{{\mathbf{M}}}
\def\mN{{\mathbf{N}}}
\def\mO{{\mathbf{O}}}
\def\mP{{\mathbf{P}}}
\def\mQ{{\mathbf{Q}}}
\def\mR{{\mathbf{R}}}
\def\mS{{\mathbf{S}}}
\def\mT{{\mathbf{T}}}
\def\mU{{\mathbf{U}}}
\def\mV{{\mathbf{V}}}
\def\mW{{\mathbf{W}}}
\def\mX{{\mathbf{X}}}
\def\mY{{\mathbf{Y}}}
\def\mZ{{\mathbf{Z}}}
\def\mBeta{{\mathbf{\beta}}}
\def\mPhi{{\mathbf{\Phi}}}
\def\mLambda{{\mathbf{\Lambda}}}
\def\mSigma{{\mathbf{\Sigma}}}


% Expectation
% \def\eE{\mathop{\mathbb{E}}\limits}
\def\eE{\mathbb{E}}

% Probability
\def\pP{\mathbb{P}}

% Tilde
\def\tf{\tilde{f}}
\def\tS{\tilde{S}}
\def\wtF{\widetilde{\mathcal{F}}}
\def\whR{\widehat{R}}
\def\tvx{\tilde{\mathbf{x}}}
\def\ty{\tilde{y}}


\def\defeq{\overset{\textup{def}}{=}}
% \def\defeq{\overset{.}{=}}
\def\defone{\overset{\text{\ding{172}}}{=}}
\def\deftwo{\overset{\text{\ding{173}}}{=}}
\def\leqone{\overset{\text{\ding{172}}}{\leq}}
\def\leqtwo{\overset{\text{\ding{173}}}{\leq}}
\def\leqthree{\overset{\text{\ding{174}}}{\leq}}
\def\leqfour{\overset{\text{\ding{175}}}{\leq}}
\def\eqone{\overset{\text{\ding{172}}}{=}}
\def\eqtwo{\overset{\text{\ding{173}}}{=}}
\def\eqthree{\overset{\text{\ding{174}}}{=}}
\def\eqfour{\overset{\text{\ding{175}}}{=}}
\def\geqfive{\overset{\text{\ding{176}}}{\geq}}

\definecolor{cvprblue}{rgb}{0.21,0.49,0.74}
\usepackage[pagebackref,breaklinks,colorlinks,allcolors=cvprblue]{hyperref}


\usepackage{hyperref}
\usepackage{graphicx}
\usepackage{caption}
\usepackage{booktabs} 
\usepackage{multirow}
\usepackage{pifont}%
\newcommand{\cmark}{\ding{51}}%
\newcommand{\xmark}{\ding{55}}%
% \usepackage{minted}
\usepackage{sidecap}

\usepackage[linesnumbered,ruled,vlined]{algorithm2e}
\usepackage{xcolor}

\def\paperID{2239} %
\def\confName{CVPR}
\def\confYear{2025}

\definecolor{mydarkblue}{RGB}{0,0,160} %
\newcommand\mycommfont[1]{\footnotesize\ttfamily\textcolor{mydarkblue}{#1}}
\SetCommentSty{mycommfont}
\SetKwInOut{Parameter}{Hyperparams}

\title{DiSciPLE: Learning Interpretable Programs for Scientific Visual Discovery}


\author{
Utkarsh Mall$^{1}$ \hspace{0.05cm} Cheng Perng Phoo$^{2}$  \hspace{0.05cm} Mia Chiquier$^{1}$ \hspace{0.05cm}
Bharath Hariharan$^{2}$ \hspace{0.05cm}  Kavita Bala$^{2}$ \hspace{0.05cm} Carl Vondrick$^{1}$\\
         $^{1}$Columbia University \qquad
        $^{2}$ Cornell University\\
        \small{\enspace Correspondence: \tt{um2171@columbia.edu}
}\\
\href{https://disciple.cs.columbia.edu/}{disciple.cs.columbia.edu}
}


\newcommand{\fix}{\marginpar{FIX}}
\newcommand{\new}{\marginpar{NEW}}

\newcommand{\best}[1]{\textbf{\textcolor{red}{#1}}}
\newcommand{\sota}[1]{\emph{\textcolor{blue}{#1}}}
\newcommand{\second}[1]{\emph{\textcolor{blue}{#1}}}

\def\disciple{DiSciPLE}

\begin{document}

\maketitle
\begin{abstract}


Visual data is used in numerous different scientific workflows ranging from remote sensing to ecology. As the amount of observation data increases, the challenge is not just to make accurate predictions but also to understand the underlying mechanisms for those predictions. 
Good interpretation is important in scientific workflows, as it allows for better decision-making by providing insights into the data. 
This paper introduces an automatic way of obtaining such interpretable-by-design models, by learning programs that interleave neural networks. We propose \disciple~(Discovering Scientific Programs using LLMs and Evolution) an evolutionary algorithm that leverages common sense and prior knowledge of large language models (LLMs) to create Python programs explaining visual data. Additionally, we propose two improvements: a program critic and a program simplifier to improve our method further to synthesize good programs. On three different real-world problems, \disciple~learns state-of-the-art programs on novel tasks with no prior literature. For example, we can learn programs with 35\% lower error than the closest non-interpretable baseline for population density estimation. The supplementary material can be found at: \href{https://disciple.cs.columbia.edu/pdf/supplementary.pdf}{https://disciple.cs.columbia.edu/pdf/supplementary.pdf}




\end{abstract}


In recent years, the growing use of deep learning across various fields~\cite{yang2021intelligent, macas2022survey, grigorescu2020survey} has highlighted the need for efficient, safe, and private deployment methods, driving the adoption of edge computing. By bringing computation closer to the data source, edge computing reduces latency, saves bandwidth, and enhances privacy and security. However, deploying deep neural networks (DNNs) on edge devices poses challenges due to limited computational resources and energy constraints~\cite{alvear2023edge}.

To tackle these challenges, researchers have focused on improving the accuracy and efficiency of DNNs for edge applications. Traditional approaches, such as manual model optimization~\cite{Marchisio_2018IJCNN_PruNet, vadera2021methods, Hanif_2022MICPRO_EfficientEmbeddedDL, chen2021quantization, matsubara2020head}, often fail to balance multiple objectives effectively. The emergence of Hardware-aware Neural Architecture Search (HW-NAS) has automated the search for optimal architectures, improving model efficiency and accuracy in resource-constrained settings~\cite{benmeziane2021comprehensive}.

HW-NAS uses machine learning to explore architecture spaces and identify designs that balance performance and resource consumption, accelerating edge DNN deployment~\cite{zhang2020fast}. Unlike traditional NAS, which prioritizes accuracy~\cite{he2021automl, elsken2019neural}, HW-NAS employs multi-objective optimization to enhance both accuracy and efficiency (e.g., latency, size)\cite{wistuba2019survey, Marchisio_2020ICCAD_NASCaps, Prabakaran_2021JIOT_BioNetExplorer}, generating Pareto-optimal solutions\cite{Kaisa1999MultiObjective}.



\subsection{Target Research Problem and Associated Challenges}

While advancements in edge DNN design have improved accuracy and computational efficiency, critical performance metrics like fairness, robustness, and generalization remain underexplored~\cite{sheng2022larger}. Fairness ensures equitable performance across diverse user groups, robustness measures reliability under varying conditions (e.g., lighting, weather, visibility)\cite{drenkow2021systematic, porrello2020robust}, and generalization evaluates performance on unseen data\cite{zhou2022domain}.

Enhancing these metrics is essential for edge DNNs, particularly in safety-critical applications like medical diagnostics~\cite{esteva2021deep}. However, achieving this faces challenges such as (1) ensuring data quality and balance~\cite{mehrabi2021survey}, (2) addressing insufficient data diversity~\cite{recht2019imagenet}, and (3) overcoming the computational limitations of edge devices~\cite{ibrahim2022robustness, feuerriegel2020fair, hickey2021fairness}.

Most existing work addresses these challenges by improving data quality~\cite{pitoura2020social} or training procedures~\cite{jain2024fairness}, with limited focus on architecture design. Studies targeting architecture often prioritize a single metric, such as fairness~\cite{sheng2022larger}. Comprehensive approaches that consider fairness, robustness, and generalization together are scarce. To fill this gap, we aim to propose a design methodology ensuring edge DNNs are not only efficient and high-performing but also fair and robust.



\subsection{Analysis: Fairness, Robustness, Generalization in Edge DNNs} \label{intro:pre-analysis}
To highlight the limitations of edge DNN designs in addressing fairness, robustness, and generalization, we evaluated 12 state-of-the-art (SOTA) edge DNNs on person classification (binary classification of images with or without a person) using a COCO dataset subset~\cite{lin2014microsoft}. We measured overfitting, sensitivity to light, and accuracy across different skin tones using the FACET dataset~\cite{gustafson2023facet}.

Figure~\ref{fig:sota_rob} presents robustness and generalization results, revealing notable gaps: validation and test accuracies differ significantly, and accuracies under varying lighting conditions drop by up to 15\%. This indicates potential overfitting and poor robustness to light changes in SOTA models. Figure~\ref{fig:sota_fair} shows a concerning fairness issue: average accuracy declines from 85.0\% for the lightest skin tone category to 70.9\% for the darkest.

Further analysis (Figure~\ref{fig:sota_graph}) demonstrates the influence of model architecture and size on fairness. Models with similar sizes but different architectures exhibit varying fairness scores, indicating that fairness is highly architecture-dependent. Moreover, increasing model size tends to enhance test accuracy but worsens fairness disparities, as larger channel sizes (depth) and more feature representations (width) amplify biases. These findings reveal the need for novel scaling strategies that explicitly target fairness and robustness without solely relying on traditional size scaling techniques.

\begin{figure}
\centering
\begin{subfigure}[b]{0.48\linewidth}
\includegraphics[width=1.\linewidth]{figures/sota_rob-gen.pdf}
    \caption{Overfitting and Sensitivity to light conditions of edge SOTA DNNs.}
    \label{fig:sota_rob}
    \end{subfigure}
%\hspace{.3cm}
\hfill
\begin{subfigure}[b]{0.48\linewidth}
\includegraphics[width=1.\linewidth]{figures/res_plot_skintone_.pdf}
    \caption{Average Accuracy of SOTA edge DNNs across 10 levels of skin tones (from lightest=1 to darkest=10) on FACET~\cite{gustafson2023facet}. }
    \label{fig:sota_fair}
\end{subfigure} 
    \caption{Evaluation of Fairness, Robustness, and Generalization of SOTA edge DNNs on FACET.}
    \label{fig:sota}
    %\vspace{-10pt}
\end{figure}





%\textit{These findings highlight the unintended consequence of current edge DNN designs: a significant bias that disadvantages individuals with darker skin tones and exhibits poor robustness and generalization capabilities.} This consequence not only raises serious ethical and reliability concerns about edge DNNs performance, but also underscores the need for a shift in the design paradigm towards creating edge DNNs that are high-performing, inclusive, and robust.


%models sharing the same architecture shown in 
%In Figure~\ref{fig:sota_graph}, we illustrate the impact of model architecture and size on its skin fairness. \textit{(1) The results reveal that model architecture and size indeed influence model fairness score}. Models with different architectures but similar sizes exhibit varied skin fairness scores. Moreover, \textit{ (2) we observe that increasing model size tends to enhance test accuracy, but leads to a reduction in the model`s skin fairness despite the benefits of larger channel sizes (depth) and more features (width)}. Observations (1) and (2) underscore the need for different model scaling approaches that can improve performance without compromising fairness in edge DNNs.

\begin{figure}
    \centering
    \includegraphics[width=\linewidth]{figures/sota_fair_ppt2.pdf}
    %\vspace{-8pt}
    \caption{Test Accuracy vs. Skin Fairness of SOTA edge DNNs: Models sharing the same architecture are connected by straight lines. The Pareto front is illustrated with a dashed line. }
    \label{fig:sota_graph}
    %\vspace{-10pt}
\end{figure}

\subsection{Contributions}

Our key contributions (listed below and summarized in Figure~\ref{fig:MoENAS_overview}), enable the design of fair, robust, and general edge DNNs.
%
%\vspace{-8pt}
%
\begin{enumerate}[leftmargin=*]
%
    \item \textbf{Dynamic Feature Extraction via Model Scaling}: We propose a scaling approach that varies the number of feature extractors dynamically (inspired by Mixture of Experts (MoE) and Switch layer architectures~\cite{fedus2022switch}), enabling adaptive feature extraction tailored to specific inputs. This flexibility indirectly benefits fairness and robustness by allowing more efficient and context-aware use of resources.
%\vspace{-5pt}    
    \item \textbf{HW-NAS with Fairness and Robustness Optimization}: We propose a HW-NAS method with a search space over switching architectures, varying the number of experts per block. Using machine-learning-based performance predictors, the search strategy identifies architectures optimized for accuracy, fairness, robustness, and model size.
%\vspace{-5pt}  
    \item \textbf{Expert Pruning for Efficiency}: We introduce an expert pruning method that helps reduce the sizes of models discovered by the search method by iteratively pruning the least used experts. This approach aims to enhance the efficiency of the resulting models without sacrificing performance.
\end{enumerate}
%\vspace{-5pt}  
%
\begin{figure*}[ht]
    \centering \includegraphics[width=.9\linewidth]{figures/BmoENAS_introver_1.pdf}
    \caption{Summary of MoENAS contributions: (1) Replace the FFN layer with a Switch FFN layer, (2) Search within the expert mixing space for optimal architectures (according to accuracy, fairness, robustness, generalization). (3) Pruning based on expert importance for better efficiency. }
    \label{fig:MoENAS_overview}
    %\vspace{-10pt}
\end{figure*}

%The rest of this paper will be structured as follows: Section 2 discusses the current state of the art in HW-NAS, MoE, fairness, and DNN robustness. Section 3 details our methodology and key contributions. Section 4 compares our model results with SOTA edge DNNs, followed by an ablation study and conclusion.



\subsection{Plasticity in Neural Networks}
In recent years, various methods have been proposed to address plasticity loss.
Several works have focused on maintaining active units \cite{abbas2023loss, elsayed2024addressing} or re-initializing dead units \cite{sokar2023dormant, dohare2024loss}.
Other studies have explored limiting deviations from the initial statistics of model parameters \cite{kumar2023maintaining, lewandowski2023curvature, elsayed2024weight}.
Additionally, some methods rely on architectural modifications \cite{nikishin2024deep, lee2024slow, lewandowski2024plastic}.  
Plasticity loss also occurs in the reinforcement learning due to its inherent non-stationary. \citet{nikishin2022primacy} proposed resetting the model, while \citet{asadi2024resetting} suggested resetting the optimizer state. 

As noted by \citet{berariu2021study}, loss of plasticity can be divided into two distinct aspects: a decreased ability of networks to minimize training loss on new data (trainability) and a decreased ability to generalize to unseen data (generalizability).
While most previous works focused on trainability, \citet{lee2024slow} addressed generalizability loss.
They demonstrated that plasticity loss also occurs under a stationary distribution, as in a warm-start learning scenario where the model is pretrained on a subset of the training data and then fine-tuned on the full dataset.

Most existing studies have focused on only one of the following challenges: trainability, generalizability, or reinforcement learning.
However, in this study, we validate our AID method across all three aspects, demonstrating its effectiveness in each scenario.



\subsection{Activation Function}
Our AID method is a stochastic approach similar to Dropout while also functioning as an activation function.
Therefore, we aim to discuss previously proposed probabilistic activation functions.
Although the field of probabilistic activation functions has not seen extensive research, two noteworthy studies exist.
The first is the Randomized ReLU (RReLU) function, introduced in the Kaggle NDSB Competition \cite{xu2015empirical}.
The original ReLU function maps all negative values to zero, whereas RReLU maps negative values linearly based on a random slope.
During testing, negative values are mapped using the mean of the slope distribution.
Their experimental results suggest that RReLU effectively prevents overfitting.
Another example of a probabilistic activation function is DropReLU \cite{liang2021drop}.
DropReLU randomly determines whether a node's activation is processed through a ReLU function or a linear function.
The authors claim that DropReLU improves the generalization performance of neural networks.
The fundamental distinction between these probabilistic activation functions and our method lies in the generality of our approach.
Unlike simple probabilistic activation functions, our method encompasses techniques such as Dropout and ReLU, providing a more comprehensive framework.

Another related approach involves activation functions designed to address plasticity loss.
\citep{abbas2023loss} proposed the Concatenated Rectified Linear Units (CReLU), which concatenates the outputs of the standard ReLU applied to the input and its negation.
This structure prevents the occurrence of dead units, thereby improving plasticity.
Additionally, trainable activation functions have also been shown to effectively mitigate plasticity loss in reinforcement learning \citep{delfosseadaptive}.
Specifically, they introduced a trainable rational activation function that prevents value overfitting and overestimation in reinforcement learning.



\begin{figure*}[ht!]
    \centering
    \includegraphics[width=0.3\textwidth]{figures/sources/mainnet_pls_acc.pdf}
    \includegraphics[width=0.3\textwidth]{figures/sources/subnet_pls_acc.pdf}
    \includegraphics[width=0.3\textwidth]{figures/sources/warm_start_dropout.pdf}
    \caption{\textbf{Left.} Random label MNIST experiment using an 8-layer MLP. Higher dropout probabilities result in significant trainability loss. 
    \textbf{Middle.} Accuracy of the subnetworks trained on random target. Each subnetworks are sampled from original network after each epoch. Subnetworks of the Dropout also experience trainability loss. \textbf{Right.} Warm-start scenario of Resnet-18 model with CIFAR100 dataset. Dropout improves generalization performance; however, the reduction in accuracy compared to the cold-start scenario is nearly identical to that of the vanilla model.}
    \label{exp_dropout}
\end{figure*}



% \begin{figure}
%     \centering
%     \includegraphics[width=0.5\linewidth]{Move_teaser.pdf}
%     \caption{Comparison of different dynamic compute approaches. length of arrow indicates residual transformation per token while width indicates velocity of transformation.}
%     \label{fig:enter-label}
% \end{figure}

\section{Method}
\label{sec:method}
Residual connections play a crucial role in shaping token representations, yet their dynamics remain underexplored in the context of efficient decoding. In this work, we delve deeper into transformer residual dynamics and investigate how modulating residual transformation velocity can improve inference efficiency in token-level processing, optimizing both dense and sparse MoE transformers.


\subsection{Residual Dynamics and Motivation for Multi-rate Residuals} \label{sec:motivation}

To analyze how hidden representations evolve across different layers of a transformer architecture, it's crucial to consider the effect of residual connections. Each transformer decoder layer typically has residual connections across attention and MLP submodules. As the residual stream $h_i$ traverses from interval $E_j$ to $E_{j+1}$, it undergoes a residual transformation given by:  
% \begin{equation}
% \label{eq:slow_residual_transformation}
% H_{E_{j+1}} = H_{E_j} \prod_{i=E_j}^{E_{j+1}} \left( I + \mathcal{A}_i \right) \left( I + \mathcal{M}_i \right) \quad \text{where} \quad \mathcal{A}_i = f(c_i, h_{i}), \mathcal{M}_i = g(h_i)
% \end{equation}

\begin{equation} \label{eq:slow_residual_transformation}
h_{E_{j+1}} = h_{E_j} + \sum_{i=E_j}^{E_{j+1}-1} \left( \mathcal{A}_i(h_i) + \mathcal{M}_i(h_i + \mathcal{A}_i(h_i)) \right) \quad \text{where} \quad \mathcal{A}_i = f(c_i, h_{i}), \mathcal{M}_i = g(h_i). 
\end{equation}

Here, \( \mathcal{A}_i \) denotes the non-linear transformation introduced by the multi-head attention mechanism at layer \( i \), while \( \mathcal{M}_i \) corresponds to the non-linear transformation of the MLP block at the same layer. These transformations depend on the input residual stream \( h_i \) and, in the case of \( \mathcal{A}_i \), the previous contextual representation \( c_i \).\footnote{Normalization layers are typically applied in practice but are omitted here for simplicity of the argument.}


% For easy tokens, the magnitude and direction of this delta transformation become progressively smaller with each successive layer as shown in \cref{fig:delta_transformation}. Consequently, it is feasible to predict these tokens after only a few residual connections, whereas harder tokens necessitate more extensive processing through additional layers.

\begin{figure}[ht]
    \centering
    \begin{subfigure}{0.48\textwidth}
        \centering
        \includegraphics[width=\textwidth]{sections/figures/residual_change.pdf}
        \caption{}
        \label{fig:residual_change}
    \end{subfigure}%
    \hfill
    \begin{subfigure}{0.48\textwidth}
        \centering
        \includegraphics[width=\textwidth]{sections/figures/alignment_wrt_dedicated_model.pdf}
        \caption{}
    \label{fig:alignment_wrt_dedicated_model}
    \end{subfigure}
    \caption{(a) As residual streams propagate through the model, the directional shifts in the residuals become progressively smaller. (b) A dedicated model with $k$ layers achieves a faster rate of change in residual streams and higher alignment than base model leveraging early exit mechanisms at layer $k$.}
    \label{fig}
\end{figure}


To examine whether residual transformations can be accelerated across layers, we conducted experiments using a diverse set of prompts on a pre-trained Phi3 model~\cite{phi3_report}. As illustrated in \cref{fig:residual_change}, we measured the directional shift in residual states as \( 1 - \mathcal{C}(h_{i-1}, h_i) \), where \(\mathcal{C}\) denotes normalized cosine similarity. This shift is notably higher in the initial layers, gradually decreasing in subsequent layers. This behavior allows traditional early exit approaches to effectively accelerate decoding by enabling earlier exits for simpler tokens. However, these approaches typically rely on a distance-based approximation, where the full residual transformation of the model is approximated by the residual transformations of the initial layers. To gain deeper insights into the distance versus velocity aspects of residual transformation, we conducted a comparative study. Specifically, we trained an early exit head at layer $k$ of the Phi3 model, which consists of 32 layers, restricting the distance traveled by each token. To accelerate the residual transformation relative to number of layers, we trained a smaller model consisting of only $k$ layers, while keeping all other hyperparameters consistent. We then compared the next-token prediction accuracy of the early exit head of the base model with that of the smaller model. To ensure an equal number of trainable parameters, we inserted low-rank adapters into the smaller model and trained only these adapters, whereas, in the distance-based approach, we trained solely the early exit head. In addition, to accelerate the residual transformation in smaller model, we distilled the residual streams from the larger model by incorporating a distillation loss ~\cite{sanh2019distilbert} between the residual state at layer \(i\) of the smaller model and the residual state at layer \(4 \times i\) of the larger model. As shown in ~\cref{fig:alignment_wrt_dedicated_model} the smaller model demonstrates a significantly faster rate of change in residual streams, leading to higher next token prediction accuracy after $k$ layers compared to the base model that employs traditional early exit mechanisms after $k$ layers \cite{schuster2022confident, chen2023eellm, varshney-etal-2024-investigating}. This experimental setup, which modifies only the rate of change in residual streams while keeping other factors constant, suggests that dense transformers, trained with a fixed number of layers, may inherently possess a slow residual transformation bias.

This observation raises an intriguing question: if the rate of change in residual streams could be accelerated relative to the number of layers, is it possible to facilitate earlier alignment for a greater proportion of tokens? Earlier alignment would be beneficial to not only facilitate dynamic computation but also for generating speculative tokens efficiently with high acceptance rates in speculative decoding setups ~\cite{leviathan2023fast, chen2023accelerating}. 

%thereby enhancing the efficiency of early exiting? 
 % This bias likely constrains the effectiveness of early exiting, particularly for easier tokens. By addressing this limitation through accelerated residual transformations, we hypothesize that it is possible to substantially improve the efficiency and accuracy of early exit strategies in transformer models.

\subsection{Multi-Rate Residual Transformation} \label{m2r2_method}

To address the slow residual transformation bias described in ~\cref{sec:motivation}, we introduce \textit{accelerated residual streams} that operate at rate $R$ relative to original slow residual stream. We pair slow residual stream, $h$ with an accelerated residual stream, $p$, which has an intrinsic bias towards earlier alignment. Relative to ~\cref{eq:slow_residual_transformation}, accelerated residual transformation from interval $E_j$ to $E_{j+1}$ can be represented as: 

% \begin{equation}
% \label{eq:fast_residual_transformation}
% P_{E_{j+1}} = P_{E_j} \prod_{i=E_j}^{E_{j+1}} \left( I + \hat{\mathcal{A}_i} \right) \left( I + \hat{\mathcal{M}_i} \right) \quad \text{where} \quad \hat{\mathcal{A}_i} = \hat{f}(c_i, P_{i}), \hat{\mathcal{M}_i} = \hat{g}(P_{i})
% \end{equation}


\begin{equation} \label{eq:fast_residual_transformation}
p_{E_{j+1}} = p_{E_j} + \sum_{i=E_j}^{E_{j+1}-1} \left( \hat{\mathcal{A}_i}(p_i) + \hat{\mathcal{M}_i}(p_i + \hat{\mathcal{A}_i}(p_i)) \right) \quad \text{where} \quad \hat{\mathcal{A}_i} = \hat{f}(c_i, p_{i}), \hat{\mathcal{M}_i} = \hat{g}(h_i), 
\end{equation}



where $\hat{\mathcal{A}_i}$ and $\hat{\mathcal{M}_i}$ denote non-linear transformation added by layer $i$ to previous accelerated residual $p_{i}$. Similar to $\mathcal{A}_i$, non-linear transformation $\hat{\mathcal{A}_i}$ attends to same context $c_i$ but uses a different transformation $\hat{f}$ for accelerating $p_{E_j}$ relative to $h_{E_j}$. 

We integrate accelerated residual transformation directly into the base network using parallel accelerator adapters such that rank of accelerator adapters $R_p << d$ where $d$ denotes base model hidden dimension. This setup allows the slow residual stream $h_{E_j}$ to pass through the base model layers while the accelerated residual stream $p_{E_j}$ utilizes these parallel adapters as shown in ~\cref{fig:m2r2_main}. Both slow and accelerated residuals are processed in same forward pass via attention masking and incur negligible additional inference latency in memory bound decoding setups, while in compute bound decoding setups where FLOPs optimization is essential, accelerated residual stream utilizes a fraction of attention heads that of slow residual (see ~\cref{sec:flops_optimization}). Additionally, to maximize the utility of accelerated residual transformations without introducing dedicated KV caches, we propose a shared caching mechanism between the slow and accelerated streams which minimally impact alignment benefits of our approach while offering substantial memory savings (see ~\cref{fig:koala_alignment}). Specifically, the attention operation on the slow residuals \( \text{MHA}(h_t, h_{\leq t}, h_{\leq t}) \) is redefined for accelerated residuals as 
\[
\hat{\mathcal{A}} = MHA(p_t, h_{<t} \oplus p_t, h_{<t} \oplus p_t),
\]
where the accelerated residual at time-step $t$, \( p_t \) attends to the slow residual’s KV cache, facilitating the reuse of contextual information across both residual streams without incurring additional caching costs. Here, \(MHA(q, k, v) \) represents multi-head attention between query \( q \), key \( k \), and value \( v \).

\begin{figure}
    \centering
    \includegraphics[width=0.8\linewidth]{sections//figures/m2r2_main2.pdf}
    \caption{Multi-rate Residuals Framework: Slow residual stream of base model is accompanied by a faster stream that operates at a $2-(J+1)\times$ rate relative to the slow stream, undergoing transformations via accelerator adapters as detailed in \cref{m2r2_method}, where J denotes number of early exit intervals. Colors within the slow and fast residual streams indicate similarity, with matching colors representing the most closely aligned residual states. At the beginning of the forward pass and at each exit point, the accelerated residual state is initialized from the corresponding slow residual state to avoid gradient conflict during training (see ~\cref{sec:grad_conflict}). Early exiting decisions are informed by the Accelerated Residual Latent Attention (ARLA) mechanism, described in \cref{method_arla}, which evaluates residual dynamics across consecutive exit gates.}
    \label{fig:m2r2_main}
\end{figure}

% Furthermore. to maximize the benefits of fast residual transformations without using dedicated KV caches, we propose sharing the fast network’s cache with the slow network. Formally speaking, We modify attention operation on slow residuals $MHA(H_t, H_{<=t}, H_{<=t})$ as $MHA(P_{t}, H_{<t} \oplus P_t, H_{<t}  \oplus P_t)$ such that accelerated residuals attend to previous slow context KV cache, where $MHA(q,k,v)$ denotes multi head attention between query, $q$, key $k$ and value $v$.


\subsection{Enhanced Early Residual Alignment}
Early residual alignment is instrumental in optimizing early exiting, speculative decoding, and Mixture-of-Experts (MoE) inference mechanisms. In this section, we provide a detailed analysis of how accelerated residuals enhance these inference setups.

% By aligning the residual states of intermediate layers with the final output representations, the model can maintain high prediction accuracy even when computations are truncated at earlier layers. This enables more reliable early exiting, reducing the overall computational cost while preserving performance. Additionally, in speculative decoding, early residual alignment allows the model to make confident predictions using faster, partial computations, thereby accelerating inference without sacrificing output quality.


\subsubsection{Early Exiting} \label{method_early_exiting}

A prevalent strategy for enabling early exiting at an intermediate layer $E_{j}$ involves approximating the residual transformation between $E_{j}$ and the final layer $N-1$ using a linear, context independent mapping, $\mathcal{T}$, such that $H_{N-1} \approx \mathcal{T}(H_{E_{j}})$. This approximation has been extensively employed in conventional approaches ~\cite{schuster2022confident, chen2023eellm, varshney-etal-2024-investigating}, providing a computationally efficient means to project the output of deeper layers from intermediate states. Specifically, residual state of layer $N-1$ with this approximation can be expressed as:


% \begin{equation}
% \label{eq: vanila_ea_assumption}
% \Phi(H_{E_{j}}) \sim H_{E_{j}} \prod_{i=E_{j}}^{N}\left( I + \mathcal{A}_i \right) \left( I + \mathcal{M}_i \right) \quad \text{where} \quad \Phi \perp C
% \end{equation}

\begin{equation} \label{eq:early_exiting}
h_{E_j} + \sum_{i=E_j}^{N-1} \left( \mathcal{A}_i(h_i) + \mathcal{M}_i(h_i + \mathcal{A}_i(h_i)) \right) \sim \mathcal{T}(h_{E_{j}})  \quad \text{where} \quad \mathcal{T} \perp c. 
\end{equation}


Here, $\mathcal{A}_i$ and $\mathcal{M}_i$ represent the residual contributions of the multi-head attention and MLP layers, respectively, while $\mathcal{T}$ remains independent of $c$, the preceding context.

This approach is inherently limited by two major factors: first, the assumption of linearity between $h_{E_{j}}$ and $h_{N-1}$ may not hold uniformly for all tokens, particularly when $E_j \ll N$. Second, the linear transformation $\mathcal{T}$ disregards the influence of the context $c$ and fails to account for the latent representations of previous contextual states. In contrast, M2R2 accelerated residual states mitigate both of these challenges by approximating the slow residual transformation of all layers via a faster residual transformation of fewer layers as:
% \begin{equation}
% H_{E_j} \prod_{i=E_j}^{N}\left( I + \mathcal{A}_i \right) \left( I + \mathcal{M}_i \right) \sim P_{E_j} \prod_{i=E_j}^{E_j+1}\left( I + \hat{\mathcal{A}_i} \right) \left( I + \hat{\mathcal{M}_i} \right)
% \end{equation}


\begin{equation} \label{eq:m2r2_approximating_ea}
h_{E_j} + \sum_{i=E_j}^{N-1} \left( \mathcal{A}_i(h_i) + \mathcal{M}_i(h_i + \mathcal{A}_i(h_i)) \right) \sim p_{E_j} + \sum_{i=E_j}^{E_{j+1}-1} \left( \hat{\mathcal{A}_i}(p_i) + \hat{\mathcal{M}_i}(p_i + \hat{\mathcal{A}_i}(p_i)) \right), 
\end{equation}

% \begin{equation} \label{eq:fast_residual_transformation}
% p_{E_{j+1}} = p_{E_j} + \sum_{i=E_j}^{E_{j+1}-1} \left( \hat{\mathcal{A}_i}(p_i) + \hat{\mathcal{M}_i}(p_i + \hat{\mathcal{A}_i}(p_i)) \right) \quad \text{where} \quad \hat{\mathcal{A}_i} = \hat{f}(c_i, p_{i}), \hat{\mathcal{M}_i} = \hat{g}(h_i) 
% \end{equation}






where $p_{E_j}$ is initialized from the slow residual state $h_{E_j}$ at each early exit interval $E_j$ using an identity transformation (see ~\cref{fig:m2r2_main}). As shown in ~\cref{fig:m2r2_residual_sim}, accelerated residuals offer a smoother, more consistent shift in residual direction across layers, in contrast to the abrupt changes typically seen at early exit points in standard early exit methods. Moreover, the normalized cosine similarity between accelerated states at early exit intervals and final residual states is substantially higher compared to traditional early exit techniques, highlighting improved alignment with final layer representations. Traditional adaptive compute methods are constrained by two principal factors: the number of tokens eligible for early exit at intermediate layers and the precision of early exit decision. If residual streams fail to saturate early, the majority of tokens remain ineligible for exit, thereby diminishing potential speedups. Additionally, imprecise delineations between tokens suitable for early exit can lead to underthinking (premature exits that adversely affect accuracy) or overthinking (unnecessary processing that compromises efficiency) ~\cite{zhou2020self, dai2020dynamic}. Enhanced early alignment using ~\cref{eq:m2r2_approximating_ea} helps to address  first issue. To address the second issue we introduce Accelerated Residual Latent Attention, which dynamically assesses the saturation of the residual stream, allowing for a more precise differentiation between tokens that can exit early and those requiring further processing.

% This results in uniform change in residual direction    
% % We keep $\mathcal{A} = \hat{\mathcal{A}}$, while $\hat{\mathcal{M}}$ is accelerated by a factor of $2 - (N_{E}+1)X$ relative to the slower residual transformation $\mathcal{M}$, where $N_E$ represents number of early exiting intervals.
% Figure~\cref{fig:rate_change_comparison} illustrates the comparative rate of change between these transformation streams.



% fig:rate_change_comparison
% - grid plot x axis -> layer id (0, 8) , y axis -> layer id -> dark color cell for max similarity , lighter for lower 
% 
-------------------------------------------------------
Let's consider residual stream $h_i$ traverses through interval $E_j$ to $E_{j+1}$ and undergoes residual transformation given by 
\begin{equation}
h_{E_{j+1}} = h_{E_j} \prod_{i=E_j}^{E_{j+1}} \left( 1 + \delta_i \right)    
\end{equation}

where $\delta_i$ denotes non-linear transformation added by layer $i$. Each non-linear transformation of layer $i$ is a function of previous contextual representation, $c_i$ and input residual stream $h_i-1$ as
$\delta_i = f(c_i, h_{i-1})$ 

One way to exit early at exit $E_j+1$ is to assume that residual transformation from $E_j+1$ to final layer $N-1$ can be approximated by a linear function $\phi$ as $h_{N-1} \sim \Phi(h_{E_j+1})$ and most conventional approaches such as \todo{cite EA papers} use this approach. In other words, 

\begin{equation}
\Phi(h_{E_j+1} \sim h_{E_j+1} \prod_{i=E_j+1}^{N} \left( 1 + \delta_i \right)   
\end{equation}

This approach suffers from two primary issues, linearity assumption from $h_E_j+1$ to $H_N-1$ if often incorrect, particularly when $E_j << N$. More importantly, linear transformation $\Phi$ doesn't consider effect of context $C_i$. M2R2  effectively addresses these issues as accelerated residual stream at interval $E_j+1$ can be represented as 

\begin{equation}
r_{E_{j+1}} = r_{E_j} \prod_{i=E_j}^{E_{j+1}} \left( 1 + \gamma_i \right)    
\end{equation}

where $\gamma_i$ denotes non-linear transformation added by layer $i$ to previous accelerated residual $r_i-1$. Similar to $\delta_i$, non-linear transformation $\gamma_i$ considers context $C_i$ as 
$\gamma_i = g(c_i, r_{i-1})$. So in summary, slow residual transformation is approximated by accelerated residual as: 

\begin{equation}
h_{E_j} \prod_{i=E_j}^{N} \left( 1 + \delta_i \right) \sim h_{E_j} \prod_{i=E_j}^{E_j+1} \left( 1 + \gamma_i \right)
\end{equation}

It's worth noting that accelerated residual $r_i$ and slow residual $h_i$ are processed concurrently at layer $i$ by constructing proper attention mask such as attention of slow residual is represented as 

$MHA(H_it, H_{i<=t}, H_{i<=t}$ while attention of fast residual is computed as 

$MHA(r_it, H_{i<=t}, H_{i<=t}$ where $MHA(q,k,v$ denotes multi head attention between query, $q$, key $k$ and value $v$.


------------------------------------------------------------------

Vertical latent attention on accelerated residual is computed as 
$MHA(S_mt, S(Ej<=i<=m)t, S(Ej<=i<=m)t)$ where $Smt$ denotes query/key/value projection in latent domain at layer $m$ at time $t$. 
------------------------------------------------------------------

Gradient conflict Avoidance: 

Let's consider $w_j$ is a trainable parameter that belongs to a layer between $E_j$ and $E_j+1$. Consider early exit loss at gate $E_j+1$, $L_j+1$, gradient propagation of $w_j$ at another trainable parameter $w_j-n$ can be gives as 

$\sum_{k=E_j-n}^{E_j} \beta_k \frac{\partial L_{E_k}}{\partial w_k}$

where $\beta_j$ denotes backward transformation coefficient for weight $w_j$ to reach gate $E_j$. 
 
On the other hand, gradient propagation in proposed approach can be represented as 

\[
\frac{\partial L_{E_j}}{\partial w_j} = 
\begin{cases} 
\beta_j \frac{\partial L_{E_j}}{\partial w_j} & \text{if } E_j \leq w_j \leq E_{j+1} \\
0 & \text{otherwise}
\end{cases}
\]







% \begin{figure}[ht]
%     \centering
%     \includegraphics[width=0.8\textwidth, height=5cm]{rate_change_comparison.png}
%     \caption{Rate of change comparison between fast and slow residual streams.}
%     \label{fig:rate_change_comparison}
% \end{figure}

%vary k and and plot EA accuracy for larger and smaller models. 

% \begin{figure}[ht]
%     \centering
%     \includegraphics[width=0.5\textwidth,height=5cm]{sections/figures/alignment_comparison_dialogsum.pdf}
%     \caption{Alignment of exited tokens for different early exit layers using traditional early exiting heads, dedicated faster networks, and faster residuals.}
%     \label{fig:small_model_early_exiting}
% \end{figure}


\textbf{Accelerated Residual Latent Attention} \label{method_arla}

In the context of residual streams, we observe that the decision to exit at a given layer can be more effectively informed by analyzing the dynamics of residual stream transformations, instead of solely relying on a classification head applied at the early exit interval $E_j$. To capture the subtle dynamics of residual acceleration, we propose a \textit{Accelerated Residual Latent Attention} (ARLA) mechanism. This approach involves making the exit decision at gate $E_j$ by attending to the residuals spanning from gate $E_{j-1}$ to $E_j$, rather than considering only the residual at gate $E_j$. To minimize the computational overhead associated with exit decision-making, the attention mechanism operates within the latent domain as depicted in ~\cref{fig:arla_arch}. Formally, for each interval $[E_j, E_{j+1}]$, the accelerated residuals are projected into Query ($Q^s_{E_j}, \ldots, Q^s_{E_{j+1}}$), Key ($K^s_{E_j}, \ldots, K^s_{E_{j+1}}$), and Value ($V^s_{E_j}, \ldots, V^s_{E_{j+1}}$) vectors, with latent dimension $d^s$ for $Q^s$, $K^s$, and $V^s$ being significantly smaller than hidden dimension of $p$.\footnote{We use $d^s = 64$ for experiments described in ~\cref{sec:experiments}.} Notably, when the router is allowed to make exit decisions at gate $E_j$ based on residual change dynamics, we observe that the attention is not confined to the residual state at $E_j$ but is distributed across residual states from $E_{j-1}$ to $E_j$, %as illustrated in Figure~\ref{fig:vertical_latent_attention_dynamics}. 
This broader focus on residual dynamics significantly reduces decision ambiguity in early exits, as demonstrated in Figure~\ref{fig:roc_arla}, which contrasts routers based on the last hidden state, and the proposed ARLA router.

%show R -> S transformation. 
%show parameter and flop overhead as compared to adapter on last hidden state.

% \begin{figure}[ht]
%     \centering
%     \includegraphics[width=0.5\textwidth,height=5cm]{sections/figures/roc_arla.pdf}
%     \caption{ROC curves of early exit decision strategies: confidence-based methods (CALM/LITE), routers based on the accelerated hidden state, and latent attention routers.}
%     \label{fig:decision_making_comparison}
% \end{figure}

% \begin{figure}[ht]
%     \centering
%     \includegraphics[width=0.5\textwidth,height=5cm]{vertical_latent_attention.png}
%     \caption{Vertical latent attention mechanism for optimizing early exit decisions by considering residuals from gate \(M\) through \(M-1\).}
%     \label{fig:vertical_latent_attention}
% \end{figure}

\begin{figure}[ht]
    \centering
    \begin{subfigure}{0.52\textwidth}
        \centering
        \includegraphics[width=\textwidth, height = 4cm]{sections/figures/arla_arch.pdf}
        \caption{Accelerated Residual Latent Attention (ARLA): Accelerated residuals between early exit gates are projected into latent domain and attention over residual states within the interval is computed to capture residual dynamics and exit decision is made based on residual saturation.}
        \label{fig:arla_arch}
    \end{subfigure}%
    \hfill
    \begin{subfigure}{0.45\textwidth}
        \centering
        \includegraphics[width=\textwidth, height = 4.5cm]{sections/figures/vla_roc.pdf}
        \caption{ROC classification curves of early exit decision strategies using a linear router used on last residual state ~\cite{schuster2022confident, varshney-etal-2024-investigating, chen2023eellm}  and using ARLA approach that considers residual dynamics. }
        \label{fig:roc_arla}
    \end{subfigure}
    \caption{Effectiveness of ARLA in capturing residual dynamics for early exiting decisions.}


\end{figure}



% \begin{figure}[ht]
%     \centering
%     \includegraphics[width=1\textwidth,height=5cm]{sections/figures/arla.pdf}
%     \caption{fig that plots 32 rows 2 cols heatmap showing attention at each gate}
%     \label{fig:vertical_latent_attention_dynamics}
% \end{figure}

\subsubsection{Self Speculative Decoding} \label{method_self_speculative_decoding}

An alternative means to exploit the early alignment properties of our approach is through the use of accelerated residual states for speculative token sampling to accelerate autoregressive decoding. Speculative decoding aims to speed up memory-bound transformer inference by employing a lightweight draft model to predict candidate tokens, while verifying speculated tokens in parallel and advancing token generation by more than one token per full model invocation \cite{leviathan2023fast, chen2023accelerating, xia2023speculative, miao2023specinfer}. Despite its effectiveness in accelerating large language models (LLMs), speculative decoding introduces substantial complexity in both deployment and training. A separate draft model must be specifically trained and aligned with the target model for each application, which increases the training load and operational complexity ~\cite{chen2023accelerating}. Additionally, this approach is resource-inefficient, as it requires both the draft and target models to be simultaneously maintained in memory during inference \cite{leviathan2023fast, chen2023accelerating}. 

One strategy to address this inefficiency is to leverage the initial layers of the target model itself to generate speculative candidates, as depicted in ~\cite{Tang2024}. While this method reduces the autoregressive overhead associated with speculation, it suffers from suboptimal acceptance rates. This occurs because the linear transformation employed for translating hidden states from layer $k$ to the final layer $N$ is typically a poor approximation, as discussed in ~\cref{sec:motivation} and ~\cref{method_early_exiting}. Our approach resolves this limitation by utilizing accelerated residuals, which demonstrate higher fidelity to their slower counterparts. By utilizing accelerated residuals operating at a rate of $N/k$, where $k$ denotes the number of layers used for candidate speculation, we are able to efficiently generate speculative tokens for decoding.\footnote{We typically set $k = 4$ to balance the trade-off between autoregressive drafting overhead and acceptance rate, as discussed in~\cref{sec:experiments}.}
 This technique not only obviates the need for multiple models during inference but also improves the overall efficiency and effectiveness of speculative decoding.

\begin{figure}
    \centering    \includegraphics[width=1\linewidth]{sections/figures/m2r2_aot_loading.pdf}
    \caption{Ahead-of-Time Expert Loading: M2R2 accelerated residual stream predicts experts required for future layers, reducing reliance on on-demand lazy loading. Speculative pre-loading is efficiently overlapped with computation of multi-head attention (MHA) and MLP transformations. Only incorrectly speculated experts are loaded lazily, resulting in faster inference steps and improved computational efficiency. Here, H indicates LBM Host while D indicates HBM Device.}
    \label{fig:moe_expert_aot_loading}
\end{figure}


\subsubsection{Ahead of Time Expert Loading:} \label{method_aot_expert_loading}

Recent advancements in sparse Mixture-of-Experts (MoE) architectures ~\cite{shazeer2017outrageously, fedus2022switch, artetxe2019massively, lepikhin2020gshard, zoph2022designing} have introduced a paradigm shift in token generation by dynamically activating only a subset of experts per input, achieving superior efficiency in comparison to dense models, particularly under memory-bound constraints of autoregressive decoding \cite{fedus2022switch, zoph2022designing}. This sparse activation approach enables MoE-based language models to generate tokens more swiftly, leveraging the efficiency of selective expert usage and avoiding the overhead of full dense layer invocation. In dense transformer models, pre-loading layers is a common strategy to enhance throughput, as computations of current layer can be overlapped with pre-loading of next layer parameters ~\cite{narayanan2021efficient, shoeybi2020megatron}. However, MoE models face a unique challenge: expert selection occurs dynamically based on previous layer’s output, making it infeasible to preload next layer’s experts in parallel. This limitation results in inherent latency, as expert loading becomes a sequential, on-demand process ~\cite{lepikhin2020gshard, fedus2022switch}.

To address this inefficiency, our method introduces a mechanism with \textit{accelerated residuals}, which not only captures key characteristics of base slower residual states but also exhibit high cosine similarity with their final counterparts (as illustrated in \cref{fig:m2r2_residual_sim}). By employing accelerated residual streams, we can effectively predict the necessary experts for future layers well in advance of their actual invocation. Specifically, using a $2\times$ accelerated residual, the experts needed for layers $2i+2$ and $2i+3$ can be identified while still computing in layer $i$, thus overcoming the bottleneck of sequential, on-demand expert selection and mitigating latency in the decoding pipeline, as shown in \cref{fig:moe_expert_aot_loading}. Note that, we use fixed set of accelerator adapters for transforming accelerated residuals (as discussed in ~\cref{m2r2_method}) while slow residual is transformed via expert routing mechanism. 

Furthermore, our approach integrates a Least Recently Used (LRU) caching strategy, which enhances memory efficiency by replacing the least recently used experts with speculated experts that are anticipated to be needed in upcoming layers. This hybrid approach of preemptive expert loading with LRU caching yields substantial improvements over traditional on-demand loading or standalone caching strategies. By minimizing cache misses and efficiently managing memory, this approach addresses both compute and memory bottlenecks, leading to faster, more resource-efficient token generation in MoE architectures. A comprehensive evaluation of this strategy, in relation to state-of-the-art methods, is provided in \cref{experiments_aot}, and the compute and memory traces on an A100 GPU are detailed in \cref{fig:moe_aot_cuda_trace}.



% Recent advancements in sparse Mixture-of-Experts (MoE) architectures have introduced the concept of utilizing distinct computational paths for different tokens \cite{shazeer2017outrageously}. This approach, wherein only a subset of experts are activated per input, enables MoE-based language models to generate tokens more swiftly compared to their dense counterparts due to memory-bound nature of auto-regressive decoding. In dense models, pre-loading layers in advance is a common strategy to enhance computational efficiency. However, this technique is not applicable to MoE models, where expert selection occurs dynamically based on the outputs of previous layers, preventing parallel pre-fetching of experts.

% Our proposed method addresses this inefficiency. Accelerated residuals, which are highly similar to their slower counterparts (see \cref{fig:similarity}), can reliably predict the necessary experts ahead of time. For instance, by utilizing $2X$ accelerated residual stream, we can predict the experts needed for the layer $2i+1$ and $2i+3$ while carrying out computation in layer $i$. This enables us to commence expert loading significantly earlier, as illustrated in \cref{expert_loading}, effectively mitigating the delays observed with the naive on-demand expert loading. Additionally, our method benefits from incorporating a Least Recently Used (LRU) strategy, where speculated experts replace those that are least recently utilized, resulting in improved performance compared to using either strategy alone. For a comprehensive evaluation, refer to \cref{moe_trace}, which provides a CUDA compute and memory trace of our approach executed on <>.



% A naive solution involves using the residual state of the previous layer along with the gating function of the next layer to predict which experts need to be loaded, and initiating the expert loading process in parallel with the attention computation of the next layer. Yet, as shown in \cref{fig:MOE_attn_vs_loading_time}, the attention computation for medium to long contexts is considerably faster than the expert loading time, making this approach inefficient.




\subsection{Training} \label{method_training}
% This approach is feasible due to the absence of gradient conflicts, as discussed in \cref{sec:grad_conflict}.

To accelerate residual streams, we employ parallel accelerator adapters as described in \cref{m2r2_method}.  For the early exiting use-case outlined in \cref{method_early_exiting}, we define the training objective for these adapters using the following loss function, which combines cross-entropy loss at each exit $E_j$ with distillation loss at each layer $i$. Loss weights coefficients $\alpha_0$ and $\alpha_1$ are employed to balance contribution of corresponding losses.

\begin{align} \label{eq:mr_loss}
L_{\text{m2r2}} = \underbrace{-\alpha_0 \sum_{j=1}^{J} \sum_{t=1}^{T} \log p_{\theta} \left( \hat{y}_t^{E_j} \mid y_{<t}, x \right)}_{\text{cross-entropy loss}} 
+ \underbrace{\alpha_1\sum_{i=1}^{E_{J-1}} \sum_{t=1}^{T} \| \mathbf{p}_{t}^{i} - \mathbf{h}_{t}^{((i - E_{j(i)}) \cdot R_i) + E_{j(i)})} \|^2}_{\text{distillation loss}}.
\end{align}

where $\hat{y}_t^{E_j}$ denotes the predictions from the accelerated residual stream at layer $E_j$ and time step $t$, $y_t$ represents the corresponding ground truth tokens, and $x$ indicates previous context tokens. The distillation loss at each layer $i$ is computed by comparing accelerated residuals at layer $i$ with slow residuals at layer $(i - E_{j(i)}) \cdot R_i + E_{j(i)}$, where $R_i$ denotes the rate of accelerated residuals at layer $i$ while $E_{j(i)}$ represents the most recent gate layer index such that $E_{j(i)} <= i$. \( J \) represents the total number of early exit gates, N denotes number of hidden layers and $E_j$ denotes layer index corresponding to gate index $j$ and \( T \) denotes the sequence length. 

In dynamic compute settings, after training of accelerator adapters, we optimize the query, key, and value parameters governing the ARLA routers (see ~\cref{method_arla}) across all exits in parallel on binary cross entropy loss between predicted decision and ground truth exiting decision. The ground truth labels for the router are determined based on whether the application of the final logit head on $\hat{y}_t^{E_j}$ yields the correct next-token prediction. 


% The objective for this optimization is defined by the following loss function:


%TODO are equations required ? 
% \begin{equation} \label{eq:arla_loss_combined}\small
%     L_{\text{arla}} = -\frac{1}{N} \sum_{t=1}^{T} \left( \sum_{j=1}^{E_n} \left[ O_t^{E_j} \log(\hat{O}_t^{E_j}) + (1 - O_t^{E_j}) \log(1 - \hat{O}_t^{E_j}) \right] \right), \quad \text{where} \quad 
%     O_t^{E_j} = \begin{cases} 
%     1, & \text{if } L(\hat{y}_t^{E_j}) = y_t^{E_j} \\
%     0, & \text{otherwise}
%     \end{cases}
% \end{equation}

% where $\hat{O}_t^{E_j}$ represents the binary predicted logits produced by the vertical latent attention router, as described in \cref{sec:arla}, at gate $E_j$ and time step $t$, and $O_t^{E_j}$ denotes the corresponding ground truth labels. The ground truth labels for the router are determined based on whether the application of the logit head on $\hat{y}_t^{E_j}$ yields the correct next-token prediction. The parameters controlling vertical latent attention are trained concurrently to ensure consistency and efficient use of computational resources.

For self-speculative decoding, as described in \cref{method_self_speculative_decoding}, the training objective remains the same as \cref{eq:mr_loss}, but with the number of intervals set to $J = 1$ and the rate of residual transformation set to $R_n = N/k$, where the first $k$ layers generate speculative candidate tokens. In the context of Ahead-of-Time Expert Loading for Mixture-of-Experts (MoE) models (see \cref{method_aot_expert_loading}), setting the rate of residual transformation to $R_n = 2$ typically offers a good trade-off between the accuracy of expert speculation and AoT pre-loading of experts. 

% Thus, we set $J = 1$ and $E_1 = 16$.


~\subsection{FLOPs Optimization} \label{sec:flops_optimization}

Naively implemented, M2R2 incurs higher FLOP overhead compared to traditional speculative decoding and early exiting approaches such as ~\cite{medusa, schuster2022confident, Tang2024}. However, modern accelerators demonstrate compute bandwidth that exceeds memory access bandwidth by an order of magnitude or more~\cite{databricksLLMInference2023, jouppi2021ten}, meaning increased FLOPs do not necessarily translate to increased decoding latency. Nevertheless, to ensure fair comparison and efficiency in compute bound scenarios, we introduce targeted optimizations.

~\textbf{Attention FLOPs Optimization} For medium-to-long context lengths, attention computation dominates FLOPs in the self-attention layer, surpassing the contribution from MLP layers. Specifically, matrix multiplications involving queries, cached keys, and cached values scale with $l_{kv} * l_{q}$ where $l_{kv}$ denotes previous context length and $l_q$ denotes current query length. Since M2R2 pairs accelerated residuals with slow residuals, a naive implementation results in twice the FLOPs consumption compared to a standard attention layer. To address this, we limit the attention of accelerated residual stream to selectively attend to the top-k most relevant tokens, identified by the slow residual stream based on top attention coefficients\footnote{We set to k = 64 and attend to top 64 tokens as identified by the slow residual stream.}. This is possible since slow and accelerated residual streams are processed in same forward pass and accelerated streams have access to attention coefficients of slow stream. Note that, the faster residual stream still retains the flexibility to assign distinct attention coefficients to these tokens. Furthermore, we design the faster residual stream to employ only 8 attention heads, compared to the 32 heads used in the slow residual stream of the Phi-3 model, reducing query, key, value, and output projection FLOPs by a factor of 1/4. ~\cref{fig:m2r2_num_heads_ablation} indicates effect of using a slicker stream on alignment. As depicted, using $\hat{n}_h = 8$ offers a good trade-off between alignment and FLOPs overhead. 

~\textbf{MLP FLOPs Optimization} The accelerator adapters operating on the accelerated residual stream are intentionally designed with lower rank than their counterparts in the base model. This reduces FLOP overhead by a factor proportional to $hiddenSize / rank$. Additionally, since the faster residual stream uses only 8 attention heads (compared to 32 in the slow residual stream of Phi-3), the subsequent MLP layers process a smaller set of activations, further reducing FLOPs by another factor of 1/4.

These optimizations significantly reduce the FLOP overhead per speculative draft generation, as illustrated in ~\cref{fig:flops_optmization}. Notably, while traditional early-exiting speculative approaches such as DEED require propagating the full slow residual state through the initial layers, incurring substantial computational costs, M2R2 achieves efficient token generation via slimmer, low-rank faster residual streams. In contrast, Medusa introduces considerable FLOP overhead due to per-head computations scaling with $d^2+dv$\footnote{Here $d$ denotes hidden state dimension while $v$ denotes vocab size.}, whereas M2R2 employs low-rank layers for both MLP and language modeling heads, maintaining computational efficiency. All experiments involving the M2R2 approach, as detailed in ~\cref{sec:experiments}, are conducted using these FLOPs optimizations.









% \[
% O_t^{E_j} = 
% \begin{cases} 
% 1, & \text{if } L(\hat{y}_t^{E_j}) = y_t^{E_j} \\
% 0, & \text{otherwise}
% \end{cases}
% \]




%add distillation
% We train accelerator adapters described in \cref{m2r2_method} to accelerate residual streams on next token prediction all in parallel since there are no gradient conflict issues as described in \cref{sec:grad_conflict}.

% \begin{align} \label{eq:mr_loss}
% L_{mr} =  & -\sum_{j = 1}^{E_n} (\sum_{t=1}^{T}\log p_{\theta} (\hat{y}_t^{E_j} | \hat{y}_{<t}, x)) \nonumber
% \end{align}

% where $\hat{y_t^{E_j}}$ denotes predicted logits obtained from accelerated residual stream at gate $E_j$ and time-step $t$ while $y_t^{E_j}$ denotes corresponding truth tokens. 

% Upon training of adapters responsible for accelerating residual streams, we train query, key, value parameters responsible for vertical latent attention of all gates in parallel as

% \begin{equation} \label{eq:arla_loss}
%     L_{arla} = -\frac{1}{N} (\sum_{t=1}^{T}(1\sum_{j=1}^{E_n} \left[ O_t^{E_j} \log(\hat{O}_t^{E_j}) + (1 - o_t^{E_j}) \log(1 - \hat{o_t}_{E_j}) \right]))
% \end{equation}

% where $\hat{O_t^{E_j}}$ denotes binary predicted logits obtained from vertical latent attention router described in \cref{sec:arla} at gate $E_j$ and timestep $t$ while $O_t^{E_j}$ denotes corresponding truth label. Truth labels for router are obtained by computing whether logit head application on $\hat{y}_t^j$ results in true next token prediction. Formally speaking, 

% $O_t^{E_j} = 1 if L(\hat{y_t^{E_j}}) == y_t^{E_j} , 0 otherwise$. 

% Parameters responsible for vertical latent attention are also trained in parallel as well. 

%todo: training slow and fast residuals together and distillation can be two training mdoes. 
%Distillation can be an ablation. 




% Although transformer decoding is memory bound on most mainstream accelerators, there could be scenarios where flop savings are crucial. For instance, on on-device settings power consumption is directly correlated with flops per decoding step and reducing flops does help with overall energy consumption. Vanilla early exiting methods help with flop reduction but suffer from mismatch between training and inference due to early exited tokens. If token at decoding step $t$, $T_t$ exited at layer $E_i$, while token $T_{t+k}$ exits at layer $E_j$ such that $E_i < E_j$, hidden state $H_{t+k}l$ does not have corresponding hidden state $H_tl$ to attend to where $E_i < l <= E_j$. One solution that's often used in literature is to rely on last hidden state available, $H_t{E_j}$, however it tends to be sub-optimal and does affect generation quality \cite{ref}.  To alleviate this mismatch while reducing flops, we train router such that attention mask between token $T_{t+k}$ and token $T_{<t+k}$ is given by: 

% \begin{equation}
%     a_{T_{{t+k}{T_{<t+k}}} = 1 if  E_{T_{<t+k}} >= E{T_{t+k}}
%     else 0
% \end{equation}

% This attention mask enables router to account for exited tokens and get trained accordingly. Since attention mechanism during decoding remains exactly same as that during training, impact on generation quality tends to be minimal as noted in \cref{fig:gen_auality_with_and_without_recompute_attention_show_flops}.  Although MoD does not suffer from training and inference mismatch, we observe that it suffers from discountinuity between pre-training and super-vised fine-tuning resulting in sub-optimal perplexity. On the other hand, our method doesn't not require pre-training , doesn't suffer from discountinuity, and achieves much better perplexity in super-vised fine-tuning and instruction tuning setups as shown in \cref{fig:Mod_vs_m2r2_loss_curves}.






% Our techniques are directly applicable in such scenarios.    




%expert loading with cuda streams in experiments
\section{Results}
\label{sec:results}

\begin{figure}[ht]
    \centering
\includegraphics[width=\linewidth]{figures/progs.png}
\caption{The best performing programs for each of the 3 benchmark problems as Python programs (left in each card) and the corresponding DAG representation on the right.
The DAG representation allows better visualization of the importance of different components.
The thickness of the red edges determine how important that component is. 
A black edge represents computation; when removed it is either the same as one of its subsequent edges or removing it could result in a bug.
}
\label{fig:mainresults}

\end{figure}

\subsection{Implementation details}
For all our main experiments we used an open-source LLM \emph{llama-3-8b-instruct}~\citep{dubey2024llama} served using the vLLM library~\citep{kwon2023vllm}.
However, in the supplementary, we also explore other open-source language models.
All the visual data in our benchmarks comes from satellite images, so to allow inferring semantic information from it, we use a black-box open-world foundational model for satellite images, GRAFT~\citep{mall2023graft}. 
Some experiments use ground-truth annotations from OpenStreetMaps~\citep{vargas2020osm} as an alternative to disentangle the effect of segmentation from discovery. 

We run our evolutionary method for $T=15$ generations with a population size of $M=100$. 
For all the problems, the input observation data comes from different geographical locations around the world. 
We split this data into three parts. Two-thirds of the easternmost observations are used to create a training-testing split. The remaining one-third of the data is use to evaluate reliability (out-of-distribution generalization). 
We also release this benchmark for future research in this area. 

\subsection{Benchmark for Visual Program Discovery for Scientific Applications}
\label{ssec:setup}
Given the novelty of the visual program discovery task, there exists no pre-existing benchmark.
We define a new benchmark for this task, drawing on scientifically relevant geospatial problems.
Concretely, we choose two different problems in \emph{Demography}: population density and poverty indicators, 
and to a problem in \emph{Climate Science}: for above ground biomass (AGB) estimation. 


It is important to note that for these problems, \emph{true relationships between variables of interest are actually unknown}.
As such, an LLM cannot be expected to produce a good program in a zero-shot manner, because it has never seen these relationships before.
This is in contrast to problems like VQA~\cite{suris-23} where the reasoning required to answer a question is well known and we can simply rely on the LLM's world knowledge.
In the case of scientific discovery, actual data is needed to discover the right reasoning.

In the following, we present the observation datasets, metrics, and overview of primitives. 

\subsubsection{Population Density}

\textbf{Observation Dataset}: 
The problem seeks to predict the population density by observing the satellite images of a region~\citep{metzger-22,metzger-24}. 
We obtain the population density values ($y_i$) for various locations in the USA by using ACS Community Surveys 5-year estimates~\citep{acs2024}.
Input observations ($x_i$) are sentinel-2 satellite images at a resolution of 10m~\citep{drusch2012sentinel}. For this experiment, we also use OpenStreetMaps masks~\citep{vargas2020osm} for 42 different land-use concepts (see supplementary) as part of the input.

\textbf{Metric and Primitives:} 
Population density values are aggregated at the county block group level. The predicted population densities are therefore also aggregated at the county block group level. The metric is the per-block group level average L2 error after applying a log transformation.
Along with the arithmetic, and logical primitives (see supplementary)
, we use open-vocabulary segmentation as a primitive.
The segmentation function returns a binary mask for an input concept. 



\subsubsection{Poverty Indicator}
\textbf{Observation Dataset:} For poverty estimation, we use data from SustainBench~\citep{yeh2021sustainbench}. 
The dataset contains coordinate location as input and wealth asset index as output.

\textbf{Metric and Primitives:} We use L2 error for each location as the evaluation metric. To obtain semantic land use information about a location, we first define a \emph{get\_satellite\_image} function, that returns a sentinel-2 satellite image for any location. This can be used in conjunction with the open-world satellite image recognition model to obtain semantic information about the world. 
Other than this we also include as primitives functions that return average annual temperature, precipitation, nightlight intensity, and elevation at the input location. 

\subsubsection{Aboveground Biomass}

\textbf{Observation Dataset:}  
Similar to poverty estimation, the observation variables are an input location and the output AGB estimate. We use NASA's GEDI ~\citep{dubayah2020gedi} to obtain the observation value for three US states. 
We use data from Massachusetts and Maine (North-East) as the train/test set and Washington (NorthWest) as the out-of-distibution set.

\textbf{Metric and Primitives:} We use L2 error as the metric and the same primitives as poverty estimation.



\subsection{Experimental Setup}
For the same set of training data we compare our best generated program with a set of baselines. 
\begin{enumerate}
    \item \textbf{Mean:} A naive baseline that use the mean of the training observation as the prediction.
    \item \textbf{Concept Bottleneck (CB):} Similar to \cite{koh20concept,yang2023language,oikarinen2023label}, we first extract a list of relevant features and train a linear classifier on it. 
    This method is interpretable due to the bottleneck, however it is not very expressive (see supplementary).
     

    \item \textbf{Deep models:} We use deep models such as ResNets~\cite{he2016resnet} as baseline (see supplementary for details). We use a small and large variant for each.
    
    \item \textbf{Zero-shot:} This baseline tests how good would LLMs be on their own in generating programs solely relying on prior knowledge without any observation. Since the generated programs can vary drastically, we report an average of 5 different zero-shot programs.

    \item \textbf{Random Search:} Instead of evolutionary search, this baseline relies on the stochasticity of LLMs to perform a random search. If ~\disciple~ is better at searching, it should do better than random searching for the same number of calls to an LLM.
\end{enumerate}

\subsection{Results and Discussion}



\begin{table*}
\small
\centering
      \caption{Performance of our programs on in-distribution (left) and out-of-distribution (right) observations across various problems in the proposed benchmark. This shows the reliability of programs produced by \disciple~(\best{red} is best and \sota{blue} is second best).
      } \label{tab:performance}      
      \begin{tabular}{l c c c c c c | c c c c c c} 
        \specialrule{.12em}{.1em}{.1em}       
        & \multicolumn{6}{c}{In distribution}
        & \multicolumn{6}{|c}{OOD}\\
        & \multicolumn{2}{c}{Population Density} & \multicolumn{2}{c}{Poverty} & \multicolumn{2}{c}{AGB}
        & \multicolumn{2}{|c}{Population Density} & \multicolumn{2}{c}{Poverty} & \multicolumn{2}{c}{AGB}
        \\
        & L2-Log & L1-Log & L1 & RMSE & L1 & RMSE
        & L2-Log & L1-Log & L1 & RMSE & L1 & RMSE\\
        \specialrule{.12em}{.1em}{.1em}
        Mean 
        & 0.6696 & 0.6540 & 1.613 & 1.836 & 42.15 & 50.65
        & 0.6734 & 0.6561 & 1.591 & 1.844 & 74.15 & 83.02
        \\
        
        CB 
        & 0.8298 & 0.7279 & 1.229 & \sota{1.476} & 26.33 & 33.49
        & 0.7951 & 0.7112 & 1.257 & 1.504 & 44.19 & 63.52
        \\
        
        Deep - Small  
        & 0.4431 & 0.5006 & 1.238 & 1.637 & 30.72 & 37.03
        & 0.6623 & 0.5967 & \sota{1.284} & \sota{1.654} & \sota{35.27} & \sota{53.06} 
        \\
        
        Deep - Large  
        & \sota{0.3974} & \sota{0.4843} & \sota{1.170} & 1.478 & \best{21.15} & \best{27.86}
        & \sota{0.4460} & \sota{0.5115}  & 1.344 & 1.741 & 35.41 & 70.30  
        \\
        
        Zero-shot 
        & 0.4702 & 0.5371 & 1.525 & 1.754 & 38.80	& 46.41
        & 0.7020 & 0.6412 & 1.510 & 1.773 & 55.11 &	64.32  
        \\

        Random Search 
        & 0.4353 & 0.5118 & 1.277 & 1.679 & 29.40	& 36.70
        & 0.6763 & 0.6298 & 1.418 & 1.840 & 42.32 & 52.53  
        \\
        
        \textbf{Ours} 
        & \best{0.2607} & \best{0.3778} & \best{1.077} & \best{1.314} & \sota{24.79} & \sota{32.99}
        & \best{0.3807} & \best{0.4426} & \best{1.134} & \best{1.420} & \best{31.10} & \best{42.93}
        \\
        \specialrule{.12em}{.1em}{.1em}
      \end{tabular} 
\end{table*}




We first test our programs on unseen \emph{in-domain} observations close to the regions used for training (\cref{tab:performance} (left)).
We observe that ~\disciple~outperforms all interpretable baselines.
It can even outperform a deep model in many cases, specifically on population density estimation, while being significantly more interpretable.
\disciple~ also outperforms zero-shot program inference from LLMs.
As discussed before, this is in line with the fact that \disciple is uncovering new relationships that may not be known to us, and by extension, to the LLM.
The performance of random search while better than zero-shot is significantly worse than ~\disciple. 
This shows that~\disciple~ is able to perform a significantly faster search, by reducing the meaningful search space.
Our evolutionary process effectively leverages data to perform this novel discovery.

\paragraph{Are our programs reliable?} 
If a program is reliable it should be able to generalize to other regions. 
\cref{tab:performance} (right) shows DiSciPLE to these baselines on such an out-of-distribution set.
Here our approach outperforms all baselines \emph{including deep networks}, suggesting that due to its interpretable-by-design representation, our method learns a model that can generalize better and overfit less to the in-distribution training data.

\begin{figure*}[ht]
    \centering
\includegraphics[width=\linewidth]{figures/qualitative_main.png}
\caption{Qualitative comparison of ~\disciple~ with other baselines on the tasks of population density. ~\disciple~ Can map to the true population density maps much more accurately than the baselines (Refer to the supplementary for more comparisons). The maps display population density as the base-10 log of people per square mile.
}
\label{fig:qualitative}
\end{figure*}

We also show these results qualitatively in \cref{fig:qualitative}, by comparing population density predictions of ~\disciple~ and the baselines to the true population density.
It is very clearly evident that ~\disciple~ can model the fine-grained changes in population in unseen regions significantly better than the baselines (refer to supplementary for more visualizations). 


\begin{figure}[ht]
    \centering
\includegraphics[width=\linewidth]{figures/scale_vert.png}         
\caption{Performance of \disciple~compared to deep baselines as we reduce the amount of training observation (in terms of L2 error). The Oracle (blue) uses a program learned from all observations but uses only partial observation for parameter training. ~\disciple~(orange) uses partial observation during evolution as well. While the errors get worse as we reduce the observation data, the drop is significantly less severe for ~\disciple~ compared to deep models, which tend to overfit.}
    \label{fig:scale}
\end{figure}


\paragraph{Are our programs data-efficient?} 
Our methods are only trained on a maximum of 4000 observations.
\cref{fig:scale} further shows that even when the amount of training data is reduced, our approach shows minimal degradation in performance compared to deep networks.
This suggests that while deep models can learn to generalize with a lot more data, our model does not need as much data to begin with, making it data-efficient.


\paragraph{Are our programs interpretable?} Our programs are interpretable-by-design as we can visualize the factors contributing to performance. 
Fig.~\ref{fig:mainresults} shows such programs (left in each card) for all the problems in our benchmark. 
An expert who is working with our method to figure out such programs can add/edit parts of the formula and figure out which/how much do each of these components matters. 

We perform this step of understanding the influence of individual operations by removing each operation in our program and measuring its effects on the final score. 
The DAGs on the right of each program show the program structure and the red edges show the influence of each component proportional to the width. 
This visualization can allow experts to understand which operations are important for the model.
For example, in the program for population density \cref{fig:mainresults}, we can see that semantic concepts such as ``highway'' and ``residential building'' are very important.






\paragraph{Can our method perform better than expert humans?}
Our method would only be useful in real-world scenarios if it can come up with stronger or comparable programs to human experts.  
We test this on the task of AGB, by providing an expert (a PhD student actively working on AGB) with a user interface with the same information as our method. 
The experts took about 1.5 hours to use their domain knowledge and iterate over their program for AGB estimation. However, the best program they could come up with had an L1 error of \textbf{37.65} on the in-distribution set and \textbf{53.20} on the OOD set (compared to \textbf{24.79} and \textbf{31.10} for ~\disciple). We figure this is primarily because experts need to spend more time on the problem. 
In general, experts would spend numerous days to come up with a good program, 
while our method can come up with a better program faster.

\paragraph{Extension to more indicators}
We also test ~\disciple~ on a larger suite of demographic indicators. Using SocialExplorer, we build a suite of 34 demography indicators. Refer to the supplementary for a list of these indicators. This includes demography information such as age group, education status, etc. In \cref{tab:moreindicators}, we report the average performance of our method compared to baselines on this data.
Since different indicators can have different scales, we first normalize all of them to have zero mean and unit standard deviation. 
These indicators are challenging to predict directly from satellite images, as evidenced by the deep model failing to perform significantly better than CB and mean baselines. 
As a result while ~\disciple~ performs better than all the baselines the improvements are not huge.
Nonetheless, ~\disciple~ performs better than every baseline.
This large-scale experiment shows the potential of applying ~\disciple~ to a wider range of problems. More details about these demographic indicators and individual performance on these is shown in the supplementary.

\begin{table}
\small
\centering
      \caption{Performance of ~\disciple~ compared to baselines on a larger suite of challenging 34 demographic indicators. Since the dataset is very challenging, the deep baseline regresses to mean, however with  ~\disciple~ we can still see some improvements.
      } \label{tab:moreindicators}      
      \begin{tabular}{l c c c c} 
      & \multicolumn{2}{c}{Test} & \multicolumn{2}{c}{OOD} \\
     & L1  & RMSE & L1 & RMSE \\
        \specialrule{.12em}{.1em}{.1em}    
     Mean & 0.8578 & 1.1519 & 0.8939 & 1.1948\\
     CB & 0.8249 & 1.1159 & 0.8771 & 1.1767\\
     Deep & 0.8527 & 1.1556 & 0.8942 & 1.1990\\
     \textbf{Ours} & \textbf{0.8159} & \textbf{1.1065} & \textbf{0.8750} & \textbf{1.1719}\\
        \specialrule{.12em}{.1em}{.1em}       
      \end{tabular} 
\end{table}

\subsection{Ablations}

\paragraph{How important is the role of feature-set prediction, critic, and simplification?}
Table~\ref{tab:ablationparts} measures the performance of our model on the task of population density as we successively add these components to the evolutionary algorithm. 
The addition of feature set prediction instead of a single feature helps, as it allows our method to learn expressive linear regression parameters instead of letting the LLM come up with them.
Further adding critic results in further improvement as the programs start covering nicher concepts resulting in better unseen and OOD generalization. 
Finally adding in simplification also improves the program. 
We posit that simplification removes irrelevant features preventing the LLM from focusing on them when performing crossovers.

\begin{table}
\small
\centering
      \caption{Performance of our method as we successively remove the components. Both critic and simplification lead to performance improvement for our method.
      } \label{tab:ablationparts}      
      \begin{tabular}{c c c c c c c} 
      & & & \multicolumn{2}{c}{Test} & \multicolumn{2}{c}{OOD} \\
      Set & Critic & Simpli. & L2 log & L1 log & L2 log & L1 log \\
        \specialrule{.12em}{.1em}{.1em}       
        \xmark & \xmark & \xmark & 0.3159 & 0.4296 & 0.4835 & 0.5178\\    
        \cmark & \xmark & \xmark & 0.2906 & 0.4049 & 0.4258 & 0.4826\\    
        \cmark & \cmark & \xmark & 0.2873 & 0.3984 & 0.4184 & 0.4684 \\    
        \cmark & \cmark & \cmark & \textbf{0.2607} & \textbf{0.3778} & \textbf{0.3807} & \textbf{0.4426}\\    
        \specialrule{.12em}{.1em}{.1em}       
      \end{tabular} 
\end{table}


\paragraph{How important are common sense and prior knowledge of LLMs?}
The two major advantages an LLM provides over traditional tree-search are: 1) better crossover and mutation as LLMs can understand the meaning of the primitives. 2) use of prior knowledge for better-guided search. 
Therefore we remove these two sources of information and test how well can our method perform. 
To remove the understanding of functions we rename them with meaningless terms and remove the descriptions.
To remove the context of the problem we remove the objective prompt.
\cref{tab:ablationcommonsense}, show the performance of our method on density estimation after removing each of these prompts. 
Without common sense, the search cannot even progress away from the initial random programs, resulting in worse-than-mean results (L1 error of 0.84 vs 0.26 for ~\disciple).
This suggests that symbolic regression models, that have no understanding of open-world primitives, would struggle to search.
If we just remove the context of the problem, the model does slightly better and can obtain results better than the mean and zero-shot programs (L1 error of 0.45). This suggests that while the search is moving in the objective's direction, it is slow. 


\begin{table}
\small
\centering
      \caption{Perfomance of our method when removing the context of the problem (objective prompt from the evolution, and when renaming and not describing the primitive functions to the LLM.
      We see significant drops in performance in both cases, suggesting that both common sense and prior knowledge of LLM are important to perform efficient evolutionary search.
      )} \label{tab:ablationcommonsense}      
      \begin{tabular}{c c c} 
      Method & L1 log & L2 log \\
        \specialrule{.12em}{.1em}{.1em}       
        No common-sense & 0.8401 & 0.7186\\    
        No problem context & 0.4498 & 0.5140\\    
        \disciple~full & \textbf{0.2607} & \textbf{0.3778}\\    
        \specialrule{.12em}{.1em}{.1em}       
      \end{tabular} 
\end{table}




\section*{Conclusion}
This paper aims to enhance our understanding of the computational complexity of computing various Shapley value variants. We found that for various ML models --- including decision trees, regression tree ensembles, weighted automata, and linear regression --- both local and global interventional and baseline SHAP can be computed in polynomial time under HMM modeled distributions. This extends popular algorithms, such as TreeSHAP, beyond their empirical distributional scope. We also establish strict complexity gaps between the various SHAP variants (baseline, interventional, and conditional) and prove the intractability of computing SHAP for tree ensembles and neural networks in simplified scenarios. Overall, we present SHAP as a versatile framework whose complexity depends on four key factors: \begin{inparaenum}[(i)] \item model type, \item SHAP variant, \item distribution modeling approach, \item and local vs. global explanations\end{inparaenum}. We believe this perspective provides deeper insight into the computational complexity of SHAP, paving the way for future work.




%We believe that our framework provides a more intricate understanding of SHAP computation complexity across different models, distributions, and variants, paving the way for further research.

Our work opens promising directions for future research. First, expanding our computational analysis to other SHAP-related metrics, such as asymmetric SHAP~\citep{frye20} and SAGE~\citep{covert2020understanding}, would be valuable. Additionally, we aim to explore more expressive distribution classes and relaxed assumptions beyond those in Section \ref{sec:tractable} while maintaining tractable SHAP computation. Finally, when exact computation is intractable (Section \ref{sec:intractable}), investigating the approximability of SHAP metrics through approximation and parameterized complexity theory~\citep{downey2012parameterized} is an important direction.

%Our work opens several promising avenues for future research on the computational properties of explainable AI methods, with a particular focus on SHAP. First, it would be interesting to broaden the computational analysis conducted in this work to include other popular SHAP-related metrics in the literature, such as asymmetric SHAP \cite{frye20} and SAGE \cite{covert2020understanding}. Also, in the future, we aim to explore more expressive distribution classes and relaxed distributional assumptions—extending beyond those examined in Section \ref{sec:tractable} —that still yield tractable SHAP computation. Finally, when exact computation proves intractable (Section \ref{sec:intractable}), it is worthwhile to theoretically investigate the question of the approximability of computing the SHAP metrics across various configurations, through the lens of approximation and parametrized complexity theory \cite{arora2009computational}.

%This paper aims to deepen our understanding of the computational complexity involved in obtaining different Shapley value variants. We found that for a variety of ML models, including decision trees, tree ensembles for regression, weighted automata, and linear regression models — computing both local and global interventional and baseline SHAP can be done in polynomial time when distributions are modeled by HMMs. This extends the distributional scope of popular algorithms like TreeSHAP, which is limited to empirical distributions. Additionally, we demonstrate a strict complexity gap between SHAP variants, showing that interventional and baseline SHAP can be strictly easier to compute than conditional SHAP. Despite these positive results, we uncovered intractability for various SHAP variants in neural networks and tree ensembles. Finally, we provided generalized complexity relations across SHAP variants. We believe that our framework offers a deeper understanding of the complexity involved in computing SHAP across various variants, models, distributions, as well as in both local and global computations, laying the groundwork for future research.
\section*{Acknowledgments}
{\textcopyright}2025 All rights reserved. The research described in this paper was carried out at the Jet Propulsion Laboratory, California Institute of Technology, under a contract with the National Aeronautics and Space Administration (80NM0018D0004).
% \clearpage
{
\small
\bibliographystyle{ieeenat_fullname}
\bibliography{indest}
}
\clearpage
\appendix


\end{document}
