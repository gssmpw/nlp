

\onecolumngrid
\section*{Supplementary Materials for:\\
Inverse Design with Dynamic Mode Decomposition}

%%%%%%%%%%%%%%%%%%%%%%%%%%%%%%%%
%%%%%%%%%%%%%%%%%%%%%%%%%%%%%%%% SECTION 1
%%%%%%%%%%%%%%%%%%%%%%%%%%%%%%%%
\section{Section 1: The ID-DMD algorithm}

To identify the ID-DMD of a system, we begin by collecting state data across various design parameters sampled over the design space. These design parameters are typically determined using uniform distributions or other design-of-experiment methods. Since design parameters can span multiple orders of magnitude, it is necessary to ensure well-conditioned matrices for subsequent calculations. To address this issue, we scale the design parameter values as ${\bar{\varepsilon}_{i}}={\alpha_{i}}{\varepsilon_{i}}$, $i\in {\mathbb{Z}^{+}}$, where ${\alpha_{i}}$ are scaling factors.

%%%%%%%%%%%%%%%%%%%%%%%%%%%%%%%%%%%%%%%%%%%%%%%% Subsection 1.1
\subsection{The regression matrix}
The ID-DMD representation is formulated as

\begin{equation}  
    \mathbf{x}_{k}=(\mathbf{A}_{0}+{\bar\varepsilon_{1}}{\mathbf{A}_{1}}+\cdots +{\bar\varepsilon_{P}}{\mathbf{A}_{P}}){\mathbf{x}_{k-1}} \label{eqS1}
\end{equation}

\noindent or in a matrix form

\begin{equation}  
    \mathbf{X'}=(\mathbf{A}_{0}+{\bar\varepsilon_{1}}{\mathbf{A}_{1}}+\cdots +{\bar\varepsilon_{P}}{\mathbf{A}_{P}})\mathbf{X} \label{eqS2}
\end{equation}

\noindent where $\bar\varepsilon_{1},\ldots,\bar\varepsilon_{P}$, $P\in {\mathbb{Z}^{+}}$, are the scaled design parameters; $\mathbf{x}_{k}$ is the state vector or snapshot at the discrete time $k$, \\

\begin{equation*} 
    \mathbf{X'}=\left[\begin{matrix}
       | & | & |  \\
    \mathbf{x}_{2} & \mathbf{x}_{3} & \cdots   \\
       | & | & |  \\
    \end{matrix} \right] \ \text{and} \ \mathbf{X}=\left[\begin{matrix}
       | & | & |  \\
    \mathbf{x}_{1} & \mathbf{x}_{2} & \cdots   \\
       | & | & |  \\
    \end{matrix} \right]
\end{equation*} 

Eq.(\ref{eq2}) can be rewritten as

\begin{equation}
    \mathbf{X'}=\mathbf{\Theta E} \label{eqS3}
\end{equation}

\noindent where 

\begin{equation*}
    \mathbf{\Theta}=[\mathbf{A}_{0}\ \mathbf{A}_{1}\ \cdots \ \mathbf{A}_{P}] \ \text{and} \ \mathbf{E}=\left[\begin{matrix}
    \mathbf{X} \\
    {\bar\varepsilon_{1}}\mathbf{X} \\
    \vdots \\
    {\bar\varepsilon_{P}}\mathbf{X} \\
\end{matrix} \right]
\end{equation*}

For different sets of scaled design parameters $\{\bar\varepsilon_{1},\ldots ,\bar\varepsilon_{P}\}$, the ID-DMD shares the same linear operators $\mathbf{\Theta}$. As a result, a unified matrix representation for the ID-DMD can be expressed as: 

\begin{equation}
\mathbf{Z}=\mathbf{\Theta \Xi} \label{eqS4}
\end{equation}

\noindent where 

\begin{equation*}
    \mathbf{Z}=[{\mathbf{X'}_{(1)}}\ {\mathbf{X'}_{(2)}}\ \cdots]\in{\mathbb{R}^{m\times n}} \ \text{and} \ \mathbf{\Xi}=[{\mathbf{E}_{(1)}}\ {\mathbf{E}}_{(2)}\cdots]\in {\mathbb{R}^{(P+1)m\times n}}, m,n\in {\mathbb{Z}^{+}}
\end{equation*}

\noindent with ${\mathbf{X'}_{(l)}}$ and ${\mathbf{E}_{(l)}}$ being matrices under the $l\in {\mathbb{Z}^{+}}$th set of design parameters.

%%%%%%%%%%%%%%%%%%%%%%%%%%%%%%%%%%%%%%%%%%%%%%%% Subsection 1.2
\subsection{Dimensional reduction}
Conducting the SVD of $\mathbf{Z}$ and $\mathbf{\Xi}$, yields

\begin{equation}
    \mathbf{Z} =\underbrace{\mathbf{\bar{U}}}_{m\times m}\mathbf{\bar\Sigma }\underbrace{\mathbf{\bar{V}}^\text{*}}_{m\times n}\approx \underbrace{\mathbf{U}}_{m\times {r}_{\text{Z}}}\mathbf{\Sigma }\underbrace{\mathbf{V}^\text{*}}_{{{r}_{\text{Z}}}\times n} \label{eqS5}
\end{equation}

\noindent and

\begin{equation}
    \underbrace{\mathbf{\Xi}}_{(P+1)m\times n}=\underbrace{\mathbf{\bar{U}}_{\Xi}}_{(P+1)m\times (P+1)m}{\mathbf{\bar\Sigma}_{\Xi}}{{\underbrace{\mathbf{\bar{V}}_{\Xi}}_{n\times n}}^\text{*}}\approx \underbrace{\mathbf{U}_{\Xi}}_{(P+1)m\times {r_{\Xi}}}{\mathbf{\Sigma}_{\Xi}}\underbrace{{\mathbf{V}_{\Xi}}^\text{*}}_{{{r}_{\Xi}}\times n} \label{eqS6}
\end{equation}

\noindent Here, “*” is the complex conjugate transpose. The truncated ranks ${r}_{\text{Z}}$ and ${r}_{\Xi}$ are hyper-parameters used in the identification of the ID-DMD. Their values, constrained by ${r}_{\text{Z}},{r}_{\Xi}\le m$, can be determined by using various techniques for hard and soft thresholding of singular values~\cite{gavish2014optimal,brunton2022data}. The matrix $\mathbf{U}_{\Xi}$ can be further organized as a block matrix:

\begin{equation*}
    \mathbf{U}_{\Xi}=\left[\begin{matrix}
    \underbrace{\mathbf{U}_{\Xi,0}}_{m\times {{r}_{\Xi}}} \\
    \underbrace{\mathbf{U}_{\Xi,1}}_{m\times {{r}_{\Xi}}} \\
    \vdots \\
    \underbrace{\mathbf{U}_{\Xi,P}}_{m\times {{r}_{\Xi}}} \\
    \end{matrix} \right]\in {\mathbb{C}^{(P+1)m\times {{r}_{\Xi}}}}
\end{equation*}

From Eq.(\ref{eqS4}), the linear operators can be evaluated as


\begin{equation}
    \underbrace{\mathbf{\Theta}}_{m\times (P+1)m}=\underbrace{\mathbf{Z}}_{m\times n}\underbrace{\mathbf{\Xi}^{\dagger}}_{n\times (P+1)m}=\underbrace{\mathbf{Z}}_{m\times n}\underbrace{\mathbf{V}_{\Xi}}_{n\times {{r}_{\Xi}}}{\mathbf{\Sigma }_{\Xi}}^{-1}\underbrace{{\mathbf{U}_{\Xi}}^{\text{*}}}_{{{r}_{\Xi}}\times (P+1)m}=\underbrace{\mathbf{Z}}_{m\times n}\underbrace{\mathbf{V}_{\Xi}}_{n\times {{r}_{\Xi}}}{{\mathbf{\Sigma }}_{\Xi}}^{-1}[\underbrace{{\mathbf{U}_{\Xi,0}}^\text{*}}_{{{r}_{\Xi}}\times m}\ \underbrace{{\mathbf{U}_{\Xi,1}}^\text{*}}_{{{r}_{\Xi}}\times m}\ \cdots \ \underbrace{{\mathbf{U}_{\Xi,P}}^\text{*}}_{{{r}_{\Xi}}\times m}] \label{eqS7}
\end{equation}

\noindent where in $\mathbf{\Theta}=[\mathbf{A}_{0}\ \mathbf{A}_{1}\ \cdots \ \mathbf{A}_{P}]$,

\begin{equation*}
    \mathbf{A}_{i}=\mathbf{Z}{\mathbf{V}_{\Xi}}{{\mathbf{\Sigma }}_{\Xi}}^{-1}{\mathbf{U}_{\Xi,i}}^\text{*},\ i\in \mathbb{Z}
\end{equation*}

Next, two Propositions will be proposed for the low-rank representation of ID-DMD. 

\textbf{Proposition 1:} The low-rank ID-DMD is obtained as

\begin{equation}
    \mathbf{\tilde{X}'}=({\mathbf{\tilde{A}}_{0}}+{\bar{\varepsilon}_{1}}{\mathbf{\tilde{A}}}_{1}+\cdots+{\bar{\varepsilon}_{P}}{\mathbf{\tilde{A}}_{P}})\mathbf{\tilde{X}} \label{eqS8}
\end{equation}

\noindent where $\mathbf{X}=\mathbf{U\tilde{X}}$, and

\begin{equation*}    
    \mathbf{\tilde{A}}_{i} =\underbrace{\mathbf{U}^\text{*}}_{{{r}_{\text{Z}}}\times m}{\mathbf{A}_{i}}\underbrace{\mathbf{U}}_{m\times {{r}_{\text{Z}}}}={\mathbf{U}^\text{*}}\mathbf{Z}{\mathbf{V}_{\Xi}}{\mathbf{\Sigma }_{\Xi}}^{-1}{\mathbf{U}_{\Xi,i}}^\text{*}\mathbf{U},\ i\in \mathbb{Z}
\end{equation*}

\textbf{Proof 1:} The ID-DMD can be projected to a low-rank space through the eigenvectors $\mathbf{U}$ as

\begin{equation}
    \begin{aligned}
    & \mathbf{X'}=({\mathbf{A}_{0}}+{{\bar{\varepsilon}}_{1}}{\mathbf{A}_{1}}+\cdots+{{\bar{\varepsilon }}_{P}}{\mathbf{A}_{P}})\mathbf{X} \\ 
    & \Rightarrow \mathbf{U\tilde{X}'}=({\mathbf{A}_{0}}+{{\bar{\varepsilon }}_{1}}{\mathbf{A}_{1}}+\cdots+{{\bar{\varepsilon}}_{P}}{\mathbf{A}_{P}})\mathbf{U\tilde{X}} \\ 
    & \Rightarrow {\mathbf{U}^{*}}\mathbf{U\tilde{X}'}={\mathbf{U}^{*}}({\mathbf{A}_{0}}+{{\bar{\varepsilon }}_{1}}{\mathbf{A}_{1}}+\cdots+{{\bar{\varepsilon}}_{P}}{\mathbf{A}_{P}})\mathbf{U\tilde{X}} \\ 
    & \Rightarrow \mathbf{\tilde{X}'}=({\mathbf{U}^{*}}{\mathbf{A}_{0}}\mathbf{U}+{{\bar{\varepsilon}}_{1}}{\mathbf{U}^{*}}{\mathbf{A}_{1}}\mathbf{U}+\cdots+{{\bar{\varepsilon}}_{P}}{\mathbf{U}^{*}}{\mathbf{A}_{P}}\mathbf{U})\mathbf{\tilde{X}} \\ 
    & \Rightarrow \mathbf{\tilde{X}'}=({{\mathbf{\tilde{A}}}_{0}}+{{\bar{\varepsilon }}_{1}}{{\mathbf{\tilde{A}}}_{1}}+\cdots+{{\bar{\varepsilon}}_{P}}{{\mathbf{\tilde{A}}}_{P}})\mathbf{\tilde{X}} \\ 
    \end{aligned} \label{eqS9}
\end{equation}

Here, it’s worth noting that ${\mathbf{U}^\text{*}}\mathbf{U}=\mathbf{I}$ while $\mathbf{U}{\mathbf{U}^\text{*}}\ne \mathbf{I}$.

\textbf{Proposition 2:} The eigenvalues of $\mathbf{\tilde{A}}_{0}+{\bar{\varepsilon}_{1}}{\mathbf{\tilde{A}}_{1}}+\cdots+{\bar{\varepsilon}_{P}}{\mathbf{\tilde{A}}_{P}}$ are the same as those of ${\mathbf{A}_{0}}+{\bar{\varepsilon}_{1}}{\mathbf{A}_{1}}+\cdots+{\bar{\varepsilon}_{P}}{\mathbf{A}_{P}}$. The eigenvectors of the full-rank operator $\mathbf{\Phi}$ are reconstructed by either the exact DMD method~\cite{tu2013dynamic,brunton2022data}:

\begin{equation}
    \mathbf{\Phi}=\mathbf{Z}{\mathbf{V}_{\Xi}}{\mathbf{\Sigma}_{\Xi}}^{-1}({\mathbf{U}_{\Xi,0}}^\text{*}+{\bar{\varepsilon}_{1}}{\mathbf{U}_{\Xi,1}}^\text{*}+\cdots+{\bar{\varepsilon}_{P}}{\mathbf{U}_{\Xi,P}}^\text{*})\mathbf{UW} \label{eqS10}
\end{equation}

\noindent or the projected DMD method ~\cite{schmid2010dynamic}

\begin{equation}
    \mathbf{\Phi}=\mathbf{UW} \label{eqS11}
\end{equation}

\noindent where $\mathbf{W}$ is the eigenvector of the low-rank operator $\mathbf{\tilde{A}}_{0}+{\bar{\varepsilon}_{1}}{\mathbf{\tilde{A}}_{1}}+\cdots+{\bar{\varepsilon}_{p}}{\mathbf{\tilde{A}}_{p}}$.

\textbf{Proof 2:} Considering the columns of $\mathbf{U}$ span the column space $\mathbf{Z}$, there is~\cite{tu2013dynamic}

\begin{equation}
    {\mathbf{U}^\text{*}}\mathbf{Z}\approx {\mathbf{U}^\text{*}}\mathbf{U\Sigma}{\mathbf{V}^\text{*}}=\mathbf{\Sigma}{\mathbf{V}^\text{*}}\Rightarrow \mathbf{U}{\mathbf{U}^\text{*}}\mathbf{Z}\approx \mathbf{U\Sigma}{\mathbf{V}^\text{*}}=\mathbf{Z} \label{eqS12}
\end{equation}

The eigenvalue decomposition of the low-rank operator can be expressed as $(\mathbf{\tilde{A}}_{0}+{\bar{\varepsilon}_{1}}{\mathbf{\tilde{A}}_{1}}+\cdots+{\bar{\varepsilon}_{P}}{\mathbf{\tilde{A}}_{P}})\mathbf{W}=\mathbf{W\Lambda}$, where $\mathbf{\Lambda}$ contains the eigenvalues, and $\mathbf{W}$ is the matrix of corresponding eigenvectors. 

Now, define the matrix $\mathbf{\Phi}$ (the exact DMD) as:

\begin{equation*}
    \mathbf{\Phi}=\mathbf{Z}{\mathbf{V}_{\Xi}}{\mathbf{\Sigma}_{\Xi}}^{-1}({\mathbf{U}_{\Xi,0}}^\text{*}+{\bar{\varepsilon}_{1}}{\mathbf{U}_{\Xi,1}}^\text{*}+\cdots+{\bar{\varepsilon}_{P}}{\mathbf{U}_{\Xi,P}}^\text{*})\mathbf{UW}
\end{equation*}

\noindent there is 

\begin{equation}
    \begin{aligned}
    & (\mathbf{A}_{0}+{\bar{\varepsilon}_{1}}{\mathbf{A}_{1}}+\cdots+{\bar{\varepsilon}_{P}}{\mathbf{A}_{P}})\mathbf{\Phi}=[\mathbf{Z}{\mathbf{V}_{\Xi}}{\mathbf{\Sigma}_{\Xi}}^{-1}({\mathbf{U}_{\Xi,0}}^\text{*}+{\bar{\varepsilon}_{1}}{\mathbf{U}_{\Xi,1}}^\text{*}+\cdots+{\bar{\varepsilon}_{P}}{\mathbf{U}_{\Xi,P}}^\text{*})]\times  \\ 
    & [\mathbf{Z}{\mathbf{V}_{\Xi}}{\mathbf{\Sigma}_{\Xi}}^{-1}({\mathbf{U}_{\Xi,0}}^\text{*}+{\bar{\varepsilon}_{1}}{\mathbf{U}_{\Xi,1}}^\text{*}+\cdots+{\bar{\varepsilon}_{P}}{\mathbf{U}_{\Xi,P}}^\text{*})\mathbf{UW}] \\ 
    & \approx [\mathbf{Z}{\mathbf{V}_{\Xi}}{\mathbf{\Sigma}_{\Xi}}^{-1}({\mathbf{U}_{\Xi,0}}^\text{*}+{\bar{\varepsilon}_{1}}{\mathbf{U}_{\Xi,1}}^\text{*}+\cdots+{\bar{\varepsilon}_{P}}{\mathbf{U}_{\Xi,P}}^\text{*})]\times \\ 
    & [\mathbf{U}\underbrace{{\mathbf{U}^\text{*}}\mathbf{Z}{\mathbf{V}_{\Xi}}{\mathbf{\Sigma}_{\Xi}}^{-1}({\mathbf{U}_{\Xi,0}}^\text{*}+{\bar{\varepsilon}_{1}}{\mathbf{U}_{\Xi,1}}^\text{*}+\cdots+{\bar{\varepsilon}_{P}}{\mathbf{U}_{\Xi,P}}^\text{*})\mathbf{U}}_{{\mathbf{U}^\text{*}}({\mathbf{A}_{0}}+{\bar{\varepsilon}_{1}}{\mathbf{A}_{1}}+\cdots+{\bar{\varepsilon}_{P}}{\mathbf{A}_{P}})\mathbf{U}=\mathbf{\tilde{A}}_{0}+{\bar{\varepsilon}_{1}}{\mathbf{\tilde{A}}_{1}}+\cdots+{\bar{\varepsilon}_{P}}{{\mathbf{\tilde{A}}}_{P}}}\mathbf{W}] \\ 
    & =\underbrace{[\mathbf{Z}{\mathbf{V}_{\Xi}}{\mathbf{\Sigma}_{\Xi}}^{-1}({\mathbf{U}_{\Xi,0}}^\text{*}+{\bar{\varepsilon}_{1}}{\mathbf{U}_{\Xi,1}}^\text{*}+\cdots+{\bar{\varepsilon}_{P}}{\mathbf{U}_{\Xi,P}}^\text{*})][\mathbf{UW}}_{\mathbf{\Phi}}\mathbf{\Lambda}]=\mathbf{\Phi \Lambda} \\ 
    \end{aligned} \label{eqS13}
\end{equation}

On the other hand, denote $\mathbf{\Phi }=\mathbf{UW}$ (the projected DMD), since $\mathbf{U}{\mathbf{U}^\text{*}}\mathbf{Z}\approx \mathbf{Z}$, there is

\begin{equation*}
    \begin{aligned}
    & \mathbf{A}_{0}+{\bar{\varepsilon}_{1}}{\mathbf{A}_{1}}+\cdots +{\bar{\varepsilon}_{P}}{\mathbf{A}_{P}} \\ 
    & =\mathbf{Z}{\mathbf{V}_{\Xi}}{\mathbf{\Sigma}_{\Xi}}^{-1}{\mathbf{U}_{\Xi,0}}^\text{*}+{\bar{\varepsilon}_{1}}\mathbf{Z}{\mathbf{V}_{\Xi}}{\mathbf{\Sigma}_{\Xi}}^{-1}{\mathbf{U}_{\Xi,1}}^\text{*}+\cdots +{\bar{\varepsilon}_{P}}\mathbf{Z}{\mathbf{V}_{\Xi}}{\mathbf{\Sigma}_{\Xi}}^{-1}{\mathbf{U}_{\Xi,P}}^\text{*} \\ 
    & \approx (\mathbf{U}{\mathbf{U}^\text{*}}\mathbf{Z}{\mathbf{V}_{\Xi}}{\mathbf{\Sigma}_{\Xi}}^{-1}{\mathbf{U}_{\Xi,0}}^\text{*}+{\bar{\varepsilon}_{1}}\mathbf{U}{\mathbf{U}^\text{*}}\mathbf{Z}{\mathbf{V}_{\Xi}}{\mathbf{\Sigma}_{\Xi}}^{-1}{\mathbf{U}_{\Xi,1}}^\text{*}+\cdots +{\bar{\varepsilon}_{P}}\mathbf{U}{\mathbf{U}^\text{*}}\mathbf{Z}{\mathbf{V}_{\Xi}}{\mathbf{\Sigma}_{\Xi}}^{-1}{\mathbf{U}_{\Xi,P}}^\text{*}) \\ 
    & =\mathbf{U}{\mathbf{U}^\text{*}}({\mathbf{A}_{0}}+{\bar{\varepsilon}_{1}}{\mathbf{A}_{1}}+\cdots+{\bar{\varepsilon}_{P}}{\mathbf{A}_{P}}) \\ 
    \end{aligned}
\end{equation*}

Then we have

\begin{equation}
    \begin{aligned}
    & ({\mathbf{A}_{0}}+{\bar{\varepsilon}_{1}}{\mathbf{A}_{1}}+\cdots+{\bar{\varepsilon}_{P}}{\mathbf{A}_{P}})\mathbf{\Phi}=({\mathbf{A}_{0}}+{\bar{\varepsilon}_{1}}{\mathbf{A}}_{1}+\cdots+{\bar{\varepsilon}_{P}}{\mathbf{A}_{P}})\mathbf{UW} \\ 
    & =\mathbf{U}\underbrace{{\mathbf{U}^\text{*}}({\mathbf{A}_{0}}+{\bar{\varepsilon}_{1}}{\mathbf{A}_{1}}+\cdots+{\bar{\varepsilon}_{P}}{\mathbf{A}_{P}})\mathbf{U}}_{\mathbf{\tilde{A}_{0}}+{\bar{\varepsilon}_{1}}{\mathbf{\tilde{A}}_{1}}+\cdots+{\bar{\varepsilon}_{P}}{\mathbf{\tilde{A}}_{P}}}\mathbf{W}=\mathbf{UW\Lambda}=\mathbf{\Phi \Lambda} \\ 
    \end{aligned} \label{eqS14}
\end{equation}

%%%%%%%%%%%%%%%%%%%%%%%%%%%%%%%%%%%%%%%%%%%%%%%% Subsection 1.3
\subsection{Reconstruction and prediction}
The predicted output response is

\begin{equation}
    \mathbf{x}_{k}=\sum\limits_{j\in \mathbb{Z}^{+}}{{\bm{\upphi}_{j}}{\text{e}^{{{s}_{j}}(k-1)}}{{b}_{j}}}=\mathbf{\Phi}\exp [\mathbf{S}(k-1)]\mathbf{b} \label{eqS16}
\end{equation}

\noindent where $\mathbf{b}={\mathbf{\Phi}^{\dagger}}{\mathbf{x}_{1}}$, ${\mathbf{x}_{1}}$ represents the initial states of the system; $\mathbf{S}$ is a diagonal matrix containing the elements ${s}_{j}={\sigma_{j}}+\text{j}{\omega_{j}}$ with ${\omega_{j}}$ being the characteristic frequency and ${\sigma_{j}}$ being the decay rate is the continuous time eigenvalues, evaluated as 

\begin{equation}
    {s}_{j}=\Delta {t}^{-1}\log ({\mathbf{\Lambda}_{(j,j)}}) \label{eqS15}
\end{equation}

\noindent where $\mathbf{\Lambda}_{(j,j)}$ are the diagonal entries of the eigenvalues, $\Delta t$ is the sampling time.

In physics, if the complex frequency has a positive real part, the system becomes unstable and diverges. To prevent this, we set ${\sigma_{j}}=0$ if ${\sigma_{j}}>0$, ensuring stability.

For model validation, the relative error~\cite{tofallis2015better} at each discrete time step $k$ is defined as

\begin{equation}
    \bm{\upeta}_{k}=\frac{\left| \mathbf{x}_{k,\text{Prediction}}-\mathbf{x}_{k,\text{True}} \right|}{\max (\left| \mathbf{x}_{k,\text{True}} \right|)}\times 100\%  \label{eqS17}
\end{equation}

\noindent and the total relative error (applied to Tab.I) is defined as the mean values of $\bm{\upeta}_{k}$ over time. The relative error is an effective measure for evaluating errors of varying amplitudes over time. 

%%%%%%%%%%%%%%%%%%%%%%%%%%%%%%%%
%%%%%%%%%%%%%%%%%%%%%%%%%%%%%%%% SECTION 2
%%%%%%%%%%%%%%%%%%%%%%%%%%%%%%%%

\section{Section 2: The pitched airfoil}
The pitched airfoil under investigation is simplified to an elliptical shape with a dimensionless length of 0.6 and a width of 0.3. The inlet airflow speed is set to 1, corresponding to a Reynolds number of 150. The simulation is performed using the Lattice Boltzmann Method~\cite{aidun2010lattice}. The configuration and simulation setup of the pitched airfoil are illustrated in Fig.\ref{S1}.

%%%%%%%%%%%%%%%%%%%%%%%%%%%%%%%% FIGURE S1
\begin{figure}[!htb]
  \centering
  \includegraphics[width=0.45\textwidth]{FigS1.png}
  \caption{
 Configuration for the vorticity simulation of a pitched airfoil.
  }
  \label{S1}
\end{figure}

In this example, random noise with a zero mean and a standard deviation of 0.3 (approximately 15\% noise) is added to the measurements (Fig.1(a)). The resulting ID-DMD for the pitched airfoil is (Fig.1(b))

\begin{equation}
    \mathbf{x}_{k}=(\mathbf{A}_{0}+\theta {\mathbf{A}_{1}}){\mathbf{x}_{k-1}} \label{eqS18}
\end{equation}

\noindent where the steady-state training data are across $\theta =\{2, 4, 6, 8, 10, 12\}{}^\circ$ over $t\in [20,50]\ \text{s}$ with the sampling time $\Delta t=0.5\ \text{s}$.

For the ID-DMD identification, we set the hyper-parameters ${r}_{\text{Z}}={r}_{\Xi}=120$, with a scaling factor of $\alpha =1$, to achieve optimal modeling and fitting performance. The ID-DMD predictions of the vorticity at, i.e. $\theta =5^\circ$, are shown in Fig.1(c) and detailed results are shown in Fig.\ref{S2}.

%%%%%%%%%%%%%%%%%%%%%%%%%%%%%%%% FIGURE S2
\begin{figure}[!htb]
  \centering
  \includegraphics[width=0.75\textwidth]{FigS2.png}
  \caption{
  Prediction of the vorticity distribution around the airfoil.
  }
  \label{S2}
\end{figure}

The design objective is to minimize the vorticity power corresponding to a specific wavelength. This objective is formulated as the following optimization problem (Fig.1(d)): 

\begin{equation}
    \theta =\arg \min ({P}_\text{air})\ \ \text{s}\text{.t}\text{.}\ \left\{\begin{aligned}
    & \lambda \ge 3.35 \\ 
    & \theta \in {\mathbb{Z}^{+}} \\ 
    \end{aligned} \right. \label{eqS19}
\end{equation}

Here, the power ${P}_\text{air}$ is defined as

\begin{equation}
    {P}_\text{air}=\frac{1}{N}\sqrt{\sum\limits_{x,y,k}{{\mathbf{x}_{k}}^{2}}} \label{eqS20}
\end{equation}

\noindent where $x,y$ are coordinates of the flow field over the blue dotted box in Fig.\ref{S1}. $N$ is the number of snapshots.

In addition, the uncertainties in the power predictions are quantified using the bagging method. In this process, we randomly select half the columns ($n/2$) of $\mathbf{Z}$ and $\mathbf{\Xi}$ to evaluate the eigenvalues and eigenvectors. Following the identification procedure of the ID-DMD, we repeat this process 30 times, and the statistical results of the power ${P}_\text{air}$ are shown in Fig.1(e). The optimal pitched angle is designed as $\theta =7^\circ$.

%%%%%%%%%%%%%%%%%%%%%%%%%%%%%%%%
%%%%%%%%%%%%%%%%%%%%%%%%%%%%%%%% SECTION 3
%%%%%%%%%%%%%%%%%%%%%%%%%%%%%%%%
\section{Section 3: ID-DMD representation of complex dynamic systems}

In this section, we provide a detailed explanation of all the examples presented in Tab.I. The 1-D Burgers' equation will be discussed separately in Section IV to facilitate a comparison of advanced machine learning methods.

%%%%%%%%%%%%%%%%%%%%%%%%%%%%%%%%%%%%%%%%%%%%%%%% Subsection 3.1
\subsection{Nonlinearly damped building}
Consider a nonlinearly damped 4-Degree-of-Freedom (4-DoF) building system as illustrated in Fig.\ref{S3}~\cite{zhu2022design}. This system is governed by the ODE as 

\begin{equation}
    \mathbf{M\ddot{x}}+ \mathbf{C\dot{x}}+\mathbf{Kx}+{{\mathbf{F}}_{\text{non}}}=\mathbf{0}  \label{eqS21}
\end{equation}

\noindent with

\begin{equation*}
    \mathbf{x}=\left[\begin{matrix}
    {{x}_{1}}(t)  \\
    {{x}_{2}}(t)  \\
    {{x}_{3}}(t)  \\
    {{x}_{4}}(t)  \\
    \end{matrix} \right],\ \mathbf{M}=\left[\begin{matrix}
    {m}_{1} & 0 & 0 & 0  \\
    0 & {m}_{2} & 0 & 0  \\
    0 & 0 & {m}_{3} & 0  \\
    0 & 0 & 0 & {m}_{4}  \\
    \end{matrix} \right],\ \mathbf{C}=\left[\begin{matrix}
    {{c}_{1}}+{{c}_{2}} & -{c}_{2} & 0 & 0  \\
    -{c}_{2} & {{c}_{2}}+{{c}_{3}} & -{c}_{3} & 0  \\
    0 & -{c}_{3} & {{c}_{3}}+{{c}_{4}} & -{c}_{4}  \\
    0 & 0 & -{c}_{4} & {c}_{4}  \\
    \end{matrix} \right]
\end{equation*}

\begin{equation*}
    \mathbf{K}=\left[ \begin{matrix}
    {{k}_{1}}+{{k}_{2}} & -{k}_{2} & 0 & 0  \\
    -{k}_{2} & {{k}_{2}}+{{k}_{3}} & -{k}_{3} & 0  \\
    0 & -{k}_{3} & {{k}_{3}}+{{k}_{4}} & -{k}_{4}  \\
    0 & 0 & -{k}_{4} & {k}_{4}  \\
    \end{matrix} \right],\ \text{and} \ {\mathbf{F}_\text{non}}=\left[ \begin{matrix}
    0  \\
    -{{c}_{\text{non}}}{{({\dot{x}_{3}}-{\dot{x}_{2}})}^{3}}  \\
    {{c}_\text{non}}{{({\dot{x}_{3}}-{\dot{x}_{2}})}^{3}}  \\
    0  \\
    \end{matrix} \right]
\end{equation*}

\noindent where the masses (Kg) of the four floors are defined as ${m}_{1}=5\times {{10}^{6}}$, ${m}_{2}=4\times {{10}^{6}}$, ${m}_{3}=3\times {{10}^{6}}$ and ${m}_{4}=2\times {{10}^{6}}$; The stiffness (N/m) of the building’s inter-floor connections are ${k}_{1}=1500\times {{10}^{6}}$, ${k}_{2}=2000\times {{10}^{6}}$, ${k}_{3}=3000\times {{10}^{6}}$ and ${k}_{4}=1000\times {{10}^{6}}$; The linear damping (Nm/s) for all stories are uniform and set to ${c}_{1}={c}_{2}={c}_{3}={c}_{4}=1\times {{10}^{5}}$. The building undergoes free vibration with initial displacements of ${{x}_{1}}(0)=1\ \text{m}$ and ${{x}_{2}}(0)={{x}_{3}}(0)={{x}_{4}}(0)=0$ and the initial velocities being all zero. The nonlinear damper ${{c}_\text{non}}\in [100,5000]\ \text{N}{\text{m}^{3}}/{\text{s}^{3}}$  placed between the masses ${m}_{2}$ and ${m}_{3}$ is the design parameter. 

\begin{figure}[!h]
  \centering
  \includegraphics[width=0.65\textwidth]{FigS3.png}
  \caption{
  Nonlinearly damped building structure and the ID-DMD prediction results.
  }
  \label{S3}
\end{figure}

To capture the system dynamics, a polynomial Koopman operator is applied. This expansion generates a set of augmented states with time delay, which is constructed to represent the dynamics of the system in a higher-dimensional space, allowing effective modeling and analysis of nonlinear behavior. The expanded states include polynomial terms up to the 8th degree:

\begin{equation*}
    \psi ({\mathbf{x}_{k}})={{[{{x}_{1}}(k),{{x}_{1}}(k-1),{{x}_{2}}(k),\ldots,{{x}_{4}}(k-1),{{x}_{2}}(k){{x}_{3}}(k),\ldots,{{x}_{2}}(k-1){{x}_{3}}^{7}(k),...]}^\text{T}}
\end{equation*}

The settings used for ID-DMD identification are listed in Tab.\ref{tab.S1}. The predicted responses at ${c}_\text{non}=500$ are shown in Fig.\ref{S3}.

%%%%%%%%%%%%%%%%%%%%%%%%%%%%%%%% TABLE S1
\linespread{1.2}
\begin{table*}[!h] 
    \centering    
\noindent
\caption{ID-DMD settings for the nonlinearly damped building}
\label{tab.S1}

    \begin{tabular}{|p{4cm}|p{9cm}|}

\hline
\makecell[l] {Training parameter} & \makecell[l] {${{c}_{\text{non}}}=\{\text{0}\text{.1, 0}\text{.8, 3, 5}\}\times {{10}^{3}}$} \\

\hline
\makecell[l] {Time period} & \makecell[l] {$t\in [0,150]\ \text{s}$} \\

\hline
\makecell[l] {Sampling time} & \makecell[l] {$\Delta t=1/64\ \text{s}$} \\

\hline
\makecell[l] {Hyper-parameters} & \makecell[l] {${r}_\text{Z}={r}_{\Xi}=100$} \\

\hline
\makecell[l] {Scaling factor for ${c}_\text{non}$} & \makecell[l] {$\alpha =0.001$} \\

\hline
\makecell[l] {ID-DMD} & \makecell[l] {$\psi (\mathbf{x}_{k})=({\mathbf{A}_{\kappa,0}}+{{c}_\text{non}}{\mathbf{A}_{\kappa,1}})\psi (\mathbf{x}_{k-1})$} \\

\hline
\end{tabular}
\end{table*}
\linespread{1}

%%%%%%%%%%%%%%%%%%%%%%%%%%%%%%%%%%%%%%%%%%%%%%%% Subsection 3.2
\subsection{Van De Pol equation}
The Van De Pol equation is 

\begin{equation}
    \ddot{x}-\mu (1-{x}^{2})\dot{x}+\bar{\omega}x=0 \label{eqS22}
\end{equation}

\noindent with the initial condition $x(0)=0.1$ and $\dot{x}(0)=0$. The variables $\mu \in [0.8,1.2]$ and $\bar{\omega}\in [0.8,1.2]$ serve as the two design parameters in this analysis. Here, a polynomial Koopman operator up to an order of 11 with 77 observables is applied to capture the steady states of the Van De Pol equation: 

\begin{equation*}
    \psi (\mathbf{x}_{k})={[x(k),x(k-1),{{x}^{2}}(k),\ldots ,{{x}^{11}}(k-1)]^\text{T}}
\end{equation*}

The settings used for ID-DMD identification are listed in Tab.\ref{tab.S2}. The predicted responses at $(\mu,\bar{\omega})=(1,1.1)$ are shown in Tab.I. 

%%%%%%%%%%%%%%%%%%%%%%%%%%%%%%%% TABLE S2
\linespread{1.2}
\begin{table*}[!ht] 
    \centering    
\noindent
\caption{ID-DMD settings for the Van De Pol equation}
\label{tab.S2}

    \begin{tabular}{|p{4cm}|p{9cm}|}

\hline
\makecell[l] {Training parameter} & \makecell[l] {$\begin{aligned}
  & (\mu,\bar{\omega})=\{\text{(0}\text{.8,0}\text{.8), (0}\text{.8,1), (0}\text{.8,1}\text{.2),}\ \text{(0}\text{.9,0}\text{.9),}\,\text{(0}\text{.9,1}\text{.1),}\ \text{(1,0}\text{.8),} \\ 
 & \text{(1,1),}\ \text{(1,1}\text{.2),}\ \text{(1}\text{.1,0}\text{.9),}\ \text{(1}\text{.1,1}\text{.1),}\ \text{(1}\text{.2,0}\text{.8),}\ \text{(1}\text{.2,1),}\ \text{(1}\text{.2,1}\text{.2)}\} \\ 
\end{aligned}$} \\

\hline
\makecell[l] {Time period} & \makecell[l] {$t\in [30,200]\ \text{s}$} \\

\hline
\makecell[l] {Sampling time} & \makecell[l] {$\Delta t=1/32\ \text{s}$} \\

\hline
\makecell[l] {Hyper-parameters} & \makecell[l] {${r}_\text{Z}={r}_{\Xi}=70$} \\

\hline
\makecell[l] {Scaling factor for $(\mu,\bar{\omega})$} & \makecell[l] {$(\alpha_{1},\ \alpha_{2})=(1,\ 1)$} \\

\hline
\makecell[l] {ID-DMD} & \makecell[l] {$\psi (\mathbf{x}_{k})=(\mathbf{A}_{\kappa,0}+\mu {\mathbf{A}_{\kappa,1}}+\bar{\omega}{\mathbf{A}_{\kappa,2}})\psi (\mathbf{x}_{k-1})$} \\

\hline
\end{tabular}
\end{table*}
\linespread{1}

%%%%%%%%%%%%%%%%o%%%%%%%%%%%%%%%%%%%%%%%%%%%%%%%% Subsection 3.3
\subsection{Incident-jet flow}
The incident-jet flow~\cite{huhn2023parametric} is represented by the governing equation

\begin{equation}
    \frac{\partial T}{\partial t}+w\nabla T=\nabla \cdot {{k}_{\text{t}}}\nabla T \label{eqS23}
\end{equation}

\noindent where the simulation domain is defined as $\{(x,y)\in [0,5]\otimes [0,5]\}$. At the boundary, the temperature is set as $T(x<0.3,y<0.3,t)=1$. The advection velocity is $w=3$, and the thermal conductivity ${{k}_\text{t}}\in [0.2,0.8]$ is the design parameter. The settings for ID-DMD identification are provided in Tab.\ref{tab.S3}. The prediction results for ${k}_\text{t}=0.3$ are illustrated in Fig.\ref{S4}.

%%%%%%%%%%%%%%%%%%%%%%%%%%%%%%%% TABLE S3
\linespread{1.2}
\begin{table*}[!ht] 
    \centering    
\noindent
\caption{ID-DMD settings for the incident-jet flow}
\label{tab.S3}

    \begin{tabular}{|p{4cm}|p{9cm}|}

\hline
\makecell[l] {Training parameter} & \makecell[l] {${k}_\text{t}=\{0.2,0.4,0.6,0.8\}$} \\

\hline
\makecell[l] {Time period} & \makecell[l] {$t\in [0,2]\ \text{s}$} \\

\hline
\makecell[l] {Sampling time} & \makecell[l] {$\Delta t=0.025\ \text{s}$} \\

\hline
\makecell[l] {Hyper-parameters} & \makecell[l] {${r}_\text{Z}={r}_{\Xi}=150$} \\

\hline
\makecell[l] {Scaling factor for ${k}_\text{t}$} & \makecell[l] {$\alpha =1$} \\

\hline
\makecell[l] {ID-DMD} & \makecell[l] {$\mathbf{x}_{k}=(\mathbf{A}_{0}+{{k}_\text{t}}{\mathbf{A}_{1}}){\mathbf{x}_{k-1}}$} \\

\hline
\end{tabular}
\end{table*}
\linespread{1}

\begin{figure}[!ht]
  \centering
  \includegraphics[width=0.65\textwidth]{FigS4.png}
  \caption{
  Prediction of the incident-jet flow.
  }
  \label{S4}
\end{figure}

%%%%%%%%%%%%%%%%%%%%%%%%%%%%%%%%%%%%%%%%%%%%%%%% Subsection 3.4
\subsection{Cavity flow}
The PDE of the Cavity flow is represented by a Navier-Stokes equation~\cite{bruneau20062d}:

\begin{equation}
    \left\{\begin{aligned}
    & \frac{\partial \mathbf{u}}{\partial t}+(\mathbf{u}\cdot \nabla)\mathbf{u}=-\nabla p+\frac{1}{{R}_\text{e}}{{\nabla}^{2}}\mathbf{u} \\ 
    & \nabla \mathbf{u}=0 \\ 
    \end{aligned} \right. \label{eqS24}
\end{equation}

\noindent where $p$ is the pressure. $\mathbf{u}(x,y,t)$ is the velocity field with $x,y\in [0,1]$. ${R}_\text{e}$ is the Reynold number. Here, the boundary conditions are $\mathbf{u}(x,y,t)=(0,0)$ at the left, bottom and right walls and $\mathbf{u}(x,y,t)=(v,0)$ on top, where $v$ is the driven velocity of the flow. The initial condition is $\mathbf{u}(x,y,0)=(0,0)$ as the flow starts from rest. 

The cavity flow design parameters are ${R}_\text{e}\in [400,800]$ and $v\in [0.4,0.8]$. The settings for ID-DMD identification are listed in Tab.\ref{tab.S4}. The ID-DMD test results for cavity flow at $({R}_\text{e},\ v)=(500,0.5)$ are shown in Fig.\ref{S5}.

%%%%%%%%%%%%%%%%%%%%%%%%%%%%%%%% TABLE S4
\linespread{1.2}
\begin{table*}[!ht] 
    \centering    
\noindent
\caption{ID-DMD settings for the Cavity flow}
\label{tab.S4}

    \begin{tabular}{|p{4cm}|p{9cm}|}

\hline
\makecell[l] {Training parameter} & \makecell[l] {$\begin{aligned}
  & (v,{R}_\text{e})=\{(0.4,400), (0.4,800), (0.5,700), (0.6,600),
  \\ 
 & (0.7,500), (0.8,400), (0.8,800)\} \\ 
\end{aligned}$} \\

\hline
\makecell[l] {Time period} & \makecell[l] {$t\in [0,30]\ \text{s}$} \\

\hline
\makecell[l] {Sampling time} & \makecell[l] {$\Delta t=0.2\ \text{s}$} \\

\hline
\makecell[l] {Hyper-parameters} & \makecell[l] {${r}_\text{Z}={r}_{\Xi}=60$} \\

\hline
\makecell[l] {Scaling factor for $(v, {{R}_{\text{e}}})$} & \makecell[l] {$(\alpha_{1},\alpha_{2})=(1,0.001)$} \\

\hline
\makecell[l] {ID-DMD} & \makecell[l] {$\mathbf{x}_{k}=(\mathbf{A}_{0}+v{\mathbf{A}_{1}}+{{R}_\text{e}}{\mathbf{A}_{2}}){\mathbf{x}_{k-1}}$} \\

\hline
\end{tabular}
\end{table*}
\linespread{1}

\begin{figure}[!ht]
  \centering
  \includegraphics[width=0.65\textwidth]{FigS5.png}
  \caption{
  Prediction of the cavity flow.
  }
  \label{S5}
\end{figure}

%%%%%%%%%%%%%%%%%%%%%%%%%%%%%%%%%%%%%%%%%%%%%%%% Subsection 3.5
\subsection{Smoke plume}
The smoke plume data used in this study is generated using the Python code available at

\url{https://github.com/Ceyron/machine-learning-and-simulation/tree/main/english/phiflow}. 

The primary design parameter is the dimensionless radius of the burner, denoted as ${r}_\text{d}$. Simulations are conducted for a range of ${{r}_{\text{d}}}\in [5,5.5]$, representing variations in burner size that influence the smoke plume dynamics. The simulation captures the intricate flow patterns associated with the smoke plume, providing a rich dataset for analysis. The ID-DMD identification method is applied to this dataset to model the system's dynamics, with detailed settings provided in Tab.\ref{tab.S5}. The testing results of the ID-DMD for the smoke plume at ${r}_\text{d}=5.2$ are shown in Fig.\ref{S6}.

%%%%%%%%%%%%%%%%%%%%%%%%%%%%%%%% TABLE S5
\linespread{1.2}
\begin{table*}[!ht] 
    \centering    
\noindent
\caption{ID-DMD settings for smoke plume}
\label{tab.S5}

    \begin{tabular}{|p{4cm}|p{9cm}|}

\hline
\makecell[l] {Training parameter} & \makecell[l] {${r}_\text{d}=\left\{ 5, 5.1, 5.4, 5.5 \right\}$} \\

\hline
\makecell[l] {Time period} & \makecell[l] {$t\in [0,150]\ \text{s}$} \\

\hline
\makecell[l] {Sampling time} & \makecell[l] {$\Delta t=1\ \text{s}$} \\

\hline
\makecell[l] {Hyper-parameters} & \makecell[l] {${r}_\text{Z}={r}_{\Xi}=350$} \\

\hline
\makecell[l] {Scaling factor for ${r}_\text{d}$} & \makecell[l] {$\alpha=0.1$} \\

\hline
\makecell[l] {ID-DMD} & \makecell[l] {$\mathbf{x}_{k}=(\mathbf{A}_{0}+{{r}_\text{d}}{\mathbf{A}_{1}}){\mathbf{x}_{k-1}}$} \\

\hline
\end{tabular}
\end{table*}
\linespread{1}

\begin{figure}[!ht]
  \centering
  \includegraphics[width=0.65\textwidth]{FigS6.png}
  \caption{
  Prediction of the smoke plume.
  }
  \label{S6}
\end{figure}

%%%%%%%%%%%%%%%%%%%%%%%%%%%%%%%%%%%%%%%%%%%%%%%% Subsection 3.6
\subsection{Droplet}
The inkjet nozzle is the core component of the droplet-based 3D printing process. The accurate implementation of various process parameters in droplet-based printing relies on the proper design and stable operation of the inkjet nozzle. The setup of the experiment is illustrated in Fig.\ref{S7}.

\begin{figure}[!ht]
  \centering
  \includegraphics[width=0.7\textwidth]{FigS7.png}
  \caption{
  Experimental setup for the droplet test. (a) Overview of the main components of the DSA Inkjet system. (b) Illustration of the experimental setup during operation. (c) Configuration of the ink control and supply system. (d) Droplet observation during the experiment.
  }
  \label{S7}
\end{figure}

For the experiment, we utilized the Droplet Shape Analyzer (DSA Inkjet) developed by Krüss (Germany) and the MH5420 inkjet nozzle manufactured by RICOH.

The goal was to control the driving voltage ${V}_\text{t}$ to produce a continuous stream of droplets. A high-speed camera was employed to precisely capture the morphology of the droplets at each moment. The experimental ink has a density of 1250 $\text{Kg/}{\text{m}^{3}}$, a viscosity of 10 cP and a surface tension coefficient of 0.04 $\text{N/m}$. 

In this study, the design parameter of interest was the driving voltage, ${V}_\text{t}$, which was varied across the range ${{V}_\text{t}}\in [15,30]\ \text{V}$. Settings for ID-DMD identification are provided in Tab.\ref{tab.S6}. The experimental results for a driving voltage of ${{V}_{\text{t}}}=21\ \text{V}$ are shown in Tab.I.

%%%%%%%%%%%%%%%%%%%%%%%%%%%%%%%% TABLE S6
\linespread{1.2}
\begin{table*}[!ht] 
    \centering    
\noindent
\caption{ID-DMD settings for the droplet test}
\label{tab.S6}

    \begin{tabular}{|p{4cm}|p{9cm}|}

\hline
\makecell[l] {Training parameter} & \makecell[l] {${V}_\text{t}=\left\{18, 24, 30 \right\}$} \\

\hline
\makecell[l] {Time period} & \makecell[l] {$t\in [0,1]\times {10}^{-4}\ \text{s}$} \\

\hline
\makecell[l] {Sampling time} & \makecell[l] {$\Delta t=1\ \upmu \text{s}$} \\

\hline
\makecell[l] {Hyper-parameters} & \makecell[l] {${r}_\text{Z}={r}_{\Xi}=200$} \\

\hline
\makecell[l] {Scaling factor for ${V}_\text{t}$} & \makecell[l] {$\alpha=0.01$} \\

\hline
\makecell[l] {ID-DMD} & \makecell[l] {$\mathbf{x}_{k}=(\mathbf{A}_{0}+{{V}_\text{t}}{\mathbf{A}_{1}}){\mathbf{x}_{k-1}}$} \\

\hline
\end{tabular}
\end{table*}
\linespread{1}

%%%%%%%%%%%%%%%%%%%%%%%%%%%%%%%%
%%%%%%%%%%%%%%%%%%%%%%%%%%%%%%%% SECTION 4
%%%%%%%%%%%%%%%%%%%%%%%%%%%%%%%%
\section{Section 4: Comparison with advanced machine learning methods}
For the Burgers' equation, the settings for the ID-DMD identification are provided in Tab.\ref{tab.S7}

%%%%%%%%%%%%%%%%%%%%%%%%%%%%%%%% TABLE S7
\linespread{1.2}
\begin{table*}[!ht] 
    \centering    
\noindent
\caption{ID-DMD settings for the Burgers' equation}
\label{tab.S7}

    \begin{tabular}{|p{4cm}|p{9cm}|}

\hline
\makecell[l] {Training parameter} & \makecell[l] {$v=\left\{0.014,0.022,0.030,0.038,0.046 \right\}$} \\

\hline
\makecell[l] {Time period} & \makecell[l] {$t\in [0,1]\ \text{s}$}\\

\hline
\makecell[l] {Sampling time} & \makecell[l] {$\Delta t=0.01\ \text{s}$}\\

\hline
\makecell[l] {Hyper-parameters} & \makecell[l] {${r}_\text{Z}={r}_{\Xi}=40$} \\

\hline
\makecell[l] {Scaling factor for ${V}_\text{t}$} & \makecell[l] {$\alpha=1$} \\

\hline
\makecell[l] {ID-DMD} & \makecell[l] {$\mathbf{x}_{k}=(\mathbf{A}_{0}+{v}{\mathbf{A}_{1}}){\mathbf{x}_{k-1}}$} \\

\hline
\end{tabular}
\end{table*}
\linespread{1}

To compare ID-DMD with advanced machine learning approaches, including Physics-Informed DeepONet (PI-DON), Physics-Informed Neural Networks (PINNs), Neural Implicit Flow (NIF), and Fourier Neural Operator (FNO), the same original training dataset $[x,t,v,s(x,t,v)]$ is used. The dataset is structured as a three-dimensional grid, where the spatial coordinates $x$ range from 0 to 1 in steps of 0.01, and the viscosity parameters $v$ take five specific values listed in Tab.\ref{tab.S7}, resulting in a dataset with dimensions $101\times 101\times 5$ and a total of 51015 data points. The Initial Conditions (IC), defined at $t=0$, consist of 505 data points ($101\times 5$), while the Boundary Conditions (BC), specified at $x=0$ and $x=1$, contribute 1010 data points ($101\times 5\times 2$). The extended architectures of parametric PI-DON, PINNs, NIF, and FNO, adapted to this dataset structure, are detailed in Fig.\ref{S8}, ensuring a consistent and fair comparison across methods.

\begin{figure}[!ht]
  \centering
  \includegraphics[width=0.9\textwidth]{FigS8.png}
  \caption{
  Extended architectures of parametric PI-DON, PINNs, NIF, and FNO.
  }
  \label{S8}
\end{figure}

\textbf{PI-DON} employs a dual-network architecture to process the input data, with distinct roles for the branch and trunk networks. The branch network is designed to capture the amplitudes $s(x, 0)$, corresponding to the initial conditions. We randomly sampled 2500 points from the remaining spatiotemporal domain for the training process. Simultaneously, the trunk network takes as input a matrix comprising the spatial coordinates $x$, time $t$, and viscosity parameter $v$, enabling the model to process the complete parametric information. The training process is governed by a physics-informed loss function, which incorporates residuals of the governing PDE along with terms ensuring adherence to the initial and boundary conditions. This ensures that the model respects the underlying physical laws during training. The PI-DON model contains 153000 trainable parameters and is trained over 20000 epochs using an initial learning rate of 0.001, which decays by a factor of 0.9 every 2000 epochs. This setup balances learning efficiency with model convergence, enabling the network to accurately capture the dynamics of the system.

\textbf{PINNs} take spatial coordinates $x$, time $t$, and viscosity $v$ as inputs, with the corresponding output data $s$ paired for supervised learning. The loss function integrates three key components: the IC, BC, and the residuals of the governing PDE. This ensures that the predicted solution remains consistent with the underlying physics and satisfies the specified constraints. For training, the input dataset consists of 100 randomly selected points from the IC and BC, combined with 10000 randomly sampled points from the interior spatiotemporal domain. These points are organized into a compact input matrix, enabling efficient processing. PINNs employ a minimalistic architecture with only 3041 trainable parameters, making the model lightweight and computationally efficient. Training is conducted over 20000 epochs with a fixed learning rate of 0.1, ensuring stable convergence while preserving computational.

\textbf{FNO} processes input data in a grid-based format, where each row of the input matrix consists of spatial coordinates $x$, time $t$, viscosity $v$, and the corresponding output $s$. This structured input allows FNO to operate effectively across both spatial and temporal domains, leveraging its unique ability to model complex operator mappings. The FNO employs Fourier transforms to represent and learn these mappings in the frequency domain. This approach not only makes the model resolution-invariant, allowing it to generalize across different grid sizes, but also enhances computational efficiency. The model’s loss function is designed to minimize the Mean Squared Error (MSE) between its predictions and the ground truth, ensuring accurate learning of the underlying dynamics.
In this example, the FNO features 52512001 trainable parameters. This large capacity allows it to capture the intricate dynamics of PDEs on discretized grids. The model is trained over 20000 epochs with a learning rate of 0.0005, enabling it to achieve stable convergence and precise results.

\textbf{NIF} utilizes an input format that separates spatial dependencies from parametric dependencies, enabling efficient modeling of complex systems. The architecture is composed of two key components: ShapeNet and ParameterNet. ShapeNet processes spatial coordinates $x$, effectively capturing spatial features, while ParameterNet encodes the temporal $t$ and viscosity $v$ parameters into a compact matrix format. This decoupling allows the model to efficiently handle spatiotemporal variations and parametric influences. The loss function is designed to minimize the MSE between the model’s predictions and the ground truth output $s$. Unlike physics-informed approaches, NIF focuses solely on data-driven accuracy, making it simple and computationally lightweight. With only 5883 trainable parameters, NIF is highly efficient and requires significantly less computational resources compared to larger models. The NIF is trained for 40000 epochs using a fixed learning rate of 0.001. This extended training period ensures convergence and enhances the model's ability to perform scalable and efficient dimensionality reduction for large-scale spatiotemporal problems.

The training details for each model are summarized in Tab.\ref{tab.S8}.

%%%%%%%%%%%%%%%%%%%%%%%%%%%%%%%% TABLE S8
\linespread{1.2}
\begin{table*}[!ht] 
    \centering    
\noindent
\caption{Training details for advanced machine learning approaches}
\label{tab.S8}

    \begin{tabular}{|p{2.5cm}|p{4cm}|p{5cm}|p{2cm}|p{2.5cm}|}

\hline
\makecell[c] {Methods} & \makecell[c] {Number of trainable \\ parameters} & \makecell[c]{Learning rate} & \makecell[c]{Total epoch}& \makecell[c]{Loss function}\\

\hline
\makecell[c] {PI-DON} & \makecell[c] {153,000} & \makecell[c] {Initial learning rate of 0.001, \\which decays by a factor of 0.9 \\every 2,000 steps} & \makecell[c] {20,000} & \makecell[c] {BC,IC,\\ PDE residual}\\

\hline
\makecell[c] {PINNs} & \makecell[c] {3,041} & \makecell[c] {0.1} & \makecell[c] {20000} & \makecell[c] {BC,IC,\\ PDE residual}\\

\hline
\makecell[c] {FNO} & \makecell[c] {52,512,001} & \makecell[c] {0.0005} & \makecell[c] {20,000} & \makecell[c] {MSE}\\

\hline
\makecell[c] {NIF} & \makecell[c] {5,883} & \makecell[c] {0.001} & \makecell[c] {20,000} & \makecell[c] {MSE}\\

\hline
\end{tabular}
\end{table*}
\linespread{1}

%%%%%%%%%%%%%%%%%%%%%%%%%%%%%%%%
%%%%%%%%%%%%%%%%%%%%%%%%%%%%%%%% SECTION 5
%%%%%%%%%%%%%%%%%%%%%%%%%%%%%%%%
\section{Section 5: Physical interpretability}
%%%%%%%%%%%%%%%%%%%%%%%%%%%%%%%%%%%%%%%%%%%%%%%% Subsection 5.1
\subsection{Linear building structure}
This section demonstrates how the ID-DMD can effectively represent complex dynamic systems while maintaining physical interpretability. The ID-DMD is inherently transparent and physically interpretable. Fig.\ref{S9} highlights the natural connections between the data-driven ID-DMD and a physical state-space model, using a 4-Degree-of-Freedom (4-DoF) linear building system as an example. The characteristic parameters of the building are detailed in Section III.A. Here, the bottom linear stiffness ${k}_\text{s}={k}_{1}$ is selected as the design parameter.

\begin{figure}[!ht]
  \centering
  \includegraphics[width=0.5\textwidth]{FigS9.png}
  \caption{
  The 4-DoF linear building system and pole placement design using the ID-DMD.
  }
  \label{S9}
\end{figure}

The physical state space model of the linear building system can be achieved from the ODE with $\mathbf{F}_\text{non}=\mathbf{0}$ in equation\ref{eqS21}:

\begin{equation}
    \left\{\begin{aligned}
    & {{m}_{1}}{\ddot{x}_{1}}+({c}_{1}+{c}_{2}){\dot{x}_{1}}-{{c}_{2}}{\dot{x}_{2}}+({k}_{1}+{k}_{2}){{x}_{1}}-{{k}_{2}}{{x}_{2}}=0 \\ 
    & {{m}_{2}}{\ddot{x}_{2}}-{{c}_{2}}{\dot{x}_{1}}+({c}_{2}+{c}_{3}){\dot{x}_{2}}-{{c}_{3}}{\dot{x}_{3}}-{{k}_{2}}{{x}_{1}}+({k}_{2}+{k}_{3}){{x}_{2}}-{{k}_{3}}{{x}_{3}}=0 \\ 
    & {{m}_{3}}{\ddot{x}_{3}}-{{c}_{3}}{\dot{x}_{2}}+({c}_{3}+{c}_{4}){\dot{x}_{3}}-{{c}_{4}}{\dot{x}_{4}}-{{k}_{3}}{{x}_{2}}+({k}_{3}+{k}_{4}){{x}_{3}}-{{k}_{4}}{{x}_{4}}=0 \\ 
    & {{m}_{4}}{\ddot{x}_{4}}-{{c}_{4}}{\dot{x}_{3}}+{{c}_{4}}{\dot{x}_{4}}-{{k}_{4}}{{x}_{3}}+{{k}_{4}}{{x}_{4}}=0 \\ 
    \end{aligned} \right. \label{eqS25}
\end{equation}

By using the central difference method

\begin{equation*}
    \dot{x}=\frac{x(k)-x(k-1)}{\Delta t}\ \text{and}\ \ddot{x}=\frac{x(k+1)-2x(k)+x(k-1)}{\Delta {t}^{2}}
\end{equation*}

\noindent there is

\begin{equation}
    \left\{\begin{aligned}
    & {{m}_{1}}{{x}_{1}}(k)=[2{{m}_{1}}-({c}_{1}+{c}_{2})\Delta t-({k}_{1}+{k}_{2})\Delta {t}^{2}]{{x}_{1}}(k-1)+[-{m}_{1}+({c}_{1}+{c}_{2})\Delta t]{{x}_{1}}(k-2) \\ 
    & \ \ \ \ +[{c}_{2}\Delta t+{k}_{2}\Delta {t}^{2}]{{x}_{2}}(k-1)+\Delta t{{c}_{2}}{{x}_{2}}(k-2) \\ 
    & {{m}_{2}}{{x}_{2}}(k)=[{c}_{2}\Delta t+{k}_{2}\Delta {t}^{2}]{{x}_{1}}(k-1)-{c}_{2}\Delta t{{x}_{1}}(k-2) \\ 
    & \ \ \ \ +[2{{m}_{2}}-({c}_{2}+{c}_{3})\Delta t-({k}_{2}+{k}_{3})\Delta {t}^{2}]{{x}_{2}}(k-1)+[-{m}_{2}+({c}_{2}+{c}_{3})\Delta t]{{x}_{2}}(k-2) \\ 
    & \ \ \ \ +[{c}_{3}\Delta t+{k}_{3}\Delta {t}^{2}]{{x}_{3}}(k-1)-{c}_{3}\Delta t{{x}_{3}}(k-2) \\ 
    & {{m}_{3}}{{x}_{3}}(k)=[{c}_{3}\Delta t+{k}_{3}\Delta {t}^{2}]{{x}_{2}}(k-1)-{c}_{3}\Delta t{{x}_{2}}(k-2) \\ 
    & \ \ \ \ +[2{m}_{3}-({c}_{3}+{c}_{4})\Delta t-({k}_{3}+{k}_{4})\Delta {t}^{2}]{{x}_{3}}(k-1)+[-{m}_{3}+({c}_{3}+{c}_{4})\Delta t]{{x}_{3}}(k-2) \\ 
    & \ \ \ \ +[{c}_{4}\Delta t+{k}_{4}\Delta {t}^{2}]{{x}_{4}}(k-1)-{c}_{4}\Delta t{{x}_{4}}(k-2) \\ 
    & {{m}_{4}}{{x}_{4}}(k)=[2{m}_{4}-{c}_{4}\Delta t-{k}_{4}\Delta {t}^{2}]{{x}_{4}}(k-1)+[-{m}_{4}+{c}_{4}\Delta t]{{x}_{4}}(k-2) \\ 
    & \ \ \ \ +[{c}_{4}\Delta t+{k}_{4}\Delta {t}^{2}]{{x}_{3}}(k-1)-{c}_{4}\Delta t{{x}_{3}}(k-2) \\ 
    \end{aligned} \right. \label{eqS26}
\end{equation}

\noindent which can be formulated as a matrix representation

\begin{equation}
    \left[\begin{matrix}
    {{x}_{1}}(k) \\
    {{x}_{1}}(k-1) \\
    {{x}_{2}}(k) \\
    \vdots \\
    \end{matrix} \right]=\left[ \begin{matrix}
    {a}_{1}-{{m}_{1}}^{-1}{k}_{s}\Delta {t}^{2} & {a}_{2} & {a}_{3} & \cdots \\
    1 & 0 & 0 & \cdots \\
    0 & 0 & {a}_{4} & \cdots \\
    \vdots  & \vdots  & \vdots  & \ddots \\
    \end{matrix} \right]\left[\begin{matrix}
    {{x}_{1}}(k-1) \\
    {{x}_{1}}(k-2) \\
    {{x}_{2}}(k-1) \\
    \vdots \\
    \end{matrix} \right] \label{eqS27}
\end{equation}

\noindent where ${a}_{1},{a}_{2},{a}_{3},{a}_{4},...$ are constants.

From the physical model to the data-driven model, the system’s structure remains consistent, highlighting the transparency and physical interpretability of the ID-DMD approach. The state vectors $\mathbf{x}_{k}$ are formulated from the building’s responses ${{x}_{1}}(k),{{x}_{2}}(k),{{x}_{3}}(k),{{x}_{4}}(k)$, along with their time-delay embeddings ${{x}_{i}}(k-1),{{x}_{i}}(k-2),...$, where $i=1,2,3,4$. Settings for ID-DMD identification are listed in Tab.\ref{tab.S9}. 

%%%%%%%%%%%%%%%%%%%%%%%%%%%%%%%% TABLE S9
\linespread{1.2}
\begin{table*}[!ht] 
    \centering    
\noindent
\caption{ID-DMD settings for the linear building}
\label{tab.S9}

    \begin{tabular}{|p{4cm}|p{9cm}|}

\hline
\makecell[l] {Training parameter} & \makecell[l] {${k}_\text{s}=\left\{1, 2, 3, 3.5 \right\}\times 10^9$} \\

\hline
\makecell[l] {Time period} & \makecell[l] {$t\in [0,8000]\ \text{s}$} \\

\hline
\makecell[l] {Sampling time} & \makecell[l] {$\Delta t=1/80\ \text{s}$} \\

\hline
\makecell[l] {Hyper-parameters} & \makecell[l] {${r}_\text{Z}={r}_{\Xi}=16$} \\

\hline
\makecell[l] {Scaling factor for ${c}_{3}$} & \makecell[l] {$\alpha=1\times 10^{-9}$} \\

\hline
\makecell[l] {ID-DMD} & \makecell[l] {$\mathbf{x}_{k}=({\mathbf{A}_{0}}+{{k}_\text{s}}{\mathbf{A}_{1}})\mathbf{x}_{k-1}$} \\

\hline
\end{tabular}
\end{table*}
\linespread{1}

This physical interpretability is demonstrated through pole placement in a 4-DoF linear building system. The desired first resonant frequency is set to $\omega_{\text{r,}1}=10\ \text{rad/s}$.  The ID-DMD-based design yields a linear stiffness of  ${k}_\text{s}=2.82\times {10}^{9}\ \text{N}/\text{m}$ $\pm 0.14\times {10}^{7}$. Key physical properties, such as mode shapes and higher-order resonant frequencies, are accurately evaluated as shown in Fig.\ref{S9}.

%%%%%%%%%%%%%%%%%%%%%%%%%%%%%%%%%%%%%%%%%%%%%%%% Subsection 5.2
\subsection{Nonlinearly damped ODE}
In this section, we explore the ID-DMD modeling of a nonlinearly damped differential equation

\begin{equation}
    \frac{{\text{d}^{2}}y}{\text{d}{t}^{2}}+0.03\frac{\text{d}y}{\text{d}t}+100y+{{c}_{3}}(\frac{\text{d}y}{\text{d}t})^{3}=0 \label{eqS28}
\end{equation}

\noindent with the initial condition $x(0)=0.01$ and $\dot{x}(0)=0$, where ${c}_{3}$ represents the nonlinear damping and serves as the design parameter with ${c}_{3}\in [1,20]$. The ID-DMD representation of the nonlinearly damped differential equation is $\psi(\mathbf{x}_{k})=(\mathbf{A}_{\kappa,0}+{{c}_{3}}{\mathbf{A}_{\kappa,1}})\psi(\mathbf{x}_{k-1})$, where the observables are obtained by the polynomial projection as

\begin{equation}
    \psi (\mathbf{x}_{k})={{[y(k),\ y(k-1),{{y}^{2}}(k),\ y(k)y(k-1),{{y}^{2}}(k-1),\ldots]}^{\text{T}}} \label{eqS29}
\end{equation}

\noindent up to the 8th order with a total number of 45 observables. Settings for ID-DMD identification are provided in Tab.\ref{tab.S10}.

%%%%%%%%%%%%%%%%%%%%%%%%%%%%%%%% TABLE S10
\linespread{1.2}
\begin{table*}[!ht] 
    \centering    
\noindent
\caption{ID-DMD settings for the nonlinearly damped ODE}
\label{tab.S10}

    \begin{tabular}{|p{4cm}|p{9cm}|}

\hline
\makecell[l] {Training parameter} & \makecell[l] {${c}_{3}=\left\{1, 10, 20 \right\}$} \\

\hline
\makecell[l] {Time period} & \makecell[l] {$t\in [0,200]\ \text{s}$} \\

\hline
\makecell[l] {Sampling time} & \makecell[l] {$\Delta t=1/32\ \text{s}$} \\

\hline
\makecell[l] {Hyper-parameters} & \makecell[l] {${r}_\text{Z}={r}_{\Xi}=35$} \\

\hline
\makecell[l] {Scaling factor for ${c}_{3}$} & \makecell[l] {$\alpha=0.1$} \\

\hline
\makecell[l] {ID-DMD} & \makecell[l] {$\psi(\mathbf{x}_{k})=(\mathbf{A}_{\kappa,0}+{{c}_{3}}{\mathbf{A}_{\kappa,1}})\psi(\mathbf{x}_{k-1})$} \\

\hline
\end{tabular}
\end{table*}
\linespread{1}

Dominant modes and their corresponding characteristic frequencies are extracted using the ID-DMD method, as shown in Fig.3. In Fig.3(a), the characteristic frequencies are evaluated across different design parameters. Frequencies that remain stable correspond to the true dominant modes, while those that vary with design parameters are identified as spurious modes. Compared to the stable dominant modes, spurious modes typically exhibit larger damping coefficients (i.e., larger real parts of the eigenvalues), causing their influence on system responses to decay rapidly. 

In Fig.3 (a) and (b), it is observed that the first mode originates from the linear states ($y(k)$, $y(k-1)$), corresponding to the characteristic frequency $\omega_\text{e}=10\ \text{rad/s}$. This can be easily verified via the ODE\ref{eqS28}, where the linear natural frequency is calculated as ${\omega_\text{r}}=10\ \text{rad/s}$.  The second mode arises from the squared projections of the states (${{y}^{2}}(k)$, $y(k)y(k-1)$, ${{y}^{2}}(k-1)$), occurring at the second-order modulation frequency ${\omega_\text{e}}=20\ \text{rad/s}$, which is twice the first-order frequency. The third stable mode appears at ${\omega_\text{e}}=30\ \text{rad/s}$, resulting from contributions of the linear states and their cubic projections (${{y}^{3}}(k)$, ${{y}^{2}}(k)y(k-1)$, $y(k){{y}^{2}}(k-1)$, ${{y}^{3}}(k-1)$). As illustrated in Fig.3(b) (iii), the modes associated with cubic projection states decrease as the nonlinear damping ${c}_{3}$ increases.

Figs.3(b) and (c) illustrate the frequency modulation characteristics of nonlinear systems~\cite{lang1996output}. In Fig.3(b), the linear states, representing the system output responses, contribute to both the first-order and third-order modes. As a result, when the system is excited by a single-tone input $u(t)=\cos(10t)$, the output response in Fig.3(c) contains the fundamental frequency and a third-order modulation component.

By introducing nonlinear damping, the specific energy loss ${E}_\text{d}=\sum\nolimits_{k=1}^{K}{{{\left| y(k) \right|}^{2}}}$ can be controlled without altering the system's settling time. As shown in Fig.3(d), an energy loss of ${{E}_\text{d}}<0.013$ is achieved when ${c}_{3}>6$. These properties are widely utilized in engineering applications for vibration control, but cannot be achieved by using linear damping.

