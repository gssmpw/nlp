\section{Related Studies}
User simulation is being used more often by task-oriented dialogue systems to test and develop dialogue techniques (\citet{gur2018user}). The Hidden Agenda Model, a user simulation that uses an organized method to replicate real-world user behaviors, was first presented by \citet{schatzmann2009hidden}. Our user simulation approach is greatly influenced by the approach presented in \citet{schatzmann2009hidden}. This is an influential work on modern dialogue systems, as well as user simulation. This model emphasizes the importance of a structured, goal-oriented user simulation, which has shaped our approach of creating detailed user profiles and behavior specifications utilizing a template. By integrating these structured elements into our user simulation, we ensure that the dialogue system can effectively manage and respond to complex user interactions in a task-oriented manner. Moreover, LLMs are integrated with these user simulations to improve the dialog system's capacity to produce responses that are human-like and to dynamically adjust to user preferences (\citet{abdullin2024synthetic}, \citet{hudevcek2023large}).