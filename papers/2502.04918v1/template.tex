% -*- Mode:TeX -*-
% LaTeX template for CinC papers                   v 1.1a 22 August 2010
%
% To use this template successfully, you must have downloaded and unpacked:
%       http://www.cinc.org/authors_kit/papers/latex.tar.gz
% or the same package in zip format:
%       http://www.cinc.org/authors_kit/papers/latex.zip
% See the README included in this package for instructions.
%
% If you have questions, comments or suggestions about this file, please
% send me a note!  George Moody (george@mit.edu)
%
\documentclass[onecolumn]{cinc}
\usepackage{graphicx}

\usepackage{comment}
\newcommand{\heading}[1]{\noindent\textbf{#1}}

\usepackage{xcolor}
\newcommand{\colnst}[1]{{\textcolor{red}{#1}}}
\newcommand{\coljma}[1]{{\textcolor{blue}{#1}}}

\usepackage{hyperref}

\begin{document}
\bibliographystyle{cinc}

\title{Explainable and externally validated machine learning for neuropsychiatric diagnosis via electrocardiograms}

\author {
Juan Miguel Lopez Alcaraz$^{1*}$, Ebenezer Oloyede$^{2,3}$, David Taylor$^{2,4}$,Wilhelm Haverkamp$^{5}$, Nils Strodthoff$^{1\dagger\#}$ \\ 
$^1$ AI4Health Division, Carl von Ossietzky Universität Oldenburg, Oldenburg, Germany. \\
\textit{juan.lopez.alcaraz@uol.de}, \textit{nils.strodthoff@uol.de} \\ 
$^2$ Pharmacy Department, Maudsley Hospital, London, UK. \\
\textit{ebenezer.oloyede@maudsley.nhs.uk} \\
$^3$ Department of Psychiatry, University of Oxford, UK. \\
$^4$ Institute of Pharmaceutical Sciences, King's College London, UK. \\
\textit{david.taylor@kcl.ac.uk} \\
$^2$ Department of Cardiology, German Heart Center of the Charité-University Medicine, Charité Campus Mitte, Berlin, Germany \\
\textit{wilhelm.haverkamp@dhzc-charite.de} \\
$*$ first author, $\dagger$ Senior author, $\#$ Corresponding author
}


\maketitle
% https://jnnp.bmj.com/


\begin{abstract}

\textbf{Background: } Electrocardiogram (ECG) analysis has emerged as a promising tool for identifying physiological changes associated with neuropsychiatric conditions. The relationship between cardiovascular health and neuropsychiatric disorders suggests that ECG abnormalities could serve as valuable biomarkers for more efficient detection, therapy monitoring, and risk stratification. However, the potential of the ECG to accurately distinguish neuropsychiatric conditions, particularly among diverse patient populations, remains underexplored.

\textbf{Methods: } This study utilized ECG markers and basic demographic data to predict neuropsychiatric conditions using machine learning models, with targets defined through ICD-10 codes. Both internal and external validation were performed using the MIMIC-IV and ECG-View datasets respectively. Performance was assessed using AUROC scores. To enhance model interpretability, Shapley values were applied to provide insights into the contributions of individual ECG features to the predictions.

\textbf{Results: } Significant predictive performance was observed for conditions within the neurological and psychiatric groups. For the neurological group, Alzheimer's disease (G30) achieved an internal AUROC of 0.813 (95\% CI: 0.812–0.814) and an external AUROC of 0.868 (95\% CI: 0.867–0.868). In the psychiatric group, unspecified dementia (F03) showed an internal AUROC of 0.849 (95\% CI: 0.848–0.849) and an external AUROC of 0.862 (95\% CI: 0.861–0.863). Discriminative features align with known ECG markers but also provide hints on potentially new markers.

\textbf{Conclusions: } ECG offers significant promise for diagnosing and monitoring neuropsychiatric conditions, with robust predictive performance across internal and external cohorts. Future work should focus on addressing potential confounders, such as therapy-related cardiotoxicity, and expanding the scope of ECG applications, including personalized care and early intervention strategies.



\end{abstract}




\subsubsection*{What is already known on this topic}
Electrocardiogram abnormalities have been explored as potential biomarkers for neuropsychiatric conditions, but their diagnostic value has not been thoroughly investigated. Recent studies have only begun to examine these abnormalities in exploratory analyses to identify potential correlations.

\subsubsection*{What this study adds}
This study demonstrates that ECG features can effectively predict neuropsychiatric conditions, extending beyond their traditional use as descriptive markers. The findings show high predictive performance across both internal and external datasets. Additionally, the use of an explainable approach allows for a deeper understanding of the correlations between specific ECG markers and neuropsychiatric conditions, confirming existing knowledge while also providing new insights. This research sheds light on the physiological changes that may underlie the interplay between cardiac features and neuropsychiatric disorders.

\subsubsection*{How this study might affect research, practice or policy}
This work lays the foundation for integrating ECG into the diagnosis and monitoring of neuropsychiatric conditions. It presents a non-invasive, cost-effective tool that could aid in more efficient detection and personalized therapy management in clinical settings, with the potential to transform both research and clinical practices.





\section{Introduction}


Neurological and psychiatric conditions are among the most challenging diseases, profoundly affecting both individuals and society. Neurological disorders such as Alzheimer's disease and Parkinson's disease lead to progressive cognitive and motor decline, severely limiting patients' independence and quality of life \cite{aarsland2021parkinson}. Psychiatric conditions like dementia and delirium further complicate mental health, often causing confusion, memory loss, and significant behavioral changes \cite{fong2022inter}. Notably, psychiatric disorders and neurological conditions are closely interrelated, as they share common neurobiological pathways, genetic factors, and overlapping cognitive, affective, and behavioral symptoms. The distinction between the two is often blurred, complicating diagnosis and treatment, and highlighting the need for integrated, multidisciplinary approaches in neuropsychiatric care \cite{taslim2024neuropsychiatric}. These conditions not only disrupt the lives of those affected but also impose a considerable burden on caregivers and healthcare systems \cite{eichel2022neuropsychiatric}. Early and fast diagnosis is crucial to effectively manage these diseases, but current diagnostic methods can be complex, costly, and sometimes inaccessible \cite{cummings2021role}. Furthermore, it is important to distinguish Alzheimer's disease from other major forms of dementia, as dementia describes a group of symptoms, whereas Alzheimer's disease refers to the underlying pathology \cite{karantzoulis2011distinguishing}. 

Electrocardiograms (ECG) have long been established as a critical tool in cardiology, used primarily to detect and monitor cardiac conditions by measuring the electrical activity of the heart. ECGs provide vital information about heart rhythms, helping to identify issues such as arrhythmias, ischemia, and other cardiac abnormalities \cite{siontis2021artificial}. While ECGs are traditionally associated with cardiac care, recent advancements in medical research have expanded their potential applications. Novel uses of ECG, supported by advancements in data analysis and machine learning, are now being explored in predictive modeling applications across fields beyond cardiology, including the estimation of laboratory values \cite{alcaraz2024cardiolab} and patient deterioration in emergency departments \cite{alcaraz2024mds}.  

The potential link between ECG changes and neurological or psychiatric conditions is an emerging area of study. For instance, conditions like Alzheimer's and Parkinson's disease may affect the autonomic nervous system \cite{gonccalves2022heart}, leading to detectable changes in heart rate variability and other ECG patterns. Similarly, psychiatric disorders such as dementia and delirium have been associated with cardiac function \cite{raberi2023postoperative}, potentially reflecting the underlying physiological disruptions. Interestingly, by identifying these subtle ECG changes, one could hope to develop new, non-invasive methods for more efficient detection and monitoring of neurological and psychiatric conditions, offering a promising complement to existing diagnostic tools.  




\section{Methods}

\subsection{Dataset}

\begin{figure*}[!ht]
    \centering
    \includegraphics[width=\textwidth]{abstract.pdf}
    \caption{Diagrammatic illustration of our proposed methodology. The MIMIC-IV-ECG dataset serves as our internal dataset, providing demographic and ECG features used as input for training a tree-based model to diagnose various neuropsychiatric conditions. For external validation, we utilize a second patient cohort from the ECG-View II dataset, extracting the same set of features and neuropsychiatric targets. The conditions are defined based on ICD-10-CM codes.}
    \label{fig:abstract}
\end{figure*}





\begin{table}[!ht]
    \centering
    \begin{tabular}{lll}
    \hline
    \textbf{Variable} & \textbf{MIMIC-IV-ECG} & \textbf{ECG-View II} \\ \hline\hline
    \textbf{Gender (\%)} &  &  \\ \hline
    Female     & 226,892 (48.50)  & 375,733 (48.44) \\
    Male   & 240,837 (51.49)  & 399,802 (51.55) \\ \hline
    \textbf{Age (\%)} &  &  \\ \hline
    Median years (IQR) & 66 (25) & 52 (25) \\ 
    Quantile 1 & 18-53 (23.83) & 18-40 (24.03) \\
    Quantile 2 & 53-66 (25.16) & 40-52 (25.75) \\
    Quantile 3 & 66-78 (25.60) & 52-65 (24.94) \\
    Quantile 4 & 78-101 (25.40) & 65-109 (25.28) \\ \hline
    \textbf{ECG features Median(IQR)} &  &  \\ \hline
    RR-interval[ms] & 769 (264)  & 857 (227) \\
    PR-interval[ms] & 158 (38)  & 158 (28) \\
    QRS-duration[ms] & 94 (23)  & 90 (14) \\
    QT-interval[ms] & 394 (68)  & 392 (48) \\
    QTc-interval[ms] & 447 (47)  & 421 (37) \\
    P-wave-axis[${}^\circ$] & 51 (32) & 53 (28) \\
    QRS-axis[${}^\circ$] & 13 (61) & 48 (49) \\
    T-wave-axis[${}^\circ$] & 42 (58)  & 44 (33) \\ \hline
    \end{tabular}%
    \caption{An overview of the variable characteristics across samples, including demographic information such as gender counts (with ratios) and the median age in years (with interquartile range, IQR), along with the age distribution represented by quantile ratios. It also includes the median (IQR) values for ECG features, such as the RR-interval, PR-interval, QRS-duration, QT-interval, and QTc-interval in milliseconds, as well as the P-wave axis, QRS axis, and T-wave axis in degrees.}
    \label{tab:descriptive}
\end{table}


Our primary dataset for training and internal evaluation was obtained from the MIMIC-IV-ECG database \cite{johnson2023mimic,MIMICIVECG2023}, a subset of a large-scale critical care dataset collected at the Beth Israel Deaconess Medical Center in Boston, Massachusetts. It includes data from patients admitted to the emergency department (ED) and intensive care unit (ICU). Target variables are based on discharge diagnoses encoded using the International Classification of Diseases Clinical Modification (ICD-10-CM). Although a broad range of neoplasm-related codes is explored, this study focuses on those with internal and external validation AUROC scores exceeding 0.7.

To create a comprehensive and harmonized feature set, ECG features from the MIMIC-IV dataset were aligned with those from the ECG-VIEW-II database \cite{kim2017ecg}, which serves as our secondary dataset for external validation. ECG-VIEW-II includes data from patients at a South Korean tertiary teaching hospital. The standardized feature set consists of ECG-derived measurements (RR-interval, PR-interval, QRS-duration, QT-interval, QTc-interval in milliseconds; P-wave-axis, QRS-axis, and T-wave-axis in degrees), along with demographic attributes (binary sex and age as a continuous variable).

For the internal dataset, stratified folds were created based on diagnoses, age, and gender distributions, using an 18:1:1 split as described in previous research \cite{strodthoff2024prospects}. A similar stratification procedure was applied to the external dataset to maintain consistency. The training process emphasizes MIMIC-IV-ECG due to its broader ethnic diversity compared to ECG-VIEW-II, which helps improve the model’s generalization across various populations, as shown in prior studies \cite{alcaraz2024estimationcardiacnoncardiacdiagnosis}, which focused on cardiac conditions, \cite{alcaraz2024electrocardiogrambaseddiagnosisliverdiseases} for liver conditions, and \cite{alcaraz2024explainablemachinelearningneoplasms} for neoplasms. This strategy ensures robust internal training and reliable external validation across ethnically and geographically diverse cohorts.


\subsection{Models and evaluation}
In this study, we create individual tree-based models using Extreme Gradient Boosting (XGBoost) to tackle binary classification tasks, with a distinct model for each selected ICD-10-CM code. To avoid overfitting, we apply early stopping with a patience of 10 iterations on the validation fold during training. Model performance is assessed using the area under the receiver operating characteristic curve (AUROC) on the validation fold. We report AUROC scores for both the internal test set and an external dataset, along with 95\% confidence intervals calculated through empirical bootstrapping with 1000 iterations.

\subsection{Explainability}
Our objective goes beyond merely assessing model performance. To gain deeper insights into the trained models, we integrate Shapley values into our workflow \cite{lundberg2020local}. These values provide a method for evaluating feature importance by measuring the individual contribution of each feature to the model's predictions.


\section{Results}

\subsection{Predictive performance}


\begin{table*}[!ht]
    \centering
    \resizebox{0.98\textwidth}{!}{%
    \begin{tabular}{lll}
    \hline
    \textbf{Code: Description} & \textbf{Internal AUROC (95\% CI) [Prev.]} & \textbf{External AUROC (95\% CI) [Prev.]} \\ \hline\hline
    
    \textbf{Neurological conditions} &  &  \\ \hline
    G30: Alzheimer's & 0.8134 (0.8124, 0.8140) [1.04\%] & 0.8678 (0.8676, 0.8682) [0.13\%] \\
    G20: Parkinson's & 0.7796 (0.7790, 0.7799) [0.94\%]  & 0.7897 (0.7900, 0.7905) [0.21\%] \\
    G931: Anoxic brain damage & 0.7283 (0.7278, 0.7312) [0.5\%]  & 0.7829 (0.7823, 0.7840) [0.05\%] \\ \hline

    \textbf{Phychiatric conditions} &  &  \\ \hline
    F01: Vascular dementia & 0.7732 (0.7724, 0.7736) [0.57\%] & 0.8120 (0.8110, 0.8117) [0.11\%] \\
    F03: Unspecified dementia &  0.8490 (0.8482, 0.8490) [2.6\%] & 0.8620 (0.8610, 0.8624) [0.12\%] \\
    F05: Delirium due to physiological condition & 0.7074 (0.7066, 0.7079) [3.41\%] & 0.7822 (0.7812, 0.7826) [0.12\%] \\ \hline

    \end{tabular}
    }
    \caption{Predictive performance results for the neuropsychiatric conditions studied. We present the internal (MIMIC-IV) and external (ECG-View) AUROC performances with 95\% confidence intervals, along with the class prevalence of each condition in both datasets.}
    \label{tab:aurocs}
\end{table*}



Table \ref{tab:aurocs} presents the predictive performance of our model for various neuropsychiatric conditions, evaluated using AUROC scores on both internal (MIMIC-IV) and external (ECG-View) test sets. The 95\% prediction intervals offer insight into the reliability of these scores. Additionally, the class prevalence for each condition within its respective dataset is provided, offering context on the population distribution. In the MIMIC-IV cohort, prevalences range from 0.5\% to 3.41\%, while the ECG-View cohort exhibits much lower prevalences, from 0.05\% to 0.21\%.

Significantly, the conditions with the highest predictive performance in the neurological group include G30: Alzheimer's, with an internal AUROC of 0.8134 (95\% CI: 0.8124–0.8140) and an external AUROC of 0.8678 (95\% CI: 0.8676–0.8682). For the psychiatric group, F03: Unspecified dementia achieved an internal AUROC of 0.8490 (95\% CI: 0.8482–0.8490) and an external AUROC of 0.8620 (95\% CI: 0.8610–0.8624). These findings highlight the reliability of our model in predicting neuropsychiatric conditions, despite differences in prevalence across datasets.



\subsection{Explainability}

\begin{figure*}[!ht]
    \centering
    \includegraphics[width=0.95\textwidth, keepaspectratio]{shapley.pdf}
    \caption{Explainability results for the investigated neurological and phychiatric conditions. The beeswarm plot visualizes through a single dot per feature and sample if the feature contributes positively (right hand side) or negatively (left hand side) to the model prediction. In addition, the color-coding allows to infer if a point is associated with high (red) or low (blue) feature values.}
    \label{fig:shapley}
\end{figure*}

Figure \ref{fig:shapley} presents the explainability results based on Shapley values. Across all investigated conditions age is the most important predictive feature, where increased age contributes positively, except for the case of anoxic brain damage. For the investigated neurological conditions, QTc interval is the most important ECG feature for Parkinson's and anoxic brain damage, where low values and high values of the marker contribute the most positively to each condition respectively. Simiarly, T-wave axis is the second most important ECG feature for the same two conditions, where low values of the marker contributes the most positively to each condition. For the investigated psychiatric conditions, RR-interval is the most important predictive ECG feature across unspecified dementia and delirium due to known physiological conditions, where low values of the marker mainly contribute to the conditions. Similarly, PR-interval is the second most important ECG feature for the same two conditions, where mainly high and low values of the marker contribute positively to the conditions.



\section{Discussion}

\subsection*{ECG biomarkers for non-cardiovascular conditions}
Detecting neuropsychiatric conditions through ECG features may initially seem unconventional, as the ECG is traditionally associated with cardiovascular diagnoses. However, the physiological interplay between the cardiovascular and neuropsychiatric systems offers a unique opportunity for diagnostic innovation. While the mechanisms linking neuropsychiatric conditions to ECG abnormalities remain incompletely understood, some known relations include autonomic nervous system dysfunction, neurodegenerative changes, vascular alterations, electrophysiological disruptions, and hormonal or metabolic factors. For example, Alzheimer's disease, Parkinson's disease, and brain damage share common ECG abnormalities, primarily reflecting autonomic dysfunction and brain-heart communication. Across these conditions, reduced heart rate variability (HRV), bradycardia due to parasympathetic dominance, and arrhythmias like atrial fibrillation \cite{barkhordarian2024atrial} are frequently observed. Similarly, medication effects, such as those from cholinesterase inhibitors in Alzheimer’s or dopamine agonists in Parkinson’s, can further influence conduction and structural heart changes \cite{huang2020comparative}.

Despite these insights, the specific relationship to certain abnormalities remains unclear. Our findings reveal specific ECG patterns as distinctive biomarkers for these conditions, identified using machine learning. Similarly, beyond diagnosis, ECG biomarkers hold promise for therapy management by monitoring treatment responses, risk stratification for medication-related adverse events, early intervention to identify stress or relapse, and longitudinal monitoring for chronic conditions. This interdisciplinary approach bridges neurology, psychiatry, and cardiology, advancing novel diagnostic and therapeutic strategies.





\subsection*{Predictive performance}

The exceptional predictive power of specific ECG features highlights their ability to reliably identify neuropsychiatric conditions from a single ECG. Consistently high AUROC scores in both internal and external validations demonstrate the robustness of these features across diverse cohorts. The distinct patterns observed across physiological systems underscore the intricate connection between cardiac and neuropsychiatric health.


\subsection*{Feature importance}

In this study, age was identified as a key factor, with older patients contributing more to most conditions. This is consistent with well-documented evidence of brain and structural degeneration associated with aging. For example, previous research has reported abnormal electroencephalograms in patients over 45 years old with conditions such as senile and arteriosclerotic psychosis, involutional psychosis, psychosis with mental deficiency, manic-depressive states, psychoneurosis, and schizophrenia \cite{doi:10.1176/ajp.101.1.82}.

In terms of neurological conditions, a recent study investigated the association of ECG markers and Alzheimer's disease and have found high hazard ratios for low QRS duration and for low PR interval \cite{isaksen2023association}, which aligns with our findings. Another study reported abormally high P wave axes\cite{gutierrez2019association}, which also aligns with our findings. For Parkinson's disease, a study that collected data from various articles and meta-analyses found a clear preponderance of higher incidence and prevalence in males \cite{georgiev2017gender}, which also aligns with our findings. Similarly, a study conducted on 981 patients demonstrated that high QT interval values are associated with Parkinson's disease, which again matches our findings. However, the authors report that high QTc interval values were also observed in their cohort \cite{rabkin2024qt}. In contrast, our analysis, based on nearly 4,000 samples, suggests that while the raw QT interval is elevated in Parkinson's patients, the QTc interval, when corrected for heart rate, appears to be lower. This discrepancy may be due to differences in sample size, patient characteristics, or the methodologies used to correct the QT interval, as well as the larger heterogeneity within our cohort.

Regarding psychiatric conditions, left QRS axis deviation have been associated with all-cause dementia patients by a recent study \cite{mao2023ventricular}, which aligns with low or negative values found in our findings. A similar study correlates with another of our findings, as the authors found that shorter QRS duration is associated with non-Alzheimer dementia \cite{isaksen2022associations}. For vascular dementia, prior work identified a significant increase of QTc values on a population group which also match our findings \cite{matei2015qt}. Finally, the high values of QTc interval for patients with delirium correlates with previous literature work on patients with delirium tremens whose also developed tachyarrhythmias and returned to sinus rhythm after appropriate treatment.

The results presented in this work provide hints on on potentially new ECG markers for neuropsychiatric conditions. This includes high values of RR interval for Alzheimer's disease, low values of QRS axis for Parkinson's disease, and low values of QRS duration for vascular dementia. We hypothesize that the high RR interval in Alzheimer's may reflect altered autonomic regulation due to parasympathetic dysfunction \cite{femminella2014autonomic}. The low QRS axis in Parkinson's disease could be linked to basal ganglia dysfunction affecting autonomic control of the heart \cite{zesiewicz2003autonomic}. Similarly, the low QRS duration in vascular dementia might result from cerebral ischemia and microvascular changes affecting cardiac conduction \cite{wang2018dysfunction}. These hypotheses warrant further investigation.






\subsection*{Therapy vs therapy-naive confounding}

Our approach naturally encompasses both therapy-naive patients and those attending follow-up visits, as ICD-10 codes represent a combination of newly assigned diagnoses and ongoing treatment cases. This creates potential confounding effects, where predictions might reflect therapy-related cardiotoxic changes rather than the neuropsychiatric condition itself. Notably, medications used for the treatment of Alzheimer’s disease are not completely safe in terms of cardiac side effects. Acetylcholinesterase inhibitors (Donepezil, Rivastigmine, Galantamine) can, in rare cases, cause bradycardia or QT interval prolongation, particularly in patients with preexisting cardiac arrhythmias \cite{howes2014cardiovascular}. In contrast, Memantine has minimal cardiovascular side effects. Additionally, some Parkinson’s disease medications have been associated with cardiac complications. Certain dopamine agonists, particularly ergot-derived agents such as pergolide and cabergoline, can increase the risk of valvular heart disease by stimulating serotonin 5-HT$_{2}B$ receptors, leading to fibrotic changes in heart valves \cite{schade2007dopamine}. In addition, antipsychotic medication is commonly used in neuropsychiatric conditions, which are associated with tachycardia, bradycardia and QTc prolongation \cite{buckley2000cardiovascular}

These therapy-related cardiac effects further complicate the distinction between disease progression and treatment influence, underscoring the need for careful monitoring, especially in patients with known cardiac conditions \cite{howes2014cardiovascular}. Future research should aim to stratify newly diagnosed cases from follow-ups, as such differentiation would enable a more precise evaluation of the model’s ability to distinguish between new diagnoses and therapy-induced patterns.


\subsection*{Potential innovations and applications}
ECG innovations for neuropsychiatric conditions hold great promise across diagnosis, therapy management, and personalized care. Distinctive ECG patterns can act as biomarkers for more efficient, non-invasive detection of neuropsychiatric disorders. Beyond diagnosis, ECG monitoring contributes to therapy management by tracking treatment responses and identifying early physiological changes related to medication efficacy or adverse effects. Risk stratification also emerges as a vital application, enabling clinicians to predict adverse events such as arrhythmias linked to neuropsychiatric treatments or stress-related cardiovascular complications.

Longitudinal monitoring through periodic or wearable ECG devices provides valuable insights into disease progression, relapse or episodes prediction, facilitating real-time assessments of mental and physical health. ECG-derived markers further support personalized medicine by helping tailor treatments to individual patients. Additionally, early intervention strategies can use subtle ECG changes to detect stress, anxiety, or relapse indicators, ensuring timely clinical responses. 



\subsection*{Limitations and future work}

ECG is a valuable tool for identifying electrical abnormalities that may correlate with neuropsychiatric conditions; however, it does not provide direct diagnostic confirmation or insights into specific neurophysiological changes underlying these disorders. Many ECG alterations are non-specific and can be attributed to factors such as stress, medication effects, or comorbid conditions, making it challenging to isolate patterns uniquely associated with neuropsychiatric conditions. Additionally, the causal mechanisms linking ECG abnormalities to neuropsychiatric disorders remain poorly understood, highlighting the need for further investigation into these complex relationships.

Future research should explore how ECG abnormalities differ across diverse demographic groups and distinguish them from normal variations, such as those related to age \cite{ott2024using}. Investigating the causal connections between ECG features and neuropsychiatric conditions will be essential for advancing our understanding \cite{alcaraz2024causalconceptts}. Further studies should focus on using raw ECG waveforms and validating findings across external datasets to improve diagnostic accuracy \cite{strodthoff2024prospects, alcaraz2024mds}. The demonstrated potential of raw ECG waveforms to outperform traditional ECG features emphasizes the importance of refining these approaches to enhance precision and reliability in detecting neuropsychiatric conditions.


\section*{Data and code availability} 
The code for dataset preprocessing and reproducing experiments is available in a dedicated \href{https://github.com/AI4HealthUOL/CardioDiag}{code repository}.

\section*{Author contributions}
JMLA, WH, and NS conceived and designed the project. JMLA conducted the full experimental analyses, with WH, and NS supervising them, and WH, EO, and DT providing critical revision of clinical intellectual content. JMLA produced the first draft, the rest of the authors revised it. All authors critically revised the content and approved the final version for publication.

\section*{Declaration of interests}
The authors declare no competing interests.

\bibliography{refs}


\end{document}

