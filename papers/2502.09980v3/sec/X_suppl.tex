

\clearpage
\setcounter{page}{1}
\maketitlesupplementary

\section{Detailed Evaluation Results}
\cref{tab:all_v2v_result_supp,tab:all_v2x_result_supp} summarize the detailed evaluation results of our \namemethod~and other baseline methods in \namedataset's \namevsplit~and \namexsplit. In addition, \cref{tab:v2v_planning,tab:v2x_planning} show the detailed planning performance. 
% skip NAVSIM result
%In addition, we also evaluate the planning performance using NAVSIM\cite{dauner2024navsim}’s Predictive Driver Model Score (PDMS). PDMS has been shown to provide open-loop evaluation results that are better aligned with close-loop evaluation results. Close-loop evaluation in our work is infeasible because our base datasets V2V4Real\cite{xu2023v2v4real} and V2X-Real\cite{xiang2024v2xreal} do not provide close-loop simulation tools. In addition, the corresponding map data is not released. Therefore, we exclude PDMS's drivable area compliance (DAC) and ego progress (EP) and include no collisions (NC), time-to-collision (TTC) and comfort (C).
For the grounding task, our \namemethod~achieves competitive results in \namevsplit~and outperforms all other baseline methods in \namexsplit. More importantly, for the notable object identification task and the planning task, our \namemethod~outperforms all other baseline methods in both \namevsplit~and \namexsplit.


% V2V split result table
\begin{table*}[t!]
\small
\setlength{\tabcolsep}{2pt}
%\renewcommand{\arraystretch}{0.7}
\begin{center}
\begin{tabular}{l ccc ccc ccc c ccc cc c}
  \hline
  \hline
  \multirow{2}{*}{Method} &
  \multicolumn{3}{c}{Q1} & \multicolumn{3}{c}{Q2} & \multicolumn{3}{c}{Q3} & \multicolumn{1}{c}{Q\textsubscript{Gr}} & \multicolumn{3}{c}{Q4} & \multicolumn{2}{c}{Q5} & \multirow{2}{*}{Comm(MB) $\downarrow$} \\
  %\cline{2-10}
  \cmidrule(lr){2-4} \cmidrule(lr){5-7} \cmidrule(lr){8-10} \cmidrule(lr){11-11} \cmidrule(lr){12-14} \cmidrule(lr){15-16}
  &
  F1 $\uparrow$ & P $\uparrow$ & R $\uparrow$ &
  F1 $\uparrow$ & P $\uparrow$ & R $\uparrow$ &
  F1 $\uparrow$ & P $\uparrow$ & R $\uparrow$ &
  F1 $\uparrow$ & 
  F1 $\uparrow$ & P $\uparrow$ & R $\uparrow$ & 
  L2$_{avg}$ (m) $\downarrow$ & CR$_{avg}$ (\%) $\downarrow$ \\
  \hline
  \hline
  \textit{No Fusion}         & 66.6 & 77.9 & 58.2 & 22.6 & 29.4 & 18.4 & 17.2 & 17.4 & 16.9 & 35.5 & 47.3 & 49.2 & 45.6 & 6.55 & 4.57 & \textbf{0} \\
  \textit{Early Fusion}      & \textbf{73.5} & \textbf{82.2} & \underline{66.5} & 23.3 & 29.1 & 19.5 & 20.8 & \underline{22.7} & 19.3 & 39.2 & 53.9 & 55.4 & 52.6 & \underline{6.20} & \underline{3.55} & 1.9208 \\
  \hline
  \scriptsize{\textit{Intermediate Fusion}} \\ 
  AttFuse~\cite{xu2022opencood}         & 70.7 & 79.6 & 63.6 & 26.4 & 31.6 & 22.7 & 18.4 & 19.6 & 17.4 & 38.5 & 56.9 & \underline{57.2} & 56.6 & 6.83 & 4.12 & \underline{0.4008} \\
  V2X-ViT~\cite{xu2022v2xvit}           & 70.8 & \underline{81.1} & 62.8 & 28.0 & 33.9 & 23.9 & \textbf{22.6} & \textbf{25.2} & 20.5 & 40.5 & \underline{57.6} & 57.0 & \textbf{58.2} & 7.08 & 4.33 & \underline{0.4008} \\
  CoBEVT~\cite{xu2022cobevt}            & \underline{72.2} & 76.8 & \textbf{68.1} & \underline{29.3} & \underline{34.7} & \underline{25.3} & \underline{21.3} & 22.1 & \underline{20.6} & \textbf{40.9} & \underline{57.6} & \underline{57.2} & \underline{58.1} & 6.72 & 3.88 & \underline{0.4008} \\
  \hline
  \scriptsize{\textit{LLM Fusion}} \\
  \namemethod~(Ours)     & 70.0 & 80.1 & 62.2 & \textbf{30.8} & \textbf{36.3} & \textbf{26.7} & 21.2 & 21.5 & \textbf{20.8} & \underline{40.7} & \textbf{59.7} & \textbf{61.9} & 57.6  & \textbf{4.99} & \textbf{3.00} & 0.4068 \\
  \hline
\end{tabular}
\vspace{-5pt}
\caption{
\namemethod's testing performance in \namedataset's \namevsplit~in comparison with baseline methods. Q1: Grounding at a reference location. Q2: Grounding behind a reference object at a location. Q3: Grounding behind a reference object in a direction. Q\textsubscript{Gr}: Average of grounding (Q1, Q2, and Q3). Q4: Notable object identification. Q5: Planning. P: Precision. R: Recall. L2: L2 distance error. CR: Collision rate. Comm: Communication cost. In each column, the \textbf{best} results are in boldface, and the \underline{second-best} results are in underline.
\vspace{-5pt}
}
\label{tab:all_v2v_result_supp}
\end{center}
\vspace{-10pt}
\end{table*}


% V2X split result table
\begin{table*}[t!]
\small
\setlength{\tabcolsep}{2pt}
%\renewcommand{\arraystretch}{0.7}
\begin{center}
\begin{tabular}{l ccc ccc ccc c ccc cc c}
  \hline
  \hline
  \multirow{2}{*}{Method} &
  \multicolumn{3}{c}{Q1} & \multicolumn{3}{c}{Q2} & \multicolumn{3}{c}{Q3} & \multicolumn{1}{c}{Q\textsubscript{Gr}} & \multicolumn{3}{c}{Q4} & \multicolumn{2}{c}{Q5} & \multirow{2}{*}{Comm(MB) $\downarrow$} \\
  %\cline{2-10}
  \cmidrule(lr){2-4} \cmidrule(lr){5-7} \cmidrule(lr){8-10} \cmidrule(lr){11-11} \cmidrule(lr){12-14} \cmidrule(lr){15-16}
  &
  F1 $\uparrow$ & P $\uparrow$ & R $\uparrow$ &
  F1 $\uparrow$ & P $\uparrow$ & R $\uparrow$ &
  F1 $\uparrow$ & P $\uparrow$ & R $\uparrow$ &
  F1 $\uparrow$ & 
  F1 $\uparrow$ & P $\uparrow$ & R $\uparrow$ & 
  L2$_{avg}$ (m) $\downarrow$ & CR$_{avg}$ (\%) $\downarrow$ \\
  \hline
  \hline
  \textit{No Fusion}         & 55.7 & \textbf{71.6} & 45.5 & 21.4 & 33.2 & 15.8 & 25.2 & 26.2 & 24.2 & 34.1 & 64.4 & 66.1 & 62.7 & 2.31 & 9.21 & \textbf{0} \\
  \textit{Early Fusion}      & \underline{59.7} & 70.6 & 51.8 & 23.3 & 34.0 & 17.7 & 26.1 & 28.0 & 24.5 & 36.4 & \underline{67.6} & \underline{69.3} & \underline{66.0} & \underline{2.12} & 8.61 & 1.9208 \\
  \hline
  \scriptsize{\textit{Intermediate Fusion}} \\ 
  AttFuse~\cite{xu2022opencood}         & 58.9 & \underline{71.1} & 50.3 & 23.9 & \underline{34.3} & 18.4 & \underline{26.3} & \textbf{28.3} & \underline{24.6} & 36.4 & 65.9 & 67.0 & 64.9 & 2.19 & \underline{8.39} & \underline{0.4008} \\
  V2X-ViT~\cite{xu2022v2xvit}           & 59.6 & 69.6 & \underline{52.1} & \underline{24.2} & 33.2 & \underline{19.1} & 26.1 & \underline{28.2} & 24.3 & \underline{36.6} & 65.0 & 64.8 & 65.3 & 2.29 & 8.86 & \underline{0.4008} \\
  \hline
  \scriptsize{\textit{LLM Fusion}} \\
  \namemethod~(Ours)     & \textbf{60.5} & 69.5 & \textbf{53.6} & \textbf{25.3} & \textbf{34.9} & \textbf{19.8} & \textbf{26.7} & 27.0 & \textbf{26.4} & \textbf{37.5} & \textbf{69.3} & \textbf{71.9} & \textbf{66.8} & \textbf{1.71} & \textbf{6.89} & 0.4068 \\
  \hline
\end{tabular}
\vspace{-5pt}
\caption{
\namemethod's testing performance in \namedataset's \namexsplit~in comparison with baseline methods. 
% Skip this duplicate description
Q1: Grounding at a reference location. Q2: Grounding behind a reference object at a location. Q3: Grounding behind a reference object in a direction. Q\textsubscript{Gr}: Average of grounding (Q1, Q2, and Q3). Q4: Notable object identification. Q5: Planning. P: Precision. R: Recall. L2: L2 distance error. CR: Collision rate. Comm: Communication cost. In each column, the \textbf{best} results are in boldface, and the \underline{second-best} results are in underline.
\vspace{-5pt}
}
\label{tab:all_v2x_result_supp}
\end{center}
\vspace{-10pt}
\end{table*}


% V2V split Planning result table
\begin{table*}[t!]
\small
%\setlength{\tabcolsep}{6pt}
%\renewcommand{\arraystretch}{0.7}
\begin{center}
\begin{tabular}{l cccc cccc }
  \hline
  \hline
  \multirow{2}{*}{Method} &
  \multicolumn{4}{c}{L2 (m)} & \multicolumn{4}{c}{CR (\%)} \\
  %\cline{2-10}
  \cmidrule(lr){2-5} \cmidrule(lr){6-9}
  &
  1s $\downarrow$ & 2s $\downarrow$ & 3s $\downarrow$ & average $\downarrow$ &
  1s $\downarrow$ & 2s $\downarrow$ & 3s $\downarrow$ & average $\downarrow$ \\
  \hline
  \hline
  \textit{No Fusion}         & 3.84 & 6.52 & 9.30 & 6.55 & 1.31 & 4.76 & 7.63 & 4.57  \\
  \textit{Early Fusion}      & \underline{3.68} & \underline{6.19} & \underline{8.74} & \underline{6.20} & 0.96 & 3.86 & \underline{5.83} & \underline{3.55} \\
  \hline
  \scriptsize{\textit{Intermediate Fusion}} \\
  AttFuse~\cite{xu2022opencood}         & 4.06 & 6.78	& 9.64 & 6.83 & 1.42 & 4.41 & 6.53 & 4.12  \\
  V2X-ViT~\cite{xu2022v2xvit}           & 4.21 & 7.05 & 9.99 & 7.08 & 1.33 & 4.82 & 6.85 & 4.33  \\
  CoBEVT~\cite{xu2022cobevt}            & 3.97 & 6.71 & 9.47 & 6.72 & \underline{0.93} & \underline{3.74} & 6.96 & 3.88  \\
  \hline
  \scriptsize{\textit{LLM Fusion}} \\
  \namemethod~(ours)     & \textbf{2.96} & \textbf{4.97} & \textbf{7.05} & \textbf{4.99} & \textbf{0.55} & \textbf{3.19} & \textbf{5.25} &  \textbf{3.00} \\
  \hline
\end{tabular}
\vspace{-5pt}
\caption{
\namemethod's planning performance in \namedataset's \namevsplit~in comparison with baseline methods. L2: L2 distance error. CR: Collision rate. In each column, the \textbf{best} results are in boldface, and the \underline{second-best} results are in underline.
\vspace{-10pt}
}
\label{tab:v2v_planning}
\end{center}
%\vspace{-15pt}
\end{table*}   


% V2X split Planning result table
\begin{table*}[t!]
\small
%\setlength{\tabcolsep}{6pt}
%\renewcommand{\arraystretch}{0.7}
\begin{center}
\begin{tabular}{l cccc cccc }
  \hline
  \hline
  \multirow{2}{*}{Method} &
  \multicolumn{4}{c}{L2 (m)} & \multicolumn{4}{c}{CR (\%)} \\
  %\cline{2-10}
  \cmidrule(lr){2-5} \cmidrule(lr){6-9}
  &
  1s $\downarrow$ & 2s $\downarrow$ & 3s $\downarrow$ & average $\downarrow$ &
  1s $\downarrow$ & 2s $\downarrow$ & 3s $\downarrow$ & average $\downarrow$ \\
  \hline
  \hline
  \textit{No Fusion}         & 1.33 & 2.28 & 3.31 & 2.31 & 2.52 & 9.54 & 15.57 & 9.21  \\
  \textit{Early Fusion}      & \underline{1.24} & \underline{2.10} & \underline{3.00} & \underline{2.12} & 3.51 & \underline{8.37} & 13.93 & 8.61  \\
  \hline
  \scriptsize{\textit{Intermediate Fusion}} \\
  AttFuse~\cite{xu2022opencood}         & 1.27 & 2.17 & 3.11 & 2.19 & 2.40 & 9.07 & \underline{13.70} & \underline{8.39} \\
  V2X-ViT~\cite{xu2022v2xvit}           & 1.34 & 2.27 & 3.25 & 2.29 & \textbf{1.41} & 9.89 & 15.28 & 8.86  \\
  \hline
  \scriptsize{\textit{LLM Fusion}} \\
  \namemethod~(ours)     & \textbf{0.99} & \textbf{1.70} & \textbf{2.45} & \textbf{1.71} & \underline{2.17} & \textbf{6.79} & \textbf{11.71} & \textbf{6.89}  \\
  \hline
\end{tabular}
\vspace{-5pt}
\caption{
\namemethod's planning performance in \namedataset's \namexsplit~in comparison with baseline methods. L2: L2 distance error. CR: Collision rate. In each column, the \textbf{best} results are in boldface, and the \underline{second-best} results are in underline.
\vspace{-10pt}
}
\label{tab:v2x_planning}
\end{center}
%\vspace{-15pt}
\end{table*}    


% The following two contains NAVSIM PDMS evaluation
% % V2V split Planning result table
% \begin{table*}[t!]
% \small
% %\setlength{\tabcolsep}{6pt}
% %\renewcommand{\arraystretch}{0.7}
% \begin{center}
% \begin{tabular}{l cccc cccc cccc}
%   \hline
%   \hline
%   \multirow{2}{*}{Method} &
%   \multicolumn{4}{c}{L2 (m)} & \multicolumn{4}{c}{CR (\%)} & \multirow{2}{*}{NC $\uparrow$} & \multirow{2}{*}{TTC $\uparrow$} & \multirow{2}{*}{C $\uparrow$} & \multirow{2}{*}{PDMS $\uparrow$} \\
%   %\cline{2-10}
%   \cmidrule(lr){2-5} \cmidrule(lr){6-9}
%   &
%   1s $\downarrow$ & 2s $\downarrow$ & 3s $\downarrow$ & average $\downarrow$ &
%   1s $\downarrow$ & 2s $\downarrow$ & 3s $\downarrow$ & average $\downarrow$ \\
%   \hline
%   \hline
%   \textit{No Fusion}         & 3.84 & 6.52 & 9.30 & 6.55 & 1.31 & 4.76 & 7.63 & 4.57 & 92.4 & 99.0 & \underline{85.6} & 87.9 \\
%   \textit{Early Fusion}      & \underline{3.68} & \underline{6.19} & \underline{8.74} & \underline{6.20} & 0.96 & 3.86 & \underline{5.83} & \underline{3.55} & \underline{94.2} & 98.8 & 81.3 & 88.3 \\
%   \hline
%   \scriptsize{\textit{Intermediate Fusion}} \\
%   AttFuse~\cite{xu2022opencood}         & 4.06 & 6.78	& 9.64 & 6.83 & 1.42 & 4.41 & 6.53 & 4.12 & 93.5 & 98.8 & \textbf{86.9} & \underline{89.2} \\
%   V2X-ViT~\cite{xu2022v2xvit}           & 4.21 & 7.05 & 9.99 & 7.08 & 1.33 & 4.82 & 6.85 & 4.33 & 93.2 & 98.8 & 74.7 & 85.7 \\
%   CoBEVT~\cite{xu2022cobevt}            & 3.97 & 6.71 & 9.47 & 6.72 & \underline{0.93} & \underline{3.74} & 6.96 & 3.88 & 93.0 & \underline{99.1} & 82.1 & 87.7 \\
%   \hline
%   \scriptsize{\textit{LLM Fusion}} \\
%   \namemethod~(ours)     & \textbf{2.96} & \textbf{4.97} & \textbf{7.05} & \textbf{4.99} & \textbf{0.55} & \textbf{3.19} & \textbf{5.25} &  \textbf{3.00} & \textbf{94.7} & \textbf{99.4} & 83.7 & \textbf{89.9} \\
%   \hline
% \end{tabular}
% \vspace{-5pt}
% \caption{
% \namemethod's planning performance in \namedataset's \namevsplit~in comparison with baseline methods. L2: L2 distance error. CR: Collision rate. NC: No collisions. TTC: Time-to-collision. C: Comfort. PDMS: Predictive driver model score. In each column, the \textbf{best} results are in boldface, and the \underline{second-best} results are in underline.
% \vspace{-10pt}
% }
% \label{tab:v2v_planning}
% \end{center}
% %\vspace{-15pt}
% \end{table*}   


% % V2X split Planning result table
% \begin{table*}[t!]
% \small
% %\setlength{\tabcolsep}{6pt}
% %\renewcommand{\arraystretch}{0.7}
% \begin{center}
% \begin{tabular}{l cccc cccc cccc}
%   \hline
%   \hline
%   \multirow{2}{*}{Method} &
%   \multicolumn{4}{c}{L2 (m)} & \multicolumn{4}{c}{CR (\%)} & \multirow{2}{*}{NC $\uparrow$} & \multirow{2}{*}{TTC $\uparrow$} & \multirow{2}{*}{C $\uparrow$} & \multirow{2}{*}{PDMS $\uparrow$}   \\
%   %\cline{2-10}
%   \cmidrule(lr){2-5} \cmidrule(lr){6-9}
%   &
%   1s $\downarrow$ & 2s $\downarrow$ & 3s $\downarrow$ & average $\downarrow$ &
%   1s $\downarrow$ & 2s $\downarrow$ & 3s $\downarrow$ & average $\downarrow$ \\
%   \hline
%   \hline
%   \textit{No Fusion}         & 1.33 & 2.28 & 3.31 & 2.31 & 2.52 & 9.54 & 15.57 & 9.21 & 84.4 & 96.9 & 79.8 & 77.7 \\
%   \textit{Early Fusion}      & \underline{1.24} & \underline{2.10} & \underline{3.00} & \underline{2.12} & 3.51 & \underline{8.37} & 13.93 & 8.61 & 86.1 & 96.5 & \textbf{88.8} & 81.1 \\
%   \hline
%   \scriptsize{\textit{Intermediate Fusion}} \\
%   AttFuse~\cite{xu2022opencood}         & 1.27 & 2.17 & 3.11 & 2.19 & 2.40 & 9.07 & \underline{13.70} & \underline{8.39} & \underline{86.3} & 97.4 & \underline{88.7} & \underline{81.9} \\
%   V2X-ViT~\cite{xu2022v2xvit}           & 1.34 & 2.27 & 3.25 & 2.29 & \textbf{1.41} & 9.89 & 15.28 & 8.86 & 84.7 & \textbf{98.2} & 84.1 & 79.8 \\
%   \hline
%   \scriptsize{\textit{LLM Fusion}} \\
%   \namemethod~(ours)     & \textbf{0.99} & \textbf{1.70} & \textbf{2.45} & \textbf{1.71} & \underline{2.17} & \textbf{6.79} & \textbf{11.71} & \textbf{6.89} & \textbf{88.3} & \underline{97.5} & 88.5 & \textbf{83.8} \\
%   \hline
% \end{tabular}
% \vspace{-5pt}
% \caption{
% \namemethod's planning performance in \namedataset's \namexsplit~in comparison with baseline methods. L2: L2 distance error. CR: Collision rate. NC: No collisions. TTC: Time-to-collision. C: Comfort. PDMS: Predictive driver model score. In each column, the \textbf{best} results are in boldface, and the \underline{second-best} results are in underline.
% \vspace{-10pt}
% }
% \label{tab:v2x_planning}
% \end{center}
% %\vspace{-15pt}
% \end{table*}    



\section{Detailed Communication Cost and Scaling Analysis}
In our centralized setting, assume that there is one centralized LLM computing node, $N_v$ CAVs, and each CAV asks $N_q$ questions at each timestep. Each CAV sends one scene-level feature map ($\leq 0.2$MB), one set of individual object detection result parameters ($\leq 0.003$MB), $N_q$ questions (each $\leq 0.0002$MB) to the LLM and receives $N_q$ answers (each $\leq 0.0002$MB) at each timestep.
Note that each CAV only needs to send the same features to the LLM  once at each timestep because the LLM node can save and reuse them to answer multiple questions from the same or different CAVs at the same timestep. The communication cost of each CAV is: $0.2 + 0.003 + (0.0002 + 0.0002)N_q = (0.203 + 0.0004N_q)$ MB. The LLM receives $N_v$ scene-level feature maps, $N_v$ set of individual object detection result parameters, $N_qN_v$ questions and returns $N_qN_v$ answers. The communication cost of the centralized LLM is $(0.2 + 0.003 + (0.0002 + 0.0002)N_q) N_v = (0.203N_v + 0.0004N_qN_v)$ MB. 


%%Because the size of a question or a answer is $1000$ times smaller than the scene-level feature, the communication cost is still dominated by the scene-level feature size and proportional to the number of CAVs $N_v$ and scale linearly when $N_q$ is small. 
%Based on our QA design, a CAV only needs to ask one or few questions to obtain useful information for driving safety at each timestep. 
%In this centralized setting, the communication cost of each CAV does not change with $N_v$: $0.2 + 0.003 + (0.0002 + 0.0002)N_q = 0.203 + 0.0004N_q$MB.

Alternatively, one can also consider a decentralized setting that deploys one LLM in each CAV. In this setting, each CAV receives the features from all other CAVs and does not need to send or receive any questions or answers. The communication cost of each CAV is $(0.2 + 0.003) (N_v - 1) = 0.203(N_v - 1)$ MB. \cref{tab:communication_cost_supp} summarizes the communication cost and scaling analysis in the aforementioned settings. There could be more different decentralized settings. Which setting works best in terms of communication costs is beyond the current focus of our work.

% Communication cost
\begin{table}[!t]
\small
\setlength{\tabcolsep}{3pt}
%\renewcommand{\arraystretch}{0.7}
\begin{center}
\begin{tabular}{l c c }
  \hline
  \hline
  Setting  & Each CAV & Centralized LLM \\
  \hline
  \hline
  Centralized  & $0.203 + 0.0004N_q$ & $0.203N_v + 0.0004N_qN_v$ \\
  Decentralized      & $0.203(N_v - 1)$ & -  \\
  \hline
\end{tabular}
\vspace{-5pt}
\caption{
Communication cost (MB) and scaling analysis. $N_v$: number of CAVs. $N_q$: number of questions asked by each CAV at each timestep.
%\vspace{-10pt}
}
\label{tab:communication_cost_supp}
\end{center}
\vspace{-10pt}
\end{table}


  
  
\section{Planning Results with Temporal Inputs}
In the main paper, all experiments use point clouds at a single frame from each CAV as the visual input to the models. In this section, we experiment with feeding visual features from $3$ consecutive frames, the current one and the previous two, as the visual input to the models. \cref{tab:planning_3} shows the planning results of the new setting together with the original setting from the main paper. In general, using visual inputs from multiple frames improves planning performance. 
%However, the performance gain for \namemethod~is not as significant as other methods. We conjecture that it is because, even with a single visual input frame, the \namemethod~model is already capable of inferring future movement directions and max possible movement distances in the next 3 seconds using the point cloud feature maps and the locations of nearby detected objects.


% Planning result table
\begin{table}[t!]
\small
\setlength{\tabcolsep}{4pt}
%\renewcommand{\arraystretch}{0.7}
\begin{center}
\begin{tabular}{l cc cc }
  \hline
  \hline
  \multirow{2}{*}{Method} &
  \multicolumn{2}{c}{1 input frame} & \multicolumn{2}{c}{3 input frames} \\
  %\cline{2-10}
  \cmidrule(lr){2-3} \cmidrule(lr){4-5}
  &
  L2 (m) $\downarrow$ & CR (\%) $\downarrow$ &
  L2 (m) $\downarrow$ & CR (\%) $\downarrow$ \\
  \hline
  \hline
  \textit{No Fusion}          & 6.55 &  4.57 & 5.94 & 3.77  \\
  \textit{Early Fusion}      & \underline{6.20} &  \underline{3.55} & \underline{5.13} & \underline{3.04} \\
  \hline
  \scriptsize{\textit{Intermediate Fusion}} \\
  AttFuse~\cite{xu2022opencood}         & 6.83 & 4.12 & 6.46 & 3.50 \\
  V2X-ViT~\cite{xu2022v2xvit}           & 7.08 & 4.33 & 5.52 & 3.84 \\
  CoBEVT~\cite{xu2022cobevt}            & 6.72 & 3.88 & 6.02 & 3.40 \\
  \hline
  \scriptsize{\textit{LLM Fusion}} \\
  \namemethod~(ours)     & \textbf{4.99} &  \textbf{3.00} & \textbf{4.82} & \textbf{2.93} \\
  \hline
\end{tabular}
\vspace{-5pt}
\caption{
\namemethod's planning performance in \namedataset's \namevsplit~in comparison with baseline methods. L2: L2 distance error. CR: Collision rate. In each column, the \textbf{best} results are in boldface. and the \underline{second-best} results are in underline.
\vspace{-10pt}
}
\label{tab:planning_3}
\end{center}
%\vspace{-15pt}
\end{table}   


\section{Detailed Ablation Results}
\cref{tab:detailed_ablation_v2v} shows the detailed ablation results when using only the scene-level features or only the object-level features as input to our \namemethod. Both types of input features contribute to the final performance, and object-level features are easier for LLM to digest. Training from scratch achieves worse performance, meaning that pre-training with LLaVA's VQA tasks improves our \namemethod's performance in \namedataset.

% single ablation table, but ignore some detailed columns
\begin{table*}[!ht]
\small
\setlength{\tabcolsep}{2pt}
%\renewcommand{\arraystretch}{0.7}
\begin{center}
\begin{tabular}{l ccc ccc ccc c ccc cc c}
  \hline
  \hline
  \multirow{2}{*}{Method} &
  \multicolumn{3}{c}{Q1} & \multicolumn{3}{c}{Q2} & \multicolumn{3}{c}{Q3} & \multicolumn{1}{c}{Q\textsubscript{Gr}} & \multicolumn{3}{c}{Q4} & \multicolumn{2}{c}{Q5} & \multirow{2}{*}{Comm (MB) $\downarrow$} \\
  %\cline{2-10}
  \cmidrule(lr){2-4} \cmidrule(lr){5-7} \cmidrule(lr){8-10} \cmidrule(lr){11-11} \cmidrule(lr){12-14} \cmidrule(lr){15-16}
  &
  F1 $\uparrow$ & P $\uparrow$ & R $\uparrow$ &
  F1 $\uparrow$ & P $\uparrow$ & R $\uparrow$ &
  F1 $\uparrow$ & P $\uparrow$ & R $\uparrow$ &
  F1 $\uparrow$ &
  F1 $\uparrow$ & P $\uparrow$ & R $\uparrow$ & 
  L2$_{avg}$ (m) $\downarrow$ & CR$_{avg}$ (\%) $\downarrow$ \\
  \hline
  \hline
  Scene-level only         & 69.9 & 74.9 & \textbf{65.5} & 15.4 & 19.9 & 12.6 & 17.9 & \textbf{26.9} & 13.5 & 34.4 & 43.2 & 40.2 & 46.7 & 7.21 & 15.55 & 0.4008 \\
  Object-level only      & 69.0 & \textbf{80.9} & 60.1 & 26.9 & 34.7 & 21.9 & 17.6 & 18.3 & 16.9 & 37.8 & 52.6 & 57.3 & 48.6 & 5.24 & 7.78 & \textbf{0.0068} \\
  \hline
  Scratch & 67.6 & 77.6 & 60.0 & 26.5 & 26.4 & 26.5 & 17.2 & 16.4 & 18.2 & 37.1 & 49.3 & 52.7 & 46.3 & 6.30 & 5.01 & 0.4068 \\
  \hline
  \namemethod~(ours)     & \textbf{70.0} & 80.1 & 62.2 & \textbf{30.8} & \textbf{36.3} & \textbf{26.7} & \textbf{21.2} & 21.5 & \textbf{20.8} & \textbf{40.7} & \textbf{59.7} & \textbf{61.9} & \textbf{57.6} & \textbf{4.99} & \textbf{3.00} & 0.4068 \\
  \hline
\end{tabular}
\vspace{-5pt}
\caption{
Ablation study in \namedataset's \namevsplit. Q1: Grounding at a reference location. Q2: Grounding behind a reference object at a location. Q3: Grounding behind a reference object in a direction. Q\textsubscript{Gr}: Average of grounding (Q1, Q2, and Q3). Q4: Notable object identification. Q5: Planning. P: Precision. R: Recall. L2: L2 distance error. CR: Collision rate. Comm: Communication cost.
%In each column, the \textbf{best} results are in boldface.
\vspace{-10pt}
}
\label{tab:detailed_ablation_v2v}
\end{center}
\vspace{-10pt}
\end{table*} 
 
% Moved to main 
%\section{Evaluation Metric Details}

\section{Additional Dataset Statistics}

For the grounding questions (Q1, Q2, Q3) and the notable object identification question (Q4), a QA pair can be categorized into either a positive case or a negative case. If at least one object exists that satisfies the condition specified in the questions, the corresponding QA pair is a positive case. Otherwise, it is a negative case. \cref{tab:stats_v2v_pos_neg,tab:stats_v2x_pos_neg} summarizes the numbers of QA pairs in each category, for \namedataset's \namevsplit~and \namexsplit~respectively. This table shows that \namedataset~has sufficient positive and negative data samples in both training and testing sets for each of these QA pairs. The planning task (Q5) is excluded from \cref{tab:stats_v2v_pos_neg,tab:stats_v2x_pos_neg}, as each planning QA pair inherently includes a ground-truth future trajectory in its corresponding answer.




\begin{table}[!t]
\small
\setlength{\tabcolsep}{3.5pt}
%\renewcommand{\arraystretch}{0.7}
\begin{center}
\begin{tabular}{c | rr | rr | r}
  \hline
  \hline
  QA type & Train-Pos & Train-Neg & Test-Pos & Test-Neg & Total \\
  \hline
  \hline
  Q1 & 217403 & 137417 & 76522 & 44861 & 476203 \\
  Q2 &  17859 &  17841 &  8391 &  5491 &  49582 \\
  Q3 &   7197 &   7142 &  3082 &  2015 &  19436 \\
  Q4 &   9911 &   2379 &  2517 &   929 &  15736 \\
  \hline
  Total & 252370 & 164779 & 90512 & 53296 & 560957 
\end{tabular}
\vspace{-10pt}
\caption{
Dataset statistics of our \namedataset's \namevsplit~on positive and negative samples. 
%Q1: Grounding at a reference location. Q2: Grounding behind a reference object at a location. Q3: Grounding behind a reference object in a direction. Q4: Notable object identification.
}
\label{tab:stats_v2v_pos_neg}
\end{center}
\vspace{-10pt}
\end{table}


\begin{table}[!t]
\small
\setlength{\tabcolsep}{3.5pt}
%\renewcommand{\arraystretch}{0.7}
\begin{center}
\begin{tabular}{c | rr | rr | r}
  \hline
  \hline
  QA type & Train-Pos & Train-Neg & Test-Pos & Test-Neg & Total \\
  \hline
  \hline
  Q1 & 247447 & 247843 & 62332 & 66379 & 624001 \\
  Q2 &  84005 &  83689 & 18297 & 16936 & 202927 \\
  Q3 &  14346 &  14394 &  3421 &  3044 &  35205 \\
  Q4 &   4624 &   1650 &  1172 &   536 &  7982 \\
  \hline
  Total & 350422 & 347576 & 85222 & 86895 & 870115 
\end{tabular}
\vspace{-10pt}
\caption{
Dataset statistics of our \namedataset's \namexsplit~on positive and negative samples. 
%Q1: Grounding at a reference location. Q2: Grounding behind a reference object at a location. Q3: Grounding behind a reference object in a direction. Q4: Notable object identification.
}
\label{tab:stats_v2x_pos_neg}
\end{center}
\vspace{-10pt}
\end{table} 
  
 
We also visualize our \namevsplit~distribution of ground truth answer locations relative to the asking CAV for the grounding questions (Q1, Q2, Q3) and the notable object identification question (Q4), as shown in \cref{fig:stats_v2v_q1,fig:stats_v2v_q2,fig:stats_v2v_q3,fig:stats_v2v_q4}. In our coordinate system, $x$ is the CAV's front direction, and $y$ is the CAV's right direction. For the planning question (Q5), we show the distribution of the ending waypoints in the ground truth answer future trajectories, as shown in \cref{fig:stats_v2v_q5}. We visualize the location distribution of \namedataset's \namexsplit in \cref{fig:stats_v2x_q1,fig:stats_v2x_q2,fig:stats_v2x_q3,fig:stats_v2x_q4,fig:stats_v2x_q5}. These figures indicate that our \namedataset~has diverse spatial distributions in the driving scenes. Compared to NuScenes~\cite{caesar2019nuscenes},
%which is commonly used in single-car LLM-based autonomous driving research~\cite{deruyttere2019talk2car,wu2023nuprompt,qian2023nuscenesqa,ding2024nuinstruct,nie2023reason2drive,tian2024token,wang2024omnidrive}, 
our ~\namedataset~has larger ranges and standard deviations of the ground-truth ending waypoints, as shown in \cref{tab:stats_range_std_q5}. Therefore, the planning task in our ~\namedataset~is more challenging.

 


% Ending waypoints stats
\begin{table}[!th]
%\vspace{-10pt}
%\captionsetup{font=tiny}
\small
%\setlength{\tabcolsep}{4pt}
%\renewcommand{\arraystretch}{0.7}
\begin{center}
{%\tiny
\begin{tabular}{l ccc ccc }
  \hline
  \hline
  \multirow{2}{*}{Dataset} &
  \multicolumn{3}{c}{x: forward} & \multicolumn{3}{c}{y: right} \\
  %\cline{2-10}
  \cmidrule(lr){2-4} \cmidrule(lr){5-7}
  &
  min & max & std & 
  min & max & std \\
  \hline
  \hline
  NuScenes & -0.9 & 39.7 & 10.4 & -11.0 & 11.1 & 1.9 \\
  % v2v-split stats
  %\namedataset~(ours) & -2.1 & 177.0 & 28.5 & -24.3 & 12.0 & 2.8 \\
  % combine v2v-split and v2x-split
  \namedataset~(ours) & -2.1 & 177.0 & 28.1 & -24.3 & 12.0 & 2.4 \\
  \hline
\end{tabular}
}
\vspace{-10pt}
\caption{
Ranges and standard deviations of ground-truth ending waypoints.
}
\label{tab:stats_range_std_q5}
\end{center}
\vspace{-15pt}
\end{table}   

 

% Distribution of answer location of v2v-split
\begin{figure}[!t]
        \centering
        \begin{subfigure}[t]{0.23\textwidth}
            \centering 
            \includegraphics[width=\textwidth]{figure/stats/q1_x.png}
            \vspace{-20pt}
            \caption[]%
            {{x (meters)}}    
        \end{subfigure}
        \hfill
        \begin{subfigure}[t]{0.23\textwidth}  
            \centering 
            \includegraphics[width=\textwidth]{figure/stats/q1_y.png}
            \vspace{-20pt}
            \caption[]%
            {{y (meters)}}
        \end{subfigure}

        %\bigskip
        \begin{subfigure}[t]{0.23\textwidth}
            \centering 
            \includegraphics[width=\textwidth]{figure/stats/q1_dist.png}
            \vspace{-20pt}
            \caption[]%
            {{distance (meters)}}
        \end{subfigure}
        \hfill
        \begin{subfigure}[t]{0.23\textwidth}
            \centering 
            \includegraphics[width=\textwidth]{figure/stats/q1_angle.png}
            \vspace{-20pt}
            \caption[]%
            {{angle (degrees)}}
        \end{subfigure}
        \hfill
        
        \vspace{-10pt}
        \caption[]
        {
        The distribution of ground-truth answer locations relative to CAV in \namedataset's \namevsplit~Q1: Grounding at a reference location. 
        } 
        \label{fig:stats_v2v_q1}
        \vspace{-10pt}
\end{figure}


\begin{figure}[!t]
        \centering
        \begin{subfigure}[t]{0.23\textwidth}
            \centering 
            \includegraphics[width=\textwidth]{figure/stats/q2_x.png}
            \vspace{-20pt}
            \caption[]%
            {{x (meters)}}    
        \end{subfigure}
        \hfill
        \begin{subfigure}[t]{0.23\textwidth}  
            \centering 
            \includegraphics[width=\textwidth]{figure/stats/q2_y.png}
            \vspace{-20pt}
            \caption[]%
            {{y (meters)}}
        \end{subfigure}

        %\bigskip
        \begin{subfigure}[t]{0.23\textwidth}
            \centering 
            \includegraphics[width=\textwidth]{figure/stats/q2_dist.png}
            \vspace{-20pt}
            \caption[]%
            {{distance (meters)}}
        \end{subfigure}
        \hfill
        \begin{subfigure}[t]{0.23\textwidth}
            \centering 
            \includegraphics[width=\textwidth]{figure/stats/q2_angle.png}
            \vspace{-20pt}
            \caption[]%
            {{angle (degrees)}}
        \end{subfigure}
        \hfill
        
        \vspace{-10pt}
        \caption[]
        {
        The distribution of ground-truth answer locations relative to CAV in \namedataset's \namevsplit~Q2: Grounding behind a reference object at a location. 
        } 
        \label{fig:stats_v2v_q2}
        \vspace{-10pt}
\end{figure}

\begin{figure}[!t]
        \centering
        \begin{subfigure}[t]{0.23\textwidth}
            \centering 
            \includegraphics[width=\textwidth]{figure/stats/q3_x.png}
            \vspace{-20pt}
            \caption[]%
            {{x (meters)}}    
        \end{subfigure}
        \hfill
        \begin{subfigure}[t]{0.23\textwidth}  
            \centering 
            \includegraphics[width=\textwidth]
            {figure/stats/q3_y.png}
            \vspace{-20pt}
            \caption[]%
            {{y (meters)}}
        \end{subfigure}

        %\bigskip
        \begin{subfigure}[t]{0.23\textwidth}
            \centering 
            \includegraphics[width=\textwidth]{figure/stats/q3_dist.png}
            \vspace{-20pt}
            \caption[]%
            {{distance (meters)}}
        \end{subfigure}
        \hfill
        \begin{subfigure}[t]{0.23\textwidth}
            \centering 
            \includegraphics[width=\textwidth]{figure/stats/q3_angle.png}
            \vspace{-20pt}
            \caption[]%
            {{angle (degrees)}}
        \end{subfigure}
        \hfill
        
        \vspace{-10pt}
        \caption[]
        {
        The distribution of ground-truth answer locations relative to CAV in \namedataset's \namevsplit~Q3: Grounding behind a reference object in a direction. 
        } 
        \label{fig:stats_v2v_q3}
        \vspace{-10pt}
\end{figure}

\begin{figure}[!t]
        \centering
        \begin{subfigure}[t]{0.23\textwidth}
            \centering 
            \includegraphics[width=\textwidth]{figure/stats/q4_x.png}
            \vspace{-20pt}
            \caption[]%
            {{x (meters)}}    
        \end{subfigure}
        \hfill
        \begin{subfigure}[t]{0.23\textwidth}  
            \centering 
            \includegraphics[width=\textwidth]{figure/stats/q4_y.png}
            \vspace{-20pt}
            \caption[]%
            {{y (meters)}}
        \end{subfigure}

        %\bigskip
        \begin{subfigure}[t]{0.23\textwidth}
            \centering 
            \includegraphics[width=\textwidth]{figure/stats/q4_dist.png}
            \vspace{-20pt}
            \caption[]%
            {{distance (meters)}}
        \end{subfigure}
        \hfill
        \begin{subfigure}[t]{0.23\textwidth}
            \centering 
            \includegraphics[width=\textwidth]{figure/stats/q4_angle.png}
            \vspace{-20pt}
            \caption[]%
            {{angle (degrees)}}
        \end{subfigure}
        \hfill
        
        \vspace{-10pt}
        \caption[]
        {
        The distribution of ground-truth answer locations relative to CAV in \namedataset's \namevsplit~Q4: Notable object identification. 
        } 
        \label{fig:stats_v2v_q4}
        \vspace{-10pt}
\end{figure}

\begin{figure}[!t]
        \centering
        \begin{subfigure}[t]{0.23\textwidth}
            \centering 
            \includegraphics[width=\textwidth]{figure/stats/q5_x.png}
            \vspace{-20pt}
            \caption[]%
            {{x (meters)}}    
        \end{subfigure}
        \hfill
        \begin{subfigure}[t]{0.23\textwidth}  
            \centering 
            \includegraphics[width=\textwidth]{figure/stats/q5_y.png}
            \vspace{-20pt}
            \caption[]%
            {{y (meters)}}
        \end{subfigure}

        %\bigskip
        \begin{subfigure}[t]{0.23\textwidth}
            \centering 
            \includegraphics[width=\textwidth]{figure/stats/q5_dist.png}
            \vspace{-20pt}
            \caption[]%
            {{distance (meters)}}
        \end{subfigure}
        \hfill
        \begin{subfigure}[t]{0.23\textwidth}
            \centering 
            \includegraphics[width=\textwidth]{figure/stats/q5_angle.png}
            \vspace{-20pt}
            \caption[]%
            {{angle (degrees)}}
        \end{subfigure}
        \hfill
        
        \vspace{-10pt}
        \caption[]
        {
        The distribution of ground-truth answer locations relative to CAV in \namedataset's \namevsplit~Q5: Planning. 
        } 
        \label{fig:stats_v2v_q5}
        \vspace{-10pt}
\end{figure}


% Distribution of answer location of v2x-split
\begin{figure}[!t]
        \centering
        \begin{subfigure}[t]{0.23\textwidth}
            \centering 
            \includegraphics[width=\textwidth]{figure/stats_v2x/q1_x.png}
            \vspace{-20pt}
            \caption[]%
            {{x (meters)}}    
        \end{subfigure}
        \hfill
        \begin{subfigure}[t]{0.23\textwidth}  
            \centering 
            \includegraphics[width=\textwidth]{figure/stats_v2x/q1_y.png}
            \vspace{-20pt}
            \caption[]%
            {{y (meters)}}
        \end{subfigure}

        %\bigskip
        \begin{subfigure}[t]{0.23\textwidth}
            \centering 
            \includegraphics[width=\textwidth]{figure/stats_v2x/q1_dist.png}
            \vspace{-20pt}
            \caption[]%
            {{distance (meters)}}
        \end{subfigure}
        \hfill
        \begin{subfigure}[t]{0.23\textwidth}
            \centering 
            \includegraphics[width=\textwidth]{figure/stats_v2x/q1_angle.png}
            \vspace{-20pt}
            \caption[]%
            {{angle (degrees)}}
        \end{subfigure}
        \hfill
        
        \vspace{-10pt}
        \caption[]
        {
        The distribution of ground-truth answer locations relative to CAV in \namedataset's \namexsplit~Q1: Grounding at a reference location. 
        } 
        \label{fig:stats_v2x_q1}
        \vspace{-10pt}
\end{figure}


\begin{figure}[!t]
        \centering
        \begin{subfigure}[t]{0.23\textwidth}
            \centering 
            \includegraphics[width=\textwidth]{figure/stats_v2x/q2_x.png}
            \vspace{-20pt}
            \caption[]%
            {{x (meters)}}    
        \end{subfigure}
        \hfill
        \begin{subfigure}[t]{0.23\textwidth}  
            \centering 
            \includegraphics[width=\textwidth]{figure/stats_v2x/q2_y.png}
            \vspace{-20pt}
            \caption[]%
            {{y (meters)}}
        \end{subfigure}

        %\bigskip
        \begin{subfigure}[t]{0.23\textwidth}
            \centering 
            \includegraphics[width=\textwidth]{figure/stats_v2x/q2_dist.png}
            \vspace{-20pt}
            \caption[]%
            {{distance (meters)}}
        \end{subfigure}
        \hfill
        \begin{subfigure}[t]{0.23\textwidth}
            \centering 
            \includegraphics[width=\textwidth]{figure/stats_v2x/q2_angle.png}
            \vspace{-20pt}
            \caption[]%
            {{angle (degrees)}}
        \end{subfigure}
        \hfill
        
        \vspace{-10pt}
        \caption[]
        {
        The distribution of ground-truth answer locations relative to CAV in \namedataset's \namexsplit~Q2: Grounding behind a reference object at a location. 
        } 
        \label{fig:stats_v2x_q2}
        \vspace{-10pt}
\end{figure}

\begin{figure}[!t]
        \centering
        \begin{subfigure}[t]{0.23\textwidth}
            \centering 
            \includegraphics[width=\textwidth]{figure/stats_v2x/q3_x.png}
            \vspace{-20pt}
            \caption[]%
            {{x (meters)}}    
        \end{subfigure}
        \hfill
        \begin{subfigure}[t]{0.23\textwidth}  
            \centering 
            \includegraphics[width=\textwidth]
            {figure/stats_v2x/q3_y.png}
            \vspace{-20pt}
            \caption[]%
            {{y (meters)}}
        \end{subfigure}

        %\bigskip
        \begin{subfigure}[t]{0.23\textwidth}
            \centering 
            \includegraphics[width=\textwidth]{figure/stats_v2x/q3_dist.png}
            \vspace{-20pt}
            \caption[]%
            {{distance (meters)}}
        \end{subfigure}
        \hfill
        \begin{subfigure}[t]{0.23\textwidth}
            \centering 
            \includegraphics[width=\textwidth]{figure/stats_v2x/q3_angle.png}
            \vspace{-20pt}
            \caption[]%
            {{angle (degrees)}}
        \end{subfigure}
        \hfill
        
        \vspace{-10pt}
        \caption[]
        {
        The distribution of ground-truth answer locations relative to CAV in \namedataset's \namexsplit~Q3: Grounding behind a reference object in a direction. 
        } 
        \label{fig:stats_v2x_q3}
        \vspace{-10pt}
\end{figure}

\begin{figure}[!t]
        \centering
        \begin{subfigure}[t]{0.23\textwidth}
            \centering 
            \includegraphics[width=\textwidth]{figure/stats_v2x/q4_x.png}
            \vspace{-20pt}
            \caption[]%
            {{x (meters)}}    
        \end{subfigure}
        \hfill
        \begin{subfigure}[t]{0.23\textwidth}  
            \centering 
            \includegraphics[width=\textwidth]{figure/stats_v2x/q4_y.png}
            \vspace{-20pt}
            \caption[]%
            {{y (meters)}}
        \end{subfigure}

        %\bigskip
        \begin{subfigure}[t]{0.23\textwidth}
            \centering 
            \includegraphics[width=\textwidth]{figure/stats_v2x/q4_dist.png}
            \vspace{-20pt}
            \caption[]%
            {{distance (meters)}}
        \end{subfigure}
        \hfill
        \begin{subfigure}[t]{0.23\textwidth}
            \centering 
            \includegraphics[width=\textwidth]{figure/stats_v2x/q4_angle.png}
            \vspace{-20pt}
            \caption[]%
            {{angle (degrees)}}
        \end{subfigure}
        \hfill
        
        \vspace{-10pt}
        \caption[]
        {
        The distribution of ground-truth answer locations relative to CAV in \namedataset's \namexsplit~Q4: Notable object identification. 
        } 
        \label{fig:stats_v2x_q4}
        \vspace{-10pt}
\end{figure}

\begin{figure}[!t]
        \centering
        \begin{subfigure}[t]{0.23\textwidth}
            \centering 
            \includegraphics[width=\textwidth]{figure/stats_v2x/q5_x.png}
            \vspace{-20pt}
            \caption[]%
            {{x (meters)}}    
        \end{subfigure}
        \hfill
        \begin{subfigure}[t]{0.23\textwidth}  
            \centering 
            \includegraphics[width=\textwidth]{figure/stats_v2x/q5_y.png}
            \vspace{-20pt}
            \caption[]%
            {{y (meters)}}
        \end{subfigure}

        %\bigskip
        \begin{subfigure}[t]{0.23\textwidth}
            \centering 
            \includegraphics[width=\textwidth]{figure/stats_v2x/q5_dist.png}
            \vspace{-20pt}
            \caption[]%
            {{distance (meters)}}
        \end{subfigure}
        \hfill
        \begin{subfigure}[t]{0.23\textwidth}
            \centering 
            \includegraphics[width=\textwidth]{figure/stats_v2x/q5_angle.png}
            \vspace{-20pt}
            \caption[]%
            {{angle (degrees)}}
        \end{subfigure}
        \hfill
        
        \vspace{-10pt}
        \caption[]
        {
        The distribution of ground-truth answer locations relative to CAV in \namedataset's \namexsplit~Q5: Planning. 
        } 
        \label{fig:stats_v2x_q5}
        \vspace{-10pt}
\end{figure}


   

\section{Additional Qualitative Results}

%\TODO{Q5, also draw the ending waypoints.  Zoom-in to show the difference, remove the results that all models have the same results. Some failure cases. More QA that models have different results.}

We show more qualitative results of our proposed \namemethod~and other baseline methods in the testing set of \namedataset's grounding task in Figs. \ref{fig:supp_q1} to \ref{fig:supp_q3_2}, notable object identification task in Figs. \ref{fig:supp_q4_1} to \ref{fig:supp_q4_2}, and planning task in Figs \ref{fig:supp_q5_1} to \ref{fig:supp_q5_2}. The baseline methods include no-fusion, early-fusion, and intermediate-fusion: AttFuse~\cite{xu2022opencood}, V2X-ViT~\cite{xu2022v2xvit}, and CoBEVT~\cite{xu2022cobevt}.
% cref does not work
%\cref{fig:supp_q1,fig:supp_q2,fig:supp_q3_1, fig:supp_q3_2} show the grounding results,
Results of \namexsplit~can be seen in Figs. \ref{fig:supp_v2x_q1} to \ref{fig:supp_v2x_q5_2}.
In general, our proposed ~\namemethod's outputs are closer to the ground-truth answers, in comparison to other baseline methods' results.


%Q1 double image size
\begin{figure*}[!t]
\centering
\includegraphics[width=1\textwidth]{figure/supp/supp_q1_focus_6.pdf}
%\vspace{-20pt}
\caption[]
        {\namemethod~and baseline methods' \textit{grounding} results on \namedataset's \namevsplit~testing set.~\textcolor{magenta}{Magenta $\circ$}: reference locations in questions. \textcolor{olive}{Yellow $+$}: model output locations. \textcolor{Green}{Green $\circ$}: ground-truth answers.} 
        \label{fig:supp_q1}
        \vspace{-5pt}
\end{figure*}

\begin{figure*}[!t]
\centering
\includegraphics[width=1\textwidth]{figure/supp/supp_q2_focus_6.pdf}
%\vspace{-20pt}
\caption[]
        {\namemethod~and baseline methods' \textit{grounding} results on \namedataset's \namevsplit~testing set.~\textcolor{magenta}{Magenta $\circ$}: reference locations in questions. \textcolor{olive}{Yellow $+$}: model output locations. \textcolor{Green}{Green $\circ$}: ground-truth answers.} 
        \label{fig:supp_q2}
        \vspace{-5pt}
\end{figure*}

\begin{figure*}[!t]
\centering
\includegraphics[width=1\textwidth]{figure/supp/supp_q3_1_focus_6.pdf}
%\vspace{-20pt}
\caption[]
        {\namemethod~and baseline methods' \textit{grounding} results on \namedataset's \namevsplit~testing set.~\textcolor{magenta}{Magenta $\circ$}: reference locations in questions. \textcolor{olive}{Yellow $+$}: model output locations. \textcolor{Green}{Green $\circ$}: ground-truth answers.} 
        \label{fig:supp_q3_1}
        \vspace{-5pt}
\end{figure*}

\begin{figure*}[!t]
\centering
\includegraphics[width=1\textwidth]{figure/supp/supp_q3_2_focus_6.pdf}
%\vspace{-20pt}
\caption[]
        {\namemethod~and baseline methods' \textit{grounding} results on \namedataset's \namevsplit~testing set.~\textcolor{magenta}{Magenta $\circ$}: reference locations in questions. \textcolor{olive}{Yellow $+$}: model output locations. \textcolor{Green}{Green $\circ$}: ground-truth answers.} 
        \label{fig:supp_q3_2}
        \vspace{-5pt}
\end{figure*}

\begin{figure*}[!t]
\centering
\includegraphics[width=1\textwidth]{figure/supp/supp_q4_1_focus_6.pdf}
%\vspace{-20pt}
\caption[]
        {\namemethod~and baseline methods' \textit{notable object identification} results on \namedataset's \namevsplit~testing set.~\textcolor{magenta}{Magenta curve}: planned future trajectories in questions. \textcolor{Green}{Green $\circ$}: ground-truth notable object locations. \textcolor{olive}{Yellow $+$}: model identification outputs.} 
        \label{fig:supp_q4_1}
        \vspace{-5pt}
\end{figure*}

\begin{figure*}[!t]
\centering
\includegraphics[width=1\textwidth]{figure/supp/supp_q4_2_focus_6.pdf}
%\vspace{-20pt}
\caption[]
        {\namemethod~and baseline methods' \textit{notable object identification} results on \namedataset's \namevsplit~testing set.~\textcolor{magenta}{Magenta curve}: planned future trajectories in questions. \textcolor{Green}{Green $\circ$}: ground-truth notable object locations. \textcolor{cyan}{Cyan $\times$}: model identification outputs.} 
        \label{fig:supp_q4_2}
        \vspace{-5pt}
\end{figure*}


\begin{figure*}[!t]
\centering
\includegraphics[width=1\textwidth]{figure/supp/supp_q5_1_focus_6.pdf}
%\vspace{-20pt}
\caption[]
        {\namemethod~and baseline methods' \textit{planning} results on \namedataset's \namevsplit~testing set.
         \textcolor{Green}{Green curve}: future trajectories in ground-truth answers. \textcolor{Green}{Green $\circ$}: ending waypoints in ground-truth answers. \textcolor{olive}{Yellow curve}: model planning outputs. \textcolor{olive}{Yellow $\times$}: ending waypoints in model outputs.} 
        \label{fig:supp_q5_1}
        \vspace{-5pt}
\end{figure*}

\begin{figure*}[!t]
\centering
\includegraphics[width=1\textwidth]{figure/supp/supp_q5_2_focus_6.pdf}
%\vspace{-20pt}
\caption[]
        {\namemethod~and baseline methods' \textit{planning} results on \namedataset's \namevsplit~testing set.
         \textcolor{Green}{Green curve}: future trajectories in ground-truth answers. \textcolor{Green}{Green $\circ$}: ending waypoints in ground-truth answers. \textcolor{cyan}{Cyan curve}: model planning outputs. \textcolor{cyan}{Cyan $\times$}: ending waypoints in model outputs.} 
        \label{fig:supp_q5_2}
        \vspace{-5pt}
\end{figure*}

\begin{figure*}[!t]
\centering
\includegraphics[width=1\textwidth]{figure/supp_v2x/supp_q1.pdf}
%\vspace{-20pt}
\caption[]
        {\namemethod~and baseline methods' \textit{grounding} results on \namedataset's \namexsplit~testing set.~\textcolor{magenta}{Magenta $\circ$}: reference locations in questions. \textcolor{olive}{Yellow $\times$}: model output locations. \textcolor{Green}{Green $\circ$}: ground-truth answers.} 
        \label{fig:supp_v2x_q1}
        \vspace{-5pt}
\end{figure*}

\begin{figure*}[!t]
\centering
\includegraphics[width=1\textwidth]{figure/supp_v2x/supp_q2.pdf}
%\vspace{-20pt}
\caption[]
        {\namemethod~and baseline methods' \textit{grounding} results on \namedataset's \namexsplit~testing set.~\textcolor{magenta}{Magenta $\circ$}: reference locations in questions. \textcolor{olive}{Yellow $\times$}: model output locations. \textcolor{Green}{Green $\circ$}: ground-truth answers.} 
        \label{fig:supp_v2x_q2}
        \vspace{-5pt}
\end{figure*}

\begin{figure*}[!t]
\centering
\includegraphics[width=1\textwidth]{figure/supp_v2x/supp_q3_1.pdf}
%\vspace{-20pt}
\caption[]
        {\namemethod~and baseline methods' \textit{grounding} results on \namedataset's \namexsplit~testing set.~\textcolor{magenta}{Magenta $\circ$}: reference locations in questions. \textcolor{olive}{Yellow $\times$}: model output locations. \textcolor{Green}{Green $\circ$}: ground-truth answers.} 
        \label{fig:supp_v2x_q3_1}
        \vspace{-5pt}
\end{figure*}


\begin{figure*}[!t]
\centering
\includegraphics[width=1\textwidth]{figure/supp_v2x/supp_q4_1.pdf}
%\vspace{-20pt}
\caption[]
        {\namemethod~and baseline methods' \textit{notable object identification} results on \namedataset's \namexsplit~testing set.~\textcolor{magenta}{Magenta curve}: planned future trajectories in questions. \textcolor{Green}{Green $\circ$}: ground-truth notable object locations. \textcolor{olive}{Yellow $\times$}: model identification outputs.} 
        \label{fig:supp_v2x_q4_1}
        \vspace{-5pt}
\end{figure*}

\begin{figure*}[!t]
\centering
\includegraphics[width=1\textwidth]{figure/supp_v2x/supp_q4_2.pdf}
%\vspace{-20pt}
\caption[]
        {\namemethod~and baseline methods' \textit{notable object identification} results on \namedataset's \namexsplit~testing set.~\textcolor{magenta}{Magenta curve}: planned future trajectories in questions. \textcolor{Green}{Green $\circ$}: ground-truth notable object locations. \textcolor{cyan}{Cyan $\times$}: model identification outputs.} 
        \label{fig:supp_v2x_q4_2}
        \vspace{-5pt}
\end{figure*}


\begin{figure*}[!t]
\centering
\includegraphics[width=1\textwidth]{figure/supp_v2x/supp_q5_1.pdf}
%\vspace{-20pt}
\caption[]
        {\namemethod~and baseline methods' \textit{planning} results on \namedataset's 
\namexsplit~testing set.
         \textcolor{Green}{Green curve}: future trajectories in ground-truth answers. \textcolor{Green}{Green $\circ$}: ending waypoints in ground-truth answers. \textcolor{olive}{Yellow curve}: model planning outputs. \textcolor{olive}{Yellow $\times$}: ending waypoints in model outputs.} 
        \label{fig:supp_v2x_q5_1}
        \vspace{-5pt}
\end{figure*}

\begin{figure*}[!t]
\centering
\includegraphics[width=1\textwidth]{figure/supp_v2x/supp_q5_2.pdf}
%\vspace{-20pt}
\caption[]
        {\namemethod~and baseline methods' \textit{planning} results on \namedataset's 
\namexsplit~testing set.
         \textcolor{Green}{Green curve}: future trajectories in ground-truth answers. \textcolor{Green}{Green $\circ$}: ending waypoints in ground-truth answers. \textcolor{cyan}{Cyan curve}: model planning outputs. \textcolor{cyan}{Cyan $\times$}: ending waypoints in model outputs.} 
        \label{fig:supp_v2x_q5_2}
        \vspace{-5pt}
\end{figure*}


\section{Limitation}
\cref{fig:supp_q5_failure} shows failure cases of \namemethod's \textit{planning} results on \namedataset's testing set. In a few frames, the model generates future trajectories in the lane of the opposite traffic direction. A potential solution and future work is to include HD map information as additional input to the model. Currently, this approach is not feasible because the base dataset V2V4Real~\cite{xu2023v2v4real} has not released its HD map to the public.


\begin{figure*}[!t]
\centering
\includegraphics[width=1\textwidth]{figure/supp/supp_q5_failure.pdf}
%\vspace{-20pt}
\caption[]
        {Failure cases of \namemethod's \textit{planning} results on \namedataset's testing set.
         \textcolor{Green}{Green curve}: future trajectories in ground-truth answers. \textcolor{Green}{Green $\circ$}: ending waypoints in ground-truth answers. \textcolor{olive}{Yellow curve} and \textcolor{cyan}{Cyan curve}: model planning outputs corresponding to \textcolor{olive}{CAV\_EGO} and \textcolor{cyan}{CAV\_1}, respectively. \textcolor{olive}{Yellow $\times$} and \textcolor{cyan}{Cyan $\times$}: ending waypoints in model outputs corresponding to \textcolor{olive}{CAV\_EGO} and \textcolor{cyan}{CAV\_1}, respectively.} 
        \label{fig:supp_q5_failure}
        \vspace{-5pt}
\end{figure*}








% \begin{figure*}[!t]
% \centering
% \includegraphics[width=1\textwidth]{figure/supp/supp_grounding_1116.pdf}
% %\vspace{-20pt}
% \caption[]
%         {\namemethod~and baseline methods' \textit{grounding} results on \namedataset's testing set.~\textcolor{magenta}{Magenta $\circ$}: reference locations in questions. \textcolor{olive}{Yellow $+$}: model output locations. \textcolor{Green}{Green $\circ$}: ground-truth answers.} 
%         \label{fig:supp_grounding}
%         \vspace{-5pt}
% \end{figure*}

% \begin{figure*}[!t]
% \centering
% \includegraphics[width=1\textwidth]{figure/supp/supp_notable_1116.pdf}
% %\vspace{-20pt}
% \caption[]
%         {\namemethod~and baseline methods' \textit{notable object identification} results on \namedataset's testing set.~\textcolor{magenta}{Magenta curve}: planned future trajectories in questions. \textcolor{Green}{Green $\circ$}: ground-truth notable object locations. \textcolor{olive}{Yellow $+$} and \textcolor{cyan}{Cyan $\times$}: model identification outputs corresponding to \textcolor{olive}{CAV\_EGO} and \textcolor{cyan}{CAV\_1}, respectively.} 
%         \label{fig:supp_notable}
%         \vspace{-5pt}
% \end{figure*}

% \begin{figure*}[!t]
% \centering
% \includegraphics[width=1\textwidth]{figure/supp/supp_planning_1116.pdf}
% %\vspace{-20pt}
% \caption[]
%         {\namemethod~and baseline methods' \textit{planning} results on \namedataset's testing set.\textcolor{Green}{Green line}: future trajectories in ground-truth answers. \textcolor{olive}{Yellow curve} and \textcolor{cyan}{Cyan curve}: model planning outputs corresponding to \textcolor{olive}{CAV\_EGO} and \textcolor{cyan}{CAV\_1}, respectively.} 
%         \label{fig:supp_planning}
%         \vspace{-5pt}
% \end{figure*}




% The followings contain one question per figure

% \begin{figure*}[!t]
% \centering
% \includegraphics[width=1\textwidth]{figure/supp/supp_q1.pdf}
% %\vspace{-20pt}
% \caption[]
%         {\namemethod~and baseline methods' Q1: \textit{ Grounding at a reference location} results on \namedataset's testing set.~\textcolor{magenta}{Magenta $\circ$}: reference locations in questions. \textcolor{olive}{Yellow $+$}: model output locations. \textcolor{Green}{Green $\circ$}: ground-truth answers.} 
%         \label{fig:supp_q1}
%         \vspace{-5pt}
% \end{figure*}

% \begin{figure*}[!t]
% \centering
% \includegraphics[width=1\textwidth]{figure/supp/supp_q2.pdf}
% %\vspace{-20pt}
% \caption[]
%         {\namemethod~and baseline methods' Q2: \textit{ Grounding behind a reference object at a location} results on \namedataset's testing set.~\textcolor{magenta}{Magenta $\circ$}: reference locations in questions. \textcolor{olive}{Yellow $+$}: model output locations. \textcolor{Green}{Green $\circ$}: ground-truth answers.} 
%         \label{fig:supp_q2}
%         \vspace{-5pt}
% \end{figure*} 

% \begin{figure*}[!t]
% \centering
% \includegraphics[width=1\textwidth]{figure/supp/supp_q3_1.pdf}
% %\vspace{-20pt}
% \caption[]
%         {\namemethod~and baseline methods' Q3: \textit{ Grounding behind a reference object in a direction} results on \namedataset's testing set.~\textcolor{magenta}{Magenta $\circ$}: reference locations in questions. \textcolor{olive}{Yellow $+$}: model output locations. \textcolor{Green}{Green $\circ$}: ground-truth answers.} 
%         \label{fig:supp_q3_1}
%         \vspace{-5pt}
% \end{figure*} 

% \begin{figure*}[!t]
% \centering
% \includegraphics[width=1\textwidth]{figure/supp/supp_q3_2.pdf}
% %\vspace{-20pt}
% \caption[]
%         {\namemethod~and baseline methods' Q3: \textit{ Grounding behind a reference object in a direction} results on \namedataset's testing set.~\textcolor{magenta}{Magenta $\circ$}: reference locations in questions. \textcolor{olive}{Yellow $+$}: model output locations. \textcolor{Green}{Green $\circ$}: ground-truth answers.} 
%         \label{fig:supp_q3_2}
%         \vspace{-5pt}
% \end{figure*} 

% \begin{figure*}[!t]
% \centering
% \includegraphics[width=1\textwidth]{figure/supp/supp_q4_1.pdf}
% %\vspace{-20pt}
% \caption[]
%         {\namemethod~and baseline methods' \textit{notable object identification} results on \namedataset's testing set. ~\textcolor{magenta}{Magenta curve}: planned future trajectories in questions. \textcolor{Green}{Green $\circ$}: ground-truth notable object locations. \textcolor{olive}{Yellow $+$}: model identification outputs.
%         }         
%         \label{fig:supp_q4_1}
%         %\vspace{-10pt}
% \end{figure*}

% \begin{figure*}[!t]
% \centering
% \includegraphics[width=1\textwidth]{figure/supp/supp_q4_2.pdf}
% %\vspace{-20pt}
% \caption[]
%         {\namemethod~and baseline methods' \textit{notable object identification} results on \namedataset's testing set. ~\textcolor{magenta}{Magenta curve}: planned future trajectories in questions. \textcolor{Green}{Green $\circ$}: ground-truth notable object locations. \textcolor{cyan}{Cyan $\times$}: model identification outputs .
%         }         
%         \label{fig:supp_q4_2}
%         %\vspace{-10pt}
% \end{figure*}

% \begin{figure*}[!t]
% \centering
% \includegraphics[width=1\textwidth]{figure/supp/supp_q5_1.pdf}
% %\vspace{-20pt}
% \caption[]
%         {\namemethod~and baseline methods' \textit{planning} results on \namedataset's testing set.
%         \textcolor{Green}{Green line}: future trajectories in ground-truth answers. \textcolor{olive}{Yellow curve}: model planning outputs.}         
%         \label{fig:supp_q5_1}
%         %\vspace{-10pt}
% \end{figure*}

% \begin{figure*}[!t]
% \centering
% \includegraphics[width=1\textwidth]{figure/supp/supp_q5_2.pdf}
% %\vspace{-20pt}
% \caption[]
%         {\namemethod~and baseline methods' \textit{planning} results on \namedataset's testing set.
%         \textcolor{Green}{Green line}: future trajectories in ground-truth answers. \textcolor{cyan}{Cyan curve}: model planning outputs.}         
%         \label{fig:supp_q5_2}
%         %\vspace{-10pt}
% \end{figure*}
   