%%
%% This is file `sample-authordraft.tex',
%% generated with the docstrip utility.
%%
%% The original source files were:
%%
%% samples.dtx  (with options: `authordraft')
%% 
%% IMPORTANT NOTICE:
%% 
%% For the copyright see the source file.
%% 
%% Any modified versions of this file must be renamed
%% with new filenames distinct from sample-authordraft.tex.
%% 
%% For distribution of the original source see the terms
%% for copying and modification in the file samples.dtx.
%% 
%% This generated file may be distributed as long as the
%% original source files, as listed above, are part of the
%% same distribution. (The sources need not necessarily be
%% in the same archive or directory.)
%%
%% Commands for TeXCount
%TC:macro \cite [option:text,text]
%TC:macro \citep [option:text,text]
%TC:macro \citet [option:text,text]
%TC:envir table 0 1
%TC:envir table* 0 1
%TC:envir tabular [ignore] word
%TC:envir displaymath 0 word
%TC:envir math 0 word
%TC:envir comment 0 0
%%
%%
%% The first command in your LaTeX source must be the \documentclass command.
\documentclass[sigconf]{acmart}
\usepackage{array}

%% NOTE that a single column version may required for 
%% submission and peer review. This can be done by changing
%% the \doucmentclass[...]{acmart} in this template to 
%% \documentclass[manuscript,screen]{acmart}
%% 
%% To ensure 100% compatibility, please check the white list of
%% approved LaTeX packages to be used with the Master Article Template at
%% https://www.acm.org/publications/taps/whitelist-of-latex-packages 
%% before creating your document. The white list page provides 
%% information on how to submit additional LaTeX packages for 
%% review and adoption.
%% Fonts used in the template cannot be substituted; margin 
%% adjustments are not allowed.

%%
%% \BibTeX command to typeset BibTeX logo in the docs
\AtBeginDocument{%
  \providecommand\BibTeX{{%
    \normalfont B\kern-0.5em{\scshape i\kern-0.25em b}\kern-0.8em\TeX}}}

%% Rights management information.  This information is sent to you
%% when you complete the rights form.  These commands have SAMPLE
%% values in them; it is your responsibility as an author to replace
%% the commands and values with those provided to you when you
%% complete the rights form.
\setcopyright{acmcopyright}
\copyrightyear{2018}
\acmYear{2018}
\acmDOI{XXXXXXX.XXXXXXX}

%% These commands are for a PROCEEDINGS abstract or paper.
\acmConference[The Web Conference 2024]{Singapore}
%
%  Uncomment \acmBooktitle if th title of the proceedings is different
%  from ``Proceedings of ...''!
%
%\acmBooktitle{Woodstock '18: ACM Symposium on Neural Gaze Detection,
%  June 03--05, 2018, Woodstock, NY} 
\acmPrice{Singapore}
\acmISBN{978-1-4503-XXXX-X/18/06}


%%
%% Submission ID.
%% Use this when submitting an article to a sponsored event. You'll
%% receive a unique submission ID from the organizers
%% of the event, and this ID should be used as the parameter to this command.
%%\acmSubmissionID{123-A56-BU3}

%%
%% For managing citations, it is recommended to use bibliography
%% files in BibTeX format.
%%
%% You can then either use BibTeX with the ACM-Reference-Format style,
%% or BibLaTeX with the acmnumeric or acmauthoryear sytles, that include
%% support for advanced citation of software artefact from the
%% biblatex-software package, also separately available on CTAN.
%%
%% Look at the sample-*-biblatex.tex files for templates showcasing
%% the biblatex styles.
%%

%%
%% For managing citations, it is recommended to use bibliography
%% files in BibTeX format.
%%
%% You can then either use BibTeX with the ACM-Reference-Format style,
%% or BibLaTeX with the acmnumeric or acmauthoryear sytles, that include
%% support for advanced citation of software artefact from the
%% biblatex-software package, also separately available on CTAN.
%%
%% Look at the sample-*-biblatex.tex files for templates showcasing
%% the biblatex styles.
%%

%%
%% The majority of ACM publications use numbered citations and
%% references.  The command \citestyle{authoryear} switches to the
%% "author year" style.
%%
%% If you are preparing content for an event
%% sponsored by ACM SIGGRAPH, you must use the "author year" style of
%% citations and references.
%% Uncommenting
%% the next command will enable that style.
%%\citestyle{acmauthoryear}

%%
%% end of the preamble, start of the body of the document source.
\begin{document}

%%
%% The "title" command has an optional parameter,
%% allowing the author to define a "short title" to be used in page headers.
\title{Semantic Ads Retrieval at Walmart eCommerce with Language Models Progressively Trained on Multiple Knowledge Domains}

%%
%% The "author" command and its associated commands are used to define
%% the authors and their affiliations.
%% Of note is the shared affiliation of the first two authors, and the
%% "authornote" and "authornotemark" commands
%% used to denote shared contribution to the research.
\author{Zhaodong Wang}
\authornote{Both authors contributed equally to this research.}
\authornote{Corresponding authors.}
\email{zhaodong.wang@walmart.com}
\author{Weizhi Du}
\orcid{0000-0001-7448-8190}
\authornotemark[1]
\authornotemark[2]
\email{weizhi.du@walmart.com}
\author{Md Omar Faruk Rokon}
\orcid{0000-0002-1385-9389}
\authornotemark[1]
\email{mdomarfaruk.rokon@walmart.com}
\affiliation{%
  \institution{Walmart Global Tech}
    \streetaddress{ 860 W California Ave.}
  \city{Sunnyvale}
  \state{California}
  \country{USA}
  \postcode{94086}
}

\author{Pooshpendu Adhikary}
\author{Yanbing Xue}
\author{Jiaxuan Xu}
\email{pooshpendu.adhikary@walmart.com}
\email{yanbing.xue@walmart.com}
\email{Jiaxuan.Xu@walmart.com}
\affiliation{%
  \institution{Walmart Global Tech}
    \streetaddress{ 860 W California Ave.}
  \city{Sunnyvale}
  \state{California}
  \country{USA}
  \postcode{94086}
}
\author{Jianghong Zhou}
\author{Kuang-chih Lee}
\author{Musen Wen}
\email{jianghong.zhou@walmart.com}
\email{Kuangchih.Lee@walmart.com}
\email{musen.wen@walmart.com}

\affiliation{%
  \institution{Walmart Global Tech}
    \streetaddress{ 860 W California Ave.}
  \city{Sunnyvale}
  \state{California}
  \country{USA}
  \postcode{94086}
}

%%
%% By default, the full list of authors will be used in the page
%% headers. Often, this list is too long, and will overlap
%% other information printed in the page headers. This command allows
%% the author to define a more concise list
%% of authors' names for this purpose.

\renewcommand{\shortauthors}{Zhaodong Wang and Weizhi Du, et al.}
%%
%% The abstract is a short summary of the work to be presented in the
%% article.
\begin{abstract}
Sponsored search in e-commerce poses several unique and complex challenges. These challenges stem from factors such as the asymmetric language structure between search queries and product names, the inherent ambiguity in user search intent, and the vast volume of sparse and imbalanced search corpus data. The role of the retrieval component within a sponsored search system is pivotal, serving as the initial step that directly affects the subsequent ranking and bidding systems. In this paper, we present an end-to-end solution tailored to optimize the ads retrieval system on Walmart.com. Our approach is to pretrain the BERT-like classification model with product category information, enhancing the model's understanding of Walmart product semantics. Second, we design a two-tower Siamese Network structure for embedding structures to augment training efficiency. Third, we introduce a Human-in-the-loop Progressive Fusion Training method to ensure robust model performance. Our results demonstrate the effectiveness of this pipeline. It enhances the search relevance metric by up to 16\% compared to a baseline DSSM-based model. Moreover, our large-scale online A/B testing demonstrates that our approach surpasses the ad revenue of the existing production model. 
\end{abstract}

%%
%% The code below is generated by the tool at http://dl.acm.org/ccs.cfm.
%% Please copy and paste the code instead of the example below.
%%
\begin{CCSXML}
<ccs2012>
 <concept>
  <concept_id>00000000.0000000.0000000</concept_id>
  <concept_desc>Do Not Use This Code, Generate the Correct Terms for Your Paper</concept_desc>
  <concept_significance>500</concept_significance>
 </concept>
 <concept>
  <concept_id>00000000.00000000.00000000</concept_id>
  <concept_desc>Do Not Use This Code, Generate the Correct Terms for Your Paper</concept_desc>
  <concept_significance>300</concept_significance>
 </concept>
 <concept>
  <concept_id>00000000.00000000.00000000</concept_id>
  <concept_desc>Do Not Use This Code, Generate the Correct Terms for Your Paper</concept_desc>
  <concept_significance>100</concept_significance>
 </concept>
 <concept>
  <concept_id>00000000.00000000.00000000</concept_id>
  <concept_desc>Do Not Use This Code, Generate the Correct Terms for Your Paper</concept_desc>
  <concept_significance>100</concept_significance>
 </concept>
</ccs2012>
\end{CCSXML}


\ccsdesc[500]{Information systems~Information retrieval}
\ccsdesc[300]{Applied computing~E-commerce}
\ccsdesc{>Computing methodologies~Machine learning approaches}
\ccsdesc[100]{Information systems~Clustering and classification}


%%
%% Keywords. The author(s) should pick words that accurately describe
%% the work being presented. Separate the keywords with commas.
  
% \keywords{Attribute Recognition, E-commerce Search, BERT, Large Language Models, Named Entity Recognition, Decorative Relation Correction}
\keywords{Retrieval, Search, BERT, E-commerce Search, language models}

%% A "teaser" image appears between the author and affiliation
%% information and the body of the document, and typically spans the
%% page.

%%
%% This command processes the author and affiliation and title
%% information and builds the first part of the formatted document.
\maketitle

\section{Introduction}
In the digital age, the landscape of e-commerce has rapidly evolved, with search advertising emerging as a vital component of online marketplaces. Platforms like Walmart.com, which cater to millions of users daily, rely increasingly on sophisticated advertising mechanisms to drive their revenue streams \cite{xue2023practical}. The strategic display of advertisements, or sponsored products, is not merely a supplement to the user experience but a critical driver of customer engagement and sales. The motivation behind enhancing ad relevance is twofold: first, to improve the user experience by ensuring that customers are presented with ads that are closely aligned with their search intent, thereby fostering a more intuitive and satisfying shopping journey. Second, there is a direct economic incentive, as more relevant ads are likely to lead to higher click-through rates, culminating in increased ad-generated revenue. This is particularly significant for e-commerce giants like Walmart, where small percentage points in ad performance can translate to substantial financial outcomes.

%Problem Definition
The central challenge in search advertising is not just the real-time delivery of ads, but more importantly, the relevancy of these ads to the users' search queries. In an online marketplace as vast as Walmart.com, which serves millions of users every day, presenting sponsored products that precisely match user intent is critical. Relevant ads enhance the shopping experience by providing users with options that align with their needs, thereby increasing the click-through rates and likelihood of purchases. This relevancy is not only beneficial for customer satisfaction, but also a key driver of the economic success of e-Commerce platforms, where even minor improvements in ad relevance can lead to significant increases in ad-generated revenue.

%Challenges
Historically, the retrieval module of traditional search ad systems has faced significant hurdles. It must sift through all available ad products to find potential candidates, which are then ranked based on predicted Click-Through-Rate (pCTR) and bidding prices. The effectiveness of this retrieval module is foundational to the overall ad system's performance. Yet, despite its critical role, there is a notable lack of comprehensive guides available to industry professionals on creating and implementing end-to-end retrieval pipelines that can address the associated challenges effectively.

In the early stages of item retrieval, roughly two decades ago, collaborative filtering (CF) technology\cite{schafer2007collaborative} held sway but grappled with the challenges posed by data sparsity, cold starts, and the rapidly expanding scale of data in e-Commerce. Other classic retrieval techniques, such as content-based (CB) and BM-25 (TF-IDF and token matching)\cite{BM25}, required a lot of domain knowledge to handcraft features and struggled to generalize to new items or downstream tasks. However, in the past decade, significant progress in deep learning has revolutionized retrieval models through embedding-based approaches, demonstrating their robust representation-learning capabilities\cite{alibaba}. In 2013, Huang et al. introduced Deep Structured Semantic Models (DSMM)\cite{dssm} for the retrieval of web searches, establishing the foundation for subsequent innovations. In 2016, Covington et al. pioneered the Two-Tower embedding architecture for recommending YouTube videos\cite{youtube}. One year later in 2017,  Facebook's introduction of Faiss\cite{faiss} offered an efficient framework for retrieving document embeddings that are similar to each other by searching for the approximate nearest neighbor (ANN). Since then, the two-tower architecture that integrates dual embedding networks and nearest neighbor search has become widely accepted in different retrieval and recommendation scenarios\cite{ebay}\cite{youtube2}. In recent years, there has been significant progress and success in the application of deep learning models to various domains, such as speech recognition \cite{kamath2019deep}, computer vision \cite{du2022self}\cite{huang2022unsupervised}, and others \cite{du2021improved}. As a result, the adoption of semantic embedding based on deep learning models has emerged as the mainstream choice in many industries. The revelation of Bert (Bidirectional Encoder Representations from Transformers)\cite{2018bert} in 2018 sent shock waves through the natural language processing community, breaking records, and setting new standards. Numerous studies have demonstrated that substituting the original embedding networks with Bert, the Twin-Tower\cite{TwinBERT}(or its derived variant \cite{Quadruplet}) architecture, can yield exceptional results in information retrieval tasks. In 2022, A. Magnani, et al. proposed a novel model designed to enhance the organic search retrieval system\cite{magnani2022semantic}. However, it was observed that this model did not significantly contribute to the improvement of the advertising system on Walmart.com.

% In this work, we describe the complete end-to-end pipeline of the semantic retrieval system for Walmart sponsored products. The pipeline includes 1) Pre-training a BERT-based model as classification using labels from Catalog for Walmarts. 
% 2) Designed two-tower Siamese network structure specifically for cosine similarity loss training target 3) Progressive training across hierarchical knowledge domains to achieve general semantic understanding 4) High throughput online inference/retrieval in production deployment. 
In response to the evolving landscape of e-commerce search advertising, we introduce a holistic end-to-end pipeline that leverages the latest advancements in natural language processing and deep learning. Our novel approach, anchored by a BERT-based model, is systematically designed to unravel the complexities of e-commerce search queries, enhancing ad relevance and user experience on Walmart.com. The pipeline features:

\begin{itemize}
  \item A pre-training phase where a BERT-based classification model is refined using product categorical labels from Walmart's extensive catalog, enabling it to recognize a wide array of product categories.
  \item A two-tower Siamese network structure optimized for cosine similarity, ensuring that retrieved ads closely align with the semantic content of user queries.
  \item Progressive training across diverse knowledge domains, which equips the model with a nuanced understanding of natural language and Walmart products.
  \item The deployment of a high-throughput online inference and retrieval system capable of operating at the scale demanded by Walmart's substantial e-Commerce traffic.
\end{itemize}

This comprehensive approach not only advances the technical framework for ad retrieval systems but also promises significant improvements in user engagement and satisfaction for Walmart's online platform.

The key contributions of this work are as follows:
% \begin{itemize}
%   \item We propose a two-stage progressive training architecture for our retrieval embedding model to improve the model ability for model ability for e-Commerce understanding.

%   \item We propose a fusion knowledge training method with Siamese network architecture to achieve general semantic understanding as well as specialized e-Commerce knowledge.
%   \item We propose and demonstrate the hard negative data labeling method is beneficial to retrieval model
%   \item We showcase the scalability of our solution, detailing its successful deployment in a live production setting, supporting one of the world's largest e-Commerce platforms.
% \end{itemize}

\begin{itemize}
\item We introduce a novel two-stage progressive training framework for our retrieval embedding model, enhancing its capability for nuanced e-Commerce comprehension.

\item We develop a unique fusion knowledge training approach with human-in-the-loop mechanism for dynamic weight adjustment, utilizing a Siamese network architecture, to attain both a broad semantic understanding and targeted e-Commerce expertise.

\item We present a new method for hard negative data labeling, demonstrating its effectiveness in improving retrieval model performance.

\item We show that our approach improves both the relevance and business metrics with a substantial margin compared to the baseline approach.

\item We illustrate the scalability and robustness of our solution, highlighting its successful deployment in a real-world environment, supporting one of the world's largest e-Commerce platforms.
\end{itemize}

\section{Model Architecture AND TRAINING METHODOLOGY}
In this section, we explore the structural design and training methodologies of the language models in use. Our discussion highlights the utilization of BERT models for their superior text embedding capabilities and outlines our approach to training these models with a curated mix of datasets tailored to the e-Commerce domain. Furthermore, we detail the architecture of the Siamese network and describe the training process, which includes a human-in-the-loop mechanism for dynamic weight adjustment based on model performance feedback.

\begin{figure*}
  \centering
  \includegraphics[width=\linewidth]{TTB_dia.png}
  \caption{Two-stage progressive training architecture of retrieval embedding model.}
  \Description{Two-stage progressive training architecture of retrieval embedding model.}
  \label{fig:model_architecture}
\end{figure*}

\subsection{Models}
We utilize BERT's language model, acknowledged for producing meaningful text embeddings \cite{2018bert}. To suit our efficiency needs, DistilBERT \cite{2019distilbert}, a compact version of BERT, is employed for its balance of performance and speed. Nonetheless, the generic BERT model's training on broad corpora is insufficient for niche applications like e-Commerce, necessitating a tailored approach for sentence pair regression tasks due to their combinatorial complexity.

\begin{figure*}
  \centering
  \includegraphics[width=0.9\linewidth]{TTB_sample.png}
  \caption{Human-in-the-loop knowledge fusion sampling.}
  \Description{Human-in-the-loop knowledge fusion sampling.}
  \label{fig:model_sample}
\end{figure*}

\textbf{\textit{Model Training Architecture:}}
We propose a two-stage progressive training architecture for our retrieval embedding model consisting of 1) language model pre-training, and 2) embedding encoder training, as illustrated in Figure~\ref{fig:model_architecture}. In the initial stage of language model pretrain, DistilBERT is utilized to execute multiple multi-class classification tasks. These tasks leverage Walmart's product category data including 'Department' labels and 'Product Type' labels, assigning specific categories to each item or query on Walmart.com. Given the fixed and unvarying nature of the 'Product Type' and 'Department', a multi-class classification approach is deemed appropriate. Detailed dataset information is available in Section 2.2. Cross-entropy loss serves as the pretrain objective function, effectively leveraging Walmart's categorical label insights into the BERT transformers. In the subsequent phase, we incorporate a pooling layer atop the pre-trained DistilBERT output, generating a 384-dimensional sentence embedding. This embedding model is then integrated into a Siamese network to directly train sentence pairs on optimizing cosine similarity loss.

During the second stage of training, we incorporate a variety of knowledge domains through a process of progressive fusion training. This strategy ensures that the model not only has the ability of general semantic understanding but also acquires a deep understanding of the e-Commerce knowledge. Diverse datasets facilitate the learning across these multiple domains as shown in Table 1. We designed and implemented a framework with the human-in-the-loop dynamic fusion sampling mechanism, as shown in Figure~\ref{fig:model_sample}. This system allows for the recalibration of training and sampling weights based on the assessment of human-labeled data. Weights adjustments are governed by the equation:

\begin{equation}
w_i = \frac{1}{1+exp(10*x_i-5)},
\end{equation}

where $w_i$ represents the sampling weights for each domain, scaled between 0 and 1, and $x_i$ denotes the normalized discounted cumulative gain (nDCG) scores calculated from human-labeled data. It is evident that as the nDCG scores increase, indicating proficient performance, the corresponding domain weights decrease. This inverse relationship allows our model to de-prioritize domains where performance is already satisfactory and focus more on other domains requiring improvement.

For the encoder training, we adopt a direct approach using cosine-similarity labels, which enables us to formulate the loss function based on the mean squared error (MSE) loss as follows:
\begin{equation}
L(V_x,V_y,l) = (V_x*V_y - l)^2,
\end{equation}
where $V_x$ and $V_y$ are the input sentence pairs and $l$ is ground-truth label. To enhance the model's discriminative learning, we apply in-batch negative sampling, refining the MSE loss to a more sophisticated triplet loss for the Walmart dataset. This approach concurrently processes positives and negatives against the anchor:

\begin{equation}
L(A,P,N) = max(D(A-P)-D(A-N)+\alpha,0),
\end{equation}

where $A, P$, and $N$ denote embeddings for anchor (query), positive (relevant item), and negative (irrelevant item) sentences. $D(A, P)$ is the L2 distance between the anchor and the positive sentence embeddings, $D(A, N)$ is the L2 distance between the anchor and the negative sentence embeddings, and the margin $\alpha$ ensures that $P$ is at least $\alpha$ closer to $A$ than $N$.

\textbf{\textit{Benchmark Models:}}
Our model is benchmarked against existing models - Deep Structured Semantic Model (DSSM) \cite{dssm}. The performance metrics are meticulously compared, underscoring the superior capabilities of our trained BERT-based model in capturing semantic textual similarities. These comparisons and findings are described in Section 4.

\begin{figure}[H]
  \centering
  \includegraphics[width=\linewidth]{super_dep.png}
  \caption{Department classification label of exampled item}
  \Description{.....}
  \label{fig:super_dep}
\end{figure}
\subsection{Datasets}

% Our training methodology also benefits from a mix of pre-labeled open-source datasets and Walmart's proprietary log data. we establish a solid database using 200,000 sampled query-product type and item-product type combinations derived from Walmart's extensive internal datasets. Each query or item is uniquely paired with a corresponding product type. Employing the same methodology, we construct 200,000 query-'Department' and item-'Department' pairs, as exemplified by the data from Walmart.com depicted in Figure~\ref{fig:super_dep}. For instance, the 'Department' for the item 'Horizon Organic Nonfat High Vitamin D Milk, Half Gallon' is classified under 'Food'.

Our training methodology also benefits from a mix of pre-labeled open-source datasets and Walmart's proprietary log data. We establish a solid database using 200,000 sampled query - 'Department' and item - 'Department' pairs, as exemplified by the data from Walmart.com depicted in Figure~\ref{fig:super_dep}. For instance, the 'Department' for the item 'Horizon Organic Nonfat High Vitamin D Milk, Half Gallon' is classified under 'Food'. Employing the same methodology, we construct another 200,000 query - 'Product Type' and item - 'Product Type' combinations derived from Walmart's extensive internal datasets. Each query or item is uniquely paired with a corresponding product type.


For subsequent training, we collected five distinct datasets to cover four knowledge domains: 

1. the Stanford Natural Language Inference (SNLI) dataset \cite{bowman2015large}, an open-source dataset of 570k labeled sentence pairs, provides the fundamental base, offering a broad spectrum of sentence pairs annotated with logical relationships vital for initial language inference training. 

2. the Multi-Genre Natural Language Inference (MultiNLI) corpus \cite{williams2017broad} introduces complexity with sentence pairs from various genres, pushing the model's ability to process diverse linguistic contexts.

3. a specialized Search Engine dataset, comprised of 20 million query-item title pairs, is extracted from Google's search results for queries that specifically request pages from "site:walmart.com". This ensures that the dataset is reflective of Walmart's product relevance as determined by Google's search algorithms.

4. the Human Evaluated Walmart Search Log dataset, with 8 million query-item title pairs, involves human evaluators' assessment of the relevance, providing a nuanced understanding of search-result appropriateness within the Walmart context.

5. Walmart.com Ad Log Data, encompassing six months of user interactions, offers a comprehensive view of customer engagement and product preferences on Walmart's digital platform.


\newcolumntype{C}[1]{>{\centering\arraybackslash}p{#1}}

% \usepackage{multirow}
\begin{table}[h]
\centering
\caption{Summary of Datasets Used in Training}
\label{tab:datasets}
\begin{tabular}{|C{1.5cm}|C{3cm}|C{3cm}|}
\hline
\textbf{Domain} & \textbf{Dataset Name} & \textbf{Size} \\
\hline
{Natural Language Public Domain} & Stanford Natural Language Inference (SNLI) & 570k sentence pairs \\
\cline{2-3}
 & Multi-Genre Natural Language Inference (MultiNLI) & 430k sentence pairs \\
\hline
Search Engine Marketing Domain & Walmart-Focused Google Search Data & 20M query-item title pairs \\
\hline
Organic item Search Domain & Human Evaluated Walmart Search Logs & 8M query-item title pairs \\
\hline
Ads item Search Domain & Walmart.com Ads Log Data & 12M query-item title pairs \\
\hline
\end{tabular}
\end{table}

\textbf{\textit{Data Labeling for Ad Logs:}}
For Walmart's  Ads log data, we adopt a pseudo-labeling approach due to the impracticality of manually annotating the vast number of query-item interactions. This method infers relevance from user engagement signals, specifically click-through data. An item is considered relevant if it attracts a high number of clicks within a designated period, leveraging the assumption that user interaction signifies relevance.

\textbf{\textit{Negative QIP Identification:}}
Incorporating negative Query-Item Pairs (QIPs) is essential for accurately determining relevance. Our methodology distinguishes between two categories of negatives. 'Easy' negatives are those randomly chosen from categories that have no relation to the query. 'Hard' negatives, on the other hand, are derived from search logs, pinpointing items that are often displayed yet seldom receive clicks. By training the model to recognize these varied levels of relevance, we significantly improve the precision of search and recommendation systems. 

\section{Semantic Retrieval Online Service Architecture}
\label{sec:system_vespa}
In this section, we will introduce the system architecture for online service deployment of semantic retrieval.

The fundamental data source of retrieval service is the embedding generation pipeline, which keeps generating the embedding vector of sponsored products (ads) and customers' search queries using our two-tower Bert model. Taking the ad embedding generation as an example, we show the architecture diagram in Figure~\ref{fig:embedding_pipeline}. There are two main Airflow modules: The first one keeps scanning and fetching eligible ads and distributes them onto hundreds of Spark clusters which host the two-tower Bert model, where the corresponding embedding vectors are generated thereafter. After the vector data passes the validation phase, the second Airflow module will grab the embedding vectors and ingest them into the target fields of Vespa (the vector search engine proposed by Yahoo). 



\begin{figure}[H]
  \centering
  \includegraphics[width=1.0\linewidth]{embedding_pipeline.png}
  \caption{Pipeline architecture of generating Ads embedding}
  \Description{.....}
  \label{fig:embedding_pipeline}
\end{figure}


Once we have ads and query embedding vectors streaming into the high-speed vector search engine (Vespa), we can build the real-time ads search service on top of that. The response of a search query can be divided into multiple stages as shown in Figure~\ref{fig:retrieval_service}. Given a customer's search query, Orchestrator will first process the raw query and send the processed query to Retrieval Service, where we fetch the saved query embedding vector. The embedding vector is wrapped in a designed API format to do an approximate nearest neighbor search in Vespa. Retrieved items will go through the re-rank and sanity check stages to be displayed on the customer's shopping page.

\begin{figure}[H]
  \centering
  \includegraphics[width=1\linewidth]{retrieval_service.png}
  \caption{Retrieval service architecture based on Vespa search engine}
  \Description{.....}
  \label{fig:retrieval_service}
\end{figure}


\section{Experiments and results}
In this section, we will discuss the experiment results of comparing our two-tower semantic model with the baseline DSSM model. Due to the company's confidentiality policy, we only reveal relative improvement numbers. 

\subsection{Relevance Evaluation Metrics}
The two primary matrices are used here to evaluate the relevance of the models: Normalized Discounted Cumulative Gain (NDCG) and Irrelevant Ads Rate (IAR). The metrics are defined as follows:

\textbf{\textit{Normalized Discounted Cumulative Gain (NDCG):}}  
NDCG~\cite{jarvelin2002cumulated} is a widely-adopted relevance measurement of ordered items. We use the calculation equation as follows:

\begin{equation} 
\label{eq:DCG}
NDCG_{p} = \frac{DCG_{p}}{IDCG_{p}}
\end{equation}
where DCG is defined as:
\begin{equation} 
\label{eq:DCG}
DCG_{p} = \sum_{i=1}^{p}\frac{rel_{i}}{\log_{2}(i+1)}
\end{equation}
where p is the length of item list, $rel_{i}$ is the human score of $i\_th$ item. To make DCG scores fair across different item lists, we always clip to the same list size before calculating. And IDCG is ideal (maximum) discounted cumulative gain.


\textbf{\textit{Irrelevant Ads Rate (IAR):}}  
IAR is the percentage of irrelevant ads when we look at the first n items of a retrieval list. In particular, we want to decrease this percentage when we are looking at the first few items (e.g. top 3 or top 5), otherwise, there will be a high probability that such front irrelevant items get selected by following the ranker model due to higher retrieval scores. IAR is calculated as follows:

\begin{equation} 
\label{eq:IAR}
\frac{\sum_{i=1}^{n}I(i)}{n}
\end{equation}
where for $n$ items $I(i) = 1$ if $i\_th$ item is irrelevant else it is 0.
IAR is used to directly measure a retrieval model's ability to avoid irrelevant items.

Alongside these relevance-focused metrics, additional e-Commerce indicators are also considered, such as:

\begin{itemize}
\item CTR (Click-Through Rate): The ratio of clicks to impressions on an ad.
\item CPMV (Cost Per Thousand Views): The advertising cost per thousand views.
\item Ads Revenue: The total revenue generated from ad clicks.
\end{itemize}

\subsection{Offline simulated relevance evaluation}
To perform offline retrieval simulations, we have developed offline Faiss/Vespa databases with all of Walmart's eligible sponsored products. From the search history, we randomly selected two thousand queries according to Walmart traffic segment and utilized their embeddings to retrieve 20 items per query through an approximate nearest-neighbor search in the Faiss database. The methodology of this evaluation is shown in Figure~\ref{fig:faisseva}.

\begin{figure*}
  \centering
  \includegraphics[width=0.8\linewidth]{faisseva.png}
  \caption{Diagram of Faiss-based simulation relevance evaluation}
  \label{fig:faisseva}
\end{figure*}

For the evaluation, we collected the ground-true relevance labels of the dataset from a third-party human annotation platform specializing in annotating Walmart traffic data. They labeled query-item pairs using the three-level class method. \textit{\textbf{0}: Irrelevant; \textbf{1}: Partially Relevant; \textbf{2}: Fully Relevant}. The results of this offline evaluation are shown in Table ~\ref{tab:ndcg_iar_offline}. We observed an increase in NDCG up to 16\%, along with a decrease in IAR by about 4\%. Furthermore, our analysis revealed an average drop of 3.6\% in NDCG when not using hard negative training datasets, indicating the effectiveness of our proposed hard negative data labeling approach in enhancing model performance.

\begin{table}[H]
\caption{Offline results of Siamese network and
DSSM model}
  \label{tab:ndcg_iar_offline}
\begin{tabular}{ccc}
\toprule
Metrics Type & \begin{tabular}[c]{@{}c@{}}Siamese BERT network  \\ vs.\\  DSSM\end{tabular} \\ \hline
NDCG@5      &  \textbf{+16.10\%}                                                               \\
NDCG@10     &  +13.26\%                                                               \\
NDCG@20     & +13.42\%                                                               \\ \hline
IAR@5      &  \textbf{-4.07\%}                                                               \\
IAR@10     & -3.45\%                                                               \\
IAR@20     & -2.97\%                                                               \\

\bottomrule
\end{tabular}
\end{table}

\subsection{Online AB testing results}
After we tested the retrieval relevance superiority offline, the next steps will be production deployment (details in Section~\ref{sec:system_vespa}) and online AB testing. This section will discuss AB testing comparing our new Two-Tower Semantic model with the current production vector model (DSSM). Since the fundamental target is to improve search-ads relevance, we first perform testing on Walmart's search pages, including:(i) search in-grid, and (ii) search carousel. Results in table ~\ref{tab:search_ab_relevance}, show the relevance improvement of the final ads impression on the two above placements. Here we consider the top 4 and top 8 NDCG because the final placement of search in-grid ads on Walmart.com contains 4 slots in a row.

Notice that the NDCG improvement numbers are not exactly the same as the offline retrieval benchmark, but the superiority of our new embedding model does hold. Since Walmart.com has multiple parallel retrieval channels and the entire ads recommendation system has a funnel-like structure containing a sequence of modules like retrieval, ranking, re-ranking, etc, the final ads impressions impact during the online AB testing would be different from what we have observed from the offline evaluation which focuses on retrieval stage.

\newcolumntype{C}[1]{>{\centering\arraybackslash}p{#1}}

\begin{table}[H]
\caption{AB testing relevance improvement of Siamese Bert network on search page}
\centering
% \small % This will make the font of the table smaller
\begin{tabular}{|c|c|c|}
\hline
Search Page Module & NDCG@4   & NDCG@8   \\ \hline
Search In-grid   & +10.91\% & +9.46\%  \\ \hline
Search Carousel  & +11.87\% & +11.73\% \\ \hline
\end{tabular}
\label{tab:search_ab_relevance}
\end{table}

\newcolumntype{C}[1]{>{\centering\arraybackslash}p{#1}}

\begin{table}[H]
\caption{AB testing E-commerce metrics improvement of Siamese Bert network on search page}
\centering
% \small % This will make the font of the table smaller
\begin{tabular}{|c|c|c|}
\hline
CTR    & CPMV   & Ads Revenue \\ \hline
+7.2\% & +4.9\% & +5.16\%     \\ \hline
\end{tabular}

\label{tab:search_ab_ecommerce}
\end{table}


Besides, through the Online AB testing, we have also testified that the improved embedding retrieval model would bring in better E-commerce metrics of the search page, which is shown in Table~\ref{tab:search_ab_ecommerce}.
We have also got a significant improvement on item page modules, which is shown in Table~\ref{tab:item_ab_ecommerce}.



\begin{table}[H]
\caption{AB testing E-commerce metrics improvement of Siamese Bert network on search page}
\centering
% \small % This will make the font of the table smaller
\begin{tabular}{|c|c|c|c|}
\hline
Item Page Modules    & CTR    & CPMV   & Ads Revenue \\ \hline
Item-top Buybox      & +1.9\% & +0.8\% & +0.6\%      \\ \hline
Item-middle Carousel & +3.2\% & +6.5\% & +6.3\%      \\ \hline
Item-bottom Carousel & +2.6\% & +4.2\% & +3.5\%      \\ \hline
\end{tabular}

\label{tab:item_ab_ecommerce}
\end{table}

\subsection{Deployment of Production Launch}
Following a series of successful A/B tests that underscored the efficacy of our two-tower BERT-based semantic retrieval model, we officially launched the system into production in June 2023. This deployment marks a key advancement in Walmart.com's search advertising, introducing a more nuanced approach to understanding and matching user queries with relevant sponsored products. The rollout was the culmination of meticulous planning and optimization, ensuring that the transition into the live environment was smooth and that the model performed in alignment with our rigorous pre-launch testing standards.

Post-deployment, we have established a rigorous monitoring protocol to continuously assess the model's performance metrics, mirroring those used during the A/B testing phase. These include measures of relevance such as Normalized Discounted Cumulative Gain (NDCG) and Irrelevant Ads Rate (IAR), as well as operational metrics like eCTR, eCPMV, latency, and throughput. The results have been promising—the model's performance has proven to be stable, and the metrics have consistently followed the positive trends observed during testing. Our ongoing commitment to monitoring and fine-tuning ensures that Walmart.com's search advertising remains responsive to user needs and business goals, reinforcing our leadership in e-commerce innovation.


%\section{Limitation}


\section{Conclusion and Future}

In this paper, we presented an innovative end-to-end pipeline for the semantic retrieval of sponsored products at Walmart.com. Our approach harnesses the power of a BERT-based model within a two-tower Siamese network structure, significantly enhancing ad relevance by capturing the nuanced semantics of user queries. The application of this model in a production environment demonstrates its capability to improve the user experience and contribute to the economic objectives of e-commerce platforms.

However, our approach is not without its limitations. The current reliance on pre-computing query embeddings may restrict the full spectrum of semantic search capabilities, and the system's architecture must balance the trade-off between latency and performance, dictated by constraints on model size and embedding dimensions. Recognizing these challenges, our future work is directed toward incorporating more dynamic, real-time models that can accommodate larger embedding sizes without compromising response times.

As we continue to refine our semantic retrieval system, our goal is to maintain its position as a leading solution in the e-commerce domain, pushing the boundaries of search advertising to deliver unparalleled relevance and efficiency. We welcome the research community to build upon our work, fostering advancements that will shape the future landscape of online retail.

\bibliographystyle{ACM-Reference-Format}
%%\bibliography{sample-base}
%%
%% This is file `sample-sigconf.tex',
%% generated with the docstrip utility.
%%
%% The original source files were:
%%
%% samples.dtx  (with options: `all,proceedings,bibtex,sigconf')
%% 
%% IMPORTANT NOTICE:
%% 
%% For the copyright see the source file.
%% 
%% Any modified versions of this file must be renamed
%% with new filenames distinct from sample-sigconf.tex.
%% 
%% For distribution of the original source see the terms
%% for copying and modification in the file samples.dtx.
%% 
%% This generated file may be distributed as long as the
%% original source files, as listed above, are part of the
%% same distribution. (The sources need not necessarily be
%% in the same archive or directory.)
%%
%%
%% Commands for TeXCount
%TC:macro \cite [option:text,text]
%TC:macro \citep [option:text,text]
%TC:macro \citet [option:text,text]
%TC:envir table 0 1
%TC:envir table* 0 1
%TC:envir tabular [ignore] word
%TC:envir displaymath 0 word
%TC:envir math 0 word
%TC:envir comment 0 0
%%
%%
%% The first command in your LaTeX source must be the \documentclass
%% command.
%%
%% For submission and review of your manuscript please change the
%% command to \documentclass[manuscript, screen, review]{acmart}.
%%
%% When submitting camera ready or to TAPS, please change the command
%% to \documentclass[sigconf]{acmart} or whichever template is required
%% for your publication.
%%
%%
% \documentclass[sigconf]{acmart}
% \documentclass[sigconf, anonymous, review, screen]{acmart}
%\documentclass[sigconf,natbib=true,anonymous=true]{acmart}
\documentclass[sigconf]{acmart}
\usepackage[ruled,vlined]{algorithm2e}
\usepackage{algorithmic}
\usepackage{subfigure}
\usepackage{graphicx}
\usepackage{multirow}
\usepackage{array}
\usepackage{colortbl}
\usepackage{booktabs} 
\usepackage{xcolor}
\usepackage{amsmath}
\usepackage{bm}
\usepackage{amsthm}
\usepackage[most]{tcolorbox}
% \usepackage{amssymb}
\usepackage{pifont}

\usepackage[utf8]{inputenc}

\usepackage{cleveref}
\crefname{section}{§}{§§}
\Crefname{section}{§}{§§}
\definecolor{chart Idle}{gray}{.6}
\definecolor{chart Poor}{RGB}{242,28,28}
\definecolor{chart Ok}{RGB}{248,172,37}
\definecolor{chart Ideal}{RGB}{1,151,0}
\definecolor{chart Over}{RGB}{0,125,234}
\definecolor{lightergray}{RGB}{230,230,230}
\definecolor{DarkGreen}{RGB}{30,130,30}
\newcommand{\cmark}{\textcolor{DarkGreen}{\ding{51}}}
\newcommand{\xmark}{\textcolor{red}{\ding{55}}}
\newcommand\ourdata{ToolLens\xspace}
\newcommand\ourmodel{COLT\xspace}
\newcommand{\paratitle}[1]{\vspace{1.5ex}\noindent\textbf{#1}}
\newcommand{\ie}{\emph{i.e.,}\xspace}
\newcommand{\aka}{\emph{a.k.a.,}\xspace}
\newcommand{\eg}{\emph{e.g.,}\xspace}
\newcommand{\wrt}{\emph{w.r.t.}\xspace}
\newcommand{\wo}{\emph{w/o}\xspace}
\newcommand{\etc}{\emph{etc}}
\newcommand\oureval{COMP}
\newcommand\Oureval{COMP\xspace}
\newdimen\tempdim
\newcommand*{\Triangle}{%
  \settoheight{\tempdim}{L}%
  \tikz[x=\tempdim, y=\tempdim]\draw(0,0) -- (.5,.5) -- (0,1) --cycle;%
}
\newcommand*{\ChartBox}[3]{%
  \begingroup
    \settoheight{\tempdim}{L}%
    \edef\tempheight{\the\tempdim}%
    \settodepth{\tempdim}{g}%
    \edef\tempdepth{\the\tempdim}%
    \tikz[
      baseline=0pt,
      inner sep=0pt,
    ]
    \node[
      fill={#3!#2},
      rounded corners=1pt,
      anchor=base,
    ]{%
      \vphantom{g\"A}%
      \pgfmathsetlength{\tempdim}{#1}%
      \kern\tempdim\relax
    };%
  \endgroup
}
\newcommand*{\chart}[3]{%
  \ChartBox{20mm/3000*(#1-400)}{#2}{#3}%
}


\AtBeginDocument{%
  \providecommand\BibTeX{{%
    \normalfont B\kern-0.5em{\scshape i\kern-0.25em b}\kern-0.8em\TeX}}}
%%
%% \BibTeX command to typeset BibTeX logo in the docs
\AtBeginDocument{%
  \providecommand\BibTeX{{%
    Bib\TeX}}}


\newcommand{\xc}[1]{{\color{blue} xuchen: ``#1''}}

\newcommand{\xy}[1]{{\color{green} xuyang: ``#1''}}

\newcommand{\xp}[1]{{\color{orange} xp: ``#1''}}

\newcommand{\xujun}[1]{{\color{red} xujun: ``#1''}}

\newcommand{\zhongxiang}[1]{{\color{yellow} zhongxiang: ``#1''}}

%% Rights management information.  This information is sent to you
%% when you complete the rights form.  These commands have SAMPLE
%% values in them; it is your responsibility as an author to replace
%% the commands and values with those provided to you when you
%% complete the rights form.
\setcopyright{acmlicensed}
\copyrightyear{2018}
\acmYear{2018}
\acmDOI{XXXXXXX.XXXXXXX}

%% These commands are for a PROCEEDINGS abstract or paper.
\acmConference[Conference acronym 'XX]{Make sure to enter the correct
  conference title from your rights confirmation emai}{June 03--05,
  2018}{Woodstock, NY}
%%
%%  Uncomment \acmBooktitle if the title of the proceedings is different
%%  from ``Proceedings of ...''!
%%
%%\acmBooktitle{Woodstock '18: ACM Symposium on Neural Gaze Detection,
%%  June 03--05, 2018, Woodstock, NY}
\acmISBN{978-1-4503-XXXX-X/18/06}


%%
%% Submission ID.
%% Use this when submitting an article to a sponsored event. You'll
%% receive a unique submission ID from the organizers
%% of the event, and this ID should be used as the parameter to this command.
%%\acmSubmissionID{123-A56-BU3}

%%
%% For managing citations, it is recommended to use bibliography
%% files in BibTeX format.
%%
%% You can then either use BibTeX with the ACM-Reference-Format style,
%% or BibLaTeX with the acmnumeric or acmauthoryear sytles, that include
%% support for advanced citation of software artefact from the
%% biblatex-software package, also separately available on CTAN.
%%
%% Look at the sample-*-biblatex.tex files for templates showcasing
%% the biblatex styles.
%%

%%
%% The majority of ACM publications use numbered citations and
%% references.  The command \citestyle{authoryear} switches to the
%% "author year" style.
%%
%% If you are preparing content for an event
%% sponsored by ACM SIGGRAPH, you must use the "author year" style of
%% citations and references.
%% Uncommenting
%% the next command will enable that style.
%%\citestyle{acmauthoryear}


%%
%% end of the preamble, start of the body of the document source.

\begin{CCSXML}
<ccs2012>
   <concept>
       <concept_id>10002951.10003317.10003347.10003350</concept_id>
       <concept_desc>Information systems~Recommender systems</concept_desc>
       <concept_significance>500</concept_significance>
       </concept>
 </ccs2012>
\end{CCSXML}

\ccsdesc[500]{Information systems~Recommender systems}

\keywords{Long-term Evaluation, Content Creator, Information Asymmetry, LLM-empowered Agent}
\begin{document}

%%
%% The "title" command has an optional parameter,
%% allowing the author to define a "short title" to be used in page headers.
\title{CreAgent: LLM-enhanced Long-term Evaluation of Recommendation Strategy under Platform Information Asymmetry}

\title{Creator Matters: Evaluating Recommendation Systems under Information Asymmetry}


\title{Enhancing Long-term Recommender Systems Evaluation: A Perspective under Information Asymmetry}

\title{Towards Long-Term Evaluation of Recommender Systems: \\A Perspective under Information Asymmetry}

\title{CreAgent: LLM-Enhanced Creator Simulation for Recommender \\ System Long-term Evaluation Under Information Asymmetry}

\title{CreAgent: Towards Long-Term Evaluation of Recommender \\ System under Platform-Creator Information Asymmetry}

% \title{CreAgent: Towards Long-Term Evaluation of Recommender \\Systems Under Platform-Creator Information Asymmetry}



%%
%% The "author" command and its associated commands are used to define
%% the authors and their affiliations.
%% Of note is the shared affiliation of the first two authors, and the
%% "authornote" and "authornotemark" commands
%% used to denote shared contribution to the research.
\author{Xiaopeng Ye}
% \authornote{Both authors contributed equally to this research.}
% \orcid{1234-5678-9012}
% \authornotemark[1]
\affiliation{%
\institution{Gaoling School of Artificial Intelligence\\Renmin University of China}
 \city{Beijing}\country{China}
}
\email{xpye@ruc.edu.cn}

\author{Chen Xu}
\affiliation{%
\institution{Gaoling School of Artificial Intelligence\\Renmin University of China}
 \city{Beijing}\country{China}
}
\email{xc_chen@ruc.edu.cn}

\author{Zhongxiang Sun}
\affiliation{%
\institution{Gaoling School of Artificial Intelligence\\Renmin University of China}
 \city{Beijing}\country{China}
}
\email{sunzhongxiang@ruc.edu.cn}


\author{Jun Xu}
\authornote{Corresponding author}
\affiliation{%
\institution{\mbox{Gaoling School of Artificial Intelligence}\\Renmin University of China}
 \city{Beijing}\country{China}
}
\email{junxu@ruc.edu.cn}



\author{Gang Wang}
\affiliation{%
 \institution{Huawei Noah's Ark Lab}
  \city{Shenzhen}
  \country{China}
 }
\email{wanggang110@huawei.com}

\author{Zhenhua Dong}
\affiliation{%
 \institution{Huawei Noah's Ark Lab}
  \city{Shenzhen}
  \country{China}
 }
\email{dongzhenhua@huawei.com}


\author{Ji-Rong Wen}
\affiliation{%
 \institution{\mbox{Gaoling School of Artificial Intelligence}\\Renmin University of China}
  \city{Beijing}
  \country{China}
 }
\email{jrwen@ruc.edu.cn}

%%
%% By default, the full list of authors will be used in the page
%% headers. Often, this list is too long, and will overlap
%% other information printed in the page headers. This command allows
%% the author to define a more concise list
%% of authors' names for this purpose.
\renewcommand{\shortauthors}{Xiaopeng Ye et al.}

%%
%% The abstract is a short summary of the work to be presented in the
%% article.
% % 
% Recently, recommender systems (RS) have been widely adopted across online recommendation platforms. 
% % The evaluation of RS can be divided into two main goals: short-term performance and long-term impact.
% 然而,这样一种短视的关注可能会伤害平台长期的发展,
% 
% While this focus is effective for predicting short-term performance, it overlooks the long-term effect of RS, presenting a significant challenge in evaluating their sustained impact. 
% 
% Recognizing this limitation, previous studies have turned to recommendation simulators for efficient and cost-effective evaluations. 
\begin{abstract}
% The long-term health of multi-stakeholder recommendation systems plays a pivotal role in the success of online content platforms.
% Currently, the evaluation of recommender systems (RS) primarily emphasizes the accuracy of prediction of user immediate behavior.
% However, this myopic focus overlooks the long-term impact of RS and may harm the sustainability of platforms. To evaluate the long-term influence of RS, it is essential to consider item creators, as they continuously reshape the platform's content ecosystem and influence its future growth.
% While simulators have been recognized as a cost-efficient way for long-term evaluation, existing approaches often neglect the modeling of creators. 
% To better model creators' strategic behavior, we highlight the importance of the concept of \textit{information asymmetry}. Without considering this, the simulation of real creator strategic behavior remains incomplete.

%to more effectively simulate the behaviors of creators under information asymmetry, we draw inspiration from game theory and utilize the belief mechanism to reflect creators' limited information status.

%When evaluating the long-term impact of RS algorithm, it is often essential to consider the behavior of content creators,

%(2) such behavior is also indirectly influenced by the RS.

%for online content platforms
Ensuring the long-term sustainability of recommender systems (RS) emerges as a crucial issue. Traditional offline evaluation methods for RS typically focus on immediate user feedback, such as clicks, but they often neglect the long-term impact of content creators. On real-world content platforms, creators can strategically produce and upload new items based on user feedback and preference trends.
While previous studies have attempted to model creator behavior, they often overlook the role of information asymmetry. This asymmetry arises because creators primarily have access to feedback on the items they produce, while platforms possess data on the entire spectrum of user feedback. Current RS simulators, however, fail to account for this asymmetry, leading to inaccurate long-term evaluations.

To address this gap, we propose CreAgent, a Large Language Model (LLM)-empowered creator simulation agent. By incorporating game theory’s belief mechanism and the fast-and-slow thinking framework, CreAgent effectively simulates creator behavior under conditions of information asymmetry. Additionally, we enhance CreAgent's simulation ability by fine-tuning it using Proximal Policy Optimization (PPO).
Our credibility validation experiments show that CreAgent aligns well with the behaviors between real-world platform and creator, thus improving the reliability of long-term RS evaluations. Moreover, through the simulation of RS involving CreAgents, we can explore how fairness- and diversity-aware RS algorithms contribute to better long-term performance for various stakeholders. CreAgent and the simulation platform are publicly available at \textcolor{blue}{\url{https://github.com/shawnye2000/CreAgent}}.



%reflect creators' limited information status and the fast-and-slow thinking framework to mimic the thought patterns of real-world creators.

%Moreover, the experiments also show that CreAgent can significantly assist platforms in evaluating long-term impacts on diverse stakeholders.
% accurately simulating creator behaviors.
%to understand and replicate these complex behaviors
% The simulation platfr
% Equipped with CreAgent, we propose a simulation platform and conduct evaluation experiments to offer valuable insight into the long-term impact of RS on the platform ecosystem.
% Recognizing this limitation,  \textit{information asymmetry} is indispensable. Without considering this factor, the simulation of real creator strategic behavior remains incomplete.
% 实验发现我们的platform帮助平台更全面的评估RS算法对different stakeholder长期影响。
\end{abstract}


% \received{20 February 2007}
% \received[revised]{12 March 2009}
% \received[accepted]{5 June 2009}

\maketitle

% RecAgent  CreAgent


\section{Introduction}

\begin{figure}[t]
    \centering
    \includegraphics[width=0.95\linewidth]{fig/intro_toy3.pdf}
    \caption{(a) A platform where users favor movies, food, and sports in decreasing order, with a creator who has created sports and food items. A comparison of the creator's creation behavior under (b) limited information; and (c) full information hypothetically.
}
    \label{fig:intro}
    % \vspace{-1em}
\end{figure}

%Recommender systems (RS), as pivotal tools to mitigate information overload~\cite{lu2012recommendersystems, ko2022surveyofRS}, have seen widespread adoption across various recommendation platforms (e.g., Youtube, TikTok). Current recommendation strategies can be divided into two types based on their objective: (1) maximize immediate user response (e.g., click, purchase), and (2) enhance the long-term user engagement of the recommendation platform.  Based on \textit{machine learning}, the first strategy type aims to predict user behavior from history interaction, prioritizing short-term performance in a static RS environment.  In contrast, the second type emphasizes the enduring long-term impact of the strategies on the platform's dynamic multi-stakeholder ecosystem~\cite{surer2018multistakeholder}, which aligns with the broader principles of \textit{mechanism design}~\cite{finocchiaro2021bridging}.
%Currently, most efforts are focused on evaluating the first type~\cite{ie2019recsim, jarvelin2002ndcg, zhang2024agent4rec}, while evaluation of the long-term impact of recommendation strategies remains a crucial challenge.


% \xc{
% Recommender systems (RS) serve as widely used tools for identifying potentially interesting items on platforms like YouTube and TikTok~\cite{ko2022surveyofRS}. Evaluating the effectiveness of RS typically hinges on two types of objectives: (1) short-term, which measures the prediction accuracy of immediate user responses  (e.g., clicks, purchases)~\cite{chen2017performance_evaluation_of_RS,jarvelin2002ndcg}, and (2) long-term, \xp{which focuses on assessing the enduring impact of RS on the platform ecosystem~\cite{hohnhold2015focusing}}. Currently, most efforts are devoted to the short-term objective~\cite{chen2017performance_evaluation_of_RS}, however, how to conduct the long-term evaluation is also crucial for existing RS, for example, xxx (find an example). }
% Evaluating the effectiveness of RS typically hinges on two types of objectives: (1) short-term, which measures the prediction accuracy of immediate user responses  (e.g., clicks, purchases)~\cite{chen2017performance_evaluation_of_RS,jarvelin2002ndcg}, and (2) long-term, which focuses on assessing the enduring impact of RS on the platform ecosystem~\cite{hohnhold2015focusing}. 

% % 任务:强调long-term 很重要
% % Recommender systems (RS) serve as widely used tools for identifying potentially interesting items on platforms like YouTube and TikTok~\cite{ko2022surveyofRS}. 

%, and overlook key long-term objectives of RS 


Recently, improving the long-term sustainability of multi-stakeholder recommendation platforms~\cite{abdollahpouri2019multi, burke2016towards_multistakeholder_evaluation}, such as YouTube and TikTok, has emerged as a critical concern~\cite{sun2023take, hohnhold2015focusing, saito2024long_term_off_policy_saito}.
Traditional evaluation methods for recommendation systems (RS) typically focus on metrics that measure the prediction accuracy of immediate user responses (e.g., clicks, purchases)~\cite{zangerle2022evaluating_RS_survey,jarvelin2002ndcg}.
However,  these short-term evaluations often fail to account for long-term effects (e.g., creator retention rate~\cite{el2022quantifying_creator_economy}, long-term user engagement~\cite{zou2019RL4long-term_user_engage}), leading to potential risks for platform developments, such as the emergence of filter bubbles~\cite{nguyen2014explor_filter_bubble}. 



%Therefore, evaluating the long-term impact of RS is essential for ensuring the sustainability of the platforms~\cite{hohnhold2015focusing, sun2023take}.


% However, such myopic focus on these immediate metrics overlooks the long-term effects~\cite{hohnhold2015focusing} and can potentially harm the platforms' long-term health~\cite{kohavi2012trustworthy_online_controlled}. For example, it may cause niche creators to leave the platform~\cite{xu2023p, wu2021tfrom} and ultimately diminish user engagement~\cite{bardhan2022more, ebrahimi2023userengagement}.
% , as it can lead to outcomes like user experience degradation~\cite{hohnhold2015focusing}, creator departure~\cite{xu2023p,wu2021tfrom}, and content diversity reduction~\cite{nguyen2014explor_filter_bubble}.
% (e.g., causing creator departure~\cite{xu2023p,wu2021tfrom} and filter bubble effect~\cite{nguyen2014explor_filter_bubble})

% 之前评估RS的好坏主要依赖measures the prediction accuracy of immediate user responses  (e.g., clicks, purchases)~\cite{chen2017performance_evaluation_of_RS,jarvelin2002ndcg}
% 但是一些研究have identified that focusing solely on such an objective can be harmful to the platform (e.g., causing creator departure~\cite{xu2023p,wu2021tfrom} and filter bubble effect~\cite{nguyen2014explor_filter_bubble})
% Some studies have identified evaluating the long-term impact of RS is a crucial 且被忽视的issue, for example....


\textbf{Creator behaviors under information asymmetry.} When evaluating the long-term impact of RS, it is crucial to consider the content creators~\cite{bhargava2022creator_economy, burke2016towards_multistakeholder_evaluation}.
This is because (1) creators continuously reshape the platform's content ecosystem by uploading items over time; and (2) RS can significantly influence creators' behavior, which in turn impacts the platform's long-term development~\cite{bardhan2022more}. While previous studies have attempted to model the interaction behavior between creators and RS~\cite{ding2024fashionregen, xu2023p, zhang2024large_scale_influencer_selection}, they often overlook the fact that such interaction behaviors are acutally under the condition of \textbf{information asymmetry}~\cite{prasad2023contentprompting, ben2018strategic_content_providers, xu2024ppa}. This asymmetry arises because creators typically have access only to feedback on the items they produced, as restricted by platform policies~\cite{felt2008privacy}, while the platform has access to the full spectrum of feedback data.
%Neglecting the \textit{information asymmetry} can lead to significant inaccuracies in estimating creation behavior, which is often more strategic and analytical compared to user behaviors~\cite{ben2018strategic_content_providers}.
To better illustrate this, we will provide an example as follows. 


% When evaluating the long-term impact of RS, online testing often serves as the gold standard~\cite{zangerle2022evaluating_RS_survey}. However, due to its high costs~\cite{zangerle2022evaluating_RS_survey}, a recommendation simulator is often considered a better alternative, as it strives to simulate the online behavior of RS stakeholders (e.g., user and item creator) in an offline manner~\cite{zhang2024agent4rec}. 
% % Current works on long-term evaluation primarily focus on online user behavior learning~\cite{hohnhold2015focusing} and simulation~\cite{shi2019virtualtaobao, ie2019recsim, zhang2024agent4rec}, 
% Existing RS simulators primarily focus on user behavior simulation, however, 
% the behaviors of item creators are also important in reshaping the platform's long-term content ecosystem~\cite{bhargava2022creator_economy}. For example, YouTubers (i.e., creators on Youtube) upload new videos every day based on the user feedback and platform policies~\cite{vskripcova2017motivations_incentives,ma2022im_not_a_youtuber}. Moreover, unlike users who want immediate satisfaction~\cite{ye2024bankfair, constantinides2004influencing}, creators exhibit more strategic and analytical behavior~\cite{ben2018strategic_content_providers}. Capturing this strategic creator behavior is essential for a simulator to accurately simulate the long-term impact of RS on the platform. 

% as it simulate the online behaviors of users and item creators in an offline manner



As shown in Figure~\ref{fig:intro} (a), we consider a platform with creators, users, and three genres of items. On such a platform, we assume that the majority of users prefer the genre of movies, followed by a preference for food-related items and then sports-related items.
In a real-world scenario, the platform will only share user feedback on items created by a specific creator directly with that creator. 
Under such partial feedback, the creator that produced sports and food item categories will be more likely to create the new food rather than sports items. This is because the food receives more positive feedback from users (as illustrated in Figure~\ref{fig:intro} (b)). However, if the platform provided him with full user feedback information hypothetically (as illustrated in Figure~\ref{fig:intro} (c)), the creator would likely opt to create a movie item instead. Since the movie is welcomed by most users in the platforms.
Therefore, modeling the information asymmetry is important for simulating the creator's strategic behaviors. It is also highlighted by existing mechanism design studies~\cite{ben2018strategic_content_providers, prasad2023contentprompting}.


\textbf{Simulator for RS long-term evaluation.} Although online test~\cite{kohavi2015onlineabtest} can effectively assess the performance of RS in environments with creators, its high costs hinder the evaluation efficiency~\cite{zangerle2022evaluating_RS_survey}.
Recognizing the need for cost-efficient solutions, recent studies~\cite{zhang2024agent4rec,wang2023recagent, ie2019recsim} have identified recommendation simulators as an ideal approach for affordable long-term evaluation. However, existing simulators~\cite{zhang2024agent4rec,shi2019virtualtaobao, ie2019recsim} often focus exclusively on user behavior simulation, neglecting the impact of creators.


To achieve realistic creator behavior simulations under platform-creator information asymmetry scenarios, we propose a Large Language Model (LLM)-empowered creator agent, called \textbf{CreAgent}. Inspired by the game theory~\cite{prasad2023contentprompting}, we employ a belief module to reflect the limited information status of creators and utilize the fast-and-slow thinking mechanism~\cite{kahneman2011thinkingfastandslow} to capture the thought process under such information asymmetry. Moreover, to better help CreAgent understand the limited user feedback information, we utilize Proximal Policy Optimization (PPO)~\cite{schulman2017ppo} to fine-tune the CreAgent. 
Credibility validation experiments have confirmed CreAgent's ability to exhibit human-like strategic creator behaviors, supported by insights from real-world data patterns and behavior economics principles~\cite{conlisk1996boundedrational, kahneman2013prospecttheory}.




Equipped with the CreAgent, we design a simulated platform environment, which contains highly extensible recommendation models and modified widely-used user agents RecAgent~\cite{wang2023recagent}. 
The simulated platform environment is built using a large-scale dataset that we collected through web crawling from the YouTube website. 
Based on such a dataset, our experiments confirm that our proposed CreAgent can well align with the behaviors of real content creators under information asymmetry interaction patterns. 
Furthermore, utilizing this simulation environment, we reassess various fairness- and diversity-aware RS algorithms to analyze their long-term performance for users, creators, and the platform separately.



%\xp{To systematically evaluate the effectiveness and limitations of CreAgent, we conduct comprehensive experiments to access how well CreAgent's long-term behavior aligns with the real-world dataset distribution and human interaction patterns under information asymmetry. Scalability experiments demonstrate that CreAgent can support large-scale simulation.}
%\xp{Through evaluation experiment from three perspectives: user engagement, creator protection, and content enrichment, we aim to fully explore the potential of CreAgent in providing insight in long-term evaluation under the multi-stakeholder RS environment.}
% Moreover, we undertake experiments to assess its effectiveness in evaluating the long-term performances of RS for different stakeholders, such as promoting user engagement, protecting the creators, and enriching the content.

%and mitigating the filter bubble effect


%This is because these advanced LLMs, trained on billions of real human behavioral data points, can effectively adjust their output in response to prompts that reflect user feedback, therefore, relaxing these overly strong assumptions.



%and constrain our LLM-empowered creator agent with well-established economic theories, such as behavioral economics and identity economics.
% Specifically, our creator agent is composed of four key modules. The profile module initializes the agent's social identity and other traits using collected real-world creator data. The memory module retains essential partial user information and creation experience. Drawing from mechanism design literature, the belief module simulates the audience and skill beliefs that drive creators' strategic actions. Furthermore, the creation module applies behavioral economic theory, with "slow thinking" for analyzing content and "fast thinking" for generating targeted creations. 
% Credibility validation experiments have confirmed our LLM-empowered agent's ability to exhibit human-like strategic creation behaviors, supported by insights from behavioral economics and real-world data patterns.



%Equipped with the creator agent, we introduce CreAgent — a general recommendation simulation platform for the long-term evaluation of recommendation strategies. CreAgent also simulates a realistic recommendation platform environment, featuring highly extensive RS and modified widely-used user agents.
%To explore the potential of the simulator in evaluating long-term recommendation strategies, we undertake experiments to investigate the long-term user engagement, creator ecosystem maintenance, and filter bubble effect of different recommendation strategies.

\textbf{Contributions.} Our contribution can be summarized as follows:

(1) We emphasize the significance of considering creator strategic behavior under platform-creator information asymmetry scenario during the long-term evaluation of RS.  

(2) We introduce CreAgent, an LLM-powered agent that simulates real-world creator interactions, and an extensible simulation platform for evaluating long-term RS performance with diverse models and user agents.

(3) Extensive experiments demonstrated that our simulation platform effectively replicates real-world RS, enabling a thorough reassessment of the long-term performance of various RS algorithms.




% We use the simulation platform to evaluate three key issues: promoting long-term user engagement, protecting the creator ecosystem, and mitigating the filter bubble effect.





% 第五段:基于我们的实验,验证了xxxxxx。In such an environment of information asymmetry, established behavior economic studies have noted that the content creators' behavior tends to follow certain distinct patterns: (1) bounded rationality:xxxxxxx; and (2) prospect theory:xxxxxx.
% 所以想要一个模拟真实世界的creator,就得符合这些规律。


% 第三段:为了建模在信息不对称下的creator   行为,之前的研究主要从两个方面。一些研究利用了大模型xxx技术来模拟agent,

% 虽然有些文章已经利用LLm去建模了creator的行为,他们产生新的item,xxxx。
% 但是这些方法通常建立在信息完全的,忽略了供应商在information limitation下的行为。
% 供应商在partial feedback


% 有两个目标,一个是为了提升user 即时的体验,二是为了留住provider长远的为了发展。
% 对于user,是evaluate短期的exp,
% 对于provider,是为了evaluate长期的。
% 之前对于user的evaluation都是提出静态的单论~、cite{agent4rec},但是对于provider应该考虑多轮的。


% 为了建模
% In such an environment of information asymmetry, established behavior economic studies have noted that the content creators' behavior tends to follow certain distinct patterns: (1) bounded rationality; and (2) prospect theory.

% 第二段:背景:evaluate provider应该考虑两个问题,其实是信息不对称的场景下,原因是其只能接收到与其创造过的内容的信息,而用户交互过的其他内容,他看不见。For example,如果信息不对称情况下,会怎么样,图(a)。但是 更好的策略是图(b)。
% 这样信息不对称的问题回导致2个问题,有限理性。



% To tackle the challenge of long-term evaluation, online testing~\cite{xxx} is considered the optimal approach. However, given the substantial costs associated with online testing~\cite{xxx}, recommendation simulators have become a prevalent choice in both industry~\cite{xxx} and academic research~\cite{wang2023recagent, zhang2024agent4rec}. However, existing simulators might be limited in their effectiveness in evaluating the long-term impacts of recommendation strategies.
% Most works~\cite{wang2023recagent} primarily focus on the user-side modeling. For example, techniques like Large Language Model (LLM) have been used to simulate user behavior on real recommendation platforms. However, due to the overlook of creator behavior modeling, the long-term dynamics of the platform cannot be accurately simulated.
%  Another line of works~\cite{xu2024ppa, yao2023howbad, yao2024uwo} aims to simulate the long-term dynamics of platforms using game theory. Although rule-based methods~\cite{xu2024ppa, yao2024uwo} are applied to model creator behavior, there is still a significant discrepancy between these works and the actual creator behavior in real-world scenarios.


% 为了解决这样一个长期evaluate的问题,虽然最优的方法就是上线测试,但是测试代价太大,因此,业界都是用simulator。
% 但是之前的simulator实效,都是从user的角度。。。。。。





% Distinguished from conventional user-centric models that prioritize short-term accuracy~\cite{rendle2012bpr,he2020lightgcn, he2017neuralmf}, provider-fair recommendation models are designed to foster a healthy and sustainable provider ecosystem, thereby benefiting the platform in the long term. 
% Previous works~\cite{xxx} evaluate provider-fair models based on static provider settings, where the provider provides items to the recommendation system (RS) only once, without any further interaction.
% However, in real-world online content recommendation platforms~\cite{xxx} (e.g., Youtube, TikTok), providers gather information from user feedback and frequently refining their creation strategies to create new content.

% 第二段:详细解释不一样。platform 会随着用户的行为做出不一样的creation behaviours, 而且这种信息获取方式是信息不对称的,他不能获取所有的信息。
% 信息不对称大家不理解,我帮助大家理解(画个图)。



% In online recommendation platform, real-world content creators tend to follow certain behavior patterns~\cite{xxx}. 
% Unlike users who prioritize immediate gratification~\cite{xxx}, creators are more rationally strategic, they will adjust their content strategies based on the information provided by the platform to increase user engagement. 
% Hence, the information received by the creators is relatively limited compared to the platform (i.e., information asymmetry).
% To better illustrate the information asymmetry, we plot an toy example compared the online recommendation platform with the lemon market~\cite{}.
% Under a second-hand car (lemon) market, sellers know the true quality of the product (e.g., used cars), including whether it is a "lemon" (a term used to describe a car with hidden defects) while buyers lack complete information about the product's quality. 
% Assume there are two cars: a lemon car and a good car, which are worth 10000 and 30000 respectively.
% When buyers cannot accurately assess the quality of products, they assume an average quality. This leads to them offering prices that reflect the average quality level rather than the high-quality products.




% Contribution
% 1我们强调了importance of evaluating content creator fairness information asymmetry.
% 2我们提出了CreAgent
% 3我们的实验说明了是很好的。












 

% Recently, ChatGPT-like large language models(LLMs) have demonstrated tremendous potential in achieving human-level decision-making and generative capabilities. Leveraging this characteristic, one can utilize it to simulate real human economic behaviors, providing opportunities and conditions for creating an OCRP environment that aligns with behavioral economics.


\section{Related Work}

% 分为两类, short-term and long-term
\textbf{Evaluation of RS}. How to evaluate RS is a complex and essential task, which can be divided into short-term and long-term evaluation based on their objectives~\cite{zangerle2022evaluating_RS_survey}. Most current studies focus on the short-term objective using offline metrics~\cite{jannach2020escaping, jannach2012recommender_landscape}, relying on pre-collected log data containing users' explicit (e.g., ratings) or implicit (e.g., click) feedback to compute metrics like prediction accuracy~\cite{willmott2005MAE} and ranking metrics~\cite{jarvelin2002ndcg}. 
Existing fairness-aware~\cite{ye2024bankfair, xu2024taxation, naghiaei2022cpfair}, debiasing~\cite{chen2023bias}, and diversity-aware~\cite{carbonell1998mmr} recommendation research often evaluates using domain-specific indicators based on these short-term metrics (e.g., fairness metric~\cite{gastwirth1972giniindex}, ), which fail to adequately reflect their long-term benefits and how they influence RS in the long run, particularly in multi-stakeholder platform environments~\cite{surer2018multistakeholder}.
Due to the inefficiency and high cost of long-term online A/B test~\cite{kohavi2015onlineabtest, saito2024long_term_off_policy_saito}, offline long-term evaluation has gained significant attention in recent years, which can be divided into two main categories: (1) use short-term metric~\cite{hohnhold2015focusing} or data~\cite{saito2024long_term_off_policy_saito} to predict long-term performance, (2) create an RS simulator to replicate the real-world environment~\cite{ie2019recsim, zhang2024agent4rec, wang2023recagent} for evaluation.
In this paper, we focus on the second type. 
% Existing RS simulators use Reinforcement Learning (RL)~\cite{ie2019recsim, shi2019virtualtaobao, rohde2018recogym} or LLM~\cite{wang2023recagent, zhang2024agent4rec, zhang2024agentcf} to simulate human behavior. 

\textbf{RS simulator for long-term evaluation.} 
Most existing recommendation simulators (e.g., LLM-based simulator~\cite{wang2023recagent, zhang2024agent4rec}, reinforcement learning (RL)-based simulator~\cite{ shi2019virtualtaobao,ie2019recsim}) focus on user simulation while overlooking creators, making it difficult to capture the long-term dynamic of content platforms.
Some data-driven methods are proposed to conduct creator simulation.  SimuLine~\cite{zhang2023simuLine} applied heuristic methods to determine creators' next creation in news recommendation. 
Some works~\cite{mladenov2021recsimng, mladenov2020optimizing_long-term} assumed that creators will leave the platform if their user engagement falls below a certain threshold.
Other modeling methods used user positive feedback (e.g., click) as the gradient to update the creation state~\cite{lin2024cfd, yao2024uwo, zhan2021toward_content_provider_rec}.
% EcoAgent\cite{zhan2021toward_content_provider_rec} and RecSim NG~\cite{mladenov2021recsimng} utilized RL-based techniques (e.g., Markov Stochastic Process\cite{zhan2021toward_content_provider_rec} and Decision Process \cite{mladenov2021recsimng}) to model the transition behavior of creators. 
However, these approaches failed to align with real creation behavior because: (1) they are unable to produce authentic content (e.g., text), instead relying on embeddings to represent the content they create~\cite{zhang2023simuLine}; (2) they cannot capture the personalization of real-world creators; (3) they ignored human behavior patterns under information asymmetry, such as risk aversion in prospect theory~\cite{kahneman2013prospecttheory} and bounded rationality~\cite{selten1990bounded_rationality}.
% However, RL-based simulators for RS evaluation may be limited in flexibility and generality, as they often rely on relatively simple rules to model behaviors that could diverge from actual human~\cite{shi2019virtualtaobao,ie2019recsim, zhan2021toward_content_provider_rec}. 
% Recently, LLM-empowered agents have shown great promise in enhancing recommendation simulators~\cite{wang2023recagent, wang2024macrec,zhang2024agent4rec, zhang2024agentcf}.
% For example,
% RecAgent~\cite{wang2023recagent} and Agent4Rec~\cite{zhang2024agent4rec} incorporated human cognitive mechanisms (e.g., memory mechanism) to simulate various user behaviors on the platform.
% AgentCF~\cite{zhang2024agentcf}
% modeled user and item agents simultaneously to explore the interaction patterns between users and items.

% For example, LLM-based simulator RecAgent~\cite{wang2023recagent} and Agent4Rec~\cite{zhang2024agent4rec}, Reinforcement Learning (RL)-based simulator Virtual-Taobao~\cite{shi2019virtualtaobao}, RecSim~\cite{ie2019recsim} solely focus on simulating user's watch, click behaviors.
% % EconAgent\cite{zhan2021toward_content_provider_rec} utilized Markov decision process to model both user and creator behavior to help build a multi-stakeholder recommender. 

% previous research~\cite{zhang2024agent4rec} has indicated that

% For example, RecoGym~\cite{rohde2018recogym} provides a configurable stylized RS simulation environment for studying sequential user interaction.
% Virtual Taobao~\cite{shi2019virtualtaobao} and RecSimu~\cite{zhao2019toward_simulating} utilized generative adversarial networks (GAN) to generate virtual users to support learning policies that can be transferred to real systems
% RecSim~\cite{ie2019recsim} provides a comprehensive toolkit for effectively simulating user behaviors across various stylized RS.





\textbf{Behaviors under information asymmetry.} Creator behaviors in information asymmetry conditions
have been actively studied and emphasized in the game theory literature~\cite{yao2024uwo, lin2024cfd, xu2024ppa}. 
They typically assume that creators are totally rational, i.e., aiming to maximize their utility, which often lacks personalization and differs from real-world human behavior under risk (i.e., bounded rational~\cite{selten1990bounded_rationality}). Rule-based~\cite{xu2024ppa} and heuristic method~\cite{yao2024uwo, lin2024cfd} are applied to model the strategic behavior. These studies mainly focus on the competition among creators~\cite{yao2024uwo,lin2024cfd} and the design of better platform mechanisms\cite{mladenov2020optimizing_long-term,prasad2023contentprompting} to maximize user welfare.

% These works have greatly inspired our study, e.g., \citet{prasad2023contentprompting} assumed that creator create content based on their beliefs about their skills and audience.
% These works are unsuitable as RS evaluators for three reasons: (1) absence of explicit RS modeling~\cite{zhan2021toward_content_provider_rec, mladenov2020optimizing_long-term}; (2) use of artificial feedback like relevance scores~\cite{yao2023howbad, yao2024userwelfare_opt} instead of real user responses (e.g., clicks and views); and (3) oversimplified modeling of creator behavior~\cite{lin2024cfd, yao2024uwo}, deviating from real-world human patterns~\cite{mullainathan2000behavioral_ecnomics}.


% \citet{prasad2023contentprompting}  used a mixed-integer-programming-based prompting policy to minimize the impact of information asymmetry on the platform's social welfare.
% \citet{mladenov2020optimizing_long-term} formulated the recommendation problem as a constrained matching problem and assumed creators might leave the platform due to insufficient user engagement.







\section{PRELIMINARIES}
\label{sec:preliminaries}
% \subsection{Notations}

% \begin{table}[!ht]
%     \centering
%     \begin{tabular}{l|l}
%     \toprule
%         \textbf{Symbol} & \textbf{Description} \\ \hline 
%      $ \mathcal{U}, \mathcal{I},  \mathcal{C}$ & The user, item, creator set \\ 
%      $\mathbf{E}, \mathbf{Y}$& The exposure/interaction matrix \\ 
%        $\mathcal{D}$ & The interaction records by the platform \\ 
%   \bottomrule
%     \end{tabular}
% \end{table}

\textbf{Simulation platform.} In RS, we use $\mathcal{P} =\{\mathcal{U}, \mathcal{I},  \mathcal{C}, \mathcal{G}, \mathcal{D}, \mathbf{E}, 
\mathbf{Y}\}$ to denote the information known to the platform. $\mathcal{U} = \{u\}$, $\mathcal{I} = \{i\}$, $\mathcal{C}=\{c\}$, $\mathcal{G}=\{g\}$ is the set of users, items, creators, and item genres. Each item $i \in \mathcal{I}_g$ in RS belongs to a genre $g\in \mathcal{G}$. An item $i$ created by creator $c$ denotes as $i\in \mathcal{I}_c$.  $\mathbf{E}$, $\mathbf{Y}$ is the user-item exposure matrix and interaction matrix. 


At each time step $n \in [1,2,\cdots, N]$, some users $\mathcal{U}^n$ will visit the RS, and some creators $\mathcal{C}^n$ will create items. Each created item $i$ will be added to the item pool $\mathcal{I}$ of RS. The items created in time step $n$ is denoted as $i\in \mathcal{I}^n$.
 For each user $u$, the RS generates a list of $K$ recommended items chosen from $\mathcal{I}$. Each user $u$ can choose to click or skip these items. Then, the RS records the click and exposure interaction in $\mathbf{Y}(n)$ and $\mathbf{E}(n)$, which denotes the exposure matrix and interaction matrix of time step $n$. Specifically, $\mathbf{Y}_{u, i}(n)=1$ if user $u$ clicks item $i$ at time step $n$ and 0 otherwise, $\mathbf{E}_{u, i}(n)=1$ if item $i$ is exposed to user $u$ at time step $n$ and 0 otherwise. Also, RS will save the interaction record $\left( u, i, \mathbf{Y}_{u,i}(n)\right)$ into the interaction record $\mathcal{D}$. For every $T$ step, the RS will be re-trained using interactions in $\mathcal{D}$. 

%\textbf{Simulation environment initialization.}
At the beginning of our simulation, we utilize the dataset $\mathcal{P}^0 =\{\mathcal{U}^0, \mathcal{I}^0, \mathcal{C}^0, \mathcal{G}^0, \mathcal{D}^0\}$ collected from real-world RS to initialize the simulation environment. Specifically, the user's profile, creator's profile and initial belief will be initialized and the RS will be trained using the real-world interactions $\mathcal{D}^0$. During simulation, we use the first $n\in[1,2,\cdots, N_0-1]$ time steps to initialize a stable platform environment that reflects the real-world RS.
Then, the subsequent $n\in[N_0, \cdots, N]$ steps are used to evaluate different RS models.

\textbf{Information asymmetry.} In real-world platforms, such as YouTube, the platform can access comprehensive user behavior data (e.g., watch time, click/like history), policies~\cite{stein2013policy_youtube} and incentives~\cite{prasad2023contentprompting}, which is not disclosed to creators. Meanwhile, creators can only infer whether their content aligns with user preferences by analyzing feedback (e.g., comments, likes, and other interactions) on their created items and then adjust their creation strategies accordingly. Thus, significant information asymmetry exists in the platforms, as highlighted by previous studies~\cite{lin2024cfd, yao2024uwo, prasad2023contentprompting}. 

In this paper, we denote the platform-possessed information as $\mathcal{P}$, whereas each creator $c$ has access to a subset of this information, denoted as $\mathcal{P}_c= \{\mathcal{I}_c,  \mathbf{E}^\top_{i} ,  \mathbf{Y}^\top_{i} \mid i\in \mathcal{I}_c\}, \forall c\in \mathcal{C}$. Since $\mathcal{P}_c \neq \mathcal{P}$, this disparity indicates that there is an information asymmetry between the platform and the creators. 









% Each item in the RS belongs to a genre $g\in \mathcal{G}$. An item i with genre $g$ denotes as $i\in \mathcal{I}_g$. An item $i$ created by creator $c$ denotes as $i\in \mathcal{I}_c$. 
% $L^n_K(u)$
% For the platform, we use $\mathcal{P}^n =\{\mathcal{U}^n, \mathcal{I}^n,  \mathcal{C}^n,  
% \mathcal{E}^n, \mathbf{E}^n
% \mathcal{Y}^n, 
% \mathbf{Y}^n\}$ to denote known to the platform at time step $n \in \{1, \ldots, N\}$, where the superscript $^n$ indicates the time step $n$, $\mathcal{U}^n, \mathcal{I}^n, \mathcal{C}^n$ is the user set, item set and creator set. $\mathcal{E}^n$ and $\mathcal{Y}^n$ is the user-item exposure set and interaction set.
% $\mathbf{E}^n$ and $\mathbf{Y}^n$ is user-item exposure matrix and interaction matrix.
% We consider binary user-item interaction in this work, i.e., $\mathbf{Y}^n_{u, i}= 1$ if user $u$ has positively interacted (e.g., click) with item $i$ within the $n$-th time step, and 0 otherwise.
% The initial user set, item set, user-item interaction set, and the user-item interaction matrix before the first time step (i.e., pre-train data) is denoted as $\mathcal{U}^0$, $\mathcal{I}^0$, $\mathcal{E}^0$, and $\mathcal{Y}^0$, respectively. 

% we use $\mathcal{O}^n_c =\{ \mathcal{I}_c, \mathbf{H}_c, \mathbf{J}_c\}$ to represent the information known to creator $c$. After each item's creation, the creator will add the newly created item to the creation list, i.e., $\mathcal{I}_c = \mathcal{I}_c \cup \{i\}$. W
% \xc{$|\mathbf{x}_c|->|\mathcal{I}_c|$}. For each content creator $c$, the RS maintains a creation list $ \mathbf{x}_c = [i_1, i_2, \ldots, i_{|\mathbf{x}_c|}] $, where the creation time of each item are represented as $ t_c = [t_1, t_2, \ldots, t_{|\mathbf{x}_c|}] $. Each newly created item will be added sequentially to the end of $\mathbf{x}_c$. We denote the most recent item created by creator $c$ as $ \mathbf{x}_{c,|x_c|}$, with its creation time as $ t_{c,|x_c|}$. At the end of each time step $n$, the platform will provide each creator with the user feedback on their created items, i.e., the received item exposures $\mathbf{e}^n \in \mathbb{N}^{|\mathcal{I}_c|}$, and the received item clicks $\mathbf{y}^n  \in \mathbb{N}^{|\mathcal{I}_c|}$. 
% The creator will record it in its memory for future creation. 

\textbf{Organization.} To evaluate the long-term impact of RS under the platform-creator information asymmetry environment, we propose a simulation platform with creator behavior modeling. The overview of the simulation platform is illustrated in Figure~\ref{fig:overview}. The simulation platform can be decoupled into the LLM-empowered creator agent \textbf{CreAgent} (see Section~\ref{sec:creagent} and Figure~\ref{fig:overview}~(a)) and the \textbf{platform environment} (see Section~\ref{sec:platform} and Figure~\ref{fig:overview}~(b)). Afterwards, we conduct experiments to demonstrate the credibility of our proposed CreAgent (see Section~\ref{sec:creator_distibution_alignment}). Finally, we use CreAgent and the simulation platform to assess the long-term impact of RS models on diverse stakeholders (see Section~\ref{sec:RS_evaluation}).










\section{CreAgent}
\label{sec:creagent}
% CreAgent, as a recommendation simulation environment involving both users and content creators, is expected to effectively mirror the long-term dynamics of RS and systematically evaluate recommendation algorithms by modeling real-world supplier economic behavior with LLM-empowered generative agents.
% Specifically, CreAgent Framework consists of three modules: (1) Content Creator Modeling; (2) Platform Environment Modeling.
% Figure~\ref{fig:overview} demonstrated the overview of CreAgent's framework, which is developed by modifying LangChain, with all agents being powered by the Llama3-8B~\cite{touvron2023llama}.

\begin{figure*}
    \centering
    \includegraphics[width=0.95\linewidth]{fig/big_fig2.pdf}
    % \vspace{-1em}
    \caption{The overall workflow of our simulation platform.
 (a) CreAgent, initialized with the real-world YouTube dataset, employs a belief mechanism combined with fast-and-slow thinking to create the next item strategically based on platform-provided user feedback.
  (b) Platform environment, which consists of an extensible two-stage recommendation system and modified widely-used user agent, collects item feedback from users and sends it to CreAgent. }
  % \vspace{-1em}
    \label{fig:overview}
\end{figure*}


In this section, we leverage the human-like analyzing, generation capabilities, and world knowledge~\cite{tamkin2021understanding_capbility_of_LLM} of LLMs to simulate real-world strategic behavior of creators under information asymmetry scenarios. As demonstrated in Figure~\ref{fig:overview} (a), CreAgent consists of four modules: the profile and memory module, the game-theory-inspired belief module, and the creation module integrating fast-and-slow thinking~\cite{kahneman2011thinkingfastandslow}. Finally, we utilize PPO-based fine-tuning to enhance CreAgent's analytical and creative capabilities under limited information.


\subsection{Profile Module Design}
In this module, we initialize CreAgent's profile using a pre-collected real-world dataset, crucial for aligning agent behavior with real human actions. CreAgent's profile comprises three elements: social identity, intrinsic motivation, and creation activity.
%为什么要有3个模块


\textbf{Social identity.} In reality, content creation, as social behavior, is driven by not only economic interests but also social identity, a factor often overlooked in behavior modeling~\cite{yao2023howbad, yao2024userwelfare_opt, zhan2021toward_content_provider_rec}. To address this, we employ an LLM to identify creators' social identities by analyzing their creation history $\mathcal{I}^0_c$ and basic information. 
For example, a creator who enjoys producing lifestyle content might be identified as a ‘‘lifestyle advocate'', while one who frequently creates gaming content could be summarized as a ‘‘gaming enthusiast''.
We represent this social identity as $P^s_c$, whose format is in text, and the detailed format can be found in Appendix~\ref{app:prompt}.


%(\xc{prompt, text-format, detailed format can be seen in Appendix)
\textbf{Intrinsic motivation.}
Established studies~\cite{bi2020intrinsic_moti, hull1943drive_theory} show that intrinsic motivation is a key factor that affects creators' creation behavior, yet this motivation varies from person to person. For example, while some creators frequently produce content to obtain more revenue, others create simply to share their lives without high-profit expectations. We use LLM to help us summarize each creator's intrinsic motivation $P^m_c$ from their creation history $\mathcal{I}^0_{c}$ and frequency as one part of their profiles.
The format $P^m_c$ of is in text, and the detailed format can be seen in Appendix~\ref{app:prompt}.


%(\xc{prompt, text-format, detailed format can be seen in Appendix)
\textbf{Creation activity.}
In addition to the above two economic characteristics, creation activity (i.e., the frequency of a creator’s creation) is also an important trait of the creator. We use the collected dataset $\mathcal{I}_c^0$ to initialize the inherent activity level $p^a_{c}$ for each creator $c$. It represents the average number of items created by each creator every day.
After getting $p^a_{c}$ of each creator, the creation probability for each creator at each step is $p^a_{c}/\eta$ where $\eta = \max\{p^a\}$ is the maximum activity among all creators.


% at each time step $n$, we generate a random number $x_c \sim U(0,1)$ for each creator to decide whether he is active for creation at this step. Specifically, creator $c$ is active if $x_c\leq\eta p^a_{c}$, and inactive otherwise.
% \begin{equation}
% \text {activate}_c= \begin{cases}0 & \text {if} x_c>\lambda P^a_{c} \\ 1 & \text {if} x_c \leq \lambda P^a_{c}\end{cases} ,
% \end{equation}
% $
% p^a_{c}  =  \eta \frac{|\mathcal{I}^{0}_{c}|-1}{\sum_{i=1}^{|\mathcal{I}^{0}_c| -1} t_{i+1}-t_i},
% $
%  where $t$ represents the creation time of the $i$-th item in $\mathcal{I}^{0}_c$ and $t_1< t_2 < \cdots < t_{|\mathcal{I}^{0}_c|}$, $\eta$ a constant such that $p^a_{c}$
%   is in the interval $[0,1]$.
% In this module, we aim to reflect the memory mechanism of the real creators.
\subsection{Memory Module Design}
 Considering the two key pieces of information that creators prioritize and store in reality, we build two memories for CreAgent: feedback memory and creation memory.


\textbf{Feedback memory.}
Different from user agent memory modeling~\cite{wang2023recagent}, the user feedback received by each creator is usually recorded and maintained by the platform over a considerable period~\cite{lange2007publicly_private}. We denote the feedback memory of creator $c$ as $\mathcal{M}_c^{\text{feed}}$. Due to its position in a state of information asymmetry,
at the end of each time step $n$, the feedback memory 
$\mathcal{M}_c^{\text{feed}}$ will be updated based on the partial user feedback information provided by the platform on history creation:
\begin{equation}
  \mathcal{M}_c^{\text{feed}} = \mathcal{M}_c^{\text{feed}} \cup \left\{ \sum_{u \in\mathcal{U}^n} \mathbf{E}_{u,i}(n), \sum_{u \in\mathcal{U}^n} \mathbf{Y}_{u,i}(n) \mid i \in \mathcal{I}_c \right\}   
\end{equation}


After storing user feedback in memory $\mathcal{M}_c^{\text{feed}}$ at time step $n$, the feedback memory undergoes a refinement process to calculate the utility of each created item. Specifically, it refers to the time-average exposure and clicks on each item $i\in \mathcal{I}_c$ up to the present step $n$.
{\small
\begin{equation}
    \mathbf{z}_i(n) = \frac{1}{\theta}  \left[\beta \sum_{k=t(i)}^{n} \sum_{u \in\mathcal{U}^k} \mathbf{E}_{u,i}(k)   + (1- \beta) \sum_{k=t(i)}^{n} \sum_{u \in\mathcal{U}^k} \mathbf{Y}_{u,i}(k) \right] ,
\end{equation}}
where $\theta = \frac{1}{n-t(i)+1}$ and $t(i)$ is the creation time step of item $i\in \mathcal{I}_c$, $\beta$ is the hyper-parameters that control the importance of exposure for the creator. This utility can be configured by adjusting $\beta$ according to specific scenarios, either as exposure-based~\cite{xu2023p, patro2020fairrec} or click-based~\cite{yao2024uwo}. 
The utility $\mathbf{z}$ will be used as input for both the belief module (Section~\ref{sec:belief_module}) and the creation module (Section~\ref{sec:creation_module}) to help CreAgent make decisions.


\textbf{Creation memory.}
The creation memory $\mathcal{M}_c^{\text{cre}}$ stores the historical creation item information of the creator agent. To better reflect the human time-fading memory mechanism, we draw on prior studies~\cite{wang2023recagent} and incorporate a power function forgetting rate~\cite{wang2023recagent}.
Before each creation, creators will retrieve the most relevant and recent creation experience from the creation memory. After creation at time step $n$, the newly created item will be saved to the creation memory:
$
\mathcal{M}_c^{\text{cre}} 
 = \mathcal{M}_c^{\text{cre}} \cup \{i \mid i \in \mathcal{I}_c\cap \mathcal{I}^n\},
$ which is used to update the skill belief (Section~\ref{sec:skill_belief}) and as the input of fast thinker~(Section~\ref{sec:fast_thinker}) for future creation.

\subsection{Belief Module Design}
\label{sec:belief_module}
Under information asymmetry conditions, creators usually do not have exact information about other stakeholders (e.g., user preferences, and item genres), and their creation behavior is often driven by their limited information expectations about it (i.e., beliefs).
In this module, we draw inspiration from game theory~\cite{prasad2023contentprompting} and employ the belief mechanism to reflect the limited information status as a guidance for the strategic creation of CreAgent.
 The beliefs of CreAgent are typically formed based on previous information provided by the platform and are updated over time and with the acquisition of new information.
Specifically, the strategic creation behavior of CreAgent is mainly driven by two types of beliefs: skill belief and audience belief.

\textbf{Skill belief.} 
\label{sec:skill_belief}
Under limited information, creators have incomplete knowledge of item genres. They will gradually acquire more genre information and improve their creation proficiency in those genres during the creative process.
The skill belief represents the creator’s confidence in their ability to create each genre of items, which is defined as the percentage of their content created in each genre. 
At the start of each time step $n$, the skill belief of creator $c$ for genre $g$ will be updated based on the creation memory $\mathcal{M}^{\text{cre}}_c$: 
\begin{equation}
\mathbf{B}^{\text{skill}}_{c,g}(n) =  \frac{|\mathcal{I}_{g} \cap \mathcal{I}_c|}{|\mathcal{I}_{c}|}, \quad \forall g \in \mathcal{G}, \mathcal{I}_{g},\mathcal{I}_{c} \subset \mathcal{M}^{\text{cre}}_c.
\end{equation}
Due to some creators being more adept at producing certain types of content than others (e.g., due to specific talent, interests, and facilities), we initialize the skill belief $\mathbf{B}^{\text{skill}}_{c}(1)$ using the pre-collected dataset $\mathcal{I}_c^0$. 


\textbf{Audience belief.} The audience belief represents the creator's internal understanding and expectation of user preferences in each genre. At the start of each time step $n$, the audience belief of creator $c$ in genre $g$ will be updated based on the user feedback stored in the feedback memory $\mathcal{M}^{\text{feed}}_{c}$:
\begin{equation}
\mathbf{B}^{\text{aud}}_{c,g}(n) = \frac{\sum_{i\in \mathcal{I}_{g} \cap \mathcal{I}_c} \mathbf{z}_i(n)}{|\mathcal{I}_{g} \cap \mathcal{I}_c|},\quad \forall i \in \mathcal{I}_c.
\end{equation}
Similarly to the skill belief, we initialize the audience belief $\mathbf{B}^{\text{aud}}_{c}(1)$ using the real-world creator's history items $\mathcal{I}_c^0$.


% % \text{Util} \in \mathcal{M}^{\text{feed}}_{c}
% \frac{1}{t}\sum_{n=1}^{t} \sum_{i \in \mathcal{I}_g} \mathbf{y}_i^n,\quad \forall g \in \mathcal{G}, i \in \mathcal{I}_c, \mathbf{y}_i^n \in \mathcal{M}^{\text{feed}}_{c},
% =\frac{1}{t}\sum_{n=1}^{t} \sum_{u \in\mathcal{U}^n} \sum_{i\in \mathcal{I}_g}\mathbf{Y}^n_{u,i}, \forall g \in \mathcal{G}, i \in \mathcal{I}_c.
% At the start of each time step, the skill belief $\mathbf{B}^{\text{skill}}$ and audience belief $\mathbf{B}^{\text{aud}}$ will be updated based on the information stored in the memory.

\subsection{Creation Module Design}
\label{sec:creation_module}
% 受到思考,快与慢的启发, which state that 人类思维分为两种模式系统1(快思考)和系统2(慢思考)。系统1是快速、直觉和自动化的思维,依赖经验和本能,适合日常的简单决策。系统2是缓慢、深思熟虑且逻辑的思维,用于复杂的分析和推理。
% 内容创作者在创作过程当中往往会在有限的信息下对未来创作方向进行分析和思考,再根据分析结果生成内容。
% \textbf{Operation Status Generation.}
% % 每一个item 收到user click,
% The state $s_t$ represents the user interest information known to the creator $c$ at time $t$, which is a partial information. It includes the average number of clicks received for items created in each category and the number of clicks received for the last created item. Note that if the creator has not previously produced content in a particular category, the information available to them is marked as "unknown".
In this module, we apply the fast-and-slow thinking mechanism~\cite{kahneman2011thinkingfastandslow, yao2024hdflow, lin2024swiftsage} to equip our creator agents with human-like analysis and creation abilities. The thought process is composed of two phases~\cite{kahneman2011thinkingfastandslow}: slow thinking for strategic planning and analysis, and fast thinking for rapid content generation based on experience and instinct.

\textbf{Slow thinker.}
During each creation,  user feedback on the most recently created item directly affects the creator's judgment on whether to continue with the current creative strategy or change it.
At the start of each time step $n$, creator $c\in \mathcal{C}^n$ gets three key factors that affect his creation strategy as inputs: (1) the utility of the most recently created item $i$, i.e.,$\mathbf{z}_i(n)$
, (2) the skill belief $\mathbf{B}^{\text{skill}}_{c}$ and audience belief $\mathbf{B}^{\text{aud}}_{c}$, and (3) the social identity $P^s_c$ and intrinsic motivation $P^m_c$. Then, these inputs are then fed into the LLM via the designed Chain-of-Thought~\cite{wei2022cot} prompt $P_1$ for slow thinking:
\begin{equation}
\mathbf{A}^{\text{exp}}_c(n) = \text{LLM}\left[ P_1\left(g(i), \mathbf{z}_i(n)
,\mathbf{B}^{\text{aud}}_{c}(n), \mathbf{B}^{\text{skill}}_{c}(n), P^s_c, P^m_c \right) \right],
\end{equation}
where  $i = h(\mathcal{I}_c)$ represent the most recent item created by creator $c$, where $h(\cdot)$ fetches the new item within $(\cdot)$, and $g(i)$ denotes the genre of item $i$. $\mathbf{A}^{\text{exp}}_c(n)$ is the text-based explore/exploit action taken by the creator agent $c$ at time step $t$. Specifically, CreAgent will choose whether to continue creating within the current genre $g(i)$ or to explore another genre $\mathcal{I}/\{g(i)\}$. The detailed format of prompt $P_1$ and action $\mathbf{A}^{\text{exp}}_c$ is in Appendix~\ref{app:prompt}.
% \footnote{\url{https://anonymous.4open.science/r/CreAgent-9B4B}\label{github}}.
% Specifically, LLM will utilize its world knowledge and analytical capabilities to determine whether to continue creating in the current genre.

\textbf{Fast thinker.} 
\label{sec:fast_thinker}
After generating analytical results, the creator agent will produce content based on these findings. The content is primarily divided into four sections: item title, item genre, item tags, and item description. Before creating content, the creator agent will retrieve the creation experience $f(\mathcal{M}^{\text{cre}}_c)$ from the creation memory $\mathcal{M}^{\text{cre}}_c$ based on the action $\mathbf{A}^{\text{exp}}_c(n)$, to assist the fast thinker in the creation process.
\begin{equation}
    \mathbf{A}^{\text{cont}}_c(n) = \text{LLM}\left[ P_2\left(f\left(\mathcal{M}^{\text{cre}}_c, \mathbf{A}^{\text{exp}}_c(n)\right), P^s_c, P^m_c\right)\right],
\end{equation}
where $P_2$ is the designed prompt for fast thinking. The item $\mathbf{A}^{\text{cont}}_c(t)$ is the content of the newly created item, which contains time title, genre, tags, and description, $f(\cdot_1, \cdot_2)$ is the retriever which retrieves the most relevant and recent creation history from $(\cdot_1)$ based on $(\cdot_2)$. The detailed format of $P_2$ and $\mathbf{A}^{\text{cont}}_c$ is in Appendix~\ref{app:prompt}.
% \textsuperscript{\ref{github}}.


\subsection{Fine-tuning to Enhance Creative Ability}
\label{sec:ppo}
In this section, we use the Proximal Policy Optimization (PPO) algorithm~\cite{schulman2017ppo} to fine-tune CreAgent.  By simulating the real-world creation cycle of creating, receiving rewards, analyzing, and creating again, we aim to:
(1) improve CreAgent's understanding of creators' limited information status and user feedback; (2) enhance CreAgent's analytical and creative capabilities.

\textbf{Reward modeling.}
In real platform scenarios, creators receive user feedback on their created items from the platform and benefit from it, which also helps them adjust their existing creation strategies. In this paper, we utilize the PPO algorithm and treat the the platform environment (Section~\ref{sec:platform}) as the reward model, to replicate the process through which creators learn their creative strategies in the real world. For each created item $i\in\mathcal{I}_c$, the reward is the weighted utility until the current time step $n$, i.e., $\lambda \mathbf{z}_i(n)$,
where $\lambda$ is a constant coefficient used to ensure training stability by preventing the reward from being too large or too small.
% Since each creator has a different level of activity (creation frequency), the time since their last created item varies. To ensure this difference does not impact our agent’s training, we use $\lambda=\frac{1}{m-t(i_{\text{rec}})+1}$ to normalize the total user engagement to each time step.

\textbf{Replay buffer.}
In real-world scenarios, user feedback is often not collected immediately but takes some time to accumulate. After the creation of item $i$ at time step $n=t(i)$, we first store the state $s^c_n=(\mathbf{B}^{\text{skill}}_{c,g}(n), \mathbf{B}^{\text{aud}}_{c,g}(n))$) and action $a^c_n=(\mathbf{A}^{\text{exp}}_c(n), \mathbf{A}^{\text{cont}}_c(n))$ in the log. After accumulating $N_r$ steps util step $n+N_r$, we save the reward $r^c_{n}= \lambda \mathbf{z}_i(n+N_r)$, state, and action related to the item $(s^c_{n}, a^c_{n}, r^c_{n})$ into the replay buffer. For every $N_u$ step, we update the policy by sampling $M$ records from the replay buffer.
% (\mathbf{B}^{\text{skill}}_{c,g}(n), \mathbf{B}^{\text{aud}}_{c,g}(n) )

% (\mathbf{A}^{\text{exp}}_c(n), \mathbf{A}^{\text{cont}}_c(n) )
%%符号都是 z_c(n)

\textbf{PPO optimization.}
To avoid the LLM policy $\pi_{\mathbf{\theta}}$ with parameter $\theta$ deviating from the initial reference policy $\pi_{\mathbf{\theta}_{\text {init}}}$ too far, we follow~\cite{ouyang2022training} and introduce the KL-divergence penalty into the current reward function. Therefore, the policy optimization formula is:
\begin{equation}
\mathcal{L}_{P P O}=\mathbb{E}\left[\lambda \mathbf{z}_i(n)-\beta K L\left(\pi_{\mathbf{\theta}_{\text {init}}}, \pi_{\mathbf{\theta}}\right)\right],
\end{equation}
where $\lambda$ is the hyper-parameter to prevent final reward $\lambda \mathbf{z}_i(n)$ too high to train stably.
% where $r_t(\theta)$ The reward obtained at time step $t$ depends on the policy parameter $\theta$.

% \begin{equation}
% \mathcal{L}^{C L I P}(\theta)=\hat{\mathbb{E}}_t\left[\min \left(r_t(\theta) \hat{A}_t, \operatorname{clip}\left(r_t(\theta), 1-\epsilon, 1+\epsilon\right) \hat{A}_t\right)\right],
% \end{equation}
% where $r_t(\theta)$ is the probability ratio of the new and old policies, and $\hat{A}_t$ is the advantage estimate at time $t$, reflecting the relative benefit of the selected action~\cite{xxx}.




\section{Platform Environment}
\label{sec:platform}
In this section, we introduce a simulated platform environment that mirrors the one creators engage with. This environment is primarily for receiving items uploaded by creators and providing them with user feedback. As illustrated in Figure~\ref{fig:overview} (b), the platform environment is mainly divided into the recommendation system and user agents.

\textbf{Recommender system modeling.}
The RS of the simulation platform primarily focuses on extensibility. As shown in Figure~\ref{fig:overview}, it encompasses a dynamic item set that allows items to be added freely, a user set, a two-stage recommendation process that includes ranking and re-ranking, and a feedback log to store user feedback.
We discuss two aspects of environment construction that resonate with real-world RS, including two-stage ranking, and item-by-item recommendation.

\textbf{Two-stage ranking.}
In real industrial scenarios, recommendation lists are often generated through a multi-stage ranking process. We also consider this situation by allowing both the ranking and re-ranking algorithms to be flexibly replaced according to specific requirements. Through this design, the platform can flexibly adjust and evaluate different recommendation strategies, whether by applying only the ranking model (e.g., BPR~\cite{rendle2012bpr}) or by considering long-term objectives (e.g., fairness~\cite{wang2023surveyonfairness} and diversity~\cite{kunaver2017diversity_survey}) in the re-ranking stage.



\textbf{Item-by-item recommendation.} After generating the recommendation list, RS will recommend items to users on an item-by-item basis. This scenario is currently widely used in online recommendation platforms such as YouTube~\cite{song2012youtube} and TikTok. 







\textbf{User agents.}
Given the maturity of existing user simulator~\cite{wang2023recagent, zhang2024agent4rec, shi2019virtualtaobao}, we utilize the widely-used LLM-based user simulator RecAgent~\cite{wang2023recagent} as our user agent. RecAgent effectively simulates the human memory mechanism and short- and long-term interests by integrating sensory, short-term, and long-term memory, enabling it to simulate user behavior over time.
Also, to better reflect users' long-term behavior, we make some targeted changes. 
For the profile, we define a targeted social identity for each user, summarized by the LLM based on their interaction and comment history. The user's long-term interests and activity levels are also derived from pre-collected real-world dataset $\mathcal{U}^0$.
After being recommended, the user agent can take three actions: click the item, skip the item, or exit the platform. To ensure the credibility of the platform environment, we also conduct experiments to verify how effectively the interaction behavior of user agents aligns with that of real users (please see Appendix~).

% \subsection{The Simulation Platform for Long-term Evaluation}



\section{Creator Credibility Evaluation}
\label{sec:creator_distibution_alignment}
In this section, we conduct experiments to verify the effectiveness of CreAgent in long-term simulation by addressing the following research questions:
\\
\textbf{RQ1}: Can CreAgent well align with the real-world creator patterns?
\\
\textbf{RQ2}: Can our simulation platform well simulate the information asymmetry conditions in the real world?
\\
\textbf{RQ3}: What is the computational cost of our simulation platform?

% How reliable is the interaction behavior between CreAgent and the platform under information asymmetry?

% supported by behavior economics theories~\cite{mullainathan2000behavioral_ecnomics}.
% Before the credibility evaluation on CreAgent, we first collect real-world platform data and utilize it to initialize CreAgent's profile.

\subsection{Experimental Setups}
\textbf{Real-world dataset collection.}
We initiate a focused data collection effort to tackle the lack of recommendation datasets that include detailed information about content creators and their items. We leverage the YouTube Data API to collect item information and comment data from channels in \textit{YouTube}~\cite{song2012youtube}, the world's leading content platform, known for its diverse and influential content creators. This choice is strategic to ensure our simulation closely mirrors the real-world platform.
Specific details of the dataset are provided in Appendix~\ref{sec:dataset}.
% \footnote{\url{https://anonymous.4open.science/r/CreAgent-9B4B}\label{github}}.

\textbf{Simulation setups.} 
We utilize two A6000 GPUs for simulations, setting the number of user agents $|\mathcal{U}|=100$ and creator agents $|\mathcal{C}|=50$, with a total of $N=100$ simulation steps. All agents are powered by the Llama3-8B~\cite{touvron2023llama}, updated and executed in parallel using multi-threading. The recommendation list length $K=5$.  The RS model is trained on the real-world dataset $\mathcal{D}^0$ and retrained every $T=5$ step using all interaction data $ \mathcal{D}^0\cup\mathcal{D}$ to reflect the periodic update of real-world RS. 
This section uses the Deep Interest Network (DIN)~\cite{zhou2018din} as the base RS model for the credibility evaluation.
To better align with real-world content platform scenarios~\cite{zhang2017timeliness, liang2024ensure_timeliness} and help to cold-start new items, we consider the timeliness of recommendation by removing excessively outdated items (older than $\eta=20$ steps) from the item recommendation pool.

\textbf{Baselines.} We select several classic creator behavior simulation baselines, where the baseline methods determine the genre of the next created item during each time step. Creator Feature Dynamics (\textbf{CFD})~\cite{lin2024cfd, zhan2021toward_content_provider_rec}: creators adjust their creation strategy using user feedback as the gradient, scaled by a learning rate.
Local Better Response (\textbf{LBR})~\cite{yao2024uwo}: creators generate a random direction and evaluate its utility (impact on user feedback). If beneficial, they update their strategy incrementally; otherwise, they maintain the current strategy. 
\textbf{SimuLine}~\cite{zhang2023simuLine}: creators' next creation is determined through probabilistic sampling based on the number of likes from previous steps. 
 % Due to the lack of characterization of creator content-generation methods and activity levels in existing baselines, the activity level and content-generation method are identical to CreAgent.




\subsection{\mbox{RQ1: Real-world Dataset Alignment}}
\label{sec:dataset_alignment}
We first evaluate whether the data pattern simulated by CreAgent aligns with the real-world creator pattern, which ranges from four key aspects: preferences, diversity, activity, and content. Then, we conduct an ablation study to reveal the importance of key modules.


% Due to the lack of activity modeling and content creation modeling in the baseline methods, we compare the performance of the baseline methods and CreAgent in terms of preference and diversity alignment. 


% For additional alignment experiments, please see Appendix~\ref{app:additional_exp_on_cre}.



% \subsubsection{Creation Content Alignment}

\begin{figure}
    \centering
    \includegraphics[width=1\linewidth]{fig/creator_alignment.pdf}
    % \vspace{-1em}
    \caption{Comparison between the creation genre preference, diversity, and activity of the ground-truth and agent-simulated result.}
    % \vspace{-1em}
    \label{fig:creation_align}
\end{figure}


\begin{table}[!ht]
    \centering
    \caption{Comparison of the divergence between the simulated and real-world distributions using Jensen-Shannon divergence~\cite{menendez1997JS_div}, with genre-level for preference evaluation and individual-level for diversity.}
        % \vspace{-1em}
    \begin{tabular}{l|cc}
    \toprule
      Creator Modeling Method &Preference & Diversity \\ \midrule
      Creator Feature Dynamics~\cite{lin2024cfd, zhan2021toward_content_provider_rec} & 0.2537 & 0.7204 \\ 
       Local Better Response~\cite{yao2024uwo} 
       & 0.2833 & 0.6284 \\ 
SimuLine~\cite{zhang2023simuLine} 
& 0.3175 &  0.6949\\ \hline
       CreAgent (w/o Audience Belief) & 0.2671 & 0.6159 \\ 
        CreAgent (w/o Skill Belief) & 0.2928 & 0.5532 \\ 
        CreAgent (w/o Fast-Slow Thinker) & 0.1728 & 0.5638 \\ 
       \textbf{CreAgent} & $\mathbf{0.1667}$ & $\mathbf{0.3014}$ \\ \bottomrule
    \end{tabular}
    \label{tab:baseline}
        % \vspace{-1em}
\end{table}


\subsubsection{\textbf{Creation Preference Alignment}}
In real platforms, each creator has diverse preferences that influence their content and shape the platform’s ecosystem. To accurately replicate this in our simulation, we first need to
ensure that the simulated creators’ preference distribution aligns with the real-world dataset
% For the experiment details, please see Appendix~\ref{exp_detail:behavior}.


As shown in Figure~\ref{fig:creation_align} (a), we plot the creation genre distribution of the YouTube dataset (i.e., the light blue bar) and our simulation (i.e., the deep blue bar). Specifically, the x-axis represents item genres, and the y-axis is the proportion of creation times all CreAgents created in each genre during the first 10 environment initialization steps.
From the comparison, we observe that our creator agents ultimately achieved a genre distribution comparable to the real-world dataset, effectively replicating a similar content ecosystem. 
Additionally, we notice some differences in certain categories (e.g., CreAgents show a stronger preference for genre 7 and less for genre 6 compared to real creators), which we attribute to the influence of the LLM’s pre-trained knowledge. Table~\ref{tab:baseline} demonstrates that CreAgent exhibits greater consistency with the dataset’s preference distribution compared to the baselines.



\subsubsection{\textbf{Creation Diversity Alignment}}
% \begin{figure}
%     \centering
% \includegraphics[width=1\linewidth]{fig/provider_div_alignment.pdf}
%     \caption{Comparison between the distributions of ground-truth and agent-simulated creation diversity}
%     \label{fig:diversity}
% \end{figure}

Creator's creation diversity is also a key aspect. Some real-world creators focus on a single category of items, while others actively explore different categories. 
Figure~\ref{fig:creation_align} (b) presents a histogram of the creation diversity for creators on the simulated platform and those in the YouTube dataset. We use the entropy of genre frequencies to represent each creator’s diversity (the x-axis). The solid lines in the figure represent the Kernel density estimation (KDE) curves~\cite{wkeglarczyk2018kde}. We observe that CreAgent effectively replicate the diversity distribution observed in the YouTube dataset: most creators stick to a single genre, while a few actively explore different genres (the y-axis is on a log scale). The superiority compared to the baseline is demonstrated in Table~\ref{tab:baseline}.





\subsubsection{\textbf{Creation Activity Alignment}}

% \begin{figure}
%     \centering
%     \includegraphics[width=1\linewidth]{fig/provider_freq_alignment.pdf}
%     \caption{Comparison between the individual level distributions of ground-truth and agent-simulated creation activity.}
%     \label{fig:enter-label}
% \end{figure}

Due to the varying creation frequencies of creators in real life, we conduct experiments to evaluate the alignment of CreAgent’s creation activity with real-world creators.
As shown in Figure~\ref{fig:creation_align} (c), we plot the individual-level distribution of creator activity. The deep blue dashed line represents the average creation count per month for 643 creators in the YouTube dataset (in decreasing order), while the light blue area represents the corresponding CreAgent's total creation times under 100 steps' simulation. Since we sample 50 creators per simulation, we conduct 20 experiments to count the average creation times of all 643 creators.
We can observe that the simulation results are very consistent with the actual distribution of creation counts, both exhibiting a long-tail distribution (the y-axis is on a log scale).

\subsubsection{\textbf{Creator Content Alignment}}
To ensure that our CreAgent can effectively simulate real creators, another crucial aspect is the consistency of the content.
As shown in Figure~\ref{fig:comparison_content}, we compare the content created by a music YouTuber with that generated by the corresponding CreAgent whose profile and creation memory are initialized from the YouTuber. It shows high consistency in form and content, indicating that CreAgent aligns well with real creators in its creation content.

\begin{figure}[!ht]
    \centering
\includegraphics[width=0.9\linewidth]{fig/comparison_content.pdf}
    % \vspace{-1em}
    \caption{A comparison example between CreAgent and content generated by real-world YouTubers.}
    \label{fig:comparison_content}
        % \vspace{-1em}
\end{figure}

% \subsubsection{Case Study of Creation Skill Improvement}
In real life, creators’ creative skills improve with the increasing number of creations they produce.
To assess the effectiveness of our fine-tuning strategy (see Section~\ref{sec:ppo}) in replicating such a process, we conduct a 100-time-step PPO fine-tuning process and record the creation history of a YouTuber, \textit{James Corden}. 
The creation memory is initialized as empty for the CreAgent to illustrate the creation skill improvement better.
Figure~\ref{fig:comparison_content2} compares the content created by CreAgent at the 1st and 100th step, showing significant improvements in content length, richness, and tag usage.
\begin{figure}[ht!]
\centering
\includegraphics[width=1\linewidth]{fig/comparison_content2.pdf}
% \vspace{-1em}
\caption{An example of the impact of fine-tuning on improving the content generated by CreAgent.}
    % \vspace{-1em}
\label{fig:comparison_content2}
\end{figure}


\subsubsection{\textbf{Ablation Study}}
We conduct an ablation study to investigate the impact of several key modules of CreAgent on creator behavior simulation and alignment (i.e., audience/skill belief, fast-slow thinker), as shown in Table~\ref{tab:baseline}.  We find that both audience belief and skill belief have a significant impact on preference alignment, as they contain information about real creators. Regarding diversity alignment, removing any of these three components greatly affects the results, proving that CreAgent requires all of them to achieve alignment with real-world creative diversity.


\subsection{RQ2: Interaction Behavior Alignment}
\label{sec:strategic_alignment}
% Building on the alignment of CreAgent's creative behaviors with those observed in real datasets, we further validate CreAgent to assess whether its strategic creative behaviors align with the actions of real humans in information asymmetry situations.
Under information asymmetry, the long-term interaction behaviors of CreAgent with the platform (i.e., creation behavior) follow certain patterns, which is supported by renowned behavioral economics theory~\cite{mullainathan2000behavioral_ecnomics}. To evaluate such interaction behavior of CreAgent, we selected the two most fundamental theories from behavioral economics: bounded rational~\cite{selten1990bounded_rationality, conlisk1996boundedrational} and prospect theory~\cite{kahneman2013prospecttheory}.






% \textbf{Baselines.}
% We select Local Better Response(\textbf{LBR})~\cite{yao2024userwelfare_opt} and Creator Feature Dynamics (\textbf{CFD})~\cite{lin2024cfd} as our baselines.
% \textbf{LBR}: At each step, a creator who decides to update her strategy would first generate an exploration direction $g_i$ ; then she would evaluate whether adjusting her strategy in this direction results in a higher utility. If so, she proceeds to update her strategy along $g_i$ in a pace of $\eta$; otherwise, she maintains her current strategy.
% \textbf{CFD} is a gradient-based strategy update method. It generates gradients based on the impact of previous user interactions with the creators and uses these gradients to update the current creator’s feature vector. This updated feature vector is then projected to the action space to produce the next action.
% \textbf{Random}: Creator Agents randomly choose one genre of item to create at each step.


\subsubsection{\textbf{Bounded Rationality}}


\begin{figure}[!ht]
    \centering
    \includegraphics[width=0.98\linewidth]{fig/bounded_with_baseline.pdf}
    % \vspace{-1em}
    \caption{Accumulated reward of creator agents over time steps, normalized by random creation strategy~\cite{shen2023hyperbandit}. } 
% \vspace{-1em}  
    \label{fig:bounded}
\end{figure}
Established behavioral economic studies~\cite{conlisk1996boundedrational, selten1990bounded_rationality} have revealed that \textit{
individuals may not always be fully rational when making decisions. Their choices can be influenced by limited information, leading to suboptimal decisions.}

Figure~\ref{fig:bounded} shows the normalized cumulative rewards~\cite{shen2023hyperbandit} of CreAgent under full and partial information scenarios compared with other baselines.
Full information refers to providing CreAgent with extra information regarding the distribution of user preferences in each genre. We can observe that CreAgent under full information achieves a higher reward than CreAgent under partial information, suggesting that CreAgent under limited information cannot always make fully rational decisions to maximize its reward. Limited information leads CreAgent to make suboptimal choices, which is consistent with human behavior under limited information~\cite{conlisk1996boundedrational}.
Additionally, when comparing CreAgent with the baselines, our approach demonstrates a higher accumulated reward, which indicates that CreAgent leverages human-like analyzing and creation abilities of LLM, exhibiting stronger decision-making and creative capabilities under limited information conditions.


% The shaded area indicates the 95\% confidence intervals of $t$-distribution under 5 experiments with different random seeds. (w/ and w/o fine-tuning)


% \begin{figure}
%     \centering
%     \includegraphics[width=1\linewidth]{fig/ecnomic_behavior.pdf}
%     \caption{(a) Creator agents make suboptimal decisions under limited information. (b) Creator agent's exploitation-to-exploration ratio increases as the previous creation's reward rises.}
% \label{fig:ecnomic}
% \end{figure}


\begin{figure}
    \centering
    \includegraphics[width=1\linewidth]{fig/prospect_three_with_base.pdf}
    % \vspace{-1em}
    \caption{Agent's next action proportion under different reward levels, with VL, L, M, H, and VH for very low, low, medium, high, and very high, respectively. The exploitation ratio of a higher reward level can be higher \textcolor{red}{$\uparrow$}, equal \textcolor{green}{$=$}, or lower \textcolor{blue}{$\downarrow$} to that of a lower reward level.}
    % \vspace{-1em}
    \label{fig:prospect}
\end{figure}

\subsubsection{\textbf{Prospect Theory}}
With only limited information available, the user feedback of  creators' creation action is uncertain and at risk, presenting them with the dilemma of exploration/exploitation. They can either continue creating content they are familiar with (i.e., exploitation) or try unfamiliar fields (i.e., exploration). 
The well-known behavior economic study Prospect Theory~\cite{kahneman2013prospecttheory, kahneman2013prospect_decision} suggests that:
\textit{individuals often exhibit asymmetric behavior in decision-making under uncertainty, where they are more sensitive to losses than to equivalent gains, leading to risk-avoiding in high-reward situations and risk-embracing in low-reward situations.}

 As reported in Figure~\ref{fig:prospect}, we plot the creator agent's next action proportion at different reward levels on the last created item. In Figure~\ref{fig:prospect}(e), when CreAgents under limited information receive a relatively low reward for their previous creation, their creation strategies are more aggressive (i.e., exploration action is nearly 40\%). Conversely, if they receive a high reward, they tend to stick to their existing content strategy to avoid risk, as evidenced by increased exploitation proportion from low to high rewards. 
For comparison, as shown in Figure~\ref{fig:prospect} (a-d), traditional simulation baselines cannot align with behavior economics principles, showing a similar exploration-exploitation ratio under both low-reward and high-reward scenarios.


\subsection{RQ3: Computaional Costs}

%\subsubsection{Discussion on Large-scale Simulation}

Though currently limited to small-scale simulations, we discuss computational complexity and costs to show that our simulator can support large-scale simulations with sufficient resources. As shown in Figure~\ref{fig:large} (a), as the number of simulated CreAgents increases, the time cost per step does not rise to an unacceptable level. Even with 1000 agents, the time per step stays under 15 minutes, offering a significant computational cost advantage over online A/B testing, which usually takes weeks or months for long-term evaluations. Moreover, we observe that the average computational cost per agent decreases as the number of agents increases. Additionally, Figure~\ref{fig:large} (b) shows that increasing the number of CPU workers can further reduce the time costs through multi-threaded parallelism, enabling large-scale simulations.
\begin{figure}[ht!]
\centering
\includegraphics[width=1\linewidth]{fig/large_scle.pdf}
    % \vspace{-1em}
\caption{ (a) Time cost per agent and time cost per step w.r.t. the number of simulated agents; (b) Time cost for 100 steps' simulation w.r.t number of CPU workers.}
\label{fig:large}
\end{figure}

% \section{Additional Experiment of CreAgent}
% \label{app:additional_exp_on_cre}





\section{\mbox{Long-term Effect Evaluation}}
\label{sec:RS_evaluation}
In this section, we use CreAgent and the simulation platform to assess the long-term effect of RS on different stakeholders.
Our evaluation focuses on three key long-term objectives that the multi-stakeholder content platform concerns~\cite{ebrahimi2023userengagement, wang2023recagent, xu2023p}: Can different RS models 
(1) improve user engagement (\textbf{RQ4}), (2) protect content creators (\textbf{RQ5}), and (3) enrich content (\textbf{RQ6}) in the long run?





\subsection{Experimental Setups}

\textbf{Evaluation setups.}
\label{sec:simulation_setups}
The experimental setup in this section is identical to Section~\ref{sec:creator_distibution_alignment}.
Note that the first $N_0-1=9$ steps are for environment initialization, followed by 90 evaluation steps starting at $N_0=10$ step.
We evaluate the performance of different ranking models which start recommending at step 1, and re-ranking beginning at step $N_0$.
In this paper, we consider the practical issue of creator retention, assuming a creator leaves if no clicks are received after 5 consecutive creations~\cite{xu2023p, mladenov2020optimizing_long-term}. 
\\
\textbf{Recommendation model selection.}
In this paper, we mainly focus on evaluating three types of recommendation models: (1) basic ranking models (i.e., Random, most popular (Pop), MF~\cite{koren2009mf},  BPR~\cite{rendle2012bpr}, DIN~\cite{zhou2018din}); and (2) provider-fair models (i.e., P-MMF~\cite{xu2023p}, FairRec~\cite{patro2020fairrec}, CPFair~\cite{naghiaei2022cpfair}, TFROM~\cite{wu2021tfrom}, FairCo~\cite{morik2020fairco}); and (3) diversity-aware models (i.e., MMR~\cite{carbonell1998mmr}, APDR~\cite{teo2016apdr}). For MF and BPR, we take user and item ID as inputs. For DIN, we incorporate the embedding of the generated textual content, along with creator ID, genre ID, and the user history interaction sequence, as input features.


% \subsection{Main Results}
% Table 2 reports the long-term performance of three different recommendation strategies, evaluated under our CreAgent framework. We conducted five experiments for each model with different random seeds. The large-font numbers represent the mean value of the five experiments, and the small-font numbers indicate the standard deviations.

% Overall, the ML-based recommendation strategies (Base Model) perform poorly in terms of long-term user engagement. In particular, id-based baseline ranking models like MF and BPR fail to capture user interests effectively, leading to inaccurate recommendations for users.
% In contrast, models like DIN, which integrate multiple features such as genre and creator, capture user interests more effectively, resulting in better long-term user engagement. However, this also leads to significant creator attrition (CRR is only 0.627) and exacerbates filter bubble issues (CGD and CCD metrics are way lower than models like BPR and MF).

% For diversity-aware strategies, we find that it can alleviate the filter bubble effect and creator attrition to some extent, but it still cannot completely eliminate these issues.

% For fairness-aware strategies, we observe that they achieve a good balance in terms of long-term user engagement, creator ecosystem protection, and filter bubble effects reduction.

% \begin{minipage}[l]{0.5\columnwidth}
% \centering
% \resizebox{1\columnwidth}{!}{
%     \begin{tabular}{l|ccc}
%     \toprule
%       & TUW & CRR 
%     & CGD  \\ \hline
%      Random & $6220_{\pm159.8}$ & $\mathbf{0.900}_{\pm0.000}$  & $2.221_{\pm 0.045}$    \\ 
%        Pop & $6167_{\pm647.0}$ & $0.500_{\pm0.028}$  & $2.146_{\pm0.095}$   \\ 
%        MF & $9322_{\pm 104.7}$ & $0.710_{\pm0.014}$ & $2.190_{\pm0.035}$  \\ 
%        BPR & $8554_{\pm353.4}$ & $0.620_{\pm0.000}$  & $\mathbf{2.223}_{\pm0.101}$   \\
%        DIN  & $\mathbf{11289}_{\pm 1353}$ & $0.627_{\pm0.012}$  & $1.872_{\pm0.145}$  \\

%     \bottomrule
%     \end{tabular}
%     }
% \end{minipage}
% \begin{minipage}[r]{0.5\columnwidth}
% \centering
% \resizebox{1\columnwidth}{!}{
%     \begin{tabular}{l|cccc}
%     \toprule
%       & TUW & CRR 
%     & CGD & CCD \\ \hline
%    Base & $11289_{\pm 1353}$ & $0.627_{\pm0.012}$  & $1.872_{\pm0.145}$  & $3.100_{\pm0.161 }$ \\
%               \hline
%            \multicolumn{5}{c}{Diversity-aware Strategy}  \\ \hline
%        +MRR & $11059_{\pm 114.6}$ & $0.68_{\pm 0.028 }$  & $2.017_{\pm 0.008 }$  & $3.389_{\pm 0.025 }$ \\
%         +APDR & $13489_{\pm 215.7}$ & $0.720_{\pm 0.057}$  & $1.974_{\pm0.185}$  & $3.303_{\pm0.104}$ \\
%        \hline
%             \multicolumn{5}{c}{Fairness-aware Strategy}  \\ \hline
%        +FairRec & $14108_{\pm 530.1}$ & $0.840 _{\pm 0.028 }$   & $1.918  _{\pm0.133 }$  & $3.387  _{\pm 0.050 }$ \\
%        +FairCo & $12749_{\pm 1955}$ & $0.960_{\pm 0.000 }$   & $\mathbf{2.246}_{\pm0.018}$  & $\mathbf{3.659}_{\pm 0.002 }$ \\
%       +TFROM & $11089_{\pm 326.7}$ & $0.920_{\pm 0.028 }$   &$2.144_{\pm 0.048 }$  & $3.534_{\pm 0.009 }$ \\  
%        +P-MMF& $13865_{\pm 225.6}$ & $\mathbf{1.000}_{\pm 0.000}$   & $2.228_{\pm 0.020}$  & $3.565  _{\pm 0.019}$ \\
%        +CPFair & $\mathbf{14506}_{\pm 605.3}$
%        & $0.940 _{\pm 0.028 }$   & $2.186  _{\pm 0.002 }$  & $3.623_{\pm 0.001}$ \\

%     \bottomrule
%     \end{tabular}
%     }
% \end{minipage}

\begin{table}[!ht]
    \centering
    \setlength\tabcolsep{5pt} 
        \caption{Rankings model evaluations, with five runs per model using various seeds, display mean and standard deviation with larger and smaller numbers, respectively.}
        % \vspace{-1em}
        \resizebox{1\columnwidth}{!}{
    \begin{tabular}{l|ccc}
    \toprule
    Models  & User Welfare & Creator Retention
    & Content Diversity\\ \midrule
     Random & $6220_{\pm159.8}$ & $\mathbf{0.900}_{\pm0.000}$  & $2.221_{\pm 0.045}$   \\ 
       Pop & $6167_{\pm647.0}$ & $0.500_{\pm0.028}$  & $2.146_{\pm0.095}$  \\ 
       MF & $9322_{\pm 104.7}$ & $0.710_{\pm0.014}$ & $2.190_{\pm0.035}$  \\ 
       BPR & $8554_{\pm353.4}$ & $0.620_{\pm0.000}$  & $\mathbf{2.223}_{\pm0.101}$  \\
       DIN  & $\mathbf{11289}_{\pm 1353}$ & $0.627_{\pm0.012}$  & $1.872_{\pm0.145}$  \\
    \bottomrule
    \end{tabular}
    }
    % \vspace{-1em}
    \label{tab:ranking_model}
\end{table}



\begin{table}[!ht]
    \centering
    \setlength\tabcolsep{5pt} 
        \caption{Evaluation of fairness- and diversity-aware models with DIN as the base model. Setups are same as Table~\ref{tab:ranking_model}.}
        % \vspace{-1em}
          \resizebox{1\columnwidth}{!}{
    \begin{tabular}{l|ccc}
    \toprule
     Models & User Welfare & Creator Retention
    & Content Diversity \\ \midrule
   Base & $11289_{\pm 1353}$ & $0.627_{\pm0.012}$  & $1.872_{\pm0.145}$   \\
           \hline
           \multicolumn{4}{c}{Diversity-aware}  \\  \hline
       +MMR & $11059_{\pm 114.6}$ & $0.68_{\pm 0.028 }$  & $2.017_{\pm 0.008 }$  \\
        +APDR & $13489_{\pm 215.7}$ & $0.720_{\pm 0.057}$  & $1.974_{\pm0.185}$ \\
       \hline
            \multicolumn{4}{c}{Fairness-aware}  \\ \hline
       +FairRec & $14108_{\pm 530.1}$ & $0.840 _{\pm 0.028 }$   & $1.918  _{\pm0.133 }$   \\
       +FairCo & $12749_{\pm 1955}$ & $0.960_{\pm 0.000 }$   & $\mathbf{2.246}_{\pm0.018}$ \\
      +TFROM & $11089_{\pm 326.7}$ & $0.920_{\pm 0.028 }$   &$2.144_{\pm 0.048 }$ \\  
       +P-MMF& $13865_{\pm 225.6}$ & $\mathbf{1.000}_{\pm 0.000}$   & $2.228_{\pm 0.020}$   \\
       +CPFair & $\mathbf{14506}_{\pm 605.3}$
       & $0.940 _{\pm 0.028 }$   & $2.186  _{\pm 0.002 }$ \\

    \bottomrule
    \end{tabular}
}
% \vspace{-1em}
    \label{tab:reranking_model}
\end{table}



\begin{figure}[!ht]
    \centering
\includegraphics[width=1\linewidth]{fig/ablation.pdf}
% \vspace{-1em}
    \caption{Changes in (a) remaining creators and (b) content enrichment with the increase of time steps under models.}
    \label{fig:remain_creator}
    % \vspace{-1em}
\end{figure}


  % \vspace{-0.5em}
\subsection{RQ4: User Engagement}
\label{sec:user_engagement}
\textbf{Motivation.}
Long-term user engagement has always been one of the key goals pursued by the platform~\cite{bardhan2022more, ebrahimi2023userengagement}.
We present the Total User Welfare (TUW) metric for evaluation, which is defined as the cumulative user click number:  
\begin{equation}
    \text{Total User Welfare} = \sum_{n=N_0}^{N} \sum_{u\in \mathcal{U}^n} \sum_{i \in \mathcal{I}^n}\mathbf{Y}_{u, i}(n).
\end{equation}
\\
\textbf{Results.}
% 表1展现了一些常见的推荐策略 which mainly focus on the short-term user experience.
% 我们发现Naive的推荐策略,如random 和pop,在长期的user engagement上表现的不尽人意。而一些基于item-id的排序模型,如MF和BPR,虽然在这基础上有所提升,但是也无法完美的捕捉用户的兴趣。考虑了商品的genre特征和creator特征的模型(i.e., DIN)能够很好的捕捉用户的长期兴趣,获得了较好的user engagement.
Table~\ref{tab:ranking_model} shows the long-term user engagement for various ranking models. Simple strategies like Random and Pop perform poorly in this context. While id-based models like MF and BPR offer some improvement, they still fall short in fully grasping user interests. However, feature-based model, such as DIN, more effectively capture long-term user interests, leading to enhanced engagement.

Experiment results in Table~\ref{tab:reranking_model} suggested that diversity and fairness-aware models do not significantly harm long-term user engagement. For instance, MMR only reduced engagement by 2.03\%, and TFROM by 1.77\% compared to the base model. Interestingly, some fairness-aware models (e.g., CPFair), even showed significant improvements, increasing engagement by 28.50\% over the base model.
From the perspective of users, we attribute this phenomenon to the exploration of users’ unknown interests: Short-sighted strategies often focus on users’ existing interests while neglecting their potential interests. Fairness-aware and diversity-aware models will be more effective in discovering users’ potential interests.
% (2) Effectively protection of the creator ecosystem. 
% (3) Effectively alleviation of the filter bubble effect.
% The later two reasons will be explained in detail in Section~\ref{sec:ecosystem_maintainance} and Section~\ref{sec:filter_bubble_alleviation}.






\subsection{RQ5: Creator Protection}
\label{sec:creator_protection}
\textbf{Motivation.} 
The protection of creators is crucial for platform growth and user attraction~\cite{bardhan2022more}. However, short-sighted recommendation strategies may prevent niche creators from gaining enough exposure, potentially leading to their departure from the platform~\cite{wu2021tfrom, xu2023p, hohnhold2015focusing}. This can reduce creator diversity and hinder the long-term development~\cite{bardhan2022more}.

To assess the protection of creators of different recommendation models (\textbf{RQ5}), we propose the Creator Retention Rate (CRR) metric, which is defined as the alive creator (i.e., creators who have not yet left) number in the N-th step divided by the alive creator number in the $N_0$-th step.
% \begin{equation}
%     \text{CRR} = \frac{|\mathcal{C}^N|}{|\mathcal{C}^{10}|}.
% \end{equation}
\\
\textbf{Result.}
Table~\ref{tab:ranking_model} and Table~\ref{tab:reranking_model} show the creator protection performance of different RS models. Aside from the random strategy's high CRR (which gives equal exposure to each item), other strategies have low CRR (below 0.8), suggesting a loss of over 20\% of initial creators after 100 simulation steps.
% Given that short-sighted recommendation strategies (e.g., BPR, DIN) fail to maintain the creators' basic living allowance for ecology protection, we attempt to evaluate whether diversity and fairness strategies could address this issue.
Figure~\ref{fig:remain_creator} illustrates the remaining creator proportion as the simulation progresses. The base model (DIN) experiences a rapid decline in creators, whereas diversity-aware and fairness-aware algorithms significantly slow this departure. Notably, different fairness algorithms achieve varying degrees of creator protection. For instance, P-MMF~\cite{xu2023p}, designed to boost exposure for worst-exposed creators, retains all creators by the 100th time step, thus effectively safeguarding the platform's creator ecosystem.




% \begin{table}[t]
% \caption{
% Elo ratings for different models w.r.t. ``Coherence'', ``Relevance'', ``Comprehensiveness'' and  ``Overall'' evaluated by GPT-4 on \ourdata dataset.
% }
% \centering
% \scriptsize
% \setlength{\tabcolsep}{2pt}

% \begin{tabular}{lcp{6.8mm}cp{6.8mm}cp{6.8mm}cp{6.8mm}}
% \toprule
% & \multicolumn{8}{c}{Evaluation Aspects}
% \\ \cmidrule(rl){2-9}
% & \multicolumn{2}{c}{Coherence}           
% & \multicolumn{2}{c}{Relevance}         
% & \multicolumn{2}{c}{Comprehensiveness}   
% & \multicolumn{2}{c}{Overall}  \\ \midrule
% {BM25}        & \phantom{0}848           & \chart{848}{9}{cyan} & \phantom{0}845           & \chart{845}{8}{cyan} & \phantom{0}860           & \chart{860}{10}{cyan}  & \phantom{0}780           & \chart{780}{5}{cyan}\\

% {ANCE} & \phantom{0}934 & \chart{934}{23}{cyan}  & \phantom{0}936           & \chart{936}{23}{cyan} & \phantom{0}946           & \chart{946}{26}{cyan} & 1016           & \chart{1016}{48}{cyan}\\ 

% {TAS-B} & \phantom{0}995 & \chart{995}{41}{cyan}  & \phantom{0}991           & \chart{991}{39}{cyan} & \phantom{0}988           & \chart{988}{39}{cyan} & 1028           & \chart{1028}{60}{cyan}\\

% {coCondensor} & 1031 & \chart{1031}{54}{cyan}  & 1036           & \chart{1036}{56}{cyan} & 1041           & \chart{1041}{58}{cyan} & 1035           & \chart{1035}{55}{cyan}\\
% {Contriever} & 1076 & \chart{1076}{73}{cyan}  & 1082           & \chart{1082}{76}{cyan} & 1044           & \chart{1044}{59}{cyan} & 1046           & \chart{1046}{60}{cyan}\\
% \midrule
% {COLT (Ours)} & 1116 & \chart{1116}{93}{cyan}  & 1110           & \chart{1110}{90}{cyan} & 1121           & \chart{1121}{95}{cyan} & 1096           & \chart{1096}{82}{cyan}\\
% \bottomrule
% \end{tabular}
% \label{tab:downstream evaluation}
% \end{table}




\subsection{RQ6: Content Enrichment}
\label{sec:filter_bubble_alleviation}
% \begin{figure}
%     \centering
%     \includegraphics[width=1\linewidth]{fig/diversity_eva.pdf}
%     \caption{Evaluation of mitigating filter bubble effect: the long-term performance of different recommendation strategies in terms of content diversity.}
%     \label{fig:filter_bubble}
% \end{figure}

\textbf{Motivation.}
The enrichment of content is a critical issue for online platforms. Myopic strategies may lead to falling into a filter bubble effect~\cite{nguyen2014explor_filter_bubble, wang2023recagent}, causing decreased user experience, creator exposure unfairness~\cite{wang2023surveyonfairness}, and insufficient exploration of user interests~\cite{chen2021values_of_user_exploration}. 
To measure the level of content enrichment, we follow~\cite{wang2023recagent} and use entropy to quantify the degree of filter bubble effect~\cite{nguyen2014explor_filter_bubble}. We define content genre diversity (CGD) as the average genre entropy received by each user. It can be defined as follows:
\begin{equation}
   \text{Content Diverisity}  = -\frac{1}{\sum_{n=N_0}^N|\mathcal{U}^n|}\sum_{n=n_0}^N \sum_{u\in \mathcal{U}^n} \sum_{g\in\mathcal{G}} p_{u,g} \log \left(p_{u,g} \right),
\end{equation}
where $p_{u,g}=\frac{\sum_{n=N_0}^{N}\sum_{u\in\mathcal{U}^n} \sum_{i \in \mathcal{I}_g}\mathbf{E}_{u,i}(n)}{\sum_{n=N_0}^{N}\sum_{u\in\mathcal{U}^n} \sum_{i}\mathbf{E}_{u,i}(n)}$ is the frequency of genre $g$ recommended to user $u$.
A higher CGD value indicates greater content enrichment and a milder filter bubble effect.
\\
\textbf{Results.}
As reported in Table~\ref{tab:ranking_model}, despite the high long-term user engagement DIN achieves (mentioned in Section~\ref{sec:user_engagement}), it goes through a pronounced filter bubble effect, with a CGD value 15.8\% lower than BPR. We attribute this to its reliance on genre and creator features, which leads to greedy recommendations on the known interests of users.
We also examine the long-term impacts of diversity-aware and fairness-aware models on enriching the content, as illustrated in Figure~\ref{fig:remain_creator} (b) and Table~\ref{tab:reranking_model}.
Compared with the base model whose genre diversity (CGD) declines over time, diversity-aware strategies alleviate this, and fairness-aware strategies notably increase recommendation diversity, as evidenced by FairCo maintaining a consistently high enrichment over time.



\section{Conclusion and Future Works}
In this paper, we propose an LLM-empowered creator simulator to enhance the long-term evaluation of RS under information asymmetry scenarios. Extensive experiments demonstrate CreAgent's effectiveness in aligning with real-world creator behavior and providing valuable insights into the long-term effects of various RS models, including fairness- and diversity-aware algorithms, on the multi-stakeholder environment. 
Although CreAgent offer a promising research direction in multi-stakeholder RS simulation, there are still some limitations. For example, real creator behaviors may involve randomness (not entirely strategic) and multi-modality. We will consider these aspects in the future work.

% In future work, we will consider extending CreAgent to create more diverse forms of content (e.g., multi-modal content) to better adapt to real-world content platform scenarios. 
% 验证是否符合经济学规律


%We delve into both the capabilities and limitations of CreAgent by conducting extensive evaluations. 

\appendix

\newpage


\section*{Appendix}

\section{Detail of the Collected Youtube Dataset}
\label{sec:dataset}

Our dataset comprises a comprehensive set of 4,004 content creators, each with a unique influence level, as indicated by their follower counts varying from 10 to 12.32 K. Additionally, we included 1.97 M users and 3.97 M comments in total—spanning 0.19 M distinct items of 14 genres. Due to the inability to collect explicit preference behaviors from specific users (e.g., clicks and likes), we consider user comments as an indication of interest in the item.
Out of privacy protection concerns, we mask specific sensitive information of users and channels.  

Due to resource limitations, the number of creators and users simulated in this article is limited. Therefore, we specifically randomly sampled from the complete YouTube dataset (i.e., Big\_Youtube) to construct a more densely populated Small\_YouTube\ dataset.
To better present the details of the YouTube dataset we collected, we present the statistics of the dataset in Table~\ref{tab:youtube_dataset}.
The dataset has been shared at \textcolor{blue}{\url{https://github.com/shawnye2000/CreAgent}}.

\begin{table}[H]
    \centering
    \caption{Statistics of the collected YouTube Dataset}
    \label{tab:youtube_dataset}
     \begin{tabular}{l|cc}
    \toprule
        Dataset & Big\_Youtube & Small\_Youtube \\ \midrule
        \#Interactions & 3,970,123 & 40,479 \\ 
        \#Users & 1,967,066 & 1,571 \\ 
        \#Items & 186,164 & 64,300 \\ 
        \#Creators & 4,004 & 643 \\ 
         \#Genres & 14 & 14 \\ 
        Inter. Per User & 2.02 & 25.77 \\ 
        Item Per Creator & 46.49 & 100 \\ \bottomrule
    \end{tabular}
\end{table}






\section{Ablation Study on Different LLMs}


We employ two additional LLMs as base models to carry out experiments (that is, Mistral-7B and Qwen2-7B), to verify the capability and effectiveness of CreAgent under these LLMs and explore some new findings.
Due to the time and resource constraints during the rebuttal phase, the number of LLMs we could test is limited. We sincerely apologize for this. In the future, we will explore and evaluate CreAgent with more additional LLMs.

\subsection{Real-world data alignment evaluation}


\begin{table}[!ht]
    \centering
    \caption{Comparison of the divergence between the simulated and real-world distributions using Jensen-Shannon divergence~\cite{menendez1997JS_div}, with genre-level for preference evaluation and individual-level for diversity.}
    \begin{tabular}{l|ll}
    \toprule
       Simulation Method  & Preference  & Diversity \\ \midrule
     Creator Feature Dynamics~\cite{lin2024cfd, zhan2021toward_content_provider_rec} & 0.2537 & 0.7204 \\ 
       Local Better Response~\cite{yao2024uwo} 
       & 0.2833 & 0.6284 \\ 
SimuLine~\cite{zhang2023simuLine} 
& 0.3175 &  0.6949\\ \hline
        CreAgent(Mistral-7B) & 0.2045 & 0.4012 \\ 
        CreAgent(Qwen2-7B) & 0.1917 & 0.3979 \\ 
        CreAgent(LLama3-8B) & 0.1667 & 0.3014 \\ \bottomrule
    \end{tabular}
    
\end{table}

\textbf{Result.}. We observe varying performance when using different LLMs as the base model for our method. In terms of the categories of items created by the simulated creators, the results consistently aligned closely with real-world data. However, in the aspect of creative diversity, some models, such as Mistral-7B, exhibited weaker performance. We hypothesize that this is due to the fact that our prompts were fine-tuned specifically on LLama, leading to potential inconsistencies when applied to other models. Despite this variation, it is notable that while these models may underperform compared to LLama, CreAgent still outperformed the baselines (e.g., CFD[1], LBR[2]).




\subsection{Strategic behavior alignment evaluation}



\begin{table}[!ht]
    \centering
    \begin{tabular}{l|c}
    \toprule
   Simulation Method & Accumulated Reward\\ \midrule
          Creator Feature Dynamics~\cite{lin2024cfd, zhan2021toward_content_provider_rec} & 3.08 \\ 
     Local Better Response~\cite{yao2024uwo} & 2.51 \\ 
     SimuLine~\cite{zhang2023simuLine} 
& 3.04\\ \hline
        CreAgent(Mistral-7B) & 7.32 \\ 
        CreAgent(Qwen2-7B) & 7.98  \\ 
        CreAgent(LLama3-8B) & 8.11 \\ \bottomrule
    \end{tabular}
\end{table}



\textbf{Result.} For the strategic behavior of creator agents, we first conducted experiments to evaluate CreAgent’s reward acquisition capability using different LLMs as base models. All rewards were normalized against the random strategy, following the settings in the paper. As shown in the table, CreAgent consistently demonstrated superior analytical decision-making and creative abilities across almost all base models, achieving higher user rewards. However, we observe that not all base models achieved rewards comparable to LLama3-8B. For instance, Mistral-7B may have limitations in its post-pretraining capabilities, making it less effective at analyzing current user feedback.


\begin{table}[!ht]
    \centering
    \resizebox{1\columnwidth}{!}{
    \begin{tabular}{l|ccccc}
    \toprule
        Simulation Method & Very Low & Low & Medium & High & Very High \\ \midrule
        Random & 0.0431 & 0.0343 & 0.0356 & 0.0547 & 0.052 \\ 
Creator Feature Dynamics~\cite{lin2024cfd, zhan2021toward_content_provider_rec}& 0.8959 & 0.9867 & 0.9746 & 1.0000 & 1.0000 \\
    Local Better Response~\cite{yao2024uwo}& 0.4069 & 0.7009 & 0.6778 & 0.6102 & 0.4583 \\ 
SimuLine~\cite{zhang2023simuLine}& 0.9104&0.9712& 0.9556 & 0.9476&1.0000 \\ \hline
        CreAgent(LLama3-8B) & 0.6138 & 0.8953 & 0.9220 & 0.9250 & 0.9498 \\
        CreAgent(Qwen2-7B) & 0.7272 & 0.8057 & 0.7848 & 0.8170 & 0.8333 \\ 
        CreAgent(Mistral-7B) & 0.4822 & 0.7500 & 0.8471 & 0.8261 & 0.8235 \\ \bottomrule
    \end{tabular}}
\end{table}



\textbf{Result.} We conducted experiments to evaluate the exploration-exploitation balance of creator agents under varying reward levels, assessing whether their behavior aligns with prospect theory. The table highlights the proportion of exploitation actions taken after receiving different levels of user feedback on newly-created items.
 While different LLMs exhibited varying exploration-exploitation levels under different rewards, they all displayed patterns resembling human behavior**. Specifically, agents showed a strong inclination to explore under low rewards and a remarkable tendency to exploit under high rewards, reflecting loss-seeking under low returns and risk aversion under high returns. This behavior sharply contrasts with the random strategy and traditional embedding-based baselines. For instance, CFD shifted to 100\% exploitation at high reward levels, while LBR paradoxically reduced exploitation proportions under high rewards.



We encourage future research to utilize CreAgent and our simulation platform to explore the capabilities and limitations of LLMs in simulating human behavior.



\section{User Alignment Evaluation}
\label{sec:user_alignment}

In this section, we conduct experiments on the user agent employed in our simulation environment to validate how effectively it aligns with real-world user preferences and behaviors.


\subsection{User Item Preference Alignment}
\begin{table}[!ht]
    \centering
    \caption{User Item Preference Alignment}
    \begin{tabular}{c|cccc}
    \toprule
        1:m & Accuracy  & Precision  & Recall  & F1 Score  \\ \midrule
        1:1 & $0.630_{\pm0.031}$  & $\mathbf{0.658 }_{\pm0.023 }$  & $0.603 _{\pm0.070 }$  & $\mathbf{0.589}_{\pm0.054 }$ \\ 
        1:2 & $0.598_{\pm0.037 }$  & $0.461 _{\pm0.042 }$  & $0.523 _{\pm0.059 }$  &$0.460 _{\pm 0.047  }$    \\
        1:3 & $0.622 _{\pm0.015 }$  &  $0.373 _{\pm 0.022 }$   & $0.520 _{\pm0.031 }$ & $0.404 _{\pm0.020 }$    \\ 
        1:9 & $\mathbf{0.653} _{\pm0.040 }$  & $0.276  _{\pm0.036 }$    & $\textbf{0.740}_{\pm0.063 }$  & $0.358_{\pm0.039 }$  \\ \bottomrule
    \end{tabular}
    \label{tab:user_item_pref}
\end{table}
\textbf{Motivation.} To validate how well generative agents align with the real-world preferences, we utilze the user agents to differentiate between items that actual users have engaged with and those they have not.
Specifically, a total of 200 agents will each be randomly assigned 20 items. Among these, the ratio between items the user has interacted with (i.e., $\mathbf{Y}_{u,i}(0) = 1$) but was not utilized for profile initialization and items the user has not interacted with (i.e., $\mathbf{Y}_{u,i}(0) = 0$) is set as 1:m, with $m \in \{1, 2, 3, 9\}$. Under this setting, user agent responses (i.e., $\mathbf{Y}_{u,i}(n) = 1, n\in [1,2, \cdots, N]$) to recommended items are considered binary discrimination, taking values between 0 and 1. Then, we compute the accuracy, precision, recall, and f1-score metric to show its performance.

\textbf{Results.} Table~\ref{tab:user_item_pref} reports the empirical discrimination results across various metrics. The best performance for each metric is highlighted in bold and marked with an asterisk. 
We observe that:
The generative user agents consistently identify items that align well with user preferences, maintaining around 65.3\% accuracy and 74\% recall even when faced with 18 (i.e., $1:m=1:9$) distracting items. This high performance is attributed to the personalized profiles that accurately reflect users' true interests, demonstrating the agents' ability to encapsulate real preferences and highlighting the viability of LLM-powered generative agents in recommendation systems.

In our item-by-item recommendation setting, the user agents do not tend to click on a certain number of items in the recommendation list, as mentioned in~\cite{zhang2024agent4rec}. However, our user agent can ensure a high level of Recall (above 0.5) and accuracy (nearly above 0.6) when 1:m decreases.
This indicates that our user agent, however, maintains a certain tendency towards identifying positive examples even when the proportion of similar items decreases, which may result in some negative items being identified as positive.
We attribute this failure to LLM's inherent hallucinations that agents tend to consistently pick a set number of items. However, we emphasize that in the subsequent simulation results with recommendation algorithms, the recommendation list length is set to 5, hence a substantial proportion of recommended items align with user preferences, thereby endorsing high trustworthiness in those simulation outcomes.





\subsection{User Genre Preference Alignment}
\begin{figure}[H]
    \centering
\includegraphics[width=1\linewidth]{fig/user_genre_alighment.pdf}
    \caption{Comparison between the distributions of ground-truth and agent-simulated genre preference.}
    \label{fig:user-interest-alignment}
\end{figure}

\textbf{Motivation.} In a real-world RS, users have unique interests in different genres of items. Since these interests drive user actions such as viewing, clicking, and liking, it is crucial for our user agents to align with the preferences of real-world users.
Specifically, we aim to align the interest distribution of our user agents with the distribution observed in the real-world dataset we have collected.
\\
\textbf{Results.}
To verify the consistency between the interest distribution of our user agent and that of real-world users, we conducted experiments using our CreAgent framework. We initialized user and creator profiles with the YouTube dataset we collected, utilizing DIN as the recommendation model and simulating 100 time steps. Figure~\ref{fig:user-interest-alignment} (a) shows the interest distribution of users from the real-world dataset, where users who clicked more than five times on a category are considered to favor that category. The x-axis represents item genres, sorted from highest to lowest proportionally. Figure~\ref{fig:user-interest-alignment} (b) illustrates the interest distribution of our user agent, where we compute the proportion of clicks for each genre.
From the comparison, we can see that our user agent ultimately achieved an interest distribution similar to that of real-world users. However, we are unable to perfectly replicate the relative differences across certain genres. For example, in the \textit{Howto \& Style} (H\&S) genre, the user agent exhibited a higher preference than real users, while extremely low preference is observed in \textit{Sports} (s). 
We attribute this to the LLM's extensive prior knowledge of genres, which causes the agent to exhibit a stronger preference for certain genres of items.

\section{Details of the Prompts}
\label{app:prompt}

\subsection{Prompts for Profile Summarization}
\label{app:profile_prompts}
\begin{tcolorbox}[colback=gray!5, colframe=gray!60, coltitle=white, breakable,fonttitle=\bfseries, title=Designed Prompt for Social Identity Summarization]

\textbf{Prompt}: You are a content creator on \{platform name\} and your name is \{creator name\}. Here is the basic information about the content you have previously created.

Recent created content: {title:<title>, genre:<genre>, description:<description>}

Created content genre (the genres you have created in the past and their respective proportions): \{created genre proportion\} 

Creation frequency (the average number of items you create each day): \{creation time per day\}

Please summarize your social identity in the following format: [Social Identity]: <the specific identity>. For example, [Social Identity]: movie enthusiast.
\end{tcolorbox}

\begin{tcolorbox}[colback=gray!5, colframe=gray!60, coltitle=white, breakable,fonttitle=\bfseries, title=Designed Prompt for Intrinsic Motivation Summarization]

\textbf{Prompt}: You are a content creator on \{platform name\} and your name is \{creator name\}. Here is the basic information about the content you have previously created.

Follower number: {follower number}

Average views per video: {average views}.

Recent created content: {title:<title>, genre:<genre>, description:<description>}

Recent interaction with users (your recent interaction records with the audience in the comments section.): \{recent comments\} 

Creation frequency (the average number of items you create each day): \{creation time per day\}

Intrinsic motivation refers to whether your purpose for creating content is for profit or simply for sharing. Please summarize your intrinsic motivation in the following format: [Intrinsic Motivation]: <the specific motivation>. For example, [Intrinsic Motivation]: profit. 
\end{tcolorbox}


\subsection{Prompts for Creation Module}
\label{app:creation_prompts}

\begin{tcolorbox}[colback=gray!5, colframe=gray!60, coltitle=white, breakable,fonttitle=\bfseries, title=Designed Prompt for Fast Thinker]

\textbf{Prompt}: You are a content creator on YouTube and your nickname is \{name\}. 

\{profile text of $P_c^m$,and $P_c^a$\}

Based on the analysis: \{$\textbf{A}^{\text{exp}}_c$\}, please create ONE new content for \{name\} that fits user's interest.

You can refer to the creation history of \{name\}: \{$f(\mathcal{M}^{\text{cre}}_c)$\}

Response in JSON dictionary format.
Write {{"name": [item name], "genre": genre1|genre2|....,  "tags": [tag1, tag2, tag3], "description": "item description text"}})
\end{tcolorbox}

\begin{tcolorbox}[colback=gray!5, colframe=gray!60, coltitle=white, breakable,fonttitle=\bfseries, title=Designed Prompt for Slow Thinker]

\textbf{Prompt}: You are a content creator on YouTube and your nickname is \{name\}. 

\{profile text of $P_c^m$,and $P_c^a$\}

The average utility per item of each genre \{name\} has created is as below: \{$\mathbf{B^{\text{aud}}_c}$\}. ([unknown] means the item genre \{name\} have not explored.

Recently, \{name\} created an item of genre \{$g_i$\}, and receives \{$\mathbf{z}_i(n)$\} utility.

Due to the statistical data, \{name\}'s profile and \{name\}'s 
familiarity on each genre: {$\mathbf{B^{\text{skill}}_c}$}, \{name\} must choose one of the two actions below to obtain more user clicks:

(1) [EXPLORE] Create content in a new genre that has not been explored before, which means other genres may have a larger audience and more opportunities to profit. But it might not be \{name\}'s area of expertise and requires greater effort to create.

(2) [EXPLOIT] Sticking to creating content of a familiar genre, which means \{name\} will leverage his creative expertise to build a stable brand identity. But it might limit \{name\}'s audience reach and lead to insufficient income.

To explore a new genre, write: [EXPLORE]:: <genre name>. If so, give the specific genre name chosen from {unknown cates}.

To stick to familiar genres, write: [EXPLOIT]:: <genre name>. If so, give the specific genre name chosen from {known cates}.
                          
  Let's think step by step. Please answer concisely and strictly follow the output rules.

\textbf{Responses Example of $\mathbf{A}^{\text{exp}}_c$}: [EXPLORE]: Entertainment
\end{tcolorbox}



% \begin{figure}[!ht]
%     \centering
% \includegraphics[width=0.9\linewidth]{fig/creation_skill.pdf}
%     \caption{An example illustrating the improvement of the creator's creative skills: a comparison of the content of items created by the creator agent at the first step and at the 100th step of the simulation.}
%     \label{fig:creation_example}
% \end{figure}













\newpage
%%
%% The next two lines define the bibliography style to be used, and
%% the bibliography file.
\bibliographystyle{ACM-Reference-Format}
\bibliography{sample-base}


%%
%% If your work has an appendix, this is the place to put it.
% \appendix




\end{document}
\endinput
%%
%% End of file `sample-sigconf.tex'.



\begin{table}[!ht]
    \centering
        \caption{Long-term evaluation on different ranking models utilizing our simulation platform. We conduct five experiments for each model with different random seeds. The numbers in large and small font represent the mean value and the standard deviations of the five experiments, respectively.}
        \resizebox{.8\columnwidth}{!}{
    \begin{tabular}{l|cccc}
    \toprule
      & TUW & CRR 
    & CGD & CCD  \\ \hline
     Random & $6220_{\pm159.8}$ & $\mathbf{0.900}_{\pm0.000}$  & $2.221_{\pm 0.045}$  & $3.572_{\pm0.003}$  \\ 
       Pop & $6167_{\pm647.0}$ & $0.500_{\pm0.028}$  & $2.146_{\pm0.095}$  & $3.376_{\pm0.065 }$  \\ 
       MF & $9322_{\pm 104.7}$ & $0.710_{\pm0.014}$ & $2.190_{\pm0.035}$  & $3.578_{\pm0.032 }$  \\ 
       BPR & $8554_{\pm353.4}$ & $0.620_{\pm0.000}$  & $\mathbf{2.223}_{\pm0.101}$  & $\mathbf{3.597}_{\pm 0.051 }$  \\
       DIN  & $\mathbf{11289}_{\pm 1353}$ & $0.627_{\pm0.012}$  & $1.872_{\pm0.145}$  & $3.100_{\pm0.161 }$ \\

    \bottomrule
    \end{tabular}
    }
    \label{tab:ranking_model}
\end{table}



\begin{table}[!ht]
    \centering
        \caption{Long-term evaluation on base ranking model, fairness-aware model, and diversity-aware model utilizing our simulation platform. We use the best-performed DIN as our base ranking model. Other setups are identical with Table~\ref{tab:ranking_model}.}
          \resizebox{.8\columnwidth}{!}{
    \begin{tabular}{l|cccc}
    \toprule
      & TUW & CRR 
    & CGD & CCD \\ \hline
   Base & $11289_{\pm 1353}$ & $0.627_{\pm0.012}$  & $1.872_{\pm0.145}$  & $3.100_{\pm0.161 }$ \\
              \hline
           \multicolumn{5}{c}{Diversity-aware Strategy}  \\ \hline
       +MRR & $11059_{\pm 114.6}$ & $0.68_{\pm 0.028 }$  & $2.017_{\pm 0.008 }$  & $3.389_{\pm 0.025 }$ \\
        +APDR & $13489_{\pm 215.7}$ & $0.720_{\pm 0.057}$  & $1.974_{\pm0.185}$  & $3.303_{\pm0.104}$ \\
       \hline
            \multicolumn{5}{c}{Fairness-aware Strategy}  \\ \hline
       +FairRec & $14108_{\pm 530.1}$ & $0.840 _{\pm 0.028 }$   & $1.918  _{\pm0.133 }$  & $3.387  _{\pm 0.050 }$ \\
       +FairCo & $12749_{\pm 1955}$ & $0.960_{\pm 0.000 }$   & $\mathbf{2.246}_{\pm0.018}$  & $\mathbf{3.659}_{\pm 0.002 }$ \\
      +TFROM & $11089_{\pm 326.7}$ & $0.920_{\pm 0.028 }$   &$2.144_{\pm 0.048 }$  & $3.534_{\pm 0.009 }$ \\  
       +P-MMF& $13865_{\pm 225.6}$ & $\mathbf{1.000}_{\pm 0.000}$   & $2.228_{\pm 0.020}$  & $3.565  _{\pm 0.019}$ \\
       +CPFair & $\mathbf{14506}_{\pm 605.3}$
       & $0.940 _{\pm 0.028 }$   & $2.186  _{\pm 0.002 }$  & $3.623_{\pm 0.001}$ \\

    \bottomrule
    \end{tabular}
}
    \label{tab:reranking_model}
\end{table}


\end{document}
\endinput
%%
%% End of file `sample-authordraft.tex'.
