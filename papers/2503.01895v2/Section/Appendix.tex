\newpage
\section*{Appendix}
\section{More Details of \method}\label{sec:detail_bench}
\subsection{Details of Datasets}
We provide details of datasets in Table~\ref{tab:series} and visualization in Figure~\ref{fig:OT_visualize}.
\begin{table*}[ht]
\centering
\caption{Overview of dataset in \method, covering key variables across nine domains with weekly, or monthly frequencies\label{tab:series}}
\resizebox{0.95\textwidth}{!}{

\begin{tabular}{ccccc}
\toprule
Domain & Target & Frequency & Timestamps & Timespan \\ \midrule
Agriculture & Retail Broiler Composite    & Monthly & 496496 & 1983 - 2024/04 \\
Climate  & Drought Level & Monthly & 496496 & 1983 - 2024/04 \\
Economy & International Trade Balance & Monthly & 423423 & 1989 - 2024/04 \\
Energy & Gasoline Prices & Weekly & 14791479 & 1996 - 2024/04 \\
Health & Influenza Patients Proportion & Weekly & 13891389 & 1997 - 2024/04 \\
Security & Disaster and Emergency Grants & Monthly & 297297 & 1999 - 2024/04 \\
Employment & Unemployment Rate & Monthly & 900900 & 1950 - 2024/04 \\ 
Traffic & Travel Volulmn & Monthly & 531531 &1980 - 2024/04 \\ \bottomrule
\end{tabular}
\vspace{-3mm}
}
\end{table*}
\begin{figure*}[ht]
    \centering
    \subfloat[Agriculture]{\includegraphics[height=0.1 \linewidth]{Figures/OT_Agriculture.png}}
    \subfloat[Climate]{\includegraphics[height=0.1 \linewidth]{Figures/OT_Climate.png}}
    \subfloat[Economy]{\includegraphics[height=0.1 \linewidth]{Figures/OT_Economy.png}}
    \subfloat[Energy]{\includegraphics[height=0.1 \linewidth]{Figures/OT_Energy.png}}
    \\
    \subfloat[Flu]{\includegraphics[height=0.12 \linewidth]{Figures/OT_Health.png}}
    \subfloat[Security]{\includegraphics[height=0.1 \linewidth]{Figures/OT_Security.png}}
    \subfloat[Employment]{\includegraphics[height=0.1 \linewidth]{Figures/OT_SocialGood.png}}
    \subfloat[Traffic]{\includegraphics[height=0.1 \linewidth]{Figures/OT_Traffic.png}}
    \caption{Visualization of time-sereis data.}
    \label{fig:OT_visualize}
\end{figure*}
\subsection{Details of Reasoning Strategies}
\subsubsection{Implementation of reusing LLMs for unimodal forecasting}
Prompts used for unimodal time-series forecasting are detailed in Figure~\ref{fig:tsf_prompt}
\subsubsection{Implementation of reusing LLMs for multimodal forecasting}
Prompts used for multimodal time-series forecasting are detailed in Figure~\ref{fig:tsf_prompt_context}
\subsubsection{Implementation of chain-of-thought reasoning}
Prompts used for chain-of-thought reasoning are detailed in Figure~\ref{fig:tsf_prompt_cot}
\subsubsection{Implementation of self-consistency reasoning}
We set the default number of sampled reasoning paths in self-consistency to 3, using the median as the most consistent reasoning path. Prompts used for self-consistency reasoning are detailed in Figure~\ref{fig:self_consistency}
\subsubsection{Implementation of self-correction reasoning}
We set the default number of iterations for self-correction to 2. Prompts used for self-correction reasoning are detailed in Figure~\ref{fig:self_correction_code}
\begin{figure*}[ht]
    \centering
    \begin{tcolorbox}
    \begin{lstlisting}[breaklines=true]
base_prompt = f"""As an expert in {target_name}, predict the trends from {future_dates[0]} to {future_dates[-1]} 
based on both historical data and contextual information.

Historical data (chronological order):
{historical_str}

Contextual information:
{context_text}

Please enclose your final predictions between [PRED_START] and [PRED_END] markers exactly like this:
[PRED_START]
2024-01-01: 123.45
2024-02-01: 124.56
[PRED_END]"""
    \end{lstlisting}
    \end{tcolorbox}
    \caption{Prompt used for multimodal time-series forecasting.}
    \label{fig:tsf_prompt_context}
\end{figure*}

\begin{figure*}[ht]
    \centering
    \begin{tcolorbox}
    \begin{lstlisting}[breaklines=true]
base_prompt = f"""As an expert in {target_name}, predict the trends from {future_dates[0]} to {future_dates[-1]} based on the historical data.

Historical data (chronological order):
{historical_str}

Please enclose your final predictions between [PRED_START] and [PRED_END] markers exactly like this:
[PRED_START]
2024-01-01: 123.45
2024-02-01: 124.56
[PRED_END]"""
    \end{lstlisting}
    \end{tcolorbox}
    \caption{Prompt used for unimodal time-series forecasting.}
    \label{fig:tsf_prompt}
\end{figure*}
\begin{figure*}[ht]
    \centering
    \begin{tcolorbox}
    \begin{lstlisting}[breaklines=true]
prompt = base_prompt + "\n\nLet's approach this step by step:\n1. Analyze historical trends\n2. Identify patterns\n3. Make predictions"
    \end{lstlisting}
    \end{tcolorbox}
    \caption{Prompt used for chain-of-thought reasoning.}
    \label{fig:tsf_prompt_cot}
\end{figure*}
\begin{figure*}[ht]
    \centering
    \begin{tcolorbox}
    \begin{lstlisting}[breaklines=true]
elif method == "self_consistency":
    predictions = []
    for _ in range(3):
        pred = get_single_prediction(base_prompt, future, target_name, api_key, model)
        predictions.append(pred)
    return average_predictions(predictions)
    \end{lstlisting}
    \end{tcolorbox}
    \caption{Prompt used for self-consistency reasoning strategy.}
    \label{fig:self_consistency}
\end{figure*}
\begin{figure*}[ht]
    \centering
    \begin{tcolorbox}
    \begin{lstlisting}[breaklines=true]
elif method == "self_correction":
    current_pred = get_single_prediction(base_prompt, future, target_name, api_key, model)
    for _ in range(2):
        correction_prompt = f"{base_prompt}\n\nPrevious prediction:\n{format_prediction(current_pred)}\n\nPlease review and improve the prediction."
        current_pred = get_single_prediction(correction_prompt, future, target_name, api_key, model)
    return current_pred
    \end{lstlisting}
    \end{tcolorbox}
    \caption{Prompt used for self-correction reasoning strategy.}
    \label{fig:self_correction_code}
\end{figure*}



\subsection{Details of Models}


\subsubsection{OpenAI Model Series}
We select GPT-4o and o1-mini as representatives of System 1 and System2 models from OpenAI. Released on Mar 13 2024, GPT-4o is an autoregressive multimodal model which integrates all inputs and outputs into a unified neural network, offering faster and more cost-effective performance than its predecessors.\footnote{https://openai.com/index/gpt-4o-system-card/} In comparison, o1-mini is trained with large-scale reinforcement learning to reason, and it shows better reasoning performance over GPT-4o. \footnote{https://openai.com/index/openai-o1-mini-advancing-cost-efficient-reasoning/}

\subsubsection{Google Model Series}
Taken from the latest Gemini models, Gemini 2.0 Flash and Gemini 2.0 Flash Thinking compose a pair of System 1 and 2 models. Based on the non-thinking version, Gemini 2.0 Flash Thinking is an experimental model designed to enhance reasoning capabilities by generating its own "thinking process" during responses. This self-articulation allows the model to break down complex tasks into smaller steps, leading to more robust and accurate reasoning outcomes.  \footnote{https://ai.google.dev/gemini-api/docs/thinking}.

\subsubsection{Deepseek Model Series}
DeepSeek-V3 is an open source 671B Mixture-of-Experts (MoE) language model. It employs Multi-head Latent Attention and DeepSeekMoE architectures for efficient inference and cost-effective training. DeepSeek-R1 surpasses V3 in reasoning tasks by incorporating cold-start data before reinforcement learning (RL), enhancing readability and performance. This approach distills reasoning patterns from larger models into smaller ones, resulting in superior reasoning capabilities compared to those developed through RL on smaller models. 
 \footnote{https://github.com/deepseek-ai/DeepSeek-V3} \footnote{https://github.com/deepseek-ai/DeepSeek-R1}



\subsection{Details of Evaluations}
We use the API from popular OpenRouter~\footnote{\url{https://openrouter.ai/}} platform to perform LLM inference. We strictly follow the recommended parameters for each LLM, i.e., median values from users on OpenRouter. We detail the LLM model versions and hyperparameter settings in Figure~\ref{fig:model_params_1}.
\begin{figure*}[ht]
    \centering
    \begin{tcolorbox}
    \begin{lstlisting}[breaklines=true]
model_params = {
    "openai/o1-mini-2024-09-12": {
        "top_p": 1,
        "temperature": 0.7,
        "frequency_penalty": 0,
        "presence_penalty": 0,
        "repetition_penalty": 1,
        "top_k": 0
    },
    "google/gemini-2.0-flash-thinking-exp:free": {
        "top_p": 1,
        "temperature": 0.8,
        "repetition_penalty": 1
    },
    "deepseek/deepseek-r1": {
        "top_p": 1,
        "temperature": 0.7,
        "frequency_penalty": 0,
        "presence_penalty": 0,
        "repetition_penalty": 1,
        "top_k": 0
    },
    "openai/gpt-4o-2024-05-13": {
        "top_p": 1,
        "temperature": 0.9,
        "frequency_penalty": 0,
        "presence_penalty": 0,
        "repetition_penalty": 1,
        "top_k": 0
    },
    "google/gemini-2.0-flash-exp:free": {
        "top_p": 1,
        "temperature": 0.7,
        "frequency_penalty": 0,
        "presence_penalty": 0,
        "repetition_penalty": 1,
        "top_k": 0
    },
    "deepseek/deepseek-chat": {
        "top_p": 1,
        "temperature": 0.9,
        "frequency_penalty": 0,
        "presence_penalty": 0,
        "repetition_penalty": 1,
        "top_k": 0
    }
}
    \end{lstlisting}
    \end{tcolorbox}
    \caption{Hyperparameter settings for different models.}
    \label{fig:model_params_1}
\end{figure*}

\section{Details of Test-Time Scaling Law Experiments}\label{sec:SCALE}
\subsection{Data Setup}
We adopted the TimesNet datasets as shown in Table \ref{tab:Datasets of Test-Time Scaling Law Experiments} for our experiments in this section. All datasets listed are open source datasets.

\begin{table*}[b]
\centering
\caption{Dataset for Test-Time Scaling Law Experiments}
\resizebox{0.5\textwidth}{!}{

\begin{tabular}{cccc}
\toprule
Name & Domain & Frequency & Timestamps  \\ \midrule
Electricity & Energy    & Hourly & 26305  \\
Traffic  & Traffic & Hourly & 17545  \\
Exchange Rate & Economy & Daily & 7589  \\
ETTh1 & Energy & Hourly & 17421  \\
ETTh2 & Energy & Hourly & 17421  \\
ETTm1 & Energy & 15 Min & 69681  \\ 
ETTm2 & Energy & 15 Min & 69681 \\ \bottomrule
\end{tabular}
\vspace{-3mm}
}
\label{tab:Datasets of Test-Time Scaling Law Experiments}

\end{table*}

\subsection{Model Setup}
We utilized two contemporary popular zero-shot TSF that supports distributional forecasting, Chronos \cite{ansari2024chronos} and Moirai \cite{woo2024unifiedtraininguniversaltime}, to carry out our investigation about relationship between self-consistency and model performance. Here we take the number of sampled paths as a reflection of self-consistency. We chose the "base" version in experiments for both models.

\subsection{Hyperparameters}
The context for forecasting is set to 512 and the prediction length is set to 96, representing a short-term forecasting for 3 months. The specific numbers of sampled paths are set to 1 as a control group, then 5, 10 and doubling itself to 160 for scaling up.

\subsection{Numerical Results}
In table \ref{tab:Numerical MSE Chronos},\ref{tab:Numerical MAE Chronos},\ref{tab:Numerical MSE Moirai},\ref{tab:Numerical MAE Moirai} we include original validation results for our test-time scaling law, besides normalized ones mentioned above.

\begin{table*}[b]
\centering
\caption{Numerical Results of Chronos (MSE)}
\resizebox{0.95\textwidth}{!}{

\begin{tabular}{cccccccc}
\toprule
Num Path $\backslash$ Dataset & Electricity & Traffic & Exchange Rate & ETTh1 & ETTh2 & ETTm1 & ETTm2 \\ \midrule
1 & 101164.19 & 2.17e-5 & 2.12e-3 & 13.95 & 50.00 & 9.62 & 37.42 \\
5  & 63658.19 & 1.68e-5 & 1.14e-3 & 11.59 & 34.73 & 6.50 & 21.57  \\
10 & 59627.12 & 1.60e-5 & 1.13e-3 & 10.29 & 31.32 & 5.94 & 20.71  \\
20 & 55347.09 & 1.58e-5 & 1.02e-3 & 10.06 & 30.47 & 5.58 & 19.68  \\
40 & 54779.91 & 1.55e-5 & 1.00e-3 & 9.98 & 29.96 & 5.60 & 18.86  \\
80 & 54301.96 & 1.53e-5 & 1.00e-3 & 9.80 & 29.25 & 5.52 & 18.44 \\
160 & 54233.27 & 1.53e-5 & 0.90e-3 & 9.98 & 29.52 & 5.50 & 18.54 \\  \bottomrule
\end{tabular}
\vspace{-3mm}
}
\label{tab:Numerical MSE Chronos}

\end{table*}

\begin{table*}[b]
\centering
\caption{Numerical Results of Chronos (MAE)}
\resizebox{0.95\textwidth}{!}{

\begin{tabular}{cccccccc}
\toprule
Num Path $\backslash$ Dataset & Electricity & Traffic & Exchange Rate & ETTh1 & ETTh2 & ETTm1 & ETTm2 \\ \midrule
1 & 225.51 & 2.77e-3 & 0.033 & 2.77 & 5.12 & 2.23 & 4.33 \\
5  & 176.63 & 2.22e-3 & 0.024 & 2.51 & 4.31 & 1.83 & 3.40  \\
10 & 168.59 & 2.17e-3 & 0.023 & 2.38 & 4.11 & 1.76 & 3.32 \\
20 & 162.45 & 2.11e-3 & 0.022 & 2.35 & 4.05 & 1.69 & 3.20 \\
40 & 161.03 & 2.07e-3 & 0.022 & 2.33 & 4.05 & 1.69 & 3.13 \\
80 & 160.15 & 2.03e-3 & 0.022 & 2.29 & 3.97 & 1.67 & 3.11 \\
160 & 159.89 & 2.03e-3 & 0.022 & 2.32 & 3.98 & 1.66 & 3.11 \\  \bottomrule
\end{tabular}
\vspace{-3mm}
}
\label{tab:Numerical MAE Chronos}

\end{table*}

\begin{table*}[b]
\centering
\caption{Numerical Results of Moirai (MSE)}
\resizebox{0.95\textwidth}{!}{

\begin{tabular}{cccccccc}
\toprule
Num Path $\backslash$ Dataset & Electricity & Traffic & Exchange Rate & ETTh1 & ETTh2 & ETTm1 & ETTm2 \\ \midrule
1 & 181007.83 & 3.23e-5 & 3.80e-3 & 31.55 & 68.30 & 40.91 & 65.14 \\
5  & 88376.75 & 1.73e-5 & 1.36e-3 & 14.57 & 42.92 & 8.45 & 32.44 \\
10 & 79590.41 & 1.61e-5 & 1.23e-3 & 13.15 & 40.11 & 7.63 & 31.56 \\
20 & 73638.04 & 1.57e-5 & 1.16e-3 & 12.31 & 37.42 & 6.91 & 27.54 \\
40 & 71017.52 & 1.55e-5 & 1.11e-3 & 11.88 & 36.90 & 6.69 & 27.26 \\
80 & 69676.72 & 1.53e-5 & 1.09e-3 & 11.70 & 36.39 & 6.58 & 27.23 \\
160 & 69018.68 & 1.52e-5 & 1.08e-3 & 11.63 & 36.27 & 6.53 & 27.16 \\  \bottomrule
\end{tabular}
\vspace{-3mm}
}
\label{tab:Numerical MSE Moirai}

\end{table*}

\begin{table*}[b]
\centering
\caption{Numerical Results of Moirai (MAE)}
\resizebox{0.95\textwidth}{!}{

\begin{tabular}{cccccccc}
\toprule
Num Path $\backslash$ Dataset & Electricity & Traffic & Exchange Rate & ETTh1 & ETTh2 & ETTm1 & ETTm2 \\ \midrule
1 & 277.88 & 3.21e-3 & 0.036 & 3.78 & 5.95 & 2.97 & 5.45\\
5  & 207.14 & 2.34e-3 & 0.026 & 2.77 & 4.84 & 2.09 & 4.25 \\
10 & 196.03 & 2.23e-3 & 0.025 & 2.65 & 4.68 & 2.01 & 4.19 \\
20 & 189.85 & 2.15e-3 & 0.024 & 2.52 & 4.50 & 1.88 & 3.87 \\
40 & 185.28 & 2.11e-3 & 0.023 & 2.48 & 4.44 & 1.84 & 3.84 \\
80 & 183.44 & 2.09e-3 & 0.023 & 2.46 & 4.44 & 1.83 & 3.84 \\
160 & 182.38 & 2.07e-3 & 0.023 & 2.45 & 4.43 & 1.82 & 3.83 \\  \bottomrule
\end{tabular}
\vspace{-3mm}
}
\label{tab:Numerical MAE Moirai}

\end{table*}
\section{Details of Reasoning-Annotated TSF Datasets}\label{data_detail}
We use GPT-4o with the agriculture and climate dataset as an example to demonstrate the reasoning annotated TSF dataset we built. We organize the data in a Question and Answering format. 
\subsection{Agriculture}
\subsubsection{Unimodal TSF}
Demo is provided in Figure~\ref{fig:tsf_qa_example}.
\subsubsection{Multimodal TSF}
Demo is provided in Figure~\ref{fig:tsf_question_example}.
\subsection{Climate}
\subsubsection{Unimodal TSF}
Demo is provided in Figure~\ref{fig:tsf_drought_example}.
\subsubsection{Multimodal TSF}
Demo is provided in Figure~\ref{fig:tsf_drought_example_2}.
\begin{figure*}[ht]
    \centering
    \begin{tcolorbox}
    \begin{lstlisting}[breaklines=true]
## Question ##
"As an expert in the US Retail Broiler Composite, predict the trends from 2023-11-30 to 2024-01-31 based on the historical data.

Historical data (chronological order):
2015-11-30: 196.99; 2015-12-31: 193.15; 2016-01-31: 192.66; 2016-02-29: 187.42; ...
2023-08-31: 252.59; 2023-09-30: 246.16; 2023-10-31: 247.64

Please enclose your final predictions between [PRED_START] and [PRED_END] markers exactly like this:
[PRED_START]
2024-01-01: 123.45
2024-02-01: 124.56
[PRED_END]"

## Answer ##
"To make a prediction for the US Retail Broiler Composite prices for November 30, 2023, to January 31, 2024, let's first analyze the historical trends from previous years in these months.

Here's a summary of the data for November, December, and January over the years:
- Nov 2015: 196.99
- Dec 2015: 193.15
- Jan 2016: 192.66
...
- Nov 2022: 252.10
- Dec 2022: 247.03
- Jan 2023: 242.33

There appears to be a pattern where November values are not substantially different from December, and changes into January can vary. Recent years indicate a trend of increasing prices into December followed by a modest decrease in January.

For projection, considering recent trends showing an upward movement in prices, we'll apply a similar rate of change to the most current data points.

Given that the last available price is for October 2023 at 247.64, predicting a reasonable increase in November, a slight upward or stable trend in December, and then a small drop or stabilization in January:

[PRED_START]
2023-11-30: 249.75
2023-12-31: 250.85
2024-01-31: 248.50
[PRED_END]"
    \end{lstlisting}
    \end{tcolorbox}
    \caption{Reasoning annotated example of unimodal agriculture.}
    \label{fig:tsf_qa_example}
\end{figure*}
\begin{figure*}[ht]
    \centering
    \begin{tcolorbox}
    \begin{lstlisting}[breaklines=true]
## Question ##
"As an expert in the US Retail Broiler Composite, predict the trends from 2023-11-30 to 2024-01-31 
based on both historical data and contextual information.

Historical data (chronological order):
2015-11-30: 196.99; 2015-12-31: 193.15; 2016-01-31: 192.66; ...
2023-09-30: 246.16; 2023-10-31: 247.64

Contextual information:
2021-12-30: The United States broiler market has seen a 2% decrease in broiler-type chicks placed, ...
2023-10-26: The annual broiler per-capita consumption is reported in the Today's Beef Consumer Fall 2023 Update.

Please enclose your final predictions between [PRED_START] and [PRED_END] markers exactly like this:
[PRED_START]
2024-01-01: 123.45
2024-02-01: 124.56
[PRED_END]"

##Answer##
"Based on the historical data and contextual information provided, here is my prediction for the US Retail Broiler Composite from 2023-11-30 to 2024-01-31:\n\n1. **Historical Data Analysis**:\n   - Historically, prices have exhibited seasonal patterns with a tendency to decrease towards the end of the year and then either stabilize or slightly increase early in the year.\n   - November to January has shown some degree of variability, with occasional drops in prices in December followed by modest increases or stability in January.\n\n2. **Contextual Information**:\n   - The US broiler market faces various influencing factors like international demand, supply chain disruptions, labor shortages, and prices of feed like corn.\n   - The ongoing issues such as price-fixing charges among major poultry producers and bird flu have potential impacts on both supply and prices.\n   - The US market is connected to global trends, including export demands which are showing strong signals for growth in 2024.\n\n3. **Market Trends**:\n   - From recent reports, there is an expectation of a higher supply of chicken meat but with weaker domestic demand, potentially leading to slight price reductions.\n   - However, the international market remains robust, and overall inflation pressures could balance out some of the downward trends.\n\n4. **Conclusion**:\n   - Given the current market conditions, we can expect a slight decrease in prices towards the end of 2023 compared to the middle of the year. However, prices might stabilize or slightly increase in January 2024 as the market adjusts to new supply-demand dynamics and international pressures.\n\n[PRED_START]\n2023-11-30: 243.80\n2023-12-31: 241.50\n2024-01-31: 243.10\n[PRED_END]"
    \end{lstlisting}
    \end{tcolorbox}
    \caption{Reasoning annotated example of multimodal agriculture.}
    \label{fig:tsf_question_example}
\end{figure*}
\begin{figure*}[ht]
    \centering
    \begin{tcolorbox}
    \begin{lstlisting}[breaklines=true]
## Question ##
"As an expert in the US Drought Level, predict the trends from 2023-11-30 to 2024-01-31 
based on the historical data.

Historical data (chronological order):
2015-11-30: 3.35; 2015-12-31: 4.05; 2016-01-31: 2.04; ...
2023-09-30: 2.12; 2023-10-31: 2.17

Please enclose your final predictions between [PRED_START] and [PRED_END] markers exactly like this:
[PRED_START]
2024-01-01: 123.45
2024-02-01: 124.56
[PRED_END]"

## Answer ##
"To predict the US Drought Level trends from November 30, 2023, to January 31, 2024, 
we need to analyze historical seasonal patterns and recent trends.

### Seasonal Trends and Analysis:
1. **Late Fall to Early Winter (Nov to Jan):**
   - Historically, November to January sees diverse behavior but often a decrease from late fall to early winter.
   - Novembers have shown values like 1.75, 1.62, 2.71, 1.94, 1.94, 1.30, 2.44.
   - December and January values usually see some fluctuations, but generally, there's an inclination towards stabilizations or mild decreases.

2. **Recent Yearly Trends:**
   - 2022 and 2023 show end-of-year levels at:
     - 2022: 1.68 (Oct) to 2.44 (Nov to Dec) to 2.94 (Jan)
     - 2023: 2.17 (Oct latest data point)
   - We notice a slight rise in early winter periods, following potential weather patterns.

3. **Data Synthesis:**
   - Given slight upticks as we move into winter and the fact that the October 2023 value is 2.17, we can predict a modest rise by end of November.
   - Typical increases from November to January stay within a moderate range, respecting the patterns seen in similar past periods.

### Predictions:
Considering the above factors, here are the expected values:

[PRED_START]
2023-11-30: 2.35
2023-12-31: 2.55
2024-01-31: 2.75
[PRED_END]

This prediction reflects historical patterns, recent trends, and typical seasonal behaviors."
    \end{lstlisting}
    \end{tcolorbox}
    \caption{Reasoning annotated example of unimodal TSF for US Drought Level.}
    \label{fig:tsf_drought_example}
\end{figure*}
\begin{figure*}[ht]
    \centering
    \begin{tcolorbox}
    \begin{lstlisting}[breaklines=true]
## Question ##
"As an expert in the US Drought Level, predict the trends from 2023-11-30 to 2024-01-31 
based on both historical data and contextual information.

Historical data (chronological order):
2015-11-30: 3.35; 2015-12-31: 4.05; 2016-01-31: 2.04; ...
2023-09-30: 2.12; 2023-10-31: 2.17

Contextual information:
2021-12-30: Historical temperature, precipitation, and Palmer drought data from 1895 to present 
for climate divisions, states, and regions in the contiguous U.S. are available.
2023-10-26: The contiguous U.S. had its 18th-warmest October in the 129-year record.

Please enclose your final predictions between [PRED_START] and [PRED_END] markers exactly like this:
[PRED_START]
2024-01-01: 123.45
2024-02-01: 124.56
[PRED_END]"

## Answer ##
"Based on the historical data and contextual information provided, I will make predictions for 
the US Drought Level for the period from 2023-11-30 to 2024-01-31. I will take into account 
the trends in recent years and the seasonal patterns observed in the historical data, as well 
as the latest contextual information.

- Historical data suggests seasonal variations, with drought levels typically higher in the 
  winter months.
- Recent contextual data indicate an expected wetter-than-normal winter for the southern 
  contiguous U.S., which may alleviate drought conditions in those areas while potentially 
  leading to more pronounced drought in others.

Considering these factors, here are the predicted drought levels:

[PRED_START]
2023-11-30: 2.25
2023-12-31: 2.55
2024-01-31: 2.35
[PRED_END]"
    \end{lstlisting}
    \end{tcolorbox}
    \caption{Reasoning annotated example of multimodal TSF for US Drought Level.}
    \label{fig:tsf_drought_example_2}
\end{figure*}


