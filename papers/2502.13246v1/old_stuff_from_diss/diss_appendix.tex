\chapter{Dangerous Waters Appendix}
\label{chpt:appendix_03}

\subsection{Computational Metaphor Detection}
\label{app:computational}

Computational metaphor processing encompasses several tasks: metaphor identification (detection), interpretation (source domain identification), generation, and application \citep{shutova_models_2010,ge_survey_2023}. Metaphor identification has primarily focused on binary classification of whether or not an expression is metaphorical, with popular metaphor detection datasets including LCC \citep{mohler2016introducing}, TroFi \citep{birke2006clustering}, and the VUA metaphor corpus \citep{steen2010method}. Prior work has modeled metaphor using linguistic features, neural networks, and pretrained Transformer models such as BERT \citep{rai_survey_2020,kohli_cracking_2023}. Older feature-based methods relied on part-of-speech clustering \citep{shutova_metaphor_2010}, corpus statistics \citep{neuman_metaphor_2013}, topic transitions \citep{jang_metaphor_2016}, and abstractness and imageability lexicons \citep{tsvetkov_metaphor_2014}. Neural models for metaphor identification include architectures such as graph convolutional networks and bidirectional LSTMs with attention \citep{gao_neural_2018,mao_end--end_2019,dankers_being_2020,le_multi-task_2020}. Within the last few years, BERT-based models have become popular for this tasks and often have special architectures and training procedures to capture intuitions that metaphorical words differ in basic and contextual meanings \citep{liu_metaphor_2020,choi_melbert_2021,lin_cate_2021,aghazadeh_metaphors_2022,babieno_miss_2022,li_framebert_2023,li_finding_2024}. 
Recent research has explored question-answering and prompt-based approaches for metaphor detection, interpretation, and generation with LLMs \citep{dankin_can_2022,liu_testing_2022,joseph_newsmet_2023,lai_multilingual_2023,prystawski_psychologically-informed_2023,ichien_large_2024}. 

Metaphor interpretation and generation tasks typically benefit from combining such models with knowledge graphs (e.g., ConceptNet) and lexical resources (e.g., WordNet) \citep{stowe_metaphor_2021, chakrabarty_mermaid_2021,chakrabarty_i_2023,ge_explainable_2022,mao_metapro_2022,mao_metapro_2023}. Some computational metaphor work distinguishes between conventional (everyday) and novel (unconventional or literary) metaphors, as the latter is easier for computational models \citep{haagsma_detecting_2016,do_dinh_weeding_2018,neidlein_analysis_2020}. Prior work has also used metaphor detection as part of multitask learning frameworks. For example, \citet{le_multi-task_2020} uses word sense disambiguation as an auxiliary task to improve metaphor detection. Metaphor detection has itself been used as an auxiliary task to improve propaganda identification \citep{baleato_rodriguez_paper_2023}, political framing and stance detection \citep{huguet_cabot_pragmatics_2020}, and hate speech type and target detection \citep{lemmens_improving_2021}. 


Comparatively little NLP research has focused on metaphor analysis in political contexts. In a topic-agnostic computational study of metaphors in public Facebook posts by US politicians, \citet{prabhakaran_how_2021} find that metaphor use is associated with higher audience engagement, and that metaphor use increased among Democrats following their loss in the 2016 US presidential election. Other work on detecting dehumanizing language highlight the use of metaphors such as \textsc{vermin} and \textsc{animal}, and propose embedding-based techniques to quantify these metaphorical associations \citep{mendelsohn2020framework,engelmann_dataset_2024,zhang_beyond_2024}. \citet{zwitter_vitez_extracting_2022} develop an annotation scheme for identifying metaphors in Slovene migration discourse, and find \textsc{liquid}, \textsc{container}, and \textsc{natural phenomenon} metaphors to be the most prevalent. In their study of political speeches about immigration, \citet{card2022computational} use masked language model predictions from BERT to quantify associations between immigrants and dehumanizing metaphors (e.g. \textsc{animal} and \textsc{cargo}), and find that modern Republicans are significantly more likely than Democrats to use such metaphors in the past two decades (concepts considered:  “animals,” “cargo,” “disease,” “flood/tide,” “machines,” and “vermin”). Also higher use of dehumanizing metaphors when talking about Mexican immigrants vs. European immigrants.


\section{Metaphoricity Score Distributions}
Below are distributions of discourse-level (EMB), word-level (LLM), and combined (SUM) scores across the corpus of 204K tweets used for analysis.

\begin{figure}[htbp!]
    \centering
    \includegraphics[width=.5\columnwidth]{plots/sbert_distribution.pdf}
    \caption{Distribution of discourse-level (EMB) metaphoricity scores using SBERT}
    \label{fig:dist-sbert}
\end{figure}

\begin{figure}[htbp!]
    \centering
    \includegraphics[width=.5\columnwidth]{plots/gpt_distribution.pdf}
    \caption{Distribution of word-level (LLM) metaphoricity scores using GPT-4-Turbo}
    \label{fig:dist-gpt}
\end{figure}

\begin{figure}[htbp!]
    \centering
    \includegraphics[width=.5\columnwidth]{plots/sum_distribution.pdf}
    \caption{Distribution of combined (SUM) metaphoricity scores}
    \label{fig:dist-sum}
\end{figure}

\newpage
\section{Inter-Annotator Agreement}
This section contains more details about interannotator agreement, including comparisons across concepts and evaluation sampling strategies. 

\begin{figure}[htbp!]
    \centering
    \includegraphics[width=.5\columnwidth]{plots/agreement_full_sample.pdf}
    \caption{Interannotator agreement (Krippendorff's $\alpha$) for all metaphorical concepts. ``Overall'' refers to annotations for use of metaphorical language independent of concept.}
    \label{fig:agreement-overall}
\end{figure}

\begin{figure}[htbp!]
    \centering
    \includegraphics[width=.5\columnwidth]{plots/agreement_models.pdf}
    \caption{Interannotator agreement for each metaphorical concept separated by evaluation subsample. Blue bars (left) represent Krippendorff's $\alpha$ for tweet pairs sampled based on GPT-4 scores, and orange bars (right) represent Krippendorff's $\alpha$ for tweet pairs sampled based on SBERT scores.}
    \label{fig:agreement-models}
\end{figure}

\begin{figure}[htbp!]
    \centering
    \includegraphics[width=.5\columnwidth]{plots/agreement_person.pdf}
    \caption{Histogram of interannotator agreement for each of 196 annotators recuited via Prolific.}
    \label{fig:agreement-person}
\end{figure}


\section{Role of ideology in metaphor use}
The main portion of this paper shows results from models that include issue-generic policy frames \citep{boydstun2013identifying} as fixed effects. This appendix section shows that excluding these frames produces similar results.

\begin{figure}[htbp!]
    \centering
    \includegraphics[width=.5\columnwidth]{plots/marginal_effects_by_ideology_no_frames.pdf}
    \caption{Average marginal effects of ideology on metaphor use from regression model that excludes frames.}
    \label{fig:ideology-effect-no-frame}
\end{figure}

\begin{figure}[htbp!]
    \centering
    \includegraphics[width=.5\columnwidth]{plots/marginal_effect_of_ideology_strength_no_frames.pdf}
    \caption{Average marginal effects of ideology strength on metaphor use for liberals and conservatives. From regression model that excludes issue-generic policy frames.}
    \label{fig:magnitude-effect-no-frame}
\end{figure}


\begin{figure}[htbp!]
    \centering
    \includegraphics[width=.5\columnwidth]{plots/marginal_effects_by_frame_heatmap_long.pdf}
    \caption{Average marginal effect of issue-generic policy frames on metaphor scores for each concept. Higher values (darker red) represent that the policy frame is more strongly associated with a conceptual metaphor being cued. For example, the ``Security \& Defense'' frame is strongly associated with the use of \textsc{war} metaphors.}
    \label{fig:frame-effect}
\end{figure}

\begin{figure}
    \centering
\includegraphics[width=.5\columnwidth]{plots/marginal_effect_metaphoricity_terms.pdf}
    \caption{Average marginal effect of conservative ideology, ideology strength, and issue-generic policy frames on use of metaphorical language.}
    \label{fig:overall-metaphoricity-all-vars}
\end{figure}

\begin{figure}[htbp!]
    \centering
    \includegraphics[width=.5\columnwidth]{plots/marginal_effect_metaphoricity_ideology_frame.pdf}
    \caption{Group-average marginal effect of issue-generic policy frames on overall metaphoricity for conservatives (orange, lower bars) and liberals (blue, upper bars). For both liberals and conservatives, the ''Security \& Defense'' frame is most positively associated with using more metaphorical language, with a slightly stronger effect among conservatives. }
    \label{fig:ideology-by-frame}
\end{figure}



\begin{figure}[htbp!]
    \centering
    \includegraphics[width=.75\columnwidth]{plots/percent_liberal_by_threshold.pdf}
    \caption{Percentage of tweets written by liberals with metaphor scores greater than or equal to the threshold specified by the x-axis. Each line represents a different concept.}
    \label{fig:percent-liberal}
\end{figure}

\input{tables/metaphor_ideology_2024-05-26_frames_combined_binarized_ideology}
\input{tables/metaphor_ideology_2024-05-26_no_frames_combined_binarized_ideology}



\newpage
\section{Role of metaphor in user engagement}
The main portion of this paper shows results from models that include issue-generic policy frames and ideology as fixed effects. This appendix section shows that excluding these frames from the regression model produces similar results.

\begin{figure}[htbp!]
    \centering
    \includegraphics[width=.5\columnwidth]{plots/combined_no_frames_marginal_effects_on_engagement.pdf}
    \caption{Average marginal effects of metaphor scores on user engagement, from models that exclude issue-generic policy frames and ideology from fixed effects.}
    \label{fig:engagement-no-frame-no-ideology}
\end{figure}

\begin{figure}[htbp!]
    \centering
    \includegraphics[width=.5\columnwidth]{plots/combined_frames_marginal_effects_on_engagement.pdf}
    \caption{Average marginal effects of metaphor scores on user engagement, from models that exclude ideology but includes issue-generic policy frames as fixed effects.}
    \label{fig:engagement-no-frame}
\end{figure}

\begin{figure}[htbp!]
    \centering
    \includegraphics[width=.5\columnwidth]{plots/combined_frames_with_ideology_marginal_effects_on_favorites.pdf}
    \caption{Group-average marginal effects of metaphor scores on favorite counts (log-scaled) for liberals and conservatives }
    \label{fig:favorite-ideology}
\end{figure}


\input{tables/combined_frames_with_ideology_retweets}
\input{tables/combined_frames_retweets}
\input{tables/combined_no_frames_retweets}

\input{tables/combined_frames_with_ideology_favorites}
\input{tables/combined_frames_favorites}
\input{tables/combined_no_frames_favorites}