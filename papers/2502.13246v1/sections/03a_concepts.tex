% Please add the following required packages to your document preamble:
% \usepackage{booktabs}
% \usepackage{graphicx}
\begin{table}[t!]
\centering
\resizebox{.9\columnwidth}{!}{%
\begin{tabular}{@{}ll@{}}
\toprule
Source Domain &
  Example Expressions  \\ \midrule
\textsc{animal} &
  \begin{tabular}[c]{@{}l@{}}sheltering and feeding refugees\\ flocks, swarms, or stampedes\end{tabular}  \\ \midrule
\textsc{vermin} &
  \begin{tabular}[c]{@{}l@{}}infest or plague the country\\ crawling or scurrying in\end{tabular} \\ \midrule
\textsc{parasite} &
  \begin{tabular}[c]{@{}l@{}}leeches, scroungers, freeloaders\\ bleed the country dry\end{tabular}  \\ \midrule
\begin{tabular}[c]{@{}l@{}}\textsc{physical}\\\textsc{pressure}\end{tabular} &
  \begin{tabular}[c]{@{}l@{}}country bursting with immigrants\\ crumbling under the burden \end{tabular}  \\ \midrule
\textsc{water} &
  \begin{tabular}[c]{@{}l@{}}floods, tides, or waves \\ pouring into the country\end{tabular}  \\ \midrule
\textsc{commodity} &
  \begin{tabular}[c]{@{}l@{}}migrants as a cheap source of labor\\ being processed at the border\end{tabular}  \\ \midrule
\textsc{war} &
  \begin{tabular}[c]{@{}l@{}}immigrants as an invading army\\ hordes of immigrants marching in\end{tabular}  \\ \bottomrule
\end{tabular}%
}
\caption{Selected source domains (metaphorical concepts) for analysis. Appendix Table \ref{tab:concepts} has an expanded version with literature references for each concept.}
\label{tab:concepts_short}
\end{table}






\section{Metaphorical concepts}
We select seven source domains based on prior literature: \textsc{animal, vermin, parasite, physical pressure, water, commodity,} and \textsc{war} (Table \ref{tab:concepts_short}). 
%While all dehumanizing, e
Each concept creates a distinct logic about the perceived threat and plausible remedies. For example, \textsc{water} and \textsc{physical pressure} suggest that immigrants are a threatening force on the host country, metaphorically represented as a container \citep{charteris-black_britain_2006}. Potential solutions would reinforce the container, e.g., through border security. 
%The very existence of 
The \textsc{vermin} and \textsc{parasite} metaphors 
%are threatening, which makes 
make extermination and eradication plausible responses \citep{steuter_vermin_2010,musolff_metaphorical_2014,musolff_dehumanizing_2015}. %Although \textsc{vermin} and \textsc{parasite} are closely related concepts, they diverge slightly in the nature of the perceived threat: vermin are characterized by their large quantity and capacity to spread disease \citep{steuter_vermin_2010}, while parasites are characterized by living off a host body at the expense of the host \citep{musolff_metaphorical_2014}. 

% Three conceptual domains are related to living non-human organisms: \textsc{animal}, \textsc{vermin}, and \textsc{parasite}. Two domains are related to natural forces (\textsc{water} and \textsc{physical pressure}). The final two concepts involve objectification (\textsc{commodity}) and organized violence (\textsc{war}).

%While all of these metaphors liken immigrants to non-human entities, each creates a distinct logic about the perceived threat from immigrants and the plausible solutions to remedy such threats. For example, the \textsc{water} and \textsc{physical pressure} metaphors suggest that immigrants are a threatening force on a host country, which is in turn metaphorically represented as a container \citep{charteris-black_britain_2006}. Within these domains, \textit{reinforcement} of the container is a solution, which is metaphorically extended to more restrictive policy and border security \citep{charteris-black_britain_2006}. On the other hand, organism-related metaphors such as \textsc{vermin} and \textsc{parasite} create conceptual mappings through which the \textit{existence} of immigrants is a threat, thus making extermination and eradication plausible responses \citep{steuter_vermin_2010,musolff_metaphorical_2014,musolff_dehumanizing_2015}. Although \textsc{vermin} and \textsc{parasite} are closely related concepts, they diverge slightly in the nature of the perceived threat: vermin are characterized by their large quantity and capacity to spread disease \citep{steuter_vermin_2010}, while parasites are characterized by living off a host body at the expense of the host \citep{musolff_metaphorical_2014}. 

%Based on these unique underlying logics, some metaphorical concepts may be seen at more extreme and blatantly dehumanizing than others (e.g. \textsc{vermin} vs \textsc{commodity}), but there is also heterogeneity within each of these categories (e.g. \textit{historical waves of immigration} and \textit{immigrants pouring in} are both instances of the \textsc{water} metaphor, but the former may be viewed as more conventional and neutrally-valenced).