



\section{Introduction}


Metaphor, communication of one concept in terms of another, is abundant in political discourse. %Rooted in culture and cognition, 
Metaphor structures how we understand the world \citep{lakoff1980metaphors}, and is deployed consciously and subconsciously to structure our understanding of political issues in terms of accessible everyday concepts \citep{burgers_figurative_2016}. By creating conceptual mappings that emphasize some aspects of issues and hide others \citep{lakoff1980metaphors}, metaphor can affect public attitudes and policy preference \citep{boeynaems_effects_2017}. 

Grounded in linguistics and communication literature, we develop a new computational approach for measuring and analyzing metaphor at scale. We use this methodology to study dehumanizing metaphor in immigration discourse on social media, and analyze the relationship between metaphor use, political ideology, and user engagement. 



\begin{figure}
    \centering
    \includegraphics[width=.65\columnwidth]{figures/fig1.pdf}
    \caption{Dehumanizing sentence likening immigrants to the \textit{source domain concepts} of \textsc{water} and \textsc{vermin} via the words ``pour'' and ``infest''.}
    \label{fig:intro}
\end{figure}


We fist identify seven \textit{source domains}, concepts evoked in discussions of immigration, such as \textsc{water} or \textsc{vermin}.  (Figure \ref{fig:intro}). 
%Our method measures metaphor with respect to these source domain concepts by leveraging 
%We measure metaphors using both word-level and document-level signals \citep{brugman_metaphorical_2019}. 
We use large language models (LLMs) to detect metaphorical words along with document embeddings to detect metaphorical associations in context. Our method requires no manual annotation, but rather just (1) brief concept descriptions, and (2) a handful of example sentences that evoke each metaphorical concept. We evaluate our approach by creating a new crowdsourced dataset of 1.6K tweets labeled for metaphor, and compare several LLMs and prompting strategies. While we focus on U.S. immigration discourse on social media, our approach can be applied to 
%study metaphor in 
other political, cultural, and discursive contexts.

We then analyze metaphor usage in 400K U.S. tweets about immigration, with a specific focus on the relationship between metaphor, political ideology, and user engagement. We find that conservative ideology is associated with greater use of dehumanizing metaphor, but this effect varies across concepts. Among conservatives, more extreme ideology is associated with higher metaphor use. Surprisingly, while moderate liberals are more likely to use object-related metaphor, extreme liberal ideology is associated with higher use of creature-related metaphor. Moreover, creature-related metaphors are associated with more retweets, and this effect is primarily driven by liberals. We additionally conduct a qualitative analysis to identify diverse contexts in which liberals use such dehumanizing metaphor. Our study reveals nuanced insights only made possible by our novel approach, and highlights the importance and complexity of studying metaphor as a rhetorical strategy in politics.\footnote{We will make all annotated data, code, and model outputs available upon publication.}


% COntributions
% 1. develop new method for identifying and measuring metaphor that is theoretically grounded. 
% 2. this work focuses on immigration and dehumanizing metaphors. This method could be readily applied to other contexts: it only requires brief descriptions of concepts and a handful of example sentences that evoke the metaphor.
% 3. Create new dataset of 1600 tweets with crowdsourced metaphoricity judgments (with respect to different source domains)
% 4. Evaluate several prompts and LLMs on a new dataset of immigration tweets annotated for metaphors mapping to seven different source domain concepts. 
% 5. use a high-performing design to study the relationship between metaphor, political ideology, and user engagement. (point to hypotheses and research questions).
% 6. we show variation across concepts, and different effects by ideology. Creature vs object metaphors, etc. 

%Political metaphor research primarily focuses on politicians' speeches and mainstream news media \citep{charteris-black_britain_2006}. While social media is a popular source of political information across the globe, little is known about how ordinary people use metaphor on social media. 


%We develop a computational approach for measuring and analyzing metaphor in immigration discourse on social media, specifically Twitter. We then use this methodology to study the relationship between immigration metaphors, political ideology, and user engagement.