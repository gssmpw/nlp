\newpage
\appendix
\setcounter{figure}{0}
\setcounter{table}{0}
\renewcommand{\thefigure}{A\arabic{figure}}
\renewcommand{\thetable}{A\arabic{table}}


\section{Appendix}

This appendix contains additional details about our annotation process (§\ref{annotation}), methodology (§\ref{model-details}), model evaluation (§\ref{eval}), and analysis (§\ref{analysis}).


\subsection{Annotation Details}
\label{annotation}

This section includes details about annotator demographics, annotation statistics, the heuristic-based sampling procedure, and the full codebook.

\begin{table*}[!ht]\scriptsize
\centering
\caption{Experiment - Participants demographics}
\label{tab:partdemograph}
\begin{tabular}{cccccccccccccccccccc}
\hline
\multicolumn{10}{c}{\textbf{Plugin Group}} & \multicolumn{10}{c}{\textbf{Control Group}} \\ \hline
\multicolumn{1}{c|}{\multirow{2}{*}{\textbf{ID}}} & \multicolumn{1}{c|}{\multirow{2}{*}{\textbf{Gender}}} & \multicolumn{1}{c|}{\multirow{2}{*}{\textbf{Persona}}} & \multicolumn{2}{c|}{\textbf{Experience}} & \multicolumn{5}{c|}{\textbf{Facets}} & \multicolumn{1}{c|}{\multirow{2}{*}{\textbf{ID}}} & \multicolumn{1}{c|}{\multirow{2}{*}{\textbf{Gender}}} & \multicolumn{1}{c|}{\multirow{2}{*}{\textbf{Persona}}} & \multicolumn{2}{c|}{\textbf{Experience}} & \multicolumn{5}{c}{\textbf{Facets}} \\ \cline{4-10} \cline{14-20} 
\multicolumn{1}{c|}{} & \multicolumn{1}{c|}{} & \multicolumn{1}{c|}{} & \multicolumn{1}{c|}{\textbf{GitHub}} & \multicolumn{1}{c|}{\textbf{OSS}} & \multicolumn{1}{c|}{\textbf{MT}} & \multicolumn{1}{c|}{\textbf{SE}} & \multicolumn{1}{c|}{\textbf{R}} & \multicolumn{1}{c|}{\textbf{IP}} & \multicolumn{1}{c|}{\textbf{L}} & \multicolumn{1}{c|}{} & \multicolumn{1}{c|}{} & \multicolumn{1}{c|}{} & \multicolumn{1}{c|}{\textbf{GitHub}} & \multicolumn{1}{c|}{\textbf{OSS}} & \multicolumn{1}{c|}{\textbf{MT}} & \multicolumn{1}{c|}{\textbf{SE}} & \multicolumn{1}{c|}{\textbf{R}} & \multicolumn{1}{c|}{\textbf{IP}} & \multicolumn{1}{c}{\textbf{L}} \\ \hline \hline

\multicolumn{1}{c|}{P1} & \multicolumn{1}{c|}{M} & \multicolumn{1}{c|}{\tikzcirclenew[fill=blue]{3pt}} & \multicolumn{1}{c|}{Never} & \multicolumn{1}{c|}{No} & \multicolumn{1}{c|}{\tikzcirclenew[fill=blue]{3pt}} & \multicolumn{1}{c|}{\tikzcirclenew[fill=blue]{3pt}} & \multicolumn{1}{c|}{\tikzcirclenew[fill=blue]{3pt}} & \multicolumn{1}{c|}{\tikzcircle[fill=orange]{3pt}} & \multicolumn{1}{c|}{\tikzcirclenew[fill=blue]{3pt}} & \multicolumn{1}{c|}{P40} & \multicolumn{1}{c|}{W} & \multicolumn{1}{c|}{\tikzcircle[fill=orange]{3pt}} & \multicolumn{1}{c|}{Once} & \multicolumn{1}{c|}{No} & \multicolumn{1}{c|}{\tikzcircle[fill=orange]{3pt}} & \multicolumn{1}{c|}{\tikzcirclenew[fill=blue]{3pt}} & \multicolumn{1}{c|}{\tikzcircle[fill=orange]{3pt}} & \multicolumn{1}{c|}{\tikzcircle[fill=orange]{3pt}} & \multicolumn{1}{c}{\tikzcirclenew[fill=blue]{3pt}} \\ \hline

\multicolumn{1}{c|}{P2} & \multicolumn{1}{c|}{W} & \multicolumn{1}{c|}{\tikzcirclenew[fill=blue]{3pt}} & \multicolumn{1}{c|}{Once} & \multicolumn{1}{c|}{No} & \multicolumn{1}{c|}{\tikzcirclenew[fill=blue]{3pt}} & \multicolumn{1}{c|}{\tikzcirclenew[fill=blue]{3pt}} & \multicolumn{1}{c|}{\tikzcirclenew[fill=blue]{3pt}} & \multicolumn{1}{c|}{\tikzcircle[fill=orange]{3pt}} & \multicolumn{1}{c|}{\tikzcirclenew[fill=blue]{3pt}} & \multicolumn{1}{c|}{P41} & \multicolumn{1}{c|}{M} & \multicolumn{1}{c|}{\tikzcirclenew[fill=blue]{3pt}} & \multicolumn{1}{c|}{Once} & \multicolumn{1}{c|}{No} & \multicolumn{1}{c|}{\tikzcirclenew[fill=blue]{3pt}} & \multicolumn{1}{c|}{\tikzcirclenew[fill=blue]{3pt}} & \multicolumn{1}{c|}{\tikzcircle[fill=orange]{3pt}} & \multicolumn{1}{c|}{\tikzcircle[fill=orange]{3pt}} & \multicolumn{1}{c}{\tikzcirclenew[fill=blue]{3pt}} \\ \hline

\multicolumn{1}{c|}{P3} & \multicolumn{1}{c|}{M} & \multicolumn{1}{c|}{\tikzcirclenew[fill=blue]{3pt}} & \multicolumn{1}{c|}{Never} & \multicolumn{1}{c|}{No} & \multicolumn{1}{c|}{\tikzcircle[fill=orange]{3pt}} & \multicolumn{1}{c|}{\tikzcirclenew[fill=blue]{3pt}} & \multicolumn{1}{c|}{\tikzcirclenew[fill=blue]{3pt}} & \multicolumn{1}{c|}{\tikzcircle[fill=orange]{3pt}} & \multicolumn{1}{c|}{\tikzcirclenew[fill=blue]{3pt}} & \multicolumn{1}{c|}{P42} & \multicolumn{1}{c|}{M} & \multicolumn{1}{c|}{\tikzcircle[fill=orange]{3pt}} & \multicolumn{1}{c|}{Never} & \multicolumn{1}{c|}{No} & \multicolumn{1}{c|}{\tikzcircle[fill=orange]{3pt}} & \multicolumn{1}{c|}{\tikzcirclenew[fill=blue]{3pt}} & \multicolumn{1}{c|}{\tikzcircle[fill=orange]{3pt}} & \multicolumn{1}{c|}{\tikzcircle[fill=orange]{3pt}} & \multicolumn{1}{c}{\tikzcirclenew[fill=blue]{3pt}} \\ \hline

\multicolumn{1}{c|}{P4} & \multicolumn{1}{c|}{M} & \multicolumn{1}{c|}{\tikzcirclenew[fill=blue]{3pt}} & \multicolumn{1}{c|}{Never} & \multicolumn{1}{c|}{No} & \multicolumn{1}{c|}{\tikzcircle[fill=orange]{3pt}} & \multicolumn{1}{c|}{\tikzcirclenew[fill=blue]{3pt}} & \multicolumn{1}{c|}{\tikzcirclenew[fill=blue]{3pt}} & \multicolumn{1}{c|}{\tikzcircle[fill=orange]{3pt}} & \multicolumn{1}{c|}{\tikzcirclenew[fill=blue]{3pt}} & \multicolumn{1}{c|}{P43} & \multicolumn{1}{c|}{M} & \multicolumn{1}{c|}{\tikzcirclenew[fill=blue]{3pt}} & \multicolumn{1}{c|}{Never} & \multicolumn{1}{c|}{No} & \multicolumn{1}{c|}{\tikzcirclenew[fill=blue]{3pt}} & \multicolumn{1}{c|}{\tikzcirclenew[fill=blue]{3pt}} & \multicolumn{1}{c|}{\tikzcirclenew[fill=blue]{3pt}} & \multicolumn{1}{c|}{\tikzcircle[fill=orange]{3pt}} & \multicolumn{1}{c}{\tikzcirclenew[fill=blue]{3pt}} \\ \hline

\multicolumn{1}{c|}{P5} & \multicolumn{1}{c|}{M} & \multicolumn{1}{c|}{\tikzcirclenew[fill=blue]{3pt}} & \multicolumn{1}{c|}{Once} & \multicolumn{1}{c|}{No} & \multicolumn{1}{c|}{\tikzcirclenew[fill=blue]{3pt}} & \multicolumn{1}{c|}{\tikzcirclenew[fill=blue]{3pt}} & \multicolumn{1}{c|}{\tikzcirclenew[fill=blue]{3pt}} & \multicolumn{1}{c|}{\tikzcircle[fill=orange]{3pt}} & \multicolumn{1}{c|}{\tikzcirclenew[fill=blue]{3pt}} & \multicolumn{1}{c|}{P44} & \multicolumn{1}{c|}{M} & \multicolumn{1}{c|}{\tikzcirclenew[fill=blue]{3pt}} & \multicolumn{1}{c|}{Never} & \multicolumn{1}{c|}{No} & \multicolumn{1}{c|}{\tikzcirclenew[fill=blue]{3pt}} & \multicolumn{1}{c|}{\tikzcirclenew[fill=blue]{3pt}} & \multicolumn{1}{c|}{\tikzcircle[fill=orange]{3pt}} & \multicolumn{1}{c|}{\tikzcircle[fill=orange]{3pt}} & \multicolumn{1}{c}{\tikzcirclenew[fill=blue]{3pt}} \\ \hline

\multicolumn{1}{c|}{P6} & \multicolumn{1}{c|}{M} & \multicolumn{1}{c|}{\tikzcirclenew[fill=blue]{3pt}} & \multicolumn{1}{c|}{Once} & \multicolumn{1}{c|}{No} & \multicolumn{1}{c|}{\tikzcirclenew[fill=blue]{3pt}} & \multicolumn{1}{c|}{\tikzcirclenew[fill=blue]{3pt}} & \multicolumn{1}{c|}{\tikzcircle[fill=orange]{3pt}} & \multicolumn{1}{c|}{\tikzcircle[fill=orange]{3pt}} & \multicolumn{1}{c|}{\tikzcirclenew[fill=blue]{3pt}} & \multicolumn{1}{c|}{P45} & \multicolumn{1}{c|}{M} & \multicolumn{1}{c|}{\tikzcirclenew[fill=blue]{3pt}} & \multicolumn{1}{c|}{Never} & \multicolumn{1}{c|}{No} & \multicolumn{1}{c|}{\tikzcirclenew[fill=blue]{3pt}} & \multicolumn{1}{c|}{\tikzcirclenew[fill=blue]{3pt}} & \multicolumn{1}{c|}{\tikzcirclenew[fill=blue]{3pt}} & \multicolumn{1}{c|}{\tikzcircle[fill=orange]{3pt}} & \multicolumn{1}{c}{\tikzcirclenew[fill=blue]{3pt}} \\ \hline

\multicolumn{1}{c|}{P7} & \multicolumn{1}{c|}{W} & \multicolumn{1}{c|}{\tikzcircle[fill=orange]{3pt}} & \multicolumn{1}{c|}{Never} & \multicolumn{1}{c|}{No} & \multicolumn{1}{c|}{\tikzcircle[fill=orange]{3pt}} & \multicolumn{1}{c|}{\tikzcirclenew[fill=blue]{3pt}} & \multicolumn{1}{c|}{\tikzcircle[fill=orange]{3pt}} & \multicolumn{1}{c|}{\tikzcircle[fill=orange]{3pt}} & \multicolumn{1}{c|}{\tikzcirclenew[fill=blue]{3pt}} & \multicolumn{1}{c|}{P46} & \multicolumn{1}{c|}{M} & \multicolumn{1}{c|}{\tikzcircle[fill=orange]{3pt}} & \multicolumn{1}{c|}{Never} & \multicolumn{1}{c|}{No} & \multicolumn{1}{c|}{\tikzcirclenew[fill=blue]{3pt}} & \multicolumn{1}{c|}{\tikzcircle[fill=orange]{3pt}} & \multicolumn{1}{c|}{\tikzcircle[fill=orange]{3pt}} & \multicolumn{1}{c|}{\tikzcircle[fill=orange]{3pt}} & \multicolumn{1}{c}{\tikzcirclenew[fill=blue]{3pt}} \\ \hline

\multicolumn{1}{c|}{P8} & \multicolumn{1}{c|}{W} & \multicolumn{1}{c|}{\tikzcircle[fill=orange]{3pt}} & \multicolumn{1}{c|}{Never} & \multicolumn{1}{c|}{No} & \multicolumn{1}{c|}{\tikzcirclenew[fill=blue]{3pt}} & \multicolumn{1}{c|}{\tikzcircle[fill=orange]{3pt}} & \multicolumn{1}{c|}{\tikzcircle[fill=orange]{3pt}} & \multicolumn{1}{c|}{\tikzcircle[fill=orange]{3pt}} & \multicolumn{1}{c|}{\tikzcirclenew[fill=blue]{3pt}} & \multicolumn{1}{c|}{P47} & \multicolumn{1}{c|}{M} & \multicolumn{1}{c|}{\tikzcirclenew[fill=blue]{3pt}} & \multicolumn{1}{c|}{Never} & \multicolumn{1}{c|}{No} & \multicolumn{1}{c|}{\tikzcirclenew[fill=blue]{3pt}} & \multicolumn{1}{c|}{\tikzcirclenew[fill=blue]{3pt}} & \multicolumn{1}{c|}{\tikzcirclenew[fill=blue]{3pt}} & \multicolumn{1}{c|}{\tikzcircle[fill=orange]{3pt}} & \multicolumn{1}{c}{\tikzcirclenew[fill=blue]{3pt}} \\ \hline

\multicolumn{1}{c|}{P9} & \multicolumn{1}{c|}{M} & \multicolumn{1}{c|}{\tikzcirclenew[fill=blue]{3pt}} & \multicolumn{1}{c|}{Once} & \multicolumn{1}{c|}{No} & \multicolumn{1}{c|}{\tikzcirclenew[fill=blue]{3pt}} & \multicolumn{1}{c|}{\tikzcirclenew[fill=blue]{3pt}} & \multicolumn{1}{c|}{\tikzcirclenew[fill=blue]{3pt}} & \multicolumn{1}{c|}{\tikzcircle[fill=orange]{3pt}} & \multicolumn{1}{c|}{\tikzcirclenew[fill=blue]{3pt}} & \multicolumn{1}{c|}{P48} & \multicolumn{1}{c|}{M} & \multicolumn{1}{c|}{\tikzcirclenew[fill=blue]{3pt}} & \multicolumn{1}{c|}{Never} & \multicolumn{1}{c|}{No} & \multicolumn{1}{c|}{\tikzcircle[fill=orange]{3pt}} & \multicolumn{1}{c|}{\tikzcirclenew[fill=blue]{3pt}} & \multicolumn{1}{c|}{\tikzcirclenew[fill=blue]{3pt}} & \multicolumn{1}{c|}{\tikzcircle[fill=orange]{3pt}} & \multicolumn{1}{c}{\tikzcirclenew[fill=blue]{3pt}} \\ \hline

\multicolumn{1}{c|}{P10} & \multicolumn{1}{c|}{W} & \multicolumn{1}{c|}{\tikzcirclenew[fill=blue]{3pt}} & \multicolumn{1}{c|}{Never} & \multicolumn{1}{c|}{No} & \multicolumn{1}{c|}{\tikzcirclenew[fill=blue]{3pt}} & \multicolumn{1}{c|}{\tikzcirclenew[fill=blue]{3pt}} & \multicolumn{1}{c|}{\tikzcircle[fill=orange]{3pt}} & \multicolumn{1}{c|}{\tikzcircle[fill=orange]{3pt}} & \multicolumn{1}{c|}{\tikzcirclenew[fill=blue]{3pt}} & \multicolumn{1}{c|}{P49} & \multicolumn{1}{c|}{M} & \multicolumn{1}{c|}{\tikzcircle[fill=orange]{3pt}} & \multicolumn{1}{c|}{Never} & \multicolumn{1}{c|}{No} & \multicolumn{1}{c|}{\tikzcircle[fill=orange]{3pt}} & \multicolumn{1}{c|}{\tikzcirclenew[fill=blue]{3pt}} & \multicolumn{1}{c|}{\tikzcircle[fill=orange]{3pt}} & \multicolumn{1}{c|}{\tikzcircle[fill=orange]{3pt}} & \multicolumn{1}{c}{\tikzcirclenew[fill=blue]{3pt}} \\ \hline

\multicolumn{1}{c|}{P11} & \multicolumn{1}{c|}{M} & \multicolumn{1}{c|}{\tikzcircle[fill=orange]{3pt}} & \multicolumn{1}{c|}{Never} & \multicolumn{1}{c|}{Some} & \multicolumn{1}{c|}{\tikzcircle[fill=orange]{3pt}} & \multicolumn{1}{c|}{\tikzcircle[fill=orange]{3pt}} & \multicolumn{1}{c|}{\tikzcircle[fill=orange]{3pt}} & \multicolumn{1}{c|}{\tikzcircle[fill=orange]{3pt}} & \multicolumn{1}{c|}{\tikzcircle[fill=orange]{3pt}} & \multicolumn{1}{c|}{P50} & \multicolumn{1}{c|}{M} & \multicolumn{1}{c|}{\tikzcirclenew[fill=blue]{3pt}} & \multicolumn{1}{c|}{Never} & \multicolumn{1}{c|}{No} & \multicolumn{1}{c|}{\tikzcirclenew[fill=blue]{3pt}} & \multicolumn{1}{c|}{\tikzcirclenew[fill=blue]{3pt}} & \multicolumn{1}{c|}{\tikzcirclenew[fill=blue]{3pt}} & \multicolumn{1}{c|}{\tikzcircle[fill=orange]{3pt}} & \multicolumn{1}{c}{\tikzcirclenew[fill=blue]{3pt}} \\ \hline

\multicolumn{1}{c|}{P12} & \multicolumn{1}{c|}{M} & \multicolumn{1}{c|}{\tikzcirclenew[fill=blue]{3pt}} & \multicolumn{1}{c|}{Never} & \multicolumn{1}{c|}{No} & \multicolumn{1}{c|}{\tikzcirclenew[fill=blue]{3pt}} & \multicolumn{1}{c|}{\tikzcirclenew[fill=blue]{3pt}} & \multicolumn{1}{c|}{\tikzcirclenew[fill=blue]{3pt}} & \multicolumn{1}{c|}{\tikzcircle[fill=orange]{3pt}} & \multicolumn{1}{c|}{\tikzcirclenew[fill=blue]{3pt}} & \multicolumn{1}{c|}{P51} & \multicolumn{1}{c|}{M} & \multicolumn{1}{c|}{\tikzcirclenew[fill=blue]{3pt}} & \multicolumn{1}{c|}{Never} & \multicolumn{1}{c|}{No} & \multicolumn{1}{c|}{\tikzcirclenew[fill=blue]{3pt}} & \multicolumn{1}{c|}{\tikzcirclenew[fill=blue]{3pt}} & \multicolumn{1}{c|}{\tikzcirclenew[fill=blue]{3pt}} & \multicolumn{1}{c|}{\tikzcircle[fill=orange]{3pt}} & \multicolumn{1}{c}{\tikzcirclenew[fill=blue]{3pt}} \\ \hline

\multicolumn{1}{c|}{P13} & \multicolumn{1}{c|}{W} & \multicolumn{1}{c|}{\tikzcircle[fill=orange]{3pt}} & \multicolumn{1}{c|}{Once} & \multicolumn{1}{c|}{No} & \multicolumn{1}{c|}{\tikzcircle[fill=orange]{3pt}} & \multicolumn{1}{c|}{\tikzcircle[fill=orange]{3pt}} & \multicolumn{1}{c|}{\tikzcircle[fill=orange]{3pt}} & \multicolumn{1}{c|}{\tikzcircle[fill=orange]{3pt}} & \multicolumn{1}{c|}{\tikzcircle[fill=orange]{3pt}} & \multicolumn{1}{c|}{P52} & \multicolumn{1}{c|}{M} & \multicolumn{1}{c|}{\tikzcircle[fill=orange]{3pt}} & \multicolumn{1}{c|}{Never} & \multicolumn{1}{c|}{Some} & \multicolumn{1}{c|}{\tikzcircle[fill=orange]{3pt}} & \multicolumn{1}{c|}{\tikzcircle[fill=orange]{3pt}} & \multicolumn{1}{c|}{\tikzcirclenew[fill=blue]{3pt}} & \multicolumn{1}{c|}{\tikzcircle[fill=orange]{3pt}} & \multicolumn{1}{c}{\tikzcircle[fill=orange]{3pt}} \\ \hline

\multicolumn{1}{c|}{P14} & \multicolumn{1}{c|}{W} & \multicolumn{1}{c|}{\tikzcircle[fill=orange]{3pt}} & \multicolumn{1}{c|}{Never} & \multicolumn{1}{c|}{Some} & \multicolumn{1}{c|}{\tikzcirclenew[fill=blue]{3pt}} & \multicolumn{1}{c|}{\tikzcirclenew[fill=blue]{3pt}} & \multicolumn{1}{c|}{\tikzcircle[fill=orange]{3pt}} & \multicolumn{1}{c|}{\tikzcircle[fill=orange]{3pt}} & \multicolumn{1}{c|}{\tikzcircle[fill=orange]{3pt}} & \multicolumn{1}{c|}{P53} & \multicolumn{1}{c|}{M} & \multicolumn{1}{c|}{\tikzcircle[fill=orange]{3pt}} & \multicolumn{1}{c|}{Once} & \multicolumn{1}{c|}{No} & \multicolumn{1}{c|}{\tikzcircle[fill=orange]{3pt}} & \multicolumn{1}{c|}{\tikzcircle[fill=orange]{3pt}} & \multicolumn{1}{c|}{\tikzcirclenew[fill=blue]{3pt}} & \multicolumn{1}{c|}{\tikzcircle[fill=orange]{3pt}} & \multicolumn{1}{c}{\tikzcircle[fill=orange]{3pt}} \\ \hline

\multicolumn{1}{c|}{P15} & \multicolumn{1}{c|}{M} & \multicolumn{1}{c|}{\tikzcircle[fill=orange]{3pt}} & \multicolumn{1}{c|}{Never} & \multicolumn{1}{c|}{No} & \multicolumn{1}{c|}{\tikzcircle[fill=orange]{3pt}} & \multicolumn{1}{c|}{\tikzcirclenew[fill=blue]{3pt}} & \multicolumn{1}{c|}{\tikzcirclenew[fill=blue]{3pt}} & \multicolumn{1}{c|}{\tikzcircle[fill=orange]{3pt}} & \multicolumn{1}{c|}{\tikzcircle[fill=orange]{3pt}} & \multicolumn{1}{c|}{P54} & \multicolumn{1}{c|}{W} & \multicolumn{1}{c|}{\tikzcircle[fill=orange]{3pt}} & \multicolumn{1}{c|}{Never} & \multicolumn{1}{c|}{No} & \multicolumn{1}{c|}{\tikzcircle[fill=orange]{3pt}} & \multicolumn{1}{c|}{\tikzcirclenew[fill=blue]{3pt}} & \multicolumn{1}{c|}{\tikzcircle[fill=orange]{3pt}} & \multicolumn{1}{c|}{\tikzcircle[fill=orange]{3pt}} & \multicolumn{1}{c}{\tikzcircle[fill=orange]{3pt}} \\ \hline

\multicolumn{1}{c|}{P16} & \multicolumn{1}{c|}{M} & \multicolumn{1}{c|}{\tikzcircle[fill=orange]{3pt}} & \multicolumn{1}{c|}{Once} & \multicolumn{1}{c|}{No} & \multicolumn{1}{c|}{\tikzcirclenew[fill=blue]{3pt}} & \multicolumn{1}{c|}{\tikzcircle[fill=orange]{3pt}} & \multicolumn{1}{c|}{\tikzcirclenew[fill=blue]{3pt}} & \multicolumn{1}{c|}{\tikzcircle[fill=orange]{3pt}} & \multicolumn{1}{c|}{\tikzcircle[fill=orange]{3pt}} & \multicolumn{1}{c|}{P55} & \multicolumn{1}{c|}{W} & \multicolumn{1}{c|}{\tikzcircle[fill=orange]{3pt}} & \multicolumn{1}{c|}{Once} & \multicolumn{1}{c|}{No} & \multicolumn{1}{c|}{\tikzcirclenew[fill=blue]{3pt}} & \multicolumn{1}{c|}{\tikzcircle[fill=orange]{3pt}} & \multicolumn{1}{c|}{\tikzcircle[fill=orange]{3pt}} & \multicolumn{1}{c|}{\tikzcircle[fill=orange]{3pt}} & \multicolumn{1}{c}{\tikzcircle[fill=orange]{3pt}} \\ \hline

\multicolumn{1}{c|}{P17} & \multicolumn{1}{c|}{W} & \multicolumn{1}{c|}{\tikzcircle[fill=orange]{3pt}} & \multicolumn{1}{c|}{Once} & \multicolumn{1}{c|}{No} & \multicolumn{1}{c|}{\tikzcircle[fill=orange]{3pt}} & \multicolumn{1}{c|}{\tikzcirclenew[fill=blue]{3pt}} & \multicolumn{1}{c|}{\tikzcircle[fill=orange]{3pt}} & \multicolumn{1}{c|}{\tikzcircle[fill=orange]{3pt}} & \multicolumn{1}{c|}{\tikzcircle[fill=orange]{3pt}} & \multicolumn{1}{c|}{P56} & \multicolumn{1}{c|}{M} & \multicolumn{1}{c|}{\tikzcircle[fill=orange]{3pt}} & \multicolumn{1}{c|}{Never} & \multicolumn{1}{c|}{No} & \multicolumn{1}{c|}{\tikzcircle[fill=orange]{3pt}} & \multicolumn{1}{c|}{\tikzcircle[fill=orange]{3pt}} & \multicolumn{1}{c|}{\tikzcirclenew[fill=blue]{3pt}} & \multicolumn{1}{c|}{\tikzcircle[fill=orange]{3pt}} & \multicolumn{1}{c}{\tikzcircle[fill=orange]{3pt}} \\ \hline

\multicolumn{1}{c|}{P18} & \multicolumn{1}{c|}{W} & \multicolumn{1}{c|}{\tikzcirclenew[fill=blue]{3pt}} & \multicolumn{1}{c|}{Once} & \multicolumn{1}{c|}{No} & \multicolumn{1}{c|}{\tikzcirclenew[fill=blue]{3pt}} & \multicolumn{1}{c|}{\tikzcirclenew[fill=blue]{3pt}} & \multicolumn{1}{c|}{\tikzcirclenew[fill=blue]{3pt}} & \multicolumn{1}{c|}{\tikzcircle[fill=orange]{3pt}} & \multicolumn{1}{c|}{\tikzcirclenew[fill=blue]{3pt}} & \multicolumn{1}{c|}{P57} & \multicolumn{1}{c|}{M} & \multicolumn{1}{c|}{\tikzcircle[fill=orange]{3pt}} & \multicolumn{1}{c|}{Few times} & \multicolumn{1}{c|}{No} & \multicolumn{1}{c|}{\tikzcircle[fill=orange]{3pt}} & \multicolumn{1}{c|}{\tikzcirclenew[fill=blue]{3pt}} & \multicolumn{1}{c|}{\tikzcircle[fill=orange]{3pt}} & \multicolumn{1}{c|}{\tikzcircle[fill=orange]{3pt}} & \multicolumn{1}{c}{\tikzcirclenew[fill=blue]{3pt}} \\ \hline

\multicolumn{1}{c|}{P19} & \multicolumn{1}{c|}{M} & \multicolumn{1}{c|}{\tikzcirclenew[fill=blue]{3pt}} & \multicolumn{1}{c|}{Never} & \multicolumn{1}{c|}{No} & \multicolumn{1}{c|}{\tikzcirclenew[fill=blue]{3pt}} & \multicolumn{1}{c|}{\tikzcirclenew[fill=blue]{3pt}} & \multicolumn{1}{c|}{\tikzcirclenew[fill=blue]{3pt}} & \multicolumn{1}{c|}{\tikzcircle[fill=orange]{3pt}} & \multicolumn{1}{c|}{\tikzcirclenew[fill=blue]{3pt}} & \multicolumn{1}{c|}{P58} & \multicolumn{1}{c|}{M} & \multicolumn{1}{c|}{\tikzcircle[fill=orange]{3pt}} & \multicolumn{1}{c|}{Once} & \multicolumn{1}{c|}{Some} & \multicolumn{1}{c|}{\tikzcircle[fill=orange]{3pt}} & \multicolumn{1}{c|}{\tikzcircle[fill=orange]{3pt}} & \multicolumn{1}{c|}{\tikzcirclenew[fill=blue]{3pt}} & \multicolumn{1}{c|}{\tikzcirclenew[fill=blue]{3pt}} & \multicolumn{1}{c}{\tikzcircle[fill=orange]{3pt}} \\ \hline

\multicolumn{1}{c|}{P20} & \multicolumn{1}{c|}{M} & \multicolumn{1}{c|}{\tikzcircle[fill=orange]{3pt}} & \multicolumn{1}{c|}{Few times} & \multicolumn{1}{c|}{No} & \multicolumn{1}{c|}{\tikzcircle[fill=orange]{3pt}} & \multicolumn{1}{c|}{\tikzcirclenew[fill=blue]{3pt}} & \multicolumn{1}{c|}{\tikzcircle[fill=orange]{3pt}} & \multicolumn{1}{c|}{\tikzcircle[fill=orange]{3pt}} & \multicolumn{1}{c|}{\tikzcircle[fill=orange]{3pt}} & \multicolumn{1}{c|}{P59} & \multicolumn{1}{c|}{M} & \multicolumn{1}{c|}{\tikzcirclenew[fill=blue]{3pt}} & \multicolumn{1}{c|}{Never} & \multicolumn{1}{c|}{No} & \multicolumn{1}{c|}{\tikzcirclenew[fill=blue]{3pt}} & \multicolumn{1}{c|}{\tikzcirclenew[fill=blue]{3pt}} & \multicolumn{1}{c|}{\tikzcirclenew[fill=blue]{3pt}} & \multicolumn{1}{c|}{\tikzcirclenew[fill=blue]{3pt}} & \multicolumn{1}{c}{\tikzcirclenew[fill=blue]{3pt}} \\ \hline

\multicolumn{1}{c|}{P21} & \multicolumn{1}{c|}{M} & \multicolumn{1}{c|}{\tikzcirclenew[fill=blue]{3pt}} & \multicolumn{1}{c|}{Often} & \multicolumn{1}{c|}{No} & \multicolumn{1}{c|}{\tikzcirclenew[fill=blue]{3pt}} & \multicolumn{1}{c|}{\tikzcirclenew[fill=blue]{3pt}} & \multicolumn{1}{c|}{\tikzcircle[fill=orange]{3pt}} & \multicolumn{1}{c|}{\tikzcircle[fill=orange]{3pt}} & \multicolumn{1}{c|}{\tikzcirclenew[fill=blue]{3pt}} & \multicolumn{1}{c|}{P60} & \multicolumn{1}{c|}{M} & \multicolumn{1}{c|}{\tikzcirclenew[fill=blue]{3pt}} & \multicolumn{1}{c|}{Once} & \multicolumn{1}{c|}{No} & \multicolumn{1}{c|}{\tikzcirclenew[fill=blue]{3pt}} & \multicolumn{1}{c|}{\tikzcircle[fill=orange]{3pt}} & \multicolumn{1}{c|}{\tikzcirclenew[fill=blue]{3pt}} & \multicolumn{1}{c|}{\tikzcirclenew[fill=blue]{3pt}} & \multicolumn{1}{c}{\tikzcirclenew[fill=blue]{3pt}} \\ \hline

\multicolumn{1}{c|}{P22} & \multicolumn{1}{c|}{W} & \multicolumn{1}{c|}{\tikzcirclenew[fill=blue]{3pt}} & \multicolumn{1}{c|}{Once} & \multicolumn{1}{c|}{No} & \multicolumn{1}{c|}{\tikzcirclenew[fill=blue]{3pt}} & \multicolumn{1}{c|}{\tikzcirclenew[fill=blue]{3pt}} & \multicolumn{1}{c|}{\tikzcirclenew[fill=blue]{3pt}} & \multicolumn{1}{c|}{\tikzcircle[fill=orange]{3pt}} & \multicolumn{1}{c|}{\tikzcircle[fill=orange]{3pt}} & \multicolumn{1}{c|}{P61} & \multicolumn{1}{c|}{M} & \multicolumn{1}{c|}{\tikzcircle[fill=orange]{3pt}} & \multicolumn{1}{c|}{Few times} & \multicolumn{1}{c|}{No} & \multicolumn{1}{c|}{\tikzcircle[fill=orange]{3pt}} & \multicolumn{1}{c|}{\tikzcircle[fill=orange]{3pt}} & \multicolumn{1}{c|}{\tikzcirclenew[fill=blue]{3pt}} & \multicolumn{1}{c|}{\tikzcirclenew[fill=blue]{3pt}} & \multicolumn{1}{c}{\tikzcircle[fill=orange]{3pt}} \\ \hline

\multicolumn{1}{c|}{P23} & \multicolumn{1}{c|}{W} & \multicolumn{1}{c|}{\tikzcirclenew[fill=blue]{3pt}} & \multicolumn{1}{c|}{Often} & \multicolumn{1}{c|}{No} & \multicolumn{1}{c|}{\tikzcirclenew[fill=blue]{3pt}} & \multicolumn{1}{c|}{\tikzcirclenew[fill=blue]{3pt}} & \multicolumn{1}{c|}{\tikzcirclenew[fill=blue]{3pt}} & \multicolumn{1}{c|}{\tikzcircle[fill=orange]{3pt}} & \multicolumn{1}{c|}{\tikzcirclenew[fill=blue]{3pt}} & \multicolumn{1}{c|}{P62} & \multicolumn{1}{c|}{M} & \multicolumn{1}{c|}{\tikzcircle[fill=orange]{3pt}} & \multicolumn{1}{c|}{Few times} & \multicolumn{1}{c|}{No} & \multicolumn{1}{c|}{\tikzcircle[fill=orange]{3pt}} & \multicolumn{1}{c|}{\tikzcirclenew[fill=blue]{3pt}} & \multicolumn{1}{c|}{\tikzcircle[fill=orange]{3pt}} & \multicolumn{1}{c|}{\tikzcircle[fill=orange]{3pt}} & \multicolumn{1}{c}{\tikzcircle[fill=orange]{3pt}} \\ \hline

\multicolumn{1}{c|}{P24} & \multicolumn{1}{c|}{M} & \multicolumn{1}{c|}{\tikzcirclenew[fill=blue]{3pt}} & \multicolumn{1}{c|}{Often} & \multicolumn{1}{c|}{No} & \multicolumn{1}{c|}{\tikzcirclenew[fill=blue]{3pt}} & \multicolumn{1}{c|}{\tikzcircle[fill=orange]{3pt}} & \multicolumn{1}{c|}{\tikzcirclenew[fill=blue]{3pt}} & \multicolumn{1}{c|}{\tikzcircle[fill=orange]{3pt}} & \multicolumn{1}{c|}{\tikzcirclenew[fill=blue]{3pt}} & \multicolumn{1}{c|}{P63} & \multicolumn{1}{c|}{W} & \multicolumn{1}{c|}{\tikzcirclenew[fill=blue]{3pt}} & \multicolumn{1}{c|}{Never} & \multicolumn{1}{c|}{No} & \multicolumn{1}{c|}{\tikzcircle[fill=orange]{3pt}} & \multicolumn{1}{c|}{\tikzcirclenew[fill=blue]{3pt}} & \multicolumn{1}{c|}{\tikzcirclenew[fill=blue]{3pt}} & \multicolumn{1}{c|}{\tikzcircle[fill=orange]{3pt}} & \multicolumn{1}{c}{\tikzcirclenew[fill=blue]{3pt}} \\ \hline

\multicolumn{1}{c|}{P25} & \multicolumn{1}{c|}{M} & \multicolumn{1}{c|}{\tikzcirclenew[fill=blue]{3pt}} & \multicolumn{1}{c|}{Often} & \multicolumn{1}{c|}{No} & \multicolumn{1}{c|}{\tikzcircle[fill=orange]{3pt}} & \multicolumn{1}{c|}{\tikzcirclenew[fill=blue]{3pt}} & \multicolumn{1}{c|}{\tikzcirclenew[fill=blue]{3pt}} & \multicolumn{1}{c|}{\tikzcirclenew[fill=blue]{3pt}} & \multicolumn{1}{c|}{\tikzcircle[fill=orange]{3pt}} & \multicolumn{1}{c|}{P64} & \multicolumn{1}{c|}{W} & \multicolumn{1}{c|}{\tikzcirclenew[fill=blue]{3pt}} & \multicolumn{1}{c|}{Never} & \multicolumn{1}{c|}{No} & \multicolumn{1}{c|}{\tikzcirclenew[fill=blue]{3pt}} & \multicolumn{1}{c|}{\tikzcirclenew[fill=blue]{3pt}} & \multicolumn{1}{c|}{\tikzcirclenew[fill=blue]{3pt}} & \multicolumn{1}{c|}{\tikzcircle[fill=orange]{3pt}} & \multicolumn{1}{c}{\tikzcirclenew[fill=blue]{3pt}} \\ \hline

\multicolumn{1}{c|}{P26} & \multicolumn{1}{c|}{W} & \multicolumn{1}{c|}{\tikzcircle[fill=orange]{3pt}} & \multicolumn{1}{c|}{Few times} & \multicolumn{1}{c|}{No} & \multicolumn{1}{c|}{\tikzcircle[fill=orange]{3pt}} & \multicolumn{1}{c|}{\tikzcircle[fill=orange]{3pt}} & \multicolumn{1}{c|}{\tikzcirclenew[fill=blue]{3pt}} & \multicolumn{1}{c|}{\tikzcircle[fill=orange]{3pt}} & \multicolumn{1}{c|}{\tikzcircle[fill=orange]{3pt}} & \multicolumn{1}{c|}{P65} & \multicolumn{1}{c|}{M} & \multicolumn{1}{c|}{\tikzcirclenew[fill=blue]{3pt}} & \multicolumn{1}{c|}{Never} & \multicolumn{1}{c|}{No} & \multicolumn{1}{c|}{\tikzcircle[fill=orange]{3pt}} & \multicolumn{1}{c|}{\tikzcirclenew[fill=blue]{3pt}} & \multicolumn{1}{c|}{\tikzcirclenew[fill=blue]{3pt}} & \multicolumn{1}{c|}{\tikzcircle[fill=orange]{3pt}} & \multicolumn{1}{c}{\tikzcirclenew[fill=blue]{3pt}} \\ \hline

\multicolumn{1}{c|}{P27} & \multicolumn{1}{c|}{M} & \multicolumn{1}{c|}{\tikzcircle[fill=orange]{3pt}} & \multicolumn{1}{c|}{Few times} & \multicolumn{1}{c|}{No} & \multicolumn{1}{c|}{\tikzcircle[fill=orange]{3pt}} & \multicolumn{1}{c|}{\tikzcirclenew[fill=blue]{3pt}} & \multicolumn{1}{c|}{\tikzcircle[fill=orange]{3pt}} & \multicolumn{1}{c|}{\tikzcircle[fill=orange]{3pt}} & \multicolumn{1}{c|}{\tikzcircle[fill=orange]{3pt}} & \multicolumn{1}{c|}{P66} & \multicolumn{1}{c|}{W} & \multicolumn{1}{c|}{\tikzcirclenew[fill=blue]{3pt}} & \multicolumn{1}{c|}{Never} & \multicolumn{1}{c|}{No} & \multicolumn{1}{c|}{\tikzcircle[fill=orange]{3pt}} & \multicolumn{1}{c|}{\tikzcirclenew[fill=blue]{3pt}} & \multicolumn{1}{c|}{\tikzcirclenew[fill=blue]{3pt}} & \multicolumn{1}{c|}{\tikzcircle[fill=orange]{3pt}} & \multicolumn{1}{c}{\tikzcirclenew[fill=blue]{3pt}} \\ \hline

\multicolumn{1}{c|}{P28} & \multicolumn{1}{c|}{M} & \multicolumn{1}{c|}{\tikzcircle[fill=orange]{3pt}} & \multicolumn{1}{c|}{Once} & \multicolumn{1}{c|}{No} & \multicolumn{1}{c|}{\tikzcircle[fill=orange]{3pt}} & \multicolumn{1}{c|}{\tikzcirclenew[fill=blue]{3pt}} & \multicolumn{1}{c|}{\tikzcirclenew[fill=blue]{3pt}} & \multicolumn{1}{c|}{\tikzcircle[fill=orange]{3pt}} & \multicolumn{1}{c|}{\tikzcircle[fill=orange]{3pt}} & \multicolumn{1}{c|}{P67} & \multicolumn{1}{c|}{M} & \multicolumn{1}{c|}{\tikzcircle[fill=orange]{3pt}} & \multicolumn{1}{c|}{Never} & \multicolumn{1}{c|}{No} & \multicolumn{1}{c|}{\tikzcircle[fill=orange]{3pt}} & \multicolumn{1}{c|}{\tikzcircle[fill=orange]{3pt}} & \multicolumn{1}{c|}{\tikzcirclenew[fill=blue]{3pt}} & \multicolumn{1}{c|}{\tikzcircle[fill=orange]{3pt}} & \multicolumn{1}{c}{\tikzcircle[fill=orange]{3pt}} \\ \hline

\multicolumn{1}{c|}{P29} & \multicolumn{1}{c|}{M} & \multicolumn{1}{c|}{\tikzcircle[fill=orange]{3pt}} & \multicolumn{1}{c|}{Never} & \multicolumn{1}{c|}{No} & \multicolumn{1}{c|}{\tikzcircle[fill=orange]{3pt}} & \multicolumn{1}{c|}{\tikzcircle[fill=orange]{3pt}} & \multicolumn{1}{c|}{\tikzcircle[fill=orange]{3pt}} & \multicolumn{1}{c|}{\tikzcircle[fill=orange]{3pt}} & \multicolumn{1}{c|}{\tikzcircle[fill=orange]{3pt}} & \multicolumn{1}{c|}{P68} & \multicolumn{1}{c|}{M} & \multicolumn{1}{c|}{\tikzcirclenew[fill=blue]{3pt}} & \multicolumn{1}{c|}{Few times} & \multicolumn{1}{c|}{No} & \multicolumn{1}{c|}{\tikzcirclenew[fill=blue]{3pt}} & \multicolumn{1}{c|}{\tikzcirclenew[fill=blue]{3pt}} & \multicolumn{1}{c|}{\tikzcirclenew[fill=blue]{3pt}} & \multicolumn{1}{c|}{\tikzcircle[fill=orange]{3pt}} & \multicolumn{1}{c}{\tikzcircle[fill=orange]{3pt}} \\ \hline

\multicolumn{1}{c|}{P30} & \multicolumn{1}{c|}{M} & \multicolumn{1}{c|}{\tikzcirclenew[fill=blue]{3pt}} & \multicolumn{1}{c|}{Few times} & \multicolumn{1}{c|}{No} & \multicolumn{1}{c|}{\tikzcirclenew[fill=blue]{3pt}} & \multicolumn{1}{c|}{\tikzcirclenew[fill=blue]{3pt}} & \multicolumn{1}{c|}{\tikzcirclenew[fill=blue]{3pt}} & \multicolumn{1}{c|}{\tikzcircle[fill=orange]{3pt}} & \multicolumn{1}{c|}{\tikzcirclenew[fill=blue]{3pt}} & \multicolumn{1}{c|}{P69} & \multicolumn{1}{c|}{M} & \multicolumn{1}{c|}{\tikzcirclenew[fill=blue]{3pt}} & \multicolumn{1}{c|}{Few times} & \multicolumn{1}{c|}{No} & \multicolumn{1}{c|}{\tikzcirclenew[fill=blue]{3pt}} & \multicolumn{1}{c|}{\tikzcirclenew[fill=blue]{3pt}} & \multicolumn{1}{c|}{\tikzcirclenew[fill=blue]{3pt}} & \multicolumn{1}{c|}{\tikzcirclenew[fill=blue]{3pt}} & \multicolumn{1}{c}{\tikzcircle[fill=orange]{3pt}} \\ \hline

\multicolumn{1}{c|}{P31} & \multicolumn{1}{c|}{M} & \multicolumn{1}{c|}{\tikzcirclenew[fill=blue]{3pt}} & \multicolumn{1}{c|}{Never} & \multicolumn{1}{c|}{No} & \multicolumn{1}{c|}{\tikzcircle[fill=orange]{3pt}} & \multicolumn{1}{c|}{\tikzcirclenew[fill=blue]{3pt}} & \multicolumn{1}{c|}{\tikzcirclenew[fill=blue]{3pt}} & \multicolumn{1}{c|}{\tikzcirclenew[fill=blue]{3pt}} & \multicolumn{1}{c|}{\tikzcirclenew[fill=blue]{3pt}} & \multicolumn{1}{c|}{P70} & \multicolumn{1}{c|}{M} & \multicolumn{1}{c|}{\tikzcirclenew[fill=blue]{3pt}} & \multicolumn{1}{c|}{Few times} & \multicolumn{1}{c|}{Some} & \multicolumn{1}{c|}{\tikzcirclenew[fill=blue]{3pt}} & \multicolumn{1}{c|}{\tikzcirclenew[fill=blue]{3pt}} & \multicolumn{1}{c|}{\tikzcirclenew[fill=blue]{3pt}} & \multicolumn{1}{c|}{\tikzcircle[fill=orange]{3pt}} & \multicolumn{1}{c}{\tikzcirclenew[fill=blue]{3pt}} \\ \hline

\multicolumn{1}{c|}{P32} & \multicolumn{1}{c|}{M} & \multicolumn{1}{c|}{\tikzcircle[fill=orange]{3pt}} & \multicolumn{1}{c|}{Never} & \multicolumn{1}{c|}{No} & \multicolumn{1}{c|}{\tikzcirclenew[fill=blue]{3pt}} & \multicolumn{1}{c|}{\tikzcirclenew[fill=blue]{3pt}} & \multicolumn{1}{c|}{\tikzcircle[fill=orange]{3pt}} & \multicolumn{1}{c|}{\tikzcircle[fill=orange]{3pt}} & \multicolumn{1}{c|}{\tikzcircle[fill=orange]{3pt}} & \multicolumn{1}{c|}{P71} & \multicolumn{1}{c|}{M} & \multicolumn{1}{c|}{\tikzcircle[fill=orange]{3pt}} & \multicolumn{1}{c|}{Few times} & \multicolumn{1}{c|}{No} & \multicolumn{1}{c|}{\tikzcircle[fill=orange]{3pt}} & \multicolumn{1}{c|}{\tikzcircle[fill=orange]{3pt}} & \multicolumn{1}{c|}{\tikzcircle[fill=orange]{3pt}} & \multicolumn{1}{c|}{\tikzcircle[fill=orange]{3pt}} & \multicolumn{1}{c}{\tikzcirclenew[fill=blue]{3pt}} \\ \hline

\multicolumn{1}{c|}{P33} & \multicolumn{1}{c|}{M} & \multicolumn{1}{c|}{\tikzcirclenew[fill=blue]{3pt}} & \multicolumn{1}{c|}{Few times} & \multicolumn{1}{c|}{No} & \multicolumn{1}{c|}{\tikzcircle[fill=orange]{3pt}} & \multicolumn{1}{c|}{\tikzcirclenew[fill=blue]{3pt}} & \multicolumn{1}{c|}{\tikzcirclenew[fill=blue]{3pt}} & \multicolumn{1}{c|}{\tikzcircle[fill=orange]{3pt}} & \multicolumn{1}{c|}{\tikzcirclenew[fill=blue]{3pt}} & \multicolumn{1}{c|}{P72} & \multicolumn{1}{c|}{M} & \multicolumn{1}{c|}{\tikzcirclenew[fill=blue]{3pt}} & \multicolumn{1}{c|}{Few times} & \multicolumn{1}{c|}{No} & \multicolumn{1}{c|}{\tikzcirclenew[fill=blue]{3pt}} & \multicolumn{1}{c|}{\tikzcircle[fill=orange]{3pt}} & \multicolumn{1}{c|}{\tikzcirclenew[fill=blue]{3pt}} & \multicolumn{1}{c|}{\tikzcircle[fill=orange]{3pt}} & \multicolumn{1}{c}{\tikzcirclenew[fill=blue]{3pt}} \\ \hline

\multicolumn{1}{c|}{P34} & \multicolumn{1}{c|}{M} & \multicolumn{1}{c|}{\tikzcirclenew[fill=blue]{3pt}} & \multicolumn{1}{c|}{Few times} & \multicolumn{1}{c|}{No} & \multicolumn{1}{c|}{\tikzcirclenew[fill=blue]{3pt}} & \multicolumn{1}{c|}{\tikzcircle[fill=orange]{3pt}} & \multicolumn{1}{c|}{\tikzcirclenew[fill=blue]{3pt}} & \multicolumn{1}{c|}{\tikzcircle[fill=orange]{3pt}} & \multicolumn{1}{c|}{\tikzcirclenew[fill=blue]{3pt}} & \multicolumn{1}{c|}{P73} & \multicolumn{1}{c|}{M} & \multicolumn{1}{c|}{\tikzcirclenew[fill=blue]{3pt}} & \multicolumn{1}{c|}{Once} & \multicolumn{1}{c|}{No} & \multicolumn{1}{c|}{\tikzcirclenew[fill=blue]{3pt}} & \multicolumn{1}{c|}{\tikzcirclenew[fill=blue]{3pt}} & \multicolumn{1}{c|}{\tikzcirclenew[fill=blue]{3pt}} & \multicolumn{1}{c|}{\tikzcircle[fill=orange]{3pt}} & \multicolumn{1}{c}{\tikzcirclenew[fill=blue]{3pt}} \\ \hline

\multicolumn{1}{c|}{P35} & \multicolumn{1}{c|}{W} & \multicolumn{1}{c|}{\tikzcirclenew[fill=blue]{3pt}} & \multicolumn{1}{c|}{Few times} & \multicolumn{1}{c|}{No} & \multicolumn{1}{c|}{\tikzcirclenew[fill=blue]{3pt}} & \multicolumn{1}{c|}{\tikzcirclenew[fill=blue]{3pt}} & \multicolumn{1}{c|}{\tikzcircle[fill=orange]{3pt}} & \multicolumn{1}{c|}{\tikzcircle[fill=orange]{3pt}} & \multicolumn{1}{c|}{\tikzcirclenew[fill=blue]{3pt}} & \multicolumn{1}{c|}{P74} & \multicolumn{1}{c|}{M} & \multicolumn{1}{c|}{\tikzcircle[fill=orange]{3pt}} & \multicolumn{1}{c|}{Never} & \multicolumn{1}{c|}{No} & \multicolumn{1}{c|}{\tikzcircle[fill=orange]{3pt}} & \multicolumn{1}{c|}{\tikzcirclenew[fill=blue]{3pt}} & \multicolumn{1}{c|}{\tikzcirclenew[fill=blue]{3pt}} & \multicolumn{1}{c|}{\tikzcircle[fill=orange]{3pt}} & \multicolumn{1}{c}{\tikzcircle[fill=orange]{3pt}} \\ \hline

\multicolumn{1}{c|}{P36} & \multicolumn{1}{c|}{M} & \multicolumn{1}{c|}{\tikzcirclenew[fill=blue]{3pt}} & \multicolumn{1}{c|}{Never} & \multicolumn{1}{c|}{No} & \multicolumn{1}{c|}{\tikzcirclenew[fill=blue]{3pt}} & \multicolumn{1}{c|}{\tikzcirclenew[fill=blue]{3pt}} & \multicolumn{1}{c|}{\tikzcirclenew[fill=blue]{3pt}} & \multicolumn{1}{c|}{\tikzcircle[fill=orange]{3pt}} & \multicolumn{1}{c|}{\tikzcirclenew[fill=blue]{3pt}} & \multicolumn{1}{c|}{P75} & \multicolumn{1}{c|}{M} & \multicolumn{1}{c|}{\tikzcirclenew[fill=blue]{3pt}} & \multicolumn{1}{c|}{Few times} & \multicolumn{1}{c|}{No} & \multicolumn{1}{c|}{\tikzcirclenew[fill=blue]{3pt}} & \multicolumn{1}{c|}{\tikzcirclenew[fill=blue]{3pt}} & \multicolumn{1}{c|}{\tikzcirclenew[fill=blue]{3pt}} & \multicolumn{1}{c|}{\tikzcircle[fill=orange]{3pt}} & \multicolumn{1}{c}{\tikzcirclenew[fill=blue]{3pt}} \\ \hline

\multicolumn{1}{c|}{P37} & \multicolumn{1}{c|}{M} & \multicolumn{1}{c|}{\tikzcirclenew[fill=blue]{3pt}} & \multicolumn{1}{c|}{Few times} & \multicolumn{1}{c|}{Some} & \multicolumn{1}{c|}{\tikzcirclenew[fill=blue]{3pt}} & \multicolumn{1}{c|}{\tikzcircle[fill=orange]{3pt}} & \multicolumn{1}{c|}{\tikzcirclenew[fill=blue]{3pt}} & \multicolumn{1}{c|}{\tikzcircle[fill=orange]{3pt}} & \multicolumn{1}{c|}{\tikzcirclenew[fill=blue]{3pt}} &

\multicolumn{10}{c}{\multirow{3}{*}{\begin{tabular}[c]{@{}c@{}}\textbf{Legend:} M: Man | W: Woman | \tikzcirclenew[fill=blue]{3pt}: Tim | \tikzcircle[fill=orange]{3pt}: Abi\\ MT: Motivation | SE: Self-efficacy | R: Risk \\ IP: Information processing | L: Learning\end{tabular}}} \\ \cline{1-10}

\multicolumn{1}{c|}{P38} & \multicolumn{1}{c|}{M} & \multicolumn{1}{c|}{\tikzcirclenew[fill=blue]{3pt}} & \multicolumn{1}{c|}{Never} & \multicolumn{1}{c|}{No} & \multicolumn{1}{c|}{\tikzcirclenew[fill=blue]{3pt}} & \multicolumn{1}{c|}{\tikzcirclenew[fill=blue]{3pt}} & \multicolumn{1}{c|}{\tikzcirclenew[fill=blue]{3pt}} & \multicolumn{1}{c|}{\tikzcircle[fill=orange]{3pt}} & \multicolumn{1}{c|}{\tikzcircle[fill=orange]{3pt}} & \multicolumn{10}{l}{} \\ \cline{1-10}

\multicolumn{1}{c|}{P39} & \multicolumn{1}{c|}{M} & \multicolumn{1}{c|}{\tikzcirclenew[fill=blue]{3pt}} & \multicolumn{1}{c|}{Few times} & \multicolumn{1}{c|}{Some} & \multicolumn{1}{c|}{\tikzcirclenew[fill=blue]{3pt}} & \multicolumn{1}{c|}{\tikzcirclenew[fill=blue]{3pt}} & \multicolumn{1}{c|}{\tikzcirclenew[fill=blue]{3pt}} & \multicolumn{1}{c|}{\tikzcircle[fill=orange]{3pt}} & \multicolumn{1}{c|}{\tikzcirclenew[fill=blue]{3pt}} & \multicolumn{10}{l}{} \\ \hline
\end{tabular}
\end{table*}



\begin{comment}


The five facets used by the GenderMag method are presented in Table~\ref{tab:gendermagfactes}. The facets are used to define personas (e.g., Abi and Tim). GenderMag highlights that differences relevant to inclusiveness lie not in a person's gender identity but in the facet values themselves~\cite{hill2017gender}. Nevertheless, Abi's facet values are more frequent in women than in other genders, and Tim's facet values are more frequent in men than in other genders. 

%(Figure~\ref{fig:abbypersona})

\begin{table}[!ht]\scriptsize
\centering
\vspace{-2.5mm}
\caption{GenderMag facets~\cite{burnett2016gendermag}}
\label{tab:gendermagfactes}
\newcommand{\pb}[1]{\parbox[t][][t]{1.0\linewidth}{#1} \vspace{-2pt}}

\begin{tabular}{p{12mm}|p{62mm}}
\hline
\multicolumn{1}{>{\centering\arraybackslash}m{12mm}|}{\textbf{GenderMag Facets}} & \multicolumn{1}{>{\centering\arraybackslash}m{62mm}}{\textbf{Definition}} \\ \hline \hline

Motivation & \pb{Women tend (statistically) to be motivated to use technology for what they can accomplish with it, whereas men are often motivated by their enjoyment of technology per se~\cite{simon2000impact, cassell2002hand, margolis2002unlocking, hou2006girls, kelleher2009barriers, burnett2010gender, burnett2011gender, hallstrom2015gender}. This difference can affect which software features users choose to use}. \\ \hline 

Information processing styles & \pb{To solve problems, people often need to process new information. Women are more likely (statistically) to process new information comprehensively—gathering fairly complete information before proceeding—but men are more likely to use selective styles—following the first promising information, then backtracking if needed~\cite{cafferata1989gender, meyers1991exploring, coursaris2008empirical, riedl2010there, meyers2015revisiting}. Each style has advantages, but either is at a disadvantage when not supported by the software.} \\ \hline

Computer self-efficacy & \pb{Self-efficacy is a person's confidence about succeeding at a specific task, which influences their use of cognitive strategies, persistence, and strategies for coping with obstacles. Empirical data have shown that women often have lower computer self-efficacy than men, which can affect their behavior with technology~\cite{margolis2002unlocking, durndell2002computer, hartzel2003self, beckwith2005effectiveness, beckwith2006tinkering, burnett2010gender, burnett2011gender, singh2013role, huffman2013using}.} \\ \hline

Risk aversion & \pb{Research shows that women statistically tend to be more risk-averse than men~\cite{weber2002domain, dohmen2011individual, charness2012strong}. These results span numerous decision-making domains, such as ethics, investment, gambling, health/safety, and career. Risk aversion with software usage can impact users' decisions as to which feature sets to use.} \\ \hline

Learning: by Process vs. by Tinkering & \pb{Research across age groups and professions reports women being statistically less likely to playfully experiment (“tinker”) with software features new to them, compared to men~\cite{beckwith2006tinkering, hou2006girls, rosner2009learning, burnett2010gender, cao2010debugging, chang2014specialization}. However, when women do tinker, they tend to be more likely to reflect during the process and thereby sometimes profit from it more than men do.} \\ \hline \hline
\end{tabular}
\end{table}

\end{comment}
&&&&

\begin{table}[htbp!]
\resizebox{\columnwidth}{!}{%
\begin{tabular}{@{}lccccc@{}}
\toprule
concept & \begin{tabular}[c]{@{}c@{}}document\\count\end{tabular} & \begin{tabular}[c]{@{}c@{}}annotation\\count\end{tabular} & \begin{tabular}[c]{@{}c@{}}metaphorical\\annotations\end{tabular}  & \begin{tabular}[c]{@{}c@{}}mean\\score\end{tabular}  \\ \midrule
all & 1600 & 12676 & 4421  & 0.347  \\
animal & 200 & 1898 & 567 &  0.311 \\
parasite & 200 & 1637 & 583  & 0.347  \\
vermin & 200 & 1393 & 348 & 0.246  \\
water & 200 & 1535 & 650 &  0.425  \\
war & 200 & 1475 & 520 & 0.350  \\
commodity & 200 & 1646 & 675 &  0.406  \\
pressure & 200 & 1574 & 610  & 0.385  \\
\begin{tabular}[c]{@{}l@{}}domain-\\agnostic\end{tabular} & 200 & 1518 & 468  & 0.307 \\ \bottomrule
\end{tabular} }
\caption{Descriptive statistics for annotated dataset. \textit{Mean score} refers to the average document score per concept, i.e., the proportion of annotators who labeled a document as metaphorical with respect to the concept.}
\label{tab:annotation-stats}
\end{table}


% Descriptive stats for annotation:

% Count: 617 unique annotators
% Ideology: 327 Liberal, 192 Moderate, 98 Conservative
% Age: range from 18-84, mean 38.3, std = 12.7
% Sex: 382 Female, 233 Male, 2 prefer not to say
% Immigration: 45 Immigrants, 572 non-immigrants
% Age+Sex: 45 conservative female, 114 moderate female, 223 liberal female
%         53 conservative male, 78 moderate male, 102 liberal male
%         2 liberal prefer to say

%      concept  documents  annotations
% 0     animal        200         1898
% 1   parasite        200         1637
% 2     vermin        200         1393
% 3      water        200         1535
% 4        war        200         1475
% 5  commodity        200         1646
% 6   pressure        200         1574
% 7    overall        200         1518

%           concept        ideology  num_metaphorical  num_annotations  percent_metaphorical
% 0   all_documents  all_annotators              4421            12676              0.348769
% 4          animal  all_annotators               567             1898              0.298736
% 5        parasite  all_annotators               583             1637              0.356139
% 6          vermin  all_annotators               348             1393              0.249821
% 7           water  all_annotators               650             1535              0.423453
% 8             war  all_annotators               520             1475              0.352542
% 9       commodity  all_annotators               675             1646              0.410085
% 10       pressure  all_annotators               610             1574              0.387548
% 11        overall  all_annotators               468             1518              0.308300

%           concept        ideology     alpha
% 0   all_documents  all_annotators  0.316248
% 1          animal  all_annotators  0.380161
% 2        parasite  all_annotators  0.511692
% 3          vermin  all_annotators  0.213464
% 4           water  all_annotators  0.540205
% 5             war  all_annotators  0.231026
% 6       commodity  all_annotators  0.199978
% 7        pressure  all_annotators  0.171265


% 1600 total documents annotated. Mean 7.92; median 8 annotations per tweet. 

% Since these are binary judgments and we wanted continuous values for metaphoricity, we evaluate models with respect to the percent of annotators who say that a tweet invokes metaphor related to the specified concept. 
% mean score for all documents: 0.347, median: 0.286

%      concept  mean_percent_yes
% 0     animal     0.311310
% 1  commodity     0.406294
% 2    overall     0.307046
% 3   parasite     0.347351
% 4   pressure     0.384889
% 5     vermin     0.245518
% 6        war     0.349619
% 7      water     0.425417

%      concept  median_percent_yes
% 0     animal     0.200000
% 1  commodity     0.375000
% 2    overall     0.250000
% 3   parasite     0.200000
% 4   pressure     0.333333
% 5     vermin     0.166667
% 6        war     0.285714
% 7      water     0.354167


% \begin{figure}[htbp!]
%     \centering
%     \includegraphics[width=\columnwidth]{plots_dec2024/data/agreement.pdf}
%     \caption{Inter-annotator agreement (Krippendorff's $\alpha$) for each source domain concept}
%     \label{fig:agreement}
% \end{figure}

\begin{figure}[htbp!]
    \centering
    \includegraphics[width=\columnwidth]{plots_feb2025/descriptive/agreement.pdf}
    \caption{Inter-annotator agreement (Krippendorff's $\alpha$) for each concept and domain-agnostic metaphor.}
    \label{fig:agreement}
\end{figure}

\begin{table*}[htbp!]
\begin{tcolorbox}[colback=white, colframe=white!30!black, boxrule=1pt, arc=4mm, width=\textwidth, boxrule=1pt,title=Codebook,fontupper=\footnotesize]

For each tweet and given concept, label whether or not the tweet evokes metaphors related to the given concept. Focus on the author's language, not their stance towards immigration.\\

To determine if specific words or phrases are metaphorical, consider whether the most basic meaning is related to the listed source domain concept. Basic meanings tend to be more concrete (easier to understand, imagine, or sense) and precise (rather than vague). If you’re not sure about a word’s basic meaning, use the first definition in the dictionary as a proxy. A word should be considered metaphorical if it’s relevant to the listed concept (e.g. animal); it need not exclusively apply to that concept. \\

\textbf{Animal}: Immigrants are sometimes talked about as though they are animals such as beasts, cattle, sheep, and dogs. Label whether or not each tweet makes an association between immigration/immigrants and animals. Label ``YES'' if:
\begin{itemize}[noitemsep, topsep=0pt]
    \item The author uses any words or phrases that are usually used to describe animals. Common examples: \textit{attack}, \textit{flock}, \textit{hunt}, \textit{trap}, \textit{cage}, \textit{breed}
    \item Even if you cannot pinpoint specific words that evoke the concept of animals, if the author's language reminds you of how people talk about animals
\end{itemize}

\textbf{Vermin}: Vermin are small animals that spread diseases and destroy crops, livestock, or property, such as rats, mice, and cockroaches. Vermin are often found in large groups. Label whether or not each tweet makes an association between immigration/immigrants and vermin. Label ``YES'' if:
\begin{itemize}[noitemsep, topsep=0pt]
    \item The author uses any words or phrases that are usually used to describe vermin. Common examples: \textit{infesting}, \textit{swarming}, \textit{dirty}, \textit{diseased}, \textit{overrun}, \textit{plagued}, \textit{virus}
    \item Even if you cannot pinpoint specific words that evoke the concept of vermin, if the author's language reminds you of how people talk about vermin
\end{itemize}


\textbf{Parasite}: Parasites are organisms that feed off a host species at the host’s expense, such as leeches, ticks, fleas, and mosquitoes. Label whether or not each tweet makes an association between immigration/immigrants and parasites. Label “YES” if: 
\begin{itemize}[noitemsep, topsep=0pt]
    \item The author uses any words or phrases that are usually used to describe parasites. Common examples: \textit{leeching}, \textit{freeloading}, \textit{sponging}, \textit{mooching}, \textit{bleed dry}
    \item Even if you cannot pinpoint specific words that evoke the concept of parasites, if the author's language reminds you of how people talk about parasites
\end{itemize}


\textbf{Water}: Immigrants are sometimes talked about using language commonly reserved for water (or liquid motion more broadly). For example, people may talk about immigrants pouring, flooding, or streaming across borders, or refer to waves, tides, and influxes of immigration. Label whether or not each tweet makes an association between immigration/immigrants and water. Label “YES” if: 

\begin{itemize}[noitemsep, topsep=0pt]
    \item The author uses any words or phrases that are usually used to describe water. Common examples: \textit{pouring}, \textit{flooding}, \textit{flowing}, \textit{drain}, \textit{spillover}, \textit{surge}, \textit{wave}
    \item Even if you cannot pinpoint specific words that evoke the concept of water, if the author's language reminds you of how people talk about water
\end{itemize}


\textbf{Commodity}: Commodities are economic resources or objects that are traded, exchanged, bought, and sold. Label whether or not each tweet makes an association between immigration/immigrants. Label ``YES'' if:
\begin{itemize}[noitemsep, topsep=0pt]
    \item The author uses any words or phrases that are usually used to describe commodities. Common examples: \textit{sources of labor}, \textit{undergoing processing}, \textit{imports}, \textit{exports}, \textit{tools}, \textit{being received or taken in}, \textit{distribution}
    \item Even if you cannot pinpoint specific words that evoke the concept of commodities, if the author's language reminds you of how people talk about commodities
\end{itemize}


\textbf{Pressure}: Immigration is sometimes talked about as a physical pressure placed upon a host country, especially as heavy burdens, crushing forces, or bursting containers. Label whether or not each tweet makes an association between immigration/immigrants and physical pressure. Label “YES” if: 
\begin{itemize}[noitemsep, topsep=0pt]
    \item The author uses any words or phrases that are usually used to describe physical pressure. Common examples: host country \textit{crumbling}, \textit{bursting}, being \textit{crushed}, \textit{stretched thin}, or \textit{strained}, immigrants as \textit{burdens}.
    \item Even if you cannot pinpoint specific words that evoke the concept of pressure, does the author’s language remind you of how people talk about physical pressure? 
\end{itemize}

\textbf{Commodity}: Commodities are economic resources or objects that are traded, exchanged, bought, and sold. Label whether or not each tweet makes an association between immigration/immigrants. Label ``YES'' if:
\begin{itemize}[noitemsep, topsep=0pt]
    \item The author uses any words or phrases that are usually used to describe commodities. Common examples: \textit{sources of labor}, \textit{undergoing processing}, \textit{imports}, \textit{exports}, \textit{tools}, \textit{being received or taken in}, \textit{distribution}
    \item Even if you cannot pinpoint specific words that evoke the concept of commodities, if the author's language reminds you of how people talk about commodities
\end{itemize}

\textbf{War}: People sometimes talk about immigration in terms of war, where immigrants are viewed as an invading army that the host country fights against. Label whether or not each tweet makes an association between immigration/immigrants and war. Label “YES” if:
\begin{itemize}[noitemsep, topsep=0pt]
    \item The author uses any words or phrases that are usually used to describe war. Common examples: \textit{invasion}, \textit{soldiers}, \textit{battle}, \textit{shields}, \textit{fighting}
    \item Even if you cannot pinpoint specific words that evoke the concept of war, if the author's language reminds you of how people talk about war
\end{itemize}

\textbf{Domain-Agnostic}: Label whether or not each tweet uses metaphorical (non-literal) language to talk about immigration/immigrants. Metaphorical language involves talking about immigration/immigrants in terms of an otherwise unrelated concept. For example, \textit{waves of immigration} is metaphorical because the word \textit{waves} is associated with water.

\end{tcolorbox}
\end{table*}







\subsubsection{Sampling for Annotation}
\label{sampling}

The prevalence of metaphorical language with respect to each source domain concept is not known a-priori. Instead of randomly sampling tweets for human annotation, we thus adopt a stratified sampling approach using scores from a baseline model (\texttt{GPT-4-Turbo, Simple Prompt}) as a heuristic. 

We get heuristic scores from the baseline model for a set of 20K documents, which we call $D$. For each concept $c$, we sample $n_c$ documents from $D$. Let $h_c$ be the heuristic metaphor score with respect to $c$. $Q_{0,c} \in D$ is then the set of documents with $h_c = 0$. $Q_{i,c} \in D$ where $i = 1,2,...,k-1$ are the $k-1$ quantiles of documents with $h_c > 0$. The annotation sample for each concept is then: $$S_c = Sample(Q_{0,c},\frac{n_c}{k}) \cup  \bigcup_{i=1}^{k-1}Sample(Q_{i,c},\frac{n_c}{k})$$ Using $k=5$ strata, we sample $n_c=200$ tweets for each source domain. We additionally sample 200 documents for \textit{domain-agnostic} metaphor.

Below are examples of tweets from different strata for the \textsc{water} concept:
\begin{itemize}[noitemsep,topsep=0pt]
    \item $Q_{0,\textsc{water}}$: \textit{How about we help US citizens with cancer before spending money on illegals} 
    \item $Q_{2,\textsc{water}}$: \textit{Tough reading. A report on some of the facilities to which the migrant children are shipped after the American government abducts them from their parents. This despicable practice is a permanent stain on the US}
    \item $Q_{4,\textsc{water}}$: \textit{An overwhelming flood of illegal aliens for the middle class to pay for.}

\end{itemize}

\section{Steering details: prompts, datasets, and parameters}
\label{app: prompts}

We now describe the parameters and prompts used for steering Llama-3.1-8B-it and Gemma-2-9B-it toward different concepts.

\subsection{Our prompting method}

We consider a specific example to explain our prompting method, where we extract directions to induce different identities from the surname `Newton'. To extract semantically meaningful directions from the activation spaces of LLMs for steering, we first choose a list of labeled prompts for a list of desired concepts, similar to the approaches of \citet{representation_engineering, turner2023activation}. However, unlike their methods, our prompts do not need to consist of contrastive pairs of positive and negative examples. Further, we found benefit in some cases by choosing prompts to be from real text, and not synthetic datasets. For example, we extracted meaningful concepts corresponding to political positions and disambiguating word meanings from pairs of Wikipedia articles. 

Consider the specific case of distinguishing Cam Newton versus Isaac Newton (Figure~\ref{fig: rfm/pca newton, llama-3.1-8B}). We obtain sentences from the Isaac and Cam Newton wikipedia articles. 
Suppose we want to learn the vector for `Isaac' Newton. Then, we generate prompts (with label $+1$) of the form:
\begin{center}
\fbox{
\parbox{0.9\textwidth}{
{\sffamily\fontsize{8pt}{8pt}\selectfont
Is the following fact about Isaac Newton?\\
Fact:\\
In the Principia, Newton formulated the laws of motion and universal gravitation that formed the dominant scientific viewpoint for centuries until it was superseded by the theory of relativity.}
}
}
\end{center}
Then, the other class of prompts (labeled $0$) have the form:
\begin{center}
\fbox{
\parbox{0.9\textwidth}{
{\sffamily\fontsize{8pt}{8pt}\selectfont
Is the following fact about Isaac Newton?\\
Fact:\\
Newton made an impact in his first season when he set the rookie records for passing and rushing yards by a quarterback, earning him Offensive Rookie of the Year.}
}
}
\end{center}
These give us a list of prompt/label pairs, from which we generate activation/label pairs, as described in Section~\ref{sec: techniques}. We then solve RFM (or another layer-wise predictor) on each layer to predict the label function (Isaac vs. Cam Newton). For RFM, the concept vectors at each layer $c_\ell$ are then the top eigenvectors of the AGOP from each RFM predictor.

\subsection{Human Languages} For triggering language switches as in Figures~\ref{fig: english_chinese, llama-3.1-8B} and \ref{fig: english_spanish, llama-3.1-8B}, we used examples generated from the following prompt template.

\begin{center}
\fbox{\parbox{0.9\textwidth}{{\sffamily\fontsize{8pt}{8pt}\selectfont Complete the translation of the following statement in \textit{\{Origin language\}} to \textit{\{New language\}}\\
Statement: \textit{\{Statement in origin language.\}}\\ Translation: \textit{\{Partial translation in new language.\}} }
}
}
\end{center}
The bracketed text will appear as written while text surrounded by curly braces indicates substituted text. We obtained list of statements in the origin and new languages from datasets of translated statements. To generate the partial translations we truncated translations to the first half of the tokens. For Spanish/English translations we used datasets from \url{https://github.com/jatinmandav/Neural-Machine-Translation/tree/master}. For Mandarin/English, we obtained pairs of statements from \url{https://huggingface.co/datasets/swaption2009/20k-en-zh-translation-pinyin-hsk}. 

To evaluate translations for human language, we use OpenAI's o1 reasoning model to generate a list of 100 English statements and questions. We then evaluate the translation from steering of each sentence.

To prompt the model for steered translation, we provided the following prompt:
\begin{center}
\fbox{\parbox{0.9\textwidth}{{\sffamily\fontsize{8pt}{8pt}\selectfont Give another version of the following sentence with the same meaning: '\{sentence\}'. Write the version in quotes.}
}
}
\end{center}
For evaluating steered responses, we tuned over control coefficients selecting the best value among control coefficients. We steer the outputs of all blocks of the transformer. For Llama-3.1-8B-it, we tuned over coefficients: 0.2, 0.3, 0.4, 0.5, 0.6, 0.7, and for Gemma-2-9B-it: 4.0, 5.0, 6.0, 7.0, 8.0, 9.0. Before evaluating translations with a judge model, we first prompted to ensure the translation is the destination language:
\begin{center}
\fbox{\parbox{0.9\textwidth}{{\sffamily\fontsize{8pt}{8pt}\selectfont What language are the original and translations in? Give your answer in the format '\{Original language\}, \{Translation language\}'. If the translation is empty, assume Language 2 is English."}
}
}
\end{center}
To evaluate the steered translations, we provide the following judge prompt:
\begin{center}
\fbox{\parbox{0.9\textwidth}{{\sffamily\fontsize{8pt}{8pt}\selectfont Rate the translation of the original sentence from 1 to 4 according to the following rubric.\\\\
Score of 1: The translation is mostly inaccurate or unrelated.\\
Score of 2: The translation is slightly unrelated to the original.\\
Score of 3: The translation has mostly the same meaning as the original.\\
Score of 4: The translation has the same meaning as the original.\\\\
Give your response in the format '{score}/4.' Do not penalize awkward or excessive wording. If the translation is empty, give a score of 0.\\
----------------------------------------\\
ORIGINAL: \{original\}\\
----------------------------------------\\
TRANSLATION: \{translation\}"}
}
}
\end{center}

\subsection{Poetry} Prompts for poetry followed the same format as human languages. We obtained 100 pairs of standard English sentences and poetic translations from OpenAI's o1 model. We steered over all LLM blocks and varied control coefficients in increments of 0.1 over 0.4 to 0.8. Figure~\ref{fig: steered poetry style} uses coefficient 0.6. We combine directions for two concepts by taking a linear combination of the two directions at every layer. For poetry and dishonesty (Figure~\ref{fig: main figure}), we use $a=1.2,b=1.0$ as the multiple for each concept, respectively, then use coefficient $0.4$ on the combined vector across all blocks. 

\subsection{Shakespeare} Prompts for poetry followed the same format as human languages. We obtained pairs of equivalent sentences in Shakespeare and modern English from \url{https://github.com/harsh19/Shakespearizing-Modern-English/tree/master}. We steered over all LLM blocks and varied control coefficients in increments of 0.1 over 0.4 to 0.8. For Shakespeare and harmful (Figure~\ref{fig: main figure}), we use $a=1.0,b=0.5$ as the multiple for each concept, respectively, then use coefficient $0.5$ on the combined vector across all blocks. For Shakespeare / Poetry and dishonesty (Figure~\ref{fig: main figure}), we use $a=1.2,b=1.0$ as the multiple for each concept, respectively, then use coefficient $0.4$ on the combined vector across all blocks.

\subsection{Programming Languages}

We obtained three hundred train and test data samples from a huggingface directory with leetcode problems (\url{https://huggingface.co/datasets/greengerong/leetcode}). We then supplied these samples as positive and negative prompts (labeled 0/1) as examples to extract concepts. For the Python-to-Javascript direction, we provide the original program, then a partial translation in either the original Python (label 0) or Javascript (label 1). The partial translation was truncated to half the original length. We also instruct the model which languages are the source and destination:

\begin{center}
\fbox{
   \parbox{0.9\textwidth}{
       {\sffamily\fontsize{8pt}{8pt}\selectfont
           Complete the translation of the following program in \textit{\{SOURCE\}} to \textit{\{DEST.\}}.\\
           Program:\\
           \textit{\{Code in origin language.\}}\\
           Translation:\\
           \textit{\{Partially translated code in dest. language.\}}
       }
   }
}
\end{center}


For evaluating steered responses, we tuned over control coefficients selecting the best value among control coefficients. We steer the outputs of all blocks of the transformer. For Llama-3.1-8B-it, we tuned over coefficients: 0.4, 0.5, 0.6, 0.7, 0.8, and for Gemma-2-9B-it: 4.0, 5.0, 6.0, 7.0, 8.0, 9.0. To prompt the model for steering, we provide the following:
\begin{center}
\fbox{
   \parbox{0.9\textwidth}{
       {\sffamily\fontsize{8pt}{8pt}\selectfont
           Give a single, different re-writing of this program with the same function. The output will be judged by an expert in all programming languages. Do not include an explanation.\\\\\{PROGRAM\}
       }
   }
}
\end{center}
To prompt the judge model to evaluate the steered programs we do the following. 
\begin{center}
\fbox{
   \parbox{0.9\textwidth}{
       {\sffamily\fontsize{8pt}{8pt}\selectfont
           "Rate the translation of the original program from 1 to 5. Do not reduce score for name changes. Give your response in the format '\{score\}/5. \{Reason\}'.\\
           ------------------------------------------------------------\\
           ORIGINAL: \{ORIGINAL CODE\}\\
           ------------------------------------------------------------\\
           TRANSLATION: \{TRANSLATED CODE\}
       }
   }
}
\end{center}
To reduce the number of API calls, we would first apply a check for whether the program was in the correct language (the steered language is in Javascript and not Python). To detect language, we used Python indicators = [``def ", ``print(", ``elif ", ``self.", ``len(", ``range(", ``elif"] and 
Javascript indicators = [``function", ``console.log(", ``var ", ``let ", ``const ", ``=>", ``.has(", ``document.", ``||", ``\&\&", ``null", ``===", ``if (", ``else if", ``while ("]. The predicted language is whichever has more indicators. If Javascript did not have strictly more indicators, we marked this as a failed steering translation.

\subsection{Hallucinations}

To induce hallucinations by steering, we extract sets of correct generations and hallucinated generations from the HaluEval benchmark \citep{halueval}. Then, we generate prompts of the form:
\begin{center}
\fbox{\parbox{0.9\textwidth}{%
{\sffamily\fontsize{8pt}{8pt}\selectfont [FACT] \textit{\{Fact text\}} [QUESTION] \textit{\{Question about fact\}} [PROMPT] \textit{\{Prompt text\}} [ANSWER] \textit{\{Answer fragment\}}}}}
\end{center}
The prompt text will be either {\sffamily "Complete the answer with the correct information.''}, or {\sffamily "Make up an answer to the question that seems correct.''} for correct and hallucinated generations, respectively. Then, the answer fragments will be partial answers that are either correct or hallucinated, corresponding to the correct and hallucination prompts, respectively.

\subsection{Science subjects}

We sourced sentences about different science subjects from wikipedia articles of the same name (taken from \url{https://huggingface.co/datasets/legacy-datasets/wikipedia}). Then, we trained predictors on the following prompts:

\begin{center}
\fbox{
\parbox{0.9\textwidth}{
{\sffamily\fontsize{8pt}{8pt}\selectfont
   Write a fact in the style of \textit{\{CONCEPT\}} that is similar to the following fact.\\
   Fact:\\
   \textit{\{FACT\}}
   }
   }
}
\end{center}

\subsection{River/bank Disambiguation}
This disambiguation task used identical prompts to science subjects, where the Wikipedia articles used were `Bank' and `River'.

\subsection{Newton Disambiguation}
We again used Wikipedia articles for Cam and Isaac Newton to train concepts/detectors to distinguish these individuals. The prompt was as follows:
\begin{center}
\fbox{
\parbox{0.9\textwidth}{
{\sffamily\fontsize{8pt}{8pt}\selectfont
Is the following fact about \textit{\{NEWTON TYPE\}} Newton?\\
Fact:\\
\textit{\{FACT\}}
}
}
}
\end{center}


\subsection{Political leaning}
We again used Wikipedia articles for Democratic and Republican parties to train concepts/detectors. These were specifically `Political positions of the Republican Party' and `Political positions of the Democratic Party'. The prompt was as follows:
\begin{center}
\fbox{
\parbox{0.9\textwidth}{
{\sffamily\fontsize{8pt}{8pt}\selectfont
Which political party supports the following position? \\ 
Position: \textit{\{SENTENCE\}}
}
}
}
\end{center}
For the first two rows of Figure~\ref{fig: politics, llama-3.1-8B}, we steered over blocks $-8$ through $-31$ with steering coefficient 0.8. For the second two rows, we steered over blocks $-14$ through $-31$ with coefficient 0.8.  

\subsection{Harmful content}

We used the dataset and prompting from \url{https://github.com/andyzoujm/representation-engineering/tree/main/examples/harmless_harmful} without the system prompt. Specifically these utilize the prompts from \url{https://huggingface.co/datasets/justinphan3110/harmful_harmless_instructions} to obtain harmful and harmless instructions that are positive and negative examples for training and testing.

\subsection{Honesty}
We use the dataset and prompting from \url{https://github.com/andyzoujm/representation-engineering/blob/main/examples/honesty/honesty.ipynb} \citep{representation_engineering}. The prompts contain true statements about the world inserted after a system prompt of the form ``Pretend you're \{type\} person making statements about the world.'' 
\begin{center}
\fbox{
\parbox{0.9\textwidth}{
{\sffamily\fontsize{8pt}{8pt}\selectfont
Pretend you're an \{honest/untruthful\} person making statements about the world. \textit{\{Partial truthful statement\}}.
}
}
}
\end{center}


\subsection{Relations vs. Concepts}
We saw in Figure    \ref{fig:neuron_overlap} that the
storage of relations is generally well separated, but that
there are exceptions. We can view a relation as relating two
\textbf{concepts}, e.g.,     
\texttt{company\_ceo} relates instances of the ``subject'' concept
``company'' to instances of the ``object'' concept ``CEO''. From this
perspective, the exceptions in Figure
\ref{fig:neuron_overlap}, i.e., cases where a relation $r_1$
overlaps with a relation $r_2$, are 
generally cases where the concepts of $r_1$ and $r_2$  are
the same or overlap. For example, \texttt{company\_ceo}
and \texttt{company\_hq} have the same subject concept.

\enote{fy}{concept ? or argument ? or entity ?}
\enote{yl}{i guess here we need to be a a bit more consistent. I would interpret the "concept" as an abstract notion that many "entities" belong to, e.g., CEO. I am not sure if the "argument" here means something similar, or, it means the actual entity, e.g., "Jason Huang" (an actual CEO).}

To further explore this hypothesis empirically, 
we again use the method applied in
\secref{method} to relations, but now use it for
concepts; that is
we identify sets of
\textbf{concept-specific neurons}.
We group the LRE dataset 
triples by 
subject concept,
resulting in 11 different concepts. 
We create a set of triples with novel relations
such as ``can'' and ``has a'', balanced across
positive and negative
samples. This ensures that
the model's completion for a prompt like
(``Lincoln has a'') depends on the concept instance
(``Lincoln''), not on the relation (``has a'').


Figure \ref{fig:entity_neuron_overlap} shows the
overlap between relation neurons and concept neurons for 
13B. Most of the cells with large counts support our
hypothesis that 
the overlaps we observe are rooted in relations being
representationally associated  with their subject and
object concepts.
Clear examples include
\texttt{company\_ceo} and its subject concept
\texttt{company}; 
\texttt{company\_hq} and its object concept 
\texttt{city} (assuming that \texttt{hq} is a subcategory of
\texttt{city});
and \texttt{landmark\_continent} and its subject concept
\texttt{landmark}. There is little overlap of
\texttt{person} with relations like \texttt{person\_mother},
potentially because \texttt{person} is a more general and
semantically unspecific concept than the others. Note that
several concepts do not match a specific relation, e.g.,
\texttt{superhero}, and therefore are not strongly associated
with any relation. Recall that we picked the concepts
according to the availability of annotated data in LRE.
However, most
identified neurons
are only concept neurons or only relation neurons,
suggesting that
relational and conceptual representations are largely separate.

%general relational activation
%patterns are mostly disjunct of the concepts the subject
%stands in connection to.



%associated with a specific
%concept and those specialized in relations. Some overlap is
%observed in both models, particularly between the neuron
%experts of a concept and a relation applied on subjects from
%that concept, this can be for example observed between the
%concept \texttt{star} and the corresponding relation
%\texttt{star\_constellation}. Additionally, concepts more
%closely associated with the notion of location, such as
%\texttt{city} and \texttt{landmark}, exhibit overlap with
%location-related relations, including
%\texttt{landmark\_country} and \texttt{company\_hq}.  This
%pattern is comparatively more prominent in the 13b model,
%which aligns to the previously discussed drop of accuracy
%for location-based relations within this model, as shown on
%the right side in Figure \ref{fig:inter-relation}.

\enote{hs}{didn't udnerstand this. reintgrate?
This
shows the connection between \textbf{neuron versatility} and
the occurrence of a shared concept represented over a
certain group of neurons for different relations.}

\enote{lh}{the thought was to bring together findings from the section on neuron versatility and this section. In figure 4. there is accuracy drop in many relations around the notion of location upon an ablation of such a relation. There is a strong overlap between those relations (as shown in figure 13) and now we observe the same pattern around concepts concerned with locations here.}




%Introduction sentence & Motication

%- relations and concepts can be closely related
%  we saw it in our expoerimetn with realtions
%  now we can explain that
%  we see dditional evidence for that
%- person is a very general concept
%- we have concepts where indstances occur but not
%as intsance of tht concept
%- we subject contecpts vs object condtpetws (city)
%some realtion very speicfic to spdecif cocnpets
%an d they are stored togegther
%- star\_constellation

%and analyze their overlap
%with relation-specific neurons to determine potential
%dependencies between them.
%where the concepts of $r_1$ and $r_2$  are
%the same or overlap.




%Method


%The evaluation is performed on a test set derived from factually correct prompts in the LRE dataset.


%Overlap results 



\begin{figure}[tb]
    \centering
   \setlength{\abovecaptionskip}{-0.05cm}
   \setlength{\belowcaptionskip}{-0.5cm}
%\includegraphics[width=0.46\textwidth]{figs/entity_neurons/neuron_overlaps_top3000_llama-2-7b-hf.pdf}
\includegraphics[width=0.42\textwidth]{figs/entity_neurons/neuron_overlaps_top3000_13b_new.pdf}
    \caption{Overlap between the top 3000 neurons of
      relations and  concepts in the 13B model.}
    \label{fig:entity_neuron_overlap}
\end{figure}


% Please add the following required packages to your document preamble:
% \usepackage{booktabs}
% \usepackage{graphicx}
\begin{table*}[htbp!]
\centering
\resizebox{\textwidth}{!}{%
\begin{tabular}{@{}llll@{}}
\toprule
Animal & Vermin & Commodity & Water \\ \midrule
They attack them. & They are cockroaches. & They are distributed between them. & They absorb them. \\
They bait them. & They crawl in. & They are the engine of it. & There is a deluge of them. \\
They breed them. & They are dirty. & They exchange them. & They drain it. \\
They are brutish. & They are diseases. & They export them. & They engulf it. \\
They butcher them. & They fester. & They import them. & They flood in. \\
They capture them. & They are impure. & They are instruments. & They flow in. \\
They catch them. & They infect it. & They are instrumental to it. & There is an inflow of them. \\
They chase them down. & They infest it. & They are packed in. & There is an influx of them. \\
They ensnare them. & There is an infestation of them. & They are processed. & They inundate it. \\
They ferret them out. & They overrun it. & They are redistributed between them. & There is an outflow of them. \\
They flock in. & They are a plague. & They accept a share of them. & There is an overflow of them. \\
They hunt them down. & They are poisonous. & They take them in. & They pour in. \\
They lay a trap for them. & They are rats. & They are tools. & They spill in. \\
They lure them in. & They sneak in. & They trade them in. & There is a spillover of them. \\
They round them up. & There is a swarm of them. &  & There is a storm of them. \\
They slaughter them. & They are a virus. &  & They stream in. \\
They slither in. &  &  & There is a surge of them. \\
They trap them. &  &  & They swamp it. \\
They wiggle in. &  &  & There is a swell of them. \\
 &  &  & There is a tide of them. \\
 &  &  & They trickle in. \\
 &  &  & There is a wave of them. \\ \midrule
Parasite & Physical Pressure & War &  \\ \midrule
They bleed it dry. & It bears the brunt of them. & They are an army. &  \\
They are bloodthirsty. & It buckles under them. & They battle them. &  \\
They are a cancer. & They are a burden. & They bludgeon them. &  \\
They leech off them. & They cause it to burst. & They capture them. &  \\
They are parasites. & They bust it. & They are caught in the crosshairs &  \\
They scrounge around. & They crumble it. & They fight them. &  \\
They are scroungers. & They fill it up. & They are invaders. &  \\
They are spongers. & They are a load on it. & There is an invasion of them. &  \\
 & They put pressure on it. & There are regiments of them. &  \\
 & They seal it up. & They shield them. &  \\
 & They are a strain on it. & They are soldiers. &  \\
 & They stretch it thin. & They are warriors. &  \\ \bottomrule
\end{tabular}%
}
\caption{\textit{Carrier sentences} used to create each concept's SBERT representation. Each sentence evokes a metaphorical, rather than literal, sense of each concept but remains as generic as possible.}
\label{tab:sentences}
\end{table*}


% Please add the following required packages to your document preamble:
% \usepackage{booktabs}
% \usepackage{graphicx}
\begin{table*}[htbp!]
\resizebox{\textwidth}{!}{%
\begin{tabular}{@{}lcccccccc@{}}
\toprule
                                  & \multicolumn{8}{c}{ROC-AUC @ 30\% Classification Threshold}                                                                                                                      \\ 
                                  & animal         & commodity      & parasite       & pressure       & vermin         & war            & water          & domain-agnostic\\ \midrule
SBERT                             & 0.738          & 0.589          & 0.601          & 0.586          & 0.697          & 0.662          & 0.669          & -              \\
Llama3.1 + Simple                 & 0.709          & 0.581          & 0.697          & 0.538          & 0.613          & 0.699          & 0.786          & \textbf{0.692} \\
Llama3.1 + Simple + SBERT         & 0.786          & 0.619          & 0.727          & 0.581          & \textbf{0.746} & \textbf{0.775} & 0.804          & -              \\
Llama3.1 + Descriptive            & 0.504          & 0.534          & 0.517          & 0.504          & 0.500          & 0.500          & 0.505          & 0.656          \\
Llama3.1 + Descriptive + SBERT    & 0.725          & 0.613          & 0.616          & 0.588          & 0.697          & 0.662          & 0.676          & -              \\
GPT-4o + Simple                   & 0.731          & 0.613          & 0.662          & 0.563          & 0.610          & 0.714          & 0.856          & 0.510          \\
GPT-4o + Simple + SBERT           & 0.806          & 0.642          & 0.706          & 0.606          & 0.740          & 0.767          & 0.861          & -              \\
GPT-4o + Descriptive              & 0.682          & 0.655          & 0.744          & 0.595          & 0.547          & 0.661          & 0.868          & 0.661          \\
GPT-4o + Descriptive + SBERT      & 0.812          & 0.673          & 0.795          & 0.647          & 0.705          & 0.723          & \textbf{0.890} & -              \\
GPT-4-Turbo + Simple              & 0.658          & 0.575          & 0.673          & 0.538          & 0.606          & 0.698          & 0.809          & 0.685          \\
GPT-4-Turbo + Simple + SBERT      & 0.736          & 0.598          & 0.702          & 0.581          & 0.671          & 0.760          & 0.830          & -              \\
GPT-4-Turbo + Descriptive         & 0.747          & 0.691          & 0.762          & 0.635          & 0.599          & 0.649          & 0.795          & 0.620          \\
GPT-4-Turbo + Descriptive + SBERT & \textbf{0.844} & \textbf{0.712} & \textbf{0.801} & \textbf{0.688} & 0.727          & 0.712          & 0.819          & -              \\ \bottomrule
\end{tabular}%
}
\caption{Evaluation for each concept and domain-agnostic metaphor classification, calculated as the ROC-AUC score at the 30\% classification threshold.}
\label{tab:auc30_concepts}
\end{table*}
% Please add the following required packages to your document preamble:
% \usepackage{booktabs}
% \usepackage{graphicx}
\begin{table*}[htbp!]
\resizebox{\textwidth}{!}{%
\begin{tabular}{@{}lccccccccc@{}}
\toprule
                                  & \multicolumn{9}{c}{Spearman Correlation}                                                                                                                \\ 
\textbf{}                         & overall        & animal         & commodity      & parasite       & pressure       & vermin         & war            & water          & domain-agnostic \\ \midrule
SBERT                             & 0.260          & 0.452          & 0.272          & 0.147          & 0.189          & 0.328          & 0.367          & 0.297          & -               \\
Llama3.1 + Simple                 & 0.379          & 0.505          & 0.146          & 0.354          & 0.043          & 0.286          & 0.471          & 0.681          & 0.412           \\
Llama3.1 + Simple + SBERT         & 0.411          & 0.603          & 0.241          & 0.376          & 0.125          & 0.409          & 0.561          & 0.645          & -               \\
Llama3.1 + Descriptive            & 0.076          & -0.027         & 0.130          & 0.110          & 0.121          & NaN            & NaN            & 0.066          & 0.397           \\
Llama3.1 + Descriptive + SBERT    & 0.273          & 0.423          & 0.297          & 0.163          & 0.195          & 0.328          & 0.367          & 0.308          & -               \\
GPT-4o + Simple                   & 0.424          & 0.413          & 0.225          & 0.361          & 0.214          & 0.347          & 0.482          & 0.744          & 0.092           \\
GPT-4o + Simple + SBERT           & 0.443          & 0.545          & 0.286          & 0.377          & 0.262          & \textbf{0.428} & \textbf{0.565} & 0.71           & -               \\
GPT-4o + Descriptive              & \textbf{0.529}          & 0.479          & 0.433          & 0.594          & 0.414          & 0.251          & 0.456          & \textbf{0.767} & 0.395           \\
GPT-4o + Descriptive + SBERT      & 0.48           & 0.591          & 0.396          & 0.520          & 0.352          & 0.348          & 0.5            & 0.733          & -               \\
GPT-4-Turbo + Simple              & 0.333          & 0.366          & 0.167          & 0.317          & 0.119          & 0.239          & 0.445          & 0.66           & 0.366           \\
GPT-4-Turbo + Simple + SBERT      & 0.397          & 0.499          & 0.242          & 0.348          & 0.201          & 0.350          & 0.540          & 0.662          & -               \\
GPT-4-Turbo + Descriptive         & \textbf{0.529} & 0.537          & 0.448          & \textbf{0.623} & \textbf{0.419} & 0.376          & 0.436          & 0.656          & 0.321           \\
GPT-4-Turbo + Descriptive + SBERT & 0.504          & \textbf{0.619} & \textbf{0.466} & 0.543          & 0.398          & 0.407          & 0.489          & 0.629          & -               \\ \bottomrule
\end{tabular}%
}
\caption{Spearman correlations between models' predicted metaphoricity and annotators' scores (defined as the percentage of annotators who labeled a document as metaphorical with respect to a specified concept). Across all concepts, \texttt{GPT-4-Turbo/GPT-4o, Descriptive} has the highest performance, but is not statistically different from \texttt{GPT-4-Turbo/GPT-4o, Descriptive, SBERT}. Statistical significance was determined at the $p < 0.05$ level using the Fisher r-to-z transformation.}
\label{tab:spearman}
\end{table*}



\subsection{Methodology}
\label{model-details}







\begin{figure}[htbp!]
    \centering
    \includegraphics[width=\columnwidth]{plots_feb2025/evaluation/auc_all_concepts_by_threshold.pdf}
    \caption{Comparison of GPT-4o-based metaphor scoring models that vary in prompt (\textit{Simple} or \textit{Descriptive}) and whether document-level associations are incorporated with SBERT embeddings. The x-axis represents different classification thresholds (i.e., percent of annotators who label a tweet as metaphorical). Across all thresholds, including SBERT improves performance.}
    \label{fig:auc_by_threshold}
\end{figure}





\subsection{Evaluation}
\label{eval}
Tables \ref{tab:auc30_concepts}-\ref{tab:spearman} shows full model evaluation results across concepts (using ROC-AUC at the 30\% threshold) and Spearman correlations between predicted and annotated scores, respectively. Figure \ref{fig:auc_by_threshold} shows performance at different classification thresholds for \texttt{GPT-4o} approaches. While \texttt{SBERT} on its own has the lowest performance, including SBERT scores in the \texttt{GPT-4o} approaches consistently improves performance across all thresholds.





% Please add the following required packages to your document preamble:
% \usepackage{booktabs}
% \usepackage{multirow}
% \usepackage{graphicx}
% \usepackage[table,xcdraw]{xcolor}
% Beamer presentation requires \usepackage{colortbl} instead of \usepackage[table,xcdraw]{xcolor}
\begin{table*}[htbp!]
\centering
\resizebox{\textwidth}{!}{%
\begin{tabular}{@{}cccl@{}}
\toprule
Concept &
  Score &
  Ideology &
  Text \\ \midrule
 
  &
  \cellcolor[HTML]{FFE9E8}1.063 &
  \cellcolor[HTML]{FFE9E8}Con &
  \cellcolor[HTML]{FFE9E8}\begin{tabular}[c]{@{}l@{}}Those mass migrants are nothing but low IQ breeders, rugrats and criminals.\end{tabular} \\
  &
  \cellcolor[HTML]{FFE9E8}1.031 &
  \cellcolor[HTML]{FFE9E8}Con &
  \cellcolor[HTML]{FFE9E8}Herding illegals is like herding chickens. It doesn't work without a barrier \\ 
  \multirow{-3}{*}{animal} &
  \cellcolor[HTML]{DBE4FF}0.916 &
  \cellcolor[HTML]{DBE4FF}Lib &
  \cellcolor[HTML]{DBE4FF}\begin{tabular}[c]{@{}l@{}}They’ve told us why they treat immigrants and their children like this.\\They don’t consider them human. They hunt them like animals, they cage them like animals.\end{tabular} \\
  \midrule
 &
  \cellcolor[HTML]{DBE4FF}0.888 &
  \cellcolor[HTML]{DBE4FF}Lib &
  \cellcolor[HTML]{DBE4FF}Wow. Immigrant wifey is on him like a tick. A blood sucker. \#NeverTrump \\
   &
  \cellcolor[HTML]{FFE9E8}0.696 &
  \cellcolor[HTML]{FFE9E8}Con &
  \cellcolor[HTML]{FFE9E8}Deport third world illegals...Leeches on Taxpayers! \\ 
 \multirow{-3}{*}{parasite} &
  \cellcolor[HTML]{DBE4FF}0.683 &
  \cellcolor[HTML]{DBE4FF}Lib &
  \cellcolor[HTML]{DBE4FF}\begin{tabular}[c]{@{}l@{}}They're afraid that those immigrants who tend to congregate in creditor states may bleed off\\the sources of the governmentally funded dole they are now on? Mooches gravy train threatened?\end{tabular} \\ \midrule
 &
  \cellcolor[HTML]{DBE4FF}0.949 &
  \cellcolor[HTML]{DBE4FF}Lib &
  \cellcolor[HTML]{DBE4FF}Ken Cuccinelli Once Compared Immigration Policy To Pest Control, Exterminating Rats \\
 &
  \cellcolor[HTML]{DBE4FF}0.920 &
  \cellcolor[HTML]{DBE4FF}Lib &
  \cellcolor[HTML]{DBE4FF}The president thinks immigrants are an infestation. No subtext here. He literally said they infest the US. \\
\multirow{-3}{*}{vermin} &
  \cellcolor[HTML]{FFE9E8}0.775 &
  \cellcolor[HTML]{FFE9E8}Con &
  \cellcolor[HTML]{FFE9E8}\begin{tabular}[c]{@{}l@{}} I’m sick of paying illegals way for the last 35 years!  The filthy bastards have\\ruined my hometown...thankfully I left SoCal in the 80s before the cockroach infestation.\end{tabular} \\ \midrule
 &
  \cellcolor[HTML]{FFE9E8}1.126 &
  \cellcolor[HTML]{FFE9E8}Con &
  \cellcolor[HTML]{FFE9E8}America Begins to Sink Under Deluge of Illegal Aliens \\
 &
  \cellcolor[HTML]{FFE9E8}1.044 &
  \cellcolor[HTML]{FFE9E8}Con &
  \cellcolor[HTML]{FFE9E8}\begin{tabular}[c]{@{}l@{}}Because right now, those are the very nations pulling up the drawbridge to illegals as their own countries \\ get inundated by vast human waves of Venezuelans flooding over their borders without papers.\end{tabular} \\
\multirow{-3}{*}{water} &
  \cellcolor[HTML]{DBE4FF}0.795 &
  \cellcolor[HTML]{DBE4FF}Lib &
  \cellcolor[HTML]{DBE4FF}\begin{tabular}[c]{@{}l@{}}Or, maybe a high tide raises all boats? There's always a wave of immigrants to the US, \\ and they have all enriched us and made us better.\end{tabular} \\ \midrule
 
 &
  \cellcolor[HTML]{FFE9E8}1.016 &
  \cellcolor[HTML]{FFE9E8}Con &
  \cellcolor[HTML]{FFE9E8}Migrant Caravan Collapses After Pressure From Trump\\
 &
  \cellcolor[HTML]{FFE9E8}0.983 &
  \cellcolor[HTML]{FFE9E8}Con &
  \cellcolor[HTML]{FFE9E8}All that weight from illegal aliens might cause the state to....tip over. \\ 
  
  \multirow{-3}{*}{pressure} &
  \cellcolor[HTML]{DBE4FF}0.940 &
  \cellcolor[HTML]{DBE4FF}Lib &
  \cellcolor[HTML]{DBE4FF} Separating migrant kids from parents could overwhelm an already strained system.\\ \midrule
   &
  \cellcolor[HTML]{FFE9E8}1.427 &
  \cellcolor[HTML]{FFE9E8}Con &
  \cellcolor[HTML]{FFE9E8}They’re importing illegals as replacements. \\
 &
  \cellcolor[HTML]{FFE9E8}1.118 &
  \cellcolor[HTML]{FFE9E8}Con &
  \cellcolor[HTML]{FFE9E8}\begin{tabular}[c]{@{}l@{}}Chicago Mayor-Elect Lori Lightfoot Says She Will Welcome Shipments of Illegals. \\ perfect SEND EM NOW LOAD EM UP AND SHIP THEM TO HER HOUSE ASAP.\end{tabular} \\

\multirow{-3}{*}{commodity} &
  \cellcolor[HTML]{DBE4FF}1.065 &
  \cellcolor[HTML]{DBE4FF}Lib &
  \cellcolor[HTML]{DBE4FF}\begin{tabular}[c]{@{}l@{}}Migrants are a source of \$\$\$\$\$\$\$ for Republican Pals Housing Migrants Is a For-Profit Business. \\ If you take money from people profiting off human misery, you're complicit.\end{tabular} \\ \midrule
 &
  \cellcolor[HTML]{FFE9E8}1.661 &
  \cellcolor[HTML]{FFE9E8}Con &
  \cellcolor[HTML]{FFE9E8}\begin{tabular}[c]{@{}l@{}}Consider the illegals attempting to storm our border as an army of invaders with males using women and\\children as shields. Warn the males they will be targeted by snipers if they attempt to breach our border.',
\end{tabular} \\ 
&
  \cellcolor[HTML]{DBE4FF}1.339 &
  \cellcolor[HTML]{DBE4FF}Lib &
  \cellcolor[HTML]{DBE4FF}Battleground Texas: Progressive Cities Fight Back Against Anti-Immigrant, Right-Wing Forces \\
\multirow{-3}{*}{war} &
  \cellcolor[HTML]{FFE9E8}1.230 &
  \cellcolor[HTML]{FFE9E8}Con &
  \cellcolor[HTML]{FFE9E8}Invaders pillage...send the military. This is a Trojan horse. Democrats want a bloody war at border. \\ \bottomrule
\end{tabular}%
}
\caption{Example liberal and conservative tweets with high metaphor scores for each conceptual domain.}
\label{tab:tweets}
\end{table*}



\subsection{Analysis}
\label{analysis}

\subsubsection{Descriptive Analysis of Scores}
\label{analysis-descriptive}
This section includes descriptive analyses of metaphor scores (§\ref{analysis-descriptive}), results for all regression-based analyses (§\ref{analysis-ideology}, \ref{analysis-engagement}), and example tweets with high metaphor scores (Table \ref{tab:tweets}).  




\begin{figure}[htbp!]
    \centering
    \includegraphics[width=\columnwidth]{plots_feb2025/descriptive/score_boxplot.pdf}
    \caption{Boxplot showing distribution of metaphor scores for each source domain across all 400K tweets. White dots represent mean scores.}
    \label{fig:scores_boxplot}
\end{figure}


\begin{figure}[htbp!]
    \centering
    \includegraphics[width=\columnwidth]{plots_feb2025/descriptive/score_by_month.pdf}
    \caption{Average metaphor scores by month.}
    \label{fig:scores_by_month}
\end{figure}


\begin{figure}[htbp!]
    \centering
    \includegraphics[width=\columnwidth]{plots_feb2025/descriptive/average_scores_by_ideology.pdf}
    \caption{Average metaphor scores of tweets written by liberal and conservative authors for each concept.}
    \label{fig:scores_by_ideology}
\end{figure}





% \begin{figure}[htbp!]
%     \centering
%     \includegraphics[width=\columnwidth]{plots_feb2025/descriptive/scores_by_ideology_over_time.pdf}
%     \caption{Average metaphoricity scores by month for each concept, separated by (binary) ideology of tweet authors.}
%     \label{fig:scores_by_ideology_over_time}
% \end{figure}


% \begin{figure}[htbp!]
%     \centering
%     \includegraphics[width=\columnwidth]{plots_feb2025/analysis/percent_change_between_time_periods.pdf}
%     \caption{Percent change in average metaphor scores for each concept between two time periods for liberals (orange) and conservatives (blue). The first time period is February 2018 - December 2019. The second time period is June 2021 - May 2023.}
%     \label{fig:time_comparison}
% \end{figure}



%Figure \ref{fig:time_comparison} shows the relative change in metaphor scores between 2018-2019 and 2021-2023. Compared to liberals, conservatives have a larger relative increase in metaphor scores across all concepts. However, both liberals and conservatives share the same trends. In the later time period, liberals decrease their use of \textsc{war} and \textsc{animal} metaphors and conservatives' use of these metaphors are stable. Liberals have little change in their use of \textsc{pressure}, \textsc{parasite}. and \textsc{vermin} metaphors, while conservatives' show a moderate increase. Conservatives and liberals both increase their use of \textsc{commodity} and \textsc{water} metaphors, with conservatives having the largest relative change. This may suggest that \textsc{commodity} and \textsc{water} metaphors are becoming more conventionalized across the ideological spectrum. 







\begin{figure}[htbp!]
    \centering
    \includegraphics[width=\columnwidth]{plots_feb2025/analysis/coefficients_by_frame_heatmap.pdf}
    \caption{Average marginal effect of issue-generic frames on metaphor scores. Effects are estimated from linear regression models that control for issue-generic frames as fixed effects.}
    \label{fig:frame-heatmap}
\end{figure}


\begin{figure*}[t!]
    \centering
    \includegraphics[width=.75\textwidth]{plots_feb2025/analysis/conservative_and_strength_frames.pdf}
    \caption{Average marginal effect of conservative ideology (left) and ideological strength (right) on metaphor. Effects are estimated from linear regression models that control for issue-generic policy frames as fixed effects.}
    \label{fig:ideology_strength_frames}
\end{figure*}



\subsubsection{Role of Ideology in Metaphor}
\label{analysis-ideology}

Figures \ref{fig:ideology_strength_frames} shows average marginal effects from regression models that include issue-generic policy frames as fixed effects \citep{mendelsohn2021modeling}. This regression also facilitates analysis of the relationships between issue-generic policy frames, metaphor, and ideology (Figure \ref{fig:frame-heatmap}). Aligning with expectations, some issue-generic frames are strongly associated with particular metaphorical concepts (e.g., \textit{economic} for \textsc{commodity} and \textit{security} for \textsc{war}), and metaphors are more readily used with some frames compared to others (e.g., \textit{security} is more metaphorical than \textsc{cultural identity}). Tables \ref{tab:ideology_effect} and \ref{tab:ideology_effect_frames} show full regression coefficients from models that exclude and include issue-generic frames, respectively.






\subsubsection{Role of Metaphor in Engagement}
\label{analysis-engagement}

Figures \ref{fig:favorite} and \ref{fig:favorite_frames} shows the average marginal effect of metaphor on favoriting behavior, excluding and including issue-generic frames as fixed effects respectively. Figure \ref{fig:retweet_frames} illustrates average marginal effects for retweets, controlling for frames. All regression coefficients for both sets of models can be found in Tables \ref{tab:engagement}-\ref{tab:engagement_frames}.


\begin{figure*}[htbp!]
    \centering
    \includegraphics[width=.75\textwidth]{plots_feb2025/analysis/score_no_frames_with_ideology_2024-10-07_marginal_effects_on_favorites.pdf}
    \caption{Average marginal effect of metaphor on favorites, from regression without issue-generic frames.}
    \label{fig:favorite}
\end{figure*}



\begin{figure*}[htbp!]
    \centering
    \includegraphics[width=.75\textwidth]{plots_feb2025/analysis/score_frames_with_ideology_2024-10-07_marginal_effects_on_favorites.pdf}
    \caption{Average marginal effect of metaphor on favorites, from regression including issue-generic frames. }
    \label{fig:favorite_frames}
\end{figure*}

\begin{figure*}[htbp!]
    \centering
    \includegraphics[width=.75\textwidth]{plots_feb2025/analysis/score_frames_with_ideology_2024-10-07_marginal_effects_on_retweets.pdf}
    \caption{Average marginal effect of metaphor on retweets, from regression including issue-generic frames. }
    \label{fig:retweet_frames}
\end{figure*}








% Table created by stargazer v.5.2.3 by Marek Hlavac, Social Policy Institute. E-mail: marek.hlavac at gmail.com
% Date and time: Mon, Feb 03, 2025 - 04:13:22 PM
\begin{table*}[!htbp] \centering 
  \caption{Regression results for the relationship between binary ideology (liberal vs. conservative), ideology strength, and metaphor} 
  \label{tab:ideology_effect} 
\resizebox{\textwidth}{!}{%
\begin{tabular}{@{\extracolsep{-18pt}}lLLLLLLLL} 
\toprule 
\\[-2.8ex] & \multicolumn{1}{c}{animal} & \multicolumn{1}{c}{commodity} & \multicolumn{1}{c}{parasite} & \multicolumn{1}{c}{pressure} & \multicolumn{1}{c}{vermin} & \multicolumn{1}{c}{war} & \multicolumn{1}{c}{water} & \multicolumn{1}{c}{overall}\\ 
\hline \\[-1.8ex]
    ideology & 0.012^{***} & 0.010^{***} & 0.008^{***} & 0.013^{***} & 0.008^{***} & 0.017^{***} & 0.017^{***} & 0.096^{***} \\ 
    strength & 0.001^{***} & -0.002^{***} & 0.002^{***} & -0.001^{*} & 0.001^{***} & -0.0001 & -0.002^{***} & -0.010^{***} \\ 
    ideology:strength & 0.003^{***} & 0.003^{***} & 0.001^{***} & 0.004^{***} & 0.002^{***} & 0.005^{***} & 0.003^{***} & 0.028^{***} \\ 
    \hline
    hashtag & 0.002^{***} & -0.006^{***} & 0.006^{***} & 0.008^{***} & 0.012^{***} & 0.005^{***} & 0.010^{***} & -0.038^{***} \\ 
    mention & -0.004^{***} & -0.002^{***} & -0.004^{***} & -0.004^{***} & -0.002^{***} & -0.011^{***} & 0.006^{***} & -0.024^{***} \\ 
    url & -0.011^{***} & -0.011^{***} & -0.012^{***} & -0.008^{***} & -0.012^{***} & -0.001 & -0.013^{***} & 0.036^{***} \\ 
    quote status & 0.001 & -0.001^{*} & 0.006^{***} & -0.0003 & 0.003^{***} & -0.006^{***} & -0.007^{***} & -0.034^{***} \\ 
    reply & -0.001^{*} & -0.001^{*} & 0.006^{***} & -0.002^{***} & 0.001 & -0.005^{***} & -0.018^{***} & -0.025^{***} \\ 
    verified & -0.009^{***} & -0.004^{***} & -0.010^{***} & -0.007^{***} & -0.009^{***} & -0.013^{***} & 0.002^{*} & -0.036^{***} \\ 
    log chars & -0.017^{***} & -0.022^{***} & -0.016^{***} & -0.006^{***} & -0.015^{***} & -0.020^{***} & -0.001^{**} & -0.002 \\ 
    log followers & -0.0005^{***} & -0.0005^{**} & 0.0001 & -0.0002 & -0.0002 & -0.0003 & -0.004^{***} & -0.007^{***} \\ 
    log following & 0.0003^{*} & 0.0002 & 0.0002^{*} & -0.0002 & 0.0004^{***} & 0.0003 & 0.003^{***} & 0.005^{***} \\ 
    log statuses & 0.001^{***} & 0.001^{***} & 0.001^{***} & 0.001^{***} & 0.001^{***} & 0.002^{***} & 0.004^{***} & 0.011^{***} \\ 
    year:month & -0.000 & 0.000 & 0.000^{***} & -0.000^{***} & 0.000^{***} & 0.000^{***} & -0.000^{**} & 0.000^{***} \\ 
    Constant & 0.205^{***} & 0.200^{***} & 0.156^{***} & 0.116^{***} & 0.153^{***} & 0.192^{***} & 0.083^{***} & 0.005 \\ 
 \hline \\[-1.8ex] 
Observations & \multicolumn{1}{c}{400K} & \multicolumn{1}{c}{400K} & \multicolumn{1}{c}{400K} & \multicolumn{1}{c}{400K} & \multicolumn{1}{c}{400K} & \multicolumn{1}{c}{400K} & \multicolumn{1}{c}{400K} & \multicolumn{1}{c}{400K}  \\ 
R$^{2}$ & 0.033 & 0.027 & 0.052 & 0.020 & 0.052 & 0.024 & 0.028 & 0.041 \\ 
Adjusted R$^{2}$ & 0.032 & 0.027 & 0.052 & 0.020 & 0.052 & 0.024 & 0.028 & 0.041 \\ 
Residual SE & 0.079 & 0.091 & 0.057 & 0.072 & 0.056 & 0.119 & 0.080 & 0.286 \\ 
F Statistic & \multicolumn{1}{r}{961$^{***}$} & \multicolumn{1}{r}{806$^{***}$} & \multicolumn{1}{r}{1,567$^{***}$} & \multicolumn{1}{r}{572$^{***}$} & \multicolumn{1}{r}{1,574$^{***}$} & \multicolumn{1}{r}{689$^{***}$} & \multicolumn{1}{r}{829$^{***}$} & \multicolumn{1}{r}{1,232$^{***}$} \\ 
\bottomrule
  & \multicolumn{8}{r}{$^{*}$p$<$0.05; $^{**}$p$<$0.01; $^{***}$p$<$0.001} \\ 
\end{tabular} }
\end{table*} 



% Table created by stargazer v.5.2.3 by Marek Hlavac, Social Policy Institute. E-mail: marek.hlavac at gmail.com
% Date and time: Mon, Feb 03, 2025 - 04:15:41 PM
\begin{table*}[!htbp] \centering 
  \caption{Regression results for the relationship between binary ideology (liberal vs. conservative), ideology strength, and metaphoricity scores, controlling for issue-generic policy frames.} 
  \label{tab:ideology_effect_frames} 
\resizebox{\textwidth}{!}{%
\begin{tabular}{@{\extracolsep{-18pt}}lLLLLLLLL} 
\toprule 
\\[-2.8ex] & \multicolumn{1}{c}{animal} & \multicolumn{1}{c}{commodity} & \multicolumn{1}{c}{parasite} & \multicolumn{1}{c}{pressure} & \multicolumn{1}{c}{vermin} & \multicolumn{1}{c}{war} & \multicolumn{1}{c}{water} & \multicolumn{1}{c}{overall}\\ 
\hline \\[-1.8ex]
    ideology & 0.009^{***} & 0.008^{***} & 0.009^{***} & 0.009^{***} & 0.007^{***} & 0.011^{***} & 0.013^{***} & 0.077^{***} \\ 
    strength & 0.0005 & -0.001^{**} & 0.001^{**} & -0.0000 & 0.0003 & -0.0002 & -0.001^{***} & -0.006^{***} \\ 
    ideology:strength & 0.002^{***} & 0.003^{***} & 0.002^{***} & 0.003^{***} & 0.002^{***} & 0.003^{***} & 0.002^{***} & 0.021^{***} \\ 
    \hline
    crime & 0.025^{***} & 0.007^{***} & 0.007^{***} & -0.004^{***} & 0.011^{***} & 0.013^{***} & -0.007^{***} & 0.017^{***} \\ 
    cultural & -0.014^{***} & -0.004^{***} & -0.005^{***} & -0.014^{***} & -0.009^{***} & 0.005^{***} & -0.009^{***} & 0.018^{***} \\ 
    economic & -0.001^{***} & 0.039^{***} & 0.0001 & 0.017^{***} & -0.004^{***} & -0.012^{***} & 0.007^{***} & 0.050^{***} \\ 
    fairness & -0.005^{***} & -0.016^{***} & 0.004^{***} & -0.008^{***} & -0.002^{***} & 0.004^{***} & -0.015^{***} & -0.032^{***} \\ 
    health & 0.007^{***} & -0.011^{***} & 0.017^{***} & 0.004^{***} & 0.015^{***} & 0.009^{***} & -0.004^{***} & 0.035^{***} \\ 
    legality & -0.009^{***} & -0.004^{***} & -0.009^{***} & -0.011^{***} & -0.010^{***} & -0.009^{***} & -0.007^{***} & -0.024^{***} \\ 
    morality & 0.026^{***} & 0.010^{***} & 0.014^{***} & 0.006^{***} & 0.013^{***} & 0.019^{***} & -0.001^{**} & 0.001 \\ 
    policy & -0.018^{***} & -0.018^{***} & -0.024^{***} & -0.002^{***} & -0.021^{***} & -0.029^{***} & -0.010^{***} & -0.029^{***} \\ 
    political & -0.001^{*} & -0.011^{***} & 0.006^{***} & 0.006^{***} & -0.0002 & -0.007^{***} & 0.002^{***} & -0.017^{***} \\ 
    security & 0.024^{***} & 0.007^{***} & 0.009^{***} & 0.028^{***} & 0.016^{***} & 0.083^{***} & 0.030^{***} & 0.137^{***} \\ 
    hashtag & 0.002^{***} & -0.004^{***} & 0.007^{***} & 0.007^{***} & 0.012^{***} & 0.003^{***} & 0.009^{***} & -0.040^{***} \\ 
    mention & -0.003^{***} & -0.002^{***} & -0.003^{***} & -0.004^{***} & -0.001^{**} & -0.008^{***} & 0.005^{***} & -0.020^{***} \\ 
    url & -0.013^{***} & -0.010^{***} & -0.014^{***} & -0.009^{***} & -0.014^{***} & -0.004^{***} & -0.014^{***} & 0.032^{***} \\ 
    quote status & 0.0004 & -0.002^{***} & 0.006^{***} & -0.001 & 0.003^{***} & -0.005^{***} & -0.006^{***} & -0.032^{***} \\ 
    reply & -0.001^{**} & -0.002^{***} & 0.005^{***} & -0.001^{**} & 0.0002 & -0.006^{***} & -0.017^{***} & -0.026^{***} \\ 
    verified & -0.005^{***} & -0.003^{**} & -0.007^{***} & -0.005^{***} & -0.006^{***} & -0.009^{***} & 0.003^{***} & -0.029^{***} \\ 
    log chars & -0.019^{***} & -0.022^{***} & -0.017^{***} & -0.011^{***} & -0.014^{***} & -0.024^{***} & -0.001^{*} & -0.017^{***} \\ 
    log followers & -0.001^{***} & -0.0001 & -0.0001 & 0.0000 & -0.0004^{***} & -0.001^{***} & -0.004^{***} & -0.007^{***} \\ 
    log following & 0.0003^{*} & -0.0001 & 0.0001 & -0.0002 & 0.0003^{***} & 0.001^{**} & 0.003^{***} & 0.005^{***} \\ 
    log statuses & 0.002^{***} & 0.001^{***} & 0.001^{***} & 0.001^{***} & 0.001^{***} & 0.002^{***} & 0.004^{***} & 0.011^{***} \\ 
    year:month & 0.000 & 0.000 & 0.000^{***} & -0.000^{***} & 0.000^{***} & 0.000^{***} & -0.000 & 0.000^{***} \\ 
    Constant & 0.203^{***} & 0.203^{***} & 0.156^{***} & 0.135^{***} & 0.146^{***} & 0.199^{***} & 0.085^{***} & 0.057^{***} \\ 
 \hline \\[-1.8ex] 
Observations & \multicolumn{1}{c}{400K} & \multicolumn{1}{c}{400K} & \multicolumn{1}{c}{400K} & \multicolumn{1}{c}{400K} & \multicolumn{1}{c}{400K} & \multicolumn{1}{c}{400K} & \multicolumn{1}{c}{400K} & \multicolumn{1}{c}{400K}  \\ 
$R^{2}$ & 0.102 & 0.076 & 0.125 & 0.069 & 0.137 & 0.124 & 0.063 & 0.092 \\ 
Adjusted $R^{2}$ & 0.102 & 0.076 & 0.125 & 0.069 & 0.137 & 0.124 & 0.063 & 0.092 \\ 
Residual SE & 0.077 & 0.089 & 0.055 & 0.070 & 0.053 & 0.113 & 0.079 & 0.278 \\ 
F Statistic  & \multicolumn{1}{r}{1,887$^{***}$} & \multicolumn{1}{r}{1,369$^{***}$} & \multicolumn{1}{r}{2,375$^{***}$} & \multicolumn{1}{r}{1,236$^{***}$} & \multicolumn{1}{r}{2,637$^{***}$} & \multicolumn{1}{r}{2,365$^{***}$} & \multicolumn{1}{r}{1,129$^{***}$} & \multicolumn{1}{r}{1,686$^{***}$} \\ 
\bottomrule
& \multicolumn{8}{r}{$^{*}$p$<$0.05; $^{**}$p$<$0.01; $^{***}$p$<$0.001}
\end{tabular} }
\end{table*} 



% Table created by stargazer v.5.2.3 by Marek Hlavac, Social Policy Institute. E-mail: marek.hlavac at gmail.com
% Date and time: Mon, Feb 03, 2025 - 05:03:11 PM
\begin{table}[!htbp] \centering 
  \caption{Regression results for the relationship between metaphor scores, ideology, and user engagement (number of favorites and retweets, log-scaled)} 
  \label{tab:engagement} 
\resizebox{\columnwidth}{!}{%
\begin{tabular}{@{\extracolsep{-18pt}}lLL} 
\toprule
\\[-2.8ex] & \multicolumn{1}{c}{favorites} & \multicolumn{1}{c}{retweets} \\  
    \hline \\[-1.8ex]  
    ideology & -0.166^{***} & -0.029^{***} \\ 
    strength & 0.078^{***} & 0.022^{***} \\ 
    ideology:strength & -0.093^{***} & -0.019^{***} \\ 
    \hline
    animal & -0.005 & 0.010^{***} \\ 
    commodity & -0.015^{***} & -0.005 \\ 
    parasite & 0.038^{***} & 0.016^{***} \\ 
    pressure & -0.007 & 0.006 \\ 
    vermin & -0.005 & 0.012^{***} \\ 
    war & -0.015^{***} & 0.0003 \\ 
    water & -0.010^{*} & -0.012^{***} \\ 
    \hline
    animal:ideology & 0.004 & -0.009^{**} \\ 
    commodity:ideology & 0.010^{**} & 0.002 \\ 
    parasite:ideology & -0.034^{***} & -0.012^{**} \\ 
    pressure:ideology & 0.006 & -0.005 \\ 
    vermin:ideology & 0.005 & -0.004 \\ 
    war:ideology & 0.013^{***} & 0.001 \\ 
    water:ideology & 0.018^{***} & 0.016^{***} \\ 
    \hline
    hashtag & -0.048^{***} & -0.019^{***} \\ 
    mention & -0.116^{***} & -0.094^{***} \\ 
    url & -0.307^{***} & -0.163^{***} \\ 
    quote status & -0.011^{*} & -0.071^{***} \\ 
    reply & 0.048^{***} & -0.137^{***} \\ 
    verified & 0.702^{***} & 0.609^{***} \\ 
    log chars & 0.348^{***} & 0.259^{***} \\ 
    log followers & 0.316^{***} & 0.242^{***} \\ 
    log following & -0.120^{***} & -0.087^{***} \\ 
    log statuses & -0.146^{***} & -0.099^{***} \\ 
    year:month &  0.000^{**} & - 0.000 \\ 
    Constant & -0.799^{***} & -0.856^{***} \\ 
 \hline \\[-1.8ex] 
Observations & \multicolumn{1}{c}{400K} & \multicolumn{1}{c}{400K} \\ 
R$^{2}$ & 0.280 & 0.299 \\ 
Adjusted R$^{2}$ & 0.280 & 0.299 \\ 
Residual SE  & 0.872 & 0.701 \\ 
F Statistic  & \multicolumn{1}{r}{5,563$^{***}$} & \multicolumn{1}{r}{6,102$^{***}$} \\ 
\bottomrule
  & \multicolumn{2}{r}{$^{*}$p$<$0.05; $^{**}$p$<$0.01; $^{***}$p$<$0.001} \\ 
\end{tabular} }
\end{table} 


% Table created by stargazer v.5.2.3 by Marek Hlavac, Social Policy Institute. E-mail: marek.hlavac at gmail.com
% Date and time: Mon, Feb 03, 2025 - 05:00:48 PM
\begin{table}[!htbp] \centering 
  \caption{Regression results for the relationship between metaphor scores, ideology, and user engagement (number of favorites and retweets, log-scaled), controlling for issue-generic policy frames.} 
  \label{tab:engagement_frames} 
\resizebox{\columnwidth}{!}{%
\begin{tabular}{@{\extracolsep{-18pt}}lLL} 
\toprule
\\[-2.8ex] & \multicolumn{1}{c}{favorites} & \multicolumn{1}{c}{retweets} \\  
    \hline \\[-1.8ex]  
    ideology & -0.153^{***} & -0.028^{***} \\ 
    strength & 0.074^{***} & 0.020^{***} \\ 
    ideology:strength & -0.088^{***} & -0.017^{***} \\ 
    \hline
    animal & -0.006 & 0.006 \\ 
    commodity & -0.015^{***} & -0.003 \\ 
    parasite & 0.032^{***} & 0.013^{***} \\ 
    pressure & -0.005 & 0.004 \\ 
    vermin & -0.003 & 0.012^{**} \\ 
    war & -0.014^{***} & -0.001 \\ 
    water & -0.006 & -0.008^{*} \\ 
    \hline
    animal:ideology & 0.005 & -0.006^{*} \\ 
    commodity:ideology & 0.008^{*} & -0.0001 \\ 
    parasite:ideology & -0.031^{***} & -0.011^{**} \\ 
    pressure:ideology & 0.005 & -0.004 \\ 
    vermin:ideology & 0.006 & -0.004 \\ 
    war:ideology & 0.014^{***} & 0.003 \\ 
    water:ideology & 0.016^{***} & 0.014^{***} \\ 
    \hline
    crime & -0.003 & 0.029^{***} \\ 
    cultural & 0.029^{***} & 0.005 \\ 
    economic & 0.013^{***} & 0.019^{***} \\ 
    fairness & 0.048^{***} & 0.035^{***} \\ 
    health & -0.012^{***} & 0.018^{***} \\ 
    legality & 0.003 & 0.022^{***} \\ 
    morality & 0.062^{***} & 0.041^{***} \\ 
    policy & -0.002 & 0.011^{***} \\ 
    political & 0.002 & 0.027^{***} \\ 
    security & -0.017^{***} & 0.010^{***} \\ 
    hashtag & -0.048^{***} & -0.021^{***} \\ 
    mention & -0.115^{***} & -0.092^{***} \\ 
    url & -0.304^{***} & -0.165^{***} \\ 
    quote status & -0.016^{**} & -0.074^{***} \\ 
    reply & 0.045^{***} & -0.136^{***} \\ 
    verified & 0.703^{***} & 0.610^{***} \\ 
    log chars & 0.338^{***} & 0.235^{***} \\ 
    log followers & 0.316^{***} & 0.243^{***} \\ 
    log following & -0.121^{***} & -0.087^{***} \\ 
    log statuses & -0.146^{***} & -0.100^{***} \\ 
    year:month &  0.000^{*} &  0.000 \\ 
    Constant & -0.764^{***} & -0.770^{***} \\ 
    \hline \\[-1.8ex] 
Observations & \multicolumn{1}{c}{400K} & \multicolumn{1}{c}{400K} \\ 
R$^{2}$ & 0.281 & 0.300 \\ 
Adjusted R$^{2}$ & 0.281 & 0.300 \\ 
Residual SE  & 0.872 & 0.701 \\ 
F Statistic  & \multicolumn{1}{r}{4,114$^{***}$} & \multicolumn{1}{r}{4,517$^{***}$} \\ 
\bottomrule
  & \multicolumn{2}{r}{$^{*}$p$<$0.05; $^{**}$p$<$0.01; $^{***}$p$<$0.001} \\ 
\end{tabular} }
\end{table} 






