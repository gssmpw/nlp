\subsection{Metaphor's Role in Engagement}
\label{reg2}

We measure associations between metaphor and user engagement (favorites and retweets) and analyze how effects vary across ideologies.


\paragraph{Regression Setup} We fit linear regression models, where dependent variables are \textit{favorite and retweet counts (log-scaled)}. Independent variables include all concepts' \textit{metaphor scores}, \textit{ideology}, \textit{ideology strength}, and interactions between scores and ideology. We control for the same variables as in §\ref{reg1} (including verified status and follower counts, which strongly predict engagement). We also fit models controlling for topic-like frames. %See Appendix Tables \ref{tab:engagement}-\ref{tab:engagement_frames} for all variables.



\begin{figure}[t!]
    \centering
    \includegraphics[width=.8\columnwidth]{plots_feb2025/analysis/retweets_effects.pdf}
    \caption{Average marginal effect of metaphor scores on retweets (log-scaled) for each concept.}
    \label{fig:retweet}
\end{figure}

\begin{figure}[t!]
    \centering
    \includegraphics[width=.8\columnwidth]{plots_feb2025/analysis/retweets_by_ideology.pdf}
    \caption{Group-average marginal effects of metaphor on (log-scaled) retweets separated by ideology.}
    \label{fig:retweet_ideology}
\end{figure}


% \begin{figure*}[htbp!]
%     \centering
%     \includegraphics[width=.75\textwidth]{plots_feb2025/analysis/score_no_frames_with_ideology_2024-10-07_marginal_effects_on_retweets.pdf}
%     \caption{Average marginal effect of metaphor on retweets}
%     \label{fig:retweet}
% \end{figure*}



\paragraph{Results}


%Figure \ref{fig:retweet} shows average marginal effects of metaphor on retweets, with effects separated by ideology in Figure \ref{fig:retweet_ideology} (see Appendix Fig. \ref{fig:favorite} for the plots for favorites, and Figs. \ref{fig:favorite_frames}-\ref{fig:retweet_frames} for results when controlling for frames, and Tables \ref{tab:engagement}-\ref{tab:engagement_frames} for full regression coefficients). 


Aligned with prior evidence that source domains moderate metaphors' effects \citep{bosman_persuasive_1987}, associations between metaphor scores and user engagement vary by concept (Fig. \ref{fig:retweet}). Creature metaphors (\textsc{vermin},\textsc{parasite}, and \textsc{animal}) are associated with more retweets, with the largest effects for liberals (Fig. \ref{fig:retweet_ideology}). Only \textsc{parasite} is significantly associated with more favorites, and metaphors from some domains (e.g. \textsc{commodity}) are actually associated with fewer favorites (Appendix Fig. \ref{fig:favorite}). For both favorites and retweets, the direction of the effect diverges for only one concept: \textsc{water}, which is associated with higher engagement for conservatives, but lower for liberals. See Appendix Figs. \ref{fig:favorite_frames}-\ref{fig:retweet_frames} for results when controlling for frames, and Tables \ref{tab:engagement}-\ref{tab:engagement_frames} for full regression coefficients.



\textbf{H2} is partially supported: creature-related metaphors are linked to higher engagement. Addressing \textbf{RQ2}, the relationship between metaphor and engagement is stronger for liberals than conservatives. Crucially, all regressions reveal that relationship between metaphor, ideology, and engagement depends on metaphors' source domains.   






