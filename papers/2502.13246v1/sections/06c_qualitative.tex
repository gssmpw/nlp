\subsection{How are liberals using metaphors?}

Our quantitative analysis paints a complex picture for liberals' use of dehumanizing metaphors of immigrants.%, which seem more aligned with conservatives' attitudes towards immigrants and policy preferences. 
While conservative ideology is more associated with using these metaphors (§\ref{reg1}), liberals use them to a substantial extent (Appendix Fig. \ref{fig:scores_by_ideology}). Furthermore, more extreme liberal ideology is associated with greater use of creature-related metaphors, %(\textsc{animal}, \textsc{vermin}, \textsc{parasite}), 
and the positive relationship between creature-related metaphors and higher retweet counts is driven by liberals. We thus conduct a qualitative analysis of 25 liberal tweets with the highest scores for each concept. Examples discussed here are shown in Appendix Table \ref{tab:tweets}. 

We identify four themes: 
%\begin{enumerate}[noitemsep,topsep=0pt]
%\item 
{\bf(i.)} Straightforwardly embracing metaphors. Liberals describe migrants as \textit{a source of \$\$\$} (\textsc{commodity}) and \textit{a wave} (\textsc{water}).
%, even though they take pro-immigrant stances.
%\item 
{\bf(ii.)} Sympathetic framing, particularly to highlight humanitarian concerns. For example, a liberal overtly cues the \textsc{animal} metaphor while lamenting that ``they hunt them like animals, they cage them like animals''. Other sympathetic instances refer to \textit{feeding} and \textit{sheltering} immigrants; while these verbs can be used to talk about humans, they deny agency to the recipient and are thus more associated with animals \citep{tipler_agencys_2014}.
%\item 
{\bf(iii.)} Quoting or paraphrasing outpartisans to criticize their use of inflammatory metaphors. 
%This appears especially common with the \textsc{vermin} concept 
(e.g., 
%in Table \ref{tab:tweets}, 
paraphrasing conservative politicians who called immigrants ``rats'' and an ``infestation''.)
%\item 
{\bf(iv.)} Redirecting dehumanizing metaphors from immigrants to political opponents. For example 
talking about ``right-wing forces'' (\textsc{war} metaphor) or referring to Melania Trump (Donald Trump's wife and an immigrant) as a ``tick'' and ``blood sucker''.
%\textsc{war} may be redirected from targeting immigrants to targeting ``right-wing forces'' (Table \ref{tab:tweets}). A common sub-category combines dehumanization of immigrants with out-party dehumanization in targeting Melania Trump (Donald Trump's wife and an immigrant), e.g. referring to her as a ``tick'' and ``blood sucker''.


%\end{enumerate}






