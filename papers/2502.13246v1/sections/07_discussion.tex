\section{Discussion}


More than simply rhetorical decor, metaphors construct and reflect a deeper conceptual structuring of human experiences \citep{lakoff1980metaphors}, and are important devices for political persuasion \citep{mio_metaphor_1997}. While metaphor in politicians' speeches and mass media have been long-studied \citep{charteris-black_britain_2006}, far less is known about how metaphor is used by ordinary people on social media. This theoretical gap is largely driven by a methodological one: measuring metaphorical language at scale is a particularly challenging task.


We develop a computational approach for processing metaphor that uses LLMs and document embeddings to capture both word- and discourse-level signals, which we evaluate on a new dataset of 1600 tweets annotated for metaphor with respect to seven concepts. We apply our approach to analyze dehumanizing metaphor in 400K tweets about immigration, and investigate the relationship between metaphor, political ideology, and user engagement. 

Conservative ideology is associated with greater use of dehumanizing metaphors of immigrants, but varies by source domain. 
%Higher conservative usage suggests that metaphors are used to communicate negative attitudes towards immigrants and promote stricter immigration policies. However, t
This variability and somewhat frequent usage among liberals suggests a high degree of conventionalization in which such metaphors are accepted as ``natural'' \citep{el2001metaphors}. Moreover, compared to moderate liberals, far-left ideology is associated with lower use of objectifying metaphor but higher use of creature metaphor. We conjecture that this pattern may be due to creature metaphors evoking stronger emotions, emphasizing the importance for future metaphor research to consider the role of source domain. Finally, we show that creature-related metaphor is linked to more retweets, with the strongest effects for liberal authors. 
%Without data about who the audience is and who is engaging with this content, it is difficult to draw concrete conclusions about metaphorical framing effects. 
If we assume homophily, i.e., that a tweet's author and audience generally share the same political ideology \citep{barbera2015tweeting}, our results align with prior findings that liberals are more susceptible to the effects of metaphor \citep{hart_riots_2018,sengupta-etal-2024-analyzing}. %suggesting that conservatives and liberals may use metaphor to achieve different goals.



We further qualitatively find that liberals use dehumanizing metaphors to express pro-immigration stances, sympathize, report speech from political opponents, and target outpartisans.
While they may not intend to dehumanize, liberals still tacitly reinforce these metaphors as permissible ways to think and talk about immigrants, with potential ramifications for the treatment of immigrants \citep{el2001metaphors}. Future research could expand our methodology to distinguish between such discursive contexts and examine their social consequences.

Our work offers many avenues for future research. 
%We benefit from decades of literature about metaphorical in U.S. immigration discourse. 
Future work could adapt and evaluate our methodology for other issues, languages, and cultures. While we curate source domains from social science literature, such resources may not be available for lesser-studied contexts. Future work could thus explore developing automated methods for \textit{metaphor discovery}, possibly with the aid of external knowledge graphs and lexical resources \citep{mao_metapro_2022,mao_metapro_2023}. Future analysis-oriented work could use NLP-based measurements of metaphorical language in large-scale experiments to precisely quantify the effects of metaphor on emotions, policy preferences, and social attitudes. 


%One paragraph about liberals use of metaphors, effects of creature metaphros etc. 


%Our previous analyses suggest that the \textsc{water} and \textsc{commodity} metaphors are conventionalized across the ideological spectrum, i.e, accepted as the natural way to talk about immigrants \citep{el2001metaphors}. Metaphors related to these concepts are most favored by political moderates (§\ref{reg1}), and they are not particularly attention-grabbing, as suggested by null-to-negative effects on user engagement (§\ref{reg2}). Yet, their use continues to increase among both liberals and conservatives.


%We consider favorite and retweet metrics as outcome variables. Our data includes the number of favorites and retweets that a tweet receives, but we unfortunately lack access to information about view counts, \textit{who} is exposed to a particular tweet, and \textit{who} engages with each tweet. To understand effects on audiences, we thus rely on information about the author's ideology and assumptions of homophily: audiences exposed to conservative (liberal) tweets are primarily conservative (liberal) \citep{barbera2015tweeting}. 


%In their analysis of the UK Times newspaper from 1800-2018, \citet{taylor_metaphors_2021} find that \textsc{liquid} and \textsc{object} metaphors persisted throughout the entire period, while \textsc{animal}, \textsc{war}, and \textsc{weight} are more recent.


%Taken together, our results suggest that the use of dehumanizing metaphors is not relationship between dehumanizing metaphors and political ideology depends on the specific concepts evoked and is not solely driven by positions on immigration policy. Extreme liberals are more likely to use creature metaphors than moderate liberals, and liberals' tweets are amplified more when they use these dehumanizing creature metaphors.