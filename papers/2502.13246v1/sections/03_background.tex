\section{Background}

According to Conceptual Metaphor Theory, ``the essence of metaphor is understanding and experiencing one thing in terms of another'' \cite[p.~5]{lakoff1980metaphors}. Metaphor 
%plays a major role in political discourse: it 
helps people understand complex political issues in terms of more concrete everyday experiences \citep{burgers_figurative_2016}. Because metaphors highlight some aspects of issues while downplaying others, they 
%can affect how people interpret political issues and are thus 
are a type of framing device \citep{Entman1993,burgers_figurative_2016} %. The conceptual mappings created and reinforced by metaphor establishes pathways for 
with implications for 
policy recommendation and political action \citep{lakoff1980metaphors}.

\paragraph{Immigration Metaphor \& Ideology}

Metaphors of immigrants as \textsc{animals}, \textsc{vermin}, \textsc{objects}, and \textsc{water} have appeared in U.S. immigration debates for centuries \citep{obrien_indigestible_2003,card2022computational}. Metaphorical dehumanization is sometimes overt (e.g. calling immigrants \textit{animals}) \citep{ana_like_1999}, but can be subconscious as some metaphors are conventionalized in immigration discourse (e.g. the \textsc{water} metaphor evoked by \textit{waves of immigration}) \citep{porto_water_2022}. By emphasizing perceived threats from immigrants, dehumanizing metaphors can increase discrimination and harsh immigration policy support \citep{ana_like_1999,utych_how_2018}.

If metaphor is used to promote such policies, we would expect conservatives to use them more than liberals. This would align with conservative Twitter users tending to frame immigrants as threats \citep{mendelsohn2021modeling}, and Republicans' speeches using more dehumanizing metaphor than Democrats' \citep{card2022computational}. However, prior work 
%in corpus-assisted discourse analysis refutes this hypothesis. There are 
finds little-to-no differences between left and right-leaning newspapers' use of immigration metaphors \citep{arcimaviciene_migration_2018,benczes_migrants_2022,porto_water_2022}. Some metaphors, especially \textsc{water}, appear across the ideological spectrum, and are even reinforced by pro-immigration authors \citep{el2001metaphors}.

Beyond binary ideology, ideology strength may impact metaphor use. 
%If such metaphors are used primarily to communicate threat and advocate for stringent immigration policies, 
From an immigration stance perspective, we would expect the highest use for far-right authors, and lowest for far-left authors. However, both extremes use more negative emotional language than moderates \citep{alizadeh2019psychology,frimer2019extremists}. If metaphors communicate such emotions \citep{ortony1975metaphors}, their use may be higher for both extremes than for moderates. %If dehumanizing immigration metaphors are highly conventionalized on social media, we would not expect to see any variation between ideological moderates and hardliners.

\paragraph{Metaphorical Framing Effects}


How metaphor affects audience's attitudes and behaviors remains an open question. Critical discourse analysis asserts that discourse is not just shaped by society, but also actively constructs social realities; from this lens, metaphor inherently has strong effects on social and political systems \citep{charteris-black_britain_2006,boeynaems_effects_2017}. However, quantitative experiments have shown mixed (and sometimes irreplicable) results \citep{thibodeau_metaphors_2011,steen_when_2014,boeynaems_effects_2017,brugman_metaphorical_2019}. Metaphors' effects are vary across factors such as topic, conceptual domains, message source, political orientation, and personality \citep{bosman_persuasive_1987,mio_metaphor_1997,robins_metaphor_2000,kalmoe_fueling_2014,kalmoe_mobilizing_2019,panzeri_does_2021}.

% \looseness=-1 This equivocality is exemplified in replication attempts of a study by \citet{thibodeau_metaphors_2011}, which manipulates whether a news vignette discusses crime with \textsc{virus} or \textsc{beast} metaphors, finding that the latter increases support for harsher policies. 
% However, \citet{steen_when_2014,reijnierse_how_2015} failed to replicate this finding.
%However, \citet{steen_when_2014} find that these effects do not replicate when adding a non-metaphorical control and a pre-exposure measure of political preference. Using the same scenario, \citet{reijnierse_how_2015} conduct two experiments with conflicting results. 
%\citet{steen_when_2014} argues that we ought to not consider \textit{whether or not} metaphorical framing effects occur, but rather \textit{under what conditions} they occur. Metaphors' effects are moderated by factors such as topic, conceptual domains, message source, political orientation, and personality \citep{bosman_persuasive_1987,mio_metaphor_1997,robins_metaphor_2000,kalmoe_fueling_2014,kalmoe_mobilizing_2019,panzeri_does_2021}.



There is experimental evidence of 
immigration metaphor effects: exposure to \textsc{animal} metaphors increase support for immigration restriction %, with the effect mediated by emotions of anger and disgust 
\citep{utych_how_2018}, and exposure \textsc{water} increases border wall support \citep{jimenez2021walls}. Framing the U.S. as a \textsc{body} amplifies effects of contamination threat exposure on negative attitudes towards immigrants \citep{landau_dirt_2014}. %It is unclear if such effects hold outside of lab experiments.%when considering real-world messages on social media.%that dehumanizing metaphors can affect attitudes towards immigrants and immigration policy preferences 
Immigration metaphors' effects are moderated by contextual variables such as intergroup prejudices and ideology \citep{marshall2018scurry,mccubbins_effects_2023}. For example, \textsc{animal} and \textsc{vermin} metaphors increase support for for-profit immigration detention centers, but only among participants with anti-Latino prejudice \citep{mccubbins_effects_2023}. %Political ideology may also moderate metaphorical framing effects. 
Due to policy positions and greater sensitivity to threat and disgust \citep{jost2003political,inbar2009conservatives}, effects may be stronger among conservatives. However, prior work finds that liberals are more susceptible to metaphors' effects \citep{thibodeau_metaphors_2011,hart_riots_2018}. %\citet{hart_riots_2018} makes an argument for \textit{entrenchment}: conservatives have more fixed attitudes and opinions, ``while liberals formulate their views on a more context-dependent basis taking into account local information supplied by texts.''  %\citet{lahav2012ideological} find that threat framing has stronger effects on immigration policy attitudes for liberals compared to conservatives. 

Moreover, recent work has uncovered \textit{resistance to extreme metaphors} among conservatives. Republicans are more opposed to immigrant detention centers when exposed to dehumanizing metaphors \citep{mccubbins_effects_2023}. \citet{hart_animals_2021} find that both \textsc{animal} and \textsc{war} metaphors decrease support for anti-immigration sentiments and policies. \citet{boeynaems_attractive_2023} show that \textsc{criminal} metaphors of refugees push right-wing populist voters' opinions \textit{away} from right-wing immigration stances. Overtly inflammatory metaphors are consciously recognized by audiences, which lead to greater scrutiny of the underlying message
%and ``thus make it easier for the metaphors and their implications to be rejected'' 
\citep{hart_animals_2021}. Due to liberals' sensitivity to metaphor and conservatives' resistance to extreme metaphor, liberals may be more susceptible to the effects of dehumanizing metaphors of immigration.




\paragraph{Computational Metaphor Analysis}

Metaphor processing encompasses detection, interpretation, generation, and application tasks \citep{shutova_models_2010,rai_survey_2020,ge_survey_2023,kohli_cracking_2023}. Most NLP work focuses on detection as a word-level binary classification task \citep{birke2006clustering,steen2010method,mohler2016introducing}, using linguistic features \citep{neuman_metaphor_2013,tsvetkov_metaphor_2014,jang_metaphor_2016}, neural networks \citep{gao_neural_2018,mao_end--end_2019,dankers_being_2020,le_multi-task_2020}, and BERT models  \citep{liu_metaphor_2020,choi_melbert_2021,lin_cate_2021,aghazadeh_metaphors_2022,babieno_miss_2022,li_framebert_2023,li_finding_2024}. Recent work explores detection and generation with LLMs \citep{dankin_can_2022,liu_testing_2022,joseph_newsmet_2023,lai_multilingual_2023,prystawski_psychologically-informed_2023,ichien_large_2024}. Metaphor detection can also support related tasks, such as propaganda, framing, and hate speech detection \citep{huguet_cabot_pragmatics_2020,lemmens_improving_2021,baleato_rodriguez_paper_2023}.



Few NLP studies examine political metaphor. A study of U.S. politicians' Facebook posts finds that metaphor is associated with higher audience engagement \citep{prabhakaran_how_2021}. NLP work on dehumanization highlights metaphors such as \textsc{vermin}, using embedding-based techniques to quantify such associations \citep{mendelsohn2020framework,engelmann_dataset_2024,zhang_beyond_2024}. \citet{sengupta-etal-2024-analyzing} extend a dataset of news editorials and persuasiveness judgments with metaphor annotations, and find that liberals (but not conservatives) judge more metaphorical editorials to be more persuasive, with effects varying across source domains. Focusing on migration, \citet{zwitter_vitez_extracting_2022} create a dataset of metaphors in Slovene news, with \textsc{water} being the most prevalent source domain. \citet{card2022computational} use BERT token probabilities to quantify associations between immigrants and non-human entities, finding that in recent decades, Republicans have been more likely than Democrats to use such metaphors in political speeches.


%Grounded in the discussed linguistics and political communication literature,
Our research questions and hypotheses, guided by the aforementioned scholarship, are as follows:\footnote{%Our study is about dehumanizing metaphor in immigration discourse. 
For brevity, we use the term ``metaphor'' to refer to metaphorical dehumanization of immigrants. }
 \begin{itemize}[noitemsep,topsep=0pt]
    \item \textbf{H1}: Conservative ideology is associated with greater metaphor use.
    \item \textbf{RQ1}: Is extreme ideology more associated with metaphor than moderate ideology? 
    \item \textbf{H2}: Higher metaphor use is associated with more user engagement.
    \item \textbf{RQ2}: How does ideology moderate the relationship between metaphor and engagement?
\end{itemize}


%Metaphor is not merely a superficial rhetorical device, but rather operates at a deeper cognitive level by structuring how we think about both everyday and complex concepts. Despite coming from a different disciplinary tradition, \cite{lakoff1980metaphors} offer a view of metaphor that is strikingly compatible with both the sociological conceptualization of framing \citep{goffman1974frame}, discussing how metaphors ``structure how we perceive, how we think, and what we do'' (p.4). Aligned with the political communication perspective on framing as selective emphasis \citep{Entman1993}, \cite{lakoff1980metaphors} describe how metaphors highlight certain aspects of concepts while downplaying or hiding others. Metaphor can also play a major role in politics: metaphors enable people to understand complex political issues in terms of more concrete everyday experiences \citep{burgers_figurative_2016}. Moreover, the conceptual mappings created and reinforced by metaphor establishes pathways for policy recommendations and political action \citep{lakoff1980metaphors}. \cite{burgers_figurative_2016} further establish how metaphors can fulfill the functions of framing defined by \cite{Entman1993}: problem definition, causal interpretation, moral evaluation, and treatment recommendation, and is well-aligned with the process model of framing \citep{Scheufele1999}. As an example, \cite{burgers_figurative_2016} demonstrate how the \textsc{natural disaster} metaphor of immigration \citep{charteris-black_britain_2006} accomplishes all framing functions: it leads to the interpretation that immigration is bad, immigrants cause harm, and immigration is difficult to control and thus requires even harsher restrictions. 


% Our hypotheses for the relationship between political ideology and dehumanizing metaphors of immigrants can be summarized as follows:

% \begin{itemize}
%     \item Conservatives use more dehumanizing metaphors than liberals.
%     \item Far-right authors use more dehumanizing metaphors than moderate conservatives.
%     \item Far-left authors use more dehumanizing metaphors than moderate liberals.
% \end{itemize}


% \subsection{Immigration Metaphors and Political Ideology}

% There is a large body of critical discourse scholarship focused on the metaphorical framing of immigration, particularly in news media. Metaphors have long been used in communication about immigration as they ``facilitate listeners' grasp of an external, difficult notion of society in terms of a familiar part of life'' \citep{ana_like_1999}. Metaphors of immigration and immigrants as \textsc{animals}, \textsc{vermin}, \textsc{objects}, \textsc{war}, and \textsc{natural disasters} have appeared in U.S. immigration restriction debates for well over a century \citep{obrien_indigestible_2003,card2022computational}. In their analysis of the UK Times newspaper from 1800-2018, \citet{taylor_metaphors_2021} find that \textsc{liquid} and \textsc{object} metaphors persisted throughout the entire period, while \textsc{animal}, \textsc{war}, and \textsc{weight} are more recent. Dehumanization of immigrants through metaphor is sometimes overt (e.g. calling immigrants \textit{animals}) \citep{ana_like_1999}, but is often subconscious as some metaphors are conventionalized in how we describe immigration (e.g. the \textsc{water} metaphor cued by talking about \textit{waves} of immigration) \citep{el2001metaphors,porto_water_2022}. Dehumanizing metaphors of immigrants in particular can play a role in licensing social discrimination and harsh punitive policies \citep{ana_like_1999,arcimaviciene_migration_2018}.


% By emphasizing the perceived threat from immigrants, dehumanizing metaphors facilitate support for harsher and more restrictive immigration policies \citep{utych_how_2018}. If people use metaphor to promote particular policy positions and attitudes, we would expect that conservatives use more metaphorical language than liberals. Aligning with prior computational findings that Republican congresspeople use more dehumanizing metaphors than Democrats \citep{card2022computational} and that conservative Twitter users tend to frame immigrants as threats more than liberals do \citep{mendelsohn2021modeling}, we hypothesize that tweets from conservative authors will have more dehumanizing metaphors than tweets from liberal authors. 

% However, prior corpus-assisted critical discourse analyses suggest a basis upon which to refute this hypothesis. Comparisons of left-leaning and right-leaning press reveal only minor, if any, differences in metaphor use in discussions of immigration in the Hungarian \citep{benczes_migrants_2022}, Spanish \citep{porto_water_2022}, and United States and European Union \citep{arcimaviciene_migration_2018} contexts. Some metaphors, particularly \textsc{water}, are so conventionalized and accepted as the ``natural'' way to talk about immigration that they appear across a wide range of news sources, and are even reinforced in counter-discourses by authors sympathetic to immigrants \citep{el2001metaphors}.

% Metaphor use may also vary between ideological moderates and extremists. If dehumanizing metaphors of immigration are primarily used to communicate threat and advocate for stringent policies, we would expect to see higher usage among far-right authors compared to moderate conservatives. Moreover, extremists on both sides of the political spectrum use more negatively emotional language than moderates \citep{alizadeh2019psychology,frimer2019extremists}, with members of the far-left using even more negative language than members of the far-right \citep{frimer2019extremists}. Both political extremes experience higher amounts of fear and outgroup derogation than moderates \citep{van2015fear}. If metaphor is used to communicate vividness and emotion \citep{ortony1975metaphors}, then we may expect higher metaphor use among both the far-left and far-right compared to moderates. However, we would not expect to see differences across ideology strength if metaphors are primarily conventionalized.


%As with other framing effects, metaphorical framing effects may be amplified or attenuated by a wide range of moderating variables. In the context of immigration, metaphors have been shown to interact with intergroup prejudices in affecting policy support \citep{marshall2018scurry,mccubbins_effects_2023}. For example, the strength of American in-group identification predicts support for stricter immigration policies, but only when participants are exposed to \textsc{vermin} metaphors \citep{marshall2018scurry}. Similarly, \citet{mccubbins_effects_2023} find that the dehumanizing \textsc{animal} and \textsc{vermin} metaphors both increase support for for-profit immigration detention centers, but only among participants with anti-Latino racial prejudice. \citet{mccubbins_effects_2023} further show that these metaphors have varied effects: exposure to the \textsc{animal} condition had enduring effects after several weeks, but not the \textsc{vermin} condition. 



% \subsection{Computational Metaphor Analysis}

% Computational metaphor processing encompasses several tasks: metaphor identification (detection), interpretation (source domain identification), generation, and application \citep{shutova_models_2010,ge_survey_2023}. Metaphor identification has primarily focused on binary classification of whether or not an expression is metaphorical, with popular metaphor detection datasets including LCC \citep{mohler2016introducing}, TroFi \citep{birke2006clustering}, and the VUA metaphor corpus \citep{steen2010method}. Prior work has modeled metaphor using linguistic features, neural networks, and pretrained Transformer models such as BERT \citep{rai_survey_2020,kohli_cracking_2023}. Older feature-based methods relied on part-of-speech clustering \citep{shutova_metaphor_2010}, corpus statistics \citep{neuman_metaphor_2013}, topic transitions \citep{jang_metaphor_2016}, and abstractness and imageability lexicons \citep{tsvetkov_metaphor_2014}. Neural models for metaphor identification include architectures such as graph convolutional networks and bidirectional LSTMs with attention \citep{gao_neural_2018,mao_end--end_2019,dankers_being_2020,le_multi-task_2020}. Within the last few years, BERT-based models have become popular for this tasks and often have special architectures and training procedures to capture intuitions that metaphorical words differ in basic and contextual meanings \citep{liu_metaphor_2020,choi_melbert_2021,lin_cate_2021,aghazadeh_metaphors_2022,babieno_miss_2022,li_framebert_2023,li_finding_2024}. 
% Recent research has explored question-answering and prompt-based approaches for metaphor detection, interpretation, and generation with LLMs \citep{dankin_can_2022,liu_testing_2022,joseph_newsmet_2023,lai_multilingual_2023,prystawski_psychologically-informed_2023,ichien_large_2024}. 

% Metaphor interpretation and generation tasks typically benefit from combining such models with knowledge graphs (e.g., ConceptNet) and lexical resources (e.g., WordNet) \citep{stowe_metaphor_2021, chakrabarty_mermaid_2021,chakrabarty_i_2023,ge_explainable_2022,mao_metapro_2022,mao_metapro_2023}. Some computational metaphor work distinguishes between conventional (everyday) and novel (unconventional or literary) metaphors, as the latter is easier for computational models \citep{haagsma_detecting_2016,do_dinh_weeding_2018,neidlein_analysis_2020}. Prior work has also used metaphor detection as part of multitask learning frameworks. For example, \citet{le_multi-task_2020} uses word sense disambiguation as an auxiliary task to improve metaphor detection. Metaphor detection has itself been used as an auxiliary task to improve propaganda identification \citep{baleato_rodriguez_paper_2023}, political framing and stance detection \citep{huguet_cabot_pragmatics_2020}, and hate speech type and target detection \citep{lemmens_improving_2021}. 


% Comparatively little NLP research has focused on metaphor analysis in political contexts. In a topic-agnostic computational study of metaphors in public Facebook posts by US politicians, \citet{prabhakaran_how_2021} find that metaphor use is associated with higher audience engagement, and that metaphor use increased among Democrats following their loss in the 2016 US presidential election. Other work on detecting dehumanizing language highlight the use of metaphors such as \textsc{vermin} and \textsc{animal}, and propose embedding-based techniques to quantify these metaphorical associations \citep{mendelsohn2020framework,engelmann_dataset_2024,zhang_beyond_2024}. \citet{zwitter_vitez_extracting_2022} develop an annotation scheme for identifying metaphors in Slovene migration discourse, and find \textsc{liquid}, \textsc{container}, and \textsc{natural phenomenon} metaphors to be the most prevalent. In their study of political speeches about immigration, \citet{card2022computational} use masked language model predictions from BERT to quantify associations between immigrants and dehumanizing metaphors (e.g. \textsc{animal} and \textsc{cargo}), and find that modern Republicans are significantly more likely than Democrats to use such metaphors in the past two decades (concepts considered:  “animals,” “cargo,” “disease,” “flood/tide,” “machines,” and “vermin”). Also higher use of dehumanizing metaphors when talking about Mexican immigrants vs. European immigrants.



% Our research questions and hypotheses require a reliable and consistent methodology to measure metaphorical language at scale. While there is substantial prior NLP research on metaphor identification, it primarily focuses on binary classification of whether or not an expression is metaphorical using existing benchmark datasets \citep{shutova_metaphor_2010,ge_survey_2023,steen2010method,gao_neural_2018,choi_melbert_2021}. Such models are less suitable for our goals: they are trained on neither political nor social media data and do not automatically identify relevant source domain concepts.
