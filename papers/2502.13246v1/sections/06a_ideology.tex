\subsection{Ideology's Role in Metaphor}
\label{reg1}

We quantify the role of ideology in metaphor use with a set of linear regression models. 




\paragraph{Regression Setup} Dependent variables are metaphor scores for each concept. Fixed effects include a binary \textit{ideology} score and a continuous \textit{ideology strength} score (group-mean centered and z-score normalized), and the interaction between these two variables. These variables are derived from the original ideology estimates from \citet{mendelsohn2021modeling}. Breaking this score to separately account for direction (\textit{ideology}) and magnitude (\textit{ideology strength}) allows us to draw conclusions about the far-left, far-right, and moderates.

We control for message, author, and time variables (e.g., tweet length, follower count, year and month) as fixed effects. For robustness, we additionally specify models that control for frames already included in the Immigration Tweets Dataset.\footnote{We control for ten issue-generic policy frames\\(e.g., \textit{Security \& Defense}) that were detected by RoBERTa sufficiently well (F1 > 0.6) from \citet{mendelsohn2021modeling}.} See regression tables (\ref{tab:ideology_effect}-\ref{tab:ideology_effect_frames}) for all included variables. We assess statistical significance at the $p=0.05$ level after applying Holm-Bonferroni corrections to account for multiple comparisons.



\begin{figure}[t]
    \centering
    \includegraphics[width=.8\columnwidth]{plots_feb2025/analysis/conservative_no_frames.pdf}
    \caption{Marginal effect of conservative ideology on metaphor scores, estimated from regression models.}
    \label{fig:ideology}
\end{figure}

\begin{figure}
    \centering
    \includegraphics[width=.8\columnwidth]{plots_feb2025/analysis/strength_no_frames.pdf}
    \caption{Marginal effect of ideology strength on metaphor scores, estimated from regression models.}
    \label{fig:strength}
\end{figure}

% \begin{figure*}[htbp!]
%     \centering
%     \includegraphics[width=.75\textwidth]{plots_feb2025/analysis/conservative_and_strength_no_frames.pdf}
%     \caption{Average marginal effect of conservative ideology (left) and ideological strength (right) on metaphoricity scores, estimated from linear regression models.}
%     \label{fig:ideology_strength}
% \end{figure*}


\paragraph{Results}

Figures \ref{fig:ideology} and \ref{fig:strength} shows average marginal effects of \textit{conservative ideology} on metaphor scores and group average marginal effects of \textit{ideology strength} for liberals and conservatives, respectively. We visualize marginal effects for ease of interpretability due to the presence of interaction terms. Full regression results are in Appendix Table \ref{tab:ideology_effect}.

\textbf{H1} is supported: conservative ideology is significantly associated with higher scores for all seven concepts. We observe variation across concepts: conservative ideology is most strongly associated with \textsc{war} and \textsc{water}, and least with creature-related metaphors (\textsc{parasite}, \textsc{vermin}, \textsc{animal}).




Addressing \textbf{RQ2}, the relationship between ideology strength and metaphor differs for liberals and conservatives (Fig. \ref{fig:strength}). Among conservatives, extremism is associated with more metaphor across all concepts. For liberals, however, the relationship between strength and metaphor depends on the concept. Extreme liberal ideology is associated with lower use of \textsc{water} and \textsc{commodity} but higher use of creature-related metaphors (\textsc{parasite}, \textsc{vermin}, \textsc{animal}). In sum, both extremes are associated with greater use of creature-related metaphor, but only stronger conservative ideology is associated with greater use of object-related metaphor. These findings hold when controlling for topical frames (Appendix Table \ref{tab:ideology_effect_frames} and Fig. \ref{fig:ideology_strength_frames}).

 
%This association is weakest for objects in motion (\textsc{water} and \textsc{commodity}), suggesting that these metaphors are relatively more favored by moderate conservatives. 









