\section{Ethical Implications}

We hope this work has positive impact in drawing attention to metaphorical dehumanization, an often unnoticed form of xenophobic discrimination. Our analysis reveals that the dehumanization of immigrants is not limited to the political right. Rather, we all have a responsibility to be aware of dehumanizing metaphors and their implicit societal implications, especially in our own language.

Our primary ethical concerns relate to our reporting of dehumanizing metaphors. Even though we clearly do not endorse these dehumanizing metaphors, merely exposing them to annotators and readers risks reinforcing harmful conceptual associations. Merely reporting others' use of slurs can still harm members of targeted communities \citep{croom2011slurs}. It remains an open question if reporting dehumanizing metaphor (even to vehemently disagree with their premise) has similar effects as straightforward usage.  

We recruited hundreds of annotators to help us create our evaluation dataset. The study was deemed exempt by the \textit{Anonymized University} Institutional Review Board and annotators were fairly compensated at an average rate of \$16/hour. Nevertheless, creating this dataset involved exposing annotators to offensive and hateful social media posts. We attempt to mitigate these harms by flagging the task as sensitive on Prolific, warning participants of its potentially harmful nature, and limiting each task to just 20 tweets, of which only a few are typically overtly hateful.