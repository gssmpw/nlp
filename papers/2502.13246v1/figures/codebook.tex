\begin{table*}[htbp!]
\begin{tcolorbox}[colback=white, colframe=white!30!black, boxrule=1pt, arc=4mm, width=\textwidth, boxrule=1pt,title=Codebook,fontupper=\footnotesize]

For each tweet and given concept, label whether or not the tweet evokes metaphors related to the given concept. Focus on the author's language, not their stance towards immigration.\\

To determine if specific words or phrases are metaphorical, consider whether the most basic meaning is related to the listed source domain concept. Basic meanings tend to be more concrete (easier to understand, imagine, or sense) and precise (rather than vague). If you’re not sure about a word’s basic meaning, use the first definition in the dictionary as a proxy. A word should be considered metaphorical if it’s relevant to the listed concept (e.g. animal); it need not exclusively apply to that concept. \\

\textbf{Animal}: Immigrants are sometimes talked about as though they are animals such as beasts, cattle, sheep, and dogs. Label whether or not each tweet makes an association between immigration/immigrants and animals. Label ``YES'' if:
\begin{itemize}[noitemsep, topsep=0pt]
    \item The author uses any words or phrases that are usually used to describe animals. Common examples: \textit{attack}, \textit{flock}, \textit{hunt}, \textit{trap}, \textit{cage}, \textit{breed}
    \item Even if you cannot pinpoint specific words that evoke the concept of animals, if the author's language reminds you of how people talk about animals
\end{itemize}

\textbf{Vermin}: Vermin are small animals that spread diseases and destroy crops, livestock, or property, such as rats, mice, and cockroaches. Vermin are often found in large groups. Label whether or not each tweet makes an association between immigration/immigrants and vermin. Label ``YES'' if:
\begin{itemize}[noitemsep, topsep=0pt]
    \item The author uses any words or phrases that are usually used to describe vermin. Common examples: \textit{infesting}, \textit{swarming}, \textit{dirty}, \textit{diseased}, \textit{overrun}, \textit{plagued}, \textit{virus}
    \item Even if you cannot pinpoint specific words that evoke the concept of vermin, if the author's language reminds you of how people talk about vermin
\end{itemize}


\textbf{Parasite}: Parasites are organisms that feed off a host species at the host’s expense, such as leeches, ticks, fleas, and mosquitoes. Label whether or not each tweet makes an association between immigration/immigrants and parasites. Label “YES” if: 
\begin{itemize}[noitemsep, topsep=0pt]
    \item The author uses any words or phrases that are usually used to describe parasites. Common examples: \textit{leeching}, \textit{freeloading}, \textit{sponging}, \textit{mooching}, \textit{bleed dry}
    \item Even if you cannot pinpoint specific words that evoke the concept of parasites, if the author's language reminds you of how people talk about parasites
\end{itemize}


\textbf{Water}: Immigrants are sometimes talked about using language commonly reserved for water (or liquid motion more broadly). For example, people may talk about immigrants pouring, flooding, or streaming across borders, or refer to waves, tides, and influxes of immigration. Label whether or not each tweet makes an association between immigration/immigrants and water. Label “YES” if: 

\begin{itemize}[noitemsep, topsep=0pt]
    \item The author uses any words or phrases that are usually used to describe water. Common examples: \textit{pouring}, \textit{flooding}, \textit{flowing}, \textit{drain}, \textit{spillover}, \textit{surge}, \textit{wave}
    \item Even if you cannot pinpoint specific words that evoke the concept of water, if the author's language reminds you of how people talk about water
\end{itemize}


\textbf{Commodity}: Commodities are economic resources or objects that are traded, exchanged, bought, and sold. Label whether or not each tweet makes an association between immigration/immigrants. Label ``YES'' if:
\begin{itemize}[noitemsep, topsep=0pt]
    \item The author uses any words or phrases that are usually used to describe commodities. Common examples: \textit{sources of labor}, \textit{undergoing processing}, \textit{imports}, \textit{exports}, \textit{tools}, \textit{being received or taken in}, \textit{distribution}
    \item Even if you cannot pinpoint specific words that evoke the concept of commodities, if the author's language reminds you of how people talk about commodities
\end{itemize}


\textbf{Pressure}: Immigration is sometimes talked about as a physical pressure placed upon a host country, especially as heavy burdens, crushing forces, or bursting containers. Label whether or not each tweet makes an association between immigration/immigrants and physical pressure. Label “YES” if: 
\begin{itemize}[noitemsep, topsep=0pt]
    \item The author uses any words or phrases that are usually used to describe physical pressure. Common examples: host country \textit{crumbling}, \textit{bursting}, being \textit{crushed}, \textit{stretched thin}, or \textit{strained}, immigrants as \textit{burdens}.
    \item Even if you cannot pinpoint specific words that evoke the concept of pressure, does the author’s language remind you of how people talk about physical pressure? 
\end{itemize}

\textbf{Commodity}: Commodities are economic resources or objects that are traded, exchanged, bought, and sold. Label whether or not each tweet makes an association between immigration/immigrants. Label ``YES'' if:
\begin{itemize}[noitemsep, topsep=0pt]
    \item The author uses any words or phrases that are usually used to describe commodities. Common examples: \textit{sources of labor}, \textit{undergoing processing}, \textit{imports}, \textit{exports}, \textit{tools}, \textit{being received or taken in}, \textit{distribution}
    \item Even if you cannot pinpoint specific words that evoke the concept of commodities, if the author's language reminds you of how people talk about commodities
\end{itemize}

\textbf{War}: People sometimes talk about immigration in terms of war, where immigrants are viewed as an invading army that the host country fights against. Label whether or not each tweet makes an association between immigration/immigrants and war. Label “YES” if:
\begin{itemize}[noitemsep, topsep=0pt]
    \item The author uses any words or phrases that are usually used to describe war. Common examples: \textit{invasion}, \textit{soldiers}, \textit{battle}, \textit{shields}, \textit{fighting}
    \item Even if you cannot pinpoint specific words that evoke the concept of war, if the author's language reminds you of how people talk about war
\end{itemize}

\textbf{Domain-Agnostic}: Label whether or not each tweet uses metaphorical (non-literal) language to talk about immigration/immigrants. Metaphorical language involves talking about immigration/immigrants in terms of an otherwise unrelated concept. For example, \textit{waves of immigration} is metaphorical because the word \textit{waves} is associated with water.

\end{tcolorbox}
\end{table*}
