\section{General Diffusion Policies}\label{appdx:gbs}
DIME's maximum entropy reinforcement learning framework for training diffusion policies is not specifically restricted to denoising diffusion policies but can be extended to general diffusion policies. This can be realized using the General Bridges framework as presented in \cite{richterimproved}.
In this case, we can write the forward and backward process as 
\begin{align}
    \dd \ac_t = \left[f(\ac_t, t) + \beta u(\ac_t,\st, t)\right]\dd t + \sqrt{2\beta_t}\dd B_t, \quad \quad a_0 \sim \fpi_0(\cdot|\st), \\
    \dd \ac_t = \left[f(\ac_t, t) - \beta v(\ac_t,\st, t)\right]\dd t + \sqrt{2\beta_t}\dd B_t, \quad\quad \ac_T \sim \mathcal{N}(0,I),
\end{align}
with the drift and control functions $f, u, v: \R^d \times [0,T] \rightarrow \R^d$, the diffusion coefficient $\beta: [0,T] \rightarrow \R^+$, standard Brownian motion $(B_t)_{t\in[0,T]}$ and some target policy $\fpi_0$. Again we denote the marginal density of the forward process as $\fpi_t$ and the conditional density at time $t$ given $l$ as $\fpi_{t|l}$ for $t,l\in[0,T]$.
The backward process starts from $\bpi_T=\fpi_T\sim\mathcal{N}(0,I)$ and runs backward in time where we denote its density as $\bpi$.

The respective discretization using the Euler Maruyama (EM) \cite{sarkka2019applied} method are given by 
\begin{align}
    \ac^{n+1} = \ac^n +\left[f(\ac^n, n) + \beta u(\ac^n, \st, n)\right]\delta +\epsilon_n, \\
    \ac^{n-1} = \ac^n - \left[f(\ac^n, n) - \beta v(\ac^n,\st, n)\right]\delta + \xi_n,
\end{align}
where $\epsilon_n,\xi_n\sim\mathcal{N}(0,2\beta\delta I)$, with the constant discretization step size $\delta$ such that $N=T/\delta$ is an integer. We have used the simplified notation where we write $\ac^n$ instead of $\ac^{n\delta}$. The discretizations admit the joint distributions

\begin{align}
\fpi_{0:N}(\ac^{0:N}|\st) = \pi_0(\ac^0|s) \prod_{n=0}^{N-1}\fpi_{n+1|n}(\ac^{n+1} \big| \ac^{n},\st), \\
\bpi_{0:N}(\ac^{0:N}|\st) = \bpi_N (\ac^N|s) \prod_{n=1}^{N}\bpi_{n-1|n}(\ac^{n-1} \big| \ac^{n},\st),
\end{align}
with Gaussian kernels
\begin{align}
    \fpi_{n+1|n}(\ac^{n+1} \big| \ac^{n},\st) = \mathcal{N}(\ac^{n+1}| \ac^n +\left[f(\ac^n, n) + \beta u(\ac^n, \st, n)\right]\delta, 2\beta\delta I)\\
    \bpi_{n-1|n}( \ac^{n-1} \big| \ac^{n},\st) = \mathcal{N}(\ac^{n-1}|\ac^{n} - \left[f(\ac^{n}, n) - \beta v(\ac^{n}, \st, n)\right]\delta, 2\beta\delta I)
\end{align}


Following the same framework presented in the main text, we can now optimize the controls $u$ and $v$ using the same objective
\begin{equation}
\bar{\mathcal{L}}(u,v) = D_{\text{KL}}\left(\bpi_{0:N}(\ac^{0:N}|\st)|\fpi_{0:N}(\ac^{0:N}|\st)\right),
\end{equation}

where the target policy at time step $n=0$ is given as 
\begin{equation}
    \pi_{0}(\ac^{0}|\st) = \frac{\exp Q^{\bpi}(\st, \ac^0)}{\Z^{\bpi}(\st)}.
\end{equation}

In practice, we optimize the control functions $u$ and $v$ using parameterized neural networks. We have run preliminary results using the general bridge framework within the maximum entropy objective as suggested in our work. The learning curves can be seen in Fig. \ref{fig::appendix::prel::dbs}.



\begin{figure*}[t!]
    \centering
    \begin{minipage}[b]{0.33\textwidth}
        \centering
       \resizebox{1\textwidth}{!}{% This file was created with tikzplotlib v0.10.1.
\begin{tikzpicture}

\definecolor{darkcyan1115178}{RGB}{1,115,178}
\definecolor{darkgray176}{RGB}{176,176,176}
\begin{axis}[
legend cell align={left},
legend cell align={left},
legend style={fill opacity=0.8, draw opacity=1, text opacity=1, draw=lightgray204, at={(0.5,0.03)},  anchor=south west},
tick align=outside,
tick pos=left,
x grid style={white},
xlabel={Number Env Interactions},
xmajorgrids,
%xmin=-37798.95, xmax=1045799.95,
xmin=-0.0, xmax=1000000.0,
xtick style={color=black},
y grid style={white},
ylabel={IQM Mean Return},
ymajorgrids,
ymin=-0.05, ymax=650,
ytick style={color=black},
axis background/.style={fill=plot_background},
label style={font=\large},
tick label style={font=\large},
x axis line style={draw=none},
y axis line style={draw=none},
]

\path [draw=C1, fill=C1, opacity=0.2]
(axis cs:1,6.99064241375)
--(axis cs:1,5.53621398333333)
--(axis cs:24000,4.28067333333333)
--(axis cs:48000,5.23071248333333)
--(axis cs:72000,11.5755848333333)
--(axis cs:96000,25.6408245)
--(axis cs:120000,25.5247493333333)
--(axis cs:144000,49.0188601666667)
--(axis cs:168000,55.8191148333333)
--(axis cs:192000,64.692345)
--(axis cs:216000,74.227769)
--(axis cs:240000,76.6928327708333)
--(axis cs:264000,88.1837003333333)
--(axis cs:288000,100.492174333333)
--(axis cs:312000,104.722368833333)
--(axis cs:336000,131.590808666667)
--(axis cs:360000,119.437102833333)
--(axis cs:384000,163.30507)
--(axis cs:408000,150.17637)
--(axis cs:432000,179.395848333333)
--(axis cs:456000,195.124558333333)
--(axis cs:480000,203.97069)
--(axis cs:504000,212.606669416667)
--(axis cs:528000,226.329041666667)
--(axis cs:552000,208.9856615)
--(axis cs:576000,238.283675)
--(axis cs:600000,254.046571666667)
--(axis cs:624000,275.016495)
--(axis cs:648000,272.557551666667)
--(axis cs:672000,303.745753333333)
--(axis cs:696000,295.1025)
--(axis cs:720000,328.943138333333)
--(axis cs:744000,345.448668333333)
--(axis cs:768000,330.267365)
--(axis cs:792000,356.852763333333)
--(axis cs:816000,354.461231666667)
--(axis cs:840000,386.447986666667)
--(axis cs:864000,322.969651)
--(axis cs:888000,397.20823375)
--(axis cs:912000,401.222951666667)
--(axis cs:936000,442.04175)
--(axis cs:960000,432.004991666667)
--(axis cs:984000,480.562826666667)
--(axis cs:984000,558.374366666667)
--(axis cs:984000,558.374366666667)
--(axis cs:960000,525.832104583333)
--(axis cs:936000,529.901533333333)
--(axis cs:912000,513.201521666667)
--(axis cs:888000,474.12895)
--(axis cs:864000,469.702438333333)
--(axis cs:840000,496.014085)
--(axis cs:816000,444.1255905)
--(axis cs:792000,442.521501666667)
--(axis cs:768000,455.044088333333)
--(axis cs:744000,428.69203)
--(axis cs:720000,437.708196666667)
--(axis cs:696000,413.18612)
--(axis cs:672000,388.057755)
--(axis cs:648000,380.312158333333)
--(axis cs:624000,372.099101666667)
--(axis cs:600000,358.568943333333)
--(axis cs:576000,347.241066666667)
--(axis cs:552000,294.06811)
--(axis cs:528000,304.411563333333)
--(axis cs:504000,286.250968333333)
--(axis cs:480000,280.257595)
--(axis cs:456000,248.37012)
--(axis cs:432000,253.453026666667)
--(axis cs:408000,234.7783)
--(axis cs:384000,224.984858333333)
--(axis cs:360000,200.886195)
--(axis cs:336000,168.093413333333)
--(axis cs:312000,178.664308291667)
--(axis cs:288000,161.485955)
--(axis cs:264000,149.313381666667)
--(axis cs:240000,141.529133333333)
--(axis cs:216000,126.745533333333)
--(axis cs:192000,104.314646666667)
--(axis cs:168000,102.2518335)
--(axis cs:144000,87.278177)
--(axis cs:120000,75.0973646666667)
--(axis cs:96000,41.588288)
--(axis cs:72000,32.1586941666667)
--(axis cs:48000,15.5923155745834)
--(axis cs:24000,8.6525861)
--(axis cs:1,6.99064241375)
--cycle;

\addplot [line width =\linewidthdime, C1, mark=*, mark size=0, mark options={solid}]
table {%
1 6.26247963333333
24000 5.7611181
48000 8.44131896666667
72000 20.1185598333333
96000 33.835471
120000 43.3480071666667
144000 64.3745236666667
168000 74.728139
192000 82.733837
216000 97.4774865
240000 106.443479166667
264000 118.875678333333
288000 123.44706
312000 141.0412925
336000 146.003979333333
360000 161.940301666667
384000 187.447373333333
408000 183.521465
432000 211.75379
456000 215.336466666667
480000 240.42844
504000 243.661463333333
528000 260.393455
552000 269.220865
576000 283.278935
600000 304.192728333333
624000 322.943031666667
648000 327.473726666667
672000 345.551548333333
696000 348.384485
720000 385.68403
744000 388.59085
768000 398.13177
792000 410.87024
816000 391.699205
840000 445.783146666667
864000 432.561828333333
888000 447.10136
912000 459.732565
936000 485.271595
960000 482.201818333333
984000 528.51851
};
\addlegendentry{GB}
%DIME
\path [draw=C0, fill=C0, opacity=0.2]
(axis cs:1,6.80215591666667)
--(axis cs:1,5.5925808)
--(axis cs:24000,4.88713573333333)
--(axis cs:48000,9.748871)
--(axis cs:72000,39.178405)
--(axis cs:96000,35.577766)
--(axis cs:120000,66.3638765)
--(axis cs:144000,70.247577)
--(axis cs:168000,107.848963333333)
--(axis cs:192000,120.425612666667)
--(axis cs:216000,131.606081666667)
--(axis cs:240000,126.6783995)
--(axis cs:264000,159.943133333333)
--(axis cs:288000,172.044628333333)
--(axis cs:312000,188.191985)
--(axis cs:336000,197.75034)
--(axis cs:360000,215.573404333333)
--(axis cs:384000,244.38039)
--(axis cs:408000,265.35188)
--(axis cs:432000,256.659486666667)
--(axis cs:456000,254.555695)
--(axis cs:480000,321.560075)
--(axis cs:504000,309.784411666667)
--(axis cs:528000,319.716291666667)
--(axis cs:552000,342.35053)
--(axis cs:576000,357.640175)
--(axis cs:600000,365.597093333333)
--(axis cs:624000,379.492581666667)
--(axis cs:648000,391.800683333333)
--(axis cs:672000,439.1368195)
--(axis cs:696000,432.521263333333)
--(axis cs:720000,458.362536666667)
--(axis cs:744000,480.79676)
--(axis cs:768000,485.700503333333)
--(axis cs:792000,483.583248791667)
--(axis cs:816000,485.992758916667)
--(axis cs:840000,517.541645)
--(axis cs:864000,527.38718)
--(axis cs:888000,525.263277708333)
--(axis cs:912000,533.699495)
--(axis cs:936000,560.338303333333)
--(axis cs:960000,519.453665)
--(axis cs:984000,568.020833333333)
--(axis cs:1008000,590.470613333333)
--(axis cs:1032000,581.025416666667)
--(axis cs:1056000,583.688265)
--(axis cs:1080000,607.528891666667)
--(axis cs:1104000,620.278735)
--(axis cs:1128000,613.026586666667)
--(axis cs:1152000,638.844431666667)
--(axis cs:1176000,637.409991666667)
--(axis cs:1200000,641.941126666667)
--(axis cs:1224000,658.54966)
--(axis cs:1248000,646.231282291667)
--(axis cs:1272000,671.152858333333)
--(axis cs:1296000,660.192991666667)
--(axis cs:1320000,670.868964125)
--(axis cs:1344000,672.098633333333)
--(axis cs:1368000,667.270646666667)
--(axis cs:1392000,679.406895)
--(axis cs:1416000,667.768826166667)
--(axis cs:1440000,699.8657)
--(axis cs:1464000,703.368514541667)
--(axis cs:1488000,691.688456666667)
--(axis cs:1512000,691.02705)
--(axis cs:1536000,724.245045)
--(axis cs:1560000,642.1076)
--(axis cs:1584000,677.059261666667)
--(axis cs:1608000,722.467075)
--(axis cs:1632000,724.970475)
--(axis cs:1656000,707.042333333333)
--(axis cs:1680000,723.18196)
--(axis cs:1704000,739.733563333333)
--(axis cs:1728000,732.09224)
--(axis cs:1752000,743.118561666667)
--(axis cs:1776000,697.080626666666)
--(axis cs:1800000,709.36812)
--(axis cs:1824000,650.052771666667)
--(axis cs:1848000,716.700755)
--(axis cs:1872000,731.219691458333)
--(axis cs:1896000,703.767975)
--(axis cs:1920000,748.827876666667)
--(axis cs:1944000,769.246855)
--(axis cs:1968000,752.894191666667)
--(axis cs:1992000,737.845081666667)
--(axis cs:2016000,756.439925)
--(axis cs:2040000,764.171413333333)
--(axis cs:2064000,740.270250333333)
--(axis cs:2088000,725.118516666667)
--(axis cs:2112000,743.246291625)
--(axis cs:2136000,734.83942)
--(axis cs:2160000,786.861907708333)
--(axis cs:2184000,738.508053333333)
--(axis cs:2208000,750.911366666667)
--(axis cs:2232000,786.895078333333)
--(axis cs:2256000,798.218891666667)
--(axis cs:2280000,758.322531666667)
--(axis cs:2304000,768.420973333333)
--(axis cs:2328000,798.43544325)
--(axis cs:2352000,735.766616666667)
--(axis cs:2376000,763.731855)
--(axis cs:2400000,788.471761666667)
--(axis cs:2424000,798.863623333333)
--(axis cs:2448000,797.623013333333)
--(axis cs:2472000,798.009976666667)
--(axis cs:2496000,799.156973333333)
--(axis cs:2520000,789.180108333333)
--(axis cs:2544000,793.308771666667)
--(axis cs:2568000,766.555433333333)
--(axis cs:2592000,775.906533333333)
--(axis cs:2616000,657.870696541667)
--(axis cs:2640000,802.967105)
--(axis cs:2664000,762.239025)
--(axis cs:2688000,758.85175)
--(axis cs:2712000,832.457286666667)
--(axis cs:2736000,798.855305)
--(axis cs:2760000,790.104215)
--(axis cs:2784000,817.637935)
--(axis cs:2808000,791.100776666667)
--(axis cs:2832000,821.486588458333)
--(axis cs:2856000,798.389748333333)
--(axis cs:2880000,756.81125)
--(axis cs:2904000,809.53344)
--(axis cs:2928000,784.753078333333)
--(axis cs:2952000,778.719458333333)
--(axis cs:2976000,756.92755)
--(axis cs:3000000,792.897486666667)
--(axis cs:3000000,837.456603333333)
--(axis cs:3000000,837.456603333333)
--(axis cs:2976000,848.466266666667)
--(axis cs:2952000,849.20779)
--(axis cs:2928000,842.088306666667)
--(axis cs:2904000,854.474571666667)
--(axis cs:2880000,849.331893333333)
--(axis cs:2856000,862.497737875)
--(axis cs:2832000,869.625008333333)
--(axis cs:2808000,863.616378333333)
--(axis cs:2784000,860.982908333333)
--(axis cs:2760000,837.8695)
--(axis cs:2736000,852.905353333333)
--(axis cs:2712000,861.954153333333)
--(axis cs:2688000,830.469433333333)
--(axis cs:2664000,868.63305)
--(axis cs:2640000,864.472063333333)
--(axis cs:2616000,836.439531666667)
--(axis cs:2592000,863.806)
--(axis cs:2568000,856.3503)
--(axis cs:2544000,854.31376)
--(axis cs:2520000,848.667223333333)
--(axis cs:2496000,852.5053125)
--(axis cs:2472000,847.29574)
--(axis cs:2448000,844.57391)
--(axis cs:2424000,851.800331666667)
--(axis cs:2400000,842.753173333333)
--(axis cs:2376000,851.53189)
--(axis cs:2352000,824.298165)
--(axis cs:2328000,854.002923333333)
--(axis cs:2304000,823.149833333333)
--(axis cs:2280000,842.485828333333)
--(axis cs:2256000,850.53472)
--(axis cs:2232000,834.622511666667)
--(axis cs:2208000,832.830438333333)
--(axis cs:2184000,829.092180333333)
--(axis cs:2160000,843.589423333333)
--(axis cs:2136000,835.928383333333)
--(axis cs:2112000,828.188466666667)
--(axis cs:2088000,834.338206875)
--(axis cs:2064000,847.99647)
--(axis cs:2040000,817.802191666667)
--(axis cs:2016000,831.623071666667)
--(axis cs:1992000,808.728928708333)
--(axis cs:1968000,824.828883333333)
--(axis cs:1944000,821.340061666667)
--(axis cs:1920000,826.504741666667)
--(axis cs:1896000,818.318428333333)
--(axis cs:1872000,831.390226666667)
--(axis cs:1848000,826.4945)
--(axis cs:1824000,819.049933333333)
--(axis cs:1800000,842.232765791667)
--(axis cs:1776000,826.921123333333)
--(axis cs:1752000,834.800166666667)
--(axis cs:1728000,837.246975)
--(axis cs:1704000,853.61177)
--(axis cs:1680000,810.376866666667)
--(axis cs:1656000,808.423789083334)
--(axis cs:1632000,821.8582)
--(axis cs:1608000,815.354633333333)
--(axis cs:1584000,808.92544)
--(axis cs:1560000,812.104926666667)
--(axis cs:1536000,809.656291666667)
--(axis cs:1512000,782.288026666667)
--(axis cs:1488000,778.782291583334)
--(axis cs:1464000,778.858313333333)
--(axis cs:1440000,821.10777)
--(axis cs:1416000,783.881986666667)
--(axis cs:1392000,750.93809425)
--(axis cs:1368000,760.521313333333)
--(axis cs:1344000,766.813391666667)
--(axis cs:1320000,745.988075)
--(axis cs:1296000,745.58748175)
--(axis cs:1272000,778.078543333333)
--(axis cs:1248000,739.945458333333)
--(axis cs:1224000,752.028946666667)
--(axis cs:1200000,754.479292041667)
--(axis cs:1176000,738.838393333333)
--(axis cs:1152000,730.995366666667)
--(axis cs:1128000,726.874316666667)
--(axis cs:1104000,706.711063333333)
--(axis cs:1080000,710.372623333333)
--(axis cs:1056000,691.064566666667)
--(axis cs:1032000,651.101731666667)
--(axis cs:1008000,694.859348333333)
--(axis cs:984000,679.811011666667)
--(axis cs:960000,657.437975)
--(axis cs:936000,635.207354333333)
--(axis cs:912000,623.17085)
--(axis cs:888000,640.82537)
--(axis cs:864000,638.372956666667)
--(axis cs:840000,605.636066666667)
--(axis cs:816000,620.0484)
--(axis cs:792000,605.530201666667)
--(axis cs:768000,619.288428333333)
--(axis cs:744000,583.706463333333)
--(axis cs:720000,568.233463333333)
--(axis cs:696000,556.574435)
--(axis cs:672000,567.827056666667)
--(axis cs:648000,536.300349125)
--(axis cs:624000,521.473436666667)
--(axis cs:600000,514.082965)
--(axis cs:576000,512.673896666667)
--(axis cs:552000,503.68212)
--(axis cs:528000,495.859381666667)
--(axis cs:504000,462.220926666667)
--(axis cs:480000,454.484963333333)
--(axis cs:456000,428.183936666667)
--(axis cs:432000,393.258111666667)
--(axis cs:408000,386.389815)
--(axis cs:384000,369.733846666667)
--(axis cs:360000,331.193968333333)
--(axis cs:336000,327.351346666667)
--(axis cs:312000,296.701976666667)
--(axis cs:288000,251.529673375)
--(axis cs:264000,234.932268333333)
--(axis cs:240000,200.893656666667)
--(axis cs:216000,193.835535)
--(axis cs:192000,173.552272666667)
--(axis cs:168000,152.488876666667)
--(axis cs:144000,126.71998265)
--(axis cs:120000,102.995841)
--(axis cs:96000,77.3036046666667)
--(axis cs:72000,62.1926506666667)
--(axis cs:48000,35.3083526666667)
--(axis cs:24000,12.7237687833333)
--(axis cs:1,6.80215591666667)
--cycle;

\addplot [line width=\linewidthdime, C0, mark=*, mark size=0, mark options={solid}]
table {%
1 6.17641296666667
24000 7.56386268333333
48000 21.194537
72000 51.2078703333333
96000 60.4982363333333
120000 86.1485583333333
144000 110.007961666667
168000 134.059945
192000 145.492449333333
216000 164.83243
240000 161.520515
264000 192.983695
288000 202.72266
312000 233.335831666667
336000 261.032931666667
360000 270.949796666667
384000 306.506178333333
408000 326.736271666667
432000 321.852778333333
456000 346.82201
480000 389.972858333333
504000 387.914375
528000 412.405621666667
552000 431.808476666667
576000 435.778751666667
600000 454.327343333333
624000 452.650095
648000 468.345228333333
672000 505.85564
696000 488.718146666667
720000 507.547415
744000 526.153453333333
768000 557.257588333333
792000 547.320155
816000 551.174743333333
840000 565.919836666667
864000 586.99087
888000 582.702056666667
912000 577.851651666667
936000 603.980191666667
960000 612.742416666667
984000 628.581438333333
1008000 652.547345
1032000 621.855605
1056000 653.459033333333
1080000 677.789983333333
1104000 677.092326666667
1128000 688.166816666667
1152000 696.043668333333
1176000 689.546143333333
1200000 707.716985
1224000 716.223413333333
1248000 695.373521666667
1272000 729.041578333333
1296000 711.299655
1320000 714.779295
1344000 720.93956
1368000 718.17746
1392000 720.170855
1416000 742.45139
1440000 767.977446666667
1464000 745.921991666667
1488000 741.981485
1512000 740.075188333334
1536000 776.043163333333
1560000 730.35288
1584000 745.76982
1608000 774.030058333333
1632000 786.423141666667
1656000 775.007413333333
1680000 767.966193333333
1704000 803.248776666667
1728000 795.057825
1752000 799.094
1776000 768.546685
1800000 785.936768333333
1824000 770.381045
1848000 778.163971666667
1872000 784.5074
1896000 775.084686666667
1920000 803.85685
1944000 802.98721
1968000 799.194371666667
1992000 777.64892
2016000 809.862228333333
2040000 802.882363333333
2064000 807.031923333333
2088000 792.892843333333
2112000 784.135861666667
2136000 799.080736666667
2160000 829.289918333333
2184000 812.669866666667
2208000 798.845955
2232000 810.922945
2256000 829.769393333333
2280000 808.989303333333
2304000 787.9255
2328000 834.55804
2352000 790.163291666667
2376000 809.844736666667
2400000 823.614375
2424000 829.909856666667
2448000 827.945956666667
2472000 830.88744
2496000 832.031106666667
2520000 821.356241666667
2544000 822.67606
2568000 821.050315
2592000 838.71875
2616000 819.876546666667
2640000 836.608795
2664000 842.476748333333
2688000 800.310716666667
2712000 848.692928333333
2736000 828.082728333333
2760000 814.163598333333
2784000 843.60977
2808000 840.753683333333
2832000 848.649233333333
2856000 840.45366
2880000 812.837433333333
2904000 839.813745
2928000 813.380873333333
2952000 813.313951666667
2976000 809.99765
3000000 818.333465
};
\addlegendentry{DIME}
\end{axis}

\end{tikzpicture}
}
       \subcaption[]{DIME and GB on Dog Run}
       \label{fig::appendix_dog_rund_dbs}
    \end{minipage}\hfill
    \begin{minipage}[b]{0.33\textwidth}
        \centering
       \resizebox{1\textwidth}{!}{% This file was created with tikzplotlib v0.10.1.
\begin{tikzpicture}

\definecolor{darkcyan1115178}{RGB}{1,115,178}
\definecolor{darkgray176}{RGB}{176,176,176}

\begin{axis}[
legend cell align={left},
legend cell align={left},
legend style={fill opacity=0.8, draw opacity=1, text opacity=1, draw=lightgray204, at={(0.5,0.03)},  anchor=south west},
tick align=outside,
tick pos=left,
x grid style={white},
xlabel={Number Env Interactions},
xmajorgrids,
xmin=-0.0, xmax=1000000.0,
xtick style={color=black},
y grid style={white},
ylabel={IQM Mean Return},
ymajorgrids,
ymin=-12.11232765065, ymax=350.65256879765,
ytick style={color=black},
axis background/.style={fill=plot_background},
label style={font=\large},
tick label style={font=\large},
x axis line style={draw=none},
y axis line style={draw=none},
]

\path [draw=C1, fill=C1, opacity=0.2]
(axis cs:1,0.913481333333333)
--(axis cs:1,0.6287236)
--(axis cs:24000,0.677781566666667)
--(axis cs:48000,0.738428966666667)
--(axis cs:72000,0.790174833333333)
--(axis cs:96000,1.1543524)
--(axis cs:120000,3.23502683000002)
--(axis cs:144000,20.5413528)
--(axis cs:168000,8.54387893333333)
--(axis cs:192000,34.4303118333333)
--(axis cs:216000,72.0098966666667)
--(axis cs:240000,72.3488675833333)
--(axis cs:264000,96.707165)
--(axis cs:288000,96.000784)
--(axis cs:312000,104.958421625)
--(axis cs:336000,114.297126)
--(axis cs:360000,106.284968)
--(axis cs:384000,123.7906925)
--(axis cs:408000,123.723489166667)
--(axis cs:432000,128.761211666667)
--(axis cs:456000,111.02938625)
--(axis cs:480000,135.956141666667)
--(axis cs:504000,138.98209)
--(axis cs:528000,118.929581333333)
--(axis cs:552000,132.387325)
--(axis cs:576000,148.148588333333)
--(axis cs:600000,153.61958)
--(axis cs:624000,156.55045)
--(axis cs:648000,156.870825)
--(axis cs:672000,161.074148333333)
--(axis cs:696000,164.107246666667)
--(axis cs:720000,165.654856541667)
--(axis cs:744000,167.669441666667)
--(axis cs:768000,160.786089)
--(axis cs:792000,168.99019)
--(axis cs:816000,177.369616666667)
--(axis cs:840000,175.239106666667)
--(axis cs:864000,176.361243333333)
--(axis cs:888000,165.394583333333)
--(axis cs:912000,183.752826666667)
--(axis cs:936000,187.066958333333)
--(axis cs:960000,160.625822716667)
--(axis cs:984000,191.555848333333)
--(axis cs:984000,251.074266666667)
--(axis cs:984000,251.074266666667)
--(axis cs:960000,253.124288333333)
--(axis cs:936000,245.776626666667)
--(axis cs:912000,236.034463333333)
--(axis cs:888000,226.598225)
--(axis cs:864000,227.013098708333)
--(axis cs:840000,234.283123333333)
--(axis cs:816000,229.472495)
--(axis cs:792000,224.034361666667)
--(axis cs:768000,206.047165)
--(axis cs:744000,209.549365)
--(axis cs:720000,205.859246666667)
--(axis cs:696000,194.984201666667)
--(axis cs:672000,188.552526666667)
--(axis cs:648000,185.157259583333)
--(axis cs:624000,177.684235)
--(axis cs:600000,172.894043333333)
--(axis cs:576000,171.0584)
--(axis cs:552000,163.391113333333)
--(axis cs:528000,162.135815)
--(axis cs:504000,161.517217666667)
--(axis cs:480000,153.846013166667)
--(axis cs:456000,150.815503333333)
--(axis cs:432000,154.043445)
--(axis cs:408000,146.23077)
--(axis cs:384000,144.130305)
--(axis cs:360000,136.713208333333)
--(axis cs:336000,131.509801666667)
--(axis cs:312000,123.557279541667)
--(axis cs:288000,121.774874166667)
--(axis cs:264000,117.476811666667)
--(axis cs:240000,110.350656666667)
--(axis cs:216000,103.274480333333)
--(axis cs:192000,90.3314516666667)
--(axis cs:168000,85.1717466666667)
--(axis cs:144000,83.9523868333333)
--(axis cs:120000,60.989724)
--(axis cs:96000,26.2789549666667)
--(axis cs:72000,5.35472753333333)
--(axis cs:48000,1.06001243333333)
--(axis cs:24000,0.923337466666667)
--(axis cs:1,0.913481333333333)
--cycle;

\addplot [line width=\linewidthdime, C1, mark=*, mark size=0, mark options={solid}]
table {%
1 0.7694417
24000 0.813880766666667
48000 0.903083466666667
72000 0.985341633333333
96000 7.39312896666667
120000 31.2267919666667
144000 57.5656945
168000 45.1603116666667
192000 70.5692
216000 94.6452396666667
240000 97.5139125
264000 109.186628333333
288000 109.1955065
312000 111.598211666667
336000 123.198625
360000 125.946167
384000 133.888621666667
408000 134.214293333333
432000 139.483735
456000 142.763216666667
480000 144.338755
504000 149.75221
528000 151.421726666667
552000 147.365006666667
576000 159.676698333333
600000 161.191491666667
624000 163.406718333333
648000 169.18849
672000 173.390193333333
696000 174.364796666667
720000 181.648728333333
744000 181.343818333333
768000 179.253028333333
792000 189.492575
816000 195.565436666667
840000 198.284515
864000 193.75859
888000 187.434418333333
912000 201.805691666667
936000 209.501925
960000 208.739933333333
984000 214.710998333333
};

%DIME
\path [draw=C0, fill=C0, opacity=0.2]
(axis cs:1,0.915501966666667)
--(axis cs:1,0.633556866666667)
--(axis cs:36000,0.5493304)
--(axis cs:72000,0.843873266666667)
--(axis cs:108000,4.13696455)
--(axis cs:144000,36.1211871666667)
--(axis cs:180000,51.4064937333333)
--(axis cs:216000,81.368514)
--(axis cs:252000,97.5408251666667)
--(axis cs:288000,109.583571666667)
--(axis cs:324000,110.618288333333)
--(axis cs:360000,120.684184666667)
--(axis cs:396000,122.417020833333)
--(axis cs:432000,131.747461666667)
--(axis cs:468000,139.23245)
--(axis cs:504000,145.708203333333)
--(axis cs:540000,128.071794166667)
--(axis cs:576000,146.743565)
--(axis cs:612000,157.523090333333)
--(axis cs:648000,162.386513333333)
--(axis cs:684000,167.712168333333)
--(axis cs:720000,177.17324)
--(axis cs:756000,175.778066666667)
--(axis cs:792000,177.557476666667)
--(axis cs:828000,183.413371666667)
--(axis cs:864000,191.343878333333)
--(axis cs:900000,198.058078333333)
--(axis cs:936000,199.495351666667)
--(axis cs:972000,204.18452)
--(axis cs:1008000,211.017165)
--(axis cs:1044000,210.395921666667)
--(axis cs:1080000,225.37889)
--(axis cs:1116000,225.57669)
--(axis cs:1152000,229.055506666667)
--(axis cs:1188000,244.717408333333)
--(axis cs:1224000,235.69925925)
--(axis cs:1260000,246.761671666667)
--(axis cs:1296000,263.70817)
--(axis cs:1332000,261.655338333333)
--(axis cs:1368000,274.904988333333)
--(axis cs:1404000,281.871765)
--(axis cs:1440000,284.6845)
--(axis cs:1476000,298.373705)
--(axis cs:1512000,314.234433333333)
--(axis cs:1548000,302.586606666667)
--(axis cs:1584000,306.158113166667)
--(axis cs:1620000,329.89694)
--(axis cs:1656000,327.346265)
--(axis cs:1692000,331.977616666667)
--(axis cs:1728000,335.598585)
--(axis cs:1764000,337.5854)
--(axis cs:1800000,350.510931458333)
--(axis cs:1836000,348.138216666667)
--(axis cs:1872000,325.377726666667)
--(axis cs:1908000,375.064071666667)
--(axis cs:1944000,364.504996666667)
--(axis cs:1980000,362.576775)
--(axis cs:2016000,359.905746666667)
--(axis cs:2052000,401.565295)
--(axis cs:2088000,380.944233333333)
--(axis cs:2124000,408.07439)
--(axis cs:2160000,406.816443333333)
--(axis cs:2196000,402.540623333333)
--(axis cs:2232000,419.828846666667)
--(axis cs:2268000,419.407693333333)
--(axis cs:2304000,391.893025)
--(axis cs:2340000,426.646173333333)
--(axis cs:2376000,430.59703)
--(axis cs:2412000,440.944021375)
--(axis cs:2448000,444.535958333333)
--(axis cs:2484000,447.59097)
--(axis cs:2520000,458.220718125)
--(axis cs:2556000,454.987311666667)
--(axis cs:2592000,444.593778333333)
--(axis cs:2628000,452.355353333333)
--(axis cs:2664000,472.315071583333)
--(axis cs:2700000,468.552297125)
--(axis cs:2736000,425.893185)
--(axis cs:2772000,461.17882)
--(axis cs:2808000,438.974565)
--(axis cs:2844000,408.566183333333)
--(axis cs:2880000,440.493023333333)
--(axis cs:2916000,474.594651666667)
--(axis cs:2952000,473.13282)
--(axis cs:2988000,468.31128725)
--(axis cs:2988000,572.54452975)
--(axis cs:2988000,572.54452975)
--(axis cs:2952000,607.047773333333)
--(axis cs:2916000,581.955683333333)
--(axis cs:2880000,562.69891)
--(axis cs:2844000,570.399441666667)
--(axis cs:2808000,561.294171666667)
--(axis cs:2772000,575.424046666667)
--(axis cs:2736000,577.564579375)
--(axis cs:2700000,570.731558333333)
--(axis cs:2664000,556.645568333333)
--(axis cs:2628000,528.610636666667)
--(axis cs:2592000,572.906766666667)
--(axis cs:2556000,556.792031666667)
--(axis cs:2520000,565.548321666667)
--(axis cs:2484000,564.076643333333)
--(axis cs:2448000,554.428711666667)
--(axis cs:2412000,511.028458333333)
--(axis cs:2376000,539.327213333333)
--(axis cs:2340000,521.274179458333)
--(axis cs:2304000,507.011033333333)
--(axis cs:2268000,533.545403333333)
--(axis cs:2232000,530.436551666667)
--(axis cs:2196000,514.156165)
--(axis cs:2160000,548.6599)
--(axis cs:2124000,526.241701666667)
--(axis cs:2088000,488.449933333333)
--(axis cs:2052000,521.397975)
--(axis cs:2016000,489.724498333333)
--(axis cs:1980000,511.56874)
--(axis cs:1944000,469.79517)
--(axis cs:1908000,487.4035)
--(axis cs:1872000,452.261116666667)
--(axis cs:1836000,448.443288333333)
--(axis cs:1800000,470.013476541667)
--(axis cs:1764000,441.067136666667)
--(axis cs:1728000,425.983283333333)
--(axis cs:1692000,459.900376666667)
--(axis cs:1656000,420.555463333333)
--(axis cs:1620000,452.412428333333)
--(axis cs:1584000,426.080911666667)
--(axis cs:1548000,405.744391916667)
--(axis cs:1512000,422.468795)
--(axis cs:1476000,392.121763333333)
--(axis cs:1440000,398.927006666667)
--(axis cs:1404000,396.300881666667)
--(axis cs:1368000,391.708926666667)
--(axis cs:1332000,379.3367)
--(axis cs:1296000,374.028266666667)
--(axis cs:1260000,354.826833333333)
--(axis cs:1224000,325.77437)
--(axis cs:1188000,356.78896)
--(axis cs:1152000,352.524811666667)
--(axis cs:1116000,343.294016666667)
--(axis cs:1080000,331.116815)
--(axis cs:1044000,287.783165)
--(axis cs:1008000,319.938826666667)
--(axis cs:972000,297.00894)
--(axis cs:936000,297.43841)
--(axis cs:900000,295.160138333333)
--(axis cs:864000,278.186283333333)
--(axis cs:828000,267.117865)
--(axis cs:792000,254.04739775)
--(axis cs:756000,245.040613333333)
--(axis cs:720000,223.696456875)
--(axis cs:684000,216.1656)
--(axis cs:648000,205.401713333333)
--(axis cs:612000,201.404601666667)
--(axis cs:576000,182.55503)
--(axis cs:540000,177.786793333333)
--(axis cs:504000,170.561265208333)
--(axis cs:468000,170.450261666667)
--(axis cs:432000,164.02998)
--(axis cs:396000,153.633991666667)
--(axis cs:360000,143.37209)
--(axis cs:324000,136.525775833333)
--(axis cs:288000,129.716861683333)
--(axis cs:252000,113.1404125)
--(axis cs:216000,102.49002)
--(axis cs:180000,92.9197603333333)
--(axis cs:144000,76.8890066666667)
--(axis cs:108000,56.7536433333333)
--(axis cs:72000,4.25857906666667)
--(axis cs:36000,0.728938933333333)
--(axis cs:1,0.915501966666667)
--cycle;

\addplot [line width=\linewidthdime, C0, mark=*, mark size=0, mark options={solid}]
table {%
1 0.770562166666667
36000 0.658370533333333
72000 1.02744053333333
108000 27.4132359166667
144000 61.7389005
180000 78.3388245
216000 96.6475441666667
252000 104.292225666667
288000 118.336289333333
324000 122.156900833333
360000 127.865645666667
396000 135.589655833333
432000 144.61231
468000 150.876916666667
504000 155.106371666667
540000 161.848961666667
576000 162.873085
612000 175.66877
648000 181.774033333333
684000 189.795443333333
720000 201.694543333333
756000 209.679653333333
792000 215.950858333333
828000 226.452128333333
864000 235.790758333333
900000 245.745095
936000 251.419853333333
972000 253.923228333333
1008000 267.745066666667
1044000 253.029908333333
1080000 281.1757
1116000 290.478148333333
1152000 295.386573333333
1188000 307.510898333333
1224000 279.324226666667
1260000 298.499671666667
1296000 320.530285
1332000 314.688511666667
1368000 333.564955
1404000 344.035705
1440000 351.026203333333
1476000 334.48205
1512000 371.517083333333
1548000 359.107073333333
1584000 362.49581
1620000 383.735268333333
1656000 371.120853333333
1692000 392.875408333333
1728000 375.706403333333
1764000 385.148411666667
1800000 411.377563333333
1836000 393.022041666667
1872000 381.083618333333
1908000 428.514751666667
1944000 407.976545
1980000 440.522148333333
2016000 425.096363333333
2052000 460.246375
2088000 439.221346666667
2124000 470.482476666667
2160000 479.806223333333
2196000 467.028006666667
2232000 471.833656666667
2268000 492.305195
2304000 444.283438333333
2340000 467.734686666667
2376000 478.305563333333
2412000 479.659883333333
2448000 502.891963333333
2484000 512.254108333333
2520000 504.927955
2556000 509.89631
2592000 517.465808333333
2628000 505.843831666667
2664000 503.219516666667
2700000 525.334166666667
2736000 498.718568333333
2772000 515.273173333333
2808000 507.004858333333
2844000 490.845221666667
2880000 488.33129
2916000 530.288801666667
2952000 544.121438333333
2988000 506.351106666667
};

\end{axis}

\end{tikzpicture}
}
       \subcaption[]{DIME and GB on Humanoid Run}
       \label{fig::appendix_hum_rund_dbs}
    \end{minipage}\hfill
    \begin{minipage}[b]{0.33\textwidth}
        \centering
       \resizebox{1\textwidth}{!}{\definecolor{crimson2143940}{RGB}{214,39,40}
\definecolor{darkgray176}{RGB}{176,176,176}
\definecolor{darkorange25512714}{RGB}{255,127,14}
\definecolor{forestgreen4416044}{RGB}{44,160,44}
\definecolor{mediumpurple148103189}{RGB}{148,103,189}
\definecolor{steelblue31119180}{RGB}{31,119,180}

% This file was created with tikzplotlib v0.10.1.
\begin{tikzpicture}

\definecolor{darkcyan1115178}{RGB}{1,115,178}
\definecolor{darkgray176}{RGB}{176,176,176}

\begin{axis}[
legend cell align={left},
legend cell align={left},
legend style={fill opacity=0.8, draw opacity=1, text opacity=1, draw=lightgray204, at={(0.5,0.03)},  anchor=south west},
tick align=outside,
tick pos=left,
x grid style={white},
xlabel={Number Env Interactions},
xmajorgrids,
%xmin=-149398.95, xmax=3000000.00,
xmin=0.0, xmax=3000000.00,
xtick style={color=black},
y grid style={white},
ylabel={IQM Return},
ymajorgrids,
%ymin=-37.9926843766667, ymax=923.09120151,
ymin=0.0, ymax=637.37269548,
ytick style={color=black},
axis background/.style={fill=plot_background},
label style={font=\large},
tick label style={font=\large},
x axis line style={draw=none},
y axis line style={draw=none},
]
% BRO

\path [draw=C1, fill=C1, opacity=0.2]
(axis cs:25000,1.06436189695316)
--(axis cs:25000,0.800044970854062)
--(axis cs:50000,0.902213431331389)
--(axis cs:75000,0.998039782938274)
--(axis cs:100000,3.73818093255901)
--(axis cs:125000,14.8008643633554)
--(axis cs:150000,29.8200848291147)
--(axis cs:175000,52.536500676349)
--(axis cs:200000,65.4528765036023)
--(axis cs:225000,70.6547353460924)
--(axis cs:250000,77.9210714654044)
--(axis cs:275000,74.6925274736162)
--(axis cs:300000,101.688198709863)
--(axis cs:325000,107.281306117989)
--(axis cs:350000,115.959467727732)
--(axis cs:375000,124.884927566366)
--(axis cs:400000,129.951946684364)
--(axis cs:425000,136.152626975576)
--(axis cs:450000,136.111337375472)
--(axis cs:475000,143.874927358078)
--(axis cs:500000,148.567262987423)
--(axis cs:525000,94.5117370449074)
--(axis cs:550000,137.525904720989)
--(axis cs:575000,152.664260638624)
--(axis cs:600000,161.929930494808)
--(axis cs:625000,171.58650975503)
--(axis cs:650000,181.142500891986)
--(axis cs:675000,186.720413352857)
--(axis cs:700000,191.754617425019)
--(axis cs:725000,196.874121030149)
--(axis cs:750000,212.645896156586)
--(axis cs:775000,103.542898066904)
--(axis cs:800000,171.614039449506)
--(axis cs:825000,201.242993560958)
--(axis cs:850000,211.747982944836)
--(axis cs:875000,224.177397591092)
--(axis cs:900000,237.999427608921)
--(axis cs:925000,239.436421657841)
--(axis cs:950000,250.973926888014)
--(axis cs:975000,250.707118365674)
--(axis cs:1000000,255.7721777332)
--(axis cs:1025000,85.7462506497815)
--(axis cs:1050000,177.039576732904)
--(axis cs:1075000,226.850952562221)
--(axis cs:1100000,256.461162193783)
--(axis cs:1125000,272.077354487363)
--(axis cs:1150000,277.88091019911)
--(axis cs:1175000,302.440032278388)
--(axis cs:1200000,286.286261722487)
--(axis cs:1225000,295.097707667798)
--(axis cs:1250000,310.062918562881)
--(axis cs:1275000,310.671213558559)
--(axis cs:1300000,328.891560548388)
--(axis cs:1325000,315.261131031065)
--(axis cs:1350000,333.264653443051)
--(axis cs:1375000,330.201890430345)
--(axis cs:1400000,324.50649705123)
--(axis cs:1425000,328.421995352062)
--(axis cs:1450000,343.634546505786)
--(axis cs:1475000,339.296630224071)
--(axis cs:1500000,348.984178695933)
--(axis cs:1525000,39.355588180581)
--(axis cs:1550000,132.834147960603)
--(axis cs:1575000,220.885853788656)
--(axis cs:1600000,268.221180119728)
--(axis cs:1625000,309.413007126434)
--(axis cs:1650000,330.020327352981)
--(axis cs:1675000,328.810183948437)
--(axis cs:1700000,342.485717291448)
--(axis cs:1725000,350.558346157369)
--(axis cs:1750000,347.543642435759)
--(axis cs:1775000,362.036408929491)
--(axis cs:1800000,363.429130292646)
--(axis cs:1825000,378.249045471367)
--(axis cs:1850000,368.466818930279)
--(axis cs:1875000,395.92381129835)
--(axis cs:1900000,375.255770601223)
--(axis cs:1925000,380.466843316452)
--(axis cs:1950000,385.742622584488)
--(axis cs:1975000,385.833705784746)
--(axis cs:2000000,388.274637595042)
--(axis cs:2025000,33.5649937116025)
--(axis cs:2050000,109.201720408668)
--(axis cs:2075000,186.449238204214)
--(axis cs:2100000,240.623289727134)
--(axis cs:2125000,305.664385445925)
--(axis cs:2150000,346.718313109583)
--(axis cs:2175000,362.847691076416)
--(axis cs:2200000,374.670521199756)
--(axis cs:2225000,396.743614451708)
--(axis cs:2250000,404.508816918751)
--(axis cs:2275000,404.815724919529)
--(axis cs:2300000,408.67455149964)
--(axis cs:2325000,415.555724534537)
--(axis cs:2350000,409.515642698911)
--(axis cs:2375000,413.456404646793)
--(axis cs:2400000,410.346599855339)
--(axis cs:2425000,420.13357683042)
--(axis cs:2450000,421.419850504902)
--(axis cs:2475000,420.083915472082)
--(axis cs:2500000,425.007581951612)
--(axis cs:2525000,424.579049671924)
--(axis cs:2550000,430.75366994294)
--(axis cs:2575000,421.216107533729)
--(axis cs:2600000,429.76793736549)
--(axis cs:2625000,443.105374475612)
--(axis cs:2650000,434.722857001578)
--(axis cs:2675000,428.275377228427)
--(axis cs:2700000,425.81611641096)
--(axis cs:2725000,444.64365130971)
--(axis cs:2750000,450.413269764893)
--(axis cs:2775000,434.345878022504)
--(axis cs:2800000,443.91301212502)
--(axis cs:2825000,447.49939896642)
--(axis cs:2850000,443.697419471331)
--(axis cs:2875000,449.354555033438)
--(axis cs:2900000,450.8026245623)
--(axis cs:2925000,458.680244532348)
--(axis cs:2950000,444.916782676853)
--(axis cs:2975000,445.634457360389)
--(axis cs:3000000,429.880019721412)
--(axis cs:3000000,513.9873176354)
--(axis cs:3000000,513.9873176354)
--(axis cs:2975000,523.797044053664)
--(axis cs:2950000,517.810496299027)
--(axis cs:2925000,525.954271943104)
--(axis cs:2900000,518.551529862146)
--(axis cs:2875000,518.920986906444)
--(axis cs:2850000,505.088500000573)
--(axis cs:2825000,518.566273395142)
--(axis cs:2800000,509.237923377759)
--(axis cs:2775000,501.25737445807)
--(axis cs:2750000,505.642689921167)
--(axis cs:2725000,513.254245513279)
--(axis cs:2700000,499.87053006485)
--(axis cs:2675000,499.068414301387)
--(axis cs:2650000,499.244308228539)
--(axis cs:2625000,500.18961766769)
--(axis cs:2600000,512.531552452399)
--(axis cs:2575000,496.697448428239)
--(axis cs:2550000,495.403692454602)
--(axis cs:2525000,502.740522809381)
--(axis cs:2500000,493.846745259791)
--(axis cs:2475000,488.320572719053)
--(axis cs:2450000,487.649215256229)
--(axis cs:2425000,478.718572892562)
--(axis cs:2400000,477.202876793975)
--(axis cs:2375000,476.451092629711)
--(axis cs:2350000,481.167057793168)
--(axis cs:2325000,464.822365433921)
--(axis cs:2300000,475.130801215997)
--(axis cs:2275000,457.275923790936)
--(axis cs:2250000,463.437629737087)
--(axis cs:2225000,469.693586932296)
--(axis cs:2200000,446.464641236289)
--(axis cs:2175000,423.741760297603)
--(axis cs:2150000,412.259115726332)
--(axis cs:2125000,374.619709027751)
--(axis cs:2100000,308.103900280336)
--(axis cs:2075000,240.504027337852)
--(axis cs:2050000,147.589375617486)
--(axis cs:2025000,60.8440314157716)
--(axis cs:2000000,454.6140241901)
--(axis cs:1975000,457.782889455546)
--(axis cs:1950000,454.490153888389)
--(axis cs:1925000,460.646195395525)
--(axis cs:1900000,435.379018582715)
--(axis cs:1875000,451.043381396628)
--(axis cs:1850000,437.594051329755)
--(axis cs:1825000,441.161889244782)
--(axis cs:1800000,428.1258205615)
--(axis cs:1775000,433.047391857064)
--(axis cs:1750000,431.680207172047)
--(axis cs:1725000,419.626783286342)
--(axis cs:1700000,401.334051733191)
--(axis cs:1675000,408.532521204368)
--(axis cs:1650000,395.947013655237)
--(axis cs:1625000,381.209360225348)
--(axis cs:1600000,328.600370144145)
--(axis cs:1575000,252.603144263693)
--(axis cs:1550000,162.227344584328)
--(axis cs:1525000,55.3468262239898)
--(axis cs:1500000,410.792177821071)
--(axis cs:1475000,404.502694934256)
--(axis cs:1450000,401.813256783996)
--(axis cs:1425000,401.708044362734)
--(axis cs:1400000,395.125789323197)
--(axis cs:1375000,400.825515156559)
--(axis cs:1350000,392.647754460344)
--(axis cs:1325000,384.152222205941)
--(axis cs:1300000,397.852872002416)
--(axis cs:1275000,377.552362447173)
--(axis cs:1250000,376.011972574837)
--(axis cs:1225000,359.739845675417)
--(axis cs:1200000,344.509178975306)
--(axis cs:1175000,367.154285866658)
--(axis cs:1150000,334.799456193374)
--(axis cs:1125000,344.307719911651)
--(axis cs:1100000,296.227318401859)
--(axis cs:1075000,266.643166383648)
--(axis cs:1050000,204.584202057326)
--(axis cs:1025000,108.689802307726)
--(axis cs:1000000,318.852715650748)
--(axis cs:975000,300.456149902442)
--(axis cs:950000,319.33259299771)
--(axis cs:925000,294.26813015175)
--(axis cs:900000,297.854696283069)
--(axis cs:875000,272.025279001709)
--(axis cs:850000,263.353273415112)
--(axis cs:825000,242.24173443444)
--(axis cs:800000,200.105312087152)
--(axis cs:775000,122.615578156031)
--(axis cs:750000,259.997946552285)
--(axis cs:725000,248.381254652691)
--(axis cs:700000,235.75254015339)
--(axis cs:675000,235.151123818793)
--(axis cs:650000,227.018726493552)
--(axis cs:625000,215.921378426131)
--(axis cs:600000,197.741080859391)
--(axis cs:575000,183.93247982137)
--(axis cs:550000,160.853477460561)
--(axis cs:525000,112.900779876465)
--(axis cs:500000,176.061616899139)
--(axis cs:475000,173.423721376022)
--(axis cs:450000,162.489487033554)
--(axis cs:425000,161.259046819264)
--(axis cs:400000,147.344859501089)
--(axis cs:375000,144.659394938876)
--(axis cs:350000,136.178883223972)
--(axis cs:325000,123.99330842961)
--(axis cs:300000,113.533585440671)
--(axis cs:275000,98.6956458394326)
--(axis cs:250000,95.6263797325166)
--(axis cs:225000,90.8477919575314)
--(axis cs:200000,80.6211055761968)
--(axis cs:175000,68.9779022961458)
--(axis cs:150000,52.6168055078161)
--(axis cs:125000,37.2068416693329)
--(axis cs:100000,17.4181970842278)
--(axis cs:75000,1.58356256613591)
--(axis cs:50000,1.25625000054544)
--(axis cs:25000,1.06436189695316)
--cycle;

\addplot [line width=\linewidthother, C1, mark=*, mark size=0, mark options={solid}]
table {%
25000 0.930442202091521
50000 1.07651185871321
75000 1.25993092696276
100000 9.83030688967838
125000 25.6411418517771
150000 41.1075536267594
175000 60.3883914444651
200000 73.172436582942
225000 80.957633586035
250000 86.0890838978233
275000 86.8321797068196
300000 107.266605289419
325000 114.972760323541
350000 125.514097446969
375000 134.324845411907
400000 138.233436327078
425000 147.660369336518
450000 148.782457916996
475000 157.754543249098
500000 161.480475192749
525000 103.279940413761
550000 148.67370118324
575000 166.636231776283
600000 177.730272953984
625000 192.167083171252
650000 202.772177321453
675000 210.734125891453
700000 213.205316409278
725000 221.706877576317
750000 236.467345033334
775000 112.87857606821
800000 186.130552121398
825000 221.723558177307
850000 236.209271737391
875000 247.087024794249
900000 267.471165250301
925000 265.658479214331
950000 283.439134541734
975000 275.133379987545
1000000 284.437051242526
1025000 97.0758468913518
1050000 191.453500474107
1075000 247.925237545804
1100000 277.620995193895
1125000 307.245282942146
1150000 305.919661019263
1175000 335.107741080569
1200000 316.435912664741
1225000 327.456512993191
1250000 342.107967490487
1275000 345.094845414729
1300000 365.251258681473
1325000 349.916712906547
1350000 363.318109792082
1375000 365.501578733968
1400000 360.257954947648
1425000 365.685379120745
1450000 374.145117409919
1475000 372.055178318366
1500000 380.772709979235
1525000 47.1556118690389
1550000 147.391608013536
1575000 235.725959616489
1600000 298.100854291553
1625000 344.459721335641
1650000 364.712840460148
1675000 369.42338730188
1700000 371.318808438126
1725000 383.874329652056
1750000 390.530684839555
1775000 396.82664916639
1800000 395.408529222678
1825000 409.974986250516
1850000 401.761351799987
1875000 424.544214100914
1900000 403.427991613482
1925000 421.452086824094
1950000 420.168662455335
1975000 421.239853255038
2000000 421.557400426789
2025000 46.8497645215133
2050000 127.574579390775
2075000 212.337561217824
2100000 273.695465882477
2125000 339.060702704761
2150000 377.404140539549
2175000 392.147298960786
2200000 409.585928423078
2225000 432.930123927388
2250000 433.06954152684
2275000 430.817714378362
2300000 441.242333227845
2325000 440.26544618574
2350000 445.109444370395
2375000 445.310468474792
2400000 444.23401205299
2425000 447.974403762188
2450000 455.009836622286
2475000 453.065691276495
2500000 459.824261297809
2525000 463.614242587133
2550000 462.153633274131
2575000 458.885605928128
2600000 472.045677739907
2625000 472.465960920779
2650000 467.617349708702
2675000 463.919322456368
2700000 462.572175194156
2725000 479.223260869987
2750000 478.300271910278
2775000 467.550224472069
2800000 477.243281641589
2825000 482.77498600685
2850000 474.021199131709
2875000 484.511658494648
2900000 484.88515836079
2925000 492.783683602026
2950000 481.675038930042
2975000 484.59483344476
3000000 470.838626791646
};
\addlegendentry{BRO}
% DIME
\path [draw=C0, fill=C0, opacity=0.2]
(axis cs:1,0.915501966666667)
--(axis cs:1,0.633556866666667)
--(axis cs:36000,0.5493304)
--(axis cs:72000,0.843873266666667)
--(axis cs:108000,4.13696455)
--(axis cs:144000,36.1211871666667)
--(axis cs:180000,51.4064937333333)
--(axis cs:216000,81.368514)
--(axis cs:252000,97.5408251666667)
--(axis cs:288000,109.583571666667)
--(axis cs:324000,110.618288333333)
--(axis cs:360000,120.684184666667)
--(axis cs:396000,122.417020833333)
--(axis cs:432000,131.747461666667)
--(axis cs:468000,139.23245)
--(axis cs:504000,145.708203333333)
--(axis cs:540000,128.071794166667)
--(axis cs:576000,146.743565)
--(axis cs:612000,157.523090333333)
--(axis cs:648000,162.386513333333)
--(axis cs:684000,167.712168333333)
--(axis cs:720000,177.17324)
--(axis cs:756000,175.778066666667)
--(axis cs:792000,177.557476666667)
--(axis cs:828000,183.413371666667)
--(axis cs:864000,191.343878333333)
--(axis cs:900000,198.058078333333)
--(axis cs:936000,199.495351666667)
--(axis cs:972000,204.18452)
--(axis cs:1008000,211.017165)
--(axis cs:1044000,210.395921666667)
--(axis cs:1080000,225.37889)
--(axis cs:1116000,225.57669)
--(axis cs:1152000,229.055506666667)
--(axis cs:1188000,244.717408333333)
--(axis cs:1224000,235.69925925)
--(axis cs:1260000,246.761671666667)
--(axis cs:1296000,263.70817)
--(axis cs:1332000,261.655338333333)
--(axis cs:1368000,274.904988333333)
--(axis cs:1404000,281.871765)
--(axis cs:1440000,284.6845)
--(axis cs:1476000,298.373705)
--(axis cs:1512000,314.234433333333)
--(axis cs:1548000,302.586606666667)
--(axis cs:1584000,306.158113166667)
--(axis cs:1620000,329.89694)
--(axis cs:1656000,327.346265)
--(axis cs:1692000,331.977616666667)
--(axis cs:1728000,335.598585)
--(axis cs:1764000,337.5854)
--(axis cs:1800000,350.510931458333)
--(axis cs:1836000,348.138216666667)
--(axis cs:1872000,325.377726666667)
--(axis cs:1908000,375.064071666667)
--(axis cs:1944000,364.504996666667)
--(axis cs:1980000,362.576775)
--(axis cs:2016000,359.905746666667)
--(axis cs:2052000,401.565295)
--(axis cs:2088000,380.944233333333)
--(axis cs:2124000,408.07439)
--(axis cs:2160000,406.816443333333)
--(axis cs:2196000,402.540623333333)
--(axis cs:2232000,419.828846666667)
--(axis cs:2268000,419.407693333333)
--(axis cs:2304000,391.893025)
--(axis cs:2340000,426.646173333333)
--(axis cs:2376000,430.59703)
--(axis cs:2412000,440.944021375)
--(axis cs:2448000,444.535958333333)
--(axis cs:2484000,447.59097)
--(axis cs:2520000,458.220718125)
--(axis cs:2556000,454.987311666667)
--(axis cs:2592000,444.593778333333)
--(axis cs:2628000,452.355353333333)
--(axis cs:2664000,472.315071583333)
--(axis cs:2700000,468.552297125)
--(axis cs:2736000,425.893185)
--(axis cs:2772000,461.17882)
--(axis cs:2808000,438.974565)
--(axis cs:2844000,408.566183333333)
--(axis cs:2880000,440.493023333333)
--(axis cs:2916000,474.594651666667)
--(axis cs:2952000,473.13282)
--(axis cs:2988000,468.31128725)
--(axis cs:2988000,572.54452975)
--(axis cs:2988000,572.54452975)
--(axis cs:2952000,607.047773333333)
--(axis cs:2916000,581.955683333333)
--(axis cs:2880000,562.69891)
--(axis cs:2844000,570.399441666667)
--(axis cs:2808000,561.294171666667)
--(axis cs:2772000,575.424046666667)
--(axis cs:2736000,577.564579375)
--(axis cs:2700000,570.731558333333)
--(axis cs:2664000,556.645568333333)
--(axis cs:2628000,528.610636666667)
--(axis cs:2592000,572.906766666667)
--(axis cs:2556000,556.792031666667)
--(axis cs:2520000,565.548321666667)
--(axis cs:2484000,564.076643333333)
--(axis cs:2448000,554.428711666667)
--(axis cs:2412000,511.028458333333)
--(axis cs:2376000,539.327213333333)
--(axis cs:2340000,521.274179458333)
--(axis cs:2304000,507.011033333333)
--(axis cs:2268000,533.545403333333)
--(axis cs:2232000,530.436551666667)
--(axis cs:2196000,514.156165)
--(axis cs:2160000,548.6599)
--(axis cs:2124000,526.241701666667)
--(axis cs:2088000,488.449933333333)
--(axis cs:2052000,521.397975)
--(axis cs:2016000,489.724498333333)
--(axis cs:1980000,511.56874)
--(axis cs:1944000,469.79517)
--(axis cs:1908000,487.4035)
--(axis cs:1872000,452.261116666667)
--(axis cs:1836000,448.443288333333)
--(axis cs:1800000,470.013476541667)
--(axis cs:1764000,441.067136666667)
--(axis cs:1728000,425.983283333333)
--(axis cs:1692000,459.900376666667)
--(axis cs:1656000,420.555463333333)
--(axis cs:1620000,452.412428333333)
--(axis cs:1584000,426.080911666667)
--(axis cs:1548000,405.744391916667)
--(axis cs:1512000,422.468795)
--(axis cs:1476000,392.121763333333)
--(axis cs:1440000,398.927006666667)
--(axis cs:1404000,396.300881666667)
--(axis cs:1368000,391.708926666667)
--(axis cs:1332000,379.3367)
--(axis cs:1296000,374.028266666667)
--(axis cs:1260000,354.826833333333)
--(axis cs:1224000,325.77437)
--(axis cs:1188000,356.78896)
--(axis cs:1152000,352.524811666667)
--(axis cs:1116000,343.294016666667)
--(axis cs:1080000,331.116815)
--(axis cs:1044000,287.783165)
--(axis cs:1008000,319.938826666667)
--(axis cs:972000,297.00894)
--(axis cs:936000,297.43841)
--(axis cs:900000,295.160138333333)
--(axis cs:864000,278.186283333333)
--(axis cs:828000,267.117865)
--(axis cs:792000,254.04739775)
--(axis cs:756000,245.040613333333)
--(axis cs:720000,223.696456875)
--(axis cs:684000,216.1656)
--(axis cs:648000,205.401713333333)
--(axis cs:612000,201.404601666667)
--(axis cs:576000,182.55503)
--(axis cs:540000,177.786793333333)
--(axis cs:504000,170.561265208333)
--(axis cs:468000,170.450261666667)
--(axis cs:432000,164.02998)
--(axis cs:396000,153.633991666667)
--(axis cs:360000,143.37209)
--(axis cs:324000,136.525775833333)
--(axis cs:288000,129.716861683333)
--(axis cs:252000,113.1404125)
--(axis cs:216000,102.49002)
--(axis cs:180000,92.9197603333333)
--(axis cs:144000,76.8890066666667)
--(axis cs:108000,56.7536433333333)
--(axis cs:72000,4.25857906666667)
--(axis cs:36000,0.728938933333333)
--(axis cs:1,0.915501966666667)
--cycle;

\addplot [line width=\linewidthdime, C0, mark=*, mark size=0, mark options={solid}]
table {%
1 0.770562166666667
36000 0.658370533333333
72000 1.02744053333333
108000 27.4132359166667
144000 61.7389005
180000 78.3388245
216000 96.6475441666667
252000 104.292225666667
288000 118.336289333333
324000 122.156900833333
360000 127.865645666667
396000 135.589655833333
432000 144.61231
468000 150.876916666667
504000 155.106371666667
540000 161.848961666667
576000 162.873085
612000 175.66877
648000 181.774033333333
684000 189.795443333333
720000 201.694543333333
756000 209.679653333333
792000 215.950858333333
828000 226.452128333333
864000 235.790758333333
900000 245.745095
936000 251.419853333333
972000 253.923228333333
1008000 267.745066666667
1044000 253.029908333333
1080000 281.1757
1116000 290.478148333333
1152000 295.386573333333
1188000 307.510898333333
1224000 279.324226666667
1260000 298.499671666667
1296000 320.530285
1332000 314.688511666667
1368000 333.564955
1404000 344.035705
1440000 351.026203333333
1476000 334.48205
1512000 371.517083333333
1548000 359.107073333333
1584000 362.49581
1620000 383.735268333333
1656000 371.120853333333
1692000 392.875408333333
1728000 375.706403333333
1764000 385.148411666667
1800000 411.377563333333
1836000 393.022041666667
1872000 381.083618333333
1908000 428.514751666667
1944000 407.976545
1980000 440.522148333333
2016000 425.096363333333
2052000 460.246375
2088000 439.221346666667
2124000 470.482476666667
2160000 479.806223333333
2196000 467.028006666667
2232000 471.833656666667
2268000 492.305195
2304000 444.283438333333
2340000 467.734686666667
2376000 478.305563333333
2412000 479.659883333333
2448000 502.891963333333
2484000 512.254108333333
2520000 504.927955
2556000 509.89631
2592000 517.465808333333
2628000 505.843831666667
2664000 503.219516666667
2700000 525.334166666667
2736000 498.718568333333
2772000 515.273173333333
2808000 507.004858333333
2844000 490.845221666667
2880000 488.33129
2916000 530.288801666667
2952000 544.121438333333
2988000 506.351106666667
};
\addlegendentry{DIME}
\end{axis}

\end{tikzpicture}
}
       \subcaption[]{DIME and BRO on Humanoid Run}
       \label{fig::appendix_dime_bro_humanoid_run_long}
    \end{minipage}\hfill
    \caption{\textbf{Preliminary results for the GB sampler on the dog run (a) and humanoid run (b) environments from DMC. 
    Comparison to BRO on the humanoid run for 3 million steps. 
    }
    }
    \label{fig::appendix::prel::dbs}
\end{figure*}



%%%%%%%%%%%%%%%%%%%%%%%%%%%%%%%%%%%%%%%%%%%%%%%%%%%%%%%%%%%%%%%%%%%%%%%
\begin{figure*}[t!]
        \centering
        \begin{minipage}[t!]{\textwidth}
            \centering
            \begin{minipage}[t!]{0.32\textwidth}
            \includegraphics[width=\textwidth]{figures/illustrations/small_entropy.pdf}
            \subcaption[]{$\ \alpha < 1$}
            \end{minipage}
            \begin{minipage}[t!]{0.32\textwidth}
            \includegraphics[width=\textwidth]{figures/illustrations/normal_entropy.pdf}
            \subcaption[]{$\ \alpha = 1$}
            \end{minipage}
            \begin{minipage}[t!]{0.32\textwidth}
            \includegraphics[width=\textwidth]{figures/illustrations/big_entropy.pdf}
            \subcaption[]{$\ \alpha > 1$}
            \end{minipage}
        \end{minipage}
        \caption[ ]
        {\textbf{The effect of the reward scaling parameter $\alpha$}. The figures in (a)-(b) show diffusion processes for different $\alpha$ values starting at a prior distribution $\mathcal{N}(0,I)$ and going backward in time to approximate the target distribution $\exp{\left(Q^\pi/\alpha\right)}/Z^\pi$. Small values for $\alpha$ (a) lead to concentrated target distributions with less noise in the diffusion trajectories especially at the last time steps. The higher $\alpha$ becomes (b) and (c), the more the target distribution is smoothed and the distribution of the samples at the last time steps becomes more noisy. Therefore, the parameter $\alpha$ directly controls the exploration by enforcing noisier samples the higher $\alpha$ becomes.}
        \label{fig:entropies}
    \end{figure*}
%%%%%%%%%%%%%%%%%%%%%%%%%%%%%%%%%%%%%%%%%%%%%%%%%%%%%%%%%%%%%%%%%%%%%%%

