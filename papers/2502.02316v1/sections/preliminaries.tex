\section{Preliminaries}
\subsection{Maximum Entropy Reinforcement Learning}
\label{sec: Maximum Entropy Reinforcement Learning}
\textbf{Notation} We consider the task of learning a policy $\pi: \St \times \Ac \rightarrow \R^+$,
where $\St$ and $\Ac$ denote a continuous state and action space, respectively using reinforcement learning (RL). We formalize the RL problem using an infinite horizon Markov decision process consisting of the tuple $(\St,\Ac,r,p,\rho_{\pi}, \gamma)$, with bounded reward function $r: \St \times \Ac \rightarrow [r_{\text{min}}, r_{\text{max}}]$ and transition density $p: \St \times \St \times \Ac \rightarrow \R^+$ which denotes the likelihood for transitioning into a state $\st' \in \St$ when being in $\st \in \St$ and executing an action $\ac \in \Ac$. We follow \cite{haarnoja2018soft} and slightly overload $\rho_{\pi}$ which denotes the state and state-action marginals induced by a policy $\pi$. Moreover, $\gamma \in [0, 1)$ denotes the discount factor.
For brevity, we use $r_t\triangleq r(\st_t,\ac_t)$. Lastly, we denote objective functions that we aim to maximize as $J$ and minimize as $\mathcal{L}$.

\textbf{Control as inference.} The goal of maximum entropy reinforcement learning (MaxEnt-RL) is to jointly maximize the sum of expected rewards and entropies of a policy
\begin{equation}
\label{eq: marginal max ent}
    J(\pi) = \sum_{t=l}^{\infty} \gamma^{t-l}\E_{\rho_\pi}\left[r_t + \alpha \Ent(\pi(\ac_t|\st_t))\right],
\end{equation}
where $\Ent(\pi(\ac|\st)) = - \int \pi(\ac|\st) \log \pi(\ac|\st) \dd \ac$ is the differential entropy, and $\alpha \in \R^+$ controls the exploration exploitation trade-off \cite{haarnoja2017reinforcement}. To keep the notation uncluttered we absorb $ \alpha$ into the reward function via $r \leftarrow r/\alpha$. Defining the $Q$-function of a policy $\pi$ as
\begin{equation}
\label{eq: marginal Q soft}
Q^{\pi}(\st_t,\ac_t) = r_t + \sum_{l=1}^\infty\gamma^{l} \E_{\rho_{\pi}}\left[r_{t+l}+ \mathcal{H}\left(\pi(\ac_{t+l}|\st_{t+l})\right)\right],
\end{equation}
with $Q^{\pi}: \St \times \Ac \rightarrow \R$,
the MaxEnt objective can be cast as an approximate inference problem of the form
\begin{equation}
\label{eq: marginal control as inference}
\mathcal{L}(\pi) = D_{\text{KL}}\left(\pi(\ac_t|\st_t)\Big|\frac{\exp Q^{\pi}(\st_t,\ac_t)}{\Z^{\pi}(\st_t)}\right),
\end{equation}
in a sense that 
$
    \max_{\pi} J(\pi) = \min_{\pi} \mathcal{L}(\pi).
$
Here, $D_{\text{KL}}$ denotes the Kullback-Leibler divergence and 
\begin{equation}
\label{eq: normalizer}
\Z^{\pi}(\st) = \int \exp Q^{\pi}(\st,\ac) \dd \ac
\end{equation}
is the state-dependent normalization constant.

\textbf{Policy iteration} is a two-step iterative update scheme that is, under certain assumptions, guaranteed to converge to the optimal policy with respect to the maximum entropy objective. The two steps include policy evaluation and policy improvement. 
%
Given a policy $\pi$, policy evaluation aims to evaluate the value of $\pi$. To that end, \cite{haarnoja2018soft} showed that repeated application of the Bellman backup operator $\mathcal{T}^{\pi} Q^{k}$ with 
\begin{equation}
    \label{eq: bellman operator}
    \mathcal{T}^{\pi} Q(\st_t,\ac_t) \triangleq r_t + \gamma \E\left[Q(\st_{t+1},\ac_{t+1}) +\mathcal{H}(\ac_{t+1}|\st_{t+1})\right],
\end{equation}
converges to $Q^{\pi}$ as $k \rightarrow \infty$, starting from any $Q$.
%
To update the policy, that is, to perform the policy improvement step, the $Q$-function of the previous evaluation step, $Q^{\pi_{\text{old}}}$ is used to obtain a new policy according to 
\begin{equation}
\label{eq: marginal policy improvement}
\pi_{\text{new}} = \argmax_{\pi \in \Pi} D_{\text{KL}}\left(\pi(\ac_t|\st_t)\Big|\frac{\exp Q^{\pi_{\text{old}}}(\st_t,\ac_t)}{\Z^{\pi_{\text{old}}}(\st_t)}\right),
\end{equation}
where $\Pi$ is a set of policies such as a family of parameterized distributions.
Note that $\Z^{\pi_{\text{old}}}(\st_t)$ is not required for optimization as it is independent of $\pi$. \citet{haarnoja2018soft} showed that for all state-action pairs $(\st, \ac) \in \St \times \Ac$ it holds that $Q^{\pi_{\text{new}}}(\st,\ac) \geq Q^{\pi_{\text{old}}}(\st,\ac)$ ensuring that policy iteration converges to the optimal policy $\pi^*$ in the limit of infinite repetitions of policy evaluation and improvement.

\subsection{Denoising Diffusion Policies}
\label{sec: Denoising Diffusion Policies}
For a given state $\st \in \St$, we consider a stochastic process on the time-interval $[0,T]$ given by an Ornstein-Uhlenbeck (OU) process \footnote{Please note, for clarity, we slightly abuse notation by using $t$ to denote the time in the stochastic process. This should not be confused with the time step in RL. The distinction becomes clear when we discretize the processes.} \cite{sarkka2019applied}
\begin{equation}
\label{eq: noising process}
        \dd \ac_t  = -\beta_t \ac_t \dd t + \eta\sqrt{2\beta_t} \dd B_t, \quad a_0 \sim \fpi_0(\cdot|\st),
\end{equation}
with diffusion coefficient $\beta: [0,T]\rightarrow \R^+$, standard Brownian motion $(B_t)_{t\in[0,T]}$, and some target policy $\fpi_0$. 
For $t,l\in [0,T]$, we denote the marginal density of Eq. $\ref{eq: noising process}$ at $t$ as $\fpi_t$
% , the joint density at $s,t$ $\fpi_{s,t}$ 
and the conditional density at time $t$ given $l$ as $\fpi_{t|l}$.
Eq. \ref{eq: noising process} is commonly referred to as \textit{forward} or \textit{noising process} since, for a suitable choice of $\beta$, it holds that $\fpi_{T} \approx \mathcal{N}(0, \eta^2I)$. Denoising diffusion models leverage the fact, that the time-reversed process of Eq. \ref{eq: noising process} is given by 
\begin{equation}
\label{eq: denoising process}
        \dd \ac_t  = \left(-\beta_t \ac_t \dd t - 2\eta^2\beta_t \nabla \log \fpi_t(\ac_t|\st)\right) + \eta\sqrt{2\beta_t} \dd B_t,
\end{equation}
starting from $\bpi_T = \fpi_{T} \approx \mathcal{N}(0, \eta^2I)$ and running backwards in time \cite{nelson2020dynamical,anderson1982reverse,haussmann1986time}. For the \textit{backward}, \textit{generative} or \textit{denoising process} (Eq. \ref{eq: denoising process}), we denote the density as $\bpi$. Here, time-reversal means that the marginal densities align, i.e., $\fpi_t = \bpi_t$ for all $t\in[0, T]$. Hence, starting from $\ac_T \sim \mathcal{N}(0, \eta^2I)$, one can sample from the target policy $\fpi_0$ by simulating Eq. \ref{eq: denoising process}. However, for most densities $\fpi_0$, the scores $\left(\nabla \log \fpi_t(\ac_t|\st)\right)_{t\in[0,T]}$ are intractable, requiring numerical approximations. To address this, denoising score-matching objectives are commonly employed, that is, 
\begin{equation}
\label{eq: score matching}
\mathcal{L}_{\text{SM}}(\theta) = \E\left[\beta_t\|f^{\theta}_t(\ac_t,\st) - \nabla \log \fpi_{t|0}(\ac_t|\ac_0,\st) \|^2\right],
\end{equation}
where $t$ is sampled on $[0,T]$ and $f^{\theta}$ denotes a parameterized score network \cite{hyvarinen2005estimation,vincent2011connection}. For OU processes, the conditional densities $\nabla \log \fpi_{t|0}$ are explicitly computable, making the objective tractable for optimizing $\theta$ \cite{song2021scorebased}. Once trained, the score network $f^{\theta}$ can be used to simulate the denoising process 
\begin{equation}
\label{eq: approximate denoising process}
        \dd \ac_t  = \left(-\beta_t \ac_t \dd t - 2\eta^2\beta_t f^{\theta}_t(\ac_t,\st)\right) + \eta\sqrt{2\beta_t} \dd B_t,
\end{equation}
to obtain samples $\ac_0 \sim \ppi_0$ that are approximately distributed according to $\fpi_0$. Here, $\ppi_t$ denotes the marginal distribution of Eq. \ref{eq: approximate denoising process} at time $t$.
While score-matching techniques work well in practice, they cannot be applied to maximum entropy reinforcement learning. 
This is because the expectation in Eq. \ref{eq: score matching} requires samples $\ac_0 \sim \fpi_0 \propto \exp Q^{\pi}$ which are not available. However, in the next section, we build on recent advances in approximate inference to optimize diffusion models without requiring samples from $\ac_0$, relying instead on evaluations of  $Q^{\pi}$.

%
%
%
\section{Diffusion-Based Maximum Entropy RL}
\label{sec: Diffusion-Based Maximum Entropy RL}
Here, we explain how diffusion models can be used within a maximum entropy RL framework. To that end, we express the maximum entropy objective as an approximate inference problem for diffusion models. We then use these results to introduce a policy iteration scheme that provably converges to the optimal policy. Lastly, we propose a practical algorithm for optimizing diffusion models.
\subsection{Control as Inference for Diffusion Policies}
Directly maximizing the maximum entropy objective
\begin{equation*}
\label{eq: marginal diffusion max ent}
    J(\bpi) = \sum_{t=l}^{\infty} \gamma^{t-l}\E_{\rho_\pi}\left[r_t(\st_t,\ac^0_t) + \alpha \Ent(\bpi_0(\ac^0_t|\st_t))\right], 
\end{equation*}
for a diffusion model is difficult as the marginal entropy $\Ent(\bpi_0(\ac|\st))$ of the denoising process in Eq. \ref{eq: denoising process} is intractable.
Please note that we use superscripts for the actions to indicate the diffusion step to avoid collisions with the time step used in RL. Moreover, we will again absorb $\alpha$ into the reward and use  $r_t\triangleq r(\st_t,\ac^0_t)$.
To overcome this intractability, we propose to maximize a lower bound. We start by discretizing the stochastic processes introduced in \Cref{sec: Denoising Diffusion Policies} and use the results as a foundation to derive this lower bound. Note that while similar results can be derived from a continuous-time perspective (see e.g., \citet{berneroptimal,richterimproved,nusken2024transport}), such derivation would require a background in stochastic calculus, making it less accessible to a broader audience. %We therefore stay with the simpler, discrete-time formulation. 

The Euler-Maruyama (EM) discretization \cite{sarkka2019applied} of the noising (Eq. \ref{eq: noising process}) and denoising (Eq. \ref{eq: denoising process}) process is given by
\begin{align}
\label{eq: em discretized noising process}
        \ac^{n+1}  & = \ac^{n} -\beta_{n} \ac^{n} \delta + \epsilon_{n} \quad \text{and}
        \\ 
\label{eq: em discretized denoising process}
        \ac^{n-1}  & = \ac^{n} + \left(\beta_{n} \ac^{n} + 2\eta^2\beta_{n} \nabla \log \fpi_n(\ac^{n}|\st) \right)\delta + \xi_{n}, 
\end{align}
respectively, with $\epsilon_n,\xi_n \sim \mathcal{N}(0,2\eta^2\beta_{n}\delta I)$. Here, $\delta$ denotes a constant discretization step size such that $N = T / \delta$ is an integer. To simplify notation, we write $\ac^n$, instead of $\ac^{n\delta}$.  Under the EM discretization, the noising and denoising process admit the following joint distributions
\begin{align}
    \fpi_{0:N}(\ac^{0:N}|\st) &= \fpi_0(\ac^0|s) \prod_{n=0}^{N-1}\fpi_{n+1|n}(\ac^{n+1} \big| \ac^{n},\st), \label{eq: forward joint}\\
    \bpi_{0:N}(\ac^{0:N}|\st) &= \bpi_N (\ac^N|s) \prod_{n=1}^{N}\bpi_{n-1|n}(\ac^{n-1} \big| \ac^{n},\st), \label{eq: parameterized joint}
\end{align}
in a sense that $\fpi_{0:N}$ and $\bpi_{0:N}$ converge to the law of $(\ac_t)_{t\in[0,T]}$ in Eq. \ref{eq: noising process} and \ref{eq: denoising process}, as $\delta \rightarrow 0$, respectively \cite{doucet2022score}. Here, $\fpi_{n+1|n}$ and $\bpi_{n-1|n}$ are Gaussian transition densities that directly follow from Eq. \ref{eq: em discretized noising process} and \ref{eq: em discretized denoising process}.

To obtain a maximum entropy objective for diffusion models, we make use of the following lower bound on the marginal entropy, that is, $\Ent(\bpi_0(\ac_0|\st)) \geq \blb(\ac^{0},\st)$, where
\begin{equation}
\label{eq: entropy lower bound}
    \blb(\ac^{0},\st) =  \E_{\bpi_{0:N}}\left[\log \frac{\fpi_{1:N|0}(\ac^{1:N}|\ac^0,\st)}{\bpi_{0:N}(\ac^{0:N}|\st)}\right].
\end{equation}
Please note that similar bounds have been used, e.g., in \cite{agakov2004auxiliary,tran2015variational,ranganath2016hierarchical,maaloe2016auxiliary,arenz2018efficient}, or, more generally, follow from the data processing inequality \cite{cover1999elements}. 
A derivation can be found in Appendix \ref{APDX:DERIVATIONS}. From Eq. \ref{eq: entropy lower bound}, it directly follows that 
\begin{equation}
\label{eq: marginal max ent}
    J(\bpi) \geq \bar{J}(\bpi) = \sum_{t=l}^{\infty} \gamma^{t-l}\E_{\rho_\pi}\left[r_t + \blb(\ac^{0}_t,\st_t)\right].
\end{equation}
Next, we cast Eq. \ref{eq: marginal max ent} as an approximate inference problem to make the objective more interpretable. To that end, let us define the $Q$-function of a denoising policy $\bpi$ with respect to the maximum entropy objective $\bar J$ as
\begin{equation}
\label{eq: diffusion Q soft}
Q^{\bpi}(\st_t,\ac^0_t) = r_t + \sum_{l=1}\gamma^l \E_{\rho_{\pi}}\left[r_{t+l}+ \blb(\ac^{0}_{t+l},\st_{t+l})\right],
\end{equation}
with $Q^{\bpi}: \St \times \Ac \rightarrow \R$. With Eq. \ref{eq: diffusion Q soft} we identify the corresponding approximate inference problem as finding $\bpi$ which minimizes (please see Appendix \ref{APDX:DERIVATIONS} for derivation)
\begin{equation}
\label{eq: joint control as inference}
\bar{\mathcal{L}}(\bpi) = D_{\text{KL}}\left(\bpi_{0:N}(\ac^{0:N}|\st)|\fpi_{0:N}(\ac^{0:N}|\st)\right),
\end{equation}
where the target policy, i.e., the marginal of the noising process in Eq. \ref{eq: forward joint} is given by the exponentiated $Q$-function of the diffusion policy
\begin{equation}
    \fpi_{0}(\ac^{0}|\st) = \frac{\exp Q^{\bpi}(\st, \ac^0)}{\Z^{\bpi}(\st)}.
\end{equation}
Recall from \Cref{sec: Denoising Diffusion Policies} that we aim to time-reverse the noising process, that is, to ensure for all states $\st \in \St$, it holds that $\bpi_{0:N} = \fpi_{0:N}$. Please note that this is precisely what Eq. \ref{eq: joint control as inference} is trying to accomplish, i.e., we aim to learn a diffusion model $\bpi$, such that the denoising process time-reverses the noising process, and, in particular, has a marginal distribution given by $\pi_{0} = \exp Q^{\bpi}/\Z^{\bpi}$. Lastly, 
from the data processing inequality it directly follows that 
\begin{align}
\label{eq: data processing inequality}
  D_{\text{KL}}\bigg(\bpi_0(\ac^0|\st)&\Big|\frac{\exp Q^{\bpi}(\st,\ac^0)}{\Z^{\bpi}(\st)}\bigg) \nonumber
 \\
 & \leq D_{\text{KL}}\left(\bpi(\ac^{0:N}|\st)|\fpi(\ac^{0:N}|\st)\right),
\end{align}
which shows the approximate inference problem in Eq. \ref{eq: joint control as inference} indeed optimizes the same inference problem stated in Eq. \ref{eq: marginal control as inference}.
Next, we will use these results to develop a policy iteration scheme for diffusion models. 

\subsection{Diffusion-based Policy Iteration}
We propose a policy iteration scheme for learning an optimal maximum entropy policy, similar to \cite{haarnoja2018soft}. However, here we restrict the family of stochastic actors to diffusion policies $\bpi \in \cev{\Pi} \subset \Pi$. Throughout this section, we assume finite action spaces to enable theoretical analysis, but relax this assumption in \cref{dime_practical}. All proofs of this section are deferred to Appendix \ref{APDX:DERIVATIONS}. 
\todo[inline]{Do we need to assume finite action spaces? $\bpi$ is still assumed to be optimal. Should be relax the assumption here? FINITE ACTION SPACE}

For policy evaluation, we aim to compute the value of a policy $\bpi$. We define the Bellman backup operator as
\begin{equation}
    \label{eq: diffusion bellman operator}
    \mathcal{T}^{\bpi} Q(\st_t,\ac^0_t) \triangleq r_t + \gamma \E\left[Q(\st_{t+1},\ac^0_{t+1}) +\blb(\ac^{0}_{t+1},\st_{t+1})\right].
\end{equation}
Note that Eq. \ref{eq: diffusion bellman operator} contains the entropy-lower bound $\blb$. By applying standard convergence results for policy evaluation \cite{sutton1999reinforcement} we can obtain the value of a policy by repeatedly applying $\mathcal{T}^{\bpi}$ as established in \Cref{prop: policy evaluation}.

\begin{proposition}[Policy Evaluation]
\label{prop: policy evaluation}
Let $\mathcal{T}^{\bpi}$ be the Bellman backup operator for a diffusion policy $\bpi$ as defined in Eq. \ref{eq: diffusion bellman operator}. Further, let $Q^0: \St \times \Ac \rightarrow \R$ and $Q^{k+1} = \mathcal{T}^{\bpi}Q^k$.
Then, it holds that $\lim_{k\rightarrow \infty} Q^k =  Q^{\bpi}$ where $Q^{\bpi}$ is the $Q$ value of $\bpi$.
\end{proposition}

For the policy improvement step, we seek to improve the current policy based on its value using the $Q$-function. Formally, we need to solve the approximate inference problem
\begin{equation}
\label{eq: policy improvement objective}
\bpi^{\text{new}} = \argmin_{\bpi \in \cev{\Pi}}D_{\text{KL}}\left(\bpi_{0:N}(\ac^{0:N}|\st)|\fpi^{\text{ old}}_{0:N}(\ac^{0:N}|\st)\right),
\end{equation}
for all $\st\in\St$, where $\fpi^{\text{ old}}_{0:N}(\ac^{0:N}|\st)$ is as in Eq. \ref{eq: forward joint} with marginal density
\begin{equation}
\label{eq: old policy}
    \fpi^{\text{ old}}_{0}(\ac^{0}|\st) = \frac{\exp Q^{\bpi_{\text{old}}}(\st, \ac^0)}{\Z^{\bpi_{\text{old}}}(\st)}.
\end{equation}
Indeed, solving Eq. \ref{eq: policy improvement objective} results in a policy with higher value as established below.
\begin{proposition}[Policy Improvement]
\label{prop: policy improvement}
Let $\bpi_{\text{old}}, \bpi_{\text{new}} \in \cev{\Pi}$ be defined as in Eq. \ref{eq: old policy} and \ref{eq: policy improvement objective}, respectively. Then for all $(\st,\ac) \in \St \times \Ac$ it holds that $Q^{\bpi_{\text{new}}}(\st,\ac) \geq Q^{\bpi_{\text{old}}}(\st,\ac).$ 
\end{proposition}

Combining these results leads to the policy iteration method which alternates between policy evaluation (\Cref{prop: policy evaluation}) and policy improvement (\Cref{prop: policy improvement}) and provably converges to the optimal policy in $\cev{\Pi}$ (\Cref{prop: policy iteration}).

\begin{proposition}[Policy Iteration]
\label{prop: policy iteration}
Let $\bpi^0, \bpi^{i+1}, \bpi^i, \bpi_* \in \cev{\Pi}$. Further, let $\bpi^{i+1}$ be the policy obtained from $\bpi^{i}$ after a policy evaluation and improvement step. Then, for any starting policy $\bpi^0$ it holds that $\lim_{i\rightarrow \infty} \bpi^i =  \bpi^*$, with $\bpi^*$ such that for all $\bpi\in \cev{\Pi}$ and $(\st,\ac) \in \St \times \Ac$ it holds that $Q^{\bpi^*}(\st,\ac) \geq Q^{\bpi}(\st,\ac).$ 
\end{proposition}

However, performing policy iteration until convergence is in practice often intractable, particularly for continuous control tasks. As such, we will introduce a practical algorithm next.


\begin{figure*}[t!]
    \centering
    \resizebox{0.95\textwidth}{!}{
    \definecolor{crimson2143940}{RGB}{214,39,40}
\definecolor{darkorange25512714}{RGB}{255,127,14}
\definecolor{forestgreen4416044}{RGB}{44,160,44}
\definecolor{mediumpurple148103189}{RGB}{148,103,189}
\definecolor{steelblue31119180}{RGB}{31,119,180}
\definecolor{darkgray176}{RGB}{176,176,176}
\begin{tikzpicture} 
    \begin{axis}[%
    hide axis,
    xmin=10,
    xmax=50,
    ymin=0,
    ymax=0.4,
    legend style={
        draw=white!15!black,
        legend cell align=left,
        legend columns=-1, 
        legend style={
            draw=none,
            column sep=1ex,
            line width=0.5pt
        }
    },
    ]
    \addlegendimage{line width=2pt, color=C4}
    \addlegendentry{32};
    \addlegendimage{line width=2pt, color=C0}
    \addlegendentry{16};
    \addlegendimage{line width=2pt, color=C3}
    \addlegendentry{8};
    \addlegendimage{line width=2pt, color=C1}
    \addlegendentry{4};
    \addlegendimage{line width=2pt, color=C5}
    \addlegendentry{2};
    \end{axis}
\end{tikzpicture}}%
    
    \begin{minipage}[b]{0.26\textwidth}
        \centering
       \resizebox{1\textwidth}{!}{% This file was created with tikzplotlib v0.10.1.
\definecolor{crimson2143940}{RGB}{214,39,40}
\definecolor{darkorange25512714}{RGB}{255,127,14}
\definecolor{forestgreen4416044}{RGB}{44,160,44}
\definecolor{mediumpurple148103189}{RGB}{148,103,189}
\definecolor{steelblue31119180}{RGB}{31,119,180}
\definecolor{darkgray176}{RGB}{176,176,176}
\begin{tikzpicture}

\definecolor{darkcyan1115178}{RGB}{1,115,178}
\definecolor{darkgray176}{RGB}{176,176,176}

\begin{axis}[
legend cell align={left},
legend style={fill opacity=0.8, draw opacity=1, text opacity=1, draw=lightgray204, at={(0.03,0.03)},  anchor=north west},
tick align=outside,
tick pos=left,
x grid style={white},
xlabel={Number Env Interactions},
xmajorgrids,
%xmin=-149398.95, xmax=3137399.95,
xmin=0.0, xmax=3000000.00,
xtick style={color=black},
y grid style={white},
ylabel={IQM Return},
ymajorgrids,
ymin=-35.861410165, ymax=877.233399531667,
ytick style={color=black},
axis background/.style={fill=plot_background},
label style={font=\large},
tick label style={font=\large},
x axis line style={draw=none},
y axis line style={draw=none},
]
\path [draw=C0, fill=C0, opacity=0.2]
(axis cs:1,6.77901475791667)
--(axis cs:1,5.64291553333333)
--(axis cs:75000,19.1597246666667)
--(axis cs:150000,62.7155849916667)
--(axis cs:225000,96.3693566666667)
--(axis cs:300000,126.079227583333)
--(axis cs:375000,149.13536)
--(axis cs:450000,175.421665)
--(axis cs:525000,186.97005)
--(axis cs:600000,212.94413)
--(axis cs:675000,217.86526)
--(axis cs:750000,260.150573333333)
--(axis cs:825000,269.859375)
--(axis cs:900000,260.430376541667)
--(axis cs:975000,340.899176666667)
--(axis cs:1050000,358.012548333333)
--(axis cs:1125000,385.657351708333)
--(axis cs:1200000,379.184725208333)
--(axis cs:1275000,433.776411666667)
--(axis cs:1350000,446.441923333333)
--(axis cs:1425000,479.320458333333)
--(axis cs:1500000,498.277125)
--(axis cs:1575000,516.634003333333)
--(axis cs:1650000,513.550598333333)
--(axis cs:1725000,519.899135)
--(axis cs:1800000,540.786894291667)
--(axis cs:1875000,526.786516666667)
--(axis cs:1950000,600.207093333333)
--(axis cs:2025000,601.858965)
--(axis cs:2100000,621.464493333333)
--(axis cs:2175000,634.164003333333)
--(axis cs:2250000,632.273245)
--(axis cs:2325000,637.152511666667)
--(axis cs:2400000,647.836116666667)
--(axis cs:2475000,649.0113)
--(axis cs:2550000,678.132627083333)
--(axis cs:2625000,674.6811)
--(axis cs:2700000,665.833213333333)
--(axis cs:2775000,697.350276666667)
--(axis cs:2850000,696.739206666667)
--(axis cs:2925000,703.537966666667)
--(axis cs:3000000,687.357123333333)
--(axis cs:3000000,743.104029916667)
--(axis cs:3000000,743.104029916667)
--(axis cs:2925000,754.299846666667)
--(axis cs:2850000,731.263983333333)
--(axis cs:2775000,742.4798)
--(axis cs:2700000,725.915376666667)
--(axis cs:2625000,728.106643333333)
--(axis cs:2550000,725.096276666667)
--(axis cs:2475000,713.061273333333)
--(axis cs:2400000,697.68279)
--(axis cs:2325000,700.821681166667)
--(axis cs:2250000,687.937288333333)
--(axis cs:2175000,687.74969125)
--(axis cs:2100000,680.6291)
--(axis cs:2025000,656.15625)
--(axis cs:1950000,663.508993333333)
--(axis cs:1875000,635.86965)
--(axis cs:1800000,642.717033333333)
--(axis cs:1725000,617.434283333333)
--(axis cs:1650000,602.215845)
--(axis cs:1575000,590.692205)
--(axis cs:1500000,578.370525)
--(axis cs:1425000,564.136311666667)
--(axis cs:1350000,542.01851)
--(axis cs:1275000,536.479591666667)
--(axis cs:1200000,504.493398333333)
--(axis cs:1125000,492.82073)
--(axis cs:1050000,473.255681666667)
--(axis cs:975000,446.901898333333)
--(axis cs:900000,407.875456666667)
--(axis cs:825000,388.210466666667)
--(axis cs:750000,356.76231)
--(axis cs:675000,315.687453333333)
--(axis cs:600000,285.150741666667)
--(axis cs:525000,250.157765)
--(axis cs:450000,205.673518333333)
--(axis cs:375000,176.253973333333)
--(axis cs:300000,160.901923333333)
--(axis cs:225000,131.07506)
--(axis cs:150000,82.5371166666667)
--(axis cs:75000,23.7965627)
--(axis cs:1,6.77901475791667)
--cycle;

\addplot [line width=\linewidthdime, C0, mark=*, mark size=0, mark options={solid}]
table {%
1 6.21731321666667
75000 21.795235
150000 72.75661
225000 114.506553333333
300000 146.230666666667
375000 168.785311666667
450000 192.863928333333
525000 216.569268333333
600000 250.331875
675000 262.220136666667
750000 309.84049
825000 323.952626666667
900000 351.00367
975000 401.235143333333
1050000 429.149441666667
1125000 451.573716666667
1200000 462.784196666667
1275000 498.048691666667
1350000 503.234986666667
1425000 530.236976666667
1500000 546.020048333333
1575000 557.309815
1650000 569.905348333333
1725000 576.280983333333
1800000 596.698425
1875000 599.94853
1950000 638.555295
2025000 635.158215
2100000 660.178686666667
2175000 666.541921666667
2250000 660.835705
2325000 677.928385
2400000 678.28305
2475000 691.880483333333
2550000 701.649325
2625000 699.744298333333
2700000 694.774641666667
2775000 718.530593333333
2850000 713.703046666667
2925000 729.128026666667
3000000 713.8313
};
\path [draw=C1, fill=C1, opacity=0.2]
(axis cs:1,6.77779228333333)
--(axis cs:1,5.6454462)
--(axis cs:75000,13.192111)
--(axis cs:150000,33.373557)
--(axis cs:225000,61.2938256666667)
--(axis cs:300000,120.016403166667)
--(axis cs:375000,171.7728935)
--(axis cs:450000,220.303233333333)
--(axis cs:525000,287.51594)
--(axis cs:600000,345.985706666667)
--(axis cs:675000,375.99755)
--(axis cs:750000,385.816735)
--(axis cs:825000,484.023426666667)
--(axis cs:900000,500.420616666667)
--(axis cs:975000,528.7232)
--(axis cs:1050000,485.542121666667)
--(axis cs:1125000,548.90671)
--(axis cs:1200000,577.96613)
--(axis cs:1275000,618.55874)
--(axis cs:1350000,625.843644625)
--(axis cs:1425000,620.230685)
--(axis cs:1500000,670.084116666667)
--(axis cs:1575000,666.705233333333)
--(axis cs:1650000,698.7523)
--(axis cs:1725000,692.89836)
--(axis cs:1800000,703.88816)
--(axis cs:1875000,717.897567041667)
--(axis cs:1950000,720.886863333333)
--(axis cs:2025000,645.359551666667)
--(axis cs:2100000,595.2405225)
--(axis cs:2175000,723.713683333333)
--(axis cs:2250000,756.195295)
--(axis cs:2325000,771.088566666667)
--(axis cs:2400000,668.475967916667)
--(axis cs:2475000,743.654456666667)
--(axis cs:2550000,736.213633333333)
--(axis cs:2625000,749.79453)
--(axis cs:2700000,736.285061666667)
--(axis cs:2775000,775.64339)
--(axis cs:2850000,782.48355)
--(axis cs:2925000,780.489636666667)
--(axis cs:3000000,770.676025)
--(axis cs:3000000,827.493025)
--(axis cs:3000000,827.493025)
--(axis cs:2925000,834.8158)
--(axis cs:2850000,823.143141666667)
--(axis cs:2775000,830.781833333333)
--(axis cs:2700000,824.169479666667)
--(axis cs:2625000,833.258581333333)
--(axis cs:2550000,816.335153333333)
--(axis cs:2475000,835.308423333333)
--(axis cs:2400000,839.038746666667)
--(axis cs:2325000,819.740543333333)
--(axis cs:2250000,818.680797708333)
--(axis cs:2175000,803.737111666667)
--(axis cs:2100000,790.368157)
--(axis cs:2025000,806.313896666667)
--(axis cs:1950000,805.564308333333)
--(axis cs:1875000,789.8978)
--(axis cs:1800000,785.000679166667)
--(axis cs:1725000,767.124138333333)
--(axis cs:1650000,759.130021666667)
--(axis cs:1575000,743.323833333333)
--(axis cs:1500000,731.116993333333)
--(axis cs:1425000,701.952718333333)
--(axis cs:1350000,698.797891666667)
--(axis cs:1275000,668.436818333333)
--(axis cs:1200000,646.790078333333)
--(axis cs:1125000,633.005033333333)
--(axis cs:1050000,596.15743)
--(axis cs:975000,600.356563333333)
--(axis cs:900000,556.532176666667)
--(axis cs:825000,528.821698333333)
--(axis cs:750000,484.572864875)
--(axis cs:675000,456.144685)
--(axis cs:600000,420.356471666667)
--(axis cs:525000,362.974098333333)
--(axis cs:450000,313.987165)
--(axis cs:375000,254.932186666667)
--(axis cs:300000,184.542658333333)
--(axis cs:225000,128.248941666667)
--(axis cs:150000,81.5557691666667)
--(axis cs:75000,38.642973)
--(axis cs:1,6.77779228333333)
--cycle;

\addplot [line width=\linewidthdime, C1, mark=*, mark size=0, mark options={solid}]
table {%
1 6.21731321666667
75000 23.234224
150000 42.76285
225000 96.4487016666667
300000 161.580498333333
375000 206.671753333333
450000 257.980293333333
525000 318.376303333333
600000 386.66074
675000 412.943236666667
750000 436.81992
825000 504.845973333333
900000 526.84992
975000 566.728766666667
1050000 562.233523333333
1125000 598.26834
1200000 611.084238333333
1275000 640.080303333333
1350000 662.24455
1425000 669.740306666667
1500000 701.389226666667
1575000 699.021706666667
1650000 731.213333333333
1725000 728.181965
1800000 743.476393333333
1875000 759.708911666667
1950000 764.764993333333
2025000 779.829181666667
2100000 750.560726666667
2175000 766.374685
2250000 789.383758333333
2325000 801.652243333333
2400000 807.818806666667
2475000 802.767416666667
2550000 780.608276666667
2625000 805.666888333333
2700000 798.535888333333
2775000 805.852318333333
2850000 810.52055
2925000 821.589403333333
3000000 805.513291666667
};
\path [draw=C3, fill=C3, opacity=0.2]
(axis cs:1,6.7789808)
--(axis cs:1,5.64503465)
--(axis cs:75000,15.4860575)
--(axis cs:150000,50.3197336666667)
--(axis cs:225000,117.495146666667)
--(axis cs:300000,196.20587)
--(axis cs:375000,248.847483333333)
--(axis cs:450000,241.23715)
--(axis cs:525000,282.718593333333)
--(axis cs:600000,346.804435)
--(axis cs:675000,374.878242)
--(axis cs:750000,403.88618)
--(axis cs:825000,431.327293333333)
--(axis cs:900000,427.045318333333)
--(axis cs:975000,467.49293425)
--(axis cs:1050000,470.675713333333)
--(axis cs:1125000,498.166373333333)
--(axis cs:1200000,524.61077)
--(axis cs:1275000,526.337566666667)
--(axis cs:1350000,543.514556666667)
--(axis cs:1425000,557.9416)
--(axis cs:1500000,547.438680583334)
--(axis cs:1575000,584.22789)
--(axis cs:1650000,590.02265)
--(axis cs:1725000,596.489593333333)
--(axis cs:1800000,604.379791875)
--(axis cs:1875000,612.749605)
--(axis cs:1950000,615.377191666667)
--(axis cs:2025000,613.665313333333)
--(axis cs:2100000,623.273542291667)
--(axis cs:2175000,625.9463)
--(axis cs:2250000,612.86094)
--(axis cs:2325000,635.473057291667)
--(axis cs:2400000,637.935976666667)
--(axis cs:2475000,647.339)
--(axis cs:2550000,643.338021666667)
--(axis cs:2625000,661.016133333333)
--(axis cs:2700000,642.8757)
--(axis cs:2775000,648.861888333333)
--(axis cs:2850000,655.498358333333)
--(axis cs:2925000,668.790966666667)
--(axis cs:3000000,669.531625)
--(axis cs:3000000,797.328245)
--(axis cs:3000000,797.328245)
--(axis cs:2925000,812.381966666667)
--(axis cs:2850000,811.048568333333)
--(axis cs:2775000,805.738583333333)
--(axis cs:2700000,812.97896)
--(axis cs:2625000,815.187776666667)
--(axis cs:2550000,792.325869166667)
--(axis cs:2475000,791.855291666667)
--(axis cs:2400000,787.386453333333)
--(axis cs:2325000,755.803311666667)
--(axis cs:2250000,771.904865)
--(axis cs:2175000,763.846793333333)
--(axis cs:2100000,768.0811055)
--(axis cs:2025000,768.096478041667)
--(axis cs:1950000,762.823413333333)
--(axis cs:1875000,746.973033333333)
--(axis cs:1800000,738.54912)
--(axis cs:1725000,730.684051666667)
--(axis cs:1650000,719.498785)
--(axis cs:1575000,716.99392)
--(axis cs:1500000,695.735943333333)
--(axis cs:1425000,695.702793333333)
--(axis cs:1350000,670.685723333333)
--(axis cs:1275000,662.224783333333)
--(axis cs:1200000,658.009566666667)
--(axis cs:1125000,627.17524225)
--(axis cs:1050000,604.48534)
--(axis cs:975000,597.68689)
--(axis cs:900000,581.189353333333)
--(axis cs:825000,548.939053333333)
--(axis cs:750000,516.750406666667)
--(axis cs:675000,500.86067)
--(axis cs:600000,457.445263333333)
--(axis cs:525000,416.607634458333)
--(axis cs:450000,391.208711666667)
--(axis cs:375000,343.01117)
--(axis cs:300000,283.617185)
--(axis cs:225000,220.604813333333)
--(axis cs:150000,140.60036)
--(axis cs:75000,40.5402306666667)
--(axis cs:1,6.7789808)
--cycle;

\addplot [line width =\linewidthdime, C3, mark=*, mark size=0, mark options={solid}]
table {%
1 6.21731321666667
75000 24.6861146666667
150000 92.4884813333333
225000 160.813744333333
300000 241.682315
375000 306.650898333333
450000 321.438153333333
525000 352.84999
600000 399.418973333333
675000 436.593163333333
750000 464.165533333333
825000 493.630033333333
900000 528.104716666667
975000 545.41824
1050000 552.732928333333
1125000 566.65974
1200000 607.605821666667
1275000 604.850893333333
1350000 622.615008333333
1425000 641.52472
1500000 632.594268333333
1575000 668.763421666667
1650000 675.501586666667
1725000 685.480978333333
1800000 694.880568333333
1875000 697.229521666667
1950000 707.730915
2025000 715.729543333333
2100000 711.488441666667
2175000 711.370843333333
2250000 706.43842
2325000 714.780741666667
2400000 736.155055
2475000 744.774666666667
2550000 745.894818333333
2625000 763.089821666667
2700000 756.2565
2775000 747.494315
2850000 757.639806666667
2925000 765.57981
3000000 765.790613333333
};
\path [draw=C4, fill=C4, opacity=0.2]
(axis cs:1,6.78237173333333)
--(axis cs:1,5.64289936666667)
--(axis cs:75000,17.9192972583333)
--(axis cs:150000,63.4254316666667)
--(axis cs:225000,126.808483525)
--(axis cs:300000,171.556595)
--(axis cs:375000,233.989082708333)
--(axis cs:450000,282.56608)
--(axis cs:525000,318.172716875)
--(axis cs:600000,315.922775)
--(axis cs:675000,388.286496666667)
--(axis cs:750000,406.59078)
--(axis cs:825000,432.7052)
--(axis cs:900000,445.606696666667)
--(axis cs:975000,447.969856666667)
--(axis cs:1050000,450.773615)
--(axis cs:1125000,464.76792925)
--(axis cs:1200000,468.25359)
--(axis cs:1275000,484.874891666667)
--(axis cs:1350000,485.228771666667)
--(axis cs:1425000,492.358671666667)
--(axis cs:1500000,501.927151666667)
--(axis cs:1575000,508.970951666667)
--(axis cs:1650000,512.376186666667)
--(axis cs:1725000,502.766398333333)
--(axis cs:1800000,510.054328333333)
--(axis cs:1875000,507.44308)
--(axis cs:1950000,526.993976666667)
--(axis cs:2025000,519.400726666667)
--(axis cs:2100000,529.391925)
--(axis cs:2175000,533.250456666667)
--(axis cs:2250000,538.98909)
--(axis cs:2325000,505.917946666667)
--(axis cs:2400000,543.547425458333)
--(axis cs:2475000,500.533943333333)
--(axis cs:2550000,540.943844833333)
--(axis cs:2625000,552.334016666667)
--(axis cs:2700000,474.275015)
--(axis cs:2775000,550.181803333333)
--(axis cs:2850000,542.496643333333)
--(axis cs:2925000,561.68943)
--(axis cs:3000000,556.49008)
--(axis cs:3000000,695.666756666667)
--(axis cs:3000000,695.666756666667)
--(axis cs:2925000,703.057563333333)
--(axis cs:2850000,704.38867)
--(axis cs:2775000,680.282676666667)
--(axis cs:2700000,676.166277083334)
--(axis cs:2625000,696.227573333333)
--(axis cs:2550000,678.491101666667)
--(axis cs:2475000,665.037966666667)
--(axis cs:2400000,664.2123185)
--(axis cs:2325000,670.133466666667)
--(axis cs:2250000,666.698308333333)
--(axis cs:2175000,666.84069)
--(axis cs:2100000,643.761933333333)
--(axis cs:2025000,636.763675)
--(axis cs:1950000,633.729883333333)
--(axis cs:1875000,631.430348333333)
--(axis cs:1800000,625.901868333333)
--(axis cs:1725000,608.686382708333)
--(axis cs:1650000,601.747791666667)
--(axis cs:1575000,609.359848333333)
--(axis cs:1500000,578.848328333333)
--(axis cs:1425000,580.778946666667)
--(axis cs:1350000,577.265266666667)
--(axis cs:1275000,574.183852625)
--(axis cs:1200000,563.541845)
--(axis cs:1125000,547.654096666667)
--(axis cs:1050000,523.382258333333)
--(axis cs:975000,518.584443333333)
--(axis cs:900000,488.865946666667)
--(axis cs:825000,490.32467)
--(axis cs:750000,465.269181666667)
--(axis cs:675000,441.94512)
--(axis cs:600000,429.324076666667)
--(axis cs:525000,402.110298333333)
--(axis cs:450000,377.434965)
--(axis cs:375000,341.79572)
--(axis cs:300000,271.953795)
--(axis cs:225000,184.835081666667)
--(axis cs:150000,126.32286025)
--(axis cs:75000,34.7162325)
--(axis cs:1,6.78237173333333)
--cycle;

\addplot [line width=\linewidthdime, C4, mark=*, mark size=0, mark options={solid}]
table {%
1 6.21731321666667
75000 27.5053421666667
150000 89.2824941666667
225000 151.325434
300000 230.389185
375000 302.202218333333
450000 343.977808333333
525000 368.504448333333
600000 381.607665
675000 424.407763333333
750000 439.562763333333
825000 466.002391666667
900000 470.92938
975000 490.342795
1050000 487.374173333333
1125000 502.150376666667
1200000 515.912985
1275000 526.891398333333
1350000 518.929268333333
1425000 528.792513333333
1500000 538.664986666667
1575000 553.59301
1650000 545.294433333333
1725000 547.315965
1800000 557.023976666667
1875000 566.952106666667
1950000 573.97352
2025000 559.881641666667
2100000 579.680123333333
2175000 586.648251666667
2250000 584.81991
2325000 593.946286666667
2400000 584.503836666667
2475000 573.781458333333
2550000 597.080675
2625000 618.949693333333
2700000 595.190093333333
2775000 607.322023333333
2850000 617.890743333333
2925000 623.280883333333
3000000 613.856913333333
};
\path [draw=C5, fill=C5, opacity=0.2]
(axis cs:1,6.7789808)
--(axis cs:1,5.6454462)
--(axis cs:75000,12.4129023333333)
--(axis cs:150000,18.691406)
--(axis cs:225000,46.6979821666667)
--(axis cs:300000,82.8818396666667)
--(axis cs:375000,97.746777)
--(axis cs:450000,149.115790666667)
--(axis cs:525000,174.960733333333)
--(axis cs:600000,156.887033)
--(axis cs:675000,230.567503666667)
--(axis cs:750000,239.808616958333)
--(axis cs:825000,263.1599305)
--(axis cs:900000,255.552918333333)
--(axis cs:975000,285.179116333333)
--(axis cs:1050000,287.863883333333)
--(axis cs:1125000,313.676190666667)
--(axis cs:1200000,339.857216333333)
--(axis cs:1275000,336.764319166667)
--(axis cs:1350000,354.8411225)
--(axis cs:1425000,355.013705)
--(axis cs:1500000,382.574541666667)
--(axis cs:1575000,394.200431666667)
--(axis cs:1650000,398.402556666667)
--(axis cs:1725000,317.105807458333)
--(axis cs:1800000,405.750238333333)
--(axis cs:1875000,420.442273333333)
--(axis cs:1950000,421.989668333333)
--(axis cs:2025000,426.871725)
--(axis cs:2100000,426.939376666667)
--(axis cs:2175000,394.635271333333)
--(axis cs:2250000,444.618741791667)
--(axis cs:2325000,389.33789)
--(axis cs:2400000,448.154216666667)
--(axis cs:2475000,460.094523333333)
--(axis cs:2550000,356.416150066667)
--(axis cs:2625000,421.063358333333)
--(axis cs:2700000,484.024856666667)
--(axis cs:2775000,486.19462)
--(axis cs:2850000,483.03179)
--(axis cs:2925000,499.572325)
--(axis cs:3000000,476.448261666667)
--(axis cs:3000000,686.118883333333)
--(axis cs:3000000,686.118883333333)
--(axis cs:2925000,706.853655)
--(axis cs:2850000,694.190088333333)
--(axis cs:2775000,702.19725)
--(axis cs:2700000,693.992685)
--(axis cs:2625000,683.2654)
--(axis cs:2550000,641.942895)
--(axis cs:2475000,678.364916666667)
--(axis cs:2400000,665.692208333333)
--(axis cs:2325000,653.163586666667)
--(axis cs:2250000,652.23255)
--(axis cs:2175000,631.806883333333)
--(axis cs:2100000,606.8741)
--(axis cs:2025000,619.2095)
--(axis cs:1950000,605.40168)
--(axis cs:1875000,615.65135)
--(axis cs:1800000,591.334628333333)
--(axis cs:1725000,590.217716666667)
--(axis cs:1650000,571.499675)
--(axis cs:1575000,574.826313333333)
--(axis cs:1500000,558.05845)
--(axis cs:1425000,538.23778)
--(axis cs:1350000,508.830476666667)
--(axis cs:1275000,504.647073333333)
--(axis cs:1200000,486.207706875)
--(axis cs:1125000,464.047959583333)
--(axis cs:1050000,441.110606666667)
--(axis cs:975000,439.210568333333)
--(axis cs:900000,413.693958333333)
--(axis cs:825000,393.532926666667)
--(axis cs:750000,368.353783333333)
--(axis cs:675000,349.156483333333)
--(axis cs:600000,312.048603333333)
--(axis cs:525000,292.259208333333)
--(axis cs:450000,261.419155)
--(axis cs:375000,230.88683)
--(axis cs:300000,192.616263333333)
--(axis cs:225000,125.033163333333)
--(axis cs:150000,62.7031406666667)
--(axis cs:75000,29.7667333333333)
--(axis cs:1,6.7789808)
--cycle;

\addplot [line width=\linewidthdime, C5, mark=*, mark size=0, mark options={solid}]
table {%
1 6.21731321666667
75000 20.8227488333333
150000 39.090304
225000 83.6150591666667
300000 126.572516666667
375000 168.156125
450000 208.803878333333
525000 236.390621666667
600000 255.977703333333
675000 307.787093333333
750000 321.887206666667
825000 341.583128333333
900000 349.487541666667
975000 364.56241
1050000 370.04829
1125000 390.933865
1200000 418.560173333333
1275000 429.208185
1350000 445.484876666667
1425000 461.478426666667
1500000 481.286283333333
1575000 493.300616666667
1650000 489.189401666667
1725000 490.588285
1800000 508.003011666667
1875000 526.332678333333
1950000 521.157121666667
2025000 533.048061666667
2100000 520.758193333333
2175000 541.62375
2250000 548.168346666667
2325000 537.054751666667
2400000 566.576028333333
2475000 564.040543333333
2550000 531.352275
2625000 554.613958333333
2700000 593.460943333333
2775000 595.345438333333
2850000 585.084225
2925000 606.38759
3000000 575.508026666667
};
\path [draw=C8, fill=C8, opacity=0.2]
(axis cs:1,6.7789808)
--(axis cs:1,5.64291553333333)
--(axis cs:75000,9.4124673)
--(axis cs:150000,17.3314582333333)
--(axis cs:225000,52.9633973333333)
--(axis cs:300000,101.5769675)
--(axis cs:375000,102.222525166667)
--(axis cs:450000,142.5976928625)
--(axis cs:525000,164.963788375)
--(axis cs:600000,190.225708)
--(axis cs:675000,205.045780333333)
--(axis cs:750000,229.856895)
--(axis cs:825000,226.751546233333)
--(axis cs:900000,256.493093833333)
--(axis cs:975000,251.30497)
--(axis cs:1050000,276.553881666667)
--(axis cs:1125000,286.622266666667)
--(axis cs:1200000,289.235215)
--(axis cs:1275000,296.566369791667)
--(axis cs:1350000,302.247237333333)
--(axis cs:1425000,311.288254166667)
--(axis cs:1500000,317.945753666667)
--(axis cs:1575000,321.976207333333)
--(axis cs:1650000,326.695174)
--(axis cs:1725000,321.680736)
--(axis cs:1800000,335.983499041667)
--(axis cs:1875000,328.8955375)
--(axis cs:1950000,334.52118)
--(axis cs:2025000,345.772400625)
--(axis cs:2100000,350.838541916667)
--(axis cs:2175000,346.9468675)
--(axis cs:2250000,365.222333333333)
--(axis cs:2325000,344.562520958333)
--(axis cs:2400000,367.718743333333)
--(axis cs:2475000,376.752096666667)
--(axis cs:2550000,358.89941)
--(axis cs:2625000,377.792513333333)
--(axis cs:2700000,373.528251666667)
--(axis cs:2775000,377.020613333333)
--(axis cs:2850000,372.75245)
--(axis cs:2925000,364.258817333333)
--(axis cs:3000000,377.686428333333)
--(axis cs:3000000,554.959116666667)
--(axis cs:3000000,554.959116666667)
--(axis cs:2925000,559.12053)
--(axis cs:2850000,558.602342583334)
--(axis cs:2775000,537.087753333333)
--(axis cs:2700000,548.07257)
--(axis cs:2625000,548.963626666667)
--(axis cs:2550000,547.906388333333)
--(axis cs:2475000,542.11712)
--(axis cs:2400000,532.937713333333)
--(axis cs:2325000,513.948656666667)
--(axis cs:2250000,519.10051)
--(axis cs:2175000,524.497208333333)
--(axis cs:2100000,504.955276666667)
--(axis cs:2025000,500.758661666667)
--(axis cs:1950000,481.218491666667)
--(axis cs:1875000,475.814105)
--(axis cs:1800000,476.492296666667)
--(axis cs:1725000,478.645116666667)
--(axis cs:1650000,477.239605)
--(axis cs:1575000,489.645466666667)
--(axis cs:1500000,464.911833333333)
--(axis cs:1425000,464.36491)
--(axis cs:1350000,456.02617)
--(axis cs:1275000,443.059238333333)
--(axis cs:1200000,437.555213333333)
--(axis cs:1125000,430.863184333333)
--(axis cs:1050000,419.827471666667)
--(axis cs:975000,408.759126666667)
--(axis cs:900000,391.891775)
--(axis cs:825000,377.6072)
--(axis cs:750000,373.54594)
--(axis cs:675000,349.649803333333)
--(axis cs:600000,327.898075)
--(axis cs:525000,302.575626666667)
--(axis cs:450000,268.6613)
--(axis cs:375000,239.243746666667)
--(axis cs:300000,189.132441666667)
--(axis cs:225000,160.150965833333)
--(axis cs:150000,78.2773068333333)
--(axis cs:75000,17.4212863333333)
--(axis cs:1,6.7789808)
--cycle;

\addplot [line width=\linewidthdime, C8, mark=*, mark size=0, mark options={solid}]
table {%
1 6.21731321666667
75000 13.47092
150000 33.3836346666667
225000 98.6985431666667
300000 144.634583166667
375000 175.533951666667
450000 209.989251666667
525000 242.447783333333
600000 277.715508333333
675000 294.81058
750000 320.20364
825000 318.6075
900000 338.964641666667
975000 349.526221666667
1050000 368.305633333333
1125000 379.405123333333
1200000 380.97295
1275000 390.509931666667
1350000 398.27875
1425000 417.129783333333
1500000 414.996983333333
1575000 425.770446666667
1650000 426.320526666667
1725000 422.677443333333
1800000 429.630751666667
1875000 420.704703333333
1950000 424.294305
2025000 444.550001666667
2100000 447.655826666667
2175000 453.966486666667
2250000 457.136071666667
2325000 431.689171666667
2400000 460.328878333333
2475000 466.293323333333
2550000 458.64369
2625000 466.096148333333
2700000 458.371545
2775000 453.671963333333
2850000 462.506571666667
2925000 473.529711666667
3000000 461.190121666667
};
\path [draw=C7, fill=C7, opacity=0.2]
(axis cs:1,6.77576936666667)
--(axis cs:1,5.64289936666667)
--(axis cs:75000,3.50145868333333)
--(axis cs:150000,3.03823495)
--(axis cs:225000,7.80344773333333)
--(axis cs:300000,15.1959053333333)
--(axis cs:375000,18.5034482)
--(axis cs:450000,39.189309)
--(axis cs:525000,36.4076191666667)
--(axis cs:600000,45.5674847650004)
--(axis cs:675000,88.4275291666667)
--(axis cs:750000,90.2524866666667)
--(axis cs:825000,96.541988)
--(axis cs:900000,76.6027375)
--(axis cs:975000,119.990786266667)
--(axis cs:1050000,104.8416875)
--(axis cs:1125000,138.4211625)
--(axis cs:1200000,140.496076666667)
--(axis cs:1275000,151.037835)
--(axis cs:1350000,167.506894633334)
--(axis cs:1425000,151.268965041667)
--(axis cs:1500000,177.999501666667)
--(axis cs:1575000,150.894498333333)
--(axis cs:1650000,188.924378333333)
--(axis cs:1725000,171.972185)
--(axis cs:1800000,224.47347)
--(axis cs:1875000,206.59252)
--(axis cs:1950000,243.250116666667)
--(axis cs:2025000,238.664676666667)
--(axis cs:2100000,256.360866666667)
--(axis cs:2175000,261.019045)
--(axis cs:2250000,274.127406666667)
--(axis cs:2325000,286.401523333333)
--(axis cs:2400000,262.819526666667)
--(axis cs:2475000,302.762569791667)
--(axis cs:2550000,176.224818333333)
--(axis cs:2625000,320.492048333333)
--(axis cs:2700000,302.236468208334)
--(axis cs:2775000,278.253405)
--(axis cs:2850000,332.351931666667)
--(axis cs:2925000,359.705546666667)
--(axis cs:3000000,358.691566666667)
--(axis cs:3000000,504.15511)
--(axis cs:3000000,504.15511)
--(axis cs:2925000,524.568105)
--(axis cs:2850000,497.756562625)
--(axis cs:2775000,481.869911583334)
--(axis cs:2700000,501.89183)
--(axis cs:2625000,476.158643333333)
--(axis cs:2550000,474.759123333333)
--(axis cs:2475000,458.712175)
--(axis cs:2400000,459.580095)
--(axis cs:2325000,461.802686666667)
--(axis cs:2250000,436.69185)
--(axis cs:2175000,439.264936666667)
--(axis cs:2100000,421.15417)
--(axis cs:2025000,405.239983333333)
--(axis cs:1950000,411.747271666667)
--(axis cs:1875000,390.732111666667)
--(axis cs:1800000,384.01697)
--(axis cs:1725000,368.661723333333)
--(axis cs:1650000,349.257366666667)
--(axis cs:1575000,324.216918333333)
--(axis cs:1500000,319.73715)
--(axis cs:1425000,279.37976)
--(axis cs:1350000,304.311968333333)
--(axis cs:1275000,291.908955)
--(axis cs:1200000,270.286778333333)
--(axis cs:1125000,266.046966666667)
--(axis cs:1050000,251.565023333333)
--(axis cs:975000,242.544341666667)
--(axis cs:900000,219.097448333333)
--(axis cs:825000,214.326845)
--(axis cs:750000,198.503528625)
--(axis cs:675000,188.036166666667)
--(axis cs:600000,136.040361666667)
--(axis cs:525000,142.3553445)
--(axis cs:450000,141.369716666667)
--(axis cs:375000,95.4977814483336)
--(axis cs:300000,45.8442208333333)
--(axis cs:225000,19.1076528333333)
--(axis cs:150000,6.91921088333333)
--(axis cs:75000,8.82462750750001)
--(axis cs:1,6.77576936666667)
--cycle;

\addplot [line width=\linewidthdime, C7, mark=*, mark size=0, mark options={solid}]
table {%
1 6.21731321666667
75000 5.253077
150000 4.1307842
225000 11.116095
300000 30.8361456666667
375000 59.315322
450000 94.1836196666667
525000 90.2431186666667
600000 93.3293095
675000 134.245496666667
750000 144.621306666667
825000 153.445905
900000 147.3714875
975000 188.856203333333
1050000 189.672935833333
1125000 204.277875
1200000 207.117788333333
1275000 219.77152
1350000 233.742483333333
1425000 204.026615
1500000 250.330331666667
1575000 227.57098
1650000 259.547911666667
1725000 258.281573333333
1800000 296.176831666667
1875000 292.834295
1950000 320.9015
2025000 322.061226666667
2100000 334.527795
2175000 350.691
2250000 363.331376666667
2325000 381.825788333333
2400000 362.641175
2475000 399.489923333333
2550000 349.276133333333
2625000 413.17155
2700000 411.355768333333
2775000 370.087901666667
2850000 420.774238333333
2925000 449.816926666667
3000000 434.692828333333
};
\path [draw=C6, fill=C6, opacity=0.2]
(axis cs:1,6.9812828)
--(axis cs:1,5.55815306666667)
--(axis cs:36000,4.64270726666667)
--(axis cs:72000,4.40833533333333)
--(axis cs:108000,4.7460012)
--(axis cs:144000,5.269655)
--(axis cs:180000,5.016788)
--(axis cs:216000,5.02033486666667)
--(axis cs:252000,4.58681713333333)
--(axis cs:288000,4.591249)
--(axis cs:324000,4.70324226666667)
--(axis cs:360000,4.0491373)
--(axis cs:396000,4.07145986666667)
--(axis cs:432000,4.5005639)
--(axis cs:468000,4.40048913333333)
--(axis cs:504000,4.40163946666667)
--(axis cs:540000,5.0360962)
--(axis cs:576000,4.5731699)
--(axis cs:612000,5.30229033333333)
--(axis cs:648000,4.20323053333333)
--(axis cs:684000,5.18204053333333)
--(axis cs:720000,5.21875266666667)
--(axis cs:756000,4.718841)
--(axis cs:792000,4.4442496)
--(axis cs:828000,4.66176246666667)
--(axis cs:864000,4.58836993333333)
--(axis cs:900000,4.45616026666667)
--(axis cs:936000,4.60671153333333)
--(axis cs:972000,4.19781506666667)
--(axis cs:1008000,5.34430326666667)
--(axis cs:1044000,4.39077166666667)
--(axis cs:1080000,4.33613363333333)
--(axis cs:1116000,5.27339733333333)
--(axis cs:1152000,4.66944033333333)
--(axis cs:1188000,4.6240356)
--(axis cs:1224000,4.5440958)
--(axis cs:1260000,4.3664907)
--(axis cs:1296000,4.5793652)
--(axis cs:1332000,4.66924266666667)
--(axis cs:1368000,3.965145)
--(axis cs:1404000,4.66238746666667)
--(axis cs:1440000,4.77891566666667)
--(axis cs:1476000,4.7915225)
--(axis cs:1512000,4.24932553333333)
--(axis cs:1548000,4.3862614)
--(axis cs:1584000,5.12497203333333)
--(axis cs:1620000,4.50460833333333)
--(axis cs:1656000,4.5744674)
--(axis cs:1692000,4.8700602)
--(axis cs:1728000,4.71128333333333)
--(axis cs:1764000,4.55467146666667)
--(axis cs:1800000,5.49946813333333)
--(axis cs:1836000,4.29188066666667)
--(axis cs:1872000,4.2263158)
--(axis cs:1908000,4.81428313333333)
--(axis cs:1944000,4.34031866666667)
--(axis cs:1980000,4.5207506)
--(axis cs:2016000,5.01495883333333)
--(axis cs:2052000,4.9837772)
--(axis cs:2088000,4.70743353333333)
--(axis cs:2124000,4.79699273333333)
--(axis cs:2160000,4.69314326666667)
--(axis cs:2196000,3.8972669)
--(axis cs:2232000,4.91765086666667)
--(axis cs:2268000,4.29147166666667)
--(axis cs:2304000,5.0530026)
--(axis cs:2340000,4.58989413333333)
--(axis cs:2376000,4.712138)
--(axis cs:2412000,4.68172313333333)
--(axis cs:2448000,4.105184)
--(axis cs:2484000,4.40403613333333)
--(axis cs:2520000,4.86467473333333)
--(axis cs:2556000,4.95357226666667)
--(axis cs:2592000,5.20899723333333)
--(axis cs:2628000,4.582157)
--(axis cs:2664000,4.68247876666667)
--(axis cs:2700000,4.5075402)
--(axis cs:2736000,4.54131653333333)
--(axis cs:2772000,4.800876)
--(axis cs:2808000,4.84288156666667)
--(axis cs:2844000,4.59038823333333)
--(axis cs:2880000,4.71261006666667)
--(axis cs:2916000,4.706956)
--(axis cs:2952000,4.72729556666667)
--(axis cs:2988000,4.68146486666667)
--(axis cs:2988000,7.3866156)
--(axis cs:2988000,7.3866156)
--(axis cs:2952000,5.58287833333333)
--(axis cs:2916000,6.16278206666667)
--(axis cs:2880000,5.8543394)
--(axis cs:2844000,7.0992872)
--(axis cs:2808000,5.852884)
--(axis cs:2772000,6.5409335)
--(axis cs:2736000,5.91396676666667)
--(axis cs:2700000,5.95869233333333)
--(axis cs:2664000,6.5659872)
--(axis cs:2628000,6.01688243333333)
--(axis cs:2592000,6.61557266666667)
--(axis cs:2556000,6.4482362)
--(axis cs:2520000,6.0962796)
--(axis cs:2484000,6.1821412)
--(axis cs:2448000,6.5614479)
--(axis cs:2412000,6.25627773333333)
--(axis cs:2376000,5.72259883333333)
--(axis cs:2340000,6.25883253333333)
--(axis cs:2304000,6.86157853333333)
--(axis cs:2268000,6.00031933333333)
--(axis cs:2232000,6.2311058)
--(axis cs:2196000,5.69142976666667)
--(axis cs:2160000,6.14034273333333)
--(axis cs:2124000,6.53163243333333)
--(axis cs:2088000,6.19169873333333)
--(axis cs:2052000,7.39258266666667)
--(axis cs:2016000,5.8830814)
--(axis cs:1980000,5.56394333333333)
--(axis cs:1944000,5.70484376666667)
--(axis cs:1908000,6.6507914)
--(axis cs:1872000,6.02142733333333)
--(axis cs:1836000,5.55149913333333)
--(axis cs:1800000,6.51535853333333)
--(axis cs:1764000,5.17395066666667)
--(axis cs:1728000,6.03141086666667)
--(axis cs:1692000,6.1870729)
--(axis cs:1656000,7.14635233333333)
--(axis cs:1620000,5.63219533333333)
--(axis cs:1584000,5.76588203333333)
--(axis cs:1548000,5.38945516666667)
--(axis cs:1512000,5.65740713333333)
--(axis cs:1476000,6.3625454)
--(axis cs:1440000,5.39659866666667)
--(axis cs:1404000,5.82999093333333)
--(axis cs:1368000,5.0504985)
--(axis cs:1332000,5.43350313333333)
--(axis cs:1296000,5.33941133333333)
--(axis cs:1260000,5.557514)
--(axis cs:1224000,6.27759833333333)
--(axis cs:1188000,5.98050653333333)
--(axis cs:1152000,6.45088623333333)
--(axis cs:1116000,6.2619232)
--(axis cs:1080000,7.7436922)
--(axis cs:1044000,5.69646133333333)
--(axis cs:1008000,6.25468076666667)
--(axis cs:972000,7.14397706666667)
--(axis cs:936000,5.81546706666667)
--(axis cs:900000,7.19774586666667)
--(axis cs:864000,6.63735433333333)
--(axis cs:828000,7.05162966666667)
--(axis cs:792000,6.32014086666667)
--(axis cs:756000,6.0808052)
--(axis cs:720000,6.98344723333333)
--(axis cs:684000,7.24339393333333)
--(axis cs:648000,5.440585)
--(axis cs:612000,6.37086776666667)
--(axis cs:576000,5.88254583333333)
--(axis cs:540000,5.89970253333333)
--(axis cs:504000,5.739004)
--(axis cs:468000,6.42585966666667)
--(axis cs:432000,5.50213233333333)
--(axis cs:396000,5.60572626666667)
--(axis cs:360000,5.37717243333333)
--(axis cs:324000,5.87759166666667)
--(axis cs:288000,5.90751273333333)
--(axis cs:252000,6.02331956666667)
--(axis cs:216000,5.87698323333333)
--(axis cs:180000,6.67633166666667)
--(axis cs:144000,7.19540406666667)
--(axis cs:108000,5.47043033333333)
--(axis cs:72000,6.8538961)
--(axis cs:36000,4.96500693333333)
--(axis cs:1,6.9812828)
--cycle;

\addplot [line width=\linewidthdime, C6, mark=*, mark size=0, mark options={solid}]
table {%
1 6.43196386666667
36000 4.71051633333333
72000 5.13775976666667
108000 5.24261886666667
144000 5.51829966666667
180000 5.64658856666667
216000 5.38689296666667
252000 5.19010796666667
288000 5.39336523333333
324000 5.19317766666667
360000 4.70536583333333
396000 5.10026153333333
432000 4.97143306666667
468000 5.18650023333333
504000 4.94802313333333
540000 5.49724323333333
576000 5.1908188
612000 5.4937367
648000 4.614996
684000 5.90373756666667
720000 5.84010806666667
756000 5.74715513333333
792000 5.22414233333333
828000 5.5550948
864000 5.73854866666667
900000 5.35176193333333
936000 5.10593973333333
972000 5.73306913333333
1008000 5.55027496666667
1044000 5.13769816666667
1080000 5.70956443333333
1116000 5.51703786666667
1152000 5.28572236666667
1188000 5.45507986666667
1224000 5.24239053333333
1260000 4.8643601
1296000 4.943376
1332000 5.04164723333333
1368000 4.53417983333333
1404000 5.25825346666667
1440000 5.0463887
1476000 5.87934166666667
1512000 5.3945577
1548000 4.94977403333333
1584000 5.22829613333333
1620000 4.7993422
1656000 5.82566633333333
1692000 5.6415978
1728000 5.44530896666667
1764000 4.9331166
1800000 5.98199113333333
1836000 5.0593467
1872000 4.9263694
1908000 6.13343693333333
1944000 5.14781176666667
1980000 5.00160466666667
2016000 5.25194606666667
2052000 5.63953603333333
2088000 5.57266936666667
2124000 5.89529176666667
2160000 5.45555266666667
2196000 4.80036943333333
2232000 5.54363353333333
2268000 4.82849466666667
2304000 5.5669531
2340000 5.55033263333333
2376000 5.41993486666667
2412000 5.5280028
2448000 5.0603249
2484000 5.24133196666667
2520000 5.66627236666667
2556000 5.49692786666667
2592000 6.15267393333333
2628000 5.1039841
2664000 5.20016276666667
2700000 5.4684839
2736000 5.44102656666667
2772000 5.9158646
2808000 5.48867306666667
2844000 5.61468313333333
2880000 5.23250153333333
2916000 5.6323046
2952000 5.31495256666667
2988000 5.95452386666667
};
\end{axis}

\end{tikzpicture}
}
       \subcaption[]{}
       \label{fig::exps_new_ablations_vary_alpha::dog_run}
    \end{minipage}\hfill
    \begin{minipage}[b]{0.25\textwidth}
        \centering
       \resizebox{0.875\textwidth}{!}{% This file was created with tikzplotlib v0.10.1.
\begin{tikzpicture}

\definecolor{chocolate213940}{RGB}{213,94,0}
\definecolor{darkcyan1115178}{RGB}{1,115,178}
\definecolor{darkcyan2158115}{RGB}{2,158,115}
\definecolor{darkgray176}{RGB}{176,176,176}
\definecolor{darkorange2221435}{RGB}{222,143,5}
\definecolor{orchid204120188}{RGB}{204,120,188}
\definecolor{peru20214597}{RGB}{202,145,97}
\definecolor{pink251175228}{RGB}{251,175,228}

\definecolor{crimson2143940}{RGB}{214,39,40}
\definecolor{darkorange25512714}{RGB}{255,127,14}
\definecolor{forestgreen4416044}{RGB}{44,160,44}
\definecolor{mediumpurple148103189}{RGB}{148,103,189}
\definecolor{steelblue31119180}{RGB}{31,119,180}
\definecolor{darkgray176}{RGB}{176,176,176}

\begin{axis}[
legend cell align={left},
legend style={fill opacity=0.8, draw opacity=1, text opacity=1, draw=lightgray204, at={(0.03,0.03)},  anchor=north west},
tick align=outside,
tick pos=left,
x grid style={white},
xlabel={IQM Return},
xmajorgrids,
xmin=363.309961666667, xmax=845.547968791667,
xtick style={color=black},
y grid style={white},
ymin=-0.63, ymax=6.63,
ytick style={color=black},
% ytick={0,1,2,3,4,5,6},
% yticklabels={$10^{-02}$,$10^{-03}$,$10^{-04}$,$10^{-05}$,$10^{-08}$,$10^{-09}$,$10^{-12}$},
%ytick=\empty,
yticklabels={},
ylabel={$\alpha$ Values},
axis background/.style={fill=plot_background},
label style={font=\large},
tick label style={font=\large},
x axis line style={draw=none},
y axis line style={draw=none},
]
\draw[draw=none,fill=C0,fill opacity=0.75] (axis cs:702.50849,-0.3) rectangle (axis cs:750.875461666667,0.3);
%\addlegendimage{ybar,ybar legend,draw=none,fill=crimson2143940,fill opacity=0.75}
%\addlegendentry{1e-2}

\draw[draw=none,fill=C1,fill opacity=0.75] (axis cs:756.833583333333,0.7) rectangle (axis cs:822.584254166667,1.3);
%\addlegendimage{ybar,ybar legend,draw=none,fill=darkorange2221435,fill opacity=0.75}
%\addlegendentry{1e-3}

\draw[draw=none,fill=C3,fill opacity=0.75] (axis cs:653.580964166667,1.7) rectangle (axis cs:798.360316520834,2.3);
%\addlegendimage{ybar,ybar legend,draw=none,fill=forestgreen4416044,fill opacity=0.75}
%\addlegendentry{1e-4}

\draw[draw=none,fill=C4,fill opacity=0.75] (axis cs:562.97792,2.7) rectangle (axis cs:699.922009166667,3.3);
%\addlegendimage{ybar,ybar legend,draw=none,fill=violet,fill opacity=0.75}
%\addlegendentry{1e-5}

\draw[draw=none,fill=C5,fill opacity=0.75] (axis cs:460.2824125,3.7) rectangle (axis cs:699.551603333333,4.3);
%\addlegendimage{ybar,ybar legend,draw=none,fill=mediumpurple148103189,fill opacity=0.75}
%\addlegendentry{1e-8}

\draw[draw=none,fill=C8,fill opacity=0.75] (axis cs:389.501995,4.7) rectangle (axis cs:523.519544166667,5.3);
%\addlegendimage{ybar,ybar legend,draw=none,fill=teal,fill opacity=0.75}
%\addlegendentry{1e-9}

\draw[draw=none,fill=C7,fill opacity=0.75] (axis cs:363.309961666667,5.7) rectangle (axis cs:508.2595675,6.3);
%\addlegendimage{ybar,ybar legend,draw=none,fill=steelblue31119180,fill opacity=0.75}
%\addlegendentry{1e-12}

\path [draw=black, draw opacity=0.5, very thick]
(axis cs:728.768623333333,-0.28125)
--(axis cs:728.768623333333,0.225);

\path [draw=black, draw opacity=0.5, very thick]
(axis cs:792.855715833333,0.71875)
--(axis cs:792.855715833333,1.225);

\path [draw=black, draw opacity=0.5, very thick]
(axis cs:744.586875,1.71875)
--(axis cs:744.586875,2.225);

\path [draw=black, draw opacity=0.5, very thick]
(axis cs:621.846805,2.71875)
--(axis cs:621.846805,3.225);

\path [draw=black, draw opacity=0.5, very thick]
(axis cs:568.7912225,3.71875)
--(axis cs:568.7912225,4.225);

\path [draw=black, draw opacity=0.5, very thick]
(axis cs:468.99135,4.71875)
--(axis cs:468.99135,5.225);

\path [draw=black, draw opacity=0.5, very thick]
(axis cs:448.295465833333,5.71875)
--(axis cs:448.295465833333,6.225);

\end{axis}

% \draw ({$(current bounding box.south west)!0.4!(current bounding box.south east)$}|-{$(current bounding box.south west)!-0.1!(current bounding box.north west)$}) node[
%   scale=0.864,
%   anchor=base,
%   text=black,
%   rotate=0.0
% ]{};
\end{tikzpicture}
}
       \subcaption[]{}
       \label{fig::exps_new_ablations_vary_alpha::dog_run_pareto}
    \end{minipage}\hfill
    \begin{minipage}[b]{0.24\textwidth}
        \centering
       \resizebox{1\textwidth}{!}{% This file was created with tikzplotlib v0.10.1.
\begin{tikzpicture}

\definecolor{darkcyan1115178}{RGB}{1,115,178}
\definecolor{darkgray176}{RGB}{176,176,176}

\begin{axis}[
legend cell align={left},
legend cell align={left},
legend style={fill opacity=0.8, draw opacity=1, text opacity=1, draw=lightgray204, at={(0.5,0.03)},  anchor=south west},
tick align=outside,
tick pos=left,
x grid style={white},
xlabel={Number Env Interactions},
xmajorgrids,
xmin=-0.0, xmax=1000000.0,
xtick style={color=black},
y grid style={white},
ylabel={IQM Mean Return},
ymajorgrids,
ymin=-12.11232765065, ymax=350.65256879765,
ytick style={color=black},
axis background/.style={fill=plot_background},
label style={font=\large},
tick label style={font=\large},
x axis line style={draw=none},
y axis line style={draw=none},
]

\path [draw=C1, fill=C1, opacity=0.2]
(axis cs:1,0.913481333333333)
--(axis cs:1,0.6287236)
--(axis cs:24000,0.677781566666667)
--(axis cs:48000,0.738428966666667)
--(axis cs:72000,0.790174833333333)
--(axis cs:96000,1.1543524)
--(axis cs:120000,3.23502683000002)
--(axis cs:144000,20.5413528)
--(axis cs:168000,8.54387893333333)
--(axis cs:192000,34.4303118333333)
--(axis cs:216000,72.0098966666667)
--(axis cs:240000,72.3488675833333)
--(axis cs:264000,96.707165)
--(axis cs:288000,96.000784)
--(axis cs:312000,104.958421625)
--(axis cs:336000,114.297126)
--(axis cs:360000,106.284968)
--(axis cs:384000,123.7906925)
--(axis cs:408000,123.723489166667)
--(axis cs:432000,128.761211666667)
--(axis cs:456000,111.02938625)
--(axis cs:480000,135.956141666667)
--(axis cs:504000,138.98209)
--(axis cs:528000,118.929581333333)
--(axis cs:552000,132.387325)
--(axis cs:576000,148.148588333333)
--(axis cs:600000,153.61958)
--(axis cs:624000,156.55045)
--(axis cs:648000,156.870825)
--(axis cs:672000,161.074148333333)
--(axis cs:696000,164.107246666667)
--(axis cs:720000,165.654856541667)
--(axis cs:744000,167.669441666667)
--(axis cs:768000,160.786089)
--(axis cs:792000,168.99019)
--(axis cs:816000,177.369616666667)
--(axis cs:840000,175.239106666667)
--(axis cs:864000,176.361243333333)
--(axis cs:888000,165.394583333333)
--(axis cs:912000,183.752826666667)
--(axis cs:936000,187.066958333333)
--(axis cs:960000,160.625822716667)
--(axis cs:984000,191.555848333333)
--(axis cs:984000,251.074266666667)
--(axis cs:984000,251.074266666667)
--(axis cs:960000,253.124288333333)
--(axis cs:936000,245.776626666667)
--(axis cs:912000,236.034463333333)
--(axis cs:888000,226.598225)
--(axis cs:864000,227.013098708333)
--(axis cs:840000,234.283123333333)
--(axis cs:816000,229.472495)
--(axis cs:792000,224.034361666667)
--(axis cs:768000,206.047165)
--(axis cs:744000,209.549365)
--(axis cs:720000,205.859246666667)
--(axis cs:696000,194.984201666667)
--(axis cs:672000,188.552526666667)
--(axis cs:648000,185.157259583333)
--(axis cs:624000,177.684235)
--(axis cs:600000,172.894043333333)
--(axis cs:576000,171.0584)
--(axis cs:552000,163.391113333333)
--(axis cs:528000,162.135815)
--(axis cs:504000,161.517217666667)
--(axis cs:480000,153.846013166667)
--(axis cs:456000,150.815503333333)
--(axis cs:432000,154.043445)
--(axis cs:408000,146.23077)
--(axis cs:384000,144.130305)
--(axis cs:360000,136.713208333333)
--(axis cs:336000,131.509801666667)
--(axis cs:312000,123.557279541667)
--(axis cs:288000,121.774874166667)
--(axis cs:264000,117.476811666667)
--(axis cs:240000,110.350656666667)
--(axis cs:216000,103.274480333333)
--(axis cs:192000,90.3314516666667)
--(axis cs:168000,85.1717466666667)
--(axis cs:144000,83.9523868333333)
--(axis cs:120000,60.989724)
--(axis cs:96000,26.2789549666667)
--(axis cs:72000,5.35472753333333)
--(axis cs:48000,1.06001243333333)
--(axis cs:24000,0.923337466666667)
--(axis cs:1,0.913481333333333)
--cycle;

\addplot [line width=\linewidthdime, C1, mark=*, mark size=0, mark options={solid}]
table {%
1 0.7694417
24000 0.813880766666667
48000 0.903083466666667
72000 0.985341633333333
96000 7.39312896666667
120000 31.2267919666667
144000 57.5656945
168000 45.1603116666667
192000 70.5692
216000 94.6452396666667
240000 97.5139125
264000 109.186628333333
288000 109.1955065
312000 111.598211666667
336000 123.198625
360000 125.946167
384000 133.888621666667
408000 134.214293333333
432000 139.483735
456000 142.763216666667
480000 144.338755
504000 149.75221
528000 151.421726666667
552000 147.365006666667
576000 159.676698333333
600000 161.191491666667
624000 163.406718333333
648000 169.18849
672000 173.390193333333
696000 174.364796666667
720000 181.648728333333
744000 181.343818333333
768000 179.253028333333
792000 189.492575
816000 195.565436666667
840000 198.284515
864000 193.75859
888000 187.434418333333
912000 201.805691666667
936000 209.501925
960000 208.739933333333
984000 214.710998333333
};

%DIME
\path [draw=C0, fill=C0, opacity=0.2]
(axis cs:1,0.915501966666667)
--(axis cs:1,0.633556866666667)
--(axis cs:36000,0.5493304)
--(axis cs:72000,0.843873266666667)
--(axis cs:108000,4.13696455)
--(axis cs:144000,36.1211871666667)
--(axis cs:180000,51.4064937333333)
--(axis cs:216000,81.368514)
--(axis cs:252000,97.5408251666667)
--(axis cs:288000,109.583571666667)
--(axis cs:324000,110.618288333333)
--(axis cs:360000,120.684184666667)
--(axis cs:396000,122.417020833333)
--(axis cs:432000,131.747461666667)
--(axis cs:468000,139.23245)
--(axis cs:504000,145.708203333333)
--(axis cs:540000,128.071794166667)
--(axis cs:576000,146.743565)
--(axis cs:612000,157.523090333333)
--(axis cs:648000,162.386513333333)
--(axis cs:684000,167.712168333333)
--(axis cs:720000,177.17324)
--(axis cs:756000,175.778066666667)
--(axis cs:792000,177.557476666667)
--(axis cs:828000,183.413371666667)
--(axis cs:864000,191.343878333333)
--(axis cs:900000,198.058078333333)
--(axis cs:936000,199.495351666667)
--(axis cs:972000,204.18452)
--(axis cs:1008000,211.017165)
--(axis cs:1044000,210.395921666667)
--(axis cs:1080000,225.37889)
--(axis cs:1116000,225.57669)
--(axis cs:1152000,229.055506666667)
--(axis cs:1188000,244.717408333333)
--(axis cs:1224000,235.69925925)
--(axis cs:1260000,246.761671666667)
--(axis cs:1296000,263.70817)
--(axis cs:1332000,261.655338333333)
--(axis cs:1368000,274.904988333333)
--(axis cs:1404000,281.871765)
--(axis cs:1440000,284.6845)
--(axis cs:1476000,298.373705)
--(axis cs:1512000,314.234433333333)
--(axis cs:1548000,302.586606666667)
--(axis cs:1584000,306.158113166667)
--(axis cs:1620000,329.89694)
--(axis cs:1656000,327.346265)
--(axis cs:1692000,331.977616666667)
--(axis cs:1728000,335.598585)
--(axis cs:1764000,337.5854)
--(axis cs:1800000,350.510931458333)
--(axis cs:1836000,348.138216666667)
--(axis cs:1872000,325.377726666667)
--(axis cs:1908000,375.064071666667)
--(axis cs:1944000,364.504996666667)
--(axis cs:1980000,362.576775)
--(axis cs:2016000,359.905746666667)
--(axis cs:2052000,401.565295)
--(axis cs:2088000,380.944233333333)
--(axis cs:2124000,408.07439)
--(axis cs:2160000,406.816443333333)
--(axis cs:2196000,402.540623333333)
--(axis cs:2232000,419.828846666667)
--(axis cs:2268000,419.407693333333)
--(axis cs:2304000,391.893025)
--(axis cs:2340000,426.646173333333)
--(axis cs:2376000,430.59703)
--(axis cs:2412000,440.944021375)
--(axis cs:2448000,444.535958333333)
--(axis cs:2484000,447.59097)
--(axis cs:2520000,458.220718125)
--(axis cs:2556000,454.987311666667)
--(axis cs:2592000,444.593778333333)
--(axis cs:2628000,452.355353333333)
--(axis cs:2664000,472.315071583333)
--(axis cs:2700000,468.552297125)
--(axis cs:2736000,425.893185)
--(axis cs:2772000,461.17882)
--(axis cs:2808000,438.974565)
--(axis cs:2844000,408.566183333333)
--(axis cs:2880000,440.493023333333)
--(axis cs:2916000,474.594651666667)
--(axis cs:2952000,473.13282)
--(axis cs:2988000,468.31128725)
--(axis cs:2988000,572.54452975)
--(axis cs:2988000,572.54452975)
--(axis cs:2952000,607.047773333333)
--(axis cs:2916000,581.955683333333)
--(axis cs:2880000,562.69891)
--(axis cs:2844000,570.399441666667)
--(axis cs:2808000,561.294171666667)
--(axis cs:2772000,575.424046666667)
--(axis cs:2736000,577.564579375)
--(axis cs:2700000,570.731558333333)
--(axis cs:2664000,556.645568333333)
--(axis cs:2628000,528.610636666667)
--(axis cs:2592000,572.906766666667)
--(axis cs:2556000,556.792031666667)
--(axis cs:2520000,565.548321666667)
--(axis cs:2484000,564.076643333333)
--(axis cs:2448000,554.428711666667)
--(axis cs:2412000,511.028458333333)
--(axis cs:2376000,539.327213333333)
--(axis cs:2340000,521.274179458333)
--(axis cs:2304000,507.011033333333)
--(axis cs:2268000,533.545403333333)
--(axis cs:2232000,530.436551666667)
--(axis cs:2196000,514.156165)
--(axis cs:2160000,548.6599)
--(axis cs:2124000,526.241701666667)
--(axis cs:2088000,488.449933333333)
--(axis cs:2052000,521.397975)
--(axis cs:2016000,489.724498333333)
--(axis cs:1980000,511.56874)
--(axis cs:1944000,469.79517)
--(axis cs:1908000,487.4035)
--(axis cs:1872000,452.261116666667)
--(axis cs:1836000,448.443288333333)
--(axis cs:1800000,470.013476541667)
--(axis cs:1764000,441.067136666667)
--(axis cs:1728000,425.983283333333)
--(axis cs:1692000,459.900376666667)
--(axis cs:1656000,420.555463333333)
--(axis cs:1620000,452.412428333333)
--(axis cs:1584000,426.080911666667)
--(axis cs:1548000,405.744391916667)
--(axis cs:1512000,422.468795)
--(axis cs:1476000,392.121763333333)
--(axis cs:1440000,398.927006666667)
--(axis cs:1404000,396.300881666667)
--(axis cs:1368000,391.708926666667)
--(axis cs:1332000,379.3367)
--(axis cs:1296000,374.028266666667)
--(axis cs:1260000,354.826833333333)
--(axis cs:1224000,325.77437)
--(axis cs:1188000,356.78896)
--(axis cs:1152000,352.524811666667)
--(axis cs:1116000,343.294016666667)
--(axis cs:1080000,331.116815)
--(axis cs:1044000,287.783165)
--(axis cs:1008000,319.938826666667)
--(axis cs:972000,297.00894)
--(axis cs:936000,297.43841)
--(axis cs:900000,295.160138333333)
--(axis cs:864000,278.186283333333)
--(axis cs:828000,267.117865)
--(axis cs:792000,254.04739775)
--(axis cs:756000,245.040613333333)
--(axis cs:720000,223.696456875)
--(axis cs:684000,216.1656)
--(axis cs:648000,205.401713333333)
--(axis cs:612000,201.404601666667)
--(axis cs:576000,182.55503)
--(axis cs:540000,177.786793333333)
--(axis cs:504000,170.561265208333)
--(axis cs:468000,170.450261666667)
--(axis cs:432000,164.02998)
--(axis cs:396000,153.633991666667)
--(axis cs:360000,143.37209)
--(axis cs:324000,136.525775833333)
--(axis cs:288000,129.716861683333)
--(axis cs:252000,113.1404125)
--(axis cs:216000,102.49002)
--(axis cs:180000,92.9197603333333)
--(axis cs:144000,76.8890066666667)
--(axis cs:108000,56.7536433333333)
--(axis cs:72000,4.25857906666667)
--(axis cs:36000,0.728938933333333)
--(axis cs:1,0.915501966666667)
--cycle;

\addplot [line width=\linewidthdime, C0, mark=*, mark size=0, mark options={solid}]
table {%
1 0.770562166666667
36000 0.658370533333333
72000 1.02744053333333
108000 27.4132359166667
144000 61.7389005
180000 78.3388245
216000 96.6475441666667
252000 104.292225666667
288000 118.336289333333
324000 122.156900833333
360000 127.865645666667
396000 135.589655833333
432000 144.61231
468000 150.876916666667
504000 155.106371666667
540000 161.848961666667
576000 162.873085
612000 175.66877
648000 181.774033333333
684000 189.795443333333
720000 201.694543333333
756000 209.679653333333
792000 215.950858333333
828000 226.452128333333
864000 235.790758333333
900000 245.745095
936000 251.419853333333
972000 253.923228333333
1008000 267.745066666667
1044000 253.029908333333
1080000 281.1757
1116000 290.478148333333
1152000 295.386573333333
1188000 307.510898333333
1224000 279.324226666667
1260000 298.499671666667
1296000 320.530285
1332000 314.688511666667
1368000 333.564955
1404000 344.035705
1440000 351.026203333333
1476000 334.48205
1512000 371.517083333333
1548000 359.107073333333
1584000 362.49581
1620000 383.735268333333
1656000 371.120853333333
1692000 392.875408333333
1728000 375.706403333333
1764000 385.148411666667
1800000 411.377563333333
1836000 393.022041666667
1872000 381.083618333333
1908000 428.514751666667
1944000 407.976545
1980000 440.522148333333
2016000 425.096363333333
2052000 460.246375
2088000 439.221346666667
2124000 470.482476666667
2160000 479.806223333333
2196000 467.028006666667
2232000 471.833656666667
2268000 492.305195
2304000 444.283438333333
2340000 467.734686666667
2376000 478.305563333333
2412000 479.659883333333
2448000 502.891963333333
2484000 512.254108333333
2520000 504.927955
2556000 509.89631
2592000 517.465808333333
2628000 505.843831666667
2664000 503.219516666667
2700000 525.334166666667
2736000 498.718568333333
2772000 515.273173333333
2808000 507.004858333333
2844000 490.845221666667
2880000 488.33129
2916000 530.288801666667
2952000 544.121438333333
2988000 506.351106666667
};

\end{axis}

\end{tikzpicture}
}
       \subcaption[]{}
       \label{fig::exps_new_ablations_gauss_vs_diff_distrq::humanoid_run}
    \end{minipage}\hfill
    \begin{minipage}[b]{0.24\textwidth}
        \centering
       \resizebox{1\textwidth}{!}{% This file was created with tikzplotlib v0.10.1.
\definecolor{crimson2143940}{RGB}{214,39,40}
\definecolor{darkorange25512714}{RGB}{255,127,14}
\definecolor{forestgreen4416044}{RGB}{44,160,44}
\definecolor{mediumpurple148103189}{RGB}{148,103,189}
\definecolor{steelblue31119180}{RGB}{31,119,180}
\definecolor{darkgray176}{RGB}{176,176,176}
\begin{tikzpicture}

\definecolor{darkcyan1115178}{RGB}{1,115,178}
\definecolor{darkgray176}{RGB}{176,176,176}

\begin{axis}[
legend cell align={left},
legend style={fill opacity=0.8, draw opacity=1, text opacity=1, draw=lightgray204, at={(0.03,0.03)},  anchor=north west},
tick align=outside,
tick pos=left,
x grid style={white},
xlabel={Number Env Interactions},
xmajorgrids,
%xmin=-149398.95, xmax=3137399.95,
xmin=0.0, xmax=3000000.00,
xtick style={color=black},
y grid style={white},
ylabel={IQM Return},
ymajorgrids,
ymin=-35.861410165, ymax=877.233399531667,
ytick style={color=black},
axis background/.style={fill=plot_background},
label style={font=\large},
tick label style={font=\large},
x axis line style={draw=none},
y axis line style={draw=none},
]
\path [draw=C0, fill=C0, opacity=0.2]
(axis cs:1,6.77901475791667)
--(axis cs:1,5.64291553333333)
--(axis cs:75000,19.1597246666667)
--(axis cs:150000,62.7155849916667)
--(axis cs:225000,96.3693566666667)
--(axis cs:300000,126.079227583333)
--(axis cs:375000,149.13536)
--(axis cs:450000,175.421665)
--(axis cs:525000,186.97005)
--(axis cs:600000,212.94413)
--(axis cs:675000,217.86526)
--(axis cs:750000,260.150573333333)
--(axis cs:825000,269.859375)
--(axis cs:900000,260.430376541667)
--(axis cs:975000,340.899176666667)
--(axis cs:1050000,358.012548333333)
--(axis cs:1125000,385.657351708333)
--(axis cs:1200000,379.184725208333)
--(axis cs:1275000,433.776411666667)
--(axis cs:1350000,446.441923333333)
--(axis cs:1425000,479.320458333333)
--(axis cs:1500000,498.277125)
--(axis cs:1575000,516.634003333333)
--(axis cs:1650000,513.550598333333)
--(axis cs:1725000,519.899135)
--(axis cs:1800000,540.786894291667)
--(axis cs:1875000,526.786516666667)
--(axis cs:1950000,600.207093333333)
--(axis cs:2025000,601.858965)
--(axis cs:2100000,621.464493333333)
--(axis cs:2175000,634.164003333333)
--(axis cs:2250000,632.273245)
--(axis cs:2325000,637.152511666667)
--(axis cs:2400000,647.836116666667)
--(axis cs:2475000,649.0113)
--(axis cs:2550000,678.132627083333)
--(axis cs:2625000,674.6811)
--(axis cs:2700000,665.833213333333)
--(axis cs:2775000,697.350276666667)
--(axis cs:2850000,696.739206666667)
--(axis cs:2925000,703.537966666667)
--(axis cs:3000000,687.357123333333)
--(axis cs:3000000,743.104029916667)
--(axis cs:3000000,743.104029916667)
--(axis cs:2925000,754.299846666667)
--(axis cs:2850000,731.263983333333)
--(axis cs:2775000,742.4798)
--(axis cs:2700000,725.915376666667)
--(axis cs:2625000,728.106643333333)
--(axis cs:2550000,725.096276666667)
--(axis cs:2475000,713.061273333333)
--(axis cs:2400000,697.68279)
--(axis cs:2325000,700.821681166667)
--(axis cs:2250000,687.937288333333)
--(axis cs:2175000,687.74969125)
--(axis cs:2100000,680.6291)
--(axis cs:2025000,656.15625)
--(axis cs:1950000,663.508993333333)
--(axis cs:1875000,635.86965)
--(axis cs:1800000,642.717033333333)
--(axis cs:1725000,617.434283333333)
--(axis cs:1650000,602.215845)
--(axis cs:1575000,590.692205)
--(axis cs:1500000,578.370525)
--(axis cs:1425000,564.136311666667)
--(axis cs:1350000,542.01851)
--(axis cs:1275000,536.479591666667)
--(axis cs:1200000,504.493398333333)
--(axis cs:1125000,492.82073)
--(axis cs:1050000,473.255681666667)
--(axis cs:975000,446.901898333333)
--(axis cs:900000,407.875456666667)
--(axis cs:825000,388.210466666667)
--(axis cs:750000,356.76231)
--(axis cs:675000,315.687453333333)
--(axis cs:600000,285.150741666667)
--(axis cs:525000,250.157765)
--(axis cs:450000,205.673518333333)
--(axis cs:375000,176.253973333333)
--(axis cs:300000,160.901923333333)
--(axis cs:225000,131.07506)
--(axis cs:150000,82.5371166666667)
--(axis cs:75000,23.7965627)
--(axis cs:1,6.77901475791667)
--cycle;

\addplot [line width=\linewidthdime, C0, mark=*, mark size=0, mark options={solid}]
table {%
1 6.21731321666667
75000 21.795235
150000 72.75661
225000 114.506553333333
300000 146.230666666667
375000 168.785311666667
450000 192.863928333333
525000 216.569268333333
600000 250.331875
675000 262.220136666667
750000 309.84049
825000 323.952626666667
900000 351.00367
975000 401.235143333333
1050000 429.149441666667
1125000 451.573716666667
1200000 462.784196666667
1275000 498.048691666667
1350000 503.234986666667
1425000 530.236976666667
1500000 546.020048333333
1575000 557.309815
1650000 569.905348333333
1725000 576.280983333333
1800000 596.698425
1875000 599.94853
1950000 638.555295
2025000 635.158215
2100000 660.178686666667
2175000 666.541921666667
2250000 660.835705
2325000 677.928385
2400000 678.28305
2475000 691.880483333333
2550000 701.649325
2625000 699.744298333333
2700000 694.774641666667
2775000 718.530593333333
2850000 713.703046666667
2925000 729.128026666667
3000000 713.8313
};
\path [draw=C1, fill=C1, opacity=0.2]
(axis cs:1,6.77779228333333)
--(axis cs:1,5.6454462)
--(axis cs:75000,13.192111)
--(axis cs:150000,33.373557)
--(axis cs:225000,61.2938256666667)
--(axis cs:300000,120.016403166667)
--(axis cs:375000,171.7728935)
--(axis cs:450000,220.303233333333)
--(axis cs:525000,287.51594)
--(axis cs:600000,345.985706666667)
--(axis cs:675000,375.99755)
--(axis cs:750000,385.816735)
--(axis cs:825000,484.023426666667)
--(axis cs:900000,500.420616666667)
--(axis cs:975000,528.7232)
--(axis cs:1050000,485.542121666667)
--(axis cs:1125000,548.90671)
--(axis cs:1200000,577.96613)
--(axis cs:1275000,618.55874)
--(axis cs:1350000,625.843644625)
--(axis cs:1425000,620.230685)
--(axis cs:1500000,670.084116666667)
--(axis cs:1575000,666.705233333333)
--(axis cs:1650000,698.7523)
--(axis cs:1725000,692.89836)
--(axis cs:1800000,703.88816)
--(axis cs:1875000,717.897567041667)
--(axis cs:1950000,720.886863333333)
--(axis cs:2025000,645.359551666667)
--(axis cs:2100000,595.2405225)
--(axis cs:2175000,723.713683333333)
--(axis cs:2250000,756.195295)
--(axis cs:2325000,771.088566666667)
--(axis cs:2400000,668.475967916667)
--(axis cs:2475000,743.654456666667)
--(axis cs:2550000,736.213633333333)
--(axis cs:2625000,749.79453)
--(axis cs:2700000,736.285061666667)
--(axis cs:2775000,775.64339)
--(axis cs:2850000,782.48355)
--(axis cs:2925000,780.489636666667)
--(axis cs:3000000,770.676025)
--(axis cs:3000000,827.493025)
--(axis cs:3000000,827.493025)
--(axis cs:2925000,834.8158)
--(axis cs:2850000,823.143141666667)
--(axis cs:2775000,830.781833333333)
--(axis cs:2700000,824.169479666667)
--(axis cs:2625000,833.258581333333)
--(axis cs:2550000,816.335153333333)
--(axis cs:2475000,835.308423333333)
--(axis cs:2400000,839.038746666667)
--(axis cs:2325000,819.740543333333)
--(axis cs:2250000,818.680797708333)
--(axis cs:2175000,803.737111666667)
--(axis cs:2100000,790.368157)
--(axis cs:2025000,806.313896666667)
--(axis cs:1950000,805.564308333333)
--(axis cs:1875000,789.8978)
--(axis cs:1800000,785.000679166667)
--(axis cs:1725000,767.124138333333)
--(axis cs:1650000,759.130021666667)
--(axis cs:1575000,743.323833333333)
--(axis cs:1500000,731.116993333333)
--(axis cs:1425000,701.952718333333)
--(axis cs:1350000,698.797891666667)
--(axis cs:1275000,668.436818333333)
--(axis cs:1200000,646.790078333333)
--(axis cs:1125000,633.005033333333)
--(axis cs:1050000,596.15743)
--(axis cs:975000,600.356563333333)
--(axis cs:900000,556.532176666667)
--(axis cs:825000,528.821698333333)
--(axis cs:750000,484.572864875)
--(axis cs:675000,456.144685)
--(axis cs:600000,420.356471666667)
--(axis cs:525000,362.974098333333)
--(axis cs:450000,313.987165)
--(axis cs:375000,254.932186666667)
--(axis cs:300000,184.542658333333)
--(axis cs:225000,128.248941666667)
--(axis cs:150000,81.5557691666667)
--(axis cs:75000,38.642973)
--(axis cs:1,6.77779228333333)
--cycle;

\addplot [line width=\linewidthdime, C1, mark=*, mark size=0, mark options={solid}]
table {%
1 6.21731321666667
75000 23.234224
150000 42.76285
225000 96.4487016666667
300000 161.580498333333
375000 206.671753333333
450000 257.980293333333
525000 318.376303333333
600000 386.66074
675000 412.943236666667
750000 436.81992
825000 504.845973333333
900000 526.84992
975000 566.728766666667
1050000 562.233523333333
1125000 598.26834
1200000 611.084238333333
1275000 640.080303333333
1350000 662.24455
1425000 669.740306666667
1500000 701.389226666667
1575000 699.021706666667
1650000 731.213333333333
1725000 728.181965
1800000 743.476393333333
1875000 759.708911666667
1950000 764.764993333333
2025000 779.829181666667
2100000 750.560726666667
2175000 766.374685
2250000 789.383758333333
2325000 801.652243333333
2400000 807.818806666667
2475000 802.767416666667
2550000 780.608276666667
2625000 805.666888333333
2700000 798.535888333333
2775000 805.852318333333
2850000 810.52055
2925000 821.589403333333
3000000 805.513291666667
};
\path [draw=C3, fill=C3, opacity=0.2]
(axis cs:1,6.7789808)
--(axis cs:1,5.64503465)
--(axis cs:75000,15.4860575)
--(axis cs:150000,50.3197336666667)
--(axis cs:225000,117.495146666667)
--(axis cs:300000,196.20587)
--(axis cs:375000,248.847483333333)
--(axis cs:450000,241.23715)
--(axis cs:525000,282.718593333333)
--(axis cs:600000,346.804435)
--(axis cs:675000,374.878242)
--(axis cs:750000,403.88618)
--(axis cs:825000,431.327293333333)
--(axis cs:900000,427.045318333333)
--(axis cs:975000,467.49293425)
--(axis cs:1050000,470.675713333333)
--(axis cs:1125000,498.166373333333)
--(axis cs:1200000,524.61077)
--(axis cs:1275000,526.337566666667)
--(axis cs:1350000,543.514556666667)
--(axis cs:1425000,557.9416)
--(axis cs:1500000,547.438680583334)
--(axis cs:1575000,584.22789)
--(axis cs:1650000,590.02265)
--(axis cs:1725000,596.489593333333)
--(axis cs:1800000,604.379791875)
--(axis cs:1875000,612.749605)
--(axis cs:1950000,615.377191666667)
--(axis cs:2025000,613.665313333333)
--(axis cs:2100000,623.273542291667)
--(axis cs:2175000,625.9463)
--(axis cs:2250000,612.86094)
--(axis cs:2325000,635.473057291667)
--(axis cs:2400000,637.935976666667)
--(axis cs:2475000,647.339)
--(axis cs:2550000,643.338021666667)
--(axis cs:2625000,661.016133333333)
--(axis cs:2700000,642.8757)
--(axis cs:2775000,648.861888333333)
--(axis cs:2850000,655.498358333333)
--(axis cs:2925000,668.790966666667)
--(axis cs:3000000,669.531625)
--(axis cs:3000000,797.328245)
--(axis cs:3000000,797.328245)
--(axis cs:2925000,812.381966666667)
--(axis cs:2850000,811.048568333333)
--(axis cs:2775000,805.738583333333)
--(axis cs:2700000,812.97896)
--(axis cs:2625000,815.187776666667)
--(axis cs:2550000,792.325869166667)
--(axis cs:2475000,791.855291666667)
--(axis cs:2400000,787.386453333333)
--(axis cs:2325000,755.803311666667)
--(axis cs:2250000,771.904865)
--(axis cs:2175000,763.846793333333)
--(axis cs:2100000,768.0811055)
--(axis cs:2025000,768.096478041667)
--(axis cs:1950000,762.823413333333)
--(axis cs:1875000,746.973033333333)
--(axis cs:1800000,738.54912)
--(axis cs:1725000,730.684051666667)
--(axis cs:1650000,719.498785)
--(axis cs:1575000,716.99392)
--(axis cs:1500000,695.735943333333)
--(axis cs:1425000,695.702793333333)
--(axis cs:1350000,670.685723333333)
--(axis cs:1275000,662.224783333333)
--(axis cs:1200000,658.009566666667)
--(axis cs:1125000,627.17524225)
--(axis cs:1050000,604.48534)
--(axis cs:975000,597.68689)
--(axis cs:900000,581.189353333333)
--(axis cs:825000,548.939053333333)
--(axis cs:750000,516.750406666667)
--(axis cs:675000,500.86067)
--(axis cs:600000,457.445263333333)
--(axis cs:525000,416.607634458333)
--(axis cs:450000,391.208711666667)
--(axis cs:375000,343.01117)
--(axis cs:300000,283.617185)
--(axis cs:225000,220.604813333333)
--(axis cs:150000,140.60036)
--(axis cs:75000,40.5402306666667)
--(axis cs:1,6.7789808)
--cycle;

\addplot [line width =\linewidthdime, C3, mark=*, mark size=0, mark options={solid}]
table {%
1 6.21731321666667
75000 24.6861146666667
150000 92.4884813333333
225000 160.813744333333
300000 241.682315
375000 306.650898333333
450000 321.438153333333
525000 352.84999
600000 399.418973333333
675000 436.593163333333
750000 464.165533333333
825000 493.630033333333
900000 528.104716666667
975000 545.41824
1050000 552.732928333333
1125000 566.65974
1200000 607.605821666667
1275000 604.850893333333
1350000 622.615008333333
1425000 641.52472
1500000 632.594268333333
1575000 668.763421666667
1650000 675.501586666667
1725000 685.480978333333
1800000 694.880568333333
1875000 697.229521666667
1950000 707.730915
2025000 715.729543333333
2100000 711.488441666667
2175000 711.370843333333
2250000 706.43842
2325000 714.780741666667
2400000 736.155055
2475000 744.774666666667
2550000 745.894818333333
2625000 763.089821666667
2700000 756.2565
2775000 747.494315
2850000 757.639806666667
2925000 765.57981
3000000 765.790613333333
};
\path [draw=C4, fill=C4, opacity=0.2]
(axis cs:1,6.78237173333333)
--(axis cs:1,5.64289936666667)
--(axis cs:75000,17.9192972583333)
--(axis cs:150000,63.4254316666667)
--(axis cs:225000,126.808483525)
--(axis cs:300000,171.556595)
--(axis cs:375000,233.989082708333)
--(axis cs:450000,282.56608)
--(axis cs:525000,318.172716875)
--(axis cs:600000,315.922775)
--(axis cs:675000,388.286496666667)
--(axis cs:750000,406.59078)
--(axis cs:825000,432.7052)
--(axis cs:900000,445.606696666667)
--(axis cs:975000,447.969856666667)
--(axis cs:1050000,450.773615)
--(axis cs:1125000,464.76792925)
--(axis cs:1200000,468.25359)
--(axis cs:1275000,484.874891666667)
--(axis cs:1350000,485.228771666667)
--(axis cs:1425000,492.358671666667)
--(axis cs:1500000,501.927151666667)
--(axis cs:1575000,508.970951666667)
--(axis cs:1650000,512.376186666667)
--(axis cs:1725000,502.766398333333)
--(axis cs:1800000,510.054328333333)
--(axis cs:1875000,507.44308)
--(axis cs:1950000,526.993976666667)
--(axis cs:2025000,519.400726666667)
--(axis cs:2100000,529.391925)
--(axis cs:2175000,533.250456666667)
--(axis cs:2250000,538.98909)
--(axis cs:2325000,505.917946666667)
--(axis cs:2400000,543.547425458333)
--(axis cs:2475000,500.533943333333)
--(axis cs:2550000,540.943844833333)
--(axis cs:2625000,552.334016666667)
--(axis cs:2700000,474.275015)
--(axis cs:2775000,550.181803333333)
--(axis cs:2850000,542.496643333333)
--(axis cs:2925000,561.68943)
--(axis cs:3000000,556.49008)
--(axis cs:3000000,695.666756666667)
--(axis cs:3000000,695.666756666667)
--(axis cs:2925000,703.057563333333)
--(axis cs:2850000,704.38867)
--(axis cs:2775000,680.282676666667)
--(axis cs:2700000,676.166277083334)
--(axis cs:2625000,696.227573333333)
--(axis cs:2550000,678.491101666667)
--(axis cs:2475000,665.037966666667)
--(axis cs:2400000,664.2123185)
--(axis cs:2325000,670.133466666667)
--(axis cs:2250000,666.698308333333)
--(axis cs:2175000,666.84069)
--(axis cs:2100000,643.761933333333)
--(axis cs:2025000,636.763675)
--(axis cs:1950000,633.729883333333)
--(axis cs:1875000,631.430348333333)
--(axis cs:1800000,625.901868333333)
--(axis cs:1725000,608.686382708333)
--(axis cs:1650000,601.747791666667)
--(axis cs:1575000,609.359848333333)
--(axis cs:1500000,578.848328333333)
--(axis cs:1425000,580.778946666667)
--(axis cs:1350000,577.265266666667)
--(axis cs:1275000,574.183852625)
--(axis cs:1200000,563.541845)
--(axis cs:1125000,547.654096666667)
--(axis cs:1050000,523.382258333333)
--(axis cs:975000,518.584443333333)
--(axis cs:900000,488.865946666667)
--(axis cs:825000,490.32467)
--(axis cs:750000,465.269181666667)
--(axis cs:675000,441.94512)
--(axis cs:600000,429.324076666667)
--(axis cs:525000,402.110298333333)
--(axis cs:450000,377.434965)
--(axis cs:375000,341.79572)
--(axis cs:300000,271.953795)
--(axis cs:225000,184.835081666667)
--(axis cs:150000,126.32286025)
--(axis cs:75000,34.7162325)
--(axis cs:1,6.78237173333333)
--cycle;

\addplot [line width=\linewidthdime, C4, mark=*, mark size=0, mark options={solid}]
table {%
1 6.21731321666667
75000 27.5053421666667
150000 89.2824941666667
225000 151.325434
300000 230.389185
375000 302.202218333333
450000 343.977808333333
525000 368.504448333333
600000 381.607665
675000 424.407763333333
750000 439.562763333333
825000 466.002391666667
900000 470.92938
975000 490.342795
1050000 487.374173333333
1125000 502.150376666667
1200000 515.912985
1275000 526.891398333333
1350000 518.929268333333
1425000 528.792513333333
1500000 538.664986666667
1575000 553.59301
1650000 545.294433333333
1725000 547.315965
1800000 557.023976666667
1875000 566.952106666667
1950000 573.97352
2025000 559.881641666667
2100000 579.680123333333
2175000 586.648251666667
2250000 584.81991
2325000 593.946286666667
2400000 584.503836666667
2475000 573.781458333333
2550000 597.080675
2625000 618.949693333333
2700000 595.190093333333
2775000 607.322023333333
2850000 617.890743333333
2925000 623.280883333333
3000000 613.856913333333
};
\path [draw=C5, fill=C5, opacity=0.2]
(axis cs:1,6.7789808)
--(axis cs:1,5.6454462)
--(axis cs:75000,12.4129023333333)
--(axis cs:150000,18.691406)
--(axis cs:225000,46.6979821666667)
--(axis cs:300000,82.8818396666667)
--(axis cs:375000,97.746777)
--(axis cs:450000,149.115790666667)
--(axis cs:525000,174.960733333333)
--(axis cs:600000,156.887033)
--(axis cs:675000,230.567503666667)
--(axis cs:750000,239.808616958333)
--(axis cs:825000,263.1599305)
--(axis cs:900000,255.552918333333)
--(axis cs:975000,285.179116333333)
--(axis cs:1050000,287.863883333333)
--(axis cs:1125000,313.676190666667)
--(axis cs:1200000,339.857216333333)
--(axis cs:1275000,336.764319166667)
--(axis cs:1350000,354.8411225)
--(axis cs:1425000,355.013705)
--(axis cs:1500000,382.574541666667)
--(axis cs:1575000,394.200431666667)
--(axis cs:1650000,398.402556666667)
--(axis cs:1725000,317.105807458333)
--(axis cs:1800000,405.750238333333)
--(axis cs:1875000,420.442273333333)
--(axis cs:1950000,421.989668333333)
--(axis cs:2025000,426.871725)
--(axis cs:2100000,426.939376666667)
--(axis cs:2175000,394.635271333333)
--(axis cs:2250000,444.618741791667)
--(axis cs:2325000,389.33789)
--(axis cs:2400000,448.154216666667)
--(axis cs:2475000,460.094523333333)
--(axis cs:2550000,356.416150066667)
--(axis cs:2625000,421.063358333333)
--(axis cs:2700000,484.024856666667)
--(axis cs:2775000,486.19462)
--(axis cs:2850000,483.03179)
--(axis cs:2925000,499.572325)
--(axis cs:3000000,476.448261666667)
--(axis cs:3000000,686.118883333333)
--(axis cs:3000000,686.118883333333)
--(axis cs:2925000,706.853655)
--(axis cs:2850000,694.190088333333)
--(axis cs:2775000,702.19725)
--(axis cs:2700000,693.992685)
--(axis cs:2625000,683.2654)
--(axis cs:2550000,641.942895)
--(axis cs:2475000,678.364916666667)
--(axis cs:2400000,665.692208333333)
--(axis cs:2325000,653.163586666667)
--(axis cs:2250000,652.23255)
--(axis cs:2175000,631.806883333333)
--(axis cs:2100000,606.8741)
--(axis cs:2025000,619.2095)
--(axis cs:1950000,605.40168)
--(axis cs:1875000,615.65135)
--(axis cs:1800000,591.334628333333)
--(axis cs:1725000,590.217716666667)
--(axis cs:1650000,571.499675)
--(axis cs:1575000,574.826313333333)
--(axis cs:1500000,558.05845)
--(axis cs:1425000,538.23778)
--(axis cs:1350000,508.830476666667)
--(axis cs:1275000,504.647073333333)
--(axis cs:1200000,486.207706875)
--(axis cs:1125000,464.047959583333)
--(axis cs:1050000,441.110606666667)
--(axis cs:975000,439.210568333333)
--(axis cs:900000,413.693958333333)
--(axis cs:825000,393.532926666667)
--(axis cs:750000,368.353783333333)
--(axis cs:675000,349.156483333333)
--(axis cs:600000,312.048603333333)
--(axis cs:525000,292.259208333333)
--(axis cs:450000,261.419155)
--(axis cs:375000,230.88683)
--(axis cs:300000,192.616263333333)
--(axis cs:225000,125.033163333333)
--(axis cs:150000,62.7031406666667)
--(axis cs:75000,29.7667333333333)
--(axis cs:1,6.7789808)
--cycle;

\addplot [line width=\linewidthdime, C5, mark=*, mark size=0, mark options={solid}]
table {%
1 6.21731321666667
75000 20.8227488333333
150000 39.090304
225000 83.6150591666667
300000 126.572516666667
375000 168.156125
450000 208.803878333333
525000 236.390621666667
600000 255.977703333333
675000 307.787093333333
750000 321.887206666667
825000 341.583128333333
900000 349.487541666667
975000 364.56241
1050000 370.04829
1125000 390.933865
1200000 418.560173333333
1275000 429.208185
1350000 445.484876666667
1425000 461.478426666667
1500000 481.286283333333
1575000 493.300616666667
1650000 489.189401666667
1725000 490.588285
1800000 508.003011666667
1875000 526.332678333333
1950000 521.157121666667
2025000 533.048061666667
2100000 520.758193333333
2175000 541.62375
2250000 548.168346666667
2325000 537.054751666667
2400000 566.576028333333
2475000 564.040543333333
2550000 531.352275
2625000 554.613958333333
2700000 593.460943333333
2775000 595.345438333333
2850000 585.084225
2925000 606.38759
3000000 575.508026666667
};
\path [draw=C8, fill=C8, opacity=0.2]
(axis cs:1,6.7789808)
--(axis cs:1,5.64291553333333)
--(axis cs:75000,9.4124673)
--(axis cs:150000,17.3314582333333)
--(axis cs:225000,52.9633973333333)
--(axis cs:300000,101.5769675)
--(axis cs:375000,102.222525166667)
--(axis cs:450000,142.5976928625)
--(axis cs:525000,164.963788375)
--(axis cs:600000,190.225708)
--(axis cs:675000,205.045780333333)
--(axis cs:750000,229.856895)
--(axis cs:825000,226.751546233333)
--(axis cs:900000,256.493093833333)
--(axis cs:975000,251.30497)
--(axis cs:1050000,276.553881666667)
--(axis cs:1125000,286.622266666667)
--(axis cs:1200000,289.235215)
--(axis cs:1275000,296.566369791667)
--(axis cs:1350000,302.247237333333)
--(axis cs:1425000,311.288254166667)
--(axis cs:1500000,317.945753666667)
--(axis cs:1575000,321.976207333333)
--(axis cs:1650000,326.695174)
--(axis cs:1725000,321.680736)
--(axis cs:1800000,335.983499041667)
--(axis cs:1875000,328.8955375)
--(axis cs:1950000,334.52118)
--(axis cs:2025000,345.772400625)
--(axis cs:2100000,350.838541916667)
--(axis cs:2175000,346.9468675)
--(axis cs:2250000,365.222333333333)
--(axis cs:2325000,344.562520958333)
--(axis cs:2400000,367.718743333333)
--(axis cs:2475000,376.752096666667)
--(axis cs:2550000,358.89941)
--(axis cs:2625000,377.792513333333)
--(axis cs:2700000,373.528251666667)
--(axis cs:2775000,377.020613333333)
--(axis cs:2850000,372.75245)
--(axis cs:2925000,364.258817333333)
--(axis cs:3000000,377.686428333333)
--(axis cs:3000000,554.959116666667)
--(axis cs:3000000,554.959116666667)
--(axis cs:2925000,559.12053)
--(axis cs:2850000,558.602342583334)
--(axis cs:2775000,537.087753333333)
--(axis cs:2700000,548.07257)
--(axis cs:2625000,548.963626666667)
--(axis cs:2550000,547.906388333333)
--(axis cs:2475000,542.11712)
--(axis cs:2400000,532.937713333333)
--(axis cs:2325000,513.948656666667)
--(axis cs:2250000,519.10051)
--(axis cs:2175000,524.497208333333)
--(axis cs:2100000,504.955276666667)
--(axis cs:2025000,500.758661666667)
--(axis cs:1950000,481.218491666667)
--(axis cs:1875000,475.814105)
--(axis cs:1800000,476.492296666667)
--(axis cs:1725000,478.645116666667)
--(axis cs:1650000,477.239605)
--(axis cs:1575000,489.645466666667)
--(axis cs:1500000,464.911833333333)
--(axis cs:1425000,464.36491)
--(axis cs:1350000,456.02617)
--(axis cs:1275000,443.059238333333)
--(axis cs:1200000,437.555213333333)
--(axis cs:1125000,430.863184333333)
--(axis cs:1050000,419.827471666667)
--(axis cs:975000,408.759126666667)
--(axis cs:900000,391.891775)
--(axis cs:825000,377.6072)
--(axis cs:750000,373.54594)
--(axis cs:675000,349.649803333333)
--(axis cs:600000,327.898075)
--(axis cs:525000,302.575626666667)
--(axis cs:450000,268.6613)
--(axis cs:375000,239.243746666667)
--(axis cs:300000,189.132441666667)
--(axis cs:225000,160.150965833333)
--(axis cs:150000,78.2773068333333)
--(axis cs:75000,17.4212863333333)
--(axis cs:1,6.7789808)
--cycle;

\addplot [line width=\linewidthdime, C8, mark=*, mark size=0, mark options={solid}]
table {%
1 6.21731321666667
75000 13.47092
150000 33.3836346666667
225000 98.6985431666667
300000 144.634583166667
375000 175.533951666667
450000 209.989251666667
525000 242.447783333333
600000 277.715508333333
675000 294.81058
750000 320.20364
825000 318.6075
900000 338.964641666667
975000 349.526221666667
1050000 368.305633333333
1125000 379.405123333333
1200000 380.97295
1275000 390.509931666667
1350000 398.27875
1425000 417.129783333333
1500000 414.996983333333
1575000 425.770446666667
1650000 426.320526666667
1725000 422.677443333333
1800000 429.630751666667
1875000 420.704703333333
1950000 424.294305
2025000 444.550001666667
2100000 447.655826666667
2175000 453.966486666667
2250000 457.136071666667
2325000 431.689171666667
2400000 460.328878333333
2475000 466.293323333333
2550000 458.64369
2625000 466.096148333333
2700000 458.371545
2775000 453.671963333333
2850000 462.506571666667
2925000 473.529711666667
3000000 461.190121666667
};
\path [draw=C7, fill=C7, opacity=0.2]
(axis cs:1,6.77576936666667)
--(axis cs:1,5.64289936666667)
--(axis cs:75000,3.50145868333333)
--(axis cs:150000,3.03823495)
--(axis cs:225000,7.80344773333333)
--(axis cs:300000,15.1959053333333)
--(axis cs:375000,18.5034482)
--(axis cs:450000,39.189309)
--(axis cs:525000,36.4076191666667)
--(axis cs:600000,45.5674847650004)
--(axis cs:675000,88.4275291666667)
--(axis cs:750000,90.2524866666667)
--(axis cs:825000,96.541988)
--(axis cs:900000,76.6027375)
--(axis cs:975000,119.990786266667)
--(axis cs:1050000,104.8416875)
--(axis cs:1125000,138.4211625)
--(axis cs:1200000,140.496076666667)
--(axis cs:1275000,151.037835)
--(axis cs:1350000,167.506894633334)
--(axis cs:1425000,151.268965041667)
--(axis cs:1500000,177.999501666667)
--(axis cs:1575000,150.894498333333)
--(axis cs:1650000,188.924378333333)
--(axis cs:1725000,171.972185)
--(axis cs:1800000,224.47347)
--(axis cs:1875000,206.59252)
--(axis cs:1950000,243.250116666667)
--(axis cs:2025000,238.664676666667)
--(axis cs:2100000,256.360866666667)
--(axis cs:2175000,261.019045)
--(axis cs:2250000,274.127406666667)
--(axis cs:2325000,286.401523333333)
--(axis cs:2400000,262.819526666667)
--(axis cs:2475000,302.762569791667)
--(axis cs:2550000,176.224818333333)
--(axis cs:2625000,320.492048333333)
--(axis cs:2700000,302.236468208334)
--(axis cs:2775000,278.253405)
--(axis cs:2850000,332.351931666667)
--(axis cs:2925000,359.705546666667)
--(axis cs:3000000,358.691566666667)
--(axis cs:3000000,504.15511)
--(axis cs:3000000,504.15511)
--(axis cs:2925000,524.568105)
--(axis cs:2850000,497.756562625)
--(axis cs:2775000,481.869911583334)
--(axis cs:2700000,501.89183)
--(axis cs:2625000,476.158643333333)
--(axis cs:2550000,474.759123333333)
--(axis cs:2475000,458.712175)
--(axis cs:2400000,459.580095)
--(axis cs:2325000,461.802686666667)
--(axis cs:2250000,436.69185)
--(axis cs:2175000,439.264936666667)
--(axis cs:2100000,421.15417)
--(axis cs:2025000,405.239983333333)
--(axis cs:1950000,411.747271666667)
--(axis cs:1875000,390.732111666667)
--(axis cs:1800000,384.01697)
--(axis cs:1725000,368.661723333333)
--(axis cs:1650000,349.257366666667)
--(axis cs:1575000,324.216918333333)
--(axis cs:1500000,319.73715)
--(axis cs:1425000,279.37976)
--(axis cs:1350000,304.311968333333)
--(axis cs:1275000,291.908955)
--(axis cs:1200000,270.286778333333)
--(axis cs:1125000,266.046966666667)
--(axis cs:1050000,251.565023333333)
--(axis cs:975000,242.544341666667)
--(axis cs:900000,219.097448333333)
--(axis cs:825000,214.326845)
--(axis cs:750000,198.503528625)
--(axis cs:675000,188.036166666667)
--(axis cs:600000,136.040361666667)
--(axis cs:525000,142.3553445)
--(axis cs:450000,141.369716666667)
--(axis cs:375000,95.4977814483336)
--(axis cs:300000,45.8442208333333)
--(axis cs:225000,19.1076528333333)
--(axis cs:150000,6.91921088333333)
--(axis cs:75000,8.82462750750001)
--(axis cs:1,6.77576936666667)
--cycle;

\addplot [line width=\linewidthdime, C7, mark=*, mark size=0, mark options={solid}]
table {%
1 6.21731321666667
75000 5.253077
150000 4.1307842
225000 11.116095
300000 30.8361456666667
375000 59.315322
450000 94.1836196666667
525000 90.2431186666667
600000 93.3293095
675000 134.245496666667
750000 144.621306666667
825000 153.445905
900000 147.3714875
975000 188.856203333333
1050000 189.672935833333
1125000 204.277875
1200000 207.117788333333
1275000 219.77152
1350000 233.742483333333
1425000 204.026615
1500000 250.330331666667
1575000 227.57098
1650000 259.547911666667
1725000 258.281573333333
1800000 296.176831666667
1875000 292.834295
1950000 320.9015
2025000 322.061226666667
2100000 334.527795
2175000 350.691
2250000 363.331376666667
2325000 381.825788333333
2400000 362.641175
2475000 399.489923333333
2550000 349.276133333333
2625000 413.17155
2700000 411.355768333333
2775000 370.087901666667
2850000 420.774238333333
2925000 449.816926666667
3000000 434.692828333333
};
\path [draw=C6, fill=C6, opacity=0.2]
(axis cs:1,6.9812828)
--(axis cs:1,5.55815306666667)
--(axis cs:36000,4.64270726666667)
--(axis cs:72000,4.40833533333333)
--(axis cs:108000,4.7460012)
--(axis cs:144000,5.269655)
--(axis cs:180000,5.016788)
--(axis cs:216000,5.02033486666667)
--(axis cs:252000,4.58681713333333)
--(axis cs:288000,4.591249)
--(axis cs:324000,4.70324226666667)
--(axis cs:360000,4.0491373)
--(axis cs:396000,4.07145986666667)
--(axis cs:432000,4.5005639)
--(axis cs:468000,4.40048913333333)
--(axis cs:504000,4.40163946666667)
--(axis cs:540000,5.0360962)
--(axis cs:576000,4.5731699)
--(axis cs:612000,5.30229033333333)
--(axis cs:648000,4.20323053333333)
--(axis cs:684000,5.18204053333333)
--(axis cs:720000,5.21875266666667)
--(axis cs:756000,4.718841)
--(axis cs:792000,4.4442496)
--(axis cs:828000,4.66176246666667)
--(axis cs:864000,4.58836993333333)
--(axis cs:900000,4.45616026666667)
--(axis cs:936000,4.60671153333333)
--(axis cs:972000,4.19781506666667)
--(axis cs:1008000,5.34430326666667)
--(axis cs:1044000,4.39077166666667)
--(axis cs:1080000,4.33613363333333)
--(axis cs:1116000,5.27339733333333)
--(axis cs:1152000,4.66944033333333)
--(axis cs:1188000,4.6240356)
--(axis cs:1224000,4.5440958)
--(axis cs:1260000,4.3664907)
--(axis cs:1296000,4.5793652)
--(axis cs:1332000,4.66924266666667)
--(axis cs:1368000,3.965145)
--(axis cs:1404000,4.66238746666667)
--(axis cs:1440000,4.77891566666667)
--(axis cs:1476000,4.7915225)
--(axis cs:1512000,4.24932553333333)
--(axis cs:1548000,4.3862614)
--(axis cs:1584000,5.12497203333333)
--(axis cs:1620000,4.50460833333333)
--(axis cs:1656000,4.5744674)
--(axis cs:1692000,4.8700602)
--(axis cs:1728000,4.71128333333333)
--(axis cs:1764000,4.55467146666667)
--(axis cs:1800000,5.49946813333333)
--(axis cs:1836000,4.29188066666667)
--(axis cs:1872000,4.2263158)
--(axis cs:1908000,4.81428313333333)
--(axis cs:1944000,4.34031866666667)
--(axis cs:1980000,4.5207506)
--(axis cs:2016000,5.01495883333333)
--(axis cs:2052000,4.9837772)
--(axis cs:2088000,4.70743353333333)
--(axis cs:2124000,4.79699273333333)
--(axis cs:2160000,4.69314326666667)
--(axis cs:2196000,3.8972669)
--(axis cs:2232000,4.91765086666667)
--(axis cs:2268000,4.29147166666667)
--(axis cs:2304000,5.0530026)
--(axis cs:2340000,4.58989413333333)
--(axis cs:2376000,4.712138)
--(axis cs:2412000,4.68172313333333)
--(axis cs:2448000,4.105184)
--(axis cs:2484000,4.40403613333333)
--(axis cs:2520000,4.86467473333333)
--(axis cs:2556000,4.95357226666667)
--(axis cs:2592000,5.20899723333333)
--(axis cs:2628000,4.582157)
--(axis cs:2664000,4.68247876666667)
--(axis cs:2700000,4.5075402)
--(axis cs:2736000,4.54131653333333)
--(axis cs:2772000,4.800876)
--(axis cs:2808000,4.84288156666667)
--(axis cs:2844000,4.59038823333333)
--(axis cs:2880000,4.71261006666667)
--(axis cs:2916000,4.706956)
--(axis cs:2952000,4.72729556666667)
--(axis cs:2988000,4.68146486666667)
--(axis cs:2988000,7.3866156)
--(axis cs:2988000,7.3866156)
--(axis cs:2952000,5.58287833333333)
--(axis cs:2916000,6.16278206666667)
--(axis cs:2880000,5.8543394)
--(axis cs:2844000,7.0992872)
--(axis cs:2808000,5.852884)
--(axis cs:2772000,6.5409335)
--(axis cs:2736000,5.91396676666667)
--(axis cs:2700000,5.95869233333333)
--(axis cs:2664000,6.5659872)
--(axis cs:2628000,6.01688243333333)
--(axis cs:2592000,6.61557266666667)
--(axis cs:2556000,6.4482362)
--(axis cs:2520000,6.0962796)
--(axis cs:2484000,6.1821412)
--(axis cs:2448000,6.5614479)
--(axis cs:2412000,6.25627773333333)
--(axis cs:2376000,5.72259883333333)
--(axis cs:2340000,6.25883253333333)
--(axis cs:2304000,6.86157853333333)
--(axis cs:2268000,6.00031933333333)
--(axis cs:2232000,6.2311058)
--(axis cs:2196000,5.69142976666667)
--(axis cs:2160000,6.14034273333333)
--(axis cs:2124000,6.53163243333333)
--(axis cs:2088000,6.19169873333333)
--(axis cs:2052000,7.39258266666667)
--(axis cs:2016000,5.8830814)
--(axis cs:1980000,5.56394333333333)
--(axis cs:1944000,5.70484376666667)
--(axis cs:1908000,6.6507914)
--(axis cs:1872000,6.02142733333333)
--(axis cs:1836000,5.55149913333333)
--(axis cs:1800000,6.51535853333333)
--(axis cs:1764000,5.17395066666667)
--(axis cs:1728000,6.03141086666667)
--(axis cs:1692000,6.1870729)
--(axis cs:1656000,7.14635233333333)
--(axis cs:1620000,5.63219533333333)
--(axis cs:1584000,5.76588203333333)
--(axis cs:1548000,5.38945516666667)
--(axis cs:1512000,5.65740713333333)
--(axis cs:1476000,6.3625454)
--(axis cs:1440000,5.39659866666667)
--(axis cs:1404000,5.82999093333333)
--(axis cs:1368000,5.0504985)
--(axis cs:1332000,5.43350313333333)
--(axis cs:1296000,5.33941133333333)
--(axis cs:1260000,5.557514)
--(axis cs:1224000,6.27759833333333)
--(axis cs:1188000,5.98050653333333)
--(axis cs:1152000,6.45088623333333)
--(axis cs:1116000,6.2619232)
--(axis cs:1080000,7.7436922)
--(axis cs:1044000,5.69646133333333)
--(axis cs:1008000,6.25468076666667)
--(axis cs:972000,7.14397706666667)
--(axis cs:936000,5.81546706666667)
--(axis cs:900000,7.19774586666667)
--(axis cs:864000,6.63735433333333)
--(axis cs:828000,7.05162966666667)
--(axis cs:792000,6.32014086666667)
--(axis cs:756000,6.0808052)
--(axis cs:720000,6.98344723333333)
--(axis cs:684000,7.24339393333333)
--(axis cs:648000,5.440585)
--(axis cs:612000,6.37086776666667)
--(axis cs:576000,5.88254583333333)
--(axis cs:540000,5.89970253333333)
--(axis cs:504000,5.739004)
--(axis cs:468000,6.42585966666667)
--(axis cs:432000,5.50213233333333)
--(axis cs:396000,5.60572626666667)
--(axis cs:360000,5.37717243333333)
--(axis cs:324000,5.87759166666667)
--(axis cs:288000,5.90751273333333)
--(axis cs:252000,6.02331956666667)
--(axis cs:216000,5.87698323333333)
--(axis cs:180000,6.67633166666667)
--(axis cs:144000,7.19540406666667)
--(axis cs:108000,5.47043033333333)
--(axis cs:72000,6.8538961)
--(axis cs:36000,4.96500693333333)
--(axis cs:1,6.9812828)
--cycle;

\addplot [line width=\linewidthdime, C6, mark=*, mark size=0, mark options={solid}]
table {%
1 6.43196386666667
36000 4.71051633333333
72000 5.13775976666667
108000 5.24261886666667
144000 5.51829966666667
180000 5.64658856666667
216000 5.38689296666667
252000 5.19010796666667
288000 5.39336523333333
324000 5.19317766666667
360000 4.70536583333333
396000 5.10026153333333
432000 4.97143306666667
468000 5.18650023333333
504000 4.94802313333333
540000 5.49724323333333
576000 5.1908188
612000 5.4937367
648000 4.614996
684000 5.90373756666667
720000 5.84010806666667
756000 5.74715513333333
792000 5.22414233333333
828000 5.5550948
864000 5.73854866666667
900000 5.35176193333333
936000 5.10593973333333
972000 5.73306913333333
1008000 5.55027496666667
1044000 5.13769816666667
1080000 5.70956443333333
1116000 5.51703786666667
1152000 5.28572236666667
1188000 5.45507986666667
1224000 5.24239053333333
1260000 4.8643601
1296000 4.943376
1332000 5.04164723333333
1368000 4.53417983333333
1404000 5.25825346666667
1440000 5.0463887
1476000 5.87934166666667
1512000 5.3945577
1548000 4.94977403333333
1584000 5.22829613333333
1620000 4.7993422
1656000 5.82566633333333
1692000 5.6415978
1728000 5.44530896666667
1764000 4.9331166
1800000 5.98199113333333
1836000 5.0593467
1872000 4.9263694
1908000 6.13343693333333
1944000 5.14781176666667
1980000 5.00160466666667
2016000 5.25194606666667
2052000 5.63953603333333
2088000 5.57266936666667
2124000 5.89529176666667
2160000 5.45555266666667
2196000 4.80036943333333
2232000 5.54363353333333
2268000 4.82849466666667
2304000 5.5669531
2340000 5.55033263333333
2376000 5.41993486666667
2412000 5.5280028
2448000 5.0603249
2484000 5.24133196666667
2520000 5.66627236666667
2556000 5.49692786666667
2592000 6.15267393333333
2628000 5.1039841
2664000 5.20016276666667
2700000 5.4684839
2736000 5.44102656666667
2772000 5.9158646
2808000 5.48867306666667
2844000 5.61468313333333
2880000 5.23250153333333
2916000 5.6323046
2952000 5.31495256666667
2988000 5.95452386666667
};
\end{axis}

\end{tikzpicture}
}
       \subcaption[]{}
       \label{fig::exps_new_ablations_gauss_vs_diff_distrq::dog_run}
    \end{minipage}\hfill
    \vspace{-2.0mm}
    \caption{\textbf{Reward Scaling Sensitivity (a)-(b)}. The $\alpha$ parameter controls the exploration-exploitation trade-off. (a) shows the learning curves for varying values on DMC's dog-run task. Too high $\alpha$ values ($\alpha=0.1$) do not incentivize learning whereas too small $\alpha$ values ($\alpha\leq10^{-5}$) converge to suboptimal behavior. (b) shows the aggregated end performance for each learning curve in (a). For increasing $\alpha$ values, the end performance increases until it reaches an optimum at $\alpha=10^{-3}$ after which the performance starts dropping. \textbf{Diffusion Policy Benefit (c) and (d).} We compare DIME to a Gaussian policy with the same implementation details as DIME on the (a) humanoid-run and (b) dog-run tasks. The diffusion-based policy reaches a higher return (a) and converges faster.} \vspace{-2mm}
\end{figure*}


\subsection{DIME: A Practical Diffusion RL Algorithm}\label{dime_practical}
To obtain a practical algorithm, we use a parameterized function approximation for the $Q$-function and the policy, that is, $Q_{\phi}$ and $\ppi$, with parameters $\phi$  and $\theta$, respectively.  Here, $\ppi$ is represented by a parameterized score network, see Eq. \ref{eq: approximate denoising process}. 
%
To perform approximate policy evaluation, we can minimize the Bellman residual,  
\begin{equation}\label{eq::Bellman_Residual}
    J_Q(\phi) = \frac{1}{2}\E\left[\left(Q_{\phi}(s_t,a^0_t) - Q_{\text{target}}(s_t,a^0_t)\right)^2\right],
\end{equation}
using stochastic gradients with respect to $\phi$. We provide implementation details in \Cref{sec:implementation details}. Moreover, the expectation is computed using state-action pairs collected from environment interactions and saved in a replay buffer. 
%
For policy improvement, we solve the approximate inference problem 
\begin{equation}
\label{eq: joint control as inference2}
\mathcal{L}(\theta) = D_{\text{KL}}\left(\ppi_{0:N}(\ac^{0:N}|\st)|\fpi_{0:N}(\ac^{0:N}|\st)\right),
\end{equation}
where the target policy, i.e., the marginal of the noising process in Eq. \ref{eq: forward joint} is given by the approximate $Q$-function
\begin{equation}
    \fpi_{0}(\ac^{0}|\st) = \frac{\exp Q_{\theta}(\st, \ac^0)}{\Z_{\theta}(\st)},
\end{equation}
where states are again sampled from a replay buffer.
Further expanding $\mathcal{L}(\theta)$ yields
\begin{align}
\label{eq: expanded loss}
    \mathcal{L}(\theta) = & \E_{\ppi}\Bigg[ \log \ppi_N(a^N|s) - Q_{\phi}(\st,\ac^0)   \\ \nonumber
    & + \sum_{n=1}^N \log\frac{\ppi_{n|n-1}(\ac^{n} \big| \ac^{n-1},\st)}{\fpi_{n-1|n}(\ac^{n-1} \big| \ac^{n},\st)} \Bigg] + \log \Z_{\phi}(s),
\end{align}
showing that $\Z_{\phi}$ is not needed to minimize Eq. \ref{eq: expanded loss} as it is independent of $\theta$. Moreover, contrary to the score-matching objective (see Eq. \ref{eq: score matching}) that is commonly used to optimize diffusion models, stochastic optimization of $\mathcal{L}(\theta)$ does not need access to samples $\ac_0 \sim \exp Q_{\phi}/\Z_{\phi}$, instead relying on stochastic gradients obtained via reparameterization trick \cite{kingma2013auto} using samples from the diffusion model $\ppi$.

\subsection{Implementation Details}\label{sec:implementation details}
\textbf{Autotuning Temperature.} We follow implementations like SAC \cite{haarnoja2018softimplementations} where the reward scaling parameter $\alpha$ is not absorbed into the reward but scales the entropy term. 
Choosing $\alpha$ depends on the reward ranges and the dimensionality of the action space which requires tuning it per environment. We instead follow prior works \cite{haarnoja2018softimplementations} for auto-tuning $\alpha$ by optimizing 
\begin{equation}\label{eq:auto_tuning_alpha}
    J(\alpha) = \alpha \left( \mathcal{H}_{\text{target}} - \plb \right),
\end{equation}
where $\mathcal{H}_{\text{target}}$ is a target value for the mismatch between the noising and denoising processes measured by the log ratio. 

\textbf{Autotuning Diffusion Coefficient.} Please note that the objective function in Eq. \ref{eq: expanded loss} is fully differentiable with respect to parameters of the diffusion process. As such, we additionally treat the diffusion coefficient $\beta$ as learnable parameter that is optimized end-to-end, further reducing the need for manual hyperparameter tuning. Further details on the parameterization can be found in Appendix \ref{appdx:implementation_details}.
%

\textbf{$Q$-function.} Following \citet{bhattcrossq} we adopt the CrossQ algorithm, i.e., we use Batch Renormalization in the Q-function and avoid a target network for calculating $Q_{\text{target}}$. When updating the Q-function, the values for the current and next state-action pairs are queried in parallel. The next Q-values are used as target values where the gradients are stopped. Additionally, we employ distributional Q learning as proposed by \cite{bellemare2017distributional}. The details are described in Appendix \ref{appdx:implementation_details}.
