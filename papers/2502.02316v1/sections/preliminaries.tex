\section{Preliminaries}
\subsection{Maximum Entropy Reinforcement Learning}
\label{sec: Maximum Entropy Reinforcement Learning}
\textbf{Notation} We consider the task of learning a policy $\pi: \St \times \Ac \rightarrow \R^+$,
where $\St$ and $\Ac$ denote a continuous state and action space, respectively using reinforcement learning (RL). We formalize the RL problem using an infinite horizon Markov decision process consisting of the tuple $(\St,\Ac,r,p,\rho_{\pi}, \gamma)$, with bounded reward function $r: \St \times \Ac \rightarrow [r_{\text{min}}, r_{\text{max}}]$ and transition density $p: \St \times \St \times \Ac \rightarrow \R^+$ which denotes the likelihood for transitioning into a state $\st' \in \St$ when being in $\st \in \St$ and executing an action $\ac \in \Ac$. We follow \cite{haarnoja2018soft} and slightly overload $\rho_{\pi}$ which denotes the state and state-action marginals induced by a policy $\pi$. Moreover, $\gamma \in [0, 1)$ denotes the discount factor.
For brevity, we use $r_t\triangleq r(\st_t,\ac_t)$. Lastly, we denote objective functions that we aim to maximize as $J$ and minimize as $\mathcal{L}$.

\textbf{Control as inference.} The goal of maximum entropy reinforcement learning (MaxEnt-RL) is to jointly maximize the sum of expected rewards and entropies of a policy
\begin{equation}
\label{eq: marginal max ent}
    J(\pi) = \sum_{t=l}^{\infty} \gamma^{t-l}\E_{\rho_\pi}\left[r_t + \alpha \Ent(\pi(\ac_t|\st_t))\right],
\end{equation}
where $\Ent(\pi(\ac|\st)) = - \int \pi(\ac|\st) \log \pi(\ac|\st) \dd \ac$ is the differential entropy, and $\alpha \in \R^+$ controls the exploration exploitation trade-off \cite{haarnoja2017reinforcement}. To keep the notation uncluttered we absorb $ \alpha$ into the reward function via $r \leftarrow r/\alpha$. Defining the $Q$-function of a policy $\pi$ as
\begin{equation}
\label{eq: marginal Q soft}
Q^{\pi}(\st_t,\ac_t) = r_t + \sum_{l=1}^\infty\gamma^{l} \E_{\rho_{\pi}}\left[r_{t+l}+ \mathcal{H}\left(\pi(\ac_{t+l}|\st_{t+l})\right)\right],
\end{equation}
with $Q^{\pi}: \St \times \Ac \rightarrow \R$,
the MaxEnt objective can be cast as an approximate inference problem of the form
\begin{equation}
\label{eq: marginal control as inference}
\mathcal{L}(\pi) = D_{\text{KL}}\left(\pi(\ac_t|\st_t)\Big|\frac{\exp Q^{\pi}(\st_t,\ac_t)}{\Z^{\pi}(\st_t)}\right),
\end{equation}
in a sense that 
$
    \max_{\pi} J(\pi) = \min_{\pi} \mathcal{L}(\pi).
$
Here, $D_{\text{KL}}$ denotes the Kullback-Leibler divergence and 
\begin{equation}
\label{eq: normalizer}
\Z^{\pi}(\st) = \int \exp Q^{\pi}(\st,\ac) \dd \ac
\end{equation}
is the state-dependent normalization constant.

\textbf{Policy iteration} is a two-step iterative update scheme that is, under certain assumptions, guaranteed to converge to the optimal policy with respect to the maximum entropy objective. The two steps include policy evaluation and policy improvement. 
%
Given a policy $\pi$, policy evaluation aims to evaluate the value of $\pi$. To that end, \cite{haarnoja2018soft} showed that repeated application of the Bellman backup operator $\mathcal{T}^{\pi} Q^{k}$ with 
\begin{equation}
    \label{eq: bellman operator}
    \mathcal{T}^{\pi} Q(\st_t,\ac_t) \triangleq r_t + \gamma \E\left[Q(\st_{t+1},\ac_{t+1}) +\mathcal{H}(\ac_{t+1}|\st_{t+1})\right],
\end{equation}
converges to $Q^{\pi}$ as $k \rightarrow \infty$, starting from any $Q$.
%
To update the policy, that is, to perform the policy improvement step, the $Q$-function of the previous evaluation step, $Q^{\pi_{\text{old}}}$ is used to obtain a new policy according to 
\begin{equation}
\label{eq: marginal policy improvement}
\pi_{\text{new}} = \argmax_{\pi \in \Pi} D_{\text{KL}}\left(\pi(\ac_t|\st_t)\Big|\frac{\exp Q^{\pi_{\text{old}}}(\st_t,\ac_t)}{\Z^{\pi_{\text{old}}}(\st_t)}\right),
\end{equation}
where $\Pi$ is a set of policies such as a family of parameterized distributions.
Note that $\Z^{\pi_{\text{old}}}(\st_t)$ is not required for optimization as it is independent of $\pi$. \citet{haarnoja2018soft} showed that for all state-action pairs $(\st, \ac) \in \St \times \Ac$ it holds that $Q^{\pi_{\text{new}}}(\st,\ac) \geq Q^{\pi_{\text{old}}}(\st,\ac)$ ensuring that policy iteration converges to the optimal policy $\pi^*$ in the limit of infinite repetitions of policy evaluation and improvement.

\subsection{Denoising Diffusion Policies}
\label{sec: Denoising Diffusion Policies}
For a given state $\st \in \St$, we consider a stochastic process on the time-interval $[0,T]$ given by an Ornstein-Uhlenbeck (OU) process \footnote{Please note, for clarity, we slightly abuse notation by using $t$ to denote the time in the stochastic process. This should not be confused with the time step in RL. The distinction becomes clear when we discretize the processes.} \cite{sarkka2019applied}
\begin{equation}
\label{eq: noising process}
        \dd \ac_t  = -\beta_t \ac_t \dd t + \eta\sqrt{2\beta_t} \dd B_t, \quad a_0 \sim \fpi_0(\cdot|\st),
\end{equation}
with diffusion coefficient $\beta: [0,T]\rightarrow \R^+$, standard Brownian motion $(B_t)_{t\in[0,T]}$, and some target policy $\fpi_0$. 
For $t,l\in [0,T]$, we denote the marginal density of Eq. $\ref{eq: noising process}$ at $t$ as $\fpi_t$
% , the joint density at $s,t$ $\fpi_{s,t}$ 
and the conditional density at time $t$ given $l$ as $\fpi_{t|l}$.
Eq. \ref{eq: noising process} is commonly referred to as \textit{forward} or \textit{noising process} since, for a suitable choice of $\beta$, it holds that $\fpi_{T} \approx \mathcal{N}(0, \eta^2I)$. Denoising diffusion models leverage the fact, that the time-reversed process of Eq. \ref{eq: noising process} is given by 
\begin{equation}
\label{eq: denoising process}
        \dd \ac_t  = \left(-\beta_t \ac_t \dd t - 2\eta^2\beta_t \nabla \log \fpi_t(\ac_t|\st)\right) + \eta\sqrt{2\beta_t} \dd B_t,
\end{equation}
starting from $\bpi_T = \fpi_{T} \approx \mathcal{N}(0, \eta^2I)$ and running backwards in time \cite{nelson2020dynamical,anderson1982reverse,haussmann1986time}. For the \textit{backward}, \textit{generative} or \textit{denoising process} (Eq. \ref{eq: denoising process}), we denote the density as $\bpi$. Here, time-reversal means that the marginal densities align, i.e., $\fpi_t = \bpi_t$ for all $t\in[0, T]$. Hence, starting from $\ac_T \sim \mathcal{N}(0, \eta^2I)$, one can sample from the target policy $\fpi_0$ by simulating Eq. \ref{eq: denoising process}. However, for most densities $\fpi_0$, the scores $\left(\nabla \log \fpi_t(\ac_t|\st)\right)_{t\in[0,T]}$ are intractable, requiring numerical approximations. To address this, denoising score-matching objectives are commonly employed, that is, 
\begin{equation}
\label{eq: score matching}
\mathcal{L}_{\text{SM}}(\theta) = \E\left[\beta_t\|f^{\theta}_t(\ac_t,\st) - \nabla \log \fpi_{t|0}(\ac_t|\ac_0,\st) \|^2\right],
\end{equation}
where $t$ is sampled on $[0,T]$ and $f^{\theta}$ denotes a parameterized score network \cite{hyvarinen2005estimation,vincent2011connection}. For OU processes, the conditional densities $\nabla \log \fpi_{t|0}$ are explicitly computable, making the objective tractable for optimizing $\theta$ \cite{song2021scorebased}. Once trained, the score network $f^{\theta}$ can be used to simulate the denoising process 
\begin{equation}
\label{eq: approximate denoising process}
        \dd \ac_t  = \left(-\beta_t \ac_t \dd t - 2\eta^2\beta_t f^{\theta}_t(\ac_t,\st)\right) + \eta\sqrt{2\beta_t} \dd B_t,
\end{equation}
to obtain samples $\ac_0 \sim \ppi_0$ that are approximately distributed according to $\fpi_0$. Here, $\ppi_t$ denotes the marginal distribution of Eq. \ref{eq: approximate denoising process} at time $t$.
While score-matching techniques work well in practice, they cannot be applied to maximum entropy reinforcement learning. 
This is because the expectation in Eq. \ref{eq: score matching} requires samples $\ac_0 \sim \fpi_0 \propto \exp Q^{\pi}$ which are not available. However, in the next section, we build on recent advances in approximate inference to optimize diffusion models without requiring samples from $\ac_0$, relying instead on evaluations of  $Q^{\pi}$.

%
%
%
\section{Diffusion-Based Maximum Entropy RL}
\label{sec: Diffusion-Based Maximum Entropy RL}
Here, we explain how diffusion models can be used within a maximum entropy RL framework. To that end, we express the maximum entropy objective as an approximate inference problem for diffusion models. We then use these results to introduce a policy iteration scheme that provably converges to the optimal policy. Lastly, we propose a practical algorithm for optimizing diffusion models.
\subsection{Control as Inference for Diffusion Policies}
Directly maximizing the maximum entropy objective
\begin{equation*}
\label{eq: marginal diffusion max ent}
    J(\bpi) = \sum_{t=l}^{\infty} \gamma^{t-l}\E_{\rho_\pi}\left[r_t(\st_t,\ac^0_t) + \alpha \Ent(\bpi_0(\ac^0_t|\st_t))\right], 
\end{equation*}
for a diffusion model is difficult as the marginal entropy $\Ent(\bpi_0(\ac|\st))$ of the denoising process in Eq. \ref{eq: denoising process} is intractable.
Please note that we use superscripts for the actions to indicate the diffusion step to avoid collisions with the time step used in RL. Moreover, we will again absorb $\alpha$ into the reward and use  $r_t\triangleq r(\st_t,\ac^0_t)$.
To overcome this intractability, we propose to maximize a lower bound. We start by discretizing the stochastic processes introduced in \Cref{sec: Denoising Diffusion Policies} and use the results as a foundation to derive this lower bound. Note that while similar results can be derived from a continuous-time perspective (see e.g., \citet{berneroptimal,richterimproved,nusken2024transport}), such derivation would require a background in stochastic calculus, making it less accessible to a broader audience. %We therefore stay with the simpler, discrete-time formulation. 

The Euler-Maruyama (EM) discretization \cite{sarkka2019applied} of the noising (Eq. \ref{eq: noising process}) and denoising (Eq. \ref{eq: denoising process}) process is given by
\begin{align}
\label{eq: em discretized noising process}
        \ac^{n+1}  & = \ac^{n} -\beta_{n} \ac^{n} \delta + \epsilon_{n} \quad \text{and}
        \\ 
\label{eq: em discretized denoising process}
        \ac^{n-1}  & = \ac^{n} + \left(\beta_{n} \ac^{n} + 2\eta^2\beta_{n} \nabla \log \fpi_n(\ac^{n}|\st) \right)\delta + \xi_{n}, 
\end{align}
respectively, with $\epsilon_n,\xi_n \sim \mathcal{N}(0,2\eta^2\beta_{n}\delta I)$. Here, $\delta$ denotes a constant discretization step size such that $N = T / \delta$ is an integer. To simplify notation, we write $\ac^n$, instead of $\ac^{n\delta}$.  Under the EM discretization, the noising and denoising process admit the following joint distributions
\begin{align}
    \fpi_{0:N}(\ac^{0:N}|\st) &= \fpi_0(\ac^0|s) \prod_{n=0}^{N-1}\fpi_{n+1|n}(\ac^{n+1} \big| \ac^{n},\st), \label{eq: forward joint}\\
    \bpi_{0:N}(\ac^{0:N}|\st) &= \bpi_N (\ac^N|s) \prod_{n=1}^{N}\bpi_{n-1|n}(\ac^{n-1} \big| \ac^{n},\st), \label{eq: parameterized joint}
\end{align}
in a sense that $\fpi_{0:N}$ and $\bpi_{0:N}$ converge to the law of $(\ac_t)_{t\in[0,T]}$ in Eq. \ref{eq: noising process} and \ref{eq: denoising process}, as $\delta \rightarrow 0$, respectively \cite{doucet2022score}. Here, $\fpi_{n+1|n}$ and $\bpi_{n-1|n}$ are Gaussian transition densities that directly follow from Eq. \ref{eq: em discretized noising process} and \ref{eq: em discretized denoising process}.

To obtain a maximum entropy objective for diffusion models, we make use of the following lower bound on the marginal entropy, that is, $\Ent(\bpi_0(\ac_0|\st)) \geq \blb(\ac^{0},\st)$, where
\begin{equation}
\label{eq: entropy lower bound}
    \blb(\ac^{0},\st) =  \E_{\bpi_{0:N}}\left[\log \frac{\fpi_{1:N|0}(\ac^{1:N}|\ac^0,\st)}{\bpi_{0:N}(\ac^{0:N}|\st)}\right].
\end{equation}
Please note that similar bounds have been used, e.g., in \cite{agakov2004auxiliary,tran2015variational,ranganath2016hierarchical,maaloe2016auxiliary,arenz2018efficient}, or, more generally, follow from the data processing inequality \cite{cover1999elements}. 
A derivation can be found in Appendix \ref{APDX:DERIVATIONS}. From Eq. \ref{eq: entropy lower bound}, it directly follows that 
\begin{equation}
\label{eq: marginal max ent}
    J(\bpi) \geq \bar{J}(\bpi) = \sum_{t=l}^{\infty} \gamma^{t-l}\E_{\rho_\pi}\left[r_t + \blb(\ac^{0}_t,\st_t)\right].
\end{equation}
Next, we cast Eq. \ref{eq: marginal max ent} as an approximate inference problem to make the objective more interpretable. To that end, let us define the $Q$-function of a denoising policy $\bpi$ with respect to the maximum entropy objective $\bar J$ as
\begin{equation}
\label{eq: diffusion Q soft}
Q^{\bpi}(\st_t,\ac^0_t) = r_t + \sum_{l=1}\gamma^l \E_{\rho_{\pi}}\left[r_{t+l}+ \blb(\ac^{0}_{t+l},\st_{t+l})\right],
\end{equation}
with $Q^{\bpi}: \St \times \Ac \rightarrow \R$. With Eq. \ref{eq: diffusion Q soft} we identify the corresponding approximate inference problem as finding $\bpi$ which minimizes (please see Appendix \ref{APDX:DERIVATIONS} for derivation)
\begin{equation}
\label{eq: joint control as inference}
\bar{\mathcal{L}}(\bpi) = D_{\text{KL}}\left(\bpi_{0:N}(\ac^{0:N}|\st)|\fpi_{0:N}(\ac^{0:N}|\st)\right),
\end{equation}
where the target policy, i.e., the marginal of the noising process in Eq. \ref{eq: forward joint} is given by the exponentiated $Q$-function of the diffusion policy
\begin{equation}
    \fpi_{0}(\ac^{0}|\st) = \frac{\exp Q^{\bpi}(\st, \ac^0)}{\Z^{\bpi}(\st)}.
\end{equation}
Recall from \Cref{sec: Denoising Diffusion Policies} that we aim to time-reverse the noising process, that is, to ensure for all states $\st \in \St$, it holds that $\bpi_{0:N} = \fpi_{0:N}$. Please note that this is precisely what Eq. \ref{eq: joint control as inference} is trying to accomplish, i.e., we aim to learn a diffusion model $\bpi$, such that the denoising process time-reverses the noising process, and, in particular, has a marginal distribution given by $\pi_{0} = \exp Q^{\bpi}/\Z^{\bpi}$. Lastly, 
from the data processing inequality it directly follows that 
\begin{align}
\label{eq: data processing inequality}
  D_{\text{KL}}\bigg(\bpi_0(\ac^0|\st)&\Big|\frac{\exp Q^{\bpi}(\st,\ac^0)}{\Z^{\bpi}(\st)}\bigg) \nonumber
 \\
 & \leq D_{\text{KL}}\left(\bpi(\ac^{0:N}|\st)|\fpi(\ac^{0:N}|\st)\right),
\end{align}
which shows the approximate inference problem in Eq. \ref{eq: joint control as inference} indeed optimizes the same inference problem stated in Eq. \ref{eq: marginal control as inference}.
Next, we will use these results to develop a policy iteration scheme for diffusion models. 

\subsection{Diffusion-based Policy Iteration}
We propose a policy iteration scheme for learning an optimal maximum entropy policy, similar to \cite{haarnoja2018soft}. However, here we restrict the family of stochastic actors to diffusion policies $\bpi \in \cev{\Pi} \subset \Pi$. Throughout this section, we assume finite action spaces to enable theoretical analysis, but relax this assumption in \cref{dime_practical}. All proofs of this section are deferred to Appendix \ref{APDX:DERIVATIONS}. 
\todo[inline]{Do we need to assume finite action spaces? $\bpi$ is still assumed to be optimal. Should be relax the assumption here? FINITE ACTION SPACE}

For policy evaluation, we aim to compute the value of a policy $\bpi$. We define the Bellman backup operator as
\begin{equation}
    \label{eq: diffusion bellman operator}
    \mathcal{T}^{\bpi} Q(\st_t,\ac^0_t) \triangleq r_t + \gamma \E\left[Q(\st_{t+1},\ac^0_{t+1}) +\blb(\ac^{0}_{t+1},\st_{t+1})\right].
\end{equation}
Note that Eq. \ref{eq: diffusion bellman operator} contains the entropy-lower bound $\blb$. By applying standard convergence results for policy evaluation \cite{sutton1999reinforcement} we can obtain the value of a policy by repeatedly applying $\mathcal{T}^{\bpi}$ as established in \Cref{prop: policy evaluation}.

\begin{proposition}[Policy Evaluation]
\label{prop: policy evaluation}
Let $\mathcal{T}^{\bpi}$ be the Bellman backup operator for a diffusion policy $\bpi$ as defined in Eq. \ref{eq: diffusion bellman operator}. Further, let $Q^0: \St \times \Ac \rightarrow \R$ and $Q^{k+1} = \mathcal{T}^{\bpi}Q^k$.
Then, it holds that $\lim_{k\rightarrow \infty} Q^k =  Q^{\bpi}$ where $Q^{\bpi}$ is the $Q$ value of $\bpi$.
\end{proposition}

For the policy improvement step, we seek to improve the current policy based on its value using the $Q$-function. Formally, we need to solve the approximate inference problem
\begin{equation}
\label{eq: policy improvement objective}
\bpi^{\text{new}} = \argmin_{\bpi \in \cev{\Pi}}D_{\text{KL}}\left(\bpi_{0:N}(\ac^{0:N}|\st)|\fpi^{\text{ old}}_{0:N}(\ac^{0:N}|\st)\right),
\end{equation}
for all $\st\in\St$, where $\fpi^{\text{ old}}_{0:N}(\ac^{0:N}|\st)$ is as in Eq. \ref{eq: forward joint} with marginal density
\begin{equation}
\label{eq: old policy}
    \fpi^{\text{ old}}_{0}(\ac^{0}|\st) = \frac{\exp Q^{\bpi_{\text{old}}}(\st, \ac^0)}{\Z^{\bpi_{\text{old}}}(\st)}.
\end{equation}
Indeed, solving Eq. \ref{eq: policy improvement objective} results in a policy with higher value as established below.
\begin{proposition}[Policy Improvement]
\label{prop: policy improvement}
Let $\bpi_{\text{old}}, \bpi_{\text{new}} \in \cev{\Pi}$ be defined as in Eq. \ref{eq: old policy} and \ref{eq: policy improvement objective}, respectively. Then for all $(\st,\ac) \in \St \times \Ac$ it holds that $Q^{\bpi_{\text{new}}}(\st,\ac) \geq Q^{\bpi_{\text{old}}}(\st,\ac).$ 
\end{proposition}

Combining these results leads to the policy iteration method which alternates between policy evaluation (\Cref{prop: policy evaluation}) and policy improvement (\Cref{prop: policy improvement}) and provably converges to the optimal policy in $\cev{\Pi}$ (\Cref{prop: policy iteration}).

\begin{proposition}[Policy Iteration]
\label{prop: policy iteration}
Let $\bpi^0, \bpi^{i+1}, \bpi^i, \bpi_* \in \cev{\Pi}$. Further, let $\bpi^{i+1}$ be the policy obtained from $\bpi^{i}$ after a policy evaluation and improvement step. Then, for any starting policy $\bpi^0$ it holds that $\lim_{i\rightarrow \infty} \bpi^i =  \bpi^*$, with $\bpi^*$ such that for all $\bpi\in \cev{\Pi}$ and $(\st,\ac) \in \St \times \Ac$ it holds that $Q^{\bpi^*}(\st,\ac) \geq Q^{\bpi}(\st,\ac).$ 
\end{proposition}

However, performing policy iteration until convergence is in practice often intractable, particularly for continuous control tasks. As such, we will introduce a practical algorithm next.


\begin{figure*}[t!]
    \centering
    \resizebox{0.95\textwidth}{!}{
    \definecolor{crimson2143940}{RGB}{214,39,40}
\definecolor{darkorange25512714}{RGB}{255,127,14}
\definecolor{forestgreen4416044}{RGB}{44,160,44}
\definecolor{mediumpurple148103189}{RGB}{148,103,189}
\definecolor{steelblue31119180}{RGB}{31,119,180}
\definecolor{darkgray176}{RGB}{176,176,176}
\begin{tikzpicture} 
    \begin{axis}[%
    hide axis,
    xmin=10,
    xmax=50,
    ymin=0,
    ymax=0.4,
    legend style={
        draw=white!15!black,
        legend cell align=left,
        legend columns=-1, 
        legend style={
            draw=none,
            column sep=1ex,
            line width=0.5pt
        }
    },
    ]
    \addlegendimage{line width=2pt, color=C6}
    \addlegendentry{$10^{-01}$};
    \addlegendimage{line width=2pt, color=C0}
    \addlegendentry{$10^{-02}$};
    \addlegendimage{line width=2pt, color=C1}
    \addlegendentry{$10^{-03}$};
    \addlegendimage{line width=2pt, color=C3}
    \addlegendentry{$10^{-04}$};
    \addlegendimage{line width=2pt, color=C4}
    \addlegendentry{$10^{-05}$};
    \addlegendimage{line width=2pt, color=C5}
    \addlegendentry{$10^{-08}$};
    \addlegendimage{line width=2pt, color=C8}
    \addlegendentry{$10^{-09}$};
    \addlegendimage{line width=2pt, color=C7}
    \addlegendentry{$10^{-12}$};
    \end{axis}
\end{tikzpicture}}%
    
    \begin{minipage}[b]{0.26\textwidth}
        \centering
       \resizebox{1\textwidth}{!}{% This file was created with tikzplotlib v0.10.1.
\begin{tikzpicture}

\definecolor{darkcyan1115178}{RGB}{1,115,178}
\definecolor{darkgray176}{RGB}{176,176,176}

\begin{axis}[
legend cell align={left},
legend cell align={left},
legend style={fill opacity=0.8, draw opacity=1, text opacity=1, draw=lightgray204, at={(0.5,0.03)},  anchor=south west},
tick align=outside,
tick pos=left,
x grid style={white},
xlabel={Number Env Interactions},
xmajorgrids,
%xmin=-149398.95, xmax=3137399.95,
xmin=0.0, xmax=3000000.00,
xtick style={color=black},
y grid style={white},
ylabel={IQM Return},
ymajorgrids,
ymin=0.0, ymax=901.400008641667,
ytick style={color=black},
axis background/.style={fill=plot_background},
label style={font=\large},
tick label style={font=\large},
x axis line style={draw=none},
y axis line style={draw=none},
]
\path [draw=C3, fill=C3, opacity=0.2]
(axis cs:1,5.89064738333333)
--(axis cs:1,4.1291282)
--(axis cs:36000,4.36958118333333)
--(axis cs:72000,21.9102696666667)
--(axis cs:108000,44.5369561666667)
--(axis cs:144000,73.8133631666667)
--(axis cs:180000,92.6374976666667)
--(axis cs:216000,103.433920833333)
--(axis cs:252000,122.695540833333)
--(axis cs:288000,144.780881666667)
--(axis cs:324000,141.113765)
--(axis cs:360000,163.508661208333)
--(axis cs:396000,155.823601916667)
--(axis cs:432000,201.528308333333)
--(axis cs:468000,213.320525)
--(axis cs:504000,224.301218333333)
--(axis cs:540000,237.119743333333)
--(axis cs:576000,246.070941666667)
--(axis cs:612000,267.884333333333)
--(axis cs:648000,275.265338333333)
--(axis cs:684000,291.610186666667)
--(axis cs:720000,306.509876666667)
--(axis cs:756000,314.089796666667)
--(axis cs:792000,328.317606666667)
--(axis cs:828000,333.810906666667)
--(axis cs:864000,338.667664458333)
--(axis cs:900000,343.544436666667)
--(axis cs:936000,361.921333333333)
--(axis cs:972000,372.899663333333)
--(axis cs:1008000,380.099353333333)
--(axis cs:1044000,403.222876666667)
--(axis cs:1080000,387.989915)
--(axis cs:1116000,417.562756666667)
--(axis cs:1152000,407.567365)
--(axis cs:1188000,430.534505)
--(axis cs:1224000,431.1114)
--(axis cs:1260000,434.56934)
--(axis cs:1296000,444.288155)
--(axis cs:1332000,476.369876666667)
--(axis cs:1368000,461.040406666667)
--(axis cs:1404000,469.349473333333)
--(axis cs:1440000,488.009956666667)
--(axis cs:1476000,488.694688333333)
--(axis cs:1512000,511.883405041667)
--(axis cs:1548000,563.67264)
--(axis cs:1584000,545.585495208333)
--(axis cs:1620000,560.501143333333)
--(axis cs:1656000,535.712238333333)
--(axis cs:1692000,501.032973333333)
--(axis cs:1728000,588.495035)
--(axis cs:1764000,534.116686666667)
--(axis cs:1800000,567.87324)
--(axis cs:1836000,597.2437)
--(axis cs:1872000,527.918356666667)
--(axis cs:1908000,638.176695)
--(axis cs:1944000,630.920571208333)
--(axis cs:1980000,640.987561666667)
--(axis cs:2016000,653.72166)
--(axis cs:2052000,679.265283333333)
--(axis cs:2088000,632.715323333333)
--(axis cs:2124000,681.83093)
--(axis cs:2160000,679.575463333333)
--(axis cs:2196000,722.347776666667)
--(axis cs:2232000,667.204936666667)
--(axis cs:2268000,685.014776666667)
--(axis cs:2304000,704.751333333333)
--(axis cs:2340000,704.640791666667)
--(axis cs:2376000,719.756748333333)
--(axis cs:2412000,744.714488333333)
--(axis cs:2448000,731.197275)
--(axis cs:2484000,720.243575)
--(axis cs:2520000,669.276773333333)
--(axis cs:2556000,754.734663333333)
--(axis cs:2592000,739.97988475)
--(axis cs:2628000,728.446603333333)
--(axis cs:2664000,766.406533333333)
--(axis cs:2700000,750.326138333333)
--(axis cs:2736000,734.063323333333)
--(axis cs:2772000,772.572516666667)
--(axis cs:2808000,728.854876666667)
--(axis cs:2844000,764.823895)
--(axis cs:2880000,784.974858333333)
--(axis cs:2916000,764.92251)
--(axis cs:2952000,781.2523575)
--(axis cs:2988000,772.058598333333)
--(axis cs:2988000,858.591428333333)
--(axis cs:2988000,858.591428333333)
--(axis cs:2952000,853.033060333334)
--(axis cs:2916000,843.288837)
--(axis cs:2880000,867.902476666667)
--(axis cs:2844000,824.153933333333)
--(axis cs:2808000,864.357281666667)
--(axis cs:2772000,845.727528333333)
--(axis cs:2736000,876.22724)
--(axis cs:2700000,861.385121666667)
--(axis cs:2664000,843.982653333333)
--(axis cs:2628000,850.563796666667)
--(axis cs:2592000,834.694483333333)
--(axis cs:2556000,864.064105)
--(axis cs:2520000,822.63462)
--(axis cs:2484000,838.08071)
--(axis cs:2448000,837.947975)
--(axis cs:2412000,837.841368333333)
--(axis cs:2376000,807.278331666667)
--(axis cs:2340000,822.403533333333)
--(axis cs:2304000,832.89487)
--(axis cs:2268000,830.849501666667)
--(axis cs:2232000,837.47201)
--(axis cs:2196000,831.17095)
--(axis cs:2160000,825.700883333333)
--(axis cs:2124000,828.526442916667)
--(axis cs:2088000,770.616286666667)
--(axis cs:2052000,807.72665)
--(axis cs:2016000,751.081956666667)
--(axis cs:1980000,782.27038)
--(axis cs:1944000,781.074635)
--(axis cs:1908000,775.310213333333)
--(axis cs:1872000,764.705915)
--(axis cs:1836000,786.169281666667)
--(axis cs:1800000,774.01652)
--(axis cs:1764000,728.940376666667)
--(axis cs:1728000,761.033158333333)
--(axis cs:1692000,770.318991666667)
--(axis cs:1656000,735.083688333333)
--(axis cs:1620000,733.302023333333)
--(axis cs:1584000,737.523791666667)
--(axis cs:1548000,723.791843333333)
--(axis cs:1512000,689.69503)
--(axis cs:1476000,703.447666666667)
--(axis cs:1440000,703.884616666667)
--(axis cs:1404000,689.787128333333)
--(axis cs:1368000,663.751797083334)
--(axis cs:1332000,675.50549)
--(axis cs:1296000,676.303538333333)
--(axis cs:1260000,650.188453333333)
--(axis cs:1224000,589.741586083333)
--(axis cs:1188000,615.98814)
--(axis cs:1152000,582.320133333333)
--(axis cs:1116000,596.750455)
--(axis cs:1080000,577.708458333333)
--(axis cs:1044000,553.53348)
--(axis cs:1008000,558.394433333333)
--(axis cs:972000,528.143061666667)
--(axis cs:936000,510.096248333333)
--(axis cs:900000,495.766748333333)
--(axis cs:864000,458.83707)
--(axis cs:828000,442.794613333333)
--(axis cs:792000,429.712663333333)
--(axis cs:756000,392.22884)
--(axis cs:720000,408.753066666667)
--(axis cs:684000,370.69758)
--(axis cs:648000,363.800676666667)
--(axis cs:612000,335.425881666667)
--(axis cs:576000,315.10437775)
--(axis cs:540000,302.417836666667)
--(axis cs:504000,287.786676666667)
--(axis cs:468000,256.548557125)
--(axis cs:432000,253.588628333333)
--(axis cs:396000,225.467696666667)
--(axis cs:360000,206.161806666667)
--(axis cs:324000,181.996731666667)
--(axis cs:288000,175.301235)
--(axis cs:252000,160.583735)
--(axis cs:216000,143.503271666667)
--(axis cs:180000,121.901736666667)
--(axis cs:144000,97.5643175)
--(axis cs:108000,79.6164608333333)
--(axis cs:72000,39.3428316666667)
--(axis cs:36000,11.0066606166667)
--(axis cs:1,5.89064738333333)
--cycle;

\addplot [line width=\linewidthother, C3, mark=*, mark size=0, mark options={solid}]
table {%
1 4.8701071
36000 6.72384941666667
72000 29.3581018333333
108000 63.3330631666667
144000 86.7607035
180000 110.848872666667
216000 125.25725
252000 144.375887333333
288000 159.330578333333
324000 161.599885
360000 188.998286666667
396000 203.654086666667
432000 232.308528333333
468000 240.16393
504000 257.163325
540000 270.039501666667
576000 273.186586666667
612000 304.593976666667
648000 318.949601666667
684000 328.977761666667
720000 359.723766666667
756000 362.648261666667
792000 386.788338333333
828000 390.710991666667
864000 405.630171666667
900000 415.033181666667
936000 434.406411666667
972000 452.752266666667
1008000 470.17097
1044000 484.749701666667
1080000 482.753605
1116000 511.220326666667
1152000 506.144361666667
1188000 527.276881666667
1224000 512.964038333333
1260000 544.194691666667
1296000 563.19015
1332000 578.308555
1368000 566.009768333333
1404000 591.790591666667
1440000 590.803741666667
1476000 607.842233333333
1512000 605.637781666667
1548000 636.038028333333
1584000 643.470818333333
1620000 634.836016666667
1656000 624.311076666667
1692000 629.079153333333
1728000 673.619896666667
1764000 605.227423333333
1800000 665.612198333333
1836000 698.479258333333
1872000 658.705781666667
1908000 700.169148333333
1944000 705.618373333333
1980000 719.874233333333
2016000 711.819425
2052000 748.167208333333
2088000 720.219761666667
2124000 755.752271666667
2160000 765.831068333333
2196000 787.752566666667
2232000 759.990015
2268000 762.137346666667
2304000 780.45201
2340000 774.628483333333
2376000 758.721198333333
2412000 802.726498333333
2448000 796.593808333333
2484000 795.960206666667
2520000 746.542065
2556000 832.110631666667
2592000 787.537713333333
2628000 803.720935
2664000 810.365966666667
2700000 829.490688333333
2736000 815.02574
2772000 809.62794
2808000 814.520655
2844000 794.265555
2880000 828.936341666667
2916000 794.882741666667
2952000 819.517033333333
2988000 807.66068
};
\addlegendentry{Gaussian Policy}
\path [draw=C0, fill=C0, opacity=0.2]
(axis cs:1,6.7908015)
--(axis cs:1,5.6020231)
--(axis cs:36000,8.62983233333333)
--(axis cs:72000,39.178405)
--(axis cs:108000,58.7256123333333)
--(axis cs:144000,70.247577)
--(axis cs:180000,85.990517)
--(axis cs:216000,132.272293333333)
--(axis cs:252000,149.256846333333)
--(axis cs:288000,171.63141)
--(axis cs:324000,192.776848333333)
--(axis cs:360000,215.394498333333)
--(axis cs:396000,245.356148333333)
--(axis cs:432000,256.659486666667)
--(axis cs:468000,271.564893333333)
--(axis cs:504000,309.006485)
--(axis cs:540000,307.207731666667)
--(axis cs:576000,357.640175)
--(axis cs:612000,389.479391666667)
--(axis cs:648000,391.45075)
--(axis cs:684000,422.799776666667)
--(axis cs:720000,458.922981666667)
--(axis cs:756000,474.32947)
--(axis cs:792000,483.093959041667)
--(axis cs:828000,506.47542)
--(axis cs:864000,527.350673333333)
--(axis cs:900000,544.75695)
--(axis cs:936000,560.401128333333)
--(axis cs:972000,583.388941666667)
--(axis cs:1008000,589.799863333333)
--(axis cs:1044000,589.046265)
--(axis cs:1080000,607.424441666667)
--(axis cs:1116000,607.498523333333)
--(axis cs:1152000,638.028483333333)
--(axis cs:1188000,601.18704)
--(axis cs:1224000,659.686861666667)
--(axis cs:1260000,667.927208708333)
--(axis cs:1296000,660.504675)
--(axis cs:1332000,676.24953)
--(axis cs:1368000,665.97862)
--(axis cs:1404000,687.60902125)
--(axis cs:1440000,700.087826666667)
--(axis cs:1476000,671.29)
--(axis cs:1512000,690.782141666667)
--(axis cs:1548000,695.240922208333)
--(axis cs:1584000,677.796726666667)
--(axis cs:1620000,720.659616666667)
--(axis cs:1656000,709.123516666667)
--(axis cs:1692000,717.853556666667)
--(axis cs:1728000,732.13092)
--(axis cs:1764000,662.37287075)
--(axis cs:1800000,708.438711666667)
--(axis cs:1836000,748.553096666667)
--(axis cs:1872000,731.0669)
--(axis cs:1908000,730.051075)
--(axis cs:1944000,769.617006666667)
--(axis cs:1980000,759.68223)
--(axis cs:2016000,756.386703333333)
--(axis cs:2052000,788.513618333333)
--(axis cs:2088000,725.240085708333)
--(axis cs:2124000,763.8678)
--(axis cs:2160000,786.34295)
--(axis cs:2196000,721.03074)
--(axis cs:2232000,787.1876)
--(axis cs:2268000,768.494865)
--(axis cs:2304000,768.75273)
--(axis cs:2340000,772.603808333333)
--(axis cs:2376000,763.798678333333)
--(axis cs:2412000,815.450611666667)
--(axis cs:2448000,797.30472)
--(axis cs:2484000,796.082483333333)
--(axis cs:2520000,789.613171666667)
--(axis cs:2556000,803.113718333333)
--(axis cs:2592000,776.112966666667)
--(axis cs:2628000,800.927163333333)
--(axis cs:2664000,762.744901666667)
--(axis cs:2700000,789.174581666667)
--(axis cs:2736000,798.797758625)
--(axis cs:2772000,808.763553333333)
--(axis cs:2808000,791.100212166667)
--(axis cs:2844000,797.507851666667)
--(axis cs:2880000,756.81125)
--(axis cs:2916000,802.708426666667)
--(axis cs:2952000,778.739201666667)
--(axis cs:2988000,806.379209958333)
--(axis cs:2988000,856.395113333333)
--(axis cs:2988000,856.395113333333)
--(axis cs:2952000,849.116565)
--(axis cs:2916000,859.785973333333)
--(axis cs:2880000,849.40738)
--(axis cs:2844000,864.92043)
--(axis cs:2808000,863.616378333333)
--(axis cs:2772000,851.90444)
--(axis cs:2736000,852.977273333333)
--(axis cs:2700000,849.869583333333)
--(axis cs:2664000,868.884923333333)
--(axis cs:2628000,845.164255)
--(axis cs:2592000,863.82811)
--(axis cs:2556000,851.953161666667)
--(axis cs:2520000,848.667223333333)
--(axis cs:2484000,842.26205)
--(axis cs:2448000,844.661853333333)
--(axis cs:2412000,835.335961666667)
--(axis cs:2376000,851.43932)
--(axis cs:2340000,858.032386666667)
--(axis cs:2304000,823.622453333333)
--(axis cs:2268000,842.820591666667)
--(axis cs:2232000,833.971345)
--(axis cs:2196000,825.505923333333)
--(axis cs:2160000,843.710091666666)
--(axis cs:2124000,841.442543333333)
--(axis cs:2088000,832.747906666667)
--(axis cs:2052000,832.985603333333)
--(axis cs:2016000,831.623071666667)
--(axis cs:1980000,834.6822)
--(axis cs:1944000,821.178215)
--(axis cs:1908000,858.758488333333)
--(axis cs:1872000,831.917276666667)
--(axis cs:1836000,816.230947916667)
--(axis cs:1800000,841.425425)
--(axis cs:1764000,803.220675)
--(axis cs:1728000,837.246975)
--(axis cs:1692000,825.25186)
--(axis cs:1656000,808.263138333333)
--(axis cs:1620000,815.792903541667)
--(axis cs:1584000,808.863161666667)
--(axis cs:1548000,805.28377)
--(axis cs:1512000,782.734225)
--(axis cs:1476000,788.93601)
--(axis cs:1440000,821.10777)
--(axis cs:1404000,792.100268333333)
--(axis cs:1368000,760.665633333333)
--(axis cs:1332000,725.122591666667)
--(axis cs:1296000,745.192578333333)
--(axis cs:1260000,735.194826666667)
--(axis cs:1224000,752.493893333333)
--(axis cs:1188000,702.905228583333)
--(axis cs:1152000,730.103061666667)
--(axis cs:1116000,717.392206666667)
--(axis cs:1080000,710.234308333333)
--(axis cs:1044000,674.616426666667)
--(axis cs:1008000,694.470678333333)
--(axis cs:972000,674.61414)
--(axis cs:936000,634.1209)
--(axis cs:900000,632.24315)
--(axis cs:864000,638.30593)
--(axis cs:828000,603.69465)
--(axis cs:792000,604.798345)
--(axis cs:756000,598.060326583333)
--(axis cs:720000,568.233463333333)
--(axis cs:684000,568.143085)
--(axis cs:648000,536.52782)
--(axis cs:612000,518.389283333333)
--(axis cs:576000,512.089843333333)
--(axis cs:540000,448.484928333333)
--(axis cs:504000,461.99631)
--(axis cs:468000,408.115491666667)
--(axis cs:432000,392.529571666667)
--(axis cs:396000,384.947523333333)
--(axis cs:360000,330.958086541667)
--(axis cs:324000,307.664625)
--(axis cs:288000,251.54248)
--(axis cs:252000,236.02058)
--(axis cs:216000,194.22897)
--(axis cs:180000,159.045319333333)
--(axis cs:144000,126.739716083333)
--(axis cs:108000,94.057901)
--(axis cs:72000,62.1926506666667)
--(axis cs:36000,16.8357318333333)
--(axis cs:1,6.7908015)
--cycle;

\addplot [line width=\linewidthdime, C0, mark=*, mark size=0, mark options={solid}]
table {%
1 6.17641296666667
36000 11.6270238333333
72000 51.2078703333333
108000 80.0899
144000 110.007961666667
180000 124.469762666667
216000 164.83243
252000 183.239915
288000 202.72266
324000 244.796003333333
360000 270.949796666667
396000 311.212078333333
432000 321.852778333333
468000 339.38604
504000 387.914375
540000 377.199451666667
576000 435.778751666667
612000 454.769993333333
648000 468.345228333333
684000 494.018921666667
720000 507.547415
756000 540.829465
792000 547.320155
828000 573.053538333333
864000 586.99087
900000 587.611123333333
936000 603.980191666667
972000 631.953166666667
1008000 652.547345
1044000 630.36095
1080000 677.789983333333
1116000 674.022491666667
1152000 696.043668333333
1188000 652.41135
1224000 716.223413333333
1260000 705.611823333333
1296000 711.299655
1332000 713.511105
1368000 718.17746
1404000 743.029048333333
1440000 767.977446666667
1476000 732.657913333333
1512000 740.075188333334
1548000 755.125068333333
1584000 745.76982
1620000 774.640848333333
1656000 775.007413333333
1692000 786.750093333333
1728000 795.057825
1764000 751.749063333333
1800000 785.936768333333
1836000 798.028601666667
1872000 784.5074
1908000 803.70621
1944000 802.98721
1980000 814.507853333333
2016000 809.862228333333
2052000 820.409161666667
2088000 792.892843333333
2124000 814.52938
2160000 829.289918333333
2196000 784.440623333333
2232000 810.922945
2268000 812.483123333333
2304000 787.9255
2340000 827.978133333333
2376000 809.844736666667
2412000 830.52828
2448000 827.945956666667
2484000 819.91705
2520000 821.356241666667
2556000 831.340293333333
2592000 838.71875
2628000 832.144581666667
2664000 842.476748333333
2700000 814.85007
2736000 828.082728333333
2772000 838.083201666667
2808000 840.753683333333
2844000 837.937456666667
2880000 812.837433333333
2916000 829.931661666667
2952000 813.313951666667
2988000 833.048301666667
};
\addlegendentry{DIME}
\end{axis}

\end{tikzpicture}
}
       \subcaption[]{}
       \label{fig::exps_new_ablations_vary_alpha::dog_run}
    \end{minipage}\hfill
    \begin{minipage}[b]{0.25\textwidth}
        \centering
       \resizebox{0.875\textwidth}{!}{% This file was created with tikzplotlib v0.10.1.
\begin{tikzpicture}

\definecolor{chocolate213940}{RGB}{213,94,0}
\definecolor{darkcyan1115178}{RGB}{1,115,178}
\definecolor{darkcyan2158115}{RGB}{2,158,115}
\definecolor{darkgray176}{RGB}{176,176,176}
\definecolor{darkorange2221435}{RGB}{222,143,5}
\definecolor{orchid204120188}{RGB}{204,120,188}
\definecolor{peru20214597}{RGB}{202,145,97}
\definecolor{pink251175228}{RGB}{251,175,228}

\definecolor{crimson2143940}{RGB}{214,39,40}
\definecolor{darkorange25512714}{RGB}{255,127,14}
\definecolor{forestgreen4416044}{RGB}{44,160,44}
\definecolor{mediumpurple148103189}{RGB}{148,103,189}
\definecolor{steelblue31119180}{RGB}{31,119,180}
\definecolor{darkgray176}{RGB}{176,176,176}

\begin{axis}[
legend cell align={left},
legend style={fill opacity=0.8, draw opacity=1, text opacity=1, draw=lightgray204, at={(0.03,0.03)},  anchor=north west},
tick align=outside,
tick pos=left,
x grid style={white},
xlabel={IQM Return},
xmajorgrids,
xmin=363.309961666667, xmax=845.547968791667,
xtick style={color=black},
y grid style={white},
ymin=-0.63, ymax=6.63,
ytick style={color=black},
% ytick={0,1,2,3,4,5,6},
% yticklabels={$10^{-02}$,$10^{-03}$,$10^{-04}$,$10^{-05}$,$10^{-08}$,$10^{-09}$,$10^{-12}$},
%ytick=\empty,
yticklabels={},
ylabel={$\alpha$ Values},
axis background/.style={fill=plot_background},
label style={font=\large},
tick label style={font=\large},
x axis line style={draw=none},
y axis line style={draw=none},
]
\draw[draw=none,fill=C0,fill opacity=0.75] (axis cs:702.50849,-0.3) rectangle (axis cs:750.875461666667,0.3);
%\addlegendimage{ybar,ybar legend,draw=none,fill=crimson2143940,fill opacity=0.75}
%\addlegendentry{1e-2}

\draw[draw=none,fill=C1,fill opacity=0.75] (axis cs:756.833583333333,0.7) rectangle (axis cs:822.584254166667,1.3);
%\addlegendimage{ybar,ybar legend,draw=none,fill=darkorange2221435,fill opacity=0.75}
%\addlegendentry{1e-3}

\draw[draw=none,fill=C3,fill opacity=0.75] (axis cs:653.580964166667,1.7) rectangle (axis cs:798.360316520834,2.3);
%\addlegendimage{ybar,ybar legend,draw=none,fill=forestgreen4416044,fill opacity=0.75}
%\addlegendentry{1e-4}

\draw[draw=none,fill=C4,fill opacity=0.75] (axis cs:562.97792,2.7) rectangle (axis cs:699.922009166667,3.3);
%\addlegendimage{ybar,ybar legend,draw=none,fill=violet,fill opacity=0.75}
%\addlegendentry{1e-5}

\draw[draw=none,fill=C5,fill opacity=0.75] (axis cs:460.2824125,3.7) rectangle (axis cs:699.551603333333,4.3);
%\addlegendimage{ybar,ybar legend,draw=none,fill=mediumpurple148103189,fill opacity=0.75}
%\addlegendentry{1e-8}

\draw[draw=none,fill=C8,fill opacity=0.75] (axis cs:389.501995,4.7) rectangle (axis cs:523.519544166667,5.3);
%\addlegendimage{ybar,ybar legend,draw=none,fill=teal,fill opacity=0.75}
%\addlegendentry{1e-9}

\draw[draw=none,fill=C7,fill opacity=0.75] (axis cs:363.309961666667,5.7) rectangle (axis cs:508.2595675,6.3);
%\addlegendimage{ybar,ybar legend,draw=none,fill=steelblue31119180,fill opacity=0.75}
%\addlegendentry{1e-12}

\path [draw=black, draw opacity=0.5, very thick]
(axis cs:728.768623333333,-0.28125)
--(axis cs:728.768623333333,0.225);

\path [draw=black, draw opacity=0.5, very thick]
(axis cs:792.855715833333,0.71875)
--(axis cs:792.855715833333,1.225);

\path [draw=black, draw opacity=0.5, very thick]
(axis cs:744.586875,1.71875)
--(axis cs:744.586875,2.225);

\path [draw=black, draw opacity=0.5, very thick]
(axis cs:621.846805,2.71875)
--(axis cs:621.846805,3.225);

\path [draw=black, draw opacity=0.5, very thick]
(axis cs:568.7912225,3.71875)
--(axis cs:568.7912225,4.225);

\path [draw=black, draw opacity=0.5, very thick]
(axis cs:468.99135,4.71875)
--(axis cs:468.99135,5.225);

\path [draw=black, draw opacity=0.5, very thick]
(axis cs:448.295465833333,5.71875)
--(axis cs:448.295465833333,6.225);

\end{axis}

% \draw ({$(current bounding box.south west)!0.4!(current bounding box.south east)$}|-{$(current bounding box.south west)!-0.1!(current bounding box.north west)$}) node[
%   scale=0.864,
%   anchor=base,
%   text=black,
%   rotate=0.0
% ]{};
\end{tikzpicture}
}
       \subcaption[]{}
       \label{fig::exps_new_ablations_vary_alpha::dog_run_pareto}
    \end{minipage}\hfill
    \begin{minipage}[b]{0.24\textwidth}
        \centering
       \resizebox{1\textwidth}{!}{% This file was created with tikzplotlib v0.10.1.
\begin{tikzpicture}

\definecolor{darkcyan1115178}{RGB}{1,115,178}
\definecolor{darkgray176}{RGB}{176,176,176}

\begin{axis}[
legend cell align={left},
legend cell align={left},
legend style={fill opacity=0.8, draw opacity=1, text opacity=1, draw=lightgray204, at={(0.5,0.03)},  anchor=south west},
tick align=outside,
tick pos=left,
x grid style={white},
xlabel={Number Env Interactions},
xmajorgrids,
xmin=-0.0, xmax=1000000.0,
xtick style={color=black},
y grid style={white},
ylabel={IQM Mean Return},
ymajorgrids,
ymin=-12.11232765065, ymax=350.65256879765,
ytick style={color=black},
axis background/.style={fill=plot_background},
label style={font=\large},
tick label style={font=\large},
x axis line style={draw=none},
y axis line style={draw=none},
]

\path [draw=C1, fill=C1, opacity=0.2]
(axis cs:1,0.913481333333333)
--(axis cs:1,0.6287236)
--(axis cs:24000,0.677781566666667)
--(axis cs:48000,0.738428966666667)
--(axis cs:72000,0.790174833333333)
--(axis cs:96000,1.1543524)
--(axis cs:120000,3.23502683000002)
--(axis cs:144000,20.5413528)
--(axis cs:168000,8.54387893333333)
--(axis cs:192000,34.4303118333333)
--(axis cs:216000,72.0098966666667)
--(axis cs:240000,72.3488675833333)
--(axis cs:264000,96.707165)
--(axis cs:288000,96.000784)
--(axis cs:312000,104.958421625)
--(axis cs:336000,114.297126)
--(axis cs:360000,106.284968)
--(axis cs:384000,123.7906925)
--(axis cs:408000,123.723489166667)
--(axis cs:432000,128.761211666667)
--(axis cs:456000,111.02938625)
--(axis cs:480000,135.956141666667)
--(axis cs:504000,138.98209)
--(axis cs:528000,118.929581333333)
--(axis cs:552000,132.387325)
--(axis cs:576000,148.148588333333)
--(axis cs:600000,153.61958)
--(axis cs:624000,156.55045)
--(axis cs:648000,156.870825)
--(axis cs:672000,161.074148333333)
--(axis cs:696000,164.107246666667)
--(axis cs:720000,165.654856541667)
--(axis cs:744000,167.669441666667)
--(axis cs:768000,160.786089)
--(axis cs:792000,168.99019)
--(axis cs:816000,177.369616666667)
--(axis cs:840000,175.239106666667)
--(axis cs:864000,176.361243333333)
--(axis cs:888000,165.394583333333)
--(axis cs:912000,183.752826666667)
--(axis cs:936000,187.066958333333)
--(axis cs:960000,160.625822716667)
--(axis cs:984000,191.555848333333)
--(axis cs:984000,251.074266666667)
--(axis cs:984000,251.074266666667)
--(axis cs:960000,253.124288333333)
--(axis cs:936000,245.776626666667)
--(axis cs:912000,236.034463333333)
--(axis cs:888000,226.598225)
--(axis cs:864000,227.013098708333)
--(axis cs:840000,234.283123333333)
--(axis cs:816000,229.472495)
--(axis cs:792000,224.034361666667)
--(axis cs:768000,206.047165)
--(axis cs:744000,209.549365)
--(axis cs:720000,205.859246666667)
--(axis cs:696000,194.984201666667)
--(axis cs:672000,188.552526666667)
--(axis cs:648000,185.157259583333)
--(axis cs:624000,177.684235)
--(axis cs:600000,172.894043333333)
--(axis cs:576000,171.0584)
--(axis cs:552000,163.391113333333)
--(axis cs:528000,162.135815)
--(axis cs:504000,161.517217666667)
--(axis cs:480000,153.846013166667)
--(axis cs:456000,150.815503333333)
--(axis cs:432000,154.043445)
--(axis cs:408000,146.23077)
--(axis cs:384000,144.130305)
--(axis cs:360000,136.713208333333)
--(axis cs:336000,131.509801666667)
--(axis cs:312000,123.557279541667)
--(axis cs:288000,121.774874166667)
--(axis cs:264000,117.476811666667)
--(axis cs:240000,110.350656666667)
--(axis cs:216000,103.274480333333)
--(axis cs:192000,90.3314516666667)
--(axis cs:168000,85.1717466666667)
--(axis cs:144000,83.9523868333333)
--(axis cs:120000,60.989724)
--(axis cs:96000,26.2789549666667)
--(axis cs:72000,5.35472753333333)
--(axis cs:48000,1.06001243333333)
--(axis cs:24000,0.923337466666667)
--(axis cs:1,0.913481333333333)
--cycle;

\addplot [line width=\linewidthdime, C1, mark=*, mark size=0, mark options={solid}]
table {%
1 0.7694417
24000 0.813880766666667
48000 0.903083466666667
72000 0.985341633333333
96000 7.39312896666667
120000 31.2267919666667
144000 57.5656945
168000 45.1603116666667
192000 70.5692
216000 94.6452396666667
240000 97.5139125
264000 109.186628333333
288000 109.1955065
312000 111.598211666667
336000 123.198625
360000 125.946167
384000 133.888621666667
408000 134.214293333333
432000 139.483735
456000 142.763216666667
480000 144.338755
504000 149.75221
528000 151.421726666667
552000 147.365006666667
576000 159.676698333333
600000 161.191491666667
624000 163.406718333333
648000 169.18849
672000 173.390193333333
696000 174.364796666667
720000 181.648728333333
744000 181.343818333333
768000 179.253028333333
792000 189.492575
816000 195.565436666667
840000 198.284515
864000 193.75859
888000 187.434418333333
912000 201.805691666667
936000 209.501925
960000 208.739933333333
984000 214.710998333333
};

%DIME
\path [draw=C0, fill=C0, opacity=0.2]
(axis cs:1,0.915501966666667)
--(axis cs:1,0.633556866666667)
--(axis cs:36000,0.5493304)
--(axis cs:72000,0.843873266666667)
--(axis cs:108000,4.13696455)
--(axis cs:144000,36.1211871666667)
--(axis cs:180000,51.4064937333333)
--(axis cs:216000,81.368514)
--(axis cs:252000,97.5408251666667)
--(axis cs:288000,109.583571666667)
--(axis cs:324000,110.618288333333)
--(axis cs:360000,120.684184666667)
--(axis cs:396000,122.417020833333)
--(axis cs:432000,131.747461666667)
--(axis cs:468000,139.23245)
--(axis cs:504000,145.708203333333)
--(axis cs:540000,128.071794166667)
--(axis cs:576000,146.743565)
--(axis cs:612000,157.523090333333)
--(axis cs:648000,162.386513333333)
--(axis cs:684000,167.712168333333)
--(axis cs:720000,177.17324)
--(axis cs:756000,175.778066666667)
--(axis cs:792000,177.557476666667)
--(axis cs:828000,183.413371666667)
--(axis cs:864000,191.343878333333)
--(axis cs:900000,198.058078333333)
--(axis cs:936000,199.495351666667)
--(axis cs:972000,204.18452)
--(axis cs:1008000,211.017165)
--(axis cs:1044000,210.395921666667)
--(axis cs:1080000,225.37889)
--(axis cs:1116000,225.57669)
--(axis cs:1152000,229.055506666667)
--(axis cs:1188000,244.717408333333)
--(axis cs:1224000,235.69925925)
--(axis cs:1260000,246.761671666667)
--(axis cs:1296000,263.70817)
--(axis cs:1332000,261.655338333333)
--(axis cs:1368000,274.904988333333)
--(axis cs:1404000,281.871765)
--(axis cs:1440000,284.6845)
--(axis cs:1476000,298.373705)
--(axis cs:1512000,314.234433333333)
--(axis cs:1548000,302.586606666667)
--(axis cs:1584000,306.158113166667)
--(axis cs:1620000,329.89694)
--(axis cs:1656000,327.346265)
--(axis cs:1692000,331.977616666667)
--(axis cs:1728000,335.598585)
--(axis cs:1764000,337.5854)
--(axis cs:1800000,350.510931458333)
--(axis cs:1836000,348.138216666667)
--(axis cs:1872000,325.377726666667)
--(axis cs:1908000,375.064071666667)
--(axis cs:1944000,364.504996666667)
--(axis cs:1980000,362.576775)
--(axis cs:2016000,359.905746666667)
--(axis cs:2052000,401.565295)
--(axis cs:2088000,380.944233333333)
--(axis cs:2124000,408.07439)
--(axis cs:2160000,406.816443333333)
--(axis cs:2196000,402.540623333333)
--(axis cs:2232000,419.828846666667)
--(axis cs:2268000,419.407693333333)
--(axis cs:2304000,391.893025)
--(axis cs:2340000,426.646173333333)
--(axis cs:2376000,430.59703)
--(axis cs:2412000,440.944021375)
--(axis cs:2448000,444.535958333333)
--(axis cs:2484000,447.59097)
--(axis cs:2520000,458.220718125)
--(axis cs:2556000,454.987311666667)
--(axis cs:2592000,444.593778333333)
--(axis cs:2628000,452.355353333333)
--(axis cs:2664000,472.315071583333)
--(axis cs:2700000,468.552297125)
--(axis cs:2736000,425.893185)
--(axis cs:2772000,461.17882)
--(axis cs:2808000,438.974565)
--(axis cs:2844000,408.566183333333)
--(axis cs:2880000,440.493023333333)
--(axis cs:2916000,474.594651666667)
--(axis cs:2952000,473.13282)
--(axis cs:2988000,468.31128725)
--(axis cs:2988000,572.54452975)
--(axis cs:2988000,572.54452975)
--(axis cs:2952000,607.047773333333)
--(axis cs:2916000,581.955683333333)
--(axis cs:2880000,562.69891)
--(axis cs:2844000,570.399441666667)
--(axis cs:2808000,561.294171666667)
--(axis cs:2772000,575.424046666667)
--(axis cs:2736000,577.564579375)
--(axis cs:2700000,570.731558333333)
--(axis cs:2664000,556.645568333333)
--(axis cs:2628000,528.610636666667)
--(axis cs:2592000,572.906766666667)
--(axis cs:2556000,556.792031666667)
--(axis cs:2520000,565.548321666667)
--(axis cs:2484000,564.076643333333)
--(axis cs:2448000,554.428711666667)
--(axis cs:2412000,511.028458333333)
--(axis cs:2376000,539.327213333333)
--(axis cs:2340000,521.274179458333)
--(axis cs:2304000,507.011033333333)
--(axis cs:2268000,533.545403333333)
--(axis cs:2232000,530.436551666667)
--(axis cs:2196000,514.156165)
--(axis cs:2160000,548.6599)
--(axis cs:2124000,526.241701666667)
--(axis cs:2088000,488.449933333333)
--(axis cs:2052000,521.397975)
--(axis cs:2016000,489.724498333333)
--(axis cs:1980000,511.56874)
--(axis cs:1944000,469.79517)
--(axis cs:1908000,487.4035)
--(axis cs:1872000,452.261116666667)
--(axis cs:1836000,448.443288333333)
--(axis cs:1800000,470.013476541667)
--(axis cs:1764000,441.067136666667)
--(axis cs:1728000,425.983283333333)
--(axis cs:1692000,459.900376666667)
--(axis cs:1656000,420.555463333333)
--(axis cs:1620000,452.412428333333)
--(axis cs:1584000,426.080911666667)
--(axis cs:1548000,405.744391916667)
--(axis cs:1512000,422.468795)
--(axis cs:1476000,392.121763333333)
--(axis cs:1440000,398.927006666667)
--(axis cs:1404000,396.300881666667)
--(axis cs:1368000,391.708926666667)
--(axis cs:1332000,379.3367)
--(axis cs:1296000,374.028266666667)
--(axis cs:1260000,354.826833333333)
--(axis cs:1224000,325.77437)
--(axis cs:1188000,356.78896)
--(axis cs:1152000,352.524811666667)
--(axis cs:1116000,343.294016666667)
--(axis cs:1080000,331.116815)
--(axis cs:1044000,287.783165)
--(axis cs:1008000,319.938826666667)
--(axis cs:972000,297.00894)
--(axis cs:936000,297.43841)
--(axis cs:900000,295.160138333333)
--(axis cs:864000,278.186283333333)
--(axis cs:828000,267.117865)
--(axis cs:792000,254.04739775)
--(axis cs:756000,245.040613333333)
--(axis cs:720000,223.696456875)
--(axis cs:684000,216.1656)
--(axis cs:648000,205.401713333333)
--(axis cs:612000,201.404601666667)
--(axis cs:576000,182.55503)
--(axis cs:540000,177.786793333333)
--(axis cs:504000,170.561265208333)
--(axis cs:468000,170.450261666667)
--(axis cs:432000,164.02998)
--(axis cs:396000,153.633991666667)
--(axis cs:360000,143.37209)
--(axis cs:324000,136.525775833333)
--(axis cs:288000,129.716861683333)
--(axis cs:252000,113.1404125)
--(axis cs:216000,102.49002)
--(axis cs:180000,92.9197603333333)
--(axis cs:144000,76.8890066666667)
--(axis cs:108000,56.7536433333333)
--(axis cs:72000,4.25857906666667)
--(axis cs:36000,0.728938933333333)
--(axis cs:1,0.915501966666667)
--cycle;

\addplot [line width=\linewidthdime, C0, mark=*, mark size=0, mark options={solid}]
table {%
1 0.770562166666667
36000 0.658370533333333
72000 1.02744053333333
108000 27.4132359166667
144000 61.7389005
180000 78.3388245
216000 96.6475441666667
252000 104.292225666667
288000 118.336289333333
324000 122.156900833333
360000 127.865645666667
396000 135.589655833333
432000 144.61231
468000 150.876916666667
504000 155.106371666667
540000 161.848961666667
576000 162.873085
612000 175.66877
648000 181.774033333333
684000 189.795443333333
720000 201.694543333333
756000 209.679653333333
792000 215.950858333333
828000 226.452128333333
864000 235.790758333333
900000 245.745095
936000 251.419853333333
972000 253.923228333333
1008000 267.745066666667
1044000 253.029908333333
1080000 281.1757
1116000 290.478148333333
1152000 295.386573333333
1188000 307.510898333333
1224000 279.324226666667
1260000 298.499671666667
1296000 320.530285
1332000 314.688511666667
1368000 333.564955
1404000 344.035705
1440000 351.026203333333
1476000 334.48205
1512000 371.517083333333
1548000 359.107073333333
1584000 362.49581
1620000 383.735268333333
1656000 371.120853333333
1692000 392.875408333333
1728000 375.706403333333
1764000 385.148411666667
1800000 411.377563333333
1836000 393.022041666667
1872000 381.083618333333
1908000 428.514751666667
1944000 407.976545
1980000 440.522148333333
2016000 425.096363333333
2052000 460.246375
2088000 439.221346666667
2124000 470.482476666667
2160000 479.806223333333
2196000 467.028006666667
2232000 471.833656666667
2268000 492.305195
2304000 444.283438333333
2340000 467.734686666667
2376000 478.305563333333
2412000 479.659883333333
2448000 502.891963333333
2484000 512.254108333333
2520000 504.927955
2556000 509.89631
2592000 517.465808333333
2628000 505.843831666667
2664000 503.219516666667
2700000 525.334166666667
2736000 498.718568333333
2772000 515.273173333333
2808000 507.004858333333
2844000 490.845221666667
2880000 488.33129
2916000 530.288801666667
2952000 544.121438333333
2988000 506.351106666667
};

\end{axis}

\end{tikzpicture}
}
       \subcaption[]{}
       \label{fig::exps_new_ablations_gauss_vs_diff_distrq::humanoid_run}
    \end{minipage}\hfill
    \begin{minipage}[b]{0.24\textwidth}
        \centering
       \resizebox{1\textwidth}{!}{% This file was created with tikzplotlib v0.10.1.
\begin{tikzpicture}

\definecolor{darkcyan1115178}{RGB}{1,115,178}
\definecolor{darkgray176}{RGB}{176,176,176}

\begin{axis}[
legend cell align={left},
legend cell align={left},
legend style={fill opacity=0.8, draw opacity=1, text opacity=1, draw=lightgray204, at={(0.5,0.03)},  anchor=south west},
tick align=outside,
tick pos=left,
x grid style={white},
xlabel={Number Env Interactions},
xmajorgrids,
%xmin=-149398.95, xmax=3137399.95,
xmin=0.0, xmax=3000000.00,
xtick style={color=black},
y grid style={white},
ylabel={IQM Return},
ymajorgrids,
ymin=0.0, ymax=901.400008641667,
ytick style={color=black},
axis background/.style={fill=plot_background},
label style={font=\large},
tick label style={font=\large},
x axis line style={draw=none},
y axis line style={draw=none},
]
\path [draw=C3, fill=C3, opacity=0.2]
(axis cs:1,5.89064738333333)
--(axis cs:1,4.1291282)
--(axis cs:36000,4.36958118333333)
--(axis cs:72000,21.9102696666667)
--(axis cs:108000,44.5369561666667)
--(axis cs:144000,73.8133631666667)
--(axis cs:180000,92.6374976666667)
--(axis cs:216000,103.433920833333)
--(axis cs:252000,122.695540833333)
--(axis cs:288000,144.780881666667)
--(axis cs:324000,141.113765)
--(axis cs:360000,163.508661208333)
--(axis cs:396000,155.823601916667)
--(axis cs:432000,201.528308333333)
--(axis cs:468000,213.320525)
--(axis cs:504000,224.301218333333)
--(axis cs:540000,237.119743333333)
--(axis cs:576000,246.070941666667)
--(axis cs:612000,267.884333333333)
--(axis cs:648000,275.265338333333)
--(axis cs:684000,291.610186666667)
--(axis cs:720000,306.509876666667)
--(axis cs:756000,314.089796666667)
--(axis cs:792000,328.317606666667)
--(axis cs:828000,333.810906666667)
--(axis cs:864000,338.667664458333)
--(axis cs:900000,343.544436666667)
--(axis cs:936000,361.921333333333)
--(axis cs:972000,372.899663333333)
--(axis cs:1008000,380.099353333333)
--(axis cs:1044000,403.222876666667)
--(axis cs:1080000,387.989915)
--(axis cs:1116000,417.562756666667)
--(axis cs:1152000,407.567365)
--(axis cs:1188000,430.534505)
--(axis cs:1224000,431.1114)
--(axis cs:1260000,434.56934)
--(axis cs:1296000,444.288155)
--(axis cs:1332000,476.369876666667)
--(axis cs:1368000,461.040406666667)
--(axis cs:1404000,469.349473333333)
--(axis cs:1440000,488.009956666667)
--(axis cs:1476000,488.694688333333)
--(axis cs:1512000,511.883405041667)
--(axis cs:1548000,563.67264)
--(axis cs:1584000,545.585495208333)
--(axis cs:1620000,560.501143333333)
--(axis cs:1656000,535.712238333333)
--(axis cs:1692000,501.032973333333)
--(axis cs:1728000,588.495035)
--(axis cs:1764000,534.116686666667)
--(axis cs:1800000,567.87324)
--(axis cs:1836000,597.2437)
--(axis cs:1872000,527.918356666667)
--(axis cs:1908000,638.176695)
--(axis cs:1944000,630.920571208333)
--(axis cs:1980000,640.987561666667)
--(axis cs:2016000,653.72166)
--(axis cs:2052000,679.265283333333)
--(axis cs:2088000,632.715323333333)
--(axis cs:2124000,681.83093)
--(axis cs:2160000,679.575463333333)
--(axis cs:2196000,722.347776666667)
--(axis cs:2232000,667.204936666667)
--(axis cs:2268000,685.014776666667)
--(axis cs:2304000,704.751333333333)
--(axis cs:2340000,704.640791666667)
--(axis cs:2376000,719.756748333333)
--(axis cs:2412000,744.714488333333)
--(axis cs:2448000,731.197275)
--(axis cs:2484000,720.243575)
--(axis cs:2520000,669.276773333333)
--(axis cs:2556000,754.734663333333)
--(axis cs:2592000,739.97988475)
--(axis cs:2628000,728.446603333333)
--(axis cs:2664000,766.406533333333)
--(axis cs:2700000,750.326138333333)
--(axis cs:2736000,734.063323333333)
--(axis cs:2772000,772.572516666667)
--(axis cs:2808000,728.854876666667)
--(axis cs:2844000,764.823895)
--(axis cs:2880000,784.974858333333)
--(axis cs:2916000,764.92251)
--(axis cs:2952000,781.2523575)
--(axis cs:2988000,772.058598333333)
--(axis cs:2988000,858.591428333333)
--(axis cs:2988000,858.591428333333)
--(axis cs:2952000,853.033060333334)
--(axis cs:2916000,843.288837)
--(axis cs:2880000,867.902476666667)
--(axis cs:2844000,824.153933333333)
--(axis cs:2808000,864.357281666667)
--(axis cs:2772000,845.727528333333)
--(axis cs:2736000,876.22724)
--(axis cs:2700000,861.385121666667)
--(axis cs:2664000,843.982653333333)
--(axis cs:2628000,850.563796666667)
--(axis cs:2592000,834.694483333333)
--(axis cs:2556000,864.064105)
--(axis cs:2520000,822.63462)
--(axis cs:2484000,838.08071)
--(axis cs:2448000,837.947975)
--(axis cs:2412000,837.841368333333)
--(axis cs:2376000,807.278331666667)
--(axis cs:2340000,822.403533333333)
--(axis cs:2304000,832.89487)
--(axis cs:2268000,830.849501666667)
--(axis cs:2232000,837.47201)
--(axis cs:2196000,831.17095)
--(axis cs:2160000,825.700883333333)
--(axis cs:2124000,828.526442916667)
--(axis cs:2088000,770.616286666667)
--(axis cs:2052000,807.72665)
--(axis cs:2016000,751.081956666667)
--(axis cs:1980000,782.27038)
--(axis cs:1944000,781.074635)
--(axis cs:1908000,775.310213333333)
--(axis cs:1872000,764.705915)
--(axis cs:1836000,786.169281666667)
--(axis cs:1800000,774.01652)
--(axis cs:1764000,728.940376666667)
--(axis cs:1728000,761.033158333333)
--(axis cs:1692000,770.318991666667)
--(axis cs:1656000,735.083688333333)
--(axis cs:1620000,733.302023333333)
--(axis cs:1584000,737.523791666667)
--(axis cs:1548000,723.791843333333)
--(axis cs:1512000,689.69503)
--(axis cs:1476000,703.447666666667)
--(axis cs:1440000,703.884616666667)
--(axis cs:1404000,689.787128333333)
--(axis cs:1368000,663.751797083334)
--(axis cs:1332000,675.50549)
--(axis cs:1296000,676.303538333333)
--(axis cs:1260000,650.188453333333)
--(axis cs:1224000,589.741586083333)
--(axis cs:1188000,615.98814)
--(axis cs:1152000,582.320133333333)
--(axis cs:1116000,596.750455)
--(axis cs:1080000,577.708458333333)
--(axis cs:1044000,553.53348)
--(axis cs:1008000,558.394433333333)
--(axis cs:972000,528.143061666667)
--(axis cs:936000,510.096248333333)
--(axis cs:900000,495.766748333333)
--(axis cs:864000,458.83707)
--(axis cs:828000,442.794613333333)
--(axis cs:792000,429.712663333333)
--(axis cs:756000,392.22884)
--(axis cs:720000,408.753066666667)
--(axis cs:684000,370.69758)
--(axis cs:648000,363.800676666667)
--(axis cs:612000,335.425881666667)
--(axis cs:576000,315.10437775)
--(axis cs:540000,302.417836666667)
--(axis cs:504000,287.786676666667)
--(axis cs:468000,256.548557125)
--(axis cs:432000,253.588628333333)
--(axis cs:396000,225.467696666667)
--(axis cs:360000,206.161806666667)
--(axis cs:324000,181.996731666667)
--(axis cs:288000,175.301235)
--(axis cs:252000,160.583735)
--(axis cs:216000,143.503271666667)
--(axis cs:180000,121.901736666667)
--(axis cs:144000,97.5643175)
--(axis cs:108000,79.6164608333333)
--(axis cs:72000,39.3428316666667)
--(axis cs:36000,11.0066606166667)
--(axis cs:1,5.89064738333333)
--cycle;

\addplot [line width=\linewidthother, C3, mark=*, mark size=0, mark options={solid}]
table {%
1 4.8701071
36000 6.72384941666667
72000 29.3581018333333
108000 63.3330631666667
144000 86.7607035
180000 110.848872666667
216000 125.25725
252000 144.375887333333
288000 159.330578333333
324000 161.599885
360000 188.998286666667
396000 203.654086666667
432000 232.308528333333
468000 240.16393
504000 257.163325
540000 270.039501666667
576000 273.186586666667
612000 304.593976666667
648000 318.949601666667
684000 328.977761666667
720000 359.723766666667
756000 362.648261666667
792000 386.788338333333
828000 390.710991666667
864000 405.630171666667
900000 415.033181666667
936000 434.406411666667
972000 452.752266666667
1008000 470.17097
1044000 484.749701666667
1080000 482.753605
1116000 511.220326666667
1152000 506.144361666667
1188000 527.276881666667
1224000 512.964038333333
1260000 544.194691666667
1296000 563.19015
1332000 578.308555
1368000 566.009768333333
1404000 591.790591666667
1440000 590.803741666667
1476000 607.842233333333
1512000 605.637781666667
1548000 636.038028333333
1584000 643.470818333333
1620000 634.836016666667
1656000 624.311076666667
1692000 629.079153333333
1728000 673.619896666667
1764000 605.227423333333
1800000 665.612198333333
1836000 698.479258333333
1872000 658.705781666667
1908000 700.169148333333
1944000 705.618373333333
1980000 719.874233333333
2016000 711.819425
2052000 748.167208333333
2088000 720.219761666667
2124000 755.752271666667
2160000 765.831068333333
2196000 787.752566666667
2232000 759.990015
2268000 762.137346666667
2304000 780.45201
2340000 774.628483333333
2376000 758.721198333333
2412000 802.726498333333
2448000 796.593808333333
2484000 795.960206666667
2520000 746.542065
2556000 832.110631666667
2592000 787.537713333333
2628000 803.720935
2664000 810.365966666667
2700000 829.490688333333
2736000 815.02574
2772000 809.62794
2808000 814.520655
2844000 794.265555
2880000 828.936341666667
2916000 794.882741666667
2952000 819.517033333333
2988000 807.66068
};
\addlegendentry{Gaussian Policy}
\path [draw=C0, fill=C0, opacity=0.2]
(axis cs:1,6.7908015)
--(axis cs:1,5.6020231)
--(axis cs:36000,8.62983233333333)
--(axis cs:72000,39.178405)
--(axis cs:108000,58.7256123333333)
--(axis cs:144000,70.247577)
--(axis cs:180000,85.990517)
--(axis cs:216000,132.272293333333)
--(axis cs:252000,149.256846333333)
--(axis cs:288000,171.63141)
--(axis cs:324000,192.776848333333)
--(axis cs:360000,215.394498333333)
--(axis cs:396000,245.356148333333)
--(axis cs:432000,256.659486666667)
--(axis cs:468000,271.564893333333)
--(axis cs:504000,309.006485)
--(axis cs:540000,307.207731666667)
--(axis cs:576000,357.640175)
--(axis cs:612000,389.479391666667)
--(axis cs:648000,391.45075)
--(axis cs:684000,422.799776666667)
--(axis cs:720000,458.922981666667)
--(axis cs:756000,474.32947)
--(axis cs:792000,483.093959041667)
--(axis cs:828000,506.47542)
--(axis cs:864000,527.350673333333)
--(axis cs:900000,544.75695)
--(axis cs:936000,560.401128333333)
--(axis cs:972000,583.388941666667)
--(axis cs:1008000,589.799863333333)
--(axis cs:1044000,589.046265)
--(axis cs:1080000,607.424441666667)
--(axis cs:1116000,607.498523333333)
--(axis cs:1152000,638.028483333333)
--(axis cs:1188000,601.18704)
--(axis cs:1224000,659.686861666667)
--(axis cs:1260000,667.927208708333)
--(axis cs:1296000,660.504675)
--(axis cs:1332000,676.24953)
--(axis cs:1368000,665.97862)
--(axis cs:1404000,687.60902125)
--(axis cs:1440000,700.087826666667)
--(axis cs:1476000,671.29)
--(axis cs:1512000,690.782141666667)
--(axis cs:1548000,695.240922208333)
--(axis cs:1584000,677.796726666667)
--(axis cs:1620000,720.659616666667)
--(axis cs:1656000,709.123516666667)
--(axis cs:1692000,717.853556666667)
--(axis cs:1728000,732.13092)
--(axis cs:1764000,662.37287075)
--(axis cs:1800000,708.438711666667)
--(axis cs:1836000,748.553096666667)
--(axis cs:1872000,731.0669)
--(axis cs:1908000,730.051075)
--(axis cs:1944000,769.617006666667)
--(axis cs:1980000,759.68223)
--(axis cs:2016000,756.386703333333)
--(axis cs:2052000,788.513618333333)
--(axis cs:2088000,725.240085708333)
--(axis cs:2124000,763.8678)
--(axis cs:2160000,786.34295)
--(axis cs:2196000,721.03074)
--(axis cs:2232000,787.1876)
--(axis cs:2268000,768.494865)
--(axis cs:2304000,768.75273)
--(axis cs:2340000,772.603808333333)
--(axis cs:2376000,763.798678333333)
--(axis cs:2412000,815.450611666667)
--(axis cs:2448000,797.30472)
--(axis cs:2484000,796.082483333333)
--(axis cs:2520000,789.613171666667)
--(axis cs:2556000,803.113718333333)
--(axis cs:2592000,776.112966666667)
--(axis cs:2628000,800.927163333333)
--(axis cs:2664000,762.744901666667)
--(axis cs:2700000,789.174581666667)
--(axis cs:2736000,798.797758625)
--(axis cs:2772000,808.763553333333)
--(axis cs:2808000,791.100212166667)
--(axis cs:2844000,797.507851666667)
--(axis cs:2880000,756.81125)
--(axis cs:2916000,802.708426666667)
--(axis cs:2952000,778.739201666667)
--(axis cs:2988000,806.379209958333)
--(axis cs:2988000,856.395113333333)
--(axis cs:2988000,856.395113333333)
--(axis cs:2952000,849.116565)
--(axis cs:2916000,859.785973333333)
--(axis cs:2880000,849.40738)
--(axis cs:2844000,864.92043)
--(axis cs:2808000,863.616378333333)
--(axis cs:2772000,851.90444)
--(axis cs:2736000,852.977273333333)
--(axis cs:2700000,849.869583333333)
--(axis cs:2664000,868.884923333333)
--(axis cs:2628000,845.164255)
--(axis cs:2592000,863.82811)
--(axis cs:2556000,851.953161666667)
--(axis cs:2520000,848.667223333333)
--(axis cs:2484000,842.26205)
--(axis cs:2448000,844.661853333333)
--(axis cs:2412000,835.335961666667)
--(axis cs:2376000,851.43932)
--(axis cs:2340000,858.032386666667)
--(axis cs:2304000,823.622453333333)
--(axis cs:2268000,842.820591666667)
--(axis cs:2232000,833.971345)
--(axis cs:2196000,825.505923333333)
--(axis cs:2160000,843.710091666666)
--(axis cs:2124000,841.442543333333)
--(axis cs:2088000,832.747906666667)
--(axis cs:2052000,832.985603333333)
--(axis cs:2016000,831.623071666667)
--(axis cs:1980000,834.6822)
--(axis cs:1944000,821.178215)
--(axis cs:1908000,858.758488333333)
--(axis cs:1872000,831.917276666667)
--(axis cs:1836000,816.230947916667)
--(axis cs:1800000,841.425425)
--(axis cs:1764000,803.220675)
--(axis cs:1728000,837.246975)
--(axis cs:1692000,825.25186)
--(axis cs:1656000,808.263138333333)
--(axis cs:1620000,815.792903541667)
--(axis cs:1584000,808.863161666667)
--(axis cs:1548000,805.28377)
--(axis cs:1512000,782.734225)
--(axis cs:1476000,788.93601)
--(axis cs:1440000,821.10777)
--(axis cs:1404000,792.100268333333)
--(axis cs:1368000,760.665633333333)
--(axis cs:1332000,725.122591666667)
--(axis cs:1296000,745.192578333333)
--(axis cs:1260000,735.194826666667)
--(axis cs:1224000,752.493893333333)
--(axis cs:1188000,702.905228583333)
--(axis cs:1152000,730.103061666667)
--(axis cs:1116000,717.392206666667)
--(axis cs:1080000,710.234308333333)
--(axis cs:1044000,674.616426666667)
--(axis cs:1008000,694.470678333333)
--(axis cs:972000,674.61414)
--(axis cs:936000,634.1209)
--(axis cs:900000,632.24315)
--(axis cs:864000,638.30593)
--(axis cs:828000,603.69465)
--(axis cs:792000,604.798345)
--(axis cs:756000,598.060326583333)
--(axis cs:720000,568.233463333333)
--(axis cs:684000,568.143085)
--(axis cs:648000,536.52782)
--(axis cs:612000,518.389283333333)
--(axis cs:576000,512.089843333333)
--(axis cs:540000,448.484928333333)
--(axis cs:504000,461.99631)
--(axis cs:468000,408.115491666667)
--(axis cs:432000,392.529571666667)
--(axis cs:396000,384.947523333333)
--(axis cs:360000,330.958086541667)
--(axis cs:324000,307.664625)
--(axis cs:288000,251.54248)
--(axis cs:252000,236.02058)
--(axis cs:216000,194.22897)
--(axis cs:180000,159.045319333333)
--(axis cs:144000,126.739716083333)
--(axis cs:108000,94.057901)
--(axis cs:72000,62.1926506666667)
--(axis cs:36000,16.8357318333333)
--(axis cs:1,6.7908015)
--cycle;

\addplot [line width=\linewidthdime, C0, mark=*, mark size=0, mark options={solid}]
table {%
1 6.17641296666667
36000 11.6270238333333
72000 51.2078703333333
108000 80.0899
144000 110.007961666667
180000 124.469762666667
216000 164.83243
252000 183.239915
288000 202.72266
324000 244.796003333333
360000 270.949796666667
396000 311.212078333333
432000 321.852778333333
468000 339.38604
504000 387.914375
540000 377.199451666667
576000 435.778751666667
612000 454.769993333333
648000 468.345228333333
684000 494.018921666667
720000 507.547415
756000 540.829465
792000 547.320155
828000 573.053538333333
864000 586.99087
900000 587.611123333333
936000 603.980191666667
972000 631.953166666667
1008000 652.547345
1044000 630.36095
1080000 677.789983333333
1116000 674.022491666667
1152000 696.043668333333
1188000 652.41135
1224000 716.223413333333
1260000 705.611823333333
1296000 711.299655
1332000 713.511105
1368000 718.17746
1404000 743.029048333333
1440000 767.977446666667
1476000 732.657913333333
1512000 740.075188333334
1548000 755.125068333333
1584000 745.76982
1620000 774.640848333333
1656000 775.007413333333
1692000 786.750093333333
1728000 795.057825
1764000 751.749063333333
1800000 785.936768333333
1836000 798.028601666667
1872000 784.5074
1908000 803.70621
1944000 802.98721
1980000 814.507853333333
2016000 809.862228333333
2052000 820.409161666667
2088000 792.892843333333
2124000 814.52938
2160000 829.289918333333
2196000 784.440623333333
2232000 810.922945
2268000 812.483123333333
2304000 787.9255
2340000 827.978133333333
2376000 809.844736666667
2412000 830.52828
2448000 827.945956666667
2484000 819.91705
2520000 821.356241666667
2556000 831.340293333333
2592000 838.71875
2628000 832.144581666667
2664000 842.476748333333
2700000 814.85007
2736000 828.082728333333
2772000 838.083201666667
2808000 840.753683333333
2844000 837.937456666667
2880000 812.837433333333
2916000 829.931661666667
2952000 813.313951666667
2988000 833.048301666667
};
\addlegendentry{DIME}
\end{axis}

\end{tikzpicture}
}
       \subcaption[]{}
       \label{fig::exps_new_ablations_gauss_vs_diff_distrq::dog_run}
    \end{minipage}\hfill
    \vspace{-2.0mm}
    \caption{\textbf{Reward Scaling Sensitivity (a)-(b)}. The $\alpha$ parameter controls the exploration-exploitation trade-off. (a) shows the learning curves for varying values on DMC's dog-run task. Too high $\alpha$ values ($\alpha=0.1$) do not incentivize learning whereas too small $\alpha$ values ($\alpha\leq10^{-5}$) converge to suboptimal behavior. (b) shows the aggregated end performance for each learning curve in (a). For increasing $\alpha$ values, the end performance increases until it reaches an optimum at $\alpha=10^{-3}$ after which the performance starts dropping. \textbf{Diffusion Policy Benefit (c) and (d).} We compare DIME to a Gaussian policy with the same implementation details as DIME on the (a) humanoid-run and (b) dog-run tasks. The diffusion-based policy reaches a higher return (a) and converges faster.} \vspace{-2mm}
\end{figure*}


\subsection{DIME: A Practical Diffusion RL Algorithm}\label{dime_practical}
To obtain a practical algorithm, we use a parameterized function approximation for the $Q$-function and the policy, that is, $Q_{\phi}$ and $\ppi$, with parameters $\phi$  and $\theta$, respectively.  Here, $\ppi$ is represented by a parameterized score network, see Eq. \ref{eq: approximate denoising process}. 
%
To perform approximate policy evaluation, we can minimize the Bellman residual,  
\begin{equation}\label{eq::Bellman_Residual}
    J_Q(\phi) = \frac{1}{2}\E\left[\left(Q_{\phi}(s_t,a^0_t) - Q_{\text{target}}(s_t,a^0_t)\right)^2\right],
\end{equation}
using stochastic gradients with respect to $\phi$. We provide implementation details in \Cref{sec:implementation details}. Moreover, the expectation is computed using state-action pairs collected from environment interactions and saved in a replay buffer. 
%
For policy improvement, we solve the approximate inference problem 
\begin{equation}
\label{eq: joint control as inference2}
\mathcal{L}(\theta) = D_{\text{KL}}\left(\ppi_{0:N}(\ac^{0:N}|\st)|\fpi_{0:N}(\ac^{0:N}|\st)\right),
\end{equation}
where the target policy, i.e., the marginal of the noising process in Eq. \ref{eq: forward joint} is given by the approximate $Q$-function
\begin{equation}
    \fpi_{0}(\ac^{0}|\st) = \frac{\exp Q_{\theta}(\st, \ac^0)}{\Z_{\theta}(\st)},
\end{equation}
where states are again sampled from a replay buffer.
Further expanding $\mathcal{L}(\theta)$ yields
\begin{align}
\label{eq: expanded loss}
    \mathcal{L}(\theta) = & \E_{\ppi}\Bigg[ \log \ppi_N(a^N|s) - Q_{\phi}(\st,\ac^0)   \\ \nonumber
    & + \sum_{n=1}^N \log\frac{\ppi_{n|n-1}(\ac^{n} \big| \ac^{n-1},\st)}{\fpi_{n-1|n}(\ac^{n-1} \big| \ac^{n},\st)} \Bigg] + \log \Z_{\phi}(s),
\end{align}
showing that $\Z_{\phi}$ is not needed to minimize Eq. \ref{eq: expanded loss} as it is independent of $\theta$. Moreover, contrary to the score-matching objective (see Eq. \ref{eq: score matching}) that is commonly used to optimize diffusion models, stochastic optimization of $\mathcal{L}(\theta)$ does not need access to samples $\ac_0 \sim \exp Q_{\phi}/\Z_{\phi}$, instead relying on stochastic gradients obtained via reparameterization trick \cite{kingma2013auto} using samples from the diffusion model $\ppi$.

\subsection{Implementation Details}\label{sec:implementation details}
\textbf{Autotuning Temperature.} We follow implementations like SAC \cite{haarnoja2018softimplementations} where the reward scaling parameter $\alpha$ is not absorbed into the reward but scales the entropy term. 
Choosing $\alpha$ depends on the reward ranges and the dimensionality of the action space which requires tuning it per environment. We instead follow prior works \cite{haarnoja2018softimplementations} for auto-tuning $\alpha$ by optimizing 
\begin{equation}\label{eq:auto_tuning_alpha}
    J(\alpha) = \alpha \left( \mathcal{H}_{\text{target}} - \plb \right),
\end{equation}
where $\mathcal{H}_{\text{target}}$ is a target value for the mismatch between the noising and denoising processes measured by the log ratio. 

\textbf{Autotuning Diffusion Coefficient.} Please note that the objective function in Eq. \ref{eq: expanded loss} is fully differentiable with respect to parameters of the diffusion process. As such, we additionally treat the diffusion coefficient $\beta$ as learnable parameter that is optimized end-to-end, further reducing the need for manual hyperparameter tuning. Further details on the parameterization can be found in Appendix \ref{appdx:implementation_details}.
%

\textbf{$Q$-function.} Following \citet{bhattcrossq} we adopt the CrossQ algorithm, i.e., we use Batch Renormalization in the Q-function and avoid a target network for calculating $Q_{\text{target}}$. When updating the Q-function, the values for the current and next state-action pairs are queried in parallel. The next Q-values are used as target values where the gradients are stopped. Additionally, we employ distributional Q learning as proposed by \cite{bellemare2017distributional}. The details are described in Appendix \ref{appdx:implementation_details}.
