\begin{figure*}[t!]
    \hspace{1.8cm}
    \resizebox{0.3\textwidth}{!}{
    \definecolor{crimson2143940}{RGB}{214,39,40}
\definecolor{darkorange25512714}{RGB}{255,127,14}
\definecolor{forestgreen4416044}{RGB}{44,160,44}
\definecolor{mediumpurple148103189}{RGB}{148,103,189}
\definecolor{steelblue31119180}{RGB}{31,119,180}
\definecolor{darkgray176}{RGB}{176,176,176}
\begin{tikzpicture} 
    \begin{axis}[%
    hide axis,
    xmin=10,
    xmax=50,
    ymin=0,
    ymax=0.4,
    legend style={
        draw=white!15!black,
        legend cell align=left,
        legend columns=-1, 
        legend style={
            draw=none,
            column sep=1ex,
            line width=0.5pt
        }
    },
    ]
    \addlegendimage{line width=2pt, color=C4}
    \addlegendentry{32};
    \addlegendimage{line width=2pt, color=C0}
    \addlegendentry{16};
    \addlegendimage{line width=2pt, color=C3}
    \addlegendentry{8};
    \addlegendimage{line width=2pt, color=C1}
    \addlegendentry{4};
    \addlegendimage{line width=2pt, color=C5}
    \addlegendentry{2};
    \end{axis}
\end{tikzpicture}}%    
    \hspace{1.6cm}
    \resizebox{0.5\textwidth}{!}{
    \definecolor{ao(english)}{rgb}{0.0,0.5,0}
\begin{tikzpicture} 
    \begin{axis}[%
    hide axis,
    xmin=10,
    xmax=50,
    ymin=0,
    ymax=0.4,
    legend style={
        draw=white!15!black,
        legend cell align=left,
        legend columns=-1, 
        legend style={
            draw=none,
            column sep=1ex,
            line width=0.5pt
        }
    },
    ]
    \addlegendimage{line width=\linewidthdime, color=C0}
    \addlegendentry{DIME (ours)};
    \addlegendimage{line width=\linewidthother, color=C2}
    \addlegendentry{CrossQ};
    \addlegendimage{line width=\linewidthother, color=C3}
    \addlegendentry{QSM};
    \addlegendimage{line width=\linewidthother, color=C4}
    \addlegendentry{Diff-QL};
    \addlegendimage{line width=\linewidthother, color=C5}
    \addlegendentry{Consistency-AC};
    \addlegendimage{line width=\linewidthother, color=C6}
    \addlegendentry{DIPO};
    \end{axis}
\end{tikzpicture}}%
    
    \begin{minipage}[b]{0.25\textwidth}
        \centering
       \resizebox{1\textwidth}{!}{\definecolor{crimson2143940}{RGB}{214,39,40}
\definecolor{darkgray176}{RGB}{176,176,176}
\definecolor{darkorange25512714}{RGB}{255,127,14}
\definecolor{forestgreen4416044}{RGB}{44,160,44}
\definecolor{mediumpurple148103189}{RGB}{148,103,189}
\definecolor{steelblue31119180}{RGB}{31,119,180}

% This file was created with tikzplotlib v0.10.1.
\begin{tikzpicture}

\definecolor{darkcyan1115178}{RGB}{1,115,178}
\definecolor{darkgray176}{RGB}{176,176,176}

\begin{axis}[
legend cell align={left},
legend style={fill opacity=0.8, draw opacity=1, text opacity=1, draw=lightgray204, at={(0.03,0.03)},  anchor=north west},
tick align=outside,
tick pos=left,
x grid style={white},
xlabel={Number Env Interactions},
xmajorgrids,
%xmin=-149398.95, xmax=3000000.00,
xmin=0.0, xmax=3000000.00,
xtick style={color=black},
y grid style={white},
ylabel={IQM Return},
ymajorgrids,
%ymin=-37.9926843766667, ymax=923.09120151,
ymin=0.0, ymax=637.37269548,
ytick style={color=black},
axis background/.style={fill=plot_background},
label style={font=\large},
tick label style={font=\large},
x axis line style={draw=none},
y axis line style={draw=none},
]
\path [draw=C4, fill=C4, opacity=0.2]
(axis cs:1,0.9499512)
--(axis cs:1,0.61380104)
--(axis cs:36000,0.5091978)
--(axis cs:72000,0.78014448)
--(axis cs:108000,1.51235488)
--(axis cs:144000,33.15233536)
--(axis cs:180000,42.49638448)
--(axis cs:216000,52.13765132)
--(axis cs:252000,61.06937196)
--(axis cs:288000,97.4847088)
--(axis cs:324000,109.085078)
--(axis cs:360000,114.543702)
--(axis cs:396000,122.4942048)
--(axis cs:432000,128.871226)
--(axis cs:468000,141.603998)
--(axis cs:504000,145.473292)
--(axis cs:540000,148.618506)
--(axis cs:576000,155.97724)
--(axis cs:612000,160.68527)
--(axis cs:648000,165.718508)
--(axis cs:684000,166.509236)
--(axis cs:720000,174.137348)
--(axis cs:756000,174.262634)
--(axis cs:792000,178.9421)
--(axis cs:828000,184.653386)
--(axis cs:864000,182.44611)
--(axis cs:900000,187.235218)
--(axis cs:936000,196.986582)
--(axis cs:972000,197.314636)
--(axis cs:1008000,201.717228)
--(axis cs:1044000,201.62223)
--(axis cs:1080000,207.79759)
--(axis cs:1116000,204.843766)
--(axis cs:1152000,212.277216)
--(axis cs:1188000,221.810054)
--(axis cs:1224000,216.7921)
--(axis cs:1260000,229.114334)
--(axis cs:1296000,227.584516)
--(axis cs:1332000,235.686472)
--(axis cs:1368000,234.308532)
--(axis cs:1404000,252.275902)
--(axis cs:1440000,239.882608)
--(axis cs:1476000,245.771224)
--(axis cs:1512000,250.173308)
--(axis cs:1548000,251.61455)
--(axis cs:1584000,266.232552)
--(axis cs:1620000,257.929554)
--(axis cs:1656000,274.230988)
--(axis cs:1692000,241.24126)
--(axis cs:1728000,284.474236)
--(axis cs:1764000,288.503434)
--(axis cs:1800000,270.387816)
--(axis cs:1836000,287.790458)
--(axis cs:1872000,296.016392)
--(axis cs:1908000,303.301005500001)
--(axis cs:1944000,303.354686)
--(axis cs:1980000,302.789166)
--(axis cs:2016000,313.029178)
--(axis cs:2052000,321.501902)
--(axis cs:2088000,314.529324)
--(axis cs:2124000,327.436628)
--(axis cs:2160000,326.064636)
--(axis cs:2196000,269.621206)
--(axis cs:2232000,333.753716)
--(axis cs:2268000,337.921414)
--(axis cs:2304000,339.602288)
--(axis cs:2340000,364.843068)
--(axis cs:2376000,370.925054)
--(axis cs:2412000,362.824864)
--(axis cs:2448000,375.315076)
--(axis cs:2484000,386.297598)
--(axis cs:2520000,359.37899)
--(axis cs:2556000,396.686304)
--(axis cs:2592000,241.9026252)
--(axis cs:2628000,395.40906)
--(axis cs:2664000,403.486806)
--(axis cs:2700000,389.548974)
--(axis cs:2736000,377.496663100001)
--(axis cs:2772000,394.291604)
--(axis cs:2808000,415.150146)
--(axis cs:2844000,405.867546)
--(axis cs:2880000,418.754272)
--(axis cs:2916000,432.54288)
--(axis cs:2952000,436.790948)
--(axis cs:2988000,417.228128)
--(axis cs:2988000,653.16198)
--(axis cs:2988000,653.16198)
--(axis cs:2952000,609.04662)
--(axis cs:2916000,631.51264)
--(axis cs:2880000,620.942462)
--(axis cs:2844000,626.28872)
--(axis cs:2808000,619.28654)
--(axis cs:2772000,571.93722)
--(axis cs:2736000,578.5241)
--(axis cs:2700000,570.601692)
--(axis cs:2664000,593.20576)
--(axis cs:2628000,597.961664)
--(axis cs:2592000,587.941856)
--(axis cs:2556000,587.59352)
--(axis cs:2520000,604.182492)
--(axis cs:2484000,576.170256)
--(axis cs:2448000,567.21784)
--(axis cs:2412000,574.79071)
--(axis cs:2376000,557.9365)
--(axis cs:2340000,575.453666)
--(axis cs:2304000,535.86766)
--(axis cs:2268000,527.993912)
--(axis cs:2232000,516.516804)
--(axis cs:2196000,520.100468)
--(axis cs:2160000,494.965038)
--(axis cs:2124000,511.347942)
--(axis cs:2088000,508.787848)
--(axis cs:2052000,504.454604)
--(axis cs:2016000,486.461312)
--(axis cs:1980000,485.57528)
--(axis cs:1944000,478.008182)
--(axis cs:1908000,453.20974)
--(axis cs:1872000,450.689744)
--(axis cs:1836000,464.400156)
--(axis cs:1800000,455.43622)
--(axis cs:1764000,436.645174)
--(axis cs:1728000,467.097242)
--(axis cs:1692000,425.822572)
--(axis cs:1656000,392.302988)
--(axis cs:1620000,436.03456)
--(axis cs:1584000,448.712154)
--(axis cs:1548000,422.624314)
--(axis cs:1512000,439.734332)
--(axis cs:1476000,414.922382)
--(axis cs:1440000,413.322502)
--(axis cs:1404000,403.74151)
--(axis cs:1368000,415.533948)
--(axis cs:1332000,379.851818)
--(axis cs:1296000,388.446344)
--(axis cs:1260000,368.935652)
--(axis cs:1224000,360.67859)
--(axis cs:1188000,363.081264)
--(axis cs:1152000,345.39738)
--(axis cs:1116000,333.850172)
--(axis cs:1080000,329.383846)
--(axis cs:1044000,324.700968)
--(axis cs:1008000,308.274324)
--(axis cs:972000,299.78854)
--(axis cs:936000,295.739292)
--(axis cs:900000,288.248428)
--(axis cs:864000,276.068788)
--(axis cs:828000,259.09953)
--(axis cs:792000,224.213106)
--(axis cs:756000,252.733748)
--(axis cs:720000,229.692398)
--(axis cs:684000,233.0386)
--(axis cs:648000,230.912702)
--(axis cs:612000,212.104924)
--(axis cs:576000,194.70461)
--(axis cs:540000,196.214974)
--(axis cs:504000,178.323158)
--(axis cs:468000,170.9685)
--(axis cs:432000,155.249548)
--(axis cs:396000,145.59113)
--(axis cs:360000,136.910784)
--(axis cs:324000,127.9143828)
--(axis cs:288000,119.242298)
--(axis cs:252000,109.786947)
--(axis cs:216000,98.776286)
--(axis cs:180000,90.258806)
--(axis cs:144000,65.1756898)
--(axis cs:108000,34.58628592)
--(axis cs:72000,1.0745772)
--(axis cs:36000,0.84643344)
--(axis cs:1,0.9499512)
--cycle;

\addplot [line width=\linewidthdime, C4, mark=*, mark size=0, mark options={solid}]
table {%
1 0.79623116
36000 0.6654912
72000 0.93887308
108000 9.32506256
144000 56.9906742
180000 76.0816506
216000 89.618604
252000 100.486158
288000 109.4723688
324000 116.9026268
360000 126.1623
396000 131.5683448
432000 139.518233
468000 150.098634
504000 157.093544
540000 165.968566
576000 174.59932
612000 184.109946
648000 188.46891
684000 188.51764
720000 200.087412
756000 201.701998
792000 197.81462
828000 215.604316
864000 224.219048
900000 232.709118
936000 244.162464
972000 240.0626
1008000 255.169998
1044000 255.946374
1080000 270.746522
1116000 257.476808
1152000 271.693674
1188000 292.684574
1224000 289.728234
1260000 303.74535
1296000 299.5958
1332000 309.234742
1368000 329.518326
1404000 329.975408
1440000 333.764516
1476000 335.617264
1512000 350.108476
1548000 341.009436
1584000 363.33507
1620000 355.569656
1656000 329.712392
1692000 352.983986
1728000 385.058156
1764000 367.520394
1800000 363.345356
1836000 388.452402
1872000 381.692602
1908000 396.384682
1944000 390.14541
1980000 406.257904
2016000 407.03754
2052000 410.052536
2088000 399.707244
2124000 426.319902
2160000 418.085162
2196000 437.975366
2232000 438.7011
2268000 446.64954
2304000 426.10882
2340000 477.69212
2376000 476.455242
2412000 458.449334
2448000 466.127348
2484000 469.038356
2520000 481.244338
2556000 495.999414
2592000 452.954074
2628000 492.817316
2664000 498.786128
2700000 480.2397
2736000 481.934388
2772000 490.771756
2808000 511.724256
2844000 519.385736
2880000 520.927048
2916000 545.797786
2952000 524.560874
2988000 553.321684
};
\path [draw=C0, fill=C0, opacity=0.2]
(axis cs:1,0.915501966666667)
--(axis cs:1,0.633556866666667)
--(axis cs:36000,0.5493304)
--(axis cs:72000,0.843873266666667)
--(axis cs:108000,4.13696455)
--(axis cs:144000,36.1211871666667)
--(axis cs:180000,51.4064937333333)
--(axis cs:216000,81.368514)
--(axis cs:252000,97.5408251666667)
--(axis cs:288000,109.583571666667)
--(axis cs:324000,110.618288333333)
--(axis cs:360000,120.684184666667)
--(axis cs:396000,122.417020833333)
--(axis cs:432000,131.747461666667)
--(axis cs:468000,139.23245)
--(axis cs:504000,145.708203333333)
--(axis cs:540000,128.071794166667)
--(axis cs:576000,146.743565)
--(axis cs:612000,157.523090333333)
--(axis cs:648000,162.386513333333)
--(axis cs:684000,167.712168333333)
--(axis cs:720000,177.17324)
--(axis cs:756000,175.778066666667)
--(axis cs:792000,177.557476666667)
--(axis cs:828000,183.413371666667)
--(axis cs:864000,191.343878333333)
--(axis cs:900000,198.058078333333)
--(axis cs:936000,199.495351666667)
--(axis cs:972000,204.18452)
--(axis cs:1008000,211.017165)
--(axis cs:1044000,210.395921666667)
--(axis cs:1080000,225.37889)
--(axis cs:1116000,225.57669)
--(axis cs:1152000,229.055506666667)
--(axis cs:1188000,244.717408333333)
--(axis cs:1224000,235.69925925)
--(axis cs:1260000,246.761671666667)
--(axis cs:1296000,263.70817)
--(axis cs:1332000,261.655338333333)
--(axis cs:1368000,274.904988333333)
--(axis cs:1404000,281.871765)
--(axis cs:1440000,284.6845)
--(axis cs:1476000,298.373705)
--(axis cs:1512000,314.234433333333)
--(axis cs:1548000,302.586606666667)
--(axis cs:1584000,306.158113166667)
--(axis cs:1620000,329.89694)
--(axis cs:1656000,327.346265)
--(axis cs:1692000,331.977616666667)
--(axis cs:1728000,335.598585)
--(axis cs:1764000,337.5854)
--(axis cs:1800000,350.510931458333)
--(axis cs:1836000,348.138216666667)
--(axis cs:1872000,325.377726666667)
--(axis cs:1908000,375.064071666667)
--(axis cs:1944000,364.504996666667)
--(axis cs:1980000,362.576775)
--(axis cs:2016000,359.905746666667)
--(axis cs:2052000,401.565295)
--(axis cs:2088000,380.944233333333)
--(axis cs:2124000,408.07439)
--(axis cs:2160000,406.816443333333)
--(axis cs:2196000,402.540623333333)
--(axis cs:2232000,419.828846666667)
--(axis cs:2268000,419.407693333333)
--(axis cs:2304000,391.893025)
--(axis cs:2340000,426.646173333333)
--(axis cs:2376000,430.59703)
--(axis cs:2412000,440.944021375)
--(axis cs:2448000,444.535958333333)
--(axis cs:2484000,447.59097)
--(axis cs:2520000,458.220718125)
--(axis cs:2556000,454.987311666667)
--(axis cs:2592000,444.593778333333)
--(axis cs:2628000,452.355353333333)
--(axis cs:2664000,472.315071583333)
--(axis cs:2700000,468.552297125)
--(axis cs:2736000,425.893185)
--(axis cs:2772000,461.17882)
--(axis cs:2808000,438.974565)
--(axis cs:2844000,408.566183333333)
--(axis cs:2880000,440.493023333333)
--(axis cs:2916000,474.594651666667)
--(axis cs:2952000,473.13282)
--(axis cs:2988000,468.31128725)
--(axis cs:2988000,572.54452975)
--(axis cs:2988000,572.54452975)
--(axis cs:2952000,607.047773333333)
--(axis cs:2916000,581.955683333333)
--(axis cs:2880000,562.69891)
--(axis cs:2844000,570.399441666667)
--(axis cs:2808000,561.294171666667)
--(axis cs:2772000,575.424046666667)
--(axis cs:2736000,577.564579375)
--(axis cs:2700000,570.731558333333)
--(axis cs:2664000,556.645568333333)
--(axis cs:2628000,528.610636666667)
--(axis cs:2592000,572.906766666667)
--(axis cs:2556000,556.792031666667)
--(axis cs:2520000,565.548321666667)
--(axis cs:2484000,564.076643333333)
--(axis cs:2448000,554.428711666667)
--(axis cs:2412000,511.028458333333)
--(axis cs:2376000,539.327213333333)
--(axis cs:2340000,521.274179458333)
--(axis cs:2304000,507.011033333333)
--(axis cs:2268000,533.545403333333)
--(axis cs:2232000,530.436551666667)
--(axis cs:2196000,514.156165)
--(axis cs:2160000,548.6599)
--(axis cs:2124000,526.241701666667)
--(axis cs:2088000,488.449933333333)
--(axis cs:2052000,521.397975)
--(axis cs:2016000,489.724498333333)
--(axis cs:1980000,511.56874)
--(axis cs:1944000,469.79517)
--(axis cs:1908000,487.4035)
--(axis cs:1872000,452.261116666667)
--(axis cs:1836000,448.443288333333)
--(axis cs:1800000,470.013476541667)
--(axis cs:1764000,441.067136666667)
--(axis cs:1728000,425.983283333333)
--(axis cs:1692000,459.900376666667)
--(axis cs:1656000,420.555463333333)
--(axis cs:1620000,452.412428333333)
--(axis cs:1584000,426.080911666667)
--(axis cs:1548000,405.744391916667)
--(axis cs:1512000,422.468795)
--(axis cs:1476000,392.121763333333)
--(axis cs:1440000,398.927006666667)
--(axis cs:1404000,396.300881666667)
--(axis cs:1368000,391.708926666667)
--(axis cs:1332000,379.3367)
--(axis cs:1296000,374.028266666667)
--(axis cs:1260000,354.826833333333)
--(axis cs:1224000,325.77437)
--(axis cs:1188000,356.78896)
--(axis cs:1152000,352.524811666667)
--(axis cs:1116000,343.294016666667)
--(axis cs:1080000,331.116815)
--(axis cs:1044000,287.783165)
--(axis cs:1008000,319.938826666667)
--(axis cs:972000,297.00894)
--(axis cs:936000,297.43841)
--(axis cs:900000,295.160138333333)
--(axis cs:864000,278.186283333333)
--(axis cs:828000,267.117865)
--(axis cs:792000,254.04739775)
--(axis cs:756000,245.040613333333)
--(axis cs:720000,223.696456875)
--(axis cs:684000,216.1656)
--(axis cs:648000,205.401713333333)
--(axis cs:612000,201.404601666667)
--(axis cs:576000,182.55503)
--(axis cs:540000,177.786793333333)
--(axis cs:504000,170.561265208333)
--(axis cs:468000,170.450261666667)
--(axis cs:432000,164.02998)
--(axis cs:396000,153.633991666667)
--(axis cs:360000,143.37209)
--(axis cs:324000,136.525775833333)
--(axis cs:288000,129.716861683333)
--(axis cs:252000,113.1404125)
--(axis cs:216000,102.49002)
--(axis cs:180000,92.9197603333333)
--(axis cs:144000,76.8890066666667)
--(axis cs:108000,56.7536433333333)
--(axis cs:72000,4.25857906666667)
--(axis cs:36000,0.728938933333333)
--(axis cs:1,0.915501966666667)
--cycle;

\addplot [line width=\linewidthdime, C0, mark=*, mark size=0, mark options={solid}]
table {%
1 0.770562166666667
36000 0.658370533333333
72000 1.02744053333333
108000 27.4132359166667
144000 61.7389005
180000 78.3388245
216000 96.6475441666667
252000 104.292225666667
288000 118.336289333333
324000 122.156900833333
360000 127.865645666667
396000 135.589655833333
432000 144.61231
468000 150.876916666667
504000 155.106371666667
540000 161.848961666667
576000 162.873085
612000 175.66877
648000 181.774033333333
684000 189.795443333333
720000 201.694543333333
756000 209.679653333333
792000 215.950858333333
828000 226.452128333333
864000 235.790758333333
900000 245.745095
936000 251.419853333333
972000 253.923228333333
1008000 267.745066666667
1044000 253.029908333333
1080000 281.1757
1116000 290.478148333333
1152000 295.386573333333
1188000 307.510898333333
1224000 279.324226666667
1260000 298.499671666667
1296000 320.530285
1332000 314.688511666667
1368000 333.564955
1404000 344.035705
1440000 351.026203333333
1476000 334.48205
1512000 371.517083333333
1548000 359.107073333333
1584000 362.49581
1620000 383.735268333333
1656000 371.120853333333
1692000 392.875408333333
1728000 375.706403333333
1764000 385.148411666667
1800000 411.377563333333
1836000 393.022041666667
1872000 381.083618333333
1908000 428.514751666667
1944000 407.976545
1980000 440.522148333333
2016000 425.096363333333
2052000 460.246375
2088000 439.221346666667
2124000 470.482476666667
2160000 479.806223333333
2196000 467.028006666667
2232000 471.833656666667
2268000 492.305195
2304000 444.283438333333
2340000 467.734686666667
2376000 478.305563333333
2412000 479.659883333333
2448000 502.891963333333
2484000 512.254108333333
2520000 504.927955
2556000 509.89631
2592000 517.465808333333
2628000 505.843831666667
2664000 503.219516666667
2700000 525.334166666667
2736000 498.718568333333
2772000 515.273173333333
2808000 507.004858333333
2844000 490.845221666667
2880000 488.33129
2916000 530.288801666667
2952000 544.121438333333
2988000 506.351106666667
};
\path [draw=C3, fill=C3, opacity=0.2]
(axis cs:1,0.928116433333333)
--(axis cs:1,0.643498933333333)
--(axis cs:36000,0.547453233333333)
--(axis cs:72000,0.7631748)
--(axis cs:108000,1.27857673333333)
--(axis cs:144000,18.0771651666667)
--(axis cs:180000,51.1960535)
--(axis cs:216000,83.2625346666667)
--(axis cs:252000,94.8146593333333)
--(axis cs:288000,102.66881)
--(axis cs:324000,108.0410345)
--(axis cs:360000,114.348600833333)
--(axis cs:396000,122.733859883333)
--(axis cs:432000,123.892253183333)
--(axis cs:468000,131.722243333333)
--(axis cs:504000,133.537971666667)
--(axis cs:540000,139.154160833333)
--(axis cs:576000,139.382015)
--(axis cs:612000,145.454221666667)
--(axis cs:648000,148.812428333333)
--(axis cs:684000,155.052648333333)
--(axis cs:720000,156.984706666667)
--(axis cs:756000,159.69884)
--(axis cs:792000,157.545298333333)
--(axis cs:828000,168.235955)
--(axis cs:864000,167.973465416667)
--(axis cs:900000,175.227398333333)
--(axis cs:936000,179.510641666667)
--(axis cs:972000,177.686416666667)
--(axis cs:1008000,176.277466666667)
--(axis cs:1044000,187.905873333333)
--(axis cs:1080000,185.12539)
--(axis cs:1116000,189.15055)
--(axis cs:1152000,194.68893)
--(axis cs:1188000,202.541036666667)
--(axis cs:1224000,199.573218333333)
--(axis cs:1260000,206.027295)
--(axis cs:1296000,212.29623)
--(axis cs:1332000,217.7664935)
--(axis cs:1368000,221.254836666667)
--(axis cs:1404000,225.306431666667)
--(axis cs:1440000,219.931678333333)
--(axis cs:1476000,232.818085)
--(axis cs:1512000,233.127253333333)
--(axis cs:1548000,231.869707083333)
--(axis cs:1584000,237.394488333333)
--(axis cs:1620000,241.367248333333)
--(axis cs:1656000,240.82232325)
--(axis cs:1692000,248.902995)
--(axis cs:1728000,246.63818)
--(axis cs:1764000,255.807626666667)
--(axis cs:1800000,264.121821666667)
--(axis cs:1836000,261.530318333333)
--(axis cs:1872000,270.118195)
--(axis cs:1908000,265.353296666667)
--(axis cs:1944000,279.52212)
--(axis cs:1980000,278.518683333333)
--(axis cs:2016000,291.372876666667)
--(axis cs:2052000,265.743866666667)
--(axis cs:2088000,292.652946666667)
--(axis cs:2124000,283.64312)
--(axis cs:2160000,285.612521375)
--(axis cs:2196000,285.773671375)
--(axis cs:2232000,302.724083333333)
--(axis cs:2268000,299.02875)
--(axis cs:2304000,305.294261666667)
--(axis cs:2340000,305.441015)
--(axis cs:2376000,306.32539)
--(axis cs:2412000,320.358673333333)
--(axis cs:2448000,318.710518333333)
--(axis cs:2484000,306.148623333333)
--(axis cs:2520000,325.304713875)
--(axis cs:2556000,323.086981666667)
--(axis cs:2592000,331.5614)
--(axis cs:2628000,328.16837)
--(axis cs:2664000,333.195741666667)
--(axis cs:2700000,343.656018333333)
--(axis cs:2736000,350.113323333333)
--(axis cs:2772000,342.653121625)
--(axis cs:2808000,335.79802725)
--(axis cs:2844000,341.467496666667)
--(axis cs:2880000,352.397085583333)
--(axis cs:2916000,346.576765458334)
--(axis cs:2952000,367.973366666667)
--(axis cs:2988000,378.375255)
--(axis cs:2988000,534.717091541667)
--(axis cs:2988000,534.717091541667)
--(axis cs:2952000,512.60975)
--(axis cs:2916000,498.056816666667)
--(axis cs:2880000,520.221383333333)
--(axis cs:2844000,505.423818333333)
--(axis cs:2808000,499.364251666667)
--(axis cs:2772000,503.720745)
--(axis cs:2736000,519.617738875)
--(axis cs:2700000,522.095011666667)
--(axis cs:2664000,505.989085)
--(axis cs:2628000,494.306316666667)
--(axis cs:2592000,457.269068333333)
--(axis cs:2556000,495.22476)
--(axis cs:2520000,485.463546666667)
--(axis cs:2484000,468.292898333333)
--(axis cs:2448000,509.16633)
--(axis cs:2412000,467.284303333333)
--(axis cs:2376000,448.310605)
--(axis cs:2340000,465.57693)
--(axis cs:2304000,463.837011666667)
--(axis cs:2268000,467.645341666667)
--(axis cs:2232000,446.958516666667)
--(axis cs:2196000,421.098828333333)
--(axis cs:2160000,453.863033333333)
--(axis cs:2124000,448.96107)
--(axis cs:2088000,433.017278333333)
--(axis cs:2052000,405.473736666667)
--(axis cs:2016000,428.690563333333)
--(axis cs:1980000,420.29339)
--(axis cs:1944000,420.167988333333)
--(axis cs:1908000,394.084698333333)
--(axis cs:1872000,382.843931666667)
--(axis cs:1836000,380.243036666667)
--(axis cs:1800000,378.436626666667)
--(axis cs:1764000,361.733458333333)
--(axis cs:1728000,370.615525)
--(axis cs:1692000,356.424053333333)
--(axis cs:1656000,348.4609)
--(axis cs:1620000,347.345446666667)
--(axis cs:1584000,341.972938333333)
--(axis cs:1548000,326.682701958333)
--(axis cs:1512000,315.043398333333)
--(axis cs:1476000,333.227661666667)
--(axis cs:1440000,306.672516666667)
--(axis cs:1404000,295.06333)
--(axis cs:1368000,301.566628333333)
--(axis cs:1332000,295.629256666667)
--(axis cs:1296000,282.03694)
--(axis cs:1260000,279.7165)
--(axis cs:1224000,261.036903333333)
--(axis cs:1188000,250.65403)
--(axis cs:1152000,240.84385)
--(axis cs:1116000,234.843987833333)
--(axis cs:1080000,223.780993333333)
--(axis cs:1044000,223.898048333333)
--(axis cs:1008000,216.471291666667)
--(axis cs:972000,215.021168333333)
--(axis cs:936000,208.009106666667)
--(axis cs:900000,202.766275)
--(axis cs:864000,195.407346666667)
--(axis cs:828000,193.874763333333)
--(axis cs:792000,191.364301666667)
--(axis cs:756000,187.788975)
--(axis cs:720000,181.462628333333)
--(axis cs:684000,175.794416666667)
--(axis cs:648000,174.354433333333)
--(axis cs:612000,168.967884208333)
--(axis cs:576000,164.405495)
--(axis cs:540000,162.083313333333)
--(axis cs:504000,155.813828333333)
--(axis cs:468000,153.955994458333)
--(axis cs:432000,150.731388333333)
--(axis cs:396000,144.817186666667)
--(axis cs:360000,136.891413333333)
--(axis cs:324000,131.662519166667)
--(axis cs:288000,125.109144)
--(axis cs:252000,119.23944)
--(axis cs:216000,104.525315)
--(axis cs:180000,85.7092318333333)
--(axis cs:144000,63.4865036666667)
--(axis cs:108000,31.2898350666667)
--(axis cs:72000,1.09303543333333)
--(axis cs:36000,0.974526366666666)
--(axis cs:1,0.928116433333333)
--cycle;

\addplot [line width=\linewidthdime, C3, mark=*, mark size=0, mark options={solid}]
table {%
1 0.776568966666667
36000 0.734329766666667
72000 0.962220433333333
108000 7.34097936666667
144000 42.8375991666667
180000 72.6396718333333
216000 93.9126996666667
252000 108.946717666667
288000 114.4714665
324000 121.5707875
360000 127.114203166667
396000 135.459174
432000 139.715702333333
468000 143.396966666667
504000 147.691935
540000 153.80647
576000 154.846155
612000 159.297523333333
648000 165.48207
684000 167.043441666667
720000 172.194396666667
756000 178.59695
792000 175.081208333333
828000 182.006455
864000 186.214628333333
900000 192.088711666667
936000 196.880713333333
972000 197.742606666667
1008000 192.003625
1044000 208.656036666667
1080000 204.470508333333
1116000 212.19196
1152000 218.894098333333
1188000 229.829843333333
1224000 228.670846666667
1260000 242.780746666667
1296000 252.607206666667
1332000 261.518213333333
1368000 267.413625
1404000 270.713213333333
1440000 266.29257
1476000 283.693373333333
1512000 284.858258333333
1548000 284.222778333333
1584000 296.784831666667
1620000 299.85238
1656000 296.072308333333
1692000 305.607678333333
1728000 314.54406
1764000 310.750091666667
1800000 334.001045
1836000 329.478816666667
1872000 337.215008333333
1908000 335.525573333333
1944000 350.877255
1980000 357.808623333333
2016000 376.93922
2052000 332.013935
2088000 372.723451666667
2124000 371.845281666667
2160000 371.61298
2196000 351.251731666667
2232000 378.468386666667
2268000 381.518951666667
2304000 383.504851666667
2340000 389.339608333333
2376000 380.56536
2412000 399.967506666667
2448000 416.27761
2484000 393.52605
2520000 409.857638333333
2556000 402.593678333333
2592000 402.324781666667
2628000 419.952185
2664000 421.262773333333
2700000 439.356176666667
2736000 441.041503333333
2772000 420.4694
2808000 412.892056666667
2844000 405.399628333333
2880000 440.82055
2916000 423.547295
2952000 430.649928333333
2988000 454.255031666667
};
\path [draw=C1, fill=C1, opacity=0.2]
(axis cs:1,0.947906)
--(axis cs:1,0.647919966666667)
--(axis cs:36000,0.6789499)
--(axis cs:72000,0.793110466666667)
--(axis cs:108000,3.3914974)
--(axis cs:144000,54.3676516666667)
--(axis cs:180000,59.421407025)
--(axis cs:216000,50.2593926183334)
--(axis cs:252000,83.4890564)
--(axis cs:288000,101.594896)
--(axis cs:324000,108.548579666667)
--(axis cs:360000,118.1397425)
--(axis cs:396000,123.787584333333)
--(axis cs:432000,128.593012333333)
--(axis cs:468000,131.158335)
--(axis cs:504000,133.8391835)
--(axis cs:540000,136.66295425)
--(axis cs:576000,142.638431666667)
--(axis cs:612000,147.749983333333)
--(axis cs:648000,150.5067)
--(axis cs:684000,155.834255291667)
--(axis cs:720000,159.892413333333)
--(axis cs:756000,161.190681666667)
--(axis cs:792000,166.696113333333)
--(axis cs:828000,166.063931666667)
--(axis cs:864000,169.293395)
--(axis cs:900000,170.210048333333)
--(axis cs:936000,176.23217)
--(axis cs:972000,177.053335)
--(axis cs:1008000,179.68822)
--(axis cs:1044000,180.989955)
--(axis cs:1080000,185.627346666667)
--(axis cs:1116000,184.800308333333)
--(axis cs:1152000,189.835196666667)
--(axis cs:1188000,189.993175)
--(axis cs:1224000,195.727295)
--(axis cs:1260000,198.3563)
--(axis cs:1296000,204.50686)
--(axis cs:1332000,206.026186666667)
--(axis cs:1368000,207.690993333333)
--(axis cs:1404000,213.770815)
--(axis cs:1440000,216.53558)
--(axis cs:1476000,221.17538)
--(axis cs:1512000,225.396866666667)
--(axis cs:1548000,223.29085)
--(axis cs:1584000,226.190543333333)
--(axis cs:1620000,229.45524)
--(axis cs:1656000,240.547594125)
--(axis cs:1692000,237.377356666667)
--(axis cs:1728000,244.92856)
--(axis cs:1764000,241.331201083333)
--(axis cs:1800000,248.926193333333)
--(axis cs:1836000,250.126581666667)
--(axis cs:1872000,243.286126666667)
--(axis cs:1908000,260.144156666667)
--(axis cs:1944000,264.149852041667)
--(axis cs:1980000,261.049196666667)
--(axis cs:2016000,269.475068333333)
--(axis cs:2052000,263.147093333333)
--(axis cs:2088000,271.862876666667)
--(axis cs:2124000,281.901515)
--(axis cs:2160000,288.305405)
--(axis cs:2196000,289.735166666667)
--(axis cs:2232000,284.797276958333)
--(axis cs:2268000,290.85063)
--(axis cs:2304000,301.895263333333)
--(axis cs:2340000,295.982725)
--(axis cs:2376000,305.697658333333)
--(axis cs:2412000,297.539705)
--(axis cs:2448000,311.789013333333)
--(axis cs:2484000,300.282626666667)
--(axis cs:2520000,312.158031666667)
--(axis cs:2556000,317.706951666667)
--(axis cs:2592000,322.548186666667)
--(axis cs:2628000,329.208981666667)
--(axis cs:2664000,321.599713166667)
--(axis cs:2700000,338.786073333333)
--(axis cs:2736000,325.001185)
--(axis cs:2772000,351.621495)
--(axis cs:2808000,343.826384166667)
--(axis cs:2844000,348.149995)
--(axis cs:2880000,351.833068208333)
--(axis cs:2916000,351.185778333333)
--(axis cs:2952000,367.294763333333)
--(axis cs:2988000,368.715268333333)
--(axis cs:2988000,512.873348333333)
--(axis cs:2988000,512.873348333333)
--(axis cs:2952000,508.561191666667)
--(axis cs:2916000,476.989976666667)
--(axis cs:2880000,515.970535)
--(axis cs:2844000,504.020386666667)
--(axis cs:2808000,497.976403333333)
--(axis cs:2772000,487.081301666667)
--(axis cs:2736000,500.454253333333)
--(axis cs:2700000,469.78942)
--(axis cs:2664000,495.601416666667)
--(axis cs:2628000,490.355298333333)
--(axis cs:2592000,475.792852125001)
--(axis cs:2556000,496.105671666667)
--(axis cs:2520000,478.902593333333)
--(axis cs:2484000,476.11038)
--(axis cs:2448000,460.190753333333)
--(axis cs:2412000,467.451103333333)
--(axis cs:2376000,477.712076666667)
--(axis cs:2340000,450.359163333333)
--(axis cs:2304000,444.528726666667)
--(axis cs:2268000,467.444493333333)
--(axis cs:2232000,402.830081666667)
--(axis cs:2196000,443.08806)
--(axis cs:2160000,447.416013333333)
--(axis cs:2124000,421.195019)
--(axis cs:2088000,421.813341666667)
--(axis cs:2052000,410.999588333333)
--(axis cs:2016000,427.116483333333)
--(axis cs:1980000,410.448768333333)
--(axis cs:1944000,398.94141)
--(axis cs:1908000,408.468751666667)
--(axis cs:1872000,409.573751666667)
--(axis cs:1836000,407.032951666667)
--(axis cs:1800000,399.665885)
--(axis cs:1764000,395.969985)
--(axis cs:1728000,389.694553333333)
--(axis cs:1692000,367.71215)
--(axis cs:1656000,355.726396666667)
--(axis cs:1620000,369.686211666667)
--(axis cs:1584000,343.331725)
--(axis cs:1548000,368.540093333333)
--(axis cs:1512000,355.983961666667)
--(axis cs:1476000,339.508708333333)
--(axis cs:1440000,343.557675)
--(axis cs:1404000,323.419966666667)
--(axis cs:1368000,297.995505)
--(axis cs:1332000,310.387591666667)
--(axis cs:1296000,294.85691)
--(axis cs:1260000,292.945816666667)
--(axis cs:1224000,302.347091666667)
--(axis cs:1188000,274.295463333333)
--(axis cs:1152000,260.324991666667)
--(axis cs:1116000,259.16028)
--(axis cs:1080000,247.237515)
--(axis cs:1044000,238.244779708334)
--(axis cs:1008000,236.47276)
--(axis cs:972000,227.938841666667)
--(axis cs:936000,211.370701625)
--(axis cs:900000,205.267756666667)
--(axis cs:864000,192.81913)
--(axis cs:828000,193.648363333333)
--(axis cs:792000,185.810245)
--(axis cs:756000,181.553681666667)
--(axis cs:720000,181.732958333333)
--(axis cs:684000,172.869191666667)
--(axis cs:648000,168.037535)
--(axis cs:612000,165.799705)
--(axis cs:576000,160.297268333333)
--(axis cs:540000,157.56922)
--(axis cs:504000,150.323818333333)
--(axis cs:468000,151.13554)
--(axis cs:432000,144.710538333333)
--(axis cs:396000,138.693163333333)
--(axis cs:360000,133.396128333333)
--(axis cs:324000,127.979688333333)
--(axis cs:288000,122.242511)
--(axis cs:252000,119.632424166667)
--(axis cs:216000,105.059850333333)
--(axis cs:180000,96.6378656666667)
--(axis cs:144000,86.5932858333333)
--(axis cs:108000,58.2570568333333)
--(axis cs:72000,16.6007818279167)
--(axis cs:36000,1.081634975)
--(axis cs:1,0.947906)
--cycle;

\addplot [line width=\linewidthdime, C1, mark=*, mark size=0, mark options={solid}]
table {%
1 0.793080833333333
36000 0.883294666666667
72000 4.76931288333333
108000 28.3824999
144000 75.2608941666667
180000 86.001534
216000 96.207304
252000 106.762967666667
288000 114.760952
324000 120.907327333333
360000 127.669745833333
396000 133.548691666667
432000 137.759246666667
468000 141.729806666667
504000 144.148366666667
540000 146.592538333333
576000 150.987151666667
612000 154.977831666667
648000 157.027331666667
684000 161.639751666667
720000 168.854493333333
756000 167.093043333333
792000 174.413931666667
828000 174.56231
864000 178.588085
900000 182.39034
936000 190.79475
972000 199.272393333333
1008000 205.989418333333
1044000 204.515688333333
1080000 210.423976666667
1116000 214.800743333333
1152000 222.188451666667
1188000 225.951475
1224000 245.590386666667
1260000 237.858665
1296000 248.191033333333
1332000 255.232883333333
1368000 250.601725
1404000 267.381861666667
1440000 278.084538333333
1476000 279.528716666667
1512000 284.331836666667
1548000 296.29661
1584000 284.514331666667
1620000 301.927791666667
1656000 295.327101666667
1692000 301.619025
1728000 312.756025
1764000 320.033821666667
1800000 325.079148333333
1836000 328.638596666667
1872000 319.262848333333
1908000 339.451468333333
1944000 333.398783333333
1980000 336.603696666667
2016000 350.79915
2052000 338.051095
2088000 347.224291666667
2124000 359.089831666667
2160000 375.549081666667
2196000 370.686981666667
2232000 341.989458333333
2268000 380.5488
2304000 377.628255
2340000 374.424245
2376000 404.878603333333
2412000 391.268576666667
2448000 385.288208333333
2484000 391.828935
2520000 392.86411
2556000 410.2523
2592000 401.955721666667
2628000 401.312245
2664000 410.0424
2700000 400.1286
2736000 412.713958333333
2772000 425.748595
2808000 423.772948333333
2844000 426.526341666667
2880000 434.080936666667
2916000 413.606458333333
2952000 444.294001666667
2988000 436.393723333333
};
\path [draw=C5, fill=C5, opacity=0.2]
(axis cs:1,0.940732166666667)
--(axis cs:1,0.6478003)
--(axis cs:36000,0.708945333333333)
--(axis cs:72000,0.882928033333333)
--(axis cs:108000,11.823371285)
--(axis cs:144000,57.7883219)
--(axis cs:180000,71.69730555)
--(axis cs:216000,85.7927391916667)
--(axis cs:252000,96.3296600541667)
--(axis cs:288000,102.735528333333)
--(axis cs:324000,110.211674333333)
--(axis cs:360000,116.709841666667)
--(axis cs:396000,119.626125541667)
--(axis cs:432000,123.4610645)
--(axis cs:468000,128.330138333333)
--(axis cs:504000,131.901021666667)
--(axis cs:540000,132.850896)
--(axis cs:576000,138.2377025)
--(axis cs:612000,139.964616666667)
--(axis cs:648000,146.691756666667)
--(axis cs:684000,151.5423)
--(axis cs:720000,153.105810833333)
--(axis cs:756000,157.36603)
--(axis cs:792000,155.616353333333)
--(axis cs:828000,158.803768333333)
--(axis cs:864000,161.035038333333)
--(axis cs:900000,164.42601)
--(axis cs:936000,168.888856666667)
--(axis cs:972000,172.766408333333)
--(axis cs:1008000,168.310463333333)
--(axis cs:1044000,173.31378)
--(axis cs:1080000,171.254566666667)
--(axis cs:1116000,175.028179416667)
--(axis cs:1152000,177.285683333333)
--(axis cs:1188000,178.981825)
--(axis cs:1224000,185.745676666667)
--(axis cs:1260000,189.849713333333)
--(axis cs:1296000,190.37594)
--(axis cs:1332000,194.85043)
--(axis cs:1368000,197.89764)
--(axis cs:1404000,199.535053333333)
--(axis cs:1440000,204.646566666667)
--(axis cs:1476000,208.745936666667)
--(axis cs:1512000,212.24775)
--(axis cs:1548000,212.720858333333)
--(axis cs:1584000,215.417915)
--(axis cs:1620000,213.527841666667)
--(axis cs:1656000,217.948634)
--(axis cs:1692000,220.525628333333)
--(axis cs:1728000,221.086986666667)
--(axis cs:1764000,222.21299)
--(axis cs:1800000,228.542551666667)
--(axis cs:1836000,230.754473333333)
--(axis cs:1872000,229.862291666667)
--(axis cs:1908000,241.493213333333)
--(axis cs:1944000,236.139893333333)
--(axis cs:1980000,235.807528333333)
--(axis cs:2016000,246.207333333333)
--(axis cs:2052000,255.321871666667)
--(axis cs:2088000,252.503613333333)
--(axis cs:2124000,255.024761666667)
--(axis cs:2160000,256.944075041667)
--(axis cs:2196000,265.224298333333)
--(axis cs:2232000,264.922091666667)
--(axis cs:2268000,264.649315)
--(axis cs:2304000,269.022133333333)
--(axis cs:2340000,276.488578041667)
--(axis cs:2376000,274.54996)
--(axis cs:2412000,278.63976)
--(axis cs:2448000,286.322148333333)
--(axis cs:2484000,298.044431666667)
--(axis cs:2520000,286.347748333333)
--(axis cs:2556000,296.236086666667)
--(axis cs:2592000,300.41447)
--(axis cs:2628000,298.691346666667)
--(axis cs:2664000,303.4108365)
--(axis cs:2700000,309.891633333333)
--(axis cs:2736000,308.441696666667)
--(axis cs:2772000,298.042187833333)
--(axis cs:2808000,308.386563333333)
--(axis cs:2844000,318.800403333333)
--(axis cs:2880000,327.243005)
--(axis cs:2916000,321.896865583334)
--(axis cs:2952000,320.295991666667)
--(axis cs:2988000,317.424581666667)
--(axis cs:2988000,429.789356916667)
--(axis cs:2988000,429.789356916667)
--(axis cs:2952000,431.420461666667)
--(axis cs:2916000,426.353536666667)
--(axis cs:2880000,435.001556666667)
--(axis cs:2844000,425.128552291667)
--(axis cs:2808000,422.1013)
--(axis cs:2772000,398.909616666667)
--(axis cs:2736000,410.803781666667)
--(axis cs:2700000,402.035226666667)
--(axis cs:2664000,394.793155458334)
--(axis cs:2628000,382.885743333333)
--(axis cs:2592000,389.94345)
--(axis cs:2556000,381.754713333333)
--(axis cs:2520000,376.490486666667)
--(axis cs:2484000,381.033273333333)
--(axis cs:2448000,374.924123333333)
--(axis cs:2412000,379.739946666667)
--(axis cs:2376000,357.32582)
--(axis cs:2340000,366.917633333333)
--(axis cs:2304000,345.256439083333)
--(axis cs:2268000,355.820071666667)
--(axis cs:2232000,341.712815)
--(axis cs:2196000,342.297211666667)
--(axis cs:2160000,336.452855)
--(axis cs:2124000,317.727765)
--(axis cs:2088000,333.013816666667)
--(axis cs:2052000,317.405666666667)
--(axis cs:2016000,314.319478333333)
--(axis cs:1980000,305.25091)
--(axis cs:1944000,303.071486666667)
--(axis cs:1908000,302.607655)
--(axis cs:1872000,293.173236666667)
--(axis cs:1836000,298.205565)
--(axis cs:1800000,282.16077)
--(axis cs:1764000,276.825536666667)
--(axis cs:1728000,274.301011666667)
--(axis cs:1692000,269.110773333333)
--(axis cs:1656000,251.612386666667)
--(axis cs:1620000,253.01692)
--(axis cs:1584000,255.715893333333)
--(axis cs:1548000,250.008147291667)
--(axis cs:1512000,254.062308333333)
--(axis cs:1476000,243.327675)
--(axis cs:1440000,235.86794)
--(axis cs:1404000,230.07126)
--(axis cs:1368000,225.080308333333)
--(axis cs:1332000,221.752435)
--(axis cs:1296000,217.387876666667)
--(axis cs:1260000,214.103885)
--(axis cs:1224000,210.257248333333)
--(axis cs:1188000,202.33855)
--(axis cs:1152000,203.467211666667)
--(axis cs:1116000,202.475293333333)
--(axis cs:1080000,197.079883333333)
--(axis cs:1044000,195.578943958333)
--(axis cs:1008000,193.111286666667)
--(axis cs:972000,193.089343333333)
--(axis cs:936000,188.631041666667)
--(axis cs:900000,185.898528333333)
--(axis cs:864000,183.699506666667)
--(axis cs:828000,179.26189)
--(axis cs:792000,177.424506666667)
--(axis cs:756000,176.002111666667)
--(axis cs:720000,173.739416666667)
--(axis cs:684000,171.615881125)
--(axis cs:648000,167.823943333333)
--(axis cs:612000,162.854888333333)
--(axis cs:576000,158.918571666667)
--(axis cs:540000,154.972273333333)
--(axis cs:504000,153.733941666667)
--(axis cs:468000,149.992083333333)
--(axis cs:432000,144.521545)
--(axis cs:396000,138.514621666667)
--(axis cs:360000,135.868858333333)
--(axis cs:324000,127.512896983333)
--(axis cs:288000,126.704393875)
--(axis cs:252000,120.517828333333)
--(axis cs:216000,111.397960166667)
--(axis cs:180000,100.25853)
--(axis cs:144000,84.2088058333333)
--(axis cs:108000,55.1225927250001)
--(axis cs:72000,19.64539359875)
--(axis cs:36000,1.02158851666667)
--(axis cs:1,0.940732166666667)
--cycle;

\addplot [line width=\linewidthdime, C5, mark=*, mark size=0, mark options={solid}]
table {%
1 0.7925715
36000 0.855803933333333
72000 5.60396411666667
108000 33.0914051166667
144000 74.8232375
180000 92.4339383333333
216000 100.283691833333
252000 109.755798333333
288000 115.849478333333
324000 120.014642666667
360000 127.528121666667
396000 131.023171666667
432000 136.990835
468000 141.091798333333
504000 145.962661666667
540000 146.876438333333
576000 150.5725
612000 152.519965
648000 158.898496666667
684000 162.630385
720000 164.797036666667
756000 167.922646666667
792000 168.24887
828000 170.044915
864000 173.468905
900000 175.792385
936000 180.013041666667
972000 183.187281666667
1008000 182.257923333333
1044000 184.304235
1080000 183.416123333333
1116000 186.710705
1152000 186.99084
1188000 189.357188333333
1224000 195.45467
1260000 199.723523333333
1296000 202.634608333333
1332000 202.059755
1368000 208.158768333333
1404000 211.770758333333
1440000 215.636181666667
1476000 222.878325
1512000 229.329966666667
1548000 227.988351666667
1584000 228.8477
1620000 229.294536666667
1656000 234.159818333333
1692000 244.671071666667
1728000 242.68265
1764000 243.482716666667
1800000 254.129886666667
1836000 257.460956666667
1872000 258.341155
1908000 262.871326666667
1944000 261.509051666667
1980000 263.556001666667
2016000 274.77184
2052000 280.683546666667
2088000 285.92951
2124000 281.301891666667
2160000 295.392713333333
2196000 300.78979
2232000 302.263681666667
2268000 302.935865
2304000 296.721758333333
2340000 315.120236666667
2376000 314.844785
2412000 325.695131666667
2448000 331.553065
2484000 344.289838333333
2520000 331.968036666667
2556000 340.519526666667
2592000 347.327713333333
2628000 348.39166
2664000 338.487821666667
2700000 352.817998333333
2736000 355.928931666667
2772000 347.775408333333
2808000 362.823151666667
2844000 378.960925
2880000 387.890165
2916000 380.993608333333
2952000 378.341203333333
2988000 375.766641666667
};

\end{axis}

\end{tikzpicture}
}
       \subcaption[]{Varying the Diffusion Steps}
       \label{fig::exps_new_ablations_vary_diff_steps::humanoid_run}
    \end{minipage}\hfill
    \begin{minipage}[b]{0.256\textwidth}
        \centering
       \resizebox{0.9\textwidth}{!}{\subsection{Runtime}\label{sec:runtime_results}
\begin{figure}[t]
    \centering
    \includegraphics[width=\linewidth]{figures/total.pdf}
    \caption{End-to-end wall clock runtimes.
    Each data point is the average of $3$ trials.
    The error bars represent the standard deviations.
    Experiments terminate benchmarks when the estimated tensor contraction runtime exceeds $10^3$ seconds.}
    \label{fig:total_runtime}
\end{figure}

\begin{figure}[t]
    \centering
    \includegraphics[width=\linewidth]{figures/total_estimation.pdf}
    \caption{End-to-end runtime estimations using Equation~\ref{eq:cut_objective} for the same experiments in Figure~\ref{fig:total_runtime}.
    The estimated runtimes closely match with experimental data.}
    \label{fig:total_runtime_estimation}
\end{figure}

\begin{figure}[t]
    \centering
    \includegraphics[width=\linewidth]{figures/supremacy_breakdown.pdf}
    \caption{The runtime breakdown for the $Supremacy$ benchmarks.}
    \label{fig:runtime_breakdown}
\end{figure}

Figure~\ref{fig:total_runtime} shows the end-to-end wall clock runtime of computing $1$ million states for various benchmarks.
Experiments limit each benchmark circuit to at most half the number of qubits and gates in the uncut benchmark.
Note that these subcircuit size limits are upper bounds used in searching for cuts,
rather than the max subcircuit size produced from cutting.
Figure~\ref{fig:quantum_area} shows the actual quantum area found from cutting for the same set of experiments.

Figure~\ref{fig:total_runtime_estimation} shows the end-to-end runtime estimations obtained from Equation~\ref{eq:cut_objective}.
The estimations predict the tensor network contraction times using the heuristics number of floating point operations obtained from tensor network compilation,
divided by the backend GPU FLOPs.
The cut searching and QPU runtime estimations are the same as Figure~\ref{fig:total_runtime}.
Hence, estimations using Equation~\ref{eq:cut_objective} tend to overestimate the actual wall clock runtime,
but accurately reflect the trends.

Figure~\ref{fig:runtime_breakdown} shows the wall clock and estimation runtime breakdown of the $Supremacy$ benchmarks.
The inaccuracy in the runtime predictions is from the estimated time to contract subcircuit tensor networks.
Equation~\ref{eq:classical_runtime} consistently overestimates the GPU runtime.
We conjecture that more accurate estimations further require a holistic profiling of tensor network contraction on specific backend GPUs.
In addition, our cut searching algorithm~\ref{alg:graph_growing} finds cut solutions that balance between the quantum and classical runtimes,
showcasing its ability to find sweet tradeoff points.
Despite the efficiency of the heuristics cut search algorithm,
cuts finding still takes a significant portion of the end-to-end runtimes.}
       \subcaption[]{Runtime for 1M Steps}
       \label{fig::exps_new_ablations_runtime_diff_steps::humanoid_run}
    \end{minipage}\hfill
    \begin{minipage}[b]{0.24\textwidth}
        \centering
       \resizebox{1\textwidth}{!}{% This file was created with tikzplotlib v0.10.1.
\begin{tikzpicture}

\definecolor{darkcyan1115178}{RGB}{1,115,178}
\definecolor{darkgray176}{RGB}{176,176,176}

\begin{axis}[
legend cell align={left},
legend style={fill opacity=0.8, draw opacity=1, text opacity=1, draw=lightgray204, at={(0.03,0.03)},  anchor=north west},
tick align=outside,
tick pos=left,
x grid style={white},
xlabel={Number Env Interactions},
xmajorgrids,
xmin=-0.0, xmax=1000000.0,
xtick style={color=black},
y grid style={white},
ylabel={IQM Mean Return},
ymajorgrids,
ymin=-200, ymax=7500,
ytick style={color=black},
axis background/.style={fill=plot_background},
label style={font=\large},
tick label style={font=\large},
x axis line style={draw=none},
y axis line style={draw=none},
]


% QSM
\path [draw=C3, fill=C3, opacity=0.2]
(axis cs:10000,-29.7961098512014)
--(axis cs:10000,-49.9186350504557)
--(axis cs:30000,173.161663770676)
--(axis cs:50000,279.029960433642)
--(axis cs:70000,266.3480861485)
--(axis cs:90000,397.152654970686)
--(axis cs:110000,487.117878958272)
--(axis cs:130000,617.187458509424)
--(axis cs:150000,567.89921937483)
--(axis cs:170000,574.119107866922)
--(axis cs:190000,573.502918755166)
--(axis cs:210000,876.276257987737)
--(axis cs:230000,608.749484391587)
--(axis cs:250000,629.103305431766)
--(axis cs:270000,419.620677564111)
--(axis cs:290000,506.396584196787)
--(axis cs:310000,576.052395023353)
--(axis cs:330000,574.358670646806)
--(axis cs:350000,376.163320088398)
--(axis cs:370000,435.166112482199)
--(axis cs:390000,442.77468235012)
--(axis cs:410000,583.94213148278)
--(axis cs:430000,444.295418344042)
--(axis cs:450000,383.461751456638)
--(axis cs:470000,409.456739527006)
--(axis cs:490000,512.80538415276)
--(axis cs:510000,476.233756693556)
--(axis cs:530000,488.137329953617)
--(axis cs:550000,647.144422912228)
--(axis cs:570000,531.206000495393)
--(axis cs:590000,530.775175898701)
--(axis cs:610000,432.412955249593)
--(axis cs:630000,442.14425653564)
--(axis cs:650000,524.116734276207)
--(axis cs:670000,478.170140993633)
--(axis cs:690000,539.201434739842)
--(axis cs:710000,583.686949389129)
--(axis cs:730000,453.508190603674)
--(axis cs:750000,434.651660396228)
--(axis cs:770000,399.332567195387)
--(axis cs:790000,479.526310847403)
--(axis cs:810000,444.410120720542)
--(axis cs:830000,550.524920360026)
--(axis cs:850000,690.811503056583)
--(axis cs:870000,618.31410464909)
--(axis cs:890000,444.543126776178)
--(axis cs:910000,433.031564461695)
--(axis cs:930000,571.694012767099)
--(axis cs:950000,558.628926663259)
--(axis cs:970000,700.973482199841)
--(axis cs:990000,956.50370028018)
--(axis cs:990000,1402.10255313275)
--(axis cs:990000,1402.10255313275)
--(axis cs:970000,1225.96639572235)
--(axis cs:950000,845.340938511643)
--(axis cs:930000,899.840026896869)
--(axis cs:910000,778.0234698168)
--(axis cs:890000,1034.21582637588)
--(axis cs:870000,1086.57328531124)
--(axis cs:850000,1052.55616795301)
--(axis cs:830000,1107.63448140109)
--(axis cs:810000,950.101670298586)
--(axis cs:790000,901.257217052208)
--(axis cs:770000,676.242187983381)
--(axis cs:750000,905.910789750986)
--(axis cs:730000,1043.36282674665)
--(axis cs:710000,842.471499861488)
--(axis cs:690000,912.123038311379)
--(axis cs:670000,1004.16082995368)
--(axis cs:650000,1095.6599446235)
--(axis cs:630000,937.297353181633)
--(axis cs:610000,813.16287977439)
--(axis cs:590000,983.890832228033)
--(axis cs:570000,971.6685603651)
--(axis cs:550000,1069.24711474082)
--(axis cs:530000,797.602565213447)
--(axis cs:510000,993.627192811827)
--(axis cs:490000,924.920626393492)
--(axis cs:470000,854.460461435448)
--(axis cs:450000,834.190072010672)
--(axis cs:430000,831.803255930412)
--(axis cs:410000,873.535445052411)
--(axis cs:390000,845.930937240168)
--(axis cs:370000,798.807693975031)
--(axis cs:350000,930.310972257307)
--(axis cs:330000,972.120727401941)
--(axis cs:310000,1052.1163085665)
--(axis cs:290000,814.937863162136)
--(axis cs:270000,911.438437389455)
--(axis cs:250000,1075.78082239463)
--(axis cs:230000,1088.92745284349)
--(axis cs:210000,1134.90490263612)
--(axis cs:190000,1065.02609310553)
--(axis cs:170000,1000.20790817078)
--(axis cs:150000,891.934735946415)
--(axis cs:130000,1092.36220940125)
--(axis cs:110000,825.369627840506)
--(axis cs:90000,854.651144775252)
--(axis cs:70000,636.167779629429)
--(axis cs:50000,651.629046863317)
--(axis cs:30000,465.867656866709)
--(axis cs:10000,-29.7961098512014)
--cycle;


\addplot [line width=\linewidthother, C3, mark=*, mark size=0, mark options={solid}]
table {%
10000 -37.0920988718669
30000 345.077998399734
50000 475.591627895832
70000 448.003924265504
90000 618.29836750403
110000 587.006937276262
130000 843.162864859061
150000 779.211177676606
170000 774.597495131786
190000 842.383557262324
210000 984.891333589905
230000 806.516941123018
250000 810.38449014779
270000 656.845105082673
290000 646.799560394647
310000 791.149401526132
330000 716.541680897042
350000 626.761452402916
370000 598.468063310935
390000 627.842420633069
410000 753.220736952638
430000 646.258523150493
450000 562.904964209698
470000 534.460944465775
490000 684.320303599777
510000 709.97858019577
530000 629.821273262497
550000 931.198524577844
570000 753.570069488338
590000 781.462186416124
610000 645.071794947022
630000 666.813561501576
650000 815.8698624925
670000 743.288800497229
690000 741.040009743159
710000 683.220332846942
730000 753.540403205518
750000 653.958612116392
770000 471.682135390282
790000 683.970927576562
810000 663.53031257037
830000 847.201783246428
850000 881.467528923925
870000 835.983595530322
890000 629.884605607131
910000 587.91990671466
930000 744.828632086724
950000 622.924845218359
970000 1000.49695407778
990000 1246.67790860341
};

\path [draw=C6, fill=C6, opacity=0.2]
(axis cs:10000,-42.9513753255208)
--(axis cs:10000,-114.150451660156)
--(axis cs:50000,745.116780598958)
--(axis cs:90000,1491.61197916667)
--(axis cs:130000,2239.00219726562)
--(axis cs:170000,3415.16796875)
--(axis cs:210000,4391.36881510417)
--(axis cs:250000,3938.32633463542)
--(axis cs:290000,4880.90934244792)
--(axis cs:330000,4720.90055338542)
--(axis cs:370000,5148.65885416667)
--(axis cs:410000,5137.36751302083)
--(axis cs:450000,5315.17057291667)
--(axis cs:490000,5223.21647135417)
--(axis cs:530000,5421.86393229167)
--(axis cs:570000,5408.71533203125)
--(axis cs:610000,5371.24153645833)
--(axis cs:650000,5465.78206380208)
--(axis cs:690000,5569.40348307292)
--(axis cs:730000,5615.03011067708)
--(axis cs:770000,4157.38981119792)
--(axis cs:810000,5591.52897135417)
--(axis cs:850000,5682.55419921875)
--(axis cs:890000,5730.31673177083)
--(axis cs:930000,5820.61881510417)
--(axis cs:970000,5729.04085286458)
--(axis cs:970000,5874.52815755208)
--(axis cs:970000,5874.52815755208)
--(axis cs:930000,5968.54915364583)
--(axis cs:890000,6083.86604817708)
--(axis cs:850000,6132.14794921875)
--(axis cs:810000,6083.45865885417)
--(axis cs:770000,5992.248046875)
--(axis cs:730000,5996.2529296875)
--(axis cs:690000,5927.681640625)
--(axis cs:650000,5896.41389973958)
--(axis cs:610000,5936.44254557292)
--(axis cs:570000,5777.9697265625)
--(axis cs:530000,5974.50406901042)
--(axis cs:490000,5825.2978515625)
--(axis cs:450000,5695.69075520833)
--(axis cs:410000,5713.37581380208)
--(axis cs:370000,5607.52571614583)
--(axis cs:330000,5487.51708984375)
--(axis cs:290000,5444.14095052083)
--(axis cs:250000,5227.77766927083)
--(axis cs:210000,5089.69791666667)
--(axis cs:170000,4574.53629557292)
--(axis cs:130000,4052.80582682292)
--(axis cs:90000,3302.58064778646)
--(axis cs:50000,1319.3700764974)
--(axis cs:10000,-42.9513753255208)
--cycle;

\addplot [line width=\linewidthother, C6, mark=*, mark size=0, mark options={solid}]
table {%
10000 -65.5242004394531
50000 914.076009114583
90000 2127.48111979167
130000 2959.2509765625
170000 4043.939453125
210000 4818.74983723958
250000 4715.47867838542
290000 5125.44108072917
330000 5099.529296875
370000 5351.82926432292
410000 5340.18212890625
450000 5482.95670572917
490000 5469.17431640625
530000 5618.16031901042
570000 5653.76643880208
610000 5636.84537760417
650000 5657.68733723958
690000 5723.84163411458
730000 5808.85009765625
770000 5693.71110026042
810000 5732.22607421875
850000 5845.03531901042
890000 5867.8076171875
930000 5876.380859375
970000 5811.94140625
};


\path [draw=C0, fill=C0, opacity=0.2]
(axis cs:1,-62.1711555)
--(axis cs:1,-252.674978)
--(axis cs:24000,365.21141)
--(axis cs:48000,677.958748333333)
--(axis cs:72000,1801.30085)
--(axis cs:96000,2062.38166666667)
--(axis cs:120000,2916.60178333333)
--(axis cs:144000,2843.33536666667)
--(axis cs:168000,3465.23925)
--(axis cs:192000,4216.89095)
--(axis cs:216000,4318.5384)
--(axis cs:240000,4530.73986666667)
--(axis cs:264000,4250.95681666667)
--(axis cs:288000,5278.04958333333)
--(axis cs:312000,4934.10516666667)
--(axis cs:336000,4813.07176666667)
--(axis cs:360000,5398.40497791667)
--(axis cs:384000,4547.6037)
--(axis cs:408000,5209.24175)
--(axis cs:432000,4963.79478333333)
--(axis cs:456000,5242.8267)
--(axis cs:480000,5367.75391666667)
--(axis cs:504000,5622.12162041667)
--(axis cs:528000,5519.55295)
--(axis cs:552000,5629.9634)
--(axis cs:576000,6036.30066666667)
--(axis cs:600000,5302.80161666667)
--(axis cs:624000,5439.22103333333)
--(axis cs:648000,6063.6662)
--(axis cs:672000,5643.2313)
--(axis cs:696000,5281.07096666667)
--(axis cs:720000,5560.64907208333)
--(axis cs:744000,6268.97635)
--(axis cs:768000,5610.37709583333)
--(axis cs:792000,6057.5613)
--(axis cs:816000,6245.50366666667)
--(axis cs:840000,6712.9649)
--(axis cs:864000,6404.37901666667)
--(axis cs:888000,6238.61089625)
--(axis cs:912000,6530.98123333333)
--(axis cs:936000,6205.35933333333)
--(axis cs:960000,5535.92578333333)
--(axis cs:984000,5498.43986666667)
--(axis cs:984000,7093.78395)
--(axis cs:984000,7093.78395)
--(axis cs:960000,7233.15493333333)
--(axis cs:936000,7232.04476666667)
--(axis cs:912000,7189.6735)
--(axis cs:888000,6966.77786666667)
--(axis cs:864000,7188.0913)
--(axis cs:840000,7142.25483333333)
--(axis cs:816000,7129.08055)
--(axis cs:792000,7088.52988333333)
--(axis cs:768000,7018.76386666667)
--(axis cs:744000,7103.80593333333)
--(axis cs:720000,6995.62833333333)
--(axis cs:696000,6932.48928333333)
--(axis cs:672000,6656.77633333333)
--(axis cs:648000,7023.21331666667)
--(axis cs:624000,6769.20178333333)
--(axis cs:600000,6802.61743333333)
--(axis cs:576000,6816.77276666667)
--(axis cs:552000,6763.71495)
--(axis cs:528000,6796.36003333333)
--(axis cs:504000,6796.23258333333)
--(axis cs:480000,6532.26516666667)
--(axis cs:456000,6619.484)
--(axis cs:432000,6276.26551833334)
--(axis cs:408000,6549.43326666667)
--(axis cs:384000,6014.1474)
--(axis cs:360000,6528.94921666667)
--(axis cs:336000,5872.6242675)
--(axis cs:312000,6280.0172)
--(axis cs:288000,6270.8663)
--(axis cs:264000,6210.54168333333)
--(axis cs:240000,6132.11138333333)
--(axis cs:216000,5774.81603333333)
--(axis cs:192000,5655.92215)
--(axis cs:168000,4934.4741)
--(axis cs:144000,4760.40586666667)
--(axis cs:120000,4536.5835)
--(axis cs:96000,4166.86253333333)
--(axis cs:72000,3031.97151666667)
--(axis cs:48000,1857.61144333333)
--(axis cs:24000,780.259655)
--(axis cs:1,-62.1711555)
--cycle;
%Diff-QL
\path [draw=C4, fill=C4, opacity=0.2]
(axis cs:10000,-130.061262459449)
--(axis cs:10000,-294.604611003888)
--(axis cs:50000,326.275042185702)
--(axis cs:90000,898.237676264207)
--(axis cs:130000,490.177539853023)
--(axis cs:170000,741.258707604816)
--(axis cs:210000,1035.76388389191)
--(axis cs:250000,2200.80046777555)
--(axis cs:290000,1547.19700307794)
--(axis cs:330000,2508.0490056234)
--(axis cs:370000,2075.894687226)
--(axis cs:410000,3033.45483431003)
--(axis cs:450000,2878.88288234077)
--(axis cs:490000,3012.7257812736)
--(axis cs:530000,2892.66164134062)
--(axis cs:570000,3382.49674899932)
--(axis cs:610000,3497.86684458171)
--(axis cs:650000,3625.96597228558)
--(axis cs:690000,3994.66739804794)
--(axis cs:730000,3743.54661201208)
--(axis cs:770000,3881.23083706291)
--(axis cs:810000,3320.08177163566)
--(axis cs:850000,4169.95163453464)
--(axis cs:890000,3601.9328115736)
--(axis cs:930000,4130.8305687931)
--(axis cs:970000,4050.47691024456)
--(axis cs:970000,5718.00651914204)
--(axis cs:970000,5718.00651914204)
--(axis cs:930000,5734.23357137727)
--(axis cs:890000,5655.97199393792)
--(axis cs:850000,5722.47447521297)
--(axis cs:810000,5541.59678374078)
--(axis cs:770000,5429.72622968172)
--(axis cs:730000,5487.13507002686)
--(axis cs:690000,5271.77633856545)
--(axis cs:650000,5113.54341768429)
--(axis cs:610000,5145.01069047612)
--(axis cs:570000,5200.34631010606)
--(axis cs:530000,5051.19735760376)
--(axis cs:490000,4824.29005608974)
--(axis cs:450000,4987.26193809875)
--(axis cs:410000,4500.80700319556)
--(axis cs:370000,3903.44701061322)
--(axis cs:330000,3925.79548390077)
--(axis cs:290000,3402.61237431961)
--(axis cs:250000,3244.06105325203)
--(axis cs:210000,2263.09002882041)
--(axis cs:170000,1087.49767970868)
--(axis cs:130000,921.072506925926)
--(axis cs:90000,944.945326543133)
--(axis cs:50000,849.213491563634)
--(axis cs:10000,-130.061262459449)
--cycle;

\addplot [line width=\linewidthother, C4, mark=*, mark size=0, mark options={solid}]
table {%
10000 -217.620700367844
50000 658.986031446973
90000 921.31238444355
130000 725.990415034111
170000 920.446564283176
210000 1495.57540388902
250000 2595.63928556631
290000 2259.68789524735
330000 3532.41865746365
370000 3459.38765225526
410000 4034.19605830229
450000 4330.61553663217
490000 4421.8844852079
530000 4350.33707692418
570000 4511.92531249358
610000 4721.31184290455
650000 4933.35998439465
690000 4852.8771229063
730000 5262.50670404879
770000 5199.4792334386
810000 5096.00793005758
850000 5383.82929501314
890000 5238.56990348632
930000 5328.91238530721
970000 5075.75964235589
};
%Consistency-AC
\path [draw=C5, fill=C5, opacity=0.2]
(axis cs:10000,965.6237603047)
--(axis cs:10000,908.908840233733)
--(axis cs:50000,852.049126435344)
--(axis cs:90000,875.486252152921)
--(axis cs:130000,500.931696737407)
--(axis cs:170000,514.710857061814)
--(axis cs:210000,1044.279325465)
--(axis cs:250000,1961.45140862083)
--(axis cs:290000,1499.26681814581)
--(axis cs:330000,2168.10592634266)
--(axis cs:370000,2516.0820997895)
--(axis cs:410000,1321.48729105422)
--(axis cs:450000,3090.9658004417)
--(axis cs:490000,2190.81216652228)
--(axis cs:530000,3107.69201913887)
--(axis cs:570000,2693.47396348374)
--(axis cs:610000,3746.27564597638)
--(axis cs:650000,4119.18263169139)
--(axis cs:690000,4195.52297852915)
--(axis cs:730000,4289.4708084011)
--(axis cs:770000,4345.238427542)
--(axis cs:810000,4387.38750238895)
--(axis cs:850000,3902.16768372856)
--(axis cs:890000,4641.7158598099)
--(axis cs:930000,4155.32528421271)
--(axis cs:970000,4240.18959324642)
--(axis cs:970000,5538.14533720782)
--(axis cs:970000,5538.14533720782)
--(axis cs:930000,5700.67332554496)
--(axis cs:890000,5668.36556624199)
--(axis cs:850000,5640.31284686526)
--(axis cs:810000,5529.05800582376)
--(axis cs:770000,5405.17735840156)
--(axis cs:730000,5224.17078832163)
--(axis cs:690000,5099.30883221102)
--(axis cs:650000,5147.79228678718)
--(axis cs:610000,4762.93795194677)
--(axis cs:570000,5048.08792234344)
--(axis cs:530000,4506.30274592249)
--(axis cs:490000,4560.25919739858)
--(axis cs:450000,4918.7252413182)
--(axis cs:410000,4445.20019743965)
--(axis cs:370000,3650.0489782956)
--(axis cs:330000,3727.32283745243)
--(axis cs:290000,2776.5911647242)
--(axis cs:250000,3019.23587042212)
--(axis cs:210000,1472.40181877856)
--(axis cs:170000,1101.32104540485)
--(axis cs:130000,891.998558525105)
--(axis cs:90000,932.363015636956)
--(axis cs:50000,931.904928493765)
--(axis cs:10000,965.6237603047)
--cycle;

\addplot [line width=\linewidthother, C5, mark=*, mark size=0, mark options={solid}]
table {%
10000 951.535547757787
50000 913.385334223428
90000 913.690720068783
130000 662.919965480778
170000 805.683736270285
210000 1273.7100324479
250000 2507.87484584282
290000 2566.48605626865
330000 3332.71484972181
370000 3161.32836858126
410000 3328.3565612846
450000 4420.96612419737
490000 4015.30728299201
530000 4089.74553336842
570000 4247.76763460473
610000 4575.70630317087
650000 4854.44009534602
690000 4862.57560774361
730000 4913.54189130563
770000 4849.50236359576
810000 5031.98980684287
850000 4519.21723096191
890000 5219.73810195169
930000 5245.90468682829
970000 4808.8177644911
};
%CrossQ
\path [draw=C2, fill=C2, opacity=0.2]
(axis cs:1,941.413516666667)
--(axis cs:1,823.115726666667)
--(axis cs:24000,333.628841666667)
--(axis cs:48000,567.546983333333)
--(axis cs:72000,934.976336666667)
--(axis cs:96000,2070.99636666667)
--(axis cs:120000,2392.50538333333)
--(axis cs:144000,3770.74366666667)
--(axis cs:168000,3792.69721666667)
--(axis cs:192000,3982.96610166667)
--(axis cs:216000,5078.58621666667)
--(axis cs:240000,5417.79738333333)
--(axis cs:264000,4573.04481666667)
--(axis cs:288000,5601.10473333333)
--(axis cs:312000,5760.2226)
--(axis cs:336000,5510.97225)
--(axis cs:360000,5790.3021)
--(axis cs:384000,5816.56273333333)
--(axis cs:408000,5774.51611666667)
--(axis cs:432000,6055.93238333333)
--(axis cs:456000,5642.288)
--(axis cs:480000,6009.18794208333)
--(axis cs:504000,6049.89101666667)
--(axis cs:528000,6334.0893)
--(axis cs:552000,6319.67058333333)
--(axis cs:576000,6215.26463333334)
--(axis cs:600000,6155.3687)
--(axis cs:624000,6299.25793333333)
--(axis cs:648000,6537.97375)
--(axis cs:672000,5724.5839)
--(axis cs:696000,6212.2179)
--(axis cs:720000,6273.73016666667)
--(axis cs:744000,6553.34511666667)
--(axis cs:768000,6685.21875)
--(axis cs:792000,6708.97075)
--(axis cs:816000,6557.00825)
--(axis cs:840000,6153.00198333333)
--(axis cs:864000,6665.02423333333)
--(axis cs:888000,6524.1346)
--(axis cs:912000,6511.1801)
--(axis cs:936000,6735.80205)
--(axis cs:960000,6583.68288333333)
--(axis cs:960000,7155.76078333333)
--(axis cs:960000,7155.76078333333)
--(axis cs:936000,7157.66736666667)
--(axis cs:912000,7055.48267958333)
--(axis cs:888000,7057.4584)
--(axis cs:864000,7088.24743333333)
--(axis cs:840000,7062.91351666667)
--(axis cs:816000,7054.68065)
--(axis cs:792000,7034.30926666667)
--(axis cs:768000,6969.35058541667)
--(axis cs:744000,6916.49058333333)
--(axis cs:720000,6885.11025)
--(axis cs:696000,6875.11608333333)
--(axis cs:672000,6871.4524)
--(axis cs:648000,6862.31051666667)
--(axis cs:624000,6808.07666666667)
--(axis cs:600000,6753.21566666667)
--(axis cs:576000,6752.55155)
--(axis cs:552000,6652.24943333333)
--(axis cs:528000,6634.1595)
--(axis cs:504000,6642.21026666667)
--(axis cs:480000,6597.53292916667)
--(axis cs:456000,6505.05411666667)
--(axis cs:432000,6548.62313333333)
--(axis cs:408000,6483.90273333333)
--(axis cs:384000,6361.76496666667)
--(axis cs:360000,6312.3064)
--(axis cs:336000,6263.6461)
--(axis cs:312000,6196.40666666667)
--(axis cs:288000,6199.63041666667)
--(axis cs:264000,5918.2678)
--(axis cs:240000,5908.31406666667)
--(axis cs:216000,5690.71731666667)
--(axis cs:192000,5607.4118)
--(axis cs:168000,5269.73876666667)
--(axis cs:144000,4767.84574958334)
--(axis cs:120000,3943.84809375)
--(axis cs:96000,3287.51206666667)
--(axis cs:72000,1907.02491666667)
--(axis cs:48000,1067.2182)
--(axis cs:24000,582.80132)
--(axis cs:1,941.413516666667)
--cycle;

\addplot [line width=\linewidthother, C2, mark=*, mark size=0, mark options={solid}]
table {%
1 923.063095
24000 452.961398333333
48000 759.495248333333
72000 1383.85751
96000 2790.96551666667
120000 3235.56833333333
144000 4234.73723333333
168000 4529.82935
192000 5104.41628333333
216000 5483.9602
240000 5697.93261666667
264000 5322.18071666667
288000 5971.70121666667
312000 6003.15886666666
336000 5935.57958333333
360000 6107.29856666667
384000 6177.49776666667
408000 6200.71715
432000 6398.66745
456000 6259.73928333333
480000 6387.96505
504000 6406.48003333333
528000 6505.50245
552000 6483.32405
576000 6599.71971666667
600000 6596.5972
624000 6575.98541666667
648000 6710.54353333333
672000 6519.63145
696000 6745.33758333333
720000 6714.8432
744000 6782.15778333333
768000 6829.66063333333
792000 6838.73535
816000 6886.67181666667
840000 6898.87518333333
864000 6903.78636666667
888000 6874.85888333333
912000 6831.66331666667
936000 6967.6926
960000 6957.2253
nan 7046.72581666667
};

\addplot [line width=\linewidthdime, C0, mark=*, mark size=0, mark options={solid}]
table {%
1 -77.3694106666667
24000 530.007491666667
48000 976.729643333333
72000 2488.35606666667
96000 3323.27631666667
120000 3934.6546
144000 4213.77568333333
168000 4361.13831666667
192000 5180.62616666667
216000 5516.45076666667
240000 5621.24408333333
264000 5628.50601666667
288000 6050.94748333333
312000 5926.91643333333
336000 5483.02671666667
360000 6246.5021
384000 5517.25081666667
408000 6301.03468333333
432000 5667.26346666667
456000 6212.54341666667
480000 6078.01823333333
504000 6356.01238333333
528000 6490.4195
552000 6380.07245
576000 6659.30368333333
600000 6452.68633333333
624000 6162.16691666667
648000 6815.03981666667
672000 6081.8011
696000 6154.45971666667
720000 6617.64396666667
744000 6970.8394
768000 6534.2619
792000 6724.97323333333
816000 7036.70365
840000 7029.17525
864000 6923.756
888000 6797.23118333333
912000 6971.2461
936000 6787.00018333333
960000 6528.86113333333
984000 6668.8109
};
\end{axis}

\end{tikzpicture}
}
       \subcaption[]{Ant-v3}
       \label{fig::exps_new::ant-v3}
    \end{minipage}\hfill
    \begin{minipage}[b]{0.23\textwidth}
        \centering
       \resizebox{1\textwidth}{!}{% This file was created with tikzplotlib v0.10.1.
\begin{tikzpicture}

\definecolor{darkcyan1115178}{RGB}{1,115,178}
\definecolor{darkgray176}{RGB}{176,176,176}

\begin{axis}[
legend cell align={left},
legend style={fill opacity=0.8, draw opacity=1, text opacity=1, draw=lightgray204, at={(0.03,0.03)},  anchor=north west},
tick align=outside,
tick pos=left,
x grid style={white},
xlabel={Number Env Interactions},
xmajorgrids,
%xmin=-37798.95, xmax=1045799.95,
xmin=-0.0, xmax=1000000.0,
xtick style={color=black},
y grid style={white},
yticklabels={,0,0.5,1.0},
% yticklabels={,0,0.3,0.6,0.9,1.2,1.5},
scaled y ticks=false,
ylabel={IQM Mean Return $(\times 10^4)$},
ymajorgrids,
ymin=-551.720758666667, ymax=14000,
ytick style={color=black},
axis background/.style={fill=plot_background},
label style={font=\large},
tick label style={font=\large},
x axis line style={draw=none},
y axis line style={draw=none},
]


% QSM
\path [draw=C3, fill=C3, opacity=0.2]
(axis cs:10000,182.157737731934)
--(axis cs:10000,80.0149993896484)
--(axis cs:30000,264.590590794881)
--(axis cs:50000,327.713520532846)
--(axis cs:70000,389.757047514121)
--(axis cs:90000,374.465304091573)
--(axis cs:110000,415.808311257511)
--(axis cs:130000,445.818828933702)
--(axis cs:150000,437.449578648767)
--(axis cs:170000,465.765347114382)
--(axis cs:190000,512.231488038655)
--(axis cs:210000,530.439800328782)
--(axis cs:230000,592.418869672214)
--(axis cs:250000,629.529918189418)
--(axis cs:270000,695.858531079447)
--(axis cs:290000,681.154383627264)
--(axis cs:310000,732.691132252687)
--(axis cs:330000,720.525415032816)
--(axis cs:350000,751.157177154635)
--(axis cs:370000,1058.90308511895)
--(axis cs:390000,1152.48201340659)
--(axis cs:410000,1217.78098496879)
--(axis cs:430000,1215.57284547671)
--(axis cs:450000,1386.29344938928)
--(axis cs:470000,1510.71687410723)
--(axis cs:490000,1970.27199795676)
--(axis cs:510000,1735.87240618568)
--(axis cs:530000,2089.02772616053)
--(axis cs:550000,2796.96063000019)
--(axis cs:570000,2939.47127471759)
--(axis cs:590000,2917.06896086684)
--(axis cs:610000,2993.43753070351)
--(axis cs:630000,3171.08065694574)
--(axis cs:650000,3496.69378739939)
--(axis cs:670000,3918.91354670281)
--(axis cs:690000,4036.20338014191)
--(axis cs:710000,3683.15496377576)
--(axis cs:730000,4011.6455814058)
--(axis cs:750000,4532.71160711907)
--(axis cs:770000,3729.91397421565)
--(axis cs:790000,4053.19091903752)
--(axis cs:810000,4638.62523917089)
--(axis cs:830000,4767.77274147274)
--(axis cs:850000,4619.52196159823)
--(axis cs:870000,4580.89523236927)
--(axis cs:890000,4278.45588114803)
--(axis cs:910000,4234.97215938218)
--(axis cs:930000,4604.78532511429)
--(axis cs:950000,4782.63576578066)
--(axis cs:970000,4480.8015041186)
--(axis cs:990000,4725.53129897616)
--(axis cs:990000,5167.20200824979)
--(axis cs:990000,5167.20200824979)
--(axis cs:970000,5138.70200244167)
--(axis cs:950000,5084.70788903675)
--(axis cs:930000,4965.46042103873)
--(axis cs:910000,4919.36610083356)
--(axis cs:890000,5050.34871210079)
--(axis cs:870000,5025.14368482953)
--(axis cs:850000,5075.26446677565)
--(axis cs:830000,5041.52089141168)
--(axis cs:810000,4957.78972257585)
--(axis cs:790000,4780.47955615995)
--(axis cs:770000,4708.37278756422)
--(axis cs:750000,4961.46505690795)
--(axis cs:730000,4795.72510123481)
--(axis cs:710000,4865.78689422796)
--(axis cs:690000,4776.22280343535)
--(axis cs:670000,4794.27321833348)
--(axis cs:650000,4603.13203700005)
--(axis cs:630000,4628.68118150478)
--(axis cs:610000,4658.76579926772)
--(axis cs:590000,4267.08506438731)
--(axis cs:570000,4186.03680345928)
--(axis cs:550000,3924.49898118041)
--(axis cs:530000,3047.99110044131)
--(axis cs:510000,3158.08729927992)
--(axis cs:490000,3442.88836928664)
--(axis cs:470000,2500.30548530318)
--(axis cs:450000,3151.50112937501)
--(axis cs:430000,2236.98659566644)
--(axis cs:410000,1675.76960324451)
--(axis cs:390000,1687.29243188035)
--(axis cs:370000,1666.66589567164)
--(axis cs:350000,1352.93846442729)
--(axis cs:330000,1373.33237961228)
--(axis cs:310000,879.286734037736)
--(axis cs:290000,966.712013765772)
--(axis cs:270000,922.694618679178)
--(axis cs:250000,813.713564951291)
--(axis cs:230000,770.043229216194)
--(axis cs:210000,653.433556605237)
--(axis cs:190000,716.165097927028)
--(axis cs:170000,583.711766327363)
--(axis cs:150000,552.910858326978)
--(axis cs:130000,603.6554662964)
--(axis cs:110000,512.718499626964)
--(axis cs:90000,501.477889666955)
--(axis cs:70000,477.889219634731)
--(axis cs:50000,437.81409907341)
--(axis cs:30000,302.002698580424)
--(axis cs:10000,182.157737731934)
--cycle;


\addplot [line width=\linewidthother, C3, mark=*, mark size=0, mark options={solid}]
table {%
10000 111.830499013265
30000 284.22275129954
50000 372.517459154129
70000 425.358012616634
90000 425.495101029674
110000 465.241586913665
130000 538.39556087926
150000 498.240291842104
170000 518.519198864194
190000 605.579988499778
210000 566.189927299302
230000 689.625347860058
250000 734.274095389253
270000 800.52473511182
290000 818.537650767405
310000 788.856850515675
330000 902.640704882759
350000 915.272071243902
370000 1279.74588955584
390000 1441.07070696166
410000 1311.25597304876
430000 1629.97434629666
450000 1962.22072852402
470000 2020.28057105981
490000 2658.45732687382
510000 2412.53400113252
530000 2431.24980228556
550000 3507.28574856541
570000 3625.13760093516
590000 3712.53562648754
610000 3941.48721627792
630000 4052.33782012933
650000 4050.0650786743
670000 4622.12643093094
690000 4561.89421526275
710000 4576.12447462044
730000 4467.38393900065
750000 4757.39867893877
770000 4274.96022446519
790000 4463.05415995961
810000 4781.62895372846
830000 4904.53319464956
850000 4906.21631715912
870000 4833.4532922142
890000 4784.56572855358
910000 4610.85570173024
930000 4759.55388918524
950000 4940.35753677629
970000 4970.11714001783
990000 4991.72773064248
};
%DIPO
\path [draw=C6, fill=C6, opacity=0.2]
(axis cs:10000,252.298934936523)
--(axis cs:10000,77.5163116455078)
--(axis cs:50000,428.866027832031)
--(axis cs:90000,501.251281738281)
--(axis cs:130000,511.433715820312)
--(axis cs:170000,689.324015299479)
--(axis cs:210000,824.139933268229)
--(axis cs:250000,1192.13582356771)
--(axis cs:290000,4226.14143880208)
--(axis cs:330000,1958.87133789062)
--(axis cs:370000,2731.38736979167)
--(axis cs:410000,3882.912109375)
--(axis cs:450000,1262.87467447917)
--(axis cs:490000,3375.08577473958)
--(axis cs:530000,1527.97802734375)
--(axis cs:570000,4425.63053385417)
--(axis cs:610000,3988.42952473958)
--(axis cs:650000,3844.41805013021)
--(axis cs:690000,3141.49739583333)
--(axis cs:730000,4363.94791666667)
--(axis cs:770000,4660.08756510417)
--(axis cs:810000,5046.76285807292)
--(axis cs:850000,5075.6650390625)
--(axis cs:890000,4665.27962239583)
--(axis cs:930000,2757.5)
--(axis cs:970000,5009.00081380208)
--(axis cs:970000,5172.62239583333)
--(axis cs:970000,5172.62239583333)
--(axis cs:930000,5125.06087239583)
--(axis cs:890000,5140.7314453125)
--(axis cs:850000,5154.37060546875)
--(axis cs:810000,5187.52376302083)
--(axis cs:770000,5151.82975260417)
--(axis cs:730000,5164.22412109375)
--(axis cs:690000,5149.48307291667)
--(axis cs:650000,5185.43196614583)
--(axis cs:610000,5093.62109375)
--(axis cs:570000,5140.16389973958)
--(axis cs:530000,5153.85302734375)
--(axis cs:490000,5160.61995442708)
--(axis cs:450000,5119.77848307292)
--(axis cs:410000,5115.609375)
--(axis cs:370000,4810.72298177083)
--(axis cs:330000,5142.99430338542)
--(axis cs:290000,5127.92708333333)
--(axis cs:250000,5099.75748697917)
--(axis cs:210000,4297.60986328125)
--(axis cs:170000,958.186808268229)
--(axis cs:130000,689.46641031901)
--(axis cs:90000,611.665669759115)
--(axis cs:50000,622.799947102865)
--(axis cs:10000,252.298934936523)
--cycle;

\addplot [line width=\linewidthother, C6, mark=*, mark size=0, mark options={solid}]
table {%
10000 149.577702840169
50000 572.34336344401
90000 563.922912597656
130000 651.373392740885
170000 782.784159342448
210000 1809.14514160156
250000 3601.42049153646
290000 4895.53255208333
330000 3325.04874674479
370000 3363.06477864583
410000 5057.90169270833
450000 3874.77490234375
490000 5099.33740234375
530000 4095.30924479167
570000 4919.17268880208
610000 4531.42708333333
650000 4725.62752278646
690000 4902.90559895833
730000 4878.84391276042
770000 5069.49283854167
810000 5127.34261067708
850000 5126.6298828125
890000 5091.71321614583
930000 4430.44417317708
970000 5116.80452473958
};

%DiffQL
\path [draw=C4, fill=C4, opacity=0.2]
(axis cs:10000,293.546163687176)
--(axis cs:10000,217.145542299399)
--(axis cs:50000,256.820883331603)
--(axis cs:90000,197.370308140057)
--(axis cs:130000,342.104331447903)
--(axis cs:170000,511.076409208912)
--(axis cs:210000,549.666538968672)
--(axis cs:250000,589.650879725777)
--(axis cs:290000,620.782305920384)
--(axis cs:330000,629.499997928096)
--(axis cs:370000,738.217122050852)
--(axis cs:410000,680.328208263076)
--(axis cs:450000,915.047907141653)
--(axis cs:490000,863.379219542889)
--(axis cs:530000,1123.21213345693)
--(axis cs:570000,1217.17895016926)
--(axis cs:610000,1683.41466747135)
--(axis cs:650000,966.555781676576)
--(axis cs:690000,1717.20737353015)
--(axis cs:730000,1834.19631278279)
--(axis cs:770000,1543.19271071442)
--(axis cs:810000,2123.64652697338)
--(axis cs:850000,1879.69667726382)
--(axis cs:890000,1637.27400476639)
--(axis cs:930000,2266.26151790917)
--(axis cs:970000,1351.70004574882)
--(axis cs:970000,3884.43309640162)
--(axis cs:970000,3884.43309640162)
--(axis cs:930000,5003.55578379182)
--(axis cs:890000,4740.59629710678)
--(axis cs:850000,4494.59513553648)
--(axis cs:810000,4453.02102947443)
--(axis cs:770000,4826.66495522798)
--(axis cs:730000,4811.8674446972)
--(axis cs:690000,3711.85750114949)
--(axis cs:650000,4035.61578815578)
--(axis cs:610000,4748.15339083127)
--(axis cs:570000,2627.01002333662)
--(axis cs:530000,4071.47752661167)
--(axis cs:490000,2792.74801471492)
--(axis cs:450000,3137.62701672615)
--(axis cs:410000,2440.57962325332)
--(axis cs:370000,3315.94665452449)
--(axis cs:330000,1547.91110918304)
--(axis cs:290000,895.964459202682)
--(axis cs:250000,1201.65970265986)
--(axis cs:210000,699.584817637387)
--(axis cs:170000,576.083482334799)
--(axis cs:130000,532.705980595277)
--(axis cs:90000,351.89434339035)
--(axis cs:50000,306.773362152596)
--(axis cs:10000,293.546163687176)
--cycle;

\addplot [line width=\linewidthother, C4, mark=*, mark size=0, mark options={solid}]
table {%
10000 268.974770212736
50000 274.466482824679
90000 291.514766767929
130000 408.087537483741
170000 530.99157592105
210000 607.089514592678
250000 736.029075955359
290000 743.086236477731
330000 697.705064473837
370000 1154.21508533469
410000 1024.17036272543
450000 1747.07319538372
490000 1555.71612071742
530000 2868.35237533739
570000 1755.64934822878
610000 3294.27627527516
650000 2329.86409536969
690000 3358.93843151726
730000 3929.34252307362
770000 3656.4058982764
810000 3212.49014412031
850000 3730.30050991224
890000 3668.50341293726
930000 3725.83525616416
970000 3022.281013287
};
%Consistency-AC
\path [draw=C5, fill=C5, opacity=0.2]
(axis cs:10000,292.858150919983)
--(axis cs:10000,229.152467071198)
--(axis cs:50000,130.806285678688)
--(axis cs:90000,228.712085460993)
--(axis cs:130000,294.594504515149)
--(axis cs:170000,307.055123094247)
--(axis cs:210000,177.785048434803)
--(axis cs:250000,328.687522544302)
--(axis cs:290000,391.490041830014)
--(axis cs:330000,391.30161122707)
--(axis cs:370000,498.472939499447)
--(axis cs:410000,512.20090352241)
--(axis cs:450000,560.828497663583)
--(axis cs:490000,549.022345770201)
--(axis cs:530000,572.806477166397)
--(axis cs:570000,670.42944795788)
--(axis cs:610000,551.402611941402)
--(axis cs:650000,763.495707049077)
--(axis cs:690000,688.146546589209)
--(axis cs:730000,843.28934550514)
--(axis cs:770000,840.238660801935)
--(axis cs:810000,964.76329945409)
--(axis cs:850000,1066.7948215151)
--(axis cs:890000,1679.65193554067)
--(axis cs:930000,1636.62539493024)
--(axis cs:970000,1910.2874303201)
--(axis cs:970000,4820.54518139592)
--(axis cs:970000,4820.54518139592)
--(axis cs:930000,5071.10185058516)
--(axis cs:890000,4864.65187950617)
--(axis cs:850000,4975.09979595576)
--(axis cs:810000,4502.95764273107)
--(axis cs:770000,4773.38083569733)
--(axis cs:730000,4703.88065671435)
--(axis cs:690000,4699.45550730847)
--(axis cs:650000,4998.03001306644)
--(axis cs:610000,4930.40376499768)
--(axis cs:570000,4299.45444607873)
--(axis cs:530000,4679.13049541553)
--(axis cs:490000,1669.93849620031)
--(axis cs:450000,4167.91628098086)
--(axis cs:410000,3625.50311003133)
--(axis cs:370000,2822.1382905014)
--(axis cs:330000,3280.67058676497)
--(axis cs:290000,2116.14438322519)
--(axis cs:250000,944.856853218664)
--(axis cs:210000,719.239261832486)
--(axis cs:170000,659.483612062957)
--(axis cs:130000,472.187408293416)
--(axis cs:90000,318.567405514166)
--(axis cs:50000,292.510363591773)
--(axis cs:10000,292.858150919983)
--cycle;

\addplot [line width=\linewidthother, C5, mark=*, mark size=0, mark options={solid}]
table {%
10000 269.684995349337
50000 260.005851663909
90000 279.732591910269
130000 393.272158662902
170000 485.466644058986
210000 521.834676175907
250000 629.92290852457
290000 627.787154553149
330000 862.43912984148
370000 947.499632555948
410000 836.900794398586
450000 2053.50569297447
490000 1204.99157040755
530000 2797.7847683414
570000 2895.86861487063
610000 2050.63908181764
650000 3453.3772755086
690000 2201.91387840434
730000 2721.39632718743
770000 3207.55358055666
810000 2303.90907439295
850000 2690.46529583941
890000 3892.20898028511
930000 3999.19772921359
970000 4297.91315567424
};

%CrossQ
\path [draw=C2, fill=C2, opacity=0.2]
(axis cs:1,143.951875)
--(axis cs:1,72.573923)
--(axis cs:24000,448.765771666667)
--(axis cs:48000,709.672403333333)
--(axis cs:72000,938.033473333333)
--(axis cs:96000,1215.55739166667)
--(axis cs:120000,1074.41969)
--(axis cs:144000,1942.2129)
--(axis cs:168000,3957.23963333333)
--(axis cs:192000,4768.72611666667)
--(axis cs:216000,2240.80487333333)
--(axis cs:240000,5172.34171666667)
--(axis cs:264000,6537.25713333333)
--(axis cs:288000,4312.75263333333)
--(axis cs:312000,6854.19913333333)
--(axis cs:336000,8053.21448333333)
--(axis cs:360000,6893.88296666667)
--(axis cs:384000,6662.19173333333)
--(axis cs:408000,7526.40466666667)
--(axis cs:432000,6497.635)
--(axis cs:456000,8011.33493333333)
--(axis cs:480000,5871.27051666667)
--(axis cs:504000,8804.57846416667)
--(axis cs:528000,8450.54058333333)
--(axis cs:552000,6950.48599166667)
--(axis cs:576000,8476.7179)
--(axis cs:600000,9148.71783333333)
--(axis cs:624000,9233.51833333333)
--(axis cs:648000,8959.72366666667)
--(axis cs:672000,8704.35026666667)
--(axis cs:696000,8537.1097)
--(axis cs:720000,10121.2173333333)
--(axis cs:744000,9859.31676666667)
--(axis cs:768000,9922.65133333333)
--(axis cs:792000,10370.5441666667)
--(axis cs:816000,10004.9963333333)
--(axis cs:840000,10014.9143958333)
--(axis cs:864000,10279.4037125)
--(axis cs:888000,9089.340475)
--(axis cs:912000,9250.5895)
--(axis cs:936000,9395.3349)
--(axis cs:960000,10198.0128333333)
--(axis cs:984000,10196.5286666667)
--(axis cs:984000,10749.7001666667)
--(axis cs:984000,10749.7001666667)
--(axis cs:960000,10800.6793333333)
--(axis cs:936000,10629.722075)
--(axis cs:912000,10628.4618333333)
--(axis cs:888000,10648.8323333333)
--(axis cs:864000,10645.1028333333)
--(axis cs:840000,10677.2643333333)
--(axis cs:816000,10592.0876666667)
--(axis cs:792000,10625.3455)
--(axis cs:768000,10569.0306666667)
--(axis cs:744000,10620.795)
--(axis cs:720000,10563.5793333333)
--(axis cs:696000,10397.1985)
--(axis cs:672000,10447.6505)
--(axis cs:648000,10341.1223333333)
--(axis cs:624000,10206.6278333333)
--(axis cs:600000,10278.9731666667)
--(axis cs:576000,10108.8238333333)
--(axis cs:552000,10091.9831666667)
--(axis cs:528000,10016.3041666667)
--(axis cs:504000,9897.1825)
--(axis cs:480000,9732.41)
--(axis cs:456000,9675.42916666667)
--(axis cs:432000,9294.886)
--(axis cs:408000,9609.0575)
--(axis cs:384000,9337.37883333333)
--(axis cs:360000,9195.62183333333)
--(axis cs:336000,9437.80366666667)
--(axis cs:312000,9185.85533333333)
--(axis cs:288000,8148.24433333333)
--(axis cs:264000,8703.90233333333)
--(axis cs:240000,8336.56634041667)
--(axis cs:216000,6965.87235)
--(axis cs:192000,7228.3226)
--(axis cs:168000,5754.61455)
--(axis cs:144000,4976.56908791667)
--(axis cs:120000,3597.12991666667)
--(axis cs:96000,3544.59625)
--(axis cs:72000,1652.0952)
--(axis cs:48000,984.666866666667)
--(axis cs:24000,664.265543333333)
--(axis cs:1,143.951875)
--cycle;

\addplot [line width=\linewidthother, C2, mark=*, mark size=0, mark options={solid}]
table {%
1 88.2032706666667
24000 540.196971666667
48000 860.80465
72000 1292.71484166667
96000 2047.52073333333
120000 1759.67473333333
144000 3294.50793333333
168000 4979.28996666667
192000 6192.48266666667
216000 4864.35875
240000 6995.58205
264000 8090.83121666667
288000 6539.98868333333
312000 8535.56128333333
336000 9051.56783333333
360000 8131.064
384000 8324.96616666667
408000 9087.96116666667
432000 8171.93773333333
456000 9039.16755
480000 8901.07593333333
504000 9712.2745
528000 9355.29466666667
552000 9306.91811666667
576000 9870.30166666667
600000 9971.57016666667
624000 9897.52566666667
648000 9846.763
672000 10096.0916666667
696000 10185.2795
720000 10372.991
744000 10343.6081666667
768000 10339.5741666667
792000 10505.7395
816000 10442.4348333333
840000 10502.973
864000 10484.2701666667
888000 10537.1416666667
912000 10210.5831666667
936000 10478.8866666667
960000 10636.6118333333
984000 10574.9655
};
\path [draw=C0, fill=C0, opacity=0.2]
(axis cs:1,113.976773333333)
--(axis cs:1,98.6404566666667)
--(axis cs:24000,543.750333333333)
--(axis cs:48000,566.534624416668)
--(axis cs:72000,2296.31103333333)
--(axis cs:96000,2662.33088333333)
--(axis cs:120000,1975.78883333333)
--(axis cs:144000,5837.14748333333)
--(axis cs:168000,3430.21438333333)
--(axis cs:192000,4158.954)
--(axis cs:216000,7479.79626666667)
--(axis cs:240000,6359.88058916667)
--(axis cs:264000,7025.80608333333)
--(axis cs:288000,6883.80211666667)
--(axis cs:312000,5514.27671666667)
--(axis cs:336000,5329.96893333333)
--(axis cs:360000,8671.99278333333)
--(axis cs:384000,5899.79371666667)
--(axis cs:408000,9239.98466666667)
--(axis cs:432000,7535.93261666667)
--(axis cs:456000,9157.44256666667)
--(axis cs:480000,7577.01969583334)
--(axis cs:504000,7900.64316583333)
--(axis cs:528000,8428.22536666667)
--(axis cs:552000,10203.8236666667)
--(axis cs:576000,10557.5527666667)
--(axis cs:600000,11246.0935)
--(axis cs:624000,8563.37276666667)
--(axis cs:648000,8844.57485)
--(axis cs:672000,10909.8815)
--(axis cs:696000,10392.3738333333)
--(axis cs:720000,11044.1534166667)
--(axis cs:744000,11672.0638333333)
--(axis cs:768000,9895.88906666667)
--(axis cs:792000,8999.81066666667)
--(axis cs:816000,9395.559)
--(axis cs:840000,10824.9211666667)
--(axis cs:864000,11323.0006666667)
--(axis cs:888000,7160.82135)
--(axis cs:912000,10577.3640708333)
--(axis cs:936000,10925.7906666667)
--(axis cs:960000,11925.2005)
--(axis cs:984000,11017.3888083333)
--(axis cs:984000,12826.5963)
--(axis cs:984000,12826.5963)
--(axis cs:960000,12726.1003333333)
--(axis cs:936000,12883.1791666667)
--(axis cs:912000,12593.295)
--(axis cs:888000,12738.0145)
--(axis cs:864000,13046.7044166667)
--(axis cs:840000,12666.205)
--(axis cs:816000,12274.4813333333)
--(axis cs:792000,12408.528)
--(axis cs:768000,12658.3473333333)
--(axis cs:744000,12653.775)
--(axis cs:720000,12624.245)
--(axis cs:696000,12407.7176666667)
--(axis cs:672000,12317.322)
--(axis cs:648000,12204.0635)
--(axis cs:624000,11113.877)
--(axis cs:600000,12410.6242833333)
--(axis cs:576000,12144.7265)
--(axis cs:552000,12128.1423333333)
--(axis cs:528000,11819.2978333333)
--(axis cs:504000,11805.7681666667)
--(axis cs:480000,11685.696)
--(axis cs:456000,11524.9763333333)
--(axis cs:432000,11864.7096666667)
--(axis cs:408000,12002.0885)
--(axis cs:384000,11802.255)
--(axis cs:360000,11526.1228333333)
--(axis cs:336000,11531.6323333333)
--(axis cs:312000,10029.2376166667)
--(axis cs:288000,10970.65125)
--(axis cs:264000,11077.3695)
--(axis cs:240000,9835.649)
--(axis cs:216000,10498.6938333333)
--(axis cs:192000,8347.68483333333)
--(axis cs:168000,8218.10176666667)
--(axis cs:144000,7772.21748333333)
--(axis cs:120000,5258.9391)
--(axis cs:96000,4597.81678333333)
--(axis cs:72000,3931.3698)
--(axis cs:48000,1361.35441666667)
--(axis cs:24000,678.893635416667)
--(axis cs:1,113.976773333333)
--cycle;

\addplot [line width=\linewidthdime, C0, mark=*, mark size=0, mark options={solid}]
table {%
1 105.609650833333
24000 599.393633333333
48000 1038.91463833333
72000 3179.1111
96000 3740.46818333333
120000 3538.3839
144000 6901.18741666667
168000 5688.52053333333
192000 6955.47028333333
216000 9364.55506666667
240000 7999.80861666667
264000 9740.00183333333
288000 9611.46503333333
312000 7718.08091666667
336000 9647.00696666667
360000 10695.9388333333
384000 10194.6781166667
408000 11393.4956666667
432000 10294.479
456000 10507.0823333333
480000 9961.7985
504000 10587.061
528000 10656.12825
552000 11985.318
576000 11683.3263333333
600000 12150.2681666667
624000 9766.63133333333
648000 11347.6486666667
672000 11861.5316666667
696000 11494.537
720000 12199.9153333333
744000 12347.6205
768000 12341.5195
792000 11430.4128333333
816000 10776.4508333333
840000 12210.7925
864000 12580.2263333333
888000 11545.895
912000 11877.1126666667
936000 12222.218
960000 12442.8011666667
984000 12504.1388333333
};

\end{axis}

\end{tikzpicture}
}
       \subcaption[]{Humanoid-v3}
       \label{fig::exps::humanoid-v3}
    \end{minipage}\hfill
    \vspace{-2.0mm}
    \caption{\textbf{Varying the Number of diffusion steps (a)-(b).} The number of diffusion steps might affect the performance and the computation time. (a) shows DIME's learning curves for varying diffusion steps. \textit{Two} diffusion steps perform badly, whereas \textit{four} and \textit{eight} diffusion steps perform similar but still worse than \textit{16} and \textit{32} diffusion steps which perform similarly. (b) shows the computation time for 1MIO steps of the corresponding learning curves. The smaller the diffusion steps the less computation time is required. \textbf{Learning Curves on Gym Benchmark Suite (c)-(d).} We compare DIME against various diffusion baselines and CrossQ on the (c) \textit{Ant-v3} and (d) \textit{Humanoid-v3} from the Gym suite. While all diffusion-based methods are outperformed by DIME, DIME performs on par with CrossQ on the Ant environment. DIME performs favorably on the high-dimensional \textit{Humanoid-v3} environment where it also outperforms CrossQ.}
\end{figure*}
\section{Experiments}
We analyze DIME's algorithmic features with an intensive ablation study where we clarify the role of the reward scaling parameter $\alpha$, the effect of varying diffusion steps, and the gained performance boost when using a diffusion policy representation over a Gaussian representation. 
In a broad range of 13 sophisticated learning environments from different benchmark suits ranging from mujoco gym \cite{gymopenai}, deepmind control suit (DMC) \cite{dmcontrol}, and myo suite \cite{MyoSuite2022} we compare DIME's performance against recent diffusion-based RL methods \baseline{QSM} \cite{psenkalearning}, \baseline{Diffusion-QL} \cite{wang2023diffusion}, \baseline{Consistency-AC} \cite{dingconsistency} and \baseline{DIPO} \cite{yang2023policy}. Additionally, we compare against the state-of-the-art RL methods \baseline{CrossQ} \cite{bhattcrossq} and \baseline{BRO} \cite{nauman2024bigger}, where we have used the provided learning curves from the latter. Both methods use a Gaussian parameterized policy and have shown remarkable results. The considered environments are challenging locomotion and manipulation learning tasks with up to 39-dimensional action and 223-dimensional observation spaces.  
We have run all learning curves for 10 seeds and report the \textit{interquartile mean (IQM)} with a 95\% stratified bootstrap confidence interval as suggested by \citet{agarwal2021deep}.

\subsection{Ablation Studies}
\textbf{Exploration Control.} The parameter $\alpha$ balances the exploration-exploitation trade-off by scaling the reward signal. We analyze the effect of this parameter by comparing DIME's learning curves with different $\alpha$ values on the dog-run task from the DMC (see Fig. \ref{fig::exps_new_ablations_vary_alpha::dog_run}). Additionally, we show the performance of the last return measurements for each learning curve in Fig. \ref{fig::exps_new_ablations_vary_alpha::dog_run_pareto}. Too high $\alpha$ values ($\alpha=0.1$) do not incentivize maximizing the task's return leading to no learning at all, whereas small values ($\alpha\leq10^{-5})$ lead to suboptimal performance because the policy does not explore sufficiently. We can also see a clear trend that starting from $\alpha=10^{-12}$ the performance gradually increases until the best performance is reached for $\alpha=10^{-3}$. 

\textbf{Diffusion Policy Benefit.} We aim to analyze the performance benefits of the diffusion-parameterized policy compared to a Gaussian parameterization in the same setup by only exchanging the policy and the corresponding policy update. This comparison ensures that the Gaussian policy is trained with the identical implementation details from DIME as described in Sec. \ref{sec:implementation details} and showcases the performance benefits of a diffusion-based policy. Fig. \ref{fig::exps_new_ablations_gauss_vs_diff_distrq::humanoid_run} and  \ref{fig::exps_new_ablations_gauss_vs_diff_distrq::dog_run} show the learning curves of both versions on DMC's humanoid-run and dog-run environments. The diffusion policy's expressivity leads to a higher aggregated return in the humanoid-run and to significantly faster convergence in the high-dimensional dog-run task. We attribute this performance benefit to an improved exploration behavior.  

\textbf{Number of Diffusion Steps.} The number of diffusion steps determines how accurately the stochastic differential equations are simulated and is a hyperparameter that affects the performance. Usually, the higher the number of diffusion steps the better the model performs at the burden of higher computational costs. In Fig. \ref{fig::exps_new_ablations_vary_diff_steps::humanoid_run} we plot DIME's performance for varying diffusion steps on DMC's humanoid-run environment and report the corresponding runtimes for 1 Mio environment steps in Fig. \ref{fig::exps_new_ablations_runtime_diff_steps::humanoid_run} on an \textit{Nvidia A100} GPU machine. With an increasing number of diffusion steps, the performance and runtime increases. However, from $16$ diffusion steps on, the performance stays the same.

\subsection{Comparisson Against Baselines}
We consider environments with high dimensional observation and action spaces from three benchmark suits for a robust performance assessment (please see Appendix \ref{apdx::environment_details}). 

\textbf{Gym Environments.} Fig \ref{fig::exps_new::ant-v3} and Fig. \ref{fig::exps::humanoid-v3} show the learning curves for the \textit{An-tv3} and \textit{Humanoid-v3} tasks respectively. While the diffusion-based baselines perform reasonably well on the \textit{Ant-v3} task with DIPO outperforming the rest, they are all outperformed by DIME and CrossQ which perform comparably. On the \textit{Humanoid-v3} DIME achieves a significantly higher return than all baselines.

\begin{figure*}[t!]
    \centering  
    \begin{minipage}[b]{0.245\textwidth}
        \centering
       \resizebox{1\textwidth}{!}{% This file was created with tikzplotlib v0.10.1.
\begin{tikzpicture}

\definecolor{darkcyan1115178}{RGB}{1,115,178}
\definecolor{darkgray176}{RGB}{176,176,176}

\begin{axis}[
legend cell align={left},
legend style={fill opacity=0.8, draw opacity=1, text opacity=1, draw=lightgray204, at={(0.03,0.03)},  anchor=north west},
tick align=outside,
tick pos=left,
x grid style={white},
xlabel={Number Env Interactions},
xmajorgrids,
%xmin=-37798.95, xmax=1045799.95,
xmin=-0.0, xmax=1000000.0,
xtick style={color=black},
y grid style={white},
ylabel={IQM Mean Return},
ymajorgrids,
ymin=-0.05, ymax=650,
ytick style={color=black},
axis background/.style={fill=plot_background},
label style={font=\large},
tick label style={font=\large},
x axis line style={draw=none},
y axis line style={draw=none},
]
%Consistency-AC
\path [draw=C5, fill=C5, opacity=0.2]
(axis cs:10000,11.463926848519)
--(axis cs:10000,5.37603210746611)
--(axis cs:50000,4.39841287621437)
--(axis cs:90000,9.17009640948666)
--(axis cs:130000,12.5870741668558)
--(axis cs:170000,9.68308216662577)
--(axis cs:210000,14.8925226931926)
--(axis cs:250000,11.1820716312957)
--(axis cs:290000,11.396756534118)
--(axis cs:330000,7.99325851842966)
--(axis cs:370000,11.1976030725659)
--(axis cs:410000,10.9248649239103)
--(axis cs:450000,8.95682571708669)
--(axis cs:490000,11.3816416377384)
--(axis cs:530000,12.8031758180575)
--(axis cs:570000,5.93578222734027)
--(axis cs:610000,8.21484428635898)
--(axis cs:650000,5.83358756157396)
--(axis cs:690000,9.46063935999991)
--(axis cs:730000,11.8192892604622)
--(axis cs:770000,12.6324885084212)
--(axis cs:810000,10.7473805476358)
--(axis cs:850000,7.25710635201092)
--(axis cs:890000,9.22131175112653)
--(axis cs:930000,13.2959293771329)
--(axis cs:970000,9.07029053354509)
--(axis cs:970000,37.8713919076384)
--(axis cs:970000,37.8713919076384)
--(axis cs:930000,31.9938327977292)
--(axis cs:890000,38.3203005660843)
--(axis cs:850000,28.0270492957679)
--(axis cs:810000,30.4347320062739)
--(axis cs:770000,37.3854786322548)
--(axis cs:730000,26.5349621208757)
--(axis cs:690000,32.7462369505973)
--(axis cs:650000,32.9976763829575)
--(axis cs:610000,22.0810080347503)
--(axis cs:570000,23.99523732983)
--(axis cs:530000,31.4375655225092)
--(axis cs:490000,29.0553190783567)
--(axis cs:450000,20.2524552334538)
--(axis cs:410000,27.2952511671052)
--(axis cs:370000,37.7092802874593)
--(axis cs:330000,34.8692907577358)
--(axis cs:290000,37.059821440363)
--(axis cs:250000,32.3343063705168)
--(axis cs:210000,36.8449215660586)
--(axis cs:170000,35.6687195813457)
--(axis cs:130000,28.0108828768184)
--(axis cs:90000,16.3792435674084)
--(axis cs:50000,9.69504458234381)
--(axis cs:10000,11.463926848519)
--cycle;

\addplot [line width=\linewidthother, C5, mark=*, mark size=0, mark options={solid}]
table {%
10000 7.85790325056077
50000 6.35860284032434
90000 13.2047146607297
130000 20.2989785218371
170000 19.8783841188691
210000 25.1934057825873
250000 20.6458589247297
290000 22.4396386818743
330000 19.9098415026298
370000 23.2276422103183
410000 19.3148549082268
450000 14.1367493320261
490000 19.0624373739155
530000 21.7560782762676
570000 13.6733204746371
610000 14.5346796841287
650000 17.6235735715368
690000 21.0359604016945
730000 18.6881061488316
770000 25.008983570338
810000 20.5910562769549
850000 16.0209033037917
890000 23.7708061586054
930000 22.2085724808314
970000 23.2810248149431
};

% BRO
\path [draw=C1, fill=C1, opacity=0.2]
(axis cs:25000,10.4214828010349)
--(axis cs:25000,5.74333818509858)
--(axis cs:50000,8.09432444087523)
--(axis cs:75000,9.90836608721418)
--(axis cs:100000,13.4284553407894)
--(axis cs:125000,28.5942453949379)
--(axis cs:150000,45.3514736374534)
--(axis cs:175000,63.5466645826227)
--(axis cs:200000,75.4050965367814)
--(axis cs:225000,96.0977329143482)
--(axis cs:250000,112.144633001037)
--(axis cs:275000,69.758808886236)
--(axis cs:300000,126.88587225159)
--(axis cs:325000,151.854080295755)
--(axis cs:350000,164.689782841864)
--(axis cs:375000,176.267798146898)
--(axis cs:400000,189.528921194811)
--(axis cs:425000,199.237052941994)
--(axis cs:450000,216.76527008714)
--(axis cs:475000,221.172823432845)
--(axis cs:500000,242.047785506084)
--(axis cs:525000,131.845836334689)
--(axis cs:550000,223.08224577297)
--(axis cs:575000,252.840470629034)
--(axis cs:600000,275.886393376962)
--(axis cs:625000,285.859410268162)
--(axis cs:650000,285.091806186583)
--(axis cs:675000,313.370579224885)
--(axis cs:700000,348.91564649927)
--(axis cs:725000,356.289859109605)
--(axis cs:750000,380.482061980522)
--(axis cs:775000,118.739536778631)
--(axis cs:800000,229.14440596249)
--(axis cs:825000,328.406216187075)
--(axis cs:850000,360.007008496921)
--(axis cs:875000,417.747465127326)
--(axis cs:900000,407.011657970625)
--(axis cs:925000,402.973664549389)
--(axis cs:950000,447.23319412658)
--(axis cs:975000,442.036693024441)
--(axis cs:1000000,456.581953317235)
--(axis cs:1000000,496.59024771113)
--(axis cs:1000000,496.59024771113)
--(axis cs:975000,500.244726881671)
--(axis cs:950000,493.170094491512)
--(axis cs:925000,475.46384217299)
--(axis cs:900000,457.711206302229)
--(axis cs:875000,453.350626938457)
--(axis cs:850000,395.073608046901)
--(axis cs:825000,358.933679864953)
--(axis cs:800000,269.559873348413)
--(axis cs:775000,143.266187556657)
--(axis cs:750000,420.817298900866)
--(axis cs:725000,408.555355122614)
--(axis cs:700000,395.166196894937)
--(axis cs:675000,377.23826528108)
--(axis cs:650000,333.916215702771)
--(axis cs:625000,335.57168171148)
--(axis cs:600000,317.262483428054)
--(axis cs:575000,297.120865242452)
--(axis cs:550000,259.923563574532)
--(axis cs:525000,150.311833808402)
--(axis cs:500000,292.772584457249)
--(axis cs:475000,292.135052782577)
--(axis cs:450000,267.872285614804)
--(axis cs:425000,243.291577456831)
--(axis cs:400000,241.393645358009)
--(axis cs:375000,214.800057235155)
--(axis cs:350000,196.464908151334)
--(axis cs:325000,189.755589772141)
--(axis cs:300000,164.593901002474)
--(axis cs:275000,107.111385050975)
--(axis cs:250000,158.16812429377)
--(axis cs:225000,140.832875677263)
--(axis cs:200000,124.605494930318)
--(axis cs:175000,103.327680424693)
--(axis cs:150000,89.2303886773787)
--(axis cs:125000,54.2583886820005)
--(axis cs:100000,32.8955349031664)
--(axis cs:75000,21.7858682309204)
--(axis cs:50000,18.1647642728865)
--(axis cs:25000,10.4214828010349)
--cycle;

\addplot [line width=\linewidthother, C1, mark=*, mark size=0, mark options={solid}]
table {%
25000 7.69078524526657
50000 12.7531337620864
75000 16.4813932058173
100000 20.9576496494661
125000 40.9559750123288
150000 60.3425381404668
175000 82.1009523487325
200000 97.9155570347124
225000 116.102735256329
250000 129.831786801786
275000 87.8189239900971
300000 143.560432818343
325000 158.198346748268
350000 173.236261187279
375000 184.46223583009
400000 201.289153423342
425000 211.007935561594
450000 238.189807796344
475000 248.385497862529
500000 265.493017462384
525000 138.929177278294
550000 238.047698713837
575000 272.57812456009
600000 292.037445610574
625000 305.485935589137
650000 313.27007681098
675000 338.308375304241
700000 382.29668702033
725000 383.13758244702
750000 397.500063267137
775000 134.47735028663
800000 254.98358315637
825000 342.750590156139
850000 377.967783546172
875000 435.965517877994
900000 431.952717196056
925000 442.435174817137
950000 471.334996063751
975000 472.695410507355
1000000 469.884836753752
};
% QSM
\path [draw=C3, fill=C3, opacity=0.2]
(axis cs:10000,9.00637129942576)
--(axis cs:10000,3.31341858704885)
--(axis cs:30000,5.57260828216871)
--(axis cs:50000,8.02879546334346)
--(axis cs:70000,8.96538097100953)
--(axis cs:90000,10.0447169281542)
--(axis cs:110000,11.083192997224)
--(axis cs:130000,15.8602566473863)
--(axis cs:150000,20.0827251935657)
--(axis cs:170000,21.7714599737985)
--(axis cs:190000,21.2784577034276)
--(axis cs:210000,29.2425191401357)
--(axis cs:230000,29.1000896614077)
--(axis cs:250000,30.7142972597102)
--(axis cs:270000,37.1213915977744)
--(axis cs:290000,35.383553986563)
--(axis cs:310000,34.5792193715324)
--(axis cs:330000,37.180703958727)
--(axis cs:350000,39.9991572830434)
--(axis cs:370000,40.0996391593703)
--(axis cs:390000,45.3179218579362)
--(axis cs:410000,43.4128867200092)
--(axis cs:430000,47.2341654755311)
--(axis cs:450000,51.4904100888942)
--(axis cs:470000,56.1438953381185)
--(axis cs:490000,56.3643598825837)
--(axis cs:510000,41.8560678676204)
--(axis cs:530000,39.4311747974906)
--(axis cs:550000,39.2973707636431)
--(axis cs:570000,42.1655587598545)
--(axis cs:590000,46.474886428713)
--(axis cs:610000,35.9292819193922)
--(axis cs:630000,37.867541948254)
--(axis cs:650000,37.0515135860717)
--(axis cs:670000,37.5128013938228)
--(axis cs:690000,30.5023094831127)
--(axis cs:710000,40.438926076983)
--(axis cs:730000,41.6069058038226)
--(axis cs:750000,40.8430295724723)
--(axis cs:770000,39.7151342713554)
--(axis cs:790000,43.331485201916)
--(axis cs:810000,45.728076310577)
--(axis cs:830000,41.9312130426918)
--(axis cs:850000,31.1859364236329)
--(axis cs:870000,26.1690699101898)
--(axis cs:890000,30.3492033461859)
--(axis cs:910000,21.6742292892421)
--(axis cs:930000,15.00353959044)
--(axis cs:950000,12.6778280563617)
--(axis cs:970000,10.276360763814)
--(axis cs:990000,16.3133930613083)
--(axis cs:990000,116.662159246092)
--(axis cs:990000,116.662159246092)
--(axis cs:970000,120.475571672727)
--(axis cs:950000,113.793179632632)
--(axis cs:930000,114.084972864025)
--(axis cs:910000,112.135332782087)
--(axis cs:890000,115.757407879623)
--(axis cs:870000,111.72505328964)
--(axis cs:850000,114.856922483414)
--(axis cs:830000,117.448915056765)
--(axis cs:810000,119.459803820126)
--(axis cs:790000,115.453403901555)
--(axis cs:770000,111.913551390986)
--(axis cs:750000,104.420705974677)
--(axis cs:730000,101.661488486193)
--(axis cs:710000,116.777563783003)
--(axis cs:690000,109.956826669484)
--(axis cs:670000,109.846564750774)
--(axis cs:650000,109.69722781989)
--(axis cs:630000,104.017540868934)
--(axis cs:610000,99.3662446587841)
--(axis cs:590000,101.502700353239)
--(axis cs:570000,102.777559600424)
--(axis cs:550000,96.6413273589876)
--(axis cs:530000,95.0563691400709)
--(axis cs:510000,93.7333309362891)
--(axis cs:490000,96.5183300919841)
--(axis cs:470000,93.7539031450182)
--(axis cs:450000,95.9979843577684)
--(axis cs:430000,88.2955526826802)
--(axis cs:410000,85.5825312019854)
--(axis cs:390000,83.9140755855183)
--(axis cs:370000,79.393379419303)
--(axis cs:350000,75.9247120943175)
--(axis cs:330000,71.4454448500963)
--(axis cs:310000,69.8080548912829)
--(axis cs:290000,67.6769566217011)
--(axis cs:270000,64.9307342505208)
--(axis cs:250000,60.1687479824038)
--(axis cs:230000,57.4818564634574)
--(axis cs:210000,53.5049428912655)
--(axis cs:190000,52.0394126697339)
--(axis cs:170000,45.0678509664825)
--(axis cs:150000,40.4675526829694)
--(axis cs:130000,36.4301813118218)
--(axis cs:110000,30.0792382984102)
--(axis cs:90000,25.7764486732582)
--(axis cs:70000,19.3209997828429)
--(axis cs:50000,13.7913653403521)
--(axis cs:30000,8.02777871837219)
--(axis cs:10000,9.00637129942576)
--cycle;


\addplot [line width=\linewidthother, C3, mark=*, mark size=0, mark options={solid}]
table {%
10000 5.48336112499237
30000 6.85617107152939
50000 10.2728418062131
70000 12.3853853059312
90000 16.2610089325656
110000 19.3796129614736
130000 28.1669670593013
150000 32.534822997574
170000 36.6120637002399
190000 40.751044902772
210000 46.2025473896589
230000 51.0763229516388
250000 52.771598438105
270000 56.1948414493802
290000 59.3300952228345
310000 61.2587398424496
330000 64.4361855497106
350000 67.4036484636402
370000 71.8194046238446
390000 73.2278073683228
410000 68.1734948966614
430000 70.3253361188797
450000 78.7786298544066
470000 81.7159956085398
490000 81.7640809562731
510000 69.7941905260474
530000 71.9224686126011
550000 73.348937141401
570000 80.0750538907225
590000 83.2811232275952
610000 71.7244561219077
630000 78.5604585154285
650000 81.8548987770866
670000 81.5800458606281
690000 75.6349773808369
710000 86.9667827749601
730000 80.4607181570385
750000 82.0978496071994
770000 86.2036625367779
790000 89.4833001638254
810000 93.2583586598661
830000 89.1554655603121
850000 76.3665210100897
870000 72.2112299010237
890000 75.572761595307
910000 65.2483546448566
930000 62.0938081454536
950000 60.9241431782228
970000 63.5439467565627
990000 64.2725527717772
};

\path [draw=C2, fill=C2, opacity=0.2]
(axis cs:1,5.08870304)
--(axis cs:1,3.68965048)
--(axis cs:24000,7.62991894)
--(axis cs:48000,7.93215652)
--(axis cs:72000,6.47293496650009)
--(axis cs:96000,12.587974)
--(axis cs:120000,27.3698494)
--(axis cs:144000,28.12897)
--(axis cs:168000,36.614457)
--(axis cs:192000,62.045424)
--(axis cs:216000,52.0521822)
--(axis cs:240000,76.786258)
--(axis cs:264000,94.280188)
--(axis cs:288000,74.9881499350004)
--(axis cs:312000,71.0480906)
--(axis cs:336000,87.6450248)
--(axis cs:360000,128.128906)
--(axis cs:384000,73.9193625100026)
--(axis cs:408000,136.3864164)
--(axis cs:432000,114.8276584)
--(axis cs:456000,167.233562)
--(axis cs:480000,162.955824)
--(axis cs:504000,169.927532)
--(axis cs:528000,176.322408)
--(axis cs:552000,187.024622)
--(axis cs:576000,163.487644)
--(axis cs:600000,204.19103)
--(axis cs:624000,221.893322)
--(axis cs:648000,231.709012)
--(axis cs:672000,236.976125050001)
--(axis cs:696000,176.3649328)
--(axis cs:720000,290.38816)
--(axis cs:744000,305.915814)
--(axis cs:768000,230.663312)
--(axis cs:792000,239.332844)
--(axis cs:816000,277.67124)
--(axis cs:840000,287.474624)
--(axis cs:864000,262.481308)
--(axis cs:888000,318.412994)
--(axis cs:912000,323.347184)
--(axis cs:936000,215.92877)
--(axis cs:960000,371.48532)
--(axis cs:984000,369.806744)
--(axis cs:1008000,395.842144)
--(axis cs:1032000,250.919718510006)
--(axis cs:1056000,411.625820300001)
--(axis cs:1080000,407.048594)
--(axis cs:1104000,419.546388)
--(axis cs:1128000,454.343112)
--(axis cs:1152000,268.321736)
--(axis cs:1176000,306.649356)
--(axis cs:1200000,473.455544)
--(axis cs:1224000,468.99469)
--(axis cs:1248000,392.569902)
--(axis cs:1272000,367.470216)
--(axis cs:1296000,514.64103)
--(axis cs:1320000,525.853924)
--(axis cs:1344000,523.617598)
--(axis cs:1368000,505.30544)
--(axis cs:1392000,481.589782)
--(axis cs:1416000,546.605916)
--(axis cs:1440000,549.00744)
--(axis cs:1464000,516.795564)
--(axis cs:1488000,530.279436)
--(axis cs:1512000,537.87598)
--(axis cs:1536000,583.42678)
--(axis cs:1560000,578.29042)
--(axis cs:1584000,349.135284)
--(axis cs:1608000,524.726164)
--(axis cs:1632000,598.62813)
--(axis cs:1656000,395.3759324)
--(axis cs:1680000,578.054076)
--(axis cs:1704000,368.5988572)
--(axis cs:1728000,560.359838)
--(axis cs:1752000,605.988216)
--(axis cs:1776000,616.570372)
--(axis cs:1800000,586.88821)
--(axis cs:1824000,608.40061)
--(axis cs:1848000,601.388426)
--(axis cs:1872000,599.06138)
--(axis cs:1896000,490.962722)
--(axis cs:1920000,276.865072)
--(axis cs:1944000,538.00442)
--(axis cs:1968000,646.4052)
--(axis cs:1992000,633.225998)
--(axis cs:2016000,639.749272)
--(axis cs:2040000,529.704108)
--(axis cs:2064000,657.623612)
--(axis cs:2088000,606.7978)
--(axis cs:2112000,557.042702)
--(axis cs:2136000,642.73752)
--(axis cs:2160000,645.962118)
--(axis cs:2184000,422.8458116)
--(axis cs:2208000,580.496716)
--(axis cs:2232000,687.99286)
--(axis cs:2256000,596.495068)
--(axis cs:2280000,688.213466)
--(axis cs:2304000,616.179208)
--(axis cs:2328000,581.54152)
--(axis cs:2352000,419.4881004)
--(axis cs:2376000,502.770992)
--(axis cs:2400000,600.524793000003)
--(axis cs:2424000,664.403404)
--(axis cs:2448000,627.83858)
--(axis cs:2472000,687.23751)
--(axis cs:2496000,597.76742)
--(axis cs:2520000,656.83965)
--(axis cs:2544000,526.801084)
--(axis cs:2568000,470.997188)
--(axis cs:2592000,636.76534)
--(axis cs:2616000,569.29958)
--(axis cs:2640000,672.796042)
--(axis cs:2664000,630.420594)
--(axis cs:2688000,550.655396)
--(axis cs:2712000,644.0207)
--(axis cs:2736000,589.857666)
--(axis cs:2760000,666.78616)
--(axis cs:2784000,617.303856)
--(axis cs:2808000,658.61183)
--(axis cs:2832000,682.914396)
--(axis cs:2856000,686.516124)
--(axis cs:2880000,710.847194)
--(axis cs:2904000,521.721572)
--(axis cs:2928000,504.00449)
--(axis cs:2952000,715.43871)
--(axis cs:2976000,609.043312)
--(axis cs:3000000,697.05923)
--(axis cs:3000000,816.05479)
--(axis cs:3000000,816.05479)
--(axis cs:2976000,816.833628)
--(axis cs:2952000,820.42984)
--(axis cs:2928000,802.37344)
--(axis cs:2904000,805.968666)
--(axis cs:2880000,797.86731)
--(axis cs:2856000,826.739952)
--(axis cs:2832000,806.72521)
--(axis cs:2808000,800.223108)
--(axis cs:2784000,817.777648)
--(axis cs:2760000,810.177492)
--(axis cs:2736000,804.11192)
--(axis cs:2712000,775.78706)
--(axis cs:2688000,798.70296)
--(axis cs:2664000,800.247196)
--(axis cs:2640000,806.08496)
--(axis cs:2616000,773.087792)
--(axis cs:2592000,757.598130200002)
--(axis cs:2568000,782.15722)
--(axis cs:2544000,792.273964)
--(axis cs:2520000,757.921734)
--(axis cs:2496000,748.993818)
--(axis cs:2472000,770.14045)
--(axis cs:2448000,748.8993)
--(axis cs:2424000,745.527544)
--(axis cs:2400000,743.45228)
--(axis cs:2376000,769.958568)
--(axis cs:2352000,718.61474)
--(axis cs:2328000,794.272146)
--(axis cs:2304000,802.248602)
--(axis cs:2280000,786.811476)
--(axis cs:2256000,760.0894)
--(axis cs:2232000,790.97057)
--(axis cs:2208000,775.016552)
--(axis cs:2184000,768.423152)
--(axis cs:2160000,743.21318)
--(axis cs:2136000,765.272132)
--(axis cs:2112000,759.624902)
--(axis cs:2088000,748.89576)
--(axis cs:2064000,745.63012)
--(axis cs:2040000,722.20737)
--(axis cs:2016000,743.0933)
--(axis cs:1992000,724.294948)
--(axis cs:1968000,752.22605)
--(axis cs:1944000,728.1269)
--(axis cs:1920000,730.239636)
--(axis cs:1896000,696.265756)
--(axis cs:1872000,716.48952)
--(axis cs:1848000,722.897252)
--(axis cs:1824000,700.40674)
--(axis cs:1800000,681.427446)
--(axis cs:1776000,702.65155)
--(axis cs:1752000,679.932826)
--(axis cs:1728000,679.549276)
--(axis cs:1704000,715.56346)
--(axis cs:1680000,688.76336)
--(axis cs:1656000,683.049632)
--(axis cs:1632000,693.61822)
--(axis cs:1608000,687.3224)
--(axis cs:1584000,651.290338)
--(axis cs:1560000,674.12986)
--(axis cs:1536000,675.877516)
--(axis cs:1512000,669.480454)
--(axis cs:1488000,638.062)
--(axis cs:1464000,662.310476)
--(axis cs:1440000,652.789328)
--(axis cs:1416000,651.846968)
--(axis cs:1392000,635.23482)
--(axis cs:1368000,635.917)
--(axis cs:1344000,635.4749)
--(axis cs:1320000,616.55244)
--(axis cs:1296000,619.092528)
--(axis cs:1272000,592.115148)
--(axis cs:1248000,585.497024)
--(axis cs:1224000,596.60722)
--(axis cs:1200000,594.4673)
--(axis cs:1176000,591.790048)
--(axis cs:1152000,549.044842)
--(axis cs:1128000,568.078136)
--(axis cs:1104000,576.72876)
--(axis cs:1080000,553.44684)
--(axis cs:1056000,546.617916)
--(axis cs:1032000,543.060196)
--(axis cs:1008000,522.083816)
--(axis cs:984000,529.931406)
--(axis cs:960000,524.29467)
--(axis cs:936000,484.517842)
--(axis cs:912000,485.825776)
--(axis cs:888000,461.478972)
--(axis cs:864000,476.117322)
--(axis cs:840000,469.27586)
--(axis cs:816000,464.6017)
--(axis cs:792000,441.409176)
--(axis cs:768000,434.262428)
--(axis cs:744000,423.281106)
--(axis cs:720000,414.170984)
--(axis cs:696000,374.5067)
--(axis cs:672000,386.673746)
--(axis cs:648000,364.314222)
--(axis cs:624000,351.766138)
--(axis cs:600000,332.345748)
--(axis cs:576000,313.014292)
--(axis cs:552000,284.177572)
--(axis cs:528000,224.632766)
--(axis cs:504000,269.586118)
--(axis cs:480000,243.80664)
--(axis cs:456000,229.6212)
--(axis cs:432000,184.839836)
--(axis cs:408000,186.79067)
--(axis cs:384000,170.161898)
--(axis cs:360000,178.871194)
--(axis cs:336000,170.301876)
--(axis cs:312000,145.5524092)
--(axis cs:288000,154.151556)
--(axis cs:264000,123.806023)
--(axis cs:240000,121.281625)
--(axis cs:216000,101.8625024)
--(axis cs:192000,111.603644)
--(axis cs:168000,84.20818)
--(axis cs:144000,66.140008)
--(axis cs:120000,53.8429784)
--(axis cs:96000,36.8297434)
--(axis cs:72000,30.9384382)
--(axis cs:48000,22.3311096)
--(axis cs:24000,12.7853277)
--(axis cs:1,5.08870304)
--cycle;

\addplot [line width=\linewidthother, C2, mark=*, mark size=0, mark options={solid}]
table {%
1 4.482758
24000 8.95052342
48000 13.292312
72000 17.10010038
96000 24.2130636
120000 44.2345288
144000 44.805972
168000 58.174146
192000 86.655458
216000 76.4512708
240000 96.9551522
264000 106.345815
288000 113.5489152
312000 125.7857092
336000 149.555445
360000 155.09633
384000 135.472366
408000 168.755284
432000 164.716176
456000 176.57803
480000 198.114586
504000 210.71531
528000 197.930814
552000 218.164982
576000 245.695396
600000 260.076062
624000 279.168826
648000 288.94217
672000 311.86936
696000 311.543426
720000 342.935496
744000 358.51124
768000 356.785058
792000 346.377912
816000 384.285682
840000 396.962226
864000 403.701502
888000 404.300138
912000 421.602336
936000 371.433818
960000 458.90596
984000 466.7362
1008000 475.609888
1032000 439.628622
1056000 485.556884
1080000 486.492076
1104000 505.164156
1128000 504.46871
1152000 461.414462
1176000 503.718468
1200000 522.697416
1224000 527.268044
1248000 506.490704
1272000 510.735564
1296000 559.11353
1320000 569.6844
1344000 569.031614
1368000 555.68144
1392000 583.679534
1416000 589.009554
1440000 594.783848
1464000 590.90014
1488000 570.207448
1512000 614.117482
1536000 619.260512
1560000 610.24442
1584000 569.768914
1608000 629.699378
1632000 634.41287
1656000 609.00341
1680000 635.959072
1704000 633.569498
1728000 623.042454
1752000 639.838396
1776000 648.812748
1800000 617.114056
1824000 647.1812
1848000 656.620386
1872000 660.19625
1896000 620.346456
1920000 554.823928
1944000 630.0526
1968000 686.13246
1992000 674.612466
2016000 680.493322
2040000 656.338942
2064000 694.551632
2088000 681.743874
2112000 701.884236
2136000 702.578264
2160000 692.050634
2184000 698.019532
2208000 715.948058
2232000 733.527184
2256000 682.839552
2280000 732.036794
2304000 720.45979
2328000 736.558496
2352000 663.161484
2376000 720.5217
2400000 697.58044
2424000 708.987852
2448000 691.49205
2472000 724.504812
2496000 700.70355
2520000 701.45418
2544000 734.79906
2568000 747.058346
2592000 725.77554
2616000 725.551706
2640000 748.126478
2664000 738.848702
2688000 730.628632
2712000 701.435
2736000 745.88346
2760000 753.473008
2784000 737.01486
2808000 750.083314
2832000 739.435622
2856000 762.29916
2880000 743.762224
2904000 739.993606
2928000 735.47718
2952000 777.263996
2976000 747.561014
3000000 765.32302
};

%BRO FAST 
\path [draw=C9, fill=C9, opacity=0.2]
(axis cs:25000,5.9627663029069)
--(axis cs:25000,3.94060263678752)
--(axis cs:50000,4.95329533063607)
--(axis cs:75000,7.2229154274937)
--(axis cs:100000,8.99076950219844)
--(axis cs:125000,7.88746522576995)
--(axis cs:150000,9.9448038857203)
--(axis cs:175000,20.6593185029793)
--(axis cs:200000,23.0857594811648)
--(axis cs:225000,31.1824613187434)
--(axis cs:250000,35.3533155525059)
--(axis cs:275000,50.8855500678719)
--(axis cs:300000,70.896840543864)
--(axis cs:325000,91.822405905343)
--(axis cs:350000,113.084879165756)
--(axis cs:375000,122.774920886158)
--(axis cs:400000,141.378469992151)
--(axis cs:425000,145.33166528664)
--(axis cs:450000,151.666671178168)
--(axis cs:475000,176.85704034099)
--(axis cs:500000,182.085180871627)
--(axis cs:525000,191.751877225063)
--(axis cs:550000,196.599502475127)
--(axis cs:575000,220.690109788465)
--(axis cs:600000,233.046428298572)
--(axis cs:625000,231.563550852072)
--(axis cs:650000,248.772572748877)
--(axis cs:675000,256.393197597437)
--(axis cs:700000,277.473464178774)
--(axis cs:725000,269.557616819856)
--(axis cs:750000,290.371096291129)
--(axis cs:775000,291.060337269253)
--(axis cs:800000,309.919693983718)
--(axis cs:825000,312.742653871502)
--(axis cs:850000,309.756613917194)
--(axis cs:875000,323.530706851565)
--(axis cs:900000,327.138909058089)
--(axis cs:925000,326.036401919083)
--(axis cs:950000,343.340258361995)
--(axis cs:975000,339.367940478116)
--(axis cs:1000000,333.206963775736)
--(axis cs:1000000,421.840029904296)
--(axis cs:1000000,421.840029904296)
--(axis cs:975000,426.477170525367)
--(axis cs:950000,408.366883033887)
--(axis cs:925000,378.320544206267)
--(axis cs:900000,375.814281035147)
--(axis cs:875000,359.889404656802)
--(axis cs:850000,353.210436290579)
--(axis cs:825000,358.3961066085)
--(axis cs:800000,360.80720507426)
--(axis cs:775000,333.439747195314)
--(axis cs:750000,325.96784250545)
--(axis cs:725000,310.065576884353)
--(axis cs:700000,317.662111605769)
--(axis cs:675000,287.03975300699)
--(axis cs:650000,287.166302214974)
--(axis cs:625000,273.632184905015)
--(axis cs:600000,260.860218679205)
--(axis cs:575000,257.138188475866)
--(axis cs:550000,248.544989304806)
--(axis cs:525000,229.709147298703)
--(axis cs:500000,219.532839209569)
--(axis cs:475000,201.932086693417)
--(axis cs:450000,190.087953115992)
--(axis cs:425000,184.630699497815)
--(axis cs:400000,166.354795684718)
--(axis cs:375000,153.138164951746)
--(axis cs:350000,143.826268523414)
--(axis cs:325000,137.892671673645)
--(axis cs:300000,121.06169983147)
--(axis cs:275000,103.340431962028)
--(axis cs:250000,83.169536368522)
--(axis cs:225000,55.4017394189915)
--(axis cs:200000,51.3296215803654)
--(axis cs:175000,48.0146769335392)
--(axis cs:150000,32.2985969187401)
--(axis cs:125000,20.4689000428325)
--(axis cs:100000,20.8867322996787)
--(axis cs:75000,11.996196520421)
--(axis cs:50000,10.8775441795892)
--(axis cs:25000,5.9627663029069)
--cycle;

\addplot [line width=\linewidthother, C9, mark=*, mark size=0, mark options={solid}]
table {%
25000 4.76931211996678
50000 7.75830953287283
75000 8.77929464496367
100000 14.0284487422699
125000 12.9102899220444
150000 18.8464043718836
175000 35.6132009401844
200000 36.2773846135605
225000 45.1248353278597
250000 54.2267690563538
275000 75.9807331931164
300000 95.4707102928311
325000 115.161371324838
350000 125.108995868431
375000 136.23022767119
400000 155.6035762322
425000 162.172617805304
450000 171.636124694723
475000 188.793437151646
500000 199.208471490115
525000 213.545280886908
550000 225.127755950347
575000 241.517190907147
600000 248.454736402755
625000 252.222876139156
650000 268.133074430179
675000 271.934776224982
700000 304.489538089164
725000 291.238290843002
750000 305.63835994789
775000 317.393607025809
800000 338.060161910782
825000 340.309283163538
850000 338.836958460671
875000 345.69471729392
900000 356.261290709827
925000 355.250340096369
950000 377.839546363
975000 384.180677473021
1000000 378.582505837938
};

%Diff-QL
\path [draw=C4, fill=C4, opacity=0.2]
(axis cs:10000,10.2812425960027)
--(axis cs:10000,6.02198363619497)
--(axis cs:50000,5.6899344243029)
--(axis cs:90000,5.71575920360953)
--(axis cs:130000,6.37500854819906)
--(axis cs:170000,13.928629817174)
--(axis cs:210000,12.2830882066452)
--(axis cs:250000,13.7931790285666)
--(axis cs:290000,23.6162004623858)
--(axis cs:330000,26.429559016189)
--(axis cs:370000,16.2833425090443)
--(axis cs:410000,19.0172584116363)
--(axis cs:450000,15.4066594707813)
--(axis cs:490000,23.9038833711864)
--(axis cs:530000,26.9407819846307)
--(axis cs:570000,24.0161717051929)
--(axis cs:610000,26.1196342836227)
--(axis cs:650000,32.9924744002448)
--(axis cs:690000,19.4350844063726)
--(axis cs:730000,17.6448948034238)
--(axis cs:770000,21.3235667429693)
--(axis cs:810000,23.7898332859306)
--(axis cs:850000,23.7954429329512)
--(axis cs:890000,25.4711678656995)
--(axis cs:930000,31.1899759133195)
--(axis cs:970000,36.6733448204817)
--(axis cs:970000,65.3130953140278)
--(axis cs:970000,65.3130953140278)
--(axis cs:930000,60.3616330092323)
--(axis cs:890000,60.697914227961)
--(axis cs:850000,48.8677851258972)
--(axis cs:810000,51.4295290804306)
--(axis cs:770000,47.6241784873465)
--(axis cs:730000,58.2453930520975)
--(axis cs:690000,57.2006607227685)
--(axis cs:650000,59.7837199585962)
--(axis cs:610000,62.4661212285104)
--(axis cs:570000,63.4328975323346)
--(axis cs:530000,58.8109865624696)
--(axis cs:490000,55.6746825931867)
--(axis cs:450000,57.6806297741792)
--(axis cs:410000,57.8697332277704)
--(axis cs:370000,52.1146904114954)
--(axis cs:330000,54.0363416001345)
--(axis cs:290000,52.2557397828543)
--(axis cs:250000,37.4225579324461)
--(axis cs:210000,38.2435984905139)
--(axis cs:170000,30.8571610481571)
--(axis cs:130000,15.2134009864936)
--(axis cs:90000,12.7719523040756)
--(axis cs:50000,11.3986260605429)
--(axis cs:10000,10.2812425960027)
--cycle;

\addplot [line width = \linewidthother, C4, mark=*, mark size=0, mark options={solid}]
table {%
10000 7.89490776182443
50000 7.9854333311754
90000 8.28937577211643
130000 9.77283196987029
170000 22.2536935035431
210000 23.2246065851491
250000 24.3473219356756
290000 34.6478363417292
330000 39.8473491621079
370000 30.8162680243438
410000 38.1505004392581
450000 34.3586620840302
490000 38.1543728235634
530000 41.383479841911
570000 43.1274020679081
610000 46.0577781009326
650000 46.1689029271184
690000 38.8777959370311
730000 30.5554718391215
770000 32.88675893705
810000 34.3310860617031
850000 35.8325598568167
890000 39.505923992965
930000 40.4884218643146
970000 48.59123916707
};

%DIME
\path [draw=C0, fill=C0, opacity=0.2]
(axis cs:1,6.80215591666667)
--(axis cs:1,5.5925808)
--(axis cs:24000,4.88713573333333)
--(axis cs:48000,9.748871)
--(axis cs:72000,39.178405)
--(axis cs:96000,35.577766)
--(axis cs:120000,66.3638765)
--(axis cs:144000,70.247577)
--(axis cs:168000,107.848963333333)
--(axis cs:192000,120.425612666667)
--(axis cs:216000,131.606081666667)
--(axis cs:240000,126.6783995)
--(axis cs:264000,159.943133333333)
--(axis cs:288000,172.044628333333)
--(axis cs:312000,188.191985)
--(axis cs:336000,197.75034)
--(axis cs:360000,215.573404333333)
--(axis cs:384000,244.38039)
--(axis cs:408000,265.35188)
--(axis cs:432000,256.659486666667)
--(axis cs:456000,254.555695)
--(axis cs:480000,321.560075)
--(axis cs:504000,309.784411666667)
--(axis cs:528000,319.716291666667)
--(axis cs:552000,342.35053)
--(axis cs:576000,357.640175)
--(axis cs:600000,365.597093333333)
--(axis cs:624000,379.492581666667)
--(axis cs:648000,391.800683333333)
--(axis cs:672000,439.1368195)
--(axis cs:696000,432.521263333333)
--(axis cs:720000,458.362536666667)
--(axis cs:744000,480.79676)
--(axis cs:768000,485.700503333333)
--(axis cs:792000,483.583248791667)
--(axis cs:816000,485.992758916667)
--(axis cs:840000,517.541645)
--(axis cs:864000,527.38718)
--(axis cs:888000,525.263277708333)
--(axis cs:912000,533.699495)
--(axis cs:936000,560.338303333333)
--(axis cs:960000,519.453665)
--(axis cs:984000,568.020833333333)
--(axis cs:1008000,590.470613333333)
--(axis cs:1032000,581.025416666667)
--(axis cs:1056000,583.688265)
--(axis cs:1080000,607.528891666667)
--(axis cs:1104000,620.278735)
--(axis cs:1128000,613.026586666667)
--(axis cs:1152000,638.844431666667)
--(axis cs:1176000,637.409991666667)
--(axis cs:1200000,641.941126666667)
--(axis cs:1224000,658.54966)
--(axis cs:1248000,646.231282291667)
--(axis cs:1272000,671.152858333333)
--(axis cs:1296000,660.192991666667)
--(axis cs:1320000,670.868964125)
--(axis cs:1344000,672.098633333333)
--(axis cs:1368000,667.270646666667)
--(axis cs:1392000,679.406895)
--(axis cs:1416000,667.768826166667)
--(axis cs:1440000,699.8657)
--(axis cs:1464000,703.368514541667)
--(axis cs:1488000,691.688456666667)
--(axis cs:1512000,691.02705)
--(axis cs:1536000,724.245045)
--(axis cs:1560000,642.1076)
--(axis cs:1584000,677.059261666667)
--(axis cs:1608000,722.467075)
--(axis cs:1632000,724.970475)
--(axis cs:1656000,707.042333333333)
--(axis cs:1680000,723.18196)
--(axis cs:1704000,739.733563333333)
--(axis cs:1728000,732.09224)
--(axis cs:1752000,743.118561666667)
--(axis cs:1776000,697.080626666666)
--(axis cs:1800000,709.36812)
--(axis cs:1824000,650.052771666667)
--(axis cs:1848000,716.700755)
--(axis cs:1872000,731.219691458333)
--(axis cs:1896000,703.767975)
--(axis cs:1920000,748.827876666667)
--(axis cs:1944000,769.246855)
--(axis cs:1968000,752.894191666667)
--(axis cs:1992000,737.845081666667)
--(axis cs:2016000,756.439925)
--(axis cs:2040000,764.171413333333)
--(axis cs:2064000,740.270250333333)
--(axis cs:2088000,725.118516666667)
--(axis cs:2112000,743.246291625)
--(axis cs:2136000,734.83942)
--(axis cs:2160000,786.861907708333)
--(axis cs:2184000,738.508053333333)
--(axis cs:2208000,750.911366666667)
--(axis cs:2232000,786.895078333333)
--(axis cs:2256000,798.218891666667)
--(axis cs:2280000,758.322531666667)
--(axis cs:2304000,768.420973333333)
--(axis cs:2328000,798.43544325)
--(axis cs:2352000,735.766616666667)
--(axis cs:2376000,763.731855)
--(axis cs:2400000,788.471761666667)
--(axis cs:2424000,798.863623333333)
--(axis cs:2448000,797.623013333333)
--(axis cs:2472000,798.009976666667)
--(axis cs:2496000,799.156973333333)
--(axis cs:2520000,789.180108333333)
--(axis cs:2544000,793.308771666667)
--(axis cs:2568000,766.555433333333)
--(axis cs:2592000,775.906533333333)
--(axis cs:2616000,657.870696541667)
--(axis cs:2640000,802.967105)
--(axis cs:2664000,762.239025)
--(axis cs:2688000,758.85175)
--(axis cs:2712000,832.457286666667)
--(axis cs:2736000,798.855305)
--(axis cs:2760000,790.104215)
--(axis cs:2784000,817.637935)
--(axis cs:2808000,791.100776666667)
--(axis cs:2832000,821.486588458333)
--(axis cs:2856000,798.389748333333)
--(axis cs:2880000,756.81125)
--(axis cs:2904000,809.53344)
--(axis cs:2928000,784.753078333333)
--(axis cs:2952000,778.719458333333)
--(axis cs:2976000,756.92755)
--(axis cs:3000000,792.897486666667)
--(axis cs:3000000,837.456603333333)
--(axis cs:3000000,837.456603333333)
--(axis cs:2976000,848.466266666667)
--(axis cs:2952000,849.20779)
--(axis cs:2928000,842.088306666667)
--(axis cs:2904000,854.474571666667)
--(axis cs:2880000,849.331893333333)
--(axis cs:2856000,862.497737875)
--(axis cs:2832000,869.625008333333)
--(axis cs:2808000,863.616378333333)
--(axis cs:2784000,860.982908333333)
--(axis cs:2760000,837.8695)
--(axis cs:2736000,852.905353333333)
--(axis cs:2712000,861.954153333333)
--(axis cs:2688000,830.469433333333)
--(axis cs:2664000,868.63305)
--(axis cs:2640000,864.472063333333)
--(axis cs:2616000,836.439531666667)
--(axis cs:2592000,863.806)
--(axis cs:2568000,856.3503)
--(axis cs:2544000,854.31376)
--(axis cs:2520000,848.667223333333)
--(axis cs:2496000,852.5053125)
--(axis cs:2472000,847.29574)
--(axis cs:2448000,844.57391)
--(axis cs:2424000,851.800331666667)
--(axis cs:2400000,842.753173333333)
--(axis cs:2376000,851.53189)
--(axis cs:2352000,824.298165)
--(axis cs:2328000,854.002923333333)
--(axis cs:2304000,823.149833333333)
--(axis cs:2280000,842.485828333333)
--(axis cs:2256000,850.53472)
--(axis cs:2232000,834.622511666667)
--(axis cs:2208000,832.830438333333)
--(axis cs:2184000,829.092180333333)
--(axis cs:2160000,843.589423333333)
--(axis cs:2136000,835.928383333333)
--(axis cs:2112000,828.188466666667)
--(axis cs:2088000,834.338206875)
--(axis cs:2064000,847.99647)
--(axis cs:2040000,817.802191666667)
--(axis cs:2016000,831.623071666667)
--(axis cs:1992000,808.728928708333)
--(axis cs:1968000,824.828883333333)
--(axis cs:1944000,821.340061666667)
--(axis cs:1920000,826.504741666667)
--(axis cs:1896000,818.318428333333)
--(axis cs:1872000,831.390226666667)
--(axis cs:1848000,826.4945)
--(axis cs:1824000,819.049933333333)
--(axis cs:1800000,842.232765791667)
--(axis cs:1776000,826.921123333333)
--(axis cs:1752000,834.800166666667)
--(axis cs:1728000,837.246975)
--(axis cs:1704000,853.61177)
--(axis cs:1680000,810.376866666667)
--(axis cs:1656000,808.423789083334)
--(axis cs:1632000,821.8582)
--(axis cs:1608000,815.354633333333)
--(axis cs:1584000,808.92544)
--(axis cs:1560000,812.104926666667)
--(axis cs:1536000,809.656291666667)
--(axis cs:1512000,782.288026666667)
--(axis cs:1488000,778.782291583334)
--(axis cs:1464000,778.858313333333)
--(axis cs:1440000,821.10777)
--(axis cs:1416000,783.881986666667)
--(axis cs:1392000,750.93809425)
--(axis cs:1368000,760.521313333333)
--(axis cs:1344000,766.813391666667)
--(axis cs:1320000,745.988075)
--(axis cs:1296000,745.58748175)
--(axis cs:1272000,778.078543333333)
--(axis cs:1248000,739.945458333333)
--(axis cs:1224000,752.028946666667)
--(axis cs:1200000,754.479292041667)
--(axis cs:1176000,738.838393333333)
--(axis cs:1152000,730.995366666667)
--(axis cs:1128000,726.874316666667)
--(axis cs:1104000,706.711063333333)
--(axis cs:1080000,710.372623333333)
--(axis cs:1056000,691.064566666667)
--(axis cs:1032000,651.101731666667)
--(axis cs:1008000,694.859348333333)
--(axis cs:984000,679.811011666667)
--(axis cs:960000,657.437975)
--(axis cs:936000,635.207354333333)
--(axis cs:912000,623.17085)
--(axis cs:888000,640.82537)
--(axis cs:864000,638.372956666667)
--(axis cs:840000,605.636066666667)
--(axis cs:816000,620.0484)
--(axis cs:792000,605.530201666667)
--(axis cs:768000,619.288428333333)
--(axis cs:744000,583.706463333333)
--(axis cs:720000,568.233463333333)
--(axis cs:696000,556.574435)
--(axis cs:672000,567.827056666667)
--(axis cs:648000,536.300349125)
--(axis cs:624000,521.473436666667)
--(axis cs:600000,514.082965)
--(axis cs:576000,512.673896666667)
--(axis cs:552000,503.68212)
--(axis cs:528000,495.859381666667)
--(axis cs:504000,462.220926666667)
--(axis cs:480000,454.484963333333)
--(axis cs:456000,428.183936666667)
--(axis cs:432000,393.258111666667)
--(axis cs:408000,386.389815)
--(axis cs:384000,369.733846666667)
--(axis cs:360000,331.193968333333)
--(axis cs:336000,327.351346666667)
--(axis cs:312000,296.701976666667)
--(axis cs:288000,251.529673375)
--(axis cs:264000,234.932268333333)
--(axis cs:240000,200.893656666667)
--(axis cs:216000,193.835535)
--(axis cs:192000,173.552272666667)
--(axis cs:168000,152.488876666667)
--(axis cs:144000,126.71998265)
--(axis cs:120000,102.995841)
--(axis cs:96000,77.3036046666667)
--(axis cs:72000,62.1926506666667)
--(axis cs:48000,35.3083526666667)
--(axis cs:24000,12.7237687833333)
--(axis cs:1,6.80215591666667)
--cycle;

\addplot [line width=\linewidthdime, C0, mark=*, mark size=0, mark options={solid}]
table {%
1 6.17641296666667
24000 7.56386268333333
48000 21.194537
72000 51.2078703333333
96000 60.4982363333333
120000 86.1485583333333
144000 110.007961666667
168000 134.059945
192000 145.492449333333
216000 164.83243
240000 161.520515
264000 192.983695
288000 202.72266
312000 233.335831666667
336000 261.032931666667
360000 270.949796666667
384000 306.506178333333
408000 326.736271666667
432000 321.852778333333
456000 346.82201
480000 389.972858333333
504000 387.914375
528000 412.405621666667
552000 431.808476666667
576000 435.778751666667
600000 454.327343333333
624000 452.650095
648000 468.345228333333
672000 505.85564
696000 488.718146666667
720000 507.547415
744000 526.153453333333
768000 557.257588333333
792000 547.320155
816000 551.174743333333
840000 565.919836666667
864000 586.99087
888000 582.702056666667
912000 577.851651666667
936000 603.980191666667
960000 612.742416666667
984000 628.581438333333
1008000 652.547345
1032000 621.855605
1056000 653.459033333333
1080000 677.789983333333
1104000 677.092326666667
1128000 688.166816666667
1152000 696.043668333333
1176000 689.546143333333
1200000 707.716985
1224000 716.223413333333
1248000 695.373521666667
1272000 729.041578333333
1296000 711.299655
1320000 714.779295
1344000 720.93956
1368000 718.17746
1392000 720.170855
1416000 742.45139
1440000 767.977446666667
1464000 745.921991666667
1488000 741.981485
1512000 740.075188333334
1536000 776.043163333333
1560000 730.35288
1584000 745.76982
1608000 774.030058333333
1632000 786.423141666667
1656000 775.007413333333
1680000 767.966193333333
1704000 803.248776666667
1728000 795.057825
1752000 799.094
1776000 768.546685
1800000 785.936768333333
1824000 770.381045
1848000 778.163971666667
1872000 784.5074
1896000 775.084686666667
1920000 803.85685
1944000 802.98721
1968000 799.194371666667
1992000 777.64892
2016000 809.862228333333
2040000 802.882363333333
2064000 807.031923333333
2088000 792.892843333333
2112000 784.135861666667
2136000 799.080736666667
2160000 829.289918333333
2184000 812.669866666667
2208000 798.845955
2232000 810.922945
2256000 829.769393333333
2280000 808.989303333333
2304000 787.9255
2328000 834.55804
2352000 790.163291666667
2376000 809.844736666667
2400000 823.614375
2424000 829.909856666667
2448000 827.945956666667
2472000 830.88744
2496000 832.031106666667
2520000 821.356241666667
2544000 822.67606
2568000 821.050315
2592000 838.71875
2616000 819.876546666667
2640000 836.608795
2664000 842.476748333333
2688000 800.310716666667
2712000 848.692928333333
2736000 828.082728333333
2760000 814.163598333333
2784000 843.60977
2808000 840.753683333333
2832000 848.649233333333
2856000 840.45366
2880000 812.837433333333
2904000 839.813745
2928000 813.380873333333
2952000 813.313951666667
2976000 809.99765
3000000 818.333465
};
\end{axis}

\end{tikzpicture}
}
       \subcaption[]{Dog Run}
       \label{fig::exps_new::dog_run}
    \end{minipage}\hfill
   \begin{minipage}[b]{0.25\textwidth}
        \centering
       \resizebox{1\textwidth}{!}{% This file was created with tikzplotlib v0.10.1.
\begin{tikzpicture}

\definecolor{darkcyan1115178}{RGB}{1,115,178}
\definecolor{darkgray176}{RGB}{176,176,176}

\begin{axis}[
legend cell align={left},
legend style={fill opacity=0.8, draw opacity=1, text opacity=1, draw=lightgray204, at={(0.03,0.03)},  anchor=north west},
tick align=outside,
tick pos=left,
x grid style={white},
xlabel={Number Env Interactions},
xmajorgrids,
%xmin=-37798.95, xmax=1045799.95,
xmin=-0.0, xmax=1000000.0,
xtick style={color=black},
y grid style={white},
ylabel={IQM Mean Return},
ymajorgrids,
ymin=-0.05, ymax=1050,
ytick style={color=black},
axis background/.style={fill=plot_background},
label style={font=\large},
tick label style={font=\large},
x axis line style={draw=none},
y axis line style={draw=none},
]

%bro
\path [draw=C1, fill=C1, opacity=0.2]
(axis cs:25000,10.2142086600653)
--(axis cs:25000,6.04700327944095)
--(axis cs:50000,9.94029688888085)
--(axis cs:75000,11.1068636224524)
--(axis cs:100000,16.9477453495589)
--(axis cs:125000,39.5802984911466)
--(axis cs:150000,59.2302148316502)
--(axis cs:175000,113.028589753238)
--(axis cs:200000,147.3704086416)
--(axis cs:225000,190.571465536214)
--(axis cs:250000,211.624961023278)
--(axis cs:275000,77.3581706076325)
--(axis cs:300000,176.720352118304)
--(axis cs:325000,261.796715117588)
--(axis cs:350000,281.615906225663)
--(axis cs:375000,311.273878134784)
--(axis cs:400000,334.58819882508)
--(axis cs:425000,363.366500733538)
--(axis cs:450000,375.456753138191)
--(axis cs:475000,394.845137909973)
--(axis cs:500000,424.594709781943)
--(axis cs:525000,215.936662773535)
--(axis cs:550000,384.345074271067)
--(axis cs:575000,420.672066721253)
--(axis cs:600000,474.329412408163)
--(axis cs:625000,515.681969250442)
--(axis cs:650000,539.371658385487)
--(axis cs:675000,529.002088773883)
--(axis cs:700000,585.727462311182)
--(axis cs:725000,618.741338898094)
--(axis cs:750000,608.667863942904)
--(axis cs:775000,208.471827354458)
--(axis cs:800000,433.473373262236)
--(axis cs:825000,562.343390471746)
--(axis cs:850000,664.706303084536)
--(axis cs:875000,671.389924641079)
--(axis cs:900000,721.117394280668)
--(axis cs:925000,782.373828923206)
--(axis cs:950000,808.388753673274)
--(axis cs:975000,818.822473845957)
--(axis cs:1000000,818.064581107608)
--(axis cs:1000000,900.886165220527)
--(axis cs:1000000,900.886165220527)
--(axis cs:975000,904.094958286527)
--(axis cs:950000,897.883146995715)
--(axis cs:925000,875.887631095982)
--(axis cs:900000,852.887418348863)
--(axis cs:875000,852.890774587416)
--(axis cs:850000,816.11648164269)
--(axis cs:825000,723.643414562601)
--(axis cs:800000,527.594134775395)
--(axis cs:775000,262.31097938707)
--(axis cs:750000,819.500926715734)
--(axis cs:725000,856.208560286228)
--(axis cs:700000,821.915819061702)
--(axis cs:675000,726.655556276404)
--(axis cs:650000,767.331963658631)
--(axis cs:625000,764.068980520742)
--(axis cs:600000,656.201622360029)
--(axis cs:575000,606.956673404254)
--(axis cs:550000,449.377051830647)
--(axis cs:525000,259.765970803016)
--(axis cs:500000,591.554770313111)
--(axis cs:475000,584.38377012915)
--(axis cs:450000,536.746331545282)
--(axis cs:425000,460.061504556395)
--(axis cs:400000,427.742665982809)
--(axis cs:375000,397.549589407921)
--(axis cs:350000,372.087420058651)
--(axis cs:325000,304.813684615114)
--(axis cs:300000,273.215346345317)
--(axis cs:275000,179.034067816553)
--(axis cs:250000,259.985401505138)
--(axis cs:225000,220.24019346277)
--(axis cs:200000,180.023196822655)
--(axis cs:175000,157.139894274251)
--(axis cs:150000,113.557440798184)
--(axis cs:125000,66.9361734922718)
--(axis cs:100000,41.7506846665356)
--(axis cs:75000,20.1929642300217)
--(axis cs:50000,19.4478434490568)
--(axis cs:25000,10.2142086600653)
--cycle;

\addplot [line width=\linewidthother, C1, mark=*, mark size=0, mark options={solid}]
table {%
25000 7.10231869129834
50000 15.4444783711257
75000 14.70340794001
100000 27.4779078942657
125000 54.775790225404
150000 97.3771422833903
175000 129.720917245138
200000 162.772251426318
225000 203.280057857938
250000 233.442759544052
275000 131.617422527116
300000 236.909918285698
325000 278.229556099335
350000 322.157334441082
375000 358.990999991801
400000 378.259351627361
425000 406.377529921682
450000 447.518485760111
475000 491.014818441506
500000 516.890032974067
525000 240.051973565367
550000 422.017551844324
575000 523.106446063144
600000 565.989562870539
625000 630.820086277866
650000 650.654465704963
675000 622.137737903315
700000 701.219632968095
725000 750.169724017328
750000 710.629840649616
775000 230.975674230592
800000 494.355571893528
825000 661.634642082916
850000 757.892398952011
875000 785.019654438239
900000 807.9640580905
925000 836.063225898729
950000 872.226292387088
975000 883.226243201753
1000000 873.249684510752
};
% QSM
\path [draw=C3, fill=C3, opacity=0.2]
(axis cs:10000,8.1497493882974)
--(axis cs:10000,4.13229278723399)
--(axis cs:30000,6.13090146730344)
--(axis cs:50000,8.79878619809945)
--(axis cs:70000,6.05236637409156)
--(axis cs:90000,9.56387667692422)
--(axis cs:110000,9.97444182542192)
--(axis cs:130000,8.96258318699984)
--(axis cs:150000,10.8673870517402)
--(axis cs:170000,9.37928022401199)
--(axis cs:190000,11.681361884141)
--(axis cs:210000,16.7618835554608)
--(axis cs:230000,20.2140298144498)
--(axis cs:250000,21.4217898176379)
--(axis cs:270000,23.5433583879111)
--(axis cs:290000,24.0484079542053)
--(axis cs:310000,32.3867602311142)
--(axis cs:330000,34.7553416767155)
--(axis cs:350000,39.8927686796612)
--(axis cs:370000,37.6004802095742)
--(axis cs:390000,44.1761799200101)
--(axis cs:410000,36.0407793405056)
--(axis cs:430000,35.9965394952628)
--(axis cs:450000,46.9404334785405)
--(axis cs:470000,46.240991831596)
--(axis cs:490000,48.1880307461282)
--(axis cs:510000,46.7389883076176)
--(axis cs:530000,42.0523428181157)
--(axis cs:550000,51.1757642354084)
--(axis cs:570000,54.9996308887997)
--(axis cs:590000,45.0969574685981)
--(axis cs:610000,60.0097136113949)
--(axis cs:630000,65.2753488719478)
--(axis cs:650000,65.7512542184017)
--(axis cs:670000,69.4210400831325)
--(axis cs:690000,69.5825379968184)
--(axis cs:710000,76.8044773580095)
--(axis cs:730000,70.9139611144716)
--(axis cs:750000,63.0932088695253)
--(axis cs:770000,62.0373886758187)
--(axis cs:790000,59.1466642159106)
--(axis cs:810000,67.1545887150432)
--(axis cs:830000,55.6769537769259)
--(axis cs:850000,64.2667619447161)
--(axis cs:870000,74.425090726176)
--(axis cs:890000,68.488441076727)
--(axis cs:910000,73.8360029108314)
--(axis cs:930000,74.5372300344349)
--(axis cs:950000,68.5224583070415)
--(axis cs:970000,75.0820896506858)
--(axis cs:990000,78.574695865305)
--(axis cs:990000,183.835799071237)
--(axis cs:990000,183.835799071237)
--(axis cs:970000,172.998099695785)
--(axis cs:950000,155.884253638603)
--(axis cs:930000,167.886945105146)
--(axis cs:910000,153.430465543837)
--(axis cs:890000,153.37904861964)
--(axis cs:870000,156.192673656376)
--(axis cs:850000,148.481321975953)
--(axis cs:830000,142.378473389597)
--(axis cs:810000,141.489704162603)
--(axis cs:790000,141.406378869799)
--(axis cs:770000,135.661811812818)
--(axis cs:750000,132.747183848558)
--(axis cs:730000,137.730177560122)
--(axis cs:710000,134.846436864044)
--(axis cs:690000,117.825814041346)
--(axis cs:670000,114.139299598922)
--(axis cs:650000,119.467530159083)
--(axis cs:630000,119.011320958721)
--(axis cs:610000,111.763549056183)
--(axis cs:590000,108.433367877438)
--(axis cs:570000,102.954443494251)
--(axis cs:550000,108.897663272183)
--(axis cs:530000,101.082772004182)
--(axis cs:510000,96.9226644051747)
--(axis cs:490000,93.2510820445003)
--(axis cs:470000,93.7161139909209)
--(axis cs:450000,90.1628516554167)
--(axis cs:430000,81.8810157204915)
--(axis cs:410000,78.5117631552581)
--(axis cs:390000,70.1159185823371)
--(axis cs:370000,71.8777455967248)
--(axis cs:350000,72.5664593951692)
--(axis cs:330000,61.5820129306377)
--(axis cs:310000,61.8530923330493)
--(axis cs:290000,60.5447328056621)
--(axis cs:270000,51.6897246423658)
--(axis cs:250000,53.2513747406901)
--(axis cs:230000,47.9164908822963)
--(axis cs:210000,39.1330208217075)
--(axis cs:190000,40.8844241386842)
--(axis cs:170000,31.7859350469305)
--(axis cs:150000,29.279683903466)
--(axis cs:130000,26.6758020640244)
--(axis cs:110000,22.2997193965585)
--(axis cs:90000,15.1530526581531)
--(axis cs:70000,7.88180729841503)
--(axis cs:50000,11.4294566406558)
--(axis cs:30000,7.74555445710818)
--(axis cs:10000,8.1497493882974)
--cycle;

\addplot [line width=\linewidthother, C3, mark=*, mark size=0, mark options={solid}]
table {%
10000 5.36117243766785
30000 6.68574044108391
50000 9.86008006334305
70000 6.76395849821468
90000 11.9236217209448
110000 15.2470202518161
130000 15.8367476984179
150000 19.613413177564
170000 19.7069684432178
190000 25.1426080836573
210000 27.61641457518
230000 35.0539077082125
250000 38.5193310568444
270000 40.8034731191335
290000 44.9377495036047
310000 50.6947184614677
330000 54.4797240349735
350000 60.9747630798344
370000 57.8345958799999
390000 59.7824717067401
410000 58.7360973222452
430000 58.9730294511566
450000 73.9740516336225
470000 75.4477187392975
490000 80.7763822079449
510000 82.6236137027409
530000 76.8738983271739
550000 92.1233840381706
570000 90.4621526241926
590000 83.0486440782473
610000 95.7206927059253
630000 103.072589494739
650000 101.875447796226
670000 100.266329784446
690000 100.147207153663
710000 115.257539295381
730000 116.567092133861
750000 111.740297837878
770000 113.148976729894
790000 113.166646899428
810000 126.166696692915
830000 108.666430277627
850000 124.932734202097
870000 139.781689523586
890000 130.588935168865
910000 136.117197318191
930000 139.918419596503
950000 129.658567690631
970000 141.723887350107
990000 142.439076479759
};

%CrossQ
\path [draw=C2, fill=C2, opacity=0.2]
(axis cs:1,7.7636034)
--(axis cs:1,5.46408252)
--(axis cs:24000,4.72060906)
--(axis cs:48000,12.9723486)
--(axis cs:72000,16.5769948)
--(axis cs:96000,24.0717558)
--(axis cs:120000,24.5572644)
--(axis cs:144000,27.6140768)
--(axis cs:168000,34.2999146)
--(axis cs:192000,38.9126824)
--(axis cs:216000,60.9166462)
--(axis cs:240000,83.1809116)
--(axis cs:264000,85.349672)
--(axis cs:288000,107.167628)
--(axis cs:312000,104.1112024)
--(axis cs:336000,101.296375)
--(axis cs:360000,150.216912)
--(axis cs:384000,179.429824)
--(axis cs:408000,196.539096)
--(axis cs:432000,198.834952)
--(axis cs:456000,221.658284)
--(axis cs:480000,249.868426)
--(axis cs:504000,301.427118)
--(axis cs:528000,370.785496)
--(axis cs:552000,421.404152)
--(axis cs:576000,456.795624)
--(axis cs:600000,521.17988)
--(axis cs:624000,521.955172)
--(axis cs:648000,615.986116)
--(axis cs:672000,629.981864)
--(axis cs:696000,274.751512)
--(axis cs:720000,761.96609)
--(axis cs:744000,781.534732)
--(axis cs:768000,810.763972)
--(axis cs:792000,812.562216)
--(axis cs:816000,686.090606)
--(axis cs:840000,845.420548)
--(axis cs:864000,827.433246)
--(axis cs:888000,732.485616)
--(axis cs:912000,846.96646)
--(axis cs:936000,772.566742)
--(axis cs:960000,861.11695)
--(axis cs:984000,872.190516)
--(axis cs:984000,903.990078)
--(axis cs:984000,903.990078)
--(axis cs:960000,896.527484)
--(axis cs:936000,881.1848495)
--(axis cs:912000,888.55644)
--(axis cs:888000,877.03426)
--(axis cs:864000,873.769988)
--(axis cs:840000,889.25587)
--(axis cs:816000,861.972786)
--(axis cs:792000,869.084066)
--(axis cs:768000,852.212194)
--(axis cs:744000,872.304078)
--(axis cs:720000,841.033808)
--(axis cs:696000,838.050332)
--(axis cs:672000,838.852396)
--(axis cs:648000,846.841732)
--(axis cs:624000,831.69471)
--(axis cs:600000,818.427164)
--(axis cs:576000,763.31424)
--(axis cs:552000,744.42057)
--(axis cs:528000,724.072936)
--(axis cs:504000,672.22374)
--(axis cs:480000,667.612748)
--(axis cs:456000,569.6825)
--(axis cs:432000,518.309868)
--(axis cs:408000,483.781336)
--(axis cs:384000,409.7531)
--(axis cs:360000,314.234096)
--(axis cs:336000,248.764082)
--(axis cs:312000,241.711819900001)
--(axis cs:288000,177.649838)
--(axis cs:264000,147.690636)
--(axis cs:240000,132.684586)
--(axis cs:216000,96.65459)
--(axis cs:192000,77.0963988)
--(axis cs:168000,64.1790628)
--(axis cs:144000,65.7244832)
--(axis cs:120000,53.0565418)
--(axis cs:96000,47.2615688)
--(axis cs:72000,29.1247694)
--(axis cs:48000,25.6661808)
--(axis cs:24000,12.5510507)
--(axis cs:1,7.7636034)
--cycle;

\addplot [line width=\linewidthother, C2, mark=*, mark size=0, mark options={solid}]
table {%
1 6.4363952
24000 7.31452946
48000 18.5531912
72000 23.503585
96000 37.309537
120000 39.3057876
144000 47.6661486
168000 49.418312
192000 66.831065
216000 82.396368
240000 109.757596
264000 121.180432
288000 144.603512
312000 161.7335992
336000 186.045542
360000 234.925814
384000 295.85547
408000 314.393752
432000 345.231132
456000 375.470498
480000 435.862264
504000 482.939014
528000 542.437274
552000 578.597692
576000 598.024944
600000 691.524844
624000 709.958354
648000 729.868072
672000 761.839658
696000 671.380936
720000 807.001848
744000 823.361278
768000 838.973674
792000 843.512524
816000 828.925708
840000 868.701604
864000 850.869712
888000 858.23142
912000 870.90532
936000 872.990692
960000 879.933208
984000 889.657072
};

% Consistency-AC
\path [draw=C5, fill=C5, opacity=0.2]
(axis cs:10000,10.2347337776344)
--(axis cs:10000,5.76255027057737)
--(axis cs:50000,5.49264731069308)
--(axis cs:90000,6.77298195446768)
--(axis cs:130000,8.50087831284714)
--(axis cs:170000,8.81954475801568)
--(axis cs:210000,7.64768590439827)
--(axis cs:250000,11.9240215813979)
--(axis cs:290000,13.1343956457185)
--(axis cs:330000,11.2241753535077)
--(axis cs:370000,17.9788732559991)
--(axis cs:410000,13.3057404918195)
--(axis cs:450000,15.0612460636002)
--(axis cs:490000,15.9825794277412)
--(axis cs:530000,23.8338574068778)
--(axis cs:570000,18.5730643536114)
--(axis cs:610000,12.741926255586)
--(axis cs:650000,12.8762697107272)
--(axis cs:690000,15.1494545785346)
--(axis cs:730000,19.2037577409984)
--(axis cs:770000,10.5938637479737)
--(axis cs:810000,13.9980277167086)
--(axis cs:850000,21.5355172063689)
--(axis cs:890000,18.7554769371398)
--(axis cs:930000,15.1719657642189)
--(axis cs:970000,19.901851282819)
--(axis cs:970000,39.6152453274992)
--(axis cs:970000,39.6152453274992)
--(axis cs:930000,42.5338958922913)
--(axis cs:890000,28.1018417340848)
--(axis cs:850000,37.4896139958857)
--(axis cs:810000,28.3110239538117)
--(axis cs:770000,21.074736778364)
--(axis cs:730000,29.829804283105)
--(axis cs:690000,30.3086126309752)
--(axis cs:650000,23.3937282318619)
--(axis cs:610000,39.4395696254161)
--(axis cs:570000,31.6647356179577)
--(axis cs:530000,37.1730573100056)
--(axis cs:490000,23.6702091518794)
--(axis cs:450000,30.5004022737851)
--(axis cs:410000,21.8871698314392)
--(axis cs:370000,23.9471682791031)
--(axis cs:330000,22.043964840338)
--(axis cs:290000,15.7472688836757)
--(axis cs:250000,16.7040127321629)
--(axis cs:210000,16.4868369835571)
--(axis cs:170000,23.9883804256518)
--(axis cs:130000,16.2607071190685)
--(axis cs:90000,9.49205229806014)
--(axis cs:50000,8.98834073841812)
--(axis cs:10000,10.2347337776344)
--cycle;

\addplot [line width=\linewidthother, C5, mark=*, mark size=0, mark options={solid}]
table {%
10000 7.79275314843354
50000 7.13295527648071
90000 8.1447138219036
130000 11.4634506445582
170000 15.0416418252199
210000 12.0266352420092
250000 14.3140171567804
290000 14.1654982611565
330000 16.5854004552536
370000 20.8883532973247
410000 18.1101578823664
450000 21.9929174264873
490000 19.6986217226023
530000 29.6013647455454
570000 24.9260566287715
610000 24.4016194517091
650000 18.3022974541329
690000 22.707283498064
730000 25.8398184188164
770000 15.4180240404238
810000 21.0232945826201
850000 28.0492888232772
890000 22.7292262633204
930000 28.5038074268748
970000 29.9241082252481
};

%Diff-QL
\path [draw=C4, fill=C4, opacity=0.2]
(axis cs:10000,9.5675717185448)
--(axis cs:10000,6.43570297415532)
--(axis cs:50000,6.39292454769072)
--(axis cs:90000,7.85799215657216)
--(axis cs:130000,8.45129532322936)
--(axis cs:170000,9.92340478366562)
--(axis cs:210000,10.2043182540672)
--(axis cs:250000,11.8922183868616)
--(axis cs:290000,13.4118754200819)
--(axis cs:330000,19.1676860418557)
--(axis cs:370000,20.4850023108021)
--(axis cs:410000,18.108463772179)
--(axis cs:450000,14.8102193368213)
--(axis cs:490000,15.1245202390428)
--(axis cs:530000,20.9443094680214)
--(axis cs:570000,23.7810497634719)
--(axis cs:610000,17.5637453001429)
--(axis cs:650000,26.506527755415)
--(axis cs:690000,23.5251517979568)
--(axis cs:730000,22.0473857129122)
--(axis cs:770000,17.6459279146738)
--(axis cs:810000,23.3500420565594)
--(axis cs:850000,24.711497286649)
--(axis cs:890000,25.592792389992)
--(axis cs:930000,14.7671256984675)
--(axis cs:970000,15.9884534999535)
--(axis cs:970000,29.0055456230495)
--(axis cs:970000,29.0055456230495)
--(axis cs:930000,36.7629647709141)
--(axis cs:890000,41.5784902709886)
--(axis cs:850000,51.7961998649288)
--(axis cs:810000,54.1293181675297)
--(axis cs:770000,52.3699466857342)
--(axis cs:730000,55.4619242104188)
--(axis cs:690000,49.5899314686535)
--(axis cs:650000,44.6105294470725)
--(axis cs:610000,41.3850266265727)
--(axis cs:570000,51.6814812336051)
--(axis cs:530000,50.6837992665532)
--(axis cs:490000,55.7895102261811)
--(axis cs:450000,41.6999965840367)
--(axis cs:410000,48.9021458201465)
--(axis cs:370000,50.5149538661431)
--(axis cs:330000,49.1221120747188)
--(axis cs:290000,40.0974084339831)
--(axis cs:250000,35.4019379923738)
--(axis cs:210000,25.2011726965485)
--(axis cs:170000,16.4359443503333)
--(axis cs:130000,17.017412095866)
--(axis cs:90000,14.8114890097126)
--(axis cs:50000,8.85630704310797)
--(axis cs:10000,9.5675717185448)
--cycle;

\addplot [line width=\linewidthother, C4, mark=*, mark size=0, mark options={solid}]
table {%
10000 7.80598880751825
50000 7.65348116096914
90000 11.1361805468193
130000 12.6382871547076
170000 13.0170353804794
210000 16.8284301669059
250000 22.1535411203353
290000 24.4405202025794
330000 31.9545421677972
370000 33.4497043608387
410000 31.0562395599516
450000 26.0716537063295
490000 33.1444053430962
530000 34.6317964757709
570000 36.0896385329562
610000 27.3383741572694
650000 35.4908174211043
690000 35.066679391649
730000 37.4148853722875
770000 33.3825137812731
810000 37.4501340433743
850000 37.5631232619905
890000 34.0456637014936
930000 24.3327714396398
970000 21.9825501321156
};

% BRO Fast
\path [draw=C9, fill=C9, opacity=0.2]
(axis cs:25000,6.59893095163156)
--(axis cs:25000,5.09619654797844)
--(axis cs:50000,6.68345863249794)
--(axis cs:75000,5.76914751969117)
--(axis cs:100000,7.73662996459905)
--(axis cs:125000,10.6142803929697)
--(axis cs:150000,17.572149605008)
--(axis cs:175000,20.0007096098967)
--(axis cs:200000,34.7245856969341)
--(axis cs:225000,50.2923967631813)
--(axis cs:250000,67.4014740613568)
--(axis cs:275000,82.6332236276432)
--(axis cs:300000,95.0253721471344)
--(axis cs:325000,152.417857446646)
--(axis cs:350000,185.149404398036)
--(axis cs:375000,214.907463578651)
--(axis cs:400000,232.147257854489)
--(axis cs:425000,267.742483422585)
--(axis cs:450000,282.599702219079)
--(axis cs:475000,300.740066559168)
--(axis cs:500000,326.129405289998)
--(axis cs:525000,347.19963525316)
--(axis cs:550000,369.629871803384)
--(axis cs:575000,385.8453548143)
--(axis cs:600000,404.72948490321)
--(axis cs:625000,436.164115402096)
--(axis cs:650000,450.927673518158)
--(axis cs:675000,481.082570789142)
--(axis cs:700000,518.911588553471)
--(axis cs:725000,543.43965165776)
--(axis cs:750000,515.127441480014)
--(axis cs:775000,556.513773740481)
--(axis cs:800000,554.050580361764)
--(axis cs:825000,581.370952518125)
--(axis cs:850000,628.264924332352)
--(axis cs:875000,642.410073099679)
--(axis cs:900000,665.05811099107)
--(axis cs:925000,694.45347982084)
--(axis cs:950000,692.781711111434)
--(axis cs:975000,740.915920800821)
--(axis cs:1000000,730.163370243415)
--(axis cs:1000000,850.491912984216)
--(axis cs:1000000,850.491912984216)
--(axis cs:975000,851.55494099411)
--(axis cs:950000,817.090108033226)
--(axis cs:925000,816.682970881118)
--(axis cs:900000,802.630296505616)
--(axis cs:875000,769.753146683131)
--(axis cs:850000,767.124010808035)
--(axis cs:825000,770.906321877137)
--(axis cs:800000,723.296245389451)
--(axis cs:775000,665.889199204516)
--(axis cs:750000,662.87584991544)
--(axis cs:725000,619.738768480183)
--(axis cs:700000,648.17990200687)
--(axis cs:675000,565.579315025049)
--(axis cs:650000,573.612537118924)
--(axis cs:625000,504.264903376933)
--(axis cs:600000,523.847714666019)
--(axis cs:575000,463.1513738777)
--(axis cs:550000,456.611566512917)
--(axis cs:525000,411.124331767699)
--(axis cs:500000,388.456462766119)
--(axis cs:475000,377.537783100878)
--(axis cs:450000,354.481573067568)
--(axis cs:425000,317.458750479861)
--(axis cs:400000,298.521104719724)
--(axis cs:375000,268.749047876867)
--(axis cs:350000,243.014242513684)
--(axis cs:325000,205.707659386141)
--(axis cs:300000,178.239885294631)
--(axis cs:275000,131.946321373703)
--(axis cs:250000,104.978418432255)
--(axis cs:225000,85.1548640032376)
--(axis cs:200000,66.6738961156747)
--(axis cs:175000,27.1292687661134)
--(axis cs:150000,33.830227833184)
--(axis cs:125000,19.2242332793354)
--(axis cs:100000,13.1249047940864)
--(axis cs:75000,10.733613508223)
--(axis cs:50000,12.0105560785595)
--(axis cs:25000,6.59893095163156)
--cycle;

\addplot [line width=\linewidthother, C9, mark=*, mark size=0, mark options={solid}]
table {%
25000 5.57809569865632
50000 8.56684693338556
75000 7.48931255583469
100000 10.0953008476579
125000 13.3382399111572
150000 22.8858155288462
175000 23.3948871560536
200000 52.4471638601048
225000 66.3525107134854
250000 89.0763045814384
275000 105.89293143455
300000 129.886948444075
325000 178.377232326059
350000 211.385032394109
375000 242.432593366026
400000 268.333310367668
425000 298.591484226867
450000 319.261949933502
475000 349.439953123409
500000 360.82156145202
525000 379.370258645179
550000 420.640691220242
575000 423.74783927818
600000 457.592846163417
625000 475.522362144384
650000 508.225371366442
675000 527.406278509008
700000 600.825694237657
725000 590.805518471181
750000 607.131564579065
775000 605.662416272929
800000 616.544655586423
825000 665.928860357327
850000 698.423299023819
875000 699.597843074859
900000 734.698433639537
925000 748.284843301212
950000 750.716637940081
975000 806.485792998319
1000000 807.99027973133
};
% dime
\path [draw=C0, fill=C0, opacity=0.2]
(axis cs:1,8.8254771)
--(axis cs:1,7.17193363333333)
--(axis cs:24000,7.37208126916667)
--(axis cs:48000,13.3014672166667)
--(axis cs:72000,32.1206455)
--(axis cs:96000,34.6587273333333)
--(axis cs:120000,51.686271)
--(axis cs:144000,101.679710833333)
--(axis cs:168000,117.785922)
--(axis cs:192000,158.421218125)
--(axis cs:216000,205.082918333333)
--(axis cs:240000,230.496605)
--(axis cs:264000,263.163888333333)
--(axis cs:288000,311.846381666667)
--(axis cs:312000,310.869181666667)
--(axis cs:336000,364.34544)
--(axis cs:360000,385.017636666667)
--(axis cs:384000,425.358576666667)
--(axis cs:408000,423.61112)
--(axis cs:432000,482.584251666667)
--(axis cs:456000,489.01956)
--(axis cs:480000,517.239456666667)
--(axis cs:504000,526.25562)
--(axis cs:528000,530.429151666667)
--(axis cs:552000,542.943948333333)
--(axis cs:576000,572.66895375)
--(axis cs:600000,574.124883333333)
--(axis cs:624000,623.001761666667)
--(axis cs:648000,604.5136)
--(axis cs:672000,632.52042)
--(axis cs:696000,633.728138333333)
--(axis cs:720000,624.9184555)
--(axis cs:744000,666.828466666667)
--(axis cs:768000,710.716005)
--(axis cs:792000,701.18865)
--(axis cs:816000,710.1812)
--(axis cs:840000,759.394913333333)
--(axis cs:864000,758.725766666667)
--(axis cs:888000,745.370601666667)
--(axis cs:912000,755.952326666667)
--(axis cs:936000,791.06075)
--(axis cs:960000,794.83587)
--(axis cs:984000,806.1239)
--(axis cs:984000,905.489024291667)
--(axis cs:984000,905.489024291667)
--(axis cs:960000,914.948321666667)
--(axis cs:936000,910.255443333333)
--(axis cs:912000,912.1204)
--(axis cs:888000,901.018776666667)
--(axis cs:864000,896.934446666667)
--(axis cs:840000,895.839225)
--(axis cs:816000,907.782151666667)
--(axis cs:792000,903.52465)
--(axis cs:768000,888.863033333333)
--(axis cs:744000,902.149706666667)
--(axis cs:720000,862.377375)
--(axis cs:696000,889.146166666667)
--(axis cs:672000,881.369471666667)
--(axis cs:648000,875.178185)
--(axis cs:624000,869.317959458334)
--(axis cs:600000,871.169721666667)
--(axis cs:576000,819.984273333333)
--(axis cs:552000,823.906741666667)
--(axis cs:528000,820.378763333333)
--(axis cs:504000,798.406914375)
--(axis cs:480000,712.451448041667)
--(axis cs:456000,721.711630208334)
--(axis cs:432000,712.088615)
--(axis cs:408000,655.5924025)
--(axis cs:384000,637.304816666667)
--(axis cs:360000,551.718676666667)
--(axis cs:336000,469.755415125)
--(axis cs:312000,449.778441666667)
--(axis cs:288000,413.157218333333)
--(axis cs:264000,363.023455)
--(axis cs:240000,325.051616666667)
--(axis cs:216000,270.634826666667)
--(axis cs:192000,236.69606)
--(axis cs:168000,165.306758333333)
--(axis cs:144000,137.55399)
--(axis cs:120000,88.3899183333333)
--(axis cs:96000,73.9707755)
--(axis cs:72000,55.5478688333333)
--(axis cs:48000,26.1899863333333)
--(axis cs:24000,19.1755403333333)
--(axis cs:1,8.8254771)
--cycle;

\addplot [line width=\linewidthdime, C0, mark=*, mark size=0, mark options={solid}]
table {%
1 8.02720193333333
24000 11.1751058333333
48000 17.8923970833333
72000 43.0054263333333
96000 52.3510316666667
120000 71.8961976666667
144000 124.242545
168000 137.793095833333
192000 189.207393333333
216000 231.128405
240000 278.087873333333
264000 310.587308333333
288000 345.057091666667
312000 351.722546666667
336000 393.582313333333
360000 417.562253333333
384000 472.572126666667
408000 496.593573333333
432000 553.027008333333
456000 588.3292
480000 581.146881666667
504000 643.333966666667
528000 674.014988333333
552000 688.805435
576000 691.755891666667
600000 725.3658
624000 751.756368333333
648000 749.245966666667
672000 760.467748333333
696000 788.088031666667
720000 757.372903333333
744000 814.558813333333
768000 821.11765
792000 819.858475
816000 847.036351666667
840000 839.362808333333
864000 842.46124
888000 850.795573333333
912000 859.859966666667
936000 875.78015
960000 866.022886666667
984000 886.180788333333
};
\end{axis}

\end{tikzpicture}
}
       \subcaption[]{Dog Trot}
       \label{fig::exps_new::dog_trot}
    \end{minipage}\hfill
    \begin{minipage}[b]{0.25\textwidth}
        \centering
       \resizebox{1\textwidth}{!}{% This file was created with tikzplotlib v0.10.1.
\begin{tikzpicture}

\definecolor{darkcyan1115178}{RGB}{1,115,178}
\definecolor{darkgray176}{RGB}{176,176,176}

\begin{axis}[
legend cell align={left},
legend style={fill opacity=0.8, draw opacity=1, text opacity=1, draw=lightgray204, at={(0.03,0.03)},  anchor=north west},
tick align=outside,
tick pos=left,
x grid style={white},
xlabel={Number Env Interactions},
xmajorgrids,
xmin=-0.0, xmax=1000000.0,
xtick style={color=black},
y grid style={white},
ylabel={IQM Mean Return},
ymajorgrids,
ymin=-0.05, ymax=1050,
ytick style={color=black},
axis background/.style={fill=plot_background},
label style={font=\large},
tick label style={font=\large},
x axis line style={draw=none},
y axis line style={draw=none},
]

% bro
\path [draw=C1, fill=C1, opacity=0.2]
(axis cs:25000,18.014924526478)
--(axis cs:25000,7.4156117968486)
--(axis cs:50000,10.9086992695516)
--(axis cs:75000,9.64750008836316)
--(axis cs:100000,36.9039894140512)
--(axis cs:125000,79.2555976999825)
--(axis cs:150000,147.611755433983)
--(axis cs:175000,188.647784577087)
--(axis cs:200000,255.018226524658)
--(axis cs:225000,338.851875771426)
--(axis cs:250000,397.475084018838)
--(axis cs:275000,295.927283903393)
--(axis cs:300000,458.818622317544)
--(axis cs:325000,518.87557559647)
--(axis cs:350000,552.648057867677)
--(axis cs:375000,652.026620702615)
--(axis cs:400000,671.906700402409)
--(axis cs:425000,723.169602299029)
--(axis cs:450000,713.79560702622)
--(axis cs:475000,791.398541157166)
--(axis cs:500000,766.585260485533)
--(axis cs:525000,329.683034674713)
--(axis cs:550000,624.12609699428)
--(axis cs:575000,787.960924204837)
--(axis cs:600000,835.200160392671)
--(axis cs:625000,874.808083870318)
--(axis cs:650000,869.887928793535)
--(axis cs:675000,884.596295710542)
--(axis cs:700000,914.325490723843)
--(axis cs:725000,885.036317419056)
--(axis cs:750000,898.166515483225)
--(axis cs:775000,329.397944540565)
--(axis cs:800000,638.245465917537)
--(axis cs:825000,785.312890466362)
--(axis cs:850000,868.701836563927)
--(axis cs:875000,888.668004186219)
--(axis cs:900000,898.975943934911)
--(axis cs:925000,896.158825985528)
--(axis cs:950000,918.992537570823)
--(axis cs:975000,918.935073067224)
--(axis cs:1000000,909.710507148213)
--(axis cs:1000000,945.487740690931)
--(axis cs:1000000,945.487740690931)
--(axis cs:975000,944.869108075849)
--(axis cs:950000,939.672549908312)
--(axis cs:925000,938.396352577302)
--(axis cs:900000,929.043287461199)
--(axis cs:875000,928.291310086408)
--(axis cs:850000,910.964526739627)
--(axis cs:825000,853.688538783975)
--(axis cs:800000,711.133909576773)
--(axis cs:775000,377.662293341185)
--(axis cs:750000,928.506236005121)
--(axis cs:725000,927.97986967702)
--(axis cs:700000,933.173159788715)
--(axis cs:675000,923.654534727326)
--(axis cs:650000,916.352617416332)
--(axis cs:625000,913.680404689217)
--(axis cs:600000,899.617312995699)
--(axis cs:575000,873.603993388215)
--(axis cs:550000,736.880200000094)
--(axis cs:525000,424.035146752066)
--(axis cs:500000,876.030429950947)
--(axis cs:475000,901.008867317357)
--(axis cs:450000,878.960967109718)
--(axis cs:425000,854.289085959737)
--(axis cs:400000,842.827103440673)
--(axis cs:375000,786.948006589569)
--(axis cs:350000,714.892268252094)
--(axis cs:325000,672.962223772281)
--(axis cs:300000,532.120859243281)
--(axis cs:275000,355.437248131595)
--(axis cs:250000,522.484047879246)
--(axis cs:225000,474.583134213079)
--(axis cs:200000,380.960460343633)
--(axis cs:175000,339.257090338765)
--(axis cs:150000,280.394022963305)
--(axis cs:125000,161.766224661837)
--(axis cs:100000,83.4621247791787)
--(axis cs:75000,31.9673331088589)
--(axis cs:50000,31.8793833744225)
--(axis cs:25000,18.014924526478)
--cycle;

\addplot [line width=\linewidthother, C1, mark=*, mark size=0, mark options={solid}]
table {%
25000 11.0475360073667
50000 19.643730280123
75000 17.8517897425518
100000 58.7257586244328
125000 106.509044647381
150000 201.643048267293
175000 252.871181547128
200000 298.560797111159
225000 403.043932676284
250000 451.786930272915
275000 323.261381434117
300000 486.16205493198
325000 592.755828547265
350000 632.340562605779
375000 727.426954429982
400000 761.644335297047
425000 801.443669165292
450000 801.303115910254
475000 867.29747856679
500000 823.852193983441
525000 371.960989403267
550000 670.980995616498
575000 832.64959930713
600000 869.944317011004
625000 897.883735748895
650000 905.216919292288
675000 906.031176723022
700000 923.909914838672
725000 908.228342049178
750000 914.701943815636
775000 360.766509137989
800000 677.660213617643
825000 810.82750601302
850000 897.043928793654
875000 915.398439166291
900000 917.314638806408
925000 925.079665296688
950000 932.302704511147
975000 933.354871155295
1000000 928.374811581295
};
% QSM
\path [draw=C3, fill=C3, opacity=0.2]
(axis cs:10000,18.2317509651184)
--(axis cs:10000,9.33155124783516)
--(axis cs:30000,8.46316670378049)
--(axis cs:50000,13.4455009978265)
--(axis cs:70000,7.47428028720121)
--(axis cs:90000,11.2942173108459)
--(axis cs:110000,11.5140420992393)
--(axis cs:130000,9.05557629196361)
--(axis cs:150000,10.2782311926518)
--(axis cs:170000,12.6281240814811)
--(axis cs:190000,17.4896556622131)
--(axis cs:210000,17.7443579280691)
--(axis cs:230000,17.4862381565088)
--(axis cs:250000,18.5683783543316)
--(axis cs:270000,19.6842298016688)
--(axis cs:290000,22.6434141921333)
--(axis cs:310000,23.0559227040279)
--(axis cs:330000,25.628317140117)
--(axis cs:350000,25.8945950313742)
--(axis cs:370000,20.5891983402086)
--(axis cs:390000,23.0181811136378)
--(axis cs:410000,27.0225498435703)
--(axis cs:430000,13.6325029867509)
--(axis cs:450000,12.494005540355)
--(axis cs:470000,12.4315191328423)
--(axis cs:490000,9.33214816980198)
--(axis cs:510000,11.8047659179325)
--(axis cs:530000,10.915171404893)
--(axis cs:550000,11.1770395609069)
--(axis cs:570000,12.096438378563)
--(axis cs:590000,13.7089846086019)
--(axis cs:610000,15.0250020587488)
--(axis cs:630000,15.7638052658516)
--(axis cs:650000,10.9461873688642)
--(axis cs:670000,12.8081325989988)
--(axis cs:690000,13.2307761259247)
--(axis cs:710000,13.3577086934129)
--(axis cs:730000,12.9611241247714)
--(axis cs:750000,11.0366643018241)
--(axis cs:770000,10.1227633357069)
--(axis cs:790000,11.1318068035308)
--(axis cs:810000,12.5599768863214)
--(axis cs:830000,8.77902470664821)
--(axis cs:850000,7.8874907578863)
--(axis cs:870000,12.8560816142785)
--(axis cs:890000,11.1033020962465)
--(axis cs:910000,10.5495339407139)
--(axis cs:930000,9.86899838604956)
--(axis cs:950000,14.9429473035543)
--(axis cs:970000,11.1070599315138)
--(axis cs:990000,12.3501733423425)
--(axis cs:990000,299.556015902364)
--(axis cs:990000,299.556015902364)
--(axis cs:970000,303.815396721636)
--(axis cs:950000,295.186397272996)
--(axis cs:930000,261.908240156441)
--(axis cs:910000,244.367742203772)
--(axis cs:890000,232.928472035947)
--(axis cs:870000,280.132968633772)
--(axis cs:850000,248.992682039191)
--(axis cs:830000,235.45211517534)
--(axis cs:810000,235.682759125517)
--(axis cs:790000,252.177580464948)
--(axis cs:770000,235.586015437977)
--(axis cs:750000,209.343413377589)
--(axis cs:730000,225.287447733721)
--(axis cs:710000,219.724694736726)
--(axis cs:690000,199.014027699256)
--(axis cs:670000,217.979319683164)
--(axis cs:650000,198.211038345108)
--(axis cs:630000,181.577450514684)
--(axis cs:610000,167.897952520424)
--(axis cs:590000,157.261533898565)
--(axis cs:570000,144.730380049508)
--(axis cs:550000,139.756759389927)
--(axis cs:530000,120.279316796574)
--(axis cs:510000,138.462703833575)
--(axis cs:490000,161.508136535996)
--(axis cs:470000,160.62259564741)
--(axis cs:450000,126.760316126525)
--(axis cs:430000,145.406328612368)
--(axis cs:410000,128.010122482428)
--(axis cs:390000,118.284301751787)
--(axis cs:370000,117.319213472328)
--(axis cs:350000,109.172708501337)
--(axis cs:330000,95.9601317046164)
--(axis cs:310000,105.989309587614)
--(axis cs:290000,91.5674961022417)
--(axis cs:270000,79.7792327313024)
--(axis cs:250000,78.048632319627)
--(axis cs:230000,77.8790271421551)
--(axis cs:210000,66.6880425024194)
--(axis cs:190000,57.6336429265646)
--(axis cs:170000,52.508201348924)
--(axis cs:150000,41.3993485910569)
--(axis cs:130000,42.7326848378286)
--(axis cs:110000,34.3941446446115)
--(axis cs:90000,29.2616458129603)
--(axis cs:70000,19.604950842758)
--(axis cs:50000,22.8809635341167)
--(axis cs:30000,12.8153479809562)
--(axis cs:10000,18.2317509651184)
--cycle;

\addplot [line width=\linewidthother, C3, mark=*, mark size=0, mark options={solid}]
table {%
10000 13.1034889221191
30000 10.9056301017602
50000 17.1017313376069
70000 10.284638547649
90000 17.3941666317793
110000 20.3164715620611
130000 23.8069427018927
150000 24.5518593590847
170000 32.1034519311864
190000 37.6064343407632
210000 40.9305210510256
230000 49.254217262785
250000 50.5838058614656
270000 52.1672970069376
290000 58.3125095097456
310000 68.3241515371738
330000 65.9415421630294
350000 72.8672516606821
370000 69.7241530918416
390000 72.2556070833635
410000 75.6824835844524
430000 77.2717352332193
450000 65.3944935672089
470000 79.7555051811686
490000 79.6833752700395
510000 56.5746592372547
530000 47.7282040038094
550000 55.4435038093302
570000 64.5000387027653
590000 63.8705090967728
610000 72.2235568679113
630000 82.3147543732159
650000 91.7710620040447
670000 104.87488082566
690000 95.6484667797231
710000 106.309858478369
730000 96.8750950933709
750000 90.3701419464158
770000 101.866760317535
790000 122.806006765378
810000 105.066399037887
830000 101.437606462748
850000 108.755072393393
870000 137.927993255135
890000 112.163157949944
910000 111.621981732676
930000 120.843898397305
950000 143.437162947898
970000 148.962196158127
990000 137.441175480202
};


%CrossQ
\path [draw=C2, fill=C2, opacity=0.2]
(axis cs:1,8.68898606)
--(axis cs:1,6.22902588)
--(axis cs:24000,6.89054972)
--(axis cs:48000,18.4226323)
--(axis cs:72000,22.9457163)
--(axis cs:96000,25.8743944)
--(axis cs:120000,21.395188)
--(axis cs:144000,40.223356)
--(axis cs:168000,37.5831172)
--(axis cs:192000,55.1677062)
--(axis cs:216000,61.476312)
--(axis cs:240000,40.0427422)
--(axis cs:264000,75.9119256)
--(axis cs:288000,64.0229088)
--(axis cs:312000,69.26281094)
--(axis cs:336000,105.510503)
--(axis cs:360000,189.144518)
--(axis cs:384000,134.7545956)
--(axis cs:408000,325.44502)
--(axis cs:432000,351.371588)
--(axis cs:456000,474.697436)
--(axis cs:480000,513.066184250001)
--(axis cs:504000,565.502148)
--(axis cs:528000,575.218164)
--(axis cs:552000,558.370902)
--(axis cs:576000,618.195468)
--(axis cs:600000,628.35331)
--(axis cs:624000,499.802302)
--(axis cs:648000,665.37321)
--(axis cs:672000,666.18247)
--(axis cs:696000,699.63518)
--(axis cs:720000,730.988108)
--(axis cs:744000,718.355928)
--(axis cs:768000,718.788102)
--(axis cs:792000,710.739216)
--(axis cs:816000,749.292068)
--(axis cs:840000,771.139008)
--(axis cs:864000,760.703114)
--(axis cs:888000,797.126376)
--(axis cs:912000,775.720508)
--(axis cs:936000,784.27005)
--(axis cs:960000,819.27462)
--(axis cs:984000,765.188266)
--(axis cs:984000,923.382076)
--(axis cs:984000,923.382076)
--(axis cs:960000,915.261328)
--(axis cs:936000,910.61033)
--(axis cs:912000,913.277484)
--(axis cs:888000,925.64263)
--(axis cs:864000,905.93858)
--(axis cs:840000,905.26686)
--(axis cs:816000,903.308168)
--(axis cs:792000,903.140544)
--(axis cs:768000,907.70392)
--(axis cs:744000,883.678848)
--(axis cs:720000,900.181532)
--(axis cs:696000,876.737092)
--(axis cs:672000,886.125968)
--(axis cs:648000,878.648996)
--(axis cs:624000,868.465284)
--(axis cs:600000,850.24228)
--(axis cs:576000,856.80366)
--(axis cs:552000,791.794644)
--(axis cs:528000,809.540612)
--(axis cs:504000,784.65436)
--(axis cs:480000,702.560658)
--(axis cs:456000,729.235654)
--(axis cs:432000,614.328532)
--(axis cs:408000,544.308078)
--(axis cs:384000,491.643938)
--(axis cs:360000,485.448504)
--(axis cs:336000,416.138828)
--(axis cs:312000,303.69766)
--(axis cs:288000,275.326692)
--(axis cs:264000,245.80224)
--(axis cs:240000,201.195274)
--(axis cs:216000,156.5564)
--(axis cs:192000,124.041462)
--(axis cs:168000,96.0079408)
--(axis cs:144000,84.4066376)
--(axis cs:120000,63.3456824)
--(axis cs:96000,60.4577144)
--(axis cs:72000,47.3699138)
--(axis cs:48000,43.652593)
--(axis cs:24000,14.2879142)
--(axis cs:1,8.68898606)
--cycle;

\addplot [line width=\linewidthother, C2, mark=*, mark size=0, mark options={solid}]
table {%
1 7.32889178
24000 9.5071731
48000 28.9120332
72000 35.1405598
96000 43.6589558
120000 40.6657584
144000 65.728057
168000 65.5245644
192000 94.462425
216000 109.6346504
240000 101.034294
264000 163.8655176
288000 167.1225674
312000 190.204746
336000 252.902089
360000 354.408182
384000 302.710574
408000 470.035322
432000 486.574628
456000 579.059604
480000 615.96973
504000 685.11765
528000 729.546412
552000 699.84182
576000 776.438498
600000 782.533658
624000 735.84468
648000 805.809744
672000 815.51636
696000 826.142582
720000 845.32984
744000 842.857988
768000 852.202504
792000 810.562086
816000 875.861952
840000 865.849218
864000 866.268518
888000 884.86735
912000 861.962388
936000 860.08715
960000 891.207646
984000 881.962054
};
% BRO Fast
\path [draw=C9, fill=C9, opacity=0.2]
(axis cs:25000,9.97776513088848)
--(axis cs:25000,6.11391046517296)
--(axis cs:50000,7.84067005699427)
--(axis cs:75000,7.97392007676465)
--(axis cs:100000,8.09740419350985)
--(axis cs:125000,11.3683075429344)
--(axis cs:150000,17.6621056047365)
--(axis cs:175000,30.9096589826155)
--(axis cs:200000,47.910201076266)
--(axis cs:225000,78.5045023159613)
--(axis cs:250000,140.874381604848)
--(axis cs:275000,194.731844362717)
--(axis cs:300000,274.798919171572)
--(axis cs:325000,344.751291602038)
--(axis cs:350000,365.521699113595)
--(axis cs:375000,407.253430129088)
--(axis cs:400000,473.219916476637)
--(axis cs:425000,504.800771108831)
--(axis cs:450000,535.473298289644)
--(axis cs:475000,608.956351721477)
--(axis cs:500000,636.961078141646)
--(axis cs:525000,664.06429038882)
--(axis cs:550000,698.051719234015)
--(axis cs:575000,718.347929452072)
--(axis cs:600000,747.472114243993)
--(axis cs:625000,784.601151842962)
--(axis cs:650000,802.799047681382)
--(axis cs:675000,832.011326535075)
--(axis cs:700000,850.40572730334)
--(axis cs:725000,853.369000523579)
--(axis cs:750000,861.173146101722)
--(axis cs:775000,884.080460792316)
--(axis cs:800000,887.474452004333)
--(axis cs:825000,892.769456022152)
--(axis cs:850000,886.216717638057)
--(axis cs:875000,881.633552513841)
--(axis cs:900000,901.396587874534)
--(axis cs:925000,907.62415254351)
--(axis cs:950000,912.835655917355)
--(axis cs:975000,904.191151759364)
--(axis cs:1000000,920.460113881173)
--(axis cs:1000000,943.812686950936)
--(axis cs:1000000,943.812686950936)
--(axis cs:975000,937.258878848711)
--(axis cs:950000,933.079338659758)
--(axis cs:925000,934.131892958026)
--(axis cs:900000,933.243651987773)
--(axis cs:875000,931.420851407596)
--(axis cs:850000,930.873559174517)
--(axis cs:825000,931.36094756682)
--(axis cs:800000,930.847697404429)
--(axis cs:775000,934.552649115273)
--(axis cs:750000,928.312326487929)
--(axis cs:725000,919.603170935523)
--(axis cs:700000,925.811457762613)
--(axis cs:675000,892.164056508921)
--(axis cs:650000,883.04137510708)
--(axis cs:625000,885.462974603035)
--(axis cs:600000,826.916482071055)
--(axis cs:575000,824.056875547996)
--(axis cs:550000,824.58735351799)
--(axis cs:525000,779.581292409421)
--(axis cs:500000,769.160639545796)
--(axis cs:475000,717.916434067664)
--(axis cs:450000,684.285505160733)
--(axis cs:425000,637.518645068939)
--(axis cs:400000,627.100249165944)
--(axis cs:375000,551.107295648536)
--(axis cs:350000,517.271421378722)
--(axis cs:325000,493.866698904899)
--(axis cs:300000,416.473105896805)
--(axis cs:275000,351.660595334079)
--(axis cs:250000,280.563167335299)
--(axis cs:225000,187.660376480859)
--(axis cs:200000,132.356701653528)
--(axis cs:175000,46.1857034755477)
--(axis cs:150000,44.2732955963669)
--(axis cs:125000,24.8587209709858)
--(axis cs:100000,17.4262657529611)
--(axis cs:75000,13.6843779676719)
--(axis cs:50000,13.6387524240808)
--(axis cs:25000,9.97776513088848)
--cycle;

\addplot [line width=\linewidthother, C9, mark=*, mark size=0, mark options={solid}]
table {%
25000 7.34058439431668
50000 10.4555807168535
75000 10.1971715641144
100000 11.3992742789542
125000 16.5776348456669
150000 27.8965955940879
175000 39.166290146251
200000 89.2496778063367
225000 134.765414844787
250000 225.037718278935
275000 291.868018971821
300000 375.088678225779
325000 454.538117527969
350000 471.159614969258
375000 491.486801011652
400000 555.658283023977
425000 563.994370701313
450000 602.642447084465
475000 646.079274664456
500000 705.140587236771
525000 719.348711718216
550000 752.018452337776
575000 764.036033998937
600000 789.418615775086
625000 837.220882104169
650000 845.980448899041
675000 870.50356709159
700000 895.158388167022
725000 888.359672244478
750000 905.334801328301
775000 919.683440747763
800000 917.239573516528
825000 911.329107091897
850000 911.983938656252
875000 905.627355594472
900000 920.927673129673
925000 920.915568342594
950000 926.951472678176
975000 924.989414787479
1000000 930.312084969747
};

%Consistency-AC
\path [draw=C5, fill=C5, opacity=0.2]
(axis cs:10000,13.9533226403329)
--(axis cs:10000,8.27090727369178)
--(axis cs:50000,5.69249121050396)
--(axis cs:90000,10.5152338475697)
--(axis cs:130000,10.9375536407389)
--(axis cs:170000,6.22690740389953)
--(axis cs:210000,11.3014906911916)
--(axis cs:250000,16.298482553703)
--(axis cs:290000,18.1293971660599)
--(axis cs:330000,24.9453683085578)
--(axis cs:370000,19.5883831049148)
--(axis cs:410000,20.3095368106207)
--(axis cs:450000,24.3010707343164)
--(axis cs:490000,28.0859851580807)
--(axis cs:530000,27.0216423884408)
--(axis cs:570000,18.141512353782)
--(axis cs:610000,31.0817739114408)
--(axis cs:650000,32.3787750496134)
--(axis cs:690000,21.8813719227102)
--(axis cs:730000,25.1195975669827)
--(axis cs:770000,46.300606260906)
--(axis cs:810000,36.5904350271629)
--(axis cs:850000,30.203098270983)
--(axis cs:890000,31.8271361031776)
--(axis cs:930000,30.8932435189179)
--(axis cs:970000,37.330142057475)
--(axis cs:970000,64.5598991950489)
--(axis cs:970000,64.5598991950489)
--(axis cs:930000,38.4433572183916)
--(axis cs:890000,79.0907072936743)
--(axis cs:850000,59.872255064937)
--(axis cs:810000,72.0785347223906)
--(axis cs:770000,58.6886234972077)
--(axis cs:730000,52.7744015870593)
--(axis cs:690000,47.7908238065234)
--(axis cs:650000,46.075577391959)
--(axis cs:610000,45.2665234057959)
--(axis cs:570000,43.9278985104227)
--(axis cs:530000,49.8529358082242)
--(axis cs:490000,38.4054506789298)
--(axis cs:450000,37.9593562055989)
--(axis cs:410000,42.4354402467535)
--(axis cs:370000,38.417327872513)
--(axis cs:330000,43.1431148220945)
--(axis cs:290000,36.2033304234417)
--(axis cs:250000,32.1589129390208)
--(axis cs:210000,21.7951570438886)
--(axis cs:170000,18.0705778968587)
--(axis cs:130000,18.0486501924151)
--(axis cs:90000,20.7918630873139)
--(axis cs:50000,11.1741615155124)
--(axis cs:10000,13.9533226403329)
--cycle;

\addplot [line width=\linewidthother, C5, mark=*, mark size=0, mark options={solid}]
table {%
10000 11.1121149570123
50000 8.36267873300035
90000 15.6799097053397
130000 13.7610504672365
170000 11.4330212165174
210000 16.9928049849546
250000 24.2286977463619
290000 25.2763557953364
330000 32.5908296461738
370000 29.0028554887139
410000 30.2589472319206
450000 31.1302134699576
490000 33.6299900070377
530000 37.3900474296616
570000 29.5488275407968
610000 36.2029264345556
650000 39.8721652252178
690000 36.857939773609
730000 38.946999577021
770000 52.4025342247281
810000 51.0148669359781
850000 44.3050448139663
890000 54.4653473563726
930000 35.5171075921871
970000 50.945020626262
};

% Diff-QL
\path [draw=C4, fill=C4, opacity=0.2]
(axis cs:10000,9.68696008707864)
--(axis cs:10000,6.56568531083453)
--(axis cs:50000,6.76363941882504)
--(axis cs:90000,7.69732733635889)
--(axis cs:130000,8.58552721745373)
--(axis cs:170000,10.1044398768931)
--(axis cs:210000,10.2540178608293)
--(axis cs:250000,17.6386610881205)
--(axis cs:290000,16.8807031299624)
--(axis cs:330000,16.3407344176024)
--(axis cs:370000,17.7899514890144)
--(axis cs:410000,13.6247548254212)
--(axis cs:450000,17.5400469487565)
--(axis cs:490000,18.4221331450452)
--(axis cs:530000,16.6527493979714)
--(axis cs:570000,27.4874096743717)
--(axis cs:610000,21.4083854709022)
--(axis cs:650000,22.7097066633785)
--(axis cs:690000,17.4781099189239)
--(axis cs:730000,21.4955006954206)
--(axis cs:770000,28.1276666423761)
--(axis cs:810000,27.0237036339671)
--(axis cs:850000,25.9510052803647)
--(axis cs:890000,20.369533713009)
--(axis cs:930000,31.7353284532981)
--(axis cs:970000,24.436460110249)
--(axis cs:970000,67.9574145957911)
--(axis cs:970000,67.9574145957911)
--(axis cs:930000,70.0205206872051)
--(axis cs:890000,58.967324659436)
--(axis cs:850000,58.0159367344782)
--(axis cs:810000,60.5976991251774)
--(axis cs:770000,65.3703443083119)
--(axis cs:730000,61.8190503161534)
--(axis cs:690000,69.7385767491858)
--(axis cs:650000,57.8321933440674)
--(axis cs:610000,51.2471642570984)
--(axis cs:570000,62.4046796110762)
--(axis cs:530000,58.1534477032587)
--(axis cs:490000,52.5958157562911)
--(axis cs:450000,54.273928107082)
--(axis cs:410000,45.9571591144304)
--(axis cs:370000,47.7762289880681)
--(axis cs:330000,31.1914726660906)
--(axis cs:290000,34.2639457207186)
--(axis cs:250000,34.5028583894094)
--(axis cs:210000,31.4837310602421)
--(axis cs:170000,26.3439522357303)
--(axis cs:130000,26.9224301124756)
--(axis cs:90000,12.7055072123941)
--(axis cs:50000,10.7861526854835)
--(axis cs:10000,9.68696008707864)
--cycle;

\addplot [line width=\linewidthother, C4, mark=*, mark size=0, mark options={solid}]
table {%
10000 7.67829538100055
50000 9.0366422305616
90000 10.2820583781489
130000 11.5091247199436
170000 13.709678590562
210000 17.7822334210263
250000 26.0300454519219
290000 24.6143154795398
330000 21.3446647821671
370000 28.7597440986115
410000 19.4825033362556
450000 34.4927035786131
490000 32.8629389588676
530000 34.6257499904115
570000 40.9584319645502
610000 35.8616175883134
650000 36.8033242048668
690000 38.3971037412759
730000 39.0265233872308
770000 48.6999729772951
810000 41.497248240931
850000 40.4423920952037
890000 36.9869297747776
930000 51.7785588219035
970000 43.8368701830181
};

% DIME
\path [draw=C0, fill=C0, opacity=0.2]
(axis cs:1,9.48870533333333)
--(axis cs:1,7.8020031)
--(axis cs:24000,7.332593)
--(axis cs:48000,16.6610283333333)
--(axis cs:72000,26.8240761666667)
--(axis cs:96000,47.8983125)
--(axis cs:120000,93.250831)
--(axis cs:144000,142.432571)
--(axis cs:168000,176.74965)
--(axis cs:192000,270.900023333333)
--(axis cs:216000,348.875643333333)
--(axis cs:240000,433.5656)
--(axis cs:264000,488.92652)
--(axis cs:288000,562.175716666667)
--(axis cs:312000,624.517013333333)
--(axis cs:336000,704.394761041667)
--(axis cs:360000,697.874666666667)
--(axis cs:384000,773.287193333333)
--(axis cs:408000,832.50529)
--(axis cs:432000,792.84356)
--(axis cs:456000,834.105045)
--(axis cs:480000,889.173708333333)
--(axis cs:504000,902.864303333333)
--(axis cs:528000,910.01044)
--(axis cs:552000,899.00265)
--(axis cs:576000,891.240741666667)
--(axis cs:600000,917.167448333333)
--(axis cs:624000,895.689055)
--(axis cs:648000,906.33511)
--(axis cs:672000,921.306265)
--(axis cs:696000,914.593696666667)
--(axis cs:720000,906.40739)
--(axis cs:744000,924.653263333333)
--(axis cs:768000,889.519406666667)
--(axis cs:792000,918.308395)
--(axis cs:816000,923.07566)
--(axis cs:840000,880.981467625)
--(axis cs:864000,914.664325)
--(axis cs:888000,920.578516666667)
--(axis cs:912000,933.507271666667)
--(axis cs:936000,927.001096666667)
--(axis cs:960000,917.831836666667)
--(axis cs:984000,903.89843)
--(axis cs:984000,952.613743333333)
--(axis cs:984000,952.613743333333)
--(axis cs:960000,955.4766)
--(axis cs:936000,953.207133333333)
--(axis cs:912000,957.105826666667)
--(axis cs:888000,948.387113)
--(axis cs:864000,939.174606666667)
--(axis cs:840000,931.386883333333)
--(axis cs:816000,952.79294)
--(axis cs:792000,950.024168541667)
--(axis cs:768000,947.814046666667)
--(axis cs:744000,943.740893333333)
--(axis cs:720000,945.810367833333)
--(axis cs:696000,948.265057333333)
--(axis cs:672000,944.190535)
--(axis cs:648000,933.03308)
--(axis cs:624000,933.266063333333)
--(axis cs:600000,943.306373333333)
--(axis cs:576000,929.518101666667)
--(axis cs:552000,924.743793333333)
--(axis cs:528000,935.990806666667)
--(axis cs:504000,932.09552)
--(axis cs:480000,932.589006666667)
--(axis cs:456000,914.408735)
--(axis cs:432000,923.459587791667)
--(axis cs:408000,913.185415)
--(axis cs:384000,899.286616666667)
--(axis cs:360000,870.796128333333)
--(axis cs:336000,869.913515)
--(axis cs:312000,826.66035)
--(axis cs:288000,764.407513333333)
--(axis cs:264000,657.220633333333)
--(axis cs:240000,552.658643333333)
--(axis cs:216000,509.4753)
--(axis cs:192000,406.611285)
--(axis cs:168000,367.307701666667)
--(axis cs:144000,259.330861666667)
--(axis cs:120000,214.912223333333)
--(axis cs:96000,127.395020833333)
--(axis cs:72000,70.3632726666667)
--(axis cs:48000,36.1025271666667)
--(axis cs:24000,15.8648638333333)
--(axis cs:1,9.48870533333333)
--cycle;

\addplot [line width=\linewidthdime, C0, mark=*, mark size=0, mark options={solid}]
table {%
1 8.52862888333333
24000 10.97423325
48000 25.6190951666667
72000 43.2510993333333
96000 74.0611521666667
120000 135.826194166667
144000 206.044491666667
168000 277.98073
192000 351.200208333333
216000 437.628098333333
240000 483.048963333333
264000 569.98084
288000 660.500133333333
312000 735.636105
336000 817.525161666667
360000 815.94302
384000 855.827801666666
408000 887.508233333333
432000 901.832255
456000 869.797883333333
480000 919.742511666667
504000 917.29069
528000 921.877925
552000 916.053213333333
576000 916.499206666667
600000 934.817345
624000 921.616625
648000 922.048795
672000 932.601863333333
696000 934.52624
720000 936.314575
744000 933.926643333333
768000 928.777961666667
792000 934.017498333333
816000 938.40135
840000 918.455273333333
864000 926.101361666667
888000 938.660406666667
912000 948.406951666666
936000 945.323136666667
960000 944.68845
984000 943.967966666667
};
\end{axis}

\end{tikzpicture}
}
       \subcaption[]{Dog Walk}
       \label{fig::exps_new::dog_walk}
    \end{minipage}\hfill
    \begin{minipage}[b]{0.25\textwidth}
        \centering
       \resizebox{1\textwidth}{!}{% This file was created with tikzplotlib v0.10.1.
\begin{tikzpicture}

\definecolor{darkcyan1115178}{RGB}{1,115,178}
\definecolor{darkgray176}{RGB}{176,176,176}

\begin{axis}[
legend cell align={left},
legend style={fill opacity=0.8, draw opacity=1, text opacity=1, draw=lightgray204, at={(0.03,0.03)},  anchor=north west},
tick align=outside,
tick pos=left,
x grid style={white},
xlabel={Number Env Interactions},
xmajorgrids,
%xmin=-37798.95, xmax=1045799.95,
xmin=-0.0, xmax=1000000.0,
xtick style={color=black},
y grid style={white},
ylabel={IQM Mean Return},
ymajorgrids,
ymin=-0.05, ymax=1050,
ytick style={color=black},
axis background/.style={fill=plot_background},
label style={font=\large},
tick label style={font=\large},
x axis line style={draw=none},
y axis line style={draw=none},
]

% BRO
\path [draw=C1, fill=C1, opacity=0.2]
(axis cs:0,73.1080145480096)
--(axis cs:0,45.9395653774727)
--(axis cs:25000,122.611279033094)
--(axis cs:50000,141.366083480597)
--(axis cs:75000,199.297645632632)
--(axis cs:100000,329.808294875319)
--(axis cs:125000,351.563910945638)
--(axis cs:150000,493.712659633606)
--(axis cs:175000,543.375343209818)
--(axis cs:200000,604.373593903834)
--(axis cs:225000,674.156907979427)
--(axis cs:250000,533.575615362072)
--(axis cs:275000,683.13829479401)
--(axis cs:300000,727.343011277235)
--(axis cs:325000,816.938568344573)
--(axis cs:350000,876.203238642456)
--(axis cs:375000,848.961718599637)
--(axis cs:400000,922.439701922768)
--(axis cs:425000,896.512158009947)
--(axis cs:450000,933.311538621634)
--(axis cs:475000,940.728139176884)
--(axis cs:500000,577.877931755349)
--(axis cs:525000,773.841068682135)
--(axis cs:550000,865.046222351294)
--(axis cs:575000,946.547475348281)
--(axis cs:600000,910.269938516866)
--(axis cs:625000,943.796896934255)
--(axis cs:650000,930.003223206694)
--(axis cs:675000,956.601771236361)
--(axis cs:700000,920.278123229804)
--(axis cs:725000,941.551539144325)
--(axis cs:750000,565.355348860879)
--(axis cs:775000,797.408982582467)
--(axis cs:800000,890.931394676929)
--(axis cs:825000,951.133814706227)
--(axis cs:850000,932.658853884061)
--(axis cs:875000,955.033779398351)
--(axis cs:900000,944.896550745655)
--(axis cs:925000,973.554814172732)
--(axis cs:950000,971.655064564768)
--(axis cs:975000,955.019258495333)
--(axis cs:975000,984.677801870144)
--(axis cs:975000,984.677801870144)
--(axis cs:950000,980.852838166561)
--(axis cs:925000,982.837208264831)
--(axis cs:900000,971.676012573945)
--(axis cs:875000,979.658088480724)
--(axis cs:850000,966.689516806462)
--(axis cs:825000,973.274577061708)
--(axis cs:800000,953.572131484406)
--(axis cs:775000,892.481514805305)
--(axis cs:750000,683.171442044704)
--(axis cs:725000,967.878264775137)
--(axis cs:700000,972.523540687052)
--(axis cs:675000,980.893177390015)
--(axis cs:650000,964.700703364789)
--(axis cs:625000,973.970077987721)
--(axis cs:600000,960.31095106755)
--(axis cs:575000,974.666521447576)
--(axis cs:550000,949.368243176964)
--(axis cs:525000,846.395333834441)
--(axis cs:500000,692.23559711531)
--(axis cs:475000,962.472452073821)
--(axis cs:450000,962.043275642352)
--(axis cs:425000,942.044869600633)
--(axis cs:400000,958.140338794808)
--(axis cs:375000,941.909792366052)
--(axis cs:350000,945.00346141395)
--(axis cs:325000,926.625254951121)
--(axis cs:300000,854.595058695965)
--(axis cs:275000,812.736943943542)
--(axis cs:250000,660.941702495919)
--(axis cs:225000,808.487280304574)
--(axis cs:200000,736.972653531449)
--(axis cs:175000,635.445120488306)
--(axis cs:150000,628.80845662357)
--(axis cs:125000,463.756365404654)
--(axis cs:100000,442.633726318307)
--(axis cs:75000,296.288317697674)
--(axis cs:50000,218.103150963555)
--(axis cs:25000,185.053469319312)
--(axis cs:0,73.1080145480096)
--cycle;

\addplot [line width=\linewidthother, C1, mark=*, mark size=0, mark options={solid}]
table {%
0 61.2337908881688
25000 147.529079455849
50000 173.287019963527
75000 246.76357706707
100000 379.688397190017
125000 401.098074726485
150000 540.287270242405
175000 586.427870385666
200000 670.992586966589
225000 735.313436807604
250000 602.20584812787
275000 740.509151445669
300000 802.426067884996
325000 876.179526858781
350000 918.477897547964
375000 913.936744903587
400000 941.802274331602
425000 917.425909631293
450000 952.093725854481
475000 951.879729693205
500000 630.778945790291
525000 816.191108763568
550000 905.821640430156
575000 962.360908704493
600000 933.81058293948
625000 959.099181084497
650000 951.084486900436
675000 972.374306855342
700000 958.700616493851
725000 954.278515830531
750000 615.524355524439
775000 848.041059461547
800000 930.963319212506
825000 967.689404071816
850000 952.627142574768
875000 970.482041110093
900000 958.86644365511
925000 979.305870524743
950000 976.801719396266
975000 981.003522397514
};
% QSM
\path [draw=C3, fill=C3, opacity=0.2]
(axis cs:10000,51.3219765345255)
--(axis cs:10000,21.6657323877017)
--(axis cs:30000,26.7050110896428)
--(axis cs:50000,29.131708920002)
--(axis cs:70000,30.6965217416485)
--(axis cs:90000,39.7666545659304)
--(axis cs:110000,48.6277184315647)
--(axis cs:130000,48.4293673480987)
--(axis cs:150000,49.1066808786961)
--(axis cs:170000,35.2969372847219)
--(axis cs:190000,29.4197210267521)
--(axis cs:210000,36.5633389162714)
--(axis cs:230000,33.6255328294842)
--(axis cs:250000,37.2782465090465)
--(axis cs:270000,53.8452194939374)
--(axis cs:290000,49.7016840277647)
--(axis cs:310000,42.3349450695508)
--(axis cs:330000,45.2021179274514)
--(axis cs:350000,41.6531172085171)
--(axis cs:370000,43.5347306567709)
--(axis cs:390000,47.2875828227958)
--(axis cs:410000,44.3814357636242)
--(axis cs:430000,63.9288870658903)
--(axis cs:450000,38.7476962341538)
--(axis cs:470000,46.6535987560911)
--(axis cs:490000,48.0297505457731)
--(axis cs:510000,38.6251590171029)
--(axis cs:530000,36.5625076242089)
--(axis cs:550000,33.1682680380025)
--(axis cs:570000,34.5010987857492)
--(axis cs:590000,29.1646374481914)
--(axis cs:610000,38.8769180427446)
--(axis cs:630000,33.7753694641531)
--(axis cs:650000,30.6311508156455)
--(axis cs:670000,26.958540090394)
--(axis cs:690000,37.5792854435929)
--(axis cs:710000,29.3328613395983)
--(axis cs:730000,24.4472061411538)
--(axis cs:750000,27.2666048470364)
--(axis cs:770000,27.5133341655264)
--(axis cs:790000,30.3931852313073)
--(axis cs:810000,34.6299892643555)
--(axis cs:830000,31.9246415917118)
--(axis cs:850000,31.1297063787001)
--(axis cs:870000,24.7995303063456)
--(axis cs:890000,34.1627881201357)
--(axis cs:910000,32.2792608208331)
--(axis cs:930000,24.567536484714)
--(axis cs:950000,34.1182489665315)
--(axis cs:970000,33.0630262069941)
--(axis cs:990000,25.7919953690526)
--(axis cs:990000,287.283823448304)
--(axis cs:990000,287.283823448304)
--(axis cs:970000,263.384839745478)
--(axis cs:950000,262.050620468185)
--(axis cs:930000,240.753815330749)
--(axis cs:910000,246.147086988019)
--(axis cs:890000,240.030819567179)
--(axis cs:870000,237.402935722452)
--(axis cs:850000,237.084883026893)
--(axis cs:830000,236.195921712979)
--(axis cs:810000,225.068227008591)
--(axis cs:790000,229.153781306553)
--(axis cs:770000,219.59298101005)
--(axis cs:750000,212.969418913629)
--(axis cs:730000,218.604992315234)
--(axis cs:710000,219.15375478337)
--(axis cs:690000,225.862016177662)
--(axis cs:670000,233.250480708499)
--(axis cs:650000,243.94154104787)
--(axis cs:630000,256.752232888186)
--(axis cs:610000,223.554552485076)
--(axis cs:590000,221.151955428894)
--(axis cs:570000,249.852949124242)
--(axis cs:550000,276.254172084664)
--(axis cs:530000,270.819822343156)
--(axis cs:510000,284.461463795381)
--(axis cs:490000,279.156429409749)
--(axis cs:470000,296.267980832509)
--(axis cs:450000,291.837055643753)
--(axis cs:430000,285.101839056638)
--(axis cs:410000,241.295906304856)
--(axis cs:390000,257.017965133517)
--(axis cs:370000,252.367794574126)
--(axis cs:350000,238.459042092993)
--(axis cs:330000,230.668160422658)
--(axis cs:310000,226.380585303961)
--(axis cs:290000,226.479607118182)
--(axis cs:270000,207.104125049547)
--(axis cs:250000,208.355894637998)
--(axis cs:230000,216.292165880956)
--(axis cs:210000,197.093786898936)
--(axis cs:190000,198.591899269379)
--(axis cs:170000,161.071574708246)
--(axis cs:150000,163.77511431692)
--(axis cs:130000,130.3843093434)
--(axis cs:110000,111.73182087529)
--(axis cs:90000,80.3005352083593)
--(axis cs:70000,55.5212963743135)
--(axis cs:50000,47.6850173324347)
--(axis cs:30000,44.2510404586792)
--(axis cs:10000,51.3219765345255)
--cycle;

\addplot [line width=\linewidthother, C3, mark=*, mark size=0, mark options={solid}]
table {%
10000 34.2099332809448
30000 34.2843734423319
50000 37.0196850001812
70000 43.9557425354918
90000 55.1142652388662
110000 77.6039161045725
130000 83.5337130466942
150000 106.666662530615
170000 91.0106643037216
190000 111.326688790229
210000 113.521550387792
230000 125.208308792318
250000 110.211391238917
270000 121.851667399545
290000 126.493053966477
310000 118.437488132148
330000 123.839636700057
350000 125.368062504429
370000 132.435427052698
390000 136.142323548182
410000 135.37178979159
430000 168.500323763591
450000 153.528382020966
470000 167.383260259473
490000 159.001140080254
510000 135.391446029633
530000 120.368543662489
550000 121.441942727599
570000 115.040560257424
590000 94.6280358580163
610000 94.1950336977449
630000 93.5864093317093
650000 89.1878148608894
670000 86.7164013756685
690000 69.7266546404351
710000 52.5941541565792
730000 48.0317029250993
750000 43.7676963912066
770000 49.5317145612492
790000 46.8339547620876
810000 49.7279786941077
830000 61.3293076325622
850000 48.2079590572804
870000 46.7592567770711
890000 47.1410181723715
910000 45.8773944236504
930000 43.4441844024321
950000 50.3540771694268
970000 60.4182382000309
990000 61.1898229214753
};
\path [draw=C2, fill=C2, opacity=0.2]
(axis cs:1,26.0251696)
--(axis cs:1,16.4090114)
--(axis cs:24000,22.3758156)
--(axis cs:48000,22.4592122)
--(axis cs:72000,80.817932)
--(axis cs:96000,78.4579168100002)
--(axis cs:120000,110.4801632)
--(axis cs:144000,114.4004844)
--(axis cs:168000,108.0229932)
--(axis cs:192000,192.7424526)
--(axis cs:216000,121.7857848)
--(axis cs:240000,261.341666)
--(axis cs:264000,127.5233954)
--(axis cs:288000,282.716239)
--(axis cs:312000,344.872214)
--(axis cs:336000,288.9853128)
--(axis cs:360000,306.921788)
--(axis cs:384000,409.35112)
--(axis cs:408000,418.7591998)
--(axis cs:432000,411.955442)
--(axis cs:456000,386.839624)
--(axis cs:480000,358.148038)
--(axis cs:504000,433.219458)
--(axis cs:528000,507.7541454)
--(axis cs:552000,480.191004)
--(axis cs:576000,546.2940908)
--(axis cs:600000,567.949693415)
--(axis cs:624000,598.675120800001)
--(axis cs:648000,540.0683024)
--(axis cs:672000,616.9193024)
--(axis cs:696000,623.3655826)
--(axis cs:720000,574.1118993)
--(axis cs:744000,621.0290168)
--(axis cs:768000,626.8547344)
--(axis cs:792000,640.60761)
--(axis cs:816000,663.428268)
--(axis cs:840000,666.613918)
--(axis cs:864000,677.504396)
--(axis cs:888000,660.5894061)
--(axis cs:912000,697.4790756)
--(axis cs:936000,713.243607)
--(axis cs:960000,583.934485800002)
--(axis cs:984000,735.1165646)
--(axis cs:984000,976.39569)
--(axis cs:984000,976.39569)
--(axis cs:960000,966.144204)
--(axis cs:936000,966.90957825)
--(axis cs:912000,960.310008)
--(axis cs:888000,961.97464)
--(axis cs:864000,955.508036)
--(axis cs:840000,967.96859)
--(axis cs:816000,940.49356)
--(axis cs:792000,959.221912)
--(axis cs:768000,949.582646)
--(axis cs:744000,967.454104)
--(axis cs:720000,942.743002)
--(axis cs:696000,938.6763715)
--(axis cs:672000,959.40905)
--(axis cs:648000,942.05094)
--(axis cs:624000,946.22186)
--(axis cs:600000,933.046156)
--(axis cs:576000,922.70192)
--(axis cs:552000,930.945744)
--(axis cs:528000,930.925904)
--(axis cs:504000,945.061174)
--(axis cs:480000,934.199908)
--(axis cs:456000,881.70468)
--(axis cs:432000,865.272804)
--(axis cs:408000,870.973276)
--(axis cs:384000,866.786156)
--(axis cs:360000,860.11428)
--(axis cs:336000,877.636252)
--(axis cs:312000,865.68344)
--(axis cs:288000,870.125728)
--(axis cs:264000,769.410146)
--(axis cs:240000,674.464004)
--(axis cs:216000,533.736622)
--(axis cs:192000,516.167014)
--(axis cs:168000,366.10706)
--(axis cs:144000,466.646916)
--(axis cs:120000,308.047298)
--(axis cs:96000,215.526222)
--(axis cs:72000,184.33103)
--(axis cs:48000,137.193396)
--(axis cs:24000,64.4940632)
--(axis cs:1,26.0251696)
--cycle;

\addplot [line width=\linewidthother, C2, mark=*, mark size=0, mark options={solid}]
table {%
1 20.6663894
24000 36.9090692
48000 55.7893116
72000 129.422736
96000 158.025459
120000 241.07842
144000 273.60558
168000 218.077848
192000 362.980136
216000 314.463602
240000 530.751416
264000 455.3150192
288000 646.483958
312000 685.626048
336000 667.599164
360000 663.36616
384000 719.4979
408000 724.9229
432000 732.29881
456000 721.826314
480000 723.915122
504000 775.987796
528000 812.472952
552000 785.670436
576000 823.226172
600000 863.959634
624000 881.451358
648000 866.528702
672000 895.988846
696000 895.034742
720000 833.212236
744000 900.826744
768000 890.806558
792000 890.116532
816000 900.193172
840000 906.66386
864000 917.279282
888000 900.643422
912000 916.87393
936000 943.23761
960000 931.975996
984000 957.176536
};
%BRO Fast
\path [draw=C9, fill=C9, opacity=0.2]
(axis cs:25000,47.7829716332109)
--(axis cs:25000,28.9775051644369)
--(axis cs:50000,23.4264722372211)
--(axis cs:75000,46.9589480099527)
--(axis cs:100000,58.80772366818)
--(axis cs:125000,150.5342668846)
--(axis cs:150000,252.416205249948)
--(axis cs:175000,302.734367793395)
--(axis cs:200000,384.620773676121)
--(axis cs:225000,484.848948723385)
--(axis cs:250000,540.248989456484)
--(axis cs:275000,613.296656235961)
--(axis cs:300000,654.960733816965)
--(axis cs:325000,742.962777517825)
--(axis cs:350000,712.897861905354)
--(axis cs:375000,795.225691689737)
--(axis cs:400000,859.243088755388)
--(axis cs:425000,831.956266538049)
--(axis cs:450000,853.464624590448)
--(axis cs:475000,879.411020971994)
--(axis cs:500000,865.970581525821)
--(axis cs:525000,904.601610085587)
--(axis cs:550000,921.666013337561)
--(axis cs:575000,937.266806862332)
--(axis cs:600000,902.697004538674)
--(axis cs:625000,932.121053178369)
--(axis cs:650000,930.064406804351)
--(axis cs:675000,917.948802870043)
--(axis cs:700000,940.478202743216)
--(axis cs:725000,932.066111003816)
--(axis cs:750000,939.02375898485)
--(axis cs:775000,954.388399892497)
--(axis cs:800000,939.760806953263)
--(axis cs:825000,922.259618999241)
--(axis cs:850000,954.544360973456)
--(axis cs:875000,949.144358979252)
--(axis cs:900000,953.986433798879)
--(axis cs:925000,926.106454673432)
--(axis cs:950000,958.877793673672)
--(axis cs:975000,945.104625303882)
--(axis cs:1000000,955.213673599658)
--(axis cs:1000000,979.019290198608)
--(axis cs:1000000,979.019290198608)
--(axis cs:975000,981.858841789743)
--(axis cs:950000,983.987072950389)
--(axis cs:925000,973.134521890668)
--(axis cs:900000,975.360160814494)
--(axis cs:875000,968.389045166917)
--(axis cs:850000,972.329474339581)
--(axis cs:825000,966.498263620662)
--(axis cs:800000,970.012679334216)
--(axis cs:775000,973.243382793682)
--(axis cs:750000,971.700671976572)
--(axis cs:725000,968.196048107626)
--(axis cs:700000,964.670710753934)
--(axis cs:675000,962.33253743053)
--(axis cs:650000,962.03550986496)
--(axis cs:625000,965.372332260293)
--(axis cs:600000,963.398737095101)
--(axis cs:575000,964.2076871487)
--(axis cs:550000,951.551670239375)
--(axis cs:525000,951.352451356599)
--(axis cs:500000,944.928898431717)
--(axis cs:475000,933.087963565399)
--(axis cs:450000,924.663521217325)
--(axis cs:425000,910.64389005503)
--(axis cs:400000,932.112206307639)
--(axis cs:375000,875.205815885047)
--(axis cs:350000,858.061117575996)
--(axis cs:325000,865.537701936607)
--(axis cs:300000,814.575479759561)
--(axis cs:275000,762.599068529242)
--(axis cs:250000,683.040833782176)
--(axis cs:225000,577.743396068857)
--(axis cs:200000,515.3194675978)
--(axis cs:175000,414.376285762987)
--(axis cs:150000,343.760590859802)
--(axis cs:125000,234.39779497037)
--(axis cs:100000,156.257332741301)
--(axis cs:75000,110.879138922244)
--(axis cs:50000,69.4587791680422)
--(axis cs:25000,47.7829716332109)
--cycle;

\addplot [line width=\linewidthother, C9, mark=*, mark size=0, mark options={solid}]
table {%
25000 36.2180403465254
50000 40.1436938371026
75000 79.4670268071966
100000 99.0328792470129
125000 190.106955903158
150000 304.788052792802
175000 382.758276578445
200000 443.087216814472
225000 539.657152034908
250000 607.02170801999
275000 686.95491697461
300000 743.112350548942
325000 823.026960453213
350000 772.425109640737
375000 834.797890109431
400000 896.440621854075
425000 872.257121931913
450000 884.872510613197
475000 904.307182287083
500000 905.183981395191
525000 928.895108921565
550000 934.899904855095
575000 951.039479103
600000 938.70376390321
625000 954.369182889067
650000 952.031224287114
675000 942.404510681595
700000 953.29101941604
725000 947.075775737166
750000 955.536028904205
775000 961.654836964447
800000 956.429972422746
825000 952.129872243206
850000 962.835861194816
875000 959.945717255176
900000 966.453304929434
925000 952.204392321671
950000 971.378955704937
975000 963.641606001128
1000000 969.544787272896
};

%Consistency-AC
\path [draw=C5, fill=C5, opacity=0.2]
(axis cs:10000,44.9000675198956)
--(axis cs:10000,23.7879696761684)
--(axis cs:50000,23.9483646805979)
--(axis cs:90000,29.261058130956)
--(axis cs:130000,68.341843178748)
--(axis cs:170000,43.7900185443472)
--(axis cs:210000,57.3683943744889)
--(axis cs:250000,59.1819531432742)
--(axis cs:290000,80.5814547995002)
--(axis cs:330000,84.5799156024914)
--(axis cs:370000,117.448972055787)
--(axis cs:410000,78.2574553282741)
--(axis cs:450000,119.288079160571)
--(axis cs:490000,120.415400004597)
--(axis cs:530000,58.5624675098369)
--(axis cs:570000,83.9674561598339)
--(axis cs:610000,124.455702403331)
--(axis cs:650000,99.8957010791339)
--(axis cs:690000,118.260665264646)
--(axis cs:730000,168.41076843604)
--(axis cs:770000,147.045790546434)
--(axis cs:810000,176.594910329473)
--(axis cs:850000,171.232373652551)
--(axis cs:890000,147.852266348141)
--(axis cs:930000,169.815434806578)
--(axis cs:970000,180.336550189125)
--(axis cs:970000,262.37877239938)
--(axis cs:970000,262.37877239938)
--(axis cs:930000,209.155038267763)
--(axis cs:890000,359.296979358945)
--(axis cs:850000,274.730337864659)
--(axis cs:810000,277.871844229998)
--(axis cs:770000,266.2375687802)
--(axis cs:730000,266.383277142912)
--(axis cs:690000,200.939308177079)
--(axis cs:650000,231.51667635723)
--(axis cs:610000,203.788827858796)
--(axis cs:570000,177.114759257121)
--(axis cs:530000,167.617151298138)
--(axis cs:490000,202.959873731914)
--(axis cs:450000,166.435955045703)
--(axis cs:410000,132.740197926395)
--(axis cs:370000,152.438562164056)
--(axis cs:330000,171.007041686147)
--(axis cs:290000,125.484048312629)
--(axis cs:250000,114.563117371545)
--(axis cs:210000,93.2571770363476)
--(axis cs:170000,95.743254341775)
--(axis cs:130000,84.0610935934159)
--(axis cs:90000,70.7856769649899)
--(axis cs:50000,57.8330497029575)
--(axis cs:10000,44.9000675198956)
--cycle;

\addplot [line width=\linewidthother, C5, mark=*, mark size=0, mark options={solid}]
table {%
10000 34.5081937800948
50000 40.8907071917777
90000 49.4525163839062
130000 76.2014683860819
170000 69.7666364430611
210000 75.9195348992343
250000 86.8725352574096
290000 101.316082159348
330000 125.77784501683
370000 133.124805178347
410000 110.708907885207
450000 143.863485513549
490000 157.291900933103
530000 108.810726362188
570000 133.332295330331
610000 159.739768474094
650000 166.213979266086
690000 160.286524543178
730000 220.716344768988
770000 208.510099871276
810000 222.109203399962
850000 224.682207296754
890000 241.390817195541
930000 192.172525911573
970000 224.226538273288
};
%DIPO
\path [line width=\linewidthother, fill=C6, opacity=0.2]
(axis cs:10000,34.4404392242432)
--(axis cs:10000,18.8483285903931)
--(axis cs:50000,33.7467994689941)
--(axis cs:90000,64.6325950622559)
--(axis cs:130000,96.2688007354736)
--(axis cs:170000,96.6586990356445)
--(axis cs:210000,142.00040435791)
--(axis cs:250000,189.108467102051)
--(axis cs:290000,201.383583068848)
--(axis cs:330000,226.598579406738)
--(axis cs:370000,180.206447601318)
--(axis cs:410000,225.945945739746)
--(axis cs:450000,171.100868225098)
--(axis cs:490000,173.635200500488)
--(axis cs:530000,209.474281311035)
--(axis cs:570000,226.090682983398)
--(axis cs:610000,245.073387145996)
--(axis cs:650000,327.512603759766)
--(axis cs:690000,321.076293945312)
--(axis cs:730000,236.492256164551)
--(axis cs:770000,392.26318359375)
--(axis cs:810000,268.671691894531)
--(axis cs:850000,361.829277038574)
--(axis cs:890000,295.06861114502)
--(axis cs:930000,381.877456665039)
--(axis cs:970000,334.403472900391)
--(axis cs:970000,479.947204589844)
--(axis cs:970000,479.947204589844)
--(axis cs:930000,489.063934326172)
--(axis cs:890000,525.080978393555)
--(axis cs:850000,567.360778808594)
--(axis cs:810000,528.213340759277)
--(axis cs:770000,512.871032714844)
--(axis cs:730000,527.466506958008)
--(axis cs:690000,471.761871337891)
--(axis cs:650000,510.752105712891)
--(axis cs:610000,541.391189575195)
--(axis cs:570000,506.941925048828)
--(axis cs:530000,513.738983154297)
--(axis cs:490000,528.120834350586)
--(axis cs:450000,460.989677429199)
--(axis cs:410000,458.211853027344)
--(axis cs:370000,437.522750854492)
--(axis cs:330000,495.926376342773)
--(axis cs:290000,506.936065673828)
--(axis cs:250000,486.298942565918)
--(axis cs:210000,496.800811767578)
--(axis cs:170000,246.717056274414)
--(axis cs:130000,325.43332862854)
--(axis cs:90000,231.199010848999)
--(axis cs:50000,260.566347122192)
--(axis cs:10000,34.4404392242432)
--cycle;

\addplot [line width=\linewidthother, C6, mark=*, mark size=0, mark options={solid}]
table {%
10000 26.6443839073181
50000 117.252042770386
90000 137.037523269653
130000 187.3125
170000 171.687877655029
210000 319.400608062744
250000 329.128307342529
290000 354.159824371338
330000 361.262477874756
370000 308.864599227905
410000 342.078899383545
450000 320.111968994141
490000 350.878017425537
530000 361.606632232666
570000 366.516304016113
610000 393.232288360596
650000 419.132354736328
690000 415.055541992188
730000 386.971687316895
770000 452.567108154297
810000 393.946876525879
850000 456.403129577637
890000 409.762378692627
930000 448.147720336914
970000 407.47673034668
};
%Diff-QL
\path [draw=C4, fill=C4, opacity=0.2]
(axis cs:10000,55.2872730485069)
--(axis cs:10000,20.0092111603638)
--(axis cs:50000,29.9189860479269)
--(axis cs:90000,31.071210651156)
--(axis cs:130000,57.2394567083522)
--(axis cs:170000,69.3924798105696)
--(axis cs:210000,66.1646284163397)
--(axis cs:250000,90.9178252419078)
--(axis cs:290000,112.452979452038)
--(axis cs:330000,96.6021223078102)
--(axis cs:370000,118.988763608494)
--(axis cs:410000,126.828234403715)
--(axis cs:450000,153.000054417344)
--(axis cs:490000,126.09855129242)
--(axis cs:530000,147.125027737574)
--(axis cs:570000,116.657388900688)
--(axis cs:610000,80.9292893831094)
--(axis cs:650000,104.615616751151)
--(axis cs:690000,106.078827873186)
--(axis cs:730000,107.839053321331)
--(axis cs:770000,105.617086630559)
--(axis cs:810000,107.220666556334)
--(axis cs:850000,131.082830186569)
--(axis cs:890000,108.379688145482)
--(axis cs:930000,143.140893312415)
--(axis cs:970000,127.263105609427)
--(axis cs:970000,266.177770853396)
--(axis cs:970000,266.177770853396)
--(axis cs:930000,267.417029492215)
--(axis cs:890000,246.177130431777)
--(axis cs:850000,280.215621450552)
--(axis cs:810000,223.876193194542)
--(axis cs:770000,224.688892280307)
--(axis cs:730000,226.892220676264)
--(axis cs:690000,216.575439948778)
--(axis cs:650000,207.083674010482)
--(axis cs:610000,211.509516858772)
--(axis cs:570000,230.155567415913)
--(axis cs:530000,226.648955975999)
--(axis cs:490000,242.762228920951)
--(axis cs:450000,256.871278724807)
--(axis cs:410000,211.887105289203)
--(axis cs:370000,217.561213824491)
--(axis cs:330000,204.557330343381)
--(axis cs:290000,221.530693876698)
--(axis cs:250000,198.366404580052)
--(axis cs:210000,182.688867208411)
--(axis cs:170000,169.67412769846)
--(axis cs:130000,125.409583690819)
--(axis cs:90000,72.2551087090024)
--(axis cs:50000,39.4882398941819)
--(axis cs:10000,55.2872730485069)
--cycle;

\addplot [line width=\linewidthother, C4, mark=*, mark size=0, mark options={solid}]
table {%
10000 28.5362650935196
50000 34.094357840127
90000 47.744431392587
130000 88.0232178558393
170000 116.412645420168
210000 104.890487129115
250000 138.318432120774
290000 154.768628799602
330000 128.722163523752
370000 156.601693926222
410000 161.605598302941
450000 205.484979051596
490000 174.093794567151
530000 180.926381790349
570000 164.192563716596
610000 134.602017472479
650000 134.074462291456
690000 145.856075164606
730000 166.636357294698
770000 156.073954574204
810000 161.305695359876
850000 202.440585667559
890000 162.686321652962
930000 198.002175934886
970000 187.199117204042
};

% DIME
\path [draw=C0, fill=C0, opacity=0.2]
(axis cs:1,27.9576268333333)
--(axis cs:1,23.6187098333333)
--(axis cs:24000,42.0466743333333)
--(axis cs:48000,107.902019)
--(axis cs:72000,286.932435)
--(axis cs:96000,383.6607)
--(axis cs:120000,463.301536666667)
--(axis cs:144000,619.97162)
--(axis cs:168000,706.390068333333)
--(axis cs:192000,764.546433333333)
--(axis cs:216000,846.058143333333)
--(axis cs:240000,879.082053333333)
--(axis cs:264000,854.890494)
--(axis cs:288000,831.489558333333)
--(axis cs:312000,878.412856666667)
--(axis cs:336000,853.464935)
--(axis cs:360000,917.62294)
--(axis cs:384000,910.691493333333)
--(axis cs:408000,913.424605)
--(axis cs:432000,935.820018333333)
--(axis cs:456000,952.073296666667)
--(axis cs:480000,934.85277)
--(axis cs:504000,932.658666666667)
--(axis cs:528000,953.095216666667)
--(axis cs:552000,941.610153333333)
--(axis cs:576000,950.202084)
--(axis cs:600000,951.575886666667)
--(axis cs:624000,935.326165)
--(axis cs:648000,938.919125208333)
--(axis cs:672000,944.16811)
--(axis cs:696000,934.772826666667)
--(axis cs:720000,942.421196666667)
--(axis cs:744000,843.021497291667)
--(axis cs:768000,916.137253333333)
--(axis cs:792000,954.30555)
--(axis cs:816000,933.997633333333)
--(axis cs:840000,958.817195)
--(axis cs:864000,949.764043333333)
--(axis cs:888000,944.877466666667)
--(axis cs:912000,933.302925)
--(axis cs:936000,957.398866666667)
--(axis cs:960000,964.81924)
--(axis cs:984000,945.251328333333)
--(axis cs:984000,979.70193125)
--(axis cs:984000,979.70193125)
--(axis cs:960000,984.4912)
--(axis cs:936000,981.188868333333)
--(axis cs:912000,967.73376)
--(axis cs:888000,981.575846666667)
--(axis cs:864000,977.82401)
--(axis cs:840000,981.681953333333)
--(axis cs:816000,971.708668333333)
--(axis cs:792000,972.213991666667)
--(axis cs:768000,976.515908333333)
--(axis cs:744000,980.158934583333)
--(axis cs:720000,973.075451666667)
--(axis cs:696000,969.904716666667)
--(axis cs:672000,969.13425)
--(axis cs:648000,970.057838333334)
--(axis cs:624000,969.209176708333)
--(axis cs:600000,970.86721)
--(axis cs:576000,972.477863333333)
--(axis cs:552000,969.714733333333)
--(axis cs:528000,974.173873333333)
--(axis cs:504000,966.641046666667)
--(axis cs:480000,960.983383333334)
--(axis cs:456000,970.03623)
--(axis cs:432000,970.547446666667)
--(axis cs:408000,959.991151666667)
--(axis cs:384000,952.387888333333)
--(axis cs:360000,952.472605)
--(axis cs:336000,949.266193333333)
--(axis cs:312000,950.346965)
--(axis cs:288000,963.71035)
--(axis cs:264000,949.419965)
--(axis cs:240000,927.764166666667)
--(axis cs:216000,912.540633333333)
--(axis cs:192000,890.191373333333)
--(axis cs:168000,881.580573333333)
--(axis cs:144000,793.342316666667)
--(axis cs:120000,629.494575)
--(axis cs:96000,501.323705)
--(axis cs:72000,389.245586666667)
--(axis cs:48000,199.23498)
--(axis cs:24000,121.677379166667)
--(axis cs:1,27.9576268333333)
--cycle;

\addplot [line width=\linewidthdime, C0, mark=*, mark size=0, mark options={solid}]
table {%
1 25.6876176666667
24000 79.7251113333333
48000 164.047568333333
72000 330.423086666667
96000 441.776863333333
120000 536.288301666667
144000 708.777756666667
168000 811.861221666667
192000 828.85041
216000 877.859698333333
240000 906.229125
264000 922.399648333333
288000 935.62599
312000 929.8002
336000 929.446433333333
360000 935.169975
384000 933.398893333333
408000 934.461331666667
432000 958.981621666667
456000 964.064378333333
480000 951.138926666667
504000 952.336173333333
528000 964.801238333333
552000 960.550486666667
576000 962.05127
600000 962.367853333333
624000 958.099848333333
648000 964.94703
672000 957.015451666667
696000 958.041455
720000 964.004003333333
744000 963.176033333333
768000 960.531601666667
792000 963.301948333333
816000 958.27401
840000 971.449178333333
864000 962.75991
888000 969.21691
912000 950.258991666667
936000 970.688138333333
960000 975.16679
984000 967.598436666667
};
\end{axis}

\end{tikzpicture}
}
       \subcaption[]{Dog Stand}
       \label{fig::exps_new::dog_stand}
    \end{minipage}\hfill
    %\vspace{0.15cm}
    \begin{minipage}[b]{0.25\textwidth}
        \centering
       \resizebox{1\textwidth}{!}{% This file was created with tikzplotlib v0.10.1.
\begin{tikzpicture}

\definecolor{darkcyan1115178}{RGB}{1,115,178}
\definecolor{darkgray176}{RGB}{176,176,176}

\begin{axis}[
legend cell align={left},
legend cell align={left},
legend style={fill opacity=0.8, draw opacity=1, text opacity=1, draw=lightgray204, at={(0.5,0.03)},  anchor=south west},
tick align=outside,
tick pos=left,
x grid style={white},
xlabel={Number Env Interactions},
xmajorgrids,
xmin=-0.0, xmax=1000000.0,
xtick style={color=black},
y grid style={white},
ylabel={IQM Mean Return},
ymajorgrids,
ymin=-12.11232765065, ymax=350.65256879765,
ytick style={color=black},
axis background/.style={fill=plot_background},
label style={font=\large},
tick label style={font=\large},
x axis line style={draw=none},
y axis line style={draw=none},
]

\path [draw=C1, fill=C1, opacity=0.2]
(axis cs:1,0.913481333333333)
--(axis cs:1,0.6287236)
--(axis cs:24000,0.677781566666667)
--(axis cs:48000,0.738428966666667)
--(axis cs:72000,0.790174833333333)
--(axis cs:96000,1.1543524)
--(axis cs:120000,3.23502683000002)
--(axis cs:144000,20.5413528)
--(axis cs:168000,8.54387893333333)
--(axis cs:192000,34.4303118333333)
--(axis cs:216000,72.0098966666667)
--(axis cs:240000,72.3488675833333)
--(axis cs:264000,96.707165)
--(axis cs:288000,96.000784)
--(axis cs:312000,104.958421625)
--(axis cs:336000,114.297126)
--(axis cs:360000,106.284968)
--(axis cs:384000,123.7906925)
--(axis cs:408000,123.723489166667)
--(axis cs:432000,128.761211666667)
--(axis cs:456000,111.02938625)
--(axis cs:480000,135.956141666667)
--(axis cs:504000,138.98209)
--(axis cs:528000,118.929581333333)
--(axis cs:552000,132.387325)
--(axis cs:576000,148.148588333333)
--(axis cs:600000,153.61958)
--(axis cs:624000,156.55045)
--(axis cs:648000,156.870825)
--(axis cs:672000,161.074148333333)
--(axis cs:696000,164.107246666667)
--(axis cs:720000,165.654856541667)
--(axis cs:744000,167.669441666667)
--(axis cs:768000,160.786089)
--(axis cs:792000,168.99019)
--(axis cs:816000,177.369616666667)
--(axis cs:840000,175.239106666667)
--(axis cs:864000,176.361243333333)
--(axis cs:888000,165.394583333333)
--(axis cs:912000,183.752826666667)
--(axis cs:936000,187.066958333333)
--(axis cs:960000,160.625822716667)
--(axis cs:984000,191.555848333333)
--(axis cs:984000,251.074266666667)
--(axis cs:984000,251.074266666667)
--(axis cs:960000,253.124288333333)
--(axis cs:936000,245.776626666667)
--(axis cs:912000,236.034463333333)
--(axis cs:888000,226.598225)
--(axis cs:864000,227.013098708333)
--(axis cs:840000,234.283123333333)
--(axis cs:816000,229.472495)
--(axis cs:792000,224.034361666667)
--(axis cs:768000,206.047165)
--(axis cs:744000,209.549365)
--(axis cs:720000,205.859246666667)
--(axis cs:696000,194.984201666667)
--(axis cs:672000,188.552526666667)
--(axis cs:648000,185.157259583333)
--(axis cs:624000,177.684235)
--(axis cs:600000,172.894043333333)
--(axis cs:576000,171.0584)
--(axis cs:552000,163.391113333333)
--(axis cs:528000,162.135815)
--(axis cs:504000,161.517217666667)
--(axis cs:480000,153.846013166667)
--(axis cs:456000,150.815503333333)
--(axis cs:432000,154.043445)
--(axis cs:408000,146.23077)
--(axis cs:384000,144.130305)
--(axis cs:360000,136.713208333333)
--(axis cs:336000,131.509801666667)
--(axis cs:312000,123.557279541667)
--(axis cs:288000,121.774874166667)
--(axis cs:264000,117.476811666667)
--(axis cs:240000,110.350656666667)
--(axis cs:216000,103.274480333333)
--(axis cs:192000,90.3314516666667)
--(axis cs:168000,85.1717466666667)
--(axis cs:144000,83.9523868333333)
--(axis cs:120000,60.989724)
--(axis cs:96000,26.2789549666667)
--(axis cs:72000,5.35472753333333)
--(axis cs:48000,1.06001243333333)
--(axis cs:24000,0.923337466666667)
--(axis cs:1,0.913481333333333)
--cycle;

\addplot [line width=\linewidthdime, C1, mark=*, mark size=0, mark options={solid}]
table {%
1 0.7694417
24000 0.813880766666667
48000 0.903083466666667
72000 0.985341633333333
96000 7.39312896666667
120000 31.2267919666667
144000 57.5656945
168000 45.1603116666667
192000 70.5692
216000 94.6452396666667
240000 97.5139125
264000 109.186628333333
288000 109.1955065
312000 111.598211666667
336000 123.198625
360000 125.946167
384000 133.888621666667
408000 134.214293333333
432000 139.483735
456000 142.763216666667
480000 144.338755
504000 149.75221
528000 151.421726666667
552000 147.365006666667
576000 159.676698333333
600000 161.191491666667
624000 163.406718333333
648000 169.18849
672000 173.390193333333
696000 174.364796666667
720000 181.648728333333
744000 181.343818333333
768000 179.253028333333
792000 189.492575
816000 195.565436666667
840000 198.284515
864000 193.75859
888000 187.434418333333
912000 201.805691666667
936000 209.501925
960000 208.739933333333
984000 214.710998333333
};

%DIME
\path [draw=C0, fill=C0, opacity=0.2]
(axis cs:1,0.915501966666667)
--(axis cs:1,0.633556866666667)
--(axis cs:36000,0.5493304)
--(axis cs:72000,0.843873266666667)
--(axis cs:108000,4.13696455)
--(axis cs:144000,36.1211871666667)
--(axis cs:180000,51.4064937333333)
--(axis cs:216000,81.368514)
--(axis cs:252000,97.5408251666667)
--(axis cs:288000,109.583571666667)
--(axis cs:324000,110.618288333333)
--(axis cs:360000,120.684184666667)
--(axis cs:396000,122.417020833333)
--(axis cs:432000,131.747461666667)
--(axis cs:468000,139.23245)
--(axis cs:504000,145.708203333333)
--(axis cs:540000,128.071794166667)
--(axis cs:576000,146.743565)
--(axis cs:612000,157.523090333333)
--(axis cs:648000,162.386513333333)
--(axis cs:684000,167.712168333333)
--(axis cs:720000,177.17324)
--(axis cs:756000,175.778066666667)
--(axis cs:792000,177.557476666667)
--(axis cs:828000,183.413371666667)
--(axis cs:864000,191.343878333333)
--(axis cs:900000,198.058078333333)
--(axis cs:936000,199.495351666667)
--(axis cs:972000,204.18452)
--(axis cs:1008000,211.017165)
--(axis cs:1044000,210.395921666667)
--(axis cs:1080000,225.37889)
--(axis cs:1116000,225.57669)
--(axis cs:1152000,229.055506666667)
--(axis cs:1188000,244.717408333333)
--(axis cs:1224000,235.69925925)
--(axis cs:1260000,246.761671666667)
--(axis cs:1296000,263.70817)
--(axis cs:1332000,261.655338333333)
--(axis cs:1368000,274.904988333333)
--(axis cs:1404000,281.871765)
--(axis cs:1440000,284.6845)
--(axis cs:1476000,298.373705)
--(axis cs:1512000,314.234433333333)
--(axis cs:1548000,302.586606666667)
--(axis cs:1584000,306.158113166667)
--(axis cs:1620000,329.89694)
--(axis cs:1656000,327.346265)
--(axis cs:1692000,331.977616666667)
--(axis cs:1728000,335.598585)
--(axis cs:1764000,337.5854)
--(axis cs:1800000,350.510931458333)
--(axis cs:1836000,348.138216666667)
--(axis cs:1872000,325.377726666667)
--(axis cs:1908000,375.064071666667)
--(axis cs:1944000,364.504996666667)
--(axis cs:1980000,362.576775)
--(axis cs:2016000,359.905746666667)
--(axis cs:2052000,401.565295)
--(axis cs:2088000,380.944233333333)
--(axis cs:2124000,408.07439)
--(axis cs:2160000,406.816443333333)
--(axis cs:2196000,402.540623333333)
--(axis cs:2232000,419.828846666667)
--(axis cs:2268000,419.407693333333)
--(axis cs:2304000,391.893025)
--(axis cs:2340000,426.646173333333)
--(axis cs:2376000,430.59703)
--(axis cs:2412000,440.944021375)
--(axis cs:2448000,444.535958333333)
--(axis cs:2484000,447.59097)
--(axis cs:2520000,458.220718125)
--(axis cs:2556000,454.987311666667)
--(axis cs:2592000,444.593778333333)
--(axis cs:2628000,452.355353333333)
--(axis cs:2664000,472.315071583333)
--(axis cs:2700000,468.552297125)
--(axis cs:2736000,425.893185)
--(axis cs:2772000,461.17882)
--(axis cs:2808000,438.974565)
--(axis cs:2844000,408.566183333333)
--(axis cs:2880000,440.493023333333)
--(axis cs:2916000,474.594651666667)
--(axis cs:2952000,473.13282)
--(axis cs:2988000,468.31128725)
--(axis cs:2988000,572.54452975)
--(axis cs:2988000,572.54452975)
--(axis cs:2952000,607.047773333333)
--(axis cs:2916000,581.955683333333)
--(axis cs:2880000,562.69891)
--(axis cs:2844000,570.399441666667)
--(axis cs:2808000,561.294171666667)
--(axis cs:2772000,575.424046666667)
--(axis cs:2736000,577.564579375)
--(axis cs:2700000,570.731558333333)
--(axis cs:2664000,556.645568333333)
--(axis cs:2628000,528.610636666667)
--(axis cs:2592000,572.906766666667)
--(axis cs:2556000,556.792031666667)
--(axis cs:2520000,565.548321666667)
--(axis cs:2484000,564.076643333333)
--(axis cs:2448000,554.428711666667)
--(axis cs:2412000,511.028458333333)
--(axis cs:2376000,539.327213333333)
--(axis cs:2340000,521.274179458333)
--(axis cs:2304000,507.011033333333)
--(axis cs:2268000,533.545403333333)
--(axis cs:2232000,530.436551666667)
--(axis cs:2196000,514.156165)
--(axis cs:2160000,548.6599)
--(axis cs:2124000,526.241701666667)
--(axis cs:2088000,488.449933333333)
--(axis cs:2052000,521.397975)
--(axis cs:2016000,489.724498333333)
--(axis cs:1980000,511.56874)
--(axis cs:1944000,469.79517)
--(axis cs:1908000,487.4035)
--(axis cs:1872000,452.261116666667)
--(axis cs:1836000,448.443288333333)
--(axis cs:1800000,470.013476541667)
--(axis cs:1764000,441.067136666667)
--(axis cs:1728000,425.983283333333)
--(axis cs:1692000,459.900376666667)
--(axis cs:1656000,420.555463333333)
--(axis cs:1620000,452.412428333333)
--(axis cs:1584000,426.080911666667)
--(axis cs:1548000,405.744391916667)
--(axis cs:1512000,422.468795)
--(axis cs:1476000,392.121763333333)
--(axis cs:1440000,398.927006666667)
--(axis cs:1404000,396.300881666667)
--(axis cs:1368000,391.708926666667)
--(axis cs:1332000,379.3367)
--(axis cs:1296000,374.028266666667)
--(axis cs:1260000,354.826833333333)
--(axis cs:1224000,325.77437)
--(axis cs:1188000,356.78896)
--(axis cs:1152000,352.524811666667)
--(axis cs:1116000,343.294016666667)
--(axis cs:1080000,331.116815)
--(axis cs:1044000,287.783165)
--(axis cs:1008000,319.938826666667)
--(axis cs:972000,297.00894)
--(axis cs:936000,297.43841)
--(axis cs:900000,295.160138333333)
--(axis cs:864000,278.186283333333)
--(axis cs:828000,267.117865)
--(axis cs:792000,254.04739775)
--(axis cs:756000,245.040613333333)
--(axis cs:720000,223.696456875)
--(axis cs:684000,216.1656)
--(axis cs:648000,205.401713333333)
--(axis cs:612000,201.404601666667)
--(axis cs:576000,182.55503)
--(axis cs:540000,177.786793333333)
--(axis cs:504000,170.561265208333)
--(axis cs:468000,170.450261666667)
--(axis cs:432000,164.02998)
--(axis cs:396000,153.633991666667)
--(axis cs:360000,143.37209)
--(axis cs:324000,136.525775833333)
--(axis cs:288000,129.716861683333)
--(axis cs:252000,113.1404125)
--(axis cs:216000,102.49002)
--(axis cs:180000,92.9197603333333)
--(axis cs:144000,76.8890066666667)
--(axis cs:108000,56.7536433333333)
--(axis cs:72000,4.25857906666667)
--(axis cs:36000,0.728938933333333)
--(axis cs:1,0.915501966666667)
--cycle;

\addplot [line width=\linewidthdime, C0, mark=*, mark size=0, mark options={solid}]
table {%
1 0.770562166666667
36000 0.658370533333333
72000 1.02744053333333
108000 27.4132359166667
144000 61.7389005
180000 78.3388245
216000 96.6475441666667
252000 104.292225666667
288000 118.336289333333
324000 122.156900833333
360000 127.865645666667
396000 135.589655833333
432000 144.61231
468000 150.876916666667
504000 155.106371666667
540000 161.848961666667
576000 162.873085
612000 175.66877
648000 181.774033333333
684000 189.795443333333
720000 201.694543333333
756000 209.679653333333
792000 215.950858333333
828000 226.452128333333
864000 235.790758333333
900000 245.745095
936000 251.419853333333
972000 253.923228333333
1008000 267.745066666667
1044000 253.029908333333
1080000 281.1757
1116000 290.478148333333
1152000 295.386573333333
1188000 307.510898333333
1224000 279.324226666667
1260000 298.499671666667
1296000 320.530285
1332000 314.688511666667
1368000 333.564955
1404000 344.035705
1440000 351.026203333333
1476000 334.48205
1512000 371.517083333333
1548000 359.107073333333
1584000 362.49581
1620000 383.735268333333
1656000 371.120853333333
1692000 392.875408333333
1728000 375.706403333333
1764000 385.148411666667
1800000 411.377563333333
1836000 393.022041666667
1872000 381.083618333333
1908000 428.514751666667
1944000 407.976545
1980000 440.522148333333
2016000 425.096363333333
2052000 460.246375
2088000 439.221346666667
2124000 470.482476666667
2160000 479.806223333333
2196000 467.028006666667
2232000 471.833656666667
2268000 492.305195
2304000 444.283438333333
2340000 467.734686666667
2376000 478.305563333333
2412000 479.659883333333
2448000 502.891963333333
2484000 512.254108333333
2520000 504.927955
2556000 509.89631
2592000 517.465808333333
2628000 505.843831666667
2664000 503.219516666667
2700000 525.334166666667
2736000 498.718568333333
2772000 515.273173333333
2808000 507.004858333333
2844000 490.845221666667
2880000 488.33129
2916000 530.288801666667
2952000 544.121438333333
2988000 506.351106666667
};

\end{axis}

\end{tikzpicture}
}
       \subcaption[]{Humanoid Run}
       \label{fig::exps_new::humanoid_run}
    \end{minipage}\hfill
   \begin{minipage}[b]{0.25\textwidth}
        \centering
       \resizebox{1\textwidth}{!}{% This file was created with tikzplotlib v0.10.1.
\begin{tikzpicture}

\definecolor{darkcyan1115178}{RGB}{1,115,178}
\definecolor{darkgray176}{RGB}{176,176,176}

\begin{axis}[
legend cell align={left},
legend style={fill opacity=0.8, draw opacity=1, text opacity=1, draw=lightgray204, at={(0.03,0.03)},  anchor=north west},
tick align=outside,
tick pos=left,
x grid style={white},
xlabel={Number Env Interactions},
xmajorgrids,
%xmin=-37798.95, xmax=1045799.95,
xmin=-0.0, xmax=1000000.0,
xtick style={color=black},
y grid style={white},
ylabel={IQM Mean Return},
ymajorgrids,
ymin=-0.05, ymax=1000,
ytick style={color=black},
axis background/.style={fill=plot_background},
label style={font=\large},
tick label style={font=\large},
x axis line style={draw=none},
y axis line style={draw=none},
]

%Consistency-AC
\path [draw=C5, fill=C5, opacity=0.2]
(axis cs:10000,1.37260497353048)
--(axis cs:10000,0.961493145056)
--(axis cs:50000,1.19879377997408)
--(axis cs:90000,1.1961187435265)
--(axis cs:130000,1.00716716275698)
--(axis cs:170000,1.29737056608604)
--(axis cs:210000,1.42463189438362)
--(axis cs:250000,1.55957584082097)
--(axis cs:290000,1.39513671351692)
--(axis cs:330000,1.47640121419869)
--(axis cs:370000,1.6337902283582)
--(axis cs:410000,1.66427846205601)
--(axis cs:450000,1.34581747967536)
--(axis cs:490000,1.86082736345981)
--(axis cs:530000,1.34477360937156)
--(axis cs:570000,1.57339369209151)
--(axis cs:610000,1.32571976032438)
--(axis cs:650000,1.55901340116942)
--(axis cs:690000,1.7721945039547)
--(axis cs:730000,1.59206028787586)
--(axis cs:770000,1.69313286816616)
--(axis cs:810000,1.67038110965322)
--(axis cs:850000,1.44089776308633)
--(axis cs:890000,1.69806824631966)
--(axis cs:930000,1.41139525455554)
--(axis cs:970000,1.73069356911495)
--(axis cs:970000,2.50838735122753)
--(axis cs:970000,2.50838735122753)
--(axis cs:930000,2.96277187032518)
--(axis cs:890000,2.2988922692023)
--(axis cs:850000,2.1269277399458)
--(axis cs:810000,2.53627777832682)
--(axis cs:770000,2.38322121842704)
--(axis cs:730000,1.77918328280744)
--(axis cs:690000,2.13775928080781)
--(axis cs:650000,2.16616828585624)
--(axis cs:610000,1.97554635526195)
--(axis cs:570000,1.94561761928452)
--(axis cs:530000,1.99715245597767)
--(axis cs:490000,2.1737874479662)
--(axis cs:450000,2.53677991097283)
--(axis cs:410000,2.3058749322035)
--(axis cs:370000,2.28462396293273)
--(axis cs:330000,2.10421195542469)
--(axis cs:290000,1.69289496064689)
--(axis cs:250000,2.23195809647073)
--(axis cs:210000,2.00834536095867)
--(axis cs:170000,2.21803447791821)
--(axis cs:130000,1.46864132667266)
--(axis cs:90000,1.44964570836897)
--(axis cs:50000,1.70816624897043)
--(axis cs:10000,1.37260497353048)
--cycle;

\addplot [line width=\linewidthother, C5, mark=*, mark size=0, mark options={solid}]
table {%
10000 1.16201534312872
50000 1.4610899949912
90000 1.30891968137887
130000 1.23998562153192
170000 1.70550376919435
210000 1.65081621013609
250000 1.8666455315097
290000 1.53372674903284
330000 1.79200373212576
370000 1.92598744526825
410000 1.98507669712976
450000 1.91517070370392
490000 2.00125653561071
530000 1.73825888936788
570000 1.77734460619993
610000 1.68707558096425
650000 1.86259084351283
690000 1.95995015877981
730000 1.69073468478584
770000 2.0249162513896
810000 2.10332944399002
850000 1.78391275151606
890000 2.03793022249033
930000 2.13511618437802
970000 2.09039394703152
};

%Diffusion-QL
\path [draw=C4, fill=C4, opacity=0.2]
(axis cs:10000,1.3416994952522)
--(axis cs:10000,1.03784647816938)
--(axis cs:50000,1.2855791865615)
--(axis cs:90000,1.41191350624092)
--(axis cs:130000,1.38840968981621)
--(axis cs:170000,1.49925861126791)
--(axis cs:210000,1.76294897921637)
--(axis cs:250000,1.53312453570517)
--(axis cs:290000,1.68368744164181)
--(axis cs:330000,1.6017081382777)
--(axis cs:370000,1.71009237586969)
--(axis cs:410000,1.65267970217337)
--(axis cs:450000,1.69568278140009)
--(axis cs:490000,1.67688054712785)
--(axis cs:530000,1.82275236170989)
--(axis cs:570000,1.87130214626991)
--(axis cs:610000,1.67993105513771)
--(axis cs:650000,1.61561667066583)
--(axis cs:690000,1.72332096338567)
--(axis cs:730000,1.9461512807115)
--(axis cs:770000,2.07326854566496)
--(axis cs:810000,1.65842440452011)
--(axis cs:850000,1.91041247301554)
--(axis cs:890000,1.85103754424372)
--(axis cs:930000,1.75341459476858)
--(axis cs:970000,1.75906629165571)
--(axis cs:970000,2.08251792826071)
--(axis cs:970000,2.08251792826071)
--(axis cs:930000,2.30326721789481)
--(axis cs:890000,2.27197307907247)
--(axis cs:850000,2.40497350326838)
--(axis cs:810000,2.18139889595959)
--(axis cs:770000,2.41227497336855)
--(axis cs:730000,2.47795103704541)
--(axis cs:690000,2.32575968425842)
--(axis cs:650000,2.11079357965707)
--(axis cs:610000,2.19136312194803)
--(axis cs:570000,2.12924521416363)
--(axis cs:530000,2.13810156835361)
--(axis cs:490000,2.06697349470997)
--(axis cs:450000,2.13459989172466)
--(axis cs:410000,2.14675029377386)
--(axis cs:370000,2.26468279285388)
--(axis cs:330000,2.02210851806541)
--(axis cs:290000,2.08964711317862)
--(axis cs:250000,2.1082047254238)
--(axis cs:210000,1.96947941129895)
--(axis cs:170000,1.8164019072593)
--(axis cs:130000,1.7446863577101)
--(axis cs:90000,1.7310769192653)
--(axis cs:50000,1.67260994954055)
--(axis cs:10000,1.3416994952522)
--cycle;

\addplot [line width =\linewidthother, C4, mark=*, mark size=0, mark options={solid}]
table {%
10000 1.19013755855947
50000 1.47445533635287
90000 1.57432803604553
130000 1.56567957932563
170000 1.66299654022159
210000 1.8477213367247
250000 1.80488156864834
290000 1.89241550741572
330000 1.80211250383476
370000 1.99237134384527
410000 1.90020970587671
450000 1.91069162061636
490000 1.874526201341
530000 1.98709300034442
570000 2.0053018936821
610000 1.9229426870549
650000 1.8600522332348
690000 2.02609739263622
730000 2.20618466229058
770000 2.24883625753222
810000 1.91629667966385
850000 2.15202178049651
890000 2.05512279711245
930000 2.00520982846367
970000 1.92138138011111
};

% BRO
\path [draw=C1, fill=C1, opacity=0.2]
(axis cs:25000,1.26311089969273)
--(axis cs:25000,0.881120350809956)
--(axis cs:50000,1.22429669002549)
--(axis cs:75000,1.2540915607512)
--(axis cs:100000,1.56548233538941)
--(axis cs:125000,42.2653406461339)
--(axis cs:150000,108.370326220328)
--(axis cs:175000,155.406986562437)
--(axis cs:200000,229.299543984666)
--(axis cs:225000,264.283666214254)
--(axis cs:250000,297.919492034969)
--(axis cs:275000,315.717840915388)
--(axis cs:300000,382.60663803708)
--(axis cs:325000,433.235974765852)
--(axis cs:350000,448.251035337998)
--(axis cs:375000,474.204600574547)
--(axis cs:400000,494.104875681655)
--(axis cs:425000,513.088247825284)
--(axis cs:450000,541.177185518471)
--(axis cs:475000,552.690096109371)
--(axis cs:500000,571.459696002497)
--(axis cs:525000,415.380427592584)
--(axis cs:550000,575.596355986013)
--(axis cs:575000,597.513568608555)
--(axis cs:600000,625.773969140176)
--(axis cs:625000,638.001168275702)
--(axis cs:650000,663.963551803929)
--(axis cs:675000,670.459110891045)
--(axis cs:700000,710.345648538078)
--(axis cs:725000,705.037719809895)
--(axis cs:750000,751.613279549504)
--(axis cs:775000,433.877621302581)
--(axis cs:800000,660.549669529238)
--(axis cs:825000,772.704234687214)
--(axis cs:850000,805.058655829785)
--(axis cs:875000,830.696042960035)
--(axis cs:900000,853.083635192577)
--(axis cs:925000,878.255891967048)
--(axis cs:950000,864.663890838824)
--(axis cs:975000,903.874025694564)
--(axis cs:1000000,899.607716202274)
--(axis cs:1000000,920.313740047826)
--(axis cs:1000000,920.313740047826)
--(axis cs:975000,925.952957421694)
--(axis cs:950000,919.921491476974)
--(axis cs:925000,925.164748806119)
--(axis cs:900000,925.717722584592)
--(axis cs:875000,906.719467383778)
--(axis cs:850000,907.002250230725)
--(axis cs:825000,905.290252669135)
--(axis cs:800000,803.700888879007)
--(axis cs:775000,534.035124257512)
--(axis cs:750000,896.187503625904)
--(axis cs:725000,896.626244208851)
--(axis cs:700000,900.082220734413)
--(axis cs:675000,909.431675613549)
--(axis cs:650000,875.289963021792)
--(axis cs:625000,902.538081879944)
--(axis cs:600000,891.753933980725)
--(axis cs:575000,858.146145604334)
--(axis cs:550000,803.430110055241)
--(axis cs:525000,539.486903912702)
--(axis cs:500000,830.252850484027)
--(axis cs:475000,741.683202403739)
--(axis cs:450000,693.567342172253)
--(axis cs:425000,646.586647878359)
--(axis cs:400000,613.477567001121)
--(axis cs:375000,595.475083872012)
--(axis cs:350000,561.891903414579)
--(axis cs:325000,531.787663780241)
--(axis cs:300000,504.785345575797)
--(axis cs:275000,383.843033300178)
--(axis cs:250000,395.079009794391)
--(axis cs:225000,361.262822100209)
--(axis cs:200000,328.548854587611)
--(axis cs:175000,267.01045081722)
--(axis cs:150000,219.381538234973)
--(axis cs:125000,146.024570559908)
--(axis cs:100000,46.953705227703)
--(axis cs:75000,2.74535568656764)
--(axis cs:50000,1.77074346132751)
--(axis cs:25000,1.26311089969273)
--cycle;

\addplot [line width=\linewidthother, C1, mark=*, mark size=0, mark options={solid}]
table {%
25000 1.09209800818576
50000 1.49028631817518
75000 1.5900187341138
100000 15.5513850057632
125000 88.1672431290349
150000 165.862611581704
175000 210.233872823628
200000 276.015132592373
225000 305.297564380387
250000 341.442138300315
275000 343.805740811048
300000 425.46657802453
325000 459.559589306348
350000 480.816449894482
375000 517.421843557096
400000 536.45368034477
425000 563.593474753503
450000 601.90835944414
475000 635.963366215657
500000 690.300187872775
525000 463.52284283259
550000 675.774290682452
575000 727.501435004846
600000 758.447114385751
625000 778.280030105389
650000 782.060793794658
675000 805.130496161245
700000 824.719415060329
725000 803.450568434708
750000 838.818421894222
775000 477.785389615811
800000 732.350814882556
825000 858.718771471815
850000 874.042599445299
875000 877.806319388745
900000 901.836765841331
925000 907.139996603909
950000 895.077239285106
975000 914.636264799804
1000000 911.910486485234
};
%CrossQ
\path [draw=C2, fill=C2, opacity=0.2]
(axis cs:1,1.206611)
--(axis cs:1,0.733544166666667)
--(axis cs:24000,0.887822566666667)
--(axis cs:48000,1.13764666666667)
--(axis cs:72000,0.860453)
--(axis cs:96000,1.45535751666667)
--(axis cs:120000,1.35952953333333)
--(axis cs:144000,1.68574125)
--(axis cs:168000,1.71572826666667)
--(axis cs:192000,2.17852331666667)
--(axis cs:216000,1.8408793)
--(axis cs:240000,1.99721312666667)
--(axis cs:264000,2.21934473333333)
--(axis cs:288000,2.25771261666667)
--(axis cs:312000,2.29095871666667)
--(axis cs:336000,2.53978901666667)
--(axis cs:360000,4.19306963333333)
--(axis cs:384000,3.63596193333333)
--(axis cs:408000,47.4714064333333)
--(axis cs:432000,90.1231862333333)
--(axis cs:456000,105.595109533333)
--(axis cs:480000,118.503758833333)
--(axis cs:504000,130.987353966667)
--(axis cs:528000,138.513652666667)
--(axis cs:552000,142.465131133333)
--(axis cs:576000,184.955457266667)
--(axis cs:600000,223.277839983333)
--(axis cs:624000,231.933094583333)
--(axis cs:648000,235.460865111667)
--(axis cs:672000,251.723714133333)
--(axis cs:696000,252.288400633333)
--(axis cs:720000,256.257522366667)
--(axis cs:744000,265.2886915)
--(axis cs:768000,265.892204483333)
--(axis cs:792000,270.361925666667)
--(axis cs:816000,273.9658075)
--(axis cs:840000,279.460149666667)
--(axis cs:864000,282.75856475)
--(axis cs:888000,290.416488841667)
--(axis cs:912000,304.976288333333)
--(axis cs:936000,354.478418)
--(axis cs:960000,415.979514375)
--(axis cs:984000,413.052065)
--(axis cs:984000,680.11246)
--(axis cs:984000,680.11246)
--(axis cs:960000,671.605166666667)
--(axis cs:936000,658.205405)
--(axis cs:912000,652.080526083334)
--(axis cs:888000,640.145955)
--(axis cs:864000,630.531503333333)
--(axis cs:840000,624.168621666667)
--(axis cs:816000,610.057871666667)
--(axis cs:792000,607.71419)
--(axis cs:768000,604.484933333333)
--(axis cs:744000,593.753)
--(axis cs:720000,576.102098333333)
--(axis cs:696000,573.056458333333)
--(axis cs:672000,551.732046666667)
--(axis cs:648000,554.370033333333)
--(axis cs:624000,542.118425)
--(axis cs:600000,513.957726666667)
--(axis cs:576000,517.319851666667)
--(axis cs:552000,493.734771666667)
--(axis cs:528000,480.815375)
--(axis cs:504000,464.369066666667)
--(axis cs:480000,446.22159)
--(axis cs:456000,436.844696666667)
--(axis cs:432000,414.495293333333)
--(axis cs:408000,393.195931666667)
--(axis cs:384000,354.493778333333)
--(axis cs:360000,306.144829083333)
--(axis cs:336000,250.113496683333)
--(axis cs:312000,199.9851376)
--(axis cs:288000,133.6296575)
--(axis cs:264000,108.009550366667)
--(axis cs:240000,111.347915005417)
--(axis cs:216000,66.5665400333333)
--(axis cs:192000,57.3204184666667)
--(axis cs:168000,39.93931665)
--(axis cs:144000,2.24199943333333)
--(axis cs:120000,2.89289409916667)
--(axis cs:96000,1.96905846666667)
--(axis cs:72000,1.4871187)
--(axis cs:48000,1.89183456666667)
--(axis cs:24000,1.33993243333333)
--(axis cs:1,1.206611)
--cycle;

\addplot [line width=\linewidthother, C2, mark=*, mark size=0, mark options={solid}]
table {%
1 0.948998766666667
24000 1.0904513
48000 1.5111967
72000 1.21671865
96000 1.71067463333333
120000 1.64213188333333
144000 1.87192435
168000 1.9981613
192000 2.55025703333333
216000 2.4710261
240000 2.40687351666667
264000 3.04548316666667
288000 3.11048063333333
312000 45.7741226333333
336000 86.8169447666667
360000 133.706172033333
384000 167.440022566667
408000 223.479682683333
432000 271.612865966667
456000 299.344489833333
480000 316.125139
504000 343.2676819
528000 362.895733166667
552000 371.129335266667
576000 403.928083333333
600000 433.682061666667
624000 447.416261666667
648000 459.888256666667
672000 481.182795
696000 484.723381666667
720000 492.360653333333
744000 507.109406666667
768000 512.461105
792000 528.92938
816000 532.223985
840000 539.78343
864000 553.663991666667
888000 564.230171666667
912000 573.553388333333
936000 576.107041666667
960000 583.989765
984000 588.975246666667
};

% QSM
\path [draw=C3, fill=C3, opacity=0.2]
(axis cs:10000,0.980437511205673)
--(axis cs:10000,0.571824709574381)
--(axis cs:30000,0.92454714452227)
--(axis cs:50000,0.625464994000504)
--(axis cs:70000,1.00934768124183)
--(axis cs:90000,1.23904738937187)
--(axis cs:110000,1.65917317456964)
--(axis cs:130000,1.43436625629442)
--(axis cs:150000,1.57254073895653)
--(axis cs:170000,1.00512615224909)
--(axis cs:190000,1.49902140908345)
--(axis cs:210000,1.27124863482568)
--(axis cs:230000,1.88724621471898)
--(axis cs:250000,1.20267154889078)
--(axis cs:270000,1.75658200170939)
--(axis cs:290000,1.83277393089646)
--(axis cs:310000,1.58038452016657)
--(axis cs:330000,1.76445524495616)
--(axis cs:350000,0.763188306420122)
--(axis cs:370000,1.11490835495629)
--(axis cs:390000,1.51385273748736)
--(axis cs:410000,1.5282044495039)
--(axis cs:430000,2.08595943205102)
--(axis cs:450000,0.914354475039343)
--(axis cs:470000,2.76064391660987)
--(axis cs:490000,1.79614915916988)
--(axis cs:510000,3.837047256244)
--(axis cs:530000,14.2205738283196)
--(axis cs:550000,20.6594053212873)
--(axis cs:570000,32.1226278921192)
--(axis cs:590000,45.2927874081871)
--(axis cs:610000,57.847562998434)
--(axis cs:630000,48.4510298987203)
--(axis cs:650000,75.1096651056319)
--(axis cs:670000,85.0774934328689)
--(axis cs:690000,70.9237061284216)
--(axis cs:710000,100.56178009231)
--(axis cs:730000,98.1477566058643)
--(axis cs:750000,107.976290410489)
--(axis cs:770000,71.4434903923603)
--(axis cs:790000,108.648716585975)
--(axis cs:810000,112.772206131411)
--(axis cs:830000,117.390613196757)
--(axis cs:850000,119.573476216849)
--(axis cs:870000,117.649741076782)
--(axis cs:890000,151.226404877646)
--(axis cs:910000,180.827611920607)
--(axis cs:930000,192.221330821407)
--(axis cs:950000,173.157631803717)
--(axis cs:970000,181.093438661726)
--(axis cs:990000,201.570374228417)
--(axis cs:990000,434.345799500106)
--(axis cs:990000,434.345799500106)
--(axis cs:970000,412.304685786318)
--(axis cs:950000,408.239204986532)
--(axis cs:930000,415.911291103618)
--(axis cs:910000,412.121660608047)
--(axis cs:890000,403.132812268095)
--(axis cs:870000,402.902415146855)
--(axis cs:850000,392.621927461997)
--(axis cs:830000,375.552186746428)
--(axis cs:810000,372.439485307654)
--(axis cs:790000,365.474082116867)
--(axis cs:770000,367.539508720008)
--(axis cs:750000,366.497041417092)
--(axis cs:730000,360.422708050593)
--(axis cs:710000,339.61880406009)
--(axis cs:690000,320.227105725585)
--(axis cs:670000,318.378592133995)
--(axis cs:650000,301.232691969097)
--(axis cs:630000,294.865961761211)
--(axis cs:610000,301.036372906753)
--(axis cs:590000,286.428439508741)
--(axis cs:570000,272.598367652664)
--(axis cs:550000,252.196429150298)
--(axis cs:530000,241.798912390359)
--(axis cs:510000,237.836566396949)
--(axis cs:490000,220.883804745637)
--(axis cs:470000,189.833164870253)
--(axis cs:450000,173.378936859817)
--(axis cs:430000,141.960067212904)
--(axis cs:410000,130.491169867304)
--(axis cs:390000,138.113353883006)
--(axis cs:370000,114.373515658883)
--(axis cs:350000,98.6734778318873)
--(axis cs:330000,81.1817100092664)
--(axis cs:310000,55.1655348985092)
--(axis cs:290000,41.7045090677592)
--(axis cs:270000,29.9551248199034)
--(axis cs:250000,24.5490494274712)
--(axis cs:230000,12.0197521944106)
--(axis cs:210000,1.87259110076196)
--(axis cs:190000,2.84060800072243)
--(axis cs:170000,1.31785599440839)
--(axis cs:150000,1.80858941291542)
--(axis cs:130000,1.71737940728772)
--(axis cs:110000,2.1906831626305)
--(axis cs:90000,1.57328910677558)
--(axis cs:70000,1.52091146669651)
--(axis cs:50000,0.793696248523581)
--(axis cs:30000,1.1464251720657)
--(axis cs:10000,0.980437511205673)
--cycle;

\addplot [line width=\linewidthother, C3, mark=*, mark size=0, mark options={solid}]
table {%
10000 0.78541495402654
30000 1.04710988762478
50000 0.70152582313555
70000 1.20712324051419
90000 1.34805506153498
110000 1.99846312581455
130000 1.61551599417059
150000 1.68128851971066
170000 1.14248661406308
190000 1.99301915924415
210000 1.49244949220567
230000 2.1092782732536
250000 1.33332079474955
270000 2.16953036495838
290000 2.14913967521944
310000 7.82967944883128
330000 13.5668198581702
350000 19.1541869631948
370000 24.0859763200088
390000 31.1562172488595
410000 29.1401031316858
430000 24.2281260720111
450000 56.6642846325007
470000 77.5072600101724
490000 97.8809422932093
510000 112.49381723707
530000 115.705674211361
550000 135.126346026902
570000 153.998785736606
590000 169.508081257951
610000 188.714317894774
630000 178.842153230762
650000 207.103767327829
670000 234.943982314579
690000 215.478880368549
710000 252.25729806296
730000 269.835779044764
750000 276.77929133518
770000 242.501664952297
790000 278.567584191335
810000 284.167561173182
830000 291.989229913184
850000 308.325568595436
870000 311.007049247529
890000 334.631471865238
910000 352.716358719483
930000 361.324490045065
950000 335.533451670049
970000 354.618315367311
990000 383.55118528714
};

% BRO Fast
\path [draw=C9, fill=C9, opacity=0.2]
(axis cs:25000,0.978264646734394)
--(axis cs:25000,0.72091368704862)
--(axis cs:50000,1.05491793774696)
--(axis cs:75000,1.08596953236573)
--(axis cs:100000,1.52037023734103)
--(axis cs:125000,1.46266709484178)
--(axis cs:150000,1.59295943050481)
--(axis cs:175000,2.44387565938964)
--(axis cs:200000,4.82734418717242)
--(axis cs:225000,45.7521036599961)
--(axis cs:250000,96.2971500491572)
--(axis cs:275000,127.313312179649)
--(axis cs:300000,172.23164217402)
--(axis cs:325000,244.310475510347)
--(axis cs:350000,272.237929528696)
--(axis cs:375000,284.616125600504)
--(axis cs:400000,336.03046131455)
--(axis cs:425000,350.108012924306)
--(axis cs:450000,367.083312777869)
--(axis cs:475000,401.547543812368)
--(axis cs:500000,398.723466087545)
--(axis cs:525000,422.326711336825)
--(axis cs:550000,448.549740780043)
--(axis cs:575000,451.405956188766)
--(axis cs:600000,472.03152501358)
--(axis cs:625000,487.398821163315)
--(axis cs:650000,499.263640163204)
--(axis cs:675000,500.387984799848)
--(axis cs:700000,513.390989496869)
--(axis cs:725000,522.77598391949)
--(axis cs:750000,532.218788062258)
--(axis cs:775000,551.795361839865)
--(axis cs:800000,529.121552841192)
--(axis cs:825000,539.275670803018)
--(axis cs:850000,557.803427281245)
--(axis cs:875000,561.377003204859)
--(axis cs:900000,579.531662132037)
--(axis cs:925000,590.444251533223)
--(axis cs:950000,600.168925522562)
--(axis cs:975000,608.268283210647)
--(axis cs:1000000,614.519616978186)
--(axis cs:1000000,735.823169335893)
--(axis cs:1000000,735.823169335893)
--(axis cs:975000,711.865959383783)
--(axis cs:950000,695.27560918257)
--(axis cs:925000,657.430296749369)
--(axis cs:900000,649.319698106356)
--(axis cs:875000,652.007601523225)
--(axis cs:850000,632.863934130143)
--(axis cs:825000,625.987409169136)
--(axis cs:800000,595.349320276297)
--(axis cs:775000,604.345836930124)
--(axis cs:750000,595.350607247404)
--(axis cs:725000,598.175364336702)
--(axis cs:700000,571.672142749138)
--(axis cs:675000,556.95226489388)
--(axis cs:650000,554.676411493885)
--(axis cs:625000,536.759004981909)
--(axis cs:600000,536.534221044197)
--(axis cs:575000,529.729470656084)
--(axis cs:550000,511.906516729537)
--(axis cs:525000,497.767470130147)
--(axis cs:500000,487.382653012983)
--(axis cs:475000,479.891298872866)
--(axis cs:450000,456.51799212362)
--(axis cs:425000,450.961141558113)
--(axis cs:400000,438.468650582956)
--(axis cs:375000,425.214390055924)
--(axis cs:350000,371.188893943217)
--(axis cs:325000,386.760074697577)
--(axis cs:300000,350.614029814281)
--(axis cs:275000,314.612320092051)
--(axis cs:250000,287.625116541546)
--(axis cs:225000,237.871501428503)
--(axis cs:200000,161.210776021734)
--(axis cs:175000,33.7108401944357)
--(axis cs:150000,20.2488057753585)
--(axis cs:125000,2.17461285395191)
--(axis cs:100000,1.95421698893483)
--(axis cs:75000,1.61999913829626)
--(axis cs:50000,1.56444397965854)
--(axis cs:25000,0.978264646734394)
--cycle;

\addplot [line width=\linewidthother, C9, mark=*, mark size=0, mark options={solid}]
table {%
25000 0.850087344779465
50000 1.37241495819768
75000 1.28900845053116
100000 1.68666534684632
125000 1.75953726445738
150000 1.92706033341065
175000 6.69589452569375
200000 67.8886300575107
225000 133.781191226929
250000 197.675911176016
275000 226.385356264941
300000 269.306621226925
325000 327.306227956068
350000 333.156068045569
375000 367.799556292709
400000 391.078967120728
425000 401.16507670247
450000 405.144005457169
475000 435.548514113535
500000 442.053473077224
525000 459.619843861156
550000 479.287580745637
575000 492.585092939003
600000 504.366983727575
625000 511.62847478279
650000 531.479780052197
675000 528.889923381696
700000 543.902636318658
725000 556.181164812954
750000 563.90720323911
775000 578.816251810389
800000 572.794295668585
825000 569.156250786908
850000 595.635812152648
875000 604.61703212743
900000 608.974333826472
925000 617.406126843253
950000 636.596095382078
975000 648.334454543
1000000 664.291363975127
};
% DIME
\path [draw=C0, fill=C0, opacity=0.2]
(axis cs:1,1.31013312833333)
--(axis cs:1,0.9095194)
--(axis cs:24000,0.8378863)
--(axis cs:48000,1.26244795875)
--(axis cs:72000,1.44710566666667)
--(axis cs:96000,1.46950423333333)
--(axis cs:120000,52.7309068)
--(axis cs:144000,180.858245)
--(axis cs:168000,222.340223333333)
--(axis cs:192000,282.443328333333)
--(axis cs:216000,323.65245)
--(axis cs:240000,342.287747875)
--(axis cs:264000,369.187963333333)
--(axis cs:288000,333.333076666667)
--(axis cs:312000,405.802426666667)
--(axis cs:336000,428.907459333333)
--(axis cs:360000,425.048635)
--(axis cs:384000,449.31192)
--(axis cs:408000,453.621951666667)
--(axis cs:432000,461.93101)
--(axis cs:456000,475.758823333333)
--(axis cs:480000,479.082393333333)
--(axis cs:504000,498.892418333333)
--(axis cs:528000,500.805258333333)
--(axis cs:552000,494.37699)
--(axis cs:576000,511.273213333333)
--(axis cs:600000,523.740333333333)
--(axis cs:624000,509.624216666667)
--(axis cs:648000,546.020103333333)
--(axis cs:672000,539.069676666667)
--(axis cs:696000,536.58489)
--(axis cs:720000,553.626156666667)
--(axis cs:744000,568.595919541667)
--(axis cs:768000,583.373420083333)
--(axis cs:792000,565.642546666667)
--(axis cs:816000,597.357453333333)
--(axis cs:840000,594.23215)
--(axis cs:864000,611.45051)
--(axis cs:888000,614.696883333333)
--(axis cs:912000,624.643156666667)
--(axis cs:936000,618.072933333333)
--(axis cs:960000,620.873685)
--(axis cs:984000,636.899008333333)
--(axis cs:984000,838.46369)
--(axis cs:984000,838.46369)
--(axis cs:960000,810.906591666667)
--(axis cs:936000,795.332943333333)
--(axis cs:912000,789.9127)
--(axis cs:888000,777.729711666667)
--(axis cs:864000,782.374836416667)
--(axis cs:840000,763.31315)
--(axis cs:816000,752.925656666667)
--(axis cs:792000,729.92765)
--(axis cs:768000,729.483733333333)
--(axis cs:744000,714.319243333333)
--(axis cs:720000,705.35754)
--(axis cs:696000,710.037980708334)
--(axis cs:672000,682.733688333333)
--(axis cs:648000,654.451601666667)
--(axis cs:624000,673.877893333333)
--(axis cs:600000,648.679313333333)
--(axis cs:576000,617.81945)
--(axis cs:552000,618.793635)
--(axis cs:528000,589.75454625)
--(axis cs:504000,585.10795)
--(axis cs:480000,590.632341666667)
--(axis cs:456000,560.512006666667)
--(axis cs:432000,539.051216666667)
--(axis cs:408000,520.076401666667)
--(axis cs:384000,512.448843333333)
--(axis cs:360000,494.60856)
--(axis cs:336000,477.199193333333)
--(axis cs:312000,459.954495)
--(axis cs:288000,448.778828333333)
--(axis cs:264000,437.029621666667)
--(axis cs:240000,410.87809)
--(axis cs:216000,404.760741666667)
--(axis cs:192000,364.241176666667)
--(axis cs:168000,333.906663333333)
--(axis cs:144000,314.15384)
--(axis cs:120000,259.780796666667)
--(axis cs:96000,125.2046745)
--(axis cs:72000,6.35896226666667)
--(axis cs:48000,2.10821175)
--(axis cs:24000,1.34840183333333)
--(axis cs:1,1.31013312833333)
--cycle;

\addplot [line width=\linewidthdime, C0, mark=*, mark size=0, mark options={solid}]
table {%
1 1.1002536
24000 1.08066533333333
48000 1.65811203333333
72000 1.96847508333333
96000 33.465209
120000 165.904953766667
144000 257.245133333333
168000 298.604093333333
192000 324.22148
216000 357.115678333333
240000 378.556383333333
264000 399.992011666667
288000 409.36517
312000 433.224305
336000 451.67759
360000 461.376753333333
384000 481.441843333333
408000 485.030593333333
432000 494.752293333333
456000 503.985166666667
480000 513.270438333333
504000 527.557041666667
528000 532.037596666667
552000 530.372225
576000 541.368468333333
600000 550.344083333333
624000 541.72046
648000 573.300886666667
672000 576.685181666667
696000 582.574813333333
720000 578.890466666667
744000 611.19024
768000 611.031585
792000 600.50536
816000 653.614403333333
840000 657.22555
864000 679.262618333333
888000 679.668845
912000 699.073006666667
936000 683.055043333333
960000 703.871611666667
984000 736.242175
};
\end{axis}

\end{tikzpicture}
}
       \subcaption[]{Humanoid Walk}
       \label{fig::exps_new::humanoid_walk}
    \end{minipage}\hfill
    \begin{minipage}[b]{0.25\textwidth}
        \centering
       \resizebox{1\textwidth}{!}{% This file was created with tikzplotlib v0.10.1.
\begin{tikzpicture}

\definecolor{darkcyan1115178}{RGB}{1,115,178}
\definecolor{darkgray176}{RGB}{176,176,176}

\begin{axis}[
legend cell align={left},
legend style={fill opacity=0.8, draw opacity=1, text opacity=1, draw=lightgray204, at={(0.03,0.03)},  anchor=north west},
tick align=outside,
tick pos=left,
x grid style={white},
xlabel={Number Env Interactions},
xmajorgrids,
xmin=-0.0, xmax=1000000.0,
xtick style={color=black},
y grid style={white},
ylabel={IQM Mean Return},
ymajorgrids,
ymin=-0.05, ymax=1000,
ytick style={color=black},
axis background/.style={fill=plot_background},
label style={font=\large},
tick label style={font=\large},
x axis line style={draw=none},
y axis line style={draw=none},
]

%Diff-QL
\path [draw=C4, fill=C4, opacity=0.2]
(axis cs:10000,4.96192337097618)
--(axis cs:10000,3.8259334250842)
--(axis cs:50000,3.94075953257382)
--(axis cs:90000,5.4062363228855)
--(axis cs:130000,5.03033449267779)
--(axis cs:170000,5.40545143063867)
--(axis cs:210000,5.30027438708067)
--(axis cs:250000,5.20143634778874)
--(axis cs:290000,5.55753265722435)
--(axis cs:330000,6.0853681989464)
--(axis cs:370000,6.02444619099523)
--(axis cs:410000,6.015013126224)
--(axis cs:450000,5.47863569419768)
--(axis cs:490000,5.66204547476913)
--(axis cs:530000,6.26274740925794)
--(axis cs:570000,5.91255339165955)
--(axis cs:610000,5.768975319171)
--(axis cs:650000,6.14131492890606)
--(axis cs:690000,5.72649506458288)
--(axis cs:730000,5.59763737988894)
--(axis cs:770000,5.46977822085792)
--(axis cs:810000,5.68756101636153)
--(axis cs:850000,6.43751144120119)
--(axis cs:890000,6.22049825988976)
--(axis cs:930000,5.66373501904046)
--(axis cs:970000,6.23997404151309)
--(axis cs:970000,7.24628461022597)
--(axis cs:970000,7.24628461022597)
--(axis cs:930000,7.09187673690713)
--(axis cs:890000,7.11150897825985)
--(axis cs:850000,7.35766596147297)
--(axis cs:810000,6.94162874164136)
--(axis cs:770000,7.42103530425838)
--(axis cs:730000,6.48023505242246)
--(axis cs:690000,6.4963495382542)
--(axis cs:650000,7.11292900482563)
--(axis cs:610000,6.72665035700891)
--(axis cs:570000,6.86088374909409)
--(axis cs:530000,7.0508111920467)
--(axis cs:490000,6.63964308795522)
--(axis cs:450000,6.54542429789018)
--(axis cs:410000,6.8550325697396)
--(axis cs:370000,6.52296981489598)
--(axis cs:330000,6.85310153371002)
--(axis cs:290000,6.31301873846887)
--(axis cs:250000,6.29854961139209)
--(axis cs:210000,6.43844611354466)
--(axis cs:170000,6.4143219795259)
--(axis cs:130000,5.82980173955923)
--(axis cs:90000,6.41187453417774)
--(axis cs:50000,5.18384166368115)
--(axis cs:10000,4.96192337097618)
--cycle;

\addplot [line width=\linewidthother, C4, mark=*, mark size=0, mark options={solid}]
table {%
10000 4.35175720515098
50000 4.5555647311236
90000 5.86526526833143
130000 5.42049514601954
170000 5.93172969433109
210000 5.88202734211672
250000 5.74720881629061
290000 5.94468227702321
330000 6.45809706730742
370000 6.27185339349583
410000 6.47991487421709
450000 6.04686455645901
490000 6.13858436728607
530000 6.66999417828992
570000 6.35627156079248
610000 6.27486046814547
650000 6.63053555757709
690000 6.10183928152291
730000 6.04947634611781
770000 6.41925766649601
810000 6.322343575543
850000 6.86201255518257
890000 6.65668789567967
930000 6.41561435964
970000 6.71706132670301
};

%consistency-AC
\path [draw=C5, fill=C5, opacity=0.2]
(axis cs:10000,4.8297494634377)
--(axis cs:10000,4.30320430581047)
--(axis cs:50000,4.29220149426134)
--(axis cs:90000,4.07170278242615)
--(axis cs:130000,3.95575421822244)
--(axis cs:170000,5.24266151644775)
--(axis cs:210000,5.03869072756926)
--(axis cs:250000,4.69962337156316)
--(axis cs:290000,5.25560721336778)
--(axis cs:330000,4.80375125144969)
--(axis cs:370000,4.68673394347318)
--(axis cs:410000,5.1560971623086)
--(axis cs:450000,4.82397139420593)
--(axis cs:490000,3.77845082411751)
--(axis cs:530000,4.13717698074318)
--(axis cs:570000,3.61462127155641)
--(axis cs:610000,4.58644754349057)
--(axis cs:650000,5.36829984217047)
--(axis cs:690000,4.51531071895644)
--(axis cs:730000,5.23635805733061)
--(axis cs:770000,5.89504368947325)
--(axis cs:810000,4.55594922938491)
--(axis cs:850000,5.37451890342538)
--(axis cs:890000,6.02303599999317)
--(axis cs:930000,5.13400433408528)
--(axis cs:970000,5.92334774661684)
--(axis cs:970000,7.3958995628676)
--(axis cs:970000,7.3958995628676)
--(axis cs:930000,7.29324771822252)
--(axis cs:890000,7.49822503680304)
--(axis cs:850000,6.90933414430978)
--(axis cs:810000,7.11533293993633)
--(axis cs:770000,6.78634775522894)
--(axis cs:730000,6.48728170023077)
--(axis cs:690000,6.79673037110153)
--(axis cs:650000,6.31693092526725)
--(axis cs:610000,6.39704955218433)
--(axis cs:570000,5.89898614666445)
--(axis cs:530000,6.92197529726165)
--(axis cs:490000,6.30997365590882)
--(axis cs:450000,6.42197851999428)
--(axis cs:410000,7.32097167443668)
--(axis cs:370000,6.62597543958417)
--(axis cs:330000,7.85862764808482)
--(axis cs:290000,6.06366093186793)
--(axis cs:250000,5.93761370686074)
--(axis cs:210000,6.46925623595991)
--(axis cs:170000,6.73484605259018)
--(axis cs:130000,6.30283744948897)
--(axis cs:90000,7.01552199161076)
--(axis cs:50000,4.96446346229076)
--(axis cs:10000,4.8297494634377)
--cycle;

\addplot [line width=\linewidthother, C5, mark=*, mark size=0, mark options={solid}]
table {%
10000 4.56647688462408
50000 4.62363275267316
90000 5.38245868911971
130000 5.10876724007399
170000 5.99035632573101
210000 5.86144084123921
250000 5.2591627052729
290000 5.63911263432431
330000 5.95824196791048
370000 5.65635469152868
410000 6.33992671665792
450000 5.53400354659528
490000 5.14945321735389
530000 5.52957613900241
570000 4.817177746175
610000 5.67350231161962
650000 5.86716362314559
690000 5.69799628759027
730000 5.81708815308413
770000 6.35482500727505
810000 5.83564108466062
850000 6.21542826700275
890000 6.77812610131495
930000 6.18133414842461
970000 6.74737769499053
};

% BRO
\path [draw=C1, fill=C1, opacity=0.2]
(axis cs:25000,5.84158825436646)
--(axis cs:25000,4.3344481467084)
--(axis cs:50000,4.76001762284864)
--(axis cs:75000,5.98334415442094)
--(axis cs:100000,39.6190792419217)
--(axis cs:125000,105.905093855071)
--(axis cs:150000,140.101295452493)
--(axis cs:175000,248.671631530087)
--(axis cs:200000,316.403771299631)
--(axis cs:225000,388.724003882934)
--(axis cs:250000,389.984829730665)
--(axis cs:275000,418.790203446846)
--(axis cs:300000,504.584103017477)
--(axis cs:325000,634.493042086236)
--(axis cs:350000,651.60672413475)
--(axis cs:375000,662.878208246955)
--(axis cs:400000,690.306958764118)
--(axis cs:425000,760.471978544107)
--(axis cs:450000,798.434222706599)
--(axis cs:475000,853.170508967177)
--(axis cs:500000,881.998256683998)
--(axis cs:525000,581.353211166842)
--(axis cs:550000,724.400256263198)
--(axis cs:575000,796.44723364736)
--(axis cs:600000,827.565137465164)
--(axis cs:625000,893.258944326342)
--(axis cs:650000,840.075583545516)
--(axis cs:675000,910.760088358108)
--(axis cs:700000,924.140523754951)
--(axis cs:725000,916.24118995868)
--(axis cs:750000,908.784019703424)
--(axis cs:775000,592.759795367581)
--(axis cs:800000,747.848587557846)
--(axis cs:825000,860.599313747778)
--(axis cs:850000,910.902007325625)
--(axis cs:875000,905.132334799133)
--(axis cs:900000,928.277133534983)
--(axis cs:925000,922.608055189844)
--(axis cs:950000,922.110443482492)
--(axis cs:975000,923.756763395021)
--(axis cs:1000000,921.503586730828)
--(axis cs:1000000,951.166113635823)
--(axis cs:1000000,951.166113635823)
--(axis cs:975000,948.419878942758)
--(axis cs:950000,946.657159751399)
--(axis cs:925000,947.665453018909)
--(axis cs:900000,948.197317491826)
--(axis cs:875000,932.382172326518)
--(axis cs:850000,934.714828325199)
--(axis cs:825000,911.119923518179)
--(axis cs:800000,884.309886258941)
--(axis cs:775000,753.588137992877)
--(axis cs:750000,941.26435620658)
--(axis cs:725000,941.379404617989)
--(axis cs:700000,939.804800004331)
--(axis cs:675000,931.705985231338)
--(axis cs:650000,917.252118415197)
--(axis cs:625000,936.140219502633)
--(axis cs:600000,906.878001714666)
--(axis cs:575000,909.320391362909)
--(axis cs:550000,811.543001437235)
--(axis cs:525000,718.744276627832)
--(axis cs:500000,922.350793367806)
--(axis cs:475000,926.259974211296)
--(axis cs:450000,900.9903723611)
--(axis cs:425000,908.458881584684)
--(axis cs:400000,887.196493423702)
--(axis cs:375000,902.373102485596)
--(axis cs:350000,880.491833867954)
--(axis cs:325000,844.789769572495)
--(axis cs:300000,745.476222744396)
--(axis cs:275000,645.552252283843)
--(axis cs:250000,785.550676720431)
--(axis cs:225000,740.69045777041)
--(axis cs:200000,598.872506186218)
--(axis cs:175000,486.011330214894)
--(axis cs:150000,431.489650690268)
--(axis cs:125000,343.338474058675)
--(axis cs:100000,207.519926093904)
--(axis cs:75000,34.0484860721215)
--(axis cs:50000,10.3597780627621)
--(axis cs:25000,5.84158825436646)
--cycle;

\addplot [line width=\linewidthother, C1, mark=*, mark size=0, mark options={solid}]
table {%
25000 4.90982355385797
50000 6.37398084193334
75000 7.12770545026627
100000 119.644498440662
125000 235.379163283062
150000 313.06619198645
175000 399.142264541751
200000 456.198831387971
225000 569.913589562338
250000 582.285139085573
275000 558.386383581462
300000 640.154805381672
325000 768.876206876439
350000 782.491089337094
375000 818.335633332461
400000 822.157951680945
425000 856.766270667425
450000 878.257691590949
475000 898.182443909436
500000 909.591918646354
525000 635.769662397217
550000 770.486232954277
575000 870.692931110874
600000 872.944882689487
625000 925.766200488531
650000 880.307214687013
675000 923.106786486696
700000 932.166760412768
725000 928.082379045331
750000 928.021705692495
775000 667.453900921648
800000 825.299125421331
825000 886.338957833471
850000 927.31806444645
875000 919.922938299009
900000 941.580648863037
925000 939.977615666552
950000 934.181726433246
975000 936.928560873933
1000000 939.396947595177
};

%DIPO
\path [draw=C6, fill=C6, opacity=0.2]
(axis cs:10000,5.47130187352498)
--(axis cs:10000,3.58421317736308)
--(axis cs:50000,5.44904883702596)
--(axis cs:90000,5.70183388392131)
--(axis cs:130000,5.97973775863647)
--(axis cs:170000,3.94914166132609)
--(axis cs:210000,5.45604101816813)
--(axis cs:250000,4.13956085840861)
--(axis cs:290000,5.80351416269938)
--(axis cs:330000,5.74497191111247)
--(axis cs:370000,5.77276134490967)
--(axis cs:410000,5.79220294952393)
--(axis cs:450000,4.20266135533651)
--(axis cs:490000,5.59914716084798)
--(axis cs:530000,4.45229895909627)
--(axis cs:570000,5.59367704391479)
--(axis cs:610000,4.97433439890544)
--(axis cs:650000,6.41463263829549)
--(axis cs:690000,5.64842987060547)
--(axis cs:730000,5.32193978627523)
--(axis cs:770000,5.6864964167277)
--(axis cs:810000,5.61175139745076)
--(axis cs:850000,5.51818911234538)
--(axis cs:890000,6.67542807261149)
--(axis cs:930000,6.45228735605876)
--(axis cs:970000,5.84438180923462)
--(axis cs:970000,8.18512487411499)
--(axis cs:970000,8.18512487411499)
--(axis cs:930000,9.65129439036051)
--(axis cs:890000,7.60076411565145)
--(axis cs:850000,9.20010153452555)
--(axis cs:810000,7.90201807022095)
--(axis cs:770000,8.19028679529826)
--(axis cs:730000,9.11347579956055)
--(axis cs:690000,8.17694489161173)
--(axis cs:650000,9.32446257273356)
--(axis cs:610000,7.2840092976888)
--(axis cs:570000,8.44864495595296)
--(axis cs:530000,8.03118817011515)
--(axis cs:490000,7.75904003779093)
--(axis cs:450000,6.12641572952271)
--(axis cs:410000,8.39852237701416)
--(axis cs:370000,7.96210209528605)
--(axis cs:330000,8.43978532155355)
--(axis cs:290000,7.92700227101644)
--(axis cs:250000,5.92736752827962)
--(axis cs:210000,8.44274743398031)
--(axis cs:170000,6.78038597106934)
--(axis cs:130000,8.08517932891846)
--(axis cs:90000,7.08941618601481)
--(axis cs:50000,7.37409941355387)
--(axis cs:10000,5.47130187352498)
--cycle;

\addplot [line width=\linewidthother, C6, mark=*, mark size=0, mark options={solid}]
table {%
10000 4.84548036257426
50000 6.84691508611043
90000 6.33715867996216
130000 6.63398472468058
170000 5.89773352940877
210000 6.20029258728027
250000 5.14127540588379
290000 6.68097686767578
330000 6.59276660283407
370000 6.99519443511963
410000 6.91162141164144
450000 5.84280109405518
490000 6.6694294611613
530000 6.88347323735555
570000 6.51250727971395
610000 5.69469102223714
650000 7.60290622711182
690000 7.12087901433309
730000 6.96562115351359
770000 7.22487052281698
810000 6.66203721364339
850000 7.74426825841268
890000 7.25314935048421
930000 8.03951040903727
970000 6.19393221537272
};
% QSM
\path [draw=C3, fill=C3, opacity=0.2]
(axis cs:10000,4.33900420914093)
--(axis cs:10000,2.4651780128479)
--(axis cs:30000,3.44652996460597)
--(axis cs:50000,2.19634002074599)
--(axis cs:70000,3.6241423535413)
--(axis cs:90000,4.12928045893932)
--(axis cs:110000,6.21344227925098)
--(axis cs:130000,5.7375070829997)
--(axis cs:150000,6.10220146887271)
--(axis cs:170000,4.03109148493348)
--(axis cs:190000,5.63116646629657)
--(axis cs:210000,5.03080810344201)
--(axis cs:230000,7.30965195825745)
--(axis cs:250000,5.39512460759291)
--(axis cs:270000,8.09623451649549)
--(axis cs:290000,7.41535438327355)
--(axis cs:310000,5.9420059111613)
--(axis cs:330000,5.65732046272927)
--(axis cs:350000,3.60900096810887)
--(axis cs:370000,5.1556017666122)
--(axis cs:390000,5.84512793671208)
--(axis cs:410000,5.65589503949148)
--(axis cs:430000,8.18752960560256)
--(axis cs:450000,3.73139691536834)
--(axis cs:470000,8.15703477688892)
--(axis cs:490000,6.41390640117852)
--(axis cs:510000,10.4706356873259)
--(axis cs:530000,18.1216368363601)
--(axis cs:550000,23.9657357201565)
--(axis cs:570000,41.7001749591427)
--(axis cs:590000,57.8857786638809)
--(axis cs:610000,85.9661722439076)
--(axis cs:630000,125.87630590994)
--(axis cs:650000,174.351536772827)
--(axis cs:670000,236.601769282146)
--(axis cs:690000,267.725848172279)
--(axis cs:710000,285.533047275077)
--(axis cs:730000,299.881790706446)
--(axis cs:750000,303.793753785679)
--(axis cs:770000,334.658388396848)
--(axis cs:790000,351.114713719624)
--(axis cs:810000,353.544961181419)
--(axis cs:830000,380.505466260056)
--(axis cs:850000,358.614047811264)
--(axis cs:870000,360.851920704261)
--(axis cs:890000,413.897306355566)
--(axis cs:910000,368.546383454347)
--(axis cs:930000,398.186258507124)
--(axis cs:950000,430.240465655413)
--(axis cs:970000,437.395962122501)
--(axis cs:990000,464.920044075976)
--(axis cs:990000,546.439484579403)
--(axis cs:990000,546.439484579403)
--(axis cs:970000,535.844942865081)
--(axis cs:950000,535.591368755336)
--(axis cs:930000,535.586203994868)
--(axis cs:910000,523.231547811827)
--(axis cs:890000,519.042129738113)
--(axis cs:870000,519.593485093143)
--(axis cs:850000,505.950726426346)
--(axis cs:830000,508.81464768672)
--(axis cs:810000,500.315200285386)
--(axis cs:790000,495.500561858691)
--(axis cs:770000,471.272235642516)
--(axis cs:750000,459.688356943647)
--(axis cs:730000,448.559875610553)
--(axis cs:710000,459.547178822489)
--(axis cs:690000,440.968945167165)
--(axis cs:670000,436.531567697291)
--(axis cs:650000,419.752966895477)
--(axis cs:630000,401.592059095488)
--(axis cs:610000,387.83615018711)
--(axis cs:590000,356.176653366673)
--(axis cs:570000,355.978053229722)
--(axis cs:550000,348.932845804336)
--(axis cs:530000,299.410530064182)
--(axis cs:510000,274.744074343744)
--(axis cs:490000,285.700892003385)
--(axis cs:470000,250.994240310222)
--(axis cs:450000,203.3074887996)
--(axis cs:430000,186.878242324351)
--(axis cs:410000,140.475114370138)
--(axis cs:390000,124.722277730354)
--(axis cs:370000,115.765306625893)
--(axis cs:350000,69.0693899132733)
--(axis cs:330000,66.2683623407922)
--(axis cs:310000,53.3595680316086)
--(axis cs:290000,48.8367917301058)
--(axis cs:270000,39.498131500448)
--(axis cs:250000,24.4754031878084)
--(axis cs:230000,20.9795508608388)
--(axis cs:210000,8.31427962569849)
--(axis cs:190000,6.35382285663477)
--(axis cs:170000,4.86043204692802)
--(axis cs:150000,6.93310167389427)
--(axis cs:130000,6.51302606294727)
--(axis cs:110000,6.80922619413468)
--(axis cs:90000,4.86646908843541)
--(axis cs:70000,4.42218022463494)
--(axis cs:50000,2.67558720459541)
--(axis cs:30000,4.08750070134799)
--(axis cs:10000,4.33900420914093)
--cycle;

\addplot [line width=\linewidthother, C3, mark=*, mark size=0, mark options={solid}]
table {%
10000 3.05563910802205
30000 3.81038435796897
50000 2.43964932734768
70000 4.17688317418409
90000 4.46891485490293
110000 6.54636771773706
130000 6.09976935550731
150000 6.41879517324681
170000 4.4305923628893
190000 5.80786526129653
210000 5.67070735621502
230000 7.55548525521732
250000 5.80610744164093
270000 8.29889199360086
290000 7.80208794326863
310000 6.66885166789628
330000 6.40861692758394
350000 4.19467520396813
370000 6.23479789692432
390000 8.49190340916372
410000 21.1533482485751
430000 72.6322997421579
450000 84.7704177099094
470000 108.824778023437
490000 133.227020388059
510000 125.020685143749
530000 157.030963948638
550000 189.722506304607
570000 209.8802236
590000 214.074038810789
610000 245.309175039823
630000 264.068465243846
650000 304.044721129306
670000 353.220771829974
690000 383.319242487523
710000 404.453011356337
730000 405.362039765354
750000 414.010066876967
770000 440.267612747887
790000 440.184966882881
810000 450.689423817678
830000 469.757887824944
850000 445.784951556184
870000 456.746090896045
890000 477.245335613448
910000 458.191985147441
930000 479.795927449377
950000 494.019004153006
970000 491.803153188724
990000 517.844641802633
};

\path [draw=C2, fill=C2, opacity=0.2]
(axis cs:1,5.34036701666667)
--(axis cs:1,3.22286363333333)
--(axis cs:24000,4.1654257)
--(axis cs:48000,5.24666393333333)
--(axis cs:72000,4.64591936666667)
--(axis cs:96000,5.8350107)
--(axis cs:120000,6.3522342)
--(axis cs:144000,5.88593316666667)
--(axis cs:168000,6.01337735)
--(axis cs:192000,7.27568028333333)
--(axis cs:216000,7.19856731666667)
--(axis cs:240000,7.69419426666667)
--(axis cs:264000,13.4929694666667)
--(axis cs:288000,78.1680613333333)
--(axis cs:312000,124.493640216667)
--(axis cs:336000,185.988444416667)
--(axis cs:360000,214.275334666667)
--(axis cs:384000,231.607758616667)
--(axis cs:408000,235.024520833333)
--(axis cs:432000,269.153187666667)
--(axis cs:456000,320.152564358334)
--(axis cs:480000,357.046088233333)
--(axis cs:504000,367.435993333333)
--(axis cs:528000,400.51718)
--(axis cs:552000,421.663074333333)
--(axis cs:576000,432.090847866667)
--(axis cs:600000,432.92334175)
--(axis cs:624000,464.833185333333)
--(axis cs:648000,491.1138575)
--(axis cs:672000,425.578812666667)
--(axis cs:696000,510.1349085)
--(axis cs:720000,521.608220995833)
--(axis cs:744000,559.477275)
--(axis cs:768000,573.1572255)
--(axis cs:792000,565.714102958334)
--(axis cs:816000,618.447065895834)
--(axis cs:840000,590.483015333333)
--(axis cs:864000,608.632808)
--(axis cs:888000,629.009467916667)
--(axis cs:912000,619.0496795)
--(axis cs:936000,648.0718764)
--(axis cs:960000,645.974498875001)
--(axis cs:984000,646.250614666667)
--(axis cs:984000,878.391048916667)
--(axis cs:984000,878.391048916667)
--(axis cs:960000,890.72346)
--(axis cs:936000,891.092146666667)
--(axis cs:912000,876.57694)
--(axis cs:888000,878.219566666667)
--(axis cs:864000,876.144036666667)
--(axis cs:840000,876.768326666667)
--(axis cs:816000,874.968613333333)
--(axis cs:792000,830.389708333333)
--(axis cs:768000,846.01752)
--(axis cs:744000,846.476676666667)
--(axis cs:720000,834.53654)
--(axis cs:696000,788.476233333333)
--(axis cs:672000,761.413083333333)
--(axis cs:648000,771.340883333333)
--(axis cs:624000,782.53678)
--(axis cs:600000,758.207078333333)
--(axis cs:576000,715.569746666667)
--(axis cs:552000,714.75034)
--(axis cs:528000,696.417755)
--(axis cs:504000,685.861353333333)
--(axis cs:480000,659.178778916667)
--(axis cs:456000,633.292721666667)
--(axis cs:432000,595.028088333333)
--(axis cs:408000,547.672721666667)
--(axis cs:384000,515.22991)
--(axis cs:360000,461.556283333333)
--(axis cs:336000,427.74379)
--(axis cs:312000,392.35991)
--(axis cs:288000,364.820115)
--(axis cs:264000,280.828976666667)
--(axis cs:240000,204.868095666667)
--(axis cs:216000,138.324822166667)
--(axis cs:192000,41.05076677625)
--(axis cs:168000,7.27687691666667)
--(axis cs:144000,7.72020108333333)
--(axis cs:120000,8.40175771666667)
--(axis cs:96000,7.00733593333333)
--(axis cs:72000,6.358321)
--(axis cs:48000,7.24747713333333)
--(axis cs:24000,6.1477227)
--(axis cs:1,5.34036701666667)
--cycle;

\addplot [line width=\linewidthother, C2, mark=*, mark size=0, mark options={solid}]
table {%
1 4.24602501666667
24000 5.19598841666667
48000 6.2748922
72000 5.21161066666667
96000 6.477539
120000 7.33993245
144000 6.83759616666667
168000 6.5776056
192000 8.55243133333333
216000 35.69796735
240000 72.9202753833333
264000 137.3760429
288000 241.945704166667
312000 292.466316666667
336000 361.009081666667
360000 407.456396666667
384000 436.816575
408000 455.387958333333
432000 467.832458333333
456000 496.443083333333
480000 511.318741666667
504000 526.26603
528000 550.582276666667
552000 559.395671666667
576000 560.30239
600000 598.576001666667
624000 637.53392
648000 665.055016666667
672000 614.8467
696000 690.229475
720000 723.162151666667
744000 751.090085
768000 761.884265
792000 761.162175
816000 828.436123333333
840000 791.33198
864000 809.15274
888000 837.576188333333
912000 824.698546666667
936000 849.783773333333
960000 847.820711666667
984000 849.378658333333
};
%BRO Fast
\path [draw=C9, fill=C9, opacity=0.2]
(axis cs:25000,6.08549476917743)
--(axis cs:25000,4.25090418214268)
--(axis cs:50000,5.20454014985797)
--(axis cs:75000,4.50563451746712)
--(axis cs:100000,6.02004149282822)
--(axis cs:125000,6.3190765144823)
--(axis cs:150000,6.65708378302587)
--(axis cs:175000,6.22719694562258)
--(axis cs:200000,7.89697431723411)
--(axis cs:225000,9.76296042243521)
--(axis cs:250000,90.0106004781012)
--(axis cs:275000,147.712982025125)
--(axis cs:300000,216.224896213634)
--(axis cs:325000,263.942156998804)
--(axis cs:350000,329.101250721248)
--(axis cs:375000,381.623732245042)
--(axis cs:400000,439.207506540744)
--(axis cs:425000,484.527160689435)
--(axis cs:450000,486.945493427618)
--(axis cs:475000,535.749314025379)
--(axis cs:500000,568.425312335819)
--(axis cs:525000,617.485206462353)
--(axis cs:550000,637.837234620963)
--(axis cs:575000,693.979201984434)
--(axis cs:600000,741.167483774972)
--(axis cs:625000,732.400182619652)
--(axis cs:650000,756.175007538625)
--(axis cs:675000,761.751120314901)
--(axis cs:700000,815.040763426397)
--(axis cs:725000,786.761971479476)
--(axis cs:750000,813.35507565201)
--(axis cs:775000,836.368377165568)
--(axis cs:800000,821.764107656168)
--(axis cs:825000,882.345831464204)
--(axis cs:850000,849.079795737451)
--(axis cs:875000,848.589110092429)
--(axis cs:900000,887.661649601052)
--(axis cs:925000,881.04190644318)
--(axis cs:950000,895.221335171576)
--(axis cs:975000,889.207489797751)
--(axis cs:1000000,907.325022358299)
--(axis cs:1000000,933.243419166001)
--(axis cs:1000000,933.243419166001)
--(axis cs:975000,928.169126008049)
--(axis cs:950000,933.009412654366)
--(axis cs:925000,924.741627425314)
--(axis cs:900000,920.359603837485)
--(axis cs:875000,926.789222367815)
--(axis cs:850000,926.505013232024)
--(axis cs:825000,935.326274791148)
--(axis cs:800000,923.61582302103)
--(axis cs:775000,906.779691192902)
--(axis cs:750000,922.963397994345)
--(axis cs:725000,881.455383162539)
--(axis cs:700000,918.488238261959)
--(axis cs:675000,911.651531896846)
--(axis cs:650000,903.596929068154)
--(axis cs:625000,911.522549770243)
--(axis cs:600000,891.736207071669)
--(axis cs:575000,878.690284066257)
--(axis cs:550000,880.790890386534)
--(axis cs:525000,872.23321318146)
--(axis cs:500000,863.130232011722)
--(axis cs:475000,822.459065802988)
--(axis cs:450000,738.68126535474)
--(axis cs:425000,732.020081054958)
--(axis cs:400000,676.32974453466)
--(axis cs:375000,563.938936402977)
--(axis cs:350000,510.731755183779)
--(axis cs:325000,466.294336834721)
--(axis cs:300000,422.977260995497)
--(axis cs:275000,330.767007747037)
--(axis cs:250000,273.914568636811)
--(axis cs:225000,205.870488077758)
--(axis cs:200000,116.191780889412)
--(axis cs:175000,71.6899773148169)
--(axis cs:150000,42.0727384922957)
--(axis cs:125000,22.831687670724)
--(axis cs:100000,7.5857728511891)
--(axis cs:75000,6.93729767370663)
--(axis cs:50000,6.90923786553642)
--(axis cs:25000,6.08549476917743)
--cycle;

\addplot [line width = \linewidthother, C9, mark=*, mark size=0, mark options={solid}]
table {%
25000 5.10255336406997
50000 6.43815012986796
75000 5.72987711517771
100000 6.84922750011532
125000 6.87640727141581
150000 8.44507607130766
175000 16.2550447189992
200000 26.1494106991611
225000 78.7480113586598
250000 190.295401020248
275000 267.820241482549
300000 363.121941372008
325000 389.165381626744
350000 446.874190921746
375000 490.107392894032
400000 545.577636786693
425000 621.713033496988
450000 627.301939620001
475000 703.60107770886
500000 732.243735602308
525000 763.752039438378
550000 769.029207915871
575000 809.106676416275
600000 847.537717041167
625000 862.11096457076
650000 857.454907279811
675000 880.860247445292
700000 897.099272581282
725000 850.015625084922
750000 888.026536919742
775000 879.820331044594
800000 885.636971774688
825000 913.258041010638
850000 899.259765208553
875000 899.118671031981
900000 905.795253907484
925000 906.345601999117
950000 915.481845547343
975000 913.77265438713
1000000 921.882417374975
};

\path [draw=C0, fill=C0, opacity=0.2]
(axis cs:1,5.20993306666667)
--(axis cs:1,3.5936406)
--(axis cs:24000,4.2058623)
--(axis cs:48000,5.23006261666667)
--(axis cs:72000,5.36594268333333)
--(axis cs:96000,6.61328026666667)
--(axis cs:120000,9.0561208)
--(axis cs:144000,171.472306666667)
--(axis cs:168000,259.852013833333)
--(axis cs:192000,243.890558833333)
--(axis cs:216000,348.902713333333)
--(axis cs:240000,386.562788333333)
--(axis cs:264000,425.766295)
--(axis cs:288000,441.939373333333)
--(axis cs:312000,449.617635)
--(axis cs:336000,476.683296666667)
--(axis cs:360000,498.84464)
--(axis cs:384000,565.2414)
--(axis cs:408000,585.455676666667)
--(axis cs:432000,580.18089)
--(axis cs:456000,611.948053333333)
--(axis cs:480000,656.423524958333)
--(axis cs:504000,681.978448333333)
--(axis cs:528000,688.64665)
--(axis cs:552000,716.314023333333)
--(axis cs:576000,727.215515)
--(axis cs:600000,746.921466)
--(axis cs:624000,799.330013333333)
--(axis cs:648000,793.307166666667)
--(axis cs:672000,841.999945)
--(axis cs:696000,809.384298333333)
--(axis cs:720000,853.694216666667)
--(axis cs:744000,837.587866666667)
--(axis cs:768000,866.009655)
--(axis cs:792000,772.1309675)
--(axis cs:816000,854.23644)
--(axis cs:840000,869.971008333333)
--(axis cs:864000,867.156496666667)
--(axis cs:888000,850.691305)
--(axis cs:912000,885.945478333333)
--(axis cs:936000,889.674413333333)
--(axis cs:960000,868.547589583333)
--(axis cs:984000,899.386575)
--(axis cs:984000,923.87629)
--(axis cs:984000,923.87629)
--(axis cs:960000,921.686756666667)
--(axis cs:936000,922.945276666667)
--(axis cs:912000,919.00825)
--(axis cs:888000,909.520479041667)
--(axis cs:864000,921.016596666667)
--(axis cs:840000,926.94974)
--(axis cs:816000,913.60635)
--(axis cs:792000,911.854805)
--(axis cs:768000,910.662645)
--(axis cs:744000,914.027133333333)
--(axis cs:720000,913.651205)
--(axis cs:696000,904.445086666667)
--(axis cs:672000,908.56979)
--(axis cs:648000,907.186443375)
--(axis cs:624000,906.38215)
--(axis cs:600000,889.08579)
--(axis cs:576000,881.58292)
--(axis cs:552000,883.002653333333)
--(axis cs:528000,878.421755)
--(axis cs:504000,875.860518333333)
--(axis cs:480000,860.02895)
--(axis cs:456000,854.410473333333)
--(axis cs:432000,813.819246666667)
--(axis cs:408000,796.505116666667)
--(axis cs:384000,820.66071)
--(axis cs:360000,701.5242005)
--(axis cs:336000,704.661795)
--(axis cs:312000,679.111363333333)
--(axis cs:288000,582.90251)
--(axis cs:264000,560.106336666667)
--(axis cs:240000,530.351031666667)
--(axis cs:216000,458.961708333333)
--(axis cs:192000,406.771356375)
--(axis cs:168000,359.876575)
--(axis cs:144000,310.85279)
--(axis cs:120000,206.738235566667)
--(axis cs:96000,89.4869938333333)
--(axis cs:72000,53.3055834333333)
--(axis cs:48000,7.76078525416667)
--(axis cs:24000,6.12820666666667)
--(axis cs:1,5.20993306666667)
--cycle;

\addplot [line width=\linewidthdime, C0, mark=*, mark size=0, mark options={solid}]
table {%
1 4.38568275
24000 5.24062566666667
48000 6.47677866666667
72000 7.91531505
96000 25.2655525166667
120000 98.1754432333333
144000 256.229933333333
168000 322.34443
192000 340.580886666667
216000 396.769735
240000 445.916901666667
264000 479.943696666667
288000 500.466561666667
312000 546.373298333333
336000 570.823473333333
360000 595.413313333333
384000 693.683268333333
408000 699.74308
432000 703.851835
456000 754.911836666667
480000 788.272193333333
504000 818.17334
528000 801.183351666667
552000 848.007833333333
576000 837.287133333333
600000 836.30689
624000 877.642243333333
648000 867.30377
672000 884.991788333333
696000 864.565315
720000 892.450298333333
744000 877.33565
768000 892.189046666667
792000 887.000015
816000 887.173176666667
840000 911.01118
864000 899.757053333333
888000 886.18923
912000 900.713658333333
936000 907.11501
960000 902.474856666667
984000 909.964375
};

\end{axis}

\end{tikzpicture}
}
       \subcaption[]{Humanoid Stand}
       \label{fig::exps_new::humanoid_stand}
    \end{minipage}\hfill    
    \begin{minipage}[b]{0.25\textwidth}
    % Legend is in this subplot
        \centering
        \hspace{0.5cm}
        \vspace{0.5cm}
       \resizebox{0.75\textwidth}{!}{\definecolor{ao(english)}{rgb}{0.0,0.5,0}
\begin{tikzpicture} 
    \begin{axis}[%
    hide axis,
    xmin=10,
    xmax=50,
    ymin=0,
    ymax=0.4,
    legend style={
        draw=white!15!black,
        legend cell align=left,
        legend columns=1, 
        legend style={
            draw=none,
            column sep=1ex,
            line width=0.5pt
        }
    },
    ]
    \addlegendimage{line width=\linewidthdime, color=C0}
    \addlegendentry{DIME (ours)};
    \addlegendimage{line width=\linewidthother, color=C1}
    \addlegendentry{BRO};
    \addlegendimage{line width=\linewidthother, color=C9}
    \addlegendentry{BRO (Fast)};
    \addlegendimage{line width=\linewidthother, color=C2}
    \addlegendentry{CrossQ};
    \addlegendimage{line width=\linewidthother, color=C3}
    \addlegendentry{QSM};
    \addlegendimage{line width=\linewidthother, color=C4}
    \addlegendentry{Diff-QL};
    \addlegendimage{line width=\linewidthother, color=C5}
    \addlegendentry{Consistency-AC};
    \addlegendimage{line width=\linewidthother, color=C6}
    \addlegendentry{DIPO};
    \end{axis}
\end{tikzpicture}}       
       \label{fig::exps_new::humanoid_stand}
    \end{minipage}\hfill    
    \begin{minipage}[b]{0.25\textwidth}
        \centering
       \resizebox{1\textwidth}{!}{% This file was created with tikzplotlib v0.10.1.
\begin{tikzpicture}

\definecolor{darkcyan1115178}{RGB}{1,115,178}
\definecolor{darkgray176}{RGB}{176,176,176}

\begin{axis}[
legend cell align={left},
legend style={fill opacity=0.8, draw opacity=1, text opacity=1, draw=lightgray204, at={(0.03,0.03)},  anchor=north west},
tick align=outside,
tick pos=left,
x grid style={white},
xlabel={Number Env Interactions},
xmajorgrids,
xmin=-0.0, xmax=1000000.0,
xtick style={color=black},
y grid style={white},
ylabel={IQM Success Rate},
ymajorgrids,
ymin=-0.05, ymax=1.05,
ytick style={color=black},
axis background/.style={fill=plot_background},
label style={font=\large},
tick label style={font=\large},
x axis line style={draw=none},
y axis line style={draw=none},
]
% BRO 
\path [draw=C1, fill=C1, opacity=0.2]
(axis cs:25000,0.166666666666667)
--(axis cs:25000,0)
--(axis cs:50000,0)
--(axis cs:75000,0)
--(axis cs:100000,0.0666666666666667)
--(axis cs:125000,0.0333333333333333)
--(axis cs:150000,0.133333333333333)
--(axis cs:175000,0.133333333333333)
--(axis cs:200000,0.3)
--(axis cs:225000,0.2)
--(axis cs:250000,0.233333333333333)
--(axis cs:275000,0.0333333333333333)
--(axis cs:300000,0.2)
--(axis cs:325000,0.133333333333333)
--(axis cs:350000,0.266666666666667)
--(axis cs:375000,0.333333333333333)
--(axis cs:400000,0.2)
--(axis cs:425000,0.266666666666667)
--(axis cs:450000,0.266666666666667)
--(axis cs:475000,0.4)
--(axis cs:500000,0.5)
--(axis cs:525000,0.0333333333333333)
--(axis cs:550000,0.2)
--(axis cs:575000,0.333333333333333)
--(axis cs:600000,0.5)
--(axis cs:625000,0.3)
--(axis cs:650000,0.566666666666667)
--(axis cs:675000,0.333333333333333)
--(axis cs:700000,0.3)
--(axis cs:725000,0.5)
--(axis cs:750000,0.533333333333333)
--(axis cs:775000,0.1)
--(axis cs:800000,0.0666666666666667)
--(axis cs:825000,0.166666666666667)
--(axis cs:850000,0.4)
--(axis cs:875000,0.366666666666667)
--(axis cs:900000,0.5)
--(axis cs:925000,0.333333333333333)
--(axis cs:950000,0.533333333333333)
--(axis cs:975000,0.5)
--(axis cs:1000000,0.566666666666667)
--(axis cs:1000000,0.833333333333333)
--(axis cs:1000000,0.833333333333333)
--(axis cs:975000,0.766666666666667)
--(axis cs:950000,0.833333333333333)
--(axis cs:925000,0.733333333333333)
--(axis cs:900000,0.833333333333333)
--(axis cs:875000,0.733333333333333)
--(axis cs:850000,0.7)
--(axis cs:825000,0.666666666666667)
--(axis cs:800000,0.4)
--(axis cs:775000,0.466666666666667)
--(axis cs:750000,0.866666666666667)
--(axis cs:725000,0.733333333333333)
--(axis cs:700000,0.733333333333333)
--(axis cs:675000,0.9)
--(axis cs:650000,0.933333333333333)
--(axis cs:625000,0.733333333333333)
--(axis cs:600000,0.733333333333333)
--(axis cs:575000,0.666666666666667)
--(axis cs:550000,0.4)
--(axis cs:525000,0.266666666666667)
--(axis cs:500000,0.766666666666667)
--(axis cs:475000,0.733333333333333)
--(axis cs:450000,0.566666666666667)
--(axis cs:425000,0.766666666666667)
--(axis cs:400000,0.633333333333333)
--(axis cs:375000,0.6)
--(axis cs:350000,0.8)
--(axis cs:325000,0.4)
--(axis cs:300000,0.466666666666667)
--(axis cs:275000,0.266666666666667)
--(axis cs:250000,0.533333333333333)
--(axis cs:225000,0.4)
--(axis cs:200000,0.6)
--(axis cs:175000,0.366666666666667)
--(axis cs:150000,0.433333333333333)
--(axis cs:125000,0.433333333333333)
--(axis cs:100000,0.366666666666667)
--(axis cs:75000,0.166666666666667)
--(axis cs:50000,0.3)
--(axis cs:25000,0.166666666666667)
--cycle;

\addplot [line width=\linewidthother, C1, mark=*, mark size=0, mark options={solid}]
table {%
25000 0.0333333333333333
50000 0.1
75000 0.0666666666666667
100000 0.233333333333333
125000 0.166666666666667
150000 0.266666666666667
175000 0.233333333333333
200000 0.466666666666667
225000 0.3
250000 0.366666666666667
275000 0.133333333333333
300000 0.3
325000 0.3
350000 0.5
375000 0.5
400000 0.433333333333333
425000 0.533333333333333
450000 0.433333333333333
475000 0.5
500000 0.666666666666667
525000 0.133333333333333
550000 0.333333333333333
575000 0.533333333333333
600000 0.633333333333333
625000 0.6
650000 0.766666666666667
675000 0.633333333333333
700000 0.5
725000 0.633333333333333
750000 0.7
775000 0.266666666666667
800000 0.233333333333333
825000 0.4
850000 0.533333333333333
875000 0.6
900000 0.666666666666667
925000 0.566666666666667
950000 0.666666666666667
975000 0.633333333333333
1000000 0.733333333333333
};


% QSM
\path [draw=C3, fill=C3, opacity=0.2]
(axis cs:10000,0)
--(axis cs:10000,0)
--(axis cs:30000,0)
--(axis cs:50000,0)
--(axis cs:70000,0)
--(axis cs:90000,0)
--(axis cs:110000,0)
--(axis cs:130000,0)
--(axis cs:150000,0)
--(axis cs:170000,0)
--(axis cs:190000,0)
--(axis cs:210000,0)
--(axis cs:230000,0)
--(axis cs:250000,0)
--(axis cs:270000,0)
--(axis cs:290000,0)
--(axis cs:310000,0)
--(axis cs:330000,0)
--(axis cs:350000,0)
--(axis cs:370000,0)
--(axis cs:390000,0)
--(axis cs:410000,0)
--(axis cs:430000,0)
--(axis cs:450000,0)
--(axis cs:470000,0)
--(axis cs:490000,0)
--(axis cs:510000,0)
--(axis cs:530000,0)
--(axis cs:550000,0)
--(axis cs:570000,0)
--(axis cs:590000,0)
--(axis cs:610000,0)
--(axis cs:630000,0)
--(axis cs:650000,0)
--(axis cs:670000,0)
--(axis cs:690000,0)
--(axis cs:710000,0)
--(axis cs:730000,0)
--(axis cs:750000,0)
--(axis cs:770000,0)
--(axis cs:790000,0)
--(axis cs:810000,0)
--(axis cs:830000,0)
--(axis cs:850000,0)
--(axis cs:870000,0)
--(axis cs:890000,0)
--(axis cs:910000,0)
--(axis cs:930000,0)
--(axis cs:950000,0)
--(axis cs:970000,0)
--(axis cs:990000,0)
--(axis cs:990000,0.000521342207989052)
--(axis cs:990000,0.000521342207989052)
--(axis cs:970000,0.00208536883195621)
--(axis cs:950000,0.00834147532782484)
--(axis cs:930000,3.25600306498567e-05)
--(axis cs:910000,0.000130240122598953)
--(axis cs:890000,0.000520960490397708)
--(axis cs:870000,0.00208435058981801)
--(axis cs:850000,0.00833740235927204)
--(axis cs:830000,8.13805199868512e-06)
--(axis cs:810000,3.25522084798043e-05)
--(axis cs:790000,0.000130208831978962)
--(axis cs:770000,0.000520835335676869)
--(axis cs:750000,0.00208334131166339)
--(axis cs:730000,0.00833336524665356)
--(axis cs:710000,1.27665698543093e-07)
--(axis cs:690000,5.10613123635058e-07)
--(axis cs:670000,2.04245249454023e-06)
--(axis cs:650000,8.20159912204114e-06)
--(axis cs:630000,3.28063964881646e-05)
--(axis cs:610000,0.000130716959650575)
--(axis cs:590000,0.000522918701232507)
--(axis cs:570000,0.0020914713544092)
--(axis cs:550000,0.00836588541763679)
--(axis cs:530000,0.000130208337213844)
--(axis cs:510000,0.000520833348855376)
--(axis cs:490000,0.00208333339542151)
--(axis cs:470000,0.00833333358789484)
--(axis cs:450000,9.9341074625651e-10)
--(axis cs:430000,3.97364298502604e-09)
--(axis cs:410000,1.58945719401042e-08)
--(axis cs:390000,6.35782877604167e-08)
--(axis cs:370000,2.54313151041667e-07)
--(axis cs:350000,1.01725260416667e-06)
--(axis cs:330000,4.06901041666667e-06)
--(axis cs:310000,1.62760416666667e-05)
--(axis cs:290000,6.51041666666667e-05)
--(axis cs:270000,0.000260416666666667)
--(axis cs:250000,0.00104166666666667)
--(axis cs:230000,0.00416666666666667)
--(axis cs:210000,0)
--(axis cs:190000,0)
--(axis cs:170000,0)
--(axis cs:150000,0)
--(axis cs:130000,0)
--(axis cs:110000,0)
--(axis cs:90000,0)
--(axis cs:70000,0)
--(axis cs:50000,0)
--(axis cs:30000,0)
--(axis cs:10000,0)
--cycle;

\addplot [line width=\linewidthother, C3, mark=*, mark size=0, mark options={solid}]
table {%
10000 0
30000 0
50000 0
70000 0
90000 0
110000 0
130000 0
150000 0
170000 0
190000 0
210000 0
230000 0
250000 0
270000 0
290000 0
310000 0
330000 0
350000 0
370000 0
390000 0
410000 0
430000 0
450000 0
470000 0
490000 0
510000 0
530000 0
550000 0
570000 0
590000 0
610000 0
630000 0
650000 0
670000 0
690000 0
710000 0
730000 0
750000 0
770000 0
790000 0
810000 0
830000 0
850000 3.03164900590976e-14
870000 7.5791225147744e-15
890000 1.8947806286936e-15
910000 4.736951571734e-16
930000 1.1842378929335e-16
950000 1.98682907163554e-09
970000 4.96707267908884e-10
990000 1.24176816977221e-10
};

%CrossQ
\path [draw=C2, fill=C2, opacity=0.2]
(axis cs:1,0)
--(axis cs:1,0)
--(axis cs:24000,0)
--(axis cs:48000,0)
--(axis cs:72000,0)
--(axis cs:96000,0.0333333338300387)
--(axis cs:120000,0.0666666676600774)
--(axis cs:144000,0.233333336810271)
--(axis cs:168000,0.333333340783914)
--(axis cs:192000,0.466666676104069)
--(axis cs:216000,0.466666678587596)
--(axis cs:240000,0.400000005960464)
--(axis cs:264000,0.500000017384688)
--(axis cs:288000,0.400000010927518)
--(axis cs:312000,0.400000008443991)
--(axis cs:336000,0.26666667064031)
--(axis cs:360000,0.466666676104069)
--(axis cs:384000,0.533333346247673)
--(axis cs:408000,0.566666677594185)
--(axis cs:432000,0.766666680574417)
--(axis cs:456000,0.533333346247673)
--(axis cs:480000,0.600000023841858)
--(axis cs:504000,0.666666686534882)
--(axis cs:528000,0.600000023841858)
--(axis cs:552000,0.766666675607363)
--(axis cs:576000,0.733333349227905)
--(axis cs:600000,0.666666679084301)
--(axis cs:624000,0.600000023841858)
--(axis cs:648000,0.633333347737789)
--(axis cs:672000,0.766666680574417)
--(axis cs:696000,0.566666677594185)
--(axis cs:720000,0.800000011920929)
--(axis cs:744000,0.700000017881393)
--(axis cs:768000,0.566666677594185)
--(axis cs:792000,0.700000017881393)
--(axis cs:816000,0.833333343267441)
--(axis cs:840000,0.800000011920929)
--(axis cs:864000,0.733333349227905)
--(axis cs:888000,0.800000011920929)
--(axis cs:912000,0.800000011920929)
--(axis cs:936000,0.766666680574417)
--(axis cs:960000,0.70000001291434)
--(axis cs:984000,0.800000011920929)
--(axis cs:984000,1)
--(axis cs:984000,1)
--(axis cs:960000,0.900000005960464)
--(axis cs:936000,1)
--(axis cs:912000,1)
--(axis cs:888000,1)
--(axis cs:864000,1)
--(axis cs:840000,0.966666668653488)
--(axis cs:816000,1)
--(axis cs:792000,0.933333337306976)
--(axis cs:768000,0.833333343267441)
--(axis cs:744000,1)
--(axis cs:720000,1)
--(axis cs:696000,0.900000005960464)
--(axis cs:672000,0.966666668653488)
--(axis cs:648000,0.966666668653488)
--(axis cs:624000,0.966666668653488)
--(axis cs:600000,0.933333337306976)
--(axis cs:576000,0.966666668653488)
--(axis cs:552000,1)
--(axis cs:528000,0.800000011920929)
--(axis cs:504000,0.900000005960464)
--(axis cs:480000,0.800000011920929)
--(axis cs:456000,0.800000011920929)
--(axis cs:432000,0.966666668653488)
--(axis cs:408000,0.900000005960464)
--(axis cs:384000,0.833333343267441)
--(axis cs:360000,0.800000011920929)
--(axis cs:336000,0.766666680574417)
--(axis cs:312000,0.800000011920929)
--(axis cs:288000,0.733333349227905)
--(axis cs:264000,0.800000011920929)
--(axis cs:240000,0.800000006953875)
--(axis cs:216000,0.600000023841858)
--(axis cs:192000,0.733333349227905)
--(axis cs:168000,0.63333335518837)
--(axis cs:144000,0.566666687528292)
--(axis cs:120000,0.466666683554649)
--(axis cs:96000,0.466666678587596)
--(axis cs:72000,0.16666667163372)
--(axis cs:48000,0.200000002980232)
--(axis cs:24000,0.100000001490116)
--(axis cs:1,0)
--cycle;

\addplot [line width=\linewidthother, C2, mark=*, mark size=0, mark options={solid}]
table {%
1 0
24000 0
48000 0.0666666676600774
72000 0.0333333338300387
96000 0.233333336810271
120000 0.233333339293798
144000 0.400000010927518
168000 0.533333351214727
192000 0.600000018874804
216000 0.566666687528292
240000 0.533333346247673
264000 0.700000017881393
288000 0.600000018874804
312000 0.666666681567828
336000 0.533333343764146
360000 0.600000018874804
384000 0.666666686534882
408000 0.766666680574417
432000 0.866666674613953
456000 0.733333349227905
480000 0.700000017881393
504000 0.766666680574417
528000 0.733333349227905
552000 0.900000005960464
576000 0.866666674613953
600000 0.833333343267441
624000 0.766666680574417
648000 0.833333343267441
672000 0.866666674613953
696000 0.766666680574417
720000 0.966666668653488
744000 0.933333337306976
768000 0.766666680574417
792000 0.833333343267441
816000 0.933333337306976
840000 0.866666674613953
864000 0.900000005960464
888000 0.900000005960464
912000 0.933333337306976
936000 0.933333337306976
960000 0.800000011920929
984000 1
};

% BRO Fast
\path [draw=C9, fill=C9, opacity=0.2]
(axis cs:25000,0.166666666666667)
--(axis cs:25000,0)
--(axis cs:50000,0)
--(axis cs:75000,0.0333333333333333)
--(axis cs:100000,0)
--(axis cs:125000,0)
--(axis cs:150000,0.0666666666666667)
--(axis cs:175000,0.0666666666666667)
--(axis cs:200000,0.0666666666666667)
--(axis cs:225000,0)
--(axis cs:250000,0.0333333333333333)
--(axis cs:275000,0.1)
--(axis cs:300000,0.0666666666666667)
--(axis cs:325000,0)
--(axis cs:350000,0.133333333333333)
--(axis cs:375000,0.166666666666667)
--(axis cs:400000,0.1)
--(axis cs:425000,0.0666666666666667)
--(axis cs:450000,0.0333333333333333)
--(axis cs:475000,0.0666666666666667)
--(axis cs:500000,0.0666666666666667)
--(axis cs:525000,0.1)
--(axis cs:550000,0.2)
--(axis cs:575000,0.1)
--(axis cs:600000,0.266666666666667)
--(axis cs:625000,0.166666666666667)
--(axis cs:650000,0.166666666666667)
--(axis cs:675000,0.133333333333333)
--(axis cs:700000,0.0666666666666667)
--(axis cs:725000,0.1)
--(axis cs:750000,0.2)
--(axis cs:775000,0.1)
--(axis cs:800000,0.233333333333333)
--(axis cs:825000,0.0666666666666667)
--(axis cs:850000,0.233333333333333)
--(axis cs:875000,0.2)
--(axis cs:900000,0.166666666666667)
--(axis cs:925000,0.166666666666667)
--(axis cs:950000,0.266666666666667)
--(axis cs:975000,0.3)
--(axis cs:1000000,0.3)
--(axis cs:1000000,0.566666666666667)
--(axis cs:1000000,0.566666666666667)
--(axis cs:975000,0.633333333333333)
--(axis cs:950000,0.566666666666667)
--(axis cs:925000,0.5)
--(axis cs:900000,0.533333333333333)
--(axis cs:875000,0.4)
--(axis cs:850000,0.5)
--(axis cs:825000,0.533333333333333)
--(axis cs:800000,0.466666666666667)
--(axis cs:775000,0.533333333333333)
--(axis cs:750000,0.466666666666667)
--(axis cs:725000,0.4)
--(axis cs:700000,0.2)
--(axis cs:675000,0.5)
--(axis cs:650000,0.3)
--(axis cs:625000,0.4)
--(axis cs:600000,0.4)
--(axis cs:575000,0.466666666666667)
--(axis cs:550000,0.466666666666667)
--(axis cs:525000,0.4)
--(axis cs:500000,0.3)
--(axis cs:475000,0.4)
--(axis cs:450000,0.4)
--(axis cs:425000,0.366666666666667)
--(axis cs:400000,0.333333333333333)
--(axis cs:375000,0.4)
--(axis cs:350000,0.4)
--(axis cs:325000,0.2)
--(axis cs:300000,0.333333333333333)
--(axis cs:275000,0.2)
--(axis cs:250000,0.266666666666667)
--(axis cs:225000,0.233333333333333)
--(axis cs:200000,0.233333333333333)
--(axis cs:175000,0.333333333333333)
--(axis cs:150000,0.333333333333333)
--(axis cs:125000,0.233333333333333)
--(axis cs:100000,0.233333333333333)
--(axis cs:75000,0.233333333333333)
--(axis cs:50000,0.133333333333333)
--(axis cs:25000,0.166666666666667)
--cycle;

\addplot [line width=\linewidthother, C9, mark=*, mark size=0, mark options={solid}]
table {%
25000 0.0666666666666667
50000 0.0333333333333333
75000 0.133333333333333
100000 0.0333333333333333
125000 0.1
150000 0.2
175000 0.2
200000 0.166666666666667
225000 0.1
250000 0.133333333333333
275000 0.2
300000 0.2
325000 0.1
350000 0.233333333333333
375000 0.266666666666667
400000 0.233333333333333
425000 0.233333333333333
450000 0.233333333333333
475000 0.233333333333333
500000 0.166666666666667
525000 0.266666666666667
550000 0.3
575000 0.266666666666667
600000 0.366666666666667
625000 0.266666666666667
650000 0.2
675000 0.3
700000 0.166666666666667
725000 0.233333333333333
750000 0.3
775000 0.3
800000 0.333333333333333
825000 0.233333333333333
850000 0.366666666666667
875000 0.3
900000 0.366666666666667
925000 0.3
950000 0.433333333333333
975000 0.433333333333333
1000000 0.433333333333333
};

% DIME %
\path [draw=C0, fill=C0, opacity=0.2]
(axis cs:1,0)
--(axis cs:1,0)
--(axis cs:24000,0)
--(axis cs:48000,0)
--(axis cs:72000,0)
--(axis cs:96000,0.133333335320155)
--(axis cs:120000,0.366666672130426)
--(axis cs:144000,0.433333347241084)
--(axis cs:168000,0.366666674613953)
--(axis cs:192000,0.533333346247673)
--(axis cs:216000,0.533333346247673)
--(axis cs:240000,0.63333335518837)
--(axis cs:264000,0.533333353698254)
--(axis cs:288000,0.500000009934107)
--(axis cs:312000,0.600000023841858)
--(axis cs:336000,0.63333335518837)
--(axis cs:360000,0.633333340287209)
--(axis cs:384000,0.666666676600774)
--(axis cs:408000,0.70000001291434)
--(axis cs:432000,0.766666680574417)
--(axis cs:456000,0.70000001291434)
--(axis cs:480000,0.766666680574417)
--(axis cs:504000,0.766666680574417)
--(axis cs:528000,0.733333349227905)
--(axis cs:552000,0.800000011920929)
--(axis cs:576000,0.733333339293798)
--(axis cs:600000,0.633333350221316)
--(axis cs:624000,0.733333339293798)
--(axis cs:648000,0.766666680574417)
--(axis cs:672000,0.733333349227905)
--(axis cs:696000,0.666666686534882)
--(axis cs:720000,0.733333349227905)
--(axis cs:744000,0.800000011920929)
--(axis cs:768000,0.766666675607363)
--(axis cs:792000,0.733333349227905)
--(axis cs:816000,0.766666680574417)
--(axis cs:840000,0.666666686534882)
--(axis cs:864000,0.900000005960464)
--(axis cs:888000,0.833333343267441)
--(axis cs:912000,0.70000001291434)
--(axis cs:936000,0.966666668653488)
--(axis cs:960000,0.966666668653488)
--(axis cs:984000,0.966666668653488)
--(axis cs:984000,1)
--(axis cs:984000,1)
--(axis cs:960000,1)
--(axis cs:936000,1)
--(axis cs:912000,1)
--(axis cs:888000,1)
--(axis cs:864000,1)
--(axis cs:840000,0.966666668653488)
--(axis cs:816000,1)
--(axis cs:792000,1)
--(axis cs:768000,1)
--(axis cs:744000,1)
--(axis cs:720000,1)
--(axis cs:696000,0.933333337306976)
--(axis cs:672000,0.966666668653488)
--(axis cs:648000,0.966666668653488)
--(axis cs:624000,1)
--(axis cs:600000,0.966666668653488)
--(axis cs:576000,1)
--(axis cs:552000,1)
--(axis cs:528000,1)
--(axis cs:504000,0.966666668653488)
--(axis cs:480000,0.966666668653488)
--(axis cs:456000,0.966666668653488)
--(axis cs:432000,0.933333337306976)
--(axis cs:408000,0.933333337306976)
--(axis cs:384000,0.933333337306976)
--(axis cs:360000,0.966666668653488)
--(axis cs:336000,0.933333337306976)
--(axis cs:312000,0.866666674613953)
--(axis cs:288000,0.900000005960464)
--(axis cs:264000,0.800000011920929)
--(axis cs:240000,0.800000011920929)
--(axis cs:216000,0.800000011920929)
--(axis cs:192000,0.733333349227905)
--(axis cs:168000,0.900000005960464)
--(axis cs:144000,0.63333335518837)
--(axis cs:120000,0.60000001390775)
--(axis cs:96000,0.466666681071122)
--(axis cs:72000,0.200000005463759)
--(axis cs:48000,0.0333333338300387)
--(axis cs:24000,0)
--(axis cs:1,0)
--cycle;


\addplot [line width=\linewidthdime, C0, mark=*, mark size=0, mark options={solid}]
table {%
1 0
24000 0
48000 0
72000 0.0666666676600774
96000 0.300000004470348
120000 0.466666678587596
144000 0.566666687528292
168000 0.666666676600774
192000 0.63333335518837
216000 0.733333349227905
240000 0.733333349227905
264000 0.63333335518837
288000 0.70000001291434
312000 0.733333349227905
336000 0.766666680574417
360000 0.866666674613953
384000 0.800000011920929
408000 0.833333343267441
432000 0.833333343267441
456000 0.866666674613953
480000 0.866666674613953
504000 0.866666674613953
528000 0.900000005960464
552000 0.900000005960464
576000 0.900000005960464
600000 0.800000011920929
624000 0.933333337306976
648000 0.866666674613953
672000 0.866666674613953
696000 0.800000011920929
720000 0.900000005960464
744000 0.900000005960464
768000 0.900000005960464
792000 0.900000005960464
816000 0.933333337306976
840000 0.833333343267441
864000 1
888000 0.966666668653488
912000 0.900000005960464
936000 1
960000 1
984000 1
};
\end{axis}

\end{tikzpicture}
}
       \subcaption[]{Object Hold Hard}
       \label{fig::exps_new::myo_hand_obj_hold_hard}
    \end{minipage}\hfill
   \begin{minipage}[b]{0.25\textwidth}
        \centering
       \resizebox{1\textwidth}{!}{% This file was created with tikzplotlib v0.10.1.
\begin{tikzpicture}

\definecolor{darkcyan1115178}{RGB}{1,115,178}
\definecolor{darkgray176}{RGB}{176,176,176}

\begin{axis}[
legend cell align={left},
legend style={fill opacity=0.8, draw opacity=1, text opacity=1, draw=lightgray204, at={(0.03,0.03)},  anchor=north west},
tick align=outside,
tick pos=left,
x grid style={white},
xlabel={Number Env Interactions},
xmajorgrids,
xmin=-0.0, xmax=1000000.0,
xtick style={color=black},
y grid style={white},
ylabel={IQM Success Rate},
ymajorgrids,
ymin=-0.05, ymax=1.05,
ytick style={color=black},
axis background/.style={fill=plot_background},
label style={font=\large},
tick label style={font=\large},
x axis line style={draw=none},
y axis line style={draw=none},
]
\path [draw=C1, fill=C1, opacity=0.2]
(axis cs:25000,0.0666666666666667)
--(axis cs:25000,0)
--(axis cs:50000,0)
--(axis cs:75000,0)
--(axis cs:100000,0.1)
--(axis cs:125000,0.0666666666666667)
--(axis cs:150000,0.166666666666667)
--(axis cs:175000,0.3)
--(axis cs:200000,0.4)
--(axis cs:225000,0.4)
--(axis cs:250000,0.433333333333333)
--(axis cs:275000,0.533333333333333)
--(axis cs:300000,0.466666666666667)
--(axis cs:325000,0.666666666666667)
--(axis cs:350000,0.5)
--(axis cs:375000,0.633333333333333)
--(axis cs:400000,0.5)
--(axis cs:425000,0.633333333333333)
--(axis cs:450000,0.7)
--(axis cs:475000,0.8)
--(axis cs:500000,0.766666666666667)
--(axis cs:525000,0.666666666666667)
--(axis cs:550000,0.833333333333333)
--(axis cs:575000,0.633333333333333)
--(axis cs:600000,0.666666666666667)
--(axis cs:625000,0.8)
--(axis cs:650000,0.6)
--(axis cs:675000,0.733333333333333)
--(axis cs:700000,0.766666666666667)
--(axis cs:725000,0.8)
--(axis cs:750000,0.733333333333333)
--(axis cs:775000,0.833333333333333)
--(axis cs:800000,0.7)
--(axis cs:825000,0.866666666666667)
--(axis cs:850000,0.8)
--(axis cs:875000,0.8)
--(axis cs:900000,1)
--(axis cs:925000,0.9)
--(axis cs:950000,0.866666666666667)
--(axis cs:975000,0.833333333333333)
--(axis cs:1000000,0.966666666666667)
--(axis cs:1000000,1)
--(axis cs:1000000,1)
--(axis cs:975000,1)
--(axis cs:950000,1)
--(axis cs:925000,1)
--(axis cs:900000,1)
--(axis cs:875000,1)
--(axis cs:850000,1)
--(axis cs:825000,1)
--(axis cs:800000,0.933333333333333)
--(axis cs:775000,1)
--(axis cs:750000,0.966666666666667)
--(axis cs:725000,1)
--(axis cs:700000,1)
--(axis cs:675000,0.966666666666667)
--(axis cs:650000,0.966666666666667)
--(axis cs:625000,1)
--(axis cs:600000,0.966666666666667)
--(axis cs:575000,1)
--(axis cs:550000,1)
--(axis cs:525000,0.8)
--(axis cs:500000,0.933333333333333)
--(axis cs:475000,0.933333333333333)
--(axis cs:450000,0.9)
--(axis cs:425000,0.9)
--(axis cs:400000,0.833333333333333)
--(axis cs:375000,0.966666666666667)
--(axis cs:350000,0.8)
--(axis cs:325000,0.9)
--(axis cs:300000,0.866666666666667)
--(axis cs:275000,0.866666666666667)
--(axis cs:250000,0.866666666666667)
--(axis cs:225000,0.8)
--(axis cs:200000,0.7)
--(axis cs:175000,0.7)
--(axis cs:150000,0.6)
--(axis cs:125000,0.333333333333333)
--(axis cs:100000,0.4)
--(axis cs:75000,0.2)
--(axis cs:50000,0.133333333333333)
--(axis cs:25000,0.0666666666666667)
--cycle;

\addplot [line width=\linewidthother, C1, mark=*, mark size=0, mark options={solid}]
table {%
25000 0
50000 0.0333333333333333
75000 0.1
100000 0.266666666666667
125000 0.2
150000 0.4
175000 0.4
200000 0.533333333333333
225000 0.633333333333333
250000 0.633333333333333
275000 0.7
300000 0.666666666666667
325000 0.766666666666667
350000 0.7
375000 0.8
400000 0.766666666666667
425000 0.766666666666667
450000 0.8
475000 0.833333333333333
500000 0.833333333333333
525000 0.8
550000 0.966666666666667
575000 0.866666666666667
600000 0.833333333333333
625000 0.9
650000 0.833333333333333
675000 0.866666666666667
700000 0.933333333333333
725000 0.9
750000 0.866666666666667
775000 0.966666666666667
800000 0.833333333333333
825000 0.966666666666667
850000 0.933333333333333
875000 0.933333333333333
900000 1
925000 1
950000 0.966666666666667
975000 0.933333333333333
1000000 1
};

% QSM
\path [draw=C3, fill=C3, opacity=0.2]
(axis cs:10000,0)
--(axis cs:10000,0)
--(axis cs:30000,0)
--(axis cs:50000,0)
--(axis cs:70000,0)
--(axis cs:90000,0)
--(axis cs:110000,0)
--(axis cs:130000,0)
--(axis cs:150000,0)
--(axis cs:170000,0)
--(axis cs:190000,0)
--(axis cs:210000,0)
--(axis cs:230000,0)
--(axis cs:250000,0)
--(axis cs:270000,0)
--(axis cs:290000,0)
--(axis cs:310000,0)
--(axis cs:330000,0)
--(axis cs:350000,0)
--(axis cs:370000,0)
--(axis cs:390000,0)
--(axis cs:410000,0)
--(axis cs:430000,0)
--(axis cs:450000,0)
--(axis cs:470000,0)
--(axis cs:490000,0)
--(axis cs:510000,8.13802083333333e-06)
--(axis cs:530000,2.03450520833333e-06)
--(axis cs:550000,0.0001312255859375)
--(axis cs:570000,3.2806396484375e-05)
--(axis cs:590000,1.63396199544271e-05)
--(axis cs:610000,4.08490498860677e-06)
--(axis cs:630000,0.00573270161946615)
--(axis cs:650000,0.00937525627513726)
--(axis cs:670000,0.0022157035736988)
--(axis cs:690000,0.00414231627558668)
--(axis cs:710000,0.00962118109067281)
--(axis cs:730000,0.0193286659855706)
--(axis cs:750000,0.0375719847685468)
--(axis cs:770000,0.0267525700231393)
--(axis cs:790000,0.0393723170117786)
--(axis cs:810000,0.0198264598846436)
--(axis cs:830000,0.0272690137227395)
--(axis cs:850000,0.0146249951175074)
--(axis cs:870000,0.032926772318378)
--(axis cs:890000,0.0225162666912184)
--(axis cs:910000,0.0224756858156373)
--(axis cs:930000,0.0284738221600076)
--(axis cs:950000,0.0463168006487346)
--(axis cs:970000,0.0401949325908693)
--(axis cs:990000,0.0552095087708677)
--(axis cs:990000,0.192393823052335)
--(axis cs:990000,0.192393823052335)
--(axis cs:970000,0.211729207252334)
--(axis cs:950000,0.179187206403504)
--(axis cs:930000,0.13598595436743)
--(axis cs:910000,0.202015583115527)
--(axis cs:890000,0.154970464195094)
--(axis cs:870000,0.131356751235139)
--(axis cs:850000,0.14870089721911)
--(axis cs:830000,0.199849632444623)
--(axis cs:810000,0.201225126145462)
--(axis cs:790000,0.231549112613478)
--(axis cs:770000,0.0936708568180883)
--(axis cs:750000,0.16086433250457)
--(axis cs:730000,0.119585126116513)
--(axis cs:710000,0.120587170682969)
--(axis cs:690000,0.0718217877916686)
--(axis cs:670000,0.140947649732407)
--(axis cs:650000,0.0678764096973464)
--(axis cs:630000,0.129019221017758)
--(axis cs:610000,0.0317317965130011)
--(axis cs:590000,0.0502197265625)
--(axis cs:570000,0.0295188904802004)
--(axis cs:550000,0.0278727372487386)
--(axis cs:530000,0.0280176099141438)
--(axis cs:510000,0.0403082021077474)
--(axis cs:490000,0.0297210693359375)
--(axis cs:470000,0.0251302083333333)
--(axis cs:450000,0.00469563802083333)
--(axis cs:430000,0.0187825520833333)
--(axis cs:410000,0.00846354166666667)
--(axis cs:390000,0.000520833333333333)
--(axis cs:370000,0.00208333333333333)
--(axis cs:350000,0.00833333333333333)
--(axis cs:330000,0)
--(axis cs:310000,0)
--(axis cs:290000,0)
--(axis cs:270000,0)
--(axis cs:250000,0)
--(axis cs:230000,0)
--(axis cs:210000,0)
--(axis cs:190000,0)
--(axis cs:170000,0)
--(axis cs:150000,0)
--(axis cs:130000,0)
--(axis cs:110000,0)
--(axis cs:90000,0)
--(axis cs:70000,0)
--(axis cs:50000,0)
--(axis cs:30000,0)
--(axis cs:10000,0)
--cycle;

\addplot [line width=\linewidthother, C3, mark=*, mark size=0, mark options={solid}]
table {%
10000 0
30000 0
50000 0
70000 0
90000 0
110000 0
130000 0
150000 0
170000 0
190000 0
210000 0
230000 0
250000 0
270000 0
290000 0
310000 0
330000 0
350000 0
370000 0
390000 0
410000 0
430000 0
450000 0
470000 0.000130208333333333
490000 0.00211588541666667
510000 0.00677897135416667
530000 0.005731201171875
550000 0.00667368570963542
570000 0.00244153340657552
590000 0.0172770500183105
610000 0.00880744469662507
630000 0.0469131489594777
650000 0.0283946995933851
670000 0.0362653415650129
690000 0.0273031058726095
710000 0.0423013186161067
730000 0.0600397774503714
750000 0.0869244984045508
770000 0.0444935480066022
790000 0.129218665764833
810000 0.103795516636455
830000 0.0994477788339434
850000 0.0742211331052208
870000 0.0761621970753378
890000 0.0826176220734264
910000 0.0959438173260083
930000 0.0768326654818253
950000 0.0955798219162088
970000 0.0880728956087298
990000 0.0888375626787062
};

%CrossQ
\path [draw=C2, fill=C2, opacity=0.2]
(axis cs:1,0)
--(axis cs:1,0)
--(axis cs:24000,0)
--(axis cs:48000,0)
--(axis cs:72000,0)
--(axis cs:96000,0)
--(axis cs:120000,0)
--(axis cs:144000,0)
--(axis cs:168000,0)
--(axis cs:192000,0)
--(axis cs:216000,0)
--(axis cs:240000,0)
--(axis cs:264000,0)
--(axis cs:288000,0)
--(axis cs:312000,0)
--(axis cs:336000,0)
--(axis cs:360000,0)
--(axis cs:384000,0)
--(axis cs:408000,0)
--(axis cs:432000,0)
--(axis cs:456000,0)
--(axis cs:480000,0)
--(axis cs:504000,0)
--(axis cs:528000,0)
--(axis cs:552000,0)
--(axis cs:576000,0)
--(axis cs:600000,0.0333333338300387)
--(axis cs:624000,0)
--(axis cs:648000,0.100000001490116)
--(axis cs:672000,0)
--(axis cs:696000,0)
--(axis cs:720000,0)
--(axis cs:744000,0.0333333338300387)
--(axis cs:768000,0)
--(axis cs:792000,0.0333333338300387)
--(axis cs:816000,0.0333333338300387)
--(axis cs:840000,0.100000001490116)
--(axis cs:864000,0.100000003973643)
--(axis cs:888000,0.100000001490116)
--(axis cs:912000,0.133333335320155)
--(axis cs:936000,0.233333336810271)
--(axis cs:960000,0.43333334227403)
--(axis cs:984000,0.333333343267441)
--(axis cs:984000,0.766666680574417)
--(axis cs:984000,0.766666680574417)
--(axis cs:960000,0.766666680574417)
--(axis cs:936000,0.70000001291434)
--(axis cs:912000,0.733333349227905)
--(axis cs:888000,0.700000017881393)
--(axis cs:864000,0.733333349227905)
--(axis cs:840000,0.733333344260852)
--(axis cs:816000,0.366666674613953)
--(axis cs:792000,0.466666673620542)
--(axis cs:768000,0.500000004967054)
--(axis cs:744000,0.633333342770735)
--(axis cs:720000,0.533333346247673)
--(axis cs:696000,0.433333337306976)
--(axis cs:672000,0.400000010927518)
--(axis cs:648000,0.633333342770735)
--(axis cs:624000,0.333333343267441)
--(axis cs:600000,0.600000016391277)
--(axis cs:576000,0.200000002980232)
--(axis cs:552000,0.200000002980232)
--(axis cs:528000,0.16666667163372)
--(axis cs:504000,0.100000003973643)
--(axis cs:480000,0.300000011920929)
--(axis cs:456000,0.433333337306976)
--(axis cs:432000,0.200000007947286)
--(axis cs:408000,0.300000011920929)
--(axis cs:384000,0.0666666676600774)
--(axis cs:360000,0.200000007947286)
--(axis cs:336000,0.100000001490116)
--(axis cs:312000,0.0333333338300387)
--(axis cs:288000,0.100000001490116)
--(axis cs:264000,0.200000002980232)
--(axis cs:240000,0)
--(axis cs:216000,0.0333333338300387)
--(axis cs:192000,0.0333333338300387)
--(axis cs:168000,0)
--(axis cs:144000,0.0333333338300387)
--(axis cs:120000,0)
--(axis cs:96000,0)
--(axis cs:72000,0)
--(axis cs:48000,0)
--(axis cs:24000,0)
--(axis cs:1,0)
--cycle;

\addplot [line width=\linewidthother, C2, mark=*, mark size=0, mark options={solid}]
table {%
1 0
24000 0
48000 0
72000 0
96000 0
120000 0
144000 0
168000 0
192000 0
216000 0
240000 0
264000 0
288000 0
312000 0
336000 0
360000 0
384000 0
408000 0
432000 0
456000 0.0333333338300387
480000 0.0333333338300387
504000 0
528000 0.0333333338300387
552000 0.0333333338300387
576000 0.0333333338300387
600000 0.266666675607363
624000 0.0333333338300387
648000 0.333333338300387
672000 0.133333335320155
696000 0.100000003973643
720000 0.16666667163372
744000 0.300000009437402
768000 0.133333335320155
792000 0.200000002980232
816000 0.200000002980232
840000 0.366666674613953
864000 0.433333347241084
888000 0.400000010927518
912000 0.400000013411045
936000 0.43333334227403
960000 0.633333350221316
984000 0.600000016391277
};

% BRO FAST
\path [draw=C9, fill=C9, opacity=0.2]
(axis cs:25000,0)
--(axis cs:25000,0)
--(axis cs:50000,0)
--(axis cs:75000,0)
--(axis cs:100000,0)
--(axis cs:125000,0)
--(axis cs:150000,0)
--(axis cs:175000,0.0666666666666667)
--(axis cs:200000,0.0333333333333333)
--(axis cs:225000,0.0666666666666667)
--(axis cs:250000,0.0333333333333333)
--(axis cs:275000,0.133333333333333)
--(axis cs:300000,0.2)
--(axis cs:325000,0.166666666666667)
--(axis cs:350000,0.133333333333333)
--(axis cs:375000,0.366666666666667)
--(axis cs:400000,0.233333333333333)
--(axis cs:425000,0.366666666666667)
--(axis cs:450000,0.3)
--(axis cs:475000,0.466666666666667)
--(axis cs:500000,0.5)
--(axis cs:525000,0.566666666666667)
--(axis cs:550000,0.3)
--(axis cs:575000,0.566666666666667)
--(axis cs:600000,0.4)
--(axis cs:625000,0.466666666666667)
--(axis cs:650000,0.533333333333333)
--(axis cs:675000,0.4)
--(axis cs:700000,0.533333333333333)
--(axis cs:725000,0.533333333333333)
--(axis cs:750000,0.366666666666667)
--(axis cs:775000,0.5)
--(axis cs:800000,0.6)
--(axis cs:825000,0.633333333333333)
--(axis cs:850000,0.566666666666667)
--(axis cs:875000,0.666666666666667)
--(axis cs:900000,0.666666666666667)
--(axis cs:925000,0.666666666666667)
--(axis cs:950000,0.733333333333333)
--(axis cs:975000,0.8)
--(axis cs:1000000,0.6)
--(axis cs:1000000,0.933333333333333)
--(axis cs:1000000,0.933333333333333)
--(axis cs:975000,1)
--(axis cs:950000,1)
--(axis cs:925000,1)
--(axis cs:900000,1)
--(axis cs:875000,0.966666666666667)
--(axis cs:850000,0.9)
--(axis cs:825000,0.966666666666667)
--(axis cs:800000,1)
--(axis cs:775000,0.833333333333333)
--(axis cs:750000,0.866666666666667)
--(axis cs:725000,0.933333333333333)
--(axis cs:700000,1)
--(axis cs:675000,0.866666666666667)
--(axis cs:650000,0.966666666666667)
--(axis cs:625000,0.866666666666667)
--(axis cs:600000,0.8)
--(axis cs:575000,0.933333333333333)
--(axis cs:550000,0.866666666666667)
--(axis cs:525000,0.833333333333333)
--(axis cs:500000,0.8)
--(axis cs:475000,0.8)
--(axis cs:450000,0.866666666666667)
--(axis cs:425000,0.733333333333333)
--(axis cs:400000,0.833333333333333)
--(axis cs:375000,0.733333333333333)
--(axis cs:350000,0.733333333333333)
--(axis cs:325000,0.633333333333333)
--(axis cs:300000,0.566666666666667)
--(axis cs:275000,0.566666666666667)
--(axis cs:250000,0.533333333333333)
--(axis cs:225000,0.533333333333333)
--(axis cs:200000,0.4)
--(axis cs:175000,0.533333333333333)
--(axis cs:150000,0.233333333333333)
--(axis cs:125000,0.2)
--(axis cs:100000,0.3)
--(axis cs:75000,0.0333333333333333)
--(axis cs:50000,0.0666666666666667)
--(axis cs:25000,0)
--cycle;

\addplot [line width=\linewidthother, C9, mark=*, mark size=0, mark options={solid}]
table {%
25000 0
50000 0
75000 0
100000 0.1
125000 0.1
150000 0.0666666666666667
175000 0.3
200000 0.233333333333333
225000 0.266666666666667
250000 0.266666666666667
275000 0.3
300000 0.366666666666667
325000 0.366666666666667
350000 0.466666666666667
375000 0.6
400000 0.566666666666667
425000 0.6
450000 0.6
475000 0.666666666666667
500000 0.666666666666667
525000 0.733333333333333
550000 0.633333333333333
575000 0.766666666666667
600000 0.666666666666667
625000 0.733333333333333
650000 0.866666666666667
675000 0.6
700000 0.833333333333333
725000 0.766666666666667
750000 0.733333333333333
775000 0.733333333333333
800000 0.9
825000 0.866666666666667
850000 0.733333333333333
875000 0.866666666666667
900000 0.933333333333333
925000 0.933333333333333
950000 0.966666666666667
975000 0.966666666666667
1000000 0.8
};

% DIME %
\path [draw=C0, fill=C0, opacity=0.2]
(axis cs:1,0)
--(axis cs:1,0)
--(axis cs:24000,0)
--(axis cs:48000,0)
--(axis cs:72000,0)
--(axis cs:96000,0)
--(axis cs:120000,0.26666667064031)
--(axis cs:144000,0.333333338300387)
--(axis cs:168000,0.43333334227403)
--(axis cs:192000,0.666666676600774)
--(axis cs:216000,0.500000014901161)
--(axis cs:240000,0.700000017881393)
--(axis cs:264000,0.500000009934107)
--(axis cs:288000,0.633333345254262)
--(axis cs:312000,0.833333343267441)
--(axis cs:336000,0.800000011920929)
--(axis cs:360000,0.666666679084301)
--(axis cs:384000,0.800000011920929)
--(axis cs:408000,0.700000007947286)
--(axis cs:432000,0.833333343267441)
--(axis cs:456000,0.800000011920929)
--(axis cs:480000,0.800000001986822)
--(axis cs:504000,0.866666674613953)
--(axis cs:528000,0.766666680574417)
--(axis cs:552000,0.666666676600774)
--(axis cs:576000,0.700000007947286)
--(axis cs:600000,0.63333335518837)
--(axis cs:624000,0.733333349227905)
--(axis cs:648000,0.466666668653488)
--(axis cs:672000,0.766666680574417)
--(axis cs:696000,0.800000011920929)
--(axis cs:720000,0.833333343267441)
--(axis cs:744000,0.566666677594185)
--(axis cs:768000,0.733333349227905)
--(axis cs:792000,0.800000011920929)
--(axis cs:816000,0.866666674613953)
--(axis cs:840000,0.766666680574417)
--(axis cs:864000,0.733333349227905)
--(axis cs:888000,0.800000011920929)
--(axis cs:912000,0.700000007947286)
--(axis cs:936000,0.833333343267441)
--(axis cs:960000,0.866666674613953)
--(axis cs:984000,0.666666676600774)
--(axis cs:984000,1)
--(axis cs:984000,1)
--(axis cs:960000,1)
--(axis cs:936000,1)
--(axis cs:912000,1)
--(axis cs:888000,1)
--(axis cs:864000,0.966666668653488)
--(axis cs:840000,0.933333337306976)
--(axis cs:816000,1)
--(axis cs:792000,1)
--(axis cs:768000,1)
--(axis cs:744000,1)
--(axis cs:720000,1)
--(axis cs:696000,1)
--(axis cs:672000,0.966666668653488)
--(axis cs:648000,1)
--(axis cs:624000,0.933333337306976)
--(axis cs:600000,0.966666668653488)
--(axis cs:576000,1)
--(axis cs:552000,1)
--(axis cs:528000,1)
--(axis cs:504000,1)
--(axis cs:480000,1)
--(axis cs:456000,1)
--(axis cs:432000,1)
--(axis cs:408000,1)
--(axis cs:384000,1)
--(axis cs:360000,0.966666668653488)
--(axis cs:336000,1)
--(axis cs:312000,1)
--(axis cs:288000,0.966666668653488)
--(axis cs:264000,0.933333337306976)
--(axis cs:240000,0.900000005960464)
--(axis cs:216000,0.766666680574417)
--(axis cs:192000,0.900000005960464)
--(axis cs:168000,0.900000000993411)
--(axis cs:144000,0.733333344260852)
--(axis cs:120000,0.566666687528292)
--(axis cs:96000,0.133333335320155)
--(axis cs:72000,0.100000001490116)
--(axis cs:48000,0)
--(axis cs:24000,0)
--(axis cs:1,0)
--cycle;

\addplot [line width=\linewidthdime, C0, mark=*, mark size=0, mark options={solid}]
table {%
1 0
24000 0
48000 0
72000 0
96000 0.0333333338300387
120000 0.433333344757557
144000 0.533333346247673
168000 0.633333345254262
192000 0.766666680574417
216000 0.666666686534882
240000 0.800000011920929
264000 0.733333344260852
288000 0.800000011920929
312000 0.933333337306976
336000 0.900000005960464
360000 0.866666674613953
384000 0.933333337306976
408000 0.900000005960464
432000 0.966666668653488
456000 0.900000005960464
480000 1
504000 0.966666668653488
528000 0.933333337306976
552000 0.900000005960464
576000 0.933333337306976
600000 0.800000011920929
624000 0.833333343267441
648000 0.900000005960464
672000 0.866666674613953
696000 0.933333337306976
720000 0.933333337306976
744000 0.866666674613953
768000 0.900000005960464
792000 0.900000005960464
816000 1
840000 0.833333343267441
864000 0.866666674613953
888000 0.933333337306976
912000 0.900000005960464
936000 0.966666668653488
960000 1
984000 0.900000005960464
};

\end{axis}

\end{tikzpicture}
}
       \subcaption[]{Reach Hard}
       \label{fig::exps_new::myo_hand_reach_hard}
    \end{minipage}\hfill
    \begin{minipage}[b]{0.25\textwidth}
        \centering
       \resizebox{1\textwidth}{!}{% This file was created with tikzplotlib v0.10.1.
\begin{tikzpicture}

\definecolor{darkcyan1115178}{RGB}{1,115,178}
\definecolor{darkgray176}{RGB}{176,176,176}

\begin{axis}[
legend cell align={left},
legend style={fill opacity=0.8, draw opacity=1, text opacity=1, draw=lightgray204, at={(0.03,0.03)},  anchor=north west},
tick align=outside,
tick pos=left,
x grid style={white},
xlabel={Number Env Interactions},
xmajorgrids,
%xmin=-37798.95, xmax=1045799.95,
xmin=-0.0, xmax=1000000.0,
xtick style={color=black},
y grid style={white},
ylabel={IQM Success Rate},
ymajorgrids,
ymin=-0.05, ymax=1.05,
ytick style={color=black},
axis background/.style={fill=plot_background},
label style={font=\large},
tick label style={font=\large},
x axis line style={draw=none},
y axis line style={draw=none},
]

% BRO
\path [draw=C1, fill=orange, opacity=0.2]
(axis cs:25000,0)
--(axis cs:25000,0)
--(axis cs:50000,0)
--(axis cs:75000,0)
--(axis cs:100000,0)
--(axis cs:125000,0)
--(axis cs:150000,0)
--(axis cs:175000,0)
--(axis cs:200000,0)
--(axis cs:225000,0)
--(axis cs:250000,0)
--(axis cs:275000,0)
--(axis cs:300000,0)
--(axis cs:325000,0)
--(axis cs:350000,0)
--(axis cs:375000,0)
--(axis cs:400000,0)
--(axis cs:425000,0)
--(axis cs:450000,0)
--(axis cs:475000,0.0333333333333333)
--(axis cs:500000,0.0333333333333333)
--(axis cs:525000,0.0333333333333333)
--(axis cs:550000,0.0333333333333333)
--(axis cs:575000,0)
--(axis cs:600000,0)
--(axis cs:625000,0.0333333333333333)
--(axis cs:650000,0.133333333333333)
--(axis cs:675000,0.0333333333333333)
--(axis cs:700000,0.166666666666667)
--(axis cs:725000,0.233333333333333)
--(axis cs:750000,0.2)
--(axis cs:775000,0.0666666666666667)
--(axis cs:800000,0.1)
--(axis cs:825000,0.133333333333333)
--(axis cs:850000,0.366666666666667)
--(axis cs:875000,0.166666666666667)
--(axis cs:900000,0.366666666666667)
--(axis cs:925000,0.366666666666667)
--(axis cs:950000,0.266666666666667)
--(axis cs:975000,0.3)
--(axis cs:1000000,0.333333333333333)
--(axis cs:1000000,1)
--(axis cs:1000000,1)
--(axis cs:975000,0.966666666666667)
--(axis cs:950000,0.9)
--(axis cs:925000,1)
--(axis cs:900000,0.966666666666667)
--(axis cs:875000,0.8)
--(axis cs:850000,0.866666666666667)
--(axis cs:825000,0.733333333333333)
--(axis cs:800000,0.733333333333333)
--(axis cs:775000,0.333333333333333)
--(axis cs:750000,0.766666666666667)
--(axis cs:725000,0.833333333333333)
--(axis cs:700000,0.833333333333333)
--(axis cs:675000,0.666666666666667)
--(axis cs:650000,0.733333333333333)
--(axis cs:625000,0.633333333333333)
--(axis cs:600000,0.5)
--(axis cs:575000,0.6)
--(axis cs:550000,0.633333333333333)
--(axis cs:525000,0.333333333333333)
--(axis cs:500000,0.366666666666667)
--(axis cs:475000,0.466666666666667)
--(axis cs:450000,0.1)
--(axis cs:425000,0.3)
--(axis cs:400000,0.233333333333333)
--(axis cs:375000,0.233333333333333)
--(axis cs:350000,0.1)
--(axis cs:325000,0.0333333333333333)
--(axis cs:300000,0)
--(axis cs:275000,0)
--(axis cs:250000,0)
--(axis cs:225000,0)
--(axis cs:200000,0)
--(axis cs:175000,0)
--(axis cs:150000,0)
--(axis cs:125000,0)
--(axis cs:100000,0)
--(axis cs:75000,0)
--(axis cs:50000,0)
--(axis cs:25000,0)
--cycle;

\addplot [line width=\linewidthother, C1, mark=*, mark size=0, mark options={solid}]
table {%
25000 0
50000 0
75000 0
100000 0
125000 0
150000 0
175000 0
200000 0
225000 0
250000 0
275000 0
300000 0
325000 0
350000 0
375000 0.0333333333333333
400000 0.0666666666666667
425000 0.1
450000 0
475000 0.2
500000 0.166666666666667
525000 0.133333333333333
550000 0.333333333333333
575000 0.266666666666667
600000 0.233333333333333
625000 0.233333333333333
650000 0.433333333333333
675000 0.3
700000 0.5
725000 0.566666666666667
750000 0.533333333333333
775000 0.2
800000 0.4
825000 0.4
850000 0.7
875000 0.466666666666667
900000 0.7
925000 0.833333333333333
950000 0.633333333333333
975000 0.7
1000000 0.766666666666667
};


% QSM
\path [draw=C3, fill=C3, opacity=0.2]
(axis cs:10000,0)
--(axis cs:10000,0)
--(axis cs:30000,0)
--(axis cs:50000,0)
--(axis cs:70000,0)
--(axis cs:90000,0)
--(axis cs:110000,0)
--(axis cs:130000,0)
--(axis cs:150000,0)
--(axis cs:170000,0)
--(axis cs:190000,0)
--(axis cs:210000,0)
--(axis cs:230000,0)
--(axis cs:250000,0)
--(axis cs:270000,0)
--(axis cs:290000,0)
--(axis cs:310000,0)
--(axis cs:330000,0)
--(axis cs:350000,0)
--(axis cs:370000,0)
--(axis cs:390000,0)
--(axis cs:410000,0)
--(axis cs:430000,0)
--(axis cs:450000,6.56291842460632e-05)
--(axis cs:470000,1.64072960615158e-05)
--(axis cs:490000,3.25521143774192e-05)
--(axis cs:510000,8.13802859435479e-06)
--(axis cs:530000,2.03450811871638e-06)
--(axis cs:550000,2.03450569339717e-06)
--(axis cs:570000,5.20230620774479e-05)
--(axis cs:590000,3.31879905241935e-05)
--(axis cs:610000,0.000927790068150839)
--(axis cs:630000,0.000525677470023865)
--(axis cs:650000,0.00052223814518988)
--(axis cs:670000,0.00013055953629747)
--(axis cs:690000,3.26398840743675e-05)
--(axis cs:710000,0.00706876716772106)
--(axis cs:730000,0.00661675929001127)
--(axis cs:750000,0.0257816320287035)
--(axis cs:770000,0.0201534921402834)
--(axis cs:790000,0.0122230567164661)
--(axis cs:810000,0.0239557941755568)
--(axis cs:830000,0.0604909352667354)
--(axis cs:850000,0.0583185404719174)
--(axis cs:870000,0.0393196087365434)
--(axis cs:890000,0.0510379400458383)
--(axis cs:910000,0.0464194536605704)
--(axis cs:930000,0.0488683444311088)
--(axis cs:950000,0.0785525890537812)
--(axis cs:970000,0.104528356833094)
--(axis cs:990000,0.0900562929602449)
--(axis cs:990000,0.396814655470077)
--(axis cs:990000,0.396814655470077)
--(axis cs:970000,0.338467918288518)
--(axis cs:950000,0.274172424297366)
--(axis cs:930000,0.342817424505524)
--(axis cs:910000,0.250617016308486)
--(axis cs:890000,0.250662349518875)
--(axis cs:870000,0.187096599146885)
--(axis cs:850000,0.210944986958025)
--(axis cs:830000,0.175389831910798)
--(axis cs:810000,0.184432572504632)
--(axis cs:790000,0.184097106586826)
--(axis cs:770000,0.139400828882727)
--(axis cs:750000,0.169743563297731)
--(axis cs:730000,0.0957063059439027)
--(axis cs:710000,0.0927717971696991)
--(axis cs:690000,0.048373805386937)
--(axis cs:670000,0.0445855877364693)
--(axis cs:650000,0.0643088599923596)
--(axis cs:630000,0.0713138509315348)
--(axis cs:610000,0.0217038164616194)
--(axis cs:590000,0.0425561273340498)
--(axis cs:570000,0.0729171752929687)
--(axis cs:550000,0.0340587243981602)
--(axis cs:530000,0.009041861825033)
--(axis cs:510000,0.0194845045918957)
--(axis cs:490000,0.0380982460609452)
--(axis cs:470000,0.0125000127281252)
--(axis cs:450000,0.0172531289358934)
--(axis cs:430000,0.0127706607502963)
--(axis cs:410000,0.0333659173338674)
--(axis cs:390000,0.00433060419124862)
--(axis cs:370000,0.000655684930582841)
--(axis cs:350000,0.00262273972233137)
--(axis cs:330000,0.0104929933945338)
--(axis cs:310000,0.00863457123438517)
--(axis cs:290000,0.00120535850524902)
--(axis cs:270000,0.00481980641682943)
--(axis cs:250000,0.00235214233398438)
--(axis cs:230000,0.0094085693359375)
--(axis cs:210000,0.00443522135416667)
--(axis cs:190000,0.000537109375)
--(axis cs:170000,0.0021484375)
--(axis cs:150000,0.00859375)
--(axis cs:130000,0.00104166666666667)
--(axis cs:110000,0.00416666666666667)
--(axis cs:90000,0)
--(axis cs:70000,0)
--(axis cs:50000,0)
--(axis cs:30000,0)
--(axis cs:10000,0)
--cycle;

\addplot [line width=\linewidthother, C3, mark=*, mark size=0, mark options={solid}]
table {%
10000 0
30000 0
50000 0
70000 0
90000 0
110000 0
130000 0
150000 0
170000 0
190000 0
210000 0
230000 0
250000 0
270000 6.51041666666667e-05
290000 1.62760416666667e-05
310000 4.06901041666667e-06
330000 1.01725260416667e-06
350000 2.54313151041667e-07
370000 6.35782877604167e-08
390000 1.58945719401042e-08
410000 0.00105006297429403
430000 0.00234584907690684
450000 0.00267183210259342
470000 0.000667958025648356
490000 0.000423304380107462
510000 0.00321554368129
530000 0.0008038859203225
550000 0.000722824054787689
570000 0.0230971183671519
590000 0.0161910098367424
610000 0.00665185554756451
630000 0.0187202714481297
650000 0.0219543273895567
670000 0.00965524851405585
690000 0.0144913160282472
710000 0.03278745818731
730000 0.0373635312134942
750000 0.0834380368174024
770000 0.0651731307261956
790000 0.0740541471778103
810000 0.0798626665104545
830000 0.116776422960222
850000 0.120309298343193
870000 0.108126961321494
890000 0.140296087114093
910000 0.145224781413192
930000 0.180289806002648
950000 0.166204916539434
970000 0.224884562468192
990000 0.208546194885458
};

%CrossQ
\path [draw=C2, fill=C2, opacity=0.2]
(axis cs:1,0)
--(axis cs:1,0)
--(axis cs:24000,0)
--(axis cs:48000,0)
--(axis cs:72000,0)
--(axis cs:96000,0)
--(axis cs:120000,0)
--(axis cs:144000,0)
--(axis cs:168000,0)
--(axis cs:192000,0)
--(axis cs:216000,0)
--(axis cs:240000,0.100000001490116)
--(axis cs:264000,0)
--(axis cs:288000,0.133333335320155)
--(axis cs:312000,0.0666666676600774)
--(axis cs:336000,0.16666667163372)
--(axis cs:360000,0.200000002980232)
--(axis cs:384000,0.133333335320155)
--(axis cs:408000,0.200000005463759)
--(axis cs:432000,0.300000011920929)
--(axis cs:456000,0.333333343267441)
--(axis cs:480000,0.466666683554649)
--(axis cs:504000,0.300000006953875)
--(axis cs:528000,0.300000006953875)
--(axis cs:552000,0.433333337306976)
--(axis cs:576000,0.466666683554649)
--(axis cs:600000,0.400000005960464)
--(axis cs:624000,0.633333335320155)
--(axis cs:648000,0.500000004967054)
--(axis cs:672000,0.533333338797092)
--(axis cs:696000,0.300000001986821)
--(axis cs:720000,0.400000015894572)
--(axis cs:744000,0.433333337306976)
--(axis cs:768000,0.566666670143604)
--(axis cs:792000,0.566666670143604)
--(axis cs:816000,0.466666668653488)
--(axis cs:840000,0.700000007947286)
--(axis cs:864000,0.466666668653488)
--(axis cs:888000,0.600000001490116)
--(axis cs:912000,0.60000001390775)
--(axis cs:936000,0.600000001490116)
--(axis cs:960000,0.533333338797092)
--(axis cs:984000,0.700000010430813)
--(axis cs:984000,1)
--(axis cs:984000,1)
--(axis cs:960000,1)
--(axis cs:936000,1)
--(axis cs:912000,1)
--(axis cs:888000,1)
--(axis cs:864000,1)
--(axis cs:840000,1)
--(axis cs:816000,1)
--(axis cs:792000,1)
--(axis cs:768000,1)
--(axis cs:744000,1)
--(axis cs:720000,1)
--(axis cs:696000,1)
--(axis cs:672000,1)
--(axis cs:648000,0.933333337306976)
--(axis cs:624000,1)
--(axis cs:600000,0.933333337306976)
--(axis cs:576000,1)
--(axis cs:552000,1)
--(axis cs:528000,0.900000000993411)
--(axis cs:504000,0.933333337306976)
--(axis cs:480000,1)
--(axis cs:456000,0.900000005960464)
--(axis cs:432000,0.966666668653488)
--(axis cs:408000,0.966666668653488)
--(axis cs:384000,0.733333344260852)
--(axis cs:360000,0.866666669646899)
--(axis cs:336000,0.666666686534882)
--(axis cs:312000,0.766666680574417)
--(axis cs:288000,0.633333350221316)
--(axis cs:264000,0.633333347737789)
--(axis cs:240000,0.500000009934107)
--(axis cs:216000,0.333333338300387)
--(axis cs:192000,0.300000004470348)
--(axis cs:168000,0.0333333338300387)
--(axis cs:144000,0.0666666676600774)
--(axis cs:120000,0)
--(axis cs:96000,0)
--(axis cs:72000,0)
--(axis cs:48000,0)
--(axis cs:24000,0)
--(axis cs:1,0)
--cycle;

\addplot [line width=\linewidthother, C2, mark=*, mark size=0, mark options={solid}]
table {%
1 0
24000 0
48000 0
72000 0
96000 0
120000 0
144000 0
168000 0
192000 0.0666666676600774
216000 0.100000001490116
240000 0.300000004470348
264000 0.266666673123837
288000 0.400000010927518
312000 0.400000010927518
336000 0.433333347241084
360000 0.533333338797092
384000 0.466666673620542
408000 0.633333340287209
432000 0.733333341777325
456000 0.70000001291434
480000 0.833333343267441
504000 0.700000010430813
528000 0.600000011424224
552000 0.866666674613953
576000 0.833333343267441
600000 0.800000011920929
624000 0.933333337306976
648000 0.800000011920929
672000 0.900000005960464
696000 0.733333336810271
720000 0.833333343267441
744000 0.866666674613953
768000 0.966666668653488
792000 0.933333337306976
816000 0.933333337306976
840000 0.966666668653488
864000 0.933333337306976
888000 0.966666668653488
912000 0.933333337306976
936000 1
960000 0.900000005960464
984000 0.900000005960464
};

\path [draw=C9, fill=C9, opacity=0.2]
(axis cs:25000,0)
--(axis cs:25000,0)
--(axis cs:50000,0)
--(axis cs:75000,0)
--(axis cs:100000,0)
--(axis cs:125000,0)
--(axis cs:150000,0)
--(axis cs:175000,0)
--(axis cs:200000,0)
--(axis cs:225000,0)
--(axis cs:250000,0)
--(axis cs:275000,0)
--(axis cs:300000,0)
--(axis cs:325000,0)
--(axis cs:350000,0)
--(axis cs:375000,0)
--(axis cs:400000,0)
--(axis cs:425000,0)
--(axis cs:450000,0)
--(axis cs:475000,0)
--(axis cs:500000,0)
--(axis cs:525000,0)
--(axis cs:550000,0)
--(axis cs:575000,0)
--(axis cs:600000,0)
--(axis cs:625000,0)
--(axis cs:650000,0)
--(axis cs:675000,0)
--(axis cs:700000,0)
--(axis cs:725000,0)
--(axis cs:750000,0)
--(axis cs:775000,0)
--(axis cs:800000,0)
--(axis cs:825000,0.0333333333333333)
--(axis cs:850000,0.0333333333333333)
--(axis cs:875000,0)
--(axis cs:900000,0)
--(axis cs:925000,0.1)
--(axis cs:950000,0.0666666666666667)
--(axis cs:975000,0.166666666666667)
--(axis cs:1000000,0.133333333333333)
--(axis cs:1000000,0.7)
--(axis cs:1000000,0.7)
--(axis cs:975000,0.666666666666667)
--(axis cs:950000,0.666666666666667)
--(axis cs:925000,0.666666666666667)
--(axis cs:900000,0.6)
--(axis cs:875000,0.7)
--(axis cs:850000,0.566666666666667)
--(axis cs:825000,0.433333333333333)
--(axis cs:800000,0.4)
--(axis cs:775000,0.5)
--(axis cs:750000,0.466666666666667)
--(axis cs:725000,0.4)
--(axis cs:700000,0.333333333333333)
--(axis cs:675000,0.433333333333333)
--(axis cs:650000,0.266666666666667)
--(axis cs:625000,0.233333333333333)
--(axis cs:600000,0.266666666666667)
--(axis cs:575000,0.1)
--(axis cs:550000,0)
--(axis cs:525000,0.0333333333333333)
--(axis cs:500000,0.0333333333333333)
--(axis cs:475000,0)
--(axis cs:450000,0)
--(axis cs:425000,0.0333333333333333)
--(axis cs:400000,0)
--(axis cs:375000,0.0333333333333333)
--(axis cs:350000,0)
--(axis cs:325000,0)
--(axis cs:300000,0)
--(axis cs:275000,0)
--(axis cs:250000,0)
--(axis cs:225000,0)
--(axis cs:200000,0)
--(axis cs:175000,0)
--(axis cs:150000,0)
--(axis cs:125000,0)
--(axis cs:100000,0)
--(axis cs:75000,0)
--(axis cs:50000,0)
--(axis cs:25000,0)
--cycle;

\addplot [line width=\linewidthother, C9, mark=*, mark size=0, mark options={solid}]
table {%
25000 0
50000 0
75000 0
100000 0
125000 0
150000 0
175000 0
200000 0
225000 0
250000 0
275000 0
300000 0
325000 0
350000 0
375000 0
400000 0
425000 0
450000 0
475000 0
500000 0
525000 0
550000 0
575000 0
600000 0.0666666666666667
625000 0.0666666666666667
650000 0.0666666666666667
675000 0.166666666666667
700000 0.0666666666666667
725000 0.133333333333333
750000 0.2
775000 0.2
800000 0.166666666666667
825000 0.166666666666667
850000 0.3
875000 0.333333333333333
900000 0.233333333333333
925000 0.3
950000 0.366666666666667
975000 0.433333333333333
1000000 0.433333333333333
};

% DIME %
\path [draw=C0, fill=C0, opacity=0.2]
(axis cs:1,0)
--(axis cs:1,0)
--(axis cs:24000,0)
--(axis cs:48000,0)
--(axis cs:72000,0)
--(axis cs:96000,0)
--(axis cs:120000,0)
--(axis cs:144000,0)
--(axis cs:168000,0)
--(axis cs:192000,0)
--(axis cs:216000,0)
--(axis cs:240000,0)
--(axis cs:264000,0)
--(axis cs:288000,0.0333333338300387)
--(axis cs:312000,0.133333335320155)
--(axis cs:336000,0.366666669646899)
--(axis cs:360000,0.466666668653488)
--(axis cs:384000,0.200000002980232)
--(axis cs:408000,0.400000010927518)
--(axis cs:432000,0.433333337306976)
--(axis cs:456000,0.66666667163372)
--(axis cs:480000,0.76666667064031)
--(axis cs:504000,0.800000011920929)
--(axis cs:528000,0.800000011920929)
--(axis cs:552000,1)
--(axis cs:576000,0.900000005960464)
--(axis cs:600000,0.966666668653488)
--(axis cs:624000,0.900000005960464)
--(axis cs:648000,0.933333337306976)
--(axis cs:672000,0.866666674613953)
--(axis cs:696000,0.966666668653488)
--(axis cs:720000,0.966666668653488)
--(axis cs:744000,0.800000001986822)
--(axis cs:768000,0.900000005960464)
--(axis cs:792000,0.966666668653488)
--(axis cs:816000,0.966666668653488)
--(axis cs:840000,0.966666668653488)
--(axis cs:864000,1)
--(axis cs:888000,0.900000005960464)
--(axis cs:912000,0.966666668653488)
--(axis cs:936000,0.966666668653488)
--(axis cs:960000,0.866666674613953)
--(axis cs:984000,1)
--(axis cs:984000,1)
--(axis cs:984000,1)
--(axis cs:960000,1)
--(axis cs:936000,1)
--(axis cs:912000,1)
--(axis cs:888000,1)
--(axis cs:864000,1)
--(axis cs:840000,1)
--(axis cs:816000,1)
--(axis cs:792000,1)
--(axis cs:768000,1)
--(axis cs:744000,1)
--(axis cs:720000,1)
--(axis cs:696000,1)
--(axis cs:672000,1)
--(axis cs:648000,1)
--(axis cs:624000,1)
--(axis cs:600000,1)
--(axis cs:576000,1)
--(axis cs:552000,1)
--(axis cs:528000,1)
--(axis cs:504000,1)
--(axis cs:480000,1)
--(axis cs:456000,1)
--(axis cs:432000,1)
--(axis cs:408000,0.966666668653488)
--(axis cs:384000,0.900000005960464)
--(axis cs:360000,1)
--(axis cs:336000,0.933333337306976)
--(axis cs:312000,0.900000005960464)
--(axis cs:288000,0.900000005960464)
--(axis cs:264000,0.566666677594185)
--(axis cs:240000,0.366666672130426)
--(axis cs:216000,0.233333336810271)
--(axis cs:192000,0.200000007947286)
--(axis cs:168000,0.133333335320155)
--(axis cs:144000,0)
--(axis cs:120000,0)
--(axis cs:96000,0)
--(axis cs:72000,0)
--(axis cs:48000,0)
--(axis cs:24000,0)
--(axis cs:1,0)
--cycle;

\addplot [line width=\linewidthdime, C0, mark=*, mark size=0, mark options={solid}]
table {%
1 0
24000 0
48000 0
72000 0
96000 0
120000 0
144000 0
168000 0.0333333338300387
192000 0
216000 0.0666666676600774
240000 0.133333335320155
264000 0.200000005463759
288000 0.466666671137015
312000 0.533333338797092
336000 0.733333344260852
360000 0.933333337306976
384000 0.566666672627131
408000 0.733333344260852
432000 0.83333333581686
456000 0.966666668653488
480000 0.966666668653488
504000 0.933333337306976
528000 0.900000005960464
552000 1
576000 1
600000 1
624000 1
648000 1
672000 1
696000 1
720000 1
744000 1
768000 1
792000 1
816000 1
840000 1
864000 1
888000 1
912000 1
936000 1
960000 1
984000 1
};
\end{axis}

\end{tikzpicture}
}
       \subcaption[]{Key Turn Hard}
       \label{fig::exps_new::myo_hand_key_turn_rndm}
    \end{minipage}\hfill
    \begin{minipage}[b]{0.25\textwidth}
        \centering
       \resizebox{1\textwidth}{!}{% This file was created with tikzplotlib v0.10.1.
\begin{tikzpicture}

\definecolor{darkcyan1115178}{RGB}{1,115,178}
\definecolor{darkgray176}{RGB}{176,176,176}

\begin{axis}[
legend cell align={left},
legend style={fill opacity=0.8, draw opacity=1, text opacity=1, draw=lightgray204, at={(0.03,0.03)},  anchor=north west},
tick align=outside,
tick pos=left,
x grid style={white},
xlabel={Number Env Interactions},
xmajorgrids,
xmin=-0.0, xmax=1000000.0,
xtick style={color=black},
y grid style={white},
ylabel={IQM Success Rate},
ymajorgrids,
ymin=-0.05, ymax=1.05,
ytick style={color=black},
axis background/.style={fill=plot_background},
label style={font=\large},
tick label style={font=\large},
x axis line style={draw=none},
y axis line style={draw=none},
]
\path [draw=C1, fill=C1, opacity=0.2]
(axis cs:25000,0.133333333333333)
--(axis cs:25000,0)
--(axis cs:50000,0.2)
--(axis cs:75000,0.0333333333333333)
--(axis cs:100000,0.266666666666667)
--(axis cs:125000,0.2)
--(axis cs:150000,0.266666666666667)
--(axis cs:175000,0.2)
--(axis cs:200000,0.266666666666667)
--(axis cs:225000,0.4)
--(axis cs:250000,0.433333333333333)
--(axis cs:275000,0.3)
--(axis cs:300000,0.533333333333333)
--(axis cs:325000,0.433333333333333)
--(axis cs:350000,0.633333333333333)
--(axis cs:375000,0.6)
--(axis cs:400000,0.533333333333333)
--(axis cs:425000,0.633333333333333)
--(axis cs:450000,0.566666666666667)
--(axis cs:475000,0.7)
--(axis cs:500000,0.6)
--(axis cs:525000,0.733333333333333)
--(axis cs:550000,0.8)
--(axis cs:575000,0.766666666666667)
--(axis cs:600000,0.666666666666667)
--(axis cs:625000,0.7)
--(axis cs:650000,0.6)
--(axis cs:675000,0.633333333333333)
--(axis cs:700000,0.8)
--(axis cs:725000,0.833333333333333)
--(axis cs:750000,0.666666666666667)
--(axis cs:775000,0.633333333333333)
--(axis cs:800000,0.8)
--(axis cs:825000,0.633333333333333)
--(axis cs:850000,0.666666666666667)
--(axis cs:875000,0.766666666666667)
--(axis cs:900000,0.766666666666667)
--(axis cs:925000,0.766666666666667)
--(axis cs:950000,0.8)
--(axis cs:975000,0.7)
--(axis cs:1000000,0.566666666666667)
--(axis cs:1000000,1)
--(axis cs:1000000,1)
--(axis cs:975000,0.933333333333333)
--(axis cs:950000,1)
--(axis cs:925000,0.966666666666667)
--(axis cs:900000,0.966666666666667)
--(axis cs:875000,0.933333333333333)
--(axis cs:850000,0.9)
--(axis cs:825000,0.833333333333333)
--(axis cs:800000,1)
--(axis cs:775000,0.966666666666667)
--(axis cs:750000,0.933333333333333)
--(axis cs:725000,1)
--(axis cs:700000,1)
--(axis cs:675000,0.866666666666667)
--(axis cs:650000,0.9)
--(axis cs:625000,1)
--(axis cs:600000,1)
--(axis cs:575000,0.9)
--(axis cs:550000,0.966666666666667)
--(axis cs:525000,1)
--(axis cs:500000,0.933333333333333)
--(axis cs:475000,0.9)
--(axis cs:450000,0.833333333333333)
--(axis cs:425000,0.966666666666667)
--(axis cs:400000,0.8)
--(axis cs:375000,0.733333333333333)
--(axis cs:350000,0.933333333333333)
--(axis cs:325000,0.9)
--(axis cs:300000,0.9)
--(axis cs:275000,0.8)
--(axis cs:250000,0.7)
--(axis cs:225000,0.766666666666667)
--(axis cs:200000,0.666666666666667)
--(axis cs:175000,0.533333333333333)
--(axis cs:150000,0.566666666666667)
--(axis cs:125000,0.5)
--(axis cs:100000,0.666666666666667)
--(axis cs:75000,0.366666666666667)
--(axis cs:50000,0.5)
--(axis cs:25000,0.133333333333333)
--cycle;

\addplot [line width=\linewidthother, C1, mark=*, mark size=0, mark options={solid}]
table {%
25000 0.0333333333333333
50000 0.333333333333333
75000 0.2
100000 0.433333333333333
125000 0.3
150000 0.433333333333333
175000 0.333333333333333
200000 0.466666666666667
225000 0.566666666666667
250000 0.566666666666667
275000 0.566666666666667
300000 0.733333333333333
325000 0.666666666666667
350000 0.766666666666667
375000 0.633333333333333
400000 0.666666666666667
425000 0.8
450000 0.666666666666667
475000 0.8
500000 0.766666666666667
525000 0.933333333333333
550000 0.866666666666667
575000 0.8
600000 0.866666666666667
625000 0.9
650000 0.733333333333333
675000 0.766666666666667
700000 0.9
725000 0.933333333333333
750000 0.8
775000 0.866666666666667
800000 0.9
825000 0.766666666666667
850000 0.8
875000 0.833333333333333
900000 0.866666666666667
925000 0.866666666666667
950000 0.933333333333333
975000 0.833333333333333
1000000 0.866666666666667
};


% QSM
\path [draw=C3, fill=C3, opacity=0.2]
(axis cs:10000,0)
--(axis cs:10000,0)
--(axis cs:30000,0)
--(axis cs:50000,0)
--(axis cs:70000,0)
--(axis cs:90000,0)
--(axis cs:110000,0)
--(axis cs:130000,0)
--(axis cs:150000,0)
--(axis cs:170000,0)
--(axis cs:190000,0)
--(axis cs:210000,0)
--(axis cs:230000,0)
--(axis cs:250000,0)
--(axis cs:270000,0)
--(axis cs:290000,1.62760416666667e-05)
--(axis cs:310000,4.06901041666667e-06)
--(axis cs:330000,1.01725260416667e-06)
--(axis cs:350000,2.54313151041667e-07)
--(axis cs:370000,3.42528025309245e-05)
--(axis cs:390000,8.56320063273112e-06)
--(axis cs:410000,0.000525106986363729)
--(axis cs:430000,0.00013127771516641)
--(axis cs:450000,3.28191866477331e-05)
--(axis cs:470000,8.20473302155733e-06)
--(axis cs:490000,0.00416871788096614)
--(axis cs:510000,0.00416838378490259)
--(axis cs:530000,0.000909873009186413)
--(axis cs:550000,0.00442720057204497)
--(axis cs:570000,0.00218100551063041)
--(axis cs:590000,0.00578206581332855)
--(axis cs:610000,0.00131123910754809)
--(axis cs:630000,0.00275319438380516)
--(axis cs:650000,0.00223887719663423)
--(axis cs:670000,0.00291346699257468)
--(axis cs:690000,0.0135569958075779)
--(axis cs:710000,0.00346086833160769)
--(axis cs:730000,0.000865217082901922)
--(axis cs:750000,0.00023003459556811)
--(axis cs:770000,5.40760676813701e-05)
--(axis cs:790000,0.000769620175320082)
--(axis cs:810000,0.00019291379818896)
--(axis cs:830000,4.81012609575051e-05)
--(axis cs:850000,1.205711238681e-05)
--(axis cs:870000,1.63288503265118e-05)
--(axis cs:890000,4.08221258162794e-06)
--(axis cs:910000,1.02055314540698e-06)
--(axis cs:930000,2.55262493958909e-07)
--(axis cs:950000,6.37845715879365e-08)
--(axis cs:970000,1.59461428969841e-08)
--(axis cs:990000,3.98842794951141e-09)
--(axis cs:990000,0.0962814468379839)
--(axis cs:990000,0.0962814468379839)
--(axis cs:970000,0.158254372054789)
--(axis cs:950000,0.172170387481672)
--(axis cs:930000,0.193584556345324)
--(axis cs:910000,0.250133692228527)
--(axis cs:890000,0.193699551315851)
--(axis cs:870000,0.183014098833528)
--(axis cs:850000,0.0863881992718888)
--(axis cs:830000,0.126751455986479)
--(axis cs:810000,0.117349110775475)
--(axis cs:790000,0.138980483653325)
--(axis cs:770000,0.125738326103371)
--(axis cs:750000,0.180583290025323)
--(axis cs:730000,0.142718609264413)
--(axis cs:710000,0.0944786319607906)
--(axis cs:690000,0.0930838767463006)
--(axis cs:670000,0.073132660921611)
--(axis cs:650000,0.127741456081621)
--(axis cs:630000,0.112539948512258)
--(axis cs:610000,0.0864812297646582)
--(axis cs:590000,0.0671846943461787)
--(axis cs:570000,0.141235237893088)
--(axis cs:550000,0.129013684327605)
--(axis cs:530000,0.0862313949479235)
--(axis cs:510000,0.0582782346563666)
--(axis cs:490000,0.117036375763128)
--(axis cs:470000,0.106060590053578)
--(axis cs:450000,0.0686259837510685)
--(axis cs:430000,0.0752373695373535)
--(axis cs:410000,0.0517972494165103)
--(axis cs:390000,0.0235402375459671)
--(axis cs:370000,0.0398777604103088)
--(axis cs:350000,0.0283308553695678)
--(axis cs:330000,0.0272120920817057)
--(axis cs:310000,0.0269131978352864)
--(axis cs:290000,0.0176458485921224)
--(axis cs:270000,0.01903076171875)
--(axis cs:250000,0.025390625)
--(axis cs:230000,0.0125651041666667)
--(axis cs:210000,0.0187565104166666)
--(axis cs:190000,0.0208333333333333)
--(axis cs:170000,0.00416666666666667)
--(axis cs:150000,0)
--(axis cs:130000,0)
--(axis cs:110000,0)
--(axis cs:90000,0)
--(axis cs:70000,0)
--(axis cs:50000,0)
--(axis cs:30000,0)
--(axis cs:10000,0)
--cycle;

\addplot [line width=\linewidthother, C3, mark=*, mark size=0, mark options={solid}]
table {%
10000 0
30000 0
50000 0
70000 0
90000 0
110000 0
130000 0
150000 0
170000 0
190000 0
210000 0.003125
230000 0.00104166666666667
250000 0.00442708333333333
270000 0.00271809895833333
290000 0.00396016438802083
310000 0.00251363118489583
330000 0.00618089040120443
350000 0.00254968007405599
370000 0.0128863056500753
390000 0.00979553461074829
410000 0.011330579717954
430000 0.0198290626828869
450000 0.0225831612323721
470000 0.0271258510688009
490000 0.0401147961005336
510000 0.0183620323584667
530000 0.00604149973569292
550000 0.033127522618391
570000 0.0514192604001475
590000 0.0306706850222336
610000 0.0173800978060629
630000 0.0336159259437523
650000 0.0254792294534724
670000 0.0188698073633681
690000 0.0531005793082047
710000 0.0313876962935438
730000 0.0370135907400526
750000 0.0217533976850132
770000 0.0338463719417446
790000 0.0409429206886467
810000 0.0269023968388283
830000 0.0317255992097071
850000 0.0162647331357601
870000 0.0582328499506067
890000 0.047891545820985
910000 0.0619728864552463
930000 0.0613265549471449
950000 0.0653316387367862
970000 0.0371662430175299
990000 0.0259582274210491
};

%CrossQ
\path [draw=C2, fill=C2, opacity=0.2]
(axis cs:1,0)
--(axis cs:1,0)
--(axis cs:24000,0)
--(axis cs:48000,0)
--(axis cs:72000,0)
--(axis cs:96000,0)
--(axis cs:120000,0)
--(axis cs:144000,0)
--(axis cs:168000,0)
--(axis cs:192000,0)
--(axis cs:216000,0)
--(axis cs:240000,0)
--(axis cs:264000,0)
--(axis cs:288000,0)
--(axis cs:312000,0)
--(axis cs:336000,0)
--(axis cs:360000,0)
--(axis cs:384000,0.0666666676600774)
--(axis cs:408000,0.100000001490116)
--(axis cs:432000,0.100000001490116)
--(axis cs:456000,0)
--(axis cs:480000,0.0666666676600774)
--(axis cs:504000,0.100000001490116)
--(axis cs:528000,0.0333333338300387)
--(axis cs:552000,0)
--(axis cs:576000,0.133333335320155)
--(axis cs:600000,0.0333333338300387)
--(axis cs:624000,0.133333335320155)
--(axis cs:648000,0.0333333338300387)
--(axis cs:672000,0.0666666676600774)
--(axis cs:696000,0.133333335320155)
--(axis cs:720000,0.0666666676600774)
--(axis cs:744000,0.0666666676600774)
--(axis cs:768000,0.100000001490116)
--(axis cs:792000,0.166666669150194)
--(axis cs:816000,0.0333333338300387)
--(axis cs:840000,0.0333333338300387)
--(axis cs:864000,0.0333333338300387)
--(axis cs:888000,0.0666666676600774)
--(axis cs:912000,0)
--(axis cs:936000,0.0666666676600774)
--(axis cs:960000,0)
--(axis cs:984000,0)
--(axis cs:984000,0.266666675607363)
--(axis cs:984000,0.266666675607363)
--(axis cs:960000,0.300000004470348)
--(axis cs:936000,0.400000005960464)
--(axis cs:912000,0.266666675607363)
--(axis cs:888000,0.566666687528292)
--(axis cs:864000,0.600000008940697)
--(axis cs:840000,0.533333346247673)
--(axis cs:816000,0.500000014901161)
--(axis cs:792000,0.533333351214727)
--(axis cs:768000,0.43333334227403)
--(axis cs:744000,0.43333334227403)
--(axis cs:720000,0.466666676104069)
--(axis cs:696000,0.533333351214727)
--(axis cs:672000,0.466666668653488)
--(axis cs:648000,0.400000005960464)
--(axis cs:624000,0.500000014901161)
--(axis cs:600000,0.333333338300387)
--(axis cs:576000,0.466666681071122)
--(axis cs:552000,0.433333344757557)
--(axis cs:528000,0.333333340783914)
--(axis cs:504000,0.466666678587596)
--(axis cs:480000,0.333333338300387)
--(axis cs:456000,0.26666667064031)
--(axis cs:432000,0.466666678587596)
--(axis cs:408000,0.333333338300387)
--(axis cs:384000,0.233333336810271)
--(axis cs:360000,0.266666675607363)
--(axis cs:336000,0.200000005463759)
--(axis cs:312000,0.200000002980232)
--(axis cs:288000,0.200000002980232)
--(axis cs:264000,0.233333336810271)
--(axis cs:240000,0.300000004470348)
--(axis cs:216000,0.200000002980232)
--(axis cs:192000,0.200000005463759)
--(axis cs:168000,0.100000001490116)
--(axis cs:144000,0.200000005463759)
--(axis cs:120000,0.133333335320155)
--(axis cs:96000,0.200000002980232)
--(axis cs:72000,0.133333335320155)
--(axis cs:48000,0.100000003973643)
--(axis cs:24000,0.0333333338300387)
--(axis cs:1,0)
--cycle;

\addplot [line width=\linewidthother, C2, mark=*, mark size=0, mark options={solid}]
table {%
1 0
24000 0
48000 0
72000 0.0333333338300387
96000 0.0333333338300387
120000 0.0333333338300387
144000 0.0666666676600774
168000 0
192000 0.0666666676600774
216000 0.100000001490116
240000 0.100000001490116
264000 0.0666666676600774
288000 0.0666666676600774
312000 0.0666666676600774
336000 0.0666666676600774
360000 0.100000001490116
384000 0.166666669150194
408000 0.233333336810271
432000 0.26666667064031
456000 0.100000001490116
480000 0.200000002980232
504000 0.26666667064031
528000 0.166666669150194
552000 0.166666669150194
576000 0.300000004470348
600000 0.166666669150194
624000 0.300000004470348
648000 0.233333336810271
672000 0.200000002980232
696000 0.333333340783914
720000 0.233333336810271
744000 0.26666667064031
768000 0.300000004470348
792000 0.366666674613953
816000 0.233333339293798
840000 0.266666673123837
864000 0.26666667064031
888000 0.333333343267441
912000 0.100000001490116
936000 0.233333336810271
960000 0.133333335320155
984000 0.100000001490116
};

% BRO Fast
\path [draw=C9, fill=C9, opacity=0.2]
(axis cs:25000,0.1)
--(axis cs:25000,0)
--(axis cs:50000,0)
--(axis cs:75000,0.133333333333333)
--(axis cs:100000,0.133333333333333)
--(axis cs:125000,0.0666666666666667)
--(axis cs:150000,0.133333333333333)
--(axis cs:175000,0.266666666666667)
--(axis cs:200000,0.466666666666667)
--(axis cs:225000,0.4)
--(axis cs:250000,0.266666666666667)
--(axis cs:275000,0.4)
--(axis cs:300000,0.366666666666667)
--(axis cs:325000,0.333333333333333)
--(axis cs:350000,0.433333333333333)
--(axis cs:375000,0.333333333333333)
--(axis cs:400000,0.3)
--(axis cs:425000,0.233333333333333)
--(axis cs:450000,0.4)
--(axis cs:475000,0.4)
--(axis cs:500000,0.6)
--(axis cs:525000,0.333333333333333)
--(axis cs:550000,0.533333333333333)
--(axis cs:575000,0.6)
--(axis cs:600000,0.4)
--(axis cs:625000,0.466666666666667)
--(axis cs:650000,0.533333333333333)
--(axis cs:675000,0.466666666666667)
--(axis cs:700000,0.5)
--(axis cs:725000,0.5)
--(axis cs:750000,0.533333333333333)
--(axis cs:775000,0.5)
--(axis cs:800000,0.433333333333333)
--(axis cs:825000,0.466666666666667)
--(axis cs:850000,0.5)
--(axis cs:875000,0.566666666666667)
--(axis cs:900000,0.566666666666667)
--(axis cs:925000,0.366666666666667)
--(axis cs:950000,0.533333333333333)
--(axis cs:975000,0.533333333333333)
--(axis cs:1000000,0.6)
--(axis cs:1000000,0.966666666666667)
--(axis cs:1000000,0.966666666666667)
--(axis cs:975000,0.833333333333333)
--(axis cs:950000,0.833333333333333)
--(axis cs:925000,0.766666666666667)
--(axis cs:900000,0.733333333333333)
--(axis cs:875000,0.833333333333333)
--(axis cs:850000,0.766666666666667)
--(axis cs:825000,0.8)
--(axis cs:800000,0.766666666666667)
--(axis cs:775000,0.7)
--(axis cs:750000,0.766666666666667)
--(axis cs:725000,0.8)
--(axis cs:700000,0.8)
--(axis cs:675000,0.8)
--(axis cs:650000,0.633333333333333)
--(axis cs:625000,0.8)
--(axis cs:600000,0.7)
--(axis cs:575000,0.833333333333333)
--(axis cs:550000,0.9)
--(axis cs:525000,0.6)
--(axis cs:500000,0.833333333333333)
--(axis cs:475000,0.6)
--(axis cs:450000,0.666666666666667)
--(axis cs:425000,0.6)
--(axis cs:400000,0.733333333333333)
--(axis cs:375000,0.7)
--(axis cs:350000,0.666666666666667)
--(axis cs:325000,0.6)
--(axis cs:300000,0.6)
--(axis cs:275000,0.733333333333333)
--(axis cs:250000,0.666666666666667)
--(axis cs:225000,0.566666666666667)
--(axis cs:200000,0.666666666666667)
--(axis cs:175000,0.633333333333333)
--(axis cs:150000,0.466666666666667)
--(axis cs:125000,0.433333333333333)
--(axis cs:100000,0.366666666666667)
--(axis cs:75000,0.366666666666667)
--(axis cs:50000,0.2)
--(axis cs:25000,0.1)
--cycle;

\addplot [line width=\linewidthother, C9, mark=*, mark size=0, mark options={solid}]
table {%
25000 0
50000 0.1
75000 0.266666666666667
100000 0.233333333333333
125000 0.233333333333333
150000 0.266666666666667
175000 0.433333333333333
200000 0.566666666666667
225000 0.433333333333333
250000 0.466666666666667
275000 0.533333333333333
300000 0.466666666666667
325000 0.466666666666667
350000 0.533333333333333
375000 0.533333333333333
400000 0.5
425000 0.366666666666667
450000 0.5
475000 0.5
500000 0.733333333333333
525000 0.533333333333333
550000 0.733333333333333
575000 0.666666666666667
600000 0.566666666666667
625000 0.633333333333333
650000 0.6
675000 0.666666666666667
700000 0.633333333333333
725000 0.7
750000 0.666666666666667
775000 0.6
800000 0.6
825000 0.666666666666667
850000 0.666666666666667
875000 0.766666666666667
900000 0.633333333333333
925000 0.6
950000 0.666666666666667
975000 0.733333333333333
1000000 0.8
};

% DIME %
\path [draw=C0, fill=C0, opacity=0.2]
(axis cs:1,0)
--(axis cs:1,0)
--(axis cs:24000,0)
--(axis cs:48000,0)
--(axis cs:72000,0)
--(axis cs:96000,0)
--(axis cs:120000,0)
--(axis cs:144000,0.166666669150194)
--(axis cs:168000,0.133333335320155)
--(axis cs:192000,0.0666666676600774)
--(axis cs:216000,0.26666667064031)
--(axis cs:240000,0.26666667064031)
--(axis cs:264000,0.233333339293798)
--(axis cs:288000,0.200000002980232)
--(axis cs:312000,0.333333338300387)
--(axis cs:336000,0.200000002980232)
--(axis cs:360000,0.366666672130426)
--(axis cs:384000,0.366666674613953)
--(axis cs:408000,0.300000004470348)
--(axis cs:432000,0.466666678587596)
--(axis cs:456000,0.43333334227403)
--(axis cs:480000,0.466666678587596)
--(axis cs:504000,0.533333341280619)
--(axis cs:528000,0.400000005960464)
--(axis cs:552000,0.566666677594185)
--(axis cs:576000,0.500000014901161)
--(axis cs:600000,0.400000005960464)
--(axis cs:624000,0.200000007947286)
--(axis cs:648000,0.300000006953875)
--(axis cs:672000,0.366666674613953)
--(axis cs:696000,0.400000005960464)
--(axis cs:720000,0.533333346247673)
--(axis cs:744000,0.43333334227403)
--(axis cs:768000,0.43333334227403)
--(axis cs:792000,0.500000014901161)
--(axis cs:816000,0.600000018874804)
--(axis cs:840000,0.500000014901161)
--(axis cs:864000,0.43333334227403)
--(axis cs:888000,0.43333334227403)
--(axis cs:912000,0.400000010927518)
--(axis cs:936000,0.400000005960464)
--(axis cs:960000,0.566666677594185)
--(axis cs:984000,0.366666674613953)
--(axis cs:984000,0.833333343267441)
--(axis cs:984000,0.833333343267441)
--(axis cs:960000,1)
--(axis cs:936000,0.900000005960464)
--(axis cs:912000,0.766666680574417)
--(axis cs:888000,0.766666680574417)
--(axis cs:864000,0.766666680574417)
--(axis cs:840000,0.900000005960464)
--(axis cs:816000,0.800000011920929)
--(axis cs:792000,0.866666674613953)
--(axis cs:768000,0.766666680574417)
--(axis cs:744000,0.733333349227905)
--(axis cs:720000,0.833333343267441)
--(axis cs:696000,0.833333338300387)
--(axis cs:672000,0.766666680574417)
--(axis cs:648000,0.666666686534882)
--(axis cs:624000,0.666666681692004)
--(axis cs:600000,0.666666681567828)
--(axis cs:576000,0.733333349227905)
--(axis cs:552000,0.833333343267441)
--(axis cs:528000,0.733333349227905)
--(axis cs:504000,0.866666674613953)
--(axis cs:480000,0.766666680574417)
--(axis cs:456000,0.833333343267441)
--(axis cs:432000,0.733333349227905)
--(axis cs:408000,0.733333344260852)
--(axis cs:384000,0.566666687528292)
--(axis cs:360000,0.466666673620542)
--(axis cs:336000,0.533333351214727)
--(axis cs:312000,0.600000023841858)
--(axis cs:288000,0.466666678587596)
--(axis cs:264000,0.600000023841858)
--(axis cs:240000,0.566666687528292)
--(axis cs:216000,0.500000014901161)
--(axis cs:192000,0.333333338300387)
--(axis cs:168000,0.533333351214727)
--(axis cs:144000,0.400000013411045)
--(axis cs:120000,0.26666667064031)
--(axis cs:96000,0.400000010927518)
--(axis cs:72000,0.133333335320155)
--(axis cs:48000,0.0333333338300387)
--(axis cs:24000,0)
--(axis cs:1,0)
--cycle;

\addplot [line width=\linewidthdime, C0, mark=*, mark size=0, mark options={solid}]
table {%
1 0
24000 0
48000 0
72000 0
96000 0.133333335320155
120000 0.100000001490116
144000 0.200000002980232
168000 0.366666674613953
192000 0.166666669150194
216000 0.366666672130426
240000 0.433333344757557
264000 0.500000017384688
288000 0.300000004470348
312000 0.500000014901161
336000 0.400000008443991
360000 0.400000005960464
384000 0.466666678587596
408000 0.500000012417634
432000 0.600000018874804
456000 0.60000001390775
480000 0.633333350221316
504000 0.700000017881393
528000 0.566666682561239
552000 0.766666680574417
576000 0.63333335518837
600000 0.500000014901161
624000 0.466666681071122
648000 0.500000017384688
672000 0.60000001390775
696000 0.566666677594185
720000 0.666666686534882
744000 0.600000018874804
768000 0.60000001390775
792000 0.733333349227905
816000 0.700000017881393
840000 0.700000017881393
864000 0.633333350221316
888000 0.633333350221316
912000 0.633333350221316
936000 0.666666676600774
960000 0.833333343267441
984000 0.70000001291434
};
\end{axis}

\end{tikzpicture}
}
       \subcaption[]{Pen Twirl Hard}
       \label{fig::exps_new::myo_hand_pen_twirl_rndm}
    \end{minipage}\hfill
    \vspace{-2mm}
    \caption{\textbf{Training curves on DMC's dog, humanoid tasks, and the hand environments from the MYO Suite.} DIME performs favorably on the high-dimensional dog tasks where it significantly outperforms all baselines (dog-run) or converges faster to the final performance. On the humanoid tasks, DIME outperforms all diffusion-based baselines, CrossQ and BRO Fast, and performs on par with BRO on the humanoid-stand task and slightly worse on the humanoid-run and humanoid-walk tasks. In the MYO SUITE environments, DIME performs consistently on all tasks, either outperforming the baselines or performing on par.}
    \label{fig::exps_new::dmc_tasks_myo}
\end{figure*}

\textbf{DMC: Dog and Humanoid Tasks (Fig. \ref{fig::exps_new::dmc_tasks_myo}).} 
We benchmark on DMC suit's challenging \textit{dog} and \textit{humanoid} environments, where we additionally consider BRO and BRO Fast as a Gaussian-based policy baseline. 
BRO Fast is identical to BRO but differs only in the update-to-data (UTD) ratio of two as DIME and CrossQ. 
Please note that we used the online available learning curves provided by the official implementation for BRO. 
DIME outperforms all baselines significantly on the \textit{dog-run} environment and converges faster to the same end performance on the remaining dog environments (see Fig. \ref{fig::exps_new::dog_run} - \ref{fig::exps_new::dog_stand}). BRO has slightly higher average performance on the \textit{humanoid-run} and \textit{humanoid-walk} (see Fig. \ref{fig::exps_new::humanoid_walk} - \ref{fig::exps_new::humanoid_run})) tasks indicating that DIME performs favorably on more high-dimensional tasks like the dog environments and tasks from the myo suite. However, DIME's asymptotic behavior in the \textit{humanoid-run} achieves slightly higher aggregated performance than BRO, where we have run both algorithms for 3M steps (Fig. \ref{fig::appendix_dime_bro_humanoid_run_long}). 
However, BRO requires full parameter resets leading to performance drops during training and it is run with a UTD ratio of 10 which is 5 times higher than DIME. This leads to longer training times. 
As reported in their paper \cite{nauman2024bigger}, BRO needs an average training time of 8.5h whereas DIME trains in approximately 4.5h with 16 diffusion steps on the humanoid-run with the same Hardware (\textit{Nvidia A100}).  

\textbf{MYO Suite (Fig. \ref{fig::exps_new::dmc_tasks_myo}).} Except for \textit{pen twirl hard} (Fig. \ref{fig::exps_new::myo_hand_pen_twirl_rndm}), DIME consistently outperforms BRO and BRO Fast in that it converges to a higher or faster to the end success rate. 
DIME also consistently outperforms CrossQ in terms of the achieved success rates on all the tasks except for the object hold hard task \ref{fig::exps_new::myo_hand_obj_hold_hard} where DIME converges faster. 


