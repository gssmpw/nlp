%Version 3 December 2023
% See section 11 of the User Manual for version history
%
%%%%%%%%%%%%%%%%%%%%%%%%%%%%%%%%%%%%%%%%%%%%%%%%%%%%%%%%%%%%%%%%%%%%%%
%%                                                                 %%
%% Please do not use \input{...} to include other tex files.       %%
%% Submit your LaTeX manuscript as one .tex document.              %%
%%                                                                 %%
%% All additional figures and files should be attached             %%
%% separately and not embedded in the \TeX\ document itself.       %%
%%                                                                 %%
%%%%%%%%%%%%%%%%%%%%%%%%%%%%%%%%%%%%%%%%%%%%%%%%%%%%%%%%%%%%%%%%%%%%%

% \documentclass[referee,sn-basic]{sn-jnl}% referee option is meant for double line spacing

%%=======================================================%%
%% to print line numbers in the margin use lineno option %%
%%=======================================================%%

%%\documentclass[lineno,sn-basic]{sn-jnl}% Basic Springer Nature Reference Style/Chemistry Reference Style

%%======================================================%%
%% to compile with pdflatex/xelatex use pdflatex option %%
%%======================================================%%

% \documentclass[pdflatex,sn-basic]{sn-jnl}% Basic Springer Nature Reference Style/Chemistry Reference Style


%%Note: the following reference styles support Namedate and Numbered referencing. By default the style follows the most common style. To switch between the options you can add or remove “Numbered” in the optional parenthesis. 
%%The option is available for: sn-basic.bst, sn-vancouver.bst, sn-chicago.bst%  
 
%%\documentclass[pdflatex,sn-nature]{sn-jnl}% Style for submissions to Nature Portfolio journals
%%\documentclass[pdflatex,sn-basic]{sn-jnl}% Basic Springer Nature Reference Style/Chemistry Reference Style
% \documentclass[pdflatex,sn-mathphys-num]{sn-jnl}% Math and Physical Sciences Numbered Reference Style 
%%\documentclass[pdflatex,sn-mathphys-ay]{sn-jnl}% Math and Physical Sciences Author Year Reference Style
%%\documentclass[pdflatex,sn-aps]{sn-jnl}% American Physical Society (APS) Reference Style
%%\documentclass[pdflatex,sn-vancouver,Numbered]{sn-jnl}% Vancouver Reference Style
%%\documentclass[pdflatex,sn-apa]{sn-jnl}% APA Reference Style 
%%\documentclass[pdflatex,sn-chicago]{sn-jnl}% Chicago-based Humanities Reference Style

%%%% Standard Packages
%%<additional latex packages if required can be included here>
\PassOptionsToPackage{table}{xcolor}
\documentclass[default]{sn-jnl}% Default
\usepackage{graphicx}%
\usepackage{multirow}%
\usepackage{amsmath,amssymb,amsfonts}%
\usepackage{amsthm}%
\usepackage{mathrsfs}%
\usepackage[title]{appendix}%
\usepackage{xcolor}%
\usepackage{textcomp}%
\usepackage{manyfoot}%
\usepackage{booktabs}%
\usepackage{algorithm}%
\usepackage{algorithmicx}%
\usepackage{algpseudocode}%
\usepackage{listings}%
\usepackage{caption}


%%%%
% \usepackage{lineno} 

\usepackage[table]{xcolor}
%\usepackage{colortbl}
% \renewcommand\linenumberfont{\normalfont\small}
%%%%%=============================================================================%%%%
%%%%  Remarks: This template is provided to aid authors with the preparation
%%%%  of original research articles intended for submission to journals published 
%%%%  by Springer Nature. The guidance has been prepared in partnership with 
%%%%  production teams to conform to Springer Nature technical requirements. 
%%%%  Editorial and presentation requirements differ among journal portfolios and 
%%%%  research disciplines. You may find sections in this template are irrelevant 
%%%%  to your work and are empowered to omit any such section if allowed by the 
%%%%  journal you intend to submit to. The submission guidelines and policies 
%%%%  of the journal take precedence. A detailed User Manual is available in the 
%%%%  template package for technical guidance.
%%%%%=============================================================================%%%%

%% as per the requirement new theorem styles can be included as shown below
%\theoremstyle{thmstyleone}%
\newtheorem{theorem}{Theorem}%  meant for continuous numbers
%%\newtheorem{theorem}{Theorem}[section]% meant for sectionwise numbers
%% optional argument [theorem] produces theorem numbering sequence instead of independent numbers for Proposition
\newtheorem{proposition}[theorem]{Proposition}% 
%%\newtheorem{proposition}{Proposition}% to get separate numbers for theorem and proposition etc.

%\theoremstyle{thmstyletwo}%
\newtheorem{example}{Example}%
\newtheorem{remark}{Remark}%

%\theoremstyle{thmstylethree}%
\newtheorem{definition}{Definition}%

\definecolor{cusyellow}{HTML}{D8D6C2}
\definecolor{cusyellowl}{HTML}{EBEADE}

\raggedbottom

\begin{document}

\title[Generalizable Cervical Cancer Screening via Large-scale Pretraining and Test-Time Adaptation]{\fontsize{14}{16}\selectfont Generalizable Cervical Cancer Screening via Large-scale Pretraining and Test-Time Adaptation}


\author[1]{\fnm{Hao} \sur{Jiang}}\email{hjiangaz@connect.ust.hk}\equalcont{These authors contributed equally to this work.}

\author[1]{\fnm{Cheng} \sur{Jin}}\email{cjinag@connect.ust.hk}\equalcont{These authors contributed equally to this work.}

\author[2]{\fnm{Huangjing} \sur{Lin}}\email{hjlin@cse.cuhk.edu.hk}\equalcont{These authors contributed equally to this work.}

\author[2,3]{\fnm{Yanning} \sur{Zhou}}\email{amandayzhou@tencent.com}

\author[2]{\fnm{Xi} \sur{Wang}}\email{xiwang@cse.cuhk.edu.hk
}

\author[1]{\fnm{Jiabo} \sur{Ma}}\email{jmabq@connect.ust.hk}

\author[4]{\fnm{Li} \sur{Ding}}\email{dingli6@mail.sysu.edu.cn}

\author[5,6,7]{\fnm{Jun} \sur{Hou}}\email{houjun0709@126.com}

\author[1]{\fnm{Runsheng} \sur{Liu}}\email{rliuar@connect.ust.hk}

\author[1]{\fnm{Zhizhong} \sur{Chai}}\email{zchaiab@connect.ust.hk}

\author[1,8]{\fnm{Luyang} \sur{Luo}}\email{luyang\_luo@hms.harvard.edu}

\author[4]{\fnm{Huijuan} \sur{Shi}}\email{shihj@mail.sysu.edu.cn}

\author[9]{\fnm{Yinling} \sur{Qian}}\email{yl.qian@siat.ac.cn}

\author[9]{\fnm{Qiong} \sur{Wang}}\email{wangqiong@siat.ac.cn}

\author*[5,6,7]{\fnm{Changzhong} \sur{Li}}\email{lichangzhong@163.com}

\author*[4]{\fnm{Anjia} \sur{Han}}\email{hananjia@mail.sysu.edu.cn}

\author*[10]{\fnm{Ronald Cheong Kin} \sur{Chan}}\email{ronaldckchan@cuhk.edu.hk}

\author*[1,11,12,13,14]{\fnm{Hao} \sur{Chen}}\email{jhc@cse.ust.hk}


\affil[1]{\orgdiv{Department of Computer Science and Engineering}, \orgname{The Hong Kong University of Science and Technology}, \orgaddress{\city{Hong Kong SAR}, \country{China}}}

\affil[2]{\orgdiv{Department of Computer Science and Engineering}, \orgname{The Chinese University of Hong Kong}, \orgaddress{\city{Hong Kong SAR}, \country{China}}}

\affil[3]{\orgname{Tencent AI Lab}, \orgaddress{\state{Shenzhen}, \country{China}}}

\affil[4]{\orgdiv{Department of Pathology}, \orgname{The First Affiliated Hospital, Sun Yat-sen University}, \orgaddress{\state{Guangzhou}, \country{China}}}

\affil[5]{\orgdiv{Center of Obstetrics and Gynecology}, \orgname{Peking University Shenzhen Hospital}, \orgaddress{\state{Shenzhen}, \country{China}}}

\affil[6]{\orgdiv{Institute of Obstetrics and Gynecology}, \orgname{Shenzhen PKU-HKUST Medical Center}, \orgaddress{\state{Shenzhen}, \country{China}}}

\affil[7]{\orgname{Shenzhen Key Laboratory on Technology for Early Diagnosis of Major Gynecologic Diseases}, \orgaddress{\state{Shenzhen}, \country{China}}}

\affil[8]{\orgdiv{Department of Biomedical Informatics}, \orgname{Harvard University}, \orgaddress{\country{USA}}}

\affil[9]{\orgdiv{Shenzhen Institute of Advanced Technology}, \orgname{Chinese Academy of Sciences}, \orgaddress{\state{Shenzhen}, \country{China}}}

\affil[10]{\orgdiv{Department of Anatomical and Cellular Pathology}, \orgname{The Chinese University of Hong Kong}, \orgaddress{\city{Hong Kong SAR}, \country{China}}}

\affil[11]{\orgdiv{Department of Chemical and Biological Engineering}, \orgname{The Hong Kong University of Science and Technology}, \orgaddress{\city{Hong Kong SAR}, \country{China}}}

\affil[12]{\orgname{HKUST Shenzhen-Hong Kong Collaborative Innovation Research Institute}, \orgaddress{\state{Shenzhen}, \country{China}}}

\affil[13]{\orgdiv{Division of Life Science}, \orgname{The Hong Kong University of Science and Technology}, \orgaddress{\city{Hong Kong SAR}, \country{China}}}

\affil[14]{\orgdiv{State Key Laboratory of Molecular Neuroscience}, \orgname{The Hong Kong University of Science and Technology}, \orgaddress{\city{Hong Kong SAR}, \country{China}}}


\clearpage

\abstract{Cervical cancer is a leading malignancy in female reproductive system. While AI-assisted cytology offers a cost-effective and non-invasive screening solution, current systems struggle with generalizability in complex clinical scenarios. To address this issue, we introduced \textbf{Smart-CCS}, a generalizable \textbf{C}ervical \textbf{C}ancer \textbf{S}creening paradigm based on pretraining and adaptation to create robust and generalizable screening systems.
To develop and validate Smart-CCS, we first curated a large-scale, multi-center dataset named CCS-127K, which comprises a total of 127,471 cervical cytology whole-slide images collected from 48 medical centers. 
By leveraging large-scale self-supervised pretraining, our CCS models are equipped with strong generalization capability, potentially generalizing across diverse scenarios. Then, we incorporated test-time adaptation to specifically optimize the trained CCS model for complex clinical settings, which adapts and refines predictions, improving real-world applicability. We conducted large-scale system evaluation among various cohorts. In retrospective cohorts, Smart-CCS achieved an overall area under the curve (AUC) value of 0.965 and sensitivity of 0.913 for cancer screening on 11 internal test datasets. In external testing, system performance maintained high at 0.950 AUC across 6 independent test datasets. In prospective cohorts, our Smart-CCS achieved AUCs of 0.947, 0.924, and 0.986 in three prospective centers, respectively. Moreover, the system demonstrated superior sensitivity in diagnosing cervical cancer, confirming the accuracy of our cancer screening results by using histology findings for validation. Interpretability analysis with cell and slide predictions further indicated that the system's decision-making aligns with clinical practice.
Smart-CCS represents a significant advancement in cervical cancer screening and highlights the potential for generalizable screening in real-word practice across diverse clinical contexts.}

\maketitle

\section*{Introduction}\label{sec1}
\section{Introduction}
\label{sec:intro}
%
In this study, we consider semi-supervised anomaly detection (AD) using the $k$-nearest neighbor (NN) approach~\citep{breunig2000lof,ramaswamy2000efficient,mehrotra2017anomaly}.
%
Semi-supervised AD detects anomalies using only normal training instances.
%
In many practical cases, such as industrial AD, anomalous instances are rare, making semi-supervised AD essential.
%
We focus on the $k$-nearest neighbor anomaly detection ($k$NNAD) among various semi-supervised AD methods.
%
The $k$NNAD approach is simple yet effective, offering flexibility, minimal data assumptions, and adaptability to different distance metrics.

An important challenge in semi-supervised AD is quantifying the reliability of detected anomalies~\citep{barnett1994outliers,chandola2009anomaly,montgomery2020introduction}.
%
Without anomalous training instances in training, estimating detection accuracy is challenging.
%
Furthermore, modeling anomaly distributions is difficult since similar anomalies may not occur repeatedly.
%
To address this issue, we formulate semi-supervised $k$NNAD as a statistical test to quantify false AD probability using $p$-values.
%
If the $p$-values are accurately calculated and, anomalies with $p$-values below a desired significance level (e.g., 5\%) can be detected, ensuring that the detected anomalies are statistically significantly different from normal instances in the specified significance level.

However, a critical challenge emerges when formulating semi-supervised AD as a statistical test.
%
The primary issue is conducting both detection and testing of anomalies on the same data, which makes accurate $p$-value calculation intractable. 
%
In traditional statistics, selecting and evaluating a hypothesis on the same data causes selection bias in $p$-values, leading to inaccuracies—a problem known as \emph{double dipping}~\citep{breiman1992little,kriegeskorte2009circular,benjamini2020selective}.
%
In semi-supervised AD, since only normal instances are available, both the detection and evaluation must rely on the same data.
%
Thus, a naive statistical test formulation cannot avoid the double dipping issue.

To address this issue, we employ \emph{Selective Inference (SI)}, a statistical framework gaining attention in the past decade~\citep{fithian2014optimal,taylor2015statistical,lee2016exact}.
%
SI ensures valid statistical inferences after data-driven selection of hypotheses by correcting selection biases, ensuring accurate \( p \)-values and confidence intervals.
%
SI was originally designed to assess feature selection reliability, enabling accurate significance evaluation even when selection and evaluation use the same dataset~\citep{lee2016exact,tibshirani2016exact,duy2022more}.
%
The key principle of SI is to perform statistical inference conditioned on the selected hypothesis.
%
By using conditional probability distributions, SI effectively mitigates selection bias from double dipping.
%
In this study, we propose \emph{Statistically Significant $k$NNAD (Stat-$k$NNAD)}, a method that performs statistical hypothesis test conditioned on anomalies detected by the $k$NNAD algorithm.
%
The Stat-$k$NNAD offers theoretical guarantees and precise quantification of false anomaly detection probability.

Our contributions are summarized as follows.
%
First, we formulate semi-supervised AD using $k$NNAD as a statistical test within the SI framework, enabling accurate reliability quantification of detected anomalies.
%
While $k$NNAD is widely used, no existing method theoretically and accurately quantifies the false identification probability of detected anomalies.
%
Second, to enable conditional inference for $k$NNAD, we decompose it into tractable selection events (linear or quadratic inequalities) within the SI framework, Notably, applying $k$NNAD in deep learning-based latent spaces requires representing complex deep learning operations in a tractable form, posing a significant technical challenge (details in \S~\ref{sec:SI}). 
%
Finally, through experiments on various datasets and industrial product AD, we validate the effectiveness of Stat-$k$NNAD. Specifically, for industrial product images, we show that applying $k$NNAD in the latent space of a pretrained CNN effectively addresses practical challenges.
%
A more comprehensive discussion on related work, as well as the scope and limitations of this work, is presented in \S\ref{sec:relatedWorks}.


\section*{Results}\label{sec2}
\documentclass{article}
\usepackage[margin=1in]{geometry} % Adjust as needed
\usepackage{booktabs}
\usepackage{multirow}
\usepackage{graphicx}

\begin{document}

\section*{Table of Code-Enhanced Reasoning Methods}

\begin{table}[ht!]
\centering
\resizebox{\textwidth}{!}{
\begin{tabular}{@{}lllcccccc@{}}
\toprule
\textbf{Method} & \textbf{Model} & \textbf{Settings} & \textbf{GSM8K} & \textbf{SVAMP} & \textbf{MATH} & \textbf{MultiArith} & \textbf{SingleEq} & \textbf{AddSub} \\
\midrule

% DIRECT
\multirow{1}{*}{\textbf{DIRECT}} 
& Codex & Few-shot Direct Prompting 
& 19.7 & 69.9 & -- & 44.0 & 86.8 & 93.1 \\
\midrule

% CoT
\multirow{10}{*}{\textbf{CoT}} 
& UL2-20B      & Few-shot Chain-of-Thought  & 4.1   & 12.6  & --   & 10.7   & --    & 18.2 \\
& LaMDA-137B   & Few-shot Chain-of-Thought  & 17.1  & 39.9  & --   & 51.8   & --    & 52.9 \\
& Codex        & Few-shot Chain-of-Thought  & 65.6  & 74.8  & --   & 95.9   & 89.1  & 86.0 \\
& PaLM-540B    & Few-shot Chain-of-Thought  & 56.9  & 79.0  & --   & 94.7   & 92.3  & 91.9 \\
& Minerva-540B & Few-shot Chain-of-Thought  & 58.8  & --    & --   & --     & --    & --   \\
& GPT-4        & Few-shot Chain-of-Thought  & 92.0  & 97.0  & --   & --     & --    & --   \\
& GPT-4o-mini  & 0-shot Chain-of-Thought    & --    & --    & 50.60 & --     & --    & --   \\
& Llama3.1-8B  & 0-shot Chain-of-Thought    & --    & --    & 18.28 & --     & --    & --   \\
& -            & 0-shot Chain-of-Thought    & --    & 78.20 & --   & 96.67  & 93.11 & 86.08 \\
& -            & Few-shot Chain-of-Thought  & --    & 77.10 & --   & 98.50  & 95.47 & 90.63 \\
\midrule

% PAL
\multirow{4}{*}{\textbf{PAL}} 
& Codex        & Few-shot Program-aided LM & 72.0  & 79.4  & --   & 99.2   & 96.1  & 92.5 \\
& GPT-4o-mini  & 0-shot Program-aided LM   & --    & --    & 36.56 & --     & --    & --   \\
& Llama3.1-8B  & 0-shot Program-aided LM   & --    & --    & 11.71 & --     & --    & --   \\
& -            & Few-shot Program-aided LM & --    & 79.50 & --   & 97.00  & 97.64 & 89.11 \\
\midrule

% PoT
\multirow{3}{*}{\textbf{PoT}}
& Codex        & Few-shot Program of Thought                      & 71.6  & 85.2 & -- & -- & -- & -- \\
& Codex        & Few-shot Program of Thought + Self-Consistency   & 80.0  & 89.1 & -- & -- & -- & -- \\
& GPT-4        & Few-shot Program of Thought                      & 97.2  & 97.4 & -- & -- & -- & -- \\
\midrule

% MathCoder-L
\multirow{3}{*}{\textbf{MathCoder-L}}
& Llama-2-7B   & 0-shot Code Interleaving  & 64.2 & 71.5 & 23.3 & -- & -- & -- \\
& Llama-2-13B  & 0-shot Code Interleaving  & 72.6 & 76.9 & 29.9 & -- & -- & -- \\
& Llama-2-70B  & 0-shot Code Interleaving  & 83.9 & 84.9 & 45.1 & -- & -- & -- \\
\midrule

% MathCoder-CL
\multirow{3}{*}{\textbf{MathCoder-CL}}
& CodeLlama-7B   & 0-shot Code Interleaving & 67.8 & 70.7 & 30.2 & -- & -- & -- \\
& CodeLlama-13B  & 0-shot Code Interleaving & 74.1 & 78.0 & 35.9 & -- & -- & -- \\
& CodeLlama-34B  & 0-shot Code Interleaving & 81.7 & 82.5 & 45.2 & -- & -- & -- \\
\midrule

% CodePlan
\multirow{3}{*}{\textbf{CodePlan}}
& Mistral-7B     & Few-shot Code-form planning & 59.5 & 61.4 & 34.3 & -- & -- & -- \\
& Llama-2-7B     & Few-shot Code-form planning & 33.8 & 41.5 & 20.8 & -- & -- & -- \\
& Llama-2-13B    & Few-shot Code-form planning & 49.5 & 53.4 & 27.4 & -- & -- & -- \\
\midrule

% CodeNL
\multirow{2}{*}{\textbf{CodeNL}}
& GPT-4o-mini  & 0-shot Code Prompting & -- & -- & 50.90 & -- & -- & -- \\
& Llama3.1-8B  & 0-shot Code Prompting & -- & -- & 14.88 & -- & -- & -- \\
\midrule

% NLCode
\multirow{2}{*}{\textbf{NLCode}}
& GPT-4o-mini  & 0-shot Code Prompting & -- & -- & 47.58 & -- & -- & -- \\
& Llama3.1-8B  & 0-shot Code Prompting & -- & -- & 18.35 & -- & -- & -- \\
\midrule

% INC-Math
\multirow{2}{*}{\textbf{INC-Math}}
& GPT-4o-mini  & 0-shot Code Prompting & -- & -- & 51.44 & -- & -- & -- \\
& Llama3.1-8B  & 0-shot Code Prompting & -- & -- & 16.69 & -- & -- & -- \\
\midrule

% CoC (Python)
\multirow{1}{*}{\textbf{CoC (Python)}}
& text-davinci-003 & Few-shot Code Interleaving with Python Execution 
& -- & -- & -- & -- & -- & -- \\
\midrule

% Code (+self-debug)
\multirow{2}{*}{\textbf{Code (+self-debug)}}
& - & 0-shot Code Prompting with self-debug 
& -- & 79.40 & -- & 96.67 & 97.64 & 91.65 \\
& - & Few-shot Code Prompting with self-debug 
& -- & 79.60 & -- & 97.33 & 97.44 & 91.39 \\
\bottomrule
\end{tabular}
}
\caption{Performance of various code-enhanced reasoning methods on multiple benchmarks. ``--'' indicates no reported result.}
\label{tab:code_reasoning_methods}
\end{table}

\end{document}


\section*{Discussion}\label{sec3}
Cervical cancer has been targeted for global elimination under the WHO initiative \cite{world2020global}. Leveraging AI to assist cytologists can significantly accelerate cervical cancer screening, especially in large-scale precancerous screening conditions. In recent years, several AI-based computational cytology studies preliminarily explored its feasibility, particularly a recent study demonstrated the effectiveness of AI-assisted cytology in cancer identification and tumor origin prediction, highlighting the potential of AI in computational cytology \cite{tian2024prediction,rassy2024predicting, li2024new}. However, previous AI-assisted CCS systems often faced challenges in generalization and robustness across different clinical scenarios, particularly concerning variations in slide preparation and imaging protocols \cite{cheng2021robust,wang2024artificial,zhu2021hybrid,wu2024development}. 

In this study, we introduced Smart-CCS, a generalizable cervical cancer screening paradigm. This pioneering paradigm consists of three stages: self-supervised pretraining, screening model finetuning, and test-time adaptation. It leverages: 1) large-scale self-supervised pretraining for robust cytology representations, 2) effective utilization of both cell-level and slide-level supervision, and 3) model adaptation during clinical evaluation. To support this, we constructed a large, multi-center dataset, CCS-127K, which includes 127,471 cervical cytology WSIs from 48 centers. To the best of our knowledge, this is the first comprehensive paradigm for cervical cancer screening with multi-center and prospective validation.

The experimental results of internal testing, external testing, and prospective studies revealed the extensive effectiveness and potential clinical benefits of the proposed Smart-CCS for generalizing cancer screening. 
Notably, the significant improvements in cell-level and WSI-level downstream tasks demonstrated the efficacy of pretraining in capturing general cytology information. For retrospective evaluation of abnormal cell detection, we compared SOTA detection models and selected the best as the first step in WSI classification model. In terms of internal testing, Smart-CCS achieved an overall AUC of 0.965 (95\% CI: 0.961–0.969) across 11 centers for cancer screening. 
In external testing, Smart-CCS achieved an AUC from 0.880 (95\% CI: 0.859–0.901) to 0.941 (95\% CI: 0.927–0.955)  across 6 external centers with high sensitivities, validating its applicability in diverse clinical scenarios. Additionally, prospective studies in PC1-PC3 yielded high AUC scores of 0.947 (95\% CI: 0.933–0.961), 0.924 (95\% CI: 0.910–0.938), and 0.986 (95\% CI: 0.979–0.993), respectively. Further histological evaluation and interpretability analysis yielded consistent conclusions with clinical knowledge, which highlights the reliability and superiority of the proposed Smart-CCS paradigm.


This study has several limitations. 
First, the availability of large-scale data is critical for pretraining robust and generalizable models. Although we have constructed one of the largest cytology datasets to date, comprising 127,471 WSIs, it remains relatively small compared to datasets commonly used in the histology domain. For instance, Virchow2 was pretrained on 3.1M WSIs \cite{vorontsov2024foundation}, and CONCH utilized 1.17M image-caption pairs \cite{lu2024visual}, leveraging extensive public histology datasets such as TCGA \cite{weinstein2013cancer}. Similarly, recent advancements in cytology have also focused on scaling up data resources to enhance model development \cite{yu2023ai}. To address this limitation, we are actively working to incorporate larger datasets for both pretraining and validation, with the goal of advancing cancer screening and computational cytology.
Second, from a methodological perspective, the ability to distinguish individual cell instances is critical for guiding accurate screening decisions. While we have developed a detection-based WSI classification framework, integrating fine-grained cellular information could further improve the modeling of cell-to-WSI relationships. For example, transitioning from patch-level pretraining to cell-level pretraining or incorporating quantified morphological features, such as cytoplasmic segmentation masks and nucleus-to-cytoplasm ratios, may enhance the performance and interpretability of the Smart-CCS system \cite{zhu2021hybrid}.
Third, regarding system validation, although this study included independent testing across 11 external centers and prospective evaluation in 3 centers, broader deployment and validation in diverse clinical settings are essential to facilitate widespread adoption. Future evaluations should consider a wider range of demographic variables, including ethnicity, age, geographic regions, and variations in slide preparation and imaging protocols, to ensure the system's generalizability and robustness across heterogeneous clinical scenarios.

Ultimately, to establish a trustworthy cancer screening system for large-scale clinical applications, several in-depth exploration directions follow this study. First, establishing a unified cytology WSI benchmark is essential to tackle unique challenges such as cell instance distinguishability and ambiguous categorization of atypical findings (e.g., ASC-US, ASC-H) \cite{jiang2024holistic}.  
Secondly, while Smart-CCS demonstrated the effectiveness for cervical cancer screening, the paradigm has significant potential for broader applications. Generalizing Smart-CCS to other cancers, such as urine \cite{wu2024development}, thyroid \cite{wang2024deep}, and pleural effusion samples \cite{tian2024prediction}, could contribute to a cytology foundation model, enhancing its applicability in computational cytology. Finally, while cytology primarily focuses on cancer screening, large-scale data-driven approaches could expand to more clinical applications, including cancer diagnosis \cite{wu2024development}, survival prognosis \cite{sigel2017cytology,valletti2021gastric}, biomarker prediction \cite{tian2024prediction,rassy2024predicting,li2024new} and molecular-level discovery \cite{caputo2024current, ohori2024molecular}.


In summary, the Smart-CCS paradigm demonstrates strong potential for advancing cervical cancer screening while paving the way for broader applications in computational cytology.



\section*{Methods}\label{sec4}
\section{Statistical Test for Anomaly Candidates}
\label{sec:SI}
%
In this section, we describe a statistical test for calculating $p$-values for potential anomalies identified in the 1st stage.
%
In the 2nd stage, anomalies are identified by selecting only candidates with $p$-values smaller than the significance level $\alpha$ (e.g., 0.05), allowing for appropriate control of the false detection probability.

\subsection{Main Selection Events}
%
The core idea of SI is to conduct statistical inference based on conditional distribution of the test-statistic conditional on the hypothesis selection event.
%
In our problem, it is necessary to consider two selection events to account for the following two facts: 
%
\begin{description}
 %
 \item[SE1] The test statistic in Eq.\eqref{eq:test_statistic} depends on the selection of $k$-nearest neighbors.
       %
 \item[SE2] The anomaly candidates are selected because the anomaly score in Eq.\ref{eq:anomaly_definition} is grater than the threshold $\theta$.
       %
\end{description}

Before discussing these two selection events, we introduce some additional notations. 
%
Let us denote the $(1+n)d$-dimensional vector obtained by concatenating the test instance $\bm x^{\rm test}$ and $n$ training instance $\bm x_1, \ldots, \bm x_n$, all of which are $d$-dimensional vectors, as  
\begin{align}
\bm y = {\rm vec} \left( \bm x^{\rm test}, \bm x_1, \ldots, \bm x_n \right) \in \RR^{(1+n)d},
\end{align}
where ${\rm vec}$ is the operation that concatenates multiple vectors into a single column vector.  
%
Similarly, the $(1+n)d$-dimensional vector obtained by concatenating $1+n$ $d$-dimensional random vectors is denoted as  
\begin{align}
\bm Y = {\rm vec} \left( \bm X^{\rm test}, \bm X_1, \ldots, \bm X_n \right) \in \RR^{(1+n)d}.
\end{align}
%
With these notations, we rewrite the test statistic in Eq.\eqref{eq:test_statistic} as $T(\bm Y) = \left\| \bm X^{\rm test} - \bar{\bm X}^{\text{$k$NN}} \right\|_1$.

\paragraph{SE1: Selection event for $k$ neighbors}
%
Let the index of the $k$th nearest neighbor in the observed data be denoted as $(k)$.
%
The event that specific $k$ training instances are selected as the $k$-nearest neighbors of the test example is expressed as  
\begin{align}
 \label{eq:k_neighbors_condition}
 {\rm dist}(\bm X^{\rm test}, \bm X_{(\bar{k})}) \le {\rm dist}(\bm X^{\rm test}, \bm X_{(\underline{k})})
\end{align}
for all $(\bar{k}, \underline{k}) \in \{1, \ldots, k\} \times \{k+1, \ldots, n\}$.
%
With a slight abuse of notations, we denote the set of indices for the $k$ nearest neighbors of the test instance $\bm X^{\rm test}$ and $n$ training instances $\bm X_1, \ldots, \bm X_n$ sampled from the statistical model in \S\ref{subsec:statistical_model_data} as $\cN_{\bm Y}$ (in \S\ref{sec:setup}, we only denote this as $\cN$).
%
Hereafter, the conditions in Eq.\eqref{eq:k_neighbors_condition} is represented as $\cN_{\bm Y} = \cN_{\bm y}.$

\paragraph{SE2: Selection event for anomaly screening}
%
Since the statistical test in the 2nd stage is performed only on test instances selected in the 1st-stage anomaly screening, it is essential to consider the selection events associated with it.
%
A test instance is selected in the 1st-stage if its anomaly score, as defined in Eq.~\eqref{eq:anomaly_definition}, exceeds a threshold $\theta$.
%
Because the anomaly score in Eq.~\eqref{eq:anomaly_definition} is calculated based on the $k$-the nearest neighbor instance, the condition on the $k$-the nearest neighbor instance must also be incorporated.
%
Specifically, by conditioning on
\begin{align}
 \label{eq:event2_cond1}
 {\rm dist}(\bm X^{\rm test}, \bm X_{(k)})
 \ge
 {\rm dist}(\bm X^{\rm test}, \bm X_{(k^\prime)})
\end{align}
for $k^\prime = 1, \ldots, k-1$, and 
\begin{align}
 \label{eq:event2_cond2}
 {\rm dist}(\bm X^{\rm test}, \bm X_{(k)})
 \le
 {\rm dist}(\bm X^{\rm test}, \bm X_{(k^\prime)})
\end{align}
for $k^\prime = k+1, \ldots, n$, we can consider only cases where the $k$-the nearest neighbor is the same as the observed case.
%
Furthermore, the condition for the anomaly score is written as
\begin{align}
 \label{eq:event2_cond3}
 \log {\rm dist}\left(\bm X^{\rm test}, \bm X_{(k)} \right) - \frac{\log k}{d} \ge \theta.
\end{align}
%
With the conditions in Eqs.\eqref{eq:event2_cond1}-\eqref{eq:event2_cond3}, we can characterize the selection event that the test case $\bm X^{\rm test}$ is selected as an anomaly candidate in the 1st-stage anomaly screening.
%
Hereafter, these conditions are collectively represented as $\cK_{\bm Y} = \cK_{\bm y}.$

\paragraph{SI with main selection events}
%
The SI taking into account the above two types of selection events is performed based on the sampling distribution of the following conditional test-statistic: 
\begin{align}
 \label{eq:cond_dist1}
 T(\bm Y) \mid \left\{\cN_{\bm Y} = \cN_{\bm y}, \cK_{\bm Y} = \cK_{\bm y} \right\}.
\end{align}
%
Performing statistical inference based on the conditional test-statistic in Eq.\eqref{eq:cond_dist1} means that we consider only cases where the randomness of the data $\bm Y$ satisfies $\cN_{\bm Y} = \cN_{\bm y}$ and $\cK_{\bm Y} = \cK_{\bm y}$, which enables us to circumvent the selection bias associated with the above two selection events.

\subsection{Additional Selection Events}
%
To make the computation of SI tractable, it is common in the SI literature to introduce additional selection events besides the main selection events mentioned above.
%
In our problem, it is necessary to introduce the following two additional selection events:
%
\begin{description}
 %
 \item[SE3] A selection event to make the computation of the $L_1$ norm in the test statistic tractable.
 %	     
 \item[SE4] A selection event related to the sufficient statistic for the nuisance component.
 %
\end{description}
%
We note that introducing these additional selection events does not affect the control of the false detection probability, but tends to reduce the power (true detection probability) of the test.

\paragraph{SE3: Selection event for $L_1$ norm.}
%
SI can be applied when the test statistic $T(\bm Y)$ can be expressed as a linear function of the data $\bm Y$.
%
In our problem, the test statistic $T(\bm Y)$ can be expressed as a linear function of $\bm Y$ by introducing additional conditions.
%
Specifically, to remove the absolute value operator in the definition of $L_1$ norm, we fix the sign of each dimension by condition, which can be written as
\begin{align}
\label{eq:event3}
 {\rm sgn}
(
X^{\rm test}_{\cdot j}
-
\bar{X}^{\text{$k$NN}}_{\cdot j}
)
=
{\rm sgn}
(
 x^{\rm test}_{\cdot j} 
-
\bar{x}^{\text{$k$NN}}_{\cdot j}
)
\end{align}
for all $j \in [d]$. 
%
Together with the condition $\cN_{\bm Y} = \cN_{\bm y}$, the test statistic $T(\bm Y)$ can be expressed as a linear function of $\bm Y$ as  
\begin{align}
 T(\bm Y) = \bm \eta^\top \bm Y
\end{align}
using a certain vector $\bm \eta \in \RR^{(1+n)d}$.
%
Hereafter, the condition in Eq.\eqref{eq:event3} is represented as $\cS_{\bm Y} = \cS_{\bm y}.$

\paragraph{SE4: Selection event for nuisance component.}
%
Finally, to make SI tractable, it is necessary to condition on the sufficient statistic of the nuisance component of the test statistic $T(\bm Y) = \bm \eta^\top \bm Y$.
%
Specifically, the nuisance parameter of the test statistic is expressed as
\begin{align}
 \cQ_{\bm Y}
 :=
 \left(
 I_{(1+n)d}
 -
 \frac{
 \tilde{\Sigma} \bm \eta \bm \eta^\top
 }{
 \bm \eta^\top \tilde{\Sigma} \bm \eta
 }
 \right) \bm Y,
\end{align}
where
$\tilde{\Sigma} \in \RR^{(1+n)d \times (1+n)d}$
is a block-diagonal matrix with $\Sigma$ in each $d \times d$ diagonal block.
%
The conditioning on the nuisance component $\cQ_{\bm Y}$ is a standard practice of SI literature to make the computation tractable~\footnote{
For example, $\cQ_{\bm Y}$ corresponds to $\bm z$ defined in \S\ref{sec:relatedWorks}, Eq.(5.2) in the seminal SI paper \citet{lee2016exact}.
}
%
Hereafter, we denote this selection event as $\cQ_{\bm Y} = \cQ_{\bm y}$.

\subsection{Selective $p$-values}
%
By conditioning on
$\cN_{\bm Y} = \cN_{\bm y}$,
$\cK_{\bm Y} = \cK_{\bm y}$,
$\cS_{\bm Y} = \cS_{\bm y}$,
and 
$\cQ_{\bm Y} = \cQ_{\bm y}$,
we can derive the exact sampling distribution of the test statistic $T(\bm Y)$ under null distribution ${\rm H}_0$, which enables us to compute the valid $p$-value.
%
\begin{definition}[Selective $p$-values]
 The selective $p$-value for a test instance $\bm x^{\rm test}$ is defined as
 \begin{align}
  \label{eq:selective_pvalue}
  p_{\rm selective}
  :=
  \PP_{\rm H_0}
  \left(
  T(\bm Y) \ge T(\bm y)
  \middle |
  \begin{gathered}
   \cN_{\bm{Y}} = \cN_{\bm{y}},\\
   \cK_{\bm{Y}} = \cK_{\bm{y}},\\
   \cS_{\bm{Y}} = \cS_{\bm{y}},\\
   \cQ_{\bm{Y}} = \cQ_{\bm{y}}
  \end{gathered}
  \right).
 \end{align}
\end{definition}
%
The selective $p$-values in Eq.\eqref{eq:selective_pvalue} correctly quantifies the false detection probability as formally stated in the following theorem.
%
\begin{theorem}
 \label{theo:main}
 The selective $p$-values defined in Eq.\eqref{eq:selective_pvalue} satisfies
 \begin{align}
  \label{eq:theorem_a}
  \PP_{\rm H_0}
  \left(
  p_{\rm selective}
  \le
  \alpha
  \middle |
  \begin{gathered}
   \cN_{\bm{Y}} = \cN_{\bm{y}},\\
   \cK_{\bm{Y}} = \cK_{\bm{y}},\\
   \cS_{\bm{Y}} = \cS_{\bm{y}},\\
   \cQ_{\bm{Y}} = \cQ_{\bm{y}}
  \end{gathered}
  \right)
  =
  \alpha,
  ~
  \forall \alpha \in (0, 1).
 \end{align}
 %
 Furthermore, the property in Eq.\eqref{eq:theorem_a} indicates the selective $p$-values are valid in the (unconditional) marginal sampling distribution, i.e., 
 \begin{align}
  \label{eq:theorem_b}
  \PP_{\rm H_0}
  \left(
  p_{\rm selective}
  \le
  \alpha
  \right)
  =
  \alpha,
  ~
  \forall \alpha \in (0, 1).
 \end{align}
\end{theorem}
%
The proof of Theorem~\ref{theo:main} is given in Appendix~\ref{app:proof_theo_main}.

\subsection{Selection Event for Data-driven selection of $k$}
%
In the case of the data-driven option for determining the number of neighbors $k$, its effect must also be appropriately considered as a selection event.  
%
For example, consider the scenario where $k_1, \ldots, k_K$ are candidate values for $k$, and the candidate that maximizes the anomaly score in Eq. \eqref{eq:anomaly_definition} is selected.  
%
Let the selected $k \in \{k_1, \ldots, k_K\}$ be denoted as $k^*$.
%
Then, the selection event is simply given by $\log {\rm dist}(\bm x^{\rm test}, \bm x_{(k^*)}) - \frac{\log k^*}{d} \ge \log {\rm dist}(\bm x^{\rm test}, \bm x_{(k_t)}) - \frac{\log k_t}{d}, \forall t \in [K]$.
%
In the case of data-driven option to determine $k$, in addition to the four selection events mentioned above, this event must also be incorporated as an additional condition.

\subsection{Selection Event for Deep Learning Models}
%
When using $k$NNAD with feature representations from a pre-trained deep learning model, the influence of the model should be considered as a selection event.
%
SI for deep learning has been discussed in prior studies, and tools like the software developed by \citet{katsuoka2025si4onnx} facilitate the analysis of selection events in these models.
%
In this study, we employ methods from earlier research to calculate selective $p$-values, taking into account selection events related to deep learning models.
%
The basic idea in these methods involves decomposing the model into components and representing each as a piecewise linear function.
%
For example, operations in a CNN such as convolution, ReLU activation, max pooling, and up-sampling are represented as piecewise linear functions.
%
In the experiment, we utilize the feature representation of a CNN model pre-trained on the ImageNet database.
%
This model is represented precisely as a composition of piecewise linear functions, facilitating accurate computation of selective $p$-values through parametric programming.
%
Details on the selection events concerning the deep learning model are discussed further in Appendix~\ref{app:selection_events_of_dnn}.

\subsection{Computing Selective $p$-values}
\label{subsec:computing_selective_p}
%
Calculating selective $p$-values is complex, but we effectively use methods from existing SI research.
%
We specifically use the parametric programming (pp)-based method from previous studies~\citep{sugiyama2020more,le2021parametric,duy2022more}.
%
In SI, statistical inference is based on the probability measure within the subspace $\mathcal{Z}$ of the data space $\mathbb{R}^{(1+n)d}$ where selection event conditions are met.
%
By conditioning on the selection event for the nuisance component, $\mathcal{Q}_{\bm{Y}} = \mathcal{Q}_{\bm{y}}$, $\mathcal{Z}$ reduces to a one-dimensional subspace.
%
The selection events are formulated as unions of intersections of linear or quadratic inequalities, suitable when using $L_1$ or $L_2$ distances for $k$-nearest neighbors.
%
$\mathcal{Z}$ consists of finite number of intervals along a line in the $(1+n)d$-dimensional space, and the pp-based method systematically enumerates all intervals that meet these conditions.

Since the noise is Gaussian, the test statistic $T(\bm{Y})$ under the null hypothesis ${\rm H}_0$ follows a one-dimensional truncated Gaussian distribution within the subspace $\mathcal{Z}$, comprising finite intervals along a line.
%
The selective $p$-value is calculated as the tail probability of this truncated distribution.
%
Early SI research often simplified calculations by assuming $\mathcal{Z}$ as a single interval under additional conditions, which still controls the false detection probability but reduces detection power.
%
In our problem, a similar simplification can be considered by enforcing $\mathcal{Z}$ to be a single interval.
%
In the experiments in \S\ref{sec:experiments}, we conduct an ablation study comparing this simple approach (denoted as {\tt w/o-pp}) as one of the baselines.




\section{Numerical Experiments}
\label{sec:Experiment}

\subsection{Methods for Comparison}
In our experiments, we compared the proposed method (\texttt{Proposed}) using $p_k^{\text{selective}}$ in~(\ref{psel_z}) with the following methods 
% over-conditioning (\texttt{OC}) method, 
% SI for optimal partitioning with over-conditioning~\citep{duy2020computing} (\texttt{OptSeg-SI-oc}), %OC DP
% SI that removed over-conditionning from OptSeg-SI-oc~\citep{duy2020computing} (\texttt{OptSeg-SI}), %Parametric DP
% naive test (\texttt{Naive}) 
% and Bonferroni correction (\texttt{Bonferroni}), 
% The details of these methods for comparison are provided in the Appendix~\ref{app:methods}.
in terms of type I error rate control and power.
\begin{itemize}
  \item \texttt{OC}: In this method, that is, a simple extension of SI literature to our setting, 
  we consider $p$-values with additional conditioning (over-conditioning) described in Appendix~\ref{subsec:over-conditioning}.
  \item \texttt{OptSeg-SI-oc}, \texttt{OptSeg-SI}~\citep{duy2020computing}: 
  These methods use $p$-values conditioned only on the dynamic programming algorithm, disregarding the conditioning on simulated annealing.
  \item \texttt{Naive}: This method is a conventional statistical inference.
  \item \texttt{Bonferroni}: This method applies Bonferroni correction for multiple testing correction.
\end{itemize}
The details of these comparison methods are provided in Appendix~\ref{Detailed_descriptions_of_comparison_methods}.

\subsection{Synthetic Data Experiments}
\label{subsec:Synthetic_Data_Experiments}
\textbf{Experimental setup.}
In all synthetic experiments, we set window size $M \in \{512, 1024\}$, 
the number of frequencies $D = \left\lfloor\frac{M}{2} \right\rfloor + 1$, 
the length of sequence $N = M \cdot T$, where $T$ was specified for each experiment, 
the sampling rate $f_s = 20480$, 
and each element of mean vector $\bm{s}$ as 
\begin{equation}
  s_n = \sum_{d \in \{d_1, d_2, d_3\}} A_n^{(d)} \sin\left(\omega^{(d)}(n-1)\right) ~~~ (1 \leq n \leq N), \notag
\end{equation}
where frequencies $d_1, d_2, d_3 \in \{0, ..., D-1\}$ were randomly selected without replacement for each simulation, %such that $|\{d_1, d_2, d_3\}| = 3$
$A_n^{(d)}$ was defined for each experiment, 
and $\omega^{(d)} \in \left\{2 \pi (\frac{f_s}{M}) d \, \big{|} \, d = 0, \dots, \left\lfloor\frac{M}{2} \right\rfloor\right\}$.
We used BIC for the choice of penalty parameters $\bm{\beta}$ and $\gamma$ as indicated in Appendix~\ref{app:penalty}, 
and set the parameters of simulated annealing as $c_0^+=1000, \lambda^+=1.5, \chi(c_0)=0.5$ and $\lambda=0.8$ in Section~\ref{subsec:SA}. 
After detecting CP candidates, a CP location $\tau_k^{\text{det}}$ randomly selected from $\bm{\tau}^{\text{det}}$ was tested at the significance level $\alpha = 0.05$.

In the experiments conducted to evaluate the control of type I error rate, 
we generated $1000$ null sequences, which did not contain true CPs in the frequency domain, $\bm{x}=(x_1, \dots, x_N)^\top \sim \mathcal{N}\left(\bm{s}, \sigma^2 I_N\right)$, where $A_n^{(d)} = A^{(d)}$ 
was randomly sampled from $\halfopen{0}{1}$ for $d$ in each simulation, 
and $\sigma=1$, for each $T \in \{40, 60, 80, 100\}$.

Regarding the experiments to compare the power, 
we generated sequences $\bm{x}=(x_1, \dots, x_N)^\top \sim \mathcal{N}\left(\bm{s}, \sigma^2 I_N\right)$, 
where
\begin{equation}
  A_n^{(d)} = 
  \begin{cases}
    A^{(d)} & \left(1 \leq t \leq M \cdot t_1^{(d)}\right) \\
    \vspace{-3mm}\\
    A^{(d)} + \Delta & \left(M \cdot t_1^{(d)} + 1 \leq t \leq M \cdot t_2^{(d)}\right) \\
    \vspace{-3mm}\\
    A^{(d)} + 2\Delta & \left(M \cdot t_2^{(d)}+1 \leq t \leq T\right)
  \end{cases}, \notag
\end{equation}
with $A^{(d_1)}, A^{(d_2)}, A^{(d_3)} \in \halfopen{0}{1}$ which were randomly sampled in each simulation, $\left(t_1^{(d_1)}, t_1^{(d_2)}, t_1^{(d_3)}\right) = (18, 20, 22), \left(t_2^{(d_1)}, t_2^{(d_2)}, t_2^{(d_3)}\right) = (38, 40, 42)$, an intensity of the change $\Delta \in \{0.04, 0.08, 0.12, 0.16\}$ and $\sigma=1$, for $T=60$.
In each case, we ran $1000$ trials. 
Since we tested only when a CP candidate location was correctly detected, the power was defined as follows
\begin{equation}
  \text{Power (or Conditional Power)} = \frac{\# \text{ correctly detected \& rejected}}{\# \text{ correctly detected}}. \notag
\end{equation}
We considered the CP candidate location $\tau_k^{\text{det}}$ to be correctly detected if it satisfied the following two conditions:
\vspace{-3mm}
\begin{itemize}
\item The set $\mathcal{D}^{\text{det}}$ of frequencies containing at least one CP candidate was a subset of $\{d_1, d_2, d_3\}$.
\vspace{-2mm}
\item For $\mathcal{D}^{\text{det}}$ satisfying the above condition, either $\underset{d \in \mathcal{D}^{\text{det}}}{\min}{t_1^{(d)}} \leq \tau_k^{\text{det}} \leq \underset{d \in \mathcal{D}^{\text{det}}}{\max} {t_1^{(d)}}$ or $\underset{d \in \mathcal{D}^{\text{det}}}{\min}{t_2^{(d)}} \leq \tau_k^{\text{det}} \leq \underset{d \in \mathcal{D}^{\text{det}}}{\max} {t_2^{(d)}}$ held, 
      that is, $\tau_k^{\text{det}}$ was detected within the true CP locations for the frequencies in $\mathcal{D}^{\text{det}}$.
\end{itemize}
\vspace{-3mm}

\textbf{Experimental results.}
The results of experiments regarding the control of the type I error rate are shown in Figure~\ref{fig_fpr}.
The \texttt{Proposed}, \texttt{OC}, and \texttt{Bonferroni} successfully controlled the type I error rate below the significance level, 
whereas the \texttt{OptSeg-SI}, \texttt{OptSeg-SI-oc}, and \texttt{Naive} could not. 
That was because the \texttt{OptSeg-SI} and \texttt{OptSeg-SI-oc} 
used $p$-values conditioned only on the dynamic programming algorithm, 
excluding the conditioning on simulated annealing,
and the \texttt{Naive} employed conventional $p$-values without conditioning. 
Since the \texttt{OptSeg-SI}, \texttt{OptSeg-SI-oc}, and \texttt{Naive} failed to control the type I error rate, 
we omitted the analysis of their power.
The results of power experiments are shown in Figure~\ref{fig_tpr}. 
Based on these results, the \texttt{Proposed} was the most powerful of all methods that controlled the type I error rate.
The power of the \texttt{OC} was lower than that of the \texttt{Proposed} due to redundant conditions (see Appendix~\ref{app:truncation} for details). %, which caused the loss of power.
Furthermore, the \texttt{Bonferroni} method had the lowest power because it was a highly conservative approach that accounted for the huge number of all possible hypotheses.
Additionally, we provide the computational time of the \texttt{Proposed} in both experiments and the information on the computer resources in Appendix~\ref{Computational_Time}.
% used in the experiments

\begin{figure}[H]
  \centering
  \begin{minipage}[t]{0.45\hsize}
      \centering
      \includegraphics[width=0.95\textwidth]{figure/exp/tmlr2025_fpr512_skip.pdf}
      \caption*{(a) $M=512$}
  \end{minipage}
  \hfill
  \begin{minipage}[t]{0.45\hsize}
      \centering
      \includegraphics[width=0.95\textwidth]{figure/exp/tmlr2025_fpr1024_skip.pdf}
      \caption*{(b) $M=1024$}
  \end{minipage}
  \caption{Type I Error Rate}
  \label{fig_fpr}
\end{figure}
% \vspace{-5mm}
\begin{figure}[H]
  \centering
  \begin{minipage}[t]{0.45\hsize}
      \centering
      \includegraphics[width=0.95\textwidth]{figure/exp/tmlr2025_tpr512.pdf}
      \caption*{(a) $M=512$}
  \end{minipage}
  \hfill
  \begin{minipage}[t]{0.45\hsize}
      \centering
      \includegraphics[width=0.95\textwidth]{figure/exp/tmlr2025_tpr1024.pdf}
      \caption*{(b) $M=1024$}
  \end{minipage}
  \caption{Power}
  \label{fig_tpr}
\end{figure}

\textbf{Robustness of type I error rate control.}
We also conducted the following experiments to investigate the robustness of the \texttt{Proposed} in terms of type I error rate control.
\begin{itemize}
  \item Unknown noise variance: We considered the case where the variance $\sigma^2$ was estimated from the same data. 
  \item Non-Gaussian noise: We also considered the case where the noise followed the five types of standardized non-Gaussian distributions.
  \item Correlated noise: Furthermore, we considered the sequence whose noise was correlated, i.e., the covariance matrix $\Sigma \neq \sigma^2 I_N$. 
        In this case, although the test statistic did not theoretically follow a $\chi$-distribution, 
        we conducted the hypothesis testing using our proposed framework.
\end{itemize}
These details and results are shown in Appendix~\ref{Robustness_of_Type_I_Error_Rate_Control}.

\subsection{Real Data Experiments}
To demonstrate the practical applicability of the \texttt{Proposed}, we applied the \texttt{Proposed}, \texttt{OC}, and \texttt{Naive} to a real-world dataset.
We used the set No.2 of the IMS bearing dataset, 
which is provided by the Center for Intelligent Maintenance Systems (IMS), University of Cincinnati~\citep{qiu2006wavelet} 
and is available from the Prognostic data repository of NASA~\citep{Lee2007bearing}. 
The experimental apparatus consisted of four identical bearings installed on a common shaft, driven at a constant rotation speed by an AC motor under applied radial loading.
In this dataset, the vibration signals were measured using accelerometers until the outer race of bearing~1 failed at the end of the experiment, 
as shown in Figure~\ref{fig_bearing_signals}. 
The analysis for the time signal of bearing~1 in the frequency domain had revealed that anomalies were detected in the harmonics of the characteristic frequency ($236$ Hz) associated with the outer race failure (Ball Pass Frequency Outer race, BPFO) on 3--4~days~\citep{gousseau2016analysis}. 
Based on this previous study, we conducted CP candidate selection for the sensor data of bearing~1 in the frequency domain (1400--4000 Hz) for two periods: 
0.25--2.25~days when no anomalies existed 
and 2.25--4.25~days when the BPFO harmonics exhibited anomalous patterns. 
Subsequently, we tested the detected CP candidate locations to evaluate whether each of them was a genuine anomaly.
Since the signal with 20480 samples per second had been recorded every 10 minutes, 
we computed the DFT of $M = 1024$ consecutive points in the 20480 samples and repeated the procedure $T = 288$ times. %$N = MT$ 
Additionally, the variance $\sigma^2$ was estimated from the data on 0--0.25~days that was not used in all experiments.
The results for the signal of bearing~1 on 0.25--2.25~days and 2.25--4.25~days are shown in Figure~\ref{fig_bearing1}.
In panel~(a), the time variation of a frequency spectrum (1920 Hz) where a CP candidate location was falsely detected is shown for the period of 0.25--2.25~days when the frequency anomaly did not actually exist. 
It shows that $p$-values of the \texttt{Proposed} and \texttt{OC} are above the significance level $0.05$, 
and therefore the result provides the validity of the inference, 
while $p$-value of the \texttt{Naive} is too small. 
In panel~(b), the time variations of the 8th and 15th harmonics of the BPFO where CP candidate locations were correctly detected are presented for the period of 2.25--4.25~days when the frequency anomaly truly existed.
In this case, $p$-values of the \texttt{Proposed} are below the significance level $\frac{0.05}{2} = 0.025$ decided by Bonferroni correction,
thus it indicates that the inference is valid. 
In contrast, $p$-values of the \texttt{OC} are too large, 
due to the loss of power caused by the redundant conditions. 
In addition, since even the time sequences of the healthy bearings~2, 3, and 4 had been reported to indicate frequency characteristics associated with the outer race fault in bearing~1~\citep{gousseau2016analysis}, 
we performed the same analysis for the three signals.
The results are shown in Appendix~\ref{More_Results_on_Real_Data_Experiment}.
% 記載した 𝑝 値は, その周辺の周波数成分や, 15, 16 次以外の高調波の変化点も考慮した値となっていることに注意されたい.

\begin{figure}[t]
  \centering
  \begin{minipage}[t]{0.24\hsize}
      \centering
      \includegraphics[width=0.95\textwidth]{figure/exp/tmlr2025_bearing1.pdf}
      \captionsetup{justification=centering}
      \caption*{(a) Bearing 1}
  \end{minipage}
  \begin{minipage}[t]{0.24\hsize}
    \centering
    \includegraphics[width=0.95\textwidth]{figure/exp/tmlr2025_bearing2.pdf}
    \captionsetup{justification=centering}
    \caption*{(b) Bearing 2}
  \end{minipage}
  \begin{minipage}[t]{0.24\hsize}
    \centering
    \includegraphics[width=0.95\textwidth]{figure/exp/tmlr2025_bearing3.pdf}
    \captionsetup{justification=centering}
    \caption*{(c) Bearing 3}
  \end{minipage}
  \begin{minipage}[t]{0.24\hsize}
    \centering
    \includegraphics[width=0.95\textwidth]{figure/exp/tmlr2025_bearing4.pdf}
    \captionsetup{justification=centering}
    \caption*{(d) Bearing 4}
  \end{minipage}
  \caption{Vibration signals of the four bearings.}
  \label{fig_bearing_signals}
\end{figure}

\begin{figure}[t]
  \centering
  \begin{minipage}[t]{0.4\hsize}
    \centering
    \includegraphics[width=0.95\textwidth]{figure/exp/tmlr2025_bearing1_fpr.pdf}
    \caption*{(a) Inference on a falsely detected CP candidate location for 1920 Hz (around the 8th harmonic) on 0.25--2.25 days}
  \end{minipage}
  \hfill
  \begin{minipage}[t]{0.4\hsize}
    \centering
    \includegraphics[width=0.95\textwidth]{figure/exp/tmlr2025_bearing1_tpr.pdf}
    \caption*{(b) Inferences on truely detected CP candidate locations for 1880 Hz (the 8th harmonic, above) and 3540 Hz (the 15th harmonic, below) on 2.25--4.25~days}
  \end{minipage}
  \caption{Results of the CP candidate selection for the signal of bearing~1 in the frequency domain and the subsequent inference for the detected CP candidate locations.
  In panel (b), note that $p$-values for the first CP candidate location were actually computed by considering not only a CP candidate of the 15th harmonic but also a CP candidate of 1900 Hz (around the 8th harmonic).}
  \label{fig_bearing1}
\end{figure}


\bibliographystyle{unsrt}
\bibliography{sn-bibliography}

\begin{appendices}
\section{Conclusion}
\label{sec:conclusion}

In this paper, we developed a statistical inference method to quantify the reliability of detected CP locations in the frequency domain. 
%
Our proposed framework contributes to various fields where time-frequency analysis is widely employed, such as condition monitoring of machine systems using vibration, electrical, and acoustic signals, and medical diagnosis based on biosignals.
%
We conducted comprehensive experiments on both synthetic and real-world datasets. 
%
The results theoretically confirmed that our method provided an unbiased evaluation based on SI framework and demonstrated its superior performance compared to existing mothods.
%
As future works, we will extend our method to the case of multi-dimensional sequences. 
%
For instance, analyzing different types of time series obtained from multiple sensors would reveal anomalies unattainable through the univariate approach, and reliability guarantee for the detections also provide a valuable future contribution. 

\end{appendices}



\end{document}