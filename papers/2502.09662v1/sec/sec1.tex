Cervical cancer, which originates in the cervix, is the fourth most common cancer among women \cite{siegel2024cancer, siegel2023cancer, cohen2019cervical, denny2024cervical}. 
To combat this global health challenge, the World Health Organization (WHO) has announced the cervical cancer elimination initiative, with the objective of 70\% screening coverage among women aged 35-45 worldwide \cite{world2020global}.
Besides, the 5-year relative survival rate for early-stage cervical cancer is 91\%, while the rate significantly drops to 19\% for the late metastatic stage \cite{ref_NCI1}. Therefore, cervical cancer is highly preventable and curable, with early detection being crucial for effective treatment. 

Cytological examination provides a non-invasive, effective, and cost-efficient alternative, making it particularly suitable for widespread screening of precancerous conditions. It is recommended for the early detection of precancerous neoplasia and cervical cancer \cite{kim2023smart,ouh2021discrepancy}. The typical workflow of cervical cancer screening (CCS) is shown in Extended Data Fig. \ref{F1_intro}(a), where cytologists examine cervical cells and identify suspicious cells using a microscope or digital slides. Reporting of cervical cytology results is guided by the Bethesda System (TBS) \cite{nayar2015bethesda}, which provides widely accepted guidelines for standardized manual interpretation. 

However, several challenges affect the efficacy of manual CCS, summarized in Extended Data Fig. \ref{F1_intro}(b). First, cell-level identification is inherently challenging due to cytomorphology similarities with different cell types (squamous and glandular), locations (superficial, intermediate, and basal), and neoplasia (metaplastic, koilocytotic, dyskeratotic) \cite{plissiti2018sipakmed}. Second, each specimen typically contains 20,000 to 50,000 cells with sparsely distributed lesion cells \cite{lin2021dual}, and digitized slides can measure up to 100,000 $\times$ 100,000 pixels. Therefore, examining cytology specimens is tedious and time-consuming for cytologists. Third, patient-level screening results can be significantly affected by the specimen quality and cytologist experience, which can lead to lower reproducibility and objectivity.

With recent advancements, artificial intelligence (AI) is expected to revolutionize the field of computational cytology \cite{jiang2023deep}. Leveraging advanced neural network architectures such as CNNs, RNNs, and Transformers, AI has achieved promising results in several typical cytology tasks, including classification (e.g., HErlev \cite{jantzen2005pap}, SIPaKMeD \cite{plissiti2018sipakmed}), detection (e.g., CERVIX93 \cite{phoulady2018new}), and segmentation (e.g., ISBI \cite{lu2015improved}, CPS \cite{jiang2023donet,liu2024gains}). Even beyond these tasks, recent studies have demonstrated the efficacy of AI-powered cytology analysis in tumor identification and origin prediction \cite{tian2024prediction}.

In the context of AI-assisted cervical cancer screening (CCS), whole slide image (WSI) analysis has emerged as a promising solution through abnormal cell identification and slide-level classification. Unlike histology, where tissues are continuous and regional, cytology involves discrete cell objects and scattered diagnostic clues, necessitating tailored WSI analysis approaches \cite{jiang2023systematic}. Early efforts, such as the dual-path network for cell-level prediction and rule-based WSI classifiers \cite{lin2021dual}, laid the foundation in this field. 
Following this, subsequent studies incorporated more specialized knowledge to provide enriched cytology characteristics and guidance, thereby improving screening performance. For example, Cheng et al. designed a multi-resolution strategy to learn low-resolution patch features, followed by the refinement of high-resolution cell features \cite{cheng2021robust}. Another study encapsulated the statistics of detected cells and then trained a WSI classifier \cite{wang2024artificial}. Additionally, some studies introduced refined segmentation of the nucleus and cytoplasm to characterize cytology morphological semantics \cite{zhu2021hybrid,yu2023ai}.
These methods generally follow a two-step CCS scheme: abnormal cell detection, where high-confidence candidates are identified, and slide-level aggregation for final prediction, illustrated in Extended Data Fig. \ref{F1_intro}(c). While effective on high-quality data \cite{lin2021dual, wang2024artificial}, this CCS scheme faces significant challenges in real-world scenarios. Variations in patient demographics, sample preparation, and staining protocols across institutions often result in inconsistencies between training data and clinical evaluation, leading to a dramatic decline in screening performance \cite{cheng2021robust,zhu2021hybrid}. Addressing these challenges is critical to improve the robustness and generalizability of AI-driven solutions for cancer screening.

Building on the recent success of large-scale pretraining and downstream adaptation, which demonstrates strong generalization and adaptation capabilities \cite{chen2024towards,ma2024towards}, we explore this insight to develop tailored pretraining strategies for computational cytology. 
Firstly, this study proposes the \textbf{Smart-CCS}, a generalizable \textbf{C}ervical \textbf{C}ancer \textbf{S}creening paradigm. 
This paradigm works under a comprehensive strategy that integrates large-scale self-supervised pretraining to capture generalizable feature representations, finetunes the task-specific WSI classification model for cancer screening, and incorporates test-time adaptation to further optimize performance across diverse clinical settings. 
Then, we curated one of the largest multi-center cervical cytology datasets to develop this Smart-CCS system. Finally, the system was retrospectively and prospectively evaluated to demonstrate its clinical applicability and reliability. To the best of our knowledge, Smart-CCS represents the first comprehensive CCS paradigm involving pretraining and adaptation in computational cytology.
