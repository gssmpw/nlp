Cervical cancer has been targeted for global elimination under the WHO initiative \cite{world2020global}. Leveraging AI to assist cytologists can significantly accelerate cervical cancer screening, especially in large-scale precancerous screening conditions. In recent years, several AI-based computational cytology studies preliminarily explored its feasibility, particularly a recent study demonstrated the effectiveness of AI-assisted cytology in cancer identification and tumor origin prediction, highlighting the potential of AI in computational cytology \cite{tian2024prediction,rassy2024predicting, li2024new}. However, previous AI-assisted CCS systems often faced challenges in generalization and robustness across different clinical scenarios, particularly concerning variations in slide preparation and imaging protocols \cite{cheng2021robust,wang2024artificial,zhu2021hybrid,wu2024development}. 

In this study, we introduced Smart-CCS, a generalizable cervical cancer screening paradigm. This pioneering paradigm consists of three stages: self-supervised pretraining, screening model finetuning, and test-time adaptation. It leverages: 1) large-scale self-supervised pretraining for robust cytology representations, 2) effective utilization of both cell-level and slide-level supervision, and 3) model adaptation during clinical evaluation. To support this, we constructed a large, multi-center dataset, CCS-127K, which includes 127,471 cervical cytology WSIs from 48 centers. To the best of our knowledge, this is the first comprehensive paradigm for cervical cancer screening with multi-center and prospective validation.

The experimental results of internal testing, external testing, and prospective studies revealed the extensive effectiveness and potential clinical benefits of the proposed Smart-CCS for generalizing cancer screening. 
Notably, the significant improvements in cell-level and WSI-level downstream tasks demonstrated the efficacy of pretraining in capturing general cytology information. For retrospective evaluation of abnormal cell detection, we compared SOTA detection models and selected the best as the first step in WSI classification model. In terms of internal testing, Smart-CCS achieved an overall AUC of 0.965 (95\% CI: 0.961–0.969) across 11 centers for cancer screening. 
In external testing, Smart-CCS achieved an AUC from 0.880 (95\% CI: 0.859–0.901) to 0.941 (95\% CI: 0.927–0.955)  across 6 external centers with high sensitivities, validating its applicability in diverse clinical scenarios. Additionally, prospective studies in PC1-PC3 yielded high AUC scores of 0.947 (95\% CI: 0.933–0.961), 0.924 (95\% CI: 0.910–0.938), and 0.986 (95\% CI: 0.979–0.993), respectively. Further histological evaluation and interpretability analysis yielded consistent conclusions with clinical knowledge, which highlights the reliability and superiority of the proposed Smart-CCS paradigm.


This study has several limitations. 
First, the availability of large-scale data is critical for pretraining robust and generalizable models. Although we have constructed one of the largest cytology datasets to date, comprising 127,471 WSIs, it remains relatively small compared to datasets commonly used in the histology domain. For instance, Virchow2 was pretrained on 3.1M WSIs \cite{vorontsov2024foundation}, and CONCH utilized 1.17M image-caption pairs \cite{lu2024visual}, leveraging extensive public histology datasets such as TCGA \cite{weinstein2013cancer}. Similarly, recent advancements in cytology have also focused on scaling up data resources to enhance model development \cite{yu2023ai}. To address this limitation, we are actively working to incorporate larger datasets for both pretraining and validation, with the goal of advancing cancer screening and computational cytology.
Second, from a methodological perspective, the ability to distinguish individual cell instances is critical for guiding accurate screening decisions. While we have developed a detection-based WSI classification framework, integrating fine-grained cellular information could further improve the modeling of cell-to-WSI relationships. For example, transitioning from patch-level pretraining to cell-level pretraining or incorporating quantified morphological features, such as cytoplasmic segmentation masks and nucleus-to-cytoplasm ratios, may enhance the performance and interpretability of the Smart-CCS system \cite{zhu2021hybrid}.
Third, regarding system validation, although this study included independent testing across 11 external centers and prospective evaluation in 3 centers, broader deployment and validation in diverse clinical settings are essential to facilitate widespread adoption. Future evaluations should consider a wider range of demographic variables, including ethnicity, age, geographic regions, and variations in slide preparation and imaging protocols, to ensure the system's generalizability and robustness across heterogeneous clinical scenarios.

Ultimately, to establish a trustworthy cancer screening system for large-scale clinical applications, several in-depth exploration directions follow this study. First, establishing a unified cytology WSI benchmark is essential to tackle unique challenges such as cell instance distinguishability and ambiguous categorization of atypical findings (e.g., ASC-US, ASC-H) \cite{jiang2024holistic}.  
Secondly, while Smart-CCS demonstrated the effectiveness for cervical cancer screening, the paradigm has significant potential for broader applications. Generalizing Smart-CCS to other cancers, such as urine \cite{wu2024development}, thyroid \cite{wang2024deep}, and pleural effusion samples \cite{tian2024prediction}, could contribute to a cytology foundation model, enhancing its applicability in computational cytology. Finally, while cytology primarily focuses on cancer screening, large-scale data-driven approaches could expand to more clinical applications, including cancer diagnosis \cite{wu2024development}, survival prognosis \cite{sigel2017cytology,valletti2021gastric}, biomarker prediction \cite{tian2024prediction,rassy2024predicting,li2024new} and molecular-level discovery \cite{caputo2024current, ohori2024molecular}.


In summary, the Smart-CCS paradigm demonstrates strong potential for advancing cervical cancer screening while paving the way for broader applications in computational cytology.

