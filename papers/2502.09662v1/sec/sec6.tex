%%%%%%%%%%%%%%%%%%%%%%%%%%%%%%%%%%%%%%%%%%%%%%%%%%%%%%%%%%%%%%%%%%%%%%%%%%%%%%%%%%%%%%%%
\newpage
\section*{Extended Data}\label{secA1}

\setcounter{figure}{0}
\begin{figure*}[h]
\renewcommand{\figurename}{Extended Data Fig.}
    \centering
    \includegraphics[width=1.0\linewidth]{fig/F1_v6.pdf}
    \caption{\textbf{Overview of cervical cancer screening and computational cytology.} \textbf{a}. Illustration of cervical cells infected with human papillomavirus (HPV), leading to cervical cancer. The cytology sample collection involves sampling, centrifugation, staining, imaging, for cytologist examinations with screening reports. \textbf{b}. Key challenges in CCS include cytomorphology similarity, sparse abnormal cell distribution, identifying abnormal cells in gigapixel-sized whole slide images (WSI), and data variability. \textbf{c}. A general AI-assisted cancer screening pipeline, comprising a cell detector and a slide classifier, provides quantitative and visualized predictions for both cell-level and slide-level screening.}
    \label{F1_intro}
\end{figure*}


\begin{figure*}[h]
\renewcommand{\figurename}{Extended Data Fig.}
    \centering
    \includegraphics[width=1.0\linewidth]{fig/SF_method.pdf}
    \caption{\textbf{The conceptual illustration of proposed Smart-CCS paradigm.} It consists of three sequential stages: 1) large-scale self-supervised pretraining, 2) CCS model finetuning, and 3) test-time adaptation. }
    \label{SF_method}
\end{figure*}

\clearpage
\begin{figure*}[h]
\renewcommand{\figurename}{Extended Data Fig.}
    \centering
    \includegraphics[width=1.0\linewidth]{fig/SF_tsne.pdf}
    \caption{\textbf{Visualization of the effectiveness of pretraining from SIPaKMeD \cite{plissiti2018sipakmed} data using t-SNE, showcasing cervical cell images and feature distributions both without and with pretraining.} Five colors (dark blue, light blue, light orange, deep orange, green) denote five cell categories: superficial-intermediate (normal), parabasal (normal), metaplastic (benign), koilocytotic (abnormal), dyskeratotic (abnormal). Cell images are extracted with LVD-142M pretrained ViT-Large and our CCS-127K pretrained ViT-Large encoder to obtain 1024-dimensional features, subsequently reduced by t-SNE and plotted as a scatter plot. }
    \label{SF_tsne}
\end{figure*}

\clearpage
\begin{figure*}[h]
\renewcommand{\figurename}{Extended Data Fig.}
    \centering
    \includegraphics[width=1.0\linewidth]{fig/SF_det_v3.pdf}
    \caption{\textbf{Cervical cell detection and classification performance.} a) Performance comparison of per-class AP50 of detectors, demonstrating consistent performance improvement across all categories. b) Confusion matrix illustrating the classification performance of detectors, showing distinguishability between cell types such as glandular cells and squamous cells, as well as category correlations between ASC-US and LSIL, ASC-H and HSIL.}
    \label{SF_det}
\end{figure*}

\clearpage
\begin{figure*}[h]
\renewcommand{\figurename}{Extended Data Fig.}
    \centering
    \includegraphics[width=1.0\linewidth]{fig/SF2_platform.pdf}
    \caption{\textbf{The integrated and interactive interface of the developed cervical cancer screening system.} This integrated CCS system provides both local and global views of cytology samples, highlighting screened suspicious cells (LSIL, HSIL, etc.) and ultimately delivering TBS diagnostic suggestions (ASC-H).
    }
    \label{SF_integrate}
\end{figure*}


\clearpage
\begin{figure*}[h]
\renewcommand{\figurename}{Extended Data Fig.}
    \centering
    \includegraphics[width=1.0\linewidth]{fig/SF8_patching.pdf}
    \caption{\textbf{Whole slide image preprocessing.} Central circular foreground is extracted from the whole-slide image (69,888 pixels in the given example) using CLAM toolkit \cite{lu2021data}, and then cut into 1,200$\times$1,200 patches for further processing. }
    \label{SF_preprocessing}
\end{figure*}


\clearpage
\begin{figure*}[h]
\renewcommand{\figurename}{Extended Data Fig.}
    \centering
    \includegraphics[width=1.0\linewidth]{fig/SF9_qualitycontrol_v4.pdf}
    \caption{\textbf{Workflow of quality control for WSI.} Quality control for WSI involves considerations of cellularity ($>$5,000 per slide \cite{nayar2015bethesda}), preparation, staining, and scanning. Typical samples that are excluded include poor staining quality, out-of-focus, artifacts, impurities, image corruption, dried specimens, and interfering labeling.}
    \label{SF_qualitycontrol}
\end{figure*}

\clearpage
\begin{table}[h] 
\renewcommand{\arraystretch}{1.5}
\renewcommand{\tablename}{Extended Data Table.}
\centering 
\caption{\textbf{Abbreviations and nomenclature in this study.}}
\begin{tabular}{c|c} 
\hline
\rowcolor{cusyellow} \textbf{Abbreviations} & \textbf{Nomenclature} \\ 
\hline
 CC & Cervical cancer \\ 
\rowcolor{cusyellowl} CCS & Cervical cancer screening \\ 
 NILM & Negative for intraepithelial lesion or malignancy \\ 
\rowcolor{cusyellowl} ASC-US & Atypical squamous cells of undetermined significance \\ 
 LSIL & Low–grade squamous intraepithelial lesion \\ 
\rowcolor{cusyellowl} ASC-H & Atypical squamous cells cannot exclude an HSIL \\ 
 HSIL & High–grade squamous intraepithelial lesions \\ 
\rowcolor{cusyellowl} SCC & Squamous cell carcinoma \\ 
 AGC & Atypical glandular cells \\ 
\rowcolor{cusyellowl} AGC-FN & Atypical glandular cells, favor neoplastic \\ 
 AGC-NOS & Atypical glandular cells, not otherwise specified \\ 
\rowcolor{cusyellowl} AIS & Adenocarcinoma in situ \\ 
 ADC & Adenocarcinoma \\ 
\rowcolor{cusyellowl} TBS & The Bethesda system \\ 
 SIL & Squamous intraepithelial lesion \\ 
\rowcolor{cusyellowl} HPV & Human papillomavirus \\ 
 ECA & Epithelial cell abnormalities \\ 
\rowcolor{cusyellowl} CIN & Cervical intraepithelial neoplasia \\ 
\hline 
\end{tabular} 
\label{ST_abb}
\end{table}


\clearpage
\begin{table}[h] 
\renewcommand{\arraystretch}{1.5}
\renewcommand{\tablename}{Extended Data Table.}
\centering 
\caption{\textbf{Investigation of the effectiveness and scaling laws of self-supervised pretraining for cell classification tasks (Top-1 accuracy)}. Three cell classification datasets are evaluated, SIPaKMeD\cite{plissiti2018sipakmed}, HErlev \cite{jantzen2005pap}, and CCS-Cell. The best results are highlighted in bold.}
\begin{tabular}{c|ccc} 
\hline
\rowcolor{cusyellow} \textbf{Pretrain settings }& \textbf{SIPaKMeD} &\textbf{HErlev} & \textbf{CCS-Cell} \\ 
\hline
0M (w/o Pretrain)&0.926 (0.908 - 0.944) & 0.842 (0.789 - 0.895)&0.827 (0.810 - 0.844)\\
\rowcolor{cusyellowl} 1M& 0.936 (0.919 - 0.953) & 0.887 (0.841 - 0.933) & 0.859 (0.843 - 0.875) \\
6M&0.937 (0.920 - 0.954) & 0.890 (0.845 - 0.935)&0.863 (0.847 - 0.879)\\
\rowcolor{cusyellowl} 30M&0.940 (0.924 - 0.956) & 0.895 (0.851 - 0.939)&0.872 (0.857 - 0.887)\\
60M&0.942 (0.926 - 0.958) & 0.905 (0.863 - 0.947)&\textbf{0.883 (0.868 - 0.898)}\\
\rowcolor{cusyellowl} 100M&\textbf{0.960 (0.946 - 0.974)} & \textbf{0.914 (0.873 - 0.955)}&\textbf{0.883 (0.868 - 0.898)}\\
\hline
\end{tabular} 
\label{ST_pretrain_cell}
\end{table}

\clearpage
\begin{table}[h] 
\renewcommand{\arraystretch}{1.5}
\renewcommand{\tablename}{Extended Data Table.}
\centering 
\caption{\textbf{Investigation of the effectiveness and scaling laws of self-supervised pretraining for WSI classification tasks (Top-1 accuracy)}. Three retrospective centers are included, and we report the overall fine-grained classification results (ALL) along with the performance of different groups (ECA, ASC-US+, LSIL+, and HSIL+).}
\begin{tabular}{cc|ccc} 
\hline
\rowcolor{cusyellow} \multicolumn{2}{c|}{\multirow{1}{*}{\textbf{Pretrain settings}}} & \textbf{CCS-127K RC2}& \textbf{CCS-127K RC4} & \textbf{CCS-127K RC5} \\ 
\hline
\multirow{1}{*}{0M }&ECA &0.945 (0.936 - 0.954) &0.857 (0.840 - 0.873)& 0.712 (0.684 - 0.740)\\
& ASC-US+  & 0.940 (0.930 - 0.949) & 0.841 (0.824 - 0.858) & 0.692 (0.664 - 0.721)\\
& LSIL+ & 0.949 (0.940 - 0.958) & 0.842 (0.824 - 0.859) & 0.788 (0.763 - 0.813) \\
& HSIL+ & 0.986 (0.981 - 0.991) & 0.932 (0.920 - 0.944) & 0.887 (0.868 - 0.906) \\
& ALL & 0.895 (0.883 - 0.908) & 0.687 (0.665 - 0.709) & 0.638 (0.608 - 0.667) \\

 \rowcolor{cusyellowl} \multirow{1}{*}{1M}&ECA & 0.943 (0.934 - 0.952) & 0.887 (0.872 - 0.899) & 0.932 (0.916 - 0.947) \\
\rowcolor{cusyellowl} & ASC-US+ & 0.935 (0.925 - 0.944) & 0.873 (0.857 - 0.886) & 0.917 (0.900 - 0.934) \\
\rowcolor{cusyellowl} & LSIL+  & 0.942 (0.932 - 0.951) & 0.837 (0.820 - 0.852)  & 0.876 (0.856 - 0.896) \\
\rowcolor{cusyellowl} & HSIL+   & 0.983 (0.978 - 0.988) & 0.916 (0.903 - 0.927)  & 0.946 (0.933 - 0.960)\\
\rowcolor{cusyellowl} & ALL     & 0.889 (0.876 - 0.902) & 0.711 (0.689 - 0.729)  & 0.792 (0.767 - 0.816)\\

\multirow{1}{*}{6M}&ECA  & 0.952 (0.943 - 0.961) & 0.875 (0.859 - 0.891)& 0.937 (0.922 - 0.952) \\
&ASC-US+  & 0.942 (0.933 - 0.951) & 0.858 (0.842 - 0.875) & 0.927 (0.911 - 0.943)\\
&LSIL+    & 0.945 (0.935 - 0.954) & 0.832 (0.814 - 0.850) & 0.895 (0.876 - 0.914)\\
&HSIL+   & 0.985 (0.980 - 0.990) & 0.917 (0.904 - 0.930) & 0.948 (0.935 - 0.962) \\
&ALL     & 0.897 (0.884 - 0.909) & 0.693 (0.671 - 0.715) & 0.808 (0.784 - 0.832) \\

  \rowcolor{cusyellowl} \multirow{1}{*}{30M}&ECA & 0.952 (0.944 - 0.961) & 0.882 (0.866 - 0.897) & 0.944 (0.929 - 0.958) \\
  \rowcolor{cusyellowl} &ASC-US+ & 0.945 (0.936 - 0.954) & 0.868 (0.852 - 0.884) & 0.936 (0.921 - 0.951) \\
  \rowcolor{cusyellowl} &LSIL+  & 0.945 (0.936 - 0.954) & 0.848 (0.830 - 0.865)  & 0.884 (0.865 - 0.904) \\
  \rowcolor{cusyellowl} &HSIL+    & 0.984 (0.979 - 0.989) & 0.927 (0.915 - 0.940) & 0.946 (0.933 - 0.960)\\
  \rowcolor{cusyellowl} &ALL    & 0.896 (0.884 - 0.908) & 0.717 (0.695 - 0.738) & 0.808 (0.784 - 0.832) \\ 

\multirow{1}{*}{60M}&ECA   & 0.954 (0.946 - 0.963) & 0.915 (0.902 - 0.929)& 0.949 (0.936 - 0.963)\\
&ASC-US+  & 0.947 (0.938 - 0.956) & 0.900 (0.886 - 0.914)& 0.941 (0.926 - 0.955)\\
&LSIL+   & 0.947 (0.938 - 0.956) & 0.857 (0.840 - 0.874)  &  0.905 (0.887 - 0.923)\\
&HSIL+  & 0.985 (0.981 - 0.990) & 0.930 (0.918 - 0.942)  & 0.946 (0.932 - 0.959) \\
&ALL & 0.907 (0.895 - 0.919) & 0.742 (0.721 - 0.763) & 0.816 (0.792 - 0.840) \\

  \rowcolor{cusyellowl} \multirow{1}{*}{100M}&ECA  & 0.950 (0.942 - 0.959) & 0.915 (0.902 - 0.929) & 0.958 (0.946 - 0.970) \\
  \rowcolor{cusyellowl} &ASC-US+  & 0.942 (0.933 - 0.951) & 0.898 (0.884 - 0.913) & 0.949 (0.936 - 0.963) \\
  \rowcolor{cusyellowl} &LSIL+ & 0.951 (0.943 - 0.960) & 0.855 (0.838 - 0.872) & 0.914 (0.897 - 0.931)\\
  \rowcolor{cusyellowl} &HSIL+   & 0.985 (0.981 - 0.990) & 0.927 (0.915 - 0.940) & 0.956 (0.944 - 0.969)\\
  \rowcolor{cusyellowl} &ALL & 0.902 (0.890 - 0.913) & 0.742 (0.721 - 0.763) & 0.835 (0.812 - 0.857) \\ 

\hline 
\end{tabular} 
\label{ST_pretrain_wsi}
\end{table}


\clearpage
\begin{table}[h] 
\renewcommand{\arraystretch}{1.5}
\renewcommand{\tablename}{Extended Data Table.}
\centering 
\caption{\textbf{Ablation experiments for self-supervised pretraining backbone (ViT-Large and ViT-Gaint) and algorithm (DINOv2 \cite{oquabdinov2} and MoCov3 \cite{chen2021empirical}) (Top-1 accuracy)}. The best results are in bold.}
\begin{tabular}{c|ccc} 
\hline
 \rowcolor{cusyellow} \textbf{Pretrain settings} & \textbf{SIPaKMeD }&\textbf{HErlev} & \textbf{CCS-Cell} \\ 
\hline
w/o Pretrain&0.926 (0.908 - 0.944) & 0.842 (0.789 - 0.895)&0.827 (0.810 - 0.844)\\
 \rowcolor{cusyellowl} ViT-Large+DINOv2 &\textbf{0.942 (0.926 - 0.958)} & 0.905 (0.863 - 0.947)&\textbf{0.883 (0.868 - 0.898)}\\
ViT-Gaint+DINOv2 &0.934 (0.917 - 0.951) & \textbf{0.906 (0.864 - 0.948)}&0.877 (0.862 - 0.892)\\
 \rowcolor{cusyellowl}ViT-Large+MoCov3 &0.927 (0.909 - 0.945) & 0.892 (0.847 - 0.937)&0.868 (0.852 - 0.884)\\
\hline 
\end{tabular} 
\label{ST_pretrain_ablation}
\end{table}




\clearpage
\begin{table}[h] 
\renewcommand{\arraystretch}{1.5}
\renewcommand{\tablename}{Extended Data Table.}
\centering 
\caption{\textbf{Overall performance and comparison of SOTA  methods for abnormal cell detection.} The best results are in bold.}
    \begin{tabular}{c|ccc}
    \hline
    \rowcolor{cusyellow} \textbf{Model} & \textbf{mAP} & \textbf{AP50} & \textbf{mAR} \\ 
    \hline
    \textbf{YOLOv3} & 0.134 (0.122–0.146) & 0.263 (0.248–0.278) & 0.349 (0.333–0.365) \\ 
     \rowcolor{cusyellowl} \textbf{Faster R-CNN} & 0.206 (0.192–0.220) & 0.347 (0.331–0.363) & 0.432 (0.415–0.449) \\ 
    \textbf{RetinaNet} & 0.201 (0.187–0.215) & 0.333 (0.317–0.349) & 0.463 (0.446–0.480) \\ 
     \rowcolor{cusyellowl} \textbf{DETR} & 0.209 (0.195–0.223) & 0.390 (0.374–0.406) & 0.470 (0.453–0.487) \\ 
    \textbf{DDETR} & \textbf{0.239 (0.225–0.253)} & \textbf{0.406 (0.389–0.423)} & \textbf{0.490 (0.473–0.507)} \\ 
    \hline
\end{tabular}
\label{ST_det}
\end{table}

\begin{table}[h] 
\renewcommand{\arraystretch}{1.5}
\renewcommand{\tablename}{Extended Data Table.}
\centering 
\caption{\textbf{Class-wise performance and comparison of SOTA methods for abnormal cell detection.} The best results are in bold.}
    \begin{tabular}{c|ccc}
    \hline
    \rowcolor{cusyellow} \textbf{Model} & \textbf{ASC-US} & \textbf{LSIL} & \textbf{ASC-H} \\ 
    \hline
    \textbf{YOLOv3} & 0.316 (0.300–0.332) & 0.375 (0.359–0.391) & 0.125 (0.114–0.136) \\ 
    \rowcolor{cusyellowl} \textbf{Faster R-CNN} & 0.350 (0.334–0.366) & 0.494 (0.477–0.511) & 0.138 (0.126–0.150) \\ 
    \textbf{RetinaNet} & 0.360 (0.344–0.376) & 0.491 (0.474–0.508) & 0.138 (0.126–0.150) \\ 
    \rowcolor{cusyellowl} \textbf{DETR} & 0.431 (0.414–0.448) & 0.513 (0.496–0.530) & \textbf{0.219 (0.205–0.233)} \\ 
    \textbf{DDETR} & \textbf{0.447 (0.430–0.464)} & \textbf{0.549 (0.532–0.566)} & 0.210 (0.196–0.224) \\ 
    \hline
\end{tabular}
\label{ST_det_p1}
\end{table}

\begin{table}[h] 
\renewcommand{\arraystretch}{1.5}
\renewcommand{\tablename}{Extended Data Table.}
\centering 
\caption{\textbf{Class-wise performance and comparison of SOTA methods for abnormal cell detection (continued).} The best results are in bold.}
    \begin{tabular}{c|ccc}
    \hline
    \rowcolor{cusyellow} \textbf{Model} & \textbf{HSIL} & \textbf{SCC} & \textbf{AGC} \\ 
    \hline
    \textbf{YOLOv3} & 0.253 (0.238–0.268) & 0.021 (0.016–0.026) & 0.488 (0.471–0.505) \\ 
    \rowcolor{cusyellowl} \textbf{Faster R-CNN} & 0.336 (0.320–0.352) & 0.136 (0.124–0.148) & 0.631 (0.615–0.647) \\ 
    \textbf{RetinaNet} & 0.314 (0.298–0.330) & 0.077 (0.068–0.086) & 0.618 (0.602–0.634) \\ 
    \rowcolor{cusyellowl} \textbf{DETR} & 0.397 (0.380–0.414) & 0.096 (0.086–0.106) & 0.671 (0.655–0.687) \\ 
    \textbf{DDETR} & \textbf{0.408 (0.391–0.425)} & \textbf{0.124 (0.113–0.135)} & \textbf{0.701 (0.686–0.716)} \\ 
    \hline
\end{tabular}
\label{ST_det_p2}
\end{table}

\clearpage
\begin{table}[h] 
\renewcommand{\arraystretch}{2}
\renewcommand{\tablename}{Extended Data Table.}
\centering 
\caption{\textbf{Internal testing results of Smart-CCS in retrospective study (RC1-RC3).}}
\begin{tabular}{cc|c|c|c} 
\hline
\rowcolor{cusyellow} \multicolumn{2}{c|}{\multirow{1}{*}{\textbf{
 Metrics}}} & \textbf{RC1} & \textbf{RC2} & \textbf{RC3} \\ 
\hline
\multirow{1}{*}{ECA}&Accuracy (95\% CI) & 0.663 (0.644–0.682) & 0.940 (0.931–0.949) & 0.968 (0.960–0.976) \\
& AUC (95\% CI) & 0.767 (0.750–0.784) & 0.971 (0.965–0.978) & 0.990 (0.985–0.995) \\
& F1 Score (95\% CI) & 0.710 (0.692–0.728) & 0.940 (0.931–0.950) & 0.968 (0.960–0.976) \\
& Sensitivity (95\% CI) & 0.713 (0.695–0.731) & 0.859 (0.845–0.873) & 0.938 (0.927–0.949) \\
& Specificity (95\% CI) & 0.654 (0.635–0.673) & 0.961 (0.954–0.969) & 0.978 (0.971–0.985) \\

 \rowcolor{cusyellowl}  \multirow{1}{*}{ASC-US+ }&Accuracy (95\% CI) & 0.726 (0.709–0.743) & 0.934 (0.924–0.944) & 0.967 (0.959–0.975) \\
 \rowcolor{cusyellowl}  & AUC (95\% CI) & 0.767 (0.750–0.784) & 0.967 (0.960–0.974) & 0.990 (0.985–0.995)\\
 \rowcolor{cusyellowl}  & F1 Score (95\% CI)  & 0.759 (0.742–0.776) & 0.934 (0.925–0.944) & 0.967 (0.959–0.975)  \\
 \rowcolor{cusyellowl}  & Sensitivity (95\% CI) & 0.626 (0.607–0.645) & 0.852 (0.838–0.866) & 0.932 (0.920–0.944) \\
 \rowcolor{cusyellowl}  & Specificity (95\% CI)      & 0.743 (0.726–0.760) & 0.954 (0.946–0.962) & 0.978 (0.971–0.985) \\

\multirow{1}{*}{LSIL+}&Accuracy (95\% CI) & 0.807 (0.792–0.822) & 0.911 (0.900–0.922) & 0.848 (0.831–0.865) \\
& AUC (95\% CI) & 0.902 (0.890–0.914) & 0.975 (0.969–0.982) & 0.926 (0.914–0.938) \\
& F1 Score (95\% CI) & 0.859 (0.845–0.873) & 0.923 (0.913–0.933) & 0.896 (0.882–0.910)  \\
& Sensitivity (95\% CI) & 0.816 (0.801–0.831) & 0.929 (0.919–0.939) & 0.957 (0.948–0.966)  \\
& Specificity (95\% CI) & 0.807 (0.792–0.822) & 0.909 (0.898–0.921) & 0.844 (0.827–0.861) \\

 \rowcolor{cusyellowl}   \multirow{1}{*}{HSIL+}&Accuracy (95\% CI) & 0.967 (0.960–0.974) & 0.970 (0.964–0.977) & 0.898 (0.884–0.912)  \\
 \rowcolor{cusyellowl}  & AUC (95\% CI) & 0.943 (0.934–0.952) & 0.982 (0.977–0.987) & 0.934 (0.923–0.945) \\
 \rowcolor{cusyellowl}  & F1 Score (95\% CI)  & 0.979 (0.973–0.985) & 0.976 (0.970–0.982) & 0.936 (0.925–0.947) \\
 \rowcolor{cusyellowl}  & Sensitivity (95\% CI) & 0.750 (0.733–0.767) & 0.826 (0.811–0.841) & 0.800 (0.782–0.818)\\
 \rowcolor{cusyellowl}  & Specificity (95\% CI) & 0.968 (0.961–0.975) & 0.973 (0.967–0.980) & 0.899 (0.885–0.913)\\
\hline
\end{tabular} 
\label{ST_wsi}
\end{table}



\clearpage
\begin{table}[h] 
\renewcommand{\arraystretch}{2}
\renewcommand{\tablename}{Extended Data Table.}
\centering 
\caption{\textbf{Internal testing results of Smart-CCS in retrospective study (RC4-RC6).}}
\begin{tabular}{cc|c|c|c} 
\hline
 \rowcolor{cusyellow} \multicolumn{2}{c|}{\multirow{1}{*}{\textbf{
 Metrics}}} & \textbf{RC4} & \textbf{RC5} & \textbf{RC6}\\ 
\hline
\multirow{1}{*}{ECA}&Accuracy (95\% CI) & 0.894 (0.879–0.908) & 0.943 (0.928–0.957) & 0.821 (0.796–0.846)  \\
& AUC (95\% CI) & 0.968 (0.959–0.976) & 0.985 (0.977–0.992) & 0.890 (0.870–0.910) \\
& F1 Score (95\% CI) & 0.894 (0.879–0.908) & 0.943 (0.928–0.957) & 0.839 (0.816–0.863)  \\
& Sensitivity (95\% CI) & 0.961 (0.951–0.970) & 0.948 (0.934–0.961) & 0.814 (0.789–0.839) \\
& Specificity (95\% CI) & 0.837 (0.819–0.855) & 0.938 (0.923–0.952) & 0.822 (0.798–0.847)  \\

 \rowcolor{cusyellowl}  \multirow{1}{*}{ASC-US+ }&Accuracy (95\% CI) & 0.877 (0.861–0.893) & 0.925 (0.908–0.941) & 0.823 (0.799–0.848) \\
 \rowcolor{cusyellowl}  & AUC (95\% CI)  & 0.957 (0.947–0.967) & 0.978 (0.968–0.987) & 0.888 (0.868–0.909)  \\
 \rowcolor{cusyellowl}  & F1 Score (95\% CI)  & 0.877 (0.862–0.893)& 0.925 (0.908–0.941) & 0.841 (0.817–0.864) \\
 \rowcolor{cusyellowl}  & Sensitivity (95\% CI) & 0.961 (0.952–0.971) & 0.947 (0.933–0.961) & 0.800 (0.774–0.826) \\
 \rowcolor{cusyellowl}  & Specificity (95\% CI)   & 0.812 (0.794–0.831)    & 0.905 (0.887–0.923) & 0.828 (0.803–0.852) ) \\

\multirow{1}{*}{LSIL+}&Accuracy (95\% CI) & 0.828 (0.810–0.846) & 0.870 (0.849–0.891) & 0.831 (0.807–0.855) \\
& AUC (95\% CI) & 0.916 (0.903–0.930) & 0.965 (0.953–0.976) & 0.946 (0.931–0.960) \\
& F1 Score (95\% CI) & 0.838 (0.820–0.855)& 0.875 (0.855–0.896) & 0.864 (0.842–0.886)  \\
& Sensitivity (95\% CI)& 0.842 (0.824–0.859) & 0.957 (0.945–0.970) & 0.935 (0.919–0.951) \\
& Specificity (95\% CI) & 0.824 (0.806–0.842) & 0.837 (0.814–0.860) & 0.821 (0.797–0.846) \\

 \rowcolor{cusyellowl}   \multirow{1}{*}{HSIL+}&Accuracy (95\% CI) & 0.878 (0.862–0.894)& 0.911 (0.893–0.928) & 0.909 (0.891–0.928) \\
 \rowcolor{cusyellowl}  & AUC (95\% CI) & 0.918 (0.904–0.931) & 0.972 (0.962–0.983) & 0.973 (0.963–0.984)  \\
 \rowcolor{cusyellowl}  & F1 Score (95\% CI)   & 0.895 (0.880–0.909)& 0.918 (0.901–0.935) & 0.930 (0.913–0.946) \\
 \rowcolor{cusyellowl}  & Sensitivity (95\% CI)  & 0.818 (0.800–0.837)& 0.895 (0.876–0.914) & 0.897 (0.878–0.917) \\
 \rowcolor{cusyellowl}  & Specificity (95\% CI)  &  0.884 (0.869–0.899) & 0.913 (0.896–0.930) & 0.910 (0.892–0.928) \\

\hline 
\end{tabular} 
\label{ST_wsi_2}
\end{table}


\clearpage
\begin{table}[h] 
\renewcommand{\arraystretch}{2}
\renewcommand{\tablename}{Extended Data Table.}
\centering 
\caption{\textbf{Internal testing results of Smart-CCS in retrospective study (RC7-RC17).} Note: R14-R17 were merged due to the limited sample size at these four centers.}
\begin{tabular}{cc|c|c} 
\hline
 \rowcolor{cusyellow} \multicolumn{2}{c|}{\multirow{1}{*}{\textbf{
 Metrics}}}  & \textbf{RC7 }& \textbf{RC14-17 }\\ 
\hline
\multirow{1}{*}{ECA}&Accuracy (95\% CI) & 0.942 (0.926–0.958) & 0.968 (0.952–0.984) \\
& AUC (95\% CI)  & 0.971 (0.960–0.982) & 0.971 (0.956–0.986) \\
& F1 Score (95\% CI) & 0.942 (0.926–0.958) & 0.968 (0.952–0.984) \\
& Sensitivity (95\% CI)  & 0.857 (0.833–0.881) & 0.940 (0.918–0.961) \\
& Specificity (95\% CI)  & 0.968 (0.956–0.980) & 0.977 (0.964–0.991) \\

 \rowcolor{cusyellowl}  \multirow{1}{*}{ASC-US+ }&Accuracy (95\% CI) & 0.948 (0.933–0.963) & 0.970 (0.955–0.986) \\
 \rowcolor{cusyellowl}  & AUC (95\% CI)  & 0.971 (0.960–0.982) & 0.980 (0.967–0.992) \\
 \rowcolor{cusyellowl}  & F1 Score (95\% CI)  & 0.948 (0.933–0.963) & 0.970 (0.955–0.986) \\
 \rowcolor{cusyellowl}  & Sensitivity (95\% CI)  & 0.887 (0.866–0.908) & 0.947 (0.927–0.968) \\
 \rowcolor{cusyellowl}  & Specificity (95\% CI)    & 0.965 (0.953–0.977) & 0.978 (0.964–0.991) \\

\multirow{1}{*}{LSIL+}&Accuracy (95\% CI)  & 0.818 (0.792–0.844) & 0.968 (0.952–0.984) \\
& AUC (95\% CI) & 0.888 (0.867–0.909) & 0.996 (0.989–1.000) \\
& F1 Score (95\% CI) & 0.846 (0.822–0.870) & 0.969 (0.953–0.985) \\
& Sensitivity (95\% CI) & 0.795 (0.768–0.822) & 1.000 (1.000–1.000) \\
& Specificity (95\% CI) & 0.820 (0.794–0.846) & 0.960 (0.942–0.977) \\

 \rowcolor{cusyellowl}   \multirow{1}{*}{HSIL+}&Accuracy (95\% CI) & 0.890 (0.869–0.911) & 0.921 (0.897–0.946) \\
 \rowcolor{cusyellowl}  & AUC (95\% CI)  & 0.932 (0.915–0.949) & 0.994 (0.987–1.001) \\
 \rowcolor{cusyellowl}  & F1 Score (95\% CI)  & 0.913 (0.894–0.932) & 0.939 (0.917–0.961) \\
 \rowcolor{cusyellowl}  & Sensitivity (95\% CI)  & 0.778 (0.750–0.806) & 1.000 (1.000–1.000) \\
 \rowcolor{cusyellowl}  & Specificity (95\% CI)  & 0.895 (0.874–0.916) & 0.918 (0.893–0.943) \\

\hline 
\end{tabular} 
\label{ST_wsi_3}
\end{table}


\clearpage
\begin{table}[h] 
\renewcommand{\arraystretch}{2}
\renewcommand{\tablename}{Extended Data Table.}
\centering 
\caption{\textbf{External testing results of Smart-CCS in the retrospective study.} Base refers to the typical two-step CCS model, w/ P is adding pretraining, w/ P\&A denotes our proposed Smart-CCS with pretraining and adaptation.}
\begin{tabular}{cc|ccc} 
\hline
 \rowcolor{cusyellow} \multicolumn{2}{c|}{\multirow{1}{*}{\textbf{Metrics}}} & \textbf{Base} &\textbf{w/ P }&\textbf{w/ P\&A (ours)}\\ 
\hline
\multirow{1}{*}{RC8}&Accuracy (95\% CI) & 0.680 (0.650–0.711)&0.802 (0.776–0.828)&0.837 (0.820–0.854)\\
& AUC (95\% CI) &0.812 (0.787–0.837) & 0.851 (0.828–0.874)& 0.923 (0.911–0.935)\\
& F1 (95\% CI) &0.750 (0.722–0.778)&0.840 (0.816–0.864) &0.867 (0.851–0.882)\\
& Sensitivity (95\% CI) &0.762 (0.735–0.790)& 0.775 (0.748–0.802) & 0.891 (0.877–0.905)\\
& Specificity (95\% CI) &0.673 (0.642–0.703)& 0.805 (0.779–0.830)&0.832 (0.815–0.849)\\




 \rowcolor{cusyellowl}  \multirow{1}{*}{RC9}&Accuracy (95\% CI) & 0.812 (0.785–0.838)&0.897 (0.876–0.918)&0.938 (0.927–0.949)\\
 \rowcolor{cusyellowl} & AUC (95\% CI) &0.897 (0.877–0.918) &0.930 (0.913–0.948)&0.963 (0.954–0.972)\\
 \rowcolor{cusyellowl} & F1 (95\% CI) & 0.823 (0.797–0.849) &0.898 (0.878–0.919)&0.937 (0.926–0.949) \\
 \rowcolor{cusyellowl} & Sensitivity (95\% CI) & 0.821 (0.795–0.847) &0.816 (0.789–0.842) &0.833 (0.816–0.851)\\
 \rowcolor{cusyellowl} & Specificity (95\% CI) & 0.809 (0.783–0.836)&0.921 (0.903–0.939) &0.967 (0.958–0.975)\\




\multirow{1}{*}{RC10}&Accuracy (95\% CI) &0.741 (0.712–0.770) &0.871 (0.849–0.893) &0.906 (0.892–0.919)\\
& AUC (95\% CI) &0.830 (0.805–0.855)&0.911 (0.892–0.930) &0.953 (0.943–0.963)\\
& F1 (95\% CI) &0.771 (0.743–0.799)& 0.879 (0.857–0.901)& 0.911 (0.898–0.924)\\
& Sensitivity (95\% CI) &0.781 (0.753–0.808)& 0.808 (0.782–0.834) &0.892 (0.878–0.907) \\
& Specificity (95\% CI) &0.733 (0.704–0.762)&0.884 (0.863–0.905)& 0.909 (0.895–0.922) \\



 \rowcolor{cusyellowl}  \multirow{1}{*}{RC11 }&Accuracy (95\% CI) & 0.815 (0.788–0.841) &0.886 (0.864–0.907)& 0.921 (0.908–0.934)\\
 \rowcolor{cusyellowl} & AUC (95\% CI) & 0.875 (0.853–0.898)&0.935 (0.919–0.952)&0.962 (0.953–0.971)\\
 \rowcolor{cusyellowl} & F1 (95\% CI) &0.815 (0.789–0.841) &0.887 (0.865–0.908) &0.921 (0.908–0.933)\\
 \rowcolor{cusyellowl} & Sensitivity (95\% CI) & 0.761 (0.732–0.790)   &0.895 (0.874–0.916) &0.871 (0.855–0.887)\\
 \rowcolor{cusyellowl} & Specificity (95\% CI) & 0.846 (0.822–0.871)  &0.880 (0.859–0.902)&0.951 (0.941–0.961)\\




\multirow{1}{*}{RC12}&Accuracy (95\% CI) & 0.804 (0.771–0.836)  &0.828 (0.797–0.859)&0.871 (0.851–0.891)\\
& AUC (95\% CI) &0.871 (0.843–0.898)& 0.916 (0.894–0.939) &0.941 (0.927–0.955) \\
& F1 (95\% CI) &0.804 (0.772–0.836)& 0.826 (0.796–0.857) & 0.871 (0.851–0.891)\\
& Sensitivity (95\% CI) &0.786 (0.752–0.819)&0.899 (0.875–0.924)& 0.884 (0.865–0.903)\\
& Specificity (95\% CI) &0.824 (0.793–0.855) & 0.747 (0.712–0.783)  &0.857 (0.836–0.878)\\



 \rowcolor{cusyellowl}  \multirow{1}{*}{RC13}&Accuracy (95\% CI) & 0.644 (0.522–0.766)&0.793 (0.761–0.825)& 0.828 (0.804–0.852)\\
 \rowcolor{cusyellowl} & AUC (95\% CI) & 0.635 (0.513–0.758)  &0.815 (0.784–0.846) &0.880 (0.859–0.901)\\
 \rowcolor{cusyellowl} & F1 (95\% CI) & 0.644 (0.522–0.766) &0.814 (0.783–0.845)&0.846 (0.823–0.869)\\
 \rowcolor{cusyellowl} & Sensitivity (95\% CI) & 0.659 (0.621–0.697)&0.950 (0.939–0.960)&0.771 (0.744–0.798)\\
 \rowcolor{cusyellowl} & Specificity (95\% CI) & 0.594 (0.468–0.719) &0.816 (0.785–0.847) &0.838 (0.814–0.861)\\

\hline 
\end{tabular} 
\label{ST_ext}
\end{table}



\clearpage
\begin{table}[h] 
\renewcommand{\arraystretch}{2}
\renewcommand{\tablename}{Extended Data Table.}
\centering 
\caption{\textbf{Performance comparisons of different classification methods in WSI classification task (AUC (95\% CI)).} The best results are in bold.}
\begin{tabular}{c|cc} 
\hline
 \rowcolor{cusyellow} \textbf{Methods} & \textbf{Internal Test}  & \textbf{External Test} \\ 
\hline
MeanMIL & 0.965 (0.961–0.969)&0.950 (0.945–0.954) \\
 \rowcolor{cusyellowl} MaxMIL & 0.957 (0.953–0.961) &0.933 (0.928–0.938)\\
ABMIL \cite{ilse2018attention} & 0.965 (0.961–0.969) &0.947 (0.942–0.952) \\
 \rowcolor{cusyellowl} DSMIL \cite{li2021dual} & 0.959 (0.955–0.964) & 0.938 (0.933–0.943)  \\
CLAM-SB \cite{shao2021transmil} & 0.964 (0.960–0.968) & 0.919 (0.914–0.925) \\
 \rowcolor{cusyellowl}  TransMIL \cite{lu2021data} & 0.966 (0.962–0.970)  & 0.951 (0.946–0.955)\\
S4MIL \cite{fillioux2023structured} &\textbf{0.969 (0.966–0.973)} & \textbf{0.953 (0.948–0.957)}\\
\hline 
\end{tabular} 
\label{ST_clas}
\end{table}

\clearpage
\begin{table}[h] 
\renewcommand{\arraystretch}{2}
\renewcommand{\tablename}{Extended Data Table.}
\centering 
\caption{\textbf{Performance of cervical cytology cancer screening in the prospective study across three centers (PC1-PC3).}}
\begin{tabular}{c|ccc} 
\hline
 \rowcolor{cusyellow} \textbf{Metrics} & \textbf{PC1 ($N$=998)} & \textbf{PC2 ($N$=1,311)} & \textbf{PC3 ($N$=1,044)}\\ 
\hline
Accuracy (95\% CI) & 0.877 (0.856–0.897)&0.862 (0.843–0.881) &0.950 (0.937–0.963)\\
 \rowcolor{cusyellowl} AUC (95\% CI) &0.947 (0.933–0.961)   & 0.924 (0.910–0.938)& 0.986 (0.979–0.993) \\
F1 Score (95\% CI) & 0.877 (0.857–0.898) & 0.867 (0.848–0.885)   &0.951 (0.938–0.964)\\
 \rowcolor{cusyellowl} Sensitivity (95\% CI) &0.893 (0.874–0.912) & 0.881 (0.864–0.899) & 0.946 (0.932–0.960)\\
Specificity (95\% CI) &0.864 (0.843–0.885)& 0.855 (0.836–0.874)  &0.952 (0.939–0.965)\\

\hline 
\end{tabular} 
\label{ST_pros}
\end{table}


\newpage
\begin{table}[h] 
\renewcommand{\arraystretch}{1.5}
\renewcommand{\tablename}{Extended Data Table.}
\centering 
\caption{\textbf{Statistics of the CCS-127K dataset with the number of WSIs and corresponding patches at each center.}}
\begin{tabular}{c|c|c|c|c|c} 
\hline
 \rowcolor{cusyellow} \textbf{Center} & \textbf{WSI} & \textbf{Patch} & \textbf{Center} & \textbf{WSI} & \textbf{Patch} \\
\hline
 RC1 & 12,598 & 26,980,754   & RC26 & 60 & 63,215\\
 \rowcolor{cusyellowl}  RC2 & 14,529 & 17,400,701&  RC27 & 239 & 102,381\\
 RC3 & 9,039  & 27,416,532 & RC28 & 33& 22,435\\
 \rowcolor{cusyellowl}  RC4 &8,432 & 14,811,441 &  RC29 & 20 & 23,921 \\
 RC5 & 17,623 & 33,845,451& RC30 & 13 & 63,745\\
 \rowcolor{cusyellowl}  RC6 & 5,773 & 15,692,295 &  RC31 & 10& 5,527\\
 RC7 & 8,139 & 31,573,365 & RC32 & 5,516 & 2,290,698\\
 \rowcolor{cusyellowl}  RC8 & 1,873 & 2,002,922&  RC33 & 160 &354,062\\
 RC9 & 2,013 & 6,085,661 & RC34 & 1,319 & 1,732,829\\
 \rowcolor{cusyellowl}  RC10 &  1,797 & 1,776,018 &  RC35 & 543 & 1,921,444 \\
 RC11 & 3,400 & 8,420,141 & RC36 & 491 & 2,970,771\\
 \rowcolor{cusyellowl}  RC12 & 1,097 & 1,867,367&  RC37 & 6& 10,337\\
 RC13 &  1,876 & 371,098 & RC38 & 39 & 42,763 \\
 \rowcolor{cusyellowl}  RC14 & 12,823 & 12,832,172 &  RC39 & 13& 14,433\\
 RC15 & 657 & 979,460 & RC40 & 127 & 181,802 \\
 \rowcolor{cusyellowl}  RC16 &588 & 68,299 &  RC41 & 190 & 63,359\\
 RC17 & 707 & 231,009 & RC42 & 7,002 & 9,228,021\\
 \rowcolor{cusyellowl}  RC18 & 491 & 451,371 &  RC43 & 330 & 54,372\\
 RC19 &494 & 300,671 &  RC44 & 451 & 428,447 \\
 \rowcolor{cusyellowl}  RC20 & 933 & 1,749,139 &  RC45 & 14 & 45,227 \\
 RC21 & 234 & 85,197 & PC1 & 998 & 421,901\\
 \rowcolor{cusyellowl}  RC22 & 2,045 & 372,867 &  PC2 & 1,311 & 788,410\\
 RC23 & 192 & 223,526 & PC3 &1,044 & 602,583\\
 \rowcolor{cusyellowl}  RC24 & 108 & 51,613&  Total & 127,471& 227,266,352\\
RC25 & 81 & 244,599& & & \\
\hline
\end{tabular} 
\label{ST_patches}
\end{table}

\newpage
\begin{table}[h] 
\renewcommand{\arraystretch}{1.5}
\renewcommand{\tablename}{Extended Data Table.}
\centering 
\caption{\textbf{Hyperparameters used for self-supervised pretraining.}}

\begin{tabular}{c|c|p{2cm}c} 
\hline 
 \rowcolor{cusyellow} & \textbf{Hyperparameters} & \textbf{Value} \\
\hline 
\multirow{7}{*}{Model} & Layer number  & 24 \\
& Feature dimension & 1,024 \\
& Patch size & 14 \\
& Heads number & 16 \\
& FFN layer & MLP \\
& Drop path ratio & 0.4 \\
& Layer scale  & 1.00e-05 \\
\hline 
 \rowcolor{cusyellowl} Loss weight & DINO & 1 \\
 \rowcolor{cusyellowl}& iBOT & 1 \\
\hline 
\multirow{12}{*}{Optimization} & Teacher momentum & 0.994\\
& Total batch size & 1,024 \\
& Base learning rate & 1.00e-04 \\
& Minimum learning rate & 1.00e-06 \\
& Global crops scale & 0.32, 1.0 \\
& Global crops size & 224 \\
& Local crops scale & 0.02, 0.32 \\
& Local crops size & 98 \\
& Local crops number & 8 \\
& Gradient clip & 3\\
& Warmup iterations & 50,000 \\
& Total iterations & 500,000\\ 
\hline 
\end{tabular} 
\label{ST_pretrain_para}
\end{table}

\clearpage
\begin{table}[h] 
\renewcommand{\arraystretch}{1.5}
\renewcommand{\tablename}{Extended Data Table.}
\centering 
\caption{\textbf{Hyperparameters used for cell detector and WSI classifier in CCS model.}}
\begin{tabular}{c|c} 
\hline
 \rowcolor{cusyellow} \multicolumn{2}{c}{\multirow{1}{*}{\textbf{Abnormal Cell Detector}}} \\ 
\hline
backbone & ResNet50\\ 
 \rowcolor{cusyellowl} num\_encoder\_layers & 6\\ 
num\_decoder\_layers & 6\\ 
 \rowcolor{cusyellowl} num\_queries & 300\\ 
num\_classes & 7\\ 
 \rowcolor{cusyellowl} dropout & 0.1\\ 
epoch & 100\\ 

\hline
 \rowcolor{cusyellow} \multicolumn{2}{c}{\multirow{1}{*}{\textbf{WSI Classifier}}} \\ 
\hline
classifier & ABMIL, MeanMIL, MaxMIL, CLAM, DSMIL, TransMIL, S4MIL\\ 
 \rowcolor{cusyellowl} top-k & 50\\ 
epoch & 50\\ 
 \rowcolor{cusyellowl} num\_classes & 7\\ 
in\_dim & 1,024\\ 
\hline 
\end{tabular} 
\label{ST_det_para}
\end{table}
