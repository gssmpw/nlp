%%%%%%%% ICML 2025 EXAMPLE LATEX SUBMISSION FILE %%%%%%%%%%%%%%%%%

\documentclass{article}
%%%%%%%%%%%%%%%%%%%%%%%%%%%%%%%%%%%%%%%%%%%%%%%%%%%%%%%%%%%%%%%%%%%%%%%%%%%%%%

%% Beautiful mathematics
\usepackage{amsmath, amssymb, amsfonts} 
\usepackage{nicefrac}
\usepackage{mathtools}
\usepackage{bm, bbm}
\usepackage[scr=boondoxo,scrscaled=1.05]{mathalfa}

%% References in the correct format 
%\usepackage[square,numbers]{natbib}
%\def\bibfont{\footnotesize} % fix to have the same font size as without natbib

\usepackage[sort, compress, space]{cite}            


%% Enumerate nicely 
\usepackage{enumitem}

%% Different color comments and commenting large parts of the text
\usepackage{xcolor}
\usepackage{comment}
\usepackage{soul}

%% Hyper references
\usepackage{hyperref}
\usepackage{cleveref}
%\usepackage[numbers]{natbib}

\usepackage{tikz}
%\usepackage{thm-restate}
%% Appendix package
%\usepackage{appendix}

%% Random text to test spacing 
\usepackage{blindtext}

\usepackage{afterpage}

\usepackage{algorithm, algorithmic}    



\usepackage{dsfont}

\usepackage{tikz}
\usepackage{graphicx}
\usepackage{tikzscale}
\usepackage{pgfplots}
\pgfplotsset{compat=newest}
\usepackage{xfrac}

\usepackage{thm-restate}

%\usepackage{subcaption}

\usepackage{balance}

\usepackage{cite}
\usepackage{amsmath,amssymb,amsfonts}
\usepackage{balance}
\usepackage{algorithmic}
\usepackage{graphicx}
\usepackage{textcomp}
\usepackage{xcolor}
\usepackage{amsmath}
\usepackage{amssymb}
\usepackage[mathscr]{euscript}
\usepackage{comment}
\usepackage{xcolor}
\usepackage{enumitem} 
\usepackage{amsthm}



% Recommended, but optional, packages for figures and better typesetting:
\usepackage{microtype}
\usepackage{graphicx}
% \usepackage{subfigure}
\usepackage{booktabs} % for professional tables

% hyperref makes hyperlinks in the resulting PDF.
% If your build breaks (sometimes temporarily if a hyperlink spans a page)
% please comment out the following usepackage line and replace
% \usepackage{icml2025} with \usepackage[nohyperref]{icml2025} above.
\usepackage{hyperref}


% Attempt to make hyperref and algorithmic work together better:
\newcommand{\theHalgorithm}{\arabic{algorithm}}
% \usepackage{algorithmic}

% \newcommand{\OURS}{\texttt{QuantSpec}\xspace}
\newcommand{\OURS}{QuantSpec}


\newcommand {\amir}[1]{{\color{red}\sf{[Amir: #1]}}}
\newcommand {\aditya}[1]{{\color{red}\sf{[Aditya: #1]}}}
\newcommand {\sehoon}[1]{{\color{blue}Sehoon: #1}}
\newcommand {\rishabh}[1]{{\color{purple}\sf{[Rishabh: #1]}}}
\newcommand {\haocheng}[1]{{\color{orange}\sf{[Haocheng: #1]}}}
\newcommand {\coleman}[1]{{\color{red}\sf{[Coleman: #1]}}}
\newcommand {\maxhorton}[1]{{\color{red}\sf{[Max: #1]}}}
\newcommand {\mahyar}[1]{{\color{red}\sf{[Mahyar: #1]}}}

% Use the following line for the initial blind version submitted for review:
% \usepackage{icml2025}

\usepackage[arxiv]{icml2025}

% If accepted, instead use the following line for the camera-ready submission:
% \usepackage[accepted]{icml2025}

% For theorems and such
\usepackage{amsmath}
\usepackage{amssymb}
\usepackage{mathtools}
\usepackage{amsthm}

% if you use cleveref..
\usepackage[capitalize,noabbrev]{cleveref}

%%%%%%%%%%%%%%%%%%%%%%%%%%%%%%%%
% THEOREMS
%%%%%%%%%%%%%%%%%%%%%%%%%%%%%%%%
\theoremstyle{plain}
\newtheorem{theorem}{Theorem}[section]
\newtheorem{proposition}[theorem]{Proposition}
\newtheorem{lemma}[theorem]{Lemma}
\newtheorem{corollary}[theorem]{Corollary}
\theoremstyle{definition}
\newtheorem{definition}[theorem]{Definition}
\newtheorem{assumption}[theorem]{Assumption}
\theoremstyle{remark}
\newtheorem{remark}[theorem]{Remark}

% Todonotes is useful during development; simply uncomment the next line
%    and comment out the line below the next line to turn off comments
%\usepackage[disable,textsize=tiny]{todonotes}
\usepackage[textsize=tiny]{todonotes}


% The \icmltitle you define below is probably too long as a header.
% Therefore, a short form for the running title is supplied here:
\icmltitlerunning{\OURS: Self-Speculative Decoding with Hierarchical Quantized KV Cache}

\begin{document}

\twocolumn[
\icmltitle{\OURS: Self-Speculative Decoding with Hierarchical Quantized KV Cache}

% It is OKAY to include author information, even for blind
% submissions: the style file will automatically remove it for you
% unless you've provided the [accepted] option to the icml2025
% package.

% List of affiliations: The first argument should be a (short)
% identifier you will use later to specify author affiliations
% Academic affiliations should list Department, University, City, Region, Country
% Industry affiliations should list Company, City, Region, Country

% You can specify symbols, otherwise they are numbered in order.
% Ideally, you should not use this facility. Affiliations will be numbered
% in order of appearance and this is the preferred way.
\icmlsetsymbol{equal}{*}

\begin{icmlauthorlist}
\icmlauthor{Rishabh Tiwari}{ucb,equal}
\icmlauthor{Haocheng Xi}{ucb,equal}
\icmlauthor{Aditya Tomar}{ucb,equal}
\icmlauthor{Coleman Hooper}{ucb}
\icmlauthor{Sehoon Kim}{ucb}
\icmlauthor{Maxwell Horton}{apple}
\icmlauthor{Mahyar Najibi}{apple}
\icmlauthor{Michael W. Mahoney}{ucb,icsi,lbnl}
\icmlauthor{Kurt Keutzer}{ucb}
\icmlauthor{Amir Gholami}{ucb,icsi}
\end{icmlauthorlist}

\icmlaffiliation{ucb}{UC Berkeley}
\icmlaffiliation{apple}{Apple}
\icmlaffiliation{icsi}{ICSI}
\icmlaffiliation{lbnl}{LBNL}

\icmlcorrespondingauthor{Amir Gholami}{amirgh@berkeley.edu}

% You may provide any keywords that you
% find helpful for describing your paper; these are used to populate
% the "keywords" metadata in the PDF but will not be shown in the document
\icmlkeywords{Machine Learning, ICML}

\vskip 0.3in
]

% this must go after the closing bracket ] following \twocolumn[ ...

% This command actually creates the footnote in the first column
% listing the affiliations and the copyright notice.
% The command takes one argument, which is text to display at the start of the footnote.
% The \icmlEqualContribution command is standard text for equal contribution.
% Remove it (just {}) if you do not need this facility.

%\printAffiliationsAndNotice{}  % leave blank if no need to mention equal contribution
\printAffiliationsAndNotice{\icmlEqualContribution \\{Apple team members served in an advisory role.}\\} % otherwise use the standard text.
\begin{abstract}
Test-time compute scaling has emerged as a new axis along which to improve model accuracy, where additional computation is used at inference time to allow the model to think longer for more challenging problems. 
One promising approach for test-time compute scaling is search against a process reward model, where a model generates multiple potential candidates at each step of the search, and these partial trajectories are then scored by a separate reward model in order to guide the search process.
The diversity of trajectories in the tree search process affects the accuracy of the search, since increasing diversity promotes more exploration.
However, this diversity comes at a cost, as divergent trajectories have less KV sharing, which means they consume more memory and slow down the search process.
Previous search methods either do not perform sufficient exploration, or else explore diverse trajectories but have high latency.
We address this challenge by proposing Efficient Tree Search (\ours), which promotes KV sharing by pruning redundant trajectories while maintaining necessary diverse trajectories. 
\ours incorporates a linear programming cost model to promote KV cache sharing by penalizing the number of nodes retained, while incorporating a semantic coverage term into the cost model to ensure that we retain trajectories which are semantically different.
We demonstrate how \ours can achieve \textbf{1.8}$\times$ reduction in average KV cache size during the search process, leading to \textbf{1.4}$\times$ increased throughput relative to prior state-of-the-art methods, with minimal accuracy degradation and without requiring any custom kernel implementation. Code is available at: \url{https://github.com/SqueezeAILab/ETS}.

\end{abstract}


\section{Introduction}
%%%%%%%%%%%%%%%%%%%%%%%%%%%%%%%%%%%%
\begin{figure}[t]
    \centering
    \includegraphics[width=\linewidth]{figs/Teaser.pdf}
    \caption{Throughput in tokens/sec of various decoding methods. \OURS\ achieves $>1.78\times$ speedup over the autoregressive baseline across several context lengths. Benchmarked on LWM-Text-Chat-128k.}
    \label{fig:placeholder}
    \vspace{-0.2cm}
\end{figure}
%%%%%%%%%%%%%%%%%%%%%%%%%%%%%%%%%%%%


Large Language Models (LLMs) have been widely used in recent years, revolutionizing natural language processing (NLP) and artificial intelligence (AI) applications. 
As their applications expand, there is a growing demand to deploy LLMs in long-context settings -- handling extended text inputs such as document summarization, lengthy conversations, or comprehensive instructions. 
The model must maintain coherence in such contexts and track intricate details across extended sequences. 
However, long-context inference presents significant challenges in terms of efficiency and scalability.
For example, token eviction~\cite{zhang2024h2o,ge2023model,liu2024scissorhands} and KV cache quantization~\cite{liu2024kivi,kang2024gear,hooper2024kvquant} have been proposed to improve the efficiency for long-context inference. However, they often entail noticeable degradation in generation quality.


One promising alternative to enhance the efficiency of LLMs while preserving generation quality is speculative decoding~\cite{leviathan2023fast,chen2023accelerating,kim2024speculative}. 
This method accelerates inference by using a smaller (draft) model to rapidly generate candidate tokens, and uses the original (target) model to verify these tokens to ensure generation quality. However, the efficient application of speculative decoding in long-context settings has not been thoroughly explored.

Traditional speculative decoding approaches often rely on using smaller models as the draft model in order to minimize the memory-bandwidth overhead of loading the \textit{model weights} of the larger target model.
In long-context scenarios, however, the primary bottleneck shifts from model weights to the \textit{KV cache}, which grows linearly with the context length. Additionally, since small models do not usually possess good long-context understanding ability, the acceptance rates of the candidate tokens by the target model drop significantly, leading to suboptimal speedup.
Moreover, traditional speculative decoding methods maintain the KV cache for both the target model and the draft model, causing a large memory footprint. Therefore, finding a solution that both optimizes the KV cache's memory efficiency and improves the acceptance rate within speculative decoding is essential for performant LLMs in long-context applications.


To mitigate these issues and to enable efficient and accurate long-context inference, we propose \OURS, a self-speculative decoding method that utilizes 4-bit weights and a 4-bit hierarchical KV cache to speedup long-context inference. In particular we make the following contributions: 
\begin{itemize}
    \item We perform a comprehensive analysis of LLM inference to identify bottlenecks across various context lengths, demonstrating that quantizing the KV cache improves efficiency for long contexts, while quantizing model weights is more beneficial for short contexts (see Section \ref{sec:multi-regime-analysis-inference-bottlenecks}).
    \item We introduce a novel hierarchical quantization technique that enables bit-sharing between the target and draft models' KV caches, eliminating the need for additional memory for the draft model (see Section \ref{subsec:Hierarchical_kv_cache}).
    \item We propose a double full-precision cache buffer used for storing the most recent KV cache in full precision to improve acceptance rates and also eliminate wasteful quantization and dequantization operations (see Section \ref{subsec:full_precision_buffer}).
    \item We show that using a quantized KV cache leads to better acceptance rates between the target and the draft model, and thus leads to better overall speedups (see Section \ref{sec:speedup}).
    \item We implement custom CUDA kernels for attention with our hierarchical quantized KV cache achieving up to $\sim 2.88 \times$ speedups at 4-bit precision relative to FP16 FlashAttention kernels. (see Section \ref{sec:kernel_speedups})
\end{itemize}
\section{Related Work}

\subsection{Efficient Long-Context Inference}
An important challenge in optimizing long-context inference lies in reducing memory and computation requirements while retaining high performance on tasks that involve long sequences.
Sparse attention mechanisms~\citep{liu2021sparseattn,xiao2023streamllm,yao2024sirllm,tang2024quest,yang2024doublesparse,liu2024scissorhands,ge2023model,jiang2024minference} have been widely adopted to manage the quadratic complexity of traditional full attention in long contexts. 
% By selectively attending only to a subset of tokens, sparse attention reduces both computational load and memory usage, enabling the processing of longer sequences. 
These techniques typically maintain efficiency by dropping non-essential Key-Value (KV) pairs from the cache.
% Another line of research maintains the full key-value pairs but dynamically loads them from HBM, and usually achieves higher performance at the cost of higher memory consumption.
Token pruning~\citep{fu2024lazyllm} selectively computes the KV for tokens relevant for next token prediction. KV Prediction~\citep{horton2024kvpredictionimprovedtime} improves prompt processing time by predicting the KV cache needed for autoregressive generation. 
Retrieval-augmented generation~\citep{tan2024lloco,liu2024retrievalattention} enhances the accuracy of language model outputs by combining generative models with external retrieval mechanisms whose context length is very long.


\subsection{Quantization}
Quantization has emerged as a powerful technique to reduce the memory footprint and computational complexity in large-scale neural networks.
Weight-only quantization~\cite{lin2024awq,kim2023squeezellm,shao2023omniquant,chee2024quip} focuses on reducing the precision of model weights to reduce the memory requirements of the model.
As models grow larger, the memory footprint of KV caches can become substantial, especially for long input sequences. KV cache quantization~\cite{liu2024kivi,hooper2024kvquant,kang2024gear} addresses this issue by quantizing the key and value caches to enable longer sequence inference.

\subsection{Speculative Decoding}
Speculative decoding has become an important technique for improving the inference efficiency of LLMs~\cite{leviathan2023fast,chen2023accelerating,kim2024speculative}. It uses a smaller draft model to rapidly generate candidate tokens, which are then verified by a larger target model to ensure correctness. Parallelization in speculative decoding has also been studied to enhance the efficiency by predicting multiple tokens at one time~\cite{cai2024medusa,bhendawade2024speculative,li2024eagle,chen2024sequoia}. We include additional related works in Appendix~\ref{appendix:rel_works}.

\textbf{Self-speculative decoding} is the class of speculative decoding methods in which the draft model shares the same architecture as that of target model for better alignment. Recent works like Magicdec \cite{magicdec} and TriForce~\cite{sun2024triforce} have shown that self-speculation with sparse KV can effectively speedup the draft model in long-context settings, where KV is the main bottleneck. While this design avoids loading the entire KV cache throughout the autoregressive generation process, KV cache sparsification can lead to noticeable performance degradation as evidenced in previous works~\cite{zhang2024h2o,liu2024scissorhands,ge2023model,zhou2024sirius}.
This can potentially yield a mismatch between the draft and target model's predictions (i.e., lower acceptance rate), which is a critical factor in overall speedup.
\OURS\ addresses this limitation by proposing a draft model with a novel hierarchical quantized KV cache, which maintains a higher acceptance rate between the draft and target models, therefore leading to better speedup. Note that our method can be combined with sparse KV methods \cite{sun2024triforce, magicdec} for additional speedup, which we leave for future work.
\section{LLM Inference Bottlenecks}

\subsection{Arithmetic Intensity}
\label{sec:multi-regime-analysis-inference-bottlenecks}
To understand the primary bottlenecks in LLM inference and to motivate our method, we perform a thorough analysis of inference under several different regimes. These regimes include a combination of small versus large batch sizes and short versus long context lengths during both the prefill and decoding stages. We use arithmetic intensity as the central metric in our analysis, where arithmetic intensity is defined as the number of floating point operations (FLOPs) that can be performed per byte loaded from memory, or memory operations (MOPs) \cite{williams2009roofline}:
\[\text{Arithmetic Intensity} = \frac{\#\text{ FLOPs}}{\#\text{ MOPs}}.\] 

Arithmetic intensity allows us to classify which regimes of LLM inference are compute-bound or memory-bound and determine appropriate optimizations to improve latency. Compute-bound operations are limited by the hardware's peak FLOP/s (FLOPs per second) performance and benefit from algorithmic improvements that reduce computational complexity (e.g., subquadratic attention).
On the other hand, memory-bound operations are limited by the hardware's memory bandwidth (GB/s) and benefit from techniques that optimize memory load-store operations, such as quantizing the weights of a model even if they are later scaled up to a higher precision during computation to preserve accuracy.

For a finer-grained analysis, we break down the major operations in the Transformer into two categories: \textbf{linear}, which consists of the weight-to-activation matrix multiplications (i.e., $W_Q, W_K, W_V, W_{out}$, \texttt{mlp\_up\_proj}, \texttt{mlp\_down\_proj}, and the linear classification layer), and \textbf{attention}, which consists of the activation-to-activation matrix multiplications (i.e., query $\times$ key and attention weights $\times$ values). Note that the \textbf{aggregate} of all Transformer operations includes the above operations as well as non-linear operations like activation functions in the feed-forward network, softmax in the attention mechanism, and layer normalization. Because we are interested in studying the linear and attention operations, we do not explicitly focus on the non-linear operations and classification layer in our asymptotic analysis, although we include them in our final results.

\subsubsection{Asymptotic Analysis of Arithmetic Intensity for Prefill and Decoding}

% ASYMPTOTIC ANALYSIS TABLE
%%%%%%%%%%%%%%%%%%%%%%%%%%%%%%%%%%%%%%%%%%%%%%%%%%%%%%%
\newcolumntype{Y}{>{\centering\arraybackslash}X}

% Define colors for headers
\definecolor{headergray}{RGB}{240,240,240}

\begin{table*}[h]
\caption{Asymptotic analysis of arithmetic intensity for linear, attention, and aggregate operations under prefill and decoding for batch size $B$, sequence length $S_L$, hidden dimension $d$, and generation length of $k$ tokens.}\label{tab:asymptotic_analysis}
\vspace{2mm}
\centering
\begin{tabularx}{\textwidth}{l|Y|Y|Y}
  \toprule
  \multicolumn{4}{c}{\textbf{Prefill}} \\
  \midrule
   & \cellcolor{headergray}Linear & \cellcolor{headergray}Attention & \cellcolor{headergray}Aggregate \\
  \midrule

  \cellcolor{headergray}FLOPs 
    & $\mathcal{O}(B \cdot S_L \cdot d^2)$ 
    & $\mathcal{O}(B \cdot {S_L}^2 \cdot d)$
    & $\! \! \mathcal{O}(B \!\cdot\! S_L \!\cdot\! d^2) + \mathcal{O}(B \!\cdot\! {S_L}^2 \!\cdot\! d) \!$ \\


  \midrule
  \cellcolor{headergray}MOPs 
    & $\underbrace{\mathcal{O}(B \cdot S_L \cdot d)}_{\text{activations}}\; + \;\underbrace{\mathcal{O}(d^2)}_{\text{weights}}$
    & $\!\underbrace{\mathcal{O}(B \cdot S_L)}_{\text{flash-attn scores}} + \! \! \! \underbrace{\mathcal{O}(B \cdot S_L \cdot d)}_{\text{activations }\{Q, C_K, C_V\}}$
    &  $\mathcal{O}(B \cdot S_L \cdot d) + \mathcal{O}(d^2)$ \\

  \midrule
  \cellcolor{headergray}Arithmetic Intensity 
      & 
        $\begin{array}{ll}
            \approx \begin{cases}
                \mathcal{O}(B \cdot S_L), & S_L \ll d \\
                \mathcal{O}(d), & S_L \gg d
            \end{cases}
        \end{array}$
      & 
        $\begin{array}{ll}
            \approx  \begin{cases}
                \mathcal{O}(S_L), & S_L \ll d \\
                \mathcal{O}(S_L), & S_L \gg d
            \end{cases}
        \end{array}$
      & 
        $\begin{array}{ll}
            \approx \begin{cases}
                \mathcal{O}(B \cdot S_L), & S_L \ll d \\
                \mathcal{O}(S_L), & S_L \gg d
            \end{cases}
        \end{array}$ \\
  \midrule
  \multicolumn{4}{c}{\textbf{Decode}} \\
  \midrule
   & \cellcolor{headergray}Linear & \cellcolor{headergray}Attention & \cellcolor{headergray}Aggregate \\
  \midrule

  \cellcolor{headergray}FLOPs 
    & $\mathcal{O}(k \cdot B \cdot d^2)$ 
    & $\mathcal{O}(k \cdot B \cdot S_L \cdot d)$ 
    & $\! \! \mathcal{O}(k \!\cdot\! B \!\cdot\! d^2) + \mathcal{O}(k \!\cdot\! B \!\cdot\! S_L \!\cdot\! d)$ \\

  \midrule
  \cellcolor{headergray}MOPs 
    & $\underbrace{\mathcal{O}(k \cdot B \cdot d)}_{\text{activations}}\; + \;\underbrace{\mathcal{O}(k \cdot d^2)}_{\text{weights}}$
    & $\! \underbrace{\mathcal{O}(k \!\cdot\! B \!\cdot\! S_L)}_{\text{attention scores}} + \!\underbrace{\mathcal{O}(k \!\cdot\! B \!\cdot\! S_L \!\cdot\! d)}_{\text{activations }\{C_K, C_V\}}$
    & $\mathcal{O}(k \cdot d^2) + \mathcal{O}(k \cdot B \cdot S_L \cdot d)$ \\

  \midrule
  \cellcolor{headergray}Arithmetic Intensity 
      & 
        $\begin{array}{ll}
            \approx \begin{cases}
                \mathcal{O}(B), & S_L \ll d \\
                \mathcal{O}(B), & S_L \gg d
            \end{cases}
        \end{array}$
      & 
        $\begin{array}{ll}
            \approx  \begin{cases}
                \mathcal{O}(1), & S_L \ll d \\
                \mathcal{O}(1), & S_L \gg d
            \end{cases}
        \end{array}$
      & 
        $\begin{array}{ll}
            \approx \begin{cases}
                \mathcal{O}(B), & S_L \ll d \\
                \mathcal{O}(1), & S_L \gg d
            \end{cases}
        \end{array}$ \\
  \bottomrule
\end{tabularx}
\end{table*}
%%%%%%%%%%%%%%%%%%%%%%%%%%%%%%%%%%%%%%%%%%%%%%%%%%%%%%%

During \textbf{prefill}, the model weights are only loaded once to process all tokens in the input and generate the first token. Because the context length can range from a couple thousand to hundreds of thousands of tokens, this phase consists of large matrix-matrix multiplications (matmuls) with high arithmetic intensities. Table \ref{tab:asymptotic_analysis} shows asymptotic analysis of arithmetic intensity for prefill and decoding broken up into linear, attention, and aggregate operations for batch size $B$, sequence length $S_L$, hidden dimension $d$, and a generation length of $k$ tokens. During prefill, the aggregate arithmetic intensity is similar to the arithmetic intensity of the linear projections when $S_L \ll d$ because self-attention is relatively inexpensive for short contexts. Thus the linear projections dominate latency in this regime. However, as the context length increases and $S_L \gg d$, the aggregate arithmetic intensity reflects the arithmetic intensity of attention, which begins to dominate latency since self-attention incurs additional cost with longer context lengths. Note that our analysis assumes the use of FlashAttention \cite{flashattention}, such that the attention scores matrix which grows on the order of $\mathcal{O}(B \cdot {S_L}^2)$ is never fully materialized, and thus the memory operations for this matrix are limited to $\mathcal{O}(B \cdot S_L)$.

On the other hand, in the \textbf{decoding} stage, generating $k$ tokens requires loading and storing the weights and KV cache $k$ times. Since the input at each iteration is a single token per sequence in the batch ($x\in\mathbb{R}^{B\times 1\times d}$), these operations mainly consist of small matmuls with low arithmetic intensity. For short context lengths where $S_L \ll d$, the aggregate arithmetic intensity for decoding again reflects the arithmetic intensity of the linear projections as loading and storing a small KV cache is relatively inexpensive compared to loading and storing the model weights. However, as the context length grows ($S_L \gg d$), the load-store operations for the large KV cache exacerbate and dominate latency, and the aggregate arithmetic intensity reflects the arithmetic intensity of attention. Ultimately, the aggregate arithmetic intensity for decoding is much lower than that of prefill:
\[
\underbrace{
\begin{cases}
    \mathcal{O}(B \cdot S_L), & S_L \ll d \\
    \mathcal{O}(S_L), & S_L \gg d
\end{cases}}_{\text{prefill}}
\quad \gg \quad
\underbrace{
\begin{cases}
    \mathcal{O}(B), & S_L \ll d \\
    \mathcal{O}(1), & S_L \gg d
\end{cases}}_{\text{decode}}.
\]
While the aggregate arithmetic intensity for prefill scales proportionally to the context length which can be in the hundreds of thousands, the aggregate arithmetic intensity for decoding does not scale with the context length at all. Moreover, using larger batch sizes only seems to increase the arithmetic intensity for decoding in the short-context setting. For long contexts, decoding has an extremely low arithmetic intensity irrespective of the batch size since every sequence in the batch undergoes self-attention separately and therefore cannot benefit from batching in the same way linear layers do.

% Combined Arithmetic Intensity Analysis Figures
%%%%%%%%%%%%%%%%%%%%%%%%%%%%%%%%%%%%
\begin{figure*}[t]
    \centering
    \includegraphics[width=\linewidth]{figs/decode_ai_analysis.png}
    \caption{Breakdown of how arithmetic intensity changes during decoding as the context length and batch size are scaled logarithmically for linear, attention, and aggregate operations. All regimes lie below the ridge plane and thus are memory-bound. The ridge plane is calculated for an NVIDIA A6000 GPU. The colors for the linear and attention surface plots simply represent the magnitude of the arithmetic intensity. The aggregate plot is colored by attention's runtime as a percentage of the total latency. Prefill results in Appendix~\ref{appendix:prefill_ai_analysis}.}
    \label{fig:decode_arithmetic_intensity_analysis}
\end{figure*}
%%%%%%%%%%%%%%%%%%%%%%%%%%%%%%%%%%%%

\subsubsection{Compute versus Memory-Bound Regimes}
The asymptotic analysis suggests that in general, decoding suffers from low arithmetic intensities compared to prefill in all regimes. However, to decide which optimizations will most effectively improve latency, all regimes must be classified as either compute-bound or memory-bound. 
Whether an operation is compute or memory-bound depends on the hardware it is being run on as well as the magnitude of the arithmetic intensity achieved by the operation.

We utilize an analytical roofline model \cite{williams2009roofline, kim2023squeezellm, kim2023stackoptimizationtransformerinference} to help determine which regimes are compute or memory-bound in a practical inference setting. The roofline model defines a \textit{ridge point} which is calculated as
\[\frac{\text{peak compute performance (FLOP/s)}}{\text{peak memory-BW (GB/s)}}.\]
Note that the ridge point has the same units as arithmetic intensity (FLOPs/byte). In the roofline model, any operation with an arithmetic intensity smaller than the ridge point 
is memory-bound, and any operation with an arithmetic intensity greater than the ridge point is compute-bound. For our analysis, we extrapolate this to a \textit{ridge plane} and use hardware specifications for an NVIDIA A6000 GPU to study inference for the Llama-2-7B model in 16 bit precision. 

For optimizing speculative decoding, we specifically focus on the decoding phase, although we include results for prefill in Appendix~\ref{appendix:prefill_ai_analysis}. Figure~\ref{fig:decode_arithmetic_intensity_analysis} shows the arithmetic intensity for generating 1k tokens at different context lengths and batch sizes for the Linear/Attention components as well as the aggregate arithmetic intensity.
To decide the ideal quantization strategy for different regimes, we consider the aggregate arithmetic intensity, which is colored by the percentage of the total latency taken up by attention and provides a complete view of decoding in all regimes. 
Based on these results, we can clearly see that in the small batch + short context regime, the memory operations for the linear projections dominate latency, so \textbf{weight quantization} could provide considerable speedup in this regime. In the small batch + long context, large batch + short context, and large batch + long context regimes, attention dominates latency due to the expensive load-store operations for the large KV cache. \textbf{KV cache quantization} could help provide performance improvements in these regimes. In the small batch + medium context and short context + medium batch regimes, the linear and attention operations are approximately equivalent in their contributions to total latency. Thus, both \textbf{weight and KV cache quantization} are ideal here.
\section{Implementation details}
\label{sec:detail} 
% Although RLHF methods have been successful in fine-tuning language models, adapting them to diffusion models has been challenging based on the fact that sampling from diffusion models is inefficient and training diffusion models is expensive. 
% Therefore, our goal is to derive a simple yet effective approach that can directly optimize diffusion models to align with human preferences.
% 

% \se{i think you need to start by sating that's the setting. you have a reference diffusion model, you have a dataset of prompt,image1,image2 and annotations on the winning one, and the goal is to align the model to the human preferences}

In this section, we present our iterative preference optimization framework, along with its implementation details and pipeline. An overview of the algorithm is illustrated in Fig. \ref{fig:Figure_pipeline}. Specifically, the data collection and annotation process are described in detail in Section \ref{collect dataset}, while Section \ref{reward_training} elaborates on the training methodology for the Critic model. Finally, Section \ref{pipeline_training} introduces a comprehensive framework for aligning human feedback, encompassing the training of DPO on paired data and KTO on single data.
% \textcolor{green}{



% In order to obtain a comprehensive model that can handle paired data sorting and single video scoring, we fine-tune the pre-trained model Vila\cite{lin2023vila}, which already has strong image and video instruction following capabilities, comprehension capabilities, and multi-image processing capabilities, which are essential for training our reward model. During training, restructure the prompts and add detailed reasons before scoring point by point or ranking in pairs. Construct a chain of thoughts. After training, the model will give detailed reasons before giving an answer, improving the accuracy of the final result while providing richer natural language feedback.

% In our experiments, we start from the Vila 13B/40B pre-trained checkpoint and fine-tune it for 5 epochs on the labeled dataset to train the reward model. We use a learning rate of 2e-5, 32 GPUs, and a batch size of 4 per GPU. In addition, to reduce the impact of scaling and center cropping preprocessing on the consistency of text and video, we use pad training. In order to enhance the understanding of video actions, 16 frames are extracted from each video. Other parameters are the same as Vila default settings.

\subsection{Iterative training pipeline} \label{pipeline_training}


% \begin{algorithm}[t]
% \small
% \DontPrintSemicolon
%     \KwInput{Prompt Set $\mathcal{M} = \{x_0,\cdots,x_n \}$, Video Diffusion Model $\mathcal{G}(\cdot)$, Video Reward Model $\mathcal{R}(\cdot)$, Curriculum Update Interval $K$} 
%    \KwOutput{Preference-aligned Video Diffusion Model $\mathcal{G}^*(\cdot)$} 
%     $\mathrm{step} = 0$ \\
%     \For{$x_i$ \text{in} $\mathcal{M}$}{
%         \textcolor{commentGreen}{// Online Preference Sample Generation}
%         $\mathcal{V} = \{y_1, y_2, \cdots, y_N \} \sim \mathcal{G}(x_i)$ 
%         $\mathcal{S} = \{\mathcal{R}(y_1), \mathcal{R}(y_2), \cdots, \mathcal{R}(y_N) \}$
%         $\tilde{y}_{w} = \mathcal{V}_{max_{i} \mathcal{R}(y_i)}; $
%         $\tilde{y}_{l} = \mathcal{V}_{min_{i} \mathcal{R}(y_i)}$\\
%         $\mathcal{L}_{\textrm{DPO}} = \mathbb{E}\left[\log \sigma\left(\beta \log \frac{\mathcal{G}_{\theta}\left(\tilde{y}_{w} \mid x\right)}{\mathcal{G}_{\mathrm{ref}\left(\tilde{y}_{w} \mid x\right)}}-\beta\log \frac{\mathcal{G}_{\theta}\left(\tilde{y}_{l} \mid x\right)}{\mathcal{G}_{\mathrm{ref}}\left(\tilde{y}_{l} \mid x\right)}\right)\right]$

%         \textcolor{commentGreen}{// Update $\mathcal{G}_{\theta}$ with DPO loss}
    
%         $\mathcal{G}_{\theta} \leftarrow \mathcal{G}_{\theta} + \nabla_{\mathcal{G}_{\theta}}\mathcal{L}_{\mathrm{DPO}} $ 

%         $\mathrm{step} = \mathrm{step} + 1$

%         \textcolor{commentGreen}{// Curriculum Preference Update}
        
%         \If{$(\mathrm{step}$ $\bmod$ $K)$ $=$ 0}{
%          $\mathcal{G}_{\mathrm{ref}} = \mathcal{G}_{\theta}$

%         }
       
%     }

% \caption{Online Video Preference Optimization (OnlineVPO). }
% \label{algo:online_preferecence}
% \end{algorithm}
% \vspace{-0.2cm}

Iterative preference optimization provides substantial advantages over offline learning methods. The iterative learning framework we propose is a \textbf{versatile} and \textbf{general-purpose} approach, capable of seamlessly integrating various offline learning algorithms into an iterative setting, thereby achieving significant performance enhancements. Specifically, we investigate two distinct offline preference alignment methods: the \textbf{KTO algorithm}, which utilizes pointwise scoring, and the \textbf{DPO algorithm}, which is based on pairwise ranking.

\paragraph{Iterative DPO} In the iterative preference optimization process employing Direct Preference Optimization (DPO), the system follows a structured workflow: during each iteration, the current optimal policy model generates multiple video outputs for a given prompt. These generated videos are subsequently evaluated and ranked by a critic model based on predefined quality metrics. The ranked outputs then serve as training data for preference alignment, where the policy model is fine-tuned to better align with the desired preferences. This cyclic process of generation, evaluation, and optimization continues iteratively, progressively enhancing the model's performance through successive refinement cycles.
\begin{itemize}
\setlength{\itemsep}{2pt}
\item \textit{\textbf{Video Generation}.} We use CogvideoX-2B as the initial checkpoint, and generate at least three random videos for the same prompt by setting different seeds to to form a dataset $\mathcal{V}$ = $\{(x_k, y_{k_1}, y_{k_2}, y_{k_3},...)\}_{k=1}^N$,  where $x_k$ represents the prompt and $y_k$ represents the corresponding video generated by the model. 

\item \textit{\textbf{Pairwise Ranking}}.
To enhance ranking reliability and mitigate potential biases in the Critic model, we implement a robust pairwise comparison protocol with position swapping during inference: for each video pair, we conduct two reciprocal evaluations by alternating their presentation order, retaining only consistent results as valid entries for the preference dataset. Formally, we construct the pairwise dataset as $\mathcal{D} = \{(x_k, y_{k_w}, y_{k_l})\}_{k=1}^N$, where $y_{k_w}$ and $y_{k_l}$ denote the preferred (winning) and dispreferred (losing) samples, respectively. This rigorous validation process ensures higher data quality and reduces positional bias in the collected preference pairs.

\item  \textit{\textbf{Preference learning}.} 
During training, we employ a selective sampling strategy to construct high-quality preference pairs by extracting only the highest-ranked and lowest-ranked samples from the critic model's evaluations, forming the final paired data $(y^{w}, y^{l})$ while deliberately excluding intermediate-ranked samples to ensure clearer preference distinctions. For optimization, we utilize the diffusion-DPO loss as the primary objective function and introduce two auxiliary negative log-likelihood (NLL) loss terms following ~\cite{pang2024iterativereasoningpreferenceoptimization}: the first term, weighted by 0.2, regularizes the preferred samples, while the second term, scaled by 0.1, anchors the model to real video data distributions. This dual regularization strategy preserves the structural integrity and format of generated outputs while preventing undesirable degradation in the log-probability of selected responses, as demonstrated in ~\cite{pal2024smaugfixingfailuremodes}, ~\cite{grattafiori2024llama3herdmodels}, and ~\cite{pang2024iterativereasoningpreferenceoptimization}.
% \begin{equation}
% \begin{aligned}
% \mathcal{L}(\theta) = L_\text{DPO-Diffusion}(\theta) \\
% &+ 
% \lambda \cdot \mathbb{E}_{(\textbf{\textit{y}}, x) \sim \mathcal{D}^{\mathrm{sample}}}
% \left[-\log p_{\theta}(y^w|x)\right] \\
% &+
% \lambda \cdot \mathbb{E}_{(\textbf{\textit{y}}, x) \sim \mathcal{D}^{\mathrm{real}}}
% \left[-\log p_{\theta}(y|x)\right],
% \end{aligned}
% \end{equation}
\begin{equation}
\begin{aligned}
\mathcal{L}(\theta) &= L_{diffusion-dpo}(\theta) \\
&\quad + \lambda_1 \cdot \mathbb{E}_{(\textbf{\textit{y}}, x) \sim \mathcal{D}^{\mathrm{sample}}} \bigl[-\log p_{\theta}(y^w|x)\bigr] \\
&\quad + \lambda_2 \cdot \mathbb{E}_{(\textbf{\textit{y}}, x) \sim \mathcal{D}^{\mathrm{real}}} \bigl[-\log p_{\theta}(y|x)\bigr],
\end{aligned}
\end{equation}
\end{itemize}

\paragraph{Iterative KTO} KTO~\cite{ethayarajh2024ktomodelalignmentprospect} is an offline learning method that leverages pointwise scoring data, eliminating the need for preference-based datasets. By directly utilizing binary-labeled data, KTO effectively trains algorithms while demonstrating heightened sensitivity to negative samples. This makes it particularly adept at handling imbalanced datasets with uneven distributions of positive and negative samples. Furthermore, KTO achieves significant performance improvements even without relying on the SFT stage. In scenarios where data is imbalanced, paired sorting data is unavailable, or preference data is noisy, KTO stands out as a more effective and robust solution.
\begin{itemize}
\setlength{\itemsep}{2pt}
\item ~\textbf{\textit{Video Generation}.} Similar to the DPO training framework, we adopt an identical video sampling strategy in this study. In alignment with the KTO~\cite{ethayarajh2024ktomodelalignmentprospect} methodology, empirical evidence suggests that sampling multiple videos for a given prompt yields superior results compared to single-video sampling. Therefore, we maintain consistency with the DPO sampling protocol throughout our implementation.

\item \textit{\textbf{Pointwise scoring}}.
The Critic model is employed to evaluate and assign scores to all generated text-video pairs 
$(y, x)$. Each data sample is categorized into three distinct quality levels: "Good", "Normal", and "Bad". Based on this classification, we designate samples labeled as "Good" and "Normal" as positive instances, while those marked as "Bad" are treated as negative instances for subsequent analysis and model training.
% The reward model is used to score all the generated text and video data $(y, x)$. Each sample has three different scores: "Good", "Normal" and "Bad". For "Good" and "Normal", we label them as positive samples, and "Bad" as negative samples.

\item  \textit{\textbf{Preference learning}.} 
We use the Eq.~\ref{eq:dkto_loss} for training. KTO needs to estimate the average reward Q during training. When the batch size is small, the Q value is not accurately estimated and the model training will be unstable. We set the batch size to 128 and achieved good results. To stabilize the training, we also sampled data from the real data VidGen-1M~\cite{tan2024vidgen1mlargescaledatasettexttovideo} and added negative log-likelihood loss.
\begin{equation}
\begin{aligned}
\mathcal{L}(\theta) &= L_{diffusion-kto}(\theta) \\
&\quad + \lambda \cdot \mathbb{E}_{(\textbf{\textit{y}}, x) \sim \mathcal{D}^{\mathrm{real}}} \bigl[-\log p_{\theta}(y|x)\bigr],
\end{aligned}
\end{equation}

\end{itemize}

Both DPO and KTO employ an iterative learning framework that cyclically executes three core steps: (1) strategic data sampling, (2) model-based preference labeling, and (3) alignment optimization. The process incorporates early stopping based on validation metrics while adaptively optimizing strategies for each component to maximize training efficiency and model performance.

\begin{table*}[t]
\centering
\caption{
Accuracy versus KV cache size for REBASE as well as \ours. 
Results are provided for MATH500 and GSM8K for the Llemma-34B and Mistral-7B-SFT models.
We report the KV cache size reduction (``KV Red.'') for each width (relative to REBASE), where higher is better since it implies a reduction in memory consumption.
}
\label{tab:results}
\scriptsize
\setlength{\tabcolsep}{5pt}
\renewcommand{\arraystretch}{1.2}
%
%----------------- LEFT SUB-TABLE -----------------%
\vspace{1.5mm}
\begin{subtable}{0.47\linewidth}
    \centering
    \begin{tabular}{c|cc|cc|cc}
    \toprule
    \multirow{2}{*}{\textbf{Method}} & \multicolumn{2}{c|}{\textbf{Width=16}} & \multicolumn{2}{c|}{\textbf{Width=64}} & \multicolumn{2}{c}{\textbf{Width=256}} \\
    \cline{2-7}
     & \textbf{Acc.} & \textbf{KV Red.} & \textbf{Acc.} & \textbf{KV Red.} & \textbf{Acc.} & \textbf{KV Red.} \\
    \midrule
    \multicolumn{7}{c}{\textbf{Llemma-34B}} \\
    \midrule
    REBASE       &   47.2         & 1$\times$   &           50.8 & 1$\times$  &   52.0         & 1$\times$  \\
    \hd \textbf{\ours}   &  \textbf{47.0}          & \textbf{1.2}$\times$  &  \textbf{51.2}          & \textbf{1.5}$\times$  &    \textbf{52.8}        & \textbf{1.8}$\times$  \\

    \midrule
    \multicolumn{7}{c}{\textbf{Mistral-7B-SFT}} \\
    \midrule
    REBASE      &     38.8       & 1$\times$   &  43.4          & 1$\times$   &     42.4       & 1$\times$   \\
    \hd \textbf{\ours}   &   \textbf{39.4}         & \textbf{1.3}$\times$  &  \textbf{43.2}          & \textbf{1.3}$\times$  &    \textbf{42.2}        & \textbf{1.7}$\times$  \\
    \bottomrule
    \end{tabular}
    % \vspace{1ex}
    \newline
    \newline
    \small\textbf{MATH500}
\end{subtable}
% \hfill
\hspace{3mm}
%----------------- RIGHT SUB-TABLE -----------------%
\begin{subtable}[t]{0.47\linewidth}
    \centering
    \begin{tabular}{c|cc|cc|cc}
    \toprule
    \multirow{2}{*}{\textbf{Method}} & \multicolumn{2}{c|}{\textbf{Width=16}} & \multicolumn{2}{c|}{\textbf{Width=64}} & \multicolumn{2}{c}{\textbf{Width=256}} \\
    \cline{2-7}
     & \textbf{Acc.} & \textbf{KV Red.} & \textbf{Acc.} & \textbf{KV Red.} & \textbf{Acc.} & \textbf{KV Red.} \\
    \midrule
    \multicolumn{7}{c}{\textbf{Llemma-34B}} \\
    \midrule
    REBASE       &    87.7        & 1$\times$   &   89.0         & 1$\times$   &      89.3      & 1$\times$   \\
    \hd \textbf{\ours}   &    \textbf{87.5}        & \textbf{1.5}$\times$  &   \textbf{89.3}         & \textbf{1.7}$\times$ &   \textbf{89.3}          & \textbf{1.8}$\times$  \\
    \midrule
    \multicolumn{7}{c}{\textbf{Mistral-7B-SFT}} \\
    \midrule
    REBASE       &     88.6       & 1$\times$   &  89.1          & 1$\times$   &   90.1        & 1$\times$   \\
    \hd \textbf{\ours}   &    \textbf{88.3}        & \textbf{1.2}$\times$  &   \textbf{89.2}         &  \textbf{1.6}$\times$  &     \textbf{89.6}       & \textbf{1.3}$\times$  \\
    \bottomrule
    \end{tabular}
    % \vspace{1ex}
    \newline
    \newline
    \small\textbf{GSM8K}
\end{subtable}
\end{table*}

\section{Evaluation}

\subsection{Experimental Details}

\label{sec:experimental-details}

We leverage the open-source REBASE code for the balanced sampling implementation \cite{wu2024inference}, and we use SGLang for serving models \cite{zheng2023efficiently}.
For all experiments in our evaluation, we leverage temperature sampling with a temperature of 1.0 and we fix the REBASE temperature at 0.2, which are the default settings in the open-source code for \cite{wu2024inference}.
We use the final PRM score at each step as the reward for that step, and we select the final answer with weighted majority voting using the final PRM score for each trajectory as the weight.
We use these aggregation strategies since they have been shown to outperform other methods of aggregating trajectories to determine the final response \cite{beeching2024scalingtesttimecompute}.
As in \cite{wu2024inference}, we reduce the search width each time a retained trajectory completes.
We set $\lambda_d = 1$ throughout our evaluation, and we sweep over $\lambda_b \in [1,2]$ (with increasing $\lambda_b$ corresponding to more aggressive KV compression) and select the largest value of $\lambda_b$ which doesn't degrade accuracy by greater than $0.2\%$.

We provide results for two groups of models. We leverage the Llemma-34B model (finetuned on Metamath) along with the Llemma-34B PRM from \cite{wu2024inference}.
We also use the Mistral-7B model finetuned on Metamath as well as the corresponding Mistral-7B PRM from the Math-Shepherd paper \cite{wang2024math}.
We evaluate these models for search widths of 16, 64, and 256, and we report results on the MATH500 and GSM8K \cite{cobbe2021training} datasets.

We compare against several baseline search strategies.
We include results for beam search both with 4 trajectories retained at each step and with  $\sqrt{N}$ trajectories retained at each step, where $N$ is the initial width of the search, as in \cite{snell2024scaling}.
We also include comparisons with DVTS both with 4 trajectories retained at each step and with $\sqrt{N}$ trajectories retained at each step (where the number of trajectories retained at each step is also the same as the number of separate subtrees), as in \cite{beeching2024scalingtesttimecompute}.
Finally, we provide comparisons against REBASE, which serves as our strongest baseline due to its high accuracy relative to search width \cite{wu2024inference}.



\subsection {Results}



Figure \ref{fig:results} provides evaluation of our methodology as well as comparisons against several baseline methods.
We report accuracy results relative to efficiency (in terms of total KV cache size), and we provide results for the Llemma-34B model for both MATH500 and GSM8K datasets.
We also report results for the beam search, DVTS, and REBASE baseline methods.
Our results demonstrate that our approach, which considers both diversity and efficiency, is able to attain a better accuracy versus efficiency trade-off than existing search strategies.

Table \ref{tab:results} shows results for the Llemma-34B and Mistral-7B models.
We report results on both MATH500 as well as GSM8K.
We provide accuracy results as well as profiled KV cache compression estimates.
These results highlight how the benefits of our search strategy are consistent across different model families and datasets, as we are able to maintain accuracy while obtaining consistent efficiency benefits relative to the REBASE baseline \cite{wu2024inference}.

\subsection{Throughput Benchmarking}

\begin{table}[t!]
\caption{
Throughput for our approach relative to REBASE \cite{wu2024inference}.
Results were measured on NVIDIA H100 GPUs using the Llemma-34B model, evaluated on 100 samples from the MATH500 test set (with the accuracy reported for the full test set).
We report throughput improvements using a beam width of 256.
We also include the reduction in KV cache size (normalized to REBASE), as well as the accuracy for each approach.
}
% \vspace{-5mm}
\scriptsize
% \vspace{1mm}
\vspace{2mm}
\label{tab:throughput}
\centering{
\resizebox{0.9\linewidth}{!}{
\begin{tabular}{c|c|c|c}
\toprule
   \textbf{Method} & \textbf{Accuracy} & \textbf{KV Reduction} & \textbf{Throughput}\\
   \midrule
   \midrule
   REBASE  & 52.0 & 1$\times$ & 1$\times$  \\
   
   % \hc \ours  & & & \\ 
   \hd \textbf{\ours}  & \textbf{52.8} & \textbf{1.8}$\times$ &  \textbf{1.4}$\times$ \\
\bottomrule
\end{tabular}
}
}
         
\end{table}



Table \ref{tab:throughput} provides measured throughput for REBASE as well as \ours on H100 NVL GPUs.
We benchmark both REBASE and \ours using [4,8,16,32] parallel threads (which is representative for the serving scenario with a batch size equal to the number of threads) and select the best configuration for each.
We run benchmarking with the main LLM and the PRM each on a separate H100 NVL GPU, and for \ours we co-locate the embedding model on the same GPU as the reward model.
We observe {1.4}$\times$ increased throughput relative to the baseline REBASE method, demonstrating how the increased KV cache sharing from our algorithm translates to higher throughput, without requiring any custom kernels.

\subsection{Ablation}

\begin{table}[t!]
\caption{
Ablation for our methodology. We include results on MATH500 for different beam widths with the Llemma-34B model, and we report KV budget estimates.
We compare \ours with only applying the KV budget term in the cost model (``\ours-KV'').
}
\scriptsize
\vspace{2.5mm}
\label{tab:ablations}
\centering{
\begin{tabular}{c|cc|cc|cc}
    \toprule
    \multirow{2}{*}{\textbf{Method}} & \multicolumn{2}{c|}{\textbf{Width=16}} & \multicolumn{2}{c|}{\textbf{Width=64}} & \multicolumn{2}{c}{\textbf{Width=256}} \\
    \cline{2-7}
     & \textbf{Acc.} & \textbf{KV Red.} & \textbf{Acc.} & \textbf{KV Red.} & \textbf{Acc.} & \textbf{KV Red.} \\
    \midrule
    REBASE       &   47.2         & 1$\times$   &           50.8 & 1$\times$  &   52.0         & 1$\times$  \\
    \hc \ours-KV &  \textbf{47.2}          & \textbf{1.3}$\times$  &  51.4          & 1.3$\times$  &    52.8        & 1.7$\times$  \\
    \hd \textbf{\ours}   &  47.0          & 1.2$\times$  &  \textbf{51.2}          & \textbf{1.5}$\times$  &    \textbf{52.8}        & \textbf{1.8}$\times$  \\
    \bottomrule
\end{tabular}
}
         
\end{table}

Table \ref{tab:ablations} also provides an ablation for our method.
We report results for the Llemma-34B model on the MATH500 dataset when using our cost model method with only the term that promotes KV cache sharing and with both terms together.
For the results with only our KV cache sharing term in the cost model, we fix $\lambda_d = 0$ and sweep over $\lambda_b \in [0.75,1.25]$, selecting the largest value of $\lambda_b$ which doesn't degrade accuracy by greater than $0.2\%$.
These results highlight that the diversity term allows us to push to more aggressive compression without degrading accuracy.
This occurs since without the diversity term, our method cannot distinguish redundant trajectories from necessary diverse trajectories when determining which subset to retain.

\section{Conclusion}

Computational efficiency is a key bottleneck for exploiting test-time scaling in order to enhance model accuracy by allowing models to think longer before responding.
An emerging approach for exploiting test-time scaling is through tree search against a verifier.
A key challenge with existing tree search methods is the trade-off between efficiency and accuracy; high accuracy with tree search necessitates diverse trajectories, but retaining diverse trajectories leads to high inference costs due to reduced KV cache sharing in the tree.
We perform profiling which demonstrates the importance of KV cache sharing, and show that existing efficiency metrics like FLOPs and model calls are insufficient for assessing the efficiency trade-offs between tree search methods due to the impacts of KV sharing.
We then propose \ours, a search strategy which promotes KV cache sharing while retaining diverse trajectories in order to attain high accuracy.
\ours encourages KV cache sharing in the tree search by penalizing divergent branches in the tree.
Our method also incorporates a coverage term which ensures that semantically diverse trajectories are maintained, even while we prune unnecessary redundant trajectories.
By retaining diverse trajectories, we are able to perform sufficient exploration to retain the accuracy benefits of diverse tree search.
The combination of these components of our method allows us to retain necessary diversity while pruning out redundancy in order to enable accurate and efficient tree search.
\ours achieves \textbf{1.8}$\times$ reduction in average KV cache size during the search process, which translates to \textbf{1.4}$\times$ increased throughput, with minimal accuracy loss and without requiring custom kernel implementations.
Our method demonstrates the potential of leveraging efficiency considerations in the search process to enable accurate and efficient search for test-time scaling.



\section*{Acknowledgements}
We acknowledge gracious support from Apple team.
We also appreciate the support from Microsoft through their Accelerating Foundation Model Research.
Furthermore, we appreciate support from
Google Cloud, the Google TRC team, and specifically Jonathan Caton, Divy Thakkar, and Prof. David Patterson.
Prof. Keutzer's lab is sponsored by the Intel corporation, Intel One-API, Intel VLAB team, the Intel One-API center of
excellence, as well as funding through BDD, BAIR, and Furiosa.
We appreciate great feedback and support from Ellick Chan, Saurabh Tangri, Andres
Rodriguez, and Kittur Ganesh.
Sehoon Kim would like to acknowledge the support from the Korea Foundation for Advanced Studies (KFAS).
Michael W. Mahoney would also like to acknowledge
a J. P. Morgan Chase Faculty Research Award 
as well as 
the DOE, NSF, and ONR.
Our conclusions do not necessarily reflect the position or the policy of our sponsors, and no official endorsement should be~inferred.
\bibliography{references.bib}
\bibliographystyle{icml2025}


%%%%%%%%%%%%%%%%%%%%%%%%%%%%%%%%%%%%%%%%%%%%%%%%%%%%%%%%%%%%%%%%%%%%%%%%%%%%%%%
%%%%%%%%%%%%%%%%%%%%%%%%%%%%%%%%%%%%%%%%%%%%%%%%%%%%%%%%%%%%%%%%%%%%%%%%%%%%%%%
% SUPPLEMENTAL CONTENT AS APPENDIX AFTER REFERENCES
%%%%%%%%%%%%%%%%%%%%%%%%%%%%%%%%%%%%%%%%%%%%%%%%%%%%%%%%%%%%%%%%%%%%%%%%%%%%%%%
%%%%%%%%%%%%%%%%%%%%%%%%%%%%%%%%%%%%%%%%%%%%%%%%%%%%%%%%%%%%%%%%%%%%%%%%%%%%%%%
\newpage
\appendix
\onecolumn
\appendix
\clearpage
\onecolumn
\leftline{ {\Large Appendix } }
% \section{Appendix}

\section{Attention Module's Inference Workflow}
The inference of LLMs can be divided into 2 parts: the prefill stage and the decoding stage. In the \textbf{prefill stage}, for the input sequence $X\in\mathbb{R}^{B\times S_L\times d}$, the KV cache update rule can be calculated as \[Q = XW_Q, ~~C_K = XW_K, ~~C_V = XW_V,\]
where we denote the query, key, and value weight matrices as $W_Q, W_K, W_V \in \mathbb{R}^{d\times d}$ and denote the key and value caches as $C_K$ and $C_V$ respectively. $B$ refers to batch size, $S_L$ refers to sequence length, and $d$ refers to hidden size. We then calculate the multi-head attention (MHA) as:
\[ O = \operatorname{MultiHeadAttn}(Q,~C_K,~C_V). \]

In the \textbf{decode stage}, for input token $x\in\mathbb{R}^{B\times 1\times d}$, we first calculate the query, key, and value of the current token:
\[q = xW_Q, ~~c_k = xW_K, ~~c_v = xW_V,\]
then concatenate the KV cache with the current token's key and value to update the KV cache:
\[ C_K = \operatorname{concat}(C_K, c_k), ~~C_V = \operatorname{concat}(C_V, c_v).\] 
Then, the multi-head attention output is calculated:
\[ O = \operatorname{MultiHeadAttn}(q,~C_K,~C_V). \]



\section{More Related Works}\label{appendix:rel_works}
We list some related works that we find interesting, but can not elaborate on in the related works section due to space limitations.

\paragraph{Efficient Long Context Inference}
Some research maintains the full key-value pairs but dynamically loads them from high-bandwidth memory~\cite{yang2024doublesparse,tang2024quest}, and usually achieves higher performance at the cost of higher memory consumption. Shared KV cache across tokens~\cite{nawrot2024dynamic} and layers~\cite{brandon2024reducing} provides a new way to reduce the KV cache budget through sharing.

\paragraph{Quantization} Any Precision representation~\cite{park2024any} incorporates multiple precision levels (e.g., INT2, INT4, and INT8) within a single representation, eliminating the need to store separate KV caches for each precision and allowing the framework to dynamically select the optimal precision based on the complexity of the task. Training quantization~\cite{peng2023fp8,xi2024jetfire,fishman2024fp8trillion,xi2024coatcompressingoptimizerstates} reduces the bit precision of various model parameters, gradients, and activations to accelerate training.
Attention quantization~\cite{chen2024int8attn,zhang2024sageattention,shah2024flashattention3} reduces the computational overhead associated with attention computations, which becomes dominant in the prefill stage of the long-context inference setting.

\paragraph{Speculative Decoding} Zhao et al.,~\cite{zhao2024qspec} explored complementary quantization schemes in speculative decoding with QSpec, enhancing efficiency without significant performance degradation. 
Sirius~\cite{zhou2024sirius} finds that contextual sparsity will lead to poor performance under the speculative decoding setting since the model performance is degraded, and thus it cannot accelerate LLM inference.

\section{Additional LLM Inference Bottlenecks Analysis}
\label{appendix:appendix_inference_bottlenecks_analysis}
\subsection{Prefill Arithmetic Intensity Analysis}
\label{appendix:prefill_ai_analysis}

Keeping in line with the asymptotic analysis in Table ~\ref{tab:asymptotic_analysis}, the arithmetic intensity of attention during prefill does not scale with batch size at all, as attention is unable to benefit from batching in the same way that linear operations do. Moreover, for long contexts, attention entirely dominates the linear operations due to the quadratic nature of self-attention. For short contexts however, this quadratic cost is relatively inexpensive when compared to the linear operations. As shown in Figure~\ref{fig:prefill_arithmetic_intensity_analysis}, the arithmetic intensity for all prefill operations in all regimes is above the ridge plane, which means that prefill is entirely compute-bound.

% Combined Arithmetic Intensity Analysis Figures
%%%%%%%%%%%%%%%%%%%%%%%%%%%%%%%%%%%%
\begin{figure*}[h]
    \centering
    \includegraphics[width=\linewidth]{figs/prefill_ai_analysis.png}
    \caption{During prefill, all regimes lie above the ridge plane and thus are compute-bound. }
    \label{fig:prefill_arithmetic_intensity_analysis}
\end{figure*}
%%%%%%%%%%%%%%%%%%%%%%%%%%%%%%%%%%%%

\subsection{Modern GPU Hardware VRAM Size Constraints}
\label{appendix:gpu_memory_constraints}
% KV Cache vs. GPU Memory Figure
%%%%%%%%%%%%%%%%%%%%%%%%%%%%%%%%%%%%
\begin{figure}[H]
    \centering
    \includegraphics[width=0.6\linewidth]{figs/kv_cache_vs_gpu_memory.png}
    \caption{KV cache memory usage by Llama-2-7B on a single node (8 GPUs) as context length and batch size are scaled logarithmically. The surface plot's color represents the ratio of KV cache memory to the model weights memory. The dotted-lines represent GPU DRAM capacities for several different GPUs. At ($B=16, S_L=262$k), the KV cache takes up 160x more memory than the model weights.}
    \label{fig:kv_cache_vs_gpu_memory}
\end{figure}
%%%%%%%%%%%%%%%%%%%%%%%%%%%%%%%%%%%%
The relatively higher linear arithmetic intensities observed in decoding for batch sizes greater than 8 in Figure~\ref{fig:decode_arithmetic_intensity_analysis} are misleading due to the limited VRAM sizes in modern GPUs. As shown in Figure~\ref{fig:kv_cache_vs_gpu_memory}, the size of the KV cache for a Llama-2-7B model exceeds the total VRAM capacities of a single node equipped with 8 A100/H100 GPUs with 80 GB of memory each. This means that simply scaling the batch size for decoding will not translate the memory-bound nature of generation to being compute-bound.

\section{Quantization Strategies for KV cache}
\label{appendix:quantization_strategies}
Here we provide a visualization of our quantization scheme in Figure~\ref{fig:group_quant_asymmetric_quant}. We apply asymmetric quantization for both the keys and values cache, and apply channel-wise quantization to the key cache and apply token-wise quantization to the value cache. We also provide a table to show that this quantization scheme offers the best performance in Table~\ref{tab:token_channel_wise_quant} by showing that it gives the lowest perplexity.

\begin{figure}[h!]
    \centering
    \includegraphics[width=0.6\linewidth]{figs/quant_method.png}
    \caption{We apply asymmetric and per-group quantization for both the key cache and value cache, along the channel axis and token axis, respectively. This figure describes how it works when only the upper-4 bit cache is applied.}\label{fig:group_quant_asymmetric_quant}
\end{figure}

\begin{table}[h]
    \centering
    \begin{tabular}{c|cc}
        \toprule
         & \multicolumn{2}{c}{Key Cache} \\
        \midrule
        Value Cache & token-wise & channel-wise \\
        \midrule
        token-wise & 6.587 & \textbf{6.507} \\
        channel-wise & 7.041 & 6.911 \\
        \bottomrule
    \end{tabular}
    \caption{Perplexity of Llama-2-7B on WikiText-2 dataset with different quantization strategies. Group size $G = 128$. Channel-wise quantization for key cache and token-wise quantization for value cache gives the best performance.}\label{tab:token_channel_wise_quant}
\end{table}

\label{appendix:fp_cache_buffer}
\begin{figure}[h]
    \centering
    \includegraphics[width=0.9\linewidth]{figs/DoubleCacheBuffer.pdf}
    \caption{Our KV cache with 2 full precision cache buffers for recent KV cache.}
    \label{fig:double_size_fp_buffer}
\end{figure}

\section{Compatibility with Flash Decoding}\label{appendix:flash_decoding}
Our full-precision buffer design, as shown in Figure~\ref{fig:double_size_fp_buffer}, is fully compatible with Flash Decoding~\cite{flashdecoding}, a fast attention implementation available for the decoding stage. In this setup, the quantized section divides naturally into several chunks, aligning perfectly with the structure of Flash Decoding. For the full-precision buffer, given its upper bound length of $2G$, it can be processed independently with minimal overhead. This full-precision segment can be treated as an additional chunk, which can then be summed with the quantized segments and seamlessly integrated into the Flash Decoding algorithm.



\section{Details about the Datasets Used}
\label{appendix:datasets}
We provide an overview of the datasets used in our experiments, highlighting their key characteristics.

\begin{itemize}
    \item \textbf{WikiText-2} \cite{merity2016pointer}: WikiText-2 is a widely used dataset for language modeling. It is a subset of the larger WikiText dataset and consists of high-quality, clean, and well-structured English text extracted from Wikipedia articles. 
    \item \textbf{C4} \cite{raffel2020exploring}: C4 is a large scale web-crawled language modelling dataset mostly used for pretraining LLMs.
    \item \textbf{PG-19} \cite{raecompressive2019pg19}: It is a dataset of books from Project Gutenberg, designed for long-context language modeling.
    \item \textbf{$\infty$B{\scriptsize ENCH} Sum} \cite{zhang2024inftybenchextendinglongcontext}: InfiniteBench benchmark is tailored for evaluating the capabilities of language models to process, understand, and reason over super long contexts. We used one of its summarization datasets where the task is to summarize a fake book created by core entity substitution. The average length of input prompt is $\sim$171k.
    \item \textbf{Multi-LexSum} \cite{shen2022multilexsum}: Multi-LexSum is a multi-doc summarization dataset for civil rights litigation lawsuits. The average length of prompt in this dataset is $\sim$90k.
\end{itemize}


\section{Hyperparameter Search}\label{appendix:hparam_search}
Here we present details about the hyperparameter search done to select optimal $\gamma$ for each experiment. We search $\gamma$ for each dataset and method pair using a prompt length of 8192 and use the same value for all other context length experiments. Table \ref{tab:hparam_search} shows the results of the search. We find that sparse-based methods achieves a maximum performance when $\gamma$ equals to 1, while our quantization-based method usually achieves the best performance with a larger $\gamma$, such as 4 or 6.

\begin{table*}[h]
\centering
\caption{Hyperparameter Search for Llama-2-7B-32K and LWM-Text-Chat-128k models on PG19, Multi-LexSum, and $\infty$B{\scriptsize ENCH} Sum datasets. Context length is kept as 8k.}
\label{tab:hparam_search}
\setlength{\tabcolsep}{5pt}
\renewcommand{\arraystretch}{1.2}
\resizebox{0.85\linewidth}{!}{
\begin{subtable}{0.48\linewidth}
    \centering
    \caption{Llama-2-7B-32k on PG19}
    \label{tab:llama_pg19}
    \begin{tabular}{lccc}
        \toprule
        \textbf{Method} & \textbf{$\gamma$} & \textbf{Acceptance Rate $\uparrow$} & \textbf{Speedup $\uparrow$} \\
        \midrule
        \multirow{3}{*}{StreamingLLM}
            & 1 & 90.78 & \textbf{39.1} \\
            & 2 & 89.42 & 38.5 \\
            & 3 & 90.21 & 38.66 \\
        \midrule
        \multirow{3}{*}{SnapKV}
            & 1 & 94.39 & \textbf{40.34} \\
            & 2 & 91.03 & 39.38 \\
            & 3 & 91.84 & 39.28 \\
        \midrule
        \multirow{4}{*}{\OURS}
            & 1 & 91.88 & 41.52 \\
            & 2 & 89.88 & \textbf{44.51} \\
            & 4 & 83.17 & 43.84 \\
            & 6 & 77.07 & 41.88 \\
        \bottomrule
    \end{tabular}
\end{subtable}
\hfill
\begin{subtable}{0.48\linewidth}
    \centering
    \caption{Llama-2-7B-32k on Multi-LexSum}
    \label{tab:llama_multilex}
    \begin{tabular}{lccc}
        \toprule
        \textbf{Method} & \textbf{$\gamma$} & \textbf{Acceptance Rate $\uparrow$} & \textbf{Speedup $\uparrow$} \\
        \midrule
        \multirow{3}{*}{StreamingLLM}
            & 1 & 90.78 & \textbf{39.17} \\
            & 2 & 86.82 & 37.84 \\
            & 3 & 83.29 & 36 \\
        \midrule
        \multirow{3}{*}{SnapKV}
            & 1 & 55.55 & \textbf{31.05} \\
            & 2 & 43.96 & 24.39 \\
            & 3 & 36.61 & 19.78 \\
        \midrule
        \multirow{4}{*}{\OURS}
            & 1 & 96.58 & 42.83 \\
            & 2 & 96.61 & 47.51 \\
            & 4 & 95.59 & 49.47 \\
            & 6 & 91.23 & \textbf{49.62} \\
        \bottomrule
    \end{tabular}
\end{subtable}
}

\vspace{1em}
\resizebox{0.85\linewidth}{!}{
\begin{subtable}{0.48\linewidth}
    \centering
    \caption{LWM-Text-Chat-128k on $\infty$B{\scriptsize ENCH} Sum}
    \label{tab:LWM-Text-Chat-128k_infbench}
    \begin{tabular}{lccc}
        \toprule
        \textbf{Method} & \textbf{$\gamma$} & \textbf{Acceptance Rate $\uparrow$} & \textbf{Speedup $\uparrow$} \\
        \midrule
        \multirow{3}{*}{StreamingLLM}
            & 1 & 81.79 & \textbf{37.17} \\
            & 2 & 74.86 & 34.33 \\
            & 3 & 64.48 & 30.14 \\
        \midrule
        \multirow{3}{*}{SnapKV}
            & 1 & 85.55 & \textbf{38.27} \\
            & 2 & 82.92 & 36.97 \\
            & 3 & 77.13 & 34.40 \\
        \midrule
        \multirow{4}{*}{\OURS}
            & 1 & 93.83 & 42.01 \\
            & 2 & 94.38 & 46.74 \\
            & 4 & 90.33 & \textbf{47.30} \\
            & 6 & 82.13 & 45.51 \\
        \bottomrule
    \end{tabular}
\end{subtable}
\hfill
\begin{subtable}{0.48\linewidth}
    \centering
    \caption{LWM-Text-Chat-128k on Multi-LexSum}
    \label{tab:LWM-Text-Chat-128k_multilex}
    \begin{tabular}{lccc}
        \toprule
        \textbf{Method} & \textbf{$\gamma$} & \textbf{Acceptance Rate $\uparrow$} & \textbf{Speedup $\uparrow$} \\
        \midrule
        \multirow{3}{*}{StreamingLLM}
            & 1 & 83.96 & \textbf{37.80} \\
            & 2 & 77.41 & 35.13 \\
            & 3 & 71.28 & 32.37 \\
        \midrule
        \multirow{3}{*}{SnapKV}
            & 1 & 89.25 & \textbf{39.04} \\
            & 2 & 83.53 & 37.22 \\
            & 3 & 80.04 & 35.48 \\
        \midrule
        \multirow{4}{*}{\OURS}
            & 1 & 95.94 & 42.79 \\
            & 2 & 95.06 & 47.10 \\
            & 4 & 92.55 & 48.15 \\
            & 6 & 87.73 & \textbf{48.20} \\
        \bottomrule
    \end{tabular}
\end{subtable}
}
\end{table*}



\section{Comparing Acceptance Rates} \label{app:acc_rate}
\begin{figure*}[h]
    \centering
    \includegraphics[width=0.6\linewidth]{figs/acc_rate.png}
    \caption{Acceptance rate of self-speculative decoding methods at different speculation length measured for model LWM-Text-Chat-128k on Multi-LexSum dataset.}
    \label{fig:acc_rate_vs_gamma}
\end{figure*}
Although the acceptance rates in Table~\ref{tab:results} for the sparse KV baselines and \OURS\ do not seem exceedingly different, this is misleading because the acceptance rates shown are for the optimal $\gamma$ values observed from our hyperparameter search in Table~\ref{tab:hparam_search}. Table~\ref{tab:results} effectively compares the acceptance rates of the baselines at very low $\gamma$ (e.g. 1) with those of \OURS\ at much higher $\gamma$ (e.g. 6). Here, we compare the acceptance rates of different self-speculative decoding algorithms. 
For fair comparison, Figure \ref{fig:acc_rate_vs_gamma} illustrates the acceptance rate between the draft and target models as a function of speculation length. We observe that \OURS\ consistently outperforms sparse KV approaches in terms of acceptance rate. Notably, as speculation length increases, the acceptance rate of sparse KV methods degrades much faster, whereas our method maintains high acceptance rates.


\begin{algorithm*}[h]
\caption{QuantSpec Algorithm} \label{alg:our_algorithm}
\begin{algorithmic}[1]
\setlength{\itemsep}{3pt}
\item[] \textbf{Input:} Model $M$, Upper 4-bit Cache $C_U$, Lower 4-bit Cache $C_L$, Full Precision Cache Buffer $[C_{F_1}, C_{F_2}]$, 
\item[] \textbf{Input:} Prefill Length $S_P$, Target Decode Length $S_D$, Prefill Context $X = [x_0, \cdots, x_{S_P-1}]$, Speculate Length $\gamma$
\item[] \textbf{Input:} Number of Layers $L$, Sensitive Layer Number $L_S$, Quantization Group Size $G$
\item[] \textbf{Function: } $\operatorname{PREFILL}$, $\operatorname{DRAFT}$, $\operatorname{TARGET}$, $\operatorname{VERITY}$, $\operatorname{QUANTIZE}$, $\operatorname{REJECTCACHE}$
\item[] \textbf{Notation: } Verified tokens $x_i$, generated draft tokens $g_i$, logits of draft model $q_i$, logits of target model $p_i$, number of tokens already been generated in total $N$, number of tokens already been generated in this speculate step $n$, number of tokens accepted in this speculate step $v$

\item[] \textbf{Prefill Stage}
\STATE $X_{S_P}, C_{KV} \leftarrow \operatorname{PREFILL}(M, X_{<S_P})$  \hfill \textcolor{gray}{$\triangleright$ KV Cache is written together for simplicity}

\STATE $\textcolor{orange!95!black}{C_U, C_L} \leftarrow \operatorname{QUANTIZE}(C_{KV_{:S_P-G}}, L_S)$ \hfill \textcolor{gray}{$\triangleright$ Prepare the hierarchical quantized KV Cache}
\STATE $\textcolor{orange!95!black}{C_{F_1}, C_{F_2}} \leftarrow C_{KV_{S_P-G:}}, \operatorname{None}$ \hfill \textcolor{gray}{$\triangleright$ Prepare the full-precision cache buffer for recent tokens}

\item[] \textbf{Decode Stage}
\WHILE{$N < S_D$}
    \STATE $n \leftarrow 0$ and $v \leftarrow 0$
    \WHILE{$n < \gamma$}
        \STATE $q_{n + 1}, \textcolor{orange!95!black}{C_{F_2}} \leftarrow \operatorname{DRAFT}(M, \textcolor{orange!95!black}{C_U, C_{F_1}, C_{F_2}, L_S}, X_{\leq S_P + N} + g_{<n})$
        \STATE Sample $g_{n + 1} \sim q_{n + 1}$ and $n \leftarrow n + 1$
    \ENDWHILE
    \STATE $p_1, \cdots p_{\gamma}, \textcolor{orange!95!black}{C_{F_2}} \leftarrow \operatorname{TARGET}(M, \textcolor{orange!95!black}{C_U, C_L, C_{F_1}, C_{F_2}}, X_{\leq S_P + N} + g_{<\gamma})$

    \FOR{$i=1$ to $\gamma$}
        \IF{$\operatorname{VERIFY}(g_i, p_i, q_i)$}
            \STATE $x_{N + i} \leftarrow g_i$ and $v \leftarrow v + 1$ 
        \ELSE
            \STATE $x_{N + i} \leftarrow \operatorname{CORRECT}(p_i, q_i)$ and $v \leftarrow v + 1$ 
            \STATE $\textcolor{orange!95!black}{C_{F_2}} \leftarrow \operatorname{REJECTCACHE}(\textcolor{orange!95!black}{C_{F_2}}, i)$ \hfill \textcolor{gray}{$\triangleright$ Clear the rejected KV cache from the full precision cache buffer}
            
            Break
        \ENDIF
        \IF{$i = \gamma$}
            \STATE $x_{N + \gamma + 1} \leftarrow p_{\gamma + 1}$ and $N \leftarrow N + 1$
        \ENDIF
    \ENDFOR

    $N \leftarrow N + v$ 
    \IF{$C_{F_2}$ is full}
        \STATE Concatenate $\operatorname{QUANTIZE(C_{F_1})}$ with \textcolor{orange!95!black}{$C_U \text{~and~} C_L$}  \hfill \textcolor{gray}{$\triangleright$ Quantize the first half of the full precision cache buffer}
        \STATE $\textcolor{orange!95!black}{C_{F_1} \leftarrow C_{F_2~:-G}}$ and $\textcolor{orange!95!black}{C_{F_2} \leftarrow C_{F_2~-G:}}$ \hfill \textcolor{gray}{$\triangleright$ Move the second part to the first part of full precision buffer}
    \ENDIF
\ENDWHILE

\end{algorithmic}
\end{algorithm*}



\end{document}