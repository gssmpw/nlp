\section{Related work}
\label{sec:related_work}

\textbf{Spiking Neural Networks (SNNs)}: 
Designing and training SNNs is challenging due to their sensitivity to hyperparameters such as membrane threshold and synaptic latency, both of which significantly impact performance ____. Consequently, many existing methods focus on achieving low-latency inference with improved convergence while maintaining accuracy ____. Traditional approaches like surrogate gradient learning ____ and temporal coding ____ have been further enhanced by advanced techniques ____. Notably, ____ introduced a gradient re-weighting mechanism to improve the temporal efficiency of SNN training.
%
To eliminate the need for manual threshold selection, ____ proposed a data-driven approach for threshold selection and potential initialization. Their method also facilitates the conversion of trained ANNs into SNNs, enabling efficient and high-performance training even in low-latency settings ($T=1,2,4$). These advancements are crucial for real-time applications. Beyond energy efficiency, SNNs have also been explored for their inherent robustness against adversarial ____ and model inversion attacks ____, further reinforcing their potential towards robust AI.

\textbf{Membership Inference Attacks: }The threat of Membership Inference Attacks (MIAs) was first demonstrated by ____ in a simple Machine Learning-as-a-Service (MLaaS) black-box setting. Since then, extensive research has explored the privacy risks associated with diverse neural network architectures for a wide range of applications ____. Despite significant advancements and robustness characteristics of SNNs ____, their vulnerability to MIAs remains largely unclear and underexplored ____.

The inconsistencies in evaluation metrics and experimental settings in existing studies have made direct comparisons of MIA techniques challenging ____. However, ____ presented MIA from first principles, emphasizing the importance of analyzing the Receiver Operating Characteristic (ROC) curve in attack's assessments. The ROC fully captures the tradeoff between True Positive Rate (TPR) and False Positive Rate (FPR) of the membership data across different classification thresholds. Reporting TPR under extremely low FPR conditions ($\leq$1\% and $\leq$ 0.1\%) is particularly crucial, as attackers prioritize confidently identifying members over overall accuracy. More recently, ____ proposed a state-of-the-art attack, called robust MIA (RMIA), and generalized all other existing MIAs under the umbrella of their attack formulation. RMIA also achieved highly effective attack performance with a limited number of shadow/reference models - auxiliary models trained on data with similar properties to the target model's training data.

\textbf{Concurrent Work: } While existing research primarily focuses on traditional ANNs, the membership privacy risks in SNNs remain largely unexamined. A recent study by ____ explored the robustness of SNNs against MIAs, incorporating diverse experimental settings and assessing the impact of data augmentation. However, despite these contributions, the study suffers from several critical limitations. It relies on biased evaluation metrics such as balanced accuracy, which can obscure the true effectiveness of MIAs, and employs outdated training techniques for SNNs ____. Nowadays, in many research papers, AUC and TPR at very low FPR are the main metrics to study the performance of MIAs.
Additionally, the evaluation is conducted on simple datasets, failing to provide meaningful insights into real-world scenarios. 
%
More importantly, the study neglects key advancements in attack methodologies, such as RMIA, limiting the comprehensiveness of its findings. Furthermore, the analysis is restricted to timestep variations in neuromorphic datasets, lacking a systematic investigation of static datasets. These shortcomings underscore the insufficiency of existing efforts in rigorously assessing membership privacy risks in SNNs. A more sound evaluation is necessary to bridge this gap and uncover the true privacy vulnerabilities of SNNs.

\vspace{-0.3cm}