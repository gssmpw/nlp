The robotics community has made  remarkable strides in recent years, boasting  advances that enable quadrupeds to hike mountain trails and perform parkour~\cite{miki_learning_2022, hoeller_anymal_2024}, autonomous drones to race at champion-level ability~\cite{kaufmann_champion-level_2023}, and robot hands to solve Rubik's Cubes~\cite{openai_solving_2019}. Shared across these feats is the process of simulated policy learning and direct deployment in the real world, termed \textit{Sim2Real}.

However, these remarkable Sim2Real innovations now target expensive, space-intensive, and high-maintenance systems, including quadrupeds, humanoids, drones, and dexterous manipulators. Such pricey, high-end platforms make it impossible for the general population of robotics builders and users to access the state-of-the-art techniques they employ, such as domain randomization and sensor simulation~\cite{kaufmann_champion-level_2023, huang_datt_2023, liao_berkeley_2024, hoeller_anymal_2024}.

In contrast, small-scale autonomous RC cars offer low-cost\footnote{About 3000 USD or less. See~\cite{samak_autodrive_2023} for a comparative analysis.} yet performative platforms that can leverage modern robotics methods, as we demonstate in this paper. Used by classrooms, national racing competitions, and enthusiasts, 
platforms like the F1Tenth~\cite{okelly_f1tenth_2020} have established themselves as go-to, entry-level platforms for beginning roboticists and researchers alike. Crucially, these platforms still provide sufficient onboard real estate to house components like the battery and compute necessary for modern algorithms as well as sufficient mobility to exercise complex dynamics and environment interactions, like drifting or off-roading~\cite{williams_information_2017, han_model_2024, han_dynamics_2024, datar_toward_2024}. In short, small-scale wheeled robots have high algorithmic potential at relatively low cost and hardware complexity.

A survey of existing ecosystems for low-cost wheeled platforms indicates that current simulation capabilities are isolated and outdated
(see \Cref{tab:integrations}). As a result, education~\cite{horvath_teaching_2024} and research opportunities~\cite{evans_unifying_2024} limit practitioners to model-based or model-free methods with low-fidelity simulation. With the advent of high-fidelity, open-source simulators like Isaac Lab~\cite{mittal_orbit_2023}, which are geared toward modern robotic learning, these limited ecosystems no longer need to be the status quo.

This work aims to support education and research by making modern robotics methods available to the broader community. To accomplish this, \textit{we contribute a Sim2Real framework for 
open-source wheeled robots that has been integrated with a state-of-the-art, research-grounded  simulation ecosystem, Isaac Lab. }Modern methods we implement and evaluate include: massive parallelization, domain randomization, sensor simulation, and end-to-end learning. We demonstrate these methods using three policy types trained in simulation and deployed on low-cost, open-source platforms:

\begin{enumerate}
    \item The \textbf{Drifting Policy} ($\pdrift$) performs a controlled drift through extensive domain randomization, an approach in modern Sim2Real for tasks with uncertain and unstable dynamics. This is the first work to demonstrate direct Sim2Real transfer for drifting without online fine-tuning.
    \item The \textbf{Elevation Policy} ($\pelev$) traverses three-dimensional features with spatial reasoning. This demonstration highlights end-to-end training for modern tasks that tightly couples perception and control. Wheeled Lab is the first to provide integration of an accessible elevation-based Sim2Real pipeline with massive parallelization.
    \item The \textbf{Visual Policy} ($\pvis$) traverses visual features with semantic understanding. This demonstration highlights simulation, end-to-end training, and deployment with cameras, a sensor modality gaining substantial research attention due to its pre-training potential. Wheeled Lab provides a lower cost alternative for visual Sim2Real compared to existing integrations.
\end{enumerate}

Demonstrated policies also exhibit generalization to robustness and recovery behaviors during testing.

By integrating low-cost, open-source platforms with Isaac Lab, this work addresses a current need in robotics to extend modern principles in Sim2Real to resource-constrained audiences. 
