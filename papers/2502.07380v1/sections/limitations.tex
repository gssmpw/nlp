
While an objective of Wheeled Lab is to support education, this work has not yet run a study to evaluate how comprehensible the pipeline is to robotics students. Similarly to F1Tenth, we aim to construct an open curriculum around Wheeled Lab to help introduce difficult or interdisciplinary topics that may be sources of confusion. This study will help reveal what steps in the Sim2Real process might be most challenging or unfamiliar.

While iterative testing for unique conditions can serve pedagogical purposes, the appeal of building, training, and deploying a platform that ``just works" can be inspirational for beginner roboticists. However, the policies presented here in their current open-source state must be met with patience. Thus, custom low-cost wheeled platforms invite yet another frontier of modern robot learning: adaptation~\cite{bousmalis_using_2018}. For instance, the development of a drifting pipeline which adequately adapts to friction on deployment can help lower the barrier-to-entry even further.

Due to the lack of access to facilities with meaningful elevation features, our evaluation of $\pelev$ used a motion capture system, a comparatively inaccessible piece of equipment for our target audience. We aim to establish a benchmark in the future which still evaluates elevation-based training using onboard compute and sensing while remaining accessible to individuals in a variety of geographical regions.
