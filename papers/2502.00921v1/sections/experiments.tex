
\section{Experiments}\label{sec:experiments}
As many authors~\citep{ho2020denoising,meng2022sdedit,9879163,raya2023spontaneous,georgiev2023journey,sclocchi2024probinglatenthierarchicalstructure,sclocchi2024phasetransitiondiffusionmodels,biroli2024dynamical,li2024criticalwindowsnonasymptotictheory} have already empirically studied critical windows in the context of diffusion, we focus on experiments on critical windows for LLMs. In Section~\ref{sec:structured_output}, we validate our theory on outputs with a hierarchical structure, showing strong agreement with Section~\ref{sec:hierarchy}. In Section~\ref{sec:cot_experiments}, we probe critical windows for~\llamainstruct,~\phiinstruct, and~\qweninstruct~in real-world reasoning benchmarks.

\subsection{Structured output experiments}\label{sec:structured_output}
To verify our theory for $\Tbefore,\Tafter$, we have to compute the total variation between truncated responses from an LLM. This usually would take a large number of samples, so to circumvent this issue, we restrict the diversity of the LLM's generations and force the LLM to generate tokens in a structured format. In particular, we have~\llamainstruct\footnote{Default sampling parameters of temperature of $0.6$ and top-p sampling of $0.9$}~respond to following prompt, which asks it to answer a series of fill-in-the-blank questions in a structured format. We also prefill the model's generations with \texttt{\textbackslash n\textbackslash n 1.} to ensure that the outputs comport to this format. To compute $\Tbefore,\Tafter$, we look at when the generations diverge based on the first occurrence of the identifying information. For example, the $\Tbefore$ of the first critical window is \texttt{1. The }, because the first answer has not appeared in the generation, and the $\Tafter$ of the first critical window is \texttt{1. The P} or \texttt{1. The N}, because that uniquely identifies the answer. Figure~\ref{fig:app:structured_llama} plots the probability of obtaining the same answers as the original generation after truncating different amounts from the generation in the forward-reverse experiments, computed with $10,000$ generations. Our theory predicts that jumps in the probability will occur at $\Tbefore,\Tafter$ which represent when the model has committed to a particular answer to a question in the generation. These predictions are validated with our experiments, as the jumps in probability, representing the model localizing to a more specific set of answers, occur exactly at $\Tbefore,\Tafter$.
\begin{tcolorbox}[colback=blue!5!white,colframe=blue!75!black,title=Structured Output Prompt]
Complete the following by choosing only one option for each blank. The options are provided in parentheses, and your response must match the exact case and meaning of the chosen option. Respond with only the completed sentence, no explanations or additional text.

1. The (Pirate/Ninja) jumped across the ship. \\
2. She adopted a (Dog/Cat) from the shelter. \\
3. The (River/Bridge) sparkled under the sun. \\
4. A (Dragon/Knight) guarded the castle gates. \\
5. He ordered (Pizza/Sushi) for dinner.
\end{tcolorbox}
\begin{figure}[H] 
    \centering
    \includegraphics[width=0.32\linewidth]{figures/ex_2.png}
    \includegraphics[width=0.32\linewidth]{figures/ex_3.png}
    \includegraphics[width=0.32\linewidth]{figures/ex_4.png}
    \caption{Structured output plots for \llamainstruct.}
    \label{fig:app:structured_llama}
\end{figure}

\subsection{Chain of thought experiments}\label{sec:cot_experiments}
 We then identify critical windows for~\llamainstruct,~\phiinstruct, and~\qweninstruct~on $7$ different math and reasoning benchmarks on which performance is known to improve with chain of thought reasoning \citep{lanham2023measuringfaithfulnesschainofthoughtreasoning}: ARC Challenge and Easy \citep{clark2018think}, AQua \citep{ling2017program}, LogiQA \citep{liu2020logiqa}, MMLU \citep{hendrycks2021measuring}, and TruthfulQA \citep{lin2022truthfulqa} multiple-choice benchmarks and the MATH benchmark from \citep{hendrycksmath2021}.\footnote{See Appendix~\ref{app:cot_experiment_details} for more results across models and datasets and a discussion on the effect of temperature on critical windows.} In the forward-reverse experiments, we take the original generation, truncate a fraction of tokens, and check if resampling yields the same answer, using a direct text comparison for the multiple choice benchmarks and the prm800k grader for MATH \citep{lightman2023lets}. We do this for $400$ questions from each dataset and resample at each truncation fraction $100$ times. Critical windows, defined as a $>0.5$ jump in probability of obtaining the same answer in consecutive truncation fractions, appear prominently across all models and benchmarks that we tested (Figures~\ref{fig:app:llama_cw_benchmark},~\ref{fig:app:phi_cw_benchmark}, and~\ref{fig:app:qwen_cw_benchmark}); for MATH, they occur in $42.2\%,16.6\%,30.2\%$ of generations from~\llamainstruct,~\qweninstruct, and~\phiinstruct. 

  \begin{figure}[H] 
    \centering
    \includegraphics[width=12cm]{figures/llama_zz.png}
    \caption{Probability that the answer is the same as a function of the percentage of the generation remaining for different math and reasoning benchmarks for~\llamainstruct. Highlighted are generations with a 50\% increase and no 30\% decrease in the probability over one step. }
    \label{fig:app:llama_cw_benchmark}
\end{figure}
 
 These jumps occur during important steps of reasoning: in Figure~\ref{fig:cw_diff_combined}, the correct formula is first expressed in the critical window. Furthermore, we find that generations with critical windows are significantly \textit{less accurate} than generations without critical windows across all datasets and models. For \llamainstruct, critical windows result in up to \textbf{73\% lower accuracy}, and incorrect responses contain \textbf{11\%-33\% more critical windows} (Table~\ref{tab:cw_table_combined}).
\begin{table}[H]
\caption{Differences between Accuracy (Acc) without versus with critical windows and frequency of critical windows (CW) when the original generation is wrong versus right.}
\bigskip
\label{tab:cw_table_combined}
\centering 
\begin{tabular}{l|cc|cc|cc}
\toprule
\textbf{Dataset} & \multicolumn{2}{c|}{\textbf{\llamainstruct}} & \multicolumn{2}{c|}{\textbf{\phiinstruct}} & \multicolumn{2}{c}{\textbf{\qweninstruct}} \\
& $\Delta$\textbf{Acc} & $\Delta$\textbf{CW} & $\Delta$\textbf{Acc} & $\Delta$\textbf{CW} & $\Delta$\textbf{Acc} & $\Delta$\textbf{CW} \\
\midrule
AQUA-RAT & 0.42 & 0.20 & 0.36 & 0.16 & 0.03 & 0.01 \\
ARC Challenge & 0.53 & 0.22 & 0.50 & 0.24 & 0.38 & 0.11 \\
ARC Easy & 0.73 & 0.26 & 0.28 & 0.13 & 0.40 & 0.07 \\
LogiQA & 0.15 & 0.11 & 0.21 & 0.19 & 0.23 & 0.11 \\
MATH & 0.41 & 0.33 & 0.36 & 0.33 & 0.46 & 0.29 \\
MMLU & 0.45 & 0.24 & 0.38 & 0.21 & 0.26 & 0.11 \\
TruthfulQA & 0.44 & 0.20 & 0.42 & 0.23 & 0.55 & 0.23 \\
\bottomrule
\end{tabular}
\end{table}



\begin{figure}[h]
\centering
\begin{minipage}{0.7\linewidth}
    \centering
    \begin{tcolorbox}[enhanced,colback=blue!5!white,colframe=blue!75!black,title=MATH (\phiinstruct)
        ]
    {\small \textbf{Q}: If the first term of an arithmetic sequence equals the third minus the second, and the fourth term is $8$, what is the first term?\\[0.1em]
    \textbf{A}: 1. We know that the arithmetic sequence has a common difference, which we will denote as "$d$". 2. Let's denote the terms in the sequence as $T_1$, $T_2$, $T_3$, and $T_4$ for the first, second, third, and fourth terms, respectively. 3. We're given the information that $T_1 = T_3 - T_2$. Using the general formula for an arithmetic sequence, we can write this \textbf{as $T_1 = T_2 + 2d - T_2$, which simplifies to $T_1 = 2d$. 4. We're also given that $T_4 = 8$. The general formula for an arithmetic sequence tells us that $T_4 = T_1 + 3d$. So we can write $8$}$ = T_1 + 3d.$ 5. Now we have a system of two equations: 1) $T_1 = 2d$ 2) $8 = T_1 + 3d$ 6. We can substitute equation (1) into equation (2), which gives us $8 = 2d + 3d = 5d$. 7. Solving for $d$, we get $d = 8/5$. 8. Now that we know d, we can find $T_1$. Using the equation from step 1, we get $T_1 = 2d = 2 * (8/5) = 16/5$. Therefore, the first term is $T_1 = 16/5$.}
    \end{tcolorbox}
\end{minipage}
\hfill
\begin{minipage}{0.29\linewidth}
    \centering
    \includegraphics[width=\linewidth]{figures/289.png}
\end{minipage}
\caption{Example of critical window for~\phiinstruct~(bolded).}
\label{fig:cw_diff_combined}
\end{figure}
