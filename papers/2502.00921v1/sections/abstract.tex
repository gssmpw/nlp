Large language models (LLMs) can exhibit undesirable and unexpected behavior in the blink of an eye. In a recent Anthropic demo, Claude switched from coding to Googling pictures of Yellowstone, and these sudden shifts in behavior have also been observed in reasoning patterns and jailbreaks. This phenomenon is not unique to autoregressive models: in diffusion models, key features of the final output are decided in narrow ``critical windows'' of the generation process. In this work we develop a simple, unifying theory to explain this phenomenon. We show that it emerges generically as the generation process localizes to a sub-population of the distribution it models.  

While critical windows have been studied at length in diffusion models, existing theory heavily relies on strong distributional assumptions and the particulars of Gaussian diffusion. In contrast to existing work our theory (1) applies to autoregressive and diffusion models; (2) makes no distributional assumptions; (3) quantitatively improves previous bounds even when specialized to diffusions; and (4) requires basic tools and no stochastic calculus or statistical physics-based machinery. We also identify an intriguing connection to the all-or-nothing phenomenon from statistical inference. Finally, we validate our predictions empirically for LLMs and find that critical windows often coincide with failures in problem solving for various math and reasoning benchmarks.
