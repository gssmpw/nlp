\section{Related Work}
\subsection{Domain Generalization for Fault Diagnosis}
Domain Adaptation (DA) methods have achieved results comparable to supervised learning, but these approaches require data from the target domain, which still presents a significant discrepancy from real-world scenarios. Domain Generalization aims to distill knowledge from source domains and apply it to unseen target domains. To extend diagnostic knowledge to unseen domains, DG methods____ have been preliminarily deployed in the context of bearing fault diagnosis. For instance, Zhang \textit{et al.} ____ proposed a DG method utilizing conditional adversarial training to tackle distribution changes in the target domain, yielding high fault diagnosis rates. Li \textit{et al.} ____ achieved enhanced generalization through adversarial domain-enhanced training, facilitating the learning of general and enhanced features, which leads to improve model's generalization performance. For domain augmentation method, Zhao \textit{et al.} ____ used a semantic regularization-based mixup strategy to generate sufficient data to tackle the issue of imbalanced domain generalization. For metric learning method, Mo \textit{et al.} ____ considered instance-to-prototype distance, additional instance-to-instance and prototype-to-prototype distances and further proposed a distance-aware risk minimization framework through two novel losses. Despite the impressive performance of these methods, most of them predominantly explore spatial domain information while overlooking discriminative and generalizable information in the frequency domain. Our approach innovatively combines the extraction of domain-invariant features with domain augmentation. We leverage information in the Fourier space to reconstruct diverse frequency representations, resulting in a significant improvement on performance.

\subsection{Fourier-based Cross-Domain Research}
Early research ____ validated an important property of the Fourier transform: the phase components of the Fourier spectrum retain the high-level semantic information of the original signal, while the amplitude spectrum components contain low-level statistical information. Recently, in the researches of computer vision, Yang \textit{et al.} ____ proposed integrating the Fourier transform with DA by replacing partial amplitude spectra in the source domain images with those from the target domain, thereby reducing domain differences. Xu \textit{et al.} ____ introduced a Fourier-based data augmentation strategy that enables the model to capture phase information, achieving good generalization effects in unseen domains through linear interpolation of amplitude spectra which is similar to mix-up. Lin \textit{et al.} ____ uncovered that Deep Neural Networks have preference on some frequency components and used Deep Frequency Filtering (DFF) to explicitly extract the components in frequency domain of different transfer difficulties across domains in the latent space during training. The previous generalizable fault diagnosis method ____ only proposes a Fourier transform at the level of data pre-processing. To the best of our knowledge, Fourier-based data augmentation fault diagnosis is the firstly introduced in the cross-domain fault diagnosis. We propose a Fourier-based data augmentation reconstruction module and manifold deep metric learning to explore the potential information to improve the performance under unseen working conditions.


\begin{figure*}[ht]
\centerline{\includegraphics[width=\linewidth]{imgs/Fig3.png}}\caption{The overall framework proposed in this paper consists of two main modules: the The Fourier-based Augmentation Reconstruction Network and the Recognition module. In the FARNet module, there are two sub-networks: an amplitude sub-network and a phase sub-network. The phase sub-network takes the fusion information of $\mathcal{F}^{-1}(\mathcal{A}(X_{out1}),\mathcal{P}(X_{in}))$ as input, guided by the residuals of the amplitude sub-network during the training process. Both components of the Frequency-Spatial Interaction Module (FSIM) serve as fundamental blocks in these networks, facilitating the feature extraction. In the final stage, both the original data and the augmented data are fed into the recognition network for fault diagnosis.}
\label{fig3}
\end{figure*}