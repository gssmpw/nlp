\section{Introduction}

Multi-agent systems often require consensus among agents, yet such consensus can be challenging due to conflicts and disagreements. These conflicts stem from differences in the strategic priorities of the agents, making coordination difficult without communication or external mediation. A key example is the equilibrium selection problem \cite{nonCoop_2_shared}, where agents must choose one outcome from multiple Nash equilibria—stable outcomes where no agent can unilaterally improve their payoff. Without a mechanism to enforce a consensus,  these disagreements can cause suboptimal outcomes and even safety risks in critical applications \cite{infer_1, infer_2}. 

\begin{figure}
    \centering
    \includegraphics[width=1\linewidth]{Figure/Scenario.png}
    \caption{Illustration of a waypoint merging scenario in air traffic management, where multiple aircraft must coordinate their estimated time of arrival (ETAs) at a shared waypoint. Each equilibrium represents a possible sequence of arrival times that ensures safe time separation between aircraft, highlighting the need for consensus to avoid conflicts and inefficiencies.}
    \label{fig:scenarioDescription}
\end{figure}

Reaching a consensus in multi-agent systems is particularly difficult when agents are noncooperative, and this difficulty arises from several factors. First, the consensus process must be \textit{procedurally rational}, meaning noncooperative agents should have an incentive to participate. Second, in practical cases, centralized coordination may be unavailable, necessitating decentralized approaches. Additionally, noncooperative agents often avoid sharing private information and lack access to 1-on-1 communication, further complicating coordination.

Existing consensus methods often rely on centralized coordination \cite{Coop_1_monotone, Coop_2_potential, Coop_3_hetero, Coop_6, Coop_8, Coop_9}, cooperation assumptions \cite{nonCoop_4_linear, nonCoop_5_linear, nonCoop_2_shared, nonCoop_1_coord}, or direct communication \cite{Coop_4_comm_priv, Coop_5_comm_priv, Coop_7_comm_priv}, limiting their applicability for noncooperative scenarios. For example, voting—a common mechanism for achieving agreement—often leads to outcomes that some agents reject \cite{nego_shoham}. Therefore, we need a decentralized approach that achieves agreement over conflicting choices, respects individual preferences, and preserves privacy without requiring direct communication.

We propose an algorithm called the trading auction for consensus (\texttt{TACo}) that enables agents to reach an agreement in noncooperative situations. This algorithm is a structured trading-based auction where agents iteratively select choices based on self-interest. Over multiple rounds, agents attempt to persuade others to adopt their preferred choice by offering trades. When disagreements persist, they refine their negotiation strategies, making increasingly precise adjustments to their offers. The process continues until no agent sees a meaningful benefit in further negotiation, ensuring a stable outcome---a property that is provably satisfied within a finite number of steps.

\texttt{TACo} addresses the core challenges of noncooperative consensus by enabling agents to adjust their preferences based on individual benefit, while ensuring decentralization and privacy preservation. It preserves privacy by not disclosing valuations over choices or trading assets. The algorithm does not require a centralized coordinator and uses broadcast communication—a common method of sharing information in multi-agent systems—which avoids direct communication. This comprehensive design makes \texttt{TACo} applicable to a range of noncooperative multi-agent systems, including autonomous vehicles and coordinating drone operations in cities.

The contributions of this work are threefold: (i) We introduce the trading auction for consensus (\texttt{TACo}) algorithm, which enables consensus in noncooperative settings without requiring a centralized coordinator or direct communication, and while preserving privacy. (ii) We prove that \texttt{TACo} enforces an agreement in finite time, and derive an explicit upper bound on the number of steps required for all agents to reach this mutual agreement. (iii) We validate our theoretical findings through numercial experiments and demonstrate that \texttt{TACo} outperforms existing negotiation methods regarding social optimality and fairness in a practical equilibrium selection scenario.
