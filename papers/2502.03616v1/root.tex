\documentclass[journal,twoside,web]{ieeecolor}
% \documentclass[12pt,draftcls,onecolumn]{IEEEtran}
\usepackage{generic}
\usepackage{graphicx,algorithmicx,algorithm,algpseudocode,verbatim,mathtools,enumerate,bm,hyperref,dsfont,tikz,pgfplots,cleveref,subcaption,nicefrac,booktabs}
\usepackage{cite}
\usepackage{amsmath,amssymb,amsfonts}
\usepackage{graphicx}
\usepackage{hyperref}
\usepackage{balance} % for balancing columns on the final page
\usepackage[acronym]{glossaries}
\usepackage{textcomp}
\hypersetup{hidelinks}
\crefformat{equation}{(#2#1#3)}
\usepackage[fancythm,fancybb]{jphmacros2e} 

\def\BibTeX{{\rm B\kern-.05em{\sc i\kern-.025em b}\kern-.08em
    T\kern-.1667em\lower.7ex\hbox{E}\kern-.125emX}}
\markboth{\hskip25pc IEEE TRANSACTIONS AND JOURNALS TEMPLATE}
{\textit{I}\MakeLowercase{\textit{m et al.}}: Noncooperative Equilibrium Selection via a Trading-based Auction}

\usepackage{xcolor}
\newcommand{\david}[1]{{\textcolor{orange}{[David: #1]}}}
\newcommand{\jaehan}[1]{{\textcolor{blue}{[Jaehan: #1]}}}
\newcommand{\filippos}[1]{{\textcolor{purple}{[Filippos: #1]}}}

\newenvironment{customproof}[1][Proof]{%
  \par\noindent\textbf{#1.}\quad%
}{\hfill$\frQED$\\}

% \makeglossaries
\newacronym{adsb}{ADS-B}{Automatic Dependent Surveillance-Broadcast}
\newacronym{kkt}{KKT}{Karush-Kuhn-Tucker}

\pdfstringdefDisableCommands{%
  \def\\{}% Remove line breaks
  \def\texttt#1{<#1>}% Format texttt content
  \def\(#1\){#1}% Inline math: simply use the text inside
  \def\[#1\]{#1}% Display math: similarly use the text inside
}

\usepackage{bbm}
\usepackage{graphicx}
\usepackage{amsmath,amssymb,amsthm,amsfonts}

\usepackage{paralist}
\usepackage{bm}
\usepackage{xspace}
\usepackage{url}
\usepackage{prettyref}
\usepackage{boxedminipage}
\usepackage{wrapfig}
\usepackage{ifthen}
\usepackage{color}
\usepackage{xspace}

\newcommand{\ii}{{\sc Indicator-Instance}\xspace}
\newcommand{\midd}{{\sf mid}}


\usepackage{amsmath,amsthm,amsfonts,amssymb}
\usepackage{mathtools}
\usepackage{graphicx}


% \usepackage{fullpage}

\usepackage{nicefrac}

\newtheorem{inftheorem}{Informal Theorem}
\newtheorem{claim}{Claim}
\newtheorem*{definition*}{Definition}
\newtheorem{example}{Example}

\DeclareMathOperator*{\argmax}{arg\,max}
\DeclareMathOperator*{\argmin}{arg\,min}
\usepackage{subcaption}

\newtheorem{problem}{Problem}
\usepackage[utf8]{inputenc}
\newcommand{\rank}{\mathsf{rank}}
\newcommand{\tr}{\mathsf{Tr}}
\newcommand{\tv}{\mathsf{TV}}
\newcommand{\opt}{\mathsf{OPT}}
\newcommand{\rr}{\textsc{R}\space}
\newcommand{\alg}{\textsf{Alg}\space}
\newcommand{\sd}{\textsf{sd}_\lambda}
\newcommand{\lblq}{\mathfrak{lq} (X_1)}
\newcommand{\diag}{\textsf{diag}}
\newcommand{\sign}{\textsf{sgn}}
\newcommand{\BC}{\texttt{BC} }
\newcommand{\MM}{\texttt{MM} }
\newcommand{\Nexp}{N_{\mathrm{exp}}}
\newcommand{\Nrep}{N_{\mathrm{replay}}}
\newcommand{\Drep}{D_{\mathrm{replay}}}
\newcommand{\Nsim}{N_{\mathrm{sim}}}
\newcommand{\piBC}{\pi^{\texttt{BC}}}
\newcommand{\piRE}{\pi^{\texttt{RE}}}
\newcommand{\piEMM}{\pi^{\texttt{MM}}}
\newcommand{\mmd}{\texttt{Mimic-MD} }
\newcommand{\RE}{\texttt{RE} }
\newcommand{\dem}{\pi^E}
\newcommand{\Rlint}{\mathcal{R}_{\mathrm{lin,t}}}
\newcommand{\Rlipt}{\mathcal{R}_{\mathrm{lip,t}}}
\newcommand{\Rlin}{\mathcal{R}_{\mathrm{lin}}}
\newcommand{\Rlip}{\mathcal{R}_{\mathrm{lip}}}
\newcommand{\Rmax}{R_{\mathrm{max}}}
\newcommand{\Rall}{\mathcal{R}_{\mathrm{all}}}
\newcommand{\Rdet}{\mathcal{R}_{\mathrm{det}}}
\newcommand{\Fmax}{F_{\mathrm{max}}}
\newcommand{\Nmax}{\mathcal{N}_{\mathrm{max}}}
\newcommand{\piref}{\pi^{\mathrm{ref}}}
\newcommand{\green}{\text{\color{green!75!black} green}\;}
\newcommand{\thetaBC}{\widehat{\theta}^{\textsf{BC}}}
\newcommand{\ent}{\mathcal{E}_{\Theta,n,\delta}}
\newcommand{\eNt}{\mathcal{E}_{\Theta_t,\Nexp,\delta}}
\newcommand{\eNtH}{\mathcal{E}_{\Theta_t,\Nexp,\delta/H}}

\newcommand{\eref}[1]{(\ref{#1})}
\newcommand{\sref}[1]{Sec. \ref{#1}}
\newcommand{\dr}{\widehat{d}_{\mathrm{replay}}}
\newcommand{\figref}[1]{Fig. \ref{#1}}

\usepackage{xcolor}
\definecolor{expert}{HTML}{008000}
\definecolor{error}{HTML}{f96565}
\newcommand{\GKS}[1]{{\textcolor{violet}{\textbf{GKS: #1}}}}
\newcommand{\Q}[1]{{\textcolor{red}{\textbf{Question #1}}}}
\newcommand{\ZSW}[1]{{\textcolor{orange}{\textbf{ZSW: #1}}}}
\newcommand{\JAB}[1]{{\textcolor{teal}{\textbf{JAB: #1}}}}
\newcommand{\jab}[1]{{\textcolor{teal}{\textbf{JAB: #1}}}}
\newcommand{\SAN}[1]{{\textcolor{blue}{\textbf{SC: #1}}}}
\newcommand{\scnote}[1]{\SAN{#1}}
\newcommand{\norm}[1]{\left\lVert #1 \right\rVert}

\usepackage{color-edits}
\addauthor{sw}{blue}

\usepackage{thmtools}
\usepackage{thm-restate}

\usepackage{tikz}
\usetikzlibrary{arrows,calc} 
\newcommand{\tikzAngleOfLine}{\tikz@AngleOfLine}
\def\tikz@AngleOfLine(#1)(#2)#3{%
\pgfmathanglebetweenpoints{%
\pgfpointanchor{#1}{center}}{%
\pgfpointanchor{#2}{center}}
\pgfmathsetmacro{#3}{\pgfmathresult}%
}

\declaretheoremstyle[
    headfont=\normalfont\bfseries, 
    bodyfont = \normalfont\itshape]{mystyle} 
\declaretheorem[name=Theorem,style=mystyle,numberwithin=section]{thm}

% \usepackage{algorithm}
% \usepackage{algorithmic}
\usepackage[linesnumbered,algoruled,boxed,lined,noend]{algorithm2e}

\usepackage{listings}
\usepackage{amsmath}
\usepackage{amsthm}
\usepackage{tikz}
\usepackage{caption}
\usepackage{mdwmath}
\usepackage{multirow}
\usepackage{mdwtab}
\usepackage{eqparbox}
\usepackage{multicol}
\usepackage{amsfonts}
\usepackage{tikz}
\usepackage{multirow,bigstrut,threeparttable}
\usepackage{amsthm}
\usepackage{bbm}
\usepackage{epstopdf}
\usepackage{mdwmath}
\usepackage{mdwtab}
\usepackage{eqparbox}
\usetikzlibrary{topaths,calc}
\usepackage{latexsym}
\usepackage{cite}
\usepackage{amssymb}
\usepackage{bm}
\usepackage{amssymb}
\usepackage{graphicx}
\usepackage{mathrsfs}
\usepackage{epsfig}
\usepackage{psfrag}
\usepackage{setspace}
\usepackage[%dvips,
            CJKbookmarks=true,
            bookmarksnumbered=true,
            bookmarksopen=true,
%						bookmarks=false,
            colorlinks=true,
            citecolor=red,
            linkcolor=blue,
            anchorcolor=red,
            urlcolor=blue
            ]{hyperref}
%\usepackage{algorithm}
\usepackage[linesnumbered,algoruled,boxed,lined]{algorithm2e}
\usepackage{algpseudocode}
\usepackage{stfloats}
\RequirePackage[numbers]{natbib}

\usepackage{comment}
\usepackage{mathtools}
\usepackage{blkarray}
\usepackage{multirow,bigdelim,dcolumn,booktabs}

\usepackage{xparse}
\usepackage{tikz}
\usetikzlibrary{calc}
\usetikzlibrary{decorations.pathreplacing,matrix,positioning}

\usepackage[T1]{fontenc}
\usepackage[utf8]{inputenc}
\usepackage{mathtools}
\usepackage{blkarray, bigstrut}
\usepackage{gauss}

\newenvironment{mygmatrix}{\def\mathstrut{\vphantom{\big(}}\gmatrix}{\endgmatrix}

\newcommand{\tikzmark}[1]{\tikz[overlay,remember picture] \node (#1) {};}

%% Adapted form https://tex.stackexchange.com/questions/206898/braces-for-cases-in-tabular-environment/207704#207704
\newcommand*{\BraceAmplitude}{0.4em}%
\newcommand*{\VerticalOffset}{0.5ex}%  
\newcommand*{\HorizontalOffset}{0.0em}% 
\newcommand*{\blocktextwid}{3.0cm}%
\NewDocumentCommand{\InsertLeftBrace}{%
	O{} % #1 = draw options
	O{\HorizontalOffset,\VerticalOffset} % #2 = optional brace shift options
	O{\blocktextwid} % #3 = optional text width
	m   % #4 = top tikzmark
	m   % #5 = bottom tikzmark
	m   % #6 = node text
}{%
	\begin{tikzpicture}[overlay,remember picture]
	\coordinate (Brace Top)    at ($(#4.north) + (#2)$);
	\coordinate (Brace Bottom) at ($(#5.south) + (#2)$);
	\draw [decoration={brace, amplitude=\BraceAmplitude}, decorate, thick, draw=black, #1]
	(Brace Bottom) -- (Brace Top) 
	node [pos=0.5, anchor=east, align=left, text width=#3, color=black, xshift=\BraceAmplitude] {#6};
	\end{tikzpicture}%
}%
\NewDocumentCommand{\InsertRightBrace}{%
	O{} % #1 = draw options
	O{\HorizontalOffset,\VerticalOffset} % #2 = optional brace shift options
	O{\blocktextwid} % #3 = optional text width
	m   % #4 = top tikzmark
	m   % #5 = bottom tikzmark
	m   % #6 = node text
}{%
	\begin{tikzpicture}[overlay,remember picture]
	\coordinate (Brace Top)    at ($(#4.north) + (#2)$);
	\coordinate (Brace Bottom) at ($(#5.south) + (#2)$);
	\draw [decoration={brace, amplitude=\BraceAmplitude}, decorate, thick, draw=black, #1]
	(Brace Top) -- (Brace Bottom) 
	node [pos=0.5, anchor=west, align=left, text width=#3, color=black, xshift=\BraceAmplitude] {#6};
	\end{tikzpicture}%
}%
\NewDocumentCommand{\InsertTopBrace}{%
	O{} % #1 = draw options
	O{\HorizontalOffset,\VerticalOffset} % #2 = optional brace shift options
	O{\blocktextwid} % #3 = optional text width
	m   % #4 = top tikzmark
	m   % #5 = bottom tikzmark
	m   % #6 = node text
}{%
	\begin{tikzpicture}[overlay,remember picture]
	\coordinate (Brace Top)    at ($(#4.west) + (#2)$);
	\coordinate (Brace Bottom) at ($(#5.east) + (#2)$);
	\draw [decoration={brace, amplitude=\BraceAmplitude}, decorate, thick, draw=black, #1]
	(Brace Top) -- (Brace Bottom) 
	node [pos=0.5, anchor=south, align=left, text width=#3, color=black, xshift=\BraceAmplitude] {#6};
	\end{tikzpicture}%
}%

\usetikzlibrary{patterns}

\definecolor{cof}{RGB}{219,144,71}
\definecolor{pur}{RGB}{186,146,162}
\definecolor{greeo}{RGB}{91,173,69}
\definecolor{greet}{RGB}{52,111,72}

% provide arXiv number if available:
% \arxiv{cs.IT/1502.00326}

% put your definitions there:

%\newtheorem{remark}{Remark} \def\remref#1{Remark~\ref{#1}}
%\newtheorem{conjecture}{Conjecture} \def\remref#1{Remark~\ref{#1}}
%\newtheorem{example}{Example}

%\theorembodyfont{\itshape}
%\newtheorem{theorem}{Theorem}
%\newtheorem{proposition}{Proposition}
%\newtheorem{lemma}{Lemma} \def\lemref#1{Lemma~\ref{#1}}
%\newtheorem{corollary}{Corollary}


%\theorembodyfont{\rmfamily}
%\newtheorem{definition}{Definition}
%\numberwithin{equation}{section}
% \theoremstyle{plain}
% \newtheorem{theorem}{Theorem}
% \newtheorem{Example}{Example}
% \newtheorem{lemma}{Lemma}
% \newtheorem{remark}{Remark}
% \newtheorem{corollary}{Corollary}
% \newtheorem{definition}{Definition}
% \newtheorem{conjecture}{Conjecture}
% \newtheorem{question}{Question}
% \newtheorem*{induction}{Induction Hypothesis}
% \newtheorem*{folklore}{Folklore}
% \newtheorem{assumption}{Assumption}

\def \by {\bar{y}}
\def \bx {\bar{x}}
\def \bh {\bar{h}}
\def \bz {\bar{z}}
\def \cF {\mathcal{F}}
\def \bP {\mathbb{P}}
\def \bE {\mathbb{E}}
\def \bR {\mathbb{R}}
\def \bF {\mathbb{F}}
\def \cG {\mathcal{G}}
\def \cM {\mathcal{M}}
\def \cB {\mathcal{B}}
\def \cN {\mathcal{N}}
\def \var {\mathsf{Var}}
\def\1{\mathbbm{1}}
\def \FF {\mathbb{F}}


\newenvironment{keywords}
{\bgroup\leftskip 20pt\rightskip 20pt \small\noindent{\bfseries
Keywords:} \ignorespaces}%
{\par\egroup\vskip 0.25ex}
\newlength\aftertitskip     \newlength\beforetitskip
\newlength\interauthorskip  \newlength\aftermaketitskip















%%%%%%%%%%%%%%%%%%%%%%%%%%%% by Wu %%%%%%%%%%%%%%%%%%%%%%%%%%%%
\usepackage{xspace}

\newcommand{\Lip}{\mathrm{Lip}}
\newcommand{\stepa}[1]{\overset{\rm (a)}{#1}}
\newcommand{\stepb}[1]{\overset{\rm (b)}{#1}}
\newcommand{\stepc}[1]{\overset{\rm (c)}{#1}}
\newcommand{\stepd}[1]{\overset{\rm (d)}{#1}}
\newcommand{\stepe}[1]{\overset{\rm (e)}{#1}}
\newcommand{\stepf}[1]{\overset{\rm (f)}{#1}}


\newcommand{\floor}[1]{{\left\lfloor {#1} \right \rfloor}}
\newcommand{\ceil}[1]{{\left\lceil {#1} \right \rceil}}

\newcommand{\blambda}{\bar{\lambda}}
\newcommand{\reals}{\mathbb{R}}
\newcommand{\naturals}{\mathbb{N}}
\newcommand{\integers}{\mathbb{Z}}
\newcommand{\Expect}{\mathbb{E}}
\newcommand{\expect}[1]{\mathbb{E}\left[#1\right]}
\newcommand{\Prob}{\mathbb{P}}
\newcommand{\prob}[1]{\mathbb{P}\left[#1\right]}
\newcommand{\pprob}[1]{\mathbb{P}[#1]}
\newcommand{\intd}{{\rm d}}
\newcommand{\TV}{{\sf TV}}
\newcommand{\LC}{{\sf LC}}
\newcommand{\PW}{{\sf PW}}
\newcommand{\htheta}{\hat{\theta}}
\newcommand{\eexp}{{\rm e}}
\newcommand{\expects}[2]{\mathbb{E}_{#2}\left[ #1 \right]}
\newcommand{\diff}{{\rm d}}
\newcommand{\eg}{e.g.\xspace}
\newcommand{\ie}{i.e.\xspace}
\newcommand{\iid}{i.i.d.\xspace}
\newcommand{\fracp}[2]{\frac{\partial #1}{\partial #2}}
\newcommand{\fracpk}[3]{\frac{\partial^{#3} #1}{\partial #2^{#3}}}
\newcommand{\fracd}[2]{\frac{\diff #1}{\diff #2}}
\newcommand{\fracdk}[3]{\frac{\diff^{#3} #1}{\diff #2^{#3}}}
\newcommand{\renyi}{R\'enyi\xspace}
\newcommand{\lpnorm}[1]{\left\|{#1} \right\|_{p}}
\newcommand{\linf}[1]{\left\|{#1} \right\|_{\infty}}
\newcommand{\lnorm}[2]{\left\|{#1} \right\|_{{#2}}}
\newcommand{\Lploc}[1]{L^{#1}_{\rm loc}}
\newcommand{\hellinger}{d_{\rm H}}
\newcommand{\Fnorm}[1]{\lnorm{#1}{\rm F}}
%% parenthesis
\newcommand{\pth}[1]{\left( #1 \right)}
\newcommand{\qth}[1]{\left[ #1 \right]}
\newcommand{\sth}[1]{\left\{ #1 \right\}}
\newcommand{\bpth}[1]{\Bigg( #1 \Bigg)}
\newcommand{\bqth}[1]{\Bigg[ #1 \Bigg]}
\newcommand{\bsth}[1]{\Bigg\{ #1 \Bigg\}}
\newcommand{\xxx}{\textbf{xxx}\xspace}
\newcommand{\toprob}{{\xrightarrow{\Prob}}}
\newcommand{\tolp}[1]{{\xrightarrow{L^{#1}}}}
\newcommand{\toas}{{\xrightarrow{{\rm a.s.}}}}
\newcommand{\toae}{{\xrightarrow{{\rm a.e.}}}}
\newcommand{\todistr}{{\xrightarrow{{\rm D}}}}
\newcommand{\eqdistr}{{\stackrel{\rm D}{=}}}
\newcommand{\iiddistr}{{\stackrel{\text{\iid}}{\sim}}}
%\newcommand{\var}{\mathsf{var}}
\newcommand\indep{\protect\mathpalette{\protect\independenT}{\perp}}
\def\independenT#1#2{\mathrel{\rlap{$#1#2$}\mkern2mu{#1#2}}}
\newcommand{\Bern}{\text{Bern}}
\newcommand{\Poi}{\mathsf{Poi}}
\newcommand{\iprod}[2]{\left \langle #1, #2 \right\rangle}
\newcommand{\Iprod}[2]{\langle #1, #2 \rangle}
\newcommand{\indc}[1]{{\mathbf{1}_{\left\{{#1}\right\}}}}
\newcommand{\Indc}{\mathbf{1}}
\newcommand{\regoff}[1]{\textsf{Reg}_{\mathcal{F}}^{\text{off}} (#1)}
\newcommand{\regon}[1]{\textsf{Reg}_{\mathcal{F}}^{\text{on}} (#1)}

\definecolor{myblue}{rgb}{.8, .8, 1}
\definecolor{mathblue}{rgb}{0.2472, 0.24, 0.6} % mathematica's Color[1, 1--3]
\definecolor{mathred}{rgb}{0.6, 0.24, 0.442893}
\definecolor{mathyellow}{rgb}{0.6, 0.547014, 0.24}


\newcommand{\red}{\color{red}}
\newcommand{\blue}{\color{blue}}
\newcommand{\nb}[1]{{\sf\blue[#1]}}
\newcommand{\nbr}[1]{{\sf\red[#1]}}

\newcommand{\tmu}{{\tilde{\mu}}}
\newcommand{\tf}{{\tilde{f}}}
\newcommand{\tp}{\tilde{p}}
\newcommand{\tilh}{{\tilde{h}}}
\newcommand{\tu}{{\tilde{u}}}
\newcommand{\tx}{{\tilde{x}}}
\newcommand{\ty}{{\tilde{y}}}
\newcommand{\tz}{{\tilde{z}}}
\newcommand{\tA}{{\tilde{A}}}
\newcommand{\tB}{{\tilde{B}}}
\newcommand{\tC}{{\tilde{C}}}
\newcommand{\tD}{{\tilde{D}}}
\newcommand{\tE}{{\tilde{E}}}
\newcommand{\tF}{{\tilde{F}}}
\newcommand{\tG}{{\tilde{G}}}
\newcommand{\tH}{{\tilde{H}}}
\newcommand{\tI}{{\tilde{I}}}
\newcommand{\tJ}{{\tilde{J}}}
\newcommand{\tK}{{\tilde{K}}}
\newcommand{\tL}{{\tilde{L}}}
\newcommand{\tM}{{\tilde{M}}}
\newcommand{\tN}{{\tilde{N}}}
\newcommand{\tO}{{\tilde{O}}}
\newcommand{\tP}{{\tilde{P}}}
\newcommand{\tQ}{{\tilde{Q}}}
\newcommand{\tR}{{\tilde{R}}}
\newcommand{\tS}{{\tilde{S}}}
\newcommand{\tT}{{\tilde{T}}}
\newcommand{\tU}{{\tilde{U}}}
\newcommand{\tV}{{\tilde{V}}}
\newcommand{\tW}{{\tilde{W}}}
\newcommand{\tX}{{\tilde{X}}}
\newcommand{\tY}{{\tilde{Y}}}
\newcommand{\tZ}{{\tilde{Z}}}

\newcommand{\sfa}{{\mathsf{a}}}
\newcommand{\sfb}{{\mathsf{b}}}
\newcommand{\sfc}{{\mathsf{c}}}
\newcommand{\sfd}{{\mathsf{d}}}
\newcommand{\sfe}{{\mathsf{e}}}
\newcommand{\sff}{{\mathsf{f}}}
\newcommand{\sfg}{{\mathsf{g}}}
\newcommand{\sfh}{{\mathsf{h}}}
\newcommand{\sfi}{{\mathsf{i}}}
\newcommand{\sfj}{{\mathsf{j}}}
\newcommand{\sfk}{{\mathsf{k}}}
\newcommand{\sfl}{{\mathsf{l}}}
\newcommand{\sfm}{{\mathsf{m}}}
\newcommand{\sfn}{{\mathsf{n}}}
\newcommand{\sfo}{{\mathsf{o}}}
\newcommand{\sfp}{{\mathsf{p}}}
\newcommand{\sfq}{{\mathsf{q}}}
\newcommand{\sfr}{{\mathsf{r}}}
\newcommand{\sfs}{{\mathsf{s}}}
\newcommand{\sft}{{\mathsf{t}}}
\newcommand{\sfu}{{\mathsf{u}}}
\newcommand{\sfv}{{\mathsf{v}}}
\newcommand{\sfw}{{\mathsf{w}}}
\newcommand{\sfx}{{\mathsf{x}}}
\newcommand{\sfy}{{\mathsf{y}}}
\newcommand{\sfz}{{\mathsf{z}}}
\newcommand{\sfA}{{\mathsf{A}}}
\newcommand{\sfB}{{\mathsf{B}}}
\newcommand{\sfC}{{\mathsf{C}}}
\newcommand{\sfD}{{\mathsf{D}}}
\newcommand{\sfE}{{\mathsf{E}}}
\newcommand{\sfF}{{\mathsf{F}}}
\newcommand{\sfG}{{\mathsf{G}}}
\newcommand{\sfH}{{\mathsf{H}}}
\newcommand{\sfI}{{\mathsf{I}}}
\newcommand{\sfJ}{{\mathsf{J}}}
\newcommand{\sfK}{{\mathsf{K}}}
\newcommand{\sfL}{{\mathsf{L}}}
\newcommand{\sfM}{{\mathsf{M}}}
\newcommand{\sfN}{{\mathsf{N}}}
\newcommand{\sfO}{{\mathsf{O}}}
\newcommand{\sfP}{{\mathsf{P}}}
\newcommand{\sfQ}{{\mathsf{Q}}}
\newcommand{\sfR}{{\mathsf{R}}}
\newcommand{\sfS}{{\mathsf{S}}}
\newcommand{\sfT}{{\mathsf{T}}}
\newcommand{\sfU}{{\mathsf{U}}}
\newcommand{\sfV}{{\mathsf{V}}}
\newcommand{\sfW}{{\mathsf{W}}}
\newcommand{\sfX}{{\mathsf{X}}}
\newcommand{\sfY}{{\mathsf{Y}}}
\newcommand{\sfZ}{{\mathsf{Z}}}


\newcommand{\calA}{{\mathcal{A}}}
\newcommand{\calB}{{\mathcal{B}}}
\newcommand{\calC}{{\mathcal{C}}}
\newcommand{\calD}{{\mathcal{D}}}
\newcommand{\calE}{{\mathcal{E}}}
\newcommand{\calF}{{\mathcal{F}}}
\newcommand{\calG}{{\mathcal{G}}}
\newcommand{\calH}{{\mathcal{H}}}
\newcommand{\calI}{{\mathcal{I}}}
\newcommand{\calJ}{{\mathcal{J}}}
\newcommand{\calK}{{\mathcal{K}}}
\newcommand{\calL}{{\mathcal{L}}}
\newcommand{\calM}{{\mathcal{M}}}
\newcommand{\calN}{{\mathcal{N}}}
\newcommand{\calO}{{\mathcal{O}}}
\newcommand{\calP}{{\mathcal{P}}}
\newcommand{\calQ}{{\mathcal{Q}}}
\newcommand{\calR}{{\mathcal{R}}}
\newcommand{\calS}{{\mathcal{S}}}
\newcommand{\calT}{{\mathcal{T}}}
\newcommand{\calU}{{\mathcal{U}}}
\newcommand{\calV}{{\mathcal{V}}}
\newcommand{\calW}{{\mathcal{W}}}
\newcommand{\calX}{{\mathcal{X}}}
\newcommand{\calY}{{\mathcal{Y}}}
\newcommand{\calZ}{{\mathcal{Z}}}

\newcommand{\bara}{{\bar{a}}}
\newcommand{\barb}{{\bar{b}}}
\newcommand{\barc}{{\bar{c}}}
\newcommand{\bard}{{\bar{d}}}
\newcommand{\bare}{{\bar{e}}}
\newcommand{\barf}{{\bar{f}}}
\newcommand{\barg}{{\bar{g}}}
\newcommand{\barh}{{\bar{h}}}
\newcommand{\bari}{{\bar{i}}}
\newcommand{\barj}{{\bar{j}}}
\newcommand{\bark}{{\bar{k}}}
\newcommand{\barl}{{\bar{l}}}
\newcommand{\barm}{{\bar{m}}}
\newcommand{\barn}{{\bar{n}}}
\newcommand{\baro}{{\bar{o}}}
\newcommand{\barp}{{\bar{p}}}
\newcommand{\barq}{{\bar{q}}}
\newcommand{\barr}{{\bar{r}}}
\newcommand{\bars}{{\bar{s}}}
\newcommand{\bart}{{\bar{t}}}
\newcommand{\baru}{{\bar{u}}}
\newcommand{\barv}{{\bar{v}}}
\newcommand{\barw}{{\bar{w}}}
\newcommand{\barx}{{\bar{x}}}
\newcommand{\bary}{{\bar{y}}}
\newcommand{\barz}{{\bar{z}}}
\newcommand{\barA}{{\bar{A}}}
\newcommand{\barB}{{\bar{B}}}
\newcommand{\barC}{{\bar{C}}}
\newcommand{\barD}{{\bar{D}}}
\newcommand{\barE}{{\bar{E}}}
\newcommand{\barF}{{\bar{F}}}
\newcommand{\barG}{{\bar{G}}}
\newcommand{\barH}{{\bar{H}}}
\newcommand{\barI}{{\bar{I}}}
\newcommand{\barJ}{{\bar{J}}}
\newcommand{\barK}{{\bar{K}}}
\newcommand{\barL}{{\bar{L}}}
\newcommand{\barM}{{\bar{M}}}
\newcommand{\barN}{{\bar{N}}}
\newcommand{\barO}{{\bar{O}}}
\newcommand{\barP}{{\bar{P}}}
\newcommand{\barQ}{{\bar{Q}}}
\newcommand{\barR}{{\bar{R}}}
\newcommand{\barS}{{\bar{S}}}
\newcommand{\barT}{{\bar{T}}}
\newcommand{\barU}{{\bar{U}}}
\newcommand{\barV}{{\bar{V}}}
\newcommand{\barW}{{\bar{W}}}
\newcommand{\barX}{{\bar{X}}}
\newcommand{\barY}{{\bar{Y}}}
\newcommand{\barZ}{{\bar{Z}}}

\newcommand{\hX}{\hat{X}}
\newcommand{\Ent}{\mathsf{Ent}}
\newcommand{\awarm}{{A_{\text{warm}}}}
\newcommand{\thetaLS}{{\widehat{\theta}^{\text{\rm LS}}}}

\newcommand{\jiao}[1]{\langle{#1}\rangle}
\newcommand{\gaht}{\textsc{GoodActionHypTest}\;}
\newcommand{\iaht}{\textsc{InitialActionHypTest}\;}
\newcommand{\true}{\textsf{True}\;}
\newcommand{\false}{\textsf{False}\;}

% \usepackage[capitalize,noabbrev]{cleveref}
% \crefname{lemma}{Lemma}{Lemmas}
% \Crefname{lemma}{Lemma}{Lemmas}
% \crefname{thm}{Theorem}{Theorems}
% \Crefname{thm}{Theorem}{Theorems}
% \Crefname{assumption}{Assumption}{Assumptions}
% \Crefname{inftheorem}{Informal Theorem}{Informal Theorems}
% \crefformat{equation}{(#2#1#3)}

% % if you use cleveref..
% \usepackage[capitalize,noabbrev]{cleveref}
% \crefname{lemma}{Lemma}{Lemmas}
% \crefname{proposition}{Proposition}{Propositions}
% \crefname{remark}{Remark}{Remarks}
% \crefname{corollary}{Corollary}{Corollaries}
% \crefname{definition}{Definition}{Definitions}
% \crefname{conjecture}{Conjecture}{Conjectures}
% \crefname{figure}{Fig.}{Figures}


\begin{document}
\title{Noncooperative Equilibrium Selection via a Trading-based Auction}
\author{Jaehan Im, \IEEEmembership{Graduate Student Member, IEEE}, Filippos Fotiadis, \IEEEmembership{Member, IEEE}, Daniel Delahaye, \IEEEmembership{Member, IEEE}, Ufuk Topcu, \IEEEmembership{Fellow, IEEE}, David Fridovich-Keil, \IEEEmembership{Member, IEEE}
\thanks{J. Im and D. Fridovich-Keil are with the Department of Aerospace Engineering and Engineering Mechanics, The University of Texas at Austin, TX, 78712, USA (emails: jaehan.im@utexas.edu,\, dfk@utexas.edu). F. Fotiadis and U. Topcu are with the Oden Institute for Computational Engineering and Sciences, The University of Texas at Austin, TX, 78712, USA (emails:  ffotiadis@utexas.edu,\, utopcu@utexas.edu). D. Delahaye is with Ecole Nationale de l'Aviation Civile (ENAC), 31055 Toulouse, France (email: daniel@recherche.enac.fr).}
\thanks{This work was supported by a National Science Foundation CAREER award under Grant No. 2336840.}
}

\maketitle

\begin{abstract}
Noncooperative multi-agent systems often face coordination challenges due to conflicting preferences among agents. In particular, agents acting in their own self-interest can settle on different equilibria, leading to suboptimal outcomes or even safety concerns. We propose an algorithm named trading auction for consensus (\texttt{TACo}), a decentralized approach that enables noncooperative agents to reach consensus without communicating directly or disclosing private valuations. \texttt{TACo} facilitates coordination through a structured trading-based auction, where agents iteratively select choices of interest and provably reach an agreement within an \emph{a priori} bounded number of steps. A series of numerical experiments validate that the termination guarantees of \texttt{TACo} hold in practice, and show that \texttt{TACo} achieves a median performance that minimizes the total cost across all agents, while allocating resources significantly more fairly than baseline approaches.
\end{abstract}

\begin{IEEEkeywords}
% Enter key words or phrases in alphabetical order, separated by commas. Using the IEEE Thesaurus can help you find the best standardized keywords to fit your article. Use the thesaurus access request form for free access to the IEEE Thesaurus
Decentralized control, Equilibrium selection, Noncooperative games, Multi-agent systems
\end{IEEEkeywords}

\section{Introduction}
\label{sec:introduction}
The business processes of organizations are experiencing ever-increasing complexity due to the large amount of data, high number of users, and high-tech devices involved \cite{martin2021pmopportunitieschallenges, beerepoot2023biggestbpmproblems}. This complexity may cause business processes to deviate from normal control flow due to unforeseen and disruptive anomalies \cite{adams2023proceddsriftdetection}. These control-flow anomalies manifest as unknown, skipped, and wrongly-ordered activities in the traces of event logs monitored from the execution of business processes \cite{ko2023adsystematicreview}. For the sake of clarity, let us consider an illustrative example of such anomalies. Figure \ref{FP_ANOMALIES} shows a so-called event log footprint, which captures the control flow relations of four activities of a hypothetical event log. In particular, this footprint captures the control-flow relations between activities \texttt{a}, \texttt{b}, \texttt{c} and \texttt{d}. These are the causal ($\rightarrow$) relation, concurrent ($\parallel$) relation, and other ($\#$) relations such as exclusivity or non-local dependency \cite{aalst2022pmhandbook}. In addition, on the right are six traces, of which five exhibit skipped, wrongly-ordered and unknown control-flow anomalies. For example, $\langle$\texttt{a b d}$\rangle$ has a skipped activity, which is \texttt{c}. Because of this skipped activity, the control-flow relation \texttt{b}$\,\#\,$\texttt{d} is violated, since \texttt{d} directly follows \texttt{b} in the anomalous trace.
\begin{figure}[!t]
\centering
\includegraphics[width=0.9\columnwidth]{images/FP_ANOMALIES.png}
\caption{An example event log footprint with six traces, of which five exhibit control-flow anomalies.}
\label{FP_ANOMALIES}
\end{figure}

\subsection{Control-flow anomaly detection}
Control-flow anomaly detection techniques aim to characterize the normal control flow from event logs and verify whether these deviations occur in new event logs \cite{ko2023adsystematicreview}. To develop control-flow anomaly detection techniques, \revision{process mining} has seen widespread adoption owing to process discovery and \revision{conformance checking}. On the one hand, process discovery is a set of algorithms that encode control-flow relations as a set of model elements and constraints according to a given modeling formalism \cite{aalst2022pmhandbook}; hereafter, we refer to the Petri net, a widespread modeling formalism. On the other hand, \revision{conformance checking} is an explainable set of algorithms that allows linking any deviations with the reference Petri net and providing the fitness measure, namely a measure of how much the Petri net fits the new event log \cite{aalst2022pmhandbook}. Many control-flow anomaly detection techniques based on \revision{conformance checking} (hereafter, \revision{conformance checking}-based techniques) use the fitness measure to determine whether an event log is anomalous \cite{bezerra2009pmad, bezerra2013adlogspais, myers2018icsadpm, pecchia2020applicationfailuresanalysispm}. 

The scientific literature also includes many \revision{conformance checking}-independent techniques for control-flow anomaly detection that combine specific types of trace encodings with machine/deep learning \cite{ko2023adsystematicreview, tavares2023pmtraceencoding}. Whereas these techniques are very effective, their explainability is challenging due to both the type of trace encoding employed and the machine/deep learning model used \cite{rawal2022trustworthyaiadvances,li2023explainablead}. Hence, in the following, we focus on the shortcomings of \revision{conformance checking}-based techniques to investigate whether it is possible to support the development of competitive control-flow anomaly detection techniques while maintaining the explainable nature of \revision{conformance checking}.
\begin{figure}[!t]
\centering
\includegraphics[width=\columnwidth]{images/HIGH_LEVEL_VIEW.png}
\caption{A high-level view of the proposed framework for combining \revision{process mining}-based feature extraction with dimensionality reduction for control-flow anomaly detection.}
\label{HIGH_LEVEL_VIEW}
\end{figure}

\subsection{Shortcomings of \revision{conformance checking}-based techniques}
Unfortunately, the detection effectiveness of \revision{conformance checking}-based techniques is affected by noisy data and low-quality Petri nets, which may be due to human errors in the modeling process or representational bias of process discovery algorithms \cite{bezerra2013adlogspais, pecchia2020applicationfailuresanalysispm, aalst2016pm}. Specifically, on the one hand, noisy data may introduce infrequent and deceptive control-flow relations that may result in inconsistent fitness measures, whereas, on the other hand, checking event logs against a low-quality Petri net could lead to an unreliable distribution of fitness measures. Nonetheless, such Petri nets can still be used as references to obtain insightful information for \revision{process mining}-based feature extraction, supporting the development of competitive and explainable \revision{conformance checking}-based techniques for control-flow anomaly detection despite the problems above. For example, a few works outline that token-based \revision{conformance checking} can be used for \revision{process mining}-based feature extraction to build tabular data and develop effective \revision{conformance checking}-based techniques for control-flow anomaly detection \cite{singh2022lapmsh, debenedictis2023dtadiiot}. However, to the best of our knowledge, the scientific literature lacks a structured proposal for \revision{process mining}-based feature extraction using the state-of-the-art \revision{conformance checking} variant, namely alignment-based \revision{conformance checking}.

\subsection{Contributions}
We propose a novel \revision{process mining}-based feature extraction approach with alignment-based \revision{conformance checking}. This variant aligns the deviating control flow with a reference Petri net; the resulting alignment can be inspected to extract additional statistics such as the number of times a given activity caused mismatches \cite{aalst2022pmhandbook}. We integrate this approach into a flexible and explainable framework for developing techniques for control-flow anomaly detection. The framework combines \revision{process mining}-based feature extraction and dimensionality reduction to handle high-dimensional feature sets, achieve detection effectiveness, and support explainability. Notably, in addition to our proposed \revision{process mining}-based feature extraction approach, the framework allows employing other approaches, enabling a fair comparison of multiple \revision{conformance checking}-based and \revision{conformance checking}-independent techniques for control-flow anomaly detection. Figure \ref{HIGH_LEVEL_VIEW} shows a high-level view of the framework. Business processes are monitored, and event logs obtained from the database of information systems. Subsequently, \revision{process mining}-based feature extraction is applied to these event logs and tabular data input to dimensionality reduction to identify control-flow anomalies. We apply several \revision{conformance checking}-based and \revision{conformance checking}-independent framework techniques to publicly available datasets, simulated data of a case study from railways, and real-world data of a case study from healthcare. We show that the framework techniques implementing our approach outperform the baseline \revision{conformance checking}-based techniques while maintaining the explainable nature of \revision{conformance checking}.

In summary, the contributions of this paper are as follows.
\begin{itemize}
    \item{
        A novel \revision{process mining}-based feature extraction approach to support the development of competitive and explainable \revision{conformance checking}-based techniques for control-flow anomaly detection.
    }
    \item{
        A flexible and explainable framework for developing techniques for control-flow anomaly detection using \revision{process mining}-based feature extraction and dimensionality reduction.
    }
    \item{
        Application to synthetic and real-world datasets of several \revision{conformance checking}-based and \revision{conformance checking}-independent framework techniques, evaluating their detection effectiveness and explainability.
    }
\end{itemize}

The rest of the paper is organized as follows.
\begin{itemize}
    \item Section \ref{sec:related_work} reviews the existing techniques for control-flow anomaly detection, categorizing them into \revision{conformance checking}-based and \revision{conformance checking}-independent techniques.
    \item Section \ref{sec:abccfe} provides the preliminaries of \revision{process mining} to establish the notation used throughout the paper, and delves into the details of the proposed \revision{process mining}-based feature extraction approach with alignment-based \revision{conformance checking}.
    \item Section \ref{sec:framework} describes the framework for developing \revision{conformance checking}-based and \revision{conformance checking}-independent techniques for control-flow anomaly detection that combine \revision{process mining}-based feature extraction and dimensionality reduction.
    \item Section \ref{sec:evaluation} presents the experiments conducted with multiple framework and baseline techniques using data from publicly available datasets and case studies.
    \item Section \ref{sec:conclusions} draws the conclusions and presents future work.
\end{itemize}
\putsec{related}{Related Work}

\noindent \textbf{Efficient Radiance Field Rendering.}
%
The introduction of Neural Radiance Fields (NeRF)~\cite{mil:sri20} has
generated significant interest in efficient 3D scene representation and
rendering for radiance fields.
%
Over the past years, there has been a large amount of research aimed at
accelerating NeRFs through algorithmic or software
optimizations~\cite{mul:eva22,fri:yu22,che:fun23,sun:sun22}, and the
development of hardware
accelerators~\cite{lee:cho23,li:li23,son:wen23,mub:kan23,fen:liu24}.
%
The state-of-the-art method, 3D Gaussian splatting~\cite{ker:kop23}, has
further fueled interest in accelerating radiance field
rendering~\cite{rad:ste24,lee:lee24,nie:stu24,lee:rho24,ham:mel24} as it
employs rasterization primitives that can be rendered much faster than NeRFs.
%
However, previous research focused on software graphics rendering on
programmable cores or building dedicated hardware accelerators. In contrast,
\name{} investigates the potential of efficient radiance field rendering while
utilizing fixed-function units in graphics hardware.
%
To our knowledge, this is the first work that assesses the performance
implications of rendering Gaussian-based radiance fields on the hardware
graphics pipeline with software and hardware optimizations.

%%%%%%%%%%%%%%%%%%%%%%%%%%%%%%%%%%%%%%%%%%%%%%%%%%%%%%%%%%%%%%%%%%%%%%%%%%
\myparagraph{Enhancing Graphics Rendering Hardware.}
%
The performance advantage of executing graphics rendering on either
programmable shader cores or fixed-function units varies depending on the
rendering methods and hardware designs.
%
Previous studies have explored the performance implication of graphics hardware
design by developing simulation infrastructures for graphics
workloads~\cite{bar:gon06,gub:aam19,tin:sax23,arn:par13}.
%
Additionally, several studies have aimed to improve the performance of
special-purpose hardware such as ray tracing units in graphics
hardware~\cite{cho:now23,liu:cha21} and proposed hardware accelerators for
graphics applications~\cite{lu:hua17,ram:gri09}.
%
In contrast to these works, which primarily evaluate traditional graphics
workloads, our work focuses on improving the performance of volume rendering
workloads, such as Gaussian splatting, which require blending a huge number of
fragments per pixel.

%%%%%%%%%%%%%%%%%%%%%%%%%%%%%%%%%%%%%%%%%%%%%%%%%%%%%%%%%%%%%%%%%%%%%%%%%%
%
In the context of multi-sample anti-aliasing, prior work proposed reducing the
amount of redundant shading by merging fragments from adjacent triangles in a
mesh at the quad granularity~\cite{fat:bou10}.
%
While both our work and quad-fragment merging (QFM)~\cite{fat:bou10} aim to
reduce operations by merging quads, our proposed technique differs from QFM in
many aspects.
%
Our method aims to blend \emph{overlapping primitives} along the depth
direction and applies to quads from any primitive. In contrast, QFM merges quad
fragments from small (e.g., pixel-sized) triangles that \emph{share} an edge
(i.e., \emph{connected}, \emph{non-overlapping} triangles).
%
As such, QFM is not applicable to the scenes consisting of a number of
unconnected transparent triangles, such as those in 3D Gaussian splatting.
%
In addition, our method computes the \emph{exact} color for each pixel by
offloading blending operations from ROPs to shader units, whereas QFM
\emph{approximates} pixel colors by using the color from one triangle when
multiple triangles are merged into a single quad.


\section{Coordination Under Conflicting Preferences}
% In multi-agent systems, agents often face conflicting preferences when selecting among available options. These conflicts occur from differences in agents' strategic priorities. Resolving such conflicts is essential to achieving system-wide coordination, fairness, and efficiency. A key challenge emerges in noncooperative systems, where agents lack direct communication, cannot share private information, and act solely in their own self-interest. This creates a broad class of problems, which we denote as \emph{conflicting choices}. A notable example of this issue is the \emph{equilibrium selection problem}, which occurs in game-theoretic problems with multiple Nash equilibria. \filippos{Is this paragraph needed? We repeated this info enough in the preceding paragraphs.}

% \filippos{Repeating again} A prominent instance of conflicting choices in multi-agent systems is the \emph{equilibrium selection problem}, which occurs in noncooperative game settings where agents must select one equilibrium from a set of multiple possible Nash equilibria. Before explaining the problem, we first introduce the notation and framework used to model the interactions between agents.

A prominent instance of conflicting choices in multi-agent systems is the \emph{equilibrium selection problem}. Before explaining the problem, we first introduce the notation and framework used to model the interactions between agents.

\subsection{Equilibrium selection problem}

We model the interaction between agents as a noncooperative game with $n\in \intSet_{\ge0} $ agents, where each agent $i \in [n] \equiv \{1,2,\ldots, n\}$ selects an action $x_i$ from an action set $\actionPool_i$. The collective actions of all agents are represented by the joint action profile $\xSet = \{x_1, x_2, \ldots, x_n\}$. Each agent $i$ wishes to minimize a cost function $c_i(x_i, \xSet_{\neg i})$ which depends on its own action $x_i$ and those of other agents, where $\xSet_{\neg i}=\xSet\setminus x_i$. The collection of costs of all agents is denoted as $\cSet=\{c_1,c_2,\ldots,c_n\}$.

Generalized Nash equilibrium is a well-known concept that describes possible outcomes in noncooperative settings. It is a point $\mathbf{x}^\star$ where no agent has an incentive to unilaterally change their action, while adhering to constraints that are dependent on other agents' strategies.

\begin{definition}[Generalized Nash equilibrium] \label{def:GeneralizedNashEquilibrium}
A set of strategies $\xSet^\star = \{x_1^\star, x_2^\star, \ldots, x_n^\star\}$ is a generalized Nash equilibrium if the following holds for each agent $i\in[n]$,
\begin{equation}
    c_i(x_i^\star, \xSet_{\neg i}^\star) \leq c_i(x_i, \xSet_{\neg i}^\star),~\forall x_i \in \actionPool_i(\xSet_{\neg i}^\star).
\end{equation}
\end{definition}

At a generalized Nash equilibrium, each agent’s strategy is optimal given the strategies of all other agents and satisfies the feasibility constraints imposed by the other agents' actions.

There exist multiple Nash equilibria in many games, meaning that several action profiles $\xSet^\star$ may satisfy the Nash equilibrium condition. This multiplicity arises when different combinations of actions lead to  optimal outcomes for each agent. The presence of multiple equilibria introduces a coordination challenge known as the \emph{equilibrium selection problem}, where agents must agree on one equilibrium from the set of possible options to ensure efficient and safe interaction.

\begin{example}[2-agent quadratic game]

Consider two agents, $i$ and $j$, aiming to minimize a shared cost while satisfying a constraint. Each agent's action $x_i \in \mathbb{R}$ is chosen to solve:
\begin{equation} \label{eq:exampleOpt}
    \begin{array}{ll}
        \underset{x_i}{\text{minimize}} & c_i(x_i, x_j) = -(x_i + x_j - 1)^2, \\
        \text{subject to} & x_i^2 + x_j^2 \leq 1.
    \end{array}
\end{equation}

A Nash equilibrium $(x_1^\star, x_2^\star)$ satisfies the \acrfull{kkt} conditions. The Lagrangian is:
\begin{equation}
    \mathcal{L}(x_1, x_2, \mu) = - (x_1 + x_2 - 1)^2 + \mu (x_1^2 + x_2^2 - 1),
\end{equation}
where $\mu \geq 0$ is the Lagrange multiplier. The first-order conditions are:
\begin{subequations} \label{eq:KKTconditions}
\begin{align}
    &\textstyle{\frac{\partial \mathcal{L}}{\partial x_1} = -2(x_1 + x_2 - 1) + 2\mu x_1 = 0,} \\
    &\textstyle{\frac{\partial \mathcal{L}}{\partial x_2} = -2(x_1 + x_2 - 1) + 2\mu x_2 = 0,} \\
    &\mu (x_1^2 + x_2^2 - 1) = 0, \quad \mu \geq 0, \quad x_1^2 + x_2^2 \leq 1.
\end{align}
\end{subequations}

From the first two equations, we obtain:
\begin{equation}
    \textstyle{\mu (x_1 - x_2) = 0, \quad x_1 + x_2 = \frac{2}{2 - \mu}.}
\end{equation}
This yields two cases: i) $\mu = 0$: The constraint is inactive, The feasible solutions satisfying $x_1^2 + x_2^2 \leq 1$ are all points on the line segment $x_1+x_2=1$ where $0 \leq x_1, x_2 \leq 1$.  
ii) $\mu > 0$: The constraint is active, so $x_1^2 + x_2^2 = 1$. Since $\mu (x_1 - x_2) = 0$, we must have $x_1 = x_2$. Solving $2x_1^2 = 1$ gives $x_1 = x_2 = \pm \frac{1}{\sqrt{2}}$.

The critical points are:
\begin{equation}
    \textstyle{\{(x,1-x) \mid 0 \leq x \leq 1\} \cup \left\{\left(\frac{1}{\sqrt{2}}, \frac{1}{\sqrt{2}}\right), \left(-\frac{1}{\sqrt{2}}, -\frac{1}{\sqrt{2}}\right)\right\}.}
\end{equation}
However, $\{(x,1-x) \mid 0 \leq x \leq 1\}$ are local maxima, as an agent can unilaterally decrease their cost by deviating. Thus, $\left(\frac{1}{\sqrt{2}}, \frac{1}{\sqrt{2}}\right)$ and $\left(-\frac{1}{\sqrt{2}}, -\frac{1}{\sqrt{2}}\right)$ are Nash equilibria, demonstrating the existence of multiple equilibria in this game.

\end{example}

\textbf{}


In decentralized systems, achieving consensus on which equilibrium to select is nontrivial, especially in noncooperative settings where agents have different preferences over equilibria. Without coordination, there is a risk that agents may settle on different equilibria, resulting in inefficiencies or even safety risks in the system. Furthermore, agents may not be able to communicate directly or share their preferences, further complicating the equilibrium selection problem.

\section{Trading auction for consensus}

The Trading Auction for Consensus (\texttt{TACo}) algorithm is designed to address the challenge of decentralized \emph{conflicting choice} problems among non-cooperative agents. The primary goal of \texttt{TACo} is to help agents reach a consensus or agree on a single choice, even when they have conflicting preferences. The algorithm accomplishes this without requiring direct negotiation or communication between agents, and it preserves each agent's private information during the process.

\subsection{Key elements of \texttt{TACo}}

\texttt{TACo} consists of several key components: a pay matrix, an offer matrix, a cost matrix, a profit matrix, the private valuations over secondary assets, the amount of trading units, and the decrement factor. These components are central to how agents compute their profit and make decisions during \texttt{TACo}. Let $n \in \mathbb{Z}_{\geq0}$ denote the number of agents, and $m \in \mathbb{Z}_{\geq0}$ the number of choices available in the auction. For any matrix $X$, $X_{ij}$ refers to its $(i,j)$ entry.

\begin{enumerate}[\hspace{0pt}1)]
    \item \textbf{Offer matrix ($\offerList$):} The offer matrix $\offerList \in \mathbb{R}^{n \times m}$ represents the units that each agent $i$ can receive if they select choice $j$. Specifically, $\offerList_{ij}$  denotes the number of units that agent $i$ will receive when they choose choice $j$, determined by the offers made by other agents.
    
    \item \textbf{Pay matrix ($\payList$):} The pay matrix $\payList \in \mathbb{R}^{n \times m}$ captures the cost each agent incurs when selecting a particular choice. $\payList_{ij}$ is the number of units that agent $i$ has to pay to others when they choose choice $j$.
    
    \item \textbf{Cost matrix ($\costList$):} The cost matrix $\costList \in \mathbb{R}^{n \times m}$ reflects the intrinsic cost associated with each agent's choice. $\costList_{ij}$ represents the original cost experienced by agent $i$ when they select choice $j$.
    
    \item \textbf{Private valuation ($\valuation$):} Each agent $i$ assigns a private valuation, $b_i$, to the units of the asset being traded in the \texttt{TACo} algorithm. This valuation, represented as the $i$-th element of the vector $\valuation \in \mathbb{R}_{>0}^n$, reflects how much agent $i$ values each unit of the secondary asset, such as carbon emission credits. The value $b_i$ is unique to each agent and remains private, influencing how the agent perceives the offer and pay quantities when calculating their profit.
    \item \textbf{Amount of trading units ($d$):} The constant $d \in \mathbb{R}_{>0}$ represents the amount of secondary assets traded in each auction step. This public quantity directly impacts the updates of the offer and pay matrices, influencing each agent's profit calculation.
    \item \textbf{Trading unit decrement factor ($\decFactor$):} The decrement factor $\decFactor \in (0,1)$ is a parameter used to reduce the number of trading units when a specified condition is met. When triggered, the amount of trading units $d$ is scaled by the decrement factor:
    \begin{equation}
    d \leftarrow \decFactor d.
    \end{equation}
    This reduction adjusts the scale of assets exchanged in the auction, allowing for a controlled decrease in trading volume over time or under certain conditions, which can influence the convergence behavior of the auction process.
    
    \item \textbf{Profit matrix ($\profitList$):} The profit matrix $\profitList \in \mathbb{R}^{n \times m}$ represents the profit each agent gains based on their choice. Specifically, $\profitList_{ij}$ denotes the profit experienced by agent $i$ when selecting choice $j$. The profit is calculated as follows:
    \begin{equation}
        \profitList = \text{diag}(\valuation) \cdot (\offerList - \payList) - \costList.
    \end{equation}
    This equation accounts for each agent's private valuation $\textbf{b}$, the offers received $\offerList$, the payments made $\payList$, and the intrinsic costs $\costList$, resulting in the total profit for each possible choice.
\end{enumerate}

\subsection{Auction mechanism} \label{sec:AuctionMechanism}

\texttt{TACo} operates through sequential steps, where agents take turns making decisions. At each step, an agent selects the choice they value most and attempts to persuade others by offering payments. These decisions are based on the matrices $\offerList$, $\payList$, and $\costList$. Before explaining the \texttt{TACo} mechanism, we introduce the following definitions.

\begin{definition}[Agent-profit matrix tuple $(i, \profitList)$]  
The agent-profit matrix tuple $(i, \profitList)$ represents the association of agent index $i \in \mathbb{N}$ with a specific profit matrix $\profitList$ applied in the current step.
\end{definition}

A cycle refers to a sequence of auction processes that returns to a previously encountered profit matrix–playing agent pair. Once the process revisits an existing profit matrix-playing agent pair, the sequence repeats, forming a loop. An interesting property of a cycle is that it includes a state where all agents have reached an agreement, particularly in the trivial case
where they continuously select the same choice without termination. 

\begin{definition}[Cycle, $\cycle$]\label{def:cycle}  
A cycle, $\cycle$, is a finite sequence of agent-profit matrix tuples,  
\[
\cycle=\left((i^{k_1},\profitList^{k_1}), (i^{k_2},\profitList^{k_2}), \ldots, (i^{k_\ell},\profitList^{k_\ell})\right),
\]
in which a tuple $(i^{k_{\ell+1}}, \profitList^{k_{\ell+1}})$ revisits any previously encountered tuple, i.e., 
\[
(i^{k_{\ell+1}}, \profitList^{k_{\ell+1}}) = (i^{k_1}, \profitList^{k_1}).
\]
Here, $\ell$ is the length of the cycle.  
\end{definition}

\begin{definition}[Step, $k$]  
A step in the \texttt{TACo} process refers to a single update to the profit matrix $\profitList$, which occurs based on the choice made by one agent during that auction step.
\end{definition}

\begin{algorithm} [hbt!]
\caption{Trading Auction for Consensus (\texttt{TACo})}\label{alg:TACo}
\begin{algorithmic}[1]
\Require $n$: number of agents, $m$: number of choices, $\epsilon$: termination tolerance
\Require $\costList \in \mathbb{R}^{n \times m}$, $d_0 \in \mathbb{R}_{>0}$, $\decFactor \in (0,1)$

\State Initialize offer matrix $\offerList \in \mathbb{N}^{n \times m}$ to $\mathbf{0}_{n \times m}$
\State Initialize pay matrix $\payList \in \mathbb{N}^{n \times m}$ to $\mathbf{0}_{n \times m}$
\State \textit{selections} $\gets$ empty list $\in \mathbb{N}^n$
\State \textit{isConverged} $\gets$ False
\State $d \gets d_0$ \Comment{Initialize trading unit}
\State \textit{recordedStates} $\gets$ empty set \Comment{For cycle detection}

\While{not \textit{isConverged}}
    \State Choose a playing agent $i$ in a sequential manner.
    \For{each choice $j = 1$ to $m$}
        \State Compute profit for agent $i$ and choice $j$:
        \State $\profitList_{ij} \gets b_i (\offerList_{ij} - \payList_{ij}) - \costList_{ij}$
    \EndFor
    \State $j^\star \gets \underset{j \in [m]}{\arg\max}(\profitList_{ij})$ \Comment{Agent $i$ selects the choice $j^\star$ that maximizes its profit}
    \State Update the matrices based on the choice $j^\star$:
    \State \quad $\payList_{ij^\star} \gets \payList_{ij^\star} + nd$
    \State \quad $\offerList_{ij^\star} \gets \offerList_{ij^\star} + d, \forall i\in [n]$
    \State $\text{\textit{selections}}_i \gets j^\star$ \Comment{Record agent $i$'s choice}
    \If{current $((\offerList-\payList), i)$ exists in \textit{recordedStates}} \Comment{Detect cycle}
        \State Reduce trading unit: $d \gets \decFactor \cdot d$
        \State Clear \textit{recordedStates}
        \For{each agent $i\in[n]$}
            \State $\max_j \profitList_{ij} - \min_j \profitList_{ij} < \epsilon, \, \forall j \in \text{cycle}$ \Comment{Check termination condition}
        \EndFor
        \If{condition is met for all agents}
            \State \textit{isConverged} $\gets$ True
        \EndIf
    \EndIf
    \State Add current $((\offerList-\payList), i)$ to \textit{recordedStates}
\EndWhile

\State \Return $\text{mode}(\textit{selections})$ \Comment{Return the most frequent value in \textit{selections}}

\end{algorithmic}
\end{algorithm}

The following rules define the \texttt{TACo} process:

\begin{enumerate}[\hspace{1pt}1)]
    \item \textbf{Sequential play:} Agents take turns sequentially in a predefined order. For example, if there are three agents, the order of play would be agent 1 $\rightarrow$ agent 2 $\rightarrow$ agent 3, and then the sequence repeats.
    
    \item \textbf{Profit calculation:} During agent $i$'s turn, they select the choice $j$ that maximizes their profit, given by:
    \begin{equation} \label{eq:profUpdate}
    \profitList_{ij} = b_i (\offerList_{ij} - \payList_{ij}) - \costList_{ij},
    \end{equation}
    where the $i$-th row of each matrix corresponds to the information relevant to agent $i$. 
    
    \item \textbf{Matrix update:} When an agent $ i $ selects a particular option $ j $, the matrices are updated as follows:

\begin{itemize} \setlength{\itemindent}{-0.8em}
    \item The algorithm updates the payment matrix $\payList$ by incrementing the entry $\payList_{ij}$ by $ nd $, where $ n $ is the total number of agents and $ d $ is the trading unit. This reflects that agent $ i $ pays $ n $ units of the secondary asset for selecting option $ j $:
    \begin{equation} \label{eq:payUpdate}
        \payList_{ij} \gets \payList_{ij} + nd.    
    \end{equation}
    \item The algorithm updates the offer matrix $\offerList$ by incrementing the $ j $-th column by $ d $. This ensures that all agents receive an additional unit of the secondary asset in their offers for option $ j $:
    \begin{equation} \label{eq:offerUpdate}
        \offerList_{ij} \gets \offerList_{ij} + d, \quad \forall i \in [n].    
    \end{equation}
\end{itemize}

    \item \label{sec:reductionRule} \textbf{Cycle Detection and Trading Unit Reduction:} The trading unit $d$ is reduced by the decrement factor $\decFactor$ when a cycle is detected. A cycle is identified by checking whether the same pair of profit matrix $\profitList$ and active agent at a specific step has occurred in any previous steps. Tracking $\profitList$ is equivalent to tracking $(\offerList - \payList)$, since $b_i$ and $C_{ij}$ are constants based on \cref{eq:profUpdate}. Agents can independently detect cycles by monitoring changes in the publicly observable matrices $\offerList$ and $\payList$. When a cycle is detected, the trading unit is updated as $d_{r+1} \gets \decFactor d_r$, where $d_r$ denotes the trading unit after $r$ cycle detections, with the initial trading unit defined as $d_0$. Additionally, the current record of profit matrices is cleared to restart the cycle detection process.
    
    \quad This reduction of the trading unit is a common practice in real bargaining scenarios, where participants tend to adjust their bids or pricing in smaller increments over time to facilitate reaching a consensus \cite{kahneman2013prospect}. The relationship governing the trading unit after $r$ cycle detections can be expressed as:
    \begin{equation}\label{eq:dr}
        d_r = d_0 \cdot \decFactor^r.
    \end{equation}
    
    \item \textbf{$\epsilon$-termination:} The auction continues until consensus is reached. The process terminates when the difference in profit between all agents' choice options is within a fixed tolerance $\epsilon$, indicating indifference among the choices in the cycle.
\end{enumerate}

\begin{table}[hbt!]
      \caption{\texttt{TACo} running example}
      \label{tab:locations}
      \centering
      \begin{tabular}{cccccc}\toprule
        \textit{Step} & \textit{Agent, $i$} & $\offerList$ & $\payList$ & $\profitList$ & \textit{Selections} \\ \midrule
        
        1 & 1 & $\begin{bsmallmatrix} 0 & 0 \\ 0 & 0 \end{bsmallmatrix}$ & $\begin{bsmallmatrix} 0 & 0 \\ 0 & 0 \end{bsmallmatrix}$ & $\begin{bsmallmatrix} -10 & -4 \\ -7 & -9 \end{bsmallmatrix}$ & $\begin{bsmallmatrix} \underline{2} & \emptyset \end{bsmallmatrix}$ \\
        
        2 & 2 & $\begin{bsmallmatrix} 0 & 1 \\ 0 & 1 \end{bsmallmatrix}$ & $\begin{bsmallmatrix} 0 & 2 \\ 0 & 0 \end{bsmallmatrix}$ & $\begin{bsmallmatrix} -10 & -4.8 \\ -7 & -7.8 \end{bsmallmatrix}$ & $\begin{bsmallmatrix} 2 & \underline{1} \end{bsmallmatrix}$ \\

        3 & \bf{1} & $\begin{bsmallmatrix} 1 & 1 \\ 1 & 1 \end{bsmallmatrix}$ & $\begin{bsmallmatrix} 0 & 2 \\ 2 & 0 \end{bsmallmatrix}$ & $\begin{bsmallmatrix} \bf{-9.2} & \bf{-4.8} \\ \bf{-8.2} & \bf{-7.8} \end{bsmallmatrix}$ & $\begin{bsmallmatrix} \underline{2} & 1 \end{bsmallmatrix}$ \\

        4 & 2 & $\begin{bsmallmatrix} 1 & 2 \\ 1 & 2 \end{bsmallmatrix}$ & $\begin{bsmallmatrix} 0 & 4 \\ 2 & 0 \end{bsmallmatrix}$ & $\begin{bsmallmatrix} -9.2 & -5.6 \\ -8.2 & -6.6 \end{bsmallmatrix}$ & $\begin{bsmallmatrix} 2 & \underline{2} \end{bsmallmatrix}$ \\

        5 & \bf{1} & $\begin{bsmallmatrix} 1 & 3 \\ 1 & 3 \end{bsmallmatrix}$ & $\begin{bsmallmatrix} 0 & 4 \\ 2 & 2 \end{bsmallmatrix}$ & $\begin{bsmallmatrix} \bf{-9.2} & \bf{-4.8} \\ \bf{-8.2} & \bf{-7.8} \end{bsmallmatrix}$ & $\begin{bsmallmatrix} \underline{2} & 2 \end{bsmallmatrix}$ \\
        
        \bottomrule
      \end{tabular}
\end{table}
\begin{example}[\texttt{TACo} running example] \label{example:2}
    Consider a case with two agents ($n=2$) and two choices ($m=2$). The initial cost matrix is given by $\,\costList=\begin{bsmallmatrix} 10 & 4 \\ 7 & 9 \end{bsmallmatrix}$, and the agents' private valuations for the traded asset are $\valuation=\begin{bsmallmatrix} 0.8 & 1.2 \end{bsmallmatrix}$.
    At each step, agents update their selections based on their calculated profit using the \texttt{TACo} rules. \Cref{tab:locations} shows how the auction progresses over time.
    In this example, Agent 1 initially selects option 2 because it provides the highest profit. In step 2, Agent 2 selects option 1, which maximizes its own profit. At each step, the agent currently playing updates its selection, and this updated selection is \underline{underlined} in the table. The \textbf{bold} values in the profit matrix $\profitList$ highlight that the same profit-value and agent pair reappear at steps 3 and 5, signaling the occurrence of a cycle. This cycle is associated with selection 2, where all agents converge to the same option. The algorithm detects that the $\epsilon$-termination criterion is satisfied because the profit differences for all agents are zero, which is smaller than any positive $\epsilon$. Consequently, the algorithm terminates at step 5. 
\end{example}

\subsection{Properties of \texttt{TACo}}
There are several properties of the \texttt{TACo} procedure (\Cref{alg:TACo}).
\subsubsection{Procedural rationality}
\texttt{TACo} incentivizes agents to act in their self-interest by enabling them to trade and negotiate based on their individual valuations of the available options. As demonstrated in \Cref{example:2}, each agent selects the option that maximizes its profit, naturally driving the auction process toward a consensus.

\subsubsection{Privacy preservation}
One of \texttt{TACo}’s core strengths is its ability to preserve the privacy of each agent’s valuations. The trading mechanism functions without requiring agents to disclose their private valuations for the available options ($\costList$) or trading assets ($\valuation$).

\subsubsection{Independence from direct communication}
\texttt{TACo} operates under a minimal communication framework, where agents only need to broadcast their choices. This eliminates the need for direct, 1-on-1 communication, making the algorithm straightforward to implement in practical systems.

\subsubsection{Termination guarantee}
\texttt{TACo} is guaranteed to terminate with an agreement within a bounded number of rounds. The proof of termination, as well as an explicit upper bound on the number of rounds needed to terminate, is provided in the following section.
\section{Termination Proof of \texttt{TACo}} \label{sec:proof}

The proof of \texttt{TACo}'s termination (with an agreement) involves three steps. First, we show that \texttt{TACo} always enters a cycle, where the auction steps repeat in a loop as defined in \Cref{def:cycle}. Next, we demonstrate that profit differences between choices within a cycle are bounded, with the bound proportional to the trading unit per step $d$. This restricts each agent’s available choices to options with profit differences no greater than the bound. Finally, we prove that \texttt{TACo} satisfies the epsilon-termination condition and hence converges. 

The intuition behind the convergence of \texttt{TACo} lies in that, as soon as a cycle is detected, the trading unit $d$ is reduced by a fixed decrement factor $\decFactor$. However, as $d$ decreases, the profit differences within the cycle shrink and must thus eventually fall below the threshold $\epsilon$ for all agents. At this point, the agents reach a consensus by definition, and \texttt{TACo} terminates.

\begin{figure}[hbt!]
    \centering
\hspace*{-0.01\textwidth}
\resizebox{0.5\textwidth}{!}{
\begin{tikzpicture}[thick, scale=1.1][hbt!]
    % Axes
    \draw[->] (0,0) -- (7.5,0) node[below] {Step$(k)$};
    \draw[->] (0,0) -- (0,3.2) node[left] {$S_i^k$};

    % Horizontal dashed lines
    \draw[thin, dotted] (0,2.5) -- (7.3,2.5) node[anchor=east] at (0,2.5) {$S_i^{\max}$};
    \draw[thin, dotted] (0,0.5) -- (7.3,0.5) node[anchor=east] at (0,0.5) {$S_i^{\min}$};
    \draw[thin, dotted] (3,0.5+0.66) -- (4,0.5+0.66);
    \draw[thin, dotted] (4,0.5+0.66*2) -- (5,0.5+0.66*2);

    % Labels
    \node[below left] at (0,0) {$0$};
    \node[below] at (1,0) {$k$};
    \node[below] at (2,0) {$k+1$};
    \node[below] at (3,0) {$k+2$};
    \node[below] at (4,0) {$k+3$};
    \node[below] at (5,0) {$k+4$};
    \node[below] at (6,0) {$k+5$};
    \node[below] at (6.7,-0.07) {$\cdots$};
    \node[below] at (0.5,-0.07) {$\cdots$};

    % Dotted vertical lines for steps
    \draw[thin, dotted] (1,0) -- (1,2.5);
    \draw[thin, dotted] (2,0) -- (2,0.5);
    \draw[thin, dotted] (3,0) -- (3,0.5+0.66);
    \draw[thin, dotted] (4,0) -- (4,0.5+0.66*2);
    \draw[thin, dotted] (5,0) -- (5,2.5);
    \draw[thin, dotted] (6,0) -- (6,0.5);

    % Connecting dashed lines
    \draw[very thick, dashed] (1,2.5) -- (2,0.5);
    \draw[very thick, dashed] (2,0.5) -- (3,0.5+0.66);
    \draw[very thick, dashed] (3,0.5+0.66) -- (4,0.5+0.66*2);
    \draw[very thick, dashed] (4,0.5+0.66*2) -- (5,2.5);
    \draw[very thick, dashed] (5,2.5) -- (6,0.5);
    \node[right] at (6.5,1.5) {$\cdots$};
    \node[left] at (0.82,1.5) {$\cdots$};
    
    % Bracket
    \draw[decorate, decoration={brace, mirror, amplitude=8pt}, gray] 
        (0,2.48) -- (0,0.52) node[midway, left=6pt, gray] {$\textstyle d(n-1)b_i$};
    \draw[decorate, decoration={brace, mirror, amplitude=8pt}, gray]
        (3,0.52) -- (3,0.5+0.64) node[midway, right=6pt, gray] {$\textstyle db_i$};
    \draw[decorate, decoration={brace, mirror, amplitude=8pt}, gray]
        (4,0.5+0.66) -- (4,0.5+0.66*2-0.02) node[midway, right=6pt, gray] {$\textstyle db_i$};
    \draw[decorate, decoration={brace, mirror, amplitude=8pt}, gray]
        (5,0.5+0.66*2+0.02) -- (5,2.48) node[midway, right=6pt, gray] {$\textstyle db_i$};
\end{tikzpicture}
}
\caption{(Illustration of \Cref{theorem:lemma1}, $n=4$) This figure states that the row sum $S_i^k$ of the $i$-th row of the profit matrix $\profitList$ remains constant every $n$ steps. The fluctuations in the row sum are bounded by the difference $S_i^{\text{max}} - S_i^{\text{min}} = d(n-1)b_i$, where $S_i^{\text{max}}$ and $S_i^{\text{min}}$ denote the maximum and minimum row sums during the cycle. The diagram highlights the periodic nature of the row sum across steps and the incremental fluctuations $db_i$ within each step.}
\label{fig:fluctuations}
\end{figure}

\subsection{Existence of cycles in \texttt{TACo}}
We first prove that \texttt{TACo} always enters a cycle. A cycle occurs when the auction steps repeat in a loop, as defined in \Cref{def:cycle}. This result is established by analyzing bounded row sums and discrete profit increments. The following lemma shows that the row sum of the profit matrix fluctuates within a bounded range as shown in \Cref{fig:fluctuations}.

\begin{lemma}[Row sum of the profit matrix] \label{theorem:lemma1}
The row sum $S_i^k = \sum_{j=1}^m \profitList_{i,j}^k$, where $m$ is the number of choices, remains constant every $n$ steps, with fluctuations equal to:
\begin{equation}
    \smax - \smin = d(n-1)b_i,
\end{equation}
where $\smax$ and $\smin$ are the maximum and minimum row sums that can occur during the \texttt{TACo} process.
\end{lemma}

The bounded row sums imply that the actual profit values $\profitList_{ij}$ are lower bounded, as established in the next lemma.

\begin{lemma}[Lower bound of profit values] \label{theorem:lemma2}
The profit value $\profitList_{ij}$ for any agent $i$ and choice $j$ is lower bounded as
\begin{equation}
    \profitList_{ij} \geq \min\left(\min(\profitList_{i,:}^{0}), \textstyle{\frac{\smin}{m}} - d(n-1)b_i\right),
\end{equation}
where $m$ is the number of choices and  $\min(\profitList_{i,:}^{0})$ is the minimum element for row $i$ in the initial profit matrix $\profitList$.
\end{lemma}

Since the row sums of the profit matrix $\profitList$ remain constant over every $n$ steps (Lemma 1) and since $\profitList$ is lower-bounded (Lemma 2), it directly follows that  $\profitList$ is also upper-bounded. We state this consequence in the following corollary.


\begin{corollary}[Upper bound of profit values] \label{theorem:corollary1}
 The profit values in the matrix $\profitList$ are upper bounded.
\end{corollary}


Exploiting the proved upper and lower bounds on $\profitList$ derived so far, we can now formally prove that \texttt{TACo} always enters a cycle. This result is important because, by definition, a state of consensus where all agents have similar preferences -- up to a level of tolerance $\epsilon$ -- is also a cycle. Hence, the convergence of \texttt{TACo} will subsequently boil down to showing that the cycle it eventually falls into corresponds to a consensus.

\begin{theorem}[The auction always falls into a cycle] \label{theorem:theorem1}
Let $\mathcal{S}$ be the set of all possible agent-profit matrix tuples $(i, \profitList)$, where $i \in [n]$ and $\profitList$ is a profit matrix satisfying the constraints defined in 
\Cref{theorem:lemma1}, \Cref{theorem:lemma2} and \Cref{theorem:corollary1}. The \texttt{TACo} algorithm must eventually revisit a previously encountered tuple $(i, \profitList) \in \mathcal{S}$, resulting in the formation of a cycle.
\end{theorem}

\subsection{Bounded profit differences between choices in a cycle}

Having established that \texttt{TACo} always enters a cycle, we now focus on bounding the profit differences among the choices within that cycle. More importantly, we will show that the underlying bound in profit differences is \textit{proportional} to $d$; a property that will allow us to prove  \texttt{TACo} eventually converges to a state of agreement between agents, particularly since $d$ is decremented according to \eqref{eq:dr} once a cycle is detected.  To establish this property, we start by introducing the \textit{choice count matrix} (\(\Delta^\cycle\)), which tracks how many times each agent selects each choice during a cycle.

\begin{definition}[Choice count matrix, $\Delta^\cycle$] 
The choice count matrix, denoted as $\Delta^\cycle \in \mathbb{Z}_{\geq 0}^{n \times m}$, represents the number of times each choice is selected by agents during a cycle. Specifically, $\Delta^\cycle_{i,j}$ denotes the number of times agent $i$ selects choice $j$ within the cycle $\cycle$. 
\end{definition}

It follows that the matrix $\Delta^\cycle$ satisfies the following:
\begin{equation} \label{eq:choiceCountToPay}
    \payList^{t+\ell} - \payList^t = nd \cdot \Delta^\cycle,
\end{equation}
\begin{equation} \label{eq:choiceCountToOffer}
    \offerList^{t+\ell} - \offerList^t = d \cdot H_n \Delta^\cycle,
\end{equation}
where $\payList^{t+\ell} - \payList^t$ and $\offerList^{t+\ell} - \offerList^t$ represent the net changes in the pay matrix and offer matrix after each cycle of length $\ell$, respectively, and $H_n = \textbf{1}_n \textbf{1}_n^\top$ is an $n \times n$ matrix with all elements equal to 1, capturing the aggregation of offers received across all agents. Using the definition of the choice count matrix and its relationship shown in \cref{eq:choiceCountToPay} and \cref{eq:choiceCountToOffer}, we establish the structure of $\Delta^\cycle$ which shows that, within a cycle, all agents select each choice the same number of times.

\begin{lemma}[Uniform choice count matrix in a cycle] \label{theorem:lemma3}
The choice count matrix $\Delta^\cycle \in \mathbb{Z}_{\geq 0}^{n \times m}$ for a cycle  has the form
\begin{equation}
\Delta^\cycle = \begin{bmatrix}
    c_1 & c_2 & \ldots & c_m \\ 
    \vdots & \vdots & \ddots & \vdots \\ 
    c_1 & c_2 & \ldots & c_m 
\end{bmatrix},
\end{equation}
where each $ c_j \in \mathbb{Z}_{\geq 0} $ represents the number of times that choice $ j $ is selected by each agent within the cycle.
\end{lemma}

This uniform structure naturally leads to the concept of \emph{active choices}: the choices selected by agents during a cycle.

\begin{definition}[Active choices in a cycle, $A^\cycle$] The active choices in a cycle, denoted $A^\cycle$, are the row indices of the choice count matrix $\Delta^\cycle$ with non-zero values.  In other words, $A^\cycle$ is a set of indices representing the choices selected at least once by some agent within a cycle.
\end{definition}

We now analyze the profit differences within a cycle by focusing on active choices only. Leveraging \Cref{theorem:lemma1}, \Cref{theorem:lemma2}, and \Cref{theorem:lemma3}, we prove a key result that bounds these profit differences \textit{proportionally} to the amount of trading units $d$.

\begin{theorem}[Bound on profit differences within a cycle] \label{theorem:theorem2}
The profit difference among the active choices $A^\cycle$ in a cycle $\cycle$ for agent $i$ is bounded as 
\begin{equation} \label{eq:profDiff_Theorem2}
U_\cycle^i - L_\cycle^i \leq (m+1)d(n-1)b_{\max},\quad \forall i \in [n],
\end{equation}
where $m$ is the total number of choices, $d$ is the trading unit for the current cycle, $b_{\max} = \max_{i \in [n]} (b_i)$ is the highest private valuation among the agents, and $U_\cycle^i$ and $L_\cycle^i$ are the maximum and minimum profit values for the active choices of agent $i$ within cycle $\cycle$, respectively.
\end{theorem}

\subsection{Finite termination property of \texttt{TACo}}
With the profit differences bounded proportionally to $d$ (\Cref{theorem:theorem2}), we now demonstrate that \texttt{TACo} satisfies the epsilon-termination condition, leading to an approximate agreement in a finite number of steps. Recall that each time a cycle is detected -- which provably happens according to Theorem \ref{theorem:theorem1} -- the algorithm reduces the trading unit $d$ by a fixed decrement factor $\decFactor$. This reduction ensures that the profit differences between choices eventually fall below the threshold $\epsilon$, leading to termination.

\begin{theorem}[Termination of \texttt{TACo}] \label{theorem:theorem3}
The \texttt{TACo} algorithm terminates in a finite number of steps.
\end{theorem}

We also provide a practical bound on the worst-case number of steps required till termination. This bound accounts for the number of cycles needed to meet the epsilon-termination condition as well as the steps within each cycle, hence providing practical insights into the efficiency of \texttt{TACo}.

\begin{theorem}[Termination bound for \texttt{TACo}] \label{theorem:theorem4}
    The \texttt{TACo} algorithm terminates at most within the following number of steps:
    \begin{equation} \label{eq:terminationBound}
        \textstyle{
        \left\lceil \log_{\decFactor} \left(\frac{\epsilon}{(m+1)d_0(n-1)b_{\max}} \right) \right\rceil \cdot n \cdot \big((m+1)(n-1)\big)^{nm}},
    \end{equation}
    with decrement factor $\decFactor$ and initial trading unit $d_0$.
\end{theorem}

\Cref{theorem:theorem4} implies that if $\epsilon$ is increased to $\frac{\epsilon}{\decFactor}$, then the worst-case termination bound decreases by $n \cdot \big((m+1)(n-1)\big)^{nm}$ steps. 
\section{Preliminary numerical results} 
\label{sec:main/numerical}

\begin{figure}[tbp]
    \centering
    \includegraphics{figures/main-ss1-time.pdf} \hfill
    \includegraphics{figures/main-ss1-hesseval.pdf}
    \caption{
        Comparison of success rates as functions of elapsed time and Hessian evaluations for CUTEst benchmark problems.  
        \algname{ARNCG$_g$}, \algname{ARNCG$_\epsilon$}, and ``Fixed'' correspond to \Cref{alg:adap-newton-cg} with the first and second regularizers from \theoremref{thm:newton-local-rate-boosted}, and a fixed $\omega_k \equiv \sqrt{\epsilon}$, respectively.  
        For Hessian evaluations, 
        since our algorithm accesses this information only via Hessian-vector products, 
        we count multiple products involving $\nabla^2\varphi(x)$ at the same point $x$ as a single evaluation.
        }
    \label{fig:main-algoperf}
\end{figure}

In this section, we present some preliminary numerical results.\footnote{Our code is available at \url{https://github.com/miskcoo/ARNCG}.} %
Our primary goal is to provide an overall sense of our algorithm's performance and the effects of its components.
Detailed results are deferred to \Cref{sec:appendix/numerical-results}.

Since the recently proposed trust-region-type method \algname{CAT} has an optimal rate and shows competitiveness with state-of-the-art solvers~\citep{hamad2024simple}, we adopt their experimental setup and compare with it, as well as the regularized Newton-type method \algname{AN2CER} proposed by \citet{gratton2024yet}.
The experiments are conducted on the 124 unconstrained problems with more than 100 variables from the widely used CUTEst benchmark for nonlinear optimization~\citep{gould2015cutest}.
The algorithm is considered successful if it terminates with $\epsilon_k \leq \epsilon = 10^{-5}$ such that $k \leq 10^5$. If the algorithm fails to terminate within 5 hours, it is also recorded as a failure.

In \Cref{sec:appendix/numerical-results}, 
we observe that the fallback step has insignificant impact on  performance yet increases computational cost, suggesting it can be relaxed or removed.
Furthermore, $\theta \in [0.5, 1]$ balances computational efficiency and local behavior 
and a small $m_{\mathrm{max}}$ is preferable. 
Finally, the second linesearch step \eqref{eqn:smooth-line-search-sol-smaller-stepsize} and the \texttt{TERM} state of \texttt{CappedCG} are rarely taken in practice.

\figureref{fig:main-algoperf} shows our method without the fallback step (see \Cref{sec:appendix/numerical-results} for details). 
It is slightly faster than CAT and AN2CER, 
as each iteration uses only a few Hessian-vector products, 
whereas CAT relies on multiple Cholesky factorizations and AN2CER involves minimal eigenvalue computations. 
Meanwhile, our method requires a similar number of Hessian evaluations as CAT, and slightly fewer than AN2CER.
We also note that using a fixed $\omega_k = \sqrt{\epsilon}$ in \Cref{alg:adap-newton-cg}
may lead to failures when $g_k \gg \epsilon$, resulting in deteriorated performance.
Additionally, our method requires significantly less memory ($\sim$6GB) compared to CAT ($\sim$74GB) for the largest problem in the benchmark with 123200 variables, as it avoids  constructing the full Hessian.

\section{Conclusion}
In this work, we propose a simple yet effective approach, called SMILE, for graph few-shot learning with fewer tasks. Specifically, we introduce a novel dual-level mixup strategy, including within-task and across-task mixup, for enriching the diversity of nodes within each task and the diversity of tasks. Also, we incorporate the degree-based prior information to learn expressive node embeddings. Theoretically, we prove that SMILE effectively enhances the model's generalization performance. Empirically, we conduct extensive experiments on multiple benchmarks and the results suggest that SMILE significantly outperforms other baselines, including both in-domain and cross-domain few-shot settings.


\section*{References}
\bibliographystyle{IEEEtran}
\bibliography{IEEEabrv,reference}

\section*{Appendix: Proofs of Main Results}
\subsection{Lloyd-Max Algorithm}
\label{subsec:Lloyd-Max}
For a given quantization bitwidth $B$ and an operand $\bm{X}$, the Lloyd-Max algorithm finds $2^B$ quantization levels $\{\hat{x}_i\}_{i=1}^{2^B}$ such that quantizing $\bm{X}$ by rounding each scalar in $\bm{X}$ to the nearest quantization level minimizes the quantization MSE. 

The algorithm starts with an initial guess of quantization levels and then iteratively computes quantization thresholds $\{\tau_i\}_{i=1}^{2^B-1}$ and updates quantization levels $\{\hat{x}_i\}_{i=1}^{2^B}$. Specifically, at iteration $n$, thresholds are set to the midpoints of the previous iteration's levels:
\begin{align*}
    \tau_i^{(n)}=\frac{\hat{x}_i^{(n-1)}+\hat{x}_{i+1}^{(n-1)}}2 \text{ for } i=1\ldots 2^B-1
\end{align*}
Subsequently, the quantization levels are re-computed as conditional means of the data regions defined by the new thresholds:
\begin{align*}
    \hat{x}_i^{(n)}=\mathbb{E}\left[ \bm{X} \big| \bm{X}\in [\tau_{i-1}^{(n)},\tau_i^{(n)}] \right] \text{ for } i=1\ldots 2^B
\end{align*}
where to satisfy boundary conditions we have $\tau_0=-\infty$ and $\tau_{2^B}=\infty$. The algorithm iterates the above steps until convergence.

Figure \ref{fig:lm_quant} compares the quantization levels of a $7$-bit floating point (E3M3) quantizer (left) to a $7$-bit Lloyd-Max quantizer (right) when quantizing a layer of weights from the GPT3-126M model at a per-tensor granularity. As shown, the Lloyd-Max quantizer achieves substantially lower quantization MSE. Further, Table \ref{tab:FP7_vs_LM7} shows the superior perplexity achieved by Lloyd-Max quantizers for bitwidths of $7$, $6$ and $5$. The difference between the quantizers is clear at 5 bits, where per-tensor FP quantization incurs a drastic and unacceptable increase in perplexity, while Lloyd-Max quantization incurs a much smaller increase. Nevertheless, we note that even the optimal Lloyd-Max quantizer incurs a notable ($\sim 1.5$) increase in perplexity due to the coarse granularity of quantization. 

\begin{figure}[h]
  \centering
  \includegraphics[width=0.7\linewidth]{sections/figures/LM7_FP7.pdf}
  \caption{\small Quantization levels and the corresponding quantization MSE of Floating Point (left) vs Lloyd-Max (right) Quantizers for a layer of weights in the GPT3-126M model.}
  \label{fig:lm_quant}
\end{figure}

\begin{table}[h]\scriptsize
\begin{center}
\caption{\label{tab:FP7_vs_LM7} \small Comparing perplexity (lower is better) achieved by floating point quantizers and Lloyd-Max quantizers on a GPT3-126M model for the Wikitext-103 dataset.}
\begin{tabular}{c|cc|c}
\hline
 \multirow{2}{*}{\textbf{Bitwidth}} & \multicolumn{2}{|c|}{\textbf{Floating-Point Quantizer}} & \textbf{Lloyd-Max Quantizer} \\
 & Best Format & Wikitext-103 Perplexity & Wikitext-103 Perplexity \\
\hline
7 & E3M3 & 18.32 & 18.27 \\
6 & E3M2 & 19.07 & 18.51 \\
5 & E4M0 & 43.89 & 19.71 \\
\hline
\end{tabular}
\end{center}
\end{table}

\subsection{Proof of Local Optimality of LO-BCQ}
\label{subsec:lobcq_opt_proof}
For a given block $\bm{b}_j$, the quantization MSE during LO-BCQ can be empirically evaluated as $\frac{1}{L_b}\lVert \bm{b}_j- \bm{\hat{b}}_j\rVert^2_2$ where $\bm{\hat{b}}_j$ is computed from equation (\ref{eq:clustered_quantization_definition}) as $C_{f(\bm{b}_j)}(\bm{b}_j)$. Further, for a given block cluster $\mathcal{B}_i$, we compute the quantization MSE as $\frac{1}{|\mathcal{B}_{i}|}\sum_{\bm{b} \in \mathcal{B}_{i}} \frac{1}{L_b}\lVert \bm{b}- C_i^{(n)}(\bm{b})\rVert^2_2$. Therefore, at the end of iteration $n$, we evaluate the overall quantization MSE $J^{(n)}$ for a given operand $\bm{X}$ composed of $N_c$ block clusters as:
\begin{align*}
    \label{eq:mse_iter_n}
    J^{(n)} = \frac{1}{N_c} \sum_{i=1}^{N_c} \frac{1}{|\mathcal{B}_{i}^{(n)}|}\sum_{\bm{v} \in \mathcal{B}_{i}^{(n)}} \frac{1}{L_b}\lVert \bm{b}- B_i^{(n)}(\bm{b})\rVert^2_2
\end{align*}

At the end of iteration $n$, the codebooks are updated from $\mathcal{C}^{(n-1)}$ to $\mathcal{C}^{(n)}$. However, the mapping of a given vector $\bm{b}_j$ to quantizers $\mathcal{C}^{(n)}$ remains as  $f^{(n)}(\bm{b}_j)$. At the next iteration, during the vector clustering step, $f^{(n+1)}(\bm{b}_j)$ finds new mapping of $\bm{b}_j$ to updated codebooks $\mathcal{C}^{(n)}$ such that the quantization MSE over the candidate codebooks is minimized. Therefore, we obtain the following result for $\bm{b}_j$:
\begin{align*}
\frac{1}{L_b}\lVert \bm{b}_j - C_{f^{(n+1)}(\bm{b}_j)}^{(n)}(\bm{b}_j)\rVert^2_2 \le \frac{1}{L_b}\lVert \bm{b}_j - C_{f^{(n)}(\bm{b}_j)}^{(n)}(\bm{b}_j)\rVert^2_2
\end{align*}

That is, quantizing $\bm{b}_j$ at the end of the block clustering step of iteration $n+1$ results in lower quantization MSE compared to quantizing at the end of iteration $n$. Since this is true for all $\bm{b} \in \bm{X}$, we assert the following:
\begin{equation}
\begin{split}
\label{eq:mse_ineq_1}
    \tilde{J}^{(n+1)} &= \frac{1}{N_c} \sum_{i=1}^{N_c} \frac{1}{|\mathcal{B}_{i}^{(n+1)}|}\sum_{\bm{b} \in \mathcal{B}_{i}^{(n+1)}} \frac{1}{L_b}\lVert \bm{b} - C_i^{(n)}(b)\rVert^2_2 \le J^{(n)}
\end{split}
\end{equation}
where $\tilde{J}^{(n+1)}$ is the the quantization MSE after the vector clustering step at iteration $n+1$.

Next, during the codebook update step (\ref{eq:quantizers_update}) at iteration $n+1$, the per-cluster codebooks $\mathcal{C}^{(n)}$ are updated to $\mathcal{C}^{(n+1)}$ by invoking the Lloyd-Max algorithm \citep{Lloyd}. We know that for any given value distribution, the Lloyd-Max algorithm minimizes the quantization MSE. Therefore, for a given vector cluster $\mathcal{B}_i$ we obtain the following result:

\begin{equation}
    \frac{1}{|\mathcal{B}_{i}^{(n+1)}|}\sum_{\bm{b} \in \mathcal{B}_{i}^{(n+1)}} \frac{1}{L_b}\lVert \bm{b}- C_i^{(n+1)}(\bm{b})\rVert^2_2 \le \frac{1}{|\mathcal{B}_{i}^{(n+1)}|}\sum_{\bm{b} \in \mathcal{B}_{i}^{(n+1)}} \frac{1}{L_b}\lVert \bm{b}- C_i^{(n)}(\bm{b})\rVert^2_2
\end{equation}

The above equation states that quantizing the given block cluster $\mathcal{B}_i$ after updating the associated codebook from $C_i^{(n)}$ to $C_i^{(n+1)}$ results in lower quantization MSE. Since this is true for all the block clusters, we derive the following result: 
\begin{equation}
\begin{split}
\label{eq:mse_ineq_2}
     J^{(n+1)} &= \frac{1}{N_c} \sum_{i=1}^{N_c} \frac{1}{|\mathcal{B}_{i}^{(n+1)}|}\sum_{\bm{b} \in \mathcal{B}_{i}^{(n+1)}} \frac{1}{L_b}\lVert \bm{b}- C_i^{(n+1)}(\bm{b})\rVert^2_2  \le \tilde{J}^{(n+1)}   
\end{split}
\end{equation}

Following (\ref{eq:mse_ineq_1}) and (\ref{eq:mse_ineq_2}), we find that the quantization MSE is non-increasing for each iteration, that is, $J^{(1)} \ge J^{(2)} \ge J^{(3)} \ge \ldots \ge J^{(M)}$ where $M$ is the maximum number of iterations. 
%Therefore, we can say that if the algorithm converges, then it must be that it has converged to a local minimum. 
\hfill $\blacksquare$


\begin{figure}
    \begin{center}
    \includegraphics[width=0.5\textwidth]{sections//figures/mse_vs_iter.pdf}
    \end{center}
    \caption{\small NMSE vs iterations during LO-BCQ compared to other block quantization proposals}
    \label{fig:nmse_vs_iter}
\end{figure}

Figure \ref{fig:nmse_vs_iter} shows the empirical convergence of LO-BCQ across several block lengths and number of codebooks. Also, the MSE achieved by LO-BCQ is compared to baselines such as MXFP and VSQ. As shown, LO-BCQ converges to a lower MSE than the baselines. Further, we achieve better convergence for larger number of codebooks ($N_c$) and for a smaller block length ($L_b$), both of which increase the bitwidth of BCQ (see Eq \ref{eq:bitwidth_bcq}).


\subsection{Additional Accuracy Results}
%Table \ref{tab:lobcq_config} lists the various LOBCQ configurations and their corresponding bitwidths.
\begin{table}
\setlength{\tabcolsep}{4.75pt}
\begin{center}
\caption{\label{tab:lobcq_config} Various LO-BCQ configurations and their bitwidths.}
\begin{tabular}{|c||c|c|c|c||c|c||c|} 
\hline
 & \multicolumn{4}{|c||}{$L_b=8$} & \multicolumn{2}{|c||}{$L_b=4$} & $L_b=2$ \\
 \hline
 \backslashbox{$L_A$\kern-1em}{\kern-1em$N_c$} & 2 & 4 & 8 & 16 & 2 & 4 & 2 \\
 \hline
 64 & 4.25 & 4.375 & 4.5 & 4.625 & 4.375 & 4.625 & 4.625\\
 \hline
 32 & 4.375 & 4.5 & 4.625& 4.75 & 4.5 & 4.75 & 4.75 \\
 \hline
 16 & 4.625 & 4.75& 4.875 & 5 & 4.75 & 5 & 5 \\
 \hline
\end{tabular}
\end{center}
\end{table}

%\subsection{Perplexity achieved by various LO-BCQ configurations on Wikitext-103 dataset}

\begin{table} \centering
\begin{tabular}{|c||c|c|c|c||c|c||c|} 
\hline
 $L_b \rightarrow$& \multicolumn{4}{c||}{8} & \multicolumn{2}{c||}{4} & 2\\
 \hline
 \backslashbox{$L_A$\kern-1em}{\kern-1em$N_c$} & 2 & 4 & 8 & 16 & 2 & 4 & 2  \\
 %$N_c \rightarrow$ & 2 & 4 & 8 & 16 & 2 & 4 & 2 \\
 \hline
 \hline
 \multicolumn{8}{c}{GPT3-1.3B (FP32 PPL = 9.98)} \\ 
 \hline
 \hline
 64 & 10.40 & 10.23 & 10.17 & 10.15 &  10.28 & 10.18 & 10.19 \\
 \hline
 32 & 10.25 & 10.20 & 10.15 & 10.12 &  10.23 & 10.17 & 10.17 \\
 \hline
 16 & 10.22 & 10.16 & 10.10 & 10.09 &  10.21 & 10.14 & 10.16 \\
 \hline
  \hline
 \multicolumn{8}{c}{GPT3-8B (FP32 PPL = 7.38)} \\ 
 \hline
 \hline
 64 & 7.61 & 7.52 & 7.48 &  7.47 &  7.55 &  7.49 & 7.50 \\
 \hline
 32 & 7.52 & 7.50 & 7.46 &  7.45 &  7.52 &  7.48 & 7.48  \\
 \hline
 16 & 7.51 & 7.48 & 7.44 &  7.44 &  7.51 &  7.49 & 7.47  \\
 \hline
\end{tabular}
\caption{\label{tab:ppl_gpt3_abalation} Wikitext-103 perplexity across GPT3-1.3B and 8B models.}
\end{table}

\begin{table} \centering
\begin{tabular}{|c||c|c|c|c||} 
\hline
 $L_b \rightarrow$& \multicolumn{4}{c||}{8}\\
 \hline
 \backslashbox{$L_A$\kern-1em}{\kern-1em$N_c$} & 2 & 4 & 8 & 16 \\
 %$N_c \rightarrow$ & 2 & 4 & 8 & 16 & 2 & 4 & 2 \\
 \hline
 \hline
 \multicolumn{5}{|c|}{Llama2-7B (FP32 PPL = 5.06)} \\ 
 \hline
 \hline
 64 & 5.31 & 5.26 & 5.19 & 5.18  \\
 \hline
 32 & 5.23 & 5.25 & 5.18 & 5.15  \\
 \hline
 16 & 5.23 & 5.19 & 5.16 & 5.14  \\
 \hline
 \multicolumn{5}{|c|}{Nemotron4-15B (FP32 PPL = 5.87)} \\ 
 \hline
 \hline
 64  & 6.3 & 6.20 & 6.13 & 6.08  \\
 \hline
 32  & 6.24 & 6.12 & 6.07 & 6.03  \\
 \hline
 16  & 6.12 & 6.14 & 6.04 & 6.02  \\
 \hline
 \multicolumn{5}{|c|}{Nemotron4-340B (FP32 PPL = 3.48)} \\ 
 \hline
 \hline
 64 & 3.67 & 3.62 & 3.60 & 3.59 \\
 \hline
 32 & 3.63 & 3.61 & 3.59 & 3.56 \\
 \hline
 16 & 3.61 & 3.58 & 3.57 & 3.55 \\
 \hline
\end{tabular}
\caption{\label{tab:ppl_llama7B_nemo15B} Wikitext-103 perplexity compared to FP32 baseline in Llama2-7B and Nemotron4-15B, 340B models}
\end{table}

%\subsection{Perplexity achieved by various LO-BCQ configurations on MMLU dataset}


\begin{table} \centering
\begin{tabular}{|c||c|c|c|c||c|c|c|c|} 
\hline
 $L_b \rightarrow$& \multicolumn{4}{c||}{8} & \multicolumn{4}{c||}{8}\\
 \hline
 \backslashbox{$L_A$\kern-1em}{\kern-1em$N_c$} & 2 & 4 & 8 & 16 & 2 & 4 & 8 & 16  \\
 %$N_c \rightarrow$ & 2 & 4 & 8 & 16 & 2 & 4 & 2 \\
 \hline
 \hline
 \multicolumn{5}{|c|}{Llama2-7B (FP32 Accuracy = 45.8\%)} & \multicolumn{4}{|c|}{Llama2-70B (FP32 Accuracy = 69.12\%)} \\ 
 \hline
 \hline
 64 & 43.9 & 43.4 & 43.9 & 44.9 & 68.07 & 68.27 & 68.17 & 68.75 \\
 \hline
 32 & 44.5 & 43.8 & 44.9 & 44.5 & 68.37 & 68.51 & 68.35 & 68.27  \\
 \hline
 16 & 43.9 & 42.7 & 44.9 & 45 & 68.12 & 68.77 & 68.31 & 68.59  \\
 \hline
 \hline
 \multicolumn{5}{|c|}{GPT3-22B (FP32 Accuracy = 38.75\%)} & \multicolumn{4}{|c|}{Nemotron4-15B (FP32 Accuracy = 64.3\%)} \\ 
 \hline
 \hline
 64 & 36.71 & 38.85 & 38.13 & 38.92 & 63.17 & 62.36 & 63.72 & 64.09 \\
 \hline
 32 & 37.95 & 38.69 & 39.45 & 38.34 & 64.05 & 62.30 & 63.8 & 64.33  \\
 \hline
 16 & 38.88 & 38.80 & 38.31 & 38.92 & 63.22 & 63.51 & 63.93 & 64.43  \\
 \hline
\end{tabular}
\caption{\label{tab:mmlu_abalation} Accuracy on MMLU dataset across GPT3-22B, Llama2-7B, 70B and Nemotron4-15B models.}
\end{table}


%\subsection{Perplexity achieved by various LO-BCQ configurations on LM evaluation harness}

\begin{table} \centering
\begin{tabular}{|c||c|c|c|c||c|c|c|c|} 
\hline
 $L_b \rightarrow$& \multicolumn{4}{c||}{8} & \multicolumn{4}{c||}{8}\\
 \hline
 \backslashbox{$L_A$\kern-1em}{\kern-1em$N_c$} & 2 & 4 & 8 & 16 & 2 & 4 & 8 & 16  \\
 %$N_c \rightarrow$ & 2 & 4 & 8 & 16 & 2 & 4 & 2 \\
 \hline
 \hline
 \multicolumn{5}{|c|}{Race (FP32 Accuracy = 37.51\%)} & \multicolumn{4}{|c|}{Boolq (FP32 Accuracy = 64.62\%)} \\ 
 \hline
 \hline
 64 & 36.94 & 37.13 & 36.27 & 37.13 & 63.73 & 62.26 & 63.49 & 63.36 \\
 \hline
 32 & 37.03 & 36.36 & 36.08 & 37.03 & 62.54 & 63.51 & 63.49 & 63.55  \\
 \hline
 16 & 37.03 & 37.03 & 36.46 & 37.03 & 61.1 & 63.79 & 63.58 & 63.33  \\
 \hline
 \hline
 \multicolumn{5}{|c|}{Winogrande (FP32 Accuracy = 58.01\%)} & \multicolumn{4}{|c|}{Piqa (FP32 Accuracy = 74.21\%)} \\ 
 \hline
 \hline
 64 & 58.17 & 57.22 & 57.85 & 58.33 & 73.01 & 73.07 & 73.07 & 72.80 \\
 \hline
 32 & 59.12 & 58.09 & 57.85 & 58.41 & 73.01 & 73.94 & 72.74 & 73.18  \\
 \hline
 16 & 57.93 & 58.88 & 57.93 & 58.56 & 73.94 & 72.80 & 73.01 & 73.94  \\
 \hline
\end{tabular}
\caption{\label{tab:mmlu_abalation} Accuracy on LM evaluation harness tasks on GPT3-1.3B model.}
\end{table}

\begin{table} \centering
\begin{tabular}{|c||c|c|c|c||c|c|c|c|} 
\hline
 $L_b \rightarrow$& \multicolumn{4}{c||}{8} & \multicolumn{4}{c||}{8}\\
 \hline
 \backslashbox{$L_A$\kern-1em}{\kern-1em$N_c$} & 2 & 4 & 8 & 16 & 2 & 4 & 8 & 16  \\
 %$N_c \rightarrow$ & 2 & 4 & 8 & 16 & 2 & 4 & 2 \\
 \hline
 \hline
 \multicolumn{5}{|c|}{Race (FP32 Accuracy = 41.34\%)} & \multicolumn{4}{|c|}{Boolq (FP32 Accuracy = 68.32\%)} \\ 
 \hline
 \hline
 64 & 40.48 & 40.10 & 39.43 & 39.90 & 69.20 & 68.41 & 69.45 & 68.56 \\
 \hline
 32 & 39.52 & 39.52 & 40.77 & 39.62 & 68.32 & 67.43 & 68.17 & 69.30  \\
 \hline
 16 & 39.81 & 39.71 & 39.90 & 40.38 & 68.10 & 66.33 & 69.51 & 69.42  \\
 \hline
 \hline
 \multicolumn{5}{|c|}{Winogrande (FP32 Accuracy = 67.88\%)} & \multicolumn{4}{|c|}{Piqa (FP32 Accuracy = 78.78\%)} \\ 
 \hline
 \hline
 64 & 66.85 & 66.61 & 67.72 & 67.88 & 77.31 & 77.42 & 77.75 & 77.64 \\
 \hline
 32 & 67.25 & 67.72 & 67.72 & 67.00 & 77.31 & 77.04 & 77.80 & 77.37  \\
 \hline
 16 & 68.11 & 68.90 & 67.88 & 67.48 & 77.37 & 78.13 & 78.13 & 77.69  \\
 \hline
\end{tabular}
\caption{\label{tab:mmlu_abalation} Accuracy on LM evaluation harness tasks on GPT3-8B model.}
\end{table}

\begin{table} \centering
\begin{tabular}{|c||c|c|c|c||c|c|c|c|} 
\hline
 $L_b \rightarrow$& \multicolumn{4}{c||}{8} & \multicolumn{4}{c||}{8}\\
 \hline
 \backslashbox{$L_A$\kern-1em}{\kern-1em$N_c$} & 2 & 4 & 8 & 16 & 2 & 4 & 8 & 16  \\
 %$N_c \rightarrow$ & 2 & 4 & 8 & 16 & 2 & 4 & 2 \\
 \hline
 \hline
 \multicolumn{5}{|c|}{Race (FP32 Accuracy = 40.67\%)} & \multicolumn{4}{|c|}{Boolq (FP32 Accuracy = 76.54\%)} \\ 
 \hline
 \hline
 64 & 40.48 & 40.10 & 39.43 & 39.90 & 75.41 & 75.11 & 77.09 & 75.66 \\
 \hline
 32 & 39.52 & 39.52 & 40.77 & 39.62 & 76.02 & 76.02 & 75.96 & 75.35  \\
 \hline
 16 & 39.81 & 39.71 & 39.90 & 40.38 & 75.05 & 73.82 & 75.72 & 76.09  \\
 \hline
 \hline
 \multicolumn{5}{|c|}{Winogrande (FP32 Accuracy = 70.64\%)} & \multicolumn{4}{|c|}{Piqa (FP32 Accuracy = 79.16\%)} \\ 
 \hline
 \hline
 64 & 69.14 & 70.17 & 70.17 & 70.56 & 78.24 & 79.00 & 78.62 & 78.73 \\
 \hline
 32 & 70.96 & 69.69 & 71.27 & 69.30 & 78.56 & 79.49 & 79.16 & 78.89  \\
 \hline
 16 & 71.03 & 69.53 & 69.69 & 70.40 & 78.13 & 79.16 & 79.00 & 79.00  \\
 \hline
\end{tabular}
\caption{\label{tab:mmlu_abalation} Accuracy on LM evaluation harness tasks on GPT3-22B model.}
\end{table}

\begin{table} \centering
\begin{tabular}{|c||c|c|c|c||c|c|c|c|} 
\hline
 $L_b \rightarrow$& \multicolumn{4}{c||}{8} & \multicolumn{4}{c||}{8}\\
 \hline
 \backslashbox{$L_A$\kern-1em}{\kern-1em$N_c$} & 2 & 4 & 8 & 16 & 2 & 4 & 8 & 16  \\
 %$N_c \rightarrow$ & 2 & 4 & 8 & 16 & 2 & 4 & 2 \\
 \hline
 \hline
 \multicolumn{5}{|c|}{Race (FP32 Accuracy = 44.4\%)} & \multicolumn{4}{|c|}{Boolq (FP32 Accuracy = 79.29\%)} \\ 
 \hline
 \hline
 64 & 42.49 & 42.51 & 42.58 & 43.45 & 77.58 & 77.37 & 77.43 & 78.1 \\
 \hline
 32 & 43.35 & 42.49 & 43.64 & 43.73 & 77.86 & 75.32 & 77.28 & 77.86  \\
 \hline
 16 & 44.21 & 44.21 & 43.64 & 42.97 & 78.65 & 77 & 76.94 & 77.98  \\
 \hline
 \hline
 \multicolumn{5}{|c|}{Winogrande (FP32 Accuracy = 69.38\%)} & \multicolumn{4}{|c|}{Piqa (FP32 Accuracy = 78.07\%)} \\ 
 \hline
 \hline
 64 & 68.9 & 68.43 & 69.77 & 68.19 & 77.09 & 76.82 & 77.09 & 77.86 \\
 \hline
 32 & 69.38 & 68.51 & 68.82 & 68.90 & 78.07 & 76.71 & 78.07 & 77.86  \\
 \hline
 16 & 69.53 & 67.09 & 69.38 & 68.90 & 77.37 & 77.8 & 77.91 & 77.69  \\
 \hline
\end{tabular}
\caption{\label{tab:mmlu_abalation} Accuracy on LM evaluation harness tasks on Llama2-7B model.}
\end{table}

\begin{table} \centering
\begin{tabular}{|c||c|c|c|c||c|c|c|c|} 
\hline
 $L_b \rightarrow$& \multicolumn{4}{c||}{8} & \multicolumn{4}{c||}{8}\\
 \hline
 \backslashbox{$L_A$\kern-1em}{\kern-1em$N_c$} & 2 & 4 & 8 & 16 & 2 & 4 & 8 & 16  \\
 %$N_c \rightarrow$ & 2 & 4 & 8 & 16 & 2 & 4 & 2 \\
 \hline
 \hline
 \multicolumn{5}{|c|}{Race (FP32 Accuracy = 48.8\%)} & \multicolumn{4}{|c|}{Boolq (FP32 Accuracy = 85.23\%)} \\ 
 \hline
 \hline
 64 & 49.00 & 49.00 & 49.28 & 48.71 & 82.82 & 84.28 & 84.03 & 84.25 \\
 \hline
 32 & 49.57 & 48.52 & 48.33 & 49.28 & 83.85 & 84.46 & 84.31 & 84.93  \\
 \hline
 16 & 49.85 & 49.09 & 49.28 & 48.99 & 85.11 & 84.46 & 84.61 & 83.94  \\
 \hline
 \hline
 \multicolumn{5}{|c|}{Winogrande (FP32 Accuracy = 79.95\%)} & \multicolumn{4}{|c|}{Piqa (FP32 Accuracy = 81.56\%)} \\ 
 \hline
 \hline
 64 & 78.77 & 78.45 & 78.37 & 79.16 & 81.45 & 80.69 & 81.45 & 81.5 \\
 \hline
 32 & 78.45 & 79.01 & 78.69 & 80.66 & 81.56 & 80.58 & 81.18 & 81.34  \\
 \hline
 16 & 79.95 & 79.56 & 79.79 & 79.72 & 81.28 & 81.66 & 81.28 & 80.96  \\
 \hline
\end{tabular}
\caption{\label{tab:mmlu_abalation} Accuracy on LM evaluation harness tasks on Llama2-70B model.}
\end{table}

%\section{MSE Studies}
%\textcolor{red}{TODO}


\subsection{Number Formats and Quantization Method}
\label{subsec:numFormats_quantMethod}
\subsubsection{Integer Format}
An $n$-bit signed integer (INT) is typically represented with a 2s-complement format \citep{yao2022zeroquant,xiao2023smoothquant,dai2021vsq}, where the most significant bit denotes the sign.

\subsubsection{Floating Point Format}
An $n$-bit signed floating point (FP) number $x$ comprises of a 1-bit sign ($x_{\mathrm{sign}}$), $B_m$-bit mantissa ($x_{\mathrm{mant}}$) and $B_e$-bit exponent ($x_{\mathrm{exp}}$) such that $B_m+B_e=n-1$. The associated constant exponent bias ($E_{\mathrm{bias}}$) is computed as $(2^{{B_e}-1}-1)$. We denote this format as $E_{B_e}M_{B_m}$.  

\subsubsection{Quantization Scheme}
\label{subsec:quant_method}
A quantization scheme dictates how a given unquantized tensor is converted to its quantized representation. We consider FP formats for the purpose of illustration. Given an unquantized tensor $\bm{X}$ and an FP format $E_{B_e}M_{B_m}$, we first, we compute the quantization scale factor $s_X$ that maps the maximum absolute value of $\bm{X}$ to the maximum quantization level of the $E_{B_e}M_{B_m}$ format as follows:
\begin{align}
\label{eq:sf}
    s_X = \frac{\mathrm{max}(|\bm{X}|)}{\mathrm{max}(E_{B_e}M_{B_m})}
\end{align}
In the above equation, $|\cdot|$ denotes the absolute value function.

Next, we scale $\bm{X}$ by $s_X$ and quantize it to $\hat{\bm{X}}$ by rounding it to the nearest quantization level of $E_{B_e}M_{B_m}$ as:

\begin{align}
\label{eq:tensor_quant}
    \hat{\bm{X}} = \text{round-to-nearest}\left(\frac{\bm{X}}{s_X}, E_{B_e}M_{B_m}\right)
\end{align}

We perform dynamic max-scaled quantization \citep{wu2020integer}, where the scale factor $s$ for activations is dynamically computed during runtime.

\subsection{Vector Scaled Quantization}
\begin{wrapfigure}{r}{0.35\linewidth}
  \centering
  \includegraphics[width=\linewidth]{sections/figures/vsquant.jpg}
  \caption{\small Vectorwise decomposition for per-vector scaled quantization (VSQ \citep{dai2021vsq}).}
  \label{fig:vsquant}
\end{wrapfigure}
During VSQ \citep{dai2021vsq}, the operand tensors are decomposed into 1D vectors in a hardware friendly manner as shown in Figure \ref{fig:vsquant}. Since the decomposed tensors are used as operands in matrix multiplications during inference, it is beneficial to perform this decomposition along the reduction dimension of the multiplication. The vectorwise quantization is performed similar to tensorwise quantization described in Equations \ref{eq:sf} and \ref{eq:tensor_quant}, where a scale factor $s_v$ is required for each vector $\bm{v}$ that maps the maximum absolute value of that vector to the maximum quantization level. While smaller vector lengths can lead to larger accuracy gains, the associated memory and computational overheads due to the per-vector scale factors increases. To alleviate these overheads, VSQ \citep{dai2021vsq} proposed a second level quantization of the per-vector scale factors to unsigned integers, while MX \citep{rouhani2023shared} quantizes them to integer powers of 2 (denoted as $2^{INT}$).

\subsubsection{MX Format}
The MX format proposed in \citep{rouhani2023microscaling} introduces the concept of sub-block shifting. For every two scalar elements of $b$-bits each, there is a shared exponent bit. The value of this exponent bit is determined through an empirical analysis that targets minimizing quantization MSE. We note that the FP format $E_{1}M_{b}$ is strictly better than MX from an accuracy perspective since it allocates a dedicated exponent bit to each scalar as opposed to sharing it across two scalars. Therefore, we conservatively bound the accuracy of a $b+2$-bit signed MX format with that of a $E_{1}M_{b}$ format in our comparisons. For instance, we use E1M2 format as a proxy for MX4.

\begin{figure}
    \centering
    \includegraphics[width=1\linewidth]{sections//figures/BlockFormats.pdf}
    \caption{\small Comparing LO-BCQ to MX format.}
    \label{fig:block_formats}
\end{figure}

Figure \ref{fig:block_formats} compares our $4$-bit LO-BCQ block format to MX \citep{rouhani2023microscaling}. As shown, both LO-BCQ and MX decompose a given operand tensor into block arrays and each block array into blocks. Similar to MX, we find that per-block quantization ($L_b < L_A$) leads to better accuracy due to increased flexibility. While MX achieves this through per-block $1$-bit micro-scales, we associate a dedicated codebook to each block through a per-block codebook selector. Further, MX quantizes the per-block array scale-factor to E8M0 format without per-tensor scaling. In contrast during LO-BCQ, we find that per-tensor scaling combined with quantization of per-block array scale-factor to E4M3 format results in superior inference accuracy across models. 

% \vskip -30pt plus -1fil
\begin{IEEEbiography}[{\includegraphics[width=1in,height=1.25in,clip,keepaspectratio]{PersonalFigure/JI.jpg}}]{Jaehan Im} (Graduate Student Member, IEEE) was born in Seoul, Republic of Korea. He received the B.S. and M.S. degrees in Aerospace Engineering from the Korea Advanced Institute of Science and Technology (KAIST). He is currently pursuing the Ph.D. degree in the Department of Aerospace Engineering and Engineering Mechanics at the University of Texas at Austin.
\end{IEEEbiography}

% \vskip 0pt plus -1fil

\begin{IEEEbiography}[{\includegraphics[width=1in,height=1.25in,clip,keepaspectratio]{PersonalFigure/Filippos.jpg}}]{Filippos Fotiadis}
(Member, IEEE) was born in Thessaloniki, Greece. He received the PhD degree in Aerospace Engineering in 2024, and the MS degrees in Aerospace Engineering and Mathematics in 2022 and 2023, all from Georgia Tech. Prior to his graduate studies, he received a diploma in Electrical \& Computer Engineering from the Aristotle University of Thessaloniki. He is currently a postdoctoral researcher at the Oden Institute for Computational Engineering \& Sciences at the University of Texas at Austin.
His research interests are in the intersection of systems \& control theory, game theory, and learning, with applications to the security and resilience of cyber-physical systems.
\end{IEEEbiography}

% \vskip 0pt plus -1fil

\begin{IEEEbiography}[{\includegraphics[width=1in,height=1.25in,clip,keepaspectratio]{PersonalFigure/Daniel.png}}]{Daniel Delahaye} (Member, IEEE) is the head of the Optimization and Machine Learning Team of the ENAC research laboratory and he is also in charge of the research program “AI4DECARBO” in the new AI institute ANITI in Toulouse and member of the SESAR scientific committee.
He obtained his engineering degree from the ENAC school and did a master of science in signal processing from the National Polytechnic Institute of Toulouse in 1991. He received his PH.D in automatic control from the Aeronautic and Space National School in 1995 under the co-supervision of Marc Schoenauer (CMAPX). He did a post-doc at the Department of Aeronautics and Astronautics at MIT in 1996 under the supervision of Pr Amedeo Odoni. He started his career working at the French Civil Aviation Study Center (CENA) and moved to ENAC in 2008. He got his tenure in applied mathematics in 2012. He conducts research on mathematical optimization and artificial intelligence for airspace design and aircraft trajectory optimization.
\end{IEEEbiography}

% \vskip 0pt plus -1fil

\begin{IEEEbiography}[{\includegraphics[width=1in,height=1.25in,clip,keepaspectratio]{PersonalFigure/UT.jpg}}]{Ufuk Topcu}(Fellow, IEEE)  is currently a Professor
with the Department of Aerospace Engineering and
Engineering Mechanics, The University of Texas at
Austin, Austin, TX, USA, where he holds the Temple
Foundation Endowed Professorship No. 1 Professorship. He is a core Faculty Member with the Oden Institute for Computational Engineering and Sciences and
Texas Robotics and the director of the Autonomous
Systems Group. His research interests include the
theoretical and algorithmic aspects of the design and
verification of autonomous systems, typically in the
intersection of formal methods, reinforcement learning, and control theory.
\end{IEEEbiography}

% \vskip 0pt plus -1fil

\begin{IEEEbiography}[{\includegraphics[width=1in,height=1.25in,clip,keepaspectratio]{PersonalFigure/David.jpg}}]{David Fridovich-Keil} (Member, IEEE) received the B.S.E. degree in electrical engineering from Princeton University, and the Ph.D. Degree from the University of California, Berkeley. He is an Assistant Professor in the Department of Aerospace Engineering and Engineering Mechanics at the University of Texas at Austin. Fridovich-Keil is the recipient of an NSF Graduate Research Fellowship and an NSF CAREER Award.
\end{IEEEbiography}

\end{document}
