\section{Trading auction for consensus}

The Trading Auction for Consensus (\texttt{TACo}) algorithm is designed to address the challenge of decentralized \emph{conflicting choice} problems among non-cooperative agents. The primary goal of \texttt{TACo} is to help agents reach a consensus or agree on a single choice, even when they have conflicting preferences. The algorithm accomplishes this without requiring direct negotiation or communication between agents, and it preserves each agent's private information during the process.

\subsection{Key elements of \texttt{TACo}}

\texttt{TACo} consists of several key components: a pay matrix, an offer matrix, a cost matrix, a profit matrix, the private valuations over secondary assets, the amount of trading units, and the decrement factor. These components are central to how agents compute their profit and make decisions during \texttt{TACo}. Let $n \in \mathbb{Z}_{\geq0}$ denote the number of agents, and $m \in \mathbb{Z}_{\geq0}$ the number of choices available in the auction. For any matrix $X$, $X_{ij}$ refers to its $(i,j)$ entry.

\begin{enumerate}[\hspace{0pt}1)]
    \item \textbf{Offer matrix ($\offerList$):} The offer matrix $\offerList \in \mathbb{R}^{n \times m}$ represents the units that each agent $i$ can receive if they select choice $j$. Specifically, $\offerList_{ij}$  denotes the number of units that agent $i$ will receive when they choose choice $j$, determined by the offers made by other agents.
    
    \item \textbf{Pay matrix ($\payList$):} The pay matrix $\payList \in \mathbb{R}^{n \times m}$ captures the cost each agent incurs when selecting a particular choice. $\payList_{ij}$ is the number of units that agent $i$ has to pay to others when they choose choice $j$.
    
    \item \textbf{Cost matrix ($\costList$):} The cost matrix $\costList \in \mathbb{R}^{n \times m}$ reflects the intrinsic cost associated with each agent's choice. $\costList_{ij}$ represents the original cost experienced by agent $i$ when they select choice $j$.
    
    \item \textbf{Private valuation ($\valuation$):} Each agent $i$ assigns a private valuation, $b_i$, to the units of the asset being traded in the \texttt{TACo} algorithm. This valuation, represented as the $i$-th element of the vector $\valuation \in \mathbb{R}_{>0}^n$, reflects how much agent $i$ values each unit of the secondary asset, such as carbon emission credits. The value $b_i$ is unique to each agent and remains private, influencing how the agent perceives the offer and pay quantities when calculating their profit.
    \item \textbf{Amount of trading units ($d$):} The constant $d \in \mathbb{R}_{>0}$ represents the amount of secondary assets traded in each auction step. This public quantity directly impacts the updates of the offer and pay matrices, influencing each agent's profit calculation.
    \item \textbf{Trading unit decrement factor ($\decFactor$):} The decrement factor $\decFactor \in (0,1)$ is a parameter used to reduce the number of trading units when a specified condition is met. When triggered, the amount of trading units $d$ is scaled by the decrement factor:
    \begin{equation}
    d \leftarrow \decFactor d.
    \end{equation}
    This reduction adjusts the scale of assets exchanged in the auction, allowing for a controlled decrease in trading volume over time or under certain conditions, which can influence the convergence behavior of the auction process.
    
    \item \textbf{Profit matrix ($\profitList$):} The profit matrix $\profitList \in \mathbb{R}^{n \times m}$ represents the profit each agent gains based on their choice. Specifically, $\profitList_{ij}$ denotes the profit experienced by agent $i$ when selecting choice $j$. The profit is calculated as follows:
    \begin{equation}
        \profitList = \text{diag}(\valuation) \cdot (\offerList - \payList) - \costList.
    \end{equation}
    This equation accounts for each agent's private valuation $\textbf{b}$, the offers received $\offerList$, the payments made $\payList$, and the intrinsic costs $\costList$, resulting in the total profit for each possible choice.
\end{enumerate}

\subsection{Auction mechanism} \label{sec:AuctionMechanism}

\texttt{TACo} operates through sequential steps, where agents take turns making decisions. At each step, an agent selects the choice they value most and attempts to persuade others by offering payments. These decisions are based on the matrices $\offerList$, $\payList$, and $\costList$. Before explaining the \texttt{TACo} mechanism, we introduce the following definitions.

\begin{definition}[Agent-profit matrix tuple $(i, \profitList)$]  
The agent-profit matrix tuple $(i, \profitList)$ represents the association of agent index $i \in \mathbb{N}$ with a specific profit matrix $\profitList$ applied in the current step.
\end{definition}

A cycle refers to a sequence of auction processes that returns to a previously encountered profit matrix–playing agent pair. Once the process revisits an existing profit matrix-playing agent pair, the sequence repeats, forming a loop. An interesting property of a cycle is that it includes a state where all agents have reached an agreement, particularly in the trivial case
where they continuously select the same choice without termination. 

\begin{definition}[Cycle, $\cycle$]\label{def:cycle}  
A cycle, $\cycle$, is a finite sequence of agent-profit matrix tuples,  
\[
\cycle=\left((i^{k_1},\profitList^{k_1}), (i^{k_2},\profitList^{k_2}), \ldots, (i^{k_\ell},\profitList^{k_\ell})\right),
\]
in which a tuple $(i^{k_{\ell+1}}, \profitList^{k_{\ell+1}})$ revisits any previously encountered tuple, i.e., 
\[
(i^{k_{\ell+1}}, \profitList^{k_{\ell+1}}) = (i^{k_1}, \profitList^{k_1}).
\]
Here, $\ell$ is the length of the cycle.  
\end{definition}

\begin{definition}[Step, $k$]  
A step in the \texttt{TACo} process refers to a single update to the profit matrix $\profitList$, which occurs based on the choice made by one agent during that auction step.
\end{definition}

\begin{algorithm} [hbt!]
\caption{Trading Auction for Consensus (\texttt{TACo})}\label{alg:TACo}
\begin{algorithmic}[1]
\Require $n$: number of agents, $m$: number of choices, $\epsilon$: termination tolerance
\Require $\costList \in \mathbb{R}^{n \times m}$, $d_0 \in \mathbb{R}_{>0}$, $\decFactor \in (0,1)$

\State Initialize offer matrix $\offerList \in \mathbb{N}^{n \times m}$ to $\mathbf{0}_{n \times m}$
\State Initialize pay matrix $\payList \in \mathbb{N}^{n \times m}$ to $\mathbf{0}_{n \times m}$
\State \textit{selections} $\gets$ empty list $\in \mathbb{N}^n$
\State \textit{isConverged} $\gets$ False
\State $d \gets d_0$ \Comment{Initialize trading unit}
\State \textit{recordedStates} $\gets$ empty set \Comment{For cycle detection}

\While{not \textit{isConverged}}
    \State Choose a playing agent $i$ in a sequential manner.
    \For{each choice $j = 1$ to $m$}
        \State Compute profit for agent $i$ and choice $j$:
        \State $\profitList_{ij} \gets b_i (\offerList_{ij} - \payList_{ij}) - \costList_{ij}$
    \EndFor
    \State $j^\star \gets \underset{j \in [m]}{\arg\max}(\profitList_{ij})$ \Comment{Agent $i$ selects the choice $j^\star$ that maximizes its profit}
    \State Update the matrices based on the choice $j^\star$:
    \State \quad $\payList_{ij^\star} \gets \payList_{ij^\star} + nd$
    \State \quad $\offerList_{ij^\star} \gets \offerList_{ij^\star} + d, \forall i\in [n]$
    \State $\text{\textit{selections}}_i \gets j^\star$ \Comment{Record agent $i$'s choice}
    \If{current $((\offerList-\payList), i)$ exists in \textit{recordedStates}} \Comment{Detect cycle}
        \State Reduce trading unit: $d \gets \decFactor \cdot d$
        \State Clear \textit{recordedStates}
        \For{each agent $i\in[n]$}
            \State $\max_j \profitList_{ij} - \min_j \profitList_{ij} < \epsilon, \, \forall j \in \text{cycle}$ \Comment{Check termination condition}
        \EndFor
        \If{condition is met for all agents}
            \State \textit{isConverged} $\gets$ True
        \EndIf
    \EndIf
    \State Add current $((\offerList-\payList), i)$ to \textit{recordedStates}
\EndWhile

\State \Return $\text{mode}(\textit{selections})$ \Comment{Return the most frequent value in \textit{selections}}

\end{algorithmic}
\end{algorithm}

The following rules define the \texttt{TACo} process:

\begin{enumerate}[\hspace{1pt}1)]
    \item \textbf{Sequential play:} Agents take turns sequentially in a predefined order. For example, if there are three agents, the order of play would be agent 1 $\rightarrow$ agent 2 $\rightarrow$ agent 3, and then the sequence repeats.
    
    \item \textbf{Profit calculation:} During agent $i$'s turn, they select the choice $j$ that maximizes their profit, given by:
    \begin{equation} \label{eq:profUpdate}
    \profitList_{ij} = b_i (\offerList_{ij} - \payList_{ij}) - \costList_{ij},
    \end{equation}
    where the $i$-th row of each matrix corresponds to the information relevant to agent $i$. 
    
    \item \textbf{Matrix update:} When an agent $ i $ selects a particular option $ j $, the matrices are updated as follows:

\begin{itemize} \setlength{\itemindent}{-0.8em}
    \item The algorithm updates the payment matrix $\payList$ by incrementing the entry $\payList_{ij}$ by $ nd $, where $ n $ is the total number of agents and $ d $ is the trading unit. This reflects that agent $ i $ pays $ n $ units of the secondary asset for selecting option $ j $:
    \begin{equation} \label{eq:payUpdate}
        \payList_{ij} \gets \payList_{ij} + nd.    
    \end{equation}
    \item The algorithm updates the offer matrix $\offerList$ by incrementing the $ j $-th column by $ d $. This ensures that all agents receive an additional unit of the secondary asset in their offers for option $ j $:
    \begin{equation} \label{eq:offerUpdate}
        \offerList_{ij} \gets \offerList_{ij} + d, \quad \forall i \in [n].    
    \end{equation}
\end{itemize}

    \item \label{sec:reductionRule} \textbf{Cycle Detection and Trading Unit Reduction:} The trading unit $d$ is reduced by the decrement factor $\decFactor$ when a cycle is detected. A cycle is identified by checking whether the same pair of profit matrix $\profitList$ and active agent at a specific step has occurred in any previous steps. Tracking $\profitList$ is equivalent to tracking $(\offerList - \payList)$, since $b_i$ and $C_{ij}$ are constants based on \cref{eq:profUpdate}. Agents can independently detect cycles by monitoring changes in the publicly observable matrices $\offerList$ and $\payList$. When a cycle is detected, the trading unit is updated as $d_{r+1} \gets \decFactor d_r$, where $d_r$ denotes the trading unit after $r$ cycle detections, with the initial trading unit defined as $d_0$. Additionally, the current record of profit matrices is cleared to restart the cycle detection process.
    
    \quad This reduction of the trading unit is a common practice in real bargaining scenarios, where participants tend to adjust their bids or pricing in smaller increments over time to facilitate reaching a consensus \cite{kahneman2013prospect}. The relationship governing the trading unit after $r$ cycle detections can be expressed as:
    \begin{equation}\label{eq:dr}
        d_r = d_0 \cdot \decFactor^r.
    \end{equation}
    
    \item \textbf{$\epsilon$-termination:} The auction continues until consensus is reached. The process terminates when the difference in profit between all agents' choice options is within a fixed tolerance $\epsilon$, indicating indifference among the choices in the cycle.
\end{enumerate}

\begin{table}[hbt!]
      \caption{\texttt{TACo} running example}
      \label{tab:locations}
      \centering
      \begin{tabular}{cccccc}\toprule
        \textit{Step} & \textit{Agent, $i$} & $\offerList$ & $\payList$ & $\profitList$ & \textit{Selections} \\ \midrule
        
        1 & 1 & $\begin{bsmallmatrix} 0 & 0 \\ 0 & 0 \end{bsmallmatrix}$ & $\begin{bsmallmatrix} 0 & 0 \\ 0 & 0 \end{bsmallmatrix}$ & $\begin{bsmallmatrix} -10 & -4 \\ -7 & -9 \end{bsmallmatrix}$ & $\begin{bsmallmatrix} \underline{2} & \emptyset \end{bsmallmatrix}$ \\
        
        2 & 2 & $\begin{bsmallmatrix} 0 & 1 \\ 0 & 1 \end{bsmallmatrix}$ & $\begin{bsmallmatrix} 0 & 2 \\ 0 & 0 \end{bsmallmatrix}$ & $\begin{bsmallmatrix} -10 & -4.8 \\ -7 & -7.8 \end{bsmallmatrix}$ & $\begin{bsmallmatrix} 2 & \underline{1} \end{bsmallmatrix}$ \\

        3 & \bf{1} & $\begin{bsmallmatrix} 1 & 1 \\ 1 & 1 \end{bsmallmatrix}$ & $\begin{bsmallmatrix} 0 & 2 \\ 2 & 0 \end{bsmallmatrix}$ & $\begin{bsmallmatrix} \bf{-9.2} & \bf{-4.8} \\ \bf{-8.2} & \bf{-7.8} \end{bsmallmatrix}$ & $\begin{bsmallmatrix} \underline{2} & 1 \end{bsmallmatrix}$ \\

        4 & 2 & $\begin{bsmallmatrix} 1 & 2 \\ 1 & 2 \end{bsmallmatrix}$ & $\begin{bsmallmatrix} 0 & 4 \\ 2 & 0 \end{bsmallmatrix}$ & $\begin{bsmallmatrix} -9.2 & -5.6 \\ -8.2 & -6.6 \end{bsmallmatrix}$ & $\begin{bsmallmatrix} 2 & \underline{2} \end{bsmallmatrix}$ \\

        5 & \bf{1} & $\begin{bsmallmatrix} 1 & 3 \\ 1 & 3 \end{bsmallmatrix}$ & $\begin{bsmallmatrix} 0 & 4 \\ 2 & 2 \end{bsmallmatrix}$ & $\begin{bsmallmatrix} \bf{-9.2} & \bf{-4.8} \\ \bf{-8.2} & \bf{-7.8} \end{bsmallmatrix}$ & $\begin{bsmallmatrix} \underline{2} & 2 \end{bsmallmatrix}$ \\
        
        \bottomrule
      \end{tabular}
\end{table}
\begin{example}[\texttt{TACo} running example] \label{example:2}
    Consider a case with two agents ($n=2$) and two choices ($m=2$). The initial cost matrix is given by $\,\costList=\begin{bsmallmatrix} 10 & 4 \\ 7 & 9 \end{bsmallmatrix}$, and the agents' private valuations for the traded asset are $\valuation=\begin{bsmallmatrix} 0.8 & 1.2 \end{bsmallmatrix}$.
    At each step, agents update their selections based on their calculated profit using the \texttt{TACo} rules. \Cref{tab:locations} shows how the auction progresses over time.
    In this example, Agent 1 initially selects option 2 because it provides the highest profit. In step 2, Agent 2 selects option 1, which maximizes its own profit. At each step, the agent currently playing updates its selection, and this updated selection is \underline{underlined} in the table. The \textbf{bold} values in the profit matrix $\profitList$ highlight that the same profit-value and agent pair reappear at steps 3 and 5, signaling the occurrence of a cycle. This cycle is associated with selection 2, where all agents converge to the same option. The algorithm detects that the $\epsilon$-termination criterion is satisfied because the profit differences for all agents are zero, which is smaller than any positive $\epsilon$. Consequently, the algorithm terminates at step 5. 
\end{example}

\subsection{Properties of \texttt{TACo}}
There are several properties of the \texttt{TACo} procedure (\Cref{alg:TACo}).
\subsubsection{Procedural rationality}
\texttt{TACo} incentivizes agents to act in their self-interest by enabling them to trade and negotiate based on their individual valuations of the available options. As demonstrated in \Cref{example:2}, each agent selects the option that maximizes its profit, naturally driving the auction process toward a consensus.

\subsubsection{Privacy preservation}
One of \texttt{TACo}’s core strengths is its ability to preserve the privacy of each agent’s valuations. The trading mechanism functions without requiring agents to disclose their private valuations for the available options ($\costList$) or trading assets ($\valuation$).

\subsubsection{Independence from direct communication}
\texttt{TACo} operates under a minimal communication framework, where agents only need to broadcast their choices. This eliminates the need for direct, 1-on-1 communication, making the algorithm straightforward to implement in practical systems.

\subsubsection{Termination guarantee}
\texttt{TACo} is guaranteed to terminate with an agreement within a bounded number of rounds. The proof of termination, as well as an explicit upper bound on the number of rounds needed to terminate, is provided in the following section.