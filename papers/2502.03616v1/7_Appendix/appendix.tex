\begin{customproof}[Proof of \Cref{theorem:lemma1}]
 Over every $n$ steps:
1. During its turn, agent $i$ updates $\payList$ according to \cref{eq:payUpdate}, resulting in a payment of $d(n-1)b_i$ to other agents.
2. During the $n-1$ turns of other agents, agent $i$ receives $db_i$ from each, by updating the offer matrix $\offerList$ according to \cref{eq:offerUpdate}.

Thus, the total amount paid and received for agent $i$ balances out over $n$ steps, keeping the row sum $S_i^k$ constant every $n$ steps. The maximum fluctuation in $S_i^k$ occurs when agent $i$ makes its payment (reducing the row sum by $d(n-1)b_i$), and the row sum is restored during subsequent steps when agent $i$ receives offers. 
\end{customproof}

\begin{customproof}[Proof of \Cref{theorem:lemma2}]
By \Cref{theorem:lemma1}, the row sum $S_i^k$ of the $i$-th row of the profit value matrix is constant every $n$ steps, and the average profit per choice at an arbitrary step $k$ is $\frac{S_i^k}{m}$. In addition, only choices with a profit above this average can be selected by agent $i$, because the choices agent $i$ selects must maximize profit. Moreover, once a choice is selected, its profit decreases by $(n-1)d b_i$. Aggregating all this information, 
 the profit value $\profitList_{ij}$ of choice $j$ for agent $i$ cannot \textit{voluntarily} drop below:
\begin{equation}\label{eq:minprofit}
    \textstyle{\frac{\smin}{m}} - d(n-1)b_i.
\end{equation}
On the other hand, there might exist choices for agent $i$ with a profit value below \eqref{eq:minprofit} due to the way they were initialized. Combining this fact with the \eqref{eq:minprofit}, it follows that 
\begin{equation}
\profitList_{ij} \geq \min\left(\min(\profitList_{i,:}^{0}),\ \textstyle{\frac{\smin}{m}} - d(n-1)b_i\right).
\end{equation}
\end{customproof}

\begin{customproof}[Proof of \Cref{theorem:theorem1}]
By \Cref{theorem:lemma2} and \Cref{theorem:corollary1}, each row of the profit matrix $\profitList$ is bounded above and below, with bounds denoted by $U$ and $L$, respectively. According to Algorithm \ref{alg:TACo}, the profits can vary only in discrete steps of size $d b_i$, where $d > 0$ is the trading unit and $b_i > 0$ is the agent-specific weight. Thus, each element $\profitList_{ij}$ of $\mathcal{J}$ can assume only a finite number of values, given by:
\begin{equation}\label{eq:elementNum}
    \textstyle{\left\lfloor \frac{U - L}{d b_i} \right\rfloor}.
\end{equation}
Since the total number of configurations of $\profitList$ is finite, this means that the set $\mathcal{S}$ of all possible tuples $(i, \profitList)$ is finite as well. As the \texttt{TACo} algorithm iterates, it progresses through elements of this finite set $\mathcal{S}$. By the pigeonhole principle, the \texttt{TACo} algorithm must eventually revisit a previously encountered tuple $(i, \profitList)$ and hence fall into a cycle.
\end{customproof}

\begin{customproof}[Proof of \Cref{theorem:lemma3}]
For a cycle to form, the profit matrix $\mathcal{J}$ must return to a previously encountered state. By \eqref{eq:profUpdate}, this requires that the total amount an agent pays for a specific choice exactly balances the total offers received from other agents for that choice. Mathematically, this condition can be expressed as:
\begin{equation}\label{eq:dPdO}
\payList^{t+\ell} - \payList^t = \offerList^{t+\ell} - \offerList^t.
\end{equation}
From \eqref{eq:choiceCountToPay}-\eqref{eq:choiceCountToOffer}, \eqref{eq:dPdO} is equivalent to
\begin{equation}
nd \cdot \Delta^\cycle = d \cdot H_n \Delta^\cycle,
\end{equation}
from which we obtain
\begin{equation}\label{eq:eig}
(n \textbf{I}_n - H_n) \Delta^\cycle = 0,
\end{equation}
where $\textbf{I}_n$ is the $n \times n$ identity matrix.

Note that in \eqref{eq:eig}, the matrix $(n \textbf{I}_n - H_n)$ is the Laplacian matrix of a fully connected graph with $n$ nodes. As shown in \cite[p.8]{lewis2013cooperative}, for such a Laplacian one must have $\det(n \textbf{I}_n - H_n) = 0$, which indicates the existence of nontrivial solutions to \eqref{eq:eig}. Transforming $(n \textbf{I}_n - H_n)$ into row echelon form using Gaussian elimination yields:
\begin{equation}
n \cdot
\underbrace{
\begin{bmatrix}
    \textbf{I}_{n-1} & -\textbf{1}_{n-1} \\
    \textbf{0}_{n-1}^\top & 0
\end{bmatrix}}_{A} \Delta^\cycle = 0.
\end{equation}
From this structure, the first row of $A$ implies that the first and last elements of each column of $ \Delta^\cycle $ are identical. The second row of $A$ implies that the second and last elements of each column of $ \Delta^\cycle $ are also identical. By induction, all elements within each column of $ \Delta^\cycle $ are identical. Therefore,  $ \Delta^\cycle $ must have the form:
\begin{equation}
\Delta^\cycle = \begin{bsmallmatrix}
    c_1 & c_2 & \ldots & c_m \\ 
    \vdots & \vdots & \ddots & \vdots \\ 
    c_1 & c_2 & \ldots & c_m 
\end{bsmallmatrix},
\end{equation}
where $ c_j \in \mathbb{Z}_{\geq 0} $ for all $ j \in [m] $. This structure implies that, for a cycle to form, each agent must select each choice the same number of times as every other agent within the cycle. This completes the proof.
\end{customproof}

\begin{customproof}[Proof of \Cref{theorem:theorem2}]
From Theorem \ref{theorem:theorem1}, \texttt{TACo} must eventually fall into a cycle $\sigma$. 
Let the number of active choices within that cycle be $p = |A^\cycle|$, and define $C_i$ as the sum of the profits for row $i$ corresponding to the inactive choices, i.e., the choices not in the cycle $\sigma$. Within $\cycle$, the sum of profits for the active choices in row $i$ is:
\begin{equation}
S_i^k - C_i,
\end{equation}
where $C_i$ remains constant during the cycle. Following the same logic as in \Cref{theorem:lemma2}, the lower bound for the profits of the active choices in $\cycle$ becomes:
\begin{equation}
\profitList_{ij} \geq \min\left(\min(\profitList_{i,:}^{\text{0}}), \textstyle{\frac{\smin - C_i}{p}} - d(n-1)b_i\right), \quad \forall j \in A^\cycle.
\end{equation}

From \Cref{theorem:lemma3}, each active choice $j \in A^\cycle$ is selected at least once by each agent. For such a choice to be selected, it must,  at the time of selection, have a profit value greater than the average profit $\frac{\smin - C_i}{p}$ among the active choices in $\cycle$. Accordingly, no choice that has a profit value below $\frac{\smin - C_i}{p}$ may be selected by agent $i$. Therefore, strictly within the active choices in $\sigma$, we obtain the following tighter lower bound on $\mathcal{J}$:
\begin{equation} \label{eq:L_cycle}
\profitList_{ij} \geq \textstyle{\frac{\smin - C_i}{p}} - d(n-1)b_i = L_\cycle^i, \quad \forall j \in A^\cycle.
\end{equation}

Since the row sum of the profit matrix is constant over every $n$ steps, this lower bound ensures the existence of an upper bound for the profits of the active choices. The upper bound, $U_\cycle^i$, occurs when one active choice achieves a profit value above the average, while the remaining $p-1$ active choices are at their lower bound. This situation leads to the following relation:
\begin{equation}
L_\cycle^i (p-1) + U_\cycle^i = \smax - C_i.
\end{equation}
Rearranging for $U_\cycle^i-L_\cycle^i$ yields
\begin{equation}\label{eq:tempdiff}
U_\cycle^i - L_\cycle^i = \smax - C_i  - L_\cycle^i p.
\end{equation}
Substituting $L_\cycle^i$  from \cref{eq:L_cycle} in \eqref{eq:tempdiff},
\begin{align}
U_\cycle^i - L_\cycle^i &= \textstyle{\smax - C_i - p\left(\frac{\smin - C_i}{p} - d(n-1)b_i \right)} \\
&= \smax - C_i - \smin + C_i + p d(n-1)b_i\\
&= \smax - \smin + p d(n-1)b_i.
\end{align}

From \Cref{theorem:lemma1}, the row sum difference $\smax - \smin$ is equal to $d(n-1)b_i$. Substituting this result:
\begin{equation} \label{eq:priceDiff}
U_\cycle^i - L_\cycle^i = d(n-1)b_i + p d(n-1)b_i = (p+1)d(n-1)b_i.
\end{equation}

Since $p \leq m$, where $m$ is the total number of choices, the profit difference is bounded as:
\begin{equation} \label{eq:priceDiffUpper}
U_\cycle^i - L_\cycle^i \leq (m+1)d(n-1)b_i \leq (m+1)d(n-1)b_{max},
\end{equation}
for all $i\in[n]$. This concludes the proof.
\end{customproof}

\begin{customproof}[Proof of \Cref{theorem:theorem3}]
By \Cref{theorem:theorem1}, the \texttt{TACo} process will always fall into a cycle. If the process enters a cycle involving a single choice, all agents consistently select the same choice.  In this state, there is no profit difference, and the $\epsilon$-termination criterion is automatically satisfied. 

For cycles involving multiple choices, from \Cref{theorem:theorem2}, the profit difference between any two choices for agent $i$ within a cycle is bounded as
\begin{equation} \label{eq:profitBound}
\max_{(i,\profitList^k),(i,\profitList^{k'})\in\cycle} |\profitList^k_{i, j} - \profitList^{k'}_{i, j'}| \leq U_\cycle^i - L_\cycle^i \leq (m+1)d(n-1)b_{\max},
\end{equation}
for all $j, j' \in [m]$ and for all $i \in [n]$, where $k$ and $k'$ are steps within the cycle. According to the trading unit reduction rule (\cref{sec:reductionRule} in \Cref{sec:AuctionMechanism}), when a cycle is detected (i.e., when a previously encountered profit matrix $\profitList$ reappears), the trading unit $d$ is reduced by a decrement factor $\decFactor$, where $0 < \decFactor < 1$. Each reduction in $d$ proportionally decreases the profit differences across choices, ensuring convergence toward zero. 

Specifically, as denoted in \Cref{sec:AuctionMechanism}, if $d_r$ is the trading unit after $r$ cycle detections, then
\begin{equation}
d_r = d_0 \cdot \decFactor^r,
\end{equation}
where $d_0$ is the initial trading unit. After $r$ reductions, from \cref{eq:profitBound}, the maximum profit difference for any agent is bounded by the value
\begin{equation} \label{eq:maxProfDiff_reduction}
(m+1)d_r(n-1)b_{\max} = (m+1)d_0(n-1)b_{\max} \cdot \decFactor^r.
\end{equation}
Since $\decFactor < 1$, and $m$, $n$, and $b_{\max}$ are constants, the term $(m+1)d_0(n-1)b_{\max} \cdot \decFactor^r$ approaches zero as $r \to \infty$. Hence, for any $\epsilon > 0$, there exists a finite $R$ such that for all $r \geq R$: 
\begin{equation} \label{eq:maxProfDiff_epsilon}
\max_{(i,\profitList^k),(i,\profitList^{k'})\in\cycle} |\profitList^k_{i, j} - \profitList^{k'}_{i, j'}| \leq \epsilon, \quad \forall i \in [n], \forall j, k \in A^\cycle.
\end{equation}
At this point, the $\epsilon$-termination criterion is satisfied, and the \texttt{TACo} process terminates. 
\end{customproof}

\begin{customproof}[Proof of \Cref{theorem:theorem4}]
    The profit difference among the active choices for agent $i$ is given in \Cref{eq:priceDiff}. Substituting this into \Cref{eq:elementNum}, we infer that  
    the number of possible values that $\profitList_{i,j}$ can take in a cycle is at most $(p+1)(n-1)$, where $p$ is the number of active choices.
    
    Since $p \leq m$ and the number of elements in $\profitList$ is $nm$, the maximum number of distinct $\profitList$ configurations is $\big((m+1)(n-1)\big)^{nm}$. Consequently, the maximum number of unique agent-profit matrix tuples $(i, \profitList)$ in a cycle is $n \cdot \big((m+1)(n-1)\big)^{nm}$. This provides an upper bound on the number of steps required to form a cycle, as established in \Cref{theorem:theorem1}.

    The upper bound on the number of cycles $r$ required to satisfy the termination condition is derived from \Cref{eq:maxProfDiff_reduction} and \Cref{eq:maxProfDiff_epsilon}:
    \begin{subequations}
        \begin{align}
            &(m+1)d_0(n-1)b_{\max} \cdot \decFactor^r \leq \epsilon, \\
            &\textstyle{\implies r = \left\lceil \log_{\decFactor} \left(\frac{\epsilon}{(m+1)d_0(n-1)b_{\max}}\right) \right\rceil,}
        \end{align}
    \end{subequations}
    with decrement factor $\decFactor$ and initial trading unit $d_0$.

    Combining the number of cycles needed and the maximum steps per cycle, we obtain the upper bound on the number of steps needed for \texttt{TACo} to terminate:
    \begin{equation}
    \textstyle{
        \left\lceil \log_{\decFactor} \left(\frac{\epsilon}{(m+1)d_0(n-1)b_{\max}} \right) \right\rceil \cdot n \cdot \big((m+1)(n-1)\big)^{nm}.}
    \end{equation}
\end{customproof}

% \filippos{Kindly, an observation I have made and tried to correct throughout the paper in multiple spots: You should write the paper not in a way that reads easily to you, but in a way that reads easily to someone reading it for the first time. What does ``the worst-case bound for \texttt{TACo} termination" even mean? Yes, I understand you mean ``the upper bound on the number of steps needed for \texttt{TACo} to terminate", but other readers will not. }