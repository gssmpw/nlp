\section{Numerical Experiments}
\subsection{Waypoint merging game}

 We model a common noncooperative scenario in air traffic management, where $n$ aircraft converging on a waypoint must adjust their speeds to avoid collisions, as shown in \Cref{fig:scenarioDescription}. Each aircraft $i$ in this \emph{waypoint merging game} has an estimated time of arrival (ETA) $e_i$ and seeks to minimize deviations from it while ensuring safe separation. We introduce carbon emission credits as a tradable secondary asset.

Each aircraft (agent) $i$ selects an action $x_i$, representing its adjustment to the ETA. The adjusted arrival time for agent $i$ is then $u_i = e_i + x_i$. The cost structure for each agent is:
\begin{equation}\label{eq:scenarioCost}
\begin{array}{ll}
     \underset{x_i}{\mbox{minimize}} & c_i(x_i) = k_ix_i^2 \\
     \mbox{subject to} & |u_i - u_j| \geq \separationDist, \enskip \forall j \in [n] \setminus i,
\end{array}
\end{equation}
where $k_i > 0$ represents the urgency factor, indicating sensitivity to arrival time adjustments, and $\separationDist \geq 0$ is the required separation distance.

This game \Cref{eq:scenarioCost} admits multiple generalized Nash equilibria. The absolute value in the constraint introduces two conditions depending on the arrival sequence. Specifically, the sign of the terms inside the absolute value changes based on the order of $ u_i $ and $ u_j $, leading to two inverted inequalities. This leads to $n!$ distinct equilibria. Thus, coordination is essential for safe and efficient merging.

Another key challenge is the lack of direct 1-on-1 communication between aircraft. Instead, aircraft use the \acrfull{adsb} system, which broadcasts publicly available information such as positions, urgency factors ($k_i$), and ETAs ($e_i$). However, \acrshort{adsb} does not support direct negotiation, making consensus on an equilibrium difficult despite agents being able to compute all equilibria.

In our scenario, each aircraft privately determines its valuation for emission credits and employs the \texttt{TACo} algorithm to iteratively negotiate arrival adjustments. By trading emission credits and broadcasting preferences via \acrshort{adsb}, aircraft reach a consensus while acting in their own self-interest, despite the communication constraints imposed by \acrshort{adsb}.

\subsection{Baseline algorithms}
We evaluated the effectiveness of the \texttt{TACo} algorithm by comparing it with widely used negotiation approaches. While these baseline methods lack key features of \texttt{TACo}, they serve as intuitive benchmarks for multi-agent coordination. The baseline algorithms considered in this experiment are:

\begin{itemize} \setlength{\itemindent}{-0.8em} 
    \item \textbf{Voting}: Each agent votes for their preferred options, and the final decision is determined by majority rule \cite{approvalVoting}. In the event of a tie, the winner is chosen randomly.
    \item \textbf{Random dictator}: A random agent is selected, and all other agents must conform to the preferences of this chosen agent.
    \item \textbf{Utilitarian}: A central coordinator selects the option that minimizes the total costs across all agents, requiring compliance from every agent.
    \item \textbf{Egalitarian}: A central coordinator selects the option that maximizes fairness among agents, and all agents are required to comply with the decision.
\end{itemize}

The voting mechanism offers a simple, decentralized approach but can lead outcomes which are not unilaterally rational, as agents must accept majority decisions even when misaligned with their preferences. Moreover, the utilitarian and egalitarian methods optimize system-level objectives but impose centralized decisions, making them unsuitable for noncooperative and decentralized settings. Lastly, the random dictator mechanism also relies on centralized selection but ensures fairness over time in repeated interactions by rotating decision-making power among agents.

\subsection{Evaluation method}

We evaluated the performance of \texttt{TACo} across three key metrics: optimality gap, fairness, and convergence stpng.

\begin{itemize} \setlength{\itemindent}{-0.8em}
    \item \textbf{Optimality gap}: The optimality gap quantifies the deviation of the total cost incurred by the algorithm from the optimal total cost achieved by a utilitarian solution. It is defined as:
    \begin{equation}\label{eq:OG}
    \textstyle{
        \mathrm{OG}(\cSet, \cSet^\star) = \frac{\sum_{i \in [n]} c_i}{\sum_{i \in [n]} c_i^\star} - 1,
        }
    \end{equation}
    where $c_i^\star \in \cSet^\star$ is the cost of agent $i$ from the utilitarian solution, and $c_i$ is the cost of agent $i$ under the evaluated algorithm.

    \item \textbf{Gini index}: The Gini index \cite{FairnessGuide} measures fairness among agents. A smaller Gini index indicates more equitable outcomes. It is computed as:
    \begin{equation}\label{eq:GI}
    \textstyle{
        \mathrm{GI}(\textbf{c}) = \frac{1}{2n\sum_{i \in [n]} c_i} \sum_{i,j \in [n]} |c_i - c_j|.
        }
    \end{equation}

    \item \textbf{Convergence stpng}: The convergence stpng metric measures the algorithm's efficiency by counting the number of stpng required for termination.

\end{itemize}

To assess the effectiveness of \texttt{TACo}, we conducted a series of Monte Carlo simulations with 1000 trials to ensure statistical significance. Each trial incorporated randomized initial conditions, including diverse weight parameters ($k_i$) and varying initial estimated times of arrival. Agents' private valuations over secondary assets ($b_i$) were randomly assigned within a predefined range to reflect diverse private information. The decrement factor $\decFactor$ was set to 0.9, with the number of aircraft $n$ set to 4 and the number of choices $m$ set to 24.

The simulations were implemented in the Julia programming language. We used the \texttt{ParametricMCPs.jl} package \cite{juliaParametric} and the \texttt{PATH} solver for mixed complementarity problems \cite{juliaPATH} to compute the Nash equilibria of the game \Cref{eq:scenarioCost} prior to running \texttt{TACo}.
% \footnote{Code available at https://github.com/iamjaehan/TACo}

\subsection{Results}
\begin{figure}
    \centering
    \begin{subfigure}[b]{0.48\textwidth}
        \includegraphics[width=\textwidth, trim=160 0 7 0, clip]{Figure/Exp1_optgap.png}
        \caption{Optimality gap} \label{fig:exp1_optgap}
    \end{subfigure}
    \begin{subfigure}[b]{0.48\textwidth}
        \includegraphics[width=\textwidth]{Figure/Exp3_gini.png}
        \caption{Gini index} \label{fig:exp2_gini}
    \end{subfigure}
    \caption{Experimental results comparing the performance of the \texttt{TACo} algorithm with baseline methods (Voting, Utilitarian, Egalitarian, and Random dictator) across three evaluation metrics: optimality gap, Gini index, and termination stpng. \texttt{TACo} demonstrates the smallest median values for the optimality gap and Gini index, highlighting its effectiveness in achieving near-optimal and fair outcomes.}
    \label{fig:result}
\end{figure}

\subsubsection{Optimality and fairness}
As shown in \Cref{fig:exp1_optgap}, the \texttt{TACo} algorithm exhibited a consistently lower optimality gap compared to the baseline algorithms, except for Utilitarian, which achieves a zero optimality gap by definition. \texttt{TACo} showed minimal deviation from the optimal solution, with a median and third quartile value of 0\% and a maximum value of 20.3\%. This confirms that the \texttt{TACo} effectively minimizes deviations from the socially optimal solution.

\Cref{fig:exp2_gini} shows that \texttt{TACo} achieves a low Gini index, with a median of 0.181, demonstrating that the \texttt{TACo} improves fairness by redistributing costs among agents. Notably, trading-based coordination enhances fairness beyond merely selecting the most balanced equilibrium. However, in cases where the cost differences between choices were initially small, the trading process amplified disparities, resulting in occasional outliers. This suggests that the choice of trading unit $d$ should account for the relative magnitude of costs in the system.

\subsubsection{Termination proof validation}
In our experiments, \texttt{TACo} reached consensus in a median of 53 stpng, with a maximum of 380 stpng. Given an \acrshort{adsb} information exchange rate of 1 Hz, this corresponds to a median negotiation time of 53 seconds and a maximum of 6 minutes and 20 seconds. While \texttt{TACo} is iterative by design, these results suggest that its convergence remains within operationally feasible limits.

\begin{figure}
    \centering
    \includegraphics[width=1.0\linewidth]{Figure/PriceDiff_merge.png}
    \caption{Observed profit differences between choices in cycles, compared with theoretical upper bounds. The solid blue line represents the maximum upper bound derived from \Cref{eq:profDiff_Theorem2} with $m=24$, the maximum number of active choices in this scenario. Since most cycles involved only two choices, a dotted line provides an additional upper bound for $m=2$ for comparison. The results confirm that profit differences consistently remained below the theoretical bounds.}
    \label{fig:price_diff}
\end{figure}

As shown in \Cref{fig:price_diff}, the maximum observed profit difference between choices remained below the theoretical upper bound from \Cref{eq:profDiff_Theorem2}, validating that profit differences decrease as predicted. This empirical confirmation supports the theoretical termination guarantee of \texttt{TACo}.


% \filippos{To be blunt, the reporting of results in 1)-4) did not look particularly meaningful to me. In some cases TACo is better, in some others it is not. So what is the point of the simulations? Have we outperformed baselines or not? More importantly, do we even need a comparison with these baselines, or are we comparing apples with oranges since the baselines violate the properties of Section IV.C? Upon reflection, these baseline comparisons might actually be distracting.}

\subsection{\texttt{TACo} variations for faster convergence}

To accelerate the convergence speed of the \texttt{TACo} algorithm, we explored two alternative strategies: (i) adjusting the decrement factor and (ii) interrupting the algorithm to enforce consensus earlier. These experiments highlight how convergence speed can be improved, albeit with potential trade-offs in rationality, optimality, and fairness.

\begin{figure}
    \centering
    \begin{subfigure}[b]{0.47\textwidth}
        \includegraphics[width=\textwidth, trim=0 0 -16 0, clip]{Figure/Exp4_2_interruption.png}
        \caption{Termination stpng} \label{fig:exp4_1_interrupt}
    \end{subfigure}
    \begin{subfigure}[b]{0.47\textwidth}
        \includegraphics[width=\textwidth]{Figure/Exp4_1_interruption.png}
        \caption{Optimality gap and Gini index} \label{fig:exp4_2_interrupt}
    \end{subfigure}
    \caption{Effect of the decrement factor on \texttt{TACo} performance.  
(a) The termination stpng increase as the decrement factor $\decFactor$ approaches 1, with a maximum exceeding 1500 stpng. For $\decFactor \leq 0.9$, there is no notable reduction in convergence stpng.
(b) The Gini index and the optimality gap remain largely unaffected by changes in $\decFactor$, demonstrating that \texttt{TACo} maintains fairness and optimality regardless of the decrement factor.}
    \label{fig:exp4_tradingStep}
\end{figure}

\begin{figure}
    \centering
    \begin{subfigure}[b]{0.47\textwidth}
        \includegraphics[width=\textwidth, trim=0 0 0 0, clip]{Figure/Exp5_2_interruption.png}
        \caption{Termination stpng} \label{fig:exp5_1_interrupt}
    \end{subfigure}
    \begin{subfigure}[b]{0.47\textwidth}
        \includegraphics[width=\textwidth]{Figure/Exp5_1_interruption.png}
        \caption{Optimality gap and Gini index} \label{fig:exp5_2_interrupt}
    \end{subfigure}
    \caption{Effect of interruptions on \texttt{TACo} performance.  
(a) Interrupting the algorithm earlier reduces termination stpng effectively.
(b) However, earlier interruptions increase the optimality gap and Gini index. This demonstrates the trade-off between faster convergence and reduced fairness and optimality.}
    \label{fig:exp5_interrupt}
\end{figure}

\subsubsection{Effect of decrement factor on convergence}
We investigated how varying the decrement factor $\decFactor$ impacts both convergence stpng and performance. In this experiment, we tested $\decFactor$ values ranging from 0.3 to 0.99. As shown in \Cref{fig:exp4_1_interrupt}, the highest number of stpng occurred in the $\decFactor=0.99$ case, with a median of 57 and a maximum of 1680 stpng. However, for $\decFactor$ values below 0.9, there was no significant improvement in the convergence rate, even for the $\decFactor=0.3$ case.

A similar trend was observed in the Gini index and optimality gap, as shown in \Cref{fig:exp4_2_interrupt}. Just as there was no significant improvement in the convergence rate for smaller $\decFactor$ values, there was also no notable impact on optimality or fairness. Thus, as long as $\decFactor$ does not exceed 0.9, there is no significant impact on convergence stpng, optimality, or fairness.

\subsubsection{Effect of algorithm interruption on convergence}
In this experiment, we interrupted the algorithm before reaching natural consensus to simulate scenarios where agents must reach a decision within a limited time. Agents were forced to select the most commonly chosen option up to the interrupted step, sacrificing individual rationality for the sake of faster decision-making.

\Cref{fig:exp5_interrupt} shows the results of these interruptions, comparing termination stpng and performance across varying levels of interruption. The results demonstrated that interruptions significantly reduced convergence stpng, as shown in \Cref{fig:exp5_1_interrupt}. However, as expected, forced consensus increased both the optimality gap and Gini index, as seen in \Cref{fig:exp5_2_interrupt}.

In the most extreme case, where the algorithm was interrupted at the fifth step—allowing each agent only a single chance to express their preference—the median optimality gap increased from 0\% to 182.7\%, and the median Gini index rose from 0.181 to 0.449. This experiment demonstrates that \texttt{TACo} can be adapted to systems requiring faster convergence by allowing interruptions, albeit at the cost of reduced optimality, fairness, and rationality. Such adaptations may be appropriate in high-stakes environments where timely decision-making is more critical than achieving perfect rationality.
