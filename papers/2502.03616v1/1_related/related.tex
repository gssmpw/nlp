\section{Related Work}
Existing multi-agent consensus methods employ leader-follower frameworks \cite{consensus_leader_follow1,consensus_leader_follow2}, finite-time consensus algorithms \cite{consensus_finite1,consensus_finite2}, robust control techniques \cite{consensus_robust1}, auction mechanisms \cite{consensus_auction}, and game-theoretic approaches for noncooperative agents \cite{corrEq}. These have been applied in UAV coordination \cite{consensus_formation1, consensus_formation2, consensus_formation3}, traffic flow optimization \cite{corrEq}, and wireless sensor networks \cite{consensus_wireless1}. However, most rely on centralized leaders or high-bandwidth communication, limiting their applicability in fully decentralized, noncooperative settings \cite{consensus_review}.

A similar trend appears in addressing the equilibrium selection problem, where agents must coordinate among multiple conflicting outcomes. Many approaches rely on a centralized coordinator, requiring agents to share private information or follow the coordinator's directives \cite{Coop_1_monotone, Coop_2_potential, Coop_3_hetero, Coop_4_comm_priv, Coop_5_comm_priv, Coop_6, Coop_7_comm_priv, Coop_8, Coop_9}. Others assume cooperativeness or shared beliefs, which bypass individual preferences \cite{nonCoop_4_linear, nonCoop_5_linear, nonCoop_2_shared, nonCoop_1_coord}. While effective in achieving a desirable equilibrium, these approaches are less suited for fully noncooperative systems due to their reliance on direct communication and lack of procedural \emph{rationality}, where agents have incentives to participate based on self-interest.

Several works tackle equilibrium selection in noncooperative systems \cite{nonCoop_1_coord, nonCoop_2_shared, nonCoop_3_enforce, nonCoop_4_linear, nonCoop_5_linear, findNash_1, findNash_2}. These include enforcing unique equilibria by modifying utility functions \cite{nonCoop_3_enforce} or employing structured processes like the Linear Tracing Procedure, which relies on shared initial beliefs \cite{nonCoop_4_linear, nonCoop_5_linear}. While ensuring rationality during equilibrium seeking, these methods assume a coordinator or shared preferences during equilibrium selection.

Negotiation algorithms provide another approach to decentralized consensus among noncooperative agents \cite{nego_shoham}. Common methods include voting, bargaining, and auction-based mechanisms. Voting is widely used but often fails to ensure \textit{rationality}, meaning outcomes may not be acceptable to all agents \cite{nego_shoham}. Bargaining relies on direct communication and is often problem-specific \cite{nego_meyerson_bargain_auction, nego_noncoopBargain, nego_tournament}, limiting its applicability in general decentralized negotiation environments. In contrast, auctions align agents' incentives with self-interest while preserving \textit{privacy} (i.e., agents need not disclose sensitive private information) \cite{nego_bertsekas, nego_meyerson_bargain_auction, nego_rational_auction, nego_shoham, nego_bertsekas_2009}. However, auctions primarily address resource allocation rather than single-choice problems like equilibrium selection, highlighting the need for research that bridges this gap by adapting auction-based methods for decentralized, noncooperative equilibrium selection. 
