\section{Coordination Under Conflicting Preferences}
% In multi-agent systems, agents often face conflicting preferences when selecting among available options. These conflicts occur from differences in agents' strategic priorities. Resolving such conflicts is essential to achieving system-wide coordination, fairness, and efficiency. A key challenge emerges in noncooperative systems, where agents lack direct communication, cannot share private information, and act solely in their own self-interest. This creates a broad class of problems, which we denote as \emph{conflicting choices}. A notable example of this issue is the \emph{equilibrium selection problem}, which occurs in game-theoretic problems with multiple Nash equilibria. \filippos{Is this paragraph needed? We repeated this info enough in the preceding paragraphs.}

% \filippos{Repeating again} A prominent instance of conflicting choices in multi-agent systems is the \emph{equilibrium selection problem}, which occurs in noncooperative game settings where agents must select one equilibrium from a set of multiple possible Nash equilibria. Before explaining the problem, we first introduce the notation and framework used to model the interactions between agents.

A prominent instance of conflicting choices in multi-agent systems is the \emph{equilibrium selection problem}. Before explaining the problem, we first introduce the notation and framework used to model the interactions between agents.

\subsection{Equilibrium selection problem}

We model the interaction between agents as a noncooperative game with $n\in \intSet_{\ge0} $ agents, where each agent $i \in [n] \equiv \{1,2,\ldots, n\}$ selects an action $x_i$ from an action set $\actionPool_i$. The collective actions of all agents are represented by the joint action profile $\xSet = \{x_1, x_2, \ldots, x_n\}$. Each agent $i$ wishes to minimize a cost function $c_i(x_i, \xSet_{\neg i})$ which depends on its own action $x_i$ and those of other agents, where $\xSet_{\neg i}=\xSet\setminus x_i$. The collection of costs of all agents is denoted as $\cSet=\{c_1,c_2,\ldots,c_n\}$.

Generalized Nash equilibrium is a well-known concept that describes possible outcomes in noncooperative settings. It is a point $\mathbf{x}^\star$ where no agent has an incentive to unilaterally change their action, while adhering to constraints that are dependent on other agents' strategies.

\begin{definition}[Generalized Nash equilibrium] \label{def:GeneralizedNashEquilibrium}
A set of strategies $\xSet^\star = \{x_1^\star, x_2^\star, \ldots, x_n^\star\}$ is a generalized Nash equilibrium if the following holds for each agent $i\in[n]$,
\begin{equation}
    c_i(x_i^\star, \xSet_{\neg i}^\star) \leq c_i(x_i, \xSet_{\neg i}^\star),~\forall x_i \in \actionPool_i(\xSet_{\neg i}^\star).
\end{equation}
\end{definition}

At a generalized Nash equilibrium, each agent’s strategy is optimal given the strategies of all other agents and satisfies the feasibility constraints imposed by the other agents' actions.

There exist multiple Nash equilibria in many games, meaning that several action profiles $\xSet^\star$ may satisfy the Nash equilibrium condition. This multiplicity arises when different combinations of actions lead to  optimal outcomes for each agent. The presence of multiple equilibria introduces a coordination challenge known as the \emph{equilibrium selection problem}, where agents must agree on one equilibrium from the set of possible options to ensure efficient and safe interaction.

\begin{example}[2-agent quadratic game]

Consider two agents, $i$ and $j$, aiming to minimize a shared cost while satisfying a constraint. Each agent's action $x_i \in \mathbb{R}$ is chosen to solve:
\begin{equation} \label{eq:exampleOpt}
    \begin{array}{ll}
        \underset{x_i}{\text{minimize}} & c_i(x_i, x_j) = -(x_i + x_j - 1)^2, \\
        \text{subject to} & x_i^2 + x_j^2 \leq 1.
    \end{array}
\end{equation}

A Nash equilibrium $(x_1^\star, x_2^\star)$ satisfies the \acrfull{kkt} conditions. The Lagrangian is:
\begin{equation}
    \mathcal{L}(x_1, x_2, \mu) = - (x_1 + x_2 - 1)^2 + \mu (x_1^2 + x_2^2 - 1),
\end{equation}
where $\mu \geq 0$ is the Lagrange multiplier. The first-order conditions are:
\begin{subequations} \label{eq:KKTconditions}
\begin{align}
    &\textstyle{\frac{\partial \mathcal{L}}{\partial x_1} = -2(x_1 + x_2 - 1) + 2\mu x_1 = 0,} \\
    &\textstyle{\frac{\partial \mathcal{L}}{\partial x_2} = -2(x_1 + x_2 - 1) + 2\mu x_2 = 0,} \\
    &\mu (x_1^2 + x_2^2 - 1) = 0, \quad \mu \geq 0, \quad x_1^2 + x_2^2 \leq 1.
\end{align}
\end{subequations}

From the first two equations, we obtain:
\begin{equation}
    \textstyle{\mu (x_1 - x_2) = 0, \quad x_1 + x_2 = \frac{2}{2 - \mu}.}
\end{equation}
This yields two cases: i) $\mu = 0$: The constraint is inactive, The feasible solutions satisfying $x_1^2 + x_2^2 \leq 1$ are all points on the line segment $x_1+x_2=1$ where $0 \leq x_1, x_2 \leq 1$.  
ii) $\mu > 0$: The constraint is active, so $x_1^2 + x_2^2 = 1$. Since $\mu (x_1 - x_2) = 0$, we must have $x_1 = x_2$. Solving $2x_1^2 = 1$ gives $x_1 = x_2 = \pm \frac{1}{\sqrt{2}}$.

The critical points are:
\begin{equation}
    \textstyle{\{(x,1-x) \mid 0 \leq x \leq 1\} \cup \left\{\left(\frac{1}{\sqrt{2}}, \frac{1}{\sqrt{2}}\right), \left(-\frac{1}{\sqrt{2}}, -\frac{1}{\sqrt{2}}\right)\right\}.}
\end{equation}
However, $\{(x,1-x) \mid 0 \leq x \leq 1\}$ are local maxima, as an agent can unilaterally decrease their cost by deviating. Thus, $\left(\frac{1}{\sqrt{2}}, \frac{1}{\sqrt{2}}\right)$ and $\left(-\frac{1}{\sqrt{2}}, -\frac{1}{\sqrt{2}}\right)$ are Nash equilibria, demonstrating the existence of multiple equilibria in this game.

\end{example}

\textbf{}


In decentralized systems, achieving consensus on which equilibrium to select is nontrivial, especially in noncooperative settings where agents have different preferences over equilibria. Without coordination, there is a risk that agents may settle on different equilibria, resulting in inefficiencies or even safety risks in the system. Furthermore, agents may not be able to communicate directly or share their preferences, further complicating the equilibrium selection problem.
