% This must be in the first 5 lines to tell arXiv to use pdfLaTeX, which is strongly recommended.
\pdfoutput=1
% In particular, the hyperref package requires pdfLaTeX in order to break URLs across lines.

\documentclass[11pt]{article}

% Change "review" to "final" to generate the final (sometimes called camera-ready) version.
% Change to "preprint" to generate a non-anonymous version with page numbers.
\usepackage[preprint]{acl}

% Standard package includes
\usepackage{times}
\usepackage{latexsym}

% For proper rendering and hyphenation of words containing Latin characters (including in bib files)
\usepackage[T1]{fontenc}
% For Vietnamese characters
% \usepackage[T5]{fontenc}
% See https://www.latex-project.org/help/documentation/encguide.pdf for other character sets

% This assumes your files are encoded as UTF8
\usepackage[utf8]{inputenc}

% This is not strictly necessary, and may be commented out,
% but it will improve the layout of the manuscript,
% and will typically save some space.
\usepackage{microtype}

% This is also not strictly necessary, and may be commented out.
% However, it will improve the aesthetics of text in
% the typewriter font.
\usepackage{inconsolata}

%Including images in your LaTeX document requires adding
%additional package(s)
\usepackage{graphicx}
\usepackage{mdframed}
\usepackage{amsmath}
\usepackage{multirow}
\usepackage{overpic}
\usepackage{bbding}
\usepackage{booktabs} 
\usepackage{subcaption}
\usepackage{float}
\definecolor{myblue}{RGB}{56,94,124}
\definecolor{myred}{RGB}{176,35,24}
\definecolor{mygreen}{RGB}{76,123,49}
\definecolor{textred}{RGB}{176,35,24}

% If the title and author information does not fit in the area allocated, uncomment the following
%
%\setlength\titlebox{<dim>}
%
% and set <dim> to something 5cm or larger.

\title{Mitigating Hallucinations in Multimodal Spatial Relations through Constraint-Aware Prompting}

% Author information can be set in various styles:
% For several authors from the same institution:
% \author{Author 1 \and ... \and Author n \\
%         Address line \\ ... \\ Address line}
% if the names do not fit well on one line use
%         Author 1 \\ {\bf Author 2} \\ ... \\ {\bf Author n} \\
% For authors from different institutions:
% \author{Author 1 \\ Address line \\  ... \\ Address line
%         \And  ... \And
%         Author n \\ Address line \\ ... \\ Address line}
% To start a separate ``row'' of authors use \AND, as in
% \author{Author 1 \\ Address line \\  ... \\ Address line
%         \AND
%         Author 2 \\ Address line \\ ... \\ Address line \And
%         Author 3 \\ Address line \\ ... \\ Address line}

\author{Jiarui Wu \\
  University of Rochester \\
  \texttt{jwu114@u.rochester.edu} \\\And
  Zhuo Liu \\
  University of Rochester \\
  \texttt{zhuo.liu@rochester.edu} \\\And
  Hangfeng He \\
  University of Rochester \\
  \texttt{hangfeng.he@rochester.edu} \\}

%\author{
%  \textbf{First Author\textsuperscript{1}},
%  \textbf{Second Author\textsuperscript{1,2}},
%  \textbf{Third T. Author\textsuperscript{1}},
%  \textbf{Fourth Author\textsuperscript{1}},
%\\
%  \textbf{Fifth Author\textsuperscript{1,2}},
%  \textbf{Sixth Author\textsuperscript{1}},
%  \textbf{Seventh Author\textsuperscript{1}},
%  \textbf{Eighth Author \textsuperscript{1,2,3,4}},
%\\
%  \textbf{Ninth Author\textsuperscript{1}},
%  \textbf{Tenth Author\textsuperscript{1}},
%  \textbf{Eleventh E. Author\textsuperscript{1,2,3,4,5}},
%  \textbf{Twelfth Author\textsuperscript{1}},
%\\
%  \textbf{Thirteenth Author\textsuperscript{3}},
%  \textbf{Fourteenth F. Author\textsuperscript{2,4}},
%  \textbf{Fifteenth Author\textsuperscript{1}},
%  \textbf{Sixteenth Author\textsuperscript{1}},
%\\
%  \textbf{Seventeenth S. Author\textsuperscript{4,5}},
%  \textbf{Eighteenth Author\textsuperscript{3,4}},
%  \textbf{Nineteenth N. Author\textsuperscript{2,5}},
%  \textbf{Twentieth Author\textsuperscript{1}}
%\\
%\\
%  \textsuperscript{1}Affiliation 1,
%  \textsuperscript{2}Affiliation 2,
%  \textsuperscript{3}Affiliation 3,
%  \textsuperscript{4}Affiliation 4,
%  \textsuperscript{5}Affiliation 5
%\\
%  \small{
%    \textbf{Correspondence:} \href{mailto:email@domain}{email@domain}
%  }
%}

\begin{document}
\maketitle
\begin{abstract}

Spatial relation hallucinations pose a persistent challenge in large vision-language models (LVLMs), leading to generate incorrect predictions about object positions and spatial configurations within an image. To address this issue, we propose a constraint-aware prompting framework designed to reduce spatial relation hallucinations. Specifically, we introduce two types of constraints: (1) bidirectional constraint, which ensures consistency in pairwise object relations, and (2) transitivity constraint, which enforces relational dependence across multiple objects. By incorporating these constraints, LVLMs can produce more spatially coherent and consistent outputs. We evaluate our method on three widely-used spatial relation datasets, demonstrating performance improvements over existing approaches.\footnote{Our code is available at \href{https://github.com/jwu114/CAP}{https://github.com/jwu114/CAP}.} Additionally, a systematic analysis of various bidirectional relation analysis choices and transitivity reference selections highlights greater possibilities of our methods in incorporating constraints to mitigate spatial relation hallucinations.


%In this paper, we propose constraint-aware prompting methods to mitigate spatial relation hallucinations in large vision-language models (LVLMs). We introduce three distinct constraints: Bidirectional Constraint, Transitivity Constraint, and Combined Constraint, integrating mainstream reasoning techniques into our approach. Our methods are evaluated using three spatial relation datasets. The results demonstrate that our approach significantly improves the performance of language models in spatial relation tasks compared to the baseline and other evaluated methods. Additionally, we analyze variants of our proposed method and shows that these variants consistently outperform the baseline method; however, their performance may vary depending on the distribution of image data, meaning there is no universally optimal choice among the variants.

\end{abstract}

\begin{figure}[t]
    \centering
    \fbox{%
    \begin{overpic}[width=0.94\linewidth]{figures/demo_v3.png}
        \put(6.7,65.2){\bfseries{
        \begin{minipage}{0.91\linewidth}   
        \linespread{0.8}\fontsize{5}{12}\raggedright\selectfont\sloppy\textcolor{myred}{
            \rotatebox{41}{Left}
        }
        \end{minipage}}}
        \put(63.3,65.2){\bfseries{
        \begin{minipage}{0.91\linewidth}   
        \linespread{0.8}\fontsize{5}{12}\raggedright\selectfont\sloppy\textcolor{mygreen}{
            \rotatebox{41}{Right}
        }
        \end{minipage}}}
        \put(63.7,72){\bfseries{
        \begin{minipage}{0.91\linewidth}   
        \linespread{0.8}\fontsize{5}{12}\raggedright\selectfont\sloppy\textcolor{myblue}{
            \rotatebox{-36}{Right}
        }
        \end{minipage}}}
        \put(60.5,68){\bfseries{
        \begin{minipage}{0.91\linewidth}   
        \linespread{0.8}\fontsize{5}{12}\raggedright\selectfont\sloppy\textcolor{myblue}{
            \rotatebox{90}{Below}
        }
        \end{minipage}}}
        \put(-1,22){\bfseries{
        \begin{minipage}{0.91\linewidth}   \linespread{0.8}\fontsize{7}{12}\raggedright\selectfont\sloppy 
            Q: Is the transparent card holder on the right of the cell phone? 
            \\A: \textcolor{myred}{No, the transparent card holder is on the left of the cell phone. \XSolidBrush}\vspace{12pt}
            Q: \textcolor{myblue}{What is the spatial relation between the transparent card holder and the camera and between the cell phone and the camera?} Based on the answer to the previous question, is the transparent card holder on the right of the cell phone?
            \\A: \textcolor{myblue}{The transparent card holder is to the right of the camera. The cell phone is below the camera.} Based on this,  \textcolor{mygreen}{the transparent card holder is to the right of the cell phone.\qquad\qquad\qquad\qquad\qquad\;\;\; \CheckmarkBold}
        \end{minipage}}}
    \end{overpic}
    }
    \caption{Comparison between the vanilla prompt and the prompt incorporating constraint awareness (transitivity constraint). Constraint-aware content is highlighted in blue, incorrect content in red, and correct content in green. In the right image, the relations highlighted in blue corrects the incorrect relation highlighted in red.}
    \label{fig:demo}
\end{figure}

\section{Introduction}

In recent years, large vision-language models (LVLMs) have been widely adopted for tasks such as image captioning and visual question answering (VQA). While these models have demonstrated remarkable capabilities, hallucination remains a persistent challenge in multimodal systems. Even state-of-the-art models occasionally generate hallucinated responses~\citep{chang2024survey}. In this study, we focus on mitigating the hallucination in multimodal spatial relations, a challenging task that requires the cognition and reasoning ability of LVLMs about objects in the image. 

Existing research has explored various methods to enhance the performance of LVLMs in spatial relations.~\citet{zhao2023enhancing} utilized a small pretrained model to provide spatial information in guiding LVLMs.~\citet{rajabi2023towards} combined an encoder-decoder model with a trained predictor to localize objects and predict spatial relations.~\citet{chen2024spatialvlm} proposed SpatialVLM, trained on a spatial VQA dataset generated using their data generation framework. Additionally,~\citet{meng2024know} introduced ZeroVLM, which leverages a 3D reconstruction model to obtain multi-view images for improved spatial reasoning. The existing proposals mainly focus on training powerful models. While these approaches are effective in enhancing LVLMs' visual spatial relation understanding, they involve high training costs and rely heavily on high-quality training data.

Prompt enhancement is a training-free approach that has been shown to effectively mitigate hallucinations in large language models (LLMs). Numerous prompting techniques (e.g.,~\citealp{NEURIPS2022_9d560961,hu2023chain,kong-etal-2024-better,zheng2024take}) have been developed to improve response quality in reasoning tasks. However, despite their success, few methods have achieved significant improvements in multimodal spatial relation tasks~\citep{sahoo2024systematic, vatsal2024survey}.

In response to these limitations, we propose constraint-aware methods that effectively reduce spatial relation hallucinations of LVLMs. These methods are inspired by the principle that, in tasks involving structured variables, once a variable's value is determined, related variables become constrained~\citep{ning-etal-2019-partial}. Specifically, in spatial relation reasoning, establishing the spatial relationship between two objects naturally constrains the potential relationships among other objects in the scene~\citep{choi2018structured}. For instance, Figure~\ref{fig:demo} demonstrates a scenario~\citep{yang2019spatialsense} where the model initially misinterprets the spatial relation between the transparent card holder and the cell phone. By identifying the relations among the camera, the card holder, and the cell phone, the initially incorrect relation is constrained and corrected. 

We propose two constraints: bidirectional constraint and transitivity constraint. Bidirectional constraints ensure that the spatial relations between two objects remain consistent when viewed from either direction. Transitivity constraints, introduce a third object as a reference to maintain logical coherence across multiple spatial relations, reducing the likelihood of conflicting interpretations. By combining these constraints, we establish a more robust approach for visual spatial relation. 


%Transitivity constraint introduces a third object as the reference as illustrated in Figure~\ref{fig:demo}. Bidirectional constraint builds on a similar principle,  leveraging converse relations to regulate direct relations in a bidirectional relationship, this method, when combined with transitivity constraint, forms the combined constraint -- a more robust approach for spatial relation reasoning.

We compare our methods against baseline methods using three widely used spatial relation datasets. The results show that all our methods have significantly improved performance, with the combined method outperforming the other two constraint methods. These findings demonstrate the effectiveness of our approach in mitigating multimodal spatial relation hallucinations. Furthermore, we analyze the performance across different method variants, highlighting their effectiveness and the variability in performance across datasets.

\section{Constraint-Aware Prompting}
This section provides a description of our proposed methods. The demonstrated methods are designed for spatial relation binary VQA, where the input is typically an image-question pair. The underlying constraint-aware approach has the potential to be applied to a broader range of spatial relation tasks.

\begin{figure}[t]
    \centering
    \begin{overpic}[width=1\linewidth]{figures/prompt6.png}
        \put(4,33.5){\textcolor{black}{
        \begin{minipage}{0.91\linewidth}
            \fontsize{8}{12}\selectfont
            \textbf{\#\# Instructions \#\#}
            \\1. Repeat the question + \{extract and label objects\}
            \\2. \{spatial relation analysis\}
            \\3. \textcolor{blue}{Think step by step} + use "yes" or "no" to answer the question
            \\\textbf{\#\# Please output in the following format \#\#}
            \\...
            \\\textcolor{blue}{Horizontal relation} between $O_1$ and $O_2$: $O_1$ is <relation> $O_2$
            \\\textcolor{blue}{Vertical relation} between $O_1$ and $O_2$: $O_1$ is <relation> $O_2$
            \\\textcolor{blue}{Depth relation} between $O_1$ and $O_2$: $O_1$ is <relation> $O_2$
            \\...
            \\\textbf{\#\# Question \#\#}
            \\Is there <Object> <Relation> <Object> in the image?
        \end{minipage}}}
    \end{overpic}
    \caption{Template prompt skeleton. Prompting techniques are highlighted in blue. The phrase inside \{\} is the summary of omitted details, and $O_1$ and $O_2$ represent the label of objects.} 
    \label{fig:prompt8}
\end{figure}

\begin{figure}[t]
    \centering
    \fbox{%
    \begin{overpic}[width=0.94\linewidth]{figures/ab_example.png}
        \put(28.5,38){\bfseries{
        \begin{minipage}{0.91\linewidth}   
        \linespread{0.8}\fontsize{7}{12}\raggedright\selectfont\sloppy\textcolor{myred}{A}
        \end{minipage}}}
        \put(60,38){\bfseries{
        \begin{minipage}{0.91\linewidth}   
        \linespread{0.8}\fontsize{7}{12}\raggedright\selectfont\sloppy\textcolor{myred}{B}
        \end{minipage}}}
        \put(-2.2,15.8){
        \begin{minipage}{0.96\linewidth}   \linespread{0.8}\fontsize{7}{12}\raggedright\selectfont\sloppy 
            \textbf{Question: Is there a cat on the right of a rabbit in the image?} \\\vspace{6pt}
            \textbf{AB Relation:} What is the relation between A (cat) and B (rabbit)?\\
            \textbf{BA Relation:} What is the relation between B (rabbit) and A (cat)?\\
            \textbf{AB+BA Relation:} What is the relation between A (cat) and B (rabbit) and between B (rabbit) and A (cat)?\\
            \textbf{BA+AB Relation:} What is the relation between B (rabbit) and A (cat) and between A (cat) and B (rabbit)?\\
        \end{minipage}}
    \end{overpic}
    }
    \caption{Example shows how candidate objects in the question are labeled and the corresponding spatial relations in the AB, BA, AB+BA, and BA+AB orders. "Cat" is labeled as "A" because it appears earlier than "rabbit" in the question.}
    \label{fig:ab_example}
\end{figure}

\begin{table*}[t]
\centering
\scalebox{0.79}{
    \begin{tabular}{l c c c c c c c c}
    \toprule
     & \multicolumn{2}{c}{\textbf{ARO}} & \multicolumn{2}{c}{\textbf{GQA}} & \multicolumn{2}{c}{\textbf{MMRel}} & \multicolumn{2}{c}{\textbf{Average}} \\
    \textbf{Methods} & \textbf{Acc} & \textbf{F1} & \textbf{Acc} & \textbf{F1} & \textbf{Acc} & \textbf{F1} & \textbf{Acc} & \textbf{F1} \\
    \midrule
    Baseline (vanilla) & 65.10 & 70.75 & 63.03 & 67.73 & 70.96 & 75.12 & 66.37 & 71.20\\ 
    Baseline (CoT+structure) & 69.60 & 74.51 & 63.47 & 68.03 & 80.00 & 81.41 & 71.02 & 74.65\\ 
    Bidirectional (ours) & 75.33\ \raisebox{0.5ex}{\textasteriskcentered{}\textasteriskcentered{}\textasteriskcentered{}} & 79.03\ \raisebox{0.5ex}{\textasteriskcentered{}\textasteriskcentered{}\textasteriskcentered{}} & 69.47\ \raisebox{0.5ex}{\textasteriskcentered{}\textasteriskcentered{}\textasteriskcentered{}} & 73.86\ \raisebox{0.5ex}{\textasteriskcentered{}\textasteriskcentered{}\textasteriskcentered{}} & 88.50\ \raisebox{0.5ex}{\textasteriskcentered{}\textasteriskcentered{}\textasteriskcentered{}} & 89.12\ \raisebox{0.5ex}{\textasteriskcentered{}\textasteriskcentered{}\textasteriskcentered{}} & 77.77\ \raisebox{0.5ex}{\textasteriskcentered{}\textasteriskcentered{}\textasteriskcentered{}} & 80.67\ \raisebox{0.5ex}{\textasteriskcentered{}\textasteriskcentered{}\textasteriskcentered{}}\\ 
    Transitivity (ours) & 73.90\ \raisebox{0.5ex}{\textasteriskcentered{}\textasteriskcentered{}\textasteriskcentered{}} & 75.89\ \raisebox{0.5ex}{\textasteriskcentered{}\textasteriskcentered{}\textasteriskcentered{}} & 68.93\ \raisebox{0.5ex}{\textasteriskcentered{}\textasteriskcentered{}\textasteriskcentered{}} & 71.04\ \raisebox{0.5ex}{\textasteriskcentered{}\textasteriskcentered{}\textasteriskcentered{}} & 84.03\ \raisebox{0.5ex}{\textasteriskcentered{}\textasteriskcentered{}\textasteriskcentered{}} & 83.19\ \raisebox{0.5ex}{\textasteriskcentered{}\textasteriskcentered{}\textasteriskcentered{}} & 75.62\ \raisebox{0.5ex}{\textasteriskcentered{}\textasteriskcentered{}\textasteriskcentered{}} & 76.71\ \raisebox{0.5ex}{\textasteriskcentered{}\textasteriskcentered{}\textasteriskcentered{}}\\ 
    Combined (ours) & \textbf{76.67\ }\raisebox{0.5ex}{\textasteriskcentered{}\textasteriskcentered{}\textasteriskcentered{}} & \textbf{79.57\ }\raisebox{0.5ex}{\textasteriskcentered{}\textasteriskcentered{}\textasteriskcentered{}} & \textbf{70.77\ }\raisebox{0.5ex}{\textasteriskcentered{}\textasteriskcentered{}\textasteriskcentered{}} & \textbf{75.08\ }\raisebox{0.5ex}{\textasteriskcentered{}\textasteriskcentered{}\textasteriskcentered{}} & \textbf{92.70\ }\raisebox{0.5ex}{\textasteriskcentered{}\textasteriskcentered{}\textasteriskcentered{}} & \textbf{92.91\ }\raisebox{0.5ex}{\textasteriskcentered{}\textasteriskcentered{}\textasteriskcentered{}} & \textbf{80.05\ }\raisebox{0.5ex}{\textasteriskcentered{}\textasteriskcentered{}\textasteriskcentered{}} & \textbf{82.52\ }\raisebox{0.5ex}{\textasteriskcentered{}\textasteriskcentered{}\textasteriskcentered{}}\\ 
    
    \bottomrule
    \end{tabular}
}\caption{5-trial average results of our methods on three datasets using GPT-4o. The "average" column represents the overall performance across the datasets.  \raisebox{0.5ex}{\textasteriskcentered{}\textasteriskcentered{}\textasteriskcentered{}} indicates that the p-value of the one-sided t-test is less than 0.05 (comparing our methods with others and comparing the combined constraint with the other two constraints).}
\label{tab:main_results}
\end{table*}

\paragraph{Skeleton}
\label{sec:skeleton}
\noindent As illustrated in Figure~\ref{fig:prompt8}, our methods follow the structure: \textit{Instructions + Output Format + Question}. Instead of relying on a few-shot prompt, we use a zero-shot prompt with step-by-step instructions to reduce costs and specify an output format to facilitate validation and evaluation.

To minimize hallucination in intermediate steps and enhance LVLM reasoning, we incorporate various techniques. We leverage zero-shot chain-of-thought (CoT) prompting~\citep{NEURIPS2022_9d560961} to enable LVLMs to reason effectively based on detected spatial relations. In the output format, we adopt a reasoning structure~\citep{zhou2024self}, explicitly instructing LVLMs to analyze horizontal, vertical, and depth relations between objects. This approach ensures that models generate comprehensive spatial relations and engage in thorough reasoning. A detailed analysis of these techniques can be found in Appendix~\ref{sec:appendix1}.

Our methods also instruct LVLMs to label the first and second objects mentioned in the candidate VQA question as A and B, respectively. These symbols are then used to guide the models in the subsequent spatial relation analysis. An example is illustrated in Figure~\ref{fig:ab_example}.

%Inspired by the EchoPrompt~\citep{mekala-etal-2024-echoprompt}, we include the phrase "repeat the question" to improve the accuracy of object extraction. Additionally,%

\paragraph{Bidirectional Constraint}
\noindent In the method, we prompt LVLMs to generate spatial relations in the \textit{BA + AB} order. This approach ensures that LVLMs first detect the converse spatial relation (BA) and automatically refer to it when generating the direct relation (AB). The bidirectional constraint between the converse relation and the direct relation can help models mitigate hallucinations. The example response can be found in Figure~\ref{fig:example1} and the detailed template prompt can be found in Figure~\ref{fig:prompt1} in Appendix.

\paragraph{Transitivity Constraint}
\noindent This method leverages the transitivity constraint among objects to mitigate spatial relation hallucinations. Besides Object A and B, we prompt LVLMs to randomly select a reference object, denoted as Object C. The model is then instructed to generate the spatial relations in the \textit{AC + BC} order, which serve as reference relations to transitively constrain the potentially hallucinated AB relation. The example response can be found in Figure~\ref{fig:example2} and the detailed template prompt can be found in Figure~\ref{fig:prompt2} in Appendix.

% All our methods outperform both the baseline method utilizing vanilla prompts and the baseline method employing CoT and structured reasoning output, which are also incorporated into our methods. The combined constraint shows superior performance compared to individual constraints.

\section{Experiments}
\subsection{Experimental Settings}
\label{sec:settings}
\paragraph{Datasets} 
\noindent We utilize three datasets containing spatial relation data to evaluate our proposed methods. ARO~\citep{yuksekgonul2023and} consists of 50K real-world image-caption pairs, with data sourced from Visual Genome~\citep{krishna2017visual}, MSCOCO~\citep{lin2014microsoft}, and Flickr30k~\citep{young2014image}. GQA~\citep{hudson2019gqa} includes 113K images and 22M diverse visual questions based on the Visual Genome scene graph. MMRel~\citep{nie2024mmrel} contains 15K image-question pairs addressing GPT-4V-generated annotations, utilizing real images from Visual Genome and synthetic images from SDXL~\citep{podell2024sdxl} and DALL-E~\citep{betker2023improving}. We randomly sampled 600 and 200 balanced pairs of real images and spatial relation binary VQA questions from each dataset for the test and validation splits. Data preprocessing details are in Appendix~\ref{sec:appendix4}.

\paragraph{Methods} 
With the hypothesis that the combination of bidirectional and transitivity constraints can yield improved performance, we introduce the combined constraint. It integrates two constraints and performs the relation analysis in the \textit{AC + BC + BA + AB} order. In the experiment, we use the above three constraint-aware methods against two baseline methods. The first baseline uses vanilla prompts, directly asking LVLMs to answer  questions with either "yes" or "no." The other baseline is based on the vanilla prompts but leverages prompting techniques, such as CoT and structured reasoning output (CoT+structure), which are also incorporated into our methods. The detailed prompts can be found in Appendix~\ref{sec:appendix7}. For each method and dataset, we conduct five trials, calculate the average results, and perform a one-sided t-test to further ensure the reliability of the findings.

\paragraph{Model Settings}
\noindent We use GPT-4o\footnote{GPT-4o-2024-05-13}~\citep{radford2018improving} as the LVLM in all experiments. The temperature and top-p are both set to a small number $1 \times 10^{-15}$, and a fixed seed is used to get more deterministic responses.

\begin{figure*}[t]
    \centering
    \includegraphics[width=\textwidth]{figures/chart.pdf}
    \caption{The accuracy comparison of different relation analysis choices in bidirectional and combined constraints is shown. \textit{BA + AB} is the method utilized in our proposed approach. \textit{BA} and \textit{AB + BA} are the variants of our method: \textit{BA} refers to analyzing only the converse relation, while \textit{AB + BA} analyzes the direct relation first, followed by the converse relation. \textit{AB}, which only analyzes the direct relation, is not considered a bidirectional constraint, as the converse relation is not examined. For the diagram of F1 score and detailed data, refer to Appendix~\ref{sec:appendix2}.}
    \label{fig:relation}
\end{figure*}

\subsection{Results}
\noindent As shown in Table~\ref{tab:main_results}, all of our proposed methods significantly outperform two baselines in both accuracy and F1 score. This highlights the effectiveness of our constraint-aware methods. Moreover, the combined constraint achieves the highest accuracy and F1 score across the three datasets, including an impressive 92.7\% accuracy and 92.91\% F1 score on the MMRel dataset. This demonstrates that the features of bidirectional and transitivity constraints can be combined to enhance performance.

To prove that our methods are generalized to other models, we also test them using different LVLMs. The results can be found in Appendix~\ref{sec:additional_appendix}.

\section{Analysis}
\subsection{Bidirectional Relation Analysis}
\label{sec:analysis1}
When utilizing the bidirectional constraint, the key factor is ensuring that LVLMs analyze the converse relation (\textit{BA}). In our experiments, we prompted GPT to analyze spatial relations in the \textit{BA + AB} order. However, alternative approaches exist, such as analyzing in the \textit{AB + BA} order or focusing solely on \textit{BA}.

We compare the performance of these variants under bidirectional and combined constraints. As shown in Figure~\ref{fig:relation}, methods analyzing the \textit{BA} relation demonstrated higher average accuracy compared to those using only the \textit{AB} relation, reaffirming the effectiveness of bidirectional constraints. However, no single method consistently outperformed the others across datasets, with results varying between them. Additionally, we observed that within the group employing bidirectional constraints, the \textit{AB} and \textit{AB + BA} performed similarly, as did the \textit{BA} and \textit{BA + AB}. We hypothesize that when the first relation is hallucinated, the subsequent relation is likely to be affected as well. This aligns with the findings that when LLMs reach a hallucinated answer, its subsequent explanations are also likely to be false~\citep{ye2022unreliability}.

\begin{table}[t]
\centering
\scalebox{0.56}{
    \begin{tabular}{l c c c c c c c c}
    \toprule
     & \multicolumn{2}{c}{\textbf{ARO}} & \multicolumn{2}{c}{\textbf{GQA}} & \multicolumn{2}{c}{\textbf{MMRel}} & \multicolumn{2}{c}{\textbf{Average}}\\
    \textbf{Attributes} & \textbf{Acc} & \textbf{F1} & \textbf{Acc} & \textbf{F1} & \textbf{Acc} & \textbf{F1} & \textbf{Acc} & \textbf{F1}\\
    \midrule
    The largest & \textbf{75.00} & 76.92 & 65.00 & 67.49 & 83.83 & \textbf{83.25} & 74.61 & 75.89 \\ 
    The smallest & 73.33 & 75.83 & 66.00 & 67.92 & 84.00 & 82.98 & 74.44 & 75.58 \\ 
    The most top & 72.17 & 73.62 & 65.50 & 67.91 & 81.67 & 80.84 & 73.11 & 74.12 \\ 
    The central & \textbf{75.00} & 76.78 & \textbf{69.17} & 70.77 & 83.00 & 81.72 & \textbf{75.72} & 76.42 \\ 
    The most obvious & \textbf{75.00} & \textbf{77.06} & 67.33 & 69.38 & 81.83 & 80.71 & 74.72 & 75.72 \\ 
    Random & 73.90 & 75.89 & 68.93 & \textbf{71.04} & \textbf{84.03} & 83.19 & 75.62 & \textbf{76.71} \\ 
    \bottomrule
    \end{tabular}
}\caption{The comparison of different reference object selection strategies in transitivity constraints.}
\label{tab:analysis1}
\end{table}

\begin{table}[t]
\centering
\scalebox{0.56}{
    \begin{tabular}{l c c c c c c c c}
    \toprule
     & \multicolumn{2}{c}{\textbf{ARO}} & \multicolumn{2}{c}{\textbf{GQA}} & \multicolumn{2}{c}{\textbf{MMRel}} & \multicolumn{2}{c}{\textbf{Average}}\\
    \textbf{Attributes} & \textbf{Acc} & \textbf{F1} & \textbf{Acc} & \textbf{F1} & \textbf{Acc} & \textbf{F1} & \textbf{Acc} & \textbf{F1}\\
    \midrule
    The largest & 77.00 & 79.53 & 70.50 & 75.04 & 91.67 & 91.88 & 79.72 & 82.15 \\ 
    The smallest & 75.83 & 79.02 & 69.67 & 74.37 & 91.17 & 91.41 & 78.89 & 81.60 \\ 
    The most top & 76.17 & 79.25 & 68.83 & 73.48 & 92.33 & 92.60 & 79.11 & 81.78 \\ 
    The central & \textbf{78.17} & \textbf{80.82} & 68.83 & 73.62 & 90.67 & 90.91 & 79.22 & 81.78\\ 
    The most obvious & 77.67 & 80.41 & 69.67 & 73.93 & 92.33 & 92.48 & 79.89 & 82.27 \\ 
    Random & 76.67 & 79.57 & \textbf{70.77} & \textbf{75.08} & \textbf{92.70} & \textbf{92.91} & \textbf{80.05} & \textbf{82.52}\\ 
    \bottomrule
    \end{tabular}
}\caption{The comparison of different reference object selection strategies in combined constraints.}
\label{tab:analysis2}
\end{table}

\subsection{Reference Selection}
\label{sec:analysis2}
The reference object in transitivity constraints plays a crucial role. Ideally, it should not introduce new hallucinations and must be tactically positioned to challenge the originally hallucinated relation. Thus, selecting a reliable reference object is essential.

In this analysis, we evaluate different reference selection strategies  using transitivity and combined constraints. This is done through prompting "select <attribute> object" as shown in Figures~\ref{fig:prompt6} and~\ref{fig:prompt7}. The attributes are manually defined as "the largest," "the smallest," "the most top," "the central," and "the most obvious," based on the assumption that they could affect the quality of reference objects.

As shown in Tables~\ref{tab:analysis1} and~\ref{tab:analysis2}, certain datasets demonstrate a preference for specific attributes. For instance, "the central" performs better in the ARO dataset, while random selection is better in the other two datasets. Although all strategies help mitigate spatial relation hallucinations, their performance varies depending on the dataset data distribution.

Generally, we found that the largest reference object performs slightly better than the smallest reference object, and an effective reference object often challenges the originally hallucinated relation. An important insight for selecting the third object is that a large object positioned between the candidate objects is more likely to challenge the hallucinated relation and mitigate the hallucination.

\section{Conclusion}
We have proposed two powerful constraint-aware methods, bidirectional and transitivity constraints, based on the inter-constraint relations among structured variables. These methods and their combinations and variants can be easily implemented to enhance the performance of LVLMs in visual spatial relations. We hope our proposed methods and experimental results can inspire further exploration in multimodal spatial relation tasks.

\section*{Limitations}
The proposed methods are inspired by constraints found in structured variables. This insight was examined and evaluated using binary VQA, and we did not evaluate our proposals with other visual tasks. Besides that, in our current evaluation, we primarily focus on regular spatial relationships, such as horizontal and depth relations. However, real-world data encompasses a broader range of spatial relations. They will be solid aspects for us to extend in future experiments.

For future research, we aim to develop a more deterministic automatic reference object selection mechanism to replace the current random selection used in the transitivity constraint. Appendix~\ref{sec:tradeoff_analysis} shows the tradeoff between accuracy and cost across the methods. Although our methods are highly cost-efficient, we can further reduce API costs while maintaining accuracy in the following research. Appendix~\ref{sec:appendix6} analyzes several failed cases of our current methods, offering insights that reveal potential directions for improving our methods in the future.



% Bibliography entries for the entire Anthology, followed by custom entries
%\bibliography{anthology,custom}
% Custom bibliography entries only
\bibliographystyle{acl_natbib}
% This must be in the first 5 lines to tell arXiv to use pdfLaTeX, which is strongly recommended.
\pdfoutput=1
% In particular, the hyperref package requires pdfLaTeX in order to break URLs across lines.

\documentclass[11pt]{article}

% Change "review" to "final" to generate the final (sometimes called camera-ready) version.
% Change to "preprint" to generate a non-anonymous version with page numbers.
% \usepackage[review]{acl}
\usepackage[preprint]{acl}

% Standard package includes
\usepackage{times}
\usepackage{latexsym}

% For proper rendering and hyphenation of words containing Latin characters (including in bib files)
\usepackage[T1]{fontenc}
% For Vietnamese characters
% \usepackage[T5]{fontenc}
% See https://www.latex-project.org/help/documentation/encguide.pdf for other character sets

% This assumes your files are encoded as UTF8
\usepackage[utf8]{inputenc}

% This is not strictly necessary, and may be commented out,
% but it will improve the layout of the manuscript,
% and will typically save some space.
\usepackage{microtype}

% This is also not strictly necessary, and may be commented out.
% However, it will improve the aesthetics of text in
% the typewriter font.
\usepackage{inconsolata}

%Including images in your LaTeX document requires adding
%additional package(s)
\usepackage{graphicx}

% If the title and author information does not fit in the area allocated, uncomment the following
%
%\setlength\titlebox{<dim>}
%
% and set <dim> to something 5cm or larger.

%%%%%%%%%% My packages adding
% \usepackage{booktabs} 
\usepackage{xspace}
\usepackage{amsmath}
\usepackage{adjustbox,booktabs,multirow}
% \usepackage{lmodern, babel}

\newcommand{\methodname}{MEMIT-Merge\xspace}
\newcommand{\eg}{\emph{e.g}\xspace}
\newcommand{\ie}{\emph{i.e}\xspace}
\newcommand{\vs}{\emph{v.s.}\xspace}
\newcommand{\etc}{\emph{etc}\xspace}

\title{MEMIT-Merge: Addressing MEMIT's Key-Value Conflicts in Same-Subject Batch Editing for LLMs}


\author{Zilu Dong \and  Xiangqing Shen \and Rui Xia\\
        School of Computer Science and Engineering  \\ 
        Nanjing University of Science and Technology, China \\
        \{zldong, xiangqing.shen, rxia\}@njust.edu.cn}

\begin{document}
\maketitle
\begin{abstract}
    As large language models continue to scale up, knowledge editing techniques that modify models' internal knowledge without full retraining have gained significant attention. 
    MEMIT, a prominent batch editing algorithm, stands out for its capability to perform mass knowledge modifications.
    However, we uncover a critical limitation that MEMIT's editing efficacy significantly deteriorates when processing batches containing multiple edits sharing the same subject.
    Our analysis reveals that the root cause lies in
MEMIT's key value modeling framework: When multiple facts with the same subject in a batch are modeled through MEMIT's key value mechanism, identical keys (derived from the shared subject) are forced to represent different values (corresponding to different knowledge), resulting in updates conflicts during editing.
    Addressing this issue, we propose MEMIT-Merge, an enhanced approach that merges value computation processes for facts sharing the same subject, effectively resolving the performance degradation in same-subject batch editing scenarios. 
    Experimental results demonstrate that when MEMIT's edit success rate drops to around 50\% at larger batch sizes, \methodname maintains a success rate exceeding 90\%, showcasing remarkable robustness to subject entity collisions.

\end{abstract}
\section{Introduction}
\label{sec:introduction}

Knowledge editing techniques aim to update models' internal knowledge without retraining.
As large language models (LLMs) continue to scale up, the prohibitive cost of full model retraining has made knowledge editing increasingly crucial in this domain. Among prevalent editing algorithms, a class of algorithms, termed ``Locate and Edit'' methods by \citet{zhang_comprehensive_2024}, operates on the fundamental assumption that specific knowledge representations are localized within particular regions of the model's parameter space, enabling targeted modifications through precise manipulation of these identified regions.
MEMIT \cite{DBLP:conf/iclr/MengSABB23}, one of the most prominent algorithms in this class, has gained significant attention \cite{DBLP:conf/aaai/Li0SYMY24, fang_alphaedit_2024, DBLP:conf/emnlp/GuptaSA24}.
MEMIT inherits the core architectural feature of ROME \cite{DBLP:conf/nips/MengBAB22}, which localizes knowledge to specific layers and modifies the output linear layer of MLP modules to update knowledge. 
The distinctive advancement of MEMIT lies in its capability to perform batch-wise mass knowledge editing, enabling simultaneous modification of multiple knowledge instances within a single batch.

\begin{figure}[t]
    \centering
    \includegraphics[width=\columnwidth]{imgs/same_diff_editsuccess.pdf}
                   \caption{The edit success rate of the MEMIT method on same-subject and distinct-subject datasets, showing the changes with varying batch sizes. A significant decline is observed when the subjects are the same. }
    \label{fig:same_diff_editsuccess}
\end{figure}

However, our investigation reveals a critical limitation in MEMIT: When handling batches with multiple edits that share the same subject (such as ``John Smith now plays basketball.'' and ``John Smith comes from England.'' where both statements have the subject ``John Smith''), the method will exhibit significant performance degradation. In contrast, edits with different subjects (such as ``Jack Johnson now plays basketball'' and ``Paul Morand comes from China'' with subjects `Jack Johnson'' and ``Paul Morand'', respectively) maintain stable efficacy.
To systematically demonstrate this performance degradation, we constructed two contrastive datasets comprising batches with identical subjects versus fully unique subjects, named distinct-subject and same-subject, respectively. The experimental results are in Fig.~\ref{fig:same_diff_editsuccess}, where the vertical axis represents efficacy (which means the editing success rate) and the horizontal axis indicates the batch size per edit. The results reveal that MEMIT maintains a high success rate as batch size increases when editing distinct-subject cases, but exhibits significant performance degradation for same-subject cases. More detailed experimental settings can be found in Sec.~\ref{subsec:same_diff_edit}.
Batch editing of same-subject knowledge is essential in practical applications. For instance, updating multiple attributes about a specific individual (such as occupation, nationality, and affiliations) inherently requires concurrent editing of same-subject facts.


Our analysis reveals that the root cause lies in MEMIT's key-value modeling framework: when handling edits involving the same subject, identical keys (derived from shared subject representations) are forced to map to divergent values (corresponding to distinct knowledge updates). 
This results in inconsistencies during batch editing.
Specifically, MEMIT treats each knowledge instance as a key-value pair, where the input vector to the MLP's output linear layer serves as the key, and the desired output becomes the value.
\footnote{Note that the key-value here refers to the hidden state and output within the MLP module as described by \citet{DBLP:conf/nips/MengBAB22}, rather than the query, key and value in the attention module.} 
The algorithm modifies the MLP layer by computing target values based on new knowledge objects and updating weights to map existing keys to these new values.
However, MEMIT's per-fact value computation creates inherent conflicts: within a single batch, multiple entries sharing identical keys (from same-subject cases) require different value mappings. 
As MLP output linear layers cannot produce multiple outputs for identical inputs, these conflicting updates lead to the performance degradation. Furthermore, we observe that the closer the keys are to each other, the more significant the performance degradation.

To resolve this fundamental conflict, we propose \methodname, an enhanced variant of MEMIT. The core insight stems from addressing MEMIT's critical flaw: independent value computation for same-key entries. Our solution enforces value consistency by merging multiple knowledge entries sharing identical keys, \ie, the same-subject cases into a unified value representation through constrained optimization.
To validate the effectiveness of our approach, we conducted experiments using the aforementioned same-subject and distinct-subject datasets. The results demonstrate that on same-subject data, \methodname consistently outperforms the standard MEMIT as the batch size increases. 
Notably, when MEMIT's success rate drops to around 50\% at larger batch sizes, \methodname maintains a success rate exceeding 90\%. While on distinct-subject data, both methods exhibit comparable performance, with no statistically significant differences observed. 

\section{Related Work}
\label{sec:appendix_related_work}


Knowledge editing techniques for large language models (LLMs) primarily fall into two paradigms: non-parametric approaches that preserve original parameters and parametric methods that directly modify model weights. Parametric approaches, while effective for targeted updates, often introduce uncontrolled parameter perturbations that adversely affect unrelated knowledge --- a challenge addressed through various constraint mechanisms.
The parametric category features two dominant subclasses: One is ``Meta-Learning Based Methods'', such as MEND \cite{DBLP:conf/iclr/MitchellLBFM22} and MALMEN \cite{DBLP:conf/iclr/TanZF24} which train meta-networks using carefully designed datasets containing both unrelated knowledge samples and paraphrased sentences, aiming to enhance generalization while minimizing collateral damage. Another is Locate-and-Edit Methods, which includes techniques such as Knowledge Neuron (KN) \cite{DBLP:conf/acl/DaiDHSCW22}, identify critical knowledge storage locations before executing precise edits. ROME \cite{DBLP:conf/nips/MengBAB22} extends this by incorporating knowledge preservation terms in its optimization objective to maintain model integrity.

Our work builds upon MEMIT, a state-of-the-art locate-and-edit approach that enables batch knowledge editing through MLP layer modifications. Building on MEMIT, many recent methods have made modifications to parameter update methods during editing or to the architecture and location of the edits. PMET \cite{DBLP:conf/aaai/Li0SYMY24} incorporates the output of the attention layer in the calculation of parameter updates. AlphaEdit \cite{fang_alphaedit_2024} improves upon MEMIT's parameter matrix update method by projecting the update matrix into the null space of the original knowledge to mitigate interference with unrelated knowledge. UNKE \cite{deng_unke_2024} extends structured knowledge editing to unstructured editing.


\section{Problem}

\subsection{Preliminaries}
\label{preliminaries}

The MEMIT framework posits that factual knowledge is stored within the MLP layer parameters in transformer-based models. Each MLP layer in the standard architecture constitutes a two-layer neural network comprising input and output linear layers, whose parameter matrices are denoted as $W_{in}$ and $W_{out}$, respectively.
MEMIT refers to the hidden state in the middle of this two-layer neural network as the key, and the final output of the MLP as the value. The output linear layer can thus be regarded as a key-value mapping, which is the optimization target of the MEMIT method, corresponding to the parameter matrix $W_{out}$. Typically, during MEMIT editing, all knowledge to be edited can be placed in a single batch, allowing for the simultaneous editing of a large amount of knowledge.

MEMIT employs a triple $(s, r, o)$ to represent the subject, relation, and object of a piece of knowledge to be edited, and constructs a complete sentence based on these elements for the editing process. During the editing process of MEMIT, the key is determined by the subject of the sentence and the part preceding the subject. The value, on the other hand, needs to be a value that enables the model to output the object. Therefore, it is optimized backward from the object of the sentence:
%The calculation is shown in the following formula, 
\begin{equation}
\label{eq:MEMIT_v_update}
v=\arg\min_v (-\log P_v [o|(s, r)]).
\end{equation} 
%\footnote{For ease of understanding, the random prefix used for generalization in the original paper is omitted here}.
This process generates a $(k, v)$ pair representing the knowledge triplet $(s,r,o)$. The model subsequently employs the following formula to update $W_{out}$ to enable the key to map to the corresponding value:
\begin{equation}
W_{out} = W_0 +(V-W_0K)K^T(C+KK^T)^{-1}
\end{equation}
where $K$ and $V$ represent the sentences composed of the k and v values in a batch, respectively. $W_0$ denotes the original $W_out$ weights before editing, and $C$ is a constant representing the existing knowledge. In this way, the model can output the edited $o$ with $(s,r)$.

\subsection{Same subject issue in MEMIT}
\label{sec:issues in MEMIT edits}

\begin{figure*}[htbp]
    \includegraphics[width=\linewidth]{imgs/samesub.pdf}
    \caption{The architecture of MEMIT processing two same subject sentences. The left and right sides of the figure depict the processing flow of the two sentences respectively. Below, we expand the details of the MLP module to be modified, which consists of two linear layers. In MEMIT, the key is determined by the subject, resulting in identical keys on both sides. The value is optimized from the relation and object, leading to different values on each side. Consequently, the optimization target for the editable $W_{out}$ requires producing different values for the same input key.}
    \label{fig:general_process}
\end{figure*}


Normally, MEMIT is capable of maintaining its efficacy without a pronounced decline in performance when the edit batch size approaches 1,000. Moreover, the edit success rate remains relatively stable even when editing up to 10,000 knowledge triples. However, we have identified a notable issue: when the edit batch encompasses knowledge triples sharing the same subject, the editing capacity of MEMIT experiences a substantial degradation.


To verify this phenomenon, we constructed two counterfactual editing datasets. In the first dataset, the subjects of the knowledge triples are all distinct. In the second dataset, the subjects of the knowledge triples are replaced by a single, fixed subject, while all other parts of these two datasets remain identical. The details of the datasets construction are provided in App.~\ref{sec:appendix_dataset_construct}.


As illustrated in Fig.~\ref{fig:same_diff_editsuccess}, when the subjects are identical, the performance of the MEMIT method drops sharply with a batch size of only 2, and the edit success rate falls below 50\% when the batch size reaches 10. In contrast, when subjects are distinct, increasing the batch size has virtually no impact on edit success.

\section{Approach}
\label{sec:methods}


\subsection{Cause Analysis}
\label{sec:reason_analysis}

In our analysis, the degradation of editing capability caused by identical subjects is closely related to the key-value modeling of knowledge inherent in locate-and-edit class editing methods.


In the standard MEMIT, a piece of knowledge to be edited can be represented by a knowledge triplet (subject, relation, object), and a complete sentence is constructed based on this triplet for the editing process. In this paper, we use the format ``{subject}'s {relation} is {object}'' to construct the sentence. For example, the knowledge triple (John,father,Bob) is formulated into the sentence ``John's father is Bob.''.


As described in Sec.~\ref{preliminaries}, during MEMIT editing, the key is derived from the subject, while the value is determined by the object.
However, when editing multiple pieces of knowledge with the same subject but different objects in one batch, this mechanism forces the MLP to map the same key to two distinct values. As illustrated in Fig.~\ref{fig:general_process}, a given key can only produce a single fixed value through deterministic $W_{out}$. This creates a conflict when optimizing the parameter matrix, making it extremely challenging. We refer to this issue as the key collision problem. 
Consequently, when a batch contains multiple edits with the same subject, as demonstrated in Fig.~\ref{fig:same_diff_editsuccess}, the editing capability of MEMIT is significantly degraded.


To further investigate the relationship between the decline in editing capability and the distance between keys, we propose an evaluation metric: the Average Keys Distance Inside Batch (AKD). This metric is defined as the average Euclidean distance between the key values of all pairs of knowledge within a batch. It reflects the average distance between keys in the batch and is represented as

$$
AKD^{(l)}=\frac{1}{\binom{|B|}{2}}\sum_{\substack{e_1\in B\\ e_2\in B}} ||k^{(l)}_{e_1}-k^{(l)}_{e_2}||_2
$$
 
where $l$ represents the $l$-th layer,  $B$ denotes the batch of knowledge to be edited, $k^{(l)}_{e_1}$ represents the key value computed by the MLP module in the $l$-th layer for the input knowledge $e_1$. 

We compute $AKD$ for all layers of the model at the subject's last token position.
As the degree of subject variation increases across sentences, the $AKD$ value proportionally rises. Conversely, when all sentences share identical subjects, the $AKD$ value remains constant at 0.
 

We construct sentence batches using predefined templates, where batches sharing the same template exhibited similar $AKD$ values, while distinct templates yielded significantly different $AKD$ measurements. The specific templates and the corresponding values of $AKD$ are detailed in App.~\ref{sec:appendix_akd_dataset}.
For experimental validation, we select three $AKD$ groups (0, 10, 25) and conduct editing tests using Qwen2.5-1.5B-Instruct. As shown in Fig.~\ref{fig:akd}, where $AKD$ values are computed using keys from MEMIT's final editing layer, the results demonstrate an inverse relationship: lower $AKD$ values correspond to reduced editing success rates. This pattern remains consistent across other $AKD$ values, establishing a statistically significant negative correlation between $AKD$ and editing efficacy.

\begin{figure}[th]
    \centering
    \includegraphics[width=\linewidth]{imgs/akd.pdf}
    \caption{Datasets with different $AKD$ values and the results of edit efficacy. The lower the $AKD$ value, the more severe the decline in edit capability.}
    \label{fig:akd}

\end{figure}

\subsection{The MEMIT-Merge Approach}


To address the issue of a single key corresponding to multiple distinct values, we organize all the knowledge triples to be edited within a batch into multiple ``same-subject groups'' based on whether their subjects are identical. Within each group, the subjects of the knowledge triple to be edited are the same.


Compared with the optimization objective of standard MEMIT in Eq.~\ref{eq:MEMIT_v_update}, our improved version aims to optimize the knowledge triples in the same-subject group to a single value
$$v=\arg\min_v \sum_{(s, r_j, o_j)\in S} -\log P_v[o_j|(s, r_j)]$$
where $S$ represents the set of knowledge triples with the subject $s$,  $v$ is the value to be optimized in a backward manner, and $P_v$ denotes the model when the value is equal to $v$.

This approach ensures that a same-subject group gets the same value while sharing the same key, thereby significantly alleviating the decline in edit efficacy observed in standard MEMIT.


\section{Experiments}
\subsection{Dataset}
\label{subsec:dataset}


As mentioned in the Introduction section, we constructed two counterfactual knowledge editing datasets based on Wikidata, one with the same subject and the other with distinct subjects. The dataset named ``same-subject'' dataset, contains 100 editing knowledge triples, all with the subject John Smith. The other dataset, named ``distinct-subject'', has knowledge triples with subjects unique to each other, while the relations and objects remain consistent with those in the ``same-subject'' dataset. For specific details on the construction of the datasets, refer to App.~\ref{sec:appendix_dataset_construct}.

In terms of evaluation metrics, we refer to the metrics used by \citet{DBLP:conf/iclr/MengSABB23}, namely Efficacy, Paraphrase, and Specificity. Efficacy is the edit success rate, which measures the probability that the edited model produces the correct answer to the sentences used for editing. Paraphrase is measured in the same way as Efficacy, but it uses paraphrased versions of the sentences used for editing. Specificity measures the probability that facts unrelated to the edit remain consistent before and after the edit.

Although we constructed our own datasets due to the lack of specialized data on the same subject in the past, addressing same-subject issue has a practical necessity. Editing multiple attributes of an entity (\eg, updating a person’s profile) is a highly realistic demand. Furthermore, changing certain information about an entity sometimes requires a chain of changes to other information about that entity. All of these make the same subject scenarios essential for practical applications.

\subsection{Experimental Setup}
\label{subsec:setup}

We conducted experiments on three models with different architectures: Qwen2.5-1.5B-Instruct \cite{qwen_qwen25_2025}, GPT-J-6B \cite{gpt-j}, and Llama-3-8B-Instruct \cite{llama3modelcard}.

For baselines, we selected MEMIT and an improved version of MEMIT, PMET \cite{DBLP:conf/aaai/Li0SYMY24}, as our baselines. The layers to be edited for these methods were determined based on the parameter settings in the Easyedit framework \cite{wang_easyedit_2024}. Specifically, we edited layers 4 to 8 for the Qwen model and layers 3 to 8 for the GPT-J model. The token position for editing was consistent with the original papers, which is selecting the subject last token.
In addition to the MEMIT-based methods, we also included FT-L \cite{zhu_modifying_2020}, which was used for comparison in the ROME paper, as another baseline to verify that the same-subject issue exists only in methods with the MEMIT-based architecture. For FT-L, we edited layer 21 in the GPT-J and Llama model, and layer 15 in the Qwen model.

\subsection{Results when Batch Size is 100 }
\label{subsec:same_diff_edit}


\begin{table*}[th]
\centering
\small
\begin{tabular}{@{}cccccc@{}}
\toprule
Model                                  & Dataset                       & Method & Efficiency & Parapharse & Specificity \\ \midrule
\multirow{8}{*}{Qwen2.5-1.5B-Instruct} & \multirow{4}{*}{same-subject} & FT          & 0.26       & 0.23       & 0.95        \\
 &                                   & MEMIT  & 0.30 & 0.24 & 1.00 \\
 &                                   & PMET   & 0.23 & 0.17 & 0.99 \\
 &                                   & MEMIT-Merge & 0.55 & 0.36 & 0.99 \\ \cmidrule(l){2-6} 
 & \multirow{4}{*}{distinct-subject} & FT     & 0.23 & 0.21 & 0.99 \\
 &                                   & MEMIT  & 1.00 & 0.77 & 0.90 \\
 &                                   & PMET   & 0.51 & 0.40 & 0.85 \\
 &                                   & MEMIT-Merge & 1.00 & 0.77 & 0.90 \\ \midrule
\multirow{8}{*}{GPT-J-6B}               & \multirow{4}{*}{same-subject} & FT          & 0.52       & 0.19       & 0.23        \\
 &                                   & MEMIT  & 0.27 & 0.21 & 1.00 \\
 &                                   & PMET   & 0.26 & 0.21 & 0.98 \\
 &                                   & MEMIT-Merge & 0.51 & 0.32 & 1.00 \\ \cmidrule(l){2-6} 
 & \multirow{4}{*}{distinct-subject} & FT     & 0.47 & 0.28 & 0.22 \\
 &                                   & MEMIT  & 1.00 & 0.77 & 0.93 \\
 &                                   & PMET   & 0.25 & 0.25 & 0.99 \\
 &                                   & MEMIT-Merge & 1.00 & 0.77 & 0.93 \\ \midrule
\multirow{8}{*}{Llama-3-8B-Instruct}  & \multirow{4}{*}{same-subject} & FT          & 0.67       & 0.47       & 0.27        \\
 &                                   & MEMIT  & 0.38 & 0.29 & 0.98 \\
 &                                   & PMET   & 0.23 & 0.21 & 0.98 \\
 &                                   & MEMIT-Merge & 0.71 & 0.44 & 0.98 \\ \cmidrule(l){2-6} 
 & \multirow{4}{*}{distinct-subject} & FT     & 0.73 & 0.58 & 0.24 \\
 &                                   & MEMIT  & 0.99 & 0.91 & 0.82 \\
 &                                   & PMET   & 0.46 & 0.46 & 0.92 \\
 &                                   & MEMIT-Merge & 1.00 & 0.91 & 0.81 \\ \midrule
\multirow{8}{*}{Qwen2.5-7B-Instruct}   & \multirow{4}{*}{same-subject} & FT          & 0.28       & 0.23       & 0.99        \\
 &                                   & MEMIT  & 0.31 & 0.25 & 1.00 \\
 &                                   & PMET   & 0.23 & 0.18 & 0.99 \\
 &                                   & MEMIT-Merge & 0.67 & 0.43 & 0.99 \\ \cmidrule(l){2-6} 
 & \multirow{4}{*}{distinct-subject} & FT     & 0.23 & 0.22 & 0.98 \\
 &                                   & MEMIT  & 0.99 & 0.84 & 0.91 \\
 &                                   & PMET   & 0.52 & 0.47 & 0.84 \\
 &                                   & MEMIT-Merge & 1.00 & 0.86 & 0.90 \\ \bottomrule
\end{tabular}
\caption{The complete results of the four editing methods—MEMIT, \methodname, PMET, and FT-L—on the same-subject and distinct-subject datasets at a batch size of 100. All experimental results were obtained by re-running each editing method on our dataset.}
\label{table:batch100_result}
\end{table*}


We first compared the edit success rates of standard MEMIT, PMET, \methodname, and FT-L on the two datasets across several models. \footnote{The results for all baselines were obtained by running the code from the Easyedit framework on our datasets.}

Tab.~\ref{table:batch100_result} presents the complete results of several editing methods on the same subject and distinct-subject datasets at a batch size of 100. Compared with the standard MEMTI, it can be observed that our method achieves an improvement in the paraphrase metric. This improvement is likely attributable to the originally low edit success rate, and by enhancing the edit success rate, we naturally improved the paraphrase performance.
Regarding specificity, it is noticeable that standard MEMIT exhibits an anomalously high specificity when the subjects are the same. However, comparing this with the results from the distinct-subject dataset reveals that specificity is actually low under normal editing conditions. Therefore, we infer that the abnormally high specificity is due to the low edit success rate, which indicates a minimal impact on the original model. In contrast, our \methodname has a higher edit success rate, resulting in a specificity that is comparable to the distinct subject scenario and the FT method. Thus, \methodname does not negatively affect specificity.

Compared to FT-L, it can be observed that the FT-L results do not decrease when subjects are the same. This situation is consistent with our analysis in Sec.~\ref{sec:reason_analysis}, which indicates that the decline in editing performance under the same subject is a unique issue of the MEMIT editing method. Similarly, the results of PMET also show a decline in efficacy only when the subjects are the same. The ability of \methodname to successfully mitigate the decline in editing performance by merging multiple edits when computing the value for the same key further supports the conclusion that key collision is the cause of this problem.

\begin{figure}[th]
    \centering
    \includegraphics[width=\linewidth]{imgs/MEMIT-Merge_samediff.pdf}
    \caption{The results of \methodname and MEMIT methods on same-subject and distinct-subject datasets using the GPT-J-6B model, showing the changes with varying batch sizes. \methodname is capable of significantly alleviating the decline in editing performance under the same-subject condition.}
    \label{fig:kmemit_samediff}
\end{figure}

\subsection{Results with Varying Batch Sizes}

As can be seen in Fig.~\ref{fig:kmemit_samediff}, when the subjects of the editing knowledge in the edit batch are the same, the standard edit success rate plummets at a batch size of 2, whereas \methodname is able to maintain a much higher success rate, with a significantly smaller decline compared to MEMIT. This also confirms the effectiveness of our method.

\begin{figure}[th]
    \centering
    \includegraphics[width=\linewidth]{imgs/qwen_llama_samediff.pdf}
    \caption{The results of \methodname and MEMIT methods on same-subject and distinct-subject datasets using the Qwen2.5-7B-Instruct and Llama-3-8B-Instruct.}
    \label{fig:qwen_llama_samediff}
\end{figure}

In the case of distinct subjects, the editing capability of both MEMIT and \methodname does not exhibit a significant decline even at a batch size of 100, which is consistent with our previous analysis.
The results of other editing methods are given in App.~\ref{sec:appendix_varyingbatchsize}.



Additionally, the experimental results for Qwen2.5-1.5B-Instruct and Llama-3-8B-Instruct, two models with different architectures, as shown in Fig.~\ref{fig:qwen_llama_samediff}, demonstrate that the same phenomenon observed in the GPT-J model also exists in these models. Moreover, \methodname is equally capable of significantly mitigating the performance degradation of standard MEMIT under the same-subject condition. Therefore, it can be concluded that this phenomenon is universally present across different model architectures, and our method is applicable to various model structures.


\section{Conclusion}


This paper identifies the issue of significant performance degradation in MEMIT when a batch contains knowledge sharing the same subject during batch editing. We analyze the reason based on the MEMIT framework, and develop the metric $AKD$ to further analyze the cause of this phenomenon. This study identifies a critical flaw in MEMIT's batch editing: parameter update conflicts arising from identical keys requiring divergent values in same-subject scenarios. Our proposed MEMIT-Merge resolves this through key-wise value merging, significantly improving same-subject editing success while maintaining original performance on distinct-subject cases. These findings advance mass-editing techniques for evolving LLM knowledge bases.


% Entries for the entire Anthology,  followed by custom entries
\bibliography{references,  anthology, custom}


\newpage

\appendix



\section{Details of Constructing Same-Subject and Distinct-Subject Data}
\label{sec:appendix_dataset_construct}

Our dataset construction is based on Wikidata. First, we retrieve all relations and properties associated with human subject entities from Wikidata. Then, we manually filter the relations, removing those that are less commonly used, such as ID and Wikidata categories. Finally, we obtain 100 relations.

Subsequently, we select a number of individuals from Wikidata and query their corresponding objects for the knowledge triples composed of these relations. Finally, we retain only one knowledge triple for each relation, thereby obtaining 100 knowledge triples, formatted as (subject, relation, object).

We then select another 100 distinct names from Wikidata and replace the subject entities in the previously obtained 100 knowledge triples with these new names, thereby creating the distinct-subject dataset. Conversely, we replace the subject entities in the 100 knowledge triples with a single, identical name to create the same-subject dataset.

Using the template ``{subject}'s {relation} is {object},'' we construct natural language sentences from these knowledge triples, which form the edit sentences in the dataset. For example, a knowledge triple in the same-subject dataset is (John Smith, doctoral advisor, Dennis W. Sciama), which is formulated into the natural sentence John Smith's doctoral advisor is Dennis W. Sciama. In the distinct-subject dataset, the corresponding knowledge triple with the same relation and object is (Paul Morand, doctoral advisor, Dennis W. Sciama), which is formulated into the natural sentence Paul Morand's doctoral advisor is Dennis W. Sciama.

Subsequently, following the dataset metrics in \citet{DBLP:conf/nips/MengBAB22}, we add two types of questions: specificity and paraphrase. For paraphrase questions, we use the same knowledge triples as the edit sentences, but with a different template format: ``The name of the {relation} of {subject} is {object}.''. For specificity, there are two types of questions. One is completely unrelated knowledge, for which we use the prompt ``The capital city of America is''. The other type has the same relation as the edited knowledge but a different subject. For example, if the edited knowledge is (John, father, Bob), a specificity question could be (Paul, father, Eugène).

\section{Diverse $AKD$ Dataset}
\label{sec:appendix_akd_dataset}

\begin{table}[th]
\adjustbox{max width=\linewidth}{%
\begin{tabular}{@{}lll@{}}
\toprule
\textbf{dataset} & \textbf{formatting template} & \textbf{$AKD$} \\ \midrule
same-subject & \{subject\}'s \{relation\} is \{object\}                  & 0.0  \\
distinct-subject & \{subject\}'s \{relation\} is \{object\}                  & 25.8 \\
same-subject & The name of the \{relation\} of \{subject\} is \{object\} & 10.5 \\
distinct-subject & The name of the \{relation\} of \{subject\} is \{object\} & 26.2 \\ \bottomrule
\end{tabular}}
\caption{The average $AKD$ values obtained using different data and templates.}
\label{table:data_format_akd}
\end{table}

The construction of datasets with three distinct $AKD$ values, where the keys within each dataset have a relatively consistent distance between each other.



We utilize the knowledge triples from the same-subject and distinct-subject datasets collected in Sec.~\ref{sec:appendix_dataset_construct} to construct data using different natural language sentence templates. The two templates we employ are ``{subject}'s {relation} is {object}'' and ``The name of the {relation} of {subject} is {object}''.

Tab.~\ref{table:data_format_akd} presents the average $AKD$ values obtained using different data and templates with the Qwen2.5-1.5B-Instruct model. We selected several datasets with distinct $AKD$ values. Since these datasets have consistent internal templates, the keys of the multiple knowledge triples within them are relatively uniform and close in distance. Therefore, when performing batch editing on these datasets, they can be used to study the correlation between efficacy and $AKD$.

\begin{figure}[th]
    \centering
    \includegraphics[width=\linewidth]{imgs/qwen_same_more.pdf}
    \caption{Editing same-subject dataset using Qwen2.5-1.5B-Instruct with four editing methods.}
    \label{fig:qwen_same_more}
\end{figure}

\begin{figure}[th]
    \centering
    \includegraphics[width=\linewidth]{imgs/llama_same_more.pdf}
    \caption{Editing same-subject dataset using Llama-3-8B-Instruct with four editing methods.}
    \label{fig:llama_same_more}
\end{figure}


\section{More Results with Varying Batch Sizes}
\label{sec:appendix_varyingbatchsize}


Here in Fig.~\ref{fig:qwen_same_more} and Fig.~\ref{fig:llama_same_more} we demonstrate some more results about editing same subject batch with varying batch sizes.

It shows clearly that MEMIT-based methods suffers from same subject issue, while methods like FT do not.




\end{document}


\clearpage
\appendix

\section{Appendix}
\subsection{Data Preprocessing} %finished%
\label{sec:appendix4}
We preprocess the datasets to align with our experimental objectives. To focus on more challenging data, GPT-4o-mini\footnote{GPT-4o-mini-2024-07-18} is used to filter the data, ensuring that all candidate images contain at least five objects. Non-spatial relations and vague spatial relations, such as "sitting on" and "nearby," are eliminated, leaving only those with clear spatial definitions in the images. Additionally, we format the annotations so that the questions follow the structure: \textit{Is there <Object> <Relation> <Object> in the image?} This makes image-question pair have uniform formats and allows us to focus on evaluating the feasibility of the proposed method in spatial relation tasks.

In details, the preprocessing of the ARO and GQA datasets is more complex than that of MMRel, as the annotations of ARO are primarily image captions rather than VQA questions, and ARO and GQA include non-spatial relations. Our approach mainly relies on keyword filtering and manual review to ensure the sampled data contains only spatial relations. ARO features image captions with clear lexical structures: \textit{<Object> is <Relation> <Object>.} This structure enables us to easily reconstruct the captions and convert them into VQA format.

\subsection{Analysis of Prompting Techniques}
\label{sec:appendix1}
\begin{table*}[h]
\centering
\scalebox{0.79}{
    \begin{tabular}{l c c c c c c c c c}
    \toprule
     & & \multicolumn{2}{c}{\textbf{ARO}} & \multicolumn{2}{c}{\textbf{GQA}} & \multicolumn{2}{c}{\textbf{MMRel}} & \multicolumn{2}{c}{\textbf{Average}}\\
    \textbf{Methods} & \textbf{CoT} & \textbf{Acc} & \textbf{F1} & \textbf{Acc} & \textbf{F1} & \textbf{Acc} & \textbf{F1} & \textbf{Acc} & \textbf{F1}\\
    \midrule
    \multirow{2}{*}{Bidirectional} & No & \textbf{76.00} & \textbf{79.07} & \textbf{70.67} & \textbf{74.27} & 86.83 & 87.32 & \textbf{77.83} & 80.22\\ 
    & Yes & 75.33 & 79.03 & 69.47 & 73.86 & \textbf{88.50} & \textbf{89.12} & 77.77 & \textbf{80.67}\\  
    \midrule
    \multirow{2}{*}{Transitivity} & No & 70.67 & 70.57 & 67.17 & 66.21 & 76.17 & 72.12 & 71.34	& 69.63\\ 
    & Yes & \textbf{73.90} & \textbf{75.89} & \textbf{68.93} & \textbf{71.04} & \textbf{84.03} & \textbf{83.19} & \textbf{75.62} & \textbf{76.71}\\ 
    \midrule
    \multirow{2}{*}{Combined} & No & 71.50 & 76.08 & 66.00 & 71.90 & 83.50 & 84.75 & 73.67 & 77.58 \\
    & Yes & \textbf{76.67} & \textbf{79.57} & \textbf{70.77} & \textbf{75.08} & \textbf{92.70} & \textbf{92.91} & \textbf{80.05} & \textbf{82.52}\\ 
     
    \bottomrule
    \end{tabular}
}\caption{The comparison of results with or without using CoT prompting in  bidirectional, transitivity, and combined constraints.}
\label{analysis_cot}
\end{table*}

\begin{table*}[h]
\centering
\scalebox{0.79}{
    \begin{tabular}{l c c c c c c c c c}
    \toprule
     & & \multicolumn{2}{c}{\textbf{ARO}} & \multicolumn{2}{c}{\textbf{GQA}} & \multicolumn{2}{c}{\textbf{MMRel}} & \multicolumn{2}{c}{\textbf{Average}}\\
    \textbf{Methods} & \textbf{Structure} & \textbf{Acc} & \textbf{F1} & \textbf{Acc} & \textbf{F1} & \textbf{Acc} & \textbf{F1} & \textbf{Acc} & \textbf{F1}\\
    \midrule
    \multirow{2}{*}{Bidirectional} & No & 71.5 & 71.83 & 67.17 & 66.67 & 74.67 & 69.72 & 71.11	& 69.41\\ 
    & Yes & \textbf{75.33} & \textbf{79.03} & \textbf{69.47} & \textbf{73.86} & \textbf{88.50} & \textbf{89.12} & \textbf{77.77} & \textbf{80.67}\\ 
    \midrule
    \multirow{2}{*}{Transitivity} & No & 70.33 & 73.43 & 64.17 & 65.71 & 76.50 & 76.38 & 70.33	& 71.84\\ 
    & Yes & \textbf{73.90} & \textbf{75.89} & \textbf{68.93} & \textbf{71.04} & \textbf{84.03} & \textbf{83.19} & \textbf{75.62} & \textbf{76.71}\\ 
    \midrule
    \multirow{2}{*}{Combined} & No & 75.33 & 78.43 & \textbf{71.67} & 75.07 & 87.17 & 87.64 & 78.06 & 80.38\\
    & Yes & \textbf{76.67} & \textbf{79.57} & 70.77 & \textbf{75.08} & \textbf{92.70} & \textbf{92.91} & \textbf{80.05} & \textbf{82.52}\\ 
     
    \bottomrule
    \end{tabular}
}\caption{The comparison of results with or without using structured reasoning output in bidirectional, transitivity, and combined constraints.}
\label{analysis_structure}
\end{table*}

In this section, we present experiments that demonstrate the effectiveness of the prompting techniques discussed in Section~\ref{sec:skeleton}. We do not assert that these prompting techniques alone will significantly improve the accuracy of LVLMs on spatial relation tasks. Instead, we highlight how our constraint-aware methods can effectively integrate these techniques to enhance overall performance.

\paragraph{CoT}
\noindent In this analysis, we compare GPT-4o's performance using bidirectional, transitivity, and combined constraints, both with and without CoT prompting. To control for CoT, we either include or omit the phrase "think step by step" in the prompt. This analysis follows the same dataset and model settings outlined in Section~\ref{sec:settings}.

As shown in Table~\ref{analysis_cot}, the accuracy and F1 score when using CoT prompting are consistently higher than without CoT for both transitivity and combined methods. Although in some cases, bidirectional methods without CoT outperform those with CoT, the overall performance with CoT remains superior. Thus, we think using CoT can bring benefits to our methods and increases the reasoning ability of LVLMs in general.

\paragraph{Reasoning Structure}
\noindent Similarly, we evaluate the effectiveness of the reasoning structure by comparing the performance of GPT-4o with and without this structure. To create a scenario without the reasoning structure, we remove the output format from the prompt and provide additional descriptive instructions that convey the information originally included in the structured reasoning output, such as analyzing horizontal, vertical, and depth relations.

The results are shown in Table~\ref{analysis_structure}. It is evident that explicitly stating the reasoning requirements in the output format enhances the effectiveness of our method across all three datasets and methods. We hypothesize that LVLMs adhere more strictly to the output format than to descriptive instructions. Consequently, including requirements in the output format increases the likelihood that LVLMs will follow these guidelines.

\subsection{Tradeoff between Accuracy and Cost}
\label{sec:tradeoff_analysis}
\begin{table*}[h]
\centering
\scalebox{0.79}{
    \begin{tabular}{l c c c c c}
    \toprule
     & vanilla & CoT+structure & bidirectional & transitivity & combined\\
    \midrule
    Cost per 100 Questions (\$) & 0.242	& 0.377	& 0.557	& 0.630	& 0.765\\  
    \midrule
    Average Accuracy (\%) & 66.37 & 71.02 & 77.77 & 75.62 & 80.05\\    
    \bottomrule
    \end{tabular}
}\caption{The results of analyzing the tradeoff between accuracy and API cost.}
\label{tab:cost_tab}
\end{table*}
Compared to the vanilla prompt and the prompt incorporating CoT and reasoning structure, the prompts used in our method are longer. While they improve the accuracy of model performance on tasks, they also increase the API cost. Therefore, it is crucial to analyze the accuracy improvement relative to the increase in cost. To investigate this, we design an experiment in which we apply each prompting method to 1,800 questions across three datasets and calculate the average GPT-4o API cost per 100 VQA questions for each method. The results are shown in Table~\ref{tab:cost_tab}. 

From these results, we observe that the cost of our methods is approximately two to three times higher than that of the vanilla prompt. However, the bidirectional method proves to be the most cost-efficient. Therefore, when the budget is limited, users can opt for the bidirectional method to save costs while maintaining performance. When the budget allows, users can select the transitivity or combined method, depending on their preference, to achieve the best accuracy.

\subsection{Results in Other LVLMs}
\label{sec:additional_appendix}
\begin{table*}[h]
\centering
\scalebox{0.72}{
    \begin{tabular}{cc} % 2x2 grid
        % First row: ARO and GQA subtables
        \begin{subtable}[h]{0.65\linewidth}
            \centering
            \begin{tabular}{l c c c c}
                \toprule
                \textbf{Methods} & \textbf{Acc} & \textbf{Precision} & \textbf{Recall} & \textbf{F1} \\
                \midrule
                Baseline (vanilla) & 69.50 & 62.97 & \textbf{94.67} & 75.63 \\ 
                Baseline (CoT+structure) & 71.50 & 66.17 & 88.00 & 75.54 \\ 
                Bidirectional (ours) & 70.17 & 63.97 & 92.33 & 75.58 \\ 
                Transitivity (ours) & 73.33 & 66.51 & 94.00 & 77.90 \\ 
                Combined (ours) & \textbf{74.50} & \textbf{68.24} & 91.67 & \textbf{78.24} \\ 
                \bottomrule
            \end{tabular}
            \caption{ARO}
        \end{subtable}
        \hspace{0.5cm} % Increase the horizontal gap here
        
        \begin{subtable}[h]{0.65\linewidth}
            \centering
            \begin{tabular}{l c c c c}
                \toprule
                \textbf{Methods} & \textbf{Acc} & \textbf{Precision} & \textbf{Recall} & \textbf{F1} \\
                \midrule
                Baseline (vanilla) & 66.17 & 61.41 & 87.00 & 72.00 \\ 
                Baseline (CoT+structure) & 68.50 & 62.25 & \textbf{94.00} & 74.90 \\ 
                Bidirectional (ours) & 70.17 & 63.72 & 93.67 & \textbf{75.84} \\ 
                Transitivity (ours) & 70.50 & 65.04 & 88.67 & 75.04 \\ 
                Combined (ours) & \textbf{72.00} & \textbf{67.28} & 85.67 & 75.37 \\ 
                \bottomrule
            \end{tabular}
            \caption{GQA}
        \end{subtable} \\ \\

        % Second row: MMRel and Average subtables
        \begin{subtable}[h]{0.65\linewidth}
            \centering
            \begin{tabular}{l c c c c}
                \toprule
                \textbf{Methods} & \textbf{Acc} & \textbf{Precision} & \textbf{Recall} & \textbf{F1} \\
                \midrule
                Baseline (vanilla) & 61.17 & 57.11 & 89.67 & 69.78 \\ 
                Baseline (CoT+structure) & 74.17 & 70.54 & 83.00 & 76.26 \\ 
                Bidirectional (ours) & 82.50 & 77.94 & 90.67 & 83.82 \\ 
                Transitivity (ours) & 82.83 & 76.99 & \textbf{93.67} & 84.51 \\ 
                Combined (ours) & \textbf{85.00} & \textbf{80.88} & 91.67 & \textbf{85.94} \\ 
                \bottomrule
            \end{tabular}
            \caption{MMRel}
        \end{subtable}
        \hspace{0.5cm} % Increase the horizontal gap here
            
        \begin{subtable}[h]{0.65\linewidth}
            \centering
            \begin{tabular}{l c c c c}
                \toprule
                \textbf{Methods} & \textbf{Acc} & \textbf{Precision} & \textbf{Recall} & \textbf{F1} \\
                \midrule
                Baseline (vanilla) & 65.61 & 60.50 & 90.45 & 72.47 \\ 
                Baseline (CoT+structure) & 71.39 & 66.32 & 88.33 & 75.57 \\ 
                Bidirectional (ours) & 74.28 & 68.54 & \textbf{92.22} & 78.41 \\ 
                Transitivity (ours) & 75.55 & 69.51 & 92.11 & 79.15 \\ 
                Combined (ours) & \textbf{77.17} & \textbf{72.13} & 89.67 & \textbf{79.85} \\ 
                \bottomrule
            \end{tabular}
            \caption{Average}
        \end{subtable}
    \end{tabular}
}
\caption{Full results of our methods (bidirectional, transitivity, and combined constraints) on three datasets using Gemini Pro.}
\label{tab:add_exp}
\end{table*}

In addition to GPT-4o, we also use other models to evaluate our methods. Under the same experimental settings, we use Gemini Pro\footnote{gemini-1.5-pro-002}~\citep{team2024gemini} to evaluate two baseline methods and three proposed methods. The results are shown in Table~\ref{tab:add_exp}. The performance of our methods is consistently strong in Gemini Pro: the combined method outperforms the bidirectional and transitivity methods, and the three proposed methods are generally better than the two baselines. An exception is the ARO dataset, where bidirectional constraints have slightly lower accuracy than the baseline method incorporating CoT and reasoning structure.

Additionally, we test our methods on smaller (1B or 7B) open-source models, such as LLaVA~\citep{liu2024visual} and Janus Pro~\citep{chen2025janus}. We observe that our methods show potential in mitigating hallucination in these models. However, due to limitations caused by the smaller model sizes, the responses from these models are unstable. In some cases, the models fail to follow the instructions or formats written in the prompt to generate a response.

\subsection{Extended Results}
\label{sec:appendix2}
\begin{table*}[h]
\centering
\scalebox{0.72}{
    \begin{tabular}{cc} % 2x2 grid
        % First row: ARO and GQA subtables
        \begin{subtable}[h]{0.65\linewidth}
            \centering
            \begin{tabular}{l c c c c}
                \toprule
                \textbf{Methods} & \textbf{Acc} & \textbf{Precision} & \textbf{Recall} & \textbf{F1} \\
                \midrule
                Baseline (vanilla) & 65.10 & 60.90 & 84.40 & 70.75 \\ 
                Baseline (CoT+structure) & 69.60 & 64.15 & 88.87 & 74.51 \\ 
                Bidirectional (ours) & 75.33 & 68.74 & \textbf{92.93} & 79.03 \\ 
                Transitivity (ours) & 73.90 & 70.50 & 82.20 & 75.89 \\ 
                Combined (ours) & \textbf{76.67} & \textbf{70.77} & 90.87 & \textbf{79.57} \\ 
                \bottomrule
            \end{tabular}
            \caption{ARO}
        \end{subtable}
        \hspace{0.5cm} % Increase the horizontal gap here
        
        \begin{subtable}[h]{0.65\linewidth}
            \centering
            \begin{tabular}{l c c c c}
                \toprule
                \textbf{Methods} & \textbf{Acc} & \textbf{Precision} & \textbf{Recall} & \textbf{F1} \\
                \midrule
                Baseline (vanilla) & 63.03 & 60.09 & 77.60 & 67.73 \\ 
                Baseline (CoT+structure) & 63.47 & 60.48 & 77.73 & 68.03 \\ 
                Bidirectional (ours) & 69.47 & 64.57 & 86.27 & 73.86 \\ 
                Transitivity (ours) & 68.93 & \textbf{66.54} & 76.20 & 71.04 \\ 
                Combined (ours) & \textbf{70.77} & 65.43 & \textbf{88.07} & \textbf{75.08} \\ 
                \bottomrule
            \end{tabular}
            \caption{GQA}
        \end{subtable} \\ \\

        % Second row: MMRel and Average subtables
        \begin{subtable}[h]{0.65\linewidth}
            \centering
            \begin{tabular}{l c c c c}
                \toprule
                \textbf{Methods} & \textbf{Acc} & \textbf{Precision} & \textbf{Recall} & \textbf{F1} \\
                \midrule
                Baseline (vanilla) & 70.96 & 65.72 & 87.67 & 75.12 \\ 
                Baseline (CoT+structure) & 80.00 & 76.04 & 87.60 & 81.41 \\ 
                Bidirectional (ours) & 88.50 & 84.57 & 94.20 & 89.12 \\ 
                Transitivity (ours) & 84.03 & 87.80 & 79.07 & 83.19 \\ 
                Combined (ours) & \textbf{92.70} & \textbf{90.37} & \textbf{95.60} & \textbf{92.91} \\ 
                \bottomrule
            \end{tabular}
            \caption{MMRel}
        \end{subtable}
        \hspace{0.5cm} % Increase the horizontal gap here
            
        \begin{subtable}[h]{0.65\linewidth}
            \centering
            \begin{tabular}{l c c c c}
                \toprule
                \textbf{Methods} & \textbf{Acc} & \textbf{Precision} & \textbf{Recall} & \textbf{F1} \\
                \midrule
                Baseline (vanilla) & 66.37 & 62.24 & 83.22 & 71.20 \\ 
                Baseline (CoT+structure) & 71.02 & 66.89 & 84.73 & 74.65 \\ 
                Bidirectional (ours) & 77.77 & 72.63 & 91.13 & 80.67 \\ 
                Transitivity (ours) & 75.62 & 74.95 & 79.16 & 76.71 \\ 
                Combined (ours) & \textbf{80.05} & \textbf{75.52} & \textbf{91.51} & \textbf{82.52} \\ 
                \bottomrule
            \end{tabular}
            \caption{Average}
        \end{subtable}
    \end{tabular}
}
\caption{Full results of our methods (bidirectional, transitivity, and combined constraints) on three datasets using GPT-4o. This is the extended data of Table~\ref{tab:main_results}.}
\label{tab:main_results_full}
\end{table*}

\begin{figure*}[h]
    \centering
    \includegraphics[width=\textwidth]{figures/chart_f1.pdf}
    \caption{The F1 score comparison of different relation analysis choices in bidirectional and combined constraints is shown.}
    \label{fig:relation_f1}
\end{figure*}

\begin{table*}[h]
\centering
\scalebox{0.72}{
    \begin{tabular}{cc}
        % First row: ARO and GQA subtables
        \begin{subtable}[h]{0.65\linewidth}
            \centering
            \begin{tabular}{l c c c c}
                \toprule
                \textbf{Relations} & \textbf{Acc} & \textbf{Precision} & \textbf{Recall} & \textbf{F1} \\
                \midrule
                AB & 71.70 & 65.38 & 92.27 & 76.53 \\ 
                BA & 74.67 & 68.54 & 91.20 & 78.26 \\ 
                AB+BA & 71.77 & 65.38 & 92.53 & 76.62 \\ 
                BA+AB & \textbf{75.33} & \textbf{68.74} & \textbf{92.93} & \textbf{79.03} \\ 
                \bottomrule
            \end{tabular}
            \caption{ARO}
        \end{subtable}
        \hspace{-2cm}
        
        \begin{subtable}[h]{0.65\linewidth}
            \centering
            \begin{tabular}{l c c c c}
                \toprule
                \textbf{Relations} & \textbf{Acc} & \textbf{Precision} & \textbf{Recall} & \textbf{F1} \\
                \midrule
                AB & 68.03 & 63.09 & 86.93 & 73.11 \\ 
                BA & \textbf{69.87} & \textbf{64.84} & 86.80 & \textbf{74.23} \\ 
                AB+BA & 68.07 & 62.85 & \textbf{88.22} & 73.46 \\ 
                BA+AB & 69.47 & 64.57 & 86.27 & 73.86 \\ 
                \bottomrule
            \end{tabular}
            \caption{GQA}
        \end{subtable} \\ \\

        % Second row: MMRel and Average subtables
        \begin{subtable}[h]{0.65\linewidth}
            \centering
            \begin{tabular}{l c c c c}
                \toprule
                \textbf{Relations} & \textbf{Acc} & \textbf{Precision} & \textbf{Recall} & \textbf{F1} \\
                \midrule
                AB & 86.50 & 82.97 & 91.87 & 87.10 \\ 
                BA & \textbf{89.10} & \textbf{85.97} & 93.47 & \textbf{89.56} \\ 
                AB+BA & 87.20 & 83.22 & 93.20 & 87.92 \\ 
                BA+AB & 88.50 & 84.57 & \textbf{94.20} & 89.12 \\ 
                \bottomrule
            \end{tabular}
            \caption{MMRel}
        \end{subtable}
        \hspace{-2cm}
        
        \begin{subtable}[h]{0.65\linewidth}
            \centering
            \begin{tabular}{l c c c c}
                \toprule
                \textbf{Relations} & \textbf{Acc} & \textbf{Precision} & \textbf{Recall} & \textbf{F1} \\
                \midrule
                AB & 75.41 & 70.48 & 90.36 & 78.91 \\ 
                BA & \textbf{77.88} & \textbf{73.12} & 90.49 & \textbf{80.68} \\ 
                AB+BA & 75.68 & 70.48 & \textbf{91.32} & 79.33 \\ 
                BA+AB & 77.77 & 72.63 & 91.13 & 80.67 \\ 
                \bottomrule
            \end{tabular}
            \caption{Average}
        \end{subtable}
    \end{tabular}
}
\caption{Full results of the comparison between different relation choices in the bidirectional constraints.}
\label{tab:bidirection_relations_full}
\end{table*}




\begin{table*}[h]
\centering
\scalebox{0.72}{
    \begin{tabular}{cc} % 2x2 grid
        % First row: ARO and GQA subtables
        \begin{subtable}[h]{0.65\linewidth}
            \centering
            \begin{tabular}{l c c c c}
                \toprule
                \textbf{Relations} & \textbf{Acc} & \textbf{Precision} & \textbf{Recall} & \textbf{F1} \\
                \midrule
                AB & 77.00 & \textbf{71.77} & 89.00 & 79.46 \\ 
                BA & 75.83 & 71.23 & 86.67 & 78.20 \\ 
                AB+BA & \textbf{77.53} & 71.60 & \textbf{91.27} & \textbf{80.25} \\ 
                BA+AB & 76.67 & 70.77 & 90.87 & 79.57 \\ 
                \bottomrule
            \end{tabular}
            \caption{ARO}
        \end{subtable}
        \hspace{-2cm} % Increase the horizontal gap here
        
        \begin{subtable}[h]{0.65\linewidth}
            \centering
            \begin{tabular}{l c c c c}
                \toprule
                \textbf{Relations} & \textbf{Acc} & \textbf{Precision} & \textbf{Recall} & \textbf{F1} \\
                \midrule
                AB & 68.17 & 63.59 & 85.00 & 72.75 \\ 
                BA & 70.67 & \textbf{66.76} & 82.33 & 73.73 \\ 
                AB+BA & 69.03 & 64.15 & 86.33 & 73.60 \\ 
                BA+AB & \textbf{70.77} & 65.43 & \textbf{88.07} & \textbf{75.08} \\ 
                \bottomrule
            \end{tabular}
            \caption{GQA}
        \end{subtable} \\ \\

        % Second row: MMRel and Average subtables
        \begin{subtable}[h]{0.65\linewidth}
            \centering
            \begin{tabular}{l c c c c}
                \toprule
                \textbf{Relations} & \textbf{Acc} & \textbf{Precision} & \textbf{Recall} & \textbf{F1} \\
                \midrule
                AB & 88.50 & 90.81 & 85.67 & 88.16 \\ 
                BA & 87.17 & 89.12 & 84.67 & 86.84 \\ 
                AB+BA & 90.83 & \textbf{90.97} & 90.67 & 90.82 \\ 
                BA+AB & \textbf{92.70} & 90.37 & \textbf{95.60} & \textbf{92.91} \\ 
                \bottomrule
            \end{tabular}
            \caption{MMRel}
        \end{subtable}
        \hspace{-2cm} % Increase the horizontal gap here
            
        \begin{subtable}[h]{0.65\linewidth}
            \centering
            \begin{tabular}{l c c c c}
                \toprule
                \textbf{Relations} & \textbf{Acc} & \textbf{Precision} & \textbf{Recall} & \textbf{F1} \\
                \midrule
                AB & 77.89 & 75.39 & 86.56 & 80.12 \\ 
                BA & 77.89 & \textbf{75.70} & 84.56 & 79.59 \\ 
                AB+BA & 79.13 & 75.57 & 89.42 & 81.56 \\ 
                BA+AB & \textbf{80.05} & 75.52 & \textbf{91.51} & \textbf{82.52} \\ 
                \bottomrule
            \end{tabular}
            \caption{Average}
        \end{subtable}
    \end{tabular}
}
\caption{Full results of the comparison between different relation choices in the combined constraints.}
\label{tab:combined_relations_full}
\end{table*}

\begin{table*}[h]
\centering
\scalebox{0.72}{
    \begin{tabular}{cc} % 2x2 grid
        % First row: ARO and GQA subtables
        \begin{subtable}[h]{0.65\linewidth}
            \centering
            \begin{tabular}{l c c c c}
                \toprule
                \textbf{Attributes} & \textbf{Acc} & \textbf{Precision} & \textbf{Recall} & \textbf{F1} \\
                \midrule
                The largest & \textbf{75.00} & 71.43 & 83.33 & 76.92 \\ 
                The smallest & 73.33 & 69.34 & 83.67 & 75.83 \\ 
                The most top & 72.17 & 69.97 & 77.67 & 73.62 \\ 
                The central & \textbf{75.00} & \textbf{71.68} & 82.67 & 76.78 \\ 
                The most obvious & \textbf{75.00} & 71.19 & \textbf{84.00} & \textbf{77.06} \\ 
                Random & 73.90 & 70.50 & 82.20 & 75.89 \\ 
                \bottomrule
            \end{tabular}
            \caption{ARO}
        \end{subtable}
        \hspace{-0.7cm} % Increase the horizontal gap here
        
        \begin{subtable}[h]{0.65\linewidth}
            \centering
            \begin{tabular}{l c c c c}
                \toprule
                \textbf{Attributes} & \textbf{Acc} & \textbf{Precision} & \textbf{Recall} & \textbf{F1} \\
                \midrule
                The largest & 65.00 & 63.01 & 72.67 & 67.49 \\ 
                The smallest & 66.00 & 64.29 & 72.00 & 67.92 \\ 
                The most top & 65.50 & 63.48 & 73.00 & 67.91 \\ 
                The central & \textbf{69.17} & \textbf{67.27} & 74.67 & 70.77 \\ 
                The most obvious & 67.33 & 65.29 & 74.00 & 69.38 \\ 
                Random & 68.93 & 66.54 & \textbf{76.20} & \textbf{71.04} \\ 
                \bottomrule
            \end{tabular}
            \caption{GQA}
        \end{subtable} \\ \\

        % Second row: MMRel and Average subtables
        \begin{subtable}[h]{0.65\linewidth}
            \centering
            \begin{tabular}{l c c c c}
                \toprule
                \textbf{Attributes} & \textbf{Acc} & \textbf{Precision} & \textbf{Recall} & \textbf{F1} \\
                \midrule
                The largest & 83.83 & 86.38 & \textbf{80.33} & \textbf{83.25} \\ 
                The smallest & 84.00 & \textbf{88.64} & 78.00 & 82.98 \\ 
                The most top & 81.67 & 84.67 & 77.33 & 80.84 \\ 
                The central & 83.00 & 88.37 & 76.00 & 81.72 \\ 
                The most obvious & 81.83 & 86.04 & 76.00 & 80.71 \\ 
                Random & \textbf{84.03} & 87.80 & 79.07 & 83.19 \\ 
                \bottomrule
            \end{tabular}
            \caption{MMRel}
        \end{subtable}
        \hspace{-0.7cm} % Increase the horizontal gap here
            
        \begin{subtable}[h]{0.65\linewidth}
            \centering
            \begin{tabular}{l c c c c}
                \toprule
                \textbf{Attributes} & \textbf{Acc} & \textbf{Precision} & \textbf{Recall} & \textbf{F1} \\
                \midrule
                \text{The largest} & 74.61 & 73.61 & 78.78 & 75.89 \\ 
                \text{The smallest} & 74.44 & 74.76 & 77.22 & 75.58 \\ 
                \text{The most top} & 73.11 & 72.71 & 76.00 & 74.12 \\ 
                \text{The central} & \textbf{75.72} & \textbf{75.77} & 77.78 & 76.42 \\ 
                \text{The most obvious} & 74.72 & 74.84 & 78.00 & 75.72 \\ 
                \text{Random} & 75.62 & 74.95 & \textbf{79.16} & \textbf{76.71} \\ 
                \bottomrule
            \end{tabular}
            \caption{Average}
        \end{subtable}
    \end{tabular}
}
\caption{Full results of the comparison of different reference object selection strategies in transitivity constraints. This is the extended data of Table~\ref{tab:analysis1}.}
\label{tab:analysis1_full}
\end{table*}

\begin{table*}[h]
\centering
\scalebox{0.72}{
    \begin{tabular}{cc} % 2x2 grid
        % First row: ARO and GQA subtables
        \begin{subtable}[h]{0.65\linewidth}
            \centering
            \begin{tabular}{l c c c c}
                \toprule
                \textbf{Attributes} & \textbf{Acc} & \textbf{Precision} & \textbf{Recall} & \textbf{F1} \\
                \midrule
                The largest & 77.00 & 71.66 & 89.33 & 79.53 \\ 
                The smallest & 75.83 & 69.82 & 91.00 & 79.02 \\ 
                The most top & 76.17 & 70.18 & 91.00 & 79.25 \\ 
                The central & \textbf{78.17} & \textbf{72.06} & \textbf{92.00} & \textbf{80.82} \\ 
                The most obvious & 77.67 & 71.61 & 91.67 & 80.41 \\ 
                Random & 76.67 & 70.77 & 90.87 & 79.57 \\ 
                \bottomrule
            \end{tabular}
            \caption{ARO}
        \end{subtable}
        \hspace{-0.7cm} % Decrease the horizontal gap here
        
        \begin{subtable}[h]{0.65\linewidth}
            \centering
            \begin{tabular}{l c c c c}
                \toprule
                \textbf{Attributes} & \textbf{Acc} & \textbf{Precision} & \textbf{Recall} & \textbf{F1} \\
                \midrule
                The largest & 70.50 & 65.04 & \textbf{88.67} & 75.04 \\ 
                The smallest & 69.67 & 64.39 & 88.00 & 74.37 \\ 
                The most top & 68.83 & 63.95 & 86.33 & 73.48 \\ 
                The central & 68.83 & 63.81 & 87.00 & 73.62 \\ 
                The most obvious & 69.67 & 64.82 & 86.00 & 73.93 \\ 
                Random & \textbf{70.77} & \textbf{65.43} & 88.07 & \textbf{75.08} \\ 
                \bottomrule
            \end{tabular}
            \caption{GQA}
        \end{subtable} \\ \\

        % Second row: MMRel and Average subtables
        \begin{subtable}[h]{0.65\linewidth}
            \centering
            \begin{tabular}{l c c c c}
                \toprule
                \textbf{Attributes} & \textbf{Acc} & \textbf{Precision} & \textbf{Recall} & \textbf{F1} \\
                \midrule
                The largest & 91.67 & 89.56 & 94.33 & 91.88 \\ 
                The smallest & 91.17 & 88.96 & 94.00 & 91.41 \\ 
                The most top & 92.33 & 89.44 & \textbf{96.00} & 92.60 \\ 
                The central & 90.67 & 88.61 & 93.33 & 90.91 \\ 
                The most obvious & 92.33 & \textbf{90.71} & 94.33 & 92.48 \\ 
                Random & \textbf{92.70} & 90.37	& 95.60 & \textbf{92.91} \\ 
                \bottomrule
            \end{tabular}
            \caption{MMRel}
        \end{subtable}
        \hspace{-0.7cm} % Decrease the horizontal gap here
            
        \begin{subtable}[h]{0.65\linewidth}
            \centering
            \begin{tabular}{l c c c c}
                \toprule
                \textbf{Attributes} & \textbf{Acc} & \textbf{Precision} & \textbf{Recall} & \textbf{F1} \\
                \midrule
                \text{The largest} & 79.72 & 75.42 & 90.78 & 82.15 \\ 
                \text{The smallest} & 78.89 & 74.39 & 91.00 & 81.60 \\ 
                \text{The most top} & 79.11 & 74.52 & 91.11 & 81.78 \\ 
                \text{The central} & 79.22 & 74.16 & 90.78 & 81.78 \\ 
                \text{The most obvious} & 79.89 & 75.05 & 90.67 & 82.27 \\ 
                \text{Random} & \textbf{80.05} & \textbf{75.52} & \textbf{91.51} & \textbf{82.52} \\ 
                \bottomrule
            \end{tabular}
            \caption{Average}
        \end{subtable}
    \end{tabular}
}
\caption{Full results of the comparison of different reference object selection strategies in combined constraints. This is the extended data of Table~\ref{tab:analysis2}.}
\label{tab:analysis2_full}
\end{table*}

This section presents the full results, including accuracy, precision, recall, and F1 score, for all experiments conducted in the body of the paper.

The comprehensive results of the main experiment, which include a comparison of our methods against baseline approaches across four metrics, are presented in Table~\ref{tab:main_results_full}. In addition to evaluating the accuracy of various bidirectional relation analysis options, as discussed in Section~\ref{sec:analysis1}, we also assess the F1 scores for different analytical choices, as shown in Figure~\ref{fig:relation_f1}. The F1 score comparison adheres to the same criteria as the accuracy evaluations, and the results follow similar patterns. Detailed data on the relation analysis options can be found in Tables~\ref{tab:bidirection_relations_full} and~\ref{tab:combined_relations_full}. Furthermore, extended results on reference object selection analysis is available in Tables~\ref{tab:analysis1_full} and~\ref{tab:analysis2_full}.


\subsection{Failed Case Analysis}
\label{sec:appendix6}
\begin{figure*}[h]
    \centering
    \begin{overpic}[width=1\linewidth]{figures/failed1.png}
        \put(3,36){\textcolor{black}{
        \begin{minipage}{0.91\linewidth}
            \fontsize{8}{12}\selectfont
            \textbf{Q:}
            \\\#\# Instructions \#\#
            \\1. Repeat the question and then extract the objects mentioned in the question. Label the first object that appears as "Object A" and the second as "Object B".
            \\2. Describe the relative position between Object B and A and between Object A and B.
            \\3. Reference to the relationship between Object B and A and between Object A and B, and then think step by step to use 'yes' or 'no' to answer the question.
            \\
            \\\#\# Please output in the following format \#\#
            \\Question:
            \\Object A:
            \\Object B:
            \\Horizontal relation between Object B and A: B is <relation> A
            \\Vertical relation between Object B and A: B is <relation> A
            \\Depth relation between Object B and A: B is <relation> A
            \\Horizontal relation between Object A and B: A is <relation> B
            \\Vertical relation between Object A and B: A is <relation> B
            \\Depth relation between Object A and B: A is <relation> B
            \\Thinking process: 
            \\Answer: 
            \\
            \\\#\# Question \#\#
            \\Is there a building to the right of the horses in the image?
            \\
            \\\textbf{A:}
            \\Question: Is there a building to the right of the horses in the image?
            \\Object A: horses
            \\Object B: building
            \\Horizontal relation between Object B and A: B is to the right of A
            \\Vertical relation between Object B and A: B is behind A
            \\Depth relation between Object B and A: B is behind A
            \\Horizontal relation between Object A and B: A is to the left of B
            \\Vertical relation between Object A and B: A is in front of B
            \\Depth relation between Object A and B: A is in front of B
            \\Thinking process: The building is positioned to the right and behind the horses in the image.
            \\Answer: Yes
        \end{minipage}}}
    \end{overpic}
    \caption{Failed case 1.} 
    \label{fig:failed1}
\end{figure*}

\begin{figure*}[h]
    \centering
    \begin{overpic}[width=1\linewidth]{figures/failed2.png}
        \put(3,39){\textcolor{black}{
        \begin{minipage}{0.91\linewidth}
            \fontsize{8}{12}\selectfont
            \textbf{Q:}
            \\\#\# Instructions \#\#
            \\1. Repeat the question and then extract the objects mentioned in the question. Label the first object that appears as "Object A" and the second as "Object B". Select an object different from Object A or Object B in the image as "Object C"
            \\2. Describe the relative position between Object A and C and between Object B and C.
            \\3. Reference to the relationship between Object A and C and between Object B and C, and then think step by step to use 'yes' or 'no' to answer the question.
            \\
            \\\#\# Please output in the following format \#\#
            \\Question:
            \\Object A:
            \\Object B:
            \\Object C:
            \\Horizontal relation between Object A and C: A is <relation> C
            \\Vertical relation between Object A and C: A is <relation> C
            \\Depth relation between Object A and C: A is <relation> C
            \\Horizontal relation between Object B and C: B is <relation> C
            \\Vertical relation between Object B and C: B is <relation> C
            \\Depth relation between Object B and C: B is <relation> C
            \\Thinking process: 
            \\Answer: 
            \\
            \\\#\# Question \#\#
            \\Is there a cabinet to the left of the wall in the image?
            \\
            \\\textbf{A:}
            \\Question: Is there a cabinet to the left of the wall in the image?
            \\Object A: cabinet
            \\Object B: wall
            \\Object C: refrigerator
            \\Horizontal relation between Object A and C: A is to the left of C
            \\Vertical relation between Object A and C: A is above C
            \\Depth relation between Object A and C: A is in front of C
            \\Horizontal relation between Object B and C: B is to the left of C
            \\Vertical relation between Object B and C: B is above C
            \\Depth relation between Object B and C: B is in front of C
            \\Thinking process: The cabinet (Object A) is to the left of the refrigerator (Object C), and the wall (Object B) is also to the left of the refrigerator (Object C). Since both the cabinet and the wall are to the left of the refrigerator, it can be concluded that the cabinet is to the left of the wall.
            \\Answer: Yes
        \end{minipage}}}
    \end{overpic}
    \caption{Failed case 2.} 
    \label{fig:failed2}
\end{figure*}


Although our methods significantly improve the accuracy of LVLMs in reasoning about visual spatial relations, some failure cases persist. Figure~\ref{fig:failed1} illustrates a typical example. In this case, the model uses bidirectional constraints to determine the spatial relation between a building and a horse. The first issue is that GPT fails to label the initially mentioned object, the building, as "Object A," instead switching the labels of the building and the horse. However, this is not the critical error. The main issue lies in GPT's inability to correctly identify the spatial relation from the horse's perspective, incorrectly stating that the building is to the right of the horse. A secondary mistake, though not impacting the final answer, is the use of "front" and "behind" to describe a vertical relationship. This analysis suggests that if the candidate LVLM has inherent misunderstandings or hallucinations regarding the definition of spatial relations, correcting them may be challenging. 

Figure~\ref{fig:failed2} presents a typical failure case involving transitivity constraints. In this instance, GPT applies transitivity constraints to determine the horizontal relationship between a cabinet and a wall, using the refrigerator as a reference object. While the LVLM correctly identifies that "the cabinet is to the left of the refrigerator" and "the wall is to the left of the refrigerator," these spatial relations do not contribute to answering the question accurately. As a result, the final response is still hallucinated.


\subsection{Output Examples}
\label{sec:appendix5}
Figures~\ref{fig:example1} and~\ref{fig:example2} demonstrate the example outputs of bidirectional and transitivity constraints respectively.

\begin{figure*}[h]
    \centering
    \begin{overpic}[width=1\linewidth]{figures/example1.png}
        \put(3,36){\textcolor{black}{
        \begin{minipage}{0.91\linewidth}
            \fontsize{8}{12}\selectfont
            \textbf{Q:}
            \\\#\# Instructions \#\#
            \\1. Repeat the question and then extract the objects mentioned in the question. Label the first object that appears as "Object A" and the second as "Object B".
            \\2. Describe the relative position between Object B and A and between Object A and B.
            \\3. Reference to the relationship between Object B and A and between Object A and B, and then think step by step to use 'yes' or 'no' to answer the question.
            \\
            \\\#\# Please output in the following format \#\#
            \\Question:
            \\Object A:
            \\Object B:
            \\Horizontal relation between Object B and A: B is <relation> A
            \\Vertical relation between Object B and A: B is <relation> A
            \\Depth relation between Object B and A: B is <relation> A
            \\Horizontal relation between Object A and B: A is <relation> B
            \\Vertical relation between Object A and B: A is <relation> B
            \\Depth relation between Object A and B: A is <relation> B
            \\Thinking process: 
            \\Answer: 
            \\
            \\\#\# Question \#\#
            \\Is there a can left the carrot in the image?
            \\
            \\\textbf{A:}
            \\Question: Is there a can left the carrot in the image?
            \\Object A: can
            \\Object B: carrot
            \\Horizontal relation between Object B and A: B is right of A
            \\Vertical relation between Object B and A: B is at the same level as A
            \\Depth relation between Object B and A: B is in front of A
            \\Horizontal relation between Object A and B: A is left of B
            \\Vertical relation between Object A and B: A is at the same level as B
            \\Depth relation between Object A and B: A is behind B
            \\Thinking process: The carrot is to the right of the can, at the same level, and in front of it. Therefore, the can is to the left of the carrot, at the same level, and behind it.
            \\Answer: Yes
        \end{minipage}}}
    \end{overpic}
    \caption{Example output of the bidirectional constraint.} 
    \label{fig:example1}
\end{figure*}

\begin{figure*}[h]
    \centering
    \begin{overpic}[width=1\linewidth]{figures/example2.png}
        \put(3,38){\textcolor{black}{
        \begin{minipage}{0.91\linewidth}
            \fontsize{8}{12}\selectfont
            \textbf{Q:}
            \\\#\# Instructions \#\#
            \\1. Repeat the question and then extract the objects mentioned in the question. Label the first object that appears as "Object A" and the second as "Object B". Select an object different from Object A or Object B in the image as "Object C"
            \\2. Describe the relative position between Object A and C and between Object B and C.
            \\3. Reference to the relationship between Object A and C and between Object B and C, and then think step by step to use 'yes' or 'no' to answer the question.
            \\
            \\\#\# Please output in the following format \#\#
            \\Question:
            \\Object A:
            \\Object B:
            \\Object C:
            \\Horizontal relation between Object A and C: A is <relation> C
            \\Vertical relation between Object A and C: A is <relation> C
            \\Depth relation between Object A and C: A is <relation> C
            \\Horizontal relation between Object B and C: B is <relation> C
            \\Vertical relation between Object B and C: B is <relation> C
            \\Depth relation between Object B and C: B is <relation> C
            \\Thinking process: 
            \\Answer: 
            \\
            \\\#\# Question \#\#
            \\Is there a window to the right of the tree in the image?
            \\
            \\\textbf{A:}
            \\Question: Is there a window to the right of the tree in the image?
            \\Object A: window
            \\Object B: tree
            \\Object C: bicycle
            \\Horizontal relation between Object A and C: A is left of C
            \\Vertical relation between Object A and C: A is above C
            \\Depth relation between Object A and C: A is behind C
            \\Horizontal relation between Object B and C: B is right of C
            \\Vertical relation between Object B and C: B is above C
            \\Depth relation between Object B and C: B is behind C
            \\Thinking process: The window (Object A) is to the left of the bicycle (Object C), and the tree (Object B) is to the right of the bicycle (Object C). Therefore, the window is to the left of the tree.
            \\Answer: No
        \end{minipage}}}
    \end{overpic}
    \caption{Example output of the transitivity constraint.} 
    \label{fig:example2}
\end{figure*}

\subsection{Template Prompts}
\label{sec:appendix7}
The template prompts utilizing bidirectional constraints, transitivity constraints, and combined constraints can be found in Figures~\ref{fig:prompt1},~\ref{fig:prompt2}, and~\ref{fig:prompt3} respectively. The template prompt of vanilla baseline is in Figure~\ref{fig:prompt4} and that of CoT+structure baseline is in Figure~\ref{fig:prompt5}. The template prompts used in the analysis of different reference selection strategies are in Figures~\ref{fig:prompt6} and~\ref{fig:prompt7}.

\begin{figure*}[h]
    \centering
    \begin{overpic}[width=0.5\linewidth]{figures/prompt2.png}
        \put(3,49){\textcolor{black}{
        \begin{minipage}{0.455\linewidth}
            \fontsize{8}{12}\selectfont
            \textbf{\#\# Instructions \#\#}
            \\1. Repeat the question and then extract the objects mentioned in the question. Label the first object that appears as "Object A" and the second as "Object B".
            \\2. Describe the relative position \textcolor{red}{between Object B and A and between Object A and B.}
            \\3. Reference to the relationship between Object B and A and between Object A and B, and then \textcolor{blue}{think step by step} to use "yes" or "no" to answer the question.
            \\\textbf{\#\# Please output in the following format \#\#}
            \\Question:
            \\Object A:
            \\Object B:
            \\\textcolor{blue}{Horizontal relation} between Object B and A: B is <relation> A
            \\\textcolor{blue}{Vertical relation} between Object B and A: B is <relation> A
            \\\textcolor{blue}{Depth relation} between Object B and A: B is <relation> A
            \\\textcolor{blue}{Horizontal relation} between Object A and B: A is <relation> B
            \\\textcolor{blue}{Vertical relation} between Object A and B: A is <relation> B
            \\\textcolor{blue}{Depth relation} between Object A and B: A is <relation> B
            \\Thinking process: 
            \\Answer: 
            \\\textbf{\#\# Question \#\#}
            \\\{question\}
        \end{minipage}}}
    \end{overpic}
    \caption{Template prompt utilizing the bidirectional constraint. Prompting techniques are highlighted in blue. Terms related to the \textit{BA + AB} order are marked in red.} 
    \label{fig:prompt1}
\end{figure*}


\begin{figure*}[h]
    \centering
    \begin{overpic}[width=0.5\linewidth]{figures/prompt1.png}
        \put(3,49){\textcolor{black}{
        \begin{minipage}{0.455\linewidth}
            \fontsize{8}{12}\selectfont
            \textbf{\#\# Instructions \#\#}
            \\1. Repeat the question and then extract the objects mentioned in the question. Label the first object that appears as "Object A" and the second as "Object B". Select an object different from Object A or Object B in the image as "Object C".
            \\2. Describe the relative position \textcolor{red}{between Object A and C and between Object B and C.}
            \\3. Reference to the relationship between Object A and C and between Object B and C, and then \textcolor{blue}{think step by step} to use "yes" or "no" to answer the question.
            \\\textbf{\#\# Please output in the following format \#\#}
            \\Question:
            \\Object A:
            \\Object B:
            \\Object C:
            \\\textcolor{blue}{Horizontal relation} between Object A and C: A is <relation> C
            \\\textcolor{blue}{Vertical relation} between Object A and C: A is <relation> C
            \\\textcolor{blue}{Depth relation} between Object A and C: A is <relation> C
            \\\textcolor{blue}{Horizontal relation} between Object B and C: B is <relation> C
            \\\textcolor{blue}{Vertical relation} between Object B and C: B is <relation> C
            \\\textcolor{blue}{Depth relation} between Object B and C: B is <relation> C
            \\Thinking process: 
            \\Answer: 
            \\\textbf{\#\# Question \#\#}
            \\\{question\}
        \end{minipage}}}
    \end{overpic}
    \caption{Template prompt utilizing the transitivity constraint. Prompting techniques are highlighted in blue, and terms relevant to the reference relations (\textit{AC + BC} order) are highlighted in red.} 
    \label{fig:prompt2}
\end{figure*}

\begin{figure*}[h]
    \centering
    \begin{overpic}[width=0.5\linewidth]{figures/prompt3.png}
        \put(2,49){\textcolor{black}{
        \begin{minipage}{0.455\linewidth}
            \fontsize{8}{12}\selectfont
            \textbf{\#\# Instructions \#\#}
            \\1. Repeat the question and then extract the objects mentioned in the question. Label the first object that appears as "Object A" and the second as "Object B". Select an object different from Object A or Object B in the image as "Object C"
            \\2. Describe the relative position \textcolor{red}{between Object A and C and between Object B and C.}
            \\3. Reference to the result of step 2 and image, describe the relative position \textcolor{orange}{between Object B and A and between Object A and B.}
            \\4. Reference to the relationship between Object B and A and between Object A and B, and then \textcolor{blue}{think step by step} to use "yes" or "no" to answer the question.
            \\\textbf{\#\# Please output in the following format \#\#}
            \\Question:
            \\Object A:
            \\Object B:
            \\Object C:
            \\\textcolor{blue}{Horizontal relation} between Object A and C: A is <relation> C
            \\\textcolor{blue}{Vertical relation} between Object A and C: A is <relation> C
            \\\textcolor{blue}{Depth relation} between Object A and C: A is <relation> C
            \\\textcolor{blue}{Horizontal relation} between Object B and C: B is <relation> C
            \\\textcolor{blue}{Vertical relation} between Object B and C: B is <relation> C
            \\\textcolor{blue}{Depth relation} between Object B and C: B is <relation> C
            \\\textcolor{blue}{Horizontal relation} between Object B and A: B is <relation> A
            \\\textcolor{blue}{Vertical relation} between Object B and A: B is <relation> A
            \\\textcolor{blue}{Depth relation} between Object B and A: B is <relation> A
            \\\textcolor{blue}{Horizontal relation} between Object A and B: A is <relation> B
            \\\textcolor{blue}{Vertical relation} between Object A and B: A is <relation> B
            \\\textcolor{blue}{Depth relation} between Object A and B: A is <relation> B
            \\Thinking process: 
            \\Answer: 
            \\\textbf{\#\# Question \#\#}
            \\\{question\}
        \end{minipage}}}
    \end{overpic}
    \caption{Template prompt utilizing the combined constraint. Prompting techniques are highlighted in blue, terms relevant to the transitivity constraint are highlighted in red, and terms relevant to the bidirectional constraint are highlighted in orange.} 
    \label{fig:prompt3}
\end{figure*}

\begin{figure*}[h]
    \centering
    \begin{overpic}[width=0.5\linewidth]{figures/prompt5.png}
        \put(4,4){\textcolor{black}{
        \begin{minipage}{0.455\linewidth}
            \fontsize{8}{12}\selectfont
            Use 'yes' or 'no' to answer the question: \{question\}
        \end{minipage}}}
    \end{overpic}
    \caption{Template prompt of Vanilla Baseline.} 
    \label{fig:prompt4}
\end{figure*}

\begin{figure*}[t]
    \centering
    \begin{overpic}[width=0.5\linewidth]{figures/prompt4.png}
        \put(4,38){\textcolor{black}{
        \begin{minipage}{0.455\linewidth}
            \fontsize{8}{12}\selectfont
            \textbf{\#\# Instructions \#\#}
            \\1. Repeat the question and then extract the objects mentioned in the question. Label the first object that appears as "Object A" and the second as "Object B".
            \\2. Think step by step to use 'yes' or 'no' to answer the question.
            \\\textbf{\#\# Please output in the following format \#\#}
            \\Question:
            \\Object A:
            \\Object B:
            \\Thinking process: 
            \\Answer: 
            \\\textbf{\#\# Question \#\#}
            \\\{question\}
        \end{minipage}}}
    \end{overpic}
    \caption{Template prompt of CoT+Structure Baseline.} 
    \label{fig:prompt5}
\end{figure*}

\begin{figure*}[h]
    \centering
    \begin{overpic}[width=0.5\linewidth]{figures/prompt1.png}
        \put(3,49){\textcolor{black}{
        \begin{minipage}{0.455\linewidth}
            \fontsize{8}{12}\selectfont
            \textbf{\#\# Instructions \#\#}
            \\1. Repeat the question and then extract the objects mentioned in the question. Label the first object that appears as "Object A" and the second as "Object B". Select \textcolor{blue}{\{attribute\}} object different from Object A or Object B in the image as "Object C"
            \\2. Describe the relative position between Object A and C and between Object B and C.
            \\3. Reference to the relationship between Object A and C and between Object B and C, and then think step by step to use 'yes' or 'no' to answer the question.
            \\\textbf{\#\# Please output in the following format \#\#}
            \\Question:
            \\Object A:
            \\Object B:
            \\Object C:
            \\Horizontal relation between Object A and C: A is <relation> C
            \\Vertical relation between Object A and C: A is <relation> C
            \\Depth relation between Object A and C: A is <relation> C
            \\Horizontal relation between Object B and C: B is <relation> C
            \\Vertical relation between Object B and C: B is <relation> C
            \\Depth relation between Object B and C: B is <relation> C
            \\Thinking process: 
            \\Answer: 
            \\\textbf{\#\# Question \#\#}
            \\\{question\}
        \end{minipage}}}
    \end{overpic}
    \caption{Template prompt used in the reference  selection analysis of transitivity constraints. We replace \{attribute\} with the candidate attributes, such as "the largest" and "the most top."} 
    \label{fig:prompt6}
\end{figure*}

\begin{figure*}[h]
    \centering
    \begin{overpic}[width=0.5\linewidth]{figures/prompt3.png}
        \put(2,49){\textcolor{black}{
        \begin{minipage}{0.455\linewidth}
            \fontsize{8}{12}\selectfont
            \textbf{\#\# Instructions \#\#}
            \\1. Repeat the question and then extract the objects mentioned in the question. Label the first object that appears as "Object A" and the second as "Object B". Select \textcolor{blue}{\{attribute\}} object different from Object A or Object B in the image as "Object C"
            \\2. Describe the relative position between Object A and C and between Object B and C.
            \\3. Reference to the result of step 2 and image, describe the relative position between Object B and A and between Object A and B.
            \\4. Reference to the relationship between Object B and A and between Object A and B, and then think step by step to use 'yes' or 'no' to answer the question.
            \\\textbf{\#\# Please output in the following format \#\#}
            \\Question:
            \\Object A:
            \\Object B:
            \\Object C:
            \\Horizontal relation between Object A and C: A is <relation> C
            \\Vertical relation between Object A and C: A is <relation> C
            \\Depth relation between Object A and C: A is <relation> C
            \\Horizontal relation between Object B and C: B is <relation> C
            \\Vertical relation between Object B and C: B is <relation> C
            \\Depth relation between Object B and C: B is <relation> C
            \\Horizontal relation between Object B and A: B is <relation> A
            \\Vertical relation between Object B and A: B is <relation> A
            \\Depth relation between Object B and A: B is <relation> A
            \\Horizontal relation between Object A and B: A is <relation> B
            \\Vertical relation between Object A and B: A is <relation> B
            \\Depth relation between Object A and B: A is <relation> B
            \\Thinking process: 
            \\Answer: 
            \\\textbf{\#\# Question \#\#}
            \\\{question\}
        \end{minipage}}}
    \end{overpic}
    \caption{Template prompt used in the reference  selection analysis of combined constraints.} 
    \label{fig:prompt7}
\end{figure*}




\end{document}
