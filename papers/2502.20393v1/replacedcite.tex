\section{Related Work}
\label{sec_related_work}
\textbf{Interpretability of Deep Neural Network Models}: 
Interpretability methods in DNN models can be broadly classified into post-hoc and ante-hoc methods. Post-hoc methods aim to interpret model predictions after training ____. Recent efforts have highlighted the issues with post-hoc methods and their reliability in reflecting a model's reasoning ____. On the other hand, ante-hoc methods that jointly learn to explain and predict provide models that are inherently interpretable ____. Ante-hoc methods have also been found to provide interpretatations that help make the model more robust and reliable ____. We focus on this genre of methods in this work. ____ proposed Concept Bottleneck Models (CBMs), a method that uses interpretable, human-defined concepts, combining them linearly to perform classification. CBMs also allow human interventions on concept activations ____ to steer the final prediction of a model. ____ obtained the intermediate semantic concepts by replacing domain experts with Large Language Models (LLMs). This allows for ease and flexibility in obtaining the concept set.
Using LLMs to obtain concepts also allow grounding of neurons in a bottleneck layer to a human-understandable concept, an issue with CBMs that was highlighted in ____. Other concept-based methods ____ use a different notion of concepts based on prototype representations (see appendix \S A1).

\noindent \textbf{Concept-Based Incremental Learning}:
While Incremental Learning in standard supervised settings has been widely explored ____, Concept-Based Incremental Learning has remained largely unstudied. We identify ____ as an early effort in this direction; however, this work trains CBMs in a continual setting under an assumption that all concepts, including those required for unseen classes, are accessible from the first experience itself, which does not emulate a real-world setting. More recently, ____ proposed an interpretable CL method that uses part-based prototypes as concepts. As mentioned earlier, our notion of concepts allows us to go beyond parts of an object category, as in CBM-based models.