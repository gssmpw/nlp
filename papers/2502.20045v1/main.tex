% CVPR 2025 Paper Template; see https://github.com/cvpr-org/author-kit

\documentclass[10pt,twocolumn,letterpaper]{article}

%%%%%%%%% PAPER TYPE  - PLEASE UPDATE FOR FINAL VERSION
\usepackage{cvpr}              % To produce the CAMERA-READY version
% \usepackage[review]{cvpr}      % To produce the REVIEW version
% \usepackage[pagenumbers]{cvpr} % To force page numbers, e.g. for an arXiv version

% Import additional packages in the preamble file, before hyperref
\newcommand{\CG}{\mathcal{G}\xspace}
\newcommand{\CV}{\mathcal{V}\xspace}
\newcommand{\CE}{\mathcal{E}\xspace}
\newcommand{\CA}{\mathcal{A}\xspace}
\newcommand{\CF}{\mathcal{F}\xspace}
\newcommand{\CR}{\mathcal{R}\xspace}
\newcommand{\CB}{\mathcal{B}\xspace}
\newcommand{\CX}{\mathcal{X}\xspace}
\newcommand{\CK}{\mathcal{K}\xspace}
\newcommand{\CM}{\mathcal{M}\xspace}
\newcommand{\CC}{\mathcal{C}\xspace}
\newcommand{\CL}{\mathcal{L}\xspace}
\newcommand{\CI}{\mathcal{I}\xspace}
\newcommand{\CQ}{\mathcal{Q}\xspace}
\newcommand{\CO}{\mathcal{O}\xspace}
\newcommand{\CP}{\mathcal{P}\xspace}
\newcommand{\CS}{\mathcal{S}\xspace}
\newcommand{\CT}{\mathcal{T}\xspace}
\newcommand{\CJ}{\mathcal{J}\xspace}
\usepackage[para]{footmisc}
\usepackage{subfig}
% \usepackage{subcaption}
% \usepackage{array}
% \usepackage{colortbl}



% It is strongly recommended to use hyperref, especially for the review version.
% hyperref with option pagebackref eases the reviewers' job.
% Please disable hyperref *only* if you encounter grave issues, 
% e.g. with the file validation for the camera-ready version.
%
% If you comment hyperref and then uncomment it, you should delete *.aux before re-running LaTeX.
% (Or just hit 'q' on the first LaTeX run, let it finish, and you should be clear).
\definecolor{cvprblue}{rgb}{0.21,0.49,0.74}
\usepackage[pagebackref,breaklinks,colorlinks,allcolors=cvprblue]{hyperref}
\usepackage{amsmath}
%\usepackage{algorithm}
\DeclareMathOperator*{\argmax}{arg\,max}
\DeclareMathOperator*{\argmin}{arg\,min}

%%%%%%%%% PAPER ID  - PLEASE UPDATE
\def\paperID{} % *** Enter the Paper ID here
\def\confName{ICCV}
\def\confYear{2025}
\def\qquad{\hskip2em\relax}

%%%%%%%%% TITLE - PLEASE UPDATE
\title{Text2VDM: Text to Vector Displacement Maps\\for Expressive and Interactive 3D Sculpting}

%%%%%%%%% AUTHORS - PLEASE UPDATE
\author{%First Author\\
%Institution1\\
%Institution1 address\\
%{\tt\small firstauthor@i1.org}
% For a paper whose authors are all at the same institution,
% omit the following lines up until the closing ``}''.
% Additional authors and addresses can be added with ``\and'',
% just like the second author.
% To save space, use either the email address or home page, not both
%\and
%Second Author\\
%Institution2\\
%First line of institution2 address\\
%{\tt\small secondauthor@i2.org}
Hengyu Meng$^{1}$ \qquad
Duotun Wang$^{1}$ \qquad
Zhijing Shao$^{1}$ \qquad
\\
Ligang Liu$^{2}$ \qquad
Zeyu Wang$^{1,3}$ \qquad
\\
$^{1}$The Hong Kong University of Science and Technology (Guangzhou)
\\
$^{2}$University of Science and Technology of China
\\
$^{3}$The Hong Kong University of Science and Technology
}

\begin{document}

\twocolumn[{
\maketitle
\begin{center}
    \captionsetup{type=figure}
    \includegraphics[width=1\textwidth]{sec/Figures/teaser.pdf}
    \captionof{figure}{
        \textbf{Example VDM brushes generated by Tex2VDM and sculpted models in Blender.} Tex2VDM can produce high-quality brushes for surface details (top row) and geometric structures (bottom row) from text input. Users can rapidly create an expressive model from a plain shape by directly applying these brushes in Blender. Yellow underlined text highlights semantics enhanced by our framework.
    }
    \label{Fig:teaser}
\end{center}
}]


\begin{abstract}

% Recent works to jointly reconstruct 3D human and object from a single RGB image, are mostly model-based, that fail to capture the fine details of the clothed human body and object surface. In this paper, we introduce ReCHOR, a novel, model-free, first-method to produce realistic clothed human-object reconstructions from a monocular view. This is extremely challenging due to human-object occlusions, diverse interactions and depth ambiguity, as it needs to infer both 3D spatial awareness and high resolution details. Our core idea is based on estimating neural implicit representations for human and object respectively by an attention-based neural implicit model that attends to pixel-aligned features from both the global human-object image for spatial awareness and  the local separate view of human and object images for high quality details. Additionally, the network is conditioned on semantic features from an initial estimated human-object pose prior and a generative diffusion model that inpaints occluded regions, thus enabling the retrieval of details from them.
% We also propose a synthetic dataset with rendered scenes of diverse, inter-occluded 3D human and object scans, to train our network. We evaluate our method on the synthetic and real world BEHAVE dataset. Our experiments show that our method outperforms the SOTA in achieving realistic clothed human-object reconstructions.
Recent approaches to jointly reconstruct 3D humans and objects from a single RGB image represent 3D shapes with template-based or coarse models, which fail to capture details of loose clothing on human bodies. In this paper, we introduce a novel implicit approach for jointly reconstructing realistic 3D clothed humans and objects from a monocular view. For the first time, we model both the human and the object with an implicit representation, allowing to capture more realistic details such as clothing. This task is extremely challenging due to human-object occlusions and the lack of 3D information in 2D images, often leading to poor detail reconstruction and depth ambiguity. To address these problems, we propose a novel attention-based neural implicit model that leverages image pixel alignment from both the input human-object image for a global understanding of the human-object scene and from local separate views of the human and object images to improve realism with, for example, clothing details. Additionally, the network is conditioned on semantic features derived from an estimated human-object pose prior, which provides 3D spatial information about the shared space of humans and objects. To handle human occlusion caused by objects, we use a generative diffusion model that inpaints the occluded regions, recovering otherwise lost details. For training and evaluation, we introduce a synthetic dataset featuring rendered scenes of inter-occluded 3D human scans and diverse objects. Extensive evaluation on both synthetic and real-world datasets demonstrates the superior quality of the proposed human-object reconstructions over competitive methods.
\end{abstract}    
\section{Introduction}
\label{sec:intro}
% Image editing methods in diffusion models depend on user-defined control directions - users can unlock their creativity using these methods by specifying the desired manipulation through prompts~\cite{gandikota2023concept}, reference images~\cite{ruiz2022dreambooth, kumari2022customdiffusion, gal2022image, chen2024trainingfreeregionalpromptingdiffusion}, or attribute vectors~\cite{parmar2023zero,hertz2022prompt}. In this work, we ask a fundamentally different question: \emph{Can we automatically discover the underlying visual structure of a concept within diffusion model's knowledge?} %Rather than requiring user-specified controls, we aim to decompose the model's internal knowledge into meaningful directions.

% This question touches on a fundamental limitation in how we interact with diffusion models. Current control methods ~\cite{zhang2023addingconditionalcontroltexttoimage, gandikota2023concept, ye2023ipadaptertextcompatibleimage,ye2023ipadaptertextcompatibleimage, hertz2024stylealignedimagegeneration, li2023photomaker, shi2024instantbooth, chen2024trainingfreeregionalpromptingdiffusion} require users to specify their desired manipulations in advance, limiting interactive creativity. This contrasts with natural human artistic workflows, where creators dynamically explore creative ideas while jointly refining them toward meaningful artistic outcomes~\cite{hoffmann2016modeling}. This synergy between specification and exploration is not new to generative models. Early GAN architectures naturally developed disentangled latent spaces that enabled continuous\cite{harkonen2020ganspace,radford2015unsupervised, wu2021stylespace, shen2020interfacegan}, compositional control over generated images. Users could explore these spaces to discover interesting variations that would be difficult to describe in words~\cite{wu2021stylespace}, then combine them to achieve their creative goals~\cite{grabe2022towards}. 


% While diffusion models have largely superseded GANs in conditional image synthesis~\cite{dhariwal2021diffusion},  their underlying structure remains less understood. Diffusion models achieve remarkable diversity through high-dimensional latents, unlike GANs' compact latent spaces.  With a single prompt, diffusion models can generate radically different variations through different random initializations of input noise. We ask - Is it possible to discover interpretable structure within this vast space of variations?

Text-to-image diffusion models are capable of generating remarkable visual variations from a single prompt through different random initializations. However, this vast creative potential remains largely opaque to users---while we can generate diverse images, we lack understanding of the underlying structure of these variations. This presents a fundamental challenge: how can we discover and expose the latent visual capabilities encoded within these models?

\let\thefootnote\relax \footnote{$^{*}$Correspondence to \texttt{gandikota.ro@northeastern.edu}}

The challenge touches on a key limitation in how we interact with diffusion models today. Current control methods require users to explicitly specify their desired edits in advance through prompts~\cite{gandikota2023concept}, reference images~\cite{zhang2023addingconditionalcontroltexttoimage, chen2024trainingfreeregionalpromptingdiffusion, ruiz2022dreambooth,kumari2022customdiffusion, Ryu_lora, hu2021lora}, or attribute vectors~\cite{ye2023ipadaptertextcompatibleimage, hertz2024stylealignedimagegeneration, li2023photomaker, shi2024instantbooth,parmar2023zero,hertz2022prompt}. That contrasts sharply with natural human creative workflows, where artists dynamically explore creative ideas and jointly refine them toward meaningful artistic outcomes~\cite{hoffmann2016modeling}. The need for pre-specified controls creates a barrier between users and the full creative potential of these models.

Interestingly, earlier generative models like GANs~\cite{gans,karras2019style,brock2018large} naturally developed more interpretable internal structures. Their compact latent spaces often exhibited emergent disentanglement~\cite{harkonen2020ganspace,radford2015unsupervised, wu2021stylespace, shen2020interfacegan}, enabling continuous and compositional control over generated images. Users could explore these spaces to discover interesting variations that would be difficult to describe in words~\cite{wu2021stylespace}, then combine them to achieve their creative goals~\cite{grabe2022towards}.

Diffusion models have largely superseded GANs in conditional image synthesis~\cite{dhariwal2021diffusion}, achieving greater diversity through much higher-dimensional latents. And yet an understanding of the underlying structure of these larger latent spaces has remained elusive. In this work, we ask a fundamental question: \emph{Can we automatically discover the visual structure within a diffusion model's knowledge of a concept?} Rather than requiring user-specified controls, we aim to decompose the model's internal representations into expressive directions that users can explore and combine.

To address these needs, we present \textbf{SliderSpace}, a framework that brings systematic explorability to diffusion models. Given just a text prompt, SliderSpace discovers a canonical set of meaningful, diverse, and controllable directions within the model's knowledge of that concept. Each direction is implemented as a low-rank adapter~\cite{hu2021lora} that can be scaled and composed with others, allowing users to explore and smoothly combine different aspects of variation, as shown in Figure~\ref{fig:intro}.

We ground SliderSpace discovery in three key requirements for meaningful decomposition of a diffusion model's visual manifold: 
\begin{enumerate}
    \item \textbf{Unsupervised Discovery:} The decomposition process should emerge from the intrinsic structure of the model's learned representation, rather than being guided by predefined attributes. This ensures we capture the true topology of the model's knowledge space rather than projecting our assumptions onto it.
    
    \item \textbf{Semantic Orthogonality:} Each discovered control must represent a distinct semantic direction. This is enforced in a semantic feature space, like CLIP, where every slider has an orthogonal effect in embeddings. This prevents discovering multiple controls that create similar semantic effects, making the system more efficient and easier.
    
    \item \textbf{Distribution Consistency:} Directions must induce consistent transformations across both random seeds and prompt variations. 
\end{enumerate}

These requirements naturally lead to our proposed framework, which we formalize in Section~\ref{sec:method}. As we show in our experiments, SliderSpace is architecture-agnostic, working with both conventional U-Net based models like Stable Diffusion~\cite{rombach2022high, rombach2022sd20, podell2023sdxl, turbo, dmd} and recent transformer-based architectures like Flux~\cite{flux}.

We demonstrate the expressiveness of SliderSpace through three applications: First, we show how SliderSpace can decompose high-level concepts into diverse and expressive components, revealing the natural axes of variation in the model's understanding. Second, we explore artistic style variation, where SliderSpace discovers directions that match or exceed the diversity of manually curated artist lists while being judged more useful by human evaluators. Finally, we show how SliderSpace can help reverse the mode collapse commonly observed in distilled diffusion models, restoring diversity while maintaining generation speed.

Beyond providing practical creative control, SliderSpace opens new avenues for understanding and utilizing the latent capabilities of diffusion models. By mapping these models' visual potential into intuitive, composable directions, we take a step toward making their creative possibilities more accessible and interpretable to users.

% Image editing methods in diffusion models unlock the creativity of users. In this work we ask an alternate question: \emph{Can we organize and expose what of the diffusion model is already capable of?}.
% Existing methods for controlling image generation typically require users to manually specify edit directions for desired changes. This process is time-consuming, requires technical expertise, and limits the spontaneity of the creative process. For instance, if a user wants to adjust the smile of a generated person, they must explicitly request this edit, often through imprecise prompt engineering or model fine-tuning. This approach of predefined controls or manual specifications restricts users from fully exploring the latent capabilities of the model. There may be interesting stylistic variations or attributes that the model can generate, but users have no easy way to discover or utilize these.

% Natural visual disentanglement was an emergent property in the latent space of Generative Adversarial Models (GANs) \cite{harkonen2020ganspace,radford2015unsupervised, wu2021stylespace, shen2020interfacegan}. In particular, it has been observed that StyleGAN~\cite{karras2019style} stylespace neurons offer detailed control over many meaningful aspects of images that would be difficult to describe in words~\cite{wu2021stylespace}. However, diffusion models do not share such a compact latent space~\cite{park2023unsupervised}; and efforts to uncover such a space in the semantic embeddings of the text conditioning have met with limited success \nik{Nick - is there a specific citation you were thinking about?}.

% In this work we introduce \textbf{SliderSpace}, which takes a step towards uncovering an analogous low dimensional representation of diffusion models' visual breadth; in essence treating the diffusion model as many generators sharing parameters, where a particular generator is defined by a specific prompt. For a given prompt we sample many random seeds (and optionally prompt expansions using an LLM), generate the corresponding images, and apply an off the shelf feature extractor (in this work CLIP, but our method can be applied to any differentiable feature extractor). We use PCA to analyze these features, and for each of the leading $k$ principal components we train a LoRA \cite{} which causes the diffusion model to produces images which increase the feature magnitude along that component when passed back through the same feature extractor. This leads to a 'Slider' for each principal component, because each LoRA can be scaled and applied to the original diffusion model, continuously varying those visual features in the generated results (as measured, in our case, by CLIP).

% There are many other works that enhance the controllability of diffusion models. One common approach is enabling users to add spatial constraints to a generation either manually, or via a reference image \cite{zhang2023addingconditionalcontroltexttoimage, chen2024trainingfreeregionalpromptingdiffusion}, a second is leveraging more abstract embeddings (e.g. identity, style) extracted from a reference image \cite{ye2023ipadaptertextcompatibleimage, hertz2024stylealignedimagegeneration, li2023photomaker, shi2024instantbooth}, a third is finetuning a foundation model to better generate a concept important to the user \cite{ruiz2022dreambooth, kumari2022customdiffusion, Ryu_lora, hu2021lora}, and a fourth (most relevant to this work) is finding low-rank adaptors of the model based on a prompt or small training set which can be scaled to provide continous control over one aspect of generated image (e.g. night vs day, basic vs luxury, etc.) \cite{gandikota2023concept}. SliderSpace is complementary to all of these methods and offers something distinct. All of the other methods we are aware require the user (and / or model designer) to know in advance what type of control they want. In contrast SliderSpace assists users in discovering and controlling hidden capabilities present in the diffusion model's distribution of possible generations.

%We propose that truly intuitive creative control in a text-to-image model should meet three key criteria: \emph{discoverability}, \emph{intuitiveness}, and \emph{specificity}. The model should reveal controllable attributes that may not be immediately obvious, offer controls that are easy to understand and manipulate, and ensure each control affects a distinct attribute of the generated image.

% We demonstrate the utility and power of SliderSpace using three applications built on top of SDXL-DMD \cite{dmd}, because its fast generation speed lends itself well to the continuous control offered by SliderSpace.

% First, we study concept decomposition (Section \ref{sec:concept_exp}), where we learn sliders for a specific concept (e.g. 'monster', 'waterfall', 'car'). Through quantitative metrics of diversity and text alignment we demonstrate that the learned sliders dramatically boost the diversity of generations when randomly applied without harming text alignment; we also ask humans to qualitatively judge these results in a user study where they find the SliderSpace results to be more 'Diverse', 'Useful', and 'Creative' than our baselines.

% Second, we attempt to compare the automatic discoveries of SliderSpace to a large scale manual study of artistic styles (Section \ref{sec:art_exp}), open-sourced by ParrotZone \cite{parrotzone}. In this study SDXL was prompted with over 4300 artist names,  and based on visual inspection the cases of successful stylistic mimicry recorded. Quantitatively SliderSpace more closely matches the distribution of artistic variation discovered by ParrotZone than other baselines, and in our user studies was judged to be significantly more 'Diverse' and 'Useful' than the baselines. To our surprise humans even judged SliderSpace results to be slightly more 'Diverse' than the results generated by the manually discovered artist names of \cite{parrotzone}.

% Third, we attempt to use SliderSpace to reverse the mode collapse commonly observed in distilled few-step diffusion models relative to the original teacher model (Section \ref{sec:diverse_exp}). We quantitatively demonstrate that applying SliderSpace to SDXL-DMD leads to more closely matching the distribution of images by the original teacher, SDXL.

%Through extensive experiments on various state-of-the-art text-to-image models, we demonstrate that SliderSpace significantly enhances user control and creative expression in AI-assisted image generation tasks. Our method enables a range of applications, including concept decomposition and control, diversity improvement in generated images, customization dissection and edits, and the exploration of artistic styles inherent in the model.

% SliderSpace goes beyond providing a practical tool for enhanced creative control. By mapping the visual potential of diffusion models it can open new avenues for generative creativity and deepens our understanding of each model's hidden potential.
\section{Related Work}
The OS Agent refers to an agent tool that utilizes the capabilities of (M)LLMs to perform user-defined tasks within an operating system, based on native system operations. Notable web-based agents include SeeAct \citep{zheng2024gpt4visiongeneralistwebagent}, SeeClick \citep{cheng2024seeclickharnessingguigrounding}, and WebAgent \citep{gur2024realworldwebagentplanninglong}, while mobile agents consist of InfiGUIAgent \citep{liu2025infiguiagent}, AppAgent \citep{zhang2023appagentmultimodalagentssmartphone}, Mobile-Agent \citep{wang2024mobileagentautonomousmultimodalmobile}, and Mobile-Agent-v2 \citep{wang2024mobileagentv2mobiledeviceoperation}. Additionally, there are various other agents \citep{yan2023gpt4vwonderlandlargemultimodal,li2023zeroshotlanguageagentcomputer,wu2024oscopilotgeneralistcomputeragents,tan2024cradleempoweringfoundationagents,lee2024exploreselectderiverecall,hoscilowicz2024clickagentenhancinguilocation,deng2024multiturninstructionfollowingconversational,hu2024infiagent}. These OS agents are often exposed to various security risks.

% These frameworks utilize inputs to (M)LLMs that consist of carefully designed prompts, alongside two key data types: Set of Mark (SoM) annotated screenshots \citep{yang2023setofmarkpromptingunleashesextraordinary} and accessibility trees (a11y trees). Each UI element in SoM screenshots features a boundary box with a numerical label, while the a11y tree provides a hierarchical representation of UI elements and their accessibility attributes. If the data inputted into MLLMs is maliciously altered, it can pose significant security risks to the agents.
% Many studies on the security of OS Agents have been conducted, especially in terms of adversarial attacks. 
Many studies on the security of OS agents have been conducted. 
\citet{wu2024wipinewwebthreat} discovered a novel network threat called Web Indirect Prompt Injection (WIPI), which involves embedding natural language instructions in webpages to indirectly control web agents driven by large language models to execute malicious commands. \citet{wu2024dissectingadversarialrobustnessmultimodal} conducted a study shows that image perturbations can affect MLLMs to produce adversarial captioners, leading agents to pursue goals contrary to user intentions. \citet{ma2024cautionenvironmentmultimodalagents} highlighted the vulnerability of GUI agents to environmental disturbances and proposed an "Environment Injection" attack. \citet{liao2024eiaenvironmentalinjectionattack} introduced an environment injection attack scheme that injects malicious code in webpage to steal users' personal identification information (PII). However, their work was limited to passive-triggered environmental injection attacks. \citet{zhang2024attackingvisionlanguagecomputeragents} explored how to carry out pop-up window attacks on visual and language model (VLM)-based agents. Nevertheless, their discussion was confined to pop-up attacks in browser environments, missing broader research on environmental injection attacks. 
\citet{yang2024securitymatrixmultimodalagents} identified eight potential attack paths that agents on mobile devices might face but did not address active environmental injection attacks like those involving message notifications.



\section{\methodname{}: Automatic Functionality Annotation Pipeline}
\label{sec: annotation pipeline}
This section introduces \methodname{}, an annotation pipeline (Fig.~\ref{fig: anno pipeline}) that automatically produces contextual element functionality annotations used to enhance VLMs' GUI grounding capabilities.


\begin{table}[t]
\tiny
\centering
\caption{\textbf{Comparing our \methodname{} dataset with existing large-scale UI datasets.} Multi-Res means the samples are collected on devices with various resolutions. Auto Anno. means the samples are collected autonomously. \#Anno. means the number of annotated samples provided by the datasets.}
\label{tab:data comparison}
\begin{tabular}{@{}cccccccc@{}}
\toprule
Dataset & UI Type & \begin{tabular}[c]{@{}c@{}}Multi\\ Res.\end{tabular} & \begin{tabular}[c]{@{}c@{}}Real-world\\ Scenario\end{tabular} & \begin{tabular}[c]{@{}c@{}}Auto\\ Anno. \end{tabular} & \begin{tabular}[c]{@{}c@{}}Contextual\\ Functionality\\ Semantics\end{tabular} & \#Anno. & Task \\ \midrule
WebShop~\citep{yao2022webshop} & Web & \cross & \cross & \cross & \cross & 12k & Web Navigation \\
Mind2Web~\citep{deng2024mind2web} & Web & \cross & \cmark & \cross & \cross & 2.4k & Web Navigation \\
WebArena~\citep{zhou2023webarena} & Web & \cross & \cmark & \cross & \cross & 812 & Web Navigation \\
\midrule
S2W~\citep{Wang2021Screen2WordsAM} & Mobile & \cross & \cmark & \cross & \cross & 112k & Screen Summarization \\
Wid. Cap.~\citep{Li2020WidgetCG} & Mobile & \cross & \cmark & \cross & \cross & 163k & Element Captioning \\
PixelHelp~\citep{Li2020MappingNL} & Mobile & \cross & \cmark & \cross & \cross & 187 & Element Grounding \\
RICOSCA~\citep{Li2020MappingNL} & Mobile & \cross & \cmark & \cross & \cross & 295k & Action Grounding \\
MoTIF~\citep{Burns2022ADF} & Mobile & \cross & \cmark & \cross & \cross & 6k & Mobile Navigation \\
AITW~\citep{rawles2023android} & Mobile & \cross & \cmark & \cross & \cross & 715k & Mobile Navigation \\
RefExp~\citep{Bai2021UIBertLG} & Mobile & \cross & \cmark & \cross & \cross & 20.8k & Element Grounding \\
VWB~\citep{liu2024visualwebbench} & Web & \cross & \cmark & \cross & \cross & 1.5k & Elem. Ground \& Ref. \\
SeeClick Web~\citep{cheng2024seeclick} & Web & \cross & \cmark & \cmark & \cross & 271k & Element Grounding \\
UI REC/REG~\citep{hong2023cogagent} & Web & \cmark & \cmark & \cmark & \cross & 400k & Box2DOM, DOM2Box \\
Ferret-UI~\citep{you2024ferretui} & Mobile & \cmark & \cmark & \cmark & \cross & 250k & Elem. Ground \& Ref. \\
\methodname{} (ours) & Web, Mobile & \cmark & \cmark & \cmark & \cmark & 704k & Functionality Ground \& Ref. \\ \bottomrule
\end{tabular}
\end{table}



\begin{figure}[t]
    \centering
    \includegraphics[width=0.95\linewidth]{figure/AnnoPipeline3.pdf}
    \caption{\textbf{The proposed pipeline for automatic UI functionality annotation.} An LLM is utilized to predict element functionality based on the UI content changes observed during the interaction. LLM-aided rejection and verification are introduced to improve data quality. Finally, the high-quality functionality annotations will be converted to instruction-following data by applying task templates.}
    \label{fig: anno pipeline}
\end{figure}


\subsection{Collecting UI Interaction Trajectories}
Our pipeline initiates by collecting interaction trajectories, which are sequences of UI contents captured by interacting with UI elements. Each trajectory step captures all interactable elements and the accessibility tree (AXTree) that briefly outlines the UI structure, which will be used to generate functionality annotations. To amass these trajectories, we utilize the latest Common Crawl repository as the data source for web UIs and Android Emulator for mobile UIs. Note that illegal websites and Apps are excluded manually from the sources to ensure no pornographic or violent content is included in our dataset. Please refer to Sec.~\ref{sec:supp:record traj detail} for collecting details and data license.

\subsection{Functionality Annotation Based on UI Dynamics}
Subsequently, the pipeline generates functionality annotations for elements in the collected trajectories. Interacting with an element $e$, by clicking or hovering over it, triggers content changes in the UI. In turn, these changes can be used to predict the functionality $f$ of the interacted element. For instance, if clicking an element causes new buttons to appear in a column, we can predict that the element likely functions as a dropdown menu activator (an example in Fig.~\ref{fig: funcpred diff case}).
With this observation, we utilize a capable LLM (i.e., Llama-3-70B~\citep{llama3modelcard}) as a surrogate for humans to summarize an element's functionality based on the UI content changes resulting from interaction. Concretely, we generate compact content differences for AXTrees before ($s_t$) and after ($s_{t+1}$) the interaction using a file-comparing library\footnote{https://docs.python.org/3/library/difflib.html}. Then, we prompt the LLM to thoroughly analyze the UI content changes (addition, deletion, and unchanged lines), present a detailed Chain-of-Thoughts~\citep{wei2022chain} reasoning process explaining how the element affects the UI, and finally summarize the element's functionality.

In cases where element interactions significantly transform the UI and cause lengthy differences—such as navigating to a new screen—we adjust our approach by using UI description changes instead of the AXTree differences. Specifically, we prompt the same LLM to discern the UI hierarchy, describe UI regions, and finally describe the entire UI functionality. After describing the UIs before and after the interaction, the LLM analyzes the description differences, presents reasoning, and summarizes the element's functionality. This annotation process is formulated as:
\begin{equation}
    f = \text{LLM}(p_{\text{anno}}, s_t, s_{t+1})
\end{equation}

where $f$ is the predicted functionality, $p_{\text{anno}}$ is the annotation prompt (Tab.~\ref{tab:supp:funcpred manip prompt} and Tab.~\ref{tab:supp:funcpred nav prompt}). Examples of annotated elements are depicted in Fig.~\ref{fig: our dataset} and more annotation details are explained in Sec.~\ref{sec:supp:anno details}.

\subsection{Removing Invalid Samples via LLM-Aided Rejection}
The collected trajectories may contain invalid samples due to broken UIs, such as incomplete UI loading. These samples are meaningless as they contain corrupted UI content and can mislead the models trained with them.

To filter out these invalid samples, we introduce an LLM-aided rejection approach. Initially, hand-written rules are used to detect obvious broken cases, such as blank UI contents, UIs containing elements indicating content loading, and interaction targets outside of UIs. While these obvious cases constitute a large portion of the invalid samples, there are a few types that are difficult to detect with hand-written rules. For instance, interacting with a “view more” button might unexpectedly redirect the user to a login page instead of the desired information page due to website login restrictions. To identify these challenging samples, we prompt the annotating LLM to also act as a rejector. Specifically, the LLM takes the UI content changes, generated using a file-comparing library, as input, provides detailed reasoning on whether the changes are meaningful for predicting the element's functionality, and finally outputs predictability scores ranging from 0 to 3. This process is formulated as follows:
\begin{equation}
 score = \text{LLM}(p_{\text{reject}}, e, s_t, s_{t+1})
\end{equation}
where $p_{\text{reject}}$ is the rejection prompt (Tab.~\ref{tab:supp:rejection prompt}).

This approach ensures that clear and predictable samples receive higher scores, while those that are ambiguous or unpredictable receive lower scores. For instance, if a button labeled "Show More", upon interaction, clearly adds new content, this sample will considered to provide sufficient changes that can anticipate the content expansion functionality and will get a score of 3. Conversely, if clicking on a "View Profile" link fails to display the profile possibly due to web browser issues, this unpredictable sample will get a score less than 3.

After implementing empirical experiments, we deploy this LLM-based rejector to discard the bottom 30\% of samples based on their scores to strike a balance between the elimination of invalid samples and the preservation of valid ones (More details in Sec.~\ref{suc:supp:reject details}). The samples that pass the hand-written rules and the LLM rejector are subsequently submitted for functionality annotation. Please see representative rejection examples in Fig.~\ref{fig: rejection examples}.

\subsection{Improving Annotation Quality via LLM-Based Verification}
The functionality annotations produced by the LLM probably contain incorrect, ambiguous, and hallucinated samples (See a case in Fig.~\ref{fig: anno pipeline}), which probably misleads the trained VLMs and compromises evaluation accuracy. To improve dataset quality, we prompt LLMs to verify the annotations by checking whether the targeted element $e$ fulfills the intent of the annotated functionality $f$. This process presents the LLMs with the interacted element, its UI context, the UI changes induced by this element, and the functionality generated in the previous annotation process. The LLMs are then tasked with analyzing the UI content changes before predicting whether the interacted element aligns with the given functionality. If the LLMs determine that the interacted element fulfills the functionality given its UI context, the LLMs will grant a full score (An example in Fig.~\ref{fig: verif diff case}). If the interacted element is considered to mismatch the functionality, this functionality can be seen as incorrect as this mismatch indicates that it may not accurately reflect the element's actual role within the UI context.

To mitigate the potential biases in LLMs~\citep{panickssery2024llm, zheng2023judging, bai2024benchmarking}, two different LLMs (i.e., Llama-3-70B~\citep{llama3modelcard} and Mistral-7B-Instruct-v0.2~\citep{mistral}) are employed as verifiers and prompted to output 0-3 scores. The scoring process is formulated as follows:
\begin{equation}
 score = \text{LLM}(p_{\text{verify}}, e, f, s_t, s_{t+1})
\end{equation}
where $p_{\text{verify}}$ denotes the verification prompt (Tab.~\ref{tab:supp:verif prompt}). Only if the two scores are both 3s do we consider the functionality label correct (More details in Sec.~\ref{suc:supp:verif details}). Although this filtering approach seems stringent, we can make up the number of annotations through scaling. 

\begin{figure}[t]
    \centering
    \includegraphics[width=0.9\linewidth]{figure/our_dataset_img.pdf}
    \caption{Element functionality annotations generated by the proposed AutoGUI pipeline for both web and mobile viewpoints.}
    \label{fig: our dataset}
    \vspace{-5mm}
\end{figure}

\subsection{Functionality Grounding and Referring Task Generation}
\vspace{-2mm}
After rejecting, annotating, and verifying, we obtain a high-quality UI functionality dataset containing triplets of \{UI screenshot, Interacted element, Functionality\}. To convert this dataset into an instruction-following dataset for training and evaluation, we generate functionality grounding and referring tasks using diverse prompt templates (see Tab.~\ref{tab:task templates}). To mitigate the difficulty of predicting absolute values for various resolutions, the coordinates of element bounding boxes are all normalized within the range $[0,99]$ (see Fig.~\ref{fig: our dataset} for examples).

\subsection{Explore the \methodname{} Dataset}

\begin{table}[]
\centering
\small
\caption{\textbf{The statistics of the AutoGUI datasets.} The Anno. Tokens and Avg. Words columns show the total number of tokens and the average number of words for the functionality annotations regardless of task templates. The Domains/Apps column shows the number of unique web domains/mobile Apps involved in each split.}
\label{tab:simple data stats}
\begin{tabular}{@{}ccccccc@{}}
\toprule
Split & \#Tasks & Anno. Tokens & Avg. Words & Domains/Apps & Device Ratio   \\                                                                   \midrule
Train & 702k  & 17.9M        & 23.1       & 916     & Web: $54.6\%$, Mobile: $45.4\%$                                              \\ \cmidrule(r){1-6}
Test  & 2k    & 53.4k        & 22.5       & 299     & Web: $50\%$, Mobile: $50\%$                                                                                                               \\ \bottomrule
\end{tabular}
\end{table}

\begin{figure}[t]
    \centering
    \includegraphics[width=1.0\linewidth]{figure/wordcloud_token-dist-comparison.pdf}
    \caption{\textbf{Diversity of the AutoGUI dataset.} \textbf{Left}: The word cloud illustrates the ratios of the verbs representing the main intents in the functionality annotations. \textbf{Right}: Comparing the distributions of the annotation token numbers for our AutoGUI training split, SeeClick Web training data~\citep{cheng2024seeclick}, and Widget Captioning~\citep{Li2020WidgetCG}. The comparison demonstrates that our dataset covers significantly more diverse task lengths.}
    \label{fig: wordcloud and tokdistrib}
\end{figure}
\vspace{-2mm}

The \methodname{} pipeline finally collects 22.4k trajectories, from which we select 2k grounding samples (evenly divided between web and smartphone views) as the test set and remove the trajectories to which these samples belong. Subsequently, 702k samples are randomly selected from the remaining instances to constitute the training set. The statistics of our dataset in Tab.~\ref{tab:simple data stats} and Sec.~\ref{sec:supp:data stats} show that our dataset covers diverse UIs and exhibits variety in lengths and functional semantics of the annotations. Moreover, our dataset presents a unique ensemble of research challenges for developing generalist web agents in real-world settings. As shown in Tab.~\ref{tab:data comparison} and Fig.~\ref{fig: functionality vs others}, our dataset distinguishes itself from existing literature by providing functionality-rich data as well as tasks that require VLMs to discern the contextual functionalities of elements to achieve high grounding accuracy.

\section{Analysis of Data Quality}
This section analyzes the reliability of the proposed annotation pipeline and data quality.

\noindent{\textbf{Comparison with Human Annotation}} To demonstrate the superiority of the proposed automatic annotation pipeline based on open-source LLMs, $N=145$ samples (99 valid and 46 invalid) are randomly selected as a testbed for comparing the annotation correctness of a trained human annotator and the pipeline. Here, correctness is defined as $Correctness = C / (N - R)$, where $C$ and $R$ denote the numbers of correctly annotated and rejected samples, respectively. The denominator subtracts the number of rejected samples as we are more interested in the percentage of correct samples after rejecting the samples considered invalid by the annotator. The authors thoroughly check the annotation results according to the three criteria in Fig.~\ref{fig: check criteria}: 1. Context-specificity. The functionality annotations must include context-specific descriptions to ensure one-to-one mapping between the element and its annotation. 2. Appropriate details. Avoid detailing unnecessary aspects of the UIs to keep the description focused on functionality. 3. No hallucination. The annotations must not include information not grounded in the visual context of the UIs. See more details in Sec.~\ref{sec:supp:humaneval details}.

After experimenting with three runs, Tab.~\ref{tab:ablate autogui} shows that the proposed AutoGUI pipeline achieves high correctness comparable to the trained human annotator (r6 vs. r1). Without rejection and verification (r2), AutoGUI is inferior as it cannot recognize invalid samples. Notably, simply using the rules written by the authors can improve the correctness, which is further enhanced with the LLM-aided rejector (r4 vs. r3). Moreover, utilizing the annotating LLM itself to self-verify its annotations helps AutoGUI surpass the trained annotator (r5 vs. r1). Introducing another LLM verifier (i.e., Mistral-7B-Instruct-v0.2) brings a slight increase which results from Mistral recognizing Llama-3-70B’s incorrect descriptions of how dropdown menu options work. Overall, these results justify the efficacy of the AutoGUI annotation pipeline.

Qualitatively comparing the annotation patterns of the human and AutoGUI (Fig.~\ref{fig: autogui vs human}), we find that AutoGUI employs the strong LLM to generate more detailed and clear annotations which would take significantly more time for the human annotator. This result suggests that the AutoGUI pipeline can lessen the burden of collecting data for training UI-VLMs.

\noindent{\textbf{Impact of LLM Output Uncertainty}} The uncertainty of LLM outputs manifests in annotation, rejection, and verification, possibly impacting the quality of the AutoGUI dataset. To evaluate this impact, we first sample 100 valid samples to test the AutoGUI pipeline for three runs. The consistency rate is 94.5\%, indicating that 94.5\% of the samples possess consistent annotation outcomes (i.e. correct or incorrect) across the runs. We also test the LLM-aided rejector with 46 invalid samples and find that the rejection consistency over three runs is 79.3\%. This indicates that LLM uncertainty impacts this rejection process. Nevertheless, this impact is minor due to the low prevalence of invalid samples (4\% of all samples) that fail the hand-written rules.

In summary, AutoGUI exhibits annotation correctness comparable to that of human annotators and LLM output uncertainty poses a minor impact on the AutoGUI annotation process.



\begin{figure}[t]
    \centering
    \includegraphics[width=0.85\linewidth]{figure/check_criteria_img.pdf}
    \caption{The checking criteria used for comparing AutoGUI pipeline and the human annotator.}
    \label{fig: check criteria}
\end{figure}


\begin{table}[]
\small
\centering
\caption{\textbf{Comparing the AutoGUI and human annotator.} AutoGUI with the proposed rejection and verification achieves annotation correctness comparable to trained human annotators. One LLM means Llama-3-70B and Two LLMs include Mistral-7B-Instruct-v0.2 as well.}
\label{tab:ablate autogui}
\begin{tabular}{@{}ccccc@{}}
\toprule
No. & Annotator  & Rejector   & Verifier              & Correctness \\ \midrule
r1 & Human      & -          & -                     & 95.5\%      \\
r2 & Llama-3-70B & -          & -                     & 64.5\%      \\
r3 & Llama-3-70B & Rules      & -                     & 83.1\%      \\
r4 & Llama-3-70B & Rules+LLM  & -                     & 94.4\%      \\
r5 & Llama-3-70B & Rules+LLM  & One LLM            & 96.0\%      \\
r6 & Llama-3-70B & Rules+LLM & Two LLMs & \textbf{96.7\%}      \\ \bottomrule
\end{tabular}
\end{table}
\vspace{-2mm}


\section{Experiment}
\label{sec:Experiment}

\subsection{Dataset}
Outside Knowledge Visual Question Answering (OK-VQA)~\cite{marino2019ok} is a benchmark dataset designed to evaluate VQA systems that require leveraging external knowledge sources beyond the information present in an image. The dataset consists of 14,055 knowledge-based questions paired with 14,031 images from the COCO dataset~\cite{lin2014microsoft}. These questions span 10 diverse knowledge categories, including domains such as Science and Technology, Geography, Cooking and Food, and Vehicles and Transportation. The questions were crowdsourced via Amazon Mechanical Turk, ensuring they require real-world knowledge to answer, making this dataset significantly more challenging than conventional VQA datasets. 

The dataset is split into 9,009 training samples and 5,046 testing samples, with each question associated with 10 ground-truth answers annotated by human annotators. This multi-answer format helps address ambiguity and variability in responses. Table~\ref{tab:okvqa_details} outlines key statistics and the distribution of questions across various knowledge categories in the Ok-VQA dataset. Baseline evaluations on OK-VQA using state-of-the-art models like MUTAN and Bilinear Attention Networks (BAN) reveal a significant drop in performance compared to traditional VQA datasets. This performance degradation underscores the need for models with enhanced retrieval and reasoning capabilities to incorporate unstructured, open-domain knowledge effectively.

\begin{table}[h!]
    \centering
    \footnotesize
    \setlength{\tabcolsep}{4pt}
    \renewcommand{\arraystretch}{1.2}
    \caption{Key Details of the OK-VQA Dataset}
    \begin{tabular}{|p{3.2cm}|p{4.8cm}|}
        \hline
        \textbf{Attribute}                & \textbf{Details} \\
        \hline
        \textbf{Name}                     & OK-VQA (Outside Knowledge VQA) \\
        \hline
        \textbf{Source}                   & COCO Image Dataset \\
        \hline
        \textbf{Number of Questions}      & 14,055 \\
        \hline
        \textbf{Number of Images}         & 14,031 \\
        \hline
        \textbf{Question Categories}      & 10 Categories \\
        \hline
        \textbf{Categories Breakdown}     & Vehicles \& Transportation (16\%) \newline Brands, Companies \& Products (3\%) \newline Objects, Materials \& Clothing (8\%) \newline Sports \& Recreation (12\%) \newline Cooking \& Food (15\%) \newline Geography, History, Language \& Culture (3\%) \newline People \& Everyday Life (9\%) \newline Plants \& Animals (17\%) \newline Science \& Technology (2\%) \newline Weather \& Climate (3\%) \newline Other (12\%) \\
        \hline
        \textbf{Average Question Length}  & 8.1 words \\
        \hline
        \textbf{Average Answer Length}    & 1.3 words \\
        \hline
        \textbf{Unique Questions}         & 12,591 \\
        \hline
        \textbf{Unique Answers}           & 14,454 \\
        \hline
        \textbf{Answer Annotations}       & 10 answers per question \\
        \hline
        \textbf{Answer Types}             & Open-ended \\
        \hline
        \textbf{Requires External Knowledge} & Yes (e.g., Wikipedia, Common Sense, etc.) \\
        \hline
        \textbf{Typical Knowledge Sources}& Unstructured Text (Wikipedia) \\
        \hline
    \end{tabular}
    \label{tab:okvqa_details}
\end{table}

\subsection{Implementation Details}
The experiments are conducted on Google Colab using a T4 GPU. The NVIDIA T4 GPU features 16 GB of GDDR6 memory, 320 Tensor Cores, and supports mixed-precision computation, making it suitable for deep learning tasks. Due to computational constraints, we evaluate our model on a subset of 100 samples from the OK-VQA dataset~\cite{marino2019ok}.

\subsection{OOD and ID Category Splits}
In our experiments, we evaluate our approach using the OK-VQA dataset~\cite{marino2019ok}, which we split into OOD and ID subsets based on knowledge categories. The OOD categories include Vehicles and Transportation, Brands, Companies and Products, Sports and Recreation, Science and Technology, and Weather and Climate. The ID categories comprise Objects, Materials and Clothing, Cooking and Food, Geography, History, Language and Culture, People and Everyday Life, Plants and Animals, and Other. Using this split, we can assess how well the model generalizes to different categories of knowledge.

\subsection{Patch-Based Image Preprocessing}
For VQA processing, we preprocess each input image by dividing it into patches of various sizes, specifically 2×2, 3×3 and 4x4 grids. This patch-based approach captures fine-grained visual details, which can enhance feature extraction for complex queries. We then employ the BLIP-VQA model~\cite{li2022blip} to extract image representations and generate initial contextual information based on the image and the associated question.

\subsection{Retrieval-Augmented Knowledge Integration}
To incorporate external knowledge, we use  RAG~\cite{lewis2020retrieval} with external knowledge sources such as Wikipedia and DBpedia. RAG retrieves relevant information based on the question and the visual features extracted by BLIP-VQA~\cite{li2022blip}. This retrieval process supplies the model with real-world context beyond the image, which is crucial for correctly answering questions that depend on external knowledge.

\subsection{State-of-the-Art Performance Comparison}
We evaluate our proposed FilterRAG framework on the OK-VQA dataset and compare it to state-of-the-art methods (Table~\ref{table:SOTA-OK-VQA}). The baseline models, Base1 and Base2, use the BLIP-VQA model with the VQA v2~\cite{goyal2017making} and OK-VQA datasets~\cite{marino2019ok}, achieving 83.0\% and 40.0\% accuracy, respectively. The drop highlights the challenge of knowledge-based questions in OK-VQA. Our FilterRAG framework, which integrates BLIP-VQA, RAG, and external knowledge sources like Wikipedia and DBpedia, achieves 36.5\% accuracy in OOD settings. This result demonstrates the effectiveness of grounding VQA responses with external knowledge, especially for OOD scenarios. 

Compared to state-of-the-art methods, KRISP~\cite{marino2021krisp}  achieves 38.35\% with Wikipedia and ConceptNet, while MAVEx~\cite{wu2022multi} reaches 41.37\% using Wikipedia, ConceptNet, and Google Images. The highest performance comes from KAT (ensemble)~\cite{gui2021kat} at 54.41\% with Wikipedia and Frozen GPT-3. Although these models achieve higher accuracy, they often require significant computational resources. 

FilterRAG balances performance and efficiency, making it suitable for resource-constrained environments. As shown in Figure~\ref{fig:plot1_accuracy}, it achieves 37.0\% accuracy in ID settings, 36.0\% in OOD settings, and 36.5\% when combining ID and OOD data. This highlights its robustness for knowledge-intensive VQA tasks.

\begin{figure}[h!]
    \centering
    \includegraphics[width=\linewidth]{figures/plot1_accuracy_v2.pdf}
    \caption{Comparison of Model Accuracy Across Different Settings.}
    \label{fig:plot1_accuracy}
\end{figure}

\begin{table*}[t]
    \centering
    \footnotesize
    \caption{Performance Comparison of State-of-the-Art Methods on the OK-VQA Dataset}
    \label{tab:okvqa_results}
    \renewcommand{\arraystretch}{1.2}
    \setlength{\tabcolsep}{10pt}
    \begin{tabular}{l l c}
        \toprule
        \textbf{Method}                                & \textbf{External Knowledge Sources}                          & \textbf{Accuracy (\%)} \\
        \midrule
        Q-only (Marino et al., 2019)~\cite{marino2019ok}                  & —                                                          & 14.93                  \\
        MLP (Marino et al., 2019)~\cite{marino2019ok}                     & —                                                          & 20.67                  \\
        BAN (Marino et al., 2019)~\cite{marino2019ok}              & —                                                          & 25.1                  \\
        MUTAN (Marino et al., 2019)~\cite{marino2019ok}               & —                                                          & 26.41                  \\
        ClipCap (Mokady et al., 2021)~\cite{mokady2021clipcap}                 & —                                                          & 22.8                   \\
        \midrule
        BAN + AN (Marino et al., 2019~\cite{marino2019ok}                  & Wikipedia                                                  & 25.61                  \\
        BAN + KG-AUG (Li et al., 2020)~\cite{li2020boosting}        & Wikipedia + ConceptNet                                     & 26.71                  \\
        Mucko (Zhu et al., 2020)~\cite{zhu2020mucko}                      & Dense Caption                                              & 29.2                   \\
        ConceptBERT (Gardères et al., 2020)~\cite{garderes2020conceptbert}           & ConceptNet                                                 & 33.66                  \\
        KRISP (Marino et al., 2021)~\cite{marino2021krisp}                   & Wikipedia + ConceptNet                                     & 38.35                  \\
        RVL (Shevchenko et al., 2021)~\cite{shevchenko2021reasoning}                 & Wikipedia + ConceptNet                                     & 39.0                   \\
        Vis-DPR (Luo et al., 2021)~\cite{luo2021weakly}                    & Google Search                                              & 39.2                   \\
        MAVEx (Wu et al., 2022)~\cite{wu2022multi}                       & Wikipedia + ConceptNet + Google Images                    & 41.37                  \\
        PICa-Full (Yang et al., 2022)~\cite{yang2022empirical}                 & Frozen GPT-3 (175B)                                        & 48.0                   \\
        KAT (Gui et al., 2022) (Ensemble)~\cite{gui2021kat}             & Wikipedia + Frozen GPT-3 (175B)                           & 54.41                  \\
        REVIVE (Lin et al., 2022) (Ensemble)~\cite{lin2022revive}          & Wikipedia + Frozen GPT-3 (175B)                           & 58.0                   \\
        RASO (Fu et al., 2023)~\cite{fu2023generate}                        & Wikipedia + Frozen Codex                                   & 58.5                   \\
        \midrule
        \textbf{FilterRAG (Ours)}                     & Wikipedia + DBpedia (\textbf{Frozen} BLIP-VQA and GPT-Neo 1.3B)    & \textbf{36.5}          \\
        \bottomrule
    \end{tabular}
    \label{table:SOTA-OK-VQA}
\end{table*}


\subsection{Hallucination Detection via Grounding Scores}
We evaluate the grounding scores of our FilterRAG framework against baseline models to assess its ability to mitigate hallucinations by aligning answers with external knowledge. As shown in Figure~\ref{fig:plot2_grounding_score}, Base1 achieves the highest grounding score of 94.60\% on the VQA v2 dataset~\cite{goyal2017making}, indicating that BLIP performs effectively when answering general-domain questions that do not require external knowledge. In contrast, Base2, evaluated on the OK-VQA dataset~\cite{marino2019ok}, shows a significant drop to 71.70\%, highlighting the challenge of answering knowledge-based questions without access to external information, thereby increasing the likelihood of hallucinations.

\begin{figure}[h!]
    \centering
    \includegraphics[width=\linewidth]{figures/plot2_grounding_score_v2.pdf}
    \caption{Grounding Score Comparison Across Baselines and Proposed Methods.}
    \label{fig:plot2_grounding_score}
\end{figure}

To address this limitation, our proposed method integrates BLIP-VQA, RAG, and external knowledge sources such as Wikipedia and DBpedia. The grounding scores for our method are 70.06\% for In-Distribution (ID) data, 70.68\% for Out-of-Distribution (OOD) data, and 70.37\% when combining both settings. These consistent scores demonstrate that FilterRAG effectively grounds answers in retrieved knowledge, reducing hallucinations even in challenging OOD scenarios.

Although our method does not achieve the grounding performance of Base1, it provides reliable results for knowledge-intensive tasks by leveraging external knowledge sources. This makes FilterRAG a robust and efficient solution for real-world VQA applications, particularly where external knowledge and OOD generalization are critical.

\subsection{Ablation Study}
We evaluate the effect of different image grid sizes on the performance of our FilterRAG framework with BLIP-VQA and RAG in OOD scenarios. We consider three grid configurations, 2x2, 3x3, and 4x4, and evaluate their influence on accuracy and grounding score. As shown in Figure~\ref{fig:plot5_measure_grid_size}, accuracy decreases slightly as the grid size increases. The accuracy is 37.00\% for the 2x2 grid, declines to 35.00\% for the 3x3 grid, and further drops to 34.00\% for the 4x4 grid. This downward trend indicates that larger grid sizes lead to excessive fragmentation, making it challenging for the model to extract coherent and meaningful visual features.

\begin{figure}[h!]
    \centering
    \includegraphics[width=\linewidth]{figures/plot5_measure_grid_size_v2.pdf}
    \caption{Effect of Grid Sizes on Accuracy and Grounding Score.}
    \label{fig:plot5_measure_grid_size}
\end{figure}

Similarly, the grounding score follows a declining trend with increasing grid size. The grounding score is 70.06\% for the 2x2 grid, reducing to 69.20\% for the 3x3 grid and 68.07\% for the 4x4 grid. This decline suggests that finer grid divisions hinder the model’s ability to align generated answers with retrieved external knowledge, likely due to the loss of contextual coherence when images are broken into smaller patches.

Overall, the 2x2 grid size achieves the best trade-off between accuracy and grounding score. It maintains both visual coherence and effective knowledge alignment, thereby reducing the risk of hallucinations. Consequently, for OOD scenarios in the FilterRAG framework, the 2x2 grid configuration is the most effective for ensuring robust and reliable performance.

\subsection{Qualitative Analysis}
We perform a qualitative analysis of FilterRAG on the OK-VQA dataset~\cite{marino2019ok}, evaluating its performance in both In-Domain (ID) and Out-of-Distribution (OOD) settings. As illustrated in Figure~\ref{fig:Qualitative_Analysis}, FilterRAG generates accurate answers in ID scenarios where the retrieved knowledge is relevant and aligns well with the visual context. In these cases, the model effectively combines visual cues and external knowledge, resulting in well-grounded responses. These errors are frequently caused by misalignment between the visual context and the retrieved information, reflecting the challenge of handling ambiguous or novel queries outside the training distribution.

In OOD settings, FilterRAG struggles when relevant knowledge of unfamiliar concepts cannot be effectively retrieved. This often leads to hallucinations, where the model produces plausible but incorrect answers that are not supported by the retrieved evidence. This analysis highlights the critical role of reliable knowledge retrieval and precise multimodal alignment in mitigating hallucinations. Improving the quality of knowledge retrieval and refining visual-textual alignment are essential steps toward making FilterRAG more reliable in OOD contexts. Future improvements in these areas can help ensure more accurate and context-aware responses in real-world VQA applications.

%\section{Experiments}
\label{sec:experiments}

In this section, we evaluate \textbf{PLCEmbed} on two classification tasks discussed earlier: toolchain provenance identification and functionality classification. We first describe the experimental setup, then present the results for toolchain provenance (Sec.~\ref{subsec:toolchain_results}). Results for functionality classification follow in subsequent subsections.


\subsection{Experimental Setup}
\label{subsec:exp_setup}
We use the \textit{PLC-BEAD} dataset introduced in Sec.~\ref{sec:plcbead}, containing 729 Structured Text (ST) programs compiled across four toolchains. All experiments are conducted on a server equipped with Intel Core i7-7820X CPU @3.60GHz with 16GB RAM and two NVIDIA GeForce GTX Titan Xp GPUs 
% Each compiled binary is converted into a raw sequence of bytes, truncated or padded to a maximum length for uniform processing. 
% The dataset is split at the ST program level, ensuring that no program (or its compiled variants) appears in both training and test partitions.

\paragraph{Evaluation Metrics.}
We report accuracy on the test set to measure how often the predicted class (toolchain or functionality) matches the ground truth. We may also include F1 scores (macro or weighted) if relevant, especially in tasks where some classes have fewer samples. In addition, we utilize the Cohen Kappa score ($\kappa$) and the Matthews correlation coefficient (MCC) to account for the disparity in class sizes and provide a fair evaluation. 

% \paragraph{Model Variants.}
% \begin{itemize}
%     \item \textbf{PLCEmbed (CNN+Transformer).} Our main approach processes the bytes through a 1D convolutional layer to capture local patterns, then a transformer encoder for global dependencies. A final classifier outputs the predicted label.
%     \item \textbf{Baseline (CNN-only).} In addition to PLCEmbed, we evaluate a simpler CNN-based model that omits the transformer encoder. This baseline retains the same raw-byte input and convolutional blocks but lacks the self-attention layers. It allows us to compare how much improvement is gained by including global context modeling in the transformer.
% \end{itemize}






\begin{table}[!ht]
    \centering
    \caption{The Performance of \textit{PLCEmbed} on Toolchain Provenance Identification.}
    \begin{tabular}{ |p{0.7cm} | p{2.4cm} |p{0.8cm} |p{0.8cm} |p{0.8cm}| p{0.8cm} |}
        \hline
        Model & Dataset & Acc & F1 & $\kappa$ & MCC \\
        \hline
        &  \textit{PLC-BEAD} &91.77\% &91.77\% &89.01\% &89.02\% \\
        base- line& \textit{PLC-BEAD}-5\%-polluted &92.22\% &92.53\% &89.62\% &89.66\% \\
        & \textit{PLC-BEAD}-10\%-polluted &91.66\% &91.65\% &88.87\% &88.90\% \\
       
        \hline%for none, first-k (n=2000), first-k (n=20000)

       %ours& \textit{PLC-BEAD} &86.31\% &86.30\% &81.73\% &81.77\% \\
        %& \textit{PLC-BEAD}-5\%-polluted &85.73\% &85.70\% &80.95\% &80.96\% \\
       % & \textit{PLC-BEAD}-10\%-polluted &85.58\% &85.51\% &80.75\% &80.77\% \\%for none, first-k (n=2000), first-k (n=20000)
        & \textit{PLC-BEAD} &93.11\% &93.10\% &90.80\% &90.81\% \\
        ours& \textit{PLC-BEAD}-5\%-polluted &92.66\% &92.64\% &90.21\% &90.25\% \\
        & \textit{PLC-BEAD}-10\%-polluted &92.64\% &92.64\% &90.19\% &90.21\% \\
        \hline
    \end{tabular}
    \label{tab:result_tp}
\end{table}


\subsection{Toolchain Provenance Results}
\label{subsec:toolchain_results}
We first evaluate our model on the task of identifying which compiler produced a given PLC binary. 

Table~\ref{tab:result_tp} summarizes accuracy, F1-score, and other relevant metrics on the test set. The proposed framework achieves strong performance, often above 90\% accuracy for predicting the correct compiler, with some variation across compiler types. In particular, binaries originating from OpenPLC-V3 tend to exhibit distinct byte patterns that are easier to learn, whereas CoDeSys binaries sometimes contain shared library code that can overlap with features from other toolchains. Nonetheless, the attention-based layers help capture global cues, such as the file header or compiler-specific data segments, which improves disambiguation. The CNN-only baseline performs moderately well but sometimes misclassifies binaries that share structural similarities across compilers. Without the attention-based mechanism, it may overlook long-range features that differentiate compilers more definitively.
% In contrast, the CNN-only baseline underperforms by about 3--4 percentage points. This gap reflects how global attention helps the network correlate byte patterns found in the header or trailer with subtle markers in the main data region.

The results indicate a substantial gain for all compilers. In practical forensic applications, even small gains in precision can significantly cut down investigative workload. A correct toolchain assignment narrows the search for known vulnerabilities and helps analysts confirm whether an executable originated from a trusted development environment. Sec.~\ref{subsec:discussion} further explores the role of this classification in ICS forensics, including how partial matches can still provide valuable clues in legacy systems with incomplete data.

\begin{table}[!ht]
    \centering
    \caption{The Performance of \textit{PLCEmbed} on Functionality Prediction.}
    \begin{tabular}{ |p{0.7cm} | p{2.5cm} |p{0.8cm}  |p{0.8cm} |p{0.8cm}| p{0.8cm} |}
        \hline
        Model & Dataset & Acc & F1 & $\kappa$ & MCC \\
        \hline
        & \textit{OpenPLC} V3 &45.18\% &41.24\%  &39.38\% &39.84\%\\
        & \textit{OpenPLC} V2 &38.77\% &36.07\%  &33.73\% &34.08\%\\

        & \textit{PLC-BEAD} &39.28\% &37.75\%  &33.70\% &33.78\%\\
        base- line & \textit{PLC-BEAD}-5\%-polluted &38.37\% &36.68\%  &32.63\% &32.75\%\\
        & \textit{PLC-BEAD}-10\%-polluted &35.65\% &33.74\%  &29.42\% &29.58\%\\
        \hline
        & \textit{OpenPLC} V3 &46.85\% &43.42\%  &41.28\% &41.73\%\\
        & \textit{OpenPLC} V2 &41.68\% &39.34\%  &37.04\% &37.36\%\\

        & \textit{PLC-BEAD} &42.28\% &40.35\%  &36.63\% &36.83\%\\
        ours& \textit{PLC-BEAD}-5\%-polluted &39.12\% &36.72\%  &33.04\% &33.29\%\\
        & \textit{PLC-BEAD}-10\%-polluted &38.23\% &35.25\%  &31.86\% &32.15\%\\
        \hline
    \end{tabular}
    \label{tab:result_func}
\end{table}

\subsection{Functionality Classification Results}
\label{subsec:func_classification}

We now evaluate both \textbf{PLCEmbed} and the \textbf{CNN-only baseline} on \textbf{functionality classification}, where each binary must be assigned to one of the 22 functionality categories from Sec.~\ref{subsubsec:labeling}. This task is typically more challenging than toolchain provenance due to potential overlaps between functionalities (for instance, math blocks that also include timing components) and uneven class distributions.

\paragraph{Overall Performance.}
Table~\ref{tab:result_func} summarizes the performance of our models on the test set. 
First, we assessed the efficacy of the models on how well they classify the functionality of binaries compiled using a single toolchain, specifically \textit{OpenPLC V3} and \textit{OpenPLC V2}, as these toolchains have the most significant number of binaries in our dataset. 
Our model achieved an accuracy of 46.85\% and 41.68\% for \textit{OpenPLC V3} and \textit{OpenPLC V2}, respectively.

While these results demonstrate the potential of our approach, it is essential to acknowledge that the performance is not exceptionally high. This could be attributed to several factors, such as the intrinsic difficulty of the task and the diversity and complexity of PLC programs. The lower performance on metrics like Cohen's Kappa score (41.28\% and 37.04\% for \textit{OpenPLC V2} and \textit{V3}, respectively) suggests that our model is affected by class imbalance in the dataset.
We also evaluated our model on the entire \textit{\textit{PLC-BEAD}-Func} dataset, which includes binaries from all four toolchains. In this setting, our model achieved an accuracy of 42.28\% and a Cohen's Kappa score of 36.63\%.

Although the baseline CNN-only model performs reasonably well for more common categories, PLCEmbed (CNN+Transformer) achieves higher accuracy overall, suggesting that the transformer’s global context helps distinguish subtler functional differences scattered across the byte sequence. Some categories with fewer samples (for example, specialized building control blocks) exhibit higher variance in both models’ predictions.



\paragraph{Imbalanced Classes.}
Certain functionalities, such as \textit{Math} or \textit{Timing}, occur more frequently than others (for example, advanced building automation). Despite label weighting and the data filtering performed in Sec.~\ref{subsubsec:labeling}, a few categories remain underrepresented. Both PLCEmbed and the baseline see lower F1 scores in these rare classes, though PLCEmbed consistently outperforms the baseline by a small margin. This indicates that local byte patterns alone may not suffice to capture complex functionalities unless supported by a mechanism (such as self-attention) that integrates information from different parts of the binary.

\paragraph{Insights.} In practical ICS scenarios, analysts can use functionality classification to verify whether an uploaded binary aligns with the intended control logic. If a malicious actor replaced a “Timing” module with a “Network” module that exfiltrates data, for instance, a high-performing classifier might flag this discrepancy even without source code. However, these results confirm that functionality classification presents an additional layer of complexity beyond toolchain detection. 
Besides class imbalance, one of the main obstacles arises when binaries compiled with the same toolchain share significant code segments, despite having different functionalities.
This can be attributed to the use of common libraries, runtime environments, and compiler-specific optimizations that result in similar binary structures across different programs.


For instance, in our dataset, the \textit{ACTUATOR\_PUMP} and \textit{TIMER\_1} programs, both compiled using \textit{OPENPLC V3}, have different functionality labels but share a notable portion of their code segments. 
The \textit{ACTUATOR\_PUMP} program implements a pump interface controllable by an input, while the \textit{TIMER\_1} program realizes a simple timer that generates output pulses on selected days. 
Conversely, the same \textit{ACTUATOR\_PUMP} program compiled using \textit{OpenPLC V3} and \textit{GEB} exhibits differences in almost all code segments, despite having similar functionality.
 
These observations highlight the need for more advanced binary analysis techniques that go beyond simple structural analysis. Future research could explore methods that consider additional features, such as control flow graphs, data flow analysis, or ML-based approaches that can learn to distinguish between different functionalities based on higher-level patterns and abstractions.




\subsection{Robustness and Additional Experiments}
\label{subsec:robustness}

We also study the model's robustness to noise. Industrial forensics sometimes involves partially corrupted binaries or incomplete memory dumps. To simulate this, we flip 5\% and 10\% of the bytes in the test-set files at random. Table~\ref{tab:result_tp} and~\ref{tab:result_func} report toolchain and functionality accuracy under these conditions. Although performance drops compared to the unmodified files, the decrease is modest (1--3\% at most), suggesting that convolution filters and attention can tolerate moderate corruption as long as key compiler or functional patterns remain intact.



\subsection{Discussion}
\label{subsec:discussion}
Our experiments indicate that \textbf{PLCEmbed} (CNN+Transformer) surpasses the simpler \textbf{CNN-only baseline} in both toolchain provenance and functionality classification tasks. This performance gap is more pronounced when classes exhibit significant overlap in byte patterns or when only limited data is available. Although pure CNNs capture local sequences effectively, the global relationships modeled by the transformer seem essential for disambiguating bytes that occur far apart in the binary.

\paragraph{Practical Implications.}
From an ICS security perspective, a data-driven method that recovers high-level properties (compiler origin or functionality) from raw binaries offers several advantages. Investigators can identify suspicious binaries that appear to originate from unexpected compilers or that do not match the stated control logic.
Moreover, maintenance teams can detect variations introduced by updated compiler versions or emerging vendor-specific features that might be invisible without proprietary documentation.

\paragraph{Limitations.}
Although these results are promising, the dataset still represents a subset of possible ICS code scenarios, and certain rare functional categories remain difficult to classify accurately. 
Potential avenues for future work include exploring more advanced model architectures, incorporating additional sources of information (e.g., control flow graphs, memory access patterns), and developing techniques to handle class imbalance and data corruption more effectively.
Nonetheless, our findings underscore the value of raw-byte, compiler-agnostic approaches for PLC binary analysis.





% Experimental results on the PLC-BEAD dataset show that our approach outperforms a CNN-only baseline and remains robust in the face of moderate data corruption. These findings indicate that our model not only validates the usefulness of the dataset but also offers a promising pathway for automated ICS digital forensics and security analysis.

% In future work, we plan to expand the dataset further and explore complementary dynamic analysis techniques to enhance functionality prediction. We believe that releasing PLC-BEAD and the PLCEmbed framework as open-source resources will stimulate further research and lead to improved security practices for industrial control systems.

\section{Limitations and Future Work}
The proposed OpenFly platform incorporates various rendering engines/techniques to provide high-quality scenes. Specifically, this is the first attempt to use 3D GS reconstructed scenes to support real-to-sim training and testing, while in the reconstruction of large-scale areas, a few visual artifacts are inevitably present. Future work will focus on exploring more effective reconstruction methods to enhance realism in large-scale scenes. Besides, the proposed OpenFly-Agent is built upon the large VLN model architecture, which is not practical for real-time deployment on UAVs. To address this, future research should focus on developing more efficient architectures and effective quantization techniques. 


\section{Conclusion}
In this work, we present OpenFly, a platform designed for large-scale data collection in aerial Vision-and-Language Navigation (VLN). OpenFly integrates multiple rendering engines and advanced real-to-sim techniques for data generation, enabling efficient collection of diverse, high-quality aerial VLN data. The resulting large-scale dataset comprises 100k trajectories across 18 distinct scenes, spanning a wide range of altitudes and difficulty levels, which is significantly superior than existing ones. Furthermore, we propose OpenFly-Agent, a keyframe-aware aerial navigation model capable of directly predicting flight actions based on observations and language instructions. Extensive experiments validate the effectiveness of the proposed method, and establishing a comprehensive benchmark for future advancements in aerial navigation. 
%The toolchain, dataset, and code will be publicly released, providing a valuable resource for future research in this field.

{
    \small
    \bibliographystyle{ieeenat_fullname}
    \bibliography{references}
}

% WARNING: do not forget to delete the supplementary pages from your submission 
% 
\clearpage
% \setcounter{page}{1}
% \maketitlesupplementary
\begin{center}
Supplementary Material
\end{center}

% {
%     \onecolumn
%     \centering
%     \Large
%     \textbf{\thetitle}\\
%     \vspace{0.5em}Supplementary Material \\
%     \vspace{1.0em}
% }

\section{Proof of \cref{theorem:dr}}
We require some additional regularity assumptions:
\begin{assumption} 1) The number of classes $C$ is bounded w.r.t the number of samples $N$, 2) the missingness mechanism $P(A=1|Y,\theta)$, as well as its estimated counterpart $P(A=1|Y,\theta)$, are bounded below by some constant $\epsilon > 0$, 3) the quantities $P(Y|X,\theta)$ and $P(A|Y,\theta)$ are estimated using auxiliary samples independent of samples used for the sample averaging.
\label{assumption:extra}
\end{assumption}
Assumptions 1 and 2 are natural. For the missingness mechanism, the ground truth being bounded means that there is a non-vanishing proportion of samples for every class. The boundedness of the estimate can be enforced by clipping the estimate. Assumption 3 is called sample splitting in \cite{kennedy-dr}.

For convenience we use operator $\E_N$ to denote the average of $N$ samples i.e. $\frac{1}{N}\sum_{i=1}^N$. Note that this is by itself a random variable, in contrast to $\E$ which is a fixed number.

\begin{proof}[Proof of \cref{theorem:dr}] Because $C$ is bounded (assumption \ref{assumption:extra}), we can fix a class $c$ and prove the theorem.
Let us define the influence function $\phi$, parameterized by $\theta$, as
\begin{equation}
\phi(O | \theta)(c) = P(Y=c|X,\theta) + \frac{\one(A=1)}{P(A=1|Y,\theta)} (\one(Y=c) - P(Y=c|X,\theta)) - P(Y=c)
\end{equation}
As we have done in the main text, we use $\phi(O)$ to denote the same function but all estimated quantities are replaced with their truths. In other words, we use $\phi(O)$ for $\phi(O|\theta_0)$ where $\theta_0$ is the truth, given that our model contains $\theta_0$ e.g. when the model is consistent.

Recall that:
\begin{equation}
\begin{aligned}
\Psi_{dr}(\theta)(c) &= \frac{1}{N}\sum_{i=1}^N \left\{P(Y=c|X,\theta) + \frac{\one(A=1)}{P(A=1|Y,\theta)} (\one(Y=c) - P(Y=c|X,\theta))\right\}\\
&= \E_N [\phi(O|\theta)(c)] + P(Y=c)
\end{aligned}
\end{equation}

We will show that:
\begin{equation}
\Psi_{dr}(\theta)(c) - P(Y=c) = (\E_N - \E)[\phi(O)(c)] + o_P(N^{-1/2})
\label{eq:proof-linearity}
\end{equation}
To do that, we use the following decomposition
\begin{equation}
\begin{aligned}
\Psi_{dr}(\theta)(c) - P(Y=c) &= \E_N [\phi(O|\theta)(c)] \\
&= (\E_N - \E)[\phi(O)(c)] + (\E_N - \E)[\phi(O|\theta)(c) - \phi(O)(c)] + \E[\phi(O|\theta)(c)]
% &+ (\E_n - \E)[\phi(O;\theta) - \phi(O)]\\
% &+ \E[P(Y=c|X,\theta)] - \E[P(Y=c|X)] + \E[\phi(O,\theta)]
\end{aligned}
\end{equation}
and analyze the second and third term. The third term is:
\begin{equation}
\begin{aligned}
\E[\phi(O|\theta)(c)] &= \E[P(Y=c|X,\theta)] + \E\left[\frac{\one(A=1)}{P(A=1|Y,\theta)}(\one(Y=c) - P(Y=c|X,\theta))\right]- P(Y=c) \\
&= \E\left[P(Y=c|X,\theta) + \frac{P(A=1|Y)}{P(A=1|Y,\theta)}(P(Y=c|X) - P(Y=c|X,\theta))\right] - \E[P(Y=c|X)]\\
&= \E\left[(P(Y=c|X,\theta) - P(Y=c|X)) (P(A=1|Y,\theta) -P(A=1|Y)) \frac{1}{P(A=1|Y,\theta)}\right]\\
\end{aligned}
\end{equation}
by Cauchy-Schwarz inequality:
\begin{equation}
\begin{aligned}
\E[\phi(O|\theta)(c)] &\le \frac{1}{\epsilon} \|P(A=1|Y,\theta) - P(A=1|Y)\|_2 \|P(Y=c|X,\theta) - P(Y=c|X)\|_{L_2(P)}\\
&= \frac{1}{\epsilon} o_P(N^{-1/4} N^{-1/4}) = o_P(N^{-1/2})
\end{aligned}
\end{equation}
by assumption \ref{assumption:4th-root-n} and that $P(A=1|Y,\theta) > \epsilon$ (assumption \ref{assumption:extra}). The second term can be bounded by Chebyshev inequality
% \begin{equation}
% \begin{aligned}
% \E[\E_N[\phi(O|\theta)(c) - \phi(O)(c)]] &= \E[\phi(O|\theta)(c) - \phi(O)(c)]\\
% \var[\E_N[\phi(O|\theta)(c) - \phi(O)(c)]] &= \frac{1}{N}\var[\phi(O|\theta)(c) - \phi(O)(c)] \le 
% \end{aligned}
% \end{equation}
\begin{equation}
P(|(\E_N - \E)[\phi(O|\theta)(c) - \phi(O)(c)]| \ge t) \le \frac{\var[\E_N[\phi(O|\theta)(c) - \phi(O)(c)]]}{t^2} = \frac{\var[\phi(O|\theta)(c) - \phi(O)(c)]}{Nt^2}
\end{equation}
note here that $\theta$ is independent of the samples used for $\E_N$ by assumption \ref{assumption:extra}. For any $\varepsilon > 0$, by picking $t = \frac{1}{\sqrt{N\varepsilon}}$ we get
\begin{equation}
P\left(\left|\frac{(\E_N - \E)[\phi(O|\theta)(c) - \phi(O)(c)]}{N^{-1/2}}\right| \ge \frac{1}{\sqrt{\varepsilon}}\right) \le \varepsilon \var[\phi(O|\theta)(c) - \phi(O)(c)]
\end{equation}
by the definition of $O_P$, we then get
\begin{equation}
(\E_N - \E)[\phi(O|\theta)(c) - \phi(O)(c)] = O_P(N^{-1/2}\var[\phi(O|\theta)(c) - \phi(O)(c)])
\end{equation}
Because $\phi$ is a continuous function of $P(Y|X,\theta)$ and $P(A|Y,\theta)$ (given $P(A|Y,\theta) > \epsilon$, assumption \ref{assumption:extra}), by the continuous mapping theorem and the fact that $P(Y|X,\theta)$ and $P(A|Y,\theta)$ are convergent in probability (assumption \ref{assumption:4th-root-n}), we get $\var[\phi(O|\theta)(c) - \phi(O)(c)] = o_P(1)$. This gives
\begin{equation}
(\E_N - \E)[\phi(O|\theta)(c) - \phi(O)(c)] = o_P(N^{-1/2})
\end{equation}
Therefore, we have shown that the second and third term are both $o_P(N^{-1/2})$, proving \cref{eq:proof-linearity}. As the final step, multiply both sides of this equation by $\sqrt{N}$ we get:
\begin{equation}
\sqrt{N}(\Psi_{dr}(\theta)(c) - P(Y=c)) = \sqrt{N} (\E_N - \E)[\phi(O)(c)] + o_P(1) \rightsquigarrow \mathcal{N}(0, \var[\phi(O)(c)])
\end{equation}
by the central limit theorem, and $\var[\phi(O)(c)] = \E[\phi(O)(c)^2]$ because $\E[\phi(O)(c)] = 0$.
\end{proof}

While we started with the definition of $\phi$, \cref{eq:proof-linearity} shows that $\phi$ is indeed an influence function. Now we show that $\phi$ is also the efficient influence function, by using the characterization of the model's tangent space \cite{tsiatis-missingdata}. Note that the joint probability factorizes as $P(X,A,Y) = P(X)P(Y|X)P(A|Y)$, therefore the tangent space $\mathcal{T}$ factorizes as $\mathcal{T} = \mathcal{T}_{X} \oplus \mathcal{T}_{Y|X} \oplus \mathcal{T}_{A|Y}$ where $\mathcal{T}_X = \{h(X): \E[h] = 0\}$, $\mathcal{T}_{Y|X} = \{h(X,Y): \E[h|X] = 0\}$, $\mathcal{T}_{A|Y} = \{h(A,Y): \E[h|Y] = 0\}$, and the 3 subspaces are pairwise orthogonal. All influence functions are orthogonal to the tangent space, but the influence function that is also in the tangent space has the smallest variance and is called the efficient influence function. As $\phi$ is already an influence function, we need only show that $\phi$ is in $\mathcal{T}$. We write $\phi$ as
\begin{equation}
\phi(O)(c) = (P(Y=c|X) - P(Y=c)) + \left[\frac{\one(A=1)}{P(A=1|Y)} - 1\right](\one(Y=c) - P(Y=c|X)) + (\one(Y=c) - P(Y=c|X))
\end{equation}
and note that the first, second and third term are in $\mathcal{T}_X$, $\mathcal{T}_{A|Y}$ and $\mathcal{T}_{Y|X}$ respectively. Therefore, $\phi$ is indeed in $\mathcal{T}$. The efficient influence function has the smallest variance of all influence function, and therefore our estimator being asymptotically linear in $\phi$ (\cref{eq:proof-linearity}) has the smallest mean squared error in a local asymptotic minimax sense \cite{kennedy-dr, asymptoticstatistics}

\section{Further background and related work}
\paragraph{Discussion on semi-supervised EM.}
It appears that semi-supervised EM was first used for parameter estimation when the missingness mechanism is non-ignorable in \cite{ibrahim1996parameter}, but has not been used for label shift estimation.
Perhaps this is because the semi-supervised situation where additional unlabeled data is available during training is rarer than the test-time adaptation case. EM is well suited to take advantage of the extra unlabeled data to improve the classifier under very scarce and long-tailed labeled data. While the connection between pseudo-labeling and EM has been explored before \cite{entropyminimization}, the situation with label shift has not until recently \cite{simpro}. Here the application of EM is much more interesting, because other than simply giving pseudo-labeling a rigorous formulation, EM also estimates the missingness mechanism (equivalently the label distribution shift), which is important for shift correction and thus high-quality pseudo-labels \cite{acr}. The application of confidence thresholding can be seen as a sparse variant of EM \cite{neal1998view}.

\paragraph{The doubly-robust risk.} 
\label{subsec:dr-risk}
A technique that also derives from the theory of semi-parametric efficiency is orthogonal statistical learning \citep{foster2023orthogonal}. The idea is to minimize the doubly-robust risk:
\label{subsec:method-dr-risk}
\begin{equation}
\label{eq:dr-risk}
\mathcal{R}(\theta_2) = \frac{1}{N} \sum_{i=1}^N \Bigg[ l(x_i, \hat y_i|\theta_2) + \frac{\one(a_i=1)}{P(A=a_i|Y=y_i, \theta_1)} (l(x_i, y_i | \theta_2) - l(x_i, \hat y_i | \theta_2))\Bigg]
\end{equation}
where $l(x,y|\theta) = -\sum_{c=1}^C [y]_c \log P(Y=c|X=x,\theta)$ is the negative cross-entropy. 
The notation $[y]_c$ means that we are using the $c$-entry in a C-dimension probability vector $y$. 
Thus, $y_i$ denotes the one-hot label of observation $i$, while $\hat y_i$ denotes the pseudo-label, which can be one-hot or all-zero. 
Finally, we use $\theta_1$ to denote that $P(a|y,\theta_1)$ is an estimation from a previous stage, but it can be estimated with $\theta_2$ as well. 
The risk $\mathcal{R}(\theta_2)$ can be used as a training loss in a straightforward fashion. 
Similar to the doubly robust estimation of $P(Y)$, the doubly robust risk provides approximately unbiased estimation of the risk. 
This property has been used in \citep{arelabelsinformative, onnonrandommissinglabels, drst} also in the semi-supervised learning setting.
More broadly, it is at the heart of one of the core techniques in heterogenous treatment effect estimation in causal estimation \cite{kennedy2023towards, foster2023orthogonal, wager2018estimation}. 
The focus here is not the estimation of $\mathcal{R}(\theta_2)$ per se, but the quality of the learned model \cite{foster2023orthogonal}.
By using the doubly-robust risk, we can achieve an optimality result similar in spirit to our theorem \cref{theorem:dr}, but for the generalization error.
While this is appealing, in practice there are 2 problems with this approach. First, the inverse probability weight $P(A=a_i|Y=y_i,\theta_1)$ can be very large if the class ratio is highly unlabeled, making training unstable \cite{kallus2020deepmatch, pham2023stable}. 
This problem exists for our estimation as well. However, it is much easier to control for estimation than for training because of the iterative nature of model update. Secondly, we can further write $\mathcal{R}$ as:
\begin{equation}
\mathcal{R}(\theta_2) = \frac{1}{N}\sum_{i=1}^N l\left(x_i, \hat y_i + \frac{\one(a_i=1)}{P(A=a_i|Y=y_i,\theta_1)} (y_i - \hat y_i)\Bigg\vert\theta_2\right)
\end{equation}
which is a cross-entropy loss with new meta-pseudo-labels. However, these labels are not meant to be learned exactly, and furthermore they can be negative. Thus, theoretical works have to put stringent assumptions on the models. In \cref{subsec:ablation-1}, we show that experimentally that the instability problem makes doubly-robust risk performance worse than our 2-stage approach.

\section{Training and hyperparameter settings.}
\label{subsec:training-setting}
For neural network training, we follow the implementation and hyperparameter settings of \cite{simpro}. In particular, we adapt the core code of SimPro for Supervised, MLE and EM. For MLE, we update $P(A|Y)$ using the Adam optimizer with learning rate 1e-3, while for EM we use a momentum update similar to SimPro's update of $P(Y|A)$ because it has a a closed-form solution at each mini-batch. We use Wide ResNet-28-2 on all methods and all datasets in this section, including Imagenet-127, because we are motivated by the fact that stage-1's goal is not classification accuracy but the estimation of a finite-dimensional parameter. When using Wide ResNet-28-2 for Imagenet-127, we use the hyperparameters of CIFAR-100, except we lower the batch size of unlabeled data to 2 times that of labeled data instead of 8 for memory reason. We do not perform additional hyperparameter tuning. All experiments can be performed on 1 A6000 RTX GPU, and are run 3 times. We report the total variation distance between the estimated and the ground truth unlabeled class distribution, similar to its usage in Theorem 3.1 of \cite{lsc}, and the top-1 classification accuracy.

In the second stage of our algorithm, we freeze our estimation and plug it in SimPro and BOAT.
We keep exactly the same hyperparameter settings that SimPro and BOAT use. In particular, for Imagenet-127, we now use ResNet-50 and run each experiment once.
In SimPro, we set the unlabeled class distribution $P(Y|A=0)$ at the E-step;  however, we still keep a running estimate of the class distribution $P(Y)$ in the logit adjustment loss \cref{eq:simpro-la-loss}. While it is possible to use the first stage estimate in the logit adjustment loss, we observe that doing so results in lower accuracy than using the the running average. This is conceptually consistent with the role of the running average - serving not as an accurate estimate of $P(Y)$ but to make the classifier's class distribution uniform through the logit adjustment loss, which is good for the test set. Similarly, in BOAT, we only replace $\Delta_c = \log P(Y|A=1) - \log P(Y|A=0)$ in equation (4) of \cite{boat}, which is adjusting a classifier's predictions from the labeled to the unlabeled class distribution, with our SimPro + DR estimate instead of their on-the-fly estimate. 


% \section{Additional experiments}
% % \begin{table*}[t]
\centering
\caption{Total Variation Distance on CIFAR-10-LT ($N_l = 500$, $M_l = 4000$) with different class imbalance ratios $\gamma_l$ and $\gamma_u$ under five different unlabeled class distributions.}
\label{tab:cifar10-tv}
\resizebox{\textwidth}{!}{
\begin{tabular}{lccccccccccc}
\toprule
& & \multicolumn{2}{c}{consistent} & \multicolumn{2}{c}{uniform} & \multicolumn{2}{c}{reversed} & \multicolumn{2}{c}{middle} & \multicolumn{2}{c}{head-tail} \\
\cmidrule(lr){3-4} \cmidrule(lr){5-6} \cmidrule(lr){7-8} \cmidrule(lr){9-10} \cmidrule(lr){11-12}
& & $\gamma_l = 150$ & $\gamma_l = 100$ & $\gamma_l = 150$ & $\gamma_l = 100$ & $\gamma_l = 150$ & $\gamma_l = 100$ & $\gamma_l = 150$ & $\gamma_l = 100$ & $\gamma_l = 150$ & $\gamma_l = 100$ \\
Model & Estimator & $\gamma_u = 150$ & $\gamma_u = 100$ & $\gamma_u = 1$ & $\gamma_u = 1$ & $\gamma_u = 1/150$ & $\gamma_u = 1/100$ & $\gamma_u = 150$ & $\gamma_u = 100$ & $\gamma_u = 150$ & $\gamma_u = 100$ \\
\midrule
Supervised & MLLS & 0.269 ± 0.252 & 0.038 ± 0.006 & 0.251 ± 0.046 & 0.255 ± 0.060 & 0.429 ± 0.028 & 0.493 ± 0.050 & 0.333 ± 0.042 & 0.320 ± 0.009 & 0.457 ± 0.034 & 0.444 ± 0.043 \\
Supervised & RLLS & 0.043 ± 0.001 & 0.044 ± 0.010 & 0.348 ± 0.034 & 0.305 ± 0.068 & 0.769 ± 0.016 & 0.678 ± 0.028 & 0.430 ± 0.008 & 0.368 ± 0.013 & 0.539 ± 0.018 & 0.503 ± 0.020 \\
\midrule
MLE & IPW & 0.027 ± 0.001 & 0.027 ± 0.000 & 0.319 ± 0.072 & 0.243 ± 0.010 & 0.674 ± 0.020 & 0.646 ± 0.041 & 0.438 ± 0.020 & 0.454 ± 0.026 & 0.547 ± 0.049 & 0.491 ± 0.059 \\
MLE & OR & 0.045 ± 0.004 & 0.042 ± 0.000 & 0.215 ± 0.026 & 0.203 ± 0.032 & 0.433 ± 0.017 & 0.395 ± 0.033 & 0.193 ± 0.006 & 0.209 ± 0.037 & 0.307 ± 0.147 & 0.249 ± 0.130 \\
MLE & DR & 0.090 ± 0.002 & 0.079 ± 0.000 & 0.407 ± 0.027 & 0.360 ± 0.007 & 0.425 ± 0.007 & 0.421 ± 0.029 & 0.256 ± 0.001 & 0.286 ± 0.031 & 0.435 ± 0.136 & 0.362 ± 0.122 \\
\midrule
EM & IPW & 0.035 ± 0.002 & 0.040 ± 0.001 & 0.021 ± 0.001 & 0.029 ± 0.015 & 0.303 ± 0.187 & 0.091 ± 0.010 & 0.119 ± 0.011 & 0.105 ± 0.022 & 0.104 ± 0.026 & 0.104 ± 0.051 \\
EM & OR & 0.037 ± 0.003 & 0.042 ± 0.002 & 0.016 ± 0.001 & 0.024 ± 0.012 & 0.269 ± 0.183 & 0.090 ± 0.008 & 0.122 ± 0.012 & 0.103 ± 0.022 & 0.072 ± 0.012 & 0.073 ± 0.024 \\
EM & DR & 0.034 ± 0.004 & 0.037 ± 0.001 & 0.014 ± 0.001 & 0.027 ± 0.020 & 0.264 ± 0.191 & 0.092 ± 0.005 & 0.111 ± 0.019 & 0.097 ± 0.026 & 0.077 ± 0.016 & 0.073 ± 0.028 \\
\midrule
SimPro & IPW & 0.070 ± 0.011 & 0.058 ± 0.000 & 0.046 ± 0.001 & 0.049 ± 0.005 & 0.254 ± 0.074 & 0.223 ± 0.098 & 0.097 ± 0.025 & 0.067 ± 0.002 & 0.105 ± 0.066 & 0.110 ± 0.079 \\
SimPro & OR & 0.071 ± 0.012 & 0.058 ± 0.000 & 0.045 ± 0.001 & 0.049 ± 0.006 & 0.040 ± 0.003 & 0.059 ± 0.017 & 0.074 ± 0.006 & 0.075 ± 0.002 & 0.033 ± 0.003 & 0.033 ± 0.003 \\
SimPro & DR & 0.017 ± 0.004 & 0.026 ± 0.001 & 0.019 ± 0.002 & 0.018 ± 0.003 & 0.039 ± 0.003 & 0.058 ± 0.025 & 0.091 ± 0.007 & 0.031 ± 0.001 & 0.015 ± 0.003 & 0.019 ± 0.007 \\
\bottomrule
\end{tabular}
}
\end{table*}
% 

\begin{table*}[t]
\centering
\caption{Total Variation Distance on CIFAR-100-LT ($N_l = 50$, $M_l = 400$) with different class imbalance ratios $\gamma_l$ and $\gamma_u$ under five different unlabeled class distributions.}
\label{tab:cifar100-tv}
\resizebox{\textwidth}{!}{
\begin{tabular}{lccccccccccc}
\toprule
& & \multicolumn{2}{c}{consistent} & \multicolumn{2}{c}{uniform} & \multicolumn{2}{c}{reversed} & \multicolumn{2}{c}{middle} & \multicolumn{2}{c}{head-tail} \\
\cmidrule(lr){3-4} \cmidrule(lr){5-6} \cmidrule(lr){7-8} \cmidrule(lr){9-10} \cmidrule(lr){11-12}
& & $\gamma_l = 20$ & $\gamma_l = 10$ & $\gamma_l = 20$ & $\gamma_l = 10$ & $\gamma_l = 20$ & $\gamma_l = 10$ & $\gamma_l = 20$ & $\gamma_l = 10$ & $\gamma_l = 20$ & $\gamma_l = 10$ \\
Model & Estimator & $\gamma_u = 20$ & $\gamma_u = 10$ & $\gamma_u = 1$ & $\gamma_u = 1$ & $\gamma_u = 1/20$ & $\gamma_u = 1/10$ & $\gamma_u = 20$ & $\gamma_u = 10$ & $\gamma_u = 20$ & $\gamma_u = 10$ \\
\midrule
Supervised & MLLS & 0.707 ± 0.016 & 0.313 ± 0.100 & 0.445 ± 0.172 & 0.309 ± 0.119 & 0.383 ± 0.075 & 0.397 ± 0.006 & 0.570 ± 0.001 & 0.373 ± 0.107 & 0.543 ± 0.009 & 0.231 ± 0.057 \\
Supervised & RLLS & 0.520 ± 0.007 & 0.133 ± 0.003 & 0.337 ± 0.125 & 0.253 ± 0.082 & 0.424 ± 0.060 & 0.463 ± 0.003 & 0.454 ± 0.021 & 0.306 ± 0.074 & 0.460 ± 0.028 & 0.241 ± 0.040 \\
\midrule
MLE & IPW & 0.075 ± 0.000 & 0.071 ± 0.001 & 0.229 ± 0.001 & 0.167 ± 0.002 & 0.565 ± 0.005 & 0.443 ± 0.007 & 0.415 ± 0.000 & 0.311 ± 0.005 & 0.343 ± 0.000 & 0.280 ± 0.001 \\
MLE & OR & 0.065 ± 0.002 & 0.061 ± 0.001 & 0.200 ± 0.007 & 0.143 ± 0.001 & 0.526 ± 0.011 & 0.399 ± 0.023 & 0.360 ± 0.003 & 0.256 ± 0.012 & 0.328 ± 0.003 & 0.266 ± 0.005 \\
MLE & DR & 0.149 ± 0.019 & 0.145 ± 0.010 & 0.243 ± 0.004 & 0.214 ± 0.019 & 0.568 ± 0.005 & 0.464 ± 0.014 & 0.403 ± 0.014 & 0.309 ± 0.012 & 0.365 ± 0.007 & 0.320 ± 0.004 \\
\midrule
EM & IPW & 0.097 ± 0.008 & 0.092 ± 0.004 & 0.239 ± 0.007 & 0.179 ± 0.003 & 0.478 ± 0.012 & 0.329 ± 0.020 & 0.262 ± 0.016 & 0.202 ± 0.003 & 0.312 ± 0.002 & 0.227 ± 0.001 \\
EM & OR & 0.121 ± 0.007 & 0.108 ± 0.005 & 0.261 ± 0.007 & 0.189 ± 0.004 & 0.489 ± 0.013 & 0.335 ± 0.020 & 0.274 ± 0.016 & 0.211 ± 0.004 & 0.336 ± 0.003 & 0.235 ± 0.001 \\
EM & DR & 0.125 ± 0.005 & 0.111 ± 0.004 & 0.269 ± 0.007 & 0.194 ± 0.005 & 0.497 ± 0.010 & 0.336 ± 0.024 & 0.281 ± 0.019 & 0.219 ± 0.008 & 0.336 ± 0.007 & 0.233 ± 0.004 \\
\midrule
SimPro & IPW & 0.125 ± 0.001 & 0.100 ± 0.005 & 0.166 ± 0.007 & 0.141 ± 0.009 & 0.353 ± 0.023 & 0.261 ± 0.008 & 0.202 ± 0.003 & 0.158 ± 0.005 & 0.277 ± 0.009 & 0.197 ± 0.003 \\
SimPro & OR & 0.133 ± 0.005 & 0.100 ± 0.004 & 0.160 ± 0.007 & 0.138 ± 0.010 & 0.322 ± 0.014 & 0.253 ± 0.008 & 0.202 ± 0.003 & 0.156 ± 0.005 & 0.269 ± 0.006 & 0.191 ± 0.004 \\
SimPro & DR & 0.122 ± 0.003 & 0.106 ± 0.006 & 0.188 ± 0.009 & 0.149 ± 0.006 & 0.343 ± 0.023 & 0.257 ± 0.007 & 0.219 ± 0.010 & 0.172 ± 0.002 & 0.279 ± 0.007 & 0.198 ± 0.004 \\
\bottomrule
\end{tabular}
}
\end{table*}

\end{document}
