\begin{abstract}
% Zeyu comment 1107: 目前我改好的这版尽量做到了准确描述 descriptive,但是感觉还没有突出研究的动机(为什么重要、为什么难、为什么现有方法不好解决、我们的核心技术创新点在哪里)。请过一遍我的edits后再改一轮,注意逻辑的连贯性和sharpness。
% Professional 3D asset creation is an iterative and engaging process using sculpting brushes within modeling software. 
% Professional 3D asset creations require dynamic and iterative designs, with sculpting brushes serving as essential tools for precision and creativity. 
Professional 3D asset creation often requires diverse sculpting brushes to add surface details and geometric structures.
Despite recent progress in 3D generation, producing reusable sculpting brushes compatible with artists' workflows remains an open and challenging problem.
These sculpting brushes are typically represented as vector displacement maps (VDMs), which existing models cannot easily generate compared to natural images.
This paper presents Text2VDM, a novel framework for text-to-VDM brush generation through the deformation of a dense planar mesh guided by score distillation sampling (SDS).
The original SDS loss is designed for generating full objects and struggles with generating desirable sub-object structures from scratch in brush generation.
We refer to this issue as semantic coupling, which we address by introducing classifier-free guidance (CFG) weighted blending of prompt tokens to SDS, resulting in a more accurate target distribution and semantic guidance.
Experiments demonstrate that Text2VDM can generate diverse, high-quality VDM brushes for sculpting surface details and geometric structures.
Our generated brushes can be seamlessly integrated into mainstream modeling software, enabling various applications such as mesh stylization and real-time interactive modeling.

% and enable repeatedly applying the geometry pattern of surface details or 3D components 

\end{abstract}
% To address semantic coupling issues, which introduce unwanted associative semantic noise beyond the intended text description in the original score distillation sampling (SDS) process,
% causing semantic coupling in brush generation for sub-object structures without context
% Unlike traditional height maps, VDMs store 3D information in highly abstracted RGB images, making them difficult to generate directly using existing text-to-image models.
% supporting sculpting surface details and 3D components.
% This framework aims to achieve three goals: 1) accurate representation capability of brush for surface styles and 3D components, 2) precise controllability for brush generation that satisfies the expected outcomes, and 3) direct usability of brush in mainstream modeling software. 
% We first reconstruct a dense mesh from an initialized vector displacement map, allowing control over the brush volume and generation region while ensuring the generated mesh can be directly baked as a brush.
% We then use an optimization process of mesh deformation via the Laplace-Beltrami operator to generate expressive and controllable brushes.
% Professional 3D asset creation is an iterative process using sculpting brushes within modeling software. 
% While impressive progress has been made in existing 3D generation work, creating VDM brushes that align with artists' workflows and enable repeatedly applying the geometry pattern of surface details and 3D components remains an open and challenging problem.
% In this work, we present Text2VDM, a novel approach for text-to-VDM brush generation, aiming to achieve three goals: 1) represent surface styles and 3D components effectively; 2) provide precise control methods beyond text to meet expected outcomes; and 3) ensure direct usability in mainstream modeling software. We designed a two-stage framework. The first stage involves reconstructing a dense mesh from the initialized VDM, allowing for precise control over brush volume and generation region while ensuring that the resulting mesh can be directly baked as a stylized brush. In the second stage, we focus on brush optimizations, where we introduce a CFG-weighted score-distillation sampling loss to resolve the semantic coupling issues inherent in the original score-distillation sampling method. 

% We demonstrate that Text2VDM can generate diverse, high-quality surface brushes and component brushes. Our generated brushes can integrate seamlessly into existing modeling workflows, enabling various applications such as mesh stylization and real-time interactive modeling.
