\begin{table*}[h!] \small
    \caption{An example of how PRM helps identify high-quality actions that lead to correct answers. The unverified claims in the presented example are \sethlcolor{lightred}\hl{highlighted}.}
    \centering
    \begin{tabular}{p{16.5cm}}
        \toprule
        \bf \makecell[c]{MedQA Question} \\
        \midrule
\textbf{Question}: A 60-year-old woman presents to a physician for worsening shortness of breath and increasing abdominal distention over the last 3 months. She says that the shortness of breath is worse on exertion and improves with rest. While she could previously walk to the nearby store for her groceries, she now has to drive because she gets ''winded'' on the way. The patient was diagnosed with diabetes 5 years ago and is compliant with her medications. The medical history is otherwise unremarkable. The physical examination reveals gross ascites and visibly engorged periumbilical veins. Bilateral pitting edema is noted around the ankles. The finger-prick blood glucose level is 100 mg/dL. What is the mechanism of action of the anti-diabetic medication this patient is most likely taking? \textbf{A.} Binding to the alpha subunit of the insulin receptor, \textbf{B.} Closure of ATP-sensitive K-channels in the pancreatic beta-cell, \textbf{C.} Increased gene expression of GLUT-4, \textbf{D.} Glucagon-like peptide-1 receptor agonist \\
\textbf{Answer} (invisible): C \\
        \midrule
        \rowcolor[RGB]{204, 255, 204}
        \bf \makecell[c]{Action 1} \\
        % \midrule
        \textbf{[Candidate 1 -- Reasoning]}
        ...\\
        \textbf{[Candidate 1 -- Action]}\\
        \textit{Query}: What is the primary concern with the patient's condition, specifically whether heart failure is confirmed based on symptoms and physical examination findings? \faThumbsDown \\
        \textbf{[Candidate 2 -- Reasoning]}\\
        1. The patient presents with worsening shortness of breath, increasing abdominal distention, and gross ascites. These symptoms, along with visibly engorged periumbilical veins and bilateral pitting edema, are indicative of fluid overload and congestive heart failure. \\
        2. The patient's history of diabetes, which is compliant with medication, suggests that her anti-diabetic medication is likely a type of oral hypoglycemic agent rather than insulin. \\
        3. Considering the patient's age (60 years) and symptoms, it is unlikely that she is taking a glucagon-like peptide-1 (GLP-1) receptor agonist, which is typically used in younger patients with type 2 diabetes and has a different side-effect profile.\\
        4. With the remaining options, 'Binding to the alpha subunit of the insulin receptor' and 'Closure of ATP-sensitive K-channels in the pancreatic beta-cell', we can narrow down the possibilities. 'Binding to the alpha subunit of the insulin receptor' is more characteristic of insulin receptor agonists or insulin itself, which is unlikely given the patient's age and the oral nature of her medication. Therefore, \sethlcolor{lightred}\hl{'Closure of ATP-sensitive K-channels in the pancreatic beta-cell' is the most plausible mechanism of action}. \\
        5. This mechanism is consistent with the \sethlcolor{lightred}\hl{sulfonylureas class of medications}, which are commonly used in patients with type 2 diabetes. \\
        \textbf{[Candidate 2 -- Action]}\\
        \textit{Query}: What are common classes of oral anti-diabetic medications? \faThumbsUp \\
        \midrule
        \rowcolor[RGB]{204, 229, 255}
        \bf \makecell[c]{Environment Feedback 1} \\
        \textbf{[+ Documents]}\\
        Common classes of oral anti-diabetic medications include: 1. Sulfonylureas, 2. Dipeptidyl Peptidase IV (DPP IV) inhibitors (gliptins), 3. Thiazolidinediones, 4. Meglitinides, 5. Sodium-glucose co-transporter 2 (SGLT 2) inhibitors, 6. Alpha-glucosidase inhibitors, 7. Biguanides (metformin).\\
        \midrule
        \rowcolor[RGB]{204, 255, 204}
        \bf \makecell[c]{Action 2} \\
        \textbf{[Candidate 1 -- Reasoning]}
        ...\\
        \textbf{[Candidate 1 -- Action]}\\
        \textit{Query}: Is metformin commonly prescribed for type 2 diabetes and known to cause lactic acidosis as a side effect? \faThumbsDown\\
        \textbf{[Candidate 2 -- Reasoning]}
        ...\\
        \textbf{[Candidate 2 -- Action]}\\
        \textit{Query}: What is the mechanism of action of Biguanides (metformin)? \faThumbsUp\\
        \midrule
        \rowcolor[RGB]{204, 229, 255}
        \bf \makecell[c]{Environment Feedback 2} \\
        \textbf{[+ Documents]}\\
        The mechanism of action of Biguanides (metformin) is to activate the enzyme AMP-activated protein kinase (AMPK) and reduce hepatic glucose production.\\
        \midrule
        \rowcolor[RGB]{204, 255, 204}
        \bf \makecell[c]{Action 3} \\
        \textbf{[Candidate 1 -- Reasoning]}
        ...
        \textbf{[Candidate 1 -- Action]}
        \textit{Answer}: B \faThumbsDown  \\
        \textbf{[Candidate 2 -- Reasoning]}
        ...
        \textbf{[Candidate 2 -- Action]}
        \textit{Answer}: C \faThumbsUp  \\
        \bottomrule
    \end{tabular}
    \label{tab:case_study}
\end{table*}