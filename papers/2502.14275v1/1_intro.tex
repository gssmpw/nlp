\section{Introduction}
Large language models (LLMs) such as GPT-4~\cite{openai2023gpt4} and Llama3~\cite{dubey2024llama} have demonstrated remarkable generative capabilities, achieving state-of-the-art performance in various NLP tasks.
However, their ability to accurately recall and apply domain-specific knowledge remains a major challenge, particularly in high-stakes fields such as healthcare~\cite{li2022neural, bi2024decoding, yang2023large, liu2023evaluating, yan2023multimodal}. LLMs are prone to hallucinations-generating plausible but incorrect content~\cite{ji2023survey, bang2023multitask, zhang2023siren}-which raises concerns about their factual reliability~\cite{chen2023hallucination, li-etal-2024-dawn}. 
These domain-specific knowledge gaps often lead to contradictions, misinformation, and overconfident yet incorrect predictions, making them unsuitable for direct deployment in clinical decision-making. 

\section{Overview}

\revision{In this section, we first explain the foundational concept of Hausdorff distance-based penetration depth algorithms, which are essential for understanding our method (Sec.~\ref{sec:preliminary}).
We then provide a brief overview of our proposed RT-based penetration depth algorithm (Sec.~\ref{subsec:algo_overview}).}



\section{Preliminaries }
\label{sec:Preliminaries}

% Before we introduce our method, we first overview the important basics of 3D dynamic human modeling with Gaussian splatting. Then, we discuss the diffusion-based 3d generation techniques, and how they can be applied to human modeling.
% \ZY{I stopp here. TBC.}
% \subsection{Dynamic human modeling with Gaussian splatting}
\subsection{3D Gaussian Splatting}
3D Gaussian splatting~\cite{kerbl3Dgaussians} is an explicit scene representation that allows high-quality real-time rendering. The given scene is represented by a set of static 3D Gaussians, which are parameterized as follows: Gaussian center $x\in {\mathbb{R}^3}$, color $c\in {\mathbb{R}^3}$, opacity $\alpha\in {\mathbb{R}}$, spatial rotation in the form of quaternion $q\in {\mathbb{R}^4}$, and scaling factor $s\in {\mathbb{R}^3}$. Given these properties, the rendering process is represented as:
\begin{equation}
  I = Splatting(x, c, s, \alpha, q, r),
  \label{eq:splattingGA}
\end{equation}
where $I$ is the rendered image, $r$ is a set of query rays crossing the scene, and $Splatting(\cdot)$ is a differentiable rendering process. We refer readers to Kerbl et al.'s paper~\cite{kerbl3Dgaussians} for the details of Gaussian splatting. 



% \ZY{I would suggest move this part to the method part.}
% GaissianAvatar is a dynamic human generation model based on Gaussian splitting. Given a sequence of RGB images, this method utilizes fitted SMPLs and sampled points on its surface to obtain a pose-dependent feature map by a pose encoder. The pose-dependent features and a geometry feature are fed in a Gaussian decoder, which is employed to establish a functional mapping from the underlying geometry of the human form to diverse attributes of 3D Gaussians on the canonical surfaces. The parameter prediction process is articulated as follows:
% \begin{equation}
%   (\Delta x,c,s)=G_{\theta}(S+P),
%   \label{eq:gaussiandecoder}
% \end{equation}
%  where $G_{\theta}$ represents the Gaussian decoder, and $(S+P)$ is the multiplication of geometry feature S and pose feature P. Instead of optimizing all attributes of Gaussian, this decoder predicts 3D positional offset $\Delta{x} \in {\mathbb{R}^3}$, color $c\in\mathbb{R}^3$, and 3D scaling factor $ s\in\mathbb{R}^3$. To enhance geometry reconstruction accuracy, the opacity $\alpha$ and 3D rotation $q$ are set to fixed values of $1$ and $(1,0,0,0)$ respectively.
 
%  To render the canonical avatar in observation space, we seamlessly combine the Linear Blend Skinning function with the Gaussian Splatting~\cite{kerbl3Dgaussians} rendering process: 
% \begin{equation}
%   I_{\theta}=Splatting(x_o,Q,d),
%   \label{eq:splatting}
% \end{equation}
% \begin{equation}
%   x_o = T_{lbs}(x_c,p,w),
%   \label{eq:LBS}
% \end{equation}
% where $I_{\theta}$ represents the final rendered image, and the canonical Gaussian position $x_c$ is the sum of the initial position $x$ and the predicted offset $\Delta x$. The LBS function $T_{lbs}$ applies the SMPL skeleton pose $p$ and blending weights $w$ to deform $x_c$ into observation space as $x_o$. $Q$ denotes the remaining attributes of the Gaussians. With the rendering process, they can now reposition these canonical 3D Gaussians into the observation space.



\subsection{Score Distillation Sampling}
Score Distillation Sampling (SDS)~\cite{poole2022dreamfusion} builds a bridge between diffusion models and 3D representations. In SDS, the noised input is denoised in one time-step, and the difference between added noise and predicted noise is considered SDS loss, expressed as:

% \begin{equation}
%   \mathcal{L}_{SDS}(I_{\Phi}) \triangleq E_{t,\epsilon}[w(t)(\epsilon_{\phi}(z_t,y,t)-\epsilon)\frac{\partial I_{\Phi}}{\partial\Phi}],
%   \label{eq:SDSObserv}
% \end{equation}
\begin{equation}
    \mathcal{L}_{\text{SDS}}(I_{\Phi}) \triangleq \mathbb{E}_{t,\epsilon} \left[ w(t) \left( \epsilon_{\phi}(z_t, y, t) - \epsilon \right) \frac{\partial I_{\Phi}}{\partial \Phi} \right],
  \label{eq:SDSObservGA}
\end{equation}
where the input $I_{\Phi}$ represents a rendered image from a 3D representation, such as 3D Gaussians, with optimizable parameters $\Phi$. $\epsilon_{\phi}$ corresponds to the predicted noise of diffusion networks, which is produced by incorporating the noise image $z_t$ as input and conditioning it with a text or image $y$ at timestep $t$. The noise image $z_t$ is derived by introducing noise $\epsilon$ into $I_{\Phi}$ at timestep $t$. The loss is weighted by the diffusion scheduler $w(t)$. 
% \vspace{-3mm}

\subsection{Overview of the RTPD Algorithm}\label{subsec:algo_overview}
Fig.~\ref{fig:Overview} presents an overview of our RTPD algorithm.
It is grounded in the Hausdorff distance-based penetration depth calculation method (Sec.~\ref{sec:preliminary}).
%, similar to that of Tang et al.~\shortcite{SIG09HIST}.
The process consists of two primary phases: penetration surface extraction and Hausdorff distance calculation.
We leverage the RTX platform's capabilities to accelerate both of these steps.

\begin{figure*}[t]
    \centering
    \includegraphics[width=0.8\textwidth]{Image/overview.pdf}
    \caption{The overview of RT-based penetration depth calculation algorithm overview}
    \label{fig:Overview}
\end{figure*}

The penetration surface extraction phase focuses on identifying the overlapped region between two objects.
\revision{The penetration surface is defined as a set of polygons from one object, where at least one of its vertices lies within the other object. 
Note that in our work, we focus on triangles rather than general polygons, as they are processed most efficiently on the RTX platform.}
To facilitate this extraction, we introduce a ray-tracing-based \revision{Point-in-Polyhedron} test (RT-PIP), significantly accelerated through the use of RT cores (Sec.~\ref{sec:RT-PIP}).
This test capitalizes on the ray-surface intersection capabilities of the RTX platform.
%
Initially, a Geometry Acceleration Structure (GAS) is generated for each object, as required by the RTX platform.
The RT-PIP module takes the GAS of one object (e.g., $GAS_{A}$) and the point set of the other object (e.g., $P_{B}$).
It outputs a set of points (e.g., $P_{\partial B}$) representing the penetration region, indicating their location inside the opposing object.
Subsequently, a penetration surface (e.g., $\partial B$) is constructed using this point set (e.g., $P_{\partial B}$) (Sec.~\ref{subsec:surfaceGen}).
%
The generated penetration surfaces (e.g., $\partial A$ and $\partial B$) are then forwarded to the next step. 

The Hausdorff distance calculation phase utilizes the ray-surface intersection test of the RTX platform (Sec.~\ref{sec:RT-Hausdorff}) to compute the Hausdorff distance between two objects.
We introduce a novel Ray-Tracing-based Hausdorff DISTance algorithm, RT-HDIST.
It begins by generating GAS for the two penetration surfaces, $P_{\partial A}$ and $P_{\partial B}$, derived from the preceding step.
RT-HDIST processes the GAS of a penetration surface (e.g., $GAS_{\partial A}$) alongside the point set of the other penetration surface (e.g., $P_{\partial B}$) to compute the penetration depth between them.
The algorithm operates bidirectionally, considering both directions ($\partial A \to \partial B$ and $\partial B \to \partial A$).
The final penetration depth between the two objects, A and B, is determined by selecting the larger value from these two directional computations.

%In the Hausdorff distance calculation step, we compute the Hausdorff distance between given two objects using a ray-surface-intersection test. (Sec.~\ref{sec:RT-Hausdorff}) Initially, we construct the GAS for both $\partial A$ and $\partial B$ to utilize the RT-core effectively. The RT-based Hausdorff distance algorithms then determine the Hausdorff distance by processing the GAS of one object (e.g. $GAS_{\partial A}$) and set of the vertices of the other (e.g. $P_{\partial B}$). Following the Hausdorff distance definition (Eq.~\ref{equation:hausdorff_definition}), we compute the Hausdorff distance to both directions ($\partial A \to \partial B$) and ($\partial B \to \partial A$). As a result, the bigger one is the final Hausdorff distance, and also it is the penetration depth between input object $A$ and $B$.


%the proposed RT-based penetration depth calculation pipeline.
%Our proposed methods adopt Tang's Hausdorff-based penetration depth methods~\cite{SIG09HIST}. The pipeline is divided into the penetration surface extraction step and the Hausdorff distance calculation between the penetration surface steps. However, since Tang's approach is not suitable for the RT platform in detail, we modified and applied it with appropriate methods.

%The penetration surface extraction step is extracting overlapped surfaces on other objects. To utilize the RT core, we use the ray-intersection-based PIP(Point-In-Polygon) algorithms instead of collision detection between two objects which Tang et al.~\cite{SIG09HIST} used. (Sec.~\ref{sec:RT-PIP})
%RT core-based PIP test uses a ray-surface intersection test. For purpose this, we generate the GAS(Geometry Acceleration Structure) for each object. RT core-based PIP test takes the GAS of one object (e.g. $GAS_{A}$) and a set of vertex of another one (e.g. $P_{B}$). Then this computes the penetrated vertex set of another one (e.g. $P_{\partial B}$). To calculate the Hausdorff distance, these vertex sets change to objects constructed by penetrated surface (e.g. $\partial B$). Finally, the two generated overlapped surface objects $\partial A$ and $\partial B$ are used in the Hausdorff distance calculation step.

To evaluate \textit{``how knowledgeable are LLM in medicine and healthcare''}, existing benchmarks typically frame the task as a question-answering (QA) challenge, often involving clinical questions that require multi-step reasoning, indirect relationships, or external retrieval~\cite{pal2022medmcqa, malaviya-etal-2024-expertqa}. 
However, evaluating LLMs' core medical knowledge requires a direct and controlled framework-one that can systematically quantify what LLMs ``know'' without the confounding effects of reasoning or retrieval augmentation. 
Fundamental knowledge evaluation is not merely an auxiliary task-it is a prerequisite for trustworthy and effective medical foundation models~\cite{zhang2024generalist, moor2023foundation}. By ensuring that LLMs possess a solid factual foundation, we pave the way for more reliable reasoning, clinical applications, and ultimately, safer deployment in real-world healthcare settings.

To address this gap, we introduce the Medical Knowledge Judgment Dataset (\mkj), designed to systematically evaluate LLMs' inherent factual medical knowledge through one-hop binary judgment tasks. 
We construct \mkj through a systematic process of extracting knowledge triplets from Unified Medical Language System (UMLS) and transforming them into carefully templated judgment questions. 
We focus on one-hop relation and remove triplets with multi-hop nodes or multiple relationships, which ensures single and indisputable answer~\cite{weimeasuring, sun2024head}. Some examples are provided in Figure~\ref{fig:overview}.
% following methodological insights from SimpleQA~\cite{weimeasuring} and Head-to-Tail~\cite{sun2024head}. 

The UMLS serves as the ideal foundation for our dataset due to its unparalleled comprehensiveness and reliability in the medical domain. 
First, UMLS is a rigorously curated and widely trusted biomedical resource, integrating over 3.8 million concepts and 78 million relationships from authoritative medical terminologies. This ensures that the knowledge assessed in our dataset is clinically validated and standardized. 
Second, its knowledge graph (KG) structure provides an explicit, structured representation of medical knowledge, enabling precise fact-based evaluation while minimizing ambiguity and inconsistencies often found in alternative sources~\citep{abacha2017overview, malaviya-etal-2024-expertqa}. 
By leveraging UMLS as a gold-standard knowledge base, our dataset ensures high factual reliability, broad medical coverage, and systematic evaluation of LLMs’ medical expertise.

% The \mkj dataset contributes to the growing body of work on LLM evaluation by systematically leveraging medical knowledge graphs to assess LLMs' medical expertise.
To comprehensively examine LLMs’ capabilities in medical knowledge retention, we focus on four specific research questions:

\begin{itemize}[left=0pt, topsep=0pt, itemsep=0pt, partopsep=0pt,parsep=0pt]
    \item \textbf{RQ1} To what extent can LLMs accurately perform medical judgments?
    \item \textbf{RQ2} How well are LLMs calibrated in medical and healthcare contexts?
    \item \textbf{RQ3} What are the underlying reasons behind LLMs’ failure to retain certain critical medical knowledge?
    \item \textbf{RQ4} What strategies can enhance the response accuracy of LLMs?
\end{itemize}

We groups the questions into three progressive categories: Biomedical Entities (foundational concepts), Pathological Conditions (phenomena), and Clinical Practice (applications) with detailed information in Appendix~\ref{app:data_detail}. Through comprehensive analysis on \mkj, we find that LLMs, especially open-source LLMs, still struggle with basic factual medical knowledge retention as illustrated in Figure~\ref{fig:overview}. 
Our further investigation in later sections reveals critical challenges: poor calibration with frequent overconfidence in incorrect predictions as shown in Figure~\ref{fig:calibration_curves}, and notably degraded performance when handling rare medical conditions as in Figure~\ref{fig:acc_and_freq}. To address these limitations, we implement retrieval-augmented generation, which substantially improves factual accuracy in medical contexts displayed in Table~\ref{tab:sparse_dense_rag}.



% With a logical progression, we group the questions into three categories: from Biomedical Entities (fundamentals) to Pathological Conditions (phenomena), and then Clinical Practice (application), on which our experiment reveals that LLMs still struggle with factual medical knowledge retention as demonstrated in Figure~\ref{fig:overview}, and show a descending trend with three groups of questions.

% Through comprehensive experimentation and analysis of LLMs on \mkj, it is found that current LLMs remain considerably distant from achieving satisfactory performances (e.g. >90\%) as demonstrated in Figure~\ref{fig:overview}. 

% \jiaxi{Categories and corresponding semantic types to be added in the appendix.}
% 1. (covering fundamental medical components such as anatomical structures and biochemical substances)
% 2. (focusing on diseases, disorders, and their associated symptoms and manifestations)
% 3. (encompassing therapeutic procedures, treatment guidelines, and medical interventions)

% Our further analysis reveals that LLMs exhibit poor calibration level on the \mkj dataset, often being overconfident in incorrect answers. LLMs are also found to have significant performance variance across different semantic categories, particularly for rare medical conditions. To mitigate these issues, we explore retrieval-augmented generation, demonstrating its effectiveness in improving factual accuracy and reducing uncertainty in medical decision-making.


% % Original words
% First, our analysis shows that current LLMs struggle to achieve high accuracies on the MKJ benchmark, indicating a noticeable gap in their medical knowledge representation. 
% Second, our calibration analysis reveals a concerning pattern of miscalibration in the domains of medicine and healthcare, where LLMs' confidence scores show significant discrepancy with their actual performance accuracies, exhibiting both over-confidence and under-confidence patterns.
% Third, our analysis reveals that the performance degradation is particularly pronounced for questions involving rare medical conditions, which suggests that insufficient coverage of such uncommon topics in the models' pretraining corpora may be a key contributing factor to their suboptimal performances.
% Third, through further investigation, we identify that the performance degradation is particularly pronounced for rare medical conditions, suggesting insufficient coverage of these topics in the models' pretraining corpora. 
% Fourth, to mitigate these limitations, we explore the integration of Retrieval-Augmented Generation (RAG) as a potential solution. By leveraging the collected comprehensive UMLS knowledge graph triplets as a document corpus, we demonstrate significant improvements in model performance. 

% To sum up, our main contributions are as follows:
% (1) We introduce \mkj, a dataset designed to evaluate LLMs’ inherent factual knowledge in medical and healthcare domains. 
% (2) To answer the RQs, we conduct extensive experiments and find current LLMs struggle to do accurate judgments and tend to be over-confident with their predictions. We analyze the underlying reasons and improve the performances with the technique of RAG. 



% \todo{Kai's comments: (1) why evaluating knowledgeable are LLMs in healthcare is important --> understand how LLM utilizes the inherent knowledge to solve the real-world or practical clinical questions (how and if the knowledge is utilized during medical reasoning). 
% (2) refine the post-training strategies to add more knowledge and estimate if there is forgetting issue.}
