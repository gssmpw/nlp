\section{Appendix}
\label{sec:appendix}


\subsection{Dataset Details}
\label{app:data_detail}

\subsubsection{Basic information}
We extract triplets $(s, p, o)$ from UMLS, where there is a basic attribute called ``Semantic Type'' that denotes the type of an entity or concept. By doing substitution on the object of a triplet, we assess LLM's knowledge on the object $o$ with the predicate $p$ of subject $s$. Therefore, we use the semantic type of the object $o$ to denote the type of a judgment question. 

Specifically, there are 236 semantic types among the collected triplets $\mathcal{T}$. Due to the large amount of semantic types, we may only be able to list some of them here:
\textit{Organic Chemical}, 
\textit{Pharmacologic Substance}, 
\textit{Congenital Abnormality}, 
\textit{Finding}, 
\textit{Disease or Syndrome}, 
\textit{Virus}, 
\textit{Injury or Poisoning}, 
\textit{Sign or Symptom}, 
\textit{Therapeutic or Preventive Procedure}, 
\textit{Physiologic Function}, 
\textit{Idea or Concept}, 
\textit{Pathologic Function}, 
\textit{Functional Concept}, 
\textit{Molecular Function}, 
\textit{Amino Acid, Peptide, or Protein}, 
\textit{Immunologic Factor}, 
\textit{Neoplastic Process}, 
\textit{Professional or Occupational Group}, 
\textit{Clinical Drug}, 
\textit{Hormone}, 
\textit{Biologically Active Substance}, 
\textit{Indicator, Reagent, or Diagnostic Aid}, 
\textit{Intellectual Product}, 
\textit{Health Care Activity}, 
\textit{Body Part, Organ, or Organ Component}, 
\textit{Temporal Concept}, 
\textit{Diagnostic Procedure}, 
\textit{Body System}.


We further group our judgment questions in the \mkj dataset into three categories according to the semantic types, including \textit{Biomedical Entities}, \textit{Pathological Conditions}, and \textit{Clinical Practice}, as shown in Figure~\ref{fig:overview}.


\subsubsection{Statistics about \mkj dataset}
Here we provide some detailed statistics on our \mkj dataset. 

During the construction process, we collect 5820 high-quality triplets $\mathcal{T}$. There are more than 40,000 entities with 236 different semantic types. The average number of entities for each semantic type is around 160, which ensures our object substitution operation (Section~\ref{sec:data_sub}) in the data generation process is effective with little chance of introducing noise or mistake.


\begin{table*}[t!]
    \centering
    \resizebox{\textwidth}{!}{
        \begin{tabular}{l|c}
            \hline
            \textbf{Model} & \textbf{Version and Source} \\
            \toprule
            \gptthree & \url{https://platform.openai.com/docs/models/gpt-3-5-turbo\#gpt-3-5-turbo}\\
            \gptfour & \url{https://platform.openai.com/docs/models\#gpt-4o-mini} \\
            \gptfouro & \url{https://platform.openai.com/docs/models\#gpt-4o} \\
            \claudea & \url{https://docs.anthropic.com/en/docs/about-claude/models\#model-comparison-table} \\
            \claudeb & \url{https://docs.anthropic.com/en/docs/about-claude/models\#model-comparison-table} \\
            
            \midrule
            \llamaa & \url{https://huggingface.co/meta-llama/Meta-Llama-3-8B-Instruct} \\
            \llamab & \url{https://huggingface.co/meta-llama/Llama-3.1-8B-Instruct} \\
            \llamac & \url{https://huggingface.co/meta-llama/Llama-3.2-1B-Instruct}\\
            \llamad & \url{https://huggingface.co/meta-llama/Llama-3.2-3B-Instruct}\\
            \ministral & \url{https://huggingface.co/mistralai/Ministral-8B-Instruct-2410}\\
            \qwena & \url{https://huggingface.co/Qwen/Qwen2.5-0.5B-Instruct}\\
            \qwenb & \url{https://huggingface.co/Qwen/Qwen2.5-1.5B-Instruct}\\
            \qwenc & \url{https://huggingface.co/Qwen/Qwen2.5-3B-Instruct}\\
            \phia & \url{https://huggingface.co/microsoft/Phi-3-mini-4k-instruct}\\
            \phib & \url{https://huggingface.co/microsoft/Phi-3-mini-128k-instruct} \\
            \phic & \url{https://huggingface.co/microsoft/Phi-3.5-mini-instruct} \\

            \midrule
            \meditron & \url{https://huggingface.co/epfl-llm/meditron-7b} \\
            \mellama & \url{https://physionet.org/content/me-llama/1.0.0/MeLLaMA-13B/} \\
            \bottomrule
        \end{tabular}
    }
    \caption{Detailed versions and sources of LLMs used in our experiments.}
    \label{tab:app_LLMs}
\end{table*}

\subsection{Model Details}
\label{app:model_details}
\paragraph{Details of LLMs in our experiment.}
Some full names of LLMs may occupy too-much space in the paper main body. Therefore, we provide the detailed versions and sources of LLMs evaluated in our experiments, as listed in Table~\ref{tab:app_LLMs}. 


% \clearpage
\subsection{Prompts}
\label{app:prompt}

\paragraph{Prompt for Extracting Keywords.}
In section~\ref{sec:data_collect}, we extracting keywords in a problem for medical and healthcare, and the prompt is given below in Figure~\ref{fig:med_ner_prompt}.

\begin{figure}[ht]
    \centering
    \includegraphics[width=0.48\textwidth]{figures/appendix/MedicalNER_prompt.pdf}
    \caption{Prompt for extracting medical keywords.}
    \label{fig:med_ner_prompt}
\end{figure}


\paragraph{Prompt for Constructing Templates.}
In section~\ref{sec:data_temp}, we prompt LLMs to generate templates for constructing judgment questions. By utilizing few-shot prompting technique (we provide 3 examples in the context), LLMs can generate high-quality templates that are logical and coherent. The prompt is given below in Figure~\ref{fig:template_gen_prompt}.

\begin{figure}[ht]
    \centering
    \includegraphics[width=0.48\textwidth]{figures/appendix/template_gen_prompt.pdf}
    \caption{Prompt used for statement generation.}
    \label{fig:template_gen_prompt}
\end{figure}


\paragraph{Prompts for Evaluation.}
The prompts used for evaluation with zero-shot prompting and retrieval-augmented generation (RAG) are provided in Figure~\ref{fig:zo_prompt} and Figure~\ref{fig:rag_prompt}.

\begin{figure}[ht]
    \centering
    \includegraphics[width=0.48\textwidth]{figures/appendix/zo_prompt.pdf}
    \caption{Zero-shot Prompting}
    \label{fig:zo_prompt}
\end{figure}

\begin{figure}[ht]
    \centering
    \includegraphics[width=0.48\textwidth]{figures/appendix/rag_prompt.pdf}
    \caption{Prompt for retrieval-augmented generation}
    \label{fig:rag_prompt}
\end{figure}


% \clearpage
% \newpage
\subsection{Calibration curves and uncertainty quantification}
\label{app:calibration}
We list the calibration curves of all open-source LLMs we tested in this paper in Figure~\ref{fig:app_calibration_curve_1} and Figure~\ref{fig:app_calibration_curve_2}, including LLM families of Llama, Mistral, Qwen, Phi, and meidcal LLMs Meditron-7B and MeLLaMA-13B.

\begin{figure}[ht]
    \centering
    
    \begin{subfigure}{0.233\textwidth}
        \centering
        \includegraphics[width=\textwidth]{figures/calibration/calibration_curve_llama-3-8B.pdf}
        \caption{Calibration curve for \llamaa.}
        \label{fig:app_cal_fig1}
    \end{subfigure}
    \hfill
    \begin{subfigure}{0.233\textwidth}
        \centering
        \includegraphics[width=\textwidth]{figures/calibration/calibration_curve_llama-3.1-8B.pdf}
        \caption{Calibration curve for \llamab.}
        \label{fig:app_cal_fig2}
    \end{subfigure}
    
    \vspace{0.5cm}
    
    \begin{subfigure}{0.233\textwidth}
        \centering
        \includegraphics[width=\textwidth]{figures/calibration/calibration_curve_llama-3.2-1B.pdf}
        \caption{Calibration curve for \llamac.}
        \label{fig:app_cal_fig3}
    \end{subfigure}
    \hfill
    \begin{subfigure}{0.233\textwidth}
        \centering
        \includegraphics[width=\textwidth]{figures/calibration/calibration_curve_llama-3.2-3B.pdf}
        \caption{Calibration curve for \llamad.}
        \label{fig:app_cal_fig4}
    \end{subfigure}

    \vspace{0.5cm}

    \begin{subfigure}{0.233\textwidth}
        \centering
        \includegraphics[width=\textwidth]{figures/calibration/calibration_curve_ministral-8B.pdf}
        \caption{Calibration curve for \ministral.}
        \label{fig:app_cal_fig5}
    \end{subfigure}
    \hfill
    \begin{subfigure}{0.233\textwidth}
        \centering
        \includegraphics[width=\textwidth]{figures/calibration/calibration_curve_qwen2.5-0.5B.pdf}
        \caption{Calibration curve for \qwena.}
        \label{fig:app_cal_fig6}
    \end{subfigure}

    
    \caption{Calibration curves for all tested LLMs (first subfigure).}
    \label{fig:app_calibration_curve_1}
\end{figure}



\begin{figure}[htbp]
    \centering

    \begin{subfigure}{0.233\textwidth}
        \centering
        \includegraphics[width=\textwidth]{figures/calibration/calibration_curve_qwen2.5-1.5B.pdf}
        \caption{Calibration curve for \qwenb.}
        \label{fig:app_cal_fig7}
    \end{subfigure}
    \hfill
    \begin{subfigure}{0.233\textwidth}
        \centering
        \includegraphics[width=\textwidth]{figures/calibration/calibration_curve_qwen2.5-3B.pdf}
        \caption{Calibration curve for \qwenc.}
        \label{fig:app_cal_fig8}
    \end{subfigure}

    \vspace{0.5cm}
    
    \begin{subfigure}{0.233\textwidth}
        \centering
        \includegraphics[width=\textwidth]{figures/calibration/calibration_curve_phi-3-mini-4k.pdf}
        \caption{Calibration curve for \phia.}
        \label{fig:app_cal_fig9}
    \end{subfigure}
    \hfill
    \begin{subfigure}{0.233\textwidth}
        \centering
        \includegraphics[width=\textwidth]{figures/calibration/calibration_curve_phi-3-mini-128k.pdf}
        \caption{Calibration curve for \phib.}
        \label{fig:app_cal_fig10}
    \end{subfigure}

    \vspace{0.5cm}

    \begin{subfigure}{0.233\textwidth}
        \centering
        \includegraphics[width=\textwidth]{figures/calibration/calibration_curve_phi-3.5-mini.pdf}
        \caption{Calibration curve for \phic.}
        \label{fig:app_cal_fig11}
    \end{subfigure}
    \hfill
    \begin{subfigure}{0.233\textwidth}
        \centering
        \includegraphics[width=\textwidth]{figures/calibration/calibration_curve_meditron-7B.pdf}
        \caption{Calibration curve for \meditron.}
        \label{fig:app_cal_fig12}
    \end{subfigure}
    
    \vspace{0.5cm}

    \begin{subfigure}{0.233\textwidth}
        \centering
        \includegraphics[width=\textwidth]{figures/calibration/calibration_curve_MeLLaMA-13B.pdf}
        \caption{Calibration curve for \mellama.}
        \label{fig:app_cal_fig13}
    \end{subfigure}

    
    \caption{Calibration curves for all tested LLMs (second subfigure).}
    \label{fig:app_calibration_curve_2}
\end{figure}
