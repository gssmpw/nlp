\section{Detailed Statistics of {\benchmark}}\label{app: statistics}


\subsection{Problem Filtering}\label{app: filter}

During problem processing, we find some physics problems do not have definite answers for evaluation.
Two examples are given in Table~\ref{tabapp: filter_example}.
We exclude these problems for automatic evaluation in our {\benchmark}.

\begin{table}[b]
    \centering
    \footnotesize
    \caption{Estimation and Explanation Problem Example.}
    \begin{tabularx}{\textwidth}{|X|}
        \hline
        \multicolumn{1}{|c|}{Estimation Problem} \\
        \hline
        Provide the best estimate of the cosmic ray flux at sea level.
        \\
        \hline
        \multicolumn{1}{|c|}{Explanation Problem} \\
        \hline
        Water is a polar molecule. Discuss the effect of electrode polarization on its dielectric constant at high and low frequencies.
        \\
        \hline
    \end{tabularx}
    \label{tabapp: filter_example}
\end{table}




\subsection{Answer Types}\label{app: answer_type}
By carefully reviewing a large collection of problems and referring to various past benchmarks \citep{OlympiadBench2024He, OlympicArena2024huang, ugmathbench2025xu}, we classify all answers to seven categories in total: six atomic and one compound type.
Each compound type is composed of a list of atomic ones.
These types are designed to encompass a wide range of problems. 
Detailed definitions for each answer type can be found in Table~\ref{tabapp:answer type instructions}.
Similar to \citet{OlympiadBench2024He}, the compound answer type is characterized by the attribute \texttt{is\_multiple\_answers}.


\begin{table}[b]
\centering
\footnotesize
\caption{Descriptions of answer types included in evaluation prompts, only include English version for simplicity.}
\scalebox{1.0}{
\begin{tabular}{lp{9cm}}
\toprule

\textbf{Answer Type} & \textbf{English Answer Type Description} \\

\midrule
NV & a numerical value without units \\

\midrule
EX & an expression \\

\midrule
EQ & an equation \\

\midrule
INT & a range interval \\

\midrule
TF & either True or False\\

\midrule
MC & one option of a multiple choice question \\

%\midrule
%TUP & multiple numbers, seperated by comma, such as (x, y, z) \\


\bottomrule
\end{tabular}%
}
\label{tabapp:answer type instructions}
\end{table}




\subsection{Physics Reasoning Skills Annotation}\label{app: skills}

The definitions of different physics skills are as follows:
\begin{itemize}
    \item \textbf{Knowledge Recall}: Refers to the recall and understanding of basic physics concepts, formulas, and definitions. This category assesses the ability to accurately remember key physics points, such as the units, definitions, and fundamental properties of physical quantities, without requiring deep reasoning or complex calculations.
    \item \textbf{Laws Application}: Involves understanding and applying important physical laws. Students are expected to correctly describe the content, conditions, and scope of these laws and determine their applicability to specific scenarios. Examples include Newton's laws, the laws of thermodynamics, and Coulomb's law.
    \item \textbf{Math Derivation}: Focuses on the mathematical derivation and logical proof of physics formulas or principles. This requires students to use known laws or principles and apply rigorous mathematical reasoning to derive new relationships or theorems. Examples include deriving the momentum conservation equation from Newton's second law or deriving thermodynamic relations from the ideal gas law.
    \item \textbf{Practical Application}: Emphasizes the practical use of physics knowledge, laws, and derivations to solve real-world problems. This includes analyzing scenarios, building physical models, and using calculations and logical reasoning to arrive at solutions. Examples include calculating the final velocity of a car using the work-energy theorem or analyzing the motion of a charged particle using electric field equations.
\end{itemize}

The prompt used for classification is given in Table~\ref{tabapp: skill_prompt}.
Note that there are problems that fail to be classified into certain types of skill, we annotate them as "others".

\begin{table}[b]
    \centering
    \footnotesize
    \caption{The prompt for annotating physics reasoning skills, where {problem}, {solution}, and {answer} represent the components of the problem to be annotated.}
    \begin{tabularx}{\textwidth}{|X|}
        \hline
        Problem: \\
        \{problem\} \\ \\
        Solution: \\ 
        \{solution\} \\ \\ 
        Answer: \\ 
        \{answer\} \\ \\
        Classification Categories: \\
        1. Knowledge Recall: Refers to the recall and understanding of basic physics concepts, formulas, and definitions. This category assesses the ability to accurately remember key physics points, such as the units, definitions, and fundamental properties of physical quantities, without requiring deep reasoning or complex calculations. \\
        2. Laws Application: Involves understanding and applying important physical laws. Students are expected to correctly describe the content, conditions, and scope of these laws and determine their applicability to specific scenarios. Examples include Newton's laws, the laws of thermodynamics, and Coulomb's law. \\ 
        3. Math Derivation: Focuses on the mathematical derivation and logical proof of physics formulas or principles. This requires students to use known laws or principles and apply rigorous mathematical reasoning to derive new relationships or theorems. Examples include deriving the momentum conservation equation from Newton's second law or deriving thermodynamic relations from the ideal gas law. \\
        4. Practical Application: Emphasizes the practical use of physics knowledge, laws, and derivations to solve real-world problems. This includes analyzing scenarios, building physical models, and using calculations and logical reasoning to arrive at solutions. Examples include calculating the final velocity of a car using the work-energy theorem or analyzing the motion of a charged particle using electric field equations. \\ \\
        Instructions for Classification: Please classify the above problem by selecting the most appropriate category that best represents the type of thinking and approach required to address the physics problem. Consider the complexity, the need for creativity, and the depth of knowledge required. You should conclude your response with \"So, the problem can be categorized as ANSWER.\", where ANSWER should be one of the indexes in 1, 2, 3, 4. 
        \\
        \hline
    \end{tabularx}
    \label{tabapp: skill_prompt}
\end{table}


\subsection{Distribution of Problems}\label{app: distribution}

Our {\benchmark} include 3 key domains and 13 core subjects in undergraduate-level physics. 
The detailed topics across different subjects and the corresponding number of examples are listed in Table~\ref{tabapp:topics}.
There are 59 topics in total.
We illustrate the words of topics in Figure~\ref{fig:word_cloud}.
Furthermore, the distribution information of our benchmark on different subjects and reasoning skills is presented in Table~\ref{tabapp:detailed distribution}.




\begin{figure}[b]
    \centering
    \includegraphics[width=1\linewidth]{figures/wordcloud.pdf}
    \caption{Word Cloud of Topics in {\benchmark}}
    \label{fig:word_cloud}
\end{figure}


\begin{table}[hbp]
\centering
\footnotesize
\scalebox{0.8}{
\begin{tabular}[tbp]{lp{8cm}c}
\toprule
\textbf{Subject}  & \textbf{Topics} & \textbf{\# Examples} \\ 
\midrule

Classical Mechanics & Particle Dynamics & 644 \\ 
Classical Mechanics & Vibrations and Waves & 358 \\ 
Classical Mechanics & Central Force Motion & 156 \\ 
Classical Mechanics & Rigid-Body Dynamics & 360 \\ 
Classical Mechanics & Fluid Mechanics & 154 \\ 
\hline
Theoretical Mechanics & Lagrangian Formulation of Mechanics & 202 \\ 
Theoretical Mechanics & Small Vibrations of Finite System & 182 \\ 
Theoretical Mechanics & Hamiltonian Formulation of Mechanics & 254 \\ 
\hline
Relativity & General Relativity & 36 \\ 
Relativity & Special Relativity & 368 \\ 
Relativity & Relativistic Cosmology & 10 \\ 
\hline
Thermodynamics & Phase Transition and Equilibrium & 168 \\ 
Thermodynamics & Thermodynamic State and First Law of Thermodynamics & 146 \\ 
Thermodynamics & Second Law of Thermodynamics and Entropy & 218 \\ 
Thermodynamics & Thermodynamic Functions and Equilibrium Conditions & 134 \\ 
Thermodynamics & Non-equilibrium Thermodynamics & 78 \\ 
\hline
Statistical Mechanics & Ensemble Theory & 234 \\ 
Statistical Mechanics & Maxwell-Boltzmann Statistics & 294 \\ 
Statistical Mechanics & Distribution Function and Statistical Entropy & 68 \\ 
Statistical Mechanics & Fermi-Dirac Statistics & 228 \\ 
Statistical Mechanics & Bose-Einstein Statistics & 140 \\ 
Statistical Mechanics & Kinetic Theory of Gases & 156 \\ 
\hline
Classical Electromagnetism & Magstatics & 296 \\ 
Classical Electromagnetism & Electrostatics & 368 \\ 
Classical Electromagnetism & Circuit Analysis & 116 \\ 
\hline
Electrodynamics & Propagation and Radiation of Electromagnetic Waves & 194 \\ 
Electrodynamics & Relativistic Electrodynamics & 116 \\ 
Electrodynamics & Interaction of Electromagnetic Fields with Matter & 58 \\ 
\hline
Geometrical Optics & Imaging of Light & 88 \\ 
Geometrical Optics & Reflection and Refraction of Light & 28 \\ 
\hline
Wave Optics & Diffraction & 192 \\ 
Wave Optics & Interference & 110 \\ 
Wave Optics & Polarization & 302 \\ 
\hline
Quantum Mechanics & Linear Dynamics Problems & 332 \\ 
Quantum Mechanics & Basic Principles of Quantum Mechanics & 180 \\ 
Quantum Mechanics & Central Force and Scattering Problems & 192 \\ 
Quantum Mechanics & Orbital and Spin Angular Momentum Problems & 270 \\ 
Quantum Mechanics & Motion of Charged Particles in Electric and Magnetic Fields & 60 \\ 
Quantum Mechanics & Adiabatic Approximation Problems & 224 \\ 
Quantum Mechanics & Scattering Problems & 110 \\ 
Quantum Mechanics & Variational Methods and Perturbation Theory & 120 \\ 
Quantum Mechanics & Quantum Information Physics & 96 \\ 
Quantum Mechanics & Few-body Problems & 104 \\ 
Quantum Mechanics & Others & 104 \\ 
Quantum Mechanics & Quantum Optics & 246 \\ 
\hline
Atomic Physics & Atomic and Molecular Physics & 622 \\ 
Atomic Physics & Particle Physics & 394 \\ 
Atomic Physics & Nuclear Physics & 400 \\ 
Atomic Physics & Experimental Methods and Particle Beams & 414 \\ 
\hline
Solid-State Physics & Crystal Structure & 38 \\ 
Solid-State Physics & Lattice Vibrations and Mechanical Properties of Solids & 84 \\ 
Solid-State Physics & Crystal Defects and Motion & 34 \\ 
Solid-State Physics & Binding in Solids & 54 \\ 
Solid-State Physics & Transport Properties of Electrons and Holes in Solids & 60 \\ 
Solid-State Physics & Motion of Electrons in Electromagnetic Fields & 42 \\ 
Solid-State Physics & Energy Bands in Solids & 32 \\ 
\hline
Semiconductor Physics & Magnetic, Optical, and Superconducting Properties & 50 \\ 
Semiconductor Physics & Transport Properties & 80 \\ 
Semiconductor Physics & Others & 186 \\ 
Semiconductor Physics & Statistical Distribution of Electrons and Holes in Space & 56 \\ 
\bottomrule
\end{tabular}}
\caption{Topics of each subject and corresponding number of examples included in {\benchmark}.}
\label{tabapp:topics}
\end{table}




\begin{table}[!t]
\centering
\footnotesize
\scalebox{0.9}{
\begin{tabular}{lcccccc}
\toprule
& \textbf{Knowledge Recall} & \textbf{Laws Application} & \textbf{Math Derivation} & \textbf{Practical Application} & \textbf{Others} & \textbf{All} \\
\midrule
Classical Mechanics & 72 & 582 & 774 & 172 & 72 & 1,672 \\ 
Theoretical Mechanics & 30 & 124 & 448 & 20 & 16 & 638 \\ 
Relativity & 10 & 188 & 154 & 44 & 18 & 414 \\ 
Thermodynamics & 36 & 388 & 246 & 46 & 28 & 744 \\ 
Statistical Mechanics & 112 & 322 & 590 & 38 & 58 & 1,120 \\ 
\hline
Classical Electromagnetism & 60 & 338 & 314 & 44 & 24 & 780 \\ 
Electrodynamics & 28 & 142 & 156 & 38 & 4 & 368 \\ 
Geometrical Optics & 8 & 48 & 26 & 30 & 4 & 116 \\ 
Wave Optics & 52 & 236 & 130 & 160 & 26 & 604 \\ 
\hline
Quantum Mechanics & 226 & 584 & 1,052 & 100 & 76 & 2,038 \\ 
Atomic Physics & 422 & 784 & 282 & 282 & 60 & 1,830 \\ 
Solid-State Physics & 36 & 66 & 158 & 70 & 14 & 344 \\ 
Semiconductor Physics & 76 & 112 & 110 & 62 & 12 & 372 \\ 
\midrule
Grand Total & 1,168 & 3,914 & 4,440 & 1,106 & 412 & 11,040 \\
\bottomrule
\end{tabular}
}
\caption{Statistics of {\benchmark} across different subjects and physics reasoning skills.}
\label{tabapp:detailed distribution}
\end{table}
