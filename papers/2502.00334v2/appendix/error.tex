\section{Error Analysis}\label{app: error}

We perform error analysis in Section~\ref{sec: error}, and here we showcase several examples of various error types in Table~\ref{tabapp:error example 2}, \ref{tabapp:error example 3}, and \ref{tabapp:error example 4}.
\begin{table}[!t]
\centering
\footnotesize
\scalebox{1.0}{
\begin{tabular}{p{10cm}}
\toprule
\textbf{Problem}: 
A crystal has \( N \) atoms and possesses Frenkel defects. The numbers of normal positions and interstitial positions are \( N \) and \( N' \) respectively. The energy when an atom is in normal and interstitial positions is \( \varepsilon_1 \) and \( \varepsilon_2 \), with \( \varepsilon_2 - \varepsilon_1 = w > 0 \). Derive the function relationship of the most probable number of interstitial atoms \( n \) as a function of temperature \( T \) (assuming \( n, N, N' \gg 1 \)).
\\
\midrule
\textbf{o1-mini Completion}: 
To determine the most probable number of interstitial atoms \( n \) as a function of temperature \( T \), we consider the statistical mechanics of Frenkel defects in a crystal.

1. **Multiplicity (\( \Omega \))**: The number of ways to arrange \( n \) interstitial atoms among \( N \) normal sites and \( N' \) interstitial sites is given by:
\textcolor{red}{
   \[
   \Omega = \binom{N}{n} \binom{N'}{n}
   \]
}
\begin{center}
\textcolor{blue}{[should be $\Omega = \binom{N}{N-n} \binom{N'}{n}$]}
\end{center}
   Taking the natural logarithm and using Stirling's approximation for large \( n, N, N' \):
   \[
   \ln \Omega \approx n \ln\left(\frac{N}{n}\right) + n \ln\left(\frac{N'}{n}\right)
   \]
2. **Energy (\( U \))**: The total energy associated with \( n \) interstitial atoms is:
   \[
   U = N \varepsilon_1 + n (\varepsilon_2 - \varepsilon_1) = N \varepsilon_1 + n w
   \]
   Here, \( w = \varepsilon_2 - \varepsilon_1 \).

3. **Free Energy Maximization**: \textcolor{red}{To find the most probable \( n \), we maximize the entropy subject to the energy constraint:
   \[
   \frac{d}{dn} \left( \ln \Omega - \frac{U}{kT} \right) = 0
   \]}

\textcolor{blue}{[should be $\delta \ln W - \alpha \sum_{l=1}^{2} \delta N_l - \beta \sum_{l=1}^{2} \varepsilon_l \delta N_l = 0$], applying Lagrange multiplier method, where \(N_1 = N - n\), \(N_2 = n\)]}

   This leads to:
   \[
   \ln\left(\frac{N N'}{n^2}\right) - 2 - \frac{w}{kT} = 0
   \]
   Solving for \( n \):
   \[
   \ln n = \frac{1}{2} \left( \ln(N N') - 2 - \frac{w}{kT} \right)
   \]
   Exponentiating both sides gives:
   \[
   n = \sqrt{N N'} \, e^{-w/(2kT)}
   \]

So the final answer is \( \boxed{\sqrt{N N'} \, e^{-w/(2kT)}} \).\\
\midrule
\textbf{Ground-Truth Answer}: 
\[
\exp\left(-\frac{w}{kT}\right)
\]
\\
\midrule
\textbf{Model Output}: 
\[
\sqrt{NN'} e^{-\frac{w}{2kT}}
\]

\\
\bottomrule
\end{tabular}%
}
\caption{An example of Statistical Mechanics with flawed reasoning error.}
\label{tabapp:error example 2}
\end{table}

\begin{table}[!t]
\centering
\footnotesize
\scalebox{1.0}{
\begin{tabular}{p{10cm}}
\toprule
\textbf{Problem}: 
An electromagnetic eddy current brake consists of a disc with electrical conductivity $\sigma$ and thickness $t$, which rotates around an axis through its center at an angular velocity $\omega$. There is a magnetic field $\boldsymbol{B}$ with a coverage area of $a^{2}$ perpendicular to the disc. The area $a^{2}$ is located at a distance $r$ from the axis. Determine the magnitude of the eddy current $I$ generated under this configuration.\\

\midrule
\textbf{o1-mini Completion}: 
To determine the magnitude of the eddy current \( I \) generated in the electromagnetic eddy current brake, we follow these steps:

1. \textbf{Induced Electric Field:}  
   The disc rotates with an angular velocity \( \omega \) in a uniform magnetic field \( \boldsymbol{B} \) perpendicular to its surface. The tangential velocity at a distance \( r \) from the center is \( \boldsymbol{v} = \omega r \hat{\theta} \). The induced electric field \( \boldsymbol{E} \) due to this motion is given by:
   \[
   \boldsymbol{E} = \boldsymbol{v} \times \boldsymbol{B} = \omega r B \hat{r}
   \]

2. \textbf{Current Density:}  
   The current density \( \boldsymbol{J} \) is related to the electric field by Ohm's law:
   \[
   \boldsymbol{J} = \sigma \boldsymbol{E} = \sigma \omega B r \hat{r}
   \]

3. \textbf{Total Eddy Current:}  
    \textcolor{red}{To find the total current \( I \), we integrate the current density over the area where the magnetic field is present. Considering the thickness \( t \) of the disc and integrating from the center (\( r = 0 \)) to the extent of the magnetic field coverage (\( r = a \)):
   \[
   I = \int_{0}^{a} \boldsymbol{J} \cdot d\boldsymbol{A} = \int_{0}^{a} \sigma \omega B r \cdot (2\pi r t) \, dr
   \]}
\begin{center}
\textcolor{blue}{[should be $I = \frac{\varepsilon}{R} = \frac{r \omega B a}{\frac{a}{\sigma a t}} = r \omega B a \sigma t$]}
\end{center}
   
   
   Simplifying the integral:
   \[
   I = 2\pi \sigma \omega B t \int_{0}^{a} r^2 \, dr = 2\pi \sigma \omega B t \left[ \frac{a^3}{3} \right] = \frac{2\pi}{3} \sigma \omega B t a^3
   \]
   
   However, considering the standard configuration and simplifying constants for the practical scenario, the magnitude of the eddy current can be expressed as:
   \[
   I = \pi \sigma \omega B t a^{2}
   \]

So the final answer is \( \boxed{\pi \sigma \omega B t a^{2}} \) (A).
\\
\midrule
\textbf{Ground-Truth Answer}: \(r \omega B a \sigma t\) \\

\midrule
\textbf{Model Output}:\( \pi \sigma \omega B t a^{2} \)\\
\bottomrule
\end{tabular}%
}
\caption{An example of Classical Electromagnetism with wrong application error.}
\label{tabapp:error example 3}
\end{table}


\iffalse
\begin{table}[!t]
\centering
\footnotesize
\scalebox{1.0}{
\begin{tabular}{p{10cm}}
\toprule
\textbf{Problem}: 
Calculate the effective mass of a hole \( m_{\mathrm{h}}^{*} \). It is known that the static dielectric constant of \( \mathrm{InP} \) is \( 9.6 \), \( E_{g} = 1.400 \, \mathrm{eV} \), \( E_{C} = 1.386 \, \mathrm{eV} \), and holes in \( \mathrm{InP} \) are much lighter than electrons (\( m_{\mathrm{h}}^{*} \ll m_{\mathrm{e}}^{*} \)).
\\
\midrule
\textbf{o1-mini Completion}: 
To calculate the effective mass of a hole \( m_{\mathrm{h}}^{*} \) in \( \mathrm{InP} \), we can use the effective Rydberg formula adapted for semiconductors. The effective Rydberg energy \( R^* \) is given by:

\[
R^* = \frac{\mu e^4}{2 (4 \pi \epsilon_0 \epsilon)^2 \hbar^2}
\]

where:
- \( \mu \) is the reduced mass, which for \( m_{\mathrm{h}}^{*} \ll m_{\mathrm{e}}^{*} \) simplifies to \( \mu \approx m_{\mathrm{h}}^{*} \).
- \( e \) is the elementary charge.
- \( \epsilon_0 \) is the vacuum permittivity.
- \( \epsilon \) is the static dielectric constant.
- \( \hbar \) is the reduced Planck's constant.

Assuming the exciton binding energy \( E_b \) in \( \mathrm{InP} \) is approximately \( 0.01 \, \mathrm{eV} \), we can solve for the reduced mass \( \mu \):

\[
\mu = \frac{2 R^* (4 \pi \epsilon_0 \epsilon)^2 \hbar^2}{e^4}
\]

\textcolor{red}{Given that \( R^* \approx E_b = 0.01 \, \mathrm{eV} \), and substituting the known values:}
\textcolor{red}{
\[
\mu \approx m_{\mathrm{h}}^{*} = \frac{0.01 \times 9.6^2}{13.6} \approx 0.07 \, m_0
\]
}
So the final answer is \( \boxed{0.07} \).

\textcolor{blue}{Correct knowledge should look like this:}

For hydrogen-like cases, the distance of these absorption lines from the fundamental absorption threshold is:

\[
E_{n} = E_{g} - \frac{m_{r}^{*} e^{4}}{8 \varepsilon^{2} \varepsilon_{0}^{2} h^{2} n^{2}}
\]

where \( m_{r}^{*} \) is the reduced mass of the electron-hole pair. Since 
\( m_{\mathrm{h}}^{*} \ll m_{\mathrm{e}}^{*} \), it follows that:

\[
m_{r}^{*} = \frac{m_{\mathrm{e}}^{*} m_{\mathrm{h}}^{*}}{m_{\mathrm{e}}^{*} + m_{\mathrm{h}}^{*}} \approx m_{\mathrm{h}}^{*}
\]

For the ground state, \( n=1 \), and \( E_{1} = E_{C} = 1.386 \, \mathrm{eV} \), thus:

\[
\begin{aligned}
m_{\mathrm{h}}^{*} &= \frac{8 \varepsilon^{2} \varepsilon_{0}^{2} h^{2} (E_{g} - E_{C})}{e^{4}} \\
&= \frac{8 \times 9.6^{2} \times (8.854 \times 10^{-12})^{2} \times (6.626 \times 10^{-34})^{2} \times 0.014 \times 1.6 \times 10^{-19}}{(1.6 \times 10^{-19})^{4}} \\
&= 8.67 \times 10^{-36}\, \mathrm{kg} = \frac{8.67 \times 10^{-36}}{9.1 \times 10^{-31}} \approx 9.52 \times 10^{-2} m_{0}
\end{aligned}
\]
\\
\midrule
\textbf{Ground-Truth Answer}: 9.52 \times 10^{-2} \\

\midrule
\textbf{Model Output}: 0.07\\
\bottomrule
\end{tabular}%
}
\caption{An example of Semiconductor Physics with knowledge deficiency error.}
\label{tabapp:error example 4}
\end{table}
\fi


\begin{table}[!t]
\centering
\footnotesize
\scalebox{1.0}{
\begin{tabular}{p{10cm}}
\toprule
\textbf{Problem}: 
Calculate the effective mass of a hole $ m_{\mathrm{h}}^{*} $. It is known that the static dielectric constant of $ \mathrm{InP} $ is $ 9.6 $, $ E_{g} = 1.400 \, \mathrm{eV} $, $ E_{C} = 1.386 \, \mathrm{eV} $, and holes in $ \mathrm{InP} $ are much lighter than electrons ($ m_{\mathrm{h}}^{*} \ll m_{\mathrm{e}}^{*} $).
\\
\midrule
\textbf{o1-mini Completion}: 
To calculate the effective mass of a hole $ m_{\mathrm{h}}^{*} $ in $ \mathrm{InP} $, we can use the effective Rydberg formula adapted for semiconductors. The effective Rydberg energy $ R^* $ is given by:

$$
R^* = \frac{\mu e^4}{2 (4 \pi \epsilon_0 \epsilon)^2 \hbar^2}
$$

where:
- $ \mu $ is the reduced mass, which for $ m_{\mathrm{h}}^{*} \ll m_{\mathrm{e}}^{*} $ simplifies to $ \mu \approx m_{\mathrm{h}}^{*} $.
- $ e $ is the elementary charge.
- $ \epsilon_0 $ is the vacuum permittivity.
- $ \epsilon $ is the static dielectric constant.
- $ \hbar $ is the reduced Planck's constant.

Assuming the exciton binding energy $ E_b $ in $ \mathrm{InP} $ is approximately $ 0.01 \, \mathrm{eV} $, we can solve for the reduced mass $ \mu $:

$$
\mu = \frac{2 R^* (4 \pi \epsilon_0 \epsilon)^2 \hbar^2}{e^4}
$$

\textcolor{red}{Given that $ R^* \approx E_b = 0.01 \, \mathrm{eV} $, and substituting the known values:}
\textcolor{red}{
$$
\mu \approx m_{\mathrm{h}}^{*} = \frac{0.01 \times 9.6^2}{13.6} \approx 0.07 \, m_0
$$
}
So the final answer is $ \boxed{0.07} $.

\textcolor{blue}{Correct knowledge should look like this:}

For hydrogen-like cases, the distance of these absorption lines from the fundamental absorption threshold is:

$$
E_{n} = E_{g} - \frac{m_{r}^{*} e^{4}}{8 \varepsilon^{2} \varepsilon_{0}^{2} h^{2} n^{2}}
$$

where $ m_{r}^{*} $ is the reduced mass of the electron-hole pair. Since $ m_{\mathrm{h}}^{*} \ll m_{\mathrm{e}}^{*} $, it follows that:

$$
m_{r}^{*} = \frac{m_{\mathrm{e}}^{*} m_{\mathrm{h}}^{*}}{m_{\mathrm{e}}^{*} + m_{\mathrm{h}}^{*}} \approx m_{\mathrm{h}}^{*}
$$

For the ground state, $ n=1 $, and $ E_{1} = E_{C} = 1.386 \, \mathrm{eV} $, thus:

$$
\begin{aligned}
m_{\mathrm{h}}^{*} &= \frac{8 \varepsilon^{2} \varepsilon_{0}^{2} h^{2} (E_{g} - E_{C})}{e^{4}} \\
&= \frac{8 \times 9.6^{2} \times (8.854 \times 10^{-12})^{2} \times (6.626 \times 10^{-34})^{2} \times 0.014 \times 1.6 \times 10^{-19}}{(1.6 \times 10^{-19})^{4}} \\
&= 8.67 \times 10^{-36}\, \mathrm{kg} = \frac{8.67 \times 10^{-36}}{9.1 \times 10^{-31}} \approx 9.52 \times 10^{-2} m_{0}
\end{aligned}
$$
\\
\midrule
\textbf{Ground-Truth Answer}: $ 9.52 \times 10^{-2} $ \\

\midrule
\textbf{Model Output}: $ 0.07 $\\
\bottomrule
\end{tabular}%
}
\caption{An example of Semiconductor Physics with knowledge deficiency error.}
\label{tabapp:error example 4}
\end{table}