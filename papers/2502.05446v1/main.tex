%%%%%%%% ICML 2025 EXAMPLE LATEX SUBMISSION FILE %%%%%%%%%%%%%%%%%

\documentclass{article}

% Recommended, but optional, packages for figures and better typesetting:
\usepackage{microtype}
\usepackage{graphicx}
\usepackage{subfigure}
\usepackage{booktabs} % for professional tables

% hyperref makes hyperlinks in the resulting PDF.
% If your build breaks (sometimes temporarily if a hyperlink spans a page)
% please comment out the following usepackage line and replace
% \usepackage{icml2025} with \usepackage[nohyperref]{icml2025} above.
\usepackage{hyperref}

\usepackage[ruled,vlined,linesnumbered,algo2e]{algorithm2e}
% Attempt to make hyperref and algorithmic work together better:
\newcommand{\theHalgorithm}{\arabic{algorithm}}

% Use the following line for the initial blind version submitted for review:
%\usepackage{icml2025}
%\usepackage{algpseudocode}

% If accepted, instead use the following line for the camera-ready submission:
\usepackage[accepted]{icml2025}

% For theorems and such
\usepackage{amsmath}
\usepackage{amssymb}
\usepackage{mathtools}
\usepackage{amsthm}
\def\method{\text MixMin~}
\def\methodnospace{\text MixMin}
\def\genmethod{$\mathbb{R}$\text Min~}
\def\genmethodnospace{ $\mathbb{R}$\text Min}

% if you use cleveref..
\usepackage[capitalize,noabbrev]{cleveref}
\usepackage{textcomp} 
\crefformat{prop}{Prop~#2#1#3}
\crefformat{figure}{Fig~#2#1#3}
\crefformat{lem}{Lem~#2#1#3}
\crefformat{lemma}{Lem~#2#1#3}
\crefformat{section}{Sec~#2#1#3}
\crefformat{equation}{Eq~(#2#1#3)}
\crefformat{algorithm}{Alg~#2#1#3}
\crefformat{fact}{Fact~#1}
\crefformat{chapter}{Chapter~#2#1#3}
\crefalias{lem}{Lemma}
\crefformat{theorem}{Thm~#1}
\crefformat{corollary}{Cor~#2#1#3}
\crefformat{proposition}{Prop~#2#1#3}
\Crefname{appendix}{Appx}{Appx}



%%%%%%%%%%%%%%%%%%%%%%%%%%%%%%%%
% THEOREMS
%%%%%%%%%%%%%%%%%%%%%%%%%%%%%%%%
%\theoremstyle{plain}
%\newtheorem{theorem}{Theorem}[section]
%\newtheorem{proposition}[theorem]{Proposition}
%\newtheorem{lemma}[theorem]{Lemma}
%\newtheorem{corollary}[theorem]{Corollary}
%\theoremstyle{definition}
%\newtheorem{definition}[theorem]{Definition}
%\newtheorem{assumption}[theorem]{Assumption}
%\theoremstyle{remark}
%\newtheorem{remark}[theorem]{Remark}

% Todonotes is useful during development; simply uncomment the next line
%    and comment out the line below the next line to turn off comments
%\usepackage[disable,textsize=tiny]{todonotes}
\usepackage[textsize=tiny]{todonotes}


% The \icmltitle you define below is probably too long as a header.
% Therefore, a short form for the running title is supplied here:
\icmltitlerunning{SFBD: 
Training Diffusion Models with Finite Noisy Datasets}

\begin{document}

\twocolumn[
\icmltitle{Stochastic Forward–Backward Deconvolution: \\
Training Diffusion Models with Finite Noisy Datasets}

% It is OKAY to include author information, even for blind
% submissions: the style file will automatically remove it for you
% unless you've provided the [accepted] option to the icml2025
% package.

% List of affiliations: The first argument should be a (short)
% identifier you will use later to specify author affiliations
% Academic affiliations should list Department, University, City, Region, Country
% Industry affiliations should list Company, City, Region, Country

% You can specify symbols, otherwise they are numbered in order.
% Ideally, you should not use this facility. Affiliations will be numbered
% in order of appearance and this is the preferred way.
%\icmlsetsymbol{equal}{*}

\begin{icmlauthorlist}
\icmlauthor{Haoye Lu}{yyy}
\icmlauthor{Qifan Wu}{yyy}
\icmlauthor{Yaoliang Yu}{yyy}
%\icmlauthor{}{sch}
%\icmlauthor{}{sch}
\end{icmlauthorlist}

\icmlaffiliation{yyy}{Cheriton School of Computer Science, University of Waterloo, Waterloo, Canada}
%\icmlaffiliation{comp}{Company Name, Location, Country}
%\icmlaffiliation{sch}{School of ZZZ, Institute of WWW, Location, Country}

\icmlcorrespondingauthor{Haoye Lu}{haoye.lu@uwaterloo.ca}
%\icmlcorrespondingauthor{Firstname2 Lastname2}{first2.last2@www.uk}

% You may provide any keywords that you
% find helpful for describing your paper; these are used to populate
% the "keywords" metadata in the PDF but will not be shown in the document
%\icmlkeywords{Machine Learning, ICML}

%\vskip 0.3in
]

% this must go after the closing bracket ] following \twocolumn[ ...

% This command actually creates the footnote in the first column
% listing the affiliations and the copyright notice.
% The command takes one argument, which is text to display at the start of the footnote.
% The \icmlEqualContribution command is standard text for equal contribution.
% Remove it (just {}) if you do not need this facility.

\printAffiliationsAndNotice{}  % leave blank if no need to mention equal contribution
%\printAffiliationsAndNotice{\icmlEqualContribution} % otherwise use the standard text.



\begin{abstract}
Recent diffusion-based generative models achieve remarkable results by training on massive datasets, yet this practice raises concerns about memorization and copyright infringement. A proposed remedy is to train exclusively on noisy data with potential copyright issues, ensuring the model never observes original content. However, through the lens of deconvolution theory, we show that although it is theoretically feasible to learn the data distribution from noisy samples, the practical challenge of collecting sufficient samples makes successful learning nearly unattainable. To overcome this limitation, we propose to pretrain the model with a small fraction of clean data to guide the deconvolution process. Combined with our Stochastic Forward--Backward Deconvolution (SFBD) method, we attain an FID of $6.31$ on CIFAR-10 with just $4\%$ clean images (and $3.58$ with $10\%$). Theoretically, we prove that SFBD guides the model to learn the true data distribution. The result also highlights the importance of pretraining on limited but clean data or the alternative from similar datasets.  Empirical studies further support these findings and offer additional insights.
\end{abstract}

\section{Introduction}
\label{sec:intro}
\section{Introduction}


\begin{figure}[t]
\centering
\includegraphics[width=0.6\columnwidth]{figures/evaluation_desiderata_V5.pdf}
\vspace{-0.5cm}
\caption{\systemName is a platform for conducting realistic evaluations of code LLMs, collecting human preferences of coding models with real users, real tasks, and in realistic environments, aimed at addressing the limitations of existing evaluations.
}
\label{fig:motivation}
\end{figure}

\begin{figure*}[t]
\centering
\includegraphics[width=\textwidth]{figures/system_design_v2.png}
\caption{We introduce \systemName, a VSCode extension to collect human preferences of code directly in a developer's IDE. \systemName enables developers to use code completions from various models. The system comprises a) the interface in the user's IDE which presents paired completions to users (left), b) a sampling strategy that picks model pairs to reduce latency (right, top), and c) a prompting scheme that allows diverse LLMs to perform code completions with high fidelity.
Users can select between the top completion (green box) using \texttt{tab} or the bottom completion (blue box) using \texttt{shift+tab}.}
\label{fig:overview}
\end{figure*}

As model capabilities improve, large language models (LLMs) are increasingly integrated into user environments and workflows.
For example, software developers code with AI in integrated developer environments (IDEs)~\citep{peng2023impact}, doctors rely on notes generated through ambient listening~\citep{oberst2024science}, and lawyers consider case evidence identified by electronic discovery systems~\citep{yang2024beyond}.
Increasing deployment of models in productivity tools demands evaluation that more closely reflects real-world circumstances~\citep{hutchinson2022evaluation, saxon2024benchmarks, kapoor2024ai}.
While newer benchmarks and live platforms incorporate human feedback to capture real-world usage, they almost exclusively focus on evaluating LLMs in chat conversations~\citep{zheng2023judging,dubois2023alpacafarm,chiang2024chatbot, kirk2024the}.
Model evaluation must move beyond chat-based interactions and into specialized user environments.



 

In this work, we focus on evaluating LLM-based coding assistants. 
Despite the popularity of these tools---millions of developers use Github Copilot~\citep{Copilot}---existing
evaluations of the coding capabilities of new models exhibit multiple limitations (Figure~\ref{fig:motivation}, bottom).
Traditional ML benchmarks evaluate LLM capabilities by measuring how well a model can complete static, interview-style coding tasks~\citep{chen2021evaluating,austin2021program,jain2024livecodebench, white2024livebench} and lack \emph{real users}. 
User studies recruit real users to evaluate the effectiveness of LLMs as coding assistants, but are often limited to simple programming tasks as opposed to \emph{real tasks}~\citep{vaithilingam2022expectation,ross2023programmer, mozannar2024realhumaneval}.
Recent efforts to collect human feedback such as Chatbot Arena~\citep{chiang2024chatbot} are still removed from a \emph{realistic environment}, resulting in users and data that deviate from typical software development processes.
We introduce \systemName to address these limitations (Figure~\ref{fig:motivation}, top), and we describe our three main contributions below.


\textbf{We deploy \systemName in-the-wild to collect human preferences on code.} 
\systemName is a Visual Studio Code extension, collecting preferences directly in a developer's IDE within their actual workflow (Figure~\ref{fig:overview}).
\systemName provides developers with code completions, akin to the type of support provided by Github Copilot~\citep{Copilot}. 
Over the past 3 months, \systemName has served over~\completions suggestions from 10 state-of-the-art LLMs, 
gathering \sampleCount~votes from \userCount~users.
To collect user preferences,
\systemName presents a novel interface that shows users paired code completions from two different LLMs, which are determined based on a sampling strategy that aims to 
mitigate latency while preserving coverage across model comparisons.
Additionally, we devise a prompting scheme that allows a diverse set of models to perform code completions with high fidelity.
See Section~\ref{sec:system} and Section~\ref{sec:deployment} for details about system design and deployment respectively.



\textbf{We construct a leaderboard of user preferences and find notable differences from existing static benchmarks and human preference leaderboards.}
In general, we observe that smaller models seem to overperform in static benchmarks compared to our leaderboard, while performance among larger models is mixed (Section~\ref{sec:leaderboard_calculation}).
We attribute these differences to the fact that \systemName is exposed to users and tasks that differ drastically from code evaluations in the past. 
Our data spans 103 programming languages and 24 natural languages as well as a variety of real-world applications and code structures, while static benchmarks tend to focus on a specific programming and natural language and task (e.g. coding competition problems).
Additionally, while all of \systemName interactions contain code contexts and the majority involve infilling tasks, a much smaller fraction of Chatbot Arena's coding tasks contain code context, with infilling tasks appearing even more rarely. 
We analyze our data in depth in Section~\ref{subsec:comparison}.



\textbf{We derive new insights into user preferences of code by analyzing \systemName's diverse and distinct data distribution.}
We compare user preferences across different stratifications of input data (e.g., common versus rare languages) and observe which affect observed preferences most (Section~\ref{sec:analysis}).
For example, while user preferences stay relatively consistent across various programming languages, they differ drastically between different task categories (e.g. frontend/backend versus algorithm design).
We also observe variations in user preference due to different features related to code structure 
(e.g., context length and completion patterns).
We open-source \systemName and release a curated subset of code contexts.
Altogether, our results highlight the necessity of model evaluation in realistic and domain-specific settings.






\section{Related Work}
\label{sec:related}
\putsec{related}{Related Work}

\noindent \textbf{Efficient Radiance Field Rendering.}
%
The introduction of Neural Radiance Fields (NeRF)~\cite{mil:sri20} has
generated significant interest in efficient 3D scene representation and
rendering for radiance fields.
%
Over the past years, there has been a large amount of research aimed at
accelerating NeRFs through algorithmic or software
optimizations~\cite{mul:eva22,fri:yu22,che:fun23,sun:sun22}, and the
development of hardware
accelerators~\cite{lee:cho23,li:li23,son:wen23,mub:kan23,fen:liu24}.
%
The state-of-the-art method, 3D Gaussian splatting~\cite{ker:kop23}, has
further fueled interest in accelerating radiance field
rendering~\cite{rad:ste24,lee:lee24,nie:stu24,lee:rho24,ham:mel24} as it
employs rasterization primitives that can be rendered much faster than NeRFs.
%
However, previous research focused on software graphics rendering on
programmable cores or building dedicated hardware accelerators. In contrast,
\name{} investigates the potential of efficient radiance field rendering while
utilizing fixed-function units in graphics hardware.
%
To our knowledge, this is the first work that assesses the performance
implications of rendering Gaussian-based radiance fields on the hardware
graphics pipeline with software and hardware optimizations.

%%%%%%%%%%%%%%%%%%%%%%%%%%%%%%%%%%%%%%%%%%%%%%%%%%%%%%%%%%%%%%%%%%%%%%%%%%
\myparagraph{Enhancing Graphics Rendering Hardware.}
%
The performance advantage of executing graphics rendering on either
programmable shader cores or fixed-function units varies depending on the
rendering methods and hardware designs.
%
Previous studies have explored the performance implication of graphics hardware
design by developing simulation infrastructures for graphics
workloads~\cite{bar:gon06,gub:aam19,tin:sax23,arn:par13}.
%
Additionally, several studies have aimed to improve the performance of
special-purpose hardware such as ray tracing units in graphics
hardware~\cite{cho:now23,liu:cha21} and proposed hardware accelerators for
graphics applications~\cite{lu:hua17,ram:gri09}.
%
In contrast to these works, which primarily evaluate traditional graphics
workloads, our work focuses on improving the performance of volume rendering
workloads, such as Gaussian splatting, which require blending a huge number of
fragments per pixel.

%%%%%%%%%%%%%%%%%%%%%%%%%%%%%%%%%%%%%%%%%%%%%%%%%%%%%%%%%%%%%%%%%%%%%%%%%%
%
In the context of multi-sample anti-aliasing, prior work proposed reducing the
amount of redundant shading by merging fragments from adjacent triangles in a
mesh at the quad granularity~\cite{fat:bou10}.
%
While both our work and quad-fragment merging (QFM)~\cite{fat:bou10} aim to
reduce operations by merging quads, our proposed technique differs from QFM in
many aspects.
%
Our method aims to blend \emph{overlapping primitives} along the depth
direction and applies to quads from any primitive. In contrast, QFM merges quad
fragments from small (e.g., pixel-sized) triangles that \emph{share} an edge
(i.e., \emph{connected}, \emph{non-overlapping} triangles).
%
As such, QFM is not applicable to the scenes consisting of a number of
unconnected transparent triangles, such as those in 3D Gaussian splatting.
%
In addition, our method computes the \emph{exact} color for each pixel by
offloading blending operations from ROPs to shader units, whereas QFM
\emph{approximates} pixel colors by using the color from one triangle when
multiple triangles are merged into a single quad.




\section{Prelimilaries}
\label{sec:prel}
\section{Preliminary notation and problem definition}\label{sec:prel}

We begin by providing preliminary notation and formally defining our problem.

Our input is a sequence of graphs $\calG = (G_1, \ldots, G_r)$, where each snapshot $G_i = (V, E_i)$
is defined over the same set of nodes. We often denote the number of nodes and edges by $n = \abs{V}$. % and $m_i = \abs{E_i}$, or $m = \abs{E}$, if $i$ is omitted.

Given a graph $G = (V, E)$, and a set of nodes $S \subseteq V$, we define $E(S) \subseteq E$ to be the subset of edges
having both endpoints in $S$. We also write $m(S) = \abs{E(S)}$. Given a  graph sequence $\calG = (G_1, \ldots, G_r)$ with $G_i = (V, E_i)$, we write $E(S, G_i) = E_i(S)$ and $m(S, G_i) = \abs{E(S, G_i)}$.

As mentioned before, our goal is to find dense subgraphs in a temporal network, and for that
we need to quantify the density of a subgraph. More formally,
assume an unweighted graph $G = (V, E)$, and let $S \subseteq V$.
We define the \emph{density} $\dens{S, G}$ of a graph $G_i$ induced by node set $S$,
and extend this definition for a sequence of graphs $\calG = (G_1, \ldots, G_k)$
as
\[
	\dens{S, G_i} = \frac{\abs{E(S, G_i)}}{\abs{S}}
	\quad\quad \text{and}\quad\quad
	\dens{S, \calG} = \sum_{i = 1}^r \dens{S, G_i}
	\quad.
\]


We first state the problem of finding a common subgraph in a graph sequence which maximizes the sum of the densities proposed by Semertzidis et al.~\cite{semertzidis2019finding}.

\begin{problem}[Total densest subgraph problem~(\problemdts)]
\label{pr:dts}
Given a graph sequence $\calG = (G_1,\dots,G_r)$, with  $G_i = (V, E_i)$, find a common subset of vertices
$S$, 
such that $\dens{S, \calG}$ is maximized.
\end{problem}

%We refer to this problem as \problemdts.
This problem can be solved by first flattening the graph sequence into one weighted graph, where an edge weight is the number of snapshots in which an edge occurs.
The problem is then a standard (weighted) densest subgraph problem that can be solved
using the exact method given by Goldberg~\cite{goldberg1984finding} in $\bigO{n(n + m) (\log n + \log r)}$ time.



Next, we introduce our main problem where an additional fairness constraint of the induced densities is considered. 
To this end, 
we denote the difference between the maximum and minimum induced density as
\[
    \diff{S, \calG} = \max_i \dens{S, G_i} - \min_i \dens{S, G_i} \quad.
\]
Given a sequence of graph snapshots and an input parameter $\alpha$, our goal
is to find a subset of vertices,
such that the sum of the densities of subgraphs is maximized while maintaining the difference $\diff{S, \calG}$ at most $\alpha$.


\begin{problem}[Fair densest subgraph problem~(\problemcdcsm)]
\label{pr:fair-dense-min}
Given a graph sequence $\calG = (G_1,\dots,G_r)$, with  $G_i = (V, E_i)$ and  real number  $\alpha$, find a  subset of vertices
$S$, 
such that $\dens{S, \calG}$ is maximized
and $\diff{S, \calG} \leq  \alpha$.
\end{problem}

Note that for $\alpha = 0$, the \problemcdcsm problem is equivalent to finding a subgraph which induces exactly the same density over each snapshot while maximizing the total density.
On the other hand, setting $\alpha = \infty$ reduces \problemcdcsm to \problemdts.

%The main difference with \problemdts is that we expect fair density distribution among snapshots.
%In other words,  we extend the idea of temporal densest subgraph in which $\alpha = \infty$, for more generalized case where $ 0 \leq \alpha < \infty$ . 




% Next we present another variant of the fair densest subgraph problem called \problemcdcsm.
% Assume that we are given an input parameter $\alpha$ and  $S \subseteq V$ which is a solution set to the  \problemcdcsm problem.
% We define $d_{\mathit{max}}$ to be the maximum density induced by $S$ over any of the snapshot.
% In \problemcdcsm, we enforce that the density induced on each snapshot should be at least $\alpha \times d_{max}$.

% Instead of $\alpha$ of dimension $1$, we can take a vector of length $r$ as the input parameter which generalizes \problemcdcs and \problemcdcsm problems further.

Next, we present a minimization variant of \problemcdcsm problem where given an input parameter $\sigma$, the goal is to find a subset of vertices $S$ that minimizes the difference $\diff{S, \calG}$  while inducing a total density of at least $\sigma$. %We refer to this problem as \problemcdcsdiff. 
% Note that we do not consider the case of $\alpha = 0$ which corresponds to the zero-solution.


\begin{problem}[The smallest difference densest subgraph problem~(\problemcdcsdiff)]
\label{pr:diff}
Given a graph sequence $\calG = (G_1,\dots,G_r)$, with  $G_i = (V, E_i)$ and real number $\sigma$, find a common subset of vertices $S$, 
such that the density induced by $S$  over  $\calG$  is at least $\sigma$ and $\diff{S, \calG}$ is minimized.
\end{problem}

Finally, we state the minimum densest subgraph problem~\cite{jethava2015finding} where the goal is to find a common subgraph which maximizes the minimum density.

\begin{problem}[Minimum densest subgraph~(\problemdcs)]
\label{pr:dcs}
Given a graph sequence $\calG = (G_1,\dots,G_r)$, with  $G_i = (V, E_i)$, find a common subset of vertices
$S$, 
such that $\min_i \dens{S, G_i}$, the minimum density induced by  $S$  over  $\calG$  is maximized.
\end{problem}



\section{Theoretical Limit of Deconvolution}
\label{sec:deconv}
In this section, we evaluate the complexity of a deconvolution problem when the data corruption process is modelled using a forward diffusion process. Through the framework of deconvolution theory, we demonstrate that while \citet{DarasDD2024} showed that diffusion models can be trained using noisy samples, obtaining a sufficient number of samples to train high-quality models is practically infeasible.

The following two theorems establish that the optimal convergence rate for estimating the data density is $\Oc(\log n)^{-2}$. These results, derived using standard deconvolution theory~\citep{Meister2009} under a Gaussian noise assumption, highlight the inherent difficulty of the problem. We present the result for $d = 1$, which suffices to illustrate the challenge.  
\begin{restatable}{theorem}{MISEUpperBdound}
\label{thm:MISE_upper_bound}
Assume $ \Yc $ is generated according to \eqref{eq:gen_conv_samples} with $ \epsilon \sim \Nc(0, \sigma_\zeta^2) $ and $ p_\text{data} $  is a univariate distribution.  Under some weak assumptions on $ p_\text{data}$, for a sufficiently large sample size $ n $, there exists an estimator $ \hat{p}(\cdot; \mathcal{Y}) $ such that
    \begin{align}
        \mathrm{MISE}(\hat{p}, p_\textrm{data}) \leq C \cdot \frac{ \sigma_\zeta^4}{(\log n)^2},\label{eq:MISE_upper_bound}
    \end{align}
    where $ C $ is determined by $ p_\textrm{data} $.
\end{restatable}
\begin{restatable}{theorem}{MISELowerBound}
\label{thm:MISE_lower_bound}
In the same setting as \cref{thm:MISE_upper_bound}, for an arbitrary estimator $ \hat{p}(\cdot; \mathcal{Y}) $ of $p_\textrm{data}$ based on $\Yc$, 
\begin{align}
	\mathrm{MISE}(\hat{p}, p_\textrm{data}) \geq K \cdot (\log n)^{-2},
\end{align}
where $K>0$ is determined by $p_\textrm{data}$ and error distribution~$h$. 
\end{restatable}
The optimal convergence rate $\Oc(\log n)^{-2}$ indicates that reducing the MISE to one-fourth of its current value requires an additional $ n^2 - n $ samples. In contrast, under the error-free scenario, the optimal convergence rate is known to be $ \Oc(n^{-4/5})$ \citep{Wand1998}, where reducing the MISE to one-fourth of its current value would only necessitate approximately $ 4.657n $ additional samples.

The pessimistic rate indicates that effectively training a generative model using only corrupted samples with Gaussian noise is nearly impossible. Consequently, this implies that training from scratch,  using only noisy images, with the consistency loss discussed in \cref{prel:deconv_through_consist_const}, is infeasible. Notably, as indicated by \cref{eq:MISE_upper_bound}, this difficulty becomes significantly more severe with larger $\sigma_{\zeta}^2$, while a large $\sigma_{\zeta}^2$ is typically required to alter the original samples significantly to address copyright and privacy concerns.

To address the pessimistic statistical rate, we propose pretraining diffusion models on a small set of copyright-free samples. While this limited dataset can only capture a subset of the features and variations of the full true data distribution, we argue that it provides valuable prior information, enabling the model to start from a point much closer to the ground distribution compared to random weight initialization. For example, for image generation, pretraining allows the model to learn common features and structures shared among samples, such as continuity, smoothness, edges, and general appearance of typical object types.

Unfortunately, our empirical study in \cref{sec:emp} will show that the consistency loss-based method discussed in \cref{prel:deconv_through_consist_const} cannot deliver promising results even after pretraining. We suspect that this is caused by the gap between their theoretical framework and the practical implementation. As a result, we propose SFBD in \cref{sec:SFBD} to bridge such a gap. 




\section{Stochastic Forward–Backward Deconvolution}
\label{sec:SFBD}
In this section, we introduce a novel method for solving the deconvolution problem that integrates seamlessly with the existing diffusion model framework. As our approach involves iteratively applying the forward diffusion process described in \cref{eq:fwd_diff}, followed by a backward step with an optimized drift, we refer to this method as Stochastic Forward-Backward Deconvolution (SFBD), as described in \cref{alg:SFBD}. 

The proposed algorithm begins with a small set of clean data, $\Dc_\text{clean}$, for pretraining, followed by iterative optimization using a large set of noisy samples. As demonstrated in \cref{sec:emp}, decent quality images can be achieved on datasets such as CIFAR-10 \citep{Krizhevsky2009} and CelebA \citep{LiuLWT2015} using as few as 50 clean images. During pretraining, the algorithm produces a neural network denoiser, $D_{{\phiv}_0}$, which serves as the initialization for the subsequent iterative optimization process. Specifically, the algorithm alternates between the following two steps: for $k = 1, 2, \ldots K$, 
\begin{enumerate}
	\item (Backward Sampling) This step can be intuitively seen as a denoising process for samples in $\Dc_\text{noisy}$ using the backward SDE \cref{eq:anderson_bwd}. In each iteration, we use the best estimation of the score function so far induced by $D_{{\phiv}_{k-1}}$ through \cref{eq:score_repara_in_denoiser}. 
	\item (Denoiser Update) Fine-tune denoiser $D_{{\phiv}_{k-1}}$ to obtain $D_{{\phiv}_k}$ by minimizing \cref{eq:denoiser_loss} with the denoised samples obtained in the previous step. 
\end{enumerate}
The following proposition shows that when $\Dc_\text{noisy}$ contains sufficiently many samples to characterize the true noisy distribution $p_{\rm data} \conv h$, when $K \rightarrow \infty$, the diffusion model implemented by denoiser $D_{{\phiv}_K}$ has the sample distribution converging to the true $p_\text{data}$. 

\begin{algorithm}[!t]
\DontPrintSemicolon
   \caption{Stochastic Forward–Backward Deconvolution. (Given sample set $\Dc$, $p_\Dc$ denotes the corresponding empirical distribution.)}
   \label{alg:SFBD}

    \KwIn{clean data: $\Dc_\text{clean} = \{\xv^{(i)}\}_{i=1}^M$, noisy data: $\Dc_\text{noisy} = \{\yv_\tau^{(i)}\}_{i=1}^N$, number of iterations: $K$.} 

    \tcp{Initialize Denoiser}
    
    $\phiv_0 \leftarrow$ Pretrain $D_{\phiv}$ using \cref{eq:denoiser_loss} with $p_0 = p_{\Dc_\text{clean}}$

    \For{$k = 1$ to $K$}{
        \tcp{Backward Sampling}
        
        $\Ec_k \leftarrow \{ \yv_0^{(i)} : \forall \yv_\tau^{(i)} \in \Dc_\text{noisy}$, solve backward SDE \cref{eq:anderson_bwd} from $\tau$ to $0$, starting from $\yv_\tau^{(i)}$, where the score function is estimated as $\frac{D_{\phiv_{k-1}}(\xv_t, t) - \xv_t}{\sigma_t^2} \}$
        
        \tcp{Denoiser Update}
        
        $\phiv_k \leftarrow$ Train $D_{\phiv}$ by minimizing \cref{eq:denoiser_loss} with $p_0 = p_{\Ec_k}$
    }

    \KwOut{Final denoiser $D_{{\phiv}_K}$}
\end{algorithm}


\begin{restatable}{proposition}{convSFBD}
\label{prop:conv_SFBD}
	Let $p^*_t$ be the density of $\xv_t$ obtained by solving the forward diffusion process \cref{eq:fwd_diff} with $\xv_0 \sim p_\text{data}$, where we have $p^*_\zeta = p_{\rm data} \conv h$. Consider a modified \cref{alg:SFBD}, where the empirical distribution $P_{\Dc_\text{noisy}}$ is replaced with the ground truth $p^*_\zeta$.  Correspondingly, $p_{{\Ec}_k}$ becomes $p_0^{(k)}$, the distribution of $\xv_0$ induced by solving:
	\begin{align}
		\diff \xv_t = - g(t)^2 \, \sv_{\phiv_{k-1}}(\xv_t, t) \diff t + g(t) \diff \bar\wv_t, ~\xv_\zeta \sim p^*_\zeta \label{eq:conv_SFBD:bwd}
	\end{align}
	from $\zeta$ to $0$, where $\sv_{\phiv_k}(\xv_t, t) = \frac{D_{\phiv_k}(\xv_t, t) - \xv_t}{\sigma_t^2}$, $g(t) = (\frac{\diff \sigma^2_t}{\diff t})^{\sfrac{1}{2}}$ and $D_{{\phiv}_k}$ is obtained by minimizing \eqref{eq:denoiser_loss} according to \cref{alg:SFBD}. Assume $D_{{\phiv}_k}$ reaches the optimal for all~$k$. Under mild assumptions, for $k \geq 0$, we have
	\begin{align}
		\KL{p_\text{data}}{p_0^{(k)}} \geq \KL{p_\text{data}}{p_0^{(k+1)}}. \label{eq:conv_SFBD:monotone}
	\end{align}
	In addition, for all $K \geq 1$ and $\uv \in \Rb^d$, we have
	\begin{align*}
			\min_{k = 1,\ldots K}\left|\Phi_{p_\text{data}}(\uv) - \Phi_{p_0^{(k)}}(\uv)\right|\leq \exp \big(\frac{\sigma_\zeta^2}{2} \|\uv\|^2 \big) \sqrt{\frac{2 M_0}{K}},
	\end{align*}	
	 where
	\begin{align*}
		M_0 = \tfrac{1}{2} \int_0^\zeta g(t)^2 \Eb_{p_{t}^*}\big\|\nabla \log p^*_t(\xv_t) - \sv_{\phiv_0}(\xv_t, t) \big\|^2 \diff t. 
	\end{align*}
\end{restatable}
\cref{prop:conv_SFBD} shows, after sufficiently many iterations of backward sampling and denoiser updates, the distribution of denoised samples generated by the backward sampling step converges to the true data distribution at a rate of $\Oc(1 / \sqrt{K})$. Consequently, after fine-tuning the denoiser on these denoised samples during the Denoiser Update step, the diffusion model is expected to generate samples that approximately follow the data distribution, solving the deconvolution problem. It is important to note that this result describes the convergence rate of SFBD under the assumption of an infinite number of noisy samples, which is distinct from the optimal sample efficiency rate discussed in \cref{sec:deconv}.

\textbf{The importance of pretraining.} \cref{prop:conv_SFBD} also highlights the critical role of pretraining, as it allows the algorithm to begin fine-tuning from a point much closer to the true data distribution. Specifically, effective pretraining ensures that $\sv_{\phiv_0}$ closely approximates the ground-truth score, leading to a smaller  $M_0$  in \cref{prop:conv_SFBD}. This, in turn, reduces the number of iterations  $K$  required for the diffusion model to generate high-quality samples.

\textbf{The practical limits of increasing $K$.}  While \cref{prop:conv_SFBD} suggests that increasing the number of iterations $K$ can continuously improve sample quality, practical limitations come into play. Sampling errors introduced during the backward sampling process, as well as imperfections in the denoiser updates, accumulate over time. These errors eventually offset the benefits of additional iterations, as demonstrated in \cref{sec:emp}. This observation further highlights the importance of pretraining to mitigate the impact of such errors and achieve high-quality samples with fewer iterations.

\textbf{Alternative methods for backward sampling.} While the backward sampling in \cref{alg:SFBD} is presented as a naive solution to the backward SDE in \cref{eq:anderson_bwd}, the algorithm is not limited to this approach. Any backward SDE and solver yielding the same marginal distribution as \cref{eq:anderson_bwd} can be employed. Alternatives include PF-ODE, the predictor-corrector sampler \citep{SongDCKKEP2021}, DEIS \citep{ZhangC2023}, and the $2^\text{nd}$ order Heun method used in EDM \citep{KarrasAAL22}. Compared to the Euler–Maruyama method, these approaches require fewer network evaluations and offer improved error control for imperfect score estimation and step discretization. As the algorithm generates $\Ec_k$ that contains samples closer to $p_\text{data}$ with increasing $k$, clean images used for pretraining can be incorporated into $\Ec_k$ to accelerate this process. In our empirical study, this technique is applied whenever clean samples and noisy samples (prior to corruption) originate from the same distribution.

\begin{table}[!t]
\caption{Performance comparison of generative models. When $\sigma_\zeta > 0$, the models are trained on noisy images corrupted by Gaussian noise $\mathcal{N}(\zero, \sigma_\zeta^2 \Iv)$ after rescaling pixel values to $[-1, 1]$. For pretrained models, 50 clean images are randomly sampled from the training datasets for pretraining. \underline{Underscored} results are produced by this work. \textbf{Bolded} values indicate the best performance.}
\label{tb:performance_compare}
\resizebox{\columnwidth}{!}{%
\begin{tabular}{@{}lccr|ccr@{}}
\toprule
\multirow{2}{*}{Method} & \multicolumn{3}{c}{CIFAR10 (32 x 32)}                                          & \multicolumn{3}{c}{CelebA  (64 x 64)}                                              \\ \cmidrule(l){2-7} 
                        & $\sigma_\zeta$ & Pretrain             & FID    & $\sigma_\zeta$ & Pretrain & FID \\ \midrule
DDPM \citep{HoJA2020}                   & 0.0   & No                   & 4.04   & 0.0                       & No                           & \textbf{3.26}                    \\
DDIM \citep{SongME2021}                   & 0.0   & No                   & 4.16   & 0.0                       & No                           & 6.53                    \\
EDM \citep{KarrasAAL22}                    & 0.0   & No                   & \textbf{1.97}   & -                       & -                           & -                       \\ \midrule
SURE-Score \citep{AaliAKT2023}             & 0.2                       & Yes                  & 132.61 & -                         & -                            & -                       \\
AmbDiff \citep{DarasSDGDK2023}             & 0.2                       & No                   & 114.13  & -                         & -                            & -                       \\
EMDiff \citep{BaiWCS2024}            & 0.2                       & Yes                  & 86.47  & -                         & -                            & -                       \\
TweedieDiff \citep{DarasDD2024}       & 0.2                       & No                   & \underline{167.23} & 0.2                       & No                           & \underline{246.95}                  \\
TweedieDiff \citep{DarasDD2024}       & 0.2                       & Yes                  & \underline{65.21}  & 0.2                       & Yes                          & \underline{58.52}                   \\
SFBD (Ours)             & 0.2                       & Yes                  & \underline{\textbf{13.53}} & 0.2                       & Yes                          &      \underline{\textbf{6.49}}                   \\
\bottomrule
\end{tabular}
}
\vspace{-0.8em}
\end{table}

\textbf{Relationship to the consistency loss.} SFBD can be seen as an algorithm that enforces the consistency constraint across all positive time steps and time zero. Specifically, we have
\begin{restatable}{proposition}{RELCONSSFBD}
\label{prop:rel_btw_cons_SFBD}
Assume that the denoising network $D_{\phiv}$ is implemented to satisfy $D_{\phiv}(\cdot, 0) = \textrm{Id}(\cdot)$. When $r = 0$, the consistency loss in \cref{eq:consistency_loss} is equivalent to the denoising noise in \cref{eq:denoiser_loss} for $t = s$. 
\end{restatable}
The requirement that $D_{\phiv}(\cdot, 0) = \textrm{Id}(\cdot)$ is both natural and intuitive, as $D_{\phiv}(\xv_0, 0)$ approximates $\Eb[\xv_0 | \xv_0] = \xv_0$. This fact is explicitly enforced in the design of the EDM framework \citep{KarrasAAL22}, which has been widely adopted in subsequent research.

A key distinction between SFBD and the original consistency loss implementation is that SFBD does not require sampling from $p_{r|s}$ or access to the ground-truth score function induced by the unknown data distribution $p_\text{data}$. This is because, in the original implementation, $p_0 = p_\text{data}$, whereas in SFBD,  $p_0 = p_0^{(k)}$, as defined in \cref{prop:conv_SFBD}, and is obtained iteratively through the backward sampling step. As $k$  increases,  $p_0^{(k)}$  converges to  $p_\text{data}$, ensuring that the same consistency constraints are eventually enforced. Consequently, SFBD bridges the gap between theoretical formulation and practical implementation that exists in the original consistency loss framework.

%\textbf{Relationship to IPF.} The forward and backpack-flavoured optimization method is also used in the iterative proportional fitting (IPF) method to find Schrodinger's Bridge (an entropy regularized optimal transport scheme) \citep{BortoliTHD2021}. In comparison to IPF, SFBD has a fixed forward diffusion process, whereas in IPF, the forward process will be updated according to the results in previous iterations. 






%
%\section{Methods to Train Diffusion Models Using Noisy Images}
%\label{sec:mth}
%placeholder

Discuss adding mollified noise is equivalent to the 

\section{Empirical Study}
\label{sec:emp}

\begin{figure}[!t]
    \centering
    \setlength{\tabcolsep}{1pt} % Adjust table cell spacing
    \setlength{\fboxrule}{1pt} % Adjust box rule thickness
    \def\imagewidth{0.15\linewidth} % Adjust box rule thickness
    \def\imagewidthb{0.19\linewidth} % Adjust box rule thickness
    \def\celebaimgidxa{000000.png} % Adjust box rule thickness    
    \def\celebaimgidxb{000006.png} % Adjust box rule thickness
    \renewcommand{\arraystretch}{0.5}
    \centering
    \resizebox{\columnwidth}{!}{
    \begin{tabular}{c}  
        \begin{tabular}{cccccccc}
            \tiny{\makecell[c]{Noisy\\Observ.}} &
            \tiny{\makecell[c]{SURE-\\Score}} &
            \tiny{\makecell[c]{Amb-Diff}} &
            \tiny{\makecell[c]{EMDiff}} &
            \tiny{\makecell[c]{Tweedie\\ Diff w/o PT}} &
            \tiny{\makecell[c]{Tweedie\\ Diff w/ PT}} &
            \tiny{\makecell[c]{SFBD \\ (\textbf{Ours})}} &
            \tiny{\makecell[c]{Ground\\Truth}} \\
            % Row 1
            \begin{overpic}[width=\imagewidth]{figures/fig4_emscore/1-1.png}\end{overpic} & % noisy
            \begin{overpic}[width=\imagewidth]{figures/fig4_emscore/1-2.png}\end{overpic} & % SURE
            \begin{overpic}[width=\imagewidth]{figures/fig4_emscore/1-3.png}\end{overpic} & % Amb
            \begin{overpic}[width=\imagewidth]{figures/fig4_emscore/1-4.png}\end{overpic} & % EM
            \begin{overpic}[width=\imagewidth]{figures/cifar10/cifar10_tweedie_NoPT/000002.png}\end{overpic} & % Tweedie w/o PT 
            \begin{overpic}[width=\imagewidth]{figures/cifar10/cifar10_tweedie_PT/000002.png}\end{overpic} & % Tweedie w/ PT 
            \begin{overpic}[width=\imagewidth]{figures/cifar10/cifar10_SFBD_02_iter3/000002.png}\end{overpic} & % SFBD
            \begin{overpic}[width=\imagewidth]{figures/cifar10/cifar10_gt/1_gt1.png}\end{overpic} \\ % GT

            % Row 2
            \begin{overpic}[width=\imagewidth]{figures/fig4_emscore/2-1.png}\end{overpic} &
            \begin{overpic}[width=\imagewidth]{figures/fig4_emscore/2-2.png}\end{overpic} &
            \begin{overpic}[width=\imagewidth]{figures/fig4_emscore/2-3.png}\end{overpic} &
            \begin{overpic}[width=\imagewidth]{figures/fig4_emscore/2-4.png}\end{overpic} &
            \begin{overpic}[width=\imagewidth]{figures/cifar10/cifar10_tweedie_NoPT/000001.png}\end{overpic} & % Tweedie w/o PT 
            \begin{overpic}[width=\imagewidth]{figures/cifar10/cifar10_tweedie_PT/000001.png}\end{overpic} & % Tweedie w/ PT 
            \begin{overpic}[width=\imagewidth]{figures/cifar10/cifar10_SFBD_02_iter3/000001.png}\end{overpic} & % SFBD
            \begin{overpic}[width=\imagewidth]{figures/cifar10/cifar10_gt/1021_gt2.png}\end{overpic} \\

            % Row 3
            \begin{overpic}[width=\imagewidth]{figures/fig4_emscore/3-1.png}\end{overpic} &
            \begin{overpic}[width=\imagewidth]{figures/fig4_emscore/3-2.png}\end{overpic} &
            \begin{overpic}[width=\imagewidth]{figures/fig4_emscore/3-3.png}\end{overpic} &
            \begin{overpic}[width=\imagewidth]{figures/fig4_emscore/3-4.png}\end{overpic} &
            \begin{overpic}[width=\imagewidth]{figures/cifar10/cifar10_tweedie_NoPT/000000.png}\end{overpic} & % Tweedie w/o PT 
            \begin{overpic}[width=\imagewidth]{figures/cifar10/cifar10_tweedie_PT/000000.png}\end{overpic} & % Tweedie w/ PT 
            \begin{overpic}[width=\imagewidth]{figures/cifar10/cifar10_SFBD_02_iter3/000000.png}\end{overpic} & % SFBD
            \begin{overpic}[width=\imagewidth]{figures/cifar10/cifar10_gt/0_gt3.png}\end{overpic} \\
        \end{tabular}
    \end{tabular}
    }
    \vspace{0.5em}
    \resizebox{\columnwidth}{!}{
    \begin{tabular}{c}
        \begin{tabular}{cccccccc}
            \tiny{\makecell[c]{Noisy Observ.}} &
            \tiny{\makecell[c]{Tweedie\\ Diff w/o PT}} &
            \tiny{\makecell[c]{Tweedie\\ Diff w/ PT}} &
            \tiny{\makecell[c]{SFBD  (\textbf{Ours})}} &
            \tiny{\makecell[c]{Ground Truth}} \\
            % Row 1
            \begin{overpic}[width=\imagewidthb]{figures/celeba/noisy/\celebaimgidxa}\end{overpic} & % noisy
            \begin{overpic}[width=\imagewidthb]{figures/celeba/tweedie_NoPT/\celebaimgidxa}\end{overpic} & % Tweedie w/o PT 
            \begin{overpic}[width=\imagewidthb]{figures/celeba/tweedie_PT/\celebaimgidxa}\end{overpic} & % Tweedie w/ PT 
            \begin{overpic}[width=\imagewidthb]{figures/celeba/SFBD/\celebaimgidxa}\end{overpic} & % SFBD
            \begin{overpic}[width=\imagewidthb]{figures/celeba/clear/\celebaimgidxa}\end{overpic} \\ % GT

            % /Users/fieldlv/Dropbox/Apps/Overleaf/deconvolution/icml2025/figures/celeba

            % Row 2
            \begin{overpic}[width=\imagewidthb]{figures/celeba/noisy/\celebaimgidxb}\end{overpic} & % noisy
            \begin{overpic}[width=\imagewidthb]{figures/celeba/tweedie_NoPT/\celebaimgidxb}\end{overpic} & % Tweedie w/o PT 
            \begin{overpic}[width=\imagewidthb]{figures/celeba/tweedie_PT/\celebaimgidxb}\end{overpic} & % Tweedie w/ PT 
            \begin{overpic}[width=\imagewidthb]{figures/celeba/SFBD/\celebaimgidxb}\end{overpic} & % SFBD
            \begin{overpic}[width=\imagewidthb]{figures/celeba/clear/\celebaimgidxb}\end{overpic} \\ % GT

        \end{tabular}
    \end{tabular}
    }
    \caption{Denoised samples of CIFAR-10 (up) and CelebA (down). (Noise level $\sigma_\zeta = 0.2$)} 
    \label{fig:model_benchmark_visual}
\end{figure}

\begin{figure*}
\centering     %%% not \center
\subfigure[Clean Image Ratio]{\label{fig:ablation_cifar10:clean_ratio}\includegraphics[width=0.29\textwidth]{figures/charts/clean_percent_vs_fid.png}}\hfill
\subfigure[Noise Level]{\label{fig:ablation_cifar10:sigma}\includegraphics[width=0.29\textwidth]{figures/charts/noise_level_vs_fid.png}}\hfill
\subfigure[Pretraining on Similar Datasets]{\label{fig:ablation_cifar10:pretrain}\includegraphics[width=0.29\textwidth]{figures/charts/pretrain_var_dataset_vs_fid.png}}
%\vspace{-0.5em}
\caption{SFBD performance on CIFAR-10 under various conditions. Unless specified, the clean image ratio is $0.04$ and the noise level $\sigma_\zeta$ is $0.59$. In (a) and (b), FID at iteration 0 corresponds to the pretrained model. In (c), models are pretrained on clean images from the ``truck'' class, with FID at iteration 0 measuring the distance between these clean images and those used for fine-tuning. For the w/o pretraining setting, models are trained on the full CIFAR-10 dataset with $\sigma_\zeta = 0.59$.}
\label{fig:ablation_cifar10}
\end{figure*}




In this section, we demonstrate the effectiveness of the SFBD framework proposed in \cref{sec:SFBD}. Compared to other models trained on noisy datasets, SFBD consistently achieves superior performance across all benchmark settings. Additionally, we conduct ablation studies to validate our theoretical findings and offer practical insights for applying SFBD effectively.

\textbf{Datasets and evaluation metrics.} The experiments are conducted on the CIFAR-10 \citep{Krizhevsky2009} and CelebA \citep{LiuGL2022} datasets, with resolutions of  $32 \times 32$  and  $64 \times 64$, respectively. CIFAR-10 consists of 50,000 training images and 10,000 test images across 10 classes. CelebA, a dataset of human face images, includes a predefined split of 162,770 training images, 19,867 validation images, and 19,962 test images. For CelebA,  images were obtained using the preprocessing tool provided in the DDIM official repository  \citep{SongME2021}.

\begin{figure}[t]
	\centering
	\includegraphics[width=0.95\columnwidth]{figures/noise_level_visual2.pdf}
	\caption{Noisy images with different $\sigma_\zeta$.}
	\label{fig:noisy_img_var_sigma}
	\vspace{-1em}
\end{figure}


We evaluate image quality using the Frechet Inception Distance (FID), computed between the reference dataset and 50,000 images generated by the models. Generated samples for FID computation are presented in \cref{appx:sample_results}.


\textbf{Models and other configurations.} We implemented SFBD algorithms using the architectures proposed in EDM \citep{KarrasAAL22} as well as the optimizers and hyperparameter configurations therein. All models are implemented in an unconditional setting, and we also enabled the non-leaky augmentation technique \citep{KarrasAAL22} to alleviate the overfitting problem. For the backward sampling step in SFBD, we adopt the $2^\text{nd}$-order Heun method \citep{KarrasAAL22}. More information is provided in \cref{appx:expConfig}. 


\subsection{Performance Comparison}
We compare SFBD with several representative models designed for training on noisy images, as shown in \cref{tb:performance_compare}. SURE-Score \citep{AaliAKT2023} and EMDiffusion \citep{BaiWCS2024} are diffusion-based frameworks that utilize Stein’s unbiased risk estimate and the expectation-maximization (EM) method, respectively, to address general inverse problems. AmbientDiffusion \citep{DarasSDGDK2023} solves similar problems by modifying the denoising loss \eqref{eq:denoiser_loss}, requiring a denoiser to restore further corrupted samples. TweedieDiffusion \citep{DarasDD2024} implements the original consistency loss in \cref{eq:consistency_loss}.

Following the experimental setup of \citet{BaiWCS2024}, images are corrupted by adding independent Gaussian noise with a standard deviation of  $\sigma_\zeta = 0.2$  to each pixel after rescaling pixel values to $[-1, 1]$. For reference, we also include results for models trained on clean images ($\sigma_\zeta = 0$). In cases with pretraining, the models are initially trained on 50 clean images randomly sampled from the training datasets. For all results presented in this work, the same set of 50 sampled images is used.


As shown in \cref{tb:performance_compare}, SFBD produces images of significantly higher quality than all baselines, as further illustrated by the denoised images in \cref{fig:model_benchmark_visual} by evaluating the backward SDE starting from a noisy image in the training dataset. Notably, on CelebA, SFBD achieves performance comparable to DDIM, which is trained on clean images. \cref{tb:performance_compare} also shows that while TweedieDiffusion benefits from pretraining, its results remain inferior to SFBD. In fact, we observe that the original consistency loss  \eqref{eq:consistency_loss} provides limited performance improvement after pretraining; the FID begins to degrade shortly after the consistency loss is applied.
\begin{figure*}[t!]
	\centering
	\includegraphics[width=0.5\columnwidth]{figures/charts/var_config_celeba.png}~~~~
	\raisebox{-0.1em}{\includegraphics[width=1.4\columnwidth]{figures/celeba_deconv_iters2.pdf}}
	\caption{(Left) SFBD performance on CelebA under three configurations, with FID at iteration 0 for the pretrained model. (Right) Denoised samples generated by the backward SDE, starting from a noisy image in the training dataset. For cfg A, results are shown after each fine-tuning iteration, while cfg B and cfg C are shown at their minimum FID iterations.}
	\label{fig:var_config_celeba}
\end{figure*}

\subsection{Ablation Study}
\label{sec:exp:ablation}
In this section, we investigate how SFBD's performance varies with clean image ratios, noise levels, and pretraining on similar datasets. The results align with our discussion in \cref{sec:deconv} and \cref{sec:SFBD} and provide practical insights. Experiments are conducted on CIFAR-10, with the default $\sigma_\zeta = 0.59$. This noise level significantly alters the original images, aligning with our original motivation to address potential copyright concerns (see \cref{fig:noisy_img_var_sigma}). 

\textbf{Clean image ratio.} \cref{fig:ablation_cifar10:clean_ratio} shows the FID trajectories across fine-tuning iterations $k$ for different clean image ratios. With just 4\% clean images, SFBD achieves strong performance (FID: 6.31) and outperforms DDIM with 10\% clean images. While higher clean image ratios further improve performance, the gains diminish as a small amount of clean data already provides sufficient high-frequency features (e.g., edges and local details) to capture feature variations. Since these features are shared across images, additional clean data offers limited improvement.


These findings suggest that practitioners with limited clean datasets should focus on collecting more copyright-free data to enhance performance. Notably, when clean images are scarce, the marginal gains from additional fine-tuning iterations $k$ are greater than when more clean data is available. Therefore, in scenarios where acquiring clean data is challenging, increasing fine-tuning iterations can be an effective alternative to improve results.
\vspace{-0.3em}

\textbf{Noise level.} \cref{fig:ablation_cifar10:sigma} shows SFBD's sampling performance across fine-tuning iterations for different noise levels, using the values from $2^\text{nd}$ order Heun sampling in EDM \citep{KarrasAAL22}. The impact of noise on the original images is visualized in \cref{fig:noisy_img_var_sigma}. As shown in \cref{fig:ablation_cifar10:sigma}, increasing $\sigma_\zeta$ significantly degrades SFBD's performance. This is expected, as higher noise levels obscure more features in the original images. Furthermore, as suggested by \cref{thm:MISE_upper_bound}, higher $\sigma_\zeta$ demands substantially more noisy images, which cannot be compensated by pretraining on a small clean image set. Importantly, this performance drop is a mathematical limitation discussed in \cref{sec:deconv}, rather than an issue solvable by better deconvolution algorithms. In \cref{sec:exp:futher_discuss}, we show that slightly increasing pretraining clean image set can yield strong results, even at reasonably high noise levels on CelebA.

\textbf{Pretraining with clean images from similar datasets.} \cref{fig:ablation_cifar10:pretrain} evaluates SFBD's performance when fine-tuning on image sets from different classes, with the model initially pretrained on clean truck images. The results show that the closer the noisy dataset is to the truck dataset (as indicated by the FID at iter 0), the better the model performs after fine-tuning. This is expected, as similar datasets share common features that facilitate learning the target data distribution.
Interestingly, even when the pretraining dataset differs significantly from the noisy dataset, the model still outperforms the version without pretraining. This is because unrelated datasets often share fundamental features, such as edges and local structures.  \textit{Therefore, practitioners should always consider pretraining before fine-tuning on target noisy datasets, while more similar pretraining datasets yield better final sampling performance.}





\subsection{Further Discussions}

\label{sec:exp:futher_discuss}
\textbf{Additional results on CelebA.} \cref{fig:var_config_celeba} presents SFBD performance trajectories on CelebA under three configurations. While \cref{tb:performance_compare} reports results using configuration (cfg) C to align with benchmarks, this setup is impractical due to its low noise level, which fails to address copyright and privacy concerns. As illustrated in \cref{fig:var_config_celeba} (right), the low noise level allows human observers to identify individuals and recover image details, with model-denoised images nearly identical to the originals. To address this, we report results for cfgs A and B with $\sigma_\zeta = 1.38$, concealing most original image information. While pretraining on 50 clean images performs poorly, increasing the size to 1.5k (still $< 1\%$ of the training dataset) achieves impressive results. At iteration 3, the model reaches FID 5.91, outperforming DDIM trained on clean images. This supports our discussion in \cref{sec:exp:ablation}: collecting more clean data significantly boosts performance when the clean dataset is small.

\textbf{Features learned from noisy images.} As shown in \cref{fig:var_config_celeba}, when $\sigma_\zeta = 1.38$, almost all information from the original images is obscured, prompting the question: can the model learn from such noisy inputs, and how does this happen? In \cref{fig:var_config_celeba}, we plot the model’s denoised outputs in cfg A after each fine-tuning iteration. These outputs serve as samples for the next iteration, revealing what the model learns and adapts to in the process. For the first row, the pretrained model (iter 0) produces a face very different from the original, failing to recover features like a headband. This occurs because the clean dataset for pretraining lacks similar faces with headbands. Instead of random guesses, the model combines local features (e.g., face shapes, eyes) learned from the clean data with the global structure from the noisy images. This process combines previously learned features in new ways, helps the model better generalize, and gradually improves its ability to approximate the true distribution, as supported by \cref{prop:conv_SFBD}. Likewise, the model learns how to attach the glasses to the individual's face in the second row. 



\section{Conclusion}
\label{sec:disc}
\section{Discussion and Future Work} \label{sec:disc}

This paper introduces the Flexible Bivariate Beta Mixture Model (FBBMM), a novel probabilistic clustering model leveraging the flexibility of the bivariate beta distribution. Experimental results show that FBBMM outperforms popular clustering algorithms such as $k$-means, MeanShift, DBSCAN, Gaussian Mixture Models, and MBMM, particularly on nonconvex clusters. Its ability to handle a wide range of cluster shapes and correlations makes it highly effective.

FBBMM offers several advantages. Its use of the beta distribution allows for flexible cluster shapes, capturing complex structures more accurately than traditional models. It supports soft clustering, assigning probabilities to data points for belonging to clusters, which is versatile for overlapping clusters. Additionally, FBBMM is generative, capable of producing new data resembling the original dataset, useful for tasks like data augmentation and simulation.

However, FBBMM has limitations, including higher computational complexity due to iterative parameter estimation. Future work could focus on improving efficiency through parallelization or better optimization strategies, extending FBBMM to multivariate data, and enhancing robustness to noise and outliers. Applying FBBMM in diverse domains such as bioinformatics and image analysis could further validate its versatility and impact.




%\clearpage 
\section*{Impact Statement}
This paper introduces SFBD, a framework for effectively training diffusion models primarily using noisy samples. Our approach enables data sharing for generative model training while safeguarding sensitive information.

For organizations utilizing personal or copyrighted data to train their models, SFBD offers a practical solution to mitigate copyright concerns, as the model never directly accesses the original samples. This mathematically guaranteed framework can promote data-sharing by providing a secure and privacy-preserving training method.

However, improper implementation poses a risk of sensitive information leakage. A false sense of security could further exacerbate this issue, underscoring the importance of rigorous validation and responsible deployment.


\bibliography{example_paper}
\bibliographystyle{icml2025}


%%%%%%%%%%%%%%%%%%%%%%%%%%%%%%%%%%%%%%%%%%%%%%%%%%%%%%%%%%%%%%%%%%%%%%%%%%%%%%%
%%%%%%%%%%%%%%%%%%%%%%%%%%%%%%%%%%%%%%%%%%%%%%%%%%%%%%%%%%%%%%%%%%%%%%%%%%%%%%%
% APPENDIX
%%%%%%%%%%%%%%%%%%%%%%%%%%%%%%%%%%%%%%%%%%%%%%%%%%%%%%%%%%%%%%%%%%%%%%%%%%%%%%%
%%%%%%%%%%%%%%%%%%%%%%%%%%%%%%%%%%%%%%%%%%%%%%%%%%%%%%%%%%%%%%%%%%%%%%%%%%%%%%%
\newpage
\appendix
\onecolumn

\section{A Brief Introduction to the Density Convolutions}
\label{appx:intro:density_deconv}
In this section, we give a brief discussion on the density convolution and how it is related to our problem. 

For simplicity, we stick to the case when $d = 1$. Consider the data generation process in \cref{eq:gen_conv_samples}. Let $p_y$ denote the density of the distribution of the noisy samples $y^{(i)}$. Then we have 
\begin{fact}
\label{fact:conv_density}
For $\omega \in \Rb$, 
\begin{align}
	p_y(\omega) = \int p_\text{data}(x) \; h(\omega-x) \diff x = (p_\text{data} \conv h) (\omega).
\end{align} 
\end{fact}
\begin{proof}
	This is because, for all measurable function $\psi$, we have
\begin{align*}
	\int \psi(\omega) p_y(\omega) \diff \omega &= \int \int \psi(x+\epsilon) \; p_\text{data}(x) h(\epsilon) \diff x \diff \epsilon = \int \int \psi(\omega) p_\text{data}(x) h(\omega-x) \diff x \diff w\\
	&= \int \psi(w) \left[\int p_\text{data}(x) \; h(\omega - x) \diff x \right]\diff \omega. 
\end{align*}
As the equality holds for all $\psi$, we have
\(
	p_y(\omega) = \int p_\text{data}(x) \; h(\omega-x) \diff x = (p_\text{data} \conv h) (\omega).
\)
\end{proof}
As a result, according to \cref{fact:conv_density}, the density convolution is naturally involved in our setting.

Then, we provide an alternative way to show why we can recover $p_\text{data}$ given $p_y$ and $h$. (Namely, we need to deconvolute $p_y$ to obtain  $p_\text{data}$.) Our discussion can be seen a complement of the discussion following  \cref{prop:conv_identify}. Let $\phi_p$ denote the characteristic function of the random variable with distribution $p$ such that 
\begin{align}
    \phi_p(t) = \int \exp(i t \omega) \; p(\omega) \diff \omega.
\end{align}
We note that the characteristic function of a density $p$ is its Fourier transform. As a result, through the dual relationship of multiplication and convolution under Fourier transformation \citep[Lemma A.5]{Meister2009}, we have 
\begin{align}
    \phi_{p_y}(t) = \phi_{p_\text{data}}(t) \; \phi_h (t).
\end{align}
As a result, given noisy data distribution $p_y$ and noise distribution $h$, we have
\begin{align}
    \phi_{p_\text{data}} (t) = \frac{\phi_{p_y}(t)}{\phi_h(t)}. 
\end{align}
Finally, we can recover $p_\text{data}$ through an inverse Fourier transform:
\begin{align}
    p_\text{data}(x) = (2\pi)^{-1} \int \exp(-itx) \; \phi_{p_\text{data}} (t) \diff t = (2\pi)^{-1} \int \exp(-itx) \; \frac{\phi_{p_y}(t)}{\phi_h(t)} \diff t.
\end{align}
We conclude this section by summarizing the relationship between data and noisy sample distributions in \cref{fig:diagram_dist_conv}.
\begin{figure}[h]
    \centering
    \includegraphics[width=0.33\linewidth]{figures/diagram_dist_conv.pdf}
    \caption{While the corruption process is irreversible at the sample level, a bijective relationship exists between the clean and noisy data distributions.}
    \label{fig:diagram_dist_conv}
\end{figure}

\section{Proofs Related to Deconvolution Theory}
\label{appx:proof:deconv}
We first show the result suggesting it is possible to identify a distribution through its noisy version obtained by corrupting its samples by injecting independent Gaussian noises. 
\PROPCONVINDENTIFY*

\begin{lemma}
\label{appx:lem:kl_bd_char}
	Given two distributions $p$ and $q$ on $\Rb^d$. Let $\Phi_p(\uv)$ and $\Phi_q(\uv)$ be their characteristic functions. Then for all $\uv \in \Rb^d$, we have
	\begin{align}
			\bigl|\Phi_p(\uv) - \Phi_q(\uv)\bigr|
			\;\le\;
			\sqrt{2 \,D_{\mathrm{KL}}(p\;\|\;q)}.
	\end{align}
\end{lemma}


\begin{proof}
We note that 
\[
\Phi_p(\uv) \;=\; \mathbb{E}_p[\exp(i\uv^\top \xv)],
\quad
\Phi_q(\uv) \;=\; \mathbb{E}_q[\exp(i\uv^\top \xv)].
\]
Then for any $\uv \in \Rb^d$, we have
\begin{align*}
\bigl|\Phi_p(\uv) - \Phi_q(\uv)\bigr|
& \le
\left|\int_{\Rb^d} \exp(i\uv^\top\xv) p(\xv) \diff \xv - \int_{\Rb^d} \exp(i\uv^\top\xv) q(\xv) \diff \xv \right| \\
& = \left|\int_{\Rb^d} \exp(i\uv^\top\xv) \Big(p(\xv) -  q(\xv) \Big) \diff \xv \right|  \leq \int_{\Rb^d} \underbrace{\left|\exp(i\uv^\top\xv)\right|}_{=1}  \left| p(\xv)  -  q(\xv) \right| \diff \xv \\
& = \int_{\Rb^d}  \left| p(\xv)  -  q(\xv) \right| \diff \xv \\
& = 2 \; \|p - q\|_{\mathrm{TV}},
\end{align*}
where the last equality is due to Scheffe's theorem \citep[Lemma 2.1, p. 84]{Tsybakov09}). 


Then, by Pinsker's inequality \citep[Lemma 2.5, p. 88]{Tsybakov09}, we have 
\[
\bigl|\Phi_p(\uv) - \Phi_q(\uv)\bigr|
\;\le\;
2 \; \|p - q\|_{\mathrm{TV}}
\;\le\;
\sqrt{2\,D_{\mathrm{KL}}(P\;\|\;Q)}.
\]
which completes the proof. 
\end{proof}
\begin{proof}[Proof of \cref{prop:conv_identify}]
	Note that, by the convolution theorem \citep[A.4]{Meister2009},  for all $\uv \in \Rb^d$, we have 
	\begin{align*}
		\Phi_{p * h}(\uv) = \Phi_{p}(\uv) \; \Phi_{h}(\uv) = \Phi_{p}(\uv) \; \exp \big(-\frac{\sigma_\zeta^2}{2} \|\uv\|^2 \big),
	\end{align*}
	as $h\sim \Nc(\mathbf{0}, \sigma_\zeta^2\mathbf{I})$ having $\Phi_h(\uv) = \exp \big(-\frac{\sigma_\zeta^2}{2} \|\uv\|^2 \big)$. Applying \cref{appx:lem:kl_bd_char}, we have
	\begin{align}
		\exp \big(-\frac{\sigma_\zeta^2}{2}\|\uv\|^2\big) \; \Big|\Phi_{p}(\uv) - \Phi_{q}(\uv)\Big|  = \big|\Phi_{p * h}(\uv) - \Phi_{q * h}(\uv) \big| \leq \sqrt{2\,D_{\mathrm{KL}}(p \conv h\|q\conv h)}. 
	\end{align}	
	Rearranging the inequality completes the proof. 
\end{proof}


We then derive the proofs regarding the sample complexity of the deconvolution problem. 
\MISEUpperBdound*
\begin{proof}
    The result is constructed based on the work by \citet{StefanskiC1990}. In particular, assuming that $p_\text{data}$ is continuous, bounded and has two bounded integrable derivatives such that 
    \begin{align}
        \int p''_\text{data} (x) \diff x < \infty,
    \end{align}
    we can construct a kernel based estimator of $p_\text{data}$ of rate
    \begin{align}
        \frac{\lambda^4}{4} \mu^2_{K, 2} \int p''_\text{data} (x) \diff x,
    \end{align}
    where $\mu^2_{\kappa, 2}$ is a constant determined by the selected kernel $\kappa$ and $\lambda$ is a function of number of samples $n$ gradually decreasing to zero as $n \rightarrow \infty$. It is required that $\lambda$ satisfies
    \begin{align}
        \frac{1}{2 \pi n \lambda} \exp( \frac{B^2 \sigma_\zeta^2}{\lambda^2}) \rightarrow 0 \label{appx:eq:mise_upper_bound:1}
    \end{align}
    as $n \rightarrow \infty$, where $B > 0$ is a constant depending on the picked kernel $\kappa$. Here, we assume we picked a kernel with $B < 1$.    
    
    To satisfy the constraint, we choose $\lambda (n) = \frac{\sigma_\zeta}{\sqrt{\log n}}$. Plugging it into \cref{appx:eq:mise_upper_bound:1}, we have 
    \begin{align}
        \lim_{n \rightarrow \infty} \frac{1}{n \lambda} \exp( \frac{B^2 \sigma_\zeta^2}{\lambda^2}) = \lim_{n \rightarrow \infty} \frac{\sqrt{\log n}}{n {\sigma_\zeta }} \exp{(B^2 \log n)} = \lim_{n \rightarrow \infty} \frac{\sqrt{\log n}}{n^{1-B^2} {\sigma_\zeta }}. 
    \end{align}
    To show $\lim_{n \rightarrow \infty} \frac{\sqrt{\log n}}{n^{1-B^2} {\sigma_\zeta }} = 0$, it suffices to show 
    $
        \lim_{n \rightarrow \infty} \frac{\log n}{n^{2 - 2 B^2} {\sigma_\zeta^2 }} = 0
    $. By L'Hopital's rule, we have
    \begin{align}
        \lim_{n \rightarrow \infty} \frac{\log n}{n^{2 - 2 B^2} {\sigma_\zeta^2 }} =  \lim_{n \rightarrow \infty} \frac{1}{(2 - 2 B^2) n^{2 - 2 B^2} {\sigma_\zeta^2 }} = 0
    \end{align}
    As a result, $\lambda (n) = \frac{\sigma_\zeta}{\sqrt{\log n}}$ is a valid choice, which gives the convergence rate
    $\frac{\sigma_\zeta^4}{(\log n)^2}$.
\end{proof}

\MISELowerBound*
\begin{proof}
    This result is a special case of Theorem 2.14 (b) in \citep{Meister2009}. When the error density is Gaussian, we have $\gamma = 2$. In addition, in the proof of \cref{thm:MISE_upper_bound}, we assumed that $p_\text{data}$ has two bounded integrable derivatives, which equivalently assumes $p_\text{data}$ satisfies the Soblev condition with smoothness degree $\beta = 2$ (see Eq. A.8,  \citealt{Meister2009}). Then the theorem shows $\mathrm{MISE}(\hat{p}, p_\textrm{data}) \geq \textrm{const} \cdot (\log n)^{-2\beta / \gamma} = \textrm{const} \cdot (\log n)^{-2}$. 
\end{proof}







\section{Proofs Related to the Results of SFBD}
\label{appx:proof:SFBD}
\def\bwdM{\overleftarrow{Q}_{0:\zeta}}
\def\fwdM{\overrightarrow{P}_{0:\zeta}}
\def\argmin{\textrm{argmin}}

We first prove \cref{prop:conv_SFBD}, which we restate below:
\convSFBD*

To facilitate our discussions, let
\begin{itemize}
	\item $\bwdM^{\phiv_{k-1}}$: the path measure induced by the backward process \cref{eq:conv_SFBD:bwd}. In general, we use $\bwdM^{\phiv}$ to denote the path measure when the drift term is parameterized $\phiv$. 
	\item $\fwdM^{(k)}$: the path measure induced by the forward process \cref{eq:fwd_diff} with $p_0 = p_0^{(k)}$, defined in \cref{prop:conv_SFBD}. The density of its marginal distribution at time $t$ is denoted by $p_t^{(k)}$
	\item $\fwdM^*$: the path measure induced by the forward process \cref{eq:fwd_diff} with $p_0 = p_\text{data}$. 
\end{itemize}
We note that, according to \cref{alg:SFBD}, the marginal distribution of $\bwdM^{\phiv_{k-1}}$ at $t = 0$ has density $p_0^{(k)}$. 

The following lemma allows us to show that the training of the diffusion model can be seen as a process of minimizing the KL divergence of two path measures. 
\begin{lemma}[\citealt{PavonA1991}, \citealt{VargasTLL2021}]
\label{appx:lem:kl_two_path}
	Given two SDEs:
	\begin{align}
		\diff \xv_t = \fv_i(\xv_t, t) \diff t + g(t) \diff \wv_t,~~~\xv_0\sim p_0^{(i)}(\xv)~~~~ t \in [0,T]
	\end{align}
	for $i = 1, 2$. Let $P^{(i)}_{0:T}$, for $i = 1, 2$, be the path measure induced by them, respectively. Then we have,
	\begin{align}
		\KL{P^{(1)}_{0:T}}{P^{(2)}_{0:T}} = \KL{p_0^{(1)}}{p_0^{(2)}} \; + \; \Eb_{P^{(1)}_{0:T}}\Bigl[\int_0^T \frac{1}{2\,g(t)^2}\,\|\fv_1(\xv_t,t)-\fv_2(\xv_t,t)\|^2\,dt\Bigr]. 
	\end{align}
	In addition, the same result applies to a pair of backward SDEs as well, where $p_0^{(i)}$ is replaced with $p_T^{(i)}$. 
\end{lemma}
\begin{proof}
	By the disintegration theorem~(e.g., see \citealt[Appx B]{VargasTLL2021}), we have 
	 \begin{align}
	 	\KL{P_1}{P_2} = \KL{p_0^{(1)}}{p_0^{(2)}} \; + \; \Eb_{P^{(1)}_{0:T}}\left[\log \frac{\diff P^{(1)}_{0:T}(\cdot | \xv_0))}{\diff P^{(2)}_{0:T}(\cdot | \xv_0)} \right],
	 \end{align}
	 where $P^{(i)}_{0:T}(\cdot |\xv_0)$ is the conditioned path measure of $P^{(i)}_{0:T}$ given the initial point $\xv_0$. Then, applying the Girsanov theorem \citep{Kailath1971, Oksendal2003} on the second term yields the desired result. 
\end{proof}
By \cref{appx:lem:kl_two_path}, we can show that the Denoiser Update step in \cref{alg:SFBD} finds $\phiv_{k}$ minimizing $\KL{\fwdM^{(k)}}{\bwdM^{\phiv}}$. To see this, note that
\begin{align*}
	\phiv_{k} &= \underset{\phiv}{\argmin} ~ \KL{\fwdM^{(k)}}{\bwdM^{\phiv}} \\
	&= \underset{\phiv}{\argmin} ~ \KL{p_\zeta^{(k)}}{p_\zeta^*} \; + \; \Eb_{\fwdM^{(k)}}\Bigl[\int_0^\zeta \frac{g(t)^2}{2}\,\| \nabla \log p_t^{(k)}(\xv_t)- \sv_{\phiv}(\xv_t,t)\|^2\,dt\Bigr],\numberthis \label{appx:eq:min_KL_update_denoiser}
\end{align*}
where $p_t^{(k)}$ is the marginal distribution induced by the forward process \eqref{eq:fwd_diff} with the boundary condition $p^{(k)}_0$ at $t = 0$. Note that, we have applied \cref{appx:lem:kl_two_path} to the backward processes inducing $\fwdM^{(k)}$ and $\bwdM^{\phiv}$. Thus, the drift term of $\fwdM^{(k)}$ is not zero but $-g(t)^2 \, \nabla \log p_t^{(k)}(\xv_t)$ according to \cref{eq:anderson_bwd}. Since the first term of \cref{appx:eq:min_KL_update_denoiser} is a constant, the minimization results in 
\begin{align}
	\nabla \log p_t^{(k)} (\xv_t) = \sv_{{\phiv}_k}(\xv_t, t)\label{eq:appx:score_minimize_KL}
\end{align}
for all $\xv_t \in \Rb^d$ and $t \in (0,\zeta]$. In addition, we note that, the denoising loss in \cref{eq:denoiser_loss} is minimized when $\nabla \log p_t^{(k)} (\xv_t) = \sv_{\phiv}(\xv_t, t)$ for all $t > 0$; as a result, $\phiv_{k}$ minimizes $\KL{\fwdM^{(k)}}{\bwdM^{\phiv}}$ as claimed. 

Now, we are ready to prove \cref{prop:conv_SFBD}. 
\begin{proof}[Proof of \cref{prop:conv_SFBD}]


Applying \cref{appx:lem:kl_two_path} to the backward process 
\begin{align*}
	\KL{\fwdM^*}{\bwdM^{\phiv_{k-1}}} &= \underbrace{\KL{p^*_\zeta}{p^*_\zeta}}_{=0} + \Eb_{\fwdM^*}\left[\int_0^\zeta \tfrac{g(t)^2}{2} \|\nabla\log p_t^*(\xv_0) - \sv_{{\phiv}_{k-1}}(\xv_t, t) \|^2 \diff t \right]\\
	&= \Eb_{\fwdM^*}\left[\int_0^\zeta \tfrac{g(t)^2}{2} \|\nabla\log p_t^*(\xv_0) - \sv_{{\phiv}_{k-1}}(\xv_t, t) \|^2 \diff t\right] \numberthis \label{eq:conv_SFBD:proof:1}
\end{align*}

Likewise,
\begin{align*}
	\KL{\fwdM^*}{\fwdM^{(k)}} =~& \KL{p^*_\zeta}{p_\zeta^{(k)}} + \Eb_{\fwdM^*}\left[\int_0^\zeta \tfrac{g(t)^2}{2} \|\nabla \log q_t^*(\xv_t) - \nabla\log p^{(k)}_t(\xv_t) \|^2 \diff t\right] \\
	=~&\KL{p^*_\zeta}{p_\zeta^{(k)}} + \Eb_{\fwdM^*}\left[\int_0^\zeta \tfrac{g(t)^2}{2} \|\nabla \log q_t^*(\xv_t) - \sv_{{\phiv}_k}(\xv_t, t) \|^2 \diff t\right] \\
	\overset{\eqref{eq:conv_SFBD:proof:1}}{=}~& \KL{p^*_\zeta}{p_\zeta^{(k)}} + \KL{\fwdM^*}{\bwdM^{\phiv_{k}}} \numberthis\label{eq:conv_SFBD:proof:2}
\end{align*} 
where the second equality is due to the discussion on deriving \cref{eq:appx:score_minimize_KL}. 


\cref{appx:lem:kl_two_path} also implies that 
\begin{align}
	\KL{\fwdM^*}{\bwdM^{\phiv_{k-1}}} = \KL{p_\text{data}}{p_0^{(k)}} + \underbrace{\Eb_{\fwdM^*}\left[\int_0^\zeta \tfrac{1}{2} \|\bv^{(k-1)}(\xv_t, t) \|^2 \diff t \right]}_{:= \Bc_{k-1}}, \label{eq:conv_SFBD:proof:2_1}
\end{align}
where $\bv^{(k-1)}(\xv_t, t)$ is the drift of the forward process inducing $\bwdM^{\phiv_{k-1}}$.
In addition, 
\begin{align}
	\KL{\fwdM^*}{\fwdM^{(k)}} = \KL{p_\text{data}}{p_0^{(k)}} + \Eb_{\fwdM^*}\left[\int_0^\zeta \tfrac{1}{2} \|\zero - \zero \|^2 \diff t\right] = \KL{p_\text{data}}{p_0^{(k)}}. \label{eq:conv_SFBD:proof:3}
\end{align}
As a result, 
\begin{align*}
	\KL{p_\text{data}}{p_0^{(k)}}
	\overset{\eqref{eq:conv_SFBD:proof:3}}{=}& \KL{\fwdM^*}{\fwdM^{(k)}} 	
	\overset{\eqref{eq:conv_SFBD:proof:2}}{=} \KL{p^*_\zeta}{p_\zeta^{(k)}} + \KL{\fwdM^*}{\bwdM^{\phiv_{k}}} \\
	\geq & ~ \KL{\fwdM^*}{\bwdM^{\phiv_{k}}}
	\overset{\eqref{eq:conv_SFBD:proof:2_1}}{=} \KL{p_\text{data}}{p_0^{(k+1)}} + \Bc_k \\
	\geq & \KL{p_\text{data}}{p_0^{(k+1)}}
\end{align*}
which is \eqref{eq:conv_SFBD:monotone}. In addition, we have
\begin{align*}
	\KL{\fwdM^*}{\bwdM^{\phiv_{k-1}}} 
	&\overset{\eqref{eq:conv_SFBD:proof:2_1}}{=} \KL{p_\text{data}}{p_0^{(k)}} + \Bc_{k-1}
	\overset{\eqref{eq:conv_SFBD:proof:3}}{=} \KL{\fwdM^*}{\fwdM^{(k)}} + \Bc_{k-1} \\ 
	&\overset{\eqref{eq:conv_SFBD:proof:2}}{=} \KL{p^*_\zeta}{p_\zeta^{(k)}} + \KL{\fwdM^*}{\bwdM^{\phiv_{k}}}  + \Bc_{k-1} \\
	&= \KL{\fwdM^*}{\bwdM^{\phiv_{k}}}  + \big[\KL{p^*_\zeta}{p_\zeta^{(k)}} +  \Bc_{k-1}\big].                                                                                                                                               
\end{align*}	


As a result, applying this relationship recursively, we have
\begin{align}
	\KL{\fwdM^*}{\bwdM^{\phiv_{0}}} = \sum_{k = 1}^K \KL{p^*_\zeta}{p_\zeta^{(k)}} + \sum_{k=1}^{K} \Bc_{k-1} + \KL{\fwdM^*}{\bwdM^{\phiv_{K}}}. 
\end{align}
Since $\KL{\fwdM^*}{\bwdM^{\phiv_{0}}} = M_0$, we have
\begin{align}
	\sum_{k = 1}^K \KL{p_\text{data} * h}{p^{(k)} * h} = \sum_{k = 1}^K \KL{p^*_\zeta}{p_\zeta^{(k)}} \leq M_0,
\end{align}
for all $K \geq 1$. This further implies, 
\begin{align}
	\min_{k \in \{1, 2, \ldots, K\}}\KL{p_\text{data} * h}{p^{(k)} * h} \leq \frac{M_0}{K}. 
\end{align}


Applying \cref{prop:conv_identify}, we obtain,
\begin{align}
			\min_{k \in \{1, 2, \ldots, K\}} \left|\Phi_{p_\text{data}}(\uv) - \Phi_{p_0^{(k)}}(\uv)\right|\leq \exp \big(\frac{\sigma_\zeta^2}{2} \|\uv\|^2 \big) \sqrt{\frac{2 M_0}{K}}.
\end{align}
\end{proof}

We complete this section by showing the connection between our framework and the original consistency loss. 
\RELCONSSFBD*
\begin{proof}
	When $t = s$, denoising noise in \cref{eq:denoiser_loss} becomes	
	\begin{align*}
	 &\underset{p_0}{\Eb} \,  \underset{p_{s|0}}{\Eb} \left[\|D_{\phiv}(\xv_s, s) - \xv_0 \|^2 \right] = \Eb_{p_s} \Eb_{p_{0|s}}\big[ \|D_{\phiv}(\xv_s, s) - \xv_0 \|^2  \big] \\
	 =\, & \Eb_{p_s} \Eb_{p_{0|s}}\big[ \|D_{\phiv}(\xv_s, s) - \Eb_{p_{0|s}}[\xv_0] + \Eb_{p_{0|s}}[\xv_0] - \xv_0 \|^2  \big] \\
	 =\, & \Eb_{p_s} \Eb_{p_{0|s}}\big[\|D_{\phiv}(\xv_s, s) - \Eb_{p_{0|s}}[\xv_0]  \|^2 \big] + \underbrace{\Eb_{p_s} \Eb_{p_{0|s}}\big[\|\Eb_{p_{0|s}}[\xv_0]  - \xv_0\|^2 \big]}_{\text{Const.}}  \\
	 &\hspace{15em} + 2 \underbrace{\Eb_{p_s} \Eb_{p_{0|s}}\big[\inner{D_{\phiv}(\xv_s, s) - \Eb_{p_{0|s}}[\xv_0]}{\Eb_{p_{0|s}}[\xv_0] - \xv_0} \big]}_{=0} \\
	 =\, & \Eb_{p_s}\big[\|D_{\phiv}(\xv_s, s) - \Eb_{p_{0|s}}[\xv_0]  \|^2 \big] + \textrm{Const.} \\
	 =\, & \Eb_{p_s} \big[\|D_{\phiv}(\xv_s, s) - \Eb_{p_{0|s}}[D_{\phiv}(\xv_0, 0)]  \|^2 \big] + \textrm{Const.},
	 \end{align*}
	 which is the consistency loss in \cref{eq:consistency_loss} when $r = 0$. 
\end{proof}
 











\newpage
\section{Additional Sampling Results}
\label{appx:sample_results}
In this section, we present model-generated samples used for FID computation in \cref{sec:emp}. The samples are taken from the models at their fine-tuning iteration with the lowest FID.

\subsection{CIFAR-10}
\textbf{Samples for computing FIDs in \cref{fig:ablation_cifar10:clean_ratio} - Clean Image Ratio}

\FloatBarrier

\begin{figure}[h!]
    \centering
    \includegraphics[width=0.6\linewidth]{figures/samples/cifar10/ratio/0_1.png}
    \caption{Clean image ratio = 0.04 -- FID: 6.31}
\end{figure}

\begin{figure}[h!]
    \centering
    \includegraphics[width=0.6\linewidth]{figures/samples/cifar10/ratio/0_4.png}
    \caption{Clean image ratio = 0.1  -- FID: 3.58 }
\end{figure}

\begin{figure}[h!]
    \centering
    \includegraphics[width=0.6\linewidth]{figures/samples/cifar10/ratio/0_9.png}
    \caption{Clean image ratio = 0.2 -- FID: 2.98}
\end{figure}



\textbf{Samples for computing FIDs in \cref{fig:ablation_cifar10:sigma} - Noise Level}


\begin{figure}[h!]
    \centering
    \includegraphics[width=0.6\linewidth]{figures/samples/cifar10/nl/0_3.png}
    \caption{Noise level $\sigma_\zeta = 0.30$ -- FID: 3.97}
\end{figure}

\begin{figure}[h!]
    \centering
    \includegraphics[width=0.6\linewidth]{figures/samples/cifar10/nl/0_59.png}
    \caption{Noise level $\sigma_\zeta = 0.59 $ -- FID: 6.31 }
\end{figure}

\begin{figure}[h!]
    \centering
    \includegraphics[width=0.6\linewidth]{figures/samples/cifar10/nl/1_09.png}
    \caption{Noise level $\sigma_\zeta = $ 1.09 -- FID: 9.43}
\end{figure}


\begin{figure}[h!]
    \centering
    \includegraphics[width=0.6\linewidth]{figures/samples/cifar10/nl/1_92.png}
    \caption{Noise level $\sigma_\zeta = $ 1.92 -- FID: 10.91}
\end{figure}

\FloatBarrier

\textbf{Samples for computing FIDs in \cref{fig:ablation_cifar10:pretrain} - Pretraining on Similar Datasets}

\begin{figure}[h!]
    \centering
    \includegraphics[width=0.6\linewidth]{figures/samples/cifar10/transfer/automobile.png}
    \caption{Class for fine-tuning: automobile -- FID: 10.39}
\end{figure}

\begin{figure}[h!]
    \centering
    \includegraphics[width=0.6\linewidth]{figures/samples/cifar10/transfer/ship.png}
    \caption{Class for fine-tuning: ship -- FID: 19.19}
\end{figure}

\begin{figure}[h!]
    \centering
    \includegraphics[width=0.6\linewidth]{figures/samples/cifar10/transfer/horse.png}
    \caption{Class for fine-tuning: horse -- FID: 48.11}
\end{figure}
\newpage\begin{figure}[h!]
    \centering
    \includegraphics[width=0.6\linewidth]{figures/samples/cifar10/transfer/nopretrain.png}
    \caption{Class for fine-tuning: no pretrain -- FID: 155.04}
\end{figure}


\subsection{CelebA}

\begin{figure}[h!]
    \centering
    \includegraphics[width=0.8\linewidth]{figures/samples/celeba/cfga.png}
   \caption{cfg A: $\sigma_\zeta = 1.38$; 1,500 clean images for pretraining -- FID: 5.91}
\end{figure}

\begin{figure}[h!]
    \centering
    \includegraphics[width=0.8\linewidth]{figures/samples/celeba/cfgb.png}
   \caption{cfg B: $\sigma_\zeta = 1.38$; 50 clean images for pretraining -- FID: 23.63}
\end{figure}

\FloatBarrier

\begin{figure}[h!]
    \centering
    \includegraphics[width=0.8\linewidth]{figures/samples/celeba/cfgc.png}
   \caption{cfg C: $\sigma_\zeta = 0.20$; 50 clean images for pretraining -- FID: 6.48}
\end{figure}


\FloatBarrier





\newpage



\section{Experiment Configurations}
\label{appx:expConfig}
\subsection{Model Architectures}
We implemented the proposed SFBD algorithm based on the following configurations throughout our empirical studies:
\begin{table}[h]
    \centering
    \caption{Experimental Configuration for CIFAR-10 and CelebA}
    \label{tab:experiment-config}
    \renewcommand{\arraystretch}{1.2} % Adjust row height for readability
    \setlength{\tabcolsep}{6pt} % Adjust column spacing
    \begin{tabular}{p{4cm} p{4.5cm} p{4.5cm}} 
        \toprule
        \textbf{Parameter} & \textbf{CIFAR-10} & \textbf{CelebA} \\
        \midrule
        \textbf{General} & & \\
        Batch Size & 512 & 256 \\
        Loss Function & \texttt{EDMLoss} \citep{KarrasAAL22} & \texttt{EDMLoss} \citep{KarrasAAL22} \\
        Sampling Method &  $2^\text{nd}$ order Heun method (EDM) \citep{KarrasAAL22} & $2^\text{nd}$ order Heun method (EDM) \citep{KarrasAAL22}  \\
        Sampling steps & 18 & 40 \\
        \midrule 
        \textbf{Network Configuration} & & \\
        Dropout & 0.13 & 0.05 \\
        Channel Multipliers & $\{2, 2, 2\}$ & $\{1, 2, 2, 2\}$ \\
        Model Channels & 128 & 128 \\
        Resample Filter & $\{1, 1\}$ & $\{1, 3, 3, 1\}$ \\
        Channel Mult Noise & 1 & 2 \\
        \midrule
        \textbf{Optimizer Configuration} & & \\
        Optimizer Class & \texttt{Adam} \citep{KingmaBa2014} & \texttt{Adam}  \citep{KingmaBa2014} \\
        Learning Rate & 0.001 & 0.0002 \\
        Epsilon & $1 \times 10^{-8}$ & $1 \times 10^{-8}$ \\
        Betas & (0.9, 0.999) & (0.9, 0.999) \\
        \bottomrule
  \end{tabular}
\end{table}


\subsection{Datasets}
All experiments on CIFAR-10 \citep{Krizhevsky2009} use only the training set, except for the one presented in \cref{fig:ablation_cifar10:pretrain}. For this specific test, we merge the training and test sets so that each class contains a total of 6,000 images. At iteration 0, the FID computation measures the distance between clean images of trucks and those from the classes on which the model is fine-tuned. For subsequent iterations, FID is calculated in the same manner as in other experiments. Specifically, the model first generates 50,000 images, and the FID is computed between the sampled images and the images from the fine-tuning classes. All experiments on CelebA \citep{LiuLWT2015} are performed on its training set.


%%%%%%%%%%%%%%%%%%%%%%%%%%%%%%%%%%%%%%%%%%%%%%%%%%%%%%%%%%%%%%%%%%%%%%%%%%%%%%%
%%%%%%%%%%%%%%%%%%%%%%%%%%%%%%%%%%%%%%%%%%%%%%%%%%%%%%%%%%%%%%%%%%%%%%%%%%%%%%%
\end{document}


% This document was modified from the file originally made available by
% Pat Langley and Andrea Danyluk for ICML-2K. This version was created
% by Iain Murray in 2018, and modified by Alexandre Bouchard in
% 2019 and 2021 and by Csaba Szepesvari, Gang Niu and Sivan Sabato in 2022.
% Modified again in 2023 and 2024 by Sivan Sabato and Jonathan Scarlett.
% Previous contributors include Dan Roy, Lise Getoor and Tobias
% Scheffer, which was slightly modified from the 2010 version by
% Thorsten Joachims & Johannes Fuernkranz, slightly modified from the
% 2009 version by Kiri Wagstaff and Sam Roweis's 2008 version, which is
% slightly modified from Prasad Tadepalli's 2007 version which is a
% lightly changed version of the previous year's version by Andrew
% Moore, which was in turn edited from those of Kristian Kersting and
% Codrina Lauth. Alex Smola contributed to the algorithmic style files.
