In this section, we give a brief discussion on the density convolution and how it is related to our problem. 

For simplicity, we stick to the case when $d = 1$. Consider the data generation process in \cref{eq:gen_conv_samples}. Let $p_y$ denote the density of the distribution of the noisy samples $y^{(i)}$. Then we have 
\begin{fact}
\label{fact:conv_density}
For $\omega \in \Rb$, 
\begin{align}
	p_y(\omega) = \int p_\text{data}(x) \; h(\omega-x) \diff x = (p_\text{data} \conv h) (\omega).
\end{align} 
\end{fact}
\begin{proof}
	This is because, for all measurable function $\psi$, we have
\begin{align*}
	\int \psi(\omega) p_y(\omega) \diff \omega &= \int \int \psi(x+\epsilon) \; p_\text{data}(x) h(\epsilon) \diff x \diff \epsilon = \int \int \psi(\omega) p_\text{data}(x) h(\omega-x) \diff x \diff w\\
	&= \int \psi(w) \left[\int p_\text{data}(x) \; h(\omega - x) \diff x \right]\diff \omega. 
\end{align*}
As the equality holds for all $\psi$, we have
\(
	p_y(\omega) = \int p_\text{data}(x) \; h(\omega-x) \diff x = (p_\text{data} \conv h) (\omega).
\)
\end{proof}
As a result, according to \cref{fact:conv_density}, the density convolution is naturally involved in our setting.

Then, we provide an alternative way to show why we can recover $p_\text{data}$ given $p_y$ and $h$. (Namely, we need to deconvolute $p_y$ to obtain  $p_\text{data}$.) Our discussion can be seen a complement of the discussion following  \cref{prop:conv_identify}. Let $\phi_p$ denote the characteristic function of the random variable with distribution $p$ such that 
\begin{align}
    \phi_p(t) = \int \exp(i t \omega) \; p(\omega) \diff \omega.
\end{align}
We note that the characteristic function of a density $p$ is its Fourier transform. As a result, through the dual relationship of multiplication and convolution under Fourier transformation \citep[Lemma A.5]{Meister2009}, we have 
\begin{align}
    \phi_{p_y}(t) = \phi_{p_\text{data}}(t) \; \phi_h (t).
\end{align}
As a result, given noisy data distribution $p_y$ and noise distribution $h$, we have
\begin{align}
    \phi_{p_\text{data}} (t) = \frac{\phi_{p_y}(t)}{\phi_h(t)}. 
\end{align}
Finally, we can recover $p_\text{data}$ through an inverse Fourier transform:
\begin{align}
    p_\text{data}(x) = (2\pi)^{-1} \int \exp(-itx) \; \phi_{p_\text{data}} (t) \diff t = (2\pi)^{-1} \int \exp(-itx) \; \frac{\phi_{p_y}(t)}{\phi_h(t)} \diff t.
\end{align}
We conclude this section by summarizing the relationship between data and noisy sample distributions in \cref{fig:diagram_dist_conv}.
\begin{figure}[h]
    \centering
    \includegraphics[width=0.33\linewidth]{figures/diagram_dist_conv.pdf}
    \caption{While the corruption process is irreversible at the sample level, a bijective relationship exists between the clean and noisy data distributions.}
    \label{fig:diagram_dist_conv}
\end{figure}