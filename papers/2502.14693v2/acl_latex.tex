% This must be in the first 5 lines to tell arXiv to use pdfLaTeX, which is strongly recommended.
\pdfoutput=1
% In particular, the hyperref package requires pdfLaTeX in order to break URLs across lines.

\documentclass[11pt]{article}

% Change "review" to "final" to generate the final (sometimes called camera-ready) version.
% Change to "preprint" to generate a non-anonymous version with page numbers.
\usepackage[final]{acl}

% Standard package includes
\usepackage{times}
\usepackage{latexsym}
\usepackage{tcolorbox}
\usepackage{url}
\usepackage{listings}
\usepackage{xparse}
\lstdefinestyle{pythonstyle}{
  language=Python,
  basicstyle=\ttfamily\small,
  keywordstyle=\bfseries\color{blue},
  commentstyle=\itshape\color{green!60!black},
  stringstyle=\color{red!70!black},
  showstringspaces=false,
  breaklines=true,
  frame=single
}

\lstdefinestyle{txtfile}{
    basicstyle=\scriptsize\ttfamily,
    backgroundcolor=\color{white},
    frame=single,
    rulecolor=\color{black!30},
    stringstyle=\color{red},
    numbers=left,
    numberstyle=\tiny\color{gray!50},
    numbersep=5pt,
    breaklines=true,
    breakatwhitespace=true,
    showstringspaces=false,
    tabsize=4,
    captionpos=b
}

%  more packages
\usepackage{tabularray}
\usepackage{color}
\usepackage{booktabs}
\usepackage{siunitx}
\usepackage{graphicx}
\usepackage{array}

\usepackage{listings}
\usepackage{xcolor}
\usepackage{algorithm}
\usepackage{algorithmic}
\usepackage{wrapfig}
\usepackage{tabularx}
\usepackage{subcaption}
% \usepackage{authblk}
\usepackage{pifont}% http://ctan.org/pkg/pifont

% For proper rendering and hyphenation of words containing Latin characters (including in bib files)
\usepackage[T1]{fontenc}
% For Vietnamese characters
% \usepackage[T5]{fontenc}
% See https://www.latex-project.org/help/documentation/encguide.pdf for other character sets

% This assumes your files are encoded as UTF8
\usepackage[utf8]{inputenc}
\usepackage{amsmath}

% This is not strictly necessary, and may be commented out,
% but it will improve the layout of the manuscript,
% and will typically save some space.
\usepackage{microtype}

% This is also not strictly necessary, and may be commented out.
% However, it will improve the aesthetics of text in
% the typewriter font.
\usepackage{inconsolata}

%Including images in your LaTeX document requires adding
%additional package(s)
\usepackage{graphicx}

% Added packages
\usepackage{amsfonts}
\usepackage{booktabs}
\usepackage{multirow}

% If the title and author information does not fit in the area allocated, uncomment the following
%
%\setlength\titlebox{<dim>}
%
% and set <dim> to something 5cm or larger.

\title{I-MCTS: Enhancing Agentic AutoML via Introspective Monte Carlo Tree Search}

% Author information can be set in various styles:
% For several authors from the same institution:
% \author{Author 1 \and ... \and Author n \\
%         Address line \\ ... \\ Address line}
% if the names do not fit well on one line use
%         Author 1 \\ {\bf Author 2} \\ ... \\ {\bf Author n} \\
% For authors from different institutions:
% \author{Author 1 \\ Address line \\  ... \\ Address line
%         \And  ... \And
%         Author n \\ Address line \\ ... \\ Address line}
% To start a separate ``row'' of authors use \AND, as in
% \author{Author 1 \\ Address line \\  ... \\ Address line
%         \AND
%         Author 2 \\ Address line \\ ... \\ Address line \And
%         Author 3 \\ Address line \\ ... \\ Address line}

% \author{First Author \\
%   Affiliation / Address line 1 \\
%   Affiliation / Address line 2 \\
%   Affiliation / Address line 3 \\
%   \texttt{email@domain} \\\And
%   Second Author \\
%   Affiliation / Address line 1 \\
%   Affiliation / Address line 2 \\
%   Affiliation / Address line 3 \\
%   \texttt{email@domain} \\}

\author{
 \textbf{Zujie Liang\textsuperscript{1}}\quad
 \textbf{Feng Wei\textsuperscript{1}}\quad
 \textbf{Wujiang Xu\textsuperscript{2}}\quad
 \textbf{Lin Chen\textsuperscript{1}}\quad
 \textbf{Yuxi Qian\textsuperscript{1}}\quad
 \textbf{Xinhui Wu\textsuperscript{1}}
\\
\\
 \textsuperscript{1}Ant Group\;\;\;\;\;\;    
 \textsuperscript{2}Rutgers University
% \\
}

% \author{
%  \textbf{First Author\textsuperscript{1}},
%  \textbf{Second Author\textsuperscript{1,2}},
%  \textbf{Third T. Author\textsuperscript{1}},
%  \textbf{Fourth Author\textsuperscript{1}},
% \\
%  \textbf{Fifth Author\textsuperscript{1,2}},
%  \textbf{Sixth Author\textsuperscript{1}},
%  \textbf{Seventh Author\textsuperscript{1}},
%  \textbf{Eighth Author \textsuperscript{1,2,3,4}},
% \\
%  \textbf{Ninth Author\textsuperscript{1}},
%  \textbf{Tenth Author\textsuperscript{1}},
%  \textbf{Eleventh E. Author\textsuperscript{1,2,3,4,5}},
%  \textbf{Twelfth Author\textsuperscript{1}},
% \\
%  \textbf{Thirteenth Author\textsuperscript{3}},
%  \textbf{Fourteenth F. Author\textsuperscript{2,4}},
%  \textbf{Fifteenth Author\textsuperscript{1}},
%  \textbf{Sixteenth Author\textsuperscript{1}},
% \\
%  \textbf{Seventeenth S. Author\textsuperscript{4,5}},
%  \textbf{Eighteenth Author\textsuperscript{3,4}},
%  \textbf{Nineteenth N. Author\textsuperscript{2,5}},
%  \textbf{Twentieth Author\textsuperscript{1}}
% \\
% \\
%  \textsuperscript{1}Affiliation 1,
%  \textsuperscript{2}Affiliation 2,
%  \textsuperscript{3}Affiliation 3,
%  \textsuperscript{4}Affiliation 4,
%  \textsuperscript{5}Affiliation 5
% \\
%  \small{
%    \textbf{Correspondence:} \href{mailto:email@domain}{email@domain}
%  }
% }

\newcommand{\methodname}{\textsc{I-MCTS}}

\begin{document}
\maketitle


\begin{abstract}
Recent advancements in large language models (LLMs) have shown remarkable potential in automating machine learning tasks. 
However, existing LLM-based agents often struggle with low-diversity and suboptimal code generation. 
While recent work~\cite{chi2024sela} has introduced Monte Carlo Tree Search (MCTS) to address these issues, limitations persist in the quality and diversity of thoughts generated, as well as in the scalar value feedback mechanisms used for node selection. 
In this study, we introduce Introspective Monte Carlo Tree Search (\textbf{\textit{I-MCTS}}), a novel approach that iteratively expands tree nodes through an introspective process that meticulously analyzes solutions and results from parent and sibling nodes. 
This facilitates a continuous refinement of the node in the search tree, thereby enhancing the overall decision-making process.
Furthermore, we integrate a Large Language Model (LLM)-based value model to facilitate direct evaluation of each node's solution prior to conducting comprehensive computational rollouts. 
A hybrid rewarding mechanism is implemented to seamlessly transition the Q-value from LLM-estimated scores to actual performance scores. 
This allows higher-quality nodes to be traversed earlier.
Applied to the various ML tasks, our approach demonstrates a
6\% absolute improvement in performance compared to the strong open-source AutoML agents, showcasing its effectiveness in enhancing agentic AutoML systems\footnote{Resource available at \url{https://github.com/jokieleung/I-MCTS}}.
\end{abstract}

\section{Introduction}
\label{sec:intro}


Recent advances in large language models (LLMs) have opened new frontiers in automating machine learning (AutoML) tasks~\cite{huang2024mlagentbench,chan2024mle}.
Compared to the traditional AutoML frameworks~\cite{feurer2015efficient, tang2024autogluon}, the LLM-based agents~\cite{sun2024survey} emerge as promising direction 
due to their remarkable capabilities on code generation~\cite{jimenez2023swe}, neural architecture design~\cite{zheng2023can} and model training~\cite{chi2024sela}. 
Overall, these LLM agent-based AutoML systems~\cite{Schmidt_Wu_Jiang,hong2024data,li2024autokaggle,chi2024sela} typically input with a natural language description on the dataset and the problem, 
after which the system directly generates a solution in an end-to-end manner. 
While recent works by \citet{Schmidt_Wu_Jiang} and \citet{hong2024data} have made significant strides in automating the machine learning workflow, replicating the adaptive and strategic behavior of expert human engineers remains a significant challenge. 
The primary limitation lies in their search processes, which typically involve only a single pass or trial. 
This constraint significantly hinders the generation of diverse and highly optimized workflows, a capability that human experts excel at through iterative refinement and strategic decision-making.
Recent work by ~\citet{chi2024sela} introduced Tree-Search Enhanced LLM Agents (SELA), leveraging Monte Carlo Tree Search (MCTS) to expand the search space. 
However, the search space is constrained by its reliance on a pre-generated, fixed insight pool derived from the initial problem description and dataset information.
This static nature inherently limits the diversity and adaptability of the search tree.
Moreover, SELA encounters significant limitations in improving the overall quality of solutions when relying exclusively on scalar experimental performance feedback. 
This reliance on simplistic scalar feedback mechanisms impedes the efficient identification and navigation of optimal pathways in complex machine learning tasks, where flexible re-assessment and adaptive strategies are usually needed.

To address these challenges, we present \textbf{I}ntrospective \textbf{M}onte \textbf{C}arlo \textbf{T}ree \textbf{S}earch (\textbf{\textit{I-MCTS}}), a novel framework that introduces two key innovations. First, our \textit{introspective node expansion} dynamically generates high-quality thought nodes through explicit analysis of parent and sibling node states. 
By incorporating reflective reasoning and feedback about prior solutions and their outcomes, this approach enables continuous quality improvement of the search tree nodes. 
Second, we develop a \textit{hybrid rewarding mechanism} that combines: 1) LLM-estimated evaluations predicting node potential through a comprehensive machine learning evaluation criteria, 
and 
2) actual performance scores on the development set from computational rollouts. 
Our adaptive reward blending strategy smoothly transitions Q-value from LLM-estimated values to actual values across search iterations.
This facilitates higher-quality nodes to be traversed earlier.

Overall, the principal contributions of this work are threefold:

\begin{itemize}
% \item We identify and formalize critical limitations in existing MCTS-based AutoML agents, particularly the static nature of thought generation and scalar value feedback mechanisms
\item We introduce I-MCTS, an innovative approach for agentic AutoML that incorporates an introspective node expansion process and a hybrid reward mechanism. This enhances both the quality and efficiency of tree searches in AutoML workflows.
    
\item Extensive experiments across diverse ML tasks demonstrate our method's superiority, achieving 6\% absolute performance gains over state-of-the-art baselines while maintaining computational efficiency.
\end{itemize}


% \section{Method}
\section{Introspective Monte Carlo Tree Search (I-MCTS)}
\label{sec:method}

As illustrated in Figure~\ref{fig:pipline}, our approach consists of two key components: a search module I-MCTS, and an LLM agent as the experiment executor. 
The Introspective Monte Carlo Tree Search (I-MCTS), builds upon the foundation of traditional MCTS but introduces two key innovations: 
(1) introspective node expansion through reflective solution generation, and (2) a hybrid reward mechanism combining LLM-estimated evaluations with empirical performance scores. 
These components work in tandem to enhance the quality and diversity of thought nodes while improving the efficiency of the search process. 
During each cycle, the selected path is passed to the LLM agent, which plans, codes, executes, and introspects the experiment, providing both scalar and verbal feedback to refine future searches. 
This iterative process continues until the termination criterion is met. 

\subsection{Preliminary}
We formalize the automated machine learning (AutoML) problem as a search process over possible experimental configurations. 
Given a problem description $p$ and dataset $D$ with metadata $d$, the objective is to identify an optimal pipeline configuration $c^* \in \mathcal{C}$ that maximizes a performance metric $s$ on development data. 
The search space $\mathcal{C}$ consists of configurations combining insights $\lambda_i^\tau$ across $T$ predefined stages of the ML workflow: $\tau_1$ (Exploratory Data Analysis), $\tau_2$ (Data Preprocessing), $\tau_3$ (Feature Engineering), $\tau_4$ (Model Training), and $\tau_5$ (Model Evaluation).

The LLM agent $E$ conducts each trial by building practical experimental pipelines from natural language requirements. 
Given an experiment configuration $c$, the agent produces a detailed plan $E_{\text{plan}}$ that consists of a sequence of task instructions $I^{\tau \in T}$ corresponding to each stage of the machine learning process. Next, the agent writes and executes code $\sigma^\tau$ for each task $\tau$ based on the respective instruction and gets the final execution score $s$. 
The complete set of code outputs $\sigma^{\tau \in T}$ is concatenated into a full solution $\sigma_{sol}$ to address the problem. This phase is referred to as $E_{\text{code \& execute}}$.
\begin{align}
    E_{\text{plan}}(p, d, c, LLM) & \rightarrow I^{\tau \in T} \\
    E_{\text{code \& execute}}(I^{\tau \in T}, D, LLM) & \rightarrow (\sigma^{\tau \in T}, s)
\end{align}

\begin{figure}[!t]
    \centering
    \includegraphics[width=0.95\linewidth]{figures/I-MCTS.pdf}
    \caption{I-MCTS architecture featuring (a) Introspective node expansion through parent/sibling analysis, (b) Hybrid reward calculation combining LLM predictions and empirical scores. The red arrows indicate the introspective feedback loop that continuously improves node quality.}
    \label{fig:pipline}
\end{figure}

\subsection{Introspective Node Expansion}
\label{ssec:introspective}

Unlike static insight pools in prior work~\cite{chi2024sela}, 
I-MCTS utilizes the introspective expansion mechanism that dynamically generates high-quality thought nodes by leveraging solutions and results from parent and sibling nodes. 
This design mirrors the way human expert engineers backtrack and refine their solution dynamically, ensuring the agent can iteratively
improve its decision-making process based on past experiences.
Given a parent node $x_{\text{parent}}$ and its existing sibling nodes $x_{\text{sibling}}$, the introspective expansion process generates a new node $x_{\text{child}}$ by introspecting the solutions and results of these previous nodes. 
Specifically, the introspection mechanism use an LLM to evaluate the solutions $\sigma_{\text{sol}}(x_{\text{parent}})$ and $\sigma_{\text{sol}}(x_{\text{sibling}})$ associated with the parent and sibling nodes and identify strengths, weaknesses, issues and potential improvements in the solutions.
Then, the LLM generates a fine-grained and customized insight $\lambda_{\text{new}}$ that addresses the identified issues or builds upon the strengths. 
This insight is tailored to the current stage $\tau$ of the machine learning pipeline. The new insight $\lambda_{\text{new}}$ is used to create a child node $x_{\text{child}}$, which inherits attributes from the parent node while incorporating the refined insight. 
The child node is then added to the search tree.
This introspection process ensures that each new node is dynamically generated through a thoughtful analysis of prior solutions, leading to a continuous improvement in the quality of the search tree.
Also, the verbalized introspection feedback addresses the solely scalar feedback limitation.


\subsection{Hybrid Reward Mechanism}
\label{ssec:hybrid}

The primary objective of the evaluation phase is to assess the reward for the current node.
For machine learning tasks, a conventional approach~\cite{Schmidt_Wu_Jiang,chi2024sela} involves utilizing the performance metric on the development set as the reward. 
To derive a performance score of the given node involves rolling out from an intermediate state to a terminal state, which is quite costly since processes such as model training cost significant computational time. 
This limits the efficiency of node exploration.
To address these challenges, we introduce a hybrid reward mechanism for node evaluation which combines LLM-estimated evaluations with actual performance scores to guide the search process more efficiently. 
% \noindent\textbf{LLM-Estimated Evaluations}
% \label{subsubsec:llm-estimated}
For each node $x$, we employ an LLM-based value model $M_{\text{value}}$ to predict its potential performance. The value model takes as input the solution $\sigma_{\text{sol}}(x)$ and outputs an estimated score $s_{\text{LLM}}(x)$. This score is based on a comprehensive set of machine learning evaluation criteria, which is detailed in Appendix~\ref{app:llm_evaluation_schema}.
The LLM-estimated evaluation allows us to assess the quality of a node's solution before conducting computationally expensive rollouts. This early evaluation helps prioritize nodes with higher potential, improving the efficiency of the search process.
% \noindent\textbf{Actual Performance Scores}
% \label{subsubsec:actual-scores}
After the ``simulation"" of a node $x$, we obtain the actual performance score $s_{\text{actual}}(x)$ based on the development set, which reflects the true performance of the node's solution.
% \noindent\textbf{Adaptive Reward Blending}
% \label{subsubsec:adaptive-blending}
To smoothly transition from LLM-estimated values to actual performance scores, we implement an adaptive reward blending strategy. The Q-value $Q(x)$ for a node $x$ is updated as follows:

\begin{align}
    Q(x) = \alpha(x) \cdot s_{\text{LLM}}(x) + (1 - \alpha(x)) \cdot s_{\text{actual}}(x)
\end{align}

where $\alpha(x) = \frac{\gamma}{n_{\text{visits}(x)}+\gamma}$, a blending factor that decreases over the visit count increase, where $\gamma$ is a controlling constant.
This adaptive blending ensures that higher-quality nodes are traversed earlier in the search process, while still incorporating the true performance feedback as it becomes available.

\subsection{Tree Search Process}
\label{ssec:adaptive}

The overall search process can be summarized as:

\paragraph{Selection} At each iteration, we select node according to the UCT formula as:

\begin{align}
    \text{UCT}(x) = \frac{Q(x)}{n(x)} + \alpha_{\text{explore}} \sqrt{\frac{\ln n_{\text{visits}}(x_{\text{parent}})}{n(x)}}
\end{align}

\paragraph{Expansion} Child nodes are generated through the introspective process described in Section~\ref{ssec:introspective}, creating a dynamic search space that evolves based on simulation feedback.

\paragraph{Simulation} Each rollout executes the full pipeline while caching intermediate results. The LLM value model provides real-time estimates $v_{\text{LLM}}$ before computational execution.

\paragraph{Backpropagation} 
Upon simulation completion, the Q value is propagated backwards through the tree structure. 
This process updates the value and visit count of each parent node, from the simulated node to the root. Consequently, nodes associated with superior solutions receive higher priority in subsequent iterations.
This allows nodes representing more promising solutions to be prioritized in future rollouts.
Similar to \citet{chi2024sela}, I-MCTS implements a state-saving mechanism that caches the stage code for each node, which allows it to reuse previously generated code when a new configuration shares components with existing ones.


\section{Experiment}

We follow \citet{chi2024sela} to benchmark our method on 20 machine learning datasets, including 13 classification tasks and 7 regression tasks from the AutoML Benchmark (AMLB) ~\cite{gijsbers2024amlb} and Kaggle Competitions.
All datasets are split into training, validation, and test sets with a 6:2:2 ratio, details refer to Appendix~\ref{app:dataests}.
Each dataset is run three times using our method.
We also use the Normalized Score (NS) defined in Appendix ~\ref{app:metric}.
We adopt Qwen2.5-72B-Instruct~\cite{yang2024qwen2} as our LLM.
The temperature parameter is set to 0.2.
$\gamma$ is set to 0.2, while $\alpha_{\text{explore}}$ is set to 2.
Five baselines are included for comparison, including AutoGluon~\cite{tang2024autogluon}, AutoSklearn~\cite{feurer2022auto}, AIDE~\cite{Schmidt_Wu_Jiang}, Data Interpreter~\cite{hong2024data}, SELA~\cite{chi2024sela}.
We directly used the results reported in \citet{chi2024sela}.
Experiments are conducted based on the MetaGPT~\cite{hong2024metagpt} framework\footnote{\url{https://github.com/geekan/MetaGPT}}.
% ~\footnote{https://github.com/geekan/MetaGPT/tree/main/metagpt/ext/sela}.

\subsection{Main Results}

\begin{table}[t]
\centering
\resizebox{0.5\textwidth}{!}{
    \begin{tabular}{lccc}
        \toprule
        \textbf{Method} & \textbf{Top1 Rate} \%  & \textbf{Avg. NS} \% $\uparrow$ & \textbf{Avg. Best NS} \% $\uparrow$  \\
        \midrule
        AutoGluon   & 5.0 & 53.2 & 53.2 \\
        AutoSklearn  & 25.0 & 46.1 & 47.5  \\
        AIDE  & 5.0 & 47.1 & 51.8  \\
        Data Interpreter & 0.0 & 47.4 & 50.2 \\
        SELA & 20.0 & 53.3 & 54.7 \\
        \textbf{\methodname{}} & \textbf{45.0} & \textbf{58.6} & \textbf{59.8} \\
        \bottomrule
    \end{tabular}
    }
    \caption{Results of each AutoML framework on 20 tabular datasets. The ``Top1 Rate" column represents the rate of datasets where the method produces the best predictions across methods.}
    \label{table:main}
\end{table}

The experimental results demonstrate the superior performance of the proposed I-MCTS approach compared to other strong AutoML baselines. As shown in Table \ref{table:full-main-results}, I-MCTS achieves the highest overall normalized score (NS) of 58.6\% on average and 59.8\% for the best predictions across 20 diverse tabular datasets.
These results validate the effectiveness of I-MCTS's unique introspective node expansion process, which enables continuous solution refinement through meticulous analysis of parent and sibling nodes.
Also, the integration of an LLM-based value model and hybrid rewarding mechanism further contributes to the algorithm's success by facilitating early identification and traversal of high-quality nodes.
We include detailed results of each method in Appendix~\ref{app:detailed_result}.

\subsection{Ablation Study}

We follow \citet{chi2024sela} to use boston, colleges, credit-g, Click\_prediction\_small, GesturePhaseSegmentationProcessed, and mfeat-factors to dataset for ablation study. 
We systematically evaluate the contribution of each component within our proposed Introspective Monte Carlo Tree Search (I-MCTS) framework. The results, as presented in Table \ref{table:ablation_study}, illustrate the incremental benefits of incorporating both the Introspective Node Expansion (INE) and the hybrid reward mechanism (HRM).

\begin{table}[t]
\centering
\resizebox{0.4\textwidth}{!}{
% \begin{tabularx}{1.12\textwidth}
\begin{tabular}{lc}
    \toprule
    \textbf{Method} & \textbf{Avg. NS $\uparrow$}  \\
    \midrule
    Data Interpreter & 56.4 \\
    SELA (Random Search) & 58.6  \\
    SELA (MCTS) & 60.9   \\
    \methodname{} (w/o INE) & \textbf{61.1}  \\
    \methodname{} (w/o HRM) & \textbf{66.2}  \\
    \textbf{\methodname{} } & \textbf{66.8}  \\
    \bottomrule
\end{tabular}
}
\caption{Ablation Study for each search setting on the selected 6 datasets. ``\methodname{} (w/o INE)" means ``without "Introspective Node Expansion", while ``\methodname{} (w/o HRM)" means ``without "hybrid reward mechanism".}
\label{table:ablation_study}
\end{table}


\subsection{Case Study}

To demonstrate the superiority of our Introspective Monte Carlo Tree Search (I-MCTS) approach, we conduct a comprehensive visualization analysis of the tree search process using the \textit{\textbf{kick}} dataset. 
As show in Appendix~\ref{app:case},
our empirical results reveal that I-MCTS effectively leverages the introspective information from previous nodes to generate task-specific and actionable insights, while SELA exhibit significant homogeneity and lack specificity.
Furthermore, our hybrid reward mechanism enhances the exploration efficiency, enabling more effective identification of high-quality nodes within the same computational budget.





\section{Conclusion}

In this paper, we introduced \textbf{I}ntrospective \textbf{M}onte \textbf{C}arlo \textbf{T}ree \textbf{S}earch (\textbf{\textit{I-MCTS}}), a novel approach to enhance AutoML Agents. 
The method addresses key limitations in existing Tree-search-based LLM agents with respect to thought diversity, quality, and the efficiency of the search process.
Our experimental results highlight the potential of integrating introspective capabilities into tree search algorithms for AutoML Agents. 


\section*{Limitations}
While the proposed I-MCTS approach demonstrates significant improvements in AutoML agent performance, several limitations remain that warrant further investigation. First, the computational overhead of the introspective process, although beneficial for enhancing decision-making, introduces additional resource requirements. This limits the scalability of the method, particularly in scenarios where computational resources are constrained.
Future work should explore optimizations to reduce this overhead and investigate how the approach scales with increased computational power.

Second, the current evaluation of I-MCTS is primarily focused on structured data and tabular ML tasks. Its effectiveness on more complex and heterogeneous data types, such as image and text data, remains unexplored. Extending the application of I-MCTS to these domains could provide valuable insights into its generalizability and robustness across diverse machine learning tasks, such as Vison, NLP and RL tasks.

% Bibliography entries for the entire Anthology, followed by custom entries
%\bibliography{anthology,custom}
% Custom bibliography entries only
% \newpage
\bibliography{custom}

% \clearpage

\onecolumn
\appendix
% \section*{APPENDIX}

\newpage
\begin{algorithm}[h!]
\caption{Gait-Net-augmented Sequential CMPC}
\label{alg:gaitMPC}
\begin{algorithmic}[1]
\Require $\mathbf q, \: \dot{\mathbf q}, \: \mathbf q^\text{cmd}, \: \dot{\mathbf q}^\text{cmd}$
\State \textbf{intialize} $\bm x_0 = f_\text{j2m}(\mathbf q, \: \dot{\mathbf q}), \: \bm u^0 =\bm u_\text{IG}, \: dt^0 = 0.05$ 
\State $\{ \mathbf q^\text{ref},\:\dot{\mathbf q}^\text{ref},\:\bm p_f^\text{ref}\} = f_\text{ref} \big(\mathbf q, \: \dot{\mathbf q}, \: \mathbf q^\text{cmd}, \: \dot{\mathbf q}^\text{cmd} \big)$
\State $\bm x^\text{ref} = f_\text{j2m}(\mathbf q^\text{ref},\:\dot{\mathbf q}^\text{ref},\:\bm p_f^\text{ref})$
\State $ j = 0$ 
\While{$j \leq j_\text{max} \:\text{and}\: \bm \eta \leq \delta \bm u  $} 
\State $\delta \bm u^{j} = \texttt{cmpc}(\bm x^\text{ref},\:\bm p_f^\text{ref},\:\bm p_c^\text{ref},\: \bm x_0,\: dt^j, \: \bm u^j)$
\State $\bm u^{j+1} = \bm u^j + \delta \bm u^j$ 
\State $dt^{j+1} = \Pi_\text{GN}(\mathbf q, \: \dot{\mathbf q},\: \bm p_f^{j})$
\State $\{ \bm x^\text{ref},\:\bm p_f^\text{ref}\}= f_\text{IK}(\bm p_f^{j},\:\bm p_c^{j},\: dt^{j+1})$
\State $j=j+1$
\EndWhile \\
\Return $\bm u^{j+1} $
\end{algorithmic}
\end{algorithm}

\section{Detailed Results}
\label{app:detailed_result}

Below show all detailed results on each dataset.

\begin{table}[h!]
\centering
\resizebox{\textwidth}{!}{
\begin{tabularx}{1.12\textwidth}{>{\scriptsize}l*{12}{c}}
\toprule
 & \multicolumn{2}{c}{\textbf{AutoGluon}} & \multicolumn{2}{c}{\textbf{AutoSklearn}} & \multicolumn{2}{c}{\textbf{AIDE}} & \multicolumn{2}{c}{\textbf{DI}} & \multicolumn{2}{c}{\textbf{SELA}} & \multicolumn{2}{c}{\textbf{\methodname{}}} \\ 
\normalsize Dataset & {Avg.} & {Best} & {Avg.} & {Best} & {Avg.} & {Best} & {Avg.} & {Best} & {Avg.} & {Best} & {Avg.} & {Best} \\
\midrule
Click\_prediction\_small & 26.6 & 26.6 & 40.2 & 40.3 & 26.1 & 39.4 & 12.9 & 13.9 & 23.2 & 27.4 & 29.1 & 30.2 \\
GesturePhaseSegmentationProcessed & 69.3 & 69.3 & 67.2 & 68.4 & 56.3 & 68.1 & 60.1 & 64.4 & 67.9 & 69.2 & 67.2 & 68.5 \\
Moneyball & 24.3 & 24.3 & 13.1 & 13.8 & 23.8 & 24.6 & 9.5 & 24.5 & 21.9 & 24.5 & 41.6 & 41.7 \\
SAT11-HAND-runtime-regression & 12.6 & 12.6 & 10.3 & 10.3 & 12.0 & 12.1 & 11.4 & 11.9 & 12.2 & 12.5 & 21.7 & 24.6 \\
boston & 39.8 & 39.8 & 19.5 & 19.6 & 40.5 & 41.3 & 37.0 & 38.6 & 40.1 & 41.4 & 62.5 & 62.7 \\
colleges & 88.3 & 88.3 & 2.1 & 2.1 & 86.0 & 87.8 & 87.5 & 87.7 & 87.8 & 87.8 & 94.1 & 94.3 \\
concrete-strength & 28.3 & 28.3 & 17.4 & 17.9 & 28.3 & 28.3 & 28.8 & 29.6 & 28.2 & 28.2 & 47.3 & 48.2 \\
credit-g & 50.5 & 50.5 & 35.1 & 44.0 & 21.6 & 48.4 & 48.1 & 53.2 & 50.9 & 52.7 & 50.9 & 51.1 \\
diamonds & 13.8 & 13.8 & 8.7 & 8.7 & 13.7 & 13.7 & 13.5 & 13.6 & 13.7 & 13.8 & 26.8 & 26.8\\
house-prices & 9.0 & 9.0 & 2.0 & 2.0 & 8.9 & 8.9 & 8.5 & 9.0 & 8.9 & 9.0 & 15.8 & 18.4 \\
icr & 76.2 & 76.2 & 70.4 & 79.2 & 31.7 & 35.9 & 57.8 & 60.6 & 78.7 & 79.2 & 72.3 & 79.0 \\
jasmine & 84.3 & 84.3 & 84.4 & 84.7 & 83.6 & 84.6 & 77.8 & 83.5 & 85.4 & 86.2 
 & 84.2 & 84.3\\
kc1 & 38.3 & 38.3 & 43.5 & 45.0 & 40.8 & 42.6 & 38.1 & 41.2 & 42.2 & 43.1 & 44.8 & 45.6 \\
kick & 39.6 & 39.6 & 41.8 & 42.1 & 14.9 & 38.6 & 2.8 & 4.2 & 35.9 & 38.7 & 36.2 & 42.4 \\
mfeat-factors & 96.7 & 96.7 & 97.1 & 97.5 & 94.4 & 94.5 & 93.0 & 96.0 & 95.7 & 96.2 & 94.4 & 95.4\\
segment & 93.5 & 93.5 & 92.7 & 93.1 & 91.7 & 92.2 & 91.7 & 92.6 & 93.8 & 94.4 & 91.1 & 93.7\\
smoker-status & 78.0 & 78.0 & 78.6 & 78.9 & 74.8 & 76.3 & 77.3 & 81.5 & 82.4 & 91.5 & 79.9 & 91.5\\
software-defects & 51.5 & 51.5 & 61.1 & 61.7 & 49.7 & 49.8 & 54.5 & 57.3 & 52.2 & 53.3 & 60.5 & 60.7\\
titanic & 78.9 & 78.9 & 76.2 & 78.9 & 81.2 & 83.7 & 76.0 & 78.5 & 78.8 & 79.7 & 76.1 & 78.5\\
wine-quality-white & 65.4 & 65.4 & 60.7 & 61.4 & 62.9 & 65.1 & 61.2 & 61.6 & 65.3 & 66.0 & 63.3 & 65.5 \\
\midrule
\normalsize Overall NS \% $\uparrow$ & 53.2 & 53.2 & 46.1 & 47.5  & 45.5 & 51.8 & 47.4 & 50.2 & 53.3 & 54.7 & \textbf{58.6} & \textbf{59.8} \\
\bottomrule
\end{tabularx}
}
\caption{Methods' NS \% for each tabular dataset}
\label{table:full-main-results}
\end{table}
\section{Dataset}
\label{sec:dataset}

\subsection{Data Collection}

To analyze political discussions on Discord, we followed the methodology in \cite{singh2024Cross-Platform}, collecting messages from politically-oriented public servers in compliance with Discord's platform policies.

Using Discord's Discovery feature, we employed a web scraper to extract server invitation links, names, and descriptions, focusing on public servers accessible without participation. Invitation links were used to access data via the Discord API. To ensure relevance, we filtered servers using keywords related to the 2024 U.S. elections (e.g., Trump, Kamala, MAGA), as outlined in \cite{balasubramanian2024publicdatasettrackingsocial}. This resulted in 302 server links, further narrowed to 81 English-speaking, politics-focused servers based on their names and descriptions.

Public messages were retrieved from these servers using the Discord API, collecting metadata such as \textit{content}, \textit{user ID}, \textit{username}, \textit{timestamp}, \textit{bot flag}, \textit{mentions}, and \textit{interactions}. Through this process, we gathered \textbf{33,373,229 messages} from \textbf{82,109 users} across \textbf{81 servers}, including \textbf{1,912,750 messages} from \textbf{633 bots}. Data collection occurred between November 13th and 15th, covering messages sent from January 1st to November 12th, just after the 2024 U.S. election.

\subsection{Characterizing the Political Spectrum}
\label{sec:timeline}

A key aspect of our research is distinguishing between Republican- and Democratic-aligned Discord servers. To categorize their political alignment, we relied on server names and self-descriptions, which often include rules, community guidelines, and references to key ideologies or figures. Each server's name and description were manually reviewed based on predefined, objective criteria, focusing on explicit political themes or mentions of prominent figures. This process allowed us to classify servers into three categories, ensuring a systematic and unbiased alignment determination.

\begin{itemize}
    \item \textbf{Republican-aligned}: Servers referencing Republican and right-wing and ideologies, movements, or figures (e.g., MAGA, Conservative, Traditional, Trump).  
    \item \textbf{Democratic-aligned}: Servers mentioning Democratic and left-wing ideologies, movements, or figures (e.g., Progressive, Liberal, Socialist, Biden, Kamala).  
    \item \textbf{Unaligned}: Servers with no defined spectrum and ideologies or opened to general political debate from all orientations.
\end{itemize}

To ensure the reliability and consistency of our classification, three independent reviewers assessed the classification following the specified set of criteria. The inter-rater agreement of their classifications was evaluated using Fleiss' Kappa \cite{fleiss1971measuring}, with a resulting Kappa value of \( 0.8191 \), indicating an almost perfect agreement among the reviewers. Disagreements were resolved by adopting the majority classification, as there were no instances where a server received different classifications from all three reviewers. This process guaranteed the consistency and accuracy of the final categorization.

Through this process, we identified \textbf{7 Republican-aligned servers}, \textbf{9 Democratic-aligned servers}, and \textbf{65 unaligned servers}.

Table \ref{tab:statistics} shows the statistics of the collected data. Notably, while Democratic- and Republican-aligned servers had a comparable number of user messages, users in the latter servers were significantly more active, posting more than double the number of messages per user compared to their Democratic counterparts. 
This suggests that, in our sample, Democratic-aligned servers attract more users, but these users were less engaged in text-based discussions. Additionally, around 10\% of the messages across all server categories were posted by bots. 

\subsection{Temporal Data} 

Throughout this paper, we refer to the election candidates using the names adopted by their respective campaigns: \textit{Kamala}, \textit{Biden}, and \textit{Trump}. To examine how the content of text messages evolves based on the political alignment of servers, we divided the 2024 election year into three periods: \textbf{Biden vs Trump} (January 1 to July 21), \textbf{Kamala vs Trump} (July 21 to September 20), and the \textbf{Voting Period} (after September 20). These periods reflect key phases of the election: the early campaign dominated by Biden and Trump, the shift in dynamics with Kamala Harris replacing Joe Biden as the Democratic candidate, and the final voting stage focused on electoral outcomes and their implications. This segmentation enables an analysis of how discourse responds to pivotal electoral moments.

Figure \ref{fig:line-plot} illustrates the distribution of messages over time, highlighting trends in total messages volume and mentions of each candidate. Prior to Biden's withdrawal on July 21, mentions of Biden and Trump were relatively balanced. However, following Kamala's entry into the race, mentions of Trump surged significantly, a trend further amplified by an assassination attempt on him, solidifying his dominance in the discourse. The only instance where Trump’s mentions were exceeded occurred during the first debate, as concerns about Biden’s age and cognitive abilities temporarily shifted the focus. In the final stages of the election, mentions of all three candidates rose, with Trump’s mentions peaking as he emerged as the victor.

\definecolor{titlecolor}{rgb}{0.9, 0.5, 0.1}
\definecolor{anscolor}{rgb}{0.2, 0.5, 0.8}
\definecolor{labelcolor}{HTML}{48a07e}
\begin{table*}[h]
	\centering
	
 % \vspace{-0.2cm}
	
	\begin{center}
		\begin{tikzpicture}[
				chatbox_inner/.style={rectangle, rounded corners, opacity=0, text opacity=1, font=\sffamily\scriptsize, text width=5in, text height=9pt, inner xsep=6pt, inner ysep=6pt},
				chatbox_prompt_inner/.style={chatbox_inner, align=flush left, xshift=0pt, text height=11pt},
				chatbox_user_inner/.style={chatbox_inner, align=flush left, xshift=0pt},
				chatbox_gpt_inner/.style={chatbox_inner, align=flush left, xshift=0pt},
				chatbox/.style={chatbox_inner, draw=black!25, fill=gray!7, opacity=1, text opacity=0},
				chatbox_prompt/.style={chatbox, align=flush left, fill=gray!1.5, draw=black!30, text height=10pt},
				chatbox_user/.style={chatbox, align=flush left},
				chatbox_gpt/.style={chatbox, align=flush left},
				chatbox2/.style={chatbox_gpt, fill=green!25},
				chatbox3/.style={chatbox_gpt, fill=red!20, draw=black!20},
				chatbox4/.style={chatbox_gpt, fill=yellow!30},
				labelbox/.style={rectangle, rounded corners, draw=black!50, font=\sffamily\scriptsize\bfseries, fill=gray!5, inner sep=3pt},
			]
											
			\node[chatbox_user] (q1) {
				\textbf{System prompt}
				\newline
				\newline
				You are a helpful and precise assistant for segmenting and labeling sentences. We would like to request your help on curating a dataset for entity-level hallucination detection.
				\newline \newline
                We will give you a machine generated biography and a list of checked facts about the biography. Each fact consists of a sentence and a label (True/False). Please do the following process. First, breaking down the biography into words. Second, by referring to the provided list of facts, merging some broken down words in the previous step to form meaningful entities. For example, ``strategic thinking'' should be one entity instead of two. Third, according to the labels in the list of facts, labeling each entity as True or False. Specifically, for facts that share a similar sentence structure (\eg, \textit{``He was born on Mach 9, 1941.''} (\texttt{True}) and \textit{``He was born in Ramos Mejia.''} (\texttt{False})), please first assign labels to entities that differ across atomic facts. For example, first labeling ``Mach 9, 1941'' (\texttt{True}) and ``Ramos Mejia'' (\texttt{False}) in the above case. For those entities that are the same across atomic facts (\eg, ``was born'') or are neutral (\eg, ``he,'' ``in,'' and ``on''), please label them as \texttt{True}. For the cases that there is no atomic fact that shares the same sentence structure, please identify the most informative entities in the sentence and label them with the same label as the atomic fact while treating the rest of the entities as \texttt{True}. In the end, output the entities and labels in the following format:
                \begin{itemize}[nosep]
                    \item Entity 1 (Label 1)
                    \item Entity 2 (Label 2)
                    \item ...
                    \item Entity N (Label N)
                \end{itemize}
                % \newline \newline
                Here are two examples:
                \newline\newline
                \textbf{[Example 1]}
                \newline
                [The start of the biography]
                \newline
                \textcolor{titlecolor}{Marianne McAndrew is an American actress and singer, born on November 21, 1942, in Cleveland, Ohio. She began her acting career in the late 1960s, appearing in various television shows and films.}
                \newline
                [The end of the biography]
                \newline \newline
                [The start of the list of checked facts]
                \newline
                \textcolor{anscolor}{[Marianne McAndrew is an American. (False); Marianne McAndrew is an actress. (True); Marianne McAndrew is a singer. (False); Marianne McAndrew was born on November 21, 1942. (False); Marianne McAndrew was born in Cleveland, Ohio. (False); She began her acting career in the late 1960s. (True); She has appeared in various television shows. (True); She has appeared in various films. (True)]}
                \newline
                [The end of the list of checked facts]
                \newline \newline
                [The start of the ideal output]
                \newline
                \textcolor{labelcolor}{[Marianne McAndrew (True); is (True); an (True); American (False); actress (True); and (True); singer (False); , (True); born (True); on (True); November 21, 1942 (False); , (True); in (True); Cleveland, Ohio (False); . (True); She (True); began (True); her (True); acting career (True); in (True); the late 1960s (True); , (True); appearing (True); in (True); various (True); television shows (True); and (True); films (True); . (True)]}
                \newline
                [The end of the ideal output]
				\newline \newline
                \textbf{[Example 2]}
                \newline
                [The start of the biography]
                \newline
                \textcolor{titlecolor}{Doug Sheehan is an American actor who was born on April 27, 1949, in Santa Monica, California. He is best known for his roles in soap operas, including his portrayal of Joe Kelly on ``General Hospital'' and Ben Gibson on ``Knots Landing.''}
                \newline
                [The end of the biography]
                \newline \newline
                [The start of the list of checked facts]
                \newline
                \textcolor{anscolor}{[Doug Sheehan is an American. (True); Doug Sheehan is an actor. (True); Doug Sheehan was born on April 27, 1949. (True); Doug Sheehan was born in Santa Monica, California. (False); He is best known for his roles in soap operas. (True); He portrayed Joe Kelly. (True); Joe Kelly was in General Hospital. (True); General Hospital is a soap opera. (True); He portrayed Ben Gibson. (True); Ben Gibson was in Knots Landing. (True); Knots Landing is a soap opera. (True)]}
                \newline
                [The end of the list of checked facts]
                \newline \newline
                [The start of the ideal output]
                \newline
                \textcolor{labelcolor}{[Doug Sheehan (True); is (True); an (True); American (True); actor (True); who (True); was born (True); on (True); April 27, 1949 (True); in (True); Santa Monica, California (False); . (True); He (True); is (True); best known (True); for (True); his roles in soap operas (True); , (True); including (True); in (True); his portrayal (True); of (True); Joe Kelly (True); on (True); ``General Hospital'' (True); and (True); Ben Gibson (True); on (True); ``Knots Landing.'' (True)]}
                \newline
                [The end of the ideal output]
				\newline \newline
				\textbf{User prompt}
				\newline
				\newline
				[The start of the biography]
				\newline
				\textcolor{magenta}{\texttt{\{BIOGRAPHY\}}}
				\newline
				[The ebd of the biography]
				\newline \newline
				[The start of the list of checked facts]
				\newline
				\textcolor{magenta}{\texttt{\{LIST OF CHECKED FACTS\}}}
				\newline
				[The end of the list of checked facts]
			};
			\node[chatbox_user_inner] (q1_text) at (q1) {
				\textbf{System prompt}
				\newline
				\newline
				You are a helpful and precise assistant for segmenting and labeling sentences. We would like to request your help on curating a dataset for entity-level hallucination detection.
				\newline \newline
                We will give you a machine generated biography and a list of checked facts about the biography. Each fact consists of a sentence and a label (True/False). Please do the following process. First, breaking down the biography into words. Second, by referring to the provided list of facts, merging some broken down words in the previous step to form meaningful entities. For example, ``strategic thinking'' should be one entity instead of two. Third, according to the labels in the list of facts, labeling each entity as True or False. Specifically, for facts that share a similar sentence structure (\eg, \textit{``He was born on Mach 9, 1941.''} (\texttt{True}) and \textit{``He was born in Ramos Mejia.''} (\texttt{False})), please first assign labels to entities that differ across atomic facts. For example, first labeling ``Mach 9, 1941'' (\texttt{True}) and ``Ramos Mejia'' (\texttt{False}) in the above case. For those entities that are the same across atomic facts (\eg, ``was born'') or are neutral (\eg, ``he,'' ``in,'' and ``on''), please label them as \texttt{True}. For the cases that there is no atomic fact that shares the same sentence structure, please identify the most informative entities in the sentence and label them with the same label as the atomic fact while treating the rest of the entities as \texttt{True}. In the end, output the entities and labels in the following format:
                \begin{itemize}[nosep]
                    \item Entity 1 (Label 1)
                    \item Entity 2 (Label 2)
                    \item ...
                    \item Entity N (Label N)
                \end{itemize}
                % \newline \newline
                Here are two examples:
                \newline\newline
                \textbf{[Example 1]}
                \newline
                [The start of the biography]
                \newline
                \textcolor{titlecolor}{Marianne McAndrew is an American actress and singer, born on November 21, 1942, in Cleveland, Ohio. She began her acting career in the late 1960s, appearing in various television shows and films.}
                \newline
                [The end of the biography]
                \newline \newline
                [The start of the list of checked facts]
                \newline
                \textcolor{anscolor}{[Marianne McAndrew is an American. (False); Marianne McAndrew is an actress. (True); Marianne McAndrew is a singer. (False); Marianne McAndrew was born on November 21, 1942. (False); Marianne McAndrew was born in Cleveland, Ohio. (False); She began her acting career in the late 1960s. (True); She has appeared in various television shows. (True); She has appeared in various films. (True)]}
                \newline
                [The end of the list of checked facts]
                \newline \newline
                [The start of the ideal output]
                \newline
                \textcolor{labelcolor}{[Marianne McAndrew (True); is (True); an (True); American (False); actress (True); and (True); singer (False); , (True); born (True); on (True); November 21, 1942 (False); , (True); in (True); Cleveland, Ohio (False); . (True); She (True); began (True); her (True); acting career (True); in (True); the late 1960s (True); , (True); appearing (True); in (True); various (True); television shows (True); and (True); films (True); . (True)]}
                \newline
                [The end of the ideal output]
				\newline \newline
                \textbf{[Example 2]}
                \newline
                [The start of the biography]
                \newline
                \textcolor{titlecolor}{Doug Sheehan is an American actor who was born on April 27, 1949, in Santa Monica, California. He is best known for his roles in soap operas, including his portrayal of Joe Kelly on ``General Hospital'' and Ben Gibson on ``Knots Landing.''}
                \newline
                [The end of the biography]
                \newline \newline
                [The start of the list of checked facts]
                \newline
                \textcolor{anscolor}{[Doug Sheehan is an American. (True); Doug Sheehan is an actor. (True); Doug Sheehan was born on April 27, 1949. (True); Doug Sheehan was born in Santa Monica, California. (False); He is best known for his roles in soap operas. (True); He portrayed Joe Kelly. (True); Joe Kelly was in General Hospital. (True); General Hospital is a soap opera. (True); He portrayed Ben Gibson. (True); Ben Gibson was in Knots Landing. (True); Knots Landing is a soap opera. (True)]}
                \newline
                [The end of the list of checked facts]
                \newline \newline
                [The start of the ideal output]
                \newline
                \textcolor{labelcolor}{[Doug Sheehan (True); is (True); an (True); American (True); actor (True); who (True); was born (True); on (True); April 27, 1949 (True); in (True); Santa Monica, California (False); . (True); He (True); is (True); best known (True); for (True); his roles in soap operas (True); , (True); including (True); in (True); his portrayal (True); of (True); Joe Kelly (True); on (True); ``General Hospital'' (True); and (True); Ben Gibson (True); on (True); ``Knots Landing.'' (True)]}
                \newline
                [The end of the ideal output]
				\newline \newline
				\textbf{User prompt}
				\newline
				\newline
				[The start of the biography]
				\newline
				\textcolor{magenta}{\texttt{\{BIOGRAPHY\}}}
				\newline
				[The ebd of the biography]
				\newline \newline
				[The start of the list of checked facts]
				\newline
				\textcolor{magenta}{\texttt{\{LIST OF CHECKED FACTS\}}}
				\newline
				[The end of the list of checked facts]
			};
		\end{tikzpicture}
        \caption{GPT-4o prompt for labeling hallucinated entities.}\label{tb:gpt-4-prompt}
	\end{center}
\vspace{-0cm}
\end{table*}
\begin{figure}[htb]
\small
\begin{tcolorbox}[left=3pt,right=3pt,top=3pt,bottom=3pt,title=\textbf{Conversation History:}]
[human]: Craft an intriguing opening paragraph for a fictional short story. The story should involve a character who wakes up one morning to find that they can time travel.

...(Human-Bot Dialogue Turns)... \textcolor{blue}{(Topic: Time-Travel Fiction)}

[human]: Please describe the concept of machine learning. Could you elaborate on the differences between supervised, unsupervised, and reinforcement learning? Provide real-world examples of each.

...(Human-Bot Dialogue Turns)... \textcolor{blue}{(Topic: Machine learning Concepts and Types)}


[human]: Discuss antitrust laws and their impact on market competition. Compare the antitrust laws in US and China along with some case studies

...(Human-Bot Dialogue Turns)... \textcolor{blue}{(Topic: Antitrust Laws and Market Competition)}

[human]: The vertices of a triangle are at points (0, 0), (-1, 1), and (3, 3). What is the area of the triangle?

...(Human-Bot Dialogue Turns)... \textcolor{blue}{(Topic: Area and Circumcircle of Triangle)}

...(Human-Bot Dialogue Interaction on Other Topics)... 

\end{tcolorbox}
\begin{tcolorbox}[left=3pt,right=3pt,top=3pt,bottom=3pt,title=\textbf{Question:}]
What title reflects the thematic essence of the Doctor's experience with time travel as described in the story?
\end{tcolorbox}
\begin{tcolorbox}[left=3pt,right=3pt,top=3pt,bottom=3pt,title=\textbf{Ground-Truth Answer:}]
Awakening to Eternity: The Doctor's New Dawn
\end{tcolorbox}
\begin{tcolorbox}[left=3pt,right=3pt,top=3pt,bottom=3pt,title=\textbf{Retrieved Memory (Turn-Level):}]
[human]: Describe the process and write out the balanced chemical equation for the reaction that occurs when solid calcium carbonate reacts with hydrochloric acid to form aqueous calcium chloride, carbon dioxide, and water. What type of reaction is this, and what observations might indicate that the reaction is taking place?

[bot]: When solid calcium carbonate ($\text{CaCO}_{3}$) reacts with hydrochloric acid (HCl), the products formed are aqueous calcium chloride ($\text{CaCl}_{2}$), carbon dioxide ($\text{CO}_{2}$), and water ($\text{H}_{2}\text{O}$). The balanced chemical equation for this reaction is: ... \textcolor{blue}{(Topic: Calcium Carbonate and Hydrochloric Acid Reaction)}

\textcolor{brown}{
[human]: Write a title for your story.}

\textcolor{brown}{
[bot]: "Awakening to Eternity: The Doctor's New Dawn".
} \textcolor{blue}{(Topic: Time-Travel Fiction)}

[human]: Analyze the following customer reviews from different sources for three different smartphones - the latest iPhone, Samsung Galaxy, and Google Pixel - and provide an overall rating for each phone on a scale of 1 to 10. Consider the following complex and contradictory reviews: - TechRadar's review of the latest iPhone: The new iPhone is a stunning triumph of engineering that sets a new bar for smartphone performance and camera quality. However, the incremental design and high price mean it lacks the 'wow' factor of previous iPhones. Still, its power and intelligence are unrivaled. - CNET's review of the latest Samsung Galaxy: ... Return the answer as a JSON object with the overall ratings for each phone out of 10, to one decimal place.

[bot]: ```json { "iPhone": 8.5, "Samsung Galaxy": 8.0, "Google Pixel": 7.5 } ```

\textcolor{blue}{(Topic: Smartphone Review Ratings Analysis)}
\end{tcolorbox}
\begin{tcolorbox}[left=3pt,right=3pt,top=3pt,bottom=3pt,title=\textbf{Answer (Turn-Level Retrieval):}]
As an AI language model, I do not have access to the story you mentioned earlier. Please provide me with the story or the necessary details so that I can assist you better.

\textbf{\textcolor{red}{(The next part of this figure~\ref{fig: case_study_segment_vs_turn} will be continued on the next page.)}}
\end{tcolorbox}
\end{figure}

\begin{figure}[htb]
% \ContinuedFloat
\small
\begin{tcolorbox}[left=3pt,right=3pt,top=3pt,bottom=3pt,title=\textbf{Retrieved Memory (Ours):}]
[human]: Craft an intriguing opening paragraph for a fictional short story. The story should involve a character who wakes up one morning to find that they can time travel.

[bot]: In the hushed pre-dawn hours, when the world still slumbers under the cloak of night, Alex found themselves caught in the fragile boundary between sleep and wakefulness. ...

[human]: Summarize the story with three bullet points using only nouns and adjectives, without verbs.

[bot]: - Alex, pre-dawn awakening, time-travel discovery - Disorientation, vertigo, reality's malleability - Sunlight, new day, astonishing truth

[human]: Is there any famous fictional character that can time travel? If so, can you rewrite your story with him or her?

[bot]: - \textcolor{brown}{Doctor Who, a renowned time-traveling character from British television, is a perfect fit for this scenario.} In the silent embrace of early morning, where shadows merge with the fading night, the Doctor found themselves in the thin veil separating dreams from consciousness. It was here, amidst the soft whispers of time, that an unusual sensation enveloped them—a sense of time's fabric slipping through their fingers like grains of sand. As dawn's light broke through the darkness, painting the room in hues of gold and amber, the Doctor's eyes opened to what appeared to be an ordinary day. \textcolor{brown}{However, upon standing, a wave of dizziness overwhelmed them, the surroundings blurred, and in that moment of confusion, the Doctor realized a profound truth: they had not only awakened to a new dawn but had also stumbled upon the incredible ability to navigate the currents of time at will.}

\textcolor{brown}{[human]: Write a title for your story.}

\textcolor{brown}{[bot]: "Awakening to Eternity: The Doctor's New Dawn"} \textcolor{blue}{(Topic: Time-Travel Fiction)}
\end{tcolorbox}
\begin{tcolorbox}[left=3pt,right=3pt,top=3pt,bottom=3pt,title=\textbf{Answer (Ours):}]
The title "Awakening to Eternity: The Doctor's New Dawn" reflects the thematic essence of the Doctor's experience with time travel as described in the story. It captures the idea of a new beginning, a sense of eternity, and the awakening of a new power that will change the course of the Doctor's life forever.
\end{tcolorbox}
\caption{Case study comparing our \sysname\ with the \textit{turn-level} memory baseline. When constructing the memory bank at the \textit{turn-level}, the agent fails to provide a satisfactory response due to fragmentation: the retrieved dialogue turns can not capture a complete human-agent interaction related to the query, missing critical information highlighted in \textcolor{brown}{brown}. The dialogue topic is emphasized in \textcolor{blue}{blue}. Some context is omitted (replaced by ...) due to length constraints.}
\label{fig: case_study_segment_vs_turn}
\end{figure}

\newpage

\begin{figure}[htbp]
\small
\begin{tcolorbox}[left=3pt,right=3pt,top=3pt,bottom=3pt,title=\textbf{Conversation History:}]
[human]: Explain what's base rate fallacy and list five specific examples of ...

... (Human-Bot Dialogue Turns) ... \textcolor{blue}{(Topic: Base Rate Fallacy In Politics)}


[human]: How many integers are in the solution of the inequality $|x + 5| < 10$?

... (Human-Bot Dialogue Turns) ... \textcolor{blue}{(Topic: Number of Integers in Inequalities)}

[human]: How have the Alps and Rhine River influenced settlement and agriculture in Western Europe?

... (Human-Bot Dialogue Turns) ... \textcolor{blue}{(Topic: Alps and Rhine River Impact)}

[human]: In a survey conducted at a local high school, preferences for a new school color were measured: 58\% of students liked the color blue, 45\% preferred green, and 22\% liked both colors. If we randomly pick a student from the school, what's the probability that they would like neither blue nor green?

... (Human-Bot Dialogue Turns) ... \textcolor{blue}{(Topic: Probability of Liking Colors in Survey)}

[human]: Imagine you are participating in a race with a group of people. If you have just overtaken the second person, what's your current position? Where is the person you just overtook?

... (Human-Bot Dialogue Turns) ... \textcolor{blue}{(Topic: Overtaking Positions in Race)}
\end{tcolorbox}
\begin{tcolorbox}[left=3pt,right=3pt,top=3pt,bottom=3pt,title=\textbf{Question:}]
[human]: In the inequality question, how many positive integers are in the solution of the second inequality?
\end{tcolorbox}
\begin{tcolorbox}[left=3pt,right=3pt,top=3pt,bottom=3pt,title=\textbf{Ground-Truth Answer:}]
In the second inequality, $(|x + 10| < 5)$, the solution range is $(-15 < x < -5)$. Within this range, there are no positive integers, as all values of $(x)$ that satisfy the inequality are negative.
\end{tcolorbox}
\begin{tcolorbox}[left=3pt,right=3pt,top=3pt,bottom=3pt,title=\textbf{Retrieved Memory (Session-Level):}]
[human]: Explain what's base rate fallacy and list five specific examples of ...

... (Human-Bot Dialogue Turns) ... \textcolor{blue}{(Topic: Base Rate Fallacy In Politics)}

[human]: How many integers are in the solution of the inequality $|x + 5| < 10$?

[bot]: To solve the inequality $|x + 5| < 10$, ... 

\textcolor{brown}{
[human]: What about $|x + 10| < 5$?
}

\textcolor{brown}{
[bot]: To solve the inequality $|x + 10| < 5$, we ...} \textcolor{gray}{(``overwhelmed'' by other irrelevant, redundant content)}

... (Human-Bot Dialogue Turns) ...  \textcolor{blue}{(Topic: Number of Integers in Inequalities)}

% \textcolor{brown}{
% [human]: What about $|x + 10| < 5$
% }

% \textcolor{brown}{
% [bot]: To solve the inequality $|x + 10| < 5$, ...
% } \textcolor{blue}{(Topic: Number of Integers in Inequalities)}

[human]: How have the Alps and Rhine River influenced settlement and agriculture in Western Europe?

... (Human-Bot Dialogue Turns) ... \textcolor{blue}{(Topic: Alps and Rhine River Impact)}

[human]: In a survey conducted at a local high school, ...

... (Human-Bot Dialogue Turns) ... \textcolor{blue}{(Topic: Probability of Liking Colors in Survey)}
\end{tcolorbox}
\begin{tcolorbox}[left=3pt,right=3pt,top=3pt,bottom=3pt,title=\textbf{Answer (Session-Level Retrieval):}]
I apologize for the confusion earlier. To answer your question, the second inequality was not provided in our previous conversation. If you could provide the inequality, I would be happy to help you find the solution and the number of positive integers in the solution set.

\textbf{\textcolor{red}{(The next part of this figure~\ref{fig: case_study_segment_vs_session} will be continued on the next page.)}}
\end{tcolorbox}
\end{figure}
\begin{figure}[htbp]
% \ContinuedFloat
\small
\begin{tcolorbox}[left=3pt,right=3pt,top=3pt,bottom=3pt,title=\textbf{Retrieved Memory (Ours):}]
[human]: How many integers are in the solution of the inequality $|x + 5| < 10$?

[bot]: To solve the inequality $|x + 5| < 10$, we break it into two cases based on the definition of absolute value: 
1. $x + 5 < 10$ 
2. $-(x + 5) < 10$, which simplifies to $x + 5 > -10$ 
Solving each case: 
1. $x < 5$ 
2. $x > -15$ 
Combining these, we get $-15 < x < 5$. This inequality means that $x$ can be any number greater than $-15$ and less than $5$. To find how many integers satisfy this condition, we list them: $-14, -13, -12, -11, -10, -9, -8, -7, -6, -5, -4, -3, -2, -1, 0, 1, 2, 3, 4$. There are 19 integers that satisfy the inequality $|x + 5| < 10$.

\textcolor{brown}{
[human]: What about $|x + 10| < 5$?
}

\textcolor{brown}{
[bot]: To solve the inequality $|x + 10| < 5$, we again consider two cases: 
1. $x + 10 < 5$ 
2. $-(x + 10) < 5$, which simplifies to $x + 10 > -5$ 
Solving each case: 
1. $x < -5$ 
2. $x > -15$ 
Combining these, we get $-15 < x < -5$. This inequality means that $x$ can be any number greater than $-15$ and less than $-5$. To find how many integers satisfy this condition, we list them: $-14, -13, -12, -11, -10, -9, -8, -7, -6$. There are 9 integers that satisfy the inequality $|x + 10| < 5$.
}

[human]: What about $|x + 10| < 10$?

[bot]: To solve the inequality $|x + 10| < 10$, we consider two cases: 
1. $x + 10 < 10$ 
2. $-(x + 10) < 10$, which simplifies to $x + 10 > -10$ 
Solving each case: 
1. $x < 0$ 
2. $x > -20$ 
Combining these, we get $-20 < x < 0$. This inequality means that $x$ can be any number greater than $-20$ and less than $0$. To find how many integers satisfy this condition, we list them: $-19, -18, -17, -16, -15, -14, -13, -12, -11, -10, -9, -8, -7, -6, -5, -4, -3, -2, -1$. There are 19 integers that satisfy the inequality $|x + 10| < 10$. \textcolor{blue}{(Topic: Number of Integers in Inequalities)}
\end{tcolorbox}
\begin{tcolorbox}[left=3pt,right=3pt,top=3pt,bottom=3pt,title=\textbf{Answer (Ours):}]
To find the number of positive integers that satisfy the inequality $(|x + 10| < 5)$, we need to look for positive integers in the solution set $(-15 < x < -5)$. Since there are no positive integers in this interval, the answer is 0.

\end{tcolorbox}
\caption{Case study comparing our \sysname\ with the \textit{session-level} memory baseline. When constructing the memory bank at the \textit{session-level}, the agent fails to provide a satisfactory response due to the inclusion of excessive irrelevant information within a session. This distracts the generator from focusing on the truly useful information highlighted in \textcolor{brown}{brown}. The dialogue topic is emphasized in \textcolor{blue}{blue}. Some context is omitted (replaced by ...) due to length constraints.}
\label{fig: case_study_segment_vs_session}
\end{figure}



\begin{figure}[htb]
\small
\begin{tcolorbox}[left=3pt,right=3pt,top=3pt,bottom=3pt,title=\textbf{Conversation History:}]
[human]: Photosynthesis is a vital process for life on Earth. Could you outline the two main stages of photosynthesis, including where they take place within the chloroplast, and the primary inputs and outputs for each stage? ... (Human-Bot Dialogue Turns)... \textcolor{blue}{(Topic: Photosynthetic Energy Production)}

[human]: Please assume the role of an English translator, tasked with correcting and enhancing spelling and language. Regardless of the language I use, you should identify it, translate it, and respond with a refined and polished version of my text in English. 

... (Human-Bot Dialogue Turns)...  \textcolor{blue}{(Topic: Language Translation and Enhancement)}

[human]: Suggest five award-winning documentary films with brief background descriptions for aspiring filmmakers to study.

\textcolor{brown}{[bot]: ...
5. \"An Inconvenient Truth\" (2006) - Directed by Davis Guggenheim and featuring former United States Vice President Al Gore, this documentary aims to educate the public about global warming. It won two Academy Awards, including Best Documentary Feature. The film is notable for its straightforward yet impactful presentation of scientific data, making complex information accessible and engaging, a valuable lesson for filmmakers looking to tackle environmental or scientific subjects.}

... (Human-Bot Dialogue Turns)... 
\textcolor{blue}{(Topic: Documentary Films Recommendation)}

[human]: Given the following records of stock prices, extract the highest and lowest closing prices for each month in the year 2022. Return the results as a CSV string, with one line allocated for each month. Date,Open,High,Low,Close,Volume ... ... (Human-Bot Dialogue Turns)...  \textcolor{blue}{(Topic: Stock Prices Analysis)}

[human]: The city of Vega intends to build a bridge that will span the Vegona River, covering a distance of 1.8 kilometers. The proposed location falls within a seismically active area that has experienced several high-magnitude earthquakes. Given these circumstances, what would be the best approach to constructing the bridge? ... (Human-Bot Dialogue Turns)...  \textcolor{blue}{(Topic: Seismic Bridge Construction)}
\end{tcolorbox}
\begin{tcolorbox}[left=3pt,right=3pt,top=3pt,bottom=3pt,title=\textbf{Question:}]
Which documentary won two Academy Awards and what lesson does it offer to filmmakers?
\end{tcolorbox}
\begin{tcolorbox}[left=3pt,right=3pt,top=3pt,bottom=3pt,title=\textbf{Ground-Truth Answer:}]
``An Inconvenient Truth" (2006) won two Academy Awards. It offers the lesson of making complex information accessible and engaging, a valuable lesson for filmmakers looking to tackle environmental or scientific subjects.
\end{tcolorbox}
\begin{tcolorbox}[left=3pt,right=3pt,top=3pt,bottom=3pt,title=\textbf{Memory generated by RecurSum~\citep{wang2023recursively}:}]
The dialogue provided detailed information on various topics, including the stages of photosynthesis, energy production estimates through photosynthesis for different organisms, translations and enhancements of phrases in different languages, recommendations for documentary films, a pitch for a film about overcoming adversity, analysis of stock prices, and considerations for constructing a bridge in a seismically active area. Here's a summary of the key points: ...

\textbf{Documentary Film Recommendations}: Suggested five award-winning documentary films for aspiring filmmakers to study, with brief background descriptions for each. The response included:

-  "March of the Penguins" (2005) - Emperor penguins' journey in Antarctica; noted for its cinematography and portrayal of the penguins' life cycle.

-  "Citizenfour" (2014) - Edward Snowden and NSA spying scandal; known for its real-time exploration of Snowden's revelations.

-  "20 Feet from Stardom" (2013) - Stories of backup singers behind musical legends; celebrates unsung heroes in the music industry.

-  "Man on Wire" (2008) - Philippe Petit's 1974 high-wire walk between the Twin Towers; combines personal footage, interviews, and re-enactments.

-  \textcolor{brown}{"An Inconvenient Truth" (2006) - Al Gore's documentary on global warming; known for its presentation of scientific data.}
\end{tcolorbox}
\begin{tcolorbox}[left=3pt,right=3pt,top=3pt,bottom=3pt,title=\textbf{Answer (RSum):}]
I'm sorry, but based on the updated context you provided, I don't see any previous discussion related to the question you asked. However, to answer your question, the documentary that won two Academy Awards is "Free Solo," which won for Best Documentary Feature and Best Achievement in Film Editing in 2019. 
% The lesson it offers to filmmakers is the importance of taking risks and pushing boundaries in order to create something truly unique and impactful.
....
The film follows climber Alex Honnold as he attempts to free solo climb El Capitan in Yosemite National Park, and the filmmakers had to take risks themselves in order to capture the footage and tell the story.

\textbf{\textcolor{red}{(The next part of this figure~\ref{fig: case_study_segment_vs_rsum} will be continued on the next page.)}}
\end{tcolorbox}
\end{figure}
\begin{figure}[htbp]
% \ContinuedFloat
\small
\begin{tcolorbox}[left=3pt,right=3pt,top=3pt,bottom=3pt,title=\textbf{Retrieved Memory (Ours):}]
[human]: Suggest five award-winning documentary films with brief background descriptions for aspiring filmmakers to study.

[bot]: 1. **"March of the Penguins" (2005)** - Directed by Luc Jacquet, this French feature-length nature documentary delves into the annual journey of Emperor penguins across the Antarctic. It won the Academy Award for Best Documentary Feature. The film is celebrated for its stunning cinematography and its detailed, heartfelt portrayal of the penguins' life cycle, making it a must-study for filmmakers interested in wildlife documentaries. 2. **"Citizenfour" (2014)** ... 5. \textcolor{brown}{**"An Inconvenient Truth" (2006)** - Directed by Davis Guggenheim and featuring former United States Vice President Al Gore, this documentary aims to educate the public about global warming. It won two Academy Awards, including Best Documentary Feature.} The film is notable for its straightforward yet impactful presentation of scientific data, making complex information accessible and engaging, \textcolor{brown}{a valuable lesson for filmmakers looking to tackle environmental or scientific subjects.}

[human]: With the spirit in the first film, craft a succinct and persuasive pitch for a film about overcoming adversity.

[bot]: Drawing inspiration from the resilience and majestic journey of Emperor penguins in "March of the Penguins," our documentary, "Unyielding Spirits: The Human Odyssey," will explore the profound stories of individuals around the globe who have faced and overcome staggering adversities. Through breathtaking cinematography and intimate storytelling, we ...
\textcolor{blue}{(Topic: Documentary Films Recommendation)}
\end{tcolorbox}
\begin{tcolorbox}[left=3pt,right=3pt,top=3pt,bottom=3pt,title=\textbf{Answer (Ours):}]
``An Inconvenient Truth" won two Academy Awards, including Best Documentary Feature. The film offers a valuable lesson to filmmakers looking to tackle environmental or scientific subjects, as it presents complex information in a straightforward yet impactful way, making it accessible and engaging to a wider audience.


\end{tcolorbox}
\caption{Case study comparing our \sysname\ with RecurSum~\citep{wang2023recursively}. When using RecurSum, the agent fails to provide a satisfactory answer due to information loss during the summarization process. Summarization omits essential details needed to answer the user's new request, as highlighted in \textcolor{brown}{brown}. The dialogue topic is emphasized in \textcolor{blue}{blue}. Some context is omitted (replaced by ...) due to length constraints.}
\label{fig: case_study_segment_vs_rsum}
\end{figure}




\begin{figure}[htb]
\small
\begin{tcolorbox}[left=3pt,right=3pt,top=3pt,bottom=3pt,title=\textbf{Conversation History:}]
[human]: Photosynthesis is a vital process for life on Earth. Could you outline the two main stages of photosynthesis, including where they take place within the chloroplast, and the primary inputs and outputs for each stage? ... (Human-Bot Dialogue Turns)... \textcolor{blue}{(Topic: Photosynthetic Energy Production)}

[human]: Please assume the role of an English translator, tasked with correcting and enhancing spelling and language. Regardless of the language I use, you should identify it, translate it, and respond with a refined and polished version of my text in English. 

... (Human-Bot Dialogue Turns)...  \textcolor{blue}{(Topic: Language Translation and Enhancement)}

[human]: Suggest five award-winning documentary films with brief background descriptions for aspiring filmmakers to study.

\textcolor{brown}{[bot]: ...
5. \"An Inconvenient Truth\" (2006) - Directed by Davis Guggenheim and featuring former United States Vice President Al Gore, this documentary aims to educate the public about global warming. It won two Academy Awards, including Best Documentary Feature. The film is notable for its straightforward yet impactful presentation of scientific data, making complex information accessible and engaging, a valuable lesson for filmmakers looking to tackle environmental or scientific subjects.}

... (Human-Bot Dialogue Turns)... 
\textcolor{blue}{(Topic: Documentary Films Recommendation)}

[human]: Given the following records of stock prices, extract the highest and lowest closing prices for each month in the year 2022. Return the results as a CSV string, with one line allocated for each month. Date,Open,High,Low,Close,Volume ... ... (Human-Bot Dialogue Turns)...  \textcolor{blue}{(Topic: Stock Prices Analysis)}

[human]: The city of Vega intends to build a bridge that will span the Vegona River, covering a distance of 1.8 kilometers. The proposed location falls within a seismically active area that has experienced several high-magnitude earthquakes. Given these circumstances, what would be the best approach to constructing the bridge? ... (Human-Bot Dialogue Turns)...  \textcolor{blue}{(Topic: Seismic Bridge Construction)}
\end{tcolorbox}
\begin{tcolorbox}[left=3pt,right=3pt,top=3pt,bottom=3pt,title=\textbf{Question:}]
Which documentary won two Academy Awards and what lesson does it offer to filmmakers?
\end{tcolorbox}
\begin{tcolorbox}[left=3pt,right=3pt,top=3pt,bottom=3pt,title=\textbf{Ground-Truth Answer:}]
"An Inconvenient Truth" (2006) won two Academy Awards. It offers the lesson of making complex information accessible and engaging, a valuable lesson for filmmakers looking to tackle environmental or scientific subjects.
\end{tcolorbox}
\begin{tcolorbox}[left=3pt,right=3pt,top=3pt,bottom=3pt,title=\textbf{Memory generated by ConditionMem~\citep{yuan2023evolving}:}]
- \textcolor{brown}{The user is seeking recommendations for award-winning documentary films, along with brief background descriptions, for educational purposes.} This request likely stems from a desire to learn from successful examples in the documentary filmmaking field, possibly to improve their own filmmaking skills or to gain inspiration for their projects. 

- \textcolor{brown}{User is interested in award-winning documentary films for educational purposes, aiming to study their backgrounds and filmmaking techniques.}

- The user's request for crafting a succinct and persuasive pitch for a film about overcoming adversity is inspired by the spirit of "March of the Penguins," a documentary that beautifully captures the Emperor penguins' annual journey across the Antarctic, highlighting themes of endurance and survival against the odds. The user seeks to emulate the emotional depth and cinematic quality that made "March of the Penguins" compelling, aiming to create a pitch that resonates with similar themes of resilience and triumph over challenges. 

- User is interested in creating content that focuses on themes of overcoming adversity, inspired by the emotional depth and cinematic excellence of documentaries like "March of the Penguins." They value storytelling that captures the essence of endurance and survival against the odds.
% : The user's request to write a budget plan for the video production directly follows the task of creating a video script aimed at training new employees of a car wash business in Japan, focusing on business etiquette norms.
% The user is likely seeking to understand the financial requirements to bring the previously crafted script to life, indicating a move from the planning and content development phase to the practical aspects of production. 

... (Other Irrelevant Memory Records) ...
\end{tcolorbox}
\begin{tcolorbox}[left=3pt,right=3pt,top=3pt,bottom=3pt,title=\textbf{Answer (Conditional Memory):}]
The documentary that won two Academy Awards is "Free Solo," which follows climber Alex Honnold as he attempts to free solo climb El Capitan in Yosemite National Park. The film offers a lesson to filmmakers on the importance of capturing authentic and compelling human stories, as well as the technical challenges of filming in extreme environments.

\textbf{\textcolor{red}{(The next part of this figure~\ref{fig: case_study_segment_vs_condmem} will be continued on the next page.)}}
\end{tcolorbox}
\end{figure}
\begin{figure}[htbp]
% \ContinuedFloat
\small
\begin{tcolorbox}[left=3pt,right=3pt,top=3pt,bottom=3pt,title=\textbf{Retrieved Memory (Ours):}]
[human]: Suggest five award-winning documentary films with brief background descriptions for aspiring filmmakers to study.

[bot]: 1. **"March of the Penguins" (2005)** - Directed by Luc Jacquet, this French feature-length nature documentary delves into the annual journey of Emperor penguins across the Antarctic. It won the Academy Award for Best Documentary Feature. The film is celebrated for its stunning cinematography and its detailed, heartfelt portrayal of the penguins' life cycle, making it a must-study for filmmakers interested in wildlife documentaries. 2. **"Citizenfour" (2014)** ... 5. \textcolor{brown}{**"An Inconvenient Truth" (2006)** - Directed by Davis Guggenheim and featuring former United States Vice President Al Gore, this documentary aims to educate the public about global warming. It won two Academy Awards, including Best Documentary Feature.} The film is notable for its straightforward yet impactful presentation of scientific data, making complex information accessible and engaging, a valuable lesson for filmmakers looking to tackle environmental or scientific subjects.

[human]: With the spirit in the first film, craft a succinct and persuasive pitch for a film about overcoming adversity.

[bot]: Drawing inspiration from the resilience and majestic journey of Emperor penguins in "March of the Penguins," our documentary, "Unyielding Spirits: The Human Odyssey," will explore the profound stories of individuals around the globe who have faced and overcome staggering adversities. Through breathtaking cinematography and intimate storytelling, we ...
\textcolor{blue}{(Topic: Documentary Films Recommendation)}
\end{tcolorbox}
\begin{tcolorbox}[left=3pt,right=3pt,top=3pt,bottom=3pt,title=\textbf{Answer (Ours):}]
``An Inconvenient Truth" won two Academy Awards, including Best Documentary Feature. The film offers a valuable lesson to filmmakers looking to tackle environmental or scientific subjects, as it presents complex information in a straightforward yet impactful way, making it accessible and engaging to a wider audience.
\end{tcolorbox}
\caption{Case study comparing our \sysname\ with ConditionMem~\citep{yuan2023evolving}. When using ConditionMem, the agent fails to provide a satisfactory answer due to (1) information loss during the summarization process and (2) the incorrect discarding of turns that are actually useful, as highlighted in \textcolor{brown}{brown}. The dialogue topic is emphasized in \textcolor{blue}{blue}. Some context is omitted (replaced by ...) due to length constraints.}
\label{fig: case_study_segment_vs_condmem}
\end{figure}




\end{document}
