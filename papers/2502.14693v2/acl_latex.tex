% This must be in the first 5 lines to tell arXiv to use pdfLaTeX, which is strongly recommended.
\pdfoutput=1
% In particular, the hyperref package requires pdfLaTeX in order to break URLs across lines.

\documentclass[11pt]{article}

% Change "review" to "final" to generate the final (sometimes called camera-ready) version.
% Change to "preprint" to generate a non-anonymous version with page numbers.
\usepackage[final]{acl}

% Standard package includes
\usepackage{times}
\usepackage{latexsym}
\usepackage{tcolorbox}
\usepackage{url}
\usepackage{listings}
\usepackage{xparse}
\lstdefinestyle{pythonstyle}{
  language=Python,
  basicstyle=\ttfamily\small,
  keywordstyle=\bfseries\color{blue},
  commentstyle=\itshape\color{green!60!black},
  stringstyle=\color{red!70!black},
  showstringspaces=false,
  breaklines=true,
  frame=single
}

\lstdefinestyle{txtfile}{
    basicstyle=\scriptsize\ttfamily,
    backgroundcolor=\color{white},
    frame=single,
    rulecolor=\color{black!30},
    stringstyle=\color{red},
    numbers=left,
    numberstyle=\tiny\color{gray!50},
    numbersep=5pt,
    breaklines=true,
    breakatwhitespace=true,
    showstringspaces=false,
    tabsize=4,
    captionpos=b
}

%  more packages
\usepackage{tabularray}
\usepackage{color}
\usepackage{booktabs}
\usepackage{siunitx}
\usepackage{graphicx}
\usepackage{array}

\usepackage{listings}
\usepackage{xcolor}
\usepackage{algorithm}
\usepackage{algorithmic}
\usepackage{wrapfig}
\usepackage{tabularx}
\usepackage{subcaption}
% \usepackage{authblk}
\usepackage{pifont}% http://ctan.org/pkg/pifont

% For proper rendering and hyphenation of words containing Latin characters (including in bib files)
\usepackage[T1]{fontenc}
% For Vietnamese characters
% \usepackage[T5]{fontenc}
% See https://www.latex-project.org/help/documentation/encguide.pdf for other character sets

% This assumes your files are encoded as UTF8
\usepackage[utf8]{inputenc}
\usepackage{amsmath}

% This is not strictly necessary, and may be commented out,
% but it will improve the layout of the manuscript,
% and will typically save some space.
\usepackage{microtype}

% This is also not strictly necessary, and may be commented out.
% However, it will improve the aesthetics of text in
% the typewriter font.
\usepackage{inconsolata}

%Including images in your LaTeX document requires adding
%additional package(s)
\usepackage{graphicx}

% Added packages
\usepackage{amsfonts}
\usepackage{booktabs}
\usepackage{multirow}

% If the title and author information does not fit in the area allocated, uncomment the following
%
%\setlength\titlebox{<dim>}
%
% and set <dim> to something 5cm or larger.

\title{I-MCTS: Enhancing Agentic AutoML via Introspective Monte Carlo Tree Search}

% Author information can be set in various styles:
% For several authors from the same institution:
% \author{Author 1 \and ... \and Author n \\
%         Address line \\ ... \\ Address line}
% if the names do not fit well on one line use
%         Author 1 \\ {\bf Author 2} \\ ... \\ {\bf Author n} \\
% For authors from different institutions:
% \author{Author 1 \\ Address line \\  ... \\ Address line
%         \And  ... \And
%         Author n \\ Address line \\ ... \\ Address line}
% To start a separate ``row'' of authors use \AND, as in
% \author{Author 1 \\ Address line \\  ... \\ Address line
%         \AND
%         Author 2 \\ Address line \\ ... \\ Address line \And
%         Author 3 \\ Address line \\ ... \\ Address line}

% \author{First Author \\
%   Affiliation / Address line 1 \\
%   Affiliation / Address line 2 \\
%   Affiliation / Address line 3 \\
%   \texttt{email@domain} \\\And
%   Second Author \\
%   Affiliation / Address line 1 \\
%   Affiliation / Address line 2 \\
%   Affiliation / Address line 3 \\
%   \texttt{email@domain} \\}

\author{
 \textbf{Zujie Liang\textsuperscript{1}}\quad
 \textbf{Feng Wei\textsuperscript{1}}\quad
 \textbf{Wujiang Xu\textsuperscript{2}}\quad
 \textbf{Lin Chen\textsuperscript{1}}\quad
 \textbf{Yuxi Qian\textsuperscript{1}}\quad
 \textbf{Xinhui Wu\textsuperscript{1}}
\\
\\
 \textsuperscript{1}Ant Group\;\;\;\;\;\;    
 \textsuperscript{2}Rutgers University
% \\
}

% \author{
%  \textbf{First Author\textsuperscript{1}},
%  \textbf{Second Author\textsuperscript{1,2}},
%  \textbf{Third T. Author\textsuperscript{1}},
%  \textbf{Fourth Author\textsuperscript{1}},
% \\
%  \textbf{Fifth Author\textsuperscript{1,2}},
%  \textbf{Sixth Author\textsuperscript{1}},
%  \textbf{Seventh Author\textsuperscript{1}},
%  \textbf{Eighth Author \textsuperscript{1,2,3,4}},
% \\
%  \textbf{Ninth Author\textsuperscript{1}},
%  \textbf{Tenth Author\textsuperscript{1}},
%  \textbf{Eleventh E. Author\textsuperscript{1,2,3,4,5}},
%  \textbf{Twelfth Author\textsuperscript{1}},
% \\
%  \textbf{Thirteenth Author\textsuperscript{3}},
%  \textbf{Fourteenth F. Author\textsuperscript{2,4}},
%  \textbf{Fifteenth Author\textsuperscript{1}},
%  \textbf{Sixteenth Author\textsuperscript{1}},
% \\
%  \textbf{Seventeenth S. Author\textsuperscript{4,5}},
%  \textbf{Eighteenth Author\textsuperscript{3,4}},
%  \textbf{Nineteenth N. Author\textsuperscript{2,5}},
%  \textbf{Twentieth Author\textsuperscript{1}}
% \\
% \\
%  \textsuperscript{1}Affiliation 1,
%  \textsuperscript{2}Affiliation 2,
%  \textsuperscript{3}Affiliation 3,
%  \textsuperscript{4}Affiliation 4,
%  \textsuperscript{5}Affiliation 5
% \\
%  \small{
%    \textbf{Correspondence:} \href{mailto:email@domain}{email@domain}
%  }
% }

\newcommand{\methodname}{\textsc{I-MCTS}}

\begin{document}
\maketitle


\begin{abstract}
Recent advancements in large language models (LLMs) have shown remarkable potential in automating machine learning tasks. 
However, existing LLM-based agents often struggle with low-diversity and suboptimal code generation. 
While recent work~\cite{chi2024sela} has introduced Monte Carlo Tree Search (MCTS) to address these issues, limitations persist in the quality and diversity of thoughts generated, as well as in the scalar value feedback mechanisms used for node selection. 
In this study, we introduce Introspective Monte Carlo Tree Search (\textbf{\textit{I-MCTS}}), a novel approach that iteratively expands tree nodes through an introspective process that meticulously analyzes solutions and results from parent and sibling nodes. 
This facilitates a continuous refinement of the node in the search tree, thereby enhancing the overall decision-making process.
Furthermore, we integrate a Large Language Model (LLM)-based value model to facilitate direct evaluation of each node's solution prior to conducting comprehensive computational rollouts. 
A hybrid rewarding mechanism is implemented to seamlessly transition the Q-value from LLM-estimated scores to actual performance scores. 
This allows higher-quality nodes to be traversed earlier.
Applied to the various ML tasks, our approach demonstrates a
6\% absolute improvement in performance compared to the strong open-source AutoML agents, showcasing its effectiveness in enhancing agentic AutoML systems\footnote{Resource available at \url{https://github.com/jokieleung/I-MCTS}}.
\end{abstract}

\section{Introduction}
\label{sec:intro}


Recent advances in large language models (LLMs) have opened new frontiers in automating machine learning (AutoML) tasks~\cite{huang2024mlagentbench,chan2024mle}.
Compared to the traditional AutoML frameworks~\cite{feurer2015efficient, tang2024autogluon}, the LLM-based agents~\cite{sun2024survey} emerge as promising direction 
due to their remarkable capabilities on code generation~\cite{jimenez2023swe}, neural architecture design~\cite{zheng2023can} and model training~\cite{chi2024sela}. 
Overall, these LLM agent-based AutoML systems~\cite{Schmidt_Wu_Jiang,hong2024data,li2024autokaggle,chi2024sela} typically input with a natural language description on the dataset and the problem, 
after which the system directly generates a solution in an end-to-end manner. 
While recent works by \citet{Schmidt_Wu_Jiang} and \citet{hong2024data} have made significant strides in automating the machine learning workflow, replicating the adaptive and strategic behavior of expert human engineers remains a significant challenge. 
The primary limitation lies in their search processes, which typically involve only a single pass or trial. 
This constraint significantly hinders the generation of diverse and highly optimized workflows, a capability that human experts excel at through iterative refinement and strategic decision-making.
Recent work by ~\citet{chi2024sela} introduced Tree-Search Enhanced LLM Agents (SELA), leveraging Monte Carlo Tree Search (MCTS) to expand the search space. 
However, the search space is constrained by its reliance on a pre-generated, fixed insight pool derived from the initial problem description and dataset information.
This static nature inherently limits the diversity and adaptability of the search tree.
Moreover, SELA encounters significant limitations in improving the overall quality of solutions when relying exclusively on scalar experimental performance feedback. 
This reliance on simplistic scalar feedback mechanisms impedes the efficient identification and navigation of optimal pathways in complex machine learning tasks, where flexible re-assessment and adaptive strategies are usually needed.

To address these challenges, we present \textbf{I}ntrospective \textbf{M}onte \textbf{C}arlo \textbf{T}ree \textbf{S}earch (\textbf{\textit{I-MCTS}}), a novel framework that introduces two key innovations. First, our \textit{introspective node expansion} dynamically generates high-quality thought nodes through explicit analysis of parent and sibling node states. 
By incorporating reflective reasoning and feedback about prior solutions and their outcomes, this approach enables continuous quality improvement of the search tree nodes. 
Second, we develop a \textit{hybrid rewarding mechanism} that combines: 1) LLM-estimated evaluations predicting node potential through a comprehensive machine learning evaluation criteria, 
and 
2) actual performance scores on the development set from computational rollouts. 
Our adaptive reward blending strategy smoothly transitions Q-value from LLM-estimated values to actual values across search iterations.
This facilitates higher-quality nodes to be traversed earlier.

Overall, the principal contributions of this work are threefold:

\begin{itemize}
% \item We identify and formalize critical limitations in existing MCTS-based AutoML agents, particularly the static nature of thought generation and scalar value feedback mechanisms
\item We introduce I-MCTS, an innovative approach for agentic AutoML that incorporates an introspective node expansion process and a hybrid reward mechanism. This enhances both the quality and efficiency of tree searches in AutoML workflows.
    
\item Extensive experiments across diverse ML tasks demonstrate our method's superiority, achieving 6\% absolute performance gains over state-of-the-art baselines while maintaining computational efficiency.
\end{itemize}


% \section{Method}
\section{Introspective Monte Carlo Tree Search (I-MCTS)}
\label{sec:method}

As illustrated in Figure~\ref{fig:pipline}, our approach consists of two key components: a search module I-MCTS, and an LLM agent as the experiment executor. 
The Introspective Monte Carlo Tree Search (I-MCTS), builds upon the foundation of traditional MCTS but introduces two key innovations: 
(1) introspective node expansion through reflective solution generation, and (2) a hybrid reward mechanism combining LLM-estimated evaluations with empirical performance scores. 
These components work in tandem to enhance the quality and diversity of thought nodes while improving the efficiency of the search process. 
During each cycle, the selected path is passed to the LLM agent, which plans, codes, executes, and introspects the experiment, providing both scalar and verbal feedback to refine future searches. 
This iterative process continues until the termination criterion is met. 

\subsection{Preliminary}
We formalize the automated machine learning (AutoML) problem as a search process over possible experimental configurations. 
Given a problem description $p$ and dataset $D$ with metadata $d$, the objective is to identify an optimal pipeline configuration $c^* \in \mathcal{C}$ that maximizes a performance metric $s$ on development data. 
The search space $\mathcal{C}$ consists of configurations combining insights $\lambda_i^\tau$ across $T$ predefined stages of the ML workflow: $\tau_1$ (Exploratory Data Analysis), $\tau_2$ (Data Preprocessing), $\tau_3$ (Feature Engineering), $\tau_4$ (Model Training), and $\tau_5$ (Model Evaluation).

The LLM agent $E$ conducts each trial by building practical experimental pipelines from natural language requirements. 
Given an experiment configuration $c$, the agent produces a detailed plan $E_{\text{plan}}$ that consists of a sequence of task instructions $I^{\tau \in T}$ corresponding to each stage of the machine learning process. Next, the agent writes and executes code $\sigma^\tau$ for each task $\tau$ based on the respective instruction and gets the final execution score $s$. 
The complete set of code outputs $\sigma^{\tau \in T}$ is concatenated into a full solution $\sigma_{sol}$ to address the problem. This phase is referred to as $E_{\text{code \& execute}}$.
\begin{align}
    E_{\text{plan}}(p, d, c, LLM) & \rightarrow I^{\tau \in T} \\
    E_{\text{code \& execute}}(I^{\tau \in T}, D, LLM) & \rightarrow (\sigma^{\tau \in T}, s)
\end{align}

\begin{figure}[!t]
    \centering
    \includegraphics[width=0.95\linewidth]{figures/I-MCTS.pdf}
    \caption{I-MCTS architecture featuring (a) Introspective node expansion through parent/sibling analysis, (b) Hybrid reward calculation combining LLM predictions and empirical scores. The red arrows indicate the introspective feedback loop that continuously improves node quality.}
    \label{fig:pipline}
\end{figure}

\subsection{Introspective Node Expansion}
\label{ssec:introspective}

Unlike static insight pools in prior work~\cite{chi2024sela}, 
I-MCTS utilizes the introspective expansion mechanism that dynamically generates high-quality thought nodes by leveraging solutions and results from parent and sibling nodes. 
This design mirrors the way human expert engineers backtrack and refine their solution dynamically, ensuring the agent can iteratively
improve its decision-making process based on past experiences.
Given a parent node $x_{\text{parent}}$ and its existing sibling nodes $x_{\text{sibling}}$, the introspective expansion process generates a new node $x_{\text{child}}$ by introspecting the solutions and results of these previous nodes. 
Specifically, the introspection mechanism use an LLM to evaluate the solutions $\sigma_{\text{sol}}(x_{\text{parent}})$ and $\sigma_{\text{sol}}(x_{\text{sibling}})$ associated with the parent and sibling nodes and identify strengths, weaknesses, issues and potential improvements in the solutions.
Then, the LLM generates a fine-grained and customized insight $\lambda_{\text{new}}$ that addresses the identified issues or builds upon the strengths. 
This insight is tailored to the current stage $\tau$ of the machine learning pipeline. The new insight $\lambda_{\text{new}}$ is used to create a child node $x_{\text{child}}$, which inherits attributes from the parent node while incorporating the refined insight. 
The child node is then added to the search tree.
This introspection process ensures that each new node is dynamically generated through a thoughtful analysis of prior solutions, leading to a continuous improvement in the quality of the search tree.
Also, the verbalized introspection feedback addresses the solely scalar feedback limitation.


\subsection{Hybrid Reward Mechanism}
\label{ssec:hybrid}

The primary objective of the evaluation phase is to assess the reward for the current node.
For machine learning tasks, a conventional approach~\cite{Schmidt_Wu_Jiang,chi2024sela} involves utilizing the performance metric on the development set as the reward. 
To derive a performance score of the given node involves rolling out from an intermediate state to a terminal state, which is quite costly since processes such as model training cost significant computational time. 
This limits the efficiency of node exploration.
To address these challenges, we introduce a hybrid reward mechanism for node evaluation which combines LLM-estimated evaluations with actual performance scores to guide the search process more efficiently. 
% \noindent\textbf{LLM-Estimated Evaluations}
% \label{subsubsec:llm-estimated}
For each node $x$, we employ an LLM-based value model $M_{\text{value}}$ to predict its potential performance. The value model takes as input the solution $\sigma_{\text{sol}}(x)$ and outputs an estimated score $s_{\text{LLM}}(x)$. This score is based on a comprehensive set of machine learning evaluation criteria, which is detailed in Appendix~\ref{app:llm_evaluation_schema}.
The LLM-estimated evaluation allows us to assess the quality of a node's solution before conducting computationally expensive rollouts. This early evaluation helps prioritize nodes with higher potential, improving the efficiency of the search process.
% \noindent\textbf{Actual Performance Scores}
% \label{subsubsec:actual-scores}
After the ``simulation"" of a node $x$, we obtain the actual performance score $s_{\text{actual}}(x)$ based on the development set, which reflects the true performance of the node's solution.
% \noindent\textbf{Adaptive Reward Blending}
% \label{subsubsec:adaptive-blending}
To smoothly transition from LLM-estimated values to actual performance scores, we implement an adaptive reward blending strategy. The Q-value $Q(x)$ for a node $x$ is updated as follows:

\begin{align}
    Q(x) = \alpha(x) \cdot s_{\text{LLM}}(x) + (1 - \alpha(x)) \cdot s_{\text{actual}}(x)
\end{align}

where $\alpha(x) = \frac{\gamma}{n_{\text{visits}(x)}+\gamma}$, a blending factor that decreases over the visit count increase, where $\gamma$ is a controlling constant.
This adaptive blending ensures that higher-quality nodes are traversed earlier in the search process, while still incorporating the true performance feedback as it becomes available.

\subsection{Tree Search Process}
\label{ssec:adaptive}

The overall search process can be summarized as:

\paragraph{Selection} At each iteration, we select node according to the UCT formula as:

\begin{align}
    \text{UCT}(x) = \frac{Q(x)}{n(x)} + \alpha_{\text{explore}} \sqrt{\frac{\ln n_{\text{visits}}(x_{\text{parent}})}{n(x)}}
\end{align}

\paragraph{Expansion} Child nodes are generated through the introspective process described in Section~\ref{ssec:introspective}, creating a dynamic search space that evolves based on simulation feedback.

\paragraph{Simulation} Each rollout executes the full pipeline while caching intermediate results. The LLM value model provides real-time estimates $v_{\text{LLM}}$ before computational execution.

\paragraph{Backpropagation} 
Upon simulation completion, the Q value is propagated backwards through the tree structure. 
This process updates the value and visit count of each parent node, from the simulated node to the root. Consequently, nodes associated with superior solutions receive higher priority in subsequent iterations.
This allows nodes representing more promising solutions to be prioritized in future rollouts.
Similar to \citet{chi2024sela}, I-MCTS implements a state-saving mechanism that caches the stage code for each node, which allows it to reuse previously generated code when a new configuration shares components with existing ones.


\section{Experiment}

We follow \citet{chi2024sela} to benchmark our method on 20 machine learning datasets, including 13 classification tasks and 7 regression tasks from the AutoML Benchmark (AMLB) ~\cite{gijsbers2024amlb} and Kaggle Competitions.
All datasets are split into training, validation, and test sets with a 6:2:2 ratio, details refer to Appendix~\ref{app:dataests}.
Each dataset is run three times using our method.
We also use the Normalized Score (NS) defined in Appendix ~\ref{app:metric}.
We adopt Qwen2.5-72B-Instruct~\cite{yang2024qwen2} as our LLM.
The temperature parameter is set to 0.2.
$\gamma$ is set to 0.2, while $\alpha_{\text{explore}}$ is set to 2.
Five baselines are included for comparison, including AutoGluon~\cite{tang2024autogluon}, AutoSklearn~\cite{feurer2022auto}, AIDE~\cite{Schmidt_Wu_Jiang}, Data Interpreter~\cite{hong2024data}, SELA~\cite{chi2024sela}.
We directly used the results reported in \citet{chi2024sela}.
Experiments are conducted based on the MetaGPT~\cite{hong2024metagpt} framework\footnote{\url{https://github.com/geekan/MetaGPT}}.
% ~\footnote{https://github.com/geekan/MetaGPT/tree/main/metagpt/ext/sela}.

\subsection{Main Results}

\begin{table}[t]
\centering
\resizebox{0.5\textwidth}{!}{
    \begin{tabular}{lccc}
        \toprule
        \textbf{Method} & \textbf{Top1 Rate} \%  & \textbf{Avg. NS} \% $\uparrow$ & \textbf{Avg. Best NS} \% $\uparrow$  \\
        \midrule
        AutoGluon   & 5.0 & 53.2 & 53.2 \\
        AutoSklearn  & 25.0 & 46.1 & 47.5  \\
        AIDE  & 5.0 & 47.1 & 51.8  \\
        Data Interpreter & 0.0 & 47.4 & 50.2 \\
        SELA & 20.0 & 53.3 & 54.7 \\
        \textbf{\methodname{}} & \textbf{45.0} & \textbf{58.6} & \textbf{59.8} \\
        \bottomrule
    \end{tabular}
    }
    \caption{Results of each AutoML framework on 20 tabular datasets. The ``Top1 Rate" column represents the rate of datasets where the method produces the best predictions across methods.}
    \label{table:main}
\end{table}

The experimental results demonstrate the superior performance of the proposed I-MCTS approach compared to other strong AutoML baselines. As shown in Table \ref{table:full-main-results}, I-MCTS achieves the highest overall normalized score (NS) of 58.6\% on average and 59.8\% for the best predictions across 20 diverse tabular datasets.
These results validate the effectiveness of I-MCTS's unique introspective node expansion process, which enables continuous solution refinement through meticulous analysis of parent and sibling nodes.
Also, the integration of an LLM-based value model and hybrid rewarding mechanism further contributes to the algorithm's success by facilitating early identification and traversal of high-quality nodes.
We include detailed results of each method in Appendix~\ref{app:detailed_result}.

\subsection{Ablation Study}

We follow \citet{chi2024sela} to use boston, colleges, credit-g, Click\_prediction\_small, GesturePhaseSegmentationProcessed, and mfeat-factors to dataset for ablation study. 
We systematically evaluate the contribution of each component within our proposed Introspective Monte Carlo Tree Search (I-MCTS) framework. The results, as presented in Table \ref{table:ablation_study}, illustrate the incremental benefits of incorporating both the Introspective Node Expansion (INE) and the hybrid reward mechanism (HRM).

\begin{table}[t]
\centering
\resizebox{0.4\textwidth}{!}{
% \begin{tabularx}{1.12\textwidth}
\begin{tabular}{lc}
    \toprule
    \textbf{Method} & \textbf{Avg. NS $\uparrow$}  \\
    \midrule
    Data Interpreter & 56.4 \\
    SELA (Random Search) & 58.6  \\
    SELA (MCTS) & 60.9   \\
    \methodname{} (w/o INE) & \textbf{61.1}  \\
    \methodname{} (w/o HRM) & \textbf{66.2}  \\
    \textbf{\methodname{} } & \textbf{66.8}  \\
    \bottomrule
\end{tabular}
}
\caption{Ablation Study for each search setting on the selected 6 datasets. ``\methodname{} (w/o INE)" means ``without "Introspective Node Expansion", while ``\methodname{} (w/o HRM)" means ``without "hybrid reward mechanism".}
\label{table:ablation_study}
\end{table}


\subsection{Case Study}

To demonstrate the superiority of our Introspective Monte Carlo Tree Search (I-MCTS) approach, we conduct a comprehensive visualization analysis of the tree search process using the \textit{\textbf{kick}} dataset. 
As show in Appendix~\ref{app:case},
our empirical results reveal that I-MCTS effectively leverages the introspective information from previous nodes to generate task-specific and actionable insights, while SELA exhibit significant homogeneity and lack specificity.
Furthermore, our hybrid reward mechanism enhances the exploration efficiency, enabling more effective identification of high-quality nodes within the same computational budget.





\section{Conclusion}

In this paper, we introduced \textbf{I}ntrospective \textbf{M}onte \textbf{C}arlo \textbf{T}ree \textbf{S}earch (\textbf{\textit{I-MCTS}}), a novel approach to enhance AutoML Agents. 
The method addresses key limitations in existing Tree-search-based LLM agents with respect to thought diversity, quality, and the efficiency of the search process.
Our experimental results highlight the potential of integrating introspective capabilities into tree search algorithms for AutoML Agents. 


\section*{Limitations}
While the proposed I-MCTS approach demonstrates significant improvements in AutoML agent performance, several limitations remain that warrant further investigation. First, the computational overhead of the introspective process, although beneficial for enhancing decision-making, introduces additional resource requirements. This limits the scalability of the method, particularly in scenarios where computational resources are constrained.
Future work should explore optimizations to reduce this overhead and investigate how the approach scales with increased computational power.

Second, the current evaluation of I-MCTS is primarily focused on structured data and tabular ML tasks. Its effectiveness on more complex and heterogeneous data types, such as image and text data, remains unexplored. Extending the application of I-MCTS to these domains could provide valuable insights into its generalizability and robustness across diverse machine learning tasks, such as Vison, NLP and RL tasks.

% Bibliography entries for the entire Anthology, followed by custom entries
%\bibliography{anthology,custom}
% Custom bibliography entries only
% \newpage
\bibliography{custom}

% \clearpage

\onecolumn
\appendix
% \section*{APPENDIX}

\newpage

\section{\thename}
\subsection{End-to-End Driving Policy}
The overall framework of \thename{} is depicted in Fig.~\ref{fig:framework}. 
\thename{} takes multi-view image sequences as input, transforms the sensor data into scene token embeddings, outputs the probabilistic distribution of actions, and samples an action to control the vehicle. 

\boldparagraph{BEV Encoder.} 
We first employ a BEV encoder~\cite{li2022bevformer} to transform multi-view image features from the perspective view to the Bird's Eye View (BEV), obtaining a feature map in the BEV space. This feature map is then used to learn instance-level map features and agent features.

\boldparagraph{Map Head.} 
Then we utilize a group of map tokens~\cite{maptrv2, liao2022maptr, lanegap} to learn the vectorized map elements of the driving scene from the BEV feature map, including lane centerlines, lane dividers, road boundaries, arrows, traffic signals, \etc.

\boldparagraph{Agent Head.} 
Besides, a group of agent tokens~\cite{jiang2022pip} is adopted to predict the motion information of other traffic participants, including location, orientation, size, speed, and multi-mode future trajectories.

\boldparagraph{Image Encoder.} 
Apart from the above instance-level map and agent tokens, we also use an individual image encoder~\cite{vit,he2016resnet} to transform the original images into image tokens. These image tokens provide dense and rich scene information for planning, complementary to the instance-level tokens.

\begin{figure}[t]
\centering
\includegraphics[width=0.98\linewidth]{fig/post-training-2.pdf} 
\caption{\textbf{Post-training.}  $N$  workers parallelly run. The generated rollout data $(s_t,a_t, r_{t+1},s_{t+1},...)$ are recorded in a rollout buffer. Rollout data and human driving demonstrations are used in RL- and IL-training steps to fine-tune the AD policy synergistically.
}
\label{fig:post-training}
\end{figure}

\boldparagraph{Action Space.} 
To accelerate the convergence of RL training, we design a decoupled discrete action representation. 
We divide the action into two independent components: lateral action and longitudinal action. 
The action space is constructed over a short $0.5$-second time horizon, during which the vehicle's motion is approximated by assuming constant linear and angular velocities. 
Under this assumption, the lateral action $a^x$ and longitudinal action $a^y$ can be directly computed based on the current linear and angular velocities.
By combining decoupling with a limited temporal scope and simplified motion model, our approach effectively reduces the dimensionality of the action space, accelerating training convergence.


\boldparagraph{Planning Head.} 
We use $E_\text{scene}$ to denote the scene representation, which consists of map tokens, agent tokens, and image tokens. We initialize a planning embedding denoted as $E_\text{plan}$. A cascaded Transformer decoder $\phi$ takes the planning embedding $E_\text{plan}$ as the query and the scene representation $E_\text{scene}$ as both key and value.

The output of the decoder $\phi$ is then combined with navigation information $E_\text{navi}$ and ego state $E_\text{state}$ to output the probabilistic distributions of the lateral action $a^x$ and the longitudinal action $a^y$:
\begin{equation}
\begin{aligned}
     \pi(a^x\mid s) = & \text{softmax}(\text{MLP}(\phi(E_\text{plan}, E_\text{scene}) \\
    & + E_\text{navi} + E_\text{state})), \\
     \pi(a^y\mid s) = & \text{softmax}(\text{MLP}(\phi(E_\text{plan}, E_\text{scene}) \\
     & + E_\text{navi} + E_\text{state})),
\label{eq:action distribution}
\end{aligned}
\end{equation}
where $E_\text{plan}$, $E_\text{navi}$, $E_\text{state}$, and the output of $\text{MLP}$ are all of the same dimension ($1 \times D$).

The planning head also outputs the value functions $V_x(s)$ and $V_y(s)$, which estimate the expected cumulative rewards for the lateral and longitudinal actions, respectively: 
\begin{equation}
\begin{aligned}
    & V_x(s) = \text{MLP}(\phi(E_\text{plan}, E_\text{scene}) + E_\text{navi} + E_\text{state}), \\
    & V_y(s) = \text{MLP}(\phi(E_\text{plan}, E_\text{scene}) + E_\text{navi} + E_\text{state}).
\end{aligned}
\end{equation}
The value functions are used in RL training (Sec.~\ref{sec:optimization}).

\subsection{Training Paradigm}
We adopt a three-stage training paradigm: perception pre-training, planning pre-training, and reinforced post-training, as shown in Fig.~\ref{fig:framework}.

\boldparagraph{Perception Pre-Training.} 
Information in the image is sparse and low-level. In the first stage,  
the map head and the agent head explicitly output map elements and agent motion information, which are supervised with ground-truth labels. Consequently,  
map tokens and agent tokens implicitly encode the corresponding high-level information.  
In this stage, we only update the parameters of the BEV encoder, the map head, and the agent head.



\boldparagraph{Planning Pre-Training.} 
In the second stage, to prevent the unstable cold start of RL training, IL is first performed to initialize the probabilistic distribution of actions based on large-scale real-world driving demonstrations from expert drivers. In this stage, we only update the parameters of the image encoder and the planning head, while the parameters of the BEV encoder, map head, and agent head are frozen. The optimization objectives of perception tasks and planning tasks may conflict with each other. However, with the training stage and parameters decoupled, such conflicts are mostly avoided.

\boldparagraph{Reinforced Post-Training.} 
In the reinforced post-training, RL and IL synergistically fine-tune the distribution. RL aims to guide the policy to be sensitive to critical risky events and adaptive to out-of-distribution situations. IL serves as the regularization term to keep the policy's behavior similar to that of humans.

We select a large amount of risky dense-traffic clips from collected driving demonstrations. For each clip, we train an independent 3DGS model that reconstructs the clip and serves as a digital driving environment.  
As shown in Fig.~\ref{fig:post-training}, we set $N$ parallel workers.  
Each worker randomly samples a 3DGS environment and begins rollout, i.e., the AD policy controls the ego vehicle to move and iteratively interacts with the 3DGS environment. After the rollout process of this 3DGS environment ends, the generated rollout data $(s_t,a_t, r_{t+1},s_{t+1},...)$ are recorded in a rollout buffer, and the worker will sample a new 3DGS environment for another round of rollout.

As for policy optimization, we iteratively perform RL-training steps and IL-training steps. For RL-training steps, we sample data from the rollout buffer and follow the Proximal Policy Optimization (PPO) framework~\cite{PPO} to update the AD policy. For IL-training steps, we use real-world driving demonstrations to update the policy. After a fixed number of training steps, the updated AD policy is sent to every worker to replace the old one, to avoid a distribution shift between data collection and optimization.
We only update the parameters of the image encoder and the planning head. The parameters of the BEV encoder, the map head, and the agent head are frozen.  
The detailed RL design is presented below.

\subsection{Interaction Mechanism between AD Policy and 3DGS Environment}
In the 3DGS environment, the ego vehicle acts according to the AD policy. Other traffic participants act according to real-world data in a log-replay manner.  
A simplified kinematic bicycle model is employed to iteratively update the ego vehicle's pose at every $\Delta t$ seconds as follows:  
\begin{equation}
\begin{aligned}
x_{t+1}^{w} & = x_{t}^w + v_t \cos \left(\psi_{t}^w\right) \Delta t, \\
y_{t+1}^{w} & = y_{t}^w + v_t \sin \left(\psi_{t}^w\right) \Delta t, \\
\psi_{t+1}^{w} & = \psi_{t}^w + \frac{v_t}{L} \tan \left(\delta_t\right) \Delta t,
\label{equation:kinematic_model}
\end{aligned}
\end{equation}  
where $x_t^{w}$ and $y_t^{w}$ denote the position of the ego vehicle relative to the world coordinate; $\psi_t^w$ is the heading angle that defines the vehicle's orientation with respect to the world $x$-coordinate; $v_t$ is the linear velocity of the ego vehicle; $\delta_t$ is the steering angle of the front wheels; and $L$ is the wheelbase, i.e., the distance between the front and rear axles.

During the rollout process, the AD policy outputs actions $(a_t^x, a_t^y)$ for a $0.5$-second time horizon at time step $t$. We derive the linear velocity $v_t$ and steering angle $\delta_t$ based on $(a_t^x, a_t^y)$.  
Based on the kinematic model in Eq.~\ref{equation:kinematic_model},  
the pose of the ego vehicle in the world coordinate system is updated from ${p}_t = (x_{t}^w, y_{t}^w, \psi_{t}^w)$ to ${p}_{t+1} = (x_{t+1}^{w}, y_{t+1}^{w}, \psi_{t+1}^{w})$.  

Based on the updated ${p}_{t+1}$, the 3DGS environment computes the new ego vehicle's state $s_{t+1}$. The updated pose ${p}_{t+1}$ and state $s_{t+1}$ serve as the input for the next iteration of the inference process.

The 3DGS environment also generates rewards $\mathcal{R}$ (Sec.~\ref{sec:reward}) according to multi-source information (including trajectories of other agents, map information, the expert trajectory of the ego vehicle, and the parameters of Gaussians), which are used to optimize the AD policy (Sec.~\ref{sec:optimization}).

\begin{figure}[t]
\centering
\includegraphics[width=1.0\linewidth]{fig/reward.pdf} 
\caption{\textbf{Example diagram of four types of reward sources.}  (1): Collision with a dynamic obstacle ahead triggers a reward $r_{\text{dc}}$. (2): Hitting a static roadside obstacle incurs a reward $r_{\text{sc}}$. (3): Moving onto the curb exceeds the positional deviation threshold $d_{\text{max}}$, triggering a reward $r_{\text{pd}}$. (4): Drifting toward the adjacent lane exceeds the heading deviation threshold $\psi_{\text{max}}$, triggering a reward $r_{\text{hd}}$.
}
\label{fig: reward source}
\end{figure}
\subsection{Reward Modeling}
\label{sec:reward}
The reward is the source of the training signal, which determines the optimization direction of RL. The reward function is designed to guide the ego vehicle's behavior by penalizing unsafe actions and encouraging alignment with the expert trajectory. It is composed of four reward components: (1) collision with dynamic obstacles, (2) collision with static obstacles, (3) positional deviation from the expert trajectory, and (4) heading deviation from the expert trajectory:
\begin{equation}
\begin{aligned}
\mathcal{R} = \{r_{\text{dc}}, r_{\text{sc}}, r_{\text{pd}}, r_{\text{hd}}  \}. 
\end{aligned}
\end{equation}

As illustrated in Fig.~\ref{fig: reward source}, these reward components are triggered under specific conditions.  
In the 3DGS environment, dynamic collision is detected if the ego vehicle's bounding box overlaps with the annotated bounding boxes of dynamic obstacles, triggering a negative reward $r_{\text{dc}}$. Similarly, static collision is identified when the ego vehicle's bounding box overlaps with the Gaussians of static obstacles, resulting in a negative reward $r_{\text{sc}}$.  
Positional deviation is measured as the Euclidean distance between the ego vehicle's current position and the closest point on the expert trajectory. A deviation beyond a predefined threshold $d_{\text{max}}$ incurs a negative reward $r_{\text{pd}}$.  
Heading deviation is calculated as the angular difference between the ego vehicle's current heading angle $ \psi_t $ and the expert trajectory's matched heading angle $\psi_{\text{expert}}$. A deviation beyond a threshold $ \psi_{\text{max}}$ results in a negative reward $r_{\text{hd}}$.

Any of these events, including dynamic collision, static collision, excessive positional deviation, or excessive heading deviation, triggers immediate episode termination. Because after such events occur, the 3DGS environment typically generates noisy sensor data, which is detrimental to RL training.

\subsection{Policy Optimization}
\label{sec:optimization}
In the closed-loop environment, the error in each single step accumulates over time. The aforementioned rewards are not only caused by the current action but also by the actions of the preceding steps.  
The rewards are propagated forward with Generalized Advantage Estimation (GAE)~\cite{gae} to optimize the action distribution of the preceding steps.

Specifically, for each time step $t$, we store the current state $s_t$, action $a_t$, reward $r_t$, and the estimate of the value $V(s_t)$.  
Based on the decoupled action space, and considering that different rewards have different correlations to lateral and longitudinal actions, the reward $r_t$ is divided into lateral reward $r_t^x$ and longitudinal reward $r_t^y$:
\begin{equation}
\begin{aligned}
r_t^x &= r_t^{\text{sc}} + r_t^{\text{pd}} + r_t^{\text{hd}}, \\
r_t^y &= r_t^{\text{dc}}.
\label{eq:reward-decouple}
\end{aligned}
\end{equation}
Similarly, the value function $V(s_t)$ is decoupled into two components: $V_x(s_t)$ for the lateral dimension and $V_y(s_t)$ for the longitudinal dimension. These value functions estimate the expected cumulative rewards for the lateral and longitudinal actions, respectively. The advantage estimates $\hat{A}_t^x$ and $\hat{A}_t^y$ are then computed as follows:
\begin{equation}
\begin{aligned}
\delta_t^x &= r_t^x + \gamma V_x(s_{t+1}) - V_x(s_t), \\
\delta_t^y &= r_t^y + \gamma V_y(s_{t+1}) - V_y(s_t), \\
\hat{A}_t^x &= \sum_{l=0}^{\infty}(\gamma \lambda)^l \delta_{t+l}^x, \\
\hat{A}_t^y &= \sum_{l=0}^{\infty}(\gamma \lambda)^l \delta_{t+l}^y,
\label{eq:advantage}
\end{aligned}
\end{equation}
where $\delta_t^x$ and $\delta_t^y$ are the temporal difference errors for the lateral and longitudinal dimensions, $\gamma$ is the discount factor, and $\lambda$ is the GAE parameter that controls the trade-off between bias and variance.

To further clarify the relationship between the advantage estimates and the reward components, we decompose $\hat{A}_t^x$ and $\hat{A}_t^y$ based on the reward decomposition in Eq.~\ref{eq:reward-decouple} and the advantage estimation in Eq.~\ref{eq:advantage}. Specifically, we derive the following decomposition:
\begin{equation}
\begin{aligned}
\hat{A}_t^x &= \hat{A}_t^{\text{sc}} + \hat{A}_t^{\text{pd}} + \hat{A}_t^{\text{hd}}, \\
\hat{A}_t^y &= \hat{A}_t^{\text{dc}},
\end{aligned}
\end{equation}
where $\hat{A}_t^{\text{sc}}$ is the advantage estimate for avoiding static collisions, $\hat{A}_t^{\text{pd}}$ is the advantage estimate for minimizing positional deviations, $\hat{A}_t^{\text{hd}}$ is the advantage estimate for minimizing heading deviations, and $\hat{A}_t^{\text{dc}}$ is the advantage estimate for avoiding dynamic collisions.

These advantage estimates are used to guide the update of the AD policy $\pi_{\theta}$, following the PPO framework~\cite{PPO}. By leveraging the decomposed advantage estimates $\hat{A}_t^x$ and $\hat{A}_t^y$, we can independently optimize the lateral and longitudinal dimensions of the policy. This is achieved by defining separate objective functions $\mathcal{L}_x^{\text{CLIP}}(\theta)$ and $\mathcal{L}_y^{\text{CLIP}}(\theta)$ for each dimension,  as follows:
\begin{equation}
\begin{aligned}
\mathcal{L}_x^{\text{PPO}}(\theta) &= \mathbb{E}_t \left[ \min \left( \rho_t^x \hat{A}_t^x, \ \text{clip}(\rho_t^x, 1-\epsilon_x, 1+\epsilon_x) \hat{A}_t^x \right) \right], \\
\mathcal{L}_y^{\text{PPO}}(\theta) &= \mathbb{E}_t \left[ \min \left( \rho_t^y \hat{A}_t^y, \ \text{clip}(\rho_t^y, 1-\epsilon_y, 1+\epsilon_y) \hat{A}_t^y \right) \right], \\
\mathcal{L}^{\text{PPO}}(\theta) &= \mathcal{L}_x^{\text{PPO}}(\theta) + \mathcal{L}_y^{\text{PPO}}(\theta),
\end{aligned}
\end{equation}
where $\rho_t^x = \frac{\pi_{\theta}(a_t^x \mid s_t)}{\pi_{\theta_{\text{old}}}(a_t^x \mid s_t)}$ is the importance sampling ratio for the lateral dimension, $\rho_t^y = \frac{\pi_{\theta}(a_t^y \mid s_t)}{\pi_{\theta_{\text{old}}}(a_t^y \mid s_t)}$ is the importance sampling ratio for the longitudinal dimension, $\epsilon_x$ and $\epsilon_y$ are small constants that control the clipping range for the lateral and longitudinal dimensions, ensuring stable policy updates.

The clipped objective function $\mathcal{L}^{\text{PPO}}(\theta)$ prevents excessively large updates to the policy parameters $\theta$, thereby maintaining training stability.

\begin{table*}[ht]
    \centering
{
\begin{tabular}{lccccccccc}
    \toprule
    RL:IL & CR$\downarrow$ & DCR$\downarrow$ & SCR$\downarrow$ & DR$\downarrow$ & PDR$\downarrow$ & HDR$\downarrow$ &ADD$\downarrow$ & Long. Jerk$\downarrow$ & Lat. Jerk$\downarrow$ \\
    \midrule
     0:1  & 0.229 & 0.211 & 0.018 & 0.066 & 0.039 & 0.027  & 0.238 & 3.928 & 0.103\\
     1:0  & 0.143 & 0.128 & 0.015 &0.080 &0.065 &0.015 &0.345 &4.204 &0.085\\
     2:1 & 0.137 & 0.125 & 0.012 & 0.059 & 0.050 & 0.009  & 0.274 & 4.538 & 0.092\\
     4:1 & 0.089 & 0.080 & 0.009 & 0.063 & 0.042 & 0.021  & 0.257 & 4.495 & 0.082 \\
     8:1 & 0.125 & 0.116 & 0.009 & 0.084 & 0.045 & 0.039  & 0.323 & 5.285 & 0.115\\
    \bottomrule
\end{tabular}
}
    \caption{\textbf{Ablation on RL-to-IL step mixing ratios in the reinforced post-training stage.}}
    \label{tab:ratio}
\end{table*}

\subsection{Auxiliary Objective}
RL usually faces the challenge of sparse rewards, which makes the convergence process unstable and slow. To speed up convergence, we introduce auxiliary objectives that provide dense guidance to the entire action distribution.

The auxiliary objectives are designed to penalize undesirable behaviors by incorporating specific reward sources, including dynamic collisions, static collisions, positional deviations, and heading deviations. These objectives are computed based on the actions \( a_t^{x, \text{old}} \) and \( a_t^{y, \text{old}} \) selected by the old AD policy \( \pi_{\theta_{\text{old}}} \) at time step \( t \). To facilitate the evaluation of these actions, we separate the probability distribution of the action into four parts:
\begin{equation}
\begin{aligned}
\Delta \pi_y^{\text{dec}} &= \sum_{a_t^y < a_t^{y, \text{old}}} \pi_\theta(a_t^y \mid s_t), \\
\Delta \pi_y^{\text{acc}} &= \sum_{a_t^y > a_t^{y, \text{old}}} \pi_\theta(a_t^y \mid s_t), \\
\Delta \pi_x^{\text{left}} &= \sum_{a_t^x < a_t^{x, \text{old}}} \pi_\theta(a_t^x \mid s_t), \\
\Delta \pi_x^{\text{right}} &= \sum_{a_t^x > a_t^{x, \text{old}}} \pi_\theta(a_t^x \mid s_t).
\end{aligned}
\end{equation}
Here, \( \Delta \pi_y^{\text{dec}} \) represents the total probability of deceleration actions, \( \Delta \pi_y^{\text{acc}} \) represents the total probability of acceleration actions, \( \Delta \pi_x^{\text{left}} \) represents the total probability of leftward steering actions, and \( \Delta \pi_x^{\text{right}} \) represents the total probability of rightward steering actions.

\boldparagraph{Dynamic Collision Auxiliary Objective.}  
The dynamic collision auxiliary objective adjusts the longitudinal control action \(a_t^y\) based on the location of potential collisions relative to the ego vehicle. If a collision is detected ahead, the policy prioritizes deceleration actions (\(a_t^y < a_t^{y, \text{old}}\)); if a collision is detected behind, it encourages acceleration actions (\(a_t^y > a_t^{y, \text{old}}\)). To formalize this behavior, we define a directional factor \(f_\text{dc}\):
\begin{equation}
\begin{aligned}
f_\text{dc} = \begin{cases} 
1 & \text{if the collision is ahead}, \\
-1 & \text{if the collision is behind}.
\end{cases} 
\end{aligned}
\end{equation}

The auxiliary objective for dynamic collision avoidance is defined as:
\begin{equation}
\begin{aligned}
\mathcal{L}_\text{dc}(\theta_y) = \mathbb{E}_t \left[ 
    \hat{A}_t^\text{dc} \cdot f_\text{dc} \cdot (\Delta \pi_y^{\text{dec}} - \Delta \pi_y^{\text{acc}})
\right],
\end{aligned}
\end{equation}
where \(\hat{A}_t^\text{dc}\) is the advantage estimate for dynamic collision avoidance.

\boldparagraph{Static Collision Auxiliary Objective.}  
The static collision auxiliary objective adjusts the steering control action $a_t^x$ based on the proximity to static obstacles. If the static obstacle is detected on the left side, the policy promotes rightward steering actions ($a_t^x > a_t^{x,\text{old}}$); if the static obstacle is detected on the right side, it promotes leftward steering actions ($a_t^x < a_t^{x,\text{old}}$). To formalize this behavior, we define a directional factor $f_\text{sc}$:  
\begin{equation}
\begin{aligned}
f_\text{sc} = \begin{cases} 
1 & \text{if static obstacle is on the left}, \\
-1 & \text{if static obstacle is on the right}.
\end{cases} 
\end{aligned}
\end{equation}

The auxiliary objective for static collision avoidance is defined as:  
\begin{equation}
\begin{aligned}
\mathcal{L}_\text{sc}(\theta_x) = \mathbb{E}_t \left[ 
    \hat{A}_t^\text{sc} \cdot f_\text{sc} \cdot (\Delta \pi_x^{\text{right}} - \Delta \pi_x^{\text{left}})
\right],
\end{aligned}
\end{equation}  
where $\hat{A}_t^\text{sc}$ is the advantage estimate for static collision avoidance.  

\boldparagraph{Positional Deviation Auxiliary Objective.}  
The positional deviation auxiliary objective adjusts the steering control action $a_t^x$ based on the ego vehicle's lateral deviation from the expert trajectory. If the ego vehicle deviates leftward, the policy promotes rightward corrections ($a_t^x > a_t^{x,\text{old}}$); if it deviates rightward, it promotes leftward corrections ($a_t^x < a_t^{x,\text{old}}$). We formalize this with a directional factor $f_\text{pd}$:  
\begin{equation}
\begin{aligned}
f_\text{pd} = \begin{cases} 
1 & \text{if ego vehicle deviates leftward}, \\
-1 & \text{if ego vehicle deviates rightward}.
\end{cases} 
\end{aligned}
\end{equation}

The auxiliary objective for positional deviation correction is:
\begin{equation}
\begin{aligned}
\mathcal{L}_\text{pd}(\theta_x) = \mathbb{E}_t \left[ 
    \hat{A}_t^\text{pd} \cdot f_\text{pd} \cdot (\Delta \pi_x^{\text{right}} - \Delta \pi_x^{\text{left}})
\right],
\end{aligned}
\end{equation}  
where $\hat{A}_t^\text{pd}$ estimates the advantage of trajectory alignment.

\boldparagraph{Heading Deviation Auxiliary Objective.}  
The heading deviation auxiliary objective adjusts the steering control action $a_t^x$ based on the angular difference between the ego vehicle’s current heading and the expert’s reference heading. If the ego vehicle deviates counterclockwise, the policy promotes clockwise corrections ($a_t^x > a_t^{x,\text{old}}$); if it deviates clockwise, it promotes counterclockwise corrections ($a_t^x < a_t^{x,\text{old}}$). To formalize this behavior, we define a directional factor $f_\text{hd}$:  
\begin{equation}
\begin{aligned}
f_\text{hd} = \begin{cases} 
1 & \text{if ego vehicle deviates clockwise}, \\
-1 & \text{if ego vehicle deviates counterclockwise}.
\end{cases} 
\end{aligned}
\end{equation}

The auxiliary objective for heading deviation correction is then defined as:  
\begin{equation}
\begin{aligned}
\mathcal{L}_\text{hd}(\theta_x) = \mathbb{E}_t \left[ 
    \hat{A}_t^\text{hd} \cdot f_\text{hd} \cdot (\Delta \pi_x^{\text{right}} - \Delta \pi_x^{\text{left}})
\right],
\end{aligned}
\end{equation}  
where $\hat{A}_t^\text{hd}$ is the advantage estimate for heading alignment.  

\begin{table*}[ht]
\begin{center}
\centering
\resizebox{0.98\textwidth}{!}{
\begin{tabular}{cccccccccccccc}
\toprule
\multirow{2}{*}{ID} & Dynamic & Static & Position & Heading & \multirow{2}{*}{CR$\downarrow$} &\multirow{2}{*}{DCR$\downarrow$} &\multirow{2}{*}{SCR$\downarrow$} &\multirow{2}{*}{DR$\downarrow$} &\multirow{2}{*}{PDR$\downarrow$} &\multirow{2}{*}{HDR$\downarrow$} &\multirow{2}{*}{ADD$\downarrow$} &\multirow{2}{*}{Long. Jerk$\downarrow$} &\multirow{2}{*}{Lat. Jerk$\downarrow$}\\
& Collision & Collision & Deviation & Deviation & & & & & & & & & \\
\midrule
1 & \cmark  &  &  &  & 0.172 & 0.154 & 0.018 & 0.092 & 0.033 & 0.059  & 0.259 & 4.211 & 0.095 \\
2 &  & \cmark & \cmark & \cmark & 0.238 & 0.217 & 0.021 & 0.090 & 0.045 & 0.045  & 0.241 & 3.937 & 0.098 \\
3 & \cmark &  & \cmark & \cmark & 0.146 & 0.128 & 0.018 & 0.060 & 0.030 & 0.030  & 0.263 & 3.729 & 0.083\\
4 & \cmark & \cmark &  & \cmark & 0.151 & 0.142 & 0.009 & 0.069 & 0.042 & 0.027 & 0.303 & 3.938 & 0.079\\
5 & \cmark & \cmark & \cmark &  & 0.166 & 0.157 & 0.009 & 0.048 & 0.036 & 0.012 & 0.243 & 3.334 & 0.067\\
6 & \cmark & \cmark & \cmark & \cmark & 0.089 & 0.080 & 0.009 & 0.063 & 0.042 & 0.021 & 0.257 & 4.495 & 0.082 \\
\bottomrule
\end{tabular}
}
\end{center}
\vspace{-2mm}
\caption{\textbf{Ablation on reward sources.} The table shows the impact of different reward components on performance.}
\label{tab:reward_ablation}
\end{table*}

\begin{table*}[ht]
\begin{center}
\centering
\resizebox{0.98\textwidth}{!}{
\begin{tabular}{ccccccccccccccc}
\toprule
\multirow{2}{*}{ID} & \multirow{2}{*}{PPO Obj.}  & Dynamic Col. & Static Col. & Position Dev. & Heading Dev. & \multirow{2}{*}{CR$\downarrow$} & \multirow{2}{*}{DCR$\downarrow$}  & \multirow{2}{*}{SCR$\downarrow$} & \multirow{2}{*}{DR$\downarrow$} & \multirow{2}{*}{PDR$\downarrow$} & \multirow{2}{*}{HDR$\downarrow$} & \multirow{2}{*}{ADD$\downarrow$} & \multirow{2}{*}{Long. Jerk$\downarrow$} & \multirow{2}{*}{Lat. Jerk$\downarrow$} \\
& & Auxiliary Obj. & Auxiliary Obj. & Auxiliary Obj. & Auxiliary Obj. & & & & & & & & & \\
\midrule
1 &\cmark&  &  &  &  & 0.249 & 0.223 & 0.026 & 0.077 & 0.047 & 0.030  & 0.266 & 4.209 & 0.104 \\
2 &\cmark& \cmark &  &  &  & 0.178 & 0.163 & 0.015 & 0.151 & 0.101 & 0.050 & 0.301 & 3.906 & 0.085 \\
3 &\cmark&  & \cmark & \cmark & \cmark & 0.137 & 0.125 & 0.012 & 0.157 & 0.145 & 0.012 & 0.296 & 3.419 & 0.071 \\
4 &\cmark& \cmark &  & \cmark & \cmark & 0.169 & 0.151 & 0.018 & 0.075 & 0.042 & 0.033 & 0.254 & 4.450 & 0.098 \\
5 &\cmark& \cmark & \cmark &  & \cmark & 0.149 & 0.134 & 0.015 & 0.063 & 0.057 & 0.006 & 0.324 & 3.980 & 0.086 \\
6 &\cmark& \cmark & \cmark & \cmark & & 0.128 & 0.119  & 0.009 & 0.066 & 0.030 & 0.036  & 0.254 & 4.102 & 0.092 \\
7 &&\cmark  &\cmark  &\cmark  &\cmark  & 0.187 &0.175  &0.012 &0.077 &0.056  &0.021  &0.309  &5.014  &0.112  \\
8 &\cmark& \cmark & \cmark & \cmark & \cmark & 0.089 & 0.080 & 0.009 & 0.063 & 0.042 & 0.021  & 0.257 & 4.495 & 0.082 \\
\bottomrule
\end{tabular}
}
\end{center}
\vspace{-2mm}
\caption{\textbf{Ablation on auxiliary objectives.} The table shows the impact of different auxiliary objectives on performance.}
\label{tab:auxiliary_ablation}
\end{table*}

\boldparagraph{Overall Auxiliary Objectives.}  
The overall auxiliary objectives are a weighted sum of the individual objectives:
\begin{equation}
\begin{aligned}
\mathcal{L}_\text{aux}(\theta) = &\lambda_1 \mathcal{L}_\text{dc}(\theta_y) + \lambda_2 \mathcal{L}_\text{sc}(\theta_x)  + \\ 
&\lambda_3 \mathcal{L}_\text{pd}(\theta_x) +\lambda_4 \mathcal{L}_\text{hd}(\theta_x),
\end{aligned}
\end{equation}
where $\lambda_1$, $\lambda_2$, $\lambda_3$, and $\lambda_4$ are weighting coefficients that balance the contributions of each auxiliary objective.

\boldparagraph{Optimization Objective.}  
The final optimization objective combines the clipped PPO objective with the auxiliary objective:
\begin{equation}
\mathcal{L}(\theta) = \mathcal{L}^{\text{PPO}}(\theta) + \mathcal{L}_\text{aux}(\theta).
\end{equation}


\section{Detailed Results}
\label{app:detailed_result}

Below show all detailed results on each dataset.

\begin{table}[h!]
\centering
\resizebox{\textwidth}{!}{
\begin{tabularx}{1.12\textwidth}{>{\scriptsize}l*{12}{c}}
\toprule
 & \multicolumn{2}{c}{\textbf{AutoGluon}} & \multicolumn{2}{c}{\textbf{AutoSklearn}} & \multicolumn{2}{c}{\textbf{AIDE}} & \multicolumn{2}{c}{\textbf{DI}} & \multicolumn{2}{c}{\textbf{SELA}} & \multicolumn{2}{c}{\textbf{\methodname{}}} \\ 
\normalsize Dataset & {Avg.} & {Best} & {Avg.} & {Best} & {Avg.} & {Best} & {Avg.} & {Best} & {Avg.} & {Best} & {Avg.} & {Best} \\
\midrule
Click\_prediction\_small & 26.6 & 26.6 & 40.2 & 40.3 & 26.1 & 39.4 & 12.9 & 13.9 & 23.2 & 27.4 & 29.1 & 30.2 \\
GesturePhaseSegmentationProcessed & 69.3 & 69.3 & 67.2 & 68.4 & 56.3 & 68.1 & 60.1 & 64.4 & 67.9 & 69.2 & 67.2 & 68.5 \\
Moneyball & 24.3 & 24.3 & 13.1 & 13.8 & 23.8 & 24.6 & 9.5 & 24.5 & 21.9 & 24.5 & 41.6 & 41.7 \\
SAT11-HAND-runtime-regression & 12.6 & 12.6 & 10.3 & 10.3 & 12.0 & 12.1 & 11.4 & 11.9 & 12.2 & 12.5 & 21.7 & 24.6 \\
boston & 39.8 & 39.8 & 19.5 & 19.6 & 40.5 & 41.3 & 37.0 & 38.6 & 40.1 & 41.4 & 62.5 & 62.7 \\
colleges & 88.3 & 88.3 & 2.1 & 2.1 & 86.0 & 87.8 & 87.5 & 87.7 & 87.8 & 87.8 & 94.1 & 94.3 \\
concrete-strength & 28.3 & 28.3 & 17.4 & 17.9 & 28.3 & 28.3 & 28.8 & 29.6 & 28.2 & 28.2 & 47.3 & 48.2 \\
credit-g & 50.5 & 50.5 & 35.1 & 44.0 & 21.6 & 48.4 & 48.1 & 53.2 & 50.9 & 52.7 & 50.9 & 51.1 \\
diamonds & 13.8 & 13.8 & 8.7 & 8.7 & 13.7 & 13.7 & 13.5 & 13.6 & 13.7 & 13.8 & 26.8 & 26.8\\
house-prices & 9.0 & 9.0 & 2.0 & 2.0 & 8.9 & 8.9 & 8.5 & 9.0 & 8.9 & 9.0 & 15.8 & 18.4 \\
icr & 76.2 & 76.2 & 70.4 & 79.2 & 31.7 & 35.9 & 57.8 & 60.6 & 78.7 & 79.2 & 72.3 & 79.0 \\
jasmine & 84.3 & 84.3 & 84.4 & 84.7 & 83.6 & 84.6 & 77.8 & 83.5 & 85.4 & 86.2 
 & 84.2 & 84.3\\
kc1 & 38.3 & 38.3 & 43.5 & 45.0 & 40.8 & 42.6 & 38.1 & 41.2 & 42.2 & 43.1 & 44.8 & 45.6 \\
kick & 39.6 & 39.6 & 41.8 & 42.1 & 14.9 & 38.6 & 2.8 & 4.2 & 35.9 & 38.7 & 36.2 & 42.4 \\
mfeat-factors & 96.7 & 96.7 & 97.1 & 97.5 & 94.4 & 94.5 & 93.0 & 96.0 & 95.7 & 96.2 & 94.4 & 95.4\\
segment & 93.5 & 93.5 & 92.7 & 93.1 & 91.7 & 92.2 & 91.7 & 92.6 & 93.8 & 94.4 & 91.1 & 93.7\\
smoker-status & 78.0 & 78.0 & 78.6 & 78.9 & 74.8 & 76.3 & 77.3 & 81.5 & 82.4 & 91.5 & 79.9 & 91.5\\
software-defects & 51.5 & 51.5 & 61.1 & 61.7 & 49.7 & 49.8 & 54.5 & 57.3 & 52.2 & 53.3 & 60.5 & 60.7\\
titanic & 78.9 & 78.9 & 76.2 & 78.9 & 81.2 & 83.7 & 76.0 & 78.5 & 78.8 & 79.7 & 76.1 & 78.5\\
wine-quality-white & 65.4 & 65.4 & 60.7 & 61.4 & 62.9 & 65.1 & 61.2 & 61.6 & 65.3 & 66.0 & 63.3 & 65.5 \\
\midrule
\normalsize Overall NS \% $\uparrow$ & 53.2 & 53.2 & 46.1 & 47.5  & 45.5 & 51.8 & 47.4 & 50.2 & 53.3 & 54.7 & \textbf{58.6} & \textbf{59.8} \\
\bottomrule
\end{tabularx}
}
\caption{Methods' NS \% for each tabular dataset}
\label{table:full-main-results}
\end{table}
\section{Dataset Generation}
\label{sec:dataset}
\revise{
To train the proposed GNN, we constructed a dataset of building structures and a subset of these structures were subjected to fire simulations using FEA. The dataset generation process is illustrated in \figref{fig:dataset_generation_procedure}. Initially, a total of 33,000 building structures with geometrical details, material properties, and gravity loads were created. Due to randomness in generating these structures, a filter is applied to remove unreasonable data after gravity load simulation, which included 15,377 structures. A trade-off between computational feasibility and model performance is made among the remaining 17,623 structures. As further labeling structures with MIDR requires resource-intensive fire simulations via OpenSeesRT, a large proportion of 16,050 structures is selected as unlabeled dataset. On the other hand, each of the other 1,573 structures was further subjected to 30 different fire simulations, forming the labeled dataset containing $1,573\times 30 = 47,190$ fire cases.} This section details the step-by-step process for generating the dataset, including geometry creation, material property assignment, and simulations due to gravity loads and fire scenarios. 
% To train the proposed neural network, we constructed a dataset comprising building structure data and a subset of fire scenario data. The dataset generation process is illustrated in \figref{fig:dataset_generation_procedure}. 
% A total of 33,000 building structures with geometric details, material properties, and gravity loads were initially created. Out of these, 3,000 structures were selected as labeled data, and the remaining 30,000 were designated as unlabeled data. Further, about half of them filtered out due to instability under gravity loads only. 
\begin{figure*}[h!]
    \centering
    \includegraphics[width=0.8\linewidth]{figures/dataset_filter_procedure.pdf}
    \caption{Workflow for dataset generation (geometry, material property, gravity loads, and fire scenarios).}
    \label{fig:dataset_generation_procedure}
\end{figure*}

\subsection{Geometry Generation}
\label{subsec:geometry_generation}
The geometry of the building structures forms the foundation of the dataset. Regular 
\revise{3D structures} resembling multi-story parking structures or shopping malls were generated, with parameters such as building floor dimensions and story heights selected randomly. Each building structure is composed of multiple rooms, which serve as the basic unit in this study. A room herein is a cuboid space defined by specific length, width, and height. Within a structure, rooms of the same dimensions are uniformly arranged along the length, width, and height, corresponding to the $x$-, $y$-, and $z$-axes, respectively. Structures vary in room size and number of rooms along each axis. Specifically, the room length, width, and height are independently sampled from a uniform distribution within the interval $[2, 5]$ meters along the three directions of the structure. Similarly, the room number along each axis is uniformly sampled independently as an integer within the interval $[2, 7]$, i.e., the maximum number of stories of the buildings simulated in this study is 7.

To introduce variability and simulate real-world scenarios, approximately $8\%$ of structural elements (beams or columns) are randomly removed after initial geometry creation. 
\revise{Such removal is not fire-induced damage, but reflects functional diversity often observed in real buildings, such as open spaces designed for activities in shopping malls, e.g., ice skating rinks. Examples of the generated geometries are illustrated in \figref{fig:example_generated_geometry}, showcasing the diversity and realism of the dataset. This element removal does not affect the definition of room's geometry in the structure and nor does it affect the number of considered fire scenarios.} 

\revise{A range of coefficient of variation values ($3.3\%$ to $17.5\%$) was derived from prior studies that investigated the statistics of geometrical and material properties of structural components of buildings (e.g., \cite{mirza1979variations, lee2004probabilistic}). These studies provide empirical data on the natural variability in parameters such as Young's modulus, yield strength, and dimensions of structural elements due to manufacturing tolerances and material inconsistencies. By selecting $8\%$ for the removal of structural elements in our database, we aimed to maintain a level of variability that is representative of real-world uncertainties while ensuring computational feasibility. This choice ensures that the database captures realistic deviations without introducing extreme cases that may not be commonly encountered in practice.}

\begin{figure*}[h!]
    \centering
    \includegraphics[width=\linewidth]{figures/example_generated_geometry.pdf}
    \caption{Examples of generated structural geometry of different sizes (all dimensions in meters).}
    \label{fig:example_generated_geometry} 
\end{figure*}

{\blockRevise

In this study, we opted for a deterministic square, dimension of $0.1$ m, solid cross-sectional steel elements due to their simplicity in modeling and analysis. Square sections exhibit uniform geometrical properties in all directions, simplifying the computation of structural responses and avoiding complications associated with more complex shapes, such as wide-flange sections, facilitating the computational efficiency and scalability to generate a large dataset. This choice also helps to mitigate issues related to stress concentrations and facilitates a more straightforward representation of structural behavior under thermal loads. 

\textit{Remark:} The selected cross-section provides a comparable flexural rigidity to a $W 130 \times 130 \times 28.1$ wide-flange section (metric units), albeit with significantly higher axial rigidity. This cross-section is acceptable for gravity-load-designed frames under service loading conditions where the models assume fully rigid, moment-resisting beam-column connections for the evaluation of the IDR under thermal loading. This assumption is reasonable in this computational study where the primary interest is to understand the global deformation response of frames under fire conditions. The selection of uniform square cross-sections for both beams and columns, rather than adherence to standard capacity design principles, was made here primarily for computational efficiency and to reduce design parameters in the database generation process. This choice allows for simplified and scalable approach to analyze the fire-induced response of generic steel frames without the need for large section variations, where this study mainly focuses on the fire vulnerability assessment using ML-based predictions. However, if additional loading conditions, e.g., seismic or wind loads, were to be considered, larger sections, strong-column/weak-beam principle, and ductile detailing would be required in the generated buildings for realistic structural behavior under combined loading conditions. Future studies may also consider investigating the influence of variable cross-sectional dimensions and semi-rigid connections on the structural performance under fire conditions. 
} % blockRevise

\subsection{Material Properties}
Steel is chosen as the material for the structures. To reflect real-world variations, we randomly assign one of five slightly different steel material types to each structural element. \revise{
The ranges of material properties are provided in \tabref{tab:material_property_ranges} and the properties are sampled from uniform distributions of the corresponding ranges. These variations simulate differences arising from manufacturing batches or regional material properties. That these properties are at ambient temperature and change when the temperature rises due to a fire. The selection of materials with varying properties is aimed at increasing the diversity of the data. Our goal is to represent as wide a range of data as possible with a limited amount of building structure data, thereby enhancing the generalization ability of the GNN. Our assumed material property ranges are expected to be wider than the real-world conditions based on findings in \cite{mirza1979variations, lee2004probabilistic}. Therefore, we are essentially tackling a more challenging and general task. If we can solve this problem, we are confident that our method will perform equally well or even better in real-world scenarios.
}
\begin{table}[h!]
    \centering
    \caption{Material properties ranges for considered steel structures.}
    \begin{tabular}{lc}
        \toprule
        Property & Range \\
        \midrule
        Young's modulus & [168, 252] GPa \\
        Yield strength & [220, 330] MPa \\
        Strain-hardening ratio & [0.8, 1.2] \% \\
        \bottomrule
    \end{tabular}
    \label{tab:material_property_ranges}
\end{table}

\subsection{Gravity Loads}
Gravity loads are applied to columns and beams based on their \revise{influence (tributary) areas as typically conducted in structural analysis. The considered ``service'' load conditions include the column self-weight and the additional loads directly supported on the beams from their self-weight and weights of the reinforced concrete slabs, people as live load, and building content. An edge beam typically carries approximately half the gravity load supported by a parallel interior beam}. The ranges of gravity loads are listed in \tabref{tab:gravity_load_ranges}. \revise{The loads are sampled from uniform distributions of the corresponding ranges.} Structures that failed to meet an MIDR threshold of $1\%$ under gravity loads were deemed unacceptable designs and filtered out, as such configurations of randomly chosen geometry, material, and gravity load combinations were considered unrealistic from a regulatory and practicality points of view.
\begin{table}[h!]
    \centering
    \caption{Gravity load ranges for considered beams and columns.}
    \begin{tabular}{lc}
        \toprule
        Element & Range (kN/m)  \\
        \midrule
        Column & [0.5, 1.0]  \\
        Edge beam & [1.5, 4.5]  \\
        Interior beam & [3.0, 7.5]  \\
        \bottomrule
    \end{tabular}
    \label{tab:gravity_load_ranges}
\end{table} 

\subsection{Rule-based Thermal Load Generation}
\label{subsec:thermal_load_generation}
To evaluate a building's structural response during a fire event, we employed a simplified rule-based approach for thermal load generation. 
% Previous studies \cite{nan_structuralfire_2023} have demonstrated that steel structures rapidly equilibrate with surrounding gases temperatures due to efficient heat exchange. Consequently, gas temperatures can be directly used as inputs for FEA tools, e.g., OpenSees, simplifying the process of modeling thermal loads. 
% Accurately simulating temperature fields in fire scenarios poses significant challenges. Advanced thermodynamic simulations, such as those performed using Fire Dynamics Simulator (FDS) \cite{mcgrattan_fire_2000}, provide precise temperature predictions. However, these methods are hindered by high computational costs, prolonging execution times, and limited scalability, making them impractical for generating large datasets. Additionally, real-world fire loads often display substantial spatial variability across different rooms \cite{dundar_fire_2023}, resulting in scenario-specific temperature fields with limited generalizability. For example, studies on bridge fires \cite{he_study_2024} have demonstrated that environmental factors, such as wind speeds, can significantly influence temperature distributions. Furthermore, even within identical scenarios, variations in fire modeling methodologies can produce distinctly different temperature fields \cite{zhang_temperature_2020, du_new_2012}. These challenges emphasize the need for efficient and adaptable methods to generate fire temperature data.
% To address these issues, we adopted a rule-based approach to model temperature variations. 
According to \cite{spearpoint_fire_2008}, a typical fire development follows a predictable pattern. During the {\em{growth stage}}, the temperature rises slowly and approximately linearly after ignition. This is followed by the {\em{flashover stage}}, where temperatures increase rapidly to peak values. After reaching the peak, the temperature either stabilizes or continues to rise slowly until the {\em{decay stage}} begins. Inspired by this fire development pattern, we describe the temperature evolution in time, $t$, prior to the decay stage in two distinct stages:
\begin{enumerate}
    \item {\bf{Initial linear increase stage}}: For $t \in [0, t_1)$, temperature increases gradually and linearly as the fire spreads through the building. This stage represents the time before the fire directly affects a structural element.  
    \item {\bf{ISO 834 fire curve stage}}: For $t \in [t_1, t_{\thre}]$, temperature rises rapidly following the ISO 834 curve \cite{ISO834}, modeling the direct impact of the fire on the structural element. 
\end{enumerate}
The slope of the linear temperature increase, $c$, and the transition time, $t_1$, are influenced by the spatial relationship between the fire source and the structural element. For the second stage of temperature evolution, we utilize the ISO 834 curve, a widely accepted standard for fire resistance testing. This standardized fire curve describes the temperature rise over time, enabling rapid and consistent thermal fields across various scenarios. The duration of fire simulation in this study is set to $t_{\thre}=60$ minutes. This value represents the upper limit for the temperature evolution of each structural element, providing a consistent basis for analyzing the structural response to fire.

Let $(x, y, z)$ represents the midpoint of a structural element and $(x_{\subfire}, y_{\subfire}, z_{\subfire})$ the fire source point. \revise{Integer parameters $h$ and $h_{\subfire}$ correspond to the respective floor levels of the element and the fire source}. The temperature evolution for each element is expressed as follows:
\begin{enumerate}
    \item Linear increase stage ($0 < t < t_1$):
    \begin{equation}
    T(t) = c \cdot t,
    \end{equation}
    where $c$, the rate of temperature increase ($^\circ\mathrm{C}/\mathrm{min}$), depends on the height difference between the element, $h$, and the fire source, $h_{\subfire}$:
    \begin{equation}
        c = 
        \begin{cases} 
        5\left/\left(h - h_{\subfire} + 1\right)\right., & h \geq h_{\subfire}, \\
        2\left/\left(h_{\subfire} - h\right)\right., & h < h_{\subfire}.
        \end{cases}
    \end{equation}
     \item ISO 834 stage ($t \geq t_1$):
\begin{equation}
    T(t) = c \cdot t_1 + 345 \log_{10} \left(8 \left(t - t_1\right) + 1\right).
\end{equation}
\end{enumerate}

The transition (arrival) time $t_1$, marking the end of the linear stage, depends on the spatial distance between the fire source and the element. We define the following two Euclidean distances $L_p$ in the $xy$ plane and $L_s$ in the $xyz$ space:
\begin{eqnarray}
L_p & \triangleq & \sqrt{(x - x_{\subfire})^2 + (y - y_{\subfire})^2}, \\
\label{eq:Lp}
L_s & \triangleq & \sqrt{(x - x_{\subfire})^2 + (y - y_{\subfire})^2 + (z - z_{\subfire})^2}.
\label{eq:Ls}
\end{eqnarray}
Accordingly, the transition time, $t_1$, is expressed as follows:
\begin{equation}
    t_1 = 
    \begin{cases}
    \beta_{1} \cdot \left(1 - \exp\left\{- L_s\left/\alpha_{1}\right.\right\}\right), & h > h_{\subfire}, \\
    \beta_{2} \cdot \left(1 - \exp\left\{- L_p\left/\alpha_{2}\right.\right\}\right), & h = h_{\subfire}, \\
    \beta_{3} \cdot \left(1 - \exp\left\{- L_s\left/\alpha_{3}\right.\right\}\right), & h < h_{\subfire} .
    \end{cases}
    \label{eq:t1}
\end{equation}
The parameters $\beta_i$ and $\alpha_i$ for determining $t_1$ are summarized in Table~\ref{tab:fire_spread_parameters}. In this study, we take $r_{\mathrm{up}}=0.95$ and $r_{\mathrm{down}}=0.97$.
\begin{table}[ht]
    \centering
    \caption{Fire spread parameters for $t_1$ calculations.}
    \begin{tabular}{lcc}
        \toprule
        Case  & $\beta_i$ & $\alpha_i$  \\
        \midrule
        $i=1$, Upward spread & $16 \left.\left(1-r_{\mathrm{up}}^{\left|h-h_{\subfire}\right|}\right)\right/\left(1-r_{\mathrm{up}}\right)$ & $10$  \\
        $i=2$, Horizontal spread & $18$ & $18$  \\
        $i=3$, Downward spread & $30 \left.\left(1-r_{\mathrm{down}}^{\left|h-h_{\subfire}\right|}\right)\right/\left(1-r_{\mathrm{down}}\right)$ & $5$  \\
        \bottomrule
    \end{tabular}
    \label{tab:fire_spread_parameters}
\end{table}

\figref{fig:t1_curve} illustrates the $t_1$ curves for various fire scenarios: (1) fire originating on the lower floor, $h-h_{\subfire}=1$ with rapid upward spread, (2) fire on the same floor, $h=h_{\subfire}$ with the fastest spread, and (3) fire on the upper floor, $h_{\subfire}-h=1$ with slow downward spread. The exponential decay in $t_1$ reflects the accelerating fire propagation speed as the distance increases. \figref{fig:t1_curve} also indicates that the employed simplified model is consistent with the Markov chain-based dynamic model given by \cite{cheng_dynamic_2011}, where the rooms at the same floor of the fire point start flashover slightly before the corresponding upper floors. Additionally, $\beta_{1}$ and $\beta_{3}$ are the summation of a geometric sequence, where story level $h$ is the index. The common ratios $r_{\mathrm{up}}<1$ in $\beta_{1}$ and $r_{\mathrm{down}}<1$ in $\beta_{3}$ indicate that the fire speeds up to spread through the next story, which is consistent with the real-world fire spread mechanism given in \cite{hokugo_mechanism_2000}. The temperature profile within the range $t \in [0, t_{\thre}]$ is subsequently used as the thermal load in OpenSeesRT simulations to compute displacements at each structural node at time $t_{\thre}$.
\begin{figure}[h!]
    \centering
    \includegraphics[width=0.8\linewidth]{figures/m204_t1_curve.pdf}
    \caption{Three examples for the $t_1$ curve.}
    \label{fig:t1_curve}
\end{figure}

\revise{
\textit{Remark:} The effects of structural elements, such as concrete floor slabs and partitions, are not explicitly modeled in our approach. Instead, their influence is implicitly captured through the careful selection of the parameters $ \alpha, \beta, r_\mathrm{up} $, and $ r_\mathrm{down} $. This parameterization provides a unified framework for generating temperature fields. Indeed, fire propagation is governed by a multitude of factors and remains an open research question. For instance, if the fire resistance of a floor slab is enhanced by fire protective coating, the corresponding model can account for this by decreasing $\alpha_1$ \& $\alpha_3$, increasing $\beta_1$ \& $\beta_3$, and adopting larger values for $r_\mathrm{up}$ \& $r_\mathrm{down}$, which collectively slow down the vertical spread of fire. Conversely, scenarios involving higher amounts of combustible materials would warrant the opposite adjustments. This flexible and integrated approach avoids the need to design separate models for different fire propagation scenarios while still capturing the essential effects.
}

\revise{
In conclusion, our rule-based approach is a computationally efficient method for approximating fire temperature fields, enabling large-scale dataset generation to train predictive models. By combining ISO 834 fire curves with spatial considerations and embedding structural effects through parameter calibration, the method achieves a balanced trade-off between accuracy and scalability, making it a practical solution for thermal load modeling in fire scenarios. After generating the temperature of each beam or column according to the middle point, the temperature is applied as uniform thermal load to the elements of the structure in question using OpenSeesRT. 
}

% In conclusion, this rule-based approach is a computationally efficient method to approximate fire temperature fields, enabling large-scale dataset generation to train predictive models. By combining ISO 834 fire curves with spatial considerations, the method balances accuracy and scalability, making it a practical solution for thermal load modeling in fire scenarios.

% \subsection{Interstory Drift Ratio}
\subsection{OpenSeesRT Simulation}
\label{subsec:opensees_simulation}

The thermal and mechanical responses of 3D frame structures under combined fire and gravity loads are simulated using OpenSeesRT \cite{perez2024openseesrt}. \revise{In the simulation, the IDR of each node at $t_{\thre}$ is computed using the computed nodal displacements. Each structural model features six degrees of freedom per node (3 translational  and 3 rotational), with linear geometrical transformations (\texttt{geomTransf: Linear}) defining how the element local coordinate systems are mapped to the global coordinate system and assuming small displacements and rotations. Although OpenSeesRT allows a variety of options for modeling finite deformations, in the present simulations and mainly for simplicity, we did not consider large deformations. All bottom nodes (nodes on the ground) are fully constrained in all six degrees of freedom, while degrees of freedom os all other nodes are free.} Material behavior is temperature-dependent and modeled with \texttt{Steel01Thermal}, while fiber-based sections (\texttt{FiberThermal}) capture nonlinear interactions between thermal and mechanical responses at the cross-section level. \revise{Structural elements are represented as displacement-based Euler-Bernoulli beam-columns (\texttt{dispBeamColumnThermal}). This element  formulation accounts for thermal strains (temperature gradients) in the section, which is discretized into fibers. Numerical integration is used along the length of each element using three integration (Gauss) points, one at each end and the third in the middle of the element.}

{\revise{Thermal expansion of steel members plays a crucial role in IDR development. In reality, reinforced concrete floor slabs heat at a different rate than steel members due to their higher thermal mass and lower thermal conductivity. This differential heating can lead to restrained thermal expansion, introducing axial compression in beams and affecting the overall structural response. In this study, explicit {\em{composite action}} between steel members and concrete slabs is not modeled. Instead, our approach focuses on isolating the response of the steel structural frame, which is often the critical load-bearing component in fire scenarios. This assumption aligns with prior studies \cite{Possidente_2024} demonstrating that steel structures reach thermal equilibrium with surrounding gases quickly, allowing the use of uniform thermal loading in fire analysis. Future work could enhance this framework by incorporating slab-beam interaction effects, through a refined FEA for an extended dataset where constraints imposed by floor slabs are explicitly considered.}

The analysis begins with the application of gravity loads, followed by incremental thermal loads simulating the fire exposure. A static nonlinear solver using  \texttt{ExpressNewton} algorithm ensures convergence, while the \texttt{NormDispIncr} test maintains accuracy. An incremental \texttt{LoadControl} scheme with small step sizes is employed to guarantee numerical stability, using 10\% for gravity loads and 1\% for thermal loads. 

\revise{
In the thermal load analysis, uniform thermal load is applied to each beam or column, i.e., the temperature of each element is set to be that at the middle point, according to \secref{subsec:thermal_load_generation}. The \texttt{Steel01Thermal} material allows the properties (e.g., Young's modulus and yield strength) to be adjusted at increasing temperatures according to \cite{EN1993} using its Table 3.1: Reduction factors for the stress-strain relationship of carbon steel at elevated temperatures. For example, if the Young’s modulus at ambient temperature is $E_0$, then as the temperature ($T$) increases, the modulus changes as $E(T) = \eta (T) \times E_0$. \cite{EN1993} directly provides the values of $\eta(T) \in \left[0,1\right] $ at every $100 ^\circ\mathrm{C}$ interval and recommends using linear interpolation to obtain $\eta(T)$ for intermediate values of $T$.
} OpenSeesRT documentation \cite{OpenSeesThermalExamples} provides several examples of thermal analyses.

This modeling framework accommodates variations in material properties, cross-sectional geometries, and temperature profiles, providing robust simulations of structural behavior under fire conditions. The primary settings and configurations for the OpenSeesRT simulations are summarized in \tabref{tab:ops_detail}.
\begin{table}[h!]
    \centering
        \caption{Key settings of OpenSeesRT simulations.}
    \begin{tabular}{l|>{\raggedright\arraybackslash}p{0.6\linewidth}} %
    \toprule
    Modeling Aspect     & Details \\
    \midrule
    Geometry            & 3D models; 6 degrees of freedom per node \\
    Transformation      & geomTransf: Linear \\ 
    Material            & Steel01Thermal \\
    Section             & FiberThermal; Cross-section: $0.1$ m $\times$ $0.1$ m \\ 
    Element type        & {dispBeamColumnThermal} \\ 
    Loading             & Gravity loads: {beamUniform}; Thermal loads: {beamThermal} \\
    Integration scheme  & Incremental {LoadControl}; Step size: $10\%$ (gravity analysis), $1\%$ (thermal analysis) \\
    Nonlinear solver    & {ExpressNewton} algorithm; {UmfPack} solver; Convergence test: {NormDispIncr} tolerance: $10^{-8}$; Maximum \# iterations per step: $1000$. \\ 
    \bottomrule
    \end{tabular}
    \label{tab:ops_detail}
\end{table}

For each structure in the labeled dataset, 30 fire points are selected using a dual-granularity approach, \revise{i.e., two-stage sampling strategy,} to ensure they are well-distributed. Specifically, rooms are sequentially selected, with one fire point randomly chosen within each selected room. If a building is large and contains more than 30 rooms, we randomly select 30 rooms without replacement, i.e., ensuring that no more than one fire point is located in the same room. Conversely, if the building is small and has fewer than 30 rooms, all rooms are initially selected, with one fire point randomly assigned to each room. Additionally, rooms are then selected with replacement until a total of 30 fire points are assigned. \revise{The room-level sampling prioritizes selecting distinct rooms to avoid spatial clustering of fire points, while the point-level sampling ensures intra-room variability. This approach aligns with stratified sampling principles commonly used for efficient spatial representation, where multi-stage sampling strategies optimize coverage and variability, e.g., \cite{arunachalam_generalized_2023}, and enables a more comprehensive characterizing of how the structures respond under fire conditions.}
% This selection method prevents fire points from clustering too closely while maintaining an element of randomness. By distributing fire points in this manner, the 30 fire scenarios are effectively utilized, enabling a more comprehensive characterizing of how the structures respond under fire conditions.

\subsection{Summary of the Dataset Generation}
As discussed in this section and related to  \figref{fig:dataset_generation_procedure}, three key steps were considered in the development of the dataset: 
\begin{enumerate}
    \item {\bf{Filtering process}}: Structures with MIDR exceeding $1\%$ under gravity loads were excluded,  resulting in $1,573$ labeled structures retained for fire simulation and $16,050$ unlabeled structures for training the MFSP predictor.
    \item {\bf{Fire simulations}}: For each retained labeled structure, 30 fire scenarios were simulated using OpenSeesRT, yielding $47,190$ fire cases.
    \item {\bf{Data distribution check}}: MIDR distributions for labeled and unlabeled data under gravity loads were highly similar, because both datasets were generated using the same method. Under fire conditions, the MIDR distribution shifted, reflecting significant structural deformation with values reaching a maximum of about 6\%, an average of 1.70\%, and a standard deviation of 1.12\%. This step ensured a diverse and comprehensive dataset for the proposed predictive framework.
\end{enumerate}
The statistical distribution histograms for MIDR (after applying the $1\%$ filtering threshold \revise{for gravity load responses}) under different loading conditions are plotted in \figref{fig:histogram_mdr}. Figures \ref{fig:histogram_mdr}(a) and \ref{fig:histogram_mdr}(b) show the MIDR distributions of the labeled and unlabeled data, respectively, under gravity loads only. \figref{fig:histogram_mdr}(c) shows the MIDR distribution of the labeled data under the combined effects of gravity and fire loads. Fire load causes the structures to significantly deform, leading to a noticeably \revise{right-skewed} MIDR distribution.

\begin{figure*}[h!]
    \centering
    \includegraphics[width=\linewidth]{figures/histogram_mdr.pdf}
    \caption{Histograms of MIDR for labeled and unlabeled structures with gravity loads and fire cases.}
    \label{fig:histogram_mdr}
\end{figure*}

\revise{
This dataset provides the basis for training and testing the performance of the GNN-based framework. Although we employed a simplified rule-based thermal load generation method compared with conventional CFD-based simulations, the temperature field, the changes of the material properties, and the response of the structures, are all still highly nonlinear and complex. Therefore, it is still a challenging task for the NN to predict the MIDRs based on this dataset.
}
\begin{tcolorbox}[title={The Prompt used for Translation}]
You are a highly skilled translator tasked with translating various types of content from English into \{\{ language \}\}. Follow these instructions carefully to complete the translation task.

You will receive a user-bot conversation in XML format. Please follow a three-step translation process:

\begin{enumerate}
  \item \textbf{Initial Translation:} Translate the input content into \{\{ language \}\}, preserving the original intent and keeping the original paragraph and text format unchanged. Do not delete or omit any content, and ensure that all original Markdown elements (e.g., images, code blocks) are preserved.
  \item \textbf{Reflection and Feedback:} Carefully review both the source text and your translation. Provide constructive criticism and specific suggestions to improve the translation in terms of:
    \begin{enumerate}[label=(\roman*)]
      \item \textbf{Accuracy:} Correct errors of addition, mistranslation, omission, or untranslated text.
      \item \textbf{Fluency:} Apply \{\{ language \}\} grammar, spelling, and punctuation rules while avoiding unnecessary repetitions.
      \item \textbf{Style:} Ensure that the translation reflects the style of the source text and considers any relevant cultural context.
    \end{enumerate}
  \item \textbf{Refinement:} Based on your reflections, refine and polish your translation.
  \item \textbf{Fallback:} If you are not confident in translating the conversation, please return ``\texttt{<stop></stop>}''.
\end{enumerate}

\bigskip
\textbf{Output:}

For each step of the translation process, output your results within the appropriate XML tags as follows:
\begin{verbatim}
<step1_initial_translation>
[Insert your initial translation here]
</step1_initial_translation>

<step2_reflection>
[Insert your reflection on the translation, including a list 
of specific, helpful, and constructive suggestions for 
improvement. Each suggestion should address a specific 
part of the translation.]
</step2_reflection>

<step3_refined_translation>
[Insert your refined and polished translation here]
</step3_refined_translation>
\end{verbatim}

Ensure that your final translation in step 3 accurately reflects the original meaning while sounding natural in \{\{ language \}\}.

Here is the original conversation:
\label{box:trans_prompt}
\end{tcolorbox}


\begin{table*}[htbp]
    \centering
    \small
    \begin{tabular}{p{14cm}}
     \toprule
\#\#\#  Objective: \\
Generate a 5-day family travel itinerantry that satisfies all specified requirements while adhering to highly fine-grained constraints. The generated itinerary should balance real-time adaptability, strict hard attributes, and semantic soft attributes. \\

\#\#\# User Profile: \\
 - Travelers: 2 adults + 1 child (age 8) \\
 - Budget: $<=$ \$300/day (total \$1,500 for the trip) \\
 - Activity Balance: 70\% educational/cultural experiences, 20\% relaxation, 10\% family-friendly shopping. \\

\#\#\# Hard Attributes: \\
- Activity Scheduling: \\
\quad- Each activity must have a defined start and end time, ensuring there is no overlap between activities. \\
\quad- A break period from 13:00-14:30 is mandatory daily. \\
\quad- Each activity must fit within a 2-hour window unless otherwise specified. \\

- Budget Requirements: \\
\quad- Each day’s total cost (including transportation, food, and activities) must not exceed \$300. \\
\quad- Transportation is limited to metro and walking only, with a maximum of 3 metro rides per day. \\

- Location Constraints: \\
\quad- Must-visit locations: City Zoo (Day 1) and Science Museum (Day 3). \\
\quad- Activities must occur in geographically adjacent areas to minimize walking distance. \\

- Keyword Requirements: \\
\quad- Each day’s description must include specific keywords. For example: \\
\quad- Day 1: “wildlife,” “exploration,” and “interactive learning.” \\
\quad- Day 3: “science,” “innovation,” and “hands-on exhibits.” \\

- Structure Constraints: \\
\quad- Each day’s itinerary must consist of 4 sections: \\
\quad\quad- Morning activity \\
\quad\quad- Break/lunch period \\ 
\quad\quad- Afternoon activity \\
\quad\quad- Evening summary (limited to 50 words) \\

\#\#\# Soft Attributes \\
- Tone and Emotion: \\ 
\quad- Day 1: Use a tone that conveys “excitement and discovery.” \\ 
\quad- Day 3: Use a tone that conveys “curiosity and wonder.” \\
- Language Style: \\ 
\quad- Use descriptive, vivid, and family-friendly language throughout. \\
\quad- Include at least one metaphor or simile per day (e.g., "The Science Museum felt like stepping into the future!"). \\
- Visual Details: \\
\quad- Each activity must include specific sensory details (e.g., "the bright colors of the parrots at the zoo" or "the tinkling sound of water fountains at the park").

- Adaptive Adjustments (Real-time Constraints): \\
\quad- Weather Sensitivity: \\
\quad\quad- If the rain forecast exceeds 60\%, replace outdoor activities with indoor alternatives while keeping the overall tone and keywords intact. \\ 
\quad- Physical Endurance: \\
\quad\quad- If a day’s total walking distance exceeds 10 kilometers, the next day’s activities must reduce walking by 30\%. \\
\quad- Health Responsiveness: \\
\quad\quad- If a health-related issue arises (e.g., fatigue or illness), adjust the itinerary dynamically to: \\
\quad\quad- Reduce activity duration to half. \\ 
\quad\quad- Substitute the activity with a more relaxing or passive option. \\
\bottomrule
    \end{tabular}
    \caption{The complete travel planner case study.}
    \label{tab:travel_planner_case}
\end{table*}


\end{document}
