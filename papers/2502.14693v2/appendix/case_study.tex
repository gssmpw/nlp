\newpage
\section{Case Study}
\label{app:case}
We detail the case study below. The refined insight generated by I-MCTS suggests that "Developing a feature that captures the temporal difference between the vehicle's manufacturing year and purchase date (PurchDate-VehYear) would enable more precise representation of the vehicle's age at the time of purchase.", which is more task-specific.
In contrast, the insights generated by SELA exhibit significant homogeneity and lack specificity. 
\subsection{Overview of I-MCTS's search process}

\begin{lstlisting}[style=txtfile]
Number of simulations: 10
Node id: 0
Plans: 
1. Perform exploratory data analysis (EDA) on the training dataset to understand the distribution of features and identify any missing values or outliers.
2. Preprocess the data by handling missing values, encoding categorical variables, and normalizing numerical features. Ensure that the preprocessing steps are applied consistently to the train, dev, and test sets.
3. Create new features that might improve the model's performance, such as interaction terms or polynomial features, without using the target column.
4. Train a machine learning model on the preprocessed training data. Use the dev set to tune hyperparameters and validate the model's performance.
5. Evaluate the model's performance on the training, dev, and test sets using the F1 score. Save the predictions for the dev and test sets in the specified format.
Simulated: True
Score: avg score: 0.17933643065796115, simulated score: {'test_score': 0.08495018353434715, 'dev_score': 0.05279187817258883, 'score': 0.05279187817258883}, Visits: 10

	Node id: 0-0
	Plans: 
	3. Create interaction features between numerical and categorical variables. For example, you can create new features by multiplying numerical features with the encoded values of categorical features. This can help the model capture interactions that might be important for predicting the target variable. Additionally, consider creating polynomial features for numerical columns to capture non-linear relationships. For instance, you can add squared and cubed terms of numerical features to the dataset.
	Simulated: True
	Score: avg score: 0.19208087615838249, simulated score: {'test_score': 0, 'dev_score': 0, 'score': 0}, Visits: 2

		Node id: 0-0-0
		Plans: 
		4. Implement an ensemble of models using techniques such as stacking or boosting. For example, use a combination of Gradient Boosting Machines (GBM), Random Forests, and XGBoost. This can help capture different aspects of the data and improve the robustness of the model. Additionally, use cross-validation to tune hyperparameters for each model in the ensemble, ensuring that the model generalizes well to unseen data.
		Simulated: False
		Score: avg score: 0, simulated score: {}, Visits: 0

		Node id: 0-0-1
		Plans: 
		4. Introduce a neural network model with multiple hidden layers to capture complex interactions and non-linear relationships in the data. Use techniques like dropout and batch normalization to prevent overfitting. Additionally, experiment with different activation functions (e.g., ReLU, LeakyReLU) and loss functions (e.g., binary cross-entropy) to find the best configuration. Use early stopping to prevent overfitting and ensure the model generalizes well to the dev set.
		Simulated: False
		Score: avg score: 0, simulated score: {}, Visits: 0

		Node id: 0-0-2
		Plans: 
		4. Implement a Bayesian Optimization approach for hyperparameter tuning. Bayesian Optimization is a sequential design strategy for global optimization of black-box functions, which can be particularly effective in finding the optimal hyperparameters for machine learning models. This method can help in exploring the hyperparameter space more efficiently and effectively, leading to better model performance. Use a library like Optuna or Hyperopt to automate the process and integrate it with cross-validation to ensure that the model generalizes well to unseen data.
		Simulated: False
		Score: avg score: 0, simulated score: {}, Visits: 0

		Node id: 0-0-3
		Plans: 
		4. Implement a hybrid model that combines a traditional machine learning model (e.g., Random Forest or XGBoost) with a neural network. The traditional model can capture the linear and tree-based relationships, while the neural network can capture the complex and non-linear interactions. Use the predictions from the traditional model as additional features (meta-features) for the neural network. This hybrid approach can help the model capture a broader range of patterns in the data, potentially leading to improved performance on the development set.
		Simulated: False
		Score: avg score: 0, simulated score: {}, Visits: 0

		Node id: 0-0-4
		Plans: 
		4. Implement a feature selection process using Recursive Feature Elimination (RFE) or permutation importance to identify and retain the most relevant features for the model. This can help reduce overfitting by removing noise and irrelevant features, leading to a more robust model. Additionally, use a cross-validated feature selection approach to ensure that the selected features generalize well to unseen data. This can improve the model's performance on the development set by focusing on the most predictive features.
		Simulated: True
		Score: avg score: 0.38416175231676497, simulated score: {'train_score': 0.9999073645206114, 'dev_score': 0.38416175231676497, 'test_score': 0.36547776309995594, 'score': 0.38416175231676497}, Visits: 1

	Node id: 0-1
	Plans: 
	3. Introduce domain-specific feature engineering by creating new features that are relevant to the automotive industry. For example, you can create a feature that calculates the age of the vehicle at the time of purchase (PurchDate - VehYear), which might be a more meaningful representation of vehicle age than the current 'VehicleAge' feature. Additionally, consider creating a feature that represents the difference between the current and acquisition MMR prices (e.g., MMRCurrentAuctionAveragePrice - MMRAcquisitionAuctionAveragePrice), which could capture the depreciation or appreciation of the vehicle's value. These domain-specific features can provide the model with more relevant information and potentially improve its performance.
	Simulated: False
	Score: avg score: 0.0, simulated score: {'test_score': 0, 'dev_score': 0, 'score': 0}, Visits: 1

	Node id: 0-2
	Plans: 
	3. Introduce time-based features to capture temporal trends and seasonality. For example, extract the month and day of the week from the 'PurchDate' column. This can help the model understand if certain times of the year or days of the week have a significant impact on the likelihood of a bad buy. Additionally, consider creating a feature that represents the time difference between the vehicle's year and the purchase date (PurchDate - VehYear) to capture the age of the vehicle at the time of purchase more accurately. These time-based features can provide the model with additional context and potentially improve its performance.
	Simulated: True
	Score: avg score: 0.41769512423377475, simulated score: {'test_score': 0.37755545889517184, 'dev_score': 0.39700996677740863, 'score': 0.39700996677740863}, Visits: 2

		Node id: 0-2-0
		Plans: 
		4. Implement an ensemble of models, such as a Random Forest and a Gradient Boosting Machine (GBM), and use stacking to combine their predictions. This approach can help capture different aspects of the data and improve the robustness of the model. Additionally, use a more systematic hyperparameter tuning method, such as Bayesian optimization, to find the best settings for each model in the ensemble.
		Simulated: False
		Score: avg score: 0, simulated score: {}, Visits: 0

		Node id: 0-2-1
		Plans: 
		4. Implement a more sophisticated model training strategy by using a deep learning model, such as a neural network with multiple layers, to capture complex feature interactions. Additionally, incorporate techniques to handle class imbalance, such as oversampling the minority class or using class weights during training. Use K-fold cross-validation to ensure that the model generalizes well to unseen data and to avoid overfitting. Finally, experiment with different activation functions and regularization techniques to improve model performance.
		Simulated: False
		Score: avg score: 0, simulated score: {}, Visits: 0

		Node id: 0-2-2
		Plans: 
		4. Implement a more advanced model training strategy by using a neural network with attention mechanisms. Attention mechanisms can help the model focus on the most relevant features and capture complex interactions between features. Additionally, use a learning rate scheduler to dynamically adjust the learning rate during training, which can help the model converge more effectively and avoid getting stuck in local minima. Finally, incorporate early stopping to prevent overfitting by monitoring the performance on the development set and stopping training when performance starts to degrade.
		Simulated: False
		Score: avg score: 0, simulated score: {}, Visits: 0

		Node id: 0-2-3
		Plans: 
		4. Implement a more advanced model training strategy by using a neural network with a combination of convolutional and recurrent layers. Convolutional layers can help capture local patterns and feature interactions, while recurrent layers can handle sequential data and temporal dependencies. Use a learning rate scheduler to dynamically adjust the learning rate during training, which can help the model converge more effectively and avoid getting stuck in local minima. Additionally, incorporate dropout layers to prevent overfitting and improve the model's generalization ability. Finally, use K-fold cross-validation to ensure that the model generalizes well to unseen data and to avoid overfitting.
		Simulated: False
		Score: avg score: 0, simulated score: {}, Visits: 0

		Node id: 0-2-4
		Plans: 
		4. Implement a more advanced model training strategy by using a neural network with a combination of dense and embedding layers. Use embeddings for categorical variables to capture their latent representations and dense layers to model complex interactions between features. Incorporate techniques to handle class imbalance, such as focal loss, which assigns higher weights to misclassified samples, particularly from the minority class. Use a learning rate scheduler to dynamically adjust the learning rate during training, and incorporate early stopping to prevent overfitting. Finally, use K-fold cross-validation to ensure that the model generalizes well to unseen data and to avoid overfitting.
		Simulated: True
		Score: avg score: 0.43838028169014087, simulated score: {'test_score': 0.4241112828438949, 'dev_score': 0.43838028169014087, 'score': 0.43838028169014087}, Visits: 1

	Node id: 0-3
	Plans: 
	3. Introduce feature engineering based on domain knowledge and feature interactions. For example, create a feature that represents the ratio of the vehicle's odometer reading to its age (VehOdo / VehicleAge), which could provide insights into the vehicle's usage intensity. Additionally, consider creating interaction features between categorical and numerical variables, such as the interaction between the vehicle's make and its MMR prices (e.g., Make * MMRCurrentAuctionAveragePrice). These features can help the model capture more nuanced relationships and potentially improve its performance.
	Simulated: True
	Score: avg score: 0.036288232244686365, simulated score: {'test_score': 0.07784120394395433, 'dev_score': 0.036288232244686365, 'score': 0.036288232244686365}, Visits: 2

		Node id: 0-3-0
		Plans: 
		4. Implement a more robust hyperparameter tuning strategy using techniques like Grid Search or Randomized Search. This will help in finding the optimal set of hyperparameters that can improve the model's performance on the development set. Additionally, consider using cross-validation during the hyperparameter tuning process to ensure that the model generalizes well to unseen data. For example, you can use `GridSearchCV` or `RandomizedSearchCV` from `scikit-learn` to systematically explore different hyperparameter combinations and select the best model based on the F1 score on the cross-validation sets.
		Simulated: False
		Score: avg score: 0, simulated score: {}, Visits: 0

		Node id: 0-3-1
		Plans: 
		4. Implement an ensemble of models to improve the robustness and generalization of the predictions. Specifically, use a stacking ensemble where multiple base models (e.g., Random Forest, Gradient Boosting, and Logistic Regression) are trained and their predictions are combined using a meta-model (e.g., another Logistic Regression or a simple averaging). This approach can help capture different aspects of the data and reduce the risk of overfitting. Additionally, use Bayesian optimization for hyperparameter tuning to systematically explore the hyperparameter space and find the optimal settings for each base model and the meta-model.
		Simulated: False
		Score: avg score: 0, simulated score: {}, Visits: 0

		Node id: 0-3-2
		Plans: 
		4. Implement class weighting or oversampling techniques to address class imbalance in the training data. Since the target variable 'IsBadBuy' is likely imbalanced, using class weights in the loss function or oversampling the minority class can help the model learn from the underrepresented class more effectively. For example, in scikit-learn, you can set the `class_weight` parameter in the model to 'balanced' or use the `imbalanced-learn` library to perform oversampling techniques like SMOTE (Synthetic Minority Over-sampling Technique). This can improve the model's ability to predict the minority class, which is crucial for the F1 score.
		Simulated: False
		Score: avg score: 0, simulated score: {}, Visits: 0

		Node id: 0-3-3
		Plans: 
		4. Implement advanced regularization techniques such as L1 (Lasso) and L2 (Ridge) regularization to prevent overfitting and improve the model's generalization. Specifically, use Elastic Net regularization, which combines both L1 and L2 penalties, to balance the trade-off between feature selection and coefficient shrinkage. This can help in reducing the complexity of the model and improving its performance on the development set. Additionally, consider using dropout regularization if you are using neural networks, which randomly drops out (sets to zero) a number of output features of the layer during training, thus preventing the model from becoming too reliant on any single feature.
		Simulated: False
		Score: avg score: 0, simulated score: {}, Visits: 0

		Node id: 0-3-4
		Plans: 
		4. Implement a more sophisticated model architecture that can capture complex interactions and non-linear relationships, such as a Gradient Boosting Machine (GBM) or a Deep Neural Network (DNN). For GBM, use a library like XGBoost or LightGBM, which are known for their efficiency and performance in handling large datasets and capturing complex patterns. For DNN, use a multi-layer perceptron with appropriate activation functions (e.g., ReLU) and regularization techniques (e.g., dropout) to prevent overfitting. Additionally, use learning rate schedules and early stopping to optimize the training process and prevent overfitting. This can help the model better capture the nuances in the data and improve its performance on the development set.
		Simulated: False
		Score: avg score: 0.036288232244686365, simulated score: {'test_score': 0.07784120394395433, 'dev_score': 0.036288232244686365, 'score': 0.036288232244686365}, Visits: 1

	Node id: 0-4
	Plans: 
	3. Introduce feature engineering using dimensionality reduction techniques such as Principal Component Analysis (PCA) or t-Distributed Stochastic Neighbor Embedding (t-SNE) to create new features that capture the underlying structure of the data. For example, apply PCA to the numerical features to reduce dimensionality and create a set of principal components that can be used as new features. This can help in capturing the most important information in the data while reducing noise and multicollinearity. Additionally, consider using autoencoders to learn a compressed representation of the data, which can be used as input features for the model. These techniques can help in uncovering hidden patterns and improving the model's performance.
	Simulated: True
	Score: avg score: 0.22422198156666764, simulated score: {'test_score': 0.3816859221223791, 'dev_score': 0.38071487946799665, 'score': 0.38071487946799665}, Visits: 2

		Node id: 0-4-0
		Plans: 
		4. Implement an ensemble of models to improve the robustness and performance of the predictions. Specifically, use a combination of different algorithms such as Random Forest, Gradient Boosting Machines (GBM), and XGBoost. Train each model on the preprocessed data and combine their predictions using techniques like stacking or voting. This approach can help in capturing different patterns in the data and reducing overfitting, leading to better generalization and higher F1 scores on the development set.
		Simulated: False
		Score: avg score: 0, simulated score: {}, Visits: 0

		Node id: 0-4-1
		Plans: 
		4. Introduce a neural network model with dropout regularization to improve the model's ability to generalize. Specifically, use a multi-layer perceptron (MLP) with dropout layers to prevent overfitting. Dropout randomly sets a fraction of input units to 0 at each update during training, which helps in reducing overfitting and improving the model's performance on unseen data. Additionally, use early stopping to monitor the performance on the dev set and stop training when the performance starts to degrade, ensuring that the model does not overfit to the training data. This approach can help in capturing complex interactions in the data and improving the F1 score on the development set.
		Simulated: False
		Score: avg score: 0, simulated score: {}, Visits: 0

		Node id: 0-4-2
		Plans: 
		4. Introduce a deep learning model with attention mechanisms to improve the model's ability to capture complex and non-linear relationships in the data. Specifically, use a Transformer-based model or a model with self-attention layers. These models can dynamically weigh the importance of different features and interactions, which can be particularly useful for datasets with a large number of features and complex relationships. This approach can help in improving the model's performance on the development set by better capturing the underlying patterns in the data.
		Simulated: False
		Score: avg score: 0, simulated score: {}, Visits: 0

		Node id: 0-4-3
		Plans: 
		4. Implement a grid search with cross-validation to systematically tune hyperparameters for the model. Specifically, use GridSearchCV from scikit-learn to perform an exhaustive search over a specified parameter grid. This will help in finding the optimal hyperparameters for the model, which can significantly improve its performance. Additionally, consider using K-Fold cross-validation to ensure that the model is robust and generalizes well to unseen data. This approach can help in reducing overfitting and improving the F1 score on the development set.
		Simulated: False
		Score: avg score: 0, simulated score: {}, Visits: 0

		Node id: 0-4-4
		Plans: 
		4. Introduce a hybrid model that combines a traditional machine learning model (e.g., Random Forest) with a deep learning model (e.g., a Convolutional Neural Network, CNN). The CNN can be used to extract high-level features from the numerical and categorical data, which can then be concatenated with the features from the Random Forest. This hybrid approach can help in capturing both linear and non-linear relationships in the data, leading to improved performance. Specifically, use the CNN to process the numerical features and the Random Forest to handle the categorical features, and then combine their outputs using a fully connected layer. This approach can help in improving the F1 score on the development set by leveraging the strengths of both models.
		Simulated: True
		Score: avg score: 0.06772908366533864, simulated score: {'train_score': -1, 'dev_score': 0.06772908366533864, 'test_score': 0.061683220073183484, 'score': 0.06772908366533864}, Visits: 1

Generated 30 unique codes.
Best node: 0-2-4, score: {'test_score': 0.4241112828438949, 'dev_score': 0.43838028169014087, 'score': 0.43838028169014087}
Dev best node: 0-2-4, score: {'test_score': 0.4241112828438949, 'dev_score': 0.43838028169014087, 'score': 0.43838028169014087}
Grader score: 0.4241112828438949
\end{lstlisting}

\subsection{Overview of baseline SELA's search process}

\begin{lstlisting}[style=txtfile]
Number of simulations: 10
[Node 0]
Plans: 
1. Perform exploratory data analysis (EDA) on the training dataset to understand the data distribution, missing values, and potential outliers.
2. Preprocess the data by handling missing values, encoding categorical variables, and normalizing numerical features.
3. Create new features that might improve the model's performance, such as interaction terms or domain-specific features.
4. Train a machine learning model using the preprocessed and feature-engineered training data.
5. Evaluate the model's performance on the development set using the F1 score.
6. Make predictions on the test set and save the results to `test_predictions.csv`.
7. Make predictions on the development set and save the results to `dev_predictions.csv`.
8. Save the trained model to `model.pkl`.
9. Print the performance metrics (F1 scores) for the training, development, and test sets.
Simulated: True
Score: avg score: 0.13309993361366776, simulated score: {'test_score': 0.1530317613089509, 'dev_score': 0.14690841469084145, 'score': 0.14690841469084145}, Visits: 10

	[Node 0-0]
	Plans: 
	3. Before creating new features such as interaction terms or domain-specific features, analyze the distribution of 'IsBadBuy' to understand the class imbalance. This analysis should be performed on the training set after splitting the data to ensure that the insights gained are representative of the data the model will be trained on. Use these insights to guide the creation of new features that might improve the model's performance.
	Simulated: True
	Score: avg score: 0.14587155963302753, simulated score: {'test_score': 0.15175644028103044, 'dev_score': 0.14587155963302753, 'score': 0.14587155963302753}, Visits: 2

		[Node 0-0-0]
		Plans: 
		4. Train a machine learning model using the preprocessed and feature-engineered training data. Before training, analyze the distribution of 'IsBadBuy' to understand the class imbalance. If necessary, apply appropriate techniques to handle the imbalance during the data split and model training.
		Simulated: False
		Score: avg score: 0.14587155963302753, simulated score: {'test_score': 0.15175644028103044, 'dev_score': 0.14587155963302753, 'score': 0.14587155963302753}, Visits: 1

		[Node 0-0-1]
		Plans: 
		4. Train a machine learning model using the preprocessed and feature-engineered training data. Before training, explore the correlation between 'VehicleAge' and 'IsBadBuy' to determine if older vehicles are more likely to be classified as 'IsBadBuy'. Ensure that this exploration is conducted on the training set if the data is split into training and testing sets.
		Simulated: False
		Score: avg score: 0, simulated score: {}, Visits: 0

		[Node 0-0-2]
		Plans: 
		4. Train a machine learning model using the preprocessed and feature-engineered training data. Before training, investigate the relationship between 'MMRAcquisitionAuctionAveragePrice' and 'MMRAcquisitionAuctionCleanPrice' with 'IsBadBuy' to understand the pricing impact. Ensure that this analysis is conducted on the training data split to avoid data leakage.
		Simulated: False
		Score: avg score: 0, simulated score: {}, Visits: 0

		[Node 0-0-3]
		Plans: 
		4. Train a machine learning model using the preprocessed and feature-engineered training data. Before training, examine the frequency of 'Make' and 'Model' in the training data to identify which brands or models are more prone to being kicks. Ensure that this analysis is performed on the training set if the data is split.
		Simulated: False
		Score: avg score: 0, simulated score: {}, Visits: 0

		[Node 0-0-4]
		Plans: 
		4. Train a machine learning model using the preprocessed and feature-engineered training data. Before training, visualize the distribution of 'VehOdo' to understand if high mileage vehicles are more likely to be problematic. Ensure that the data is split appropriately for training and validation, and integrate the insights from the 'VehOdo' distribution into the model training process.
		Simulated: False
		Score: avg score: 0, simulated score: {}, Visits: 0

	[Node 0-1]
	Plans: 
	3. Create new features that might improve the model's performance, such as interaction terms or domain-specific features. Specifically, explore the correlation between 'VehicleAge' and 'IsBadBuy' to see if older vehicles are more likely to be classified as 'kicks'. Consider adding interaction terms or derived features based on this relationship, and ensure that any new features are integrated into the model training process, including any data splits you perform.
	Simulated: True
	Score: avg score: 0.14587155963302753, simulated score: {'test_score': 0.15175644028103044, 'dev_score': 0.14587155963302753, 'score': 0.14587155963302753}, Visits: 2

		[Node 0-1-0]
		Plans: 
		4. Train a machine learning model using the preprocessed and feature-engineered training data. Before training, analyze the distribution of 'IsBadBuy' to understand the class imbalance, and ensure that any data splitting maintains the distribution of the target variable to avoid bias in the training and validation sets.
		Simulated: False
		Score: avg score: 0, simulated score: {}, Visits: 0

		[Node 0-1-1]
		Plans: 
		4. Train a machine learning model using the preprocessed and feature-engineered training data. Before training, explore the correlation between 'VehicleAge' and 'IsBadBuy' to determine if older vehicles are more likely to be classified as 'kicks'. Ensure that this exploration is conducted on the training data split to avoid data leakage.
		Simulated: False
		Score: avg score: 0.14587155963302753, simulated score: {'test_score': 0.15175644028103044, 'dev_score': 0.14587155963302753, 'score': 0.14587155963302753}, Visits: 1

		[Node 0-1-2]
		Plans: 
		4. Train a machine learning model using the preprocessed and feature-engineered training data. Before training, investigate the relationship between 'MMRAcquisitionAuctionAveragePrice' and 'MMRAcquisitionAuctionCleanPrice' with 'IsBadBuy' to understand the pricing impact. Ensure that this analysis is conducted on the training data split to avoid data leakage.
		Simulated: False
		Score: avg score: 0, simulated score: {}, Visits: 0

		[Node 0-1-3]
		Plans: 
		4. Train a machine learning model using the preprocessed and feature-engineered training data. Before training, examine the frequency of 'Make' and 'Model' in the dataset to identify which brands or models are more prone to being kicks. Ensure that this analysis is performed on the training set if the data is split into training and testing sets.
		Simulated: False
		Score: avg score: 0, simulated score: {}, Visits: 0

		[Node 0-1-4]
		Plans: 
		4. Before training the machine learning model using the preprocessed and feature-engineered training data, visualize the distribution of 'VehOdo' to understand if high mileage vehicles are more likely to be problematic. Ensure that this visualization is performed on the training set after splitting the data into training and validation sets.
		Simulated: False
		Score: avg score: 0, simulated score: {}, Visits: 0

	[Node 0-2]
	Plans: 
	3. Create new features that might improve the model's performance, such as interaction terms or domain-specific features. Specifically, investigate the relationship between 'MMRAcquisitionAuctionAveragePrice' and 'MMRAcquisitionAuctionCleanPrice' with 'IsBadBuy' to understand the pricing impact. Consider creating interaction terms or derived features based on these insights. If the data needs to be split, ensure that the investigation of these relationships is conducted on the training set to avoid data leakage.
	Simulated: True
	Score: avg score: 0.14587155963302753, simulated score: {'test_score': 0.15175644028103044, 'dev_score': 0.14587155963302753, 'score': 0.14587155963302753}, Visits: 2

		[Node 0-2-0]
		Plans: 
		4. Train a machine learning model using the preprocessed and feature-engineered training data. Before training, analyze the distribution of 'IsBadBuy' to understand the class imbalance. If necessary, apply appropriate techniques to handle the class imbalance during the data split and model training process.
		Simulated: False
		Score: avg score: 0, simulated score: {}, Visits: 0

		[Node 0-2-1]
		Plans: 
		4. Train a machine learning model using the preprocessed and feature-engineered training data. Before training, explore the correlation between 'VehicleAge' and 'IsBadBuy' to determine if older vehicles are more likely to be classified as 'IsBadBuy'. Ensure that this analysis is conducted on the training data split to avoid data leakage.
		Simulated: False
		Score: avg score: 0, simulated score: {}, Visits: 0

		[Node 0-2-2]
		Plans: 
		4. Train a machine learning model using the preprocessed and feature-engineered training data. Before training, investigate the relationship between 'MMRAcquisitionAuctionAveragePrice' and 'MMRAcquisitionAuctionCleanPrice' with 'IsBadBuy' to understand the pricing impact. Ensure that this analysis is conducted on the training set if the data is split.
		Simulated: False
		Score: avg score: 0.14587155963302753, simulated score: {'test_score': 0.15175644028103044, 'dev_score': 0.14587155963302753, 'score': 0.14587155963302753}, Visits: 1

		[Node 0-2-3]
		Plans: 
		4. Train a machine learning model using the preprocessed and feature-engineered training data. Before training, examine the frequency of 'Make' and 'Model' in the training data to identify which brands or models are more prone to being kicks. Ensure that this analysis is integrated with the data split process, if applicable, to maintain the distribution of 'Make' and 'Model' across the training and validation sets.
		Simulated: False
		Score: avg score: 0, simulated score: {}, Visits: 0

		[Node 0-2-4]
		Plans: 
		4. Train a machine learning model using the preprocessed and feature-engineered training data. Before training, visualize the distribution of 'VehOdo' to understand if high mileage vehicles are more likely to be problematic. Ensure that the data is split appropriately for training and validation, and integrate the insights from the 'VehOdo' distribution into the model training process.
		Simulated: False
		Score: avg score: 0, simulated score: {}, Visits: 0

	[Node 0-3]
	Plans: 
	3. Create new features that might improve the model's performance, such as interaction terms or domain-specific features. Additionally, examine the frequency of 'Make' and 'Model' to identify which brands or models are more prone to being kicks. Use these insights to create new features, such as a 'HighKickProneBrand' flag or a 'ModelKickFrequency' score, which can be integrated into the model. If the data is split, ensure that the frequency analysis is performed on the training set to avoid data leakage.
	Simulated: True
	Score: avg score: 0.09650833089037442, simulated score: {'test_score': 0.15175644028103044, 'dev_score': 0.14587155963302753, 'score': 0.14587155963302753}, Visits: 2

		[Node 0-3-0]
		Plans: 
		4. Train a machine learning model using the preprocessed and feature-engineered training data. Before training, analyze the distribution of 'IsBadBuy' to understand the class imbalance. If necessary, apply appropriate techniques to handle the class imbalance, such as oversampling, undersampling, or using class weights. Ensure that the data is split into training and validation sets, and perform the class imbalance analysis on the training set before applying any balancing techniques.
		Simulated: False
		Score: avg score: 0, simulated score: {}, Visits: 0

		[Node 0-3-1]
		Plans: 
		4. Train a machine learning model using the preprocessed and feature-engineered training data. Before training, explore the correlation between 'VehicleAge' and 'IsBadBuy' to determine if older vehicles are more likely to be classified as 'IsBadBuy'. Ensure that this exploration is conducted on the training set after any data splits have been performed.
		Simulated: False
		Score: avg score: 0, simulated score: {}, Visits: 0

		[Node 0-3-2]
		Plans: 
		4. Train a machine learning model using the preprocessed and feature-engineered training data. Before training, investigate the relationship between 'MMRAcquisitionAuctionAveragePrice' and 'MMRAcquisitionAuctionCleanPrice' with 'IsBadBuy' to understand the pricing impact. Ensure that this analysis is conducted on the training set if the data is split.
		Simulated: False
		Score: avg score: 0, simulated score: {}, Visits: 0

		[Node 0-3-3]
		Plans: 
		4. Train a machine learning model using the preprocessed and feature-engineered training data. Before training, examine the frequency of 'Make' and 'Model' in the training data to identify which brands or models are more prone to being kicks. Ensure that this analysis is performed on the training set if the data is split.
		Simulated: False
		Score: avg score: 0, simulated score: {}, Visits: 0

		[Node 0-3-4]
		Plans: 
		4. Before training the machine learning model using the preprocessed and feature-engineered training data, visualize the distribution of 'VehOdo' to understand if high mileage vehicles are more likely to be problematic. Ensure that this visualization is performed on the training set after splitting the data into training and validation sets.
		Simulated: True
		Score: avg score: 0.047145102147721316, simulated score: {'test_score': 0.06040992448759439, 'dev_score': 0.047145102147721316, 'score': 0.047145102147721316}, Visits: 1

	[Node 0-4]
	Plans: 
	3. Create new features that might improve the model's performance, such as interaction terms or domain-specific features. Additionally, visualize the distribution of 'VehOdo' to see if high mileage vehicles are more likely to be problematic. If the data is being split, ensure that the visualization is performed on the training set to avoid data leakage.
	Simulated: True
	Score: avg score: 0.11584490186692196, simulated score: {'test_score': 0.14014598540145987, 'dev_score': 0.11584490186692196, 'score': 0.11584490186692196}, Visits: 1

Generated 29 unique codes.
Best node: 0, score: {'test_score': 0.1530317613089509, 'dev_score': 0.14690841469084145, 'score': 0.14690841469084145}
Dev best node: 0, score: {'test_score': 0.1530317613089509, 'dev_score': 0.14690841469084145, 'score': 0.14690841469084145}
Grader score: 0.1530317613089509
\end{lstlisting}
