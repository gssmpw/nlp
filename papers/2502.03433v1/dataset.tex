\section{Dataset}
\label{sec:dataset}

\subsection{Data Collection}

To analyze political discussions on Discord, we followed the methodology in \cite{singh2024Cross-Platform}, collecting messages from politically-oriented public servers in compliance with Discord's platform policies.

Using Discord's Discovery feature, we employed a web scraper to extract server invitation links, names, and descriptions, focusing on public servers accessible without participation. Invitation links were used to access data via the Discord API. To ensure relevance, we filtered servers using keywords related to the 2024 U.S. elections (e.g., Trump, Kamala, MAGA), as outlined in \cite{balasubramanian2024publicdatasettrackingsocial}. This resulted in 302 server links, further narrowed to 81 English-speaking, politics-focused servers based on their names and descriptions.

Public messages were retrieved from these servers using the Discord API, collecting metadata such as \textit{content}, \textit{user ID}, \textit{username}, \textit{timestamp}, \textit{bot flag}, \textit{mentions}, and \textit{interactions}. Through this process, we gathered \textbf{33,373,229 messages} from \textbf{82,109 users} across \textbf{81 servers}, including \textbf{1,912,750 messages} from \textbf{633 bots}. Data collection occurred between November 13th and 15th, covering messages sent from January 1st to November 12th, just after the 2024 U.S. election.

\subsection{Characterizing the Political Spectrum}
\label{sec:timeline}

A key aspect of our research is distinguishing between Republican- and Democratic-aligned Discord servers. To categorize their political alignment, we relied on server names and self-descriptions, which often include rules, community guidelines, and references to key ideologies or figures. Each server's name and description were manually reviewed based on predefined, objective criteria, focusing on explicit political themes or mentions of prominent figures. This process allowed us to classify servers into three categories, ensuring a systematic and unbiased alignment determination.

\begin{itemize}
    \item \textbf{Republican-aligned}: Servers referencing Republican and right-wing and ideologies, movements, or figures (e.g., MAGA, Conservative, Traditional, Trump).  
    \item \textbf{Democratic-aligned}: Servers mentioning Democratic and left-wing ideologies, movements, or figures (e.g., Progressive, Liberal, Socialist, Biden, Kamala).  
    \item \textbf{Unaligned}: Servers with no defined spectrum and ideologies or opened to general political debate from all orientations.
\end{itemize}

To ensure the reliability and consistency of our classification, three independent reviewers assessed the classification following the specified set of criteria. The inter-rater agreement of their classifications was evaluated using Fleiss' Kappa \cite{fleiss1971measuring}, with a resulting Kappa value of \( 0.8191 \), indicating an almost perfect agreement among the reviewers. Disagreements were resolved by adopting the majority classification, as there were no instances where a server received different classifications from all three reviewers. This process guaranteed the consistency and accuracy of the final categorization.

Through this process, we identified \textbf{7 Republican-aligned servers}, \textbf{9 Democratic-aligned servers}, and \textbf{65 unaligned servers}.

Table \ref{tab:statistics} shows the statistics of the collected data. Notably, while Democratic- and Republican-aligned servers had a comparable number of user messages, users in the latter servers were significantly more active, posting more than double the number of messages per user compared to their Democratic counterparts. 
This suggests that, in our sample, Democratic-aligned servers attract more users, but these users were less engaged in text-based discussions. Additionally, around 10\% of the messages across all server categories were posted by bots. 

\subsection{Temporal Data} 

Throughout this paper, we refer to the election candidates using the names adopted by their respective campaigns: \textit{Kamala}, \textit{Biden}, and \textit{Trump}. To examine how the content of text messages evolves based on the political alignment of servers, we divided the 2024 election year into three periods: \textbf{Biden vs Trump} (January 1 to July 21), \textbf{Kamala vs Trump} (July 21 to September 20), and the \textbf{Voting Period} (after September 20). These periods reflect key phases of the election: the early campaign dominated by Biden and Trump, the shift in dynamics with Kamala Harris replacing Joe Biden as the Democratic candidate, and the final voting stage focused on electoral outcomes and their implications. This segmentation enables an analysis of how discourse responds to pivotal electoral moments.

Figure \ref{fig:line-plot} illustrates the distribution of messages over time, highlighting trends in total messages volume and mentions of each candidate. Prior to Biden's withdrawal on July 21, mentions of Biden and Trump were relatively balanced. However, following Kamala's entry into the race, mentions of Trump surged significantly, a trend further amplified by an assassination attempt on him, solidifying his dominance in the discourse. The only instance where Trump’s mentions were exceeded occurred during the first debate, as concerns about Biden’s age and cognitive abilities temporarily shifted the focus. In the final stages of the election, mentions of all three candidates rose, with Trump’s mentions peaking as he emerged as the victor.