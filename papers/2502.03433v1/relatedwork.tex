\section{Related Work}
\label{sec:related_work}

\begin{figure*}[!ht]
    \centering
    \includegraphics[width=0.8\linewidth]{line.pdf}
    \caption{Volume of messages shared on Discord political servers, categorized by metadata. The first plot shows total daily messages across all servers. The second plot shows daily mentions of ``Trump'', ``Biden'', and ``Kamala''.}
    \label{fig:line-plot}
    \Description{Volume of messages shared on Discord political servers categorized by metadata. The first plot shows the total number of messages shared per day across all servers. The second plot depicts daily mentions of ``Trump'', ``Biden'', and ``Kamala''.}
\end{figure*}


The influence of social media on political engagement has been widely discussed, particularly in the context of mainstream platforms such as X and Facebook. For example, Fujiwara et al.~\cite{fujiwara2024effect} examined the role of Twitter in the 2016 and 2020 U.S. presidential elections and found that increased Twitter usage in certain counties contributed to a significant reduction in the Republican vote share. In addition, Allcott et al.~\cite{allcott24theeffects} investigated the effects of deactivating Facebook and Instagram accounts during the 2020 U.S. election. Their study revealed that while Instagram had no measurable impact on political outcomes, Facebook significantly increased user awareness of general news but also led to greater exposure to misinformation.


In more recent work, Balasubramanian et al.~\cite{balasubramanian2024publicdatasettrackingsocial} collected a dataset from X that highlighted the prominence of keywords such as \emph{Biden}, \emph{Trump}, and \emph{MAGA}, as well as the widespread use of hashtags like \#trump2024, \#maga, and \#bidenharris2024. This dataset reflected public engagement with key candidates and movements and underscored the significant influence of multimedia platforms such as YouTube and X. Moreover, the data illustrated how media outlets, including Fox News and Breitbart, shape political discourse online.

Regarding discourse analysis, several studies have employed various methodologies to explore political polarization and social media rhetoric. For instance, Stefanov et al.~\cite{valence} utilized a combination of supervised and unsupervised learning techniques to detect political polarization on Twitter. Their approach incorporated measures such as valence, graph theory, and contextual embeddings to predict political biases. Similarly, Magno and Almeida~\cite{magno2021measuring} employed word embeddings to explore cultural and social values on a global scale by analyzing large datasets of online communications. Their findings revealed correlations between online sentiment and offline cultural traits, as captured in the World Values Survey.

Hate speech on social media platforms has been extensively explored in various studies. For example, Alkomah and Ma~\cite{hate-speech-1} provided a comprehensive review of textual hate speech detection methods and datasets, while Davidson et al.~\cite{hate-speech-3} focused on automated hate speech detection and the challenges posed by offensive language. Additionally, Saha et al.~\cite{hate-speech-2} proposed novel techniques for hate speech detection that extend beyond traditional methods. In this context, Ottoni et al.~\cite{embd-2} concentrated on right-wing YouTube channels by examining the prevalence of hate speech, violence, and discrimination in both video content and user comments. Their layered methodology, which included lexical analysis, topic modeling, and implicit bias detection, revealed patterns of negative language and discriminatory bias in political discourse.

While most studies on social networks have focused on traditional platforms, there is a noticeable gap in the literature regarding the influence of Discord, particularly within the political context. Although Discord was originally designed as a communication platform for gamers, it has evolved into a multifaceted environment that accommodates a broad spectrum of social and political groups, including far-right organizations. These groups have used Discord to organize events, such as the ``Unite the Right'' rally in Charlottesville, as discussed by Roose~\cite{roose17thiswas}, leveraging its private and community-driven nature for communication and recruitment. Moreover, Heslep and Berge~\cite{heslep2024mapping} demonstrate how hate networks exploit Discord's moderation gaps and third-party tools.

Although numerous studies have investigated political debates on platforms such as Twitter and Telegram, to the best of our knowledge, no comprehensive investigation into political discourse on Discord has been conducted. Given Discord's growing prominence, this study aims to perform an extensive analysis of political discourse on the platform, using the 2024 U.S. presidential election as a case study.



\begin{table*}[H]
\small
\centering
\begin{tabular}{crrrrr} \toprule
\textbf{Alignment} & \textbf{Unique Users} & \textbf{User Messages} & \textbf{Messages/User} & \textbf{Unique Bots} & \textbf{Bot messages} \\ \midrule
\textbf{Democratic}    & 10,149                & 520,614                & 51.29                  & 69                   & 47,123                \\
\textbf{Unaligned} & 71,686                & 31,603,712             & 440.86                 & 618                  & 2,102,070             \\
\textbf{Republican}   & 4,255                 & 555,750                & 130.61                 & 35                   & 76,412                \\
\textbf{Total}   & 83,611                & 32,680,076             & 390.85                 & 675                  & 2,225,605             \\ \bottomrule
\end{tabular}
\caption{Statistics of user and bot posts across server categories. Most of the messages belong to unaligned servers. The unique users and unique bots columns do not sum to total, as some users participate in discussion on multiple server categories.}
\label{tab:statistics}
\end{table*}