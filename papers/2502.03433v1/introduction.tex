\section{Introduction}
\label{sec:introduction}

Social networks have become a powerful tool for amplifying social movements, such as protests \cite{wolfsfeld2013social} and political campaigns~\cite{groshek2017helping}, allowing messages to reach audiences as large as those of traditional media like television -- all without relying on mainstream outlets~\cite{boynton2016agenda}.   % Social movements, such as protests \cite{wolfsfeld2013social} and political campaigns~\cite{groshek2017helping}, have found a lot of success by relying on the characteristics of social media networks, allowing posts to have reach that is comparable to, or greater than television without having to rely on mainstream media~\cite{boynton2016agenda}. 
% Due to these reasons, researchers have argued that the discourse in social media, in the context of politics, is an important force behind a politician success. In fact, even as far back as the 2008 American elections, the usage of social media as part of Barack Obama's campaign strategy was already seen as an integral part of his victory~\cite{hughes2010obama}.
Due to these reasons, researchers have identified the important role of social media in political success. Indeed, during the 2008 U.S. presidential elections,  Barack Obama's use of social media as part of his campaign strategy was already regarded as a key factor in his victory~\cite{hughes2010obama}.

Since then, social media's influence on politics has only intensified. The spread of fake news, misinformation and bot-driven manipulation during the 2016 and 2020 elections~\cite{gunther2019fake,ferrara2020characterizing}, raised significant questions about the need for regulation and moderation on these platforms. As a result, platforms such as Facebook and X (formerly Twitter)~\cite{zannettou2021won} implemented interventions to curb the potential harm caused by these strategies. Despite these efforts, social media continues to enable harmful activities, such as spreading  conspiracy theories and organizing events like the January 2021 Capitol riot~\cite{o2023coming}. This highlights the importance of studying political discussions on social media platforms. 

% Although a lot has changed since then, the usage of social media for political efforts has only been intensified. Discussion around fake news spread by bots and other bad faith actors in the 2016 and 2020 elections~\cite{gunther2019fake,ferrara2020characterizing}, raised many questions about the necessity of regulation and moderation in these networks, leading to some interventions in social media platforms such as Facebook and X (formerly known as Twitter)~\cite{zannettou2021won} to curb the potential harm of these strategies. Since social media has seen usage in order to give birth to conspiracy theories and help organize dangerous movements such as ``storm the capitol'' in January 2020~\cite{o2023coming}, the importance of social media in electoral and political processes is undeniable. Thus, understanding the discourse in such platforms is paramount.

Most research so far has focused on mainstream platforms, such as YouTube, X, Facebook, Instagram, and Reddit. However, other platforms may hold significant influence in shaping voters' opinions during the electoral process, still remaining under-explored. 
% However much of the research has focused on mainstream platforms, such as YouTube, X, Facebook, Instagram, and Reddit, under-exploring smaller platforms that may still hold significant influence in shaping voters opinions over the course of the electoral calendar.
One of these platforms is Discord, which has experienced significant growth since the pandemic. Originally designed for gaming and related topics, it has since been adopted by many other communities~\cite{johnson2022embracing}.
The platform is structured into user-created and managed groups, which are called servers, where they can share text, images, videos, and voice-chat. 
Unlike mainstream social media platforms, moderation and guidelines on Discord are primarily determined by server creators \footnote{https://discord.com/community-moderation-safety}, meaning they may vary across different servers. This decentralized structure fosters less-moderated and unfiltered discussions, including those in public community servers.

Discord's unique characteristics make it a compelling subject for study. Its semi-private servers are similar to platforms like Telegram, but its recent controversies -- such as child abuse networks uncovered in Brazil~\cite{discord2024scandalbrazil} and its role in mental health discussions \cite{webmedia} -- show both its potential and risks. The growing presence of alt-right extremism on the platform~\cite{gallagher2021extreme} further highlights the importance of studying its impact on political discussions. Despite these emerging concerns, there remains a significant gap in scientific research regarding the political dimensions of this network, particularly  its role in influencing political discourse and the presence of already established ideological communities.
    
To the best of our knowledge, this study represents the first in-depth political analysis of Discord, focusing on the 2024 United States presidential election. Specifically, we aim to explore how the platform's unique characteristics shape political discussions and user interactions. We address the following research questions:

\begin{enumerate}
    \item RQ1: How does the discourse on Discord servers vary across different political spectra and electoral periods, particularly in response to major political events? 
    \item RQ2: What is the level of toxicity in political discussions on Discord, and which groups are the primary targets of hate speech across different political spectra and key electoral moments?
\end{enumerate}

We find that (1) by tracking shifts in political conversations during key campaign events, we identified distinct political valences and implicit biases in semantic associations through embedding analysis. Republican-aligned servers emphasized economic policies, while Democratic-aligned servers focused on equality-related and progressive causes. The volume of political messages surged significantly during pivotal events such as Biden's exit from the race and the presidential debates.

% the volume of messages related to political candidates during the 2024 US elections increased drastically following important events such as Biden leaving the presidential race and the presidential debates. This also took shape in changes in the political discourse on the political servers, as the discussion shifted from global issues such as economy, communism, and the Israel-Palestine conflict, to be more related to the elections over the year, focusing on topics such as the debates and the politicians related to the presidential candidates.

Additionally, we note that (2) the discussion on Republican-aligned servers was considerably more toxic than on Democratic-aligned servers, especially in the period just after Kamala Harris was announced as the Democrat candidate, which was correlated with increased toxicity and sexism on the Republican servers.

\begin{table*}[!h]
\small
\centering
\caption{Statistics of user and bot posts across server categories. Most of the messages belong to unaligned servers. The unique users and unique bots columns do not sum to total, as some users participate in discussion on multiple server categories.}
\begin{tabular}{crrrrr} \toprule
\textbf{Alignment} & \textbf{Unique Users} & \textbf{User Messages} & \textbf{Messages/User} & \textbf{Unique Bots} & \textbf{Bot messages} \\ \midrule
\textbf{ Democratic}    & 10,149                & 520,614                & 51.29                  & 69                   & 47,123                \\
\textbf{Unaligned} & 71,686                & 31,603,712             & 440.86                 & 618                  & 2,102,070             \\
\textbf{ Republican}   & 4,255                 & 555,750                & 130.61                 & 35                   & 76,412                \\
\textbf{Total}   & 83,611                & 32,680,076             & 390.85                 & 675                  & 2,225,605             \\ \bottomrule
\end{tabular}

\label{tab:statistics}
\end{table*}
