\section{Conclusion}
\label{sec:concludingremarks}

On this study, we examine political servers on Discord, highlighting consistent differences between Republican- and Democratic-aligned communities throughout 2024. Our findings show that Trump was far more discussed than Kamala Harris, based on the number of mentions. Conversations about Democratic candidates -- Kamala and Biden -- were mostly limited to Democratic-aligned servers, while Trump dominated discussions across all categories of servers, eliciting both positive and negative reactions.  While Kamala’s discourse stayed confined to Democratic-leaning spaces, Trump’s polarizing presence drove high engagement across both sides, engaging  his voter base and contributing to his strong turnout, as presented in Section \ref{sec:politicians}.

Although Discord is primarily used by younger people, who generally lean Democratic \cite{IOP2024}, we found that Republican-aligned servers had higher engagement per user. On average, users in these servers posted twice as many messages as their Democratic-aligned counterparts, resulting in a similar total volume of messages across both categories. Combined with Trump’s significantly higher visibility, this suggests that Republicans were more effective in mobilizing their voter base on Discord.

Our analysis reveals that Republican-aligned servers were significantly more toxic than Democratic-aligned ones, with non political correct insults among the most frequently used terms in those spaces. Some instances of hate speech, including racism and sexism, were notably higher in Republican servers. This trend may have been influenced by Kamala Harris’s identity as a Black woman of Jamaican and Indian descent, making her a target for attacks based on race and gender -- vulnerabilities not as easily exploited against figures like Biden or Trump. Interestingly, while sexist hate speech spiked after Harris officially entered the race, racial hate speech remained stable and even declined slightly over the election cycle. These findings, though preliminary, raise important questions. The U.S. has never elected a woman president, and this pattern might suggest that sexism, often less visible than other forms of bias, could play a larger role than previously understood.

As younger generations increasingly engage in politics, studying the platforms they use to communicate will be critical for understanding modern political movements -- not just in the U.S., but globally. This is essential to addressing challenges like radicalization and polarization that have characterized recent years. This study offers a first look at Discord’s role in contemporary political discourse. Future research should expand on these findings by exploring the broader Discord ecosystem, moving beyond self-identified political leanings and keyword-based filtering to gain deeper insights. 

\subsection{Future Works and Limitations}

One of Discord's distinctive features is the extensive use of bots, future research could focus on investigating the use of bots in the political context on Discord, examining how these automated systems influence discourse, facilitate the spread of political ideologies, or potentially exacerbate the dissemination of hate speech.

Additionally, a more in-depth analysis of the Word Embedding Association Test (WEAT) could be conducted, further exploring the intriguing implicit biases identified in this study. Future work could aim to uncover how these biases are shaped and reinforced within political discussions on Discord. This exploration could provide valuable insights into the mechanisms by which sociopolitical biases emerge and propagate through online interactions.

% \subsection{Limitations}
% \label{sec:limitations}

Despite the large amount of servers and messages collected from public groups using political keywords, it’s important to note that political discussions may also be occurring on public servers that don't explicitly identify with politics or even on private servers. As a result, we cannot fully capture how the platform, as a whole, is engaging with political topics.

Additionally, Discord does not release any user demographic data through its API, such as gender, age, or race -- only usernames are available. This lack of demographic information makes it challenging to fully understand the context of our dataset, as we are unable to analyze the characteristics of the users participating in these discussions.



\section{Ethical Considerations}
\label{sec:ethics}
The data we collected from Discord is accessed via publicly available invitation links, in full compliance with Discord’s terms of service\footnote{https://dis.gd/discord-developer-policy} \footnote{https://discord.com/terms}.  To protect the privacy of users, we employed several anonymization techniques. Specifically, we refrained from identifying individual users or focusing on any hate speech tied to specific users.

\section{Dataset Availability}
The dataset used in this study, which contains messages from public Discord servers during the 2024 U.S. presidential election, is publicly available for further research. It is anonymized to ensure user privacy and organized in a server-specific format, aligned with other relevant literature \cite{aquino2025discordunveiledcomprehensivedataset}, with each server's data stored in an individual JSON file. The dataset is published on Zenodo with DOI: \texttt{10.5281/zenodo.14807501}\footnote{\url{https://zenodo.org/records/14807501}}. For further details, refer to the Discord API documentation\footnote{\url{https://discord.com/developers/docs/reference}}. We encourage researchers to use this dataset for future studies on political discourse, social media dynamics, or related areas.


