\section{Results}
In this section, we present the findings of our study, addressing the research questions (RQs) outlined in the introduction.

To answer RQ1, we investigate user discourse to uncover implicit biases and identify thematic shifts across different political spectra and electoral periods. Specifically, we analyze two aspects of textual content from messages shared on Discord servers: in subsection \ref{sec:valence}, we examine the political valence of specific terms used in discussions, and in the subsection \ref{sec:embedding_results}, we present the embeddings obtained by training \textit{Word2Vec} models on our dataset. Both techniques enable us to explore the ideological leanings of discourse over time.

Furthermore, to address RQ2, we assess the level of toxicity in political discussions in subsection \ref{sec:hate_speech_results}. We identify the primary targets of hate speech and analyze how these dynamics vary across political groups and key electoral moments, using a multi-class speech classifier.

\subsection{Political Valence of Messages}
\label{sec:valence}
We categorize political valence into five groups, ranging from highly-Democratic to highly-Republican, based on a scale from \( [-1, +1] \).  We opt to make the center interval range [-0.33, 0.33] to account for terms that are used by both sides. Figure \ref{fig:valenceT1} illustrates the top-8 most frequent terms and top-3 most shared URLs within these groups in each period, highlighting discourse shifts during the election year. This analysis excludes links to embedded images, videos, and GIFs, focusing on textual content. We opted to use the most frequent words since we believe that those better represent the subjects commonly discussed on these categories.

\begin{figure}[h]
    \centering
    \includegraphics[width=\linewidth]{images/tabela-valencia-15.pdf}
    %\includegraphics[width=\linewidth,trim=3cm 6cm 3cm 4cm,clip]{images/ValenceT1 compressed.pdf}
    \caption{Political valence of top terms and URLs shared and discussed in Democratic and Republican-aligned Discord servers across the studied time period. }
    \label{fig:valenceT1}
\end{figure}

\subsubsection{Candidates and Politicians.\\}
\label{sec:politicians} 

In the Democratic side, we observe that throughout all periods the term ``Biden`` is relevant, while Kamala Harris gained more prominence in the discussions only post her presidential campaign announcement, in July 21st, date after Kamala continued to be a top-3 most frequent political term in the moderately-Democratic valence. Also, during the voting period Tim Walz had his first appearance, which is logic since he was then running for the vice-president position.

Even Donald Trump, the main candidate for the Republican Party, appears in the center or slightly Democratic in the valence range depending on the period, likely due to his highly divisive and controversial nature, which sparks discussions on both sides.

Furthermore, the elected vice-president, Vance, appears on the highly-Democratic valence, despite being a Republican. This is intriguing as Vance is often seen as an unpopular choice for vice president when compared to other recent candidates~\cite{vance2024impopular}. It’s noteworthy that the left has seized on this situation to generate discussions about him, while he has largely been ignored on the Republican side. Another interesting appearance in the latest periods is from terms related to Robert Kennedy Jr. in the Republican side of valence, possibly due to him dropping out of the presidential race in support of Trump, and later being picked to be the secretary of the Department of Health and Human Services. 

Lastly, \textit{Milei} shows up as a highly-Republican term during the voting period, referring to Javier Milei, president of Argentina. He is the only non-American politician in the table, and has previously shown support to Trump, being the first foreign leader to meet him since his victory in the elections.

\subsubsection{Political and Economical Ideologies. \\}
\label{sec:ideologies}

\textbf{Democratic Spectrum:\\} Up to the first semester, Democratic-aligned servers focused on Communist and Socialist discourse related terms, such as \textit{comrade}, \textit{Lenin}, \textit{Stalin}, \textit{capitalism} and \textit{revolution} while also sharing websites like \textit{marxists.org}. As the year progressed, however, these discussions and references to Marxist figures and ideals decreased in frequency. 

Discussions about the Israel-Palestine conflict were highly relevant throughout all periods, with terms such as \textit{Israel}, \textit{war}, \textit{genocide}, and \textit{Palestine} appearing in the moderately-Democratic valence. Also, the prevalence of \textit{pronouns} and appearance of \textit{feminism} in highly-Democratic discourse also emphasize an endorsement of equality-related and progressive causes, particularly with respect to the LGBTQIA+ community and women rights.

\textbf{Republican Spectrum:\\}    When examining the Republican side of the discourse, we observe a focus on economical and capitalism related issues, with terms such as \textit{libertarian}, \textit{crypto}, \textit{property}, \textit{market}, \textit{taxes} and \textit{economy}, as well as websites like \textit{mises.org} (an institute focused on economics and libertarianism),  all displaying moderate to highly-Republican valence. One  possible explanation for the prominence of these terms could be  Donald Trump's proposals and general campaign rhetoric, particularly related to taxes, tariffs, and trade, which likely sparked discussions among his supporters. Another notable aspect is that the government-related discussions are more prevalent on the Republican-side, with terms like state, government, and law appearing frequently. 

Additionally,  terms often used as politically incorrect insults, such as \textit{retarded} are among the most commonly used  in the highly-Republican valence, while no insults appear even in the top-100 terms for the highly-Democratic valence.


\subsubsection{News Sources and Associated URLs. \\}
\label{sec:news}

As previously mentioned in \ref{sec:ideologies}, marxists.org and mises.org were commonly shared respectively by the Democratic and the Republican, aligning precisely with left-wing and right-wing ideals.

Democratic-aligned servers also referenced to news outlets such as The Guardian, Reuters, Associated Press and New York Times, with the latter publicly endorsing Kamala Harris campaign~\cite{nyt2024stance}. Meanwhile, Republican-aligned servers were surprisingly associated with two well-known Argentinean media channels, Infobae and DerechaDiario (an auto-declared right-wing news outlet \footnote{In their YouTube channel description 'https://www.youtube.com/c/LaDerechaDiario', they call themselves ``An alternative media to the left-wing hegemony.''}), tying together with Milei's Valence, mentioned before in \ref{sec:politicians}. Another influential source was Infowars which raise concerns about misinformation sharing \footnote{Infowars, known for spreading conspiracy theories and fake news, is concerning due to its influence on political discourse in Republican-leaning communities. https://www.bbc.com/news/world-63243981}. 

Additionally, there was a slight deviation from New York Post links to the Republican side, but it also appears in the balanced Valence during the Voting Period. Other sources such as NBC News, BBC and Wikipedia also stood in the middle-ground of the Valence table, showing no preference from any of the polarized alignments in the Discord platform. Overall, the position of each of the news outlets somewhat reflects the popular perceptions of these sources~\footnote{https://www.allsides.com/media-bias/media-bias-chart}.

\subsubsection{Social Media and Digital Platforms.\\}
\label{sec:social_media}

A few of the URLs found in the table contained hyperlinks to posts on other social media platforms, and their Valence values help us understand the landscape of the preferences from each spectrum.

The first and most important observation is the intriguing positioning of the social media ``X'', previously known as ``Twitter'', as the terms ``x.com'', ``twitter.com'' and ``tweet'' appear heavily related to the Republican. Directly contradicting previous beliefs and studies indicating that Twitter was a predominantly Liberal-leaning platform, with some even saying it affected previous elections by benefiting Democratic candidates \cite{fujiwara2024effect}.

This may be explained by the brand's recent ``rebranding'', which might have attracted a new audience from different political alignments, and is directly corroborated by \cite{balasubramanian2024publicdatasettrackingsocial}, in which is shown that the volume of ``hashtags'' supportive of Republicans, such as \textit{\#MAGA} and \textit{\#Trump} far outweighed the ones related to Democratic backing, as of early 2024.

In contrast, Instagram remains largely non-aligned politically. Its focus on lifestyle, entertainment, and visual content tends to limit its engagement in political discourse, resulting in a more neutral stance in terms of ideological positioning, as backed up by previous studies \cite{allcott24theeffects}. Meanwhile, particularly in the Discord space, Reddit seems to have become more associated with Democrat-aligned communities, as supported by it's Valence values.

\subsection{Embedding Analysis}

The Word Embedding Association Test (WEAT) was conducted to uncover implicit biases and semantic associations in the political discourse on Discord servers, seen on Figure \ref{fig:weat}. The purpose of this analysis is to explore how key politically relevant terms are perceived and connected to positive or negative connotations across different ideological contexts and electoral periods.

To ensure comprehensive coverage of political topics, we selected terms spanning a wide range of themes. These terms were organized into five categories: Civil Rights and Liberties, Social Issues, Economic Policy, Governance and Democracy, and Candidates and Party Names. For each term, we defined three positive and three negative reference terms to capture its associations in different contexts. The only exception was the Candidates and Party Names category, where, due to the diverse scenarios and topics these names appear in, we used an extensive list of positive and negative reference terms. This approach ensures a more robust analysis, capturing a wide variety of political themes and contexts.

\begin{figure}[h]
    \centering
    \includegraphics[width=1\linewidth]{images/evaluation_plot_patches_horizontal_titles.pdf}
    \caption{Visualization of S-values computed using models trained on servers with Democratic and Republican alignment for each period of the electoral campaign. The baseline model corresponds to the used pre-trained model. A positive S-value indicates closer alignment with positive terms, while a negative S-value indicates closer alignment with negative terms. }
    \label{fig:weat}
\end{figure}

The first issue concerns Civil Rights and Liberties. Terms related to sexuality and LGBTQIA+ rights appear to be more positively biased on the Democratic side, while being mostly neutral to negative on the Republican side. A notable observation is that the Democratic party, known for its stricter stance on gun laws and more open approach to immigration, has gradually shifted over time, developing a bias more similar to the Republicans on these issues.

New trends emerge in the analysis of Social Issues, bringing several key topics to the forefront, as demonstrated in the hate speech analysis. The term \textit{feminism} stands out, with a predominantly negative bias on Republican-aligned servers. It’s also noteworthy that after Kamala Harris entered the race, both Democratic and Republican servers shifted significantly toward a negative bias on the topic of \textit{diversity}, which correlates with the observed increase in sexist hate speech during this period.

On economic policies, the trends are largely as expected. Words related to \textit{market} or \textit{economy} tend to have a more positive connotation on the Republican side. One notable situation is relating to inflation. During Biden's presidency, many of the critiques against him focused on inflation, which was an issue in late 2022 and early 2023. Interestingly, when Biden was the candidate, the bias towards the term \textit{inflation} in Democratic servers was positive, possibly reflecting that his supporters defend his decisions or the state of the economy. However, once Kamala entered the race, \textit{inflation} shifted to a negative bias, mirroring the Republican perspective.

Governance and Democracy reveal a few notable trends. The term \textit{democracy} saw a sharp increase in positive bias on Democratic servers when Kamala entered the race. This likely reflects her campaign's strong focus on democracy and freedom, which were central to her messaging. Another noteworthy term is \textit{climate}. Until the election, \textit{climate} was largely viewed positively, but as the election approached, its bias turned negative. This shift may be tied to the intensifying discussions around climate change and the evolving political climate.

Finally, we analyze the sentiment dynamics surrounding candidates and party names. In this category, the broader range of terms analyzed resulted in less pronounced differences across the sentiment spectra. However, we can still identify some notable differences. While Joe Biden was the Democratic candidate, both parties exhibited a relatively neutral stance toward Walz. However, upon Walz’s nomination as the vice-presidential candidate, sentiment on the Republican side became markedly more negative.  Interestingly, during the voting period, which coincided with the vice-presidential debate, sentiment on the Republican side shifted back toward a more positive view of Walz. Similarly, Vance experienced a smaller but noticeable shift, where the bias toward him became more positive over time on both the Democratic and Republican sides.
% This change is logical, considering both Walz and Vance performed well in the debate.


The case of \textit{Biden} presents an intriguing case. Prior to his withdraw of the race, many within his own party members were urging him to step down. As a result, his bias was notably more negative on the left during the first half of the year. However, after he stepped out, his bias became more positive and stabilized toward the middle. Conversely, \textit{Donald Trump} exhibited a similar trend. Even on the Democrat alignment, Trump’s bias became more stable after Biden’s exit, likely due to an increase in his popularity. Interestingly, \textit{Kamala} bias remained relatively stable throughout the entire electoral cycle.

\label{sec:embedding_results}

\subsection{Hate Speech Analysis}
\label{sec:hate_speech_results}

By applying the multi-class hate speech classifier in our dateset, Figure \ref{fig:bar-graph-toxic} displays the average levels of hate speech across Democratic and Republican-aligned and unaligned servers during the periods of interest. The figure provides an overview of the presence of hate speech across multiple classes, offering a general perspective on its distribution. The figure reveals a higher prevalence of hate speech in Republican-aligned and unaligned servers compared to Democratic-aligned servers. 

\begin{figure}[h]
    \centering
    \includegraphics[width=1\linewidth]{images/clusters_de_barras_coloridas_com_legenda_e_erros.pdf}
    \caption{Bar plot of Hate Speech for Democratic Aligned, Republican Aligned and Unaligned in each period. }
    \label{fig:bar-graph-toxic}
\end{figure}
\begin{figure}[h]
    \centering
    \includegraphics[width=0.82\linewidth]{images/hate_speech_radar_charts.pdf}
    \caption{Radar charts displaying the percentage of hate speech messages by period. The highest value, 0.55\%, occurred in the 'Sexism' category during the Kamala vs. Trump race.}
    \label{fig:radar}
\end{figure}

\subsubsection{Hate Speech on Unaligned Servers. \\}

Figures \ref{fig:radar} and \ref{fig:all_hate} provide a detailed analysis of the distribution of hate speech across multiple categories throughout the electoral campaign. The weekly-segmented plot reveal a striking trend in unaligned servers: a consistently high proportion of hate speech across all analyzed categories throughout the observed period. A particularly notable trend is the persistent presence of hate speech in the ``Other'' category, which remains at a significant level over time. Together, these trends, illustrated in Figure \ref{fig:bar-graph-toxic}, demonstrate how all categories cumulatively drive the high global rate of hate speech observed in unaligned servers, especially during the Biden vs. Trump and Voting Periods.

Additionally, it is plausible to hypothesize that unaligned servers, due to their lack of a clear political alignment, function as open forums where diverse opinions and groups converge, helping to explain the consistently high levels of hate speech observed in these servers, unlike explicitly aligned servers (Democratic or Republican), where a greater uniformity of opinions might be expected. Moreover, unaligned servers seems to engage in discussions beyond the scope of the American presidential race. This broader scope is reflected in the significant increase in Religion-related hate speech in early September, as seen in Figure \ref{fig:all_hate}. We hypothesize that this surge is linked to the ongoing Israel-Palestine conflict, which ignited political discussions worldwide, further fueling divisive discourse.

\subsubsection{Hate Speech on Republican and Democratic Servers. \\}
%On Republican servers, a significant shift in the distribution of hate speech becomes evident following Joe Biden's withdrawal from the campaign and Kamala Harris's entry as the Democratic candidate. As illustrated in Figures \ref{fig:radar} and \ref{fig:all_hate}, Kamala Harris's candidacy, identifying as a Black woman of Indian descent, is notably linked to a significant rise in hate speech targeting sexism an increase that persists throughout the Kamala vs. Trump period. Notably, an increase in sexist hate speech is also observed on Democratic servers following Kamala's entry, although to a lesser extent compared to Republican servers. 

On Republican servers, a significant shift in the distribution of hate speech becomes evident following Joe Biden's withdrawal from the campaign and Kamala Harris's entry as the Democratic candidate. As illustrated in Figures~\ref{fig:radar} and \ref{fig:all_hate}, Kamala Harris's candidacy coincides with a notable rise in hate speech targeting sexism, an increase that persists throughout the Kamala vs. Trump period. Notably, a similar, though less pronounced, increase in sexist hate speech is also observed on Democratic servers following her entry.

It is noteworthy that while sexism has risen, racism did not. This disparity is interesting considering that United States has yet to elect a woman as president. Social media platforms often act as mirrors of societal attitudes, amplifying latent sociocultural biases and hate-driven behaviors. These biases, reflected in the volume and nature of hate speech, are closely associated with political narratives and may ultimately impact electoral outcomes. 

Another notable trend in the figure is the significant increase in sexist hate speech observed from the second week of May to the first week of June, evident on both Republican and Democratic servers. We hypothesize that this behavior may be linked to the events of May 16, 2024, during a U.S. House Oversight Committee hearing, where a heated exchange occurred between Representatives Marjorie Taylor Greene (MTG) and Alexandria Ocasio-Cortez (AOC). The confrontation was sparked by a comment from Greene directed at Representative Jasmine Crockett. Greene criticized Crockett's appearance, claiming her false eyelashes hindered her ability to read. Such remarks, when amplified within politically charged groups on social media, may contribute to the rise in online sexism. Discussions in these spaces often devolve into misogynistic attacks and disparaging comments aimed at women in politics \footnote{https://www.theguardian.com/us-news/article/2024/may/17/aoc-v-mtg-house-hearing-chaos}.

Through this analysis, we find that Republican-affiliated servers consistently exhibit higher rates of hate speech related to racism and sexism over the observed period. This trend is clearly illustrated in Figure \ref{fig:bar-graph-toxic}, which also highlight another significant point: the notably low incidence of hate speech related to sexuality in Democrat-affiliated servers. This difference aligns with the historically progressive stance of Democrats, who actively advocate for LGBTQIA+ rights. These findings provide a quantitative basis for explaining the higher overall rates of hate speech in Republican-affiliated servers, driven predominantly by the prevalence of racism and sexism. In contrast, Democrat-affiliated servers demonstrate a stronger commitment to inclusive values, particularly regarding sexuality, further emphasizing the distinction between the two political groups in terms of hate speech dynamics.

\begin{figure}[h]
    \centering
    \includegraphics[width=1\linewidth]{images/heatmaps.pdf}
    \caption{Heatmap of Hate Speech Categories Across Weekly Periods.}
    \label{fig:all_hate}
\end{figure}

\subsubsection{Challenges of Moderation on Discord. \\}

Although Discord has not been extensively studied in academic literature and there is a noticeable gap in analyses of its dynamics, the work \cite{heslep2024mapping} explores the unique challenges of moderation on the platform. They investigate how Discord, in combination with the Disboard website, facilitates the organization of distributed hate networks, exposing significant vulnerabilities in its moderation practices. Unlike more traditional social networks such as Twitter and Facebook \cite{wilson2020hate}, Discord operates on a decentralized model where each community functions autonomously. This decentralization makes centralized moderation extremely limited and creates loopholes for the proliferation of toxic networks. Furthermore, Discord explicitly delegates much of the moderation responsibility to its users, particularly to the administrators and moderators of individual servers \footnote{https://discord.com/safety/360044103531-role-of-administrators-and-moderators-on-discord}.


