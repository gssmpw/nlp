%%%%%%%% mlsys 2024 EXAMPLE LATEX SUBMISSION FILE %%%%%%%%%%%%%%%%%

\documentclass{article}

% Recommended, but optional, packages for figures and better typesetting:
\usepackage{microtype}
\usepackage{subfigure}
\usepackage{booktabs} % for professional tables

% hyperref makes hyperlinks in the resulting PDF.
% If your build breaks (sometimes temporarily if a hyperlink spans a page)
% please comment out the following usepackage line and replace
% \usepackage{mlsys2024} with \usepackage[nohyperref]{mlsys2024} above.
\usepackage{hyperref}

% Attempt to make hyperref and algorithmic work together better:
% \newcommand{\theHalgorithm}{\arabic{algorithm}}

% Use the following line for the initial blind version submitted for review:
% \usepackage{mlsys2025}
\PassOptionsToPackage{hyphens}{url}
\usepackage[hyphens]{url}
\PassOptionsToPackage{breaklinks,colorlinks}{hyperref}
\usepackage[breaklinks,colorlinks]{hyperref}
\PassOptionsToPackage{usenames,dvipsnames}{xcolor}
\usepackage[usenames,dvipsnames]{xcolor}
\usepackage{epigraph}
\hypersetup{
  colorlinks,
  linkcolor={red!50!black},
  citecolor={blue!50!black},
  urlcolor={blue!50!black}
}
\usepackage[outdir=./data/]{epstopdf}
\PassOptionsToPackage{usenames,dvipsnames}{xcolor}
\usepackage[usenames,dvipsnames]{xcolor}
\usepackage{colortbl}
\usepackage{amsmath,amsopn,amssymb,amsthm}
% \usepackage{subfigure}
\usepackage{endnotes,microtype,xspace,graphicx,fancyvrb,multirow}
\usepackage{booktabs}
\usepackage{array,underscore,relsize}
\usepackage{libertine}
\usepackage{fancyhdr}
\usepackage{enumitem}
\usepackage{csquotes}
\usepackage[labelfont=bf,font=small]{caption}
\captionsetup[figure]{aboveskip=6pt,belowskip=-12pt}
\captionsetup[table]{aboveskip=-12pt}
\usepackage[belowskip=0pt,aboveskip=2pt]{subcaption}
% \usepackage[]{subcaption}
\usepackage{balance}
\usepackage[normalem]{ulem}
\pagestyle{fancy}
\pagenumbering{arabic}
\fancyhf{}
\renewcommand{\headrulewidth}{0pt}
% page number
\cfoot{\thepage}

\usepackage{tikz}
\usepackage{mathdots}
% \usepackage{yhmath}
\usepackage{cancel}
\usepackage{color}
\usepackage{gensymb}
\usepackage{tabularx}
\usetikzlibrary{fadings}
\usepackage{calligra}
\usepackage{arydshln} % dashed hline/cline

% for math macro and numbers
\usepackage{fp}
\usepackage{siunitx}

% text stuff
\usepackage[normalem]{ulem}
% colored boxes
\usepackage{tcolorbox}

% balance bibliography
% \usepackage{balance}
\usepackage{multicol}

% use \num{123456} -> 123,456
\sisetup{group-separator={,},group-minimum-digits={3},output-decimal-marker={.}}

% adjust spacious title
\usepackage{titling}
\renewcommand{\maketitlehookb}{\vspace{-0.3in}}
\renewcommand{\maketitlehookc}{\vspace{-0.45in}}
\setlength{\droptitle}{-0.5in}

% adjust subsections
\usepackage[compact,small]{titlesec}
\titlespacing*{\section}{0pt}{0.3\baselineskip}{0.3\baselineskip}
\titlespacing*{\subsection}{0pt}{0.2\baselineskip}{0.2\baselineskip}
\pretitle{\vspace*{\fill}\vskip 2em\begin{center}\Large\bf}
\usepackage{enumitem}
\usepackage{multirow}
\usepackage{tcolorbox}
\usepackage{moreverb}
\usepackage{cprotect}
\usepackage{fancyvrb}
\usepackage{framed}

\newcommand{\myname}[0]{RLTHF}
\newcommand{\yifei}[1]{\textcolor{purple}{Yifei: #1}}
\newcommand{\bbb}[1]{\noindent\textbf{#1}}


% If accepted, instead use the following line for the camera-ready submission:
\usepackage[accepted]{mlsys2025}

% The \mlsystitle you define below is probably too long as a header.
% Therefore, a short form for the running title is supplied here:
\mlsystitlerunning{VoLUT: Efficient Volumetric Streaming Enhanced by LUT-based Super-resolution}

\begin{document}

\twocolumn[
\mlsystitle{VoLUT: Efficient Volumetric streaming enhanced by LUT-based super-resolution}

% It is OKAY to include author information, even for blind
% submissions: the style file will automatically remove it for you
% unless you've provided the [accepted] option to the mlsys2024
% package.

% List of affiliations: The first argument should be a (short)
% identifier you will use later to specify author affiliations
% Academic affiliations should list Department, University, City, Region, Country
% Industry affiliations should list Company, City, Region, Country

% You can specify symbols, otherwise they are numbered in order.
% Ideally, you should not use this facility. Affiliations will be numbered
% in order of appearance and this is the preferred way.
% \mlsyssetsymbol{equal}{*}

% \begin{mlsysauthorlist}
% \mlsysauthor{Aeiau Zzzz}{equal,to}
% \mlsysauthor{Bauiu C.~Yyyy}{equal,to,goo}
% \mlsysauthor{Cieua Vvvvv}{goo}
% \mlsysauthor{Iaesut Saoeu}{ed}
% \mlsysauthor{Fiuea Rrrr}{to}
% \mlsysauthor{Tateu H.~Yasehe}{ed,to,goo}
% \mlsysauthor{Aaoeu Iasoh}{goo}
% \mlsysauthor{Buiui Eueu}{ed}
% \mlsysauthor{Aeuia Zzzz}{ed}
% \mlsysauthor{Bieea C.~Yyyy}{to,goo}
% \mlsysauthor{Teoau Xxxx}{ed}
% \mlsysauthor{Eee Pppp}{ed}
% \end{mlsysauthorlist}

% \mlsysaffiliation{to}{Department of Computation, University of Torontoland, Torontoland, Canada}
% \mlsysaffiliation{goo}{Googol ShallowMind, New London, Michigan, USA}
% \mlsysaffiliation{ed}{School of Computation, University of Edenborrow, Edenborrow, United Kingdom}

% \mlsyscorrespondingauthor{Cieua Vvvvv}{c.vvvvv@googol.com}
% \mlsyscorrespondingauthor{Eee Pppp}{ep@eden.co.uk}

\begin{mlsysauthorlist}
\mlsysauthor{Chendong Wang\textsuperscript{*}}{uwm}
\mlsysauthor{Anlan Zhang}{usc}
\mlsysauthor{Yifan Yang}{msra}
\mlsysauthor{Lili Qiu}{msra}
\mlsysauthor{Yuqing Yang}{msra}
\mlsysauthor{XINYANG JIANG}{msra}
\mlsysauthor{Feng Qian}{usc}
\mlsysauthor{Suman Banerjee}{uwm}
\end{mlsysauthorlist}

\mlsysaffiliation{uwm}{University of Wisconsin–Madison, Madison, WI, USA}
\mlsysaffiliation{usc}{University of Southern California, Los Angeles, CA, USA}
\mlsysaffiliation{msra}{Microsoft Research Asia, Beijing, China}
\mlsyscorrespondingauthor{Chendong Wang}{cwang747@wisc.edu}
% You may provide any keywords that you
% find helpful for describing your paper; these are used to populate
% the "keywords" metadata in the PDF but will not be shown in the document
\mlsyskeywords{Machine Learning, MLSys}


\vskip 0.3in

Multi-agent reinforcement learning (MARL) has made significant progress, largely fueled by the development of specialized testbeds that enable systematic evaluation of algorithms in controlled yet challenging scenarios. However, existing testbeds often focus on purely virtual simulations or limited robot morphologies such as robotic arms, quadrupeds, and humanoids, leaving high-mobility platforms with real-world physical constraints like drones underexplored. To bridge this gap, we present \textbf{\textit{VolleyBots}}, a new MARL testbed where multiple drones cooperate and compete in the sport of volleyball under physical dynamics. VolleyBots features a turn-based interaction model under volleyball rules, a hierarchical decision-making process that combines motion control and strategic play, and a high-fidelity simulation for seamless sim-to-real transfer. We provide a comprehensive suite of tasks ranging from single-drone drills to multi-drone cooperative and competitive tasks, accompanied by baseline evaluations of representative MARL and game-theoretic algorithms. Results in simulation show that while existing algorithms handle simple tasks effectively, they encounter difficulty in complex tasks that require both low-level control and high-level strategy. We further demonstrate zero-shot deployment of a simulation-learned policy to real-world drones, highlighting VolleyBots’ potential to propel MARL research involving agile robotic platforms. The project page is at \url{https://sites.google.com/view/thu-volleybots/home}.

]
 
\printAffiliationsAndNotice{} 

  

  
\mysection{Introduction}
\label{sec:intro}
\section{Introduction}
\label{section:introduction}

% redirection is unique and important in VR
Virtual Reality (VR) systems enable users to embody virtual avatars by mirroring their physical movements and aligning their perspective with virtual avatars' in real time. 
As the head-mounted displays (HMDs) block direct visual access to the physical world, users primarily rely on visual feedback from the virtual environment and integrate it with proprioceptive cues to control the avatar’s movements and interact within the VR space.
Since human perception is heavily influenced by visual input~\cite{gibson1933adaptation}, 
VR systems have the unique capability to control users' perception of the virtual environment and avatars by manipulating the visual information presented to them.
Leveraging this, various redirection techniques have been proposed to enable novel VR interactions, 
such as redirecting users' walking paths~\cite{razzaque2005redirected, suma2012impossible, steinicke2009estimation},
modifying reaching movements~\cite{gonzalez2022model, azmandian2016haptic, cheng2017sparse, feick2021visuo},
and conveying haptic information through visual feedback to create pseudo-haptic effects~\cite{samad2019pseudo, dominjon2005influence, lecuyer2009simulating}.
Such redirection techniques enable these interactions by manipulating the alignment between users' physical movements and their virtual avatar's actions.

% % what is hand/arm redirection, motivation of study arm-offset
% \change{\yj{i don't understand the purpose of this paragraph}
% These illusion-based techniques provide users with unique experiences in virtual environments that differ from the physical world yet maintain an immersive experience. 
% A key example is hand redirection, which shifts the virtual hand’s position away from the real hand as the user moves to enhance ergonomics during interaction~\cite{feuchtner2018ownershift, wentzel2020improving} and improve interaction performance~\cite{montano2017erg, poupyrev1996go}. 
% To increase the realism of virtual movements and strengthen the user’s sense of embodiment, hand redirection techniques often incorporate a complete virtual arm or full body alongside the redirected virtual hand, using inverse kinematics~\cite{hartfill2021analysis, ponton2024stretch} or adjustments to the virtual arm's movement as well~\cite{li2022modeling, feick2024impact}.
% }

% noticeability, motivation of predicting a probability, not a classification
However, these redirection techniques are most effective when the manipulation remains undetected~\cite{gonzalez2017model, li2022modeling}. 
If the redirection becomes too large, the user may not mitigate the conflict between the visual sensory input (redirected virtual movement) and their proprioception (actual physical movement), potentially leading to a loss of embodiment with the virtual avatar and making it difficult for the user to accurately control virtual movements to complete interaction tasks~\cite{li2022modeling, wentzel2020improving, feuchtner2018ownershift}. 
While proprioception is not absolute, users only have a general sense of their physical movements and the likelihood that they notice the redirection is probabilistic. 
This probability of detecting the redirection is referred to as \textbf{noticeability}~\cite{li2022modeling, zenner2024beyond, zenner2023detectability} and is typically estimated based on the frequency with which users detect the manipulation across multiple trials.

% version B
% Prior research has explored factors influencing the noticeability of redirected motion, including the redirection's magnitude~\cite{wentzel2020improving, poupyrev1996go}, direction~\cite{li2022modeling, feuchtner2018ownershift}, and the visual characteristics of the virtual avatar~\cite{ogawa2020effect, feick2024impact}.
% While these factors focus on the avatars, the surrounding virtual environment can also influence the users' behavior and in turn affect the noticeability of redirection.
% One such prominent external influence is through the visual channel - the users' visual attention is constantly distracted by complex visual effects and events in practical VR scenarios.
% Although some prior studies have explored how to leverage user blindness caused by visual distractions to redirect users' virtual hand~\cite{zenner2023detectability}, there remains a gap in understanding how to quantify the noticeability of redirection under visual distractions.

% visual stimuli and gaze behavior
Prior research has explored factors influencing the noticeability of redirected motion, including the redirection's magnitude~\cite{wentzel2020improving, poupyrev1996go}, direction~\cite{li2022modeling, feuchtner2018ownershift}, and the visual characteristics of the virtual avatar~\cite{ogawa2020effect, feick2024impact}.
While these factors focus on the avatars, the surrounding virtual environment can also influence the users' behavior and in turn affect the noticeability of redirection.
This, however, remains underexplored.
One such prominent external influence is through the visual channel - the users' visual attention is constantly distracted by complex visual effects and events in practical VR scenarios.
We thus want to investigate how \textbf{visual stimuli in the virtual environment} affect the noticeability of redirection.
With this, we hope to complement existing works that focus on avatars by incorporating environmental visual influences to enable more accurate control over the noticeability of redirected motions in practical VR scenarios.
% However, in realistic VR applications, the virtual environment often contains complex visual effects beyond the virtual avatar itself. 
% We argue that these visual effects can \textbf{distract users’ visual attention and thus affect the noticeability of redirection offsets}, while current research has yet taken into account.
% For instance, in a VR boxing scenario, a user’s visual attention is likely focused on their opponent rather than on their virtual body, leading to a lower noticeability of redirection offsets on their virtual movements. 
% Conversely, when reaching for an object in the center of their field of view, the user’s attention is more concentrated on the virtual hand’s movement and position to ensure successful interaction, resulting in a higher noticeability of offsets.

Since each visual event is a complex choreography of many underlying factors (type of visual effect, location, duration, etc.), it is extremely difficult to quantify or parameterize visual stimuli.
Furthermore, individuals respond differently to even the same visual events.
Prior neuroscience studies revealed that factors like age, gender, and personality can influence how quickly someone reacts to visual events~\cite{gillon2024responses, gale1997human}. 
Therefore, aiming to model visual stimuli in a way that is generalizable and applicable to different stimuli and users, we propose to use users' \textbf{gaze behavior} as an indicator of how they respond to visual stimuli.
In this paper, we used various gaze behaviors, including gaze location, saccades~\cite{krejtz2018eye}, fixations~\cite{perkhofer2019using}, and the Index of Pupil Activity (IPA)~\cite{duchowski2018index}.
These behaviors indicate both where users are looking and their cognitive activity, as looking at something does not necessarily mean they are attending to it.
Our goal is to investigate how these gaze behaviors stimulated by various visual stimuli relate to the noticeability of redirection.
With this, we contribute a model that allows designers and content creators to adjust the redirection in real-time responding to dynamic visual events in VR.

To achieve this, we conducted user studies to collect users' noticeability of redirection under various visual stimuli.
To simulate realistic VR scenarios, we adopted a dual-task design in which the participants performed redirected movements while monitoring the visual stimuli.
Specifically, participants' primary task was to report if they noticed an offset between the avatar's movement and their own, while their secondary task was to monitor and report the visual stimuli.
As realistic virtual environments often contain complex visual effects, we started with simple and controlled visual stimulus to manage the influencing factors.

% first user study, confirmation study
% collect data under no visual stimuli, different basic visual stimuli
We first conducted a confirmation study (N=16) to test whether applying visual stimuli (opacity-based) actually affects their noticeability of redirection. 
The results showed that participants were significantly less likely to detect the redirection when visual stimuli was presented $(F_{(1,15)}=5.90,~p=0.03)$.
Furthermore, by analyzing the collected gaze data, results revealed a correlation between the proposed gaze behaviors and the noticeability results $(r=-0.43)$, confirming that the gaze behaviors could be leveraged to compute the noticeability.

% data collection study
We then conducted a data collection study to obtain more accurate noticeability results through repeated measurements to better model the relationship between visual stimuli-triggered gaze behaviors and noticeability of redirection.
With the collected data, we analyzed various numerical features from the gaze behaviors to identify the most effective ones. 
We tested combinations of these features to determine the most effective one for predicting noticeability under visual stimuli.
Using the selected features, our regression model achieved a mean squared error (MSE) of 0.011 through leave-one-user-out cross-validation. 
Furthermore, we developed both a binary and a three-class classification model to categorize noticeability, which achieved an accuracy of 91.74\% and 85.62\%, respectively.

% evaluation study
To evaluate the generalizability of the regression model, we conducted an evaluation study (N=24) to test whether the model could accurately predict noticeability with new visual stimuli (color- and scale-based animations).
Specifically, we evaluated whether the model's predictions aligned with participants' responses under these unseen stimuli.
The results showed that our model accurately estimated the noticeability, achieving mean squared errors (MSE) of 0.014 and 0.012 for the color- and scale-based visual stimili, respectively, compared to participants' responses.
Since the tested visual stimuli data were not included in the training, the results suggested that the extracted gaze behavior features capture a generalizable pattern and can effectively indicate the corresponding impact on the noticeability of redirection.

% application
Based on our model, we implemented an adaptive redirection technique and demonstrated it through two applications: adaptive VR action game and opportunistic rendering.
We conducted a proof-of-concept user study (N=8) to compare our adaptive redirection technique with a static redirection, evaluating the usability and benefits of our adaptive redirection technique.
The results indicated that participants experienced less physical demand and stronger sense of embodiment and agency when using the adaptive redirection technique. 
These results demonstrated the effectiveness and usability of our model.

In summary, we make the following contributions.
% 
\begin{itemize}
    \item 
    We propose to use users' gaze behavior as a medium to quantify how visual stimuli influences the noticebility of redirection. 
    Through two user studies, we confirm that visual stimuli significantly influences noticeability and identify key gaze behavior features that are closely related to this impact.
    \item 
    We build a regression model that takes the user's gaze behavioral data as input, then computes the noticeability of redirection.
    Through an evaluation study, we verify that our model can estimate the noticeability with new participants under unseen visual stimuli.
    These findings suggest that the extracted gaze behavior features effectively capture the influence of visual stimuli on noticeability and can generalize across different users and visual stimuli.
    \item 
    We develop an adaptive redirection technique based on our regression model and implement two applications with it.
    With a proof-of-concept study, we demonstrate the effectiveness and potential usability of our regression model on real-world use cases.

\end{itemize}

% \delete{
% Virtual Reality (VR) allows the user to embody a virtual avatar by mirroring their physical movements through the avatar.
% As the user's visual access to the physical world is blocked in tasks involving motion control, they heavily rely on the visual representation of the avatar's motions to guide their proprioception.
% Similar to real-world experiences, the user is able to resolve conflicts between different sensory inputs (e.g., vision and motor control) through multisensory integration, which is essential for mitigating the sensory noise that commonly arises.
% However, it also enables unique manipulations in VR, as the system can intentionally modify the avatar's movements in relation to the user's motions to achieve specific functional outcomes,
% for example, 
% % the manipulations on the avatar's movements can 
% enabling novel interaction techniques of redirected walking~\cite{razzaque2005redirected}, redirected reaching~\cite{gonzalez2022model}, and pseudo haptics~\cite{samad2019pseudo}.
% With small adjustments to the avatar's movements, the user can maintain their sense of embodiment, due to their ability to resolve the perceptual differences.
% % However, a large mismatch between the user and avatar's movements can result in the user losing their sense of embodiment, due to an inability to resolve the perceptual differences.
% }

% \delete{
% However, multisensory integration can break when the manipulation is so intense that the user is aware of the existence of the motion offset and no longer maintains the sense of embodiment.
% Prior research studied the intensity threshold of the offset applied on the avatar's hand, beyond which the embodiment will break~\cite{li2022modeling}. 
% Studies also investigated the user's sensitivity to the offsets over time~\cite{kohm2022sensitivity}.
% Based on the findings, we argue that one crucial factor that affects to what extent the user notices the offset (i.e., \textit{noticeability}) that remains under-explored is whether the user directs their visual attention towards or away from the virtual avatar.
% Related work (e.g., Mise-unseen~\cite{marwecki2019mise}) has showcased applications where adjustments in the environment can be made in an unnoticeable manner when they happen in the area out of the user's visual field.
% We hypothesize that directing the user's visual attention away from the avatar's body, while still partially keeping the avatar within the user's field-of-view, can reduce the noticeability of the offset.
% Therefore, we conduct two user studies and implement a regression model to systematically investigate this effect.
% }

% \delete{
% In the first user study (N = 16), we test whether drawing the user's visual attention away from their body impacts the possibility of them noticing an offset that we apply to their arm motion in VR.
% We adopt a dual-task design to enable the alteration of the user's visual attention and a yes/no paradigm to measure the noticeability of motion offset. 
% The primary task for the user is to perform an arm motion and report when they perceive an offset between the avatar's virtual arm and their real arm.
% In the secondary task, we randomly render a visual animation of a ball turning from transparent to red and becoming transparent again and ask them to monitor and report when it appears.
% We control the strength of the visual stimuli by changing the duration and location of the animation.
% % By changing the time duration and location of the visual animation, we control the strengths of attraction to the users.
% As a result, we found significant differences in the noticeability of the offsets $(F_{(1,15)}=5.90,~p=0.03)$ between conditions with and without visual stimuli.
% Based on further analysis, we also identified the behavioral patterns of the user's gaze (including pupil dilation, fixations, and saccades) to be correlated with the noticeability results $(r=-0.43)$ and they may potentially serve as indicators of noticeability.
% }

% \delete{
% To further investigate how visual attention influences the noticeability, we conduct a data collection study (N = 12) and build a regression model based on the data.
% The regression model is able to calculate the noticeability of the offset applied on the user's arm under various visual stimuli based on their gaze behaviors.
% Our leave-one-out cross-validation results show that the proposed method was able to achieve a mean-squared error (MSE) of 0.012 in the probability regression task.
% }

% \delete{
% To verify the feasibility and extendability of the regression model, we conduct an evaluation study where we test new visual animations based on adjustments on scale and color and invite 24 new participants to attend the study.
% Results show that the proposed method can accurately estimate the noticeability with an MSE of 0.014 and 0.012 in the conditions of the color- and scale-based visual effects.
% Since these animations were not included in the dataset that the regression model was built on, the study demonstrates that the gaze behavioral features we extracted from the data capture a generalizable pattern of the user's visual attention and can indicate the corresponding impact on the noticeability of the offset.
% }

% \delete{
% Finally, we demonstrate applications that can benefit from the noticeability prediction model, including adaptive motion offsets and opportunistic rendering, considering the user's visual attention. 
% We conclude with discussions of our work's limitations and future research directions.
% }

% \delete{
% In summary, we make the following contributions.
% }
% % 
% \begin{itemize}
%     \item 
%     \delete{
%     We quantify the effects of the user's visual attention directed away by stimuli on their noticeability of an offset applied to the avatar's arm motion with respect to the user's physical arm. 
%     Through two user studies, we identified gaze behavioral features that are indicative of the changes in noticeability.
%     }
%     \item 
%     \delete{We build a regression model that takes the user's gaze behavioral data and the offset applied to the arm motion as input, then computes the probability of the user noticing the offset.
%     Through an evaluation study, we verified that the model needs no information about the source attracting the user's visual attention and can be generalizable in different scenarios.
%     }
%     \item 
%     \delete{We demonstrate two applications that potentially benefit from the regression model, including adaptive motion offsets and opportunistic rendering.
%     }

% \end{itemize}

\begin{comment}
However, users will lose the sense of embodiment to the virtual avatars if they notice the offset between the virtual and physical movements.
To address this, researchers have been exploring the noticing threshold of offsets with various magnitudes and proposing various redirection techniques that maintain the sense of embodiment~\cite{}.

However, when users embody virtual avatars to explore virtual environments, they encounter various visual effects and content that can attract their attention~\cite{}.
During this, the user may notice an offset when he observes the virtual movement carefully while ignoring it when the virtual contents attract his attention from the movements.
Therefore, static offset thresholds are not appropriate in dynamic scenarios.

Past research has proposed dynamic mapping techniques that adapted to users' state, such as hand moving speed~\cite{frees2007prism} or ergonomically comfortable poses~\cite{montano2017erg}, but not considering the influence of virtual content.
More specifically, PRISM~\cite{frees2007prism} proposed adjusting the C/D ratio with a non-linear mapping according to users' hand moving speed, but it might not be optimal for various virtual scenarios.
While Erg-O~\cite{montano2017erg} redirected users' virtual hands according to the virtual target's relative position to reduce physical fatigue, neglecting the change of virtual environments. 

Therefore, how to design redirection techniques in various scenarios with different visual attractions remains unknown.
To address this, we investigate how visual attention affects the noticing probability of movement offsets.
Based on our experiments, we implement a computational model that automatically computes the noticing probability of offsets under certain visual attractions.
VR application designers and developers can easily leverage our model to design redirection techniques maintaining the sense of embodiment adapt to the user's visual attention.
We implement a dynamic redirection technique with our model and demonstrate that it effectively reduces the target reaching time without reducing the sense of embodiment compared to static redirection techniques.

% Need to be refined
This paper offers the following contributions.
\begin{itemize}
    \item We investigate how visual attractions affect the noticing probability of redirection offsets.
    \item We construct a computational model to predict the noticing probability of an offset with a given visual background.
    \item We implement a dynamic redirection technique adapting to the visual background. We evaluate the technique and develop three applications to demonstrate the benefits. 
\end{itemize}



First, we conducted a controlled experiment to understand how users perceived the movement offset while subjected to various distractions.
Since hand redirection is one of the most frequently used redirections in VR interactions, we focused on the dynamic arm movements and manually added angular offsets to the' elbow joint~\cite{li2022modeling, gonzalez2022model, zenner2019estimating}. 
We employed flashing spheres in the user's field of view as distractions to attract users' visual attention.
Participants were instructed to report the appearing location of the spheres while simultaneously performing the arm movements and reporting if they perceived an offset during the movement. 
(\zhipeng{Add the results of data collection. Analyze the influence of the distance between the gaze map and the offset.}
We measured the visual attraction's magnitude with the gaze distribution on it.
Results showed that stronger distractions made it harder for users to notice the offset.)
\zhipeng{Need to rewrite. Not sure to use gaze distribution or a metric obtained from the visual content.}
Secondly, we constructed a computational model to predict the noticing probability of offsets with given visual content.
We analyzed the data from the user studies to measure the influence of visual attractions on the noticing probability of offsets.
We built a statistical model to predict the offset's noticing probability with a given visual content.
Based on the model, we implement a dynamic redirection technique to adjust the redirection offset adapted to the user's current field of view.
We evaluated the technique in a target selection task compared to no hand redirection and static hand redirection.
\zhipeng{Add the results of the evaluation.}
Results showed that the dynamic hand redirection technique significantly reduced the target selection time with similar accuracy and a comparable sense of embodiment.
Finally, we implemented three applications to demonstrate the potential benefits of the visual attention adapted dynamic redirection technique.
\end{comment}

% This one modifies arm length, not redirection
% \citeauthor{mcintosh2020iteratively} proposed an adaptation method to iteratively change the virtual avatar arm's length based on the primary tasks' performance~\cite{mcintosh2020iteratively}.



% \zhipeng{TO ADD: what is redirection}
% Redirection enables novel interactions in Virtual Reality, including redirected walking, haptic redirection, and pseudo haptics by introducing an offset to users' movement.
% \zhipeng{TO ADD: extend this sentence}
% The price of this is that users' immersiveness and embodiment in VR can be compromised when they notice the offset and perceive the virtual movement not as theirs~\cite{}.
% \zhipeng{TO ADD: extend this sentence, elaborate how the virtual environment attracts users' attention}
% Meanwhile, the visual content in the virtual environment is abundant and consistently captures users' attention, making it harder to notice the offset~\cite{}.
% While previous studies explored the noticing threshold of the offsets and optimized the redirection techniques to maintain the sense of embodiment~\cite{}, the influence of visual content on the probability of perceiving offsets remains unknown.  
% Therefore, we propose to investigate how users perceive the redirection offset when they are facing various visual attractions.


% We conducted a user study to understand how users notice the shift with visual attractions.
% We used a color-changing ball to attract the user's attention while instructing users to perform different poses with their arms and observe it meanwhile.
% \zhipeng{(Which one should be the primary task? Observe the ball should be the primary one, but if the primary task is too simple, users might allocate more attention on the secondary task and this makes the secondary task primary.)}
% \zhipeng{(We need a good and reasonable dual-task design in which users care about both their pose and the visual content, at least in the evaluation study. And we need to be able to control the visual content's magnitude and saliency maybe?)}
% We controlled the shift magnitude and direction, the user's pose, the ball's size, and the color range.
% We set the ball's color-changing interval as the independent factor.
% We collect the user's response to each shift and the color-changing times.
% Based on the collected data, we constructed a statistical model to describe the influence of visual attraction on the noticing probability.
% \zhipeng{(Are we actually controlling the attention allocation? How do we measure the attracting effect? We need uniform metrics, otherwise it is also hard for others to use our knowledge.)}
% \zhipeng{(Try to use eye gaze? The eye gaze distribution in the last five seconds to decide the attention allocation? Basically constructing a model with eye gaze distribution and noticing probability. But the user's head is moving, so the eye gaze distribution is not aligned well with the current view.)}

% \zhipeng{Saliency and EMD}
% \zhipeng{Gaze is more than just a point: Rethinking visual attention
% analysis using peripheral vision-based gaze mapping}

% Evaluation study(ideal case): based on the visual content, adjusting the redirection magnitude dynamically.

% \zhipeng{(The risk is our model's effect is trivial.)}

% Applications:
% Playing Lego while watching demo videos, we can accelerate the reaching process of bricks, and forbid the redirection during the manipulation.

% Beat saber again: but not make a lot of sense? Difficult game has complicated visual effects, while allows larger shift, but do not need large shift with high difficulty



\mysection{Background and Motivation}
\label{sec:background}
%!TEX root = 2024_auv_mola_drl6dof_main.tex
%%%%%%%%%%%%%%%%%%%%%%%%%%%%%%%%%%%%%%%%%%%%%%%%%%%%%%%%%%%%%%%%%%%%%%%
\section{Background}
\label{sec:background}

%%%%%%%%%%%%%%%%%%%%%%%%%%%%%%%%%%%%%%%%%%%%%%%%%%
%%% Reinforcement learning
\subsection{Deep Reinforcement learning}

\ac{drl} \cite{RichardSutton20} is a method that aims to train an agent's policy $\pi$ to map states into actions by interacting with the environment. This is achieved by maximizing a numerical reward signal and using a \ac{mdp} framework to regulate the interaction between the \ac{rl} agent’s policy and the environment. At each time step, the agent observes a state $\bm{s}$, takes an action $\bm{a}$, and upon transitioning to the next state, receives a reward $r$. Once the episode (i.e., process) is complete, the accumulated reward is calculated as the sum of all time steps rewards in that episode.

\ac{drl} methods can be model-based or model-free. Model-based methods use a model to predict the next state and reward, while model-free methods learn solely from experiencing the unmodeled and unknown consequences of an action. While learning from trial and error may result in less efficient learning, model-free methods have the advantage when a model is unavailable or inaccurate.

\begin{figure}[t!]
\centering
\includegraphics[width=0.45\textwidth]{figures/reward_vs_step.pdf}%
\caption{Average reward per episode over a moving window of 100 episodes obtained by the TQC, SAC, and TD3 algorithms during a $2.5\times10^6$ step training, equivalent to 3125 episodes.}
\label{fig:rewards}
\end{figure}

%%%%%%%%%%%%%%%%%%%%%%%%%%%%%%%%%%%%%%%%%%%%%%%%%%
%% 6DOF Error Computation
\subsection{\ac{6dof} Error Computation}

The position errors are determined by the difference between the current position $(x, y, z)$ and the goal position $(x_d, y_d, z_d)$ following the North-East-Down (NED) convention, computed as
%%%
% Keep to remove space between equations and paragraph
%%%
\begin{equation}
    e_x(t) = x^t - x_d^t,\; e_y(t) = y^t - y_d^t,\; e_z(t) = z^t - z_d^t.
\label{eq:errors}
\end{equation}

To compute the error in attitude, we will evaluate the difference between the current orientation and the goal attitude, both with respect to the fixed world frame. This involves representing both poses as rotation matrices ($\bm{R}\in SO(3)$) and converting their difference to exponential coordinates $[\bm{{e_\theta}}]\in so(3)$ through the matrix logarithm:
%%%
% Keep to remove space between equations and paragraph
%%%
\begin{equation}
     [\bm{{e_\theta}}(t)] = \log(\bm{R}(t)^T \cdot \bm{R}_d)
\end{equation}

Then, the skew-symmetric matrix $[\bm{{e_\theta}}(t)]$ is converted into its vector representation $\bm{{e_\theta}}(t) \in \mathbb{R}^3$, where its entries correspond to the element-wise error for the attitude, defined as
%%%
% Keep to remove space between equations and paragraph
%%%
\begin{equation}
    \begin{bmatrix} \theta_{x}^t & \theta_{y}^t & \theta_{z}^t \end{bmatrix} = \bm{{e_\theta}}(t).
    \label{eq:attitude_error}
\end{equation}

Furthermore, to provide a single metric for attitude error evaluation, we compute $\theta^t$ based on the axis-angle representation for $\bm{{e_\theta}}(t)$, as described in \eqref{eq:theta_error}. By using this metric, we obtain a global evaluation of orientation, which aligns the controller's performance with practical manual navigation comparisons.
%%%
% Keep to remove space between equations and paragraph
%%%
\begin{equation}
    \theta^t = ||\bm{{e_\theta}}(t)||
    \label{eq:theta_error}
\end{equation}



\mysection{System Overview}
\label{sec:overview}

% \begin{figure}[t]
%     \small
%     \centering
%     \includegraphics[width=0.6\textwidth]{VoluLUT/figures/arch.pdf}
%     \vspace{-.0in}
%     \caption{\small The System Architecture of \name.}
%     \label{fig:arch}
%     \vspace{-.0in}
% \end{figure}





% \name is a novel approach to streaming volumetric video content with super-resolution enhancement. It aims to deliver high-quality volumetric video-on-demand from an Internet server to client hosts. On server side, the video is segmented into chunks with fixed number of frames and further encoded with dynamic resolution (\ie point densities). \anlan{[Why does \name need to encode a video into different quality levels, given that it does continuous ABR?]} As shown in Figure~\ref{fig:arch}, Upon receiving the downsampled frames from the server, the client performs super-resolution(SR) using our efficient approach. The SR process consists of two main steps: kNN-based interpolation and colorization (\S\ref{sec:inter}), followed by fine-tuning of the interpolated points using our model-defined LUTs (\S\ref{sec:lut}). The kNN-based interpolation increases the point density of the low-resolution point clouds, while the colorization step assigns appropriate colors to the upsampled points. The fine-tuning step, which utilizes our pre-computed LUTs, further enhances the visual quality of the upsampled point clouds. The point clouds with restored resolution are then rendered for the users. 

% To achieve real-time SR performance, \name employs several optimizations tailored for SR-enhanced volumetric video streaming. These optimizations include boosting the interpolation with dilation (\S\ref{sec:inter}), transforming the reference finetuning model to Look-up Table (LUT) for efficient inference (\S\ref{sec:lut}) and reusing the knn result to avoid extra computation. These approaches ensures the quality and latency of the SR is only slightly impacted by the quality ratio, which allows continuous scale super-resolution.

% The \name makes the crucial decision to determine an optimal downsampling ratio based on the current network and buffer conditions (\S\ref{sec:ABR}). This continuous ratio is then sent to the server, which applies it to the requested video frames before transmitting them to the client. By dynamically adjusting the downsampling ratio and combining adaptive downsampling, efficient SR techniques, and optimized streaming logic, \name adapts to varying network conditions and ensures efficient bandwidth utilization while delivering high-quality volumetric video content to clients.

\name enables high-quality streaming of volumetric video content by combining adaptive bitrate streaming with super-resolution enhancement. The server segments videos into fixed-length chunks and encodes them at requested point densities. 

As shown in Figure~\ref{fig:arch}, the client processes received low-resolution frames through three stages: kNN-based interpolation with dilation to increase point density (\S\ref{sec:inter}), colorization based on spatial relationships (\S\ref{sec:inter}), and LUT-based fine-tuning to enhance visual quality (\S\ref{sec:lut}). To achieve real-time performance, \name transforms neural networks into memory-efficient LUTs and reuses kNN results across pipeline stages, ensuring stable quality and latency across different upscaling ratios.

The system dynamically selects optimal downsampling ratios based on network conditions and buffer status (\S\ref{sec:ABR}). Through this combination of adaptive downsampling and efficient super-resolution, \name delivers high-quality volumetric video while optimizing bandwidth utilization across varying network conditions.


\mysection{LUT based point cloud Super-resolution}
\label{sec:SR}


% \subsection{Enhanced Interpolation with Colorization}
% \label{sec:inter}




% \anlan{[I think there lacks a clear motivation on why we want to improve the interpolation stage, as what we essentially care about is the final SR quality. For example, a motivation could be: we try to improve the outcome of the interpolation stage because we find that the final SR quality highly depends on this intermediate result.]}

% The first stage of our SR pipeline addresses the fundamental challenge of increasing point cloud density while maintaining visual quality and computational efficiency. Traditional kNN-based interpolation approaches face critical limitations in their ability to maintain uniform point distributions, often distorting surface geometry particularly in regions with complex features (as shown in fig~\ref{fig:dila-visual}). They also require computationally expensive neighbor search operations that hinder real-time performance, while struggling to restore color information effectively. Our enhanced interpolation framework tackles these challenges through a comprehensive approach that combines dilated sampling \anlan{[We may need to briefly introduce somewhere what dilated sampling is. The reviewers may not know this concept.]}, hierarchical spatial indexing, and fast color recovery.
\begin{figure*}[t]
\small
\includegraphics[width=1\textwidth]{figures/interpolation.png} % Second image
\caption{The pipeline of two-stage Super-resolution with LUT refinement}
\label{fig:lut-pipe}
\end{figure*}





% \subsubsection{Dilated Interpolation for Uniform Upsampling}
% Our approach introduces a novel dilated interpolation technique that fundamentally rethinks point selection for interpolation. Traditional methods that rely solely on k-nearest neighbors often reinforce existing density patterns, creating clusters in already dense regions while leaving sparse areas under-sampled. Our key insight is that by carefully expanding the sampling neighborhood, we can break these density patterns and achieve more uniform point distribution. \anlan{[A potential question could be: why is uniform point distribution better than the original point distribution, as the original point distribution more faithfully follow the ground truth point distribution when the point cloud is captured?]}

% For a point cloud $P = \{p_1, p_2, \ldots, p_n\}$, we first identify a dilated neighborhood for each point $p_i$. This expanded neighborhood is defined as:
% \begin{equation}
% N_{dk}(p_i) = \{p_j \in P | \text{rank}(||p_j - p_i||) \leq d \times k\}
% \end{equation}
% where $d$ is our dilation factor (typically set to 2-4) and $k$ is the desired number of neighbors (usually 8-16). The rank function ensures we select points based on their distance \anlan{[What distance? Explain the notion $||x||$.]} ordering rather than absolute distances, making the method more robust to varying point densities.

% From this expanded pool, we select interpolation partners using a distance-weighted probability distribution:
% \begin{equation}
% P(p_j \in S_i) \propto \frac{1}{||p_j - p_i||} \cdot f(p_j)
% \end{equation}
% where $f(p_j)$ is a feature-preserving weight that considers local surface properties such as normal vectors and curvature. \anlan{[What is $S_{i}$ denoted to?]} This weighted selection strategy ensures closer points have higher selection probability, preserving local surface details, while more distant points maintain a chance of selection to help break density patterns. Surface features are preserved through the feature-aware weighting.

\begin{figure}[t]
\small
\centering
\includegraphics[width=0.48\textwidth]{figures/inter.pdf}
\vspace{-.1in}
\caption{Qualitative upsampling results (from left to right): \emph{Groundtruth}, \emph{Interpolation with dilation}, \emph{Naive knn-based interpolation}. Our method achieves more uniform point distribution while preserving geometric details.}
\vspace{-.3in}
\label{fig:dila-visual}
\end{figure}

\begin{figure}[t]
\small
\includegraphics[width=0.5\textwidth]{figures/interpolationw_dilation.png}
\caption{Interpolation with and without dilation. Receptive Field size = $k \times$ dilation. The dilated approach significantly improves point distribution uniformity and surface coverage.}
\label{fig:dilation}
\end{figure}

% \subsubsection{Hierarchical kNN Computation with Relationship Reuse}
% To make our dilated interpolation approach practical for real-time applications \anlan{[Need some evidences that the original design runs to slowly on commodity/mobile devices.]}, we implement an efficient two-layer octree structure. Unlike traditional octrees that subdivide until reaching a minimum cell size, our approach uses a fixed two-layer structure that balances spatial organization benefits against traversal overhead.

% The octree construction process begins by dividing the point cloud into eight major regions based on the point cloud centroid. Each first-layer region is then subdivided into eight sub-regions, creating leaf nodes that store rich statistical information including point count, density estimation, bounding box, centroid, average normal vector, and curvature. This octree-based spatial division enables efficient pruning during neighbor searches, as regions unlikely to contain relevant neighbors can be quickly eliminated.

% We further optimize performance through neighbor relationship reuse. For each interpolated point $p'$ generated between points $p$ and $q$, we observe that:
% \begin{equation}
% N_k(p') \approx \text{MergeAndPrune}(N_k(p), N_k(q))
% \end{equation}
% where $N_k(p')$ represents the k-nearest neighbors of $p'$. This approximation dramatically reduces computation time by avoiding full neighbor searches for interpolated points.

% \subsubsection{Fast Colorization with kNN Reuse}
% Our colorization approach leverages the spatial relationships computed during geometric interpolation to achieve efficient and accurate color assignment. For each interpolated point $p'$, we compute its color using the closest neighbour point from the original point cloud.

% The reuse of spatial relationships from the geometric interpolation phase eliminates the need for additional neighbor searches during colorization. This optimization, combined with our dilated neighbour selection, enables real-time performance while maintaining high visual quality.

% Our experimental results demonstrate that this enhanced interpolation framework achieves significantly more uniform point distribution compared to traditional kNN interpolation, faster computation through efficient neighbor reuse, and improved color preservation, particularly at object boundaries and in regions with complex texture patterns.



\begin{figure}[t]
\small
\centering
\includegraphics[width=0.48\textwidth]{figures/lut-lookup.png}
\vspace{-.2in}
\caption{ LUT look up example.}
\vspace{-.2in}
\label{fig:lut-lookup}
\end{figure}


\mysubsection{Enhanced Interpolation with Colorization}
\label{sec:inter}

Given a downsampled point cloud along with a desired upsampling ratio, we first perform interpolation to increase the point density.
The quality of the final super-resoluted point cloud critically depends on the initial interpolation stage. Our qualitative results reveal that poor initial point distributions create artifacts (Figure~\ref{fig:dila-visual}) that persist even after neural refinement. Additionally, traditional interpolation methods create a severe performance bottleneck—our measurements on GradPU show that naive kNN-based interpolation consumes over 70\% of the frame time, making real-time operation infeasible.

Central to both interpolation and subsequent refinement is the concept of receptive fields (RF)—the local spatial regions considered when processing each point. As illustrated in Figure~\ref{fig:dilation}, the receptive field determines which neighboring points influence the position and attributes of newly generated points. In traditional kNN approaches, each new point is generated by considering only its k closest neighbors, which causes two distinct problems. First, this method tends to reinforce existing density patterns because points in dense regions have closer neighbors than those in sparse regions, leading to uneven point distributions. Second, the neighbor search operations required for each new point are computationally expensive, especially as the point cloud size grows.
\name enables two optimizations in interpolation stage.

\noindent \textbf{Dilated Interpolation for Uniform Upsampling}
 Our key insight is that by carefully expanding the sampling neighborhood through dilation, we can break the artifact introduced by traditional kNN interpolation while still preserving important geometric features. As shown in Figure~\ref{fig:dilation}, our dilated approach examines a broader spatial region during interpolation, defined by a receptive field of size $k \times d$, where $k$ is the number of neighbors and $d$ is the dilation factor.

For a point cloud $P = \{p_1, p_2, \ldots, p_n\}$, we define a dilated neighborhood for each point $p_i$ as:
\begin{equation}
N_{dk}(p_i) = \{p_j \in P_{n} | P_{n} = \text{Top}_{d \times k}(||p_j - p_i||_2) \}
\end{equation}
where $d$ is the dilation factor, $k$ is the desired neighbor count, and $||x||_2$ denotes Euclidean distance. The Top function orders points by distance increasingly and keeps the first $d \times k$ ones, similar as the request of vanilla kNN. From this expanded neighborhood, we randomly select a subset $S_i$ of points for interpolation based on the target upsampling ratio requirement. 

An alternative solution may interpolate the point cloud to a higher density and perform Farthest Point Sampling~\cite{liAdjustableFarthestPoint2022} (FPS) to the target upsampling ratio. FPS iteratively samples the farthest point and updates the distance, which can preserve geometry feature but introduce unacceptable computation latency ($\geq5$ minutes to downsample a 200K points to 100K points on a commodity desktop).

\noindent \textbf{Hierarchical kNN Computation with Relationship Reuse}
To achieve real-time performance on mobile devices, we adopt an efficient two-layer octree~\cite{schnabel_octree-based_nodate} structure that balances spatial organization against traversal overhead. Our measurements show that naive dilated interpolation takes over 100ms per 100K-points-frame on an Orange Pi, making optimization essential for real-time operation.

The octree divides the point cloud into eight major regions at the first layer, with each region further subdivided into eight sub-regions. While the construction of the octree takes limited effort, its leaf nodes store a subset of the points whose neighbour points are highly likely self-contained. This hierarchical structure enables rapid neighbor search through efficient spatial pruning.

We further accelerate computation through neighbor relationship reuse. For each interpolated point $p'$ generated between points $p$ and $q$, we observe that:
\begin{equation}
N_k(p') \approx \text{MergeAndPrune}(N_k(p), N_k(q))
\end{equation}
where $N_k(p')$ represents the k-nearest neighbors of $p'$. This approximation eliminates redundant neighbor searches while maintaining accuracy.

We colorize new points based on the nearest original point, reusing spatial relationships from geometric interpolation to avoid redundant computations.


\mysubsection{Interpolation Refinement with LUT}
\label{sec:lut}

As discussed in \S~\ref{sec:pcsr}, a refinement function is required to adjust the interpolated point clouds for better visual quality. We propose an LUT-based refinement approach (shown in Figure~\ref{fig:lut-pipe}) that first captures refinement patterns through offline neural network training, then transfers this knowledge into an efficient lookup table for real-time inference.

% \subsubsection{Neural Network Training for LUT Construction}
% \label{sec:NN}

% \anlan{[I think this part needs to be moved to the front, when talking about improving SR quality.]}

% \anlan{[Question: (1) Does the training process follow the same training process applied in GradPU at a high-level? (2) Is the refinement DNN the same as that of GradPU?]}

% Unlike previous approaches that directly predict upsampled point coordinates (shown in Fig~\ref{fig:SR}), our method first learns a refinement function through a neural network, which is later transformed into a lookup table. This two stage NN-baded pointcloud super-resolution solution is firstly proposed in GradPU~\cite{he_grad-pu_2023} and achieves sota performance in terms of visual quality. This decoupling of upsampling and refinement enables more flexible system level optimization and pipelining, which allow better potential to runtime performance.

% The refinement network operates on local point neighborhoods, taking as input a patch centered at each interpolated point with its four nearest neighbors. For each point, the network learns to predict the optimal offset that would move it closer to its correct position in the high-resolution point cloud:
% \begin{equation}
% \delta_i = \text{NN}(p_i, P_h) - p_i
% \end{equation}

% To improve the network's ability to handle different refinement scenarios, we introduce Gaussian noise to the interpolated points during training. This noise injection with standard deviation of 0.02 serves two crucial purposes: it makes the network more robust to varying point positions during refinement iterations, and it helps learn smoother refinement fields around each point.

% Our training objective combines three key components. First, we ensure accurate offset prediction by minimizing the distance between predicted and ground truth offsets. Second, we encourage uniform point distribution through a repulsion term that prevents points from clustering too closely. Third, we add regularization that promotes smooth refinement fields, making the learned function more stable and suitable for LUT conversion.

% The network architecture consists of two main components. The feature extraction stage uses a simplified Point 4D Convolution that operates in 3D space, effectively capturing local geometric patterns while maintaining computational efficiency. This is followed by a lightweight MLP-based predictor that generates the final offset values.

% During training, we process patches of 256 points with a batch size of 32. The network is optimized using Adam optimizer with an initial learning rate of 1e-3, which decays by half every 20 epochs. The entire training process runs for 60 epochs, balancing between refinement accuracy and training efficiency. \anlan{[This can go to Implementation/Evaluation section.]}

% The careful design of our training process ensures that the network learns robust refinement patterns that can be effectively transferred to the lookup table format. By focusing on local geometric relationships and incorporating noise during training, we achieve a refinement function that is both accurate and generalizable. Most importantly, while the training process may be complex, the resulting knowledge can be efficiently encoded into our LUT structure, enabling real-time refinement during inference.



% \subsubsection{Position Encoding and LUT Construction}
% The key challenge in constructing a lookup table for point refinement is converting continuous 3D point positions into discrete indices while preserving geometric relationships. As shown in Figure~\ref{fig:lut-lookup}, our solution involves a three-step encoding pipeline followed by comprehensive table construction and lookup. \anlan{[We may need to clearly introduce the input and output of the LUT first.]}

% \paragraph{Position Encoding Pipeline}
% Given a point and its three nearest neighbors, our encoding process proceeds as follows:

% First, we record the raw positions of the neighborhood. As illustrated in the figure step (a), a typical neighborhood might have coordinates like (1,3,5), (1,6,5), (1,4,3), (1,5,2), representing the actual 3D positions of the points in space. These raw coordinates can span any range of values depending on the input point cloud's scale and position. 

% Next, we perform normalization to achieve translation and scale invariance, shown in step (b). The neighborhood is centered at the reference point and scaled by the neighborhood radius. This transforms our example coordinates to normalized values like (0,0,1), (0,1,1), (0,0.3,0.3), (0,0.7,1). \anlan{[How do you normalize the example coordinates to this value? The normalization process is not very clear to me.]} After this step, all neighborhoods are represented in a consistent coordinate frame within the unit cube, regardless of their original position or scale.

% Finally, in step (c), we quantize the normalized coordinates into 16 discrete levels along each dimension. Our example coordinates become (0,0,15), (0,15,15), (0,5,5), (0,11,15). We chose 16 levels based on experiments balancing precision and memory requirements - it provides sufficient geometric detail while keeping the lookup table size manageable. \anlan{[We may need some numbers to show the trade-off.]}

% \paragraph{Comprehensive LUT Construction}

% After establishing the encoding scheme, we construct the lookup table through an exhaustive enumeration process. With 16 quantization levels for each coordinate (x,y,z) and four points in each neighborhood, we generate all possible input patterns. For a single neighbor point, there are $16^{3}$ possible positions. With four neighbor points, this gives us $(16^{3})^{4}$ possible neighborhood configurations.

% The LUT construction process systematically generates each possible quantized pattern, from ((0,0,0), (0,0,0),(0,0,0),(0,0,0)) to ((15,15,15), (15,15,15), (15,15,15), (15,15,15)). These patterns are then dequantized back to normalized space and fed through our trained refinement network NN. The resulting offset is stored in the corresponding LUT entry pairing with the quantized input as the index. This process help us freeze the NN trained in \S~\ref{sec:NN} with acceptable accuracy loss and significant inference speed-up. \anlan{[How long does it take to generate the LUT? Better to report the number.]}

% For a practical point refinement LUT, the receptive field size and quantization precision must be carefully chosen as they directly impact the memory footprint and refinement quality. For point refinement in 3D space, each point requires three coordinates. Given a receptive field of size n and quantization using b bins per dimension, the LUT requires $b^{3n} \times \text{sizeof}(\text{float})$ bytes of storage, where the factor of 3n in the exponent accounts for three coordinates per point across n points.

% The bin size b determines the granularity of coordinate quantization, with each coordinate dimension divided into b discrete levels. For example, with b=16 bins, each coordinate axis is divided into 16 segments, while b=8 uses 8 bins per axis. This quantization directly affects both storage requirements and refinement precision. For n points in 3D space, we need to store $b^{3n}$ entries since each point requires three coordinates, and each coordinate can take b different values.

% Table~\ref{tab:lut_size} illustrates the memory requirements for various configurations:

% \begin{table}[t]
% \caption{Memory analysis for different LUT configurations}
% \label{tab:lut_size}
% \centering
% \begin{tabular}{@{}l r r r@{}}
% \toprule
% \multicolumn{1}{c}{RF Size (n)} & 
% \multicolumn{1}{c}{Bins (b)} & 
% \multicolumn{1}{c}{Entries ($b^{3n}$)} & 
% \multicolumn{1}{c}{Memory (4B/float)} \\ 
% \midrule
% 2 & 16 & $16^6$ & 16\,MB \\
% 2 & 8 & $8^6$ & 256\,KB \\
% 3 & 16 & $16^9$ & 4\,GB \\
% 3 & 8 & $8^9$ & 8\,MB \\
% 4 & 16 & $16^{12}$ & 1\,TB \\
% 4 & 8 & $8^{12}$ & 2\,GB \\
% \bottomrule
% \end{tabular}
% \end{table}

% Each LUT entry stores a mapping from quantized input coordinates to refinement offsets. The input coordinates are represented as n-tuples of 3D points, while the offsets are stored as floating-point values. The input space quantization follows a uniform discretization strategy, dividing each coordinate's normalized range [0,1] into equally spaced bins. This uniform quantization simplifies both the construction process and runtime lookup operations.

% The choice of bin size presents a crucial trade-off between memory efficiency and refinement precision. Reducing the number of bins from 16 to 8 decreases the memory requirement by a factor of $2^{3n}$, at the cost of reduced coordinate precision. For a receptive field size of 4, this reduction brings the memory footprint from an impractical 1 TB to a more manageable 2 GB, enabling practical deployment while maintaining acceptable refinement quality.

% \mysubsubsection{Position Encoding and LUT Construction}
% \label{sec:lut_construction}

% The key challenge in constructing a lookup table for point refinement is converting continuous 3D point positions into discrete indices while preserving geometric relationships. As shown in Figure~\ref{fig:lut-lookup}, our solution consists of an encoding pipeline for position discretization and a systematic table construction process for storing refinement offsets.

% \paragraph{LUT Construction}
% The lookup table stores precomputed refinement offsets for all possible quantized neighborhood configurations. For a receptive field of size $n$ points with 3 coordinates in 3D space and $b$ quantization bins per dimension, the total number of possible combinations for input indices is:
% \begin{equation}
% N_{entries} = b^n \times 3
% \end{equation}
% As illustrated in step (d) of Figure~\ref{fig:lut-lookup}, each LUT entry maps a sequence of quantized coordinates to a refinement offset in three dimensions:
% \begin{equation}
% \text{LUT}[\text{quantize}(\mathbf{q}_1,\dots,\mathbf{q}_n)] = \text{NN}(\mathbf{q}_1,\dots,\mathbf{q}_n)
% \end{equation}
% The memory requirement for storing these entries with 2-byte floating-point values (float16) for three coordinate offsets is:
% \begin{equation}
% M = N_{entries} \times 2\text{ bytes}
% \end{equation}
% Table~\ref{tab:lut-size} analyzes the memory requirements for different configurations. The selection of bin size $b$ presents a crucial trade-off between memory efficiency and refinement precision. Our implementation uses $b=128$ (7-bit quantization) with a receptive field size $n=4$, resulting in a 1.6 GB lookup table that stores 3D coordinate offsets in $float16$ format. This configuration achieves a good balance between memory efficiency and refinement precision, making the approach practical for real-world deployment.
\mysubsubsection{Position Encoding and LUT Construction}
\label{sec:lut_construction}
The key challenge in constructing a lookup table for point refinement is converting continuous 3D point positions into discrete indices while preserving geometric relationships. As shown in Figure~\ref{fig:lut-lookup}, we address this through a systematic encoding pipeline that transforms raw 3D coordinates into quantized indices for efficient lookup and refinement.
\paragraph{Position Encoding Pipeline}
Our encoding process consists of three key steps:

\BULLET \textbf{Position Input(Stage (a)):} The pipeline takes as input a neighborhood of 3D points represented as $(x,y,z)$ coordinates. For a receptive field of size $n$, we process the target point along with its $n-1$ neighboring points' positions in 3D space.

\BULLET \textbf{Normalization(Stage (b)):} To ensure consistent lookup behavior, we normalize the coordinates relative to the center point:
\begin{equation}
    \mathbf{n}_i = \frac{\mathbf{r}_i - \mathbf{r}_c}{R}
\end{equation}
where $\mathbf{r}_c$ is the center point coordinates and $R$ is the neighborhood radius (maximum distance from any point to the center). This transforms ensuring all points lie within the unit cube $[-1,1]^3$.

\BULLET \textbf{Quantization(Stage (c)):} The normalized coordinates are then discretized into fixed-size $b$ bins to create lookup indices:
\begin{equation}
    \mathbf{q}_i = \lfloor (\frac{\mathbf{n}_i + 1}{2}) \times (b-1) \rfloor
\end{equation}
 This step converts continuous normalized values into discrete integer indices suitable for table lookup, effectively creating a finite set of possible neighborhood configurations. 

\paragraph{LUT Construction and Usage}
The lookup table stores precomputed refinement offsets for all possible quantized neighborhood configurations. For a receptive field of size $n$ points with 3 coordinates in 3D space and $b$ quantization bins per dimension, the total number of possible combinations for input indices is:
\begin{equation}
N_{entries} = b^n \times 3
\end{equation}
As illustrated in step (d) of Figure~\ref{fig:lut-lookup}, each LUT entry maps a sequence of quantized coordinates to a refinement offset in three dimensions:
\begin{equation}
\text{LUT}[\text{quantize}(\mathbf{q}_1,\dots,\mathbf{q}_n)] = \text{NN}(\mathbf{q}_1,\dots,\mathbf{q}_n)
\end{equation}
The memory requirement for storing these entries with 2-byte floating-point values (float16) for three coordinate offsets is:
\begin{equation}
M = N_{entries} \times 2\text{ bytes}
\end{equation}
Table~\ref{tab:lut-size} analyzes the memory requirements for different configurations. The selection of bin size $b$ presents a crucial trade-off between memory efficiency and refinement precision. Our implementation uses $b=128$ (7-bit quantization) with a receptive field size $n=4$, resulting in a 1.6 GB lookup table that stores 3D coordinate offsets in $float16$ format. This configuration achieves a good balance between memory efficiency and refinement precision, making the approach practical for real-world deployment.
\begin{table}[t]
\centering
\begin{tabular}{@{}c c r r@{}}
\toprule
RF Size ($n$) & 
Bins ($b$) & 
Entries ($b^n \times 3$) & 
Size (2B/offset) \\ 
% \midrule
% 2 & 128 & $128^2 \times 3$ & 98\,KB \\
% 2 & 64 & $64^2\times3$ & 24\,KB \\
\midrule
3 & 128 & $128^3\times3$ & 12\,MB \\
3 & 64 & $64^3\times3$ & 1.5\,MB \\
\midrule
4 & 128 & $128^4\times3$ & 1.61\,GB \\
4 & 64 & $64^4\times3$ & 100\,MB \\
\midrule
5 & 128 & $128^5\times3$ & 201\,GB \\
5 & 64 & $64^5\times3$ & 6.25\,GB \\
\bottomrule
\end{tabular}
\vspace{-.1in}
\caption{Memory analysis for different LUT configurations with float16 (2B) storage}
\label{tab:lut-size}
\vspace{-.3in}
\end{table}

During runtime operation (Stage (a-f) in Figure~\ref{fig:lut-lookup}), we follow this encoding pipeline: given an interpolated point along with its neighbors, we normalize and quantize the coordinates to obtain lookup indices, retrieve the corresponding offset from the table, and apply it to refine the center point's position. Notably, the interpolated point will be placed at first in the index.


% \paragraph{Position Encoding Pipeline}
% Given a neighborhood consisting of a center point $\mathbf{p}_c$ (marked in red in Figure~\ref{fig:lut-lookup}) and its three nearest neighbors (marked in green), our encoding process transforms continuous 3D coordinates into discrete indices through three stages:

% Stage (a) records the raw positions $\mathbf{r}_i = (x_i, y_i, z_i)$ of the neighborhood points. Stage (b) performs normalization to achieve translation and scale invariance:
% \begin{equation}
%     \mathbf{n}_i = \frac{\mathbf{r}_i - \mathbf{r}_c}{R}
% \end{equation}
% where $\mathbf{r}_c$ is the center point coordinates and $R$ is the neighborhood radius (maximum distance from any point to the center). This transforms ensuring all points lie within the unit cube $[-1,1]^3$.

% Stage (c) quantizes the normalized coordinates into $b$ discrete levels per dimension:
% \begin{equation}
%     \mathbf{q}_i = \lfloor (\frac{\mathbf{n}_i + 1}{2}) \times (b-1) \rfloor
% \end{equation}
% \paragraph{LUT Construction}
% The lookup table stores precomputed refinement offsets for all possible quantized neighborhood configurations. For a receptive field of size $n$ points with 3 coordinates in 3D space and $b$ quantization bins per dimension, the total number of possible entries is:
% \begin{equation}
%     N_{entries} = b^{3n}
% \end{equation}
% As illustrated in step (d) of Figure~\ref{fig:lut-lookup}, each LUT entry maps a sequence of quantized coordinates to a refinement offset:
% \begin{equation}
%     \text{LUT}[\mathbf{q}_1,\dots,\mathbf{q}_n] = \text{NN}(\text{dequantize}(\mathbf{q}_1,\dots,\mathbf{q}_n))
% \end{equation}
% where dequantize converts discrete indices back to normalized coordinates for neural network processing.

% The memory requirement for storing these entries with 4-byte floating-point values is:
% \begin{equation}
%     M = b^{3n} \times 4\text{ bytes}
% \end{equation}
% Table~\ref{tab:lut-size} analyzes the memory requirements for different configurations.


% \begin{table}[t]

% \centering
% \begin{tabular}{@{}c c r r@{}}
% \toprule
% RF Size ($n$) & 
% Bins ($b$) & 
% Entries ($b^{3n}$) & 
% Memory (4B/float) \\ 
% \midrule
% 2 & 16 & $16^6$ & 16\,MB \\
% 2 & 8 & $8^6$ & 256\,KB \\
% 3 & 16 & $16^9$ & 4\,GB \\
% 3 & 8 & $8^9$ & 8\,MB \\
% 4 & 16 & $16^{12}$ & 1\,TB \\
% 4 & 8 & $8^{12}$ & 1.5\,GB \\
% \bottomrule
% \end{tabular}
% \vspace{-.1in}
% \caption{Memory analysis for different LUT configurations}
% \label{tab:lut-size}
% \vspace{-.2in}
% \end{table}


% During runtime operation (steps a-f in Figure~\ref{fig:lut-lookup}), we follow this encoding pipeline: given an interpolated point along with its neighbors, we normalize and quantize the coordinates to obtain lookup indices, retrieve the corresponding offset from the table, and apply it to refine the center point's position. Notably, the interpolated point will be placed at first in the index. This approach transforms the computationally intensive neural network inference into efficient table lookups, enabling real-time refinement while maintaining high quality results.

% The selection of bin size $b$ presents a crucial trade-off between memory efficiency and refinement precision. With receptive field size $n=4$, reducing $b$ from 16 to 8 decreases the memory footprint from 1 TB to 1.5 GB while introducing only minimal quality degradation, making the approach practical for real-world deployment.


\mysubsubsection{Neural Network for LUT Construction}
\label{sec:NN}

The refinement network follows the design from GradPU~\cite{he_grad-pu_2023}. Given a interpolated point as the central point $\mathbf{p}_c$ and its $n-1$ nearest neighbors $\{\mathbf{p}_i\}_{i=1}^{n-1}$, our network $\textit{NN}$ computes a refinement offset through:
\begin{equation}
    \boldsymbol{\delta} = \textit{NN}(\mathbf{p}_c, \{\mathbf{p}_i\}_{i=1}^{n-1})
\end{equation}
The receptive field size $n$ is specifically chosen to balance LUT memory requirements and refinement quality. The offset $\boldsymbol{\delta}$ represents the mean displacement between the interpolated points and their groundtruth counterparts:
\begin{equation}
\boldsymbol{\delta} = \frac{1}{|P|}\sum{p_c \in P} ||\mathbf{p}_{gt} - \mathbf{p}_c||_2
\end{equation}
Training $\textit{NN}$ is equivalent to minimize the target loss function  $\boldsymbol{\delta}$.

To assist robust LUT construction, we incorporate two key design elements into the network training. First, Gaussian noise injection ($\sigma = 0.02$) to interpolated points during training improves the network's resilience to quantization artifacts that may arise during the discretization process. Second, we constrain the network's prediction space through normalized coordinate inputs, ensuring the learned function maps well to the LUT's discrete indexing scheme. The complete network training process is detailed in Section~\ref{sec:eval-setup}.

% \paragraph{Runtime Usage}
% During inference, shown in Fig~\ref{fig:lut-lookup}, we simply follow the encoding pipeline to convert each point's neighborhood into a quantized pattern, then use this pattern as an index into our precomputed table. More specifically, given a point with neighborhood coordinates (step (a)), we apply normalization and quantization (step (b) and (c)) to obtain lookup indices, retrieve the corresponding offset (step (d) and (e)), and apply it to refine the point's position (step (f)).

% \chendong{
% \section{Interpolation Refinement with LUT}
% \label{sec:lut}
% While interpolation effectively increases point density, directly predicting 3D coordinates or residuals for point refinement often leads to outliers and shrinkage artifacts. We propose a novel LUT-based refinement approach that first captures refinement patterns through offline neural network training, then transfers this knowledge into an efficient lookup table for real-time inference.
% \subsection{Training Improvements}
% Our refinement network operates on local point neighborhoods, taking as input a patch centered at each interpolated point with its k-nearest neighbors. To enhance robustness and generalization, we introduce Gaussian noise ($\sigma = 0.02$) to the interpolated points during training, improving network resilience to position variations. The network is trained with a multi-objective loss function that combines coordinate accuracy and distribution uniformity constraints.
% \subsection{Position Encoding}
% The transformation of continuous 3D positions into discrete indices requires careful consideration of precision and efficiency. Our encoding pipeline first establishes translation and scale invariance through coordinate centering and neighborhood radius scaling, mapping all neighborhoods into a consistent unit cube representation. The normalized coordinates are then discretized through a quantization process that balances precision with memory requirements.
% \subsection{LUT Construction}
% For a practical point refinement LUT, the receptive field size and quantization precision must be carefully chosen as they directly impact the memory footprint and refinement quality. Given a receptive field of size n and quantization using b bits, the LUT requires $(2^b)^n \times d \times \text{sizeof}(\text{float})$ bytes of storage, where d represents the output dimensions of the refinement offset.
% The bin size b determines the granularity of coordinate quantization, with each input dimension divided into $2^b$ discrete levels. For example, with 8-bit quantization (b=8), each coordinate axis is divided into 256 bins, while 4-bit quantization (b=4) uses 16 bins per axis. This quantization directly affects both storage requirements and refinement precision.
% Table~\ref{tab:lut_size} illustrates the memory requirements for various configurations:
% \begin{table}[t]
% \caption{Memory analysis for different LUT configurations}
% \label{tab:lut_size}
% \centering
% \begin{tabular}{cccc}
% \toprule
% Receptive Field & Bits & Bins per Dim & Memory \
% \midrule
% 2 & 8 & 256 & 192 KB \
% 2 & 4 & 16 & 0.75 KB \
% 3 & 8 & 256 & 48 MB \
% 3 & 4 & 16 & 12 KB \
% 4 & 8 & 256 & 12 GB \
% 4 & 4 & 16 & 192 KB \
% \bottomrule
% \end{tabular}
% \end{table}
% Each LUT entry stores a mapping from quantized input coordinates to refinement offsets. The input coordinates are represented as n-dimensional integer tuples, while the offsets are stored as 3D floating-point vectors. The input space quantization follows a uniform discretization strategy, dividing the normalized coordinate range [0,1] into equally spaced bins. This uniform quantization simplifies both the construction process and runtime lookup operations.
% The choice of bin size presents a crucial trade-off between memory efficiency and refinement precision. Reducing the quantization from 8 bits to 4 bits decreases the memory requirement by a factor of $2^{4n}$, at the cost of reduced coordinate precision. For a receptive field size of 4, this reduction brings the memory footprint from an impractical 12 GB to a manageable 192 KB, enabling practical deployment while maintaining acceptable refinement quality.
% }


\mysection{Continuous Adaptive Bitrate Streaming}
\label{sec:ABR}


Existing volumetric video streaming systems~\cite{han_vivo_2020,zhang_yuzu_nodate,lee_groot_2020} are constrained by discrete quality levels, typically offering fixed point densities (e.g., 100K, 200K points per frame). We introduce a continuous adaptive bitrate (ABR) mechanism that dynamically optimizes streaming quality through fine-grained point density adjustments. This approach is made possible by our upsampling algorithm's consistent latency across varying upsampling ratios (detailed in \S~\ref{sec:eval-runtime}). The ability to support arbitrary downsampling ratios through our super-resolution pipeline enables more precise adaptation to network conditions, allowing for smoother quality transitions and better bandwidth utilization compared to traditional discrete-level approaches.

\mysubsection{MPC-based Quality Optimization}
We formulate quality adaptation using Model Predictive Control (MPC)~\cite{yinControlTheoreticApproachDynamic2015}, which optimizes streaming quality over a finite horizon of k future frames. We borrow the QoE formulation from Yuzu~\cite{zhang_yuzu_nodate} since it provides an SR-targeting definition validated by real user study. The optimization objective balances three key components: visual quality, quality variation, and stall:
\begin{equation}
\max_{r_t,...,r_{t+k}} \sum_{i=t}^{t+k} (\alpha Q(r_i) - \beta V(r_i, r_{i-1}) - \gamma S(r_i))
\end{equation}
The quality term $Q(r)$ denotes the post-SR point density viewed by the user. The variation penalty $V(r_i, r_{i-1})$ prevents rapid quality fluctuations by penalizing changes between consecutive frames, with higher weights for quality drops that are more noticeable to viewers. The stall term $S(r_i)$ ensures smooth playback by maintaining sufficient buffer levels above a minimum threshold.

The MPC solver takes network throughput estimates (computed via harmonic mean over sliding windows) and current buffer levels as input, outputting the optimal \{to-be-fetched point density, SR ratio\} pair by solving a simple constrained optimization problem.

\mysubsection{Random Downsampling}

Given a target ratio $r$ from the MPC solver, we employ random point selection for downsampling with a simple selection probability $P_{select}(p_i) = r$ for each point $p_i \in \mathcal{P}$. 
%While  farthest point sampling (FPS)~\cite{liAdjustableFarthestPoint2022} offers better structure preservation, its $O(NC)$ complexity makes it prohibitively expensive for real-time streaming scenarios, where $N$ is the number of points in the point cloud and $C$ is the number of sampled points. 
Similarly, we choose random sampling approach over FPS stated in \S~\ref{sec:inter} to avoid the high computation cost. Combined with our robust upsampling pipeline, simple random downsampling provides sufficient quality while meeting the strict latency requirements of VoD streaming.

% \subsection{Density-Aware Downsampling}
% Given a target ratio r from the MPC solver, we perform downsampling through density-aware point selection. For each point $p_i$, we compute its sampling probability based on local density:

% \begin{equation}
% P(p_i) = \min(1, \frac{r \cdot \rho(p_i)}{\eta \sum_{j} \rho(p_j)})
% \end{equation}

% Here, $\rho(p_i)$ represents local point density estimated through GPU-accelerated k-NN search, considering the k=8 nearest neighbors. The normalization factor $\eta$ ensures achieving the target point count while preserving geometric structure. This density-aware approach naturally maintains higher point density in regions of geometric complexity, such as sharp features or detailed surfaces.

% The combination of MPC-based optimization and density-aware downsampling enables our system to perform fine-grained quality adaptation. The continuous ratio adjustment provides more flexible resource utilization compared to traditional discrete quality levels, while the density-aware sampling ensures efficient yet high-quality point cloud transmission.

\mysection{Implementation}
\label{sec:impl}

We integrate all the components described in \S\ref{sec:SR} into  \name. Our implementation consists of 8.1K lines of code (LoC) in total, with 2.8K LoC for the c++ version client and 3.4K LoC for the cuda version client.

For offline training of the point cloud super-resolution models, we use PyTorch 3.7.11~\cite{paszke2017automatic} as the deep learning framework. We modify the source code of  GradPU~\cite{he_grad-pu_2023}, to incorporate our proposed interpolation with dilation and multi-LUT fusion techniques.

Our Look Up Table is generated using c++ code and stored as an npy file which is language- and platform- neutral, faciliating for future use. 
The c++ verison client pipelined is optimized for performance by leveraging multi-threading and system pipelining.  The CUDA client features parallel kNN search, interpolation, and colorization kernels based on cuKDTree~\cite{cudaKDTree}, along with efficient LUT lookup.
The server is also implemented in C++ for efficient processing and serving of volumetric video content. We develop a custom DASH-like protocol over TCP for client-server communication.



\mysection{Evaluation}
\label{sec:eval}
% \mysubsection{Evaluation Setup}
% \textbf{Volumetric Videos.} We use four point-cloud-based volumetric videos throughout our evaluations.
% \begin{itemize}
%     \item \textbf{The Long Dress (Dress) and Loot videos.} They have 300 frames (10 sec) each consisting of $\sim$100K points. We loop them (with cold caches) 10 times in our evaluations due to the short lasting time.
%     \item \textbf{The Haggle video.} It has 7,800 frames (4’20”) each consisting of $\sim$100K points.
%     \item \textbf{The Lab video} has 3,622 frames (2 min) each consisting of $\sim$100K points.
% \end{itemize}

% \textbf{PCSR Model and LUT tables.} We choose GradPU~\cite{he_grad-pu_2023} as our reference model for LUT construcution. Two models with different hyperparameters are trained with the Long Dress video. Then we transfer the models to the LUT tables as mentioned in Section ~\ref{sec:lut}. Each LUT table is around 1.5GB.

% \textbf{Metrics and Roadmap.} We thoroughly evaluate \name in terms of SR quality, SR runtime, and QoE. \S~\ref{sec:eval-inter} and \S~\ref{sec:eval-srqual} evaluates the quality improvement brought by our 3D SR optimizations using both subjective (i.e., real-user ratings) and objective (e.g., PSNR) metrics. \S~\ref{sec:eval-runtime} focuses on the runtime performance of our SR approach, from the perspectives of resource usage, and inference time. \S~\ref{sec:eval-e2e}  evaluate the end-to-end performance (e.g., QoE and data usage) of \name. 

% \textbf{Devices.} We use a commodity machine with an Intel Xeon Gold 6230 CPU @ 2.10GHz and 32GB memory as the \name server. We use two client hosts: a desktop with an Intel Core i9-10900X CPU @ 3.70GHz, an NVIDIA GeForce RTX 3080Ti GPU, and 32GB memory (the default client used in our evaluations); an Orange Pi embedded system board with a Rockchip RK3588S 8-core 64-bit processor @ 2.4GHz and 8GB memory, which is comparable to the most accessible VR headsets on the market, Meta Quest 3~\cite{metaQuestComparison2024} with Qualcomm XR2 chip ~\cite{qualcommXR22024}.

% \textbf{User traces} We use a multi-users’ 6DoF motion traces when watching videos, and replay them in some experiments.

\mysubsection{Evaluation Setup}
\label{sec:eval-setup}
\textbf{Volumetric Videos.} We use four point-cloud-based volumetric videos in our evaluations:

\BULLET \textbf{The Long Dress (Dress) and Loot Videos:} Each has 300 frames lasting 10 seconds and containing approximately 100K points. We loop these videos ten times in our evaluations due to their short duration.

\BULLET \textbf{The Haggle Video:} Comprises 7,800 frames (4.3 minutes) each containing approximately 100K points.

\BULLET\textbf{The Lab Video:} Features 3,622 frames (2 minutes) each with approximately 100K points.


\textbf{Model Training and LUT Generation.} We use GradPU~\cite{he_grad-pu_2023} as our reference model, training it exclusively on the Long Dress video. The training process involves downsampling the original frames to different densities and using pairs of low/high-resolution point clouds as training data. The trained model is then transformed into a single LUT table with $RF=4$ and $bin=128$ (approximately 1.5GB) following the process described in \S~\ref{sec:lut}. We apply this LUT for super-resolution across all test videos to evaluate its generalization capability.

\textbf{Evaluation Metrics.} We assess \name's performance using both geometric and perceptual metrics:
Point-to-point (P2P) Chamfer Distance (CD)~\cite{wuDensityawareChamferDistance2021,li_pu-gan_2019} measure geometric accuracy between upsampled and ground truth point clouds. Peak Signal-to-Noise Ratio (PSNR) evaluates the visual quality of upsampled points. These metrics are computed per-frame and averaged over all frames. Runtime performance is measured through CPU/GPU memory utilization, frame processing latency and frame per second (FPS). For streaming evaluation, we assess Quality of Experience (QoE) discussed in \S~\ref{sec:ABR} and data usage during transmission. \S~\ref{sec:eval-inter} and \S~\ref{sec:eval-srqual} present quality results, \S~\ref{sec:eval-runtime} examines computational efficiency, while \S~\ref{sec:eval-qoe} \S~\ref{sec:eval-e2e} analyzes end-to-end streaming performance.

\textbf{Network traces} We consider the following network conditions that are representative of today's wired and wireless networks. (1) Wired network with stable bandwidth (\eg, 50, 75, and 100 Mbps) and a round-trip time (RTT) of approximately 10ms. (2) Fluctuating bandwidth captured from real-world LTE networks, with average bandwidths varying from 32.5 to 176.5 Mbps and standard deviations ranging from 13.5 to 26.8 Mbps. Among these traces, we include a LTE trace with an average throughput of 32.5 Mbps to represent lower-bandwidth wireless network scenarios. 

\textbf{Devices.} Our server setup includes a commodity machine with an Intel Xeon Gold 6230 CPU @ 2.10GHz and 32GB of RAM. We utilize two client hosts: (1) a desktop with an Intel Core i9-10900X CPU @ 3.70GHz, an NVIDIA GeForce RTX 3080Ti GPU, and 32GB of RAM, serving as our standard evaluation client; (2) an Orange Pi embedded system equipped with a Rockchip RK3588S 8-core 64-bit processor @ 2.4GHz and 8GB of RAM, comparable to the Meta Quest 3~\cite{metaQuestComparison2024} with a Qualcomm XR2 chip~\cite{qualcommXR22024}.

\textbf{User Traces.} We employ multi-user 6DoF motion traces during video playback, replicating user movements in some experiments.

\textbf{Baselines for SR Quality Evaluation:}
For evaluating the quality of super-resolution (SR) techniques, we employ two primary baselines: GradPU~\cite{he_grad-pu_2023}, and a naive interpolation method with a dilation factor of 1. GradPU serves not only as a baseline for assessing SR quality but also as a benchmark for runtime performance. Additionally, we implemented Yuzu~\cite{zhang_yuzu_nodate} with its SR pipeline, for runtime comparisons and end-to-end evaluations. To ensure a fair comparison, we disable Yuzu’s cache and delta-coding mechanisms, as these features are orthogonal to our SR approach.




% \mysubsection{Interpolation improvement}
% \label{sec:eval-inter}
% We first examine how dilation scale improves kNN-based interpolation. 


% % \todo{todo}
% % How dilation improve the interpolation results and the effectiveness of the fine-tuning model?


% % Graph for closer demonstration of the point cloud uniformity improvement.

% % Chart comparing the dilation rate vs the interpolation effectiveness(in terms of CD/HD)

% \subsection{SR Quality and effectiveness of MoE structure}
% \label{sec:eval-srqual}


% How the LUT-based SR behaves compared to the traditional SR models and the groundtruth?

% Graph: approach comparison in different dataset in terms of CD. 

% How the hyper-parameters influence the SR quality?


% How Mixture of Expert design boosts the SR quality?
\mysubsection{SR Quality}


\begin{figure*}[t]
    \small
    \centering
    \begin{minipage}{.48\textwidth}
    \centering
    \includegraphics[width=1\textwidth]{figures/srquality-psnrx2.png}
    \vspace{-.3in}
    \caption{PSNR for $\times2$ SR}
    \label{fig:srquality-psnrx2}    
    \vspace{-.1in}
    \end{minipage}
    \hfill
    \begin{minipage}{.48\textwidth}
    \centering
    \includegraphics[width=1\textwidth]{figures/srquality-cdx2.png}
    \vspace{-.3in}
    \caption{ Chamfer Distance for $\times2$ SR}
    \label{fig:srquality-cdx2}  
    \vspace{-.1in}
    \end{minipage}
\end{figure*}

\begin{figure*}[t]
    \small
    \centering
    \begin{minipage}{.48\textwidth}
    \centering
    \includegraphics[width=1\textwidth]{figures/srquality-psnrx4.png}
    \vspace{-.3in}
    \caption{PSNR for $\times4$ SR}
    \label{fig:srquality-psnrx4}    
    \vspace{-.1in}
    \end{minipage}
    \hfill
    \begin{minipage}{.48\textwidth}
    \centering
    \includegraphics[width=1\textwidth]{figures/srquality-cdx4.png}
    \vspace{-.3in}
    \caption{Chamfer Distance for $\times4$ SR}
    \label{fig:srquality-cdx4}  
    \vspace{-.1in}
    \end{minipage}
\end{figure*}

% \begin{minipage}{.48\textwidth}
%     \centering
%     \begin{minipage}{.48\linewidth}
%         \centering
%         \includegraphics[width=1\textwidth]{figures/inter-gpu.png}
%         \vspace{-.2in}
%         \captionof{figure}{Interpolation FPS on commodity GPU}
%         \label{fig:eval-nq}    
%     \end{minipage}
%     \hfill
%     \begin{minipage}{.48\linewidth}
%         \centering
%         \includegraphics[width=1\textwidth]{figures/inter-pi.png}
%         \vspace{-.2in}
%         \captionof{figure}{Interpolation FPS on Orange Pi}
%         \label{fig:eval-du}  
%     \end{minipage}
%     \vspace{-.1in}
% \end{minipage}
\begin{figure*}[t]
    \begin{minipage}{0.7\textwidth}
        % \begin{minipage}{\textwidth}
        %     \begin{minipage}{0.49\textwidth}
        %         \centering
        %         \includegraphics[width=\textwidth]{figures/inter-gpu.png}
        %         \vspace{-.3in}
        %         \caption{Interpolation FPS on commodity GPU}
        %         \label{fig:eval-gpu}    
        %     \end{minipage}
        %     \hfill
        %     \begin{minipage}{0.49\textwidth}
        %         \centering
        %         \includegraphics[width=\textwidth]{figures/inter-pi.png}
        %         \vspace{-.3in}
        %         \caption{Interpolation FPS on Orange Pi}
        %         \label{fig:eval-pi}  
        %     \end{minipage}
            \centering
            \includegraphics[width=\textwidth]{figures/inter.png}
            \vspace{-.3in}
            \caption{Interpolation FPS on Orange Pi (Left) and NVDIA 3080Ti (Right)}
            \label{fig:eval-inter}
        % \end{minipage}
        % \vspace{.2in}
        
        \begin{minipage}{\textwidth}
            \begin{minipage}{0.49\textwidth}
                \centering
                \includegraphics[width=\textwidth]{figures/e2e-nq.png}
                \vspace{-.3in}
                \caption{Normalized QoE}
                \label{fig:eval-qoe}    
            \end{minipage}
            \hfill
            \begin{minipage}{0.49\textwidth}
                \centering
                \includegraphics[width=\textwidth]{figures/e2e-du.png}
                \vspace{-.3in}
                \caption{Data usage}
                \label{fig:eval-du}  
            \end{minipage}
        \end{minipage}
    \end{minipage}
    \hfill
    \begin{minipage}{0.27\textwidth}
        \centering
        \includegraphics[width=\textwidth]{figures/e2e_scatter.png}
        \vspace{-.15in}
        \caption{QoE vs. Data usage over fluctuating bandwidth (LTE traces)}
        \label{fig:scatter}
        % \vspace{-.2in}
        
        \small
        \begin{tabular}{c|l}
            \textit{H1} & VoLUT with Continuous ABR \\
            \hline
            \textit{H2} & VoLUT with Discrete ABR \\
            \hline
            \textit{H3} & VoLUT with Discrete ABR\\
            & and Yuzu SR
        \end{tabular}
        % \vspace{.05in}
        \captionof{table}{Variants of VoLUT}
        \label{tab:variants}
    \end{minipage}
\end{figure*}
To rigorously evaluate the impact of super-resolution (SR) techniques on image and geometric quality, we conduct a comprehensive set of experiments. These experiments involve multiple users watching videos while their 6 Degrees of Freedom (6DoF) motion traces are recorded. For each SR setting, we render viewports as images, denoted as \{ISR\}, and repeat this process with the original videos to capture the baseline images, denoted as \{Igt\}. We then compute the Peak Signal-to-Noise Ratio (PSNR) by comparing each image in \{ISR\} against its corresponding image in \{Igt\}. Additionally, we measure the Chamfer Distance to evaluate the geometric accuracy by comparing the SR-enhanced point clouds with the corresponding ground truth point clouds. The corresponding videos are downsampled to 100K and upsampled to $\times2$ and $\times4$.

\mysubsubsection{Interpolation with dilation}
\label{sec:eval-inter}

In this analysis, we explore the effect of varying the dilation factor ("d") within the kNN interpolation process used in SR. Specifically, we assess the PSNR outcomes for both $\times2$ and $\times4$ super-resolution settings, presented in Figures~\ref{fig:srquality-psnrx2} and \ref{fig:srquality-psnrx4}. The results indicate a clear improvement in PSNR values when dilation is increased from $K4d1$ to $K4d2$, suggesting better image quality across different upsampling ratios. Concurrently, the Chamfer Distance results, shown in Figures~\ref{fig:srquality-cdx2} and \ref{fig:srquality-cdx4}, reveal a reduction in geometric discrepancies as dilation is incorporated. These findings illustrate that enhanced dilation provides a broader spatial context during interpolation which not only improves visual clarity but also significantly enhances the geometric accuracy of the super-resolved images.

\mysubsubsection{LUT refinement}
\label{sec:eval-srqual}

The LUT refinement process targets the optimization of interpolated point cloud data by looking up the precomputed offsets stored in the Look-Up Table. This step is crucial for enhancing the final SR quality. $K4d2-lut$ represents our approach using network generated LUT.  By analyzing both PSNR and Chamfer Distance metrics post-refinement, as depicted in Figures~\ref{fig:srquality-psnrx2}, \ref{fig:srquality-psnrx4}, \ref{fig:srquality-cdx2}, and \ref{fig:srquality-cdx4}, we observe noticeable improvements in image fidelity and geometric accuracy. By comparing the GradPU and our lut results, we show that the integration of our interpolation with dilation adjustments and subsequent LUT refinement ensures that the accelerated SR process does not compromise on visual or geometric quality. 

Overall, our experimental analysis demonstrates that the applied SR techniques not only preserve but significantly enhance both the visual and geometric qualities of the images. Notably, achieving consistent PSNR values over 30 dB across various settings underscores the excellent visual quality of our SR process~\cite{thomos2005optimized,dasari2020streaming}.



\mysubsection{Runtime Performance}
\label{sec:eval-runtime}


% \begin{figure}[t]
%     \small
%     \includegraphics[width=0.5\textwidth]{figures/mem.png} % Second image
%         \caption{\small GPU memory usage.}
%         \label{fig:lut1}
% \end{figure}

% \begin{figure*}[t]
%     \small
%     \centering
%     \begin{minipage}{.48\textwidth}
%     \centering
%     \includegraphics[width=1\textwidth]{figures/mem.png} % Second image
%         \caption{GPU memory usage (3080Ti Desktop).}
%         \label{fig:gpu-mem}   
%     \vspace{-.1in}
%     \end{minipage}
%     \hfill
%     \begin{minipage}{.48\textwidth}
%     \centering
%     \includegraphics[width=1\textwidth]{figures/e2e-breakdown.png}
%     \vspace{-.3in}
%     \caption{End to End SR Runtime breakdown}
%     \label{fig:breakdown}  
%     \vspace{-.1in}
%     \end{minipage}
% \end{figure*}

% \begin{figure*}[t]
%     \small
%     \centering
%     \begin{minipage}{.48\textwidth}
%     \centering
%     \includegraphics[width=1\textwidth]{figures/GPU_runtime.png}
%     \vspace{-.3in}
%     \caption{SR Runtime on commodity GPU}
%     \label{fig:gpu}    
%     \vspace{-.1in}
%     \end{minipage}
%     \hfill
%     \begin{minipage}{.48\textwidth}
%     \centering
%     \includegraphics[width=1\textwidth]{figures/pi-runtime.png}
%     \vspace{-.3in}
%     \caption{SR Runtime on OrangePi under various upsampling ratio}
%     \label{fig:pi-runtime}  
%     \vspace{-.1in}
%     \end{minipage}
% \end{figure*}

\begin{figure*}[t]
    \small
    \centering
    \begin{minipage}{.24\textwidth}
    \centering
    \includegraphics[width=1\textwidth]{figures/mem.png}
    \caption{GPU memory usage (3080Ti Desktop).}
    \label{fig:gpu-mem}   
    \vspace{-.1in}
    \end{minipage}
    \hfill
    \begin{minipage}{.24\textwidth}
    \centering
    \includegraphics[width=1\textwidth]{figures/e2e-breakdown.png}
    \vspace{-.3in}
    \caption{End to End SR Runtime breakdown}
    \label{fig:breakdown}  
    \vspace{-.1in}
    \end{minipage}
    \hfill
    \begin{minipage}{.24\textwidth}
    \centering
    \includegraphics[width=1\textwidth]{figures/GPU_runtime.png}
    \vspace{-.3in}
    \caption{SR Runtime on commodity GPU}
    \label{fig:gpu}    
    \vspace{-.1in}
    \end{minipage}
    \hfill
    \begin{minipage}{.24\textwidth}
    \centering
    \includegraphics[width=1\textwidth]{figures/pi-runtime.png}
    \vspace{-.3in}
    \caption{SR Runtime on OrangePi under various upsampling ratio}
    \label{fig:pi-runtime}  
    \vspace{-.1in}
    \end{minipage}
\end{figure*}


We now focus on analyzing how different parts of our SR system contribute to the overall latency in both desktop and mobile settings. The experiments are conducted using 100Mbps wired netork with our continuous ABR disabled.

\textbf{Interpolation Speedup:} As shown in Figures~\ref{fig:eval-inter}, our optimized interpolation achieves significant speedups across different platforms. On the Orange Pi, we maintain 3.7$\times$-3.9$\times$ speedup over vanilla interpolation, reaching 31.2 FPS at 8$\times$ upsampling (vs vanilla's 8.0 FPS). The improvement is even more substantial on the commordity GPU empowered by the cuda implementation, where we achieve 7.5$\times$-8.1$\times$ speedup, processing at 357.1 FPS for 2$\times$ upsampling and maintaining 138.9 FPS even at 8$\times$ upsampling. This consistent performance across upsampling ratios demonstrates the effectiveness of our spatial-parallel optimization in reducing the kNN search overhead.


\textbf{GPU Memory Usage:} As shown in Figure~\ref{fig:gpu-mem},  Our approach using one LUT can improve the GPU memory usage by 86\% compared to GradPU and is comparable to Yuzu with frozen tensorflow model in c++, which is particularly beneficial for devices with limited GPU resources.

\textbf{Runtime Breakdown:} Figure~\ref{fig:breakdown} provides a detailed breakdown of the time spent in each stage of the SR process on both desktop and Orange Pi platforms. On both GPU (desktop) and mobile scenario, kNN search takes the most significant portion of time, followed by interpolation, with LUT refinement consuming the least time.

\textbf{SR Performance on Different Platforms:} Figure~\ref{fig:gpu} and Figure~\ref{fig:pi-runtime} further illustrate the SR runtime on a commodity GPU and the impact of various upsampling ratios on the Orange Pi, respectively. We show the average upsampling rate(in FPS). On the desktop (Figure~\ref{fig:gpu}), our method outperforming Yuzu by $8.4\times$ and outperforming GradPU by $46400\times$. Our approach is mainly benefited by the efficient LUT look up compared to heavy neural network inferencing even if accelerated in a frozen cpp implementation as Yuzu~\cite{zhang_yuzu_nodate} did.  As shown in Figure~\ref{fig:pi-runtime}, the upsampling speed on the Orange Pi maintains relatively stable even as the upsampling ratio increases. This is due to the fact that the main bottleneck (shown in Figure~\ref{fig:breakdown}) of our SR approach lies in the kNN based interpolation which is mainly related to the number of input points.

\mysubsection{QoE measurement under various network conditions}
\label{sec:eval-qoe}

We evaluate the QoE of our system under different network conditions, using various videos and associated motion traces. We compare our approach with Yuzu-SR, a re-implementation of Yuzu~\cite{zhang_yuzu_nodate} with caching and delta coding disabled for fair comparison. The normalized QoE results are shown in Figure~\ref{fig:eval-qoe} and the data usage (defined by the total downloaded bytes including SR models for yuzu SR and meta data) results are shown in Figure~\ref{fig:eval-du}.

\textbf{Stable Bandwidth.} We first consider a stable bandwidth scenario with a throughput of 50Mbps. Under this condition, our system achieves a normalized QoE of 100, while Yuzu-SR, on the other hand, achieves a normalized QoE of 75.8. \name is mainly benefited from the fast SR speed for any-scale upsampling compared to Yuzu-SR. 

In terms of data usage, \name can reduce it by 23\% compared to Yuzu-SR because our fine-grained ABR algorithms allows any-scale downsampling rate for transmission while Yuzu-SR's discrete SR options (1x2, 2x2, 1x3, 1x4, 4x1, 2x1) provide less optimal decision.

\textbf{Fluctuating Bandwidth.} We also evaluate the performance of our system under fluctuating bandwidth conditions using real-world LTE traces(\S~\ref{sec:eval-setup}). In this scenario, our system achieves a normalized QoE of 83 while consuming only 17\% of the data. In comparison, Yuzu-SR achieves a normalized QoE of 57 but requires 31\% of the data. Notably, QoE performance gain is higer under LTE (26) traces compared to the stable trace (24) is due to the the limited bandwidth, which pushes the systems to fetch content with lower density and introduces more SR workload. Thus \name's fast SR will benefit more under limited network resources.


% \begin{figure*}[t]
%     \small
%     \centering
%     \begin{minipage}{.48\textwidth}
%     \centering
%     \includegraphics[width=1\textwidth]{figures/e2e-nq.png}
%     \vspace{-.3in}
%     \caption{Normalized QoE}
%     \label{fig:eval-nq}    
%     \vspace{-.1in}
%     \end{minipage}
%     \hfill
%     \begin{minipage}{.48\textwidth}
%     \centering
%     \includegraphics[width=1\textwidth]{figures/e2e-du.png}
%     \vspace{-.3in}
%     \caption{Data usage}
%     \label{fig:eval-du}  
%     \vspace{-.1in}
%     \end{minipage}
% \end{figure*}

\mysubsection{Ablation study}
\label{sec:eval-e2e}
How the system is compared to Yuzu and simple Adaptation in terms of FPS and resource consumption?
% \begin{table}[t]
%     \centering
%     \begin{minipage}{0.4\columnwidth}
%         \centering
%         \includegraphics[width=1\textwidth]{figures/e2e_scatter.png}
%         % \captionsetup{size=small}
%        \vspace{-.33in}
%         \captionof{figure}{QoE vs. Data usage over fluctuating bandwidth (LTE traces)}
%         \label{fig:ablation}
%     \end{minipage}
%     \hfill
%     \begin{minipage}{.55\textwidth}
%         \centering
%         \begin{tabular}{c|l}
%             \textit{H1} & \name with Continuous ABR \\
%             \hline
%             \textit{H2} & \name with Discrete ABR \\
%             \hline
%             \textit{H3} & \name with Discrete ABR and Yuzu SR \\
%         \end{tabular}
%         \vspace{-.1in}
%         % \captionsetup{size=small}
%         \caption{Variants of \name }
%         \label{tab:ablation}
%     \end{minipage}
% \end{table}


% \begin{minipage}{.48\textwidth}
%     \centering
%     \begin{minipage}{.48\linewidth}
%         \centering
%         \includegraphics[width=1\textwidth]{figures/e2e-nq.png}
%         \vspace{-.2in}
%         \captionof{figure}{Normalized QoE}
%         \label{fig:eval-nq}    
%     \end{minipage}
%     \hfill
%     \begin{minipage}{.48\linewidth}
%         \centering
%         \includegraphics[width=1\textwidth]{figures/e2e-du.png}
%         \vspace{-.2in}
%         \captionof{figure}{Data usage}
%         \label{fig:eval-du}  
%     \end{minipage}
%     \vspace{-.1in}
% \end{minipage}


Our ablation study evaluate three variants of our system (In Table~\ref{tab:variants}). Figure~\ref{fig:scatter} shows the normalized QoE vs. data usage trade-off for these system variants under fluctuating bandwidth conditions.

Our proposed system (H1) achieves the best balance between QoE and data usage. It maintains a high normalized QoE of 98 while consuming only 31\% of the data compared to the baseline.
% This demonstrates the effectiveness of our continuous ABR approach in adapting to network conditions and our faster SR method in reducing data usage without compromising quality.
Using discrete ABR (H2) instead of continuous ABR leads a reduction of normalized QoE by 15.3\% and increases the data usage by 14\% compared to H1. This highlights the advantage of our continuous ABR approach in fine-grained bitrate adaptation, which allows better utilization of available bandwidth and reduces data consumption.

Replacing our faster SR method with Yuzu's SR (H3) results in a notable drop in QoE by 36.7\% compared to H1 while still consuming 48\% of the data. This emphasizes the faster SR speed will also benefit the stall time which is a major components of the QoE(\S~\ref{sec:ABR}).









% \section{Discussion}
% \label{sec:discussion}
% 
\textbf{Live streaming:} One promising direction is to explore the applicability of \name to live volumetric video streaming. Since \name requires minimal processing of point clouds at the server side, it has the potential to support low-latency live streaming scenarios. By leveraging the efficiency of our LUT-based super-resolution approach and the real-time performance of our optimized pipeline, \name could enable interactive and immersive live experiences, such as virtual concerts, sports events, or telepresence applications. However, live streaming introduces additional challenges, such as real-time point cloud acquisition, compression, and synchronization, which need to be carefully addressed.
\textbf{LUT for inter-frame compression:} Another interesting avenue for future research is to explore the use of LUTs for inter-frame compression in volumetric video streaming. Volumetric videos often exhibit temporal redundancy between consecutive frames, similar to traditional 2D videos. By exploiting this temporal correlation, we can potentially train an point-wise motion prediction network like pointFlowNet~\cite{behl2019pointflownet} for each video and make employs an LUT to record the referenced network for

\mysection{Related Work}
\label{sec:related}

\textbf{Volumetric video streaming:}
Volumetric video streaming has gained significant attention in recent years due to its ability to provide immersive and interactive experiences. Several studies have focused on point-cloud-based volumetric video streaming \cite{lee_groot_2020,han_vivo_2020, gul2020low, gul2020cloud, hosseini2018dynamic, park2018volumetric, qian2019toward, van2019towards}. DASH-PC \cite{hosseini2018dynamic} extends the Dynamic Adaptive Streaming over HTTP (DASH) protocol to support volumetric videos. ViVo \cite{han_vivo_2020} introduces visibility-aware optimizations to improve the streaming efficiency of volumetric videos. GROOT \cite{lee_groot_2020} focuses on optimizing point cloud compression for volumetric video streaming.MuV2 \cite{liuMuV2ScalingMultiuser2024} and Vues \cite{liu_vues_2022} applies transcoding to 3D contents at server and transmit 2D frames to clients.  YuZu \cite{zhang_yuzu_nodate} is a recently proposed volumetric video streaming system that employs deep learning-based point cloud super-resolution to enhance the visual quality of low-resolution content at the receiver's end. However, these existing works do not explore the potential of interpolation and lut-based approaches for efficient point cloud super-resolution in volumetric video streaming.

\textbf{Look-up table (LUT) based inference speed-up:}
Look-up tables have been widely used in various domains to accelerate computation and reduce memory footprint. In the context of image processing, several works have explored LUT-based approaches. Jo \etal~\cite{jo_practical_2021} and Liu\etal~\cite{liu4DLUTLearnable2022} propose LUT-based method for efficient single-image super-resolution. LUT-NN~\cite{tang_lut-nn_2023} introduces a LUT-based neural network inference framework that achieves significant speedup and memory reduction compared to traditional neural network inference. DLUX~\cite{gu2020dlux} presents a LUT-based near-bank accelerator for efficient deep learning training in data centers. Sutradhar \etal ~\cite{sutradhar2021look} explores the use of LUTs in processing-in-memory architectures for deep learning workloads. These works demonstrate the effectiveness of LUT-based approaches in various domains. However, to the best of our knowledge, no prior work has investigated the application of LUTs for point cloud super-resolution in volumetric video streaming.

% \textbf{Point cloud super-resolution:}
% Point cloud super-resolution aims to increase the resolution and quality of point cloud data. Several deep learning-based approaches have been proposed for point cloud super-resolution. PU-Net \cite{yu_pu-net_2018} introduces a point-wise convolution network for upsampling point clouds. 3PU \cite{yifan2019patch} proposes a progressive upsampling network that incrementally increases the resolution of point clouds. PU-GAN \cite{li_pu-gan_2019} employs a generative adversarial network (GAN) to generate high-resolution point clouds from low-resolution inputs.There also exists optimzation based approach~\cite{alexa2003computing, huang2013edge}. While these methods demonstrate promising results, they often rely on computationally expensive neural networks and are not optimized for real-time volumetric video streaming. In contrast, \name addresses several challenges to make point cloud super-resolution approach realtime and accessible for non-highend devices.



\mysection{Conclusion}
\label{sec:conclusion}
In this paper, we present \name, a novel system that leverages LUT-based point cloud super-resolution for efficient and high-quality volumetric video streaming. Through combining accelerated dilated interpolation and Look Up table based refinement, \name achieves real-time performance even on mobile devices, significantly reduces bandwidth requirements, and enhances the user experience. Our extensive evaluations demonstrate the effectiveness of \name in delivering high-quality volumetric video content while adapting to network conditions and user preferences. The contributions of our work lay the foundation for future research and development in the field of volumetric video streaming, opening up new possibilities for immersive and accessible volumetric experiences.

% this must go after the closing bracket ] following \twocolumn[ ...

% This command actually creates the footnote in the first column
% listing the affiliations and the copyright notice.
% The command takes one argument, which is text to display at the start of the footnote.
% The \mlsysEqualContribution command is standard text for equal contribution.
% Remove it (just {}) if you do not need this facility.




% In the unusual situation where you want a paper to appear in the
% references without citing it in the main text, use \nocite
\nocite{langley00}

% \bibliography{example_paper}

\bibliographystyle{mlsys2025}
\bibliography{reference}

%%%%%%%%%%%%%%%%%%%%%%%%%%%%%%%%%%%%%%%%%%%%%%%%%%%%%%%%%%%%%%%%%%%%%%%%%%%%%%%
%%%%%%%%%%%%%%%%%%%%%%%%%%%%%%%%%%%%%%%%%%%%%%%%%%%%%%%%%%%%%%%%%%%%%%%%%%%%%%%
% SUPPLEMENTAL CONTENT AS APPENDIX AFTER REFERENCES
%%%%%%%%%%%%%%%%%%%%%%%%%%%%%%%%%%%%%%%%%%%%%%%%%%%%%%%%%%%%%%%%%%%%%%%%%%%%%%%
%%%%%%%%%%%%%%%%%%%%%%%%%%%%%%%%%%%%%%%%%%%%%%%%%%%%%%%%%%%%%%%%%%%%%%%%%%%%%%%
% \appendix
% \section{Please add supplemental material as appendix here}
% %
% Put anything that you might normally include after the references as an appendix here, {\it not in a separate supplementary file}. Upload your final camera-ready as a single pdf, including all appendices.

%%%%%%%%%%%%%%%%%%%%%%%%%%%%%%%%%%%%%%%%%%%%%%%%%%%%%%%%%%%%%%%%%%%%%%%%%%%%%%%
%%%%%%%%%%%%%%%%%%%%%%%%%%%%%%%%%%%%%%%%%%%%%%%%%%%%%%%%%%%%%%%%%%%%%%%%%%%%%%%


\end{document}


% This document was modified from the file originally made available by
% Pat Langley and Andrea Danyluk for ICML-2K. This version was created
% by Iain Murray in 2018. It was modified from a version from Dan Roy in
% 2017, which was based on a version from Lise Getoor and Tobias
% Scheffer, which was slightly modified from the 2010 version by
% Thorsten Joachims & Johannes Fuernkranz, slightly modified from the
% 2009 version by Kiri Wagstaff and Sam Roweis's 2008 version, which is
% slightly modified from Prasad Tadepalli's 2007 version which is a
% lightly changed version of the previous year's version by Andrew
% Moore, which was in turn edited from those of Kristian Kersting and
% Codrina Lauth. Alex Smola contributed to the algorithmic style files.
