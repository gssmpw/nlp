
We integrate all the components described in \S\ref{sec:SR} into  \name. Our implementation consists of 8.1K lines of code (LoC) in total, with 2.8K LoC for the c++ version client and 3.4K LoC for the cuda version client.

For offline training of the point cloud super-resolution models, we use PyTorch 3.7.11~\cite{paszke2017automatic} as the deep learning framework. We modify the source code of  GradPU~\cite{he_grad-pu_2023}, to incorporate our proposed interpolation with dilation and multi-LUT fusion techniques.

Our Look Up Table is generated using c++ code and stored as an npy file which is language- and platform- neutral, faciliating for future use. 
The c++ verison client pipelined is optimized for performance by leveraging multi-threading and system pipelining.  The CUDA client features parallel kNN search, interpolation, and colorization kernels based on cuKDTree~\cite{cudaKDTree}, along with efficient LUT lookup.
The server is also implemented in C++ for efficient processing and serving of volumetric video content. We develop a custom DASH-like protocol over TCP for client-server communication.

