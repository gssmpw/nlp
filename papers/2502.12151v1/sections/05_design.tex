

Existing volumetric video streaming systems~\cite{han_vivo_2020,zhang_yuzu_nodate,lee_groot_2020} are constrained by discrete quality levels, typically offering fixed point densities (e.g., 100K, 200K points per frame). We introduce a continuous adaptive bitrate (ABR) mechanism that dynamically optimizes streaming quality through fine-grained point density adjustments. This approach is made possible by our upsampling algorithm's consistent latency across varying upsampling ratios (detailed in \S~\ref{sec:eval-runtime}). The ability to support arbitrary downsampling ratios through our super-resolution pipeline enables more precise adaptation to network conditions, allowing for smoother quality transitions and better bandwidth utilization compared to traditional discrete-level approaches.

\mysubsection{MPC-based Quality Optimization}
We formulate quality adaptation using Model Predictive Control (MPC)~\cite{yinControlTheoreticApproachDynamic2015}, which optimizes streaming quality over a finite horizon of k future frames. We borrow the QoE formulation from Yuzu~\cite{zhang_yuzu_nodate} since it provides an SR-targeting definition validated by real user study. The optimization objective balances three key components: visual quality, quality variation, and stall:
\begin{equation}
\max_{r_t,...,r_{t+k}} \sum_{i=t}^{t+k} (\alpha Q(r_i) - \beta V(r_i, r_{i-1}) - \gamma S(r_i))
\end{equation}
The quality term $Q(r)$ denotes the post-SR point density viewed by the user. The variation penalty $V(r_i, r_{i-1})$ prevents rapid quality fluctuations by penalizing changes between consecutive frames, with higher weights for quality drops that are more noticeable to viewers. The stall term $S(r_i)$ ensures smooth playback by maintaining sufficient buffer levels above a minimum threshold.

The MPC solver takes network throughput estimates (computed via harmonic mean over sliding windows) and current buffer levels as input, outputting the optimal \{to-be-fetched point density, SR ratio\} pair by solving a simple constrained optimization problem.

\mysubsection{Random Downsampling}

Given a target ratio $r$ from the MPC solver, we employ random point selection for downsampling with a simple selection probability $P_{select}(p_i) = r$ for each point $p_i \in \mathcal{P}$. 
%While  farthest point sampling (FPS)~\cite{liAdjustableFarthestPoint2022} offers better structure preservation, its $O(NC)$ complexity makes it prohibitively expensive for real-time streaming scenarios, where $N$ is the number of points in the point cloud and $C$ is the number of sampled points. 
Similarly, we choose random sampling approach over FPS stated in \S~\ref{sec:inter} to avoid the high computation cost. Combined with our robust upsampling pipeline, simple random downsampling provides sufficient quality while meeting the strict latency requirements of VoD streaming.

% \subsection{Density-Aware Downsampling}
% Given a target ratio r from the MPC solver, we perform downsampling through density-aware point selection. For each point $p_i$, we compute its sampling probability based on local density:

% \begin{equation}
% P(p_i) = \min(1, \frac{r \cdot \rho(p_i)}{\eta \sum_{j} \rho(p_j)})
% \end{equation}

% Here, $\rho(p_i)$ represents local point density estimated through GPU-accelerated k-NN search, considering the k=8 nearest neighbors. The normalization factor $\eta$ ensures achieving the target point count while preserving geometric structure. This density-aware approach naturally maintains higher point density in regions of geometric complexity, such as sharp features or detailed surfaces.

% The combination of MPC-based optimization and density-aware downsampling enables our system to perform fine-grained quality adaptation. The continuous ratio adjustment provides more flexible resource utilization compared to traditional discrete quality levels, while the density-aware sampling ensures efficient yet high-quality point cloud transmission.