
\textbf{Volumetric video streaming:}
Volumetric video streaming has gained significant attention in recent years due to its ability to provide immersive and interactive experiences. Several studies have focused on point-cloud-based volumetric video streaming \cite{lee_groot_2020,han_vivo_2020, gul2020low, gul2020cloud, hosseini2018dynamic, park2018volumetric, qian2019toward, van2019towards}. DASH-PC \cite{hosseini2018dynamic} extends the Dynamic Adaptive Streaming over HTTP (DASH) protocol to support volumetric videos. ViVo \cite{han_vivo_2020} introduces visibility-aware optimizations to improve the streaming efficiency of volumetric videos. GROOT \cite{lee_groot_2020} focuses on optimizing point cloud compression for volumetric video streaming.MuV2 \cite{liuMuV2ScalingMultiuser2024} and Vues \cite{liu_vues_2022} applies transcoding to 3D contents at server and transmit 2D frames to clients.  YuZu \cite{zhang_yuzu_nodate} is a recently proposed volumetric video streaming system that employs deep learning-based point cloud super-resolution to enhance the visual quality of low-resolution content at the receiver's end. However, these existing works do not explore the potential of interpolation and lut-based approaches for efficient point cloud super-resolution in volumetric video streaming.

\textbf{Look-up table (LUT) based inference speed-up:}
Look-up tables have been widely used in various domains to accelerate computation and reduce memory footprint. In the context of image processing, several works have explored LUT-based approaches. Jo \etal~\cite{jo_practical_2021} and Liu\etal~\cite{liu4DLUTLearnable2022} propose LUT-based method for efficient single-image super-resolution. LUT-NN~\cite{tang_lut-nn_2023} introduces a LUT-based neural network inference framework that achieves significant speedup and memory reduction compared to traditional neural network inference. DLUX~\cite{gu2020dlux} presents a LUT-based near-bank accelerator for efficient deep learning training in data centers. Sutradhar \etal ~\cite{sutradhar2021look} explores the use of LUTs in processing-in-memory architectures for deep learning workloads. These works demonstrate the effectiveness of LUT-based approaches in various domains. However, to the best of our knowledge, no prior work has investigated the application of LUTs for point cloud super-resolution in volumetric video streaming.

% \textbf{Point cloud super-resolution:}
% Point cloud super-resolution aims to increase the resolution and quality of point cloud data. Several deep learning-based approaches have been proposed for point cloud super-resolution. PU-Net \cite{yu_pu-net_2018} introduces a point-wise convolution network for upsampling point clouds. 3PU \cite{yifan2019patch} proposes a progressive upsampling network that incrementally increases the resolution of point clouds. PU-GAN \cite{li_pu-gan_2019} employs a generative adversarial network (GAN) to generate high-resolution point clouds from low-resolution inputs.There also exists optimzation based approach~\cite{alexa2003computing, huang2013edge}. While these methods demonstrate promising results, they often rely on computationally expensive neural networks and are not optimized for real-time volumetric video streaming. In contrast, \name addresses several challenges to make point cloud super-resolution approach realtime and accessible for non-highend devices.