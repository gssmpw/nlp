Cyber-physical systems (CPSs) integrate physical and computation processes using communication networks to enable the monitoring and control of these processes \citep{allgower2019}. 
In recent years, CPSs have seen an increase in their complexity and dependence on communication networks to ensure safe operation.
However, these new advancements led to new vulnerabilities that cyber attacks can exploit in CPS, e.g., \citep{Farwell:2011,Checkoway:2011,Greenberg:2020,Easterly:2023}.
As key components of CPSs,  sensors and actuators are particularly susceptible to attacks that can compromise the integrity and safety of these systems. 
Therefore, detecting and preventing attacks on sensors and actuators is crucial to ensure the secure and safe operation of CPS.


% Our analysis is at
% the supervisory control layer; hence, we adopt a discrete event
% modeling formalism, where system operation and communications are event-based and the controller is a supervisor. In
% contrast to prior work on sensor deception attacks for Discrete
% Event Systems (DES), where logical models are used, we
% model the system as a Probabilistic Finite-State Automaton

% In this paper, we study the problem of designing better and faster \emph{intrusion detection (ID) systems} for CPSs modeled as \emph{probabilistic discrete event systems (PDES)} \citep{Lawford:1993,Garg:1999,Lafortune:2021}. 
% An ID system monitors the presence of attacks by analyzing the behavior of the control system.
% Herein, we investigate detectors for \emph{sensor deception attacks}; an attacker that hijacks a subset of sensors to damage the CPS. 

In this paper, we analyze the security of CPSs at the supervisory control layer in the hierarchical control architecture.
Hence, we use the discrete event modeling formalism of probabilistic discrete event systems (PDES), where system operation and communications are event-based with known transition probabilities and the controller is a supervisor \citep{Lawford:1993,Garg:1999,Lafortune:2021}. 
In this context, the supervisor controls the CPS via actuator commands based on the observation of events generated by sensor readings. 
Based on event-driven models, we assume that an attacker infiltrates and manipulates the sensor communication channels between the plant and the supervisor; this type of attack is known as a sensor deception attack\footnote{Sensor attacks for short.}.
We study the design of \emph{Probabilistic Intrusion Detection (ID) Systems} to detect sensor deception attacks in CPSs. 
% Within this setting, we analyze the actions of an attacker who stealthily manipulates a subset of sensor readings to guide the system toward undesirables and unsafe behaviors while avoiding detection. 
% In the literature, this class of sensor attacks is commonly referred to as \emph{stealthy (or covert)} and \emph{deception} sensor attacks \citep{Su:2018,meira-goes:2020synthesis,tong:2022,YAO2024deception}.

In the domain of discrete event systems (DES), several works focused on dealing with cyber attacks, e.g., see \citep{Rashidinejad:2019,hadjicostis2022cybersecurity, OLIVEIRA2023100907}. Among these efforts, ID systems have received significant attention in recent years, particularly for their role in identifying and mitigating such attacks \citep{Thorsley:2006,Carvalho:2018,Lima:2019,meira-goes:2020towards,wang:2022,fritz2023detection,zhang2023robust,lin2024diagnosability, li2025diagnosability, kang2025diagnosability}. 
An ID system monitors the presence of attacks by analyzing the behavior of the controlled system. 
However, the state-of-the-art ID systems in DES has mainly focused in analyzing the controlled behavior qualitatively using logical DES models, i.e., models without probabilities or other quantitative metric.
For this reason, the current ID systems in DES are ineffective against ``smarter attacks" such as stealthy/covert sensor deception attacks \citep{Su:2018,meira-goes:2020synthesis,tong:2022,YAO2024deception}. 

Since ID systems play a crucial role in CPS, there is a need to develop quantitative frameworks for the detection of sensor attacks to complement the logical approach. 
This leads to an important question: \\
\emph{How can we find a quantitative measure to better detect sensor attacks, indicating the certainty that a given behavior originates from the system under attack?}

Our previous works addressed this question by adopting a stochastic framework and proposing a probabilistic ID method based on a property termed $\epsilon$-safety \citep{meira-goes:2020towards, Fahim2024-wodes}. 
This property captures the system's ability to analyze and identify whether an \emph{deterministic} sensor attack strategy \emph{drastically} modifies the probability of the controlled system. 
In other words, $\epsilon$-safety determines if the attacker leaves a \emph{probabilistic footprint} while attacking the controlled system. 
The parameter $\epsilon$ represents the confidence level that an observed behavior is more likely to have been generated by the system under attack than by the nominal system (a system operating under normal conditions). 

In this paper, we extend this problem by exploring a generalization on $\epsilon$-safety. 
Rather than focusing on a specific deterministic sensor attack strategy, we address a wider scope of attack strategies, including \emph{all complete, consistent, and successful} sensor attack strategies.  
While the $\epsilon$-safety framework was limited to detecting a \emph{single sensor attack strategy}, our enhanced approach introduces the notion of $\lambda$-sensor-attack detectability ($\lambda$-sa, for short), which expands the detection capability to handle \emph{multiple} sensor attack strategies. 
This generalization ensures a more robust and adaptable defense mechanism against a wider range of potential threats.

Based on the notion of $\lambda$-sensor-attack detectability, we formulate two novel problems: (i) verifying $\lambda$-sa detectability, and (ii) searching for the largest $\lambda$ such that $\lambda$-sa detectability holds. 
Moreover, we present a \emph{polynomial-time} complexity algorithm to solve these two problems. 
The solution methodology is based on methods from DES and graph theory and consists of two steps. 
First, we construct a structure called the \emph{weighted verifier}, which includes information about both the nominal system and the system under attack. 
This structure is inspired by the verifier automaton used for fault diagnosis purposes, which combines information from faulty and non-faulty systems \citep{yoo2002polynomial}.
By using the weights in the verifier automaton and solving the \emph{shortest path problem} \citep{cormen2022introduction}, we find the string with the least $\lambda$ value.
Specifically, this is achieved by finding the shortest path from the initial state to the marked states within the weighted verifier. 
Solving the shortest path problem over the verifier provides the correct value to check the $\lambda$-sa detectability.

The contributions of this paper are as follows: 
\begin{enumerate}
\item[(1)] The novel definition of $\lambda$-sensor-attack detectability that can ensure probabilistic detection for all complete, consistent, and successful sensor attack strategies. 
\item[(2)] Two new problems formulation based on the $\lambda$-sa detectability. The verification problem of $\lambda$-sa detectability and an optimal value problem to ensure $\lambda$-sa detectability.
\item[(3)] A polynomial-time solution methodology that solves both problems.
\end{enumerate}
The remainder of the paper is organized as follows. Section~\ref{sect:motivation} introduces a motivating example for the problem addressed in this study. 
Section~\ref{sect:sup} reviews the necessary definitions of supervisory control and PDES under sensor deception attacks. 
Section~\ref{sect:preliminaries} presents the modeling of the system under sensor attacks and also the class of all complete, consistent, and successful sensor attack strategies. 
The concept of $\lambda$-sa detectability and two verification problems are formulated in Section~\ref{sect:problem}. 
In Section~\ref{sect:solution}, we describe our polynomial-time complexity solution for the two verification problems. We conclude the paper in Section~\ref{sect:conclusion}. 