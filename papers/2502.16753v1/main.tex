%Version 3 October 2023
% See section 11 of the User Manual for version history
%
%%%%%%%%%%%%%%%%%%%%%%%%%%%%%%%%%%%%%%%%%%%%%%%%%%%%%%%%%%%%%%%%%%%%%%
%%                                                                 %%
%% Please do not use \input{...} to include other tex files.       %%
%% Submit your LaTeX manuscript as one .tex document.              %%
%%                                                                 %%
%% All additional figures and files should be attached             %%
%% separately and not embedded in the \TeX\ document itself.       %%
%%                                                                 %%
%%%%%%%%%%%%%%%%%%%%%%%%%%%%%%%%%%%%%%%%%%%%%%%%%%%%%%%%%%%%%%%%%%%%%

%%\documentclass[referee,sn-basic]{sn-jnl}% referee option is meant for double line spacing

%%=======================================================%%
%% to print line numbers in the margin use lineno option %%
%%=======================================================%%

%%\documentclass[lineno,sn-basic]{sn-jnl}% Basic Springer Nature Reference Style/Chemistry Reference Style

%%======================================================%%
%% to compile with pdflatex/xelatex use pdflatex option %%
%%======================================================%%

%%\documentclass[pdflatex,sn-basic]{sn-jnl}% Basic Springer Nature Reference Style/Chemistry Reference Style


%%Note: the following reference styles support Namedate and Numbered referencing. By default the style follows the most common style. To switch between the options you can add or remove “Numbered” in the optional parenthesis. 
%%The option is available for: sn-basic.bst, sn-vancouver.bst, sn-chicago.bst%  
 
%%\documentclass[sn-nature]{sn-jnl}% Style for submissions to Nature Portfolio journals
%%\documentclass[sn-basic]{sn-jnl}% Basic Springer Nature Reference Style/Chemistry Reference Style
% \documentclass[sn-mathphys-num]{sn-jnl}% Math and Physical Sciences Numbered Reference Style 
\documentclass[sn-mathphys-ay]{sn-jnl}% Math and Physical Sciences Author Year Reference Style
%%\documentclass[sn-aps]{sn-jnl}% American Physical Society (APS) Reference Style
%%\documentclass[sn-vancouver,Numbered]{sn-jnl}% Vancouver Reference Style
%%\documentclass[sn-apa]{sn-jnl}% APA Reference Style 
%%\documentclass[sn-chicago]{sn-jnl}% Chicago-based Humanities Reference Style

%%%% Standard Packages
%%<additional latex packages if required can be included here>
\usepackage[dvipsnames]{xcolor}
\usepackage{graphicx}%
\usepackage{multirow}%
\usepackage{amsmath,amssymb,amsfonts}%
\usepackage{amsthm}%
\usepackage{mathrsfs}%
\usepackage[title]{appendix}%
\usepackage{xcolor}%
\usepackage{textcomp}%
\usepackage{manyfoot}%
\usepackage{booktabs}%
\usepackage{algorithm}%
\usepackage{algorithmicx}%
\usepackage{algpseudocode}%
\usepackage{listings}%
\usepackage{subcaption}

%%%%

%%%%%=============================================================================%%%%
%%%%  Remarks: This template is provided to aid authors with the preparation
%%%%  of original research articles intended for submission to journals published 
%%%%  by Springer Nature. The guidance has been prepared in partnership with 
%%%%  production teams to conform to Springer Nature technical requirements. 
%%%%  Editorial and presentation requirements differ among journal portfolios and 
%%%%  research disciplines. You may find sections in this template are irrelevant 
%%%%  to your work and are empowered to omit any such section if allowed by the 
%%%%  journal you intend to submit to. The submission guidelines and policies 
%%%%  of the journal take precedence. A detailed User Manual is available in the 
%%%%  template package for technical guidance.
%%%%%=============================================================================%%%%

%% as per the requirement new theorem styles can be included as shown below
\theoremstyle{thmstyleone}%
\newtheorem{theorem}{Theorem}%  meant for continuous numbers
% \newtheorem{theorem}{Theorem}[section]% meant for sectionwise numbers
%% optional argument [theorem] produces theorem numbering sequence instead of independent numbers for Proposition
% \newtheorem{proposition}[theorem]{Proposition}% 
\newtheorem{proposition}{Proposition}% to get separate numbers for theorem and proposition etc.

\theoremstyle{thmstyletwo}%
\newtheorem{example}{Example}%
\newtheorem{remark}{Remark}%

\theoremstyle{thmstylethree}%
\newtheorem{definition}{Definition}%
\newtheorem{problem}{Problem}%

\raggedbottom
%%\unnumbered% uncomment this for unnumbered level heads
%%%%%% ADDITIONAL PACKAGES AND COMMANDS%%%%%%%%%%%%%%%
%\usepackage[pagewise]{lineno}
%\linenumbers

\newcommand{\arena}{$\mathcal{A}$}
\newcommand{\attackset}{\mathcal{A}}
\newcommand{\Act}{\mathcal{A}}
\newcommand{\Prob}{\mathcal{P}}
\newcommand{\lang}{\mathcal{L}}
\newcommand{\graph}{\mathcal{G}}
\newcommand{\mask}{\mathcal{M}}
\newcommand{\expval}{\mathbb{E}}
\usepackage{caption}
\usepackage{subcaption} 

\newcommand{\rmg}[1]{ {\color{blue}RMG: #1}}

%\usepackage{epsfig} % for postscript graphics files
%\usepackage{epstopdf}


\begin{document}

\title[]{Enhancing sensor attack detection in supervisory control systems modeled by probabilistic automata}

% Enhancing detection of sensor attacks in supervisory control systems modeled by probabilistic automata

%Detecting sensor attack strategies in supervisory control systems modeled by probabilistic automata
%On detecting sensor 
%Detection of classes of sensor attacks using probabilistic supervisory control framework
%%=============================================================%%
%% GivenName	-> \fnm{Joergen W.}
%% Particle	-> \spfx{van der} -> surname prefix
%% FamilyName	-> \sur{Ploeg}
%% Suffix	-> \sfx{IV}
%% \author*[1,2]{\fnm{Joergen W.} \spfx{van der} \sur{Ploeg} 
%%  \sfx{IV}}\email{iauthor@gmail.com}
%%=============================================================%%

\author*[1]{\fnm{Parastou} \sur{Fahim}}\email{pbf5107@psu.edu}

\author[2,3]{\fnm{Samuel} \sur{Oliveira}}\email{samuel.oliveira@unifap.br}

\author[1]{\fnm{R\^omulo} \sur{Meira-G\'oes}}\email{romulo@psu.edu}
% \equalcont{These authors contributed equally to this work.}

% \author[1,2]{\fnm{Third} \sur{Author}}\email{iiiauthor@gmail.com}
% \equalcont{These authors contributed equally to this work.}

\affil[1]{\orgdiv{School of Electrical Engineering and Computer Science}, \orgname{The Pennsylvania State University}, \orgaddress{\city{University Park}, \state{PA}, \country{USA}}}

\affil[2]{\orgdiv{Graduate Program in Electrical Engineering}, \orgname{Santa Catarina State University (UDESC)}, \orgaddress{\city{Joinville}, \state{SC}, \country{Brazil}}}

\affil[3]{\orgdiv{Department of Exact and Technological Sciences}, \orgname{Federal University of Amapá (UNIFAP)}, \orgaddress{\city{Macapá}, \state{AP}, \country{Brazil}}}
% \affil[2]{\orgdiv{Department}, \orgname{Organization}, \orgaddress{\street{Street}, \city{City}, \postcode{10587}, \state{State}, \country{Country}}}

% \affil[3]{\orgdiv{Department}, \orgname{Organization}, \orgaddress{\street{Street}, \city{City}, \postcode{610101}, \state{State}, \country{Country}}}

%%==================================%%
%% Sample for unstructured abstract %%
%%==================================%%

\abstract{
Sensor attacks compromise the reliability of cyber-physical systems (CPSs) by altering sensor outputs with the objective of leading the system to unsafe system states. 
This paper studies a probabilistic intrusion detection framework based on $\lambda$-sensor-attack detectability ($\lambda$-sa), a formal measure that evaluates the likelihood of a system being under attack based on observed behaviors.
Our framework enhances detection by extending its capabilities to identify multiple sensor attack strategies using probabilistic information, which enables the detection of sensor attacks that were undetected by current detection methodologies. 
We develop a polynomial-time algorithm that verifies $\lambda$-sa detectability by constructing a weighted verifier automaton and solving the shortest path problem. 
Additionally, we propose a method to determine the maximum detection confidence level ($\lambda$*) achievable by the system, ensuring the highest probability of identifying attack-induced behaviors.  
%Sensor deception is a class of attacks in control systems where an attacker manipulates sensor readings to cause damage to the system.
%In this work, we investigate the problem of designing better and faster intrusion detection systems against sensor deception attacks.
%We study this problem in the context of stochastic supervisory control theory using the notion of $\epsilon$-safety detection.
%The $\epsilon$-safety notion ensures that a sensor deception attack can be detected due to changes in the probabilistic behavior in the control system, i.e., it leaves a probability footprint. 
% The intrusion detection mechanism uses the probabilistic information of the system to identify if the system is under attack.
% Our work investigates quantitative measurements to detect this class of attacks in the context of stochastic supervisory control.
% We introduce the notion of $\epsilon$-safe systems, which is a first step to generalize qualitative intrusion detection conditions to quantitative intrusion detection conditions.
%We provide necessary and sufficient conditions to verify if a system is $\epsilon$-safe in polynomial-time complexity improving the current state-of-the-art exponential-time complexity. 
% This algorithm improves the state-of-the-art verification algorithm for $\epsilon$-safety that ran in exponential-time complexity.
}

%%================================%%
%% Sample for structured abstract %%
%%================================%%

% \abstract{\textbf{Purpose:} The abstract serves both as a general introduction to the topic and as a brief, non-technical summary of the main results and their implications. The abstract must not include subheadings (unless expressly permitted in the journal's Instructions to Authors), equations or citations. As a guide the abstract should not exceed 200 words. Most journals do not set a hard limit however authors are advised to check the author instructions for the journal they are submitting to.
% 
% \textbf{Methods:} The abstract serves both as a general introduction to the topic and as a brief, non-technical summary of the main results and their implications. The abstract must not include subheadings (unless expressly permitted in the journal's Instructions to Authors), equations or citations. As a guide the abstract should not exceed 200 words. Most journals do not set a hard limit however authors are advised to check the author instructions for the journal they are submitting to.
% 
% \textbf{Results:} The abstract serves both as a general introduction to the topic and as a brief, non-technical summary of the main results and their implications. The abstract must not include subheadings (unless expressly permitted in the journal's Instructions to Authors), equations or citations. As a guide the abstract should not exceed 200 words. Most journals do not set a hard limit however authors are advised to check the author instructions for the journal they are submitting to.
% 
% \textbf{Conclusion:} The abstract serves both as a general introduction to the topic and as a brief, non-technical summary of the main results and their implications. The abstract must not include subheadings (unless expressly permitted in the journal's Instructions to Authors), equations or citations. As a guide the abstract should not exceed 200 words. Most journals do not set a hard limit however authors are advised to check the author instructions for the journal they are submitting to.}

\keywords{Discrete Event Systems; Supervisory Control; Cybersecurity; Intrusion Detection}

%%\pacs[JEL Classification]{D8, H51}

%%\pacs[MSC Classification]{35A01, 65L10, 65L12, 65L20, 65L70}

\maketitle
\graphicspath{{Figs/}}

\section{Introduction}
\vspace{-0.5em}
%!TEX root = gcn.tex
\section{Introduction}
Graphs, representing structural data and topology, are widely used across various domains, such as social networks and merchandising transactions.
Graph convolutional networks (GCN)~\cite{iclr/KipfW17} have significantly enhanced model training on these interconnected nodes.
However, these graphs often contain sensitive information that should not be leaked to untrusted parties.
For example, companies may analyze sensitive demographic and behavioral data about users for applications ranging from targeted advertising to personalized medicine.
Given the data-centric nature and analytical power of GCN training, addressing these privacy concerns is imperative.

Secure multi-party computation (MPC)~\cite{crypto/ChaumDG87,crypto/ChenC06,eurocrypt/CiampiRSW22} is a critical tool for privacy-preserving machine learning, enabling mutually distrustful parties to collaboratively train models with privacy protection over inputs and (intermediate) computations.
While research advances (\eg,~\cite{ccs/RatheeRKCGRS20,uss/NgC21,sp21/TanKTW,uss/WatsonWP22,icml/Keller022,ccs/ABY318,folkerts2023redsec}) support secure training on convolutional neural networks (CNNs) efficiently, private GCN training with MPC over graphs remains challenging.

Graph convolutional layers in GCNs involve multiplications with a (normalized) adjacency matrix containing $\numedge$ non-zero values in a $\numnode \times \numnode$ matrix for a graph with $\numnode$ nodes and $\numedge$ edges.
The graphs are typically sparse but large.
One could use the standard Beaver-triple-based protocol to securely perform these sparse matrix multiplications by treating graph convolution as ordinary dense matrix multiplication.
However, this approach incurs $O(\numnode^2)$ communication and memory costs due to computations on irrelevant nodes.
%
Integrating existing cryptographic advances, the initial effort of SecGNN~\cite{tsc/WangZJ23,nips/RanXLWQW23} requires heavy communication or computational overhead.
Recently, CoGNN~\cite{ccs/ZouLSLXX24} optimizes the overhead in terms of  horizontal data partitioning, proposing a semi-honest secure framework.
Research for secure GCN over vertical data  remains nascent.

Current MPC studies, for GCN or not, have primarily targeted settings where participants own different data samples, \ie, horizontally partitioned data~\cite{ccs/ZouLSLXX24}.
MPC specialized for scenarios where parties hold different types of features~\cite{tkde/LiuKZPHYOZY24,icml/CastigliaZ0KBP23,nips/Wang0ZLWL23} is rare.
This paper studies $2$-party secure GCN training for these vertical partition cases, where one party holds private graph topology (\eg, edges) while the other owns private node features.
For instance, LinkedIn holds private social relationships between users, while banks own users' private bank statements.
Such real-world graph structures underpin the relevance of our focus.
To our knowledge, no prior work tackles secure GCN training in this context, which is crucial for cross-silo collaboration.


To realize secure GCN over vertically split data, we tailor MPC protocols for sparse graph convolution, which fundamentally involves sparse (adjacency) matrix multiplication.
Recent studies have begun exploring MPC protocols for sparse matrix multiplication (SMM).
ROOM~\cite{ccs/SchoppmannG0P19}, a seminal work on SMM, requires foreknowledge of sparsity types: whether the input matrices are row-sparse or column-sparse.
Unfortunately, GCN typically trains on graphs with arbitrary sparsity, where nodes have varying degrees and no specific sparsity constraints.
Moreover, the adjacency matrix in GCN often contains a self-loop operation represented by adding the identity matrix, which is neither row- nor column-sparse.
Araki~\etal~\cite{ccs/Araki0OPRT21} avoid this limitation in their scalable, secure graph analysis work, yet it does not cover vertical partition.

% and related primitives
To bridge this gap, we propose a secure sparse matrix multiplication protocol, \osmm, achieving \emph{accurate, efficient, and secure GCN training over vertical data} for the first time.

\subsection{New Techniques for Sparse Matrices}
The cost of evaluating a GCN layer is dominated by SMM in the form of $\adjmat\feamat$, where $\adjmat$ is a sparse adjacency matrix of a (directed) graph $\graph$ and $\feamat$ is a dense matrix of node features.
For unrelated nodes, which often constitute a substantial portion, the element-wise products $0\cdot x$ are always zero.
Our efficient MPC design 
avoids unnecessary secure computation over unrelated nodes by focusing on computing non-zero results while concealing the sparse topology.
We achieve this~by:
1) decomposing the sparse matrix $\adjmat$ into a product of matrices (\S\ref{sec::sgc}), including permutation and binary diagonal matrices, that can \emph{faithfully} represent the original graph topology;
2) devising specialized protocols (\S\ref{sec::smm_protocol}) for efficiently multiplying the structured matrices while hiding sparsity topology.


 
\subsubsection{Sparse Matrix Decomposition}
We decompose adjacency matrix $\adjmat$ of $\graph$ into two bipartite graphs: one represented by sparse matrix $\adjout$, linking the out-degree nodes to edges, the other 
by sparse matrix $\adjin$,
linking edges to in-degree nodes.

%\ie, we decompose $\adjmat$ into $\adjout \adjin$, where $\adjout$ and $\adjin$ are sparse matrices representing these connections.
%linking out-degree nodes to edges and edges to in-degree nodes of $\graph$, respectively.

We then permute the columns of $\adjout$ and the rows of $\adjin$ so that the permuted matrices $\adjout'$ and $\adjin'$ have non-zero positions with \emph{monotonically non-decreasing} row and column indices.
A permutation $\sigma$ is used to preserve the edge topology, leading to an initial decomposition of $\adjmat = \adjout'\sigma \adjin'$.
This is further refined into a sequence of \emph{linear transformations}, 
which can be efficiently computed by our MPC protocols for 
\emph{oblivious permutation}
%($\Pi_{\ssp}$) 
and \emph{oblivious selection-multiplication}.
% ($\Pi_\SM$)
\iffalse
Our approach leverages bipartite graph representation and the monotonicity of non-zero positions to decompose a general sparse matrix into linear transformations, enhancing the efficiency of our MPC protocols.
\fi
Our decomposition approach is not limited to GCNs but also general~SMM 
by 
%simply 
treating them 
as adjacency matrices.
%of a graph.
%Since any sparse matrix can be viewed 

%allowing the same technique to be applied.

 
\subsubsection{New Protocols for Linear Transformations}
\emph{Oblivious permutation} (OP) is a two-party protocol taking a private permutation $\sigma$ and a private vector $\xvec$ from the two parties, respectively, and generating a secret share $\l\sigma \xvec\r$ between them.
Our OP protocol employs correlated randomnesses generated in an input-independent offline phase to mask $\sigma$ and $\xvec$ for secure computations on intermediate results, requiring only $1$ round in the online phase (\cf, $\ge 2$ in previous works~\cite{ccs/AsharovHIKNPTT22, ccs/Araki0OPRT21}).

Another crucial two-party protocol in our work is \emph{oblivious selection-multiplication} (OSM).
It takes a private bit~$s$ from a party and secret share $\l x\r$ of an arithmetic number~$x$ owned by the two parties as input and generates secret share $\l sx\r$.
%between them.
%Like our OP protocol, o
Our $1$-round OSM protocol also uses pre-computed randomnesses to mask $s$ and $x$.
%for secure computations.
Compared to the Beaver-triple-based~\cite{crypto/Beaver91a} and oblivious-transfer (OT)-based approaches~\cite{pkc/Tzeng02}, our protocol saves ${\sim}50\%$ of online communication while having the same offline communication and round complexities.

By decomposing the sparse matrix into linear transformations and applying our specialized protocols, our \osmm protocol
%($\prosmm$) 
reduces the complexity of evaluating $\numnode \times \numnode$ sparse matrices with $\numedge$ non-zero values from $O(\numnode^2)$ to $O(\numedge)$.

%(\S\ref{sec::secgcn})
\subsection{\cgnn: Secure GCN made Efficient}
Supported by our new sparsity techniques, we build \cgnn, 
a two-party computation (2PC) framework for GCN inference and training over vertical
%ly split
data.
Our contributions include:

1) We are the first to explore sparsity over vertically split, secret-shared data in MPC, enabling decompositions of sparse matrices with arbitrary sparsity and isolating computations that can be performed in plaintext without sacrificing privacy.

2) We propose two efficient $2$PC primitives for OP and OSM, both optimally single-round.
Combined with our sparse matrix decomposition approach, our \osmm protocol ($\prosmm$) achieves constant-round communication costs of $O(\numedge)$, reducing memory requirements and avoiding out-of-memory errors for large matrices.
In practice, it saves $99\%+$ communication
%(Table~\ref{table:comm_smm}) 
and reduces ${\sim}72\%$ memory usage over large $(5000\times5000)$ matrices compared with using Beaver triples.
%(Table~\ref{table:mem_smm_sparse}) ${\sim}16\%$-

3) We build an end-to-end secure GCN framework for inference and training over vertically split data, maintaining accuracy on par with plaintext computations.
We will open-source our evaluation code for research and deployment.

To evaluate the performance of $\cgnn$, we conducted extensive experiments over three standard graph datasets (Cora~\cite{aim/SenNBGGE08}, Citeseer~\cite{dl/GilesBL98}, and Pubmed~\cite{ijcnlp/DernoncourtL17}),
reporting communication, memory usage, accuracy, and running time under varying network conditions, along with an ablation study with or without \osmm.
Below, we highlight our key achievements.

\textit{Communication (\S\ref{sec::comm_compare_gcn}).}
$\cgnn$ saves communication by $50$-$80\%$.
(\cf,~CoGNN~\cite{ccs/KotiKPG24}, OblivGNN~\cite{uss/XuL0AYY24}).

\textit{Memory usage (\S\ref{sec::smmmemory}).}
\cgnn alleviates out-of-memory problems of using %the standard 
Beaver-triples~\cite{crypto/Beaver91a} for large datasets.

\textit{Accuracy (\S\ref{sec::acc_compare_gcn}).}
$\cgnn$ achieves inference and training accuracy comparable to plaintext counterparts.
%training accuracy $\{76\%$, $65.1\%$, $75.2\%\}$ comparable to $\{75.7\%$, $65.4\%$, $74.5\%\}$ in plaintext.

{\textit{Computational efficiency (\S\ref{sec::time_net}).}} 
%If the network is worse in bandwidth and better in latency, $\cgnn$ shows more benefits.
$\cgnn$ is faster by $6$-$45\%$ in inference and $28$-$95\%$ in training across various networks and excels in narrow-bandwidth and low-latency~ones.

{\textit{Impact of \osmm (\S\ref{sec:ablation}).}}
Our \osmm protocol shows a $10$-$42\times$ speed-up for $5000\times 5000$ matrices and saves $10$-2$1\%$ memory for ``small'' datasets and up to $90\%$+ for larger ones.

% Cyber-physical systems (CPSs) bridge computational and communication technologies, enabling the monitoring and control of physical processes.
% However, their increase in use, complexity, and dependence on communication networks introduces new vulnerabilities that may be exploited by cyber-attacks.
% For this reason, it is crucial to identify and understand vulnerabilities in CPSs for their secure operation.

% In this paper, we study the problem of designing better and faster \emph{intrusion detection (ID) systems} for CPSs modeled as \emph{probabilistic discrete event systems (PDES)} \citep{Lawford:1993,Garg:1999,Lafortune:2021}. 
% An ID system monitors the presence of attacks by analyzing the behavior of the control system.
% Herein, we investigate detectors for \emph{sensor deception attacks}; an attacker that hijacks a subset of sensors to damage the CPS. 

% % In the context of DES, several works focused on detecting and preventing deception attacks. 
% % We separate these works in three categories: \textbf{[1]} \emph{attack synthesis} \citep{Su:2018,meira-goes:2020synthesis,tong:2022,YAO2024deception}, \textbf{[2]}~\emph{robust supervisor design} \citep{meira-goes:2023dealing,Wang:2020,Zheng2023deception}, and \textbf{[3]} \emph{intrusion detection design} \citep{Thorsley:2006,Carvalho:2018,Lima:2019,meira-goes:2020towards,wang:2022,fritz2023detection,zhang2023robust,lin2023diagnosability}.
% % Our work differs intrinsically from attack synthesis and robust supervisor design since we study intrusion detection design.
% % Within intrusion detection design, most works only consider logical, without probabilities, DES models except for \citep{Thorsley:2006,meira-goes:2020towards}.
% In DES, several works focused on detecting and preventing deception attacks, e.g., see \citep{Rashidinejad:2019,hadjicostis2022cybersecurity}. 
% We separate these works in three categories: \textbf{[1]} \emph{attack synthesis} \citep{Su:2018,meira-goes:2020synthesis,tong:2022,YAO2024deception}, \textbf{[2]}~\emph{robust supervisor design} \citep{Su:2018,wakaiki2019supervisory,meira-goes:2023dealing,Zheng2023deception}, and \textbf{[3]} \emph{intrusion detection design} \citep{Thorsley:2006,Carvalho:2018,Lima:2019,meira-goes:2020towards,wang:2022,fritz2023detection,zhang2023robust,lin2023diagnosability}.
% Our work differs intrinsically from attack synthesis and robust supervisor design since we focus on ID.
% Within ID design, most works only consider logical, without probabilities, DES models except for \citep{Thorsley:2006,meira-goes:2020towards}.

% \cite{Thorsley:2006} investigated ID systems for actuator deception attacks. 
% This attacker was modeled using a probabilistic model, leading to a probabilistic model for the controlled system.
% However, the ID system \emph{does not use} this probabilistic information to detect the attacker, i.e., the detector only uses logical models similar to those in \citep{Carvalho:2018,Lima:2019}.
% In other words, these detectors can only detect logical footprints left by the attacker.


% Of particular relevance to this paper is the work in \citep{meira-goes:2020towards} where the notion of \emph{$\epsilon$-safety} is introduced.
% $\epsilon$-safety defines a \emph{quantitative measure} for detecting sensor deception attacks that are undetectable by logical intrusion detectors in \citep{Carvalho:2018,Lima:2019}.
% The $\epsilon$ value represents the certainty a behavior originates from an attacked system.
% This new detection definition identifies attackers that significantly change the probability of the controlled system when modifying the behavior of the system but that remain undetectable by their logical behavior.
% In other words, it evaluates if the attacker leaves a \emph{probability footprint} while stealthily modifying the controlled behavior.
% % i.e., the likelihood of the attacked behavior.

% In this paper, we introduce a computationally more efficient algorithm to verify $\epsilon$-safety, i.e., to verify if the attacker leaves a probability footprint.
% \cite{meira-goes:2020towards} presented an algorithm with \emph{exponential-time complexity} to solve the $\epsilon$-safety verification problem.
% We present a \emph{polynomial-time} complexity algorithm to this verification problem.
% The complexity reduction comes from transforming the $\epsilon$-safety verification problem to the \emph{shortest path problem} \citep{cormen2022introduction}.

% Our solution methodology comprises two steps and uses methods from DES and graph theory.
% In the first step, we build a structure, denoted \emph{weighted verifier}, that includes information on the attacked and unattacked controlled systems.
% This structure was inspired by the verifier automaton used for faulty diagnosis, which has information about the faulty and nonfaulty systems \citep{yoo2002polynomial}.
% Using the weights in the verifier, we find the string with the least $\epsilon$-safety value by finding the shortest path from initial to marked states in the weighted verifier.
% % This string is obtained by finding the shortest path to \emph{detection states}, i.e.,
% The solution of the shortest path problem over the verifier provides the correct value to verify $\epsilon$-safety.

% The contributions of this paper are as follows: (i) a correction in the original problem definition of finding the largest $\epsilon$ such that $\epsilon$-safety holds; and (ii) the first, to the best of our knowledge, polynomial-time complexity algorithm to verify $\epsilon$-safety.

% The rest of the paper is structured as follows: Section~\ref{sect:preliminaries} reviews the necessary definitions of PDES and supervisory control under sensor deception attacks. 
% The concept of $\epsilon$-safety and two verification problems are formulated in Section~\ref{sect:problem}. In Section~\ref{sect:solution}, we describe our polynomial-time complexity solution for the two verification problems. 
% We conclude the paper in Section~\ref{sect:conclusion}. 
% % \begin{figure}[h]
% %     \centering
% %     \includegraphics[width=0.4\textwidth]{6.1.png} 
% %     \caption{CPS} 
% %     \label{CPS} 
% % \end{figure}



\section{Motivation Example} \label{sect:motivation}
Inspired by the problem described in \citep{meira-goes:2020towards}, we consider, as a motivating example, a scenario in which two vehicles are traveling on an infinite, discretized road. In this scenario, the vehicle in front, referred to as \emph{adv veh}, is manually driven, i.e., its actions are uncontrollable. 
The vehicle behind, referred to as \emph{ego veh}, is autonomous.
The two vehicles move in the same direction with an initial distance of 2 units between them.
When the relative distance between the vehicles becomes zero, a collision occurs.
% To simplify the scenario, \emph{adv veh} is considered to have moved out of range when the distance from \emph{ego veh} reaches three or more units.
% In this case, \emph{adv veh} is no longer considered a threat to \emph{ego veh}.


\begin{figure}[h]
    \centering
    \includegraphics[width=0.65\textwidth]{aut-veh-overview.pdf} 
    \caption{Overview of discrete collision avoidance modeling for autonomous vehicles} 
    \label{Example} 
\end{figure}

\noindent \textbf{Controlled system.} 
The goal of the $ego$ controller is to avoid collision with $adv$ by measuring the relative distance between the two cars.
Based on a discrete-state event-driven model of this system, we can use standard supervisory control theory techniques to synthesize a controller that ensures no collision.
Intuitively, this safe controller prevents $ego$ from moving ahead when the relative distance between the cars is equal to one. 

\noindent \textbf{System under attack.} 
Let us consider that $ego$ has been compromised by a sensor attacker.
The attacker hijacks the sensors in $ego$ aiming to cause a collision between $ego$ and $adv$. 
The attacker might use two different attack strategies to cause the collision between the cars.
The first attack strategy, $att_1$, will immediately insert a fictitious reading of relative distance equal to $2$ when $ego$ is only $1$ cell away from $adv$. 
In this manner, the controller allows $ego$ to move forward and collide with $adv$. 
On the other hand, the second attack strategy, $att_2$, only makes the insertion the second time $ego$ and $adv$ are $1$ cell away.
The first attack strategy ``eagerly" changes the nominal behavior to reach a collision whereas the second attack strategy allows the nominal behavior to happen before changing it.

\noindent \textbf{ID systems.}
Using probabilistic information about the system allows detection analysis beyond the logical ID systems.
We discuss ID mechanisms options below. 

\noindent \textbf{(1) Logical ID systems.} 
Logical ID mechanisms rely on monitoring the behavior of the controlled system to determine whether an attack has occurred or not. 
These mechanisms identify a sensor attack when the observed behavior deviates from the nominal behavior \citep{Carvalho:2018, Lima:2019, lin2024diagnosability}. 
% However, when an attacker manipulates sensor observations in such a way that it mimics nominal behavior, such malicious actions may go undetected. 
Since a relative distance of $2$ is possible in the collision avoidance system, both $att_1$ and $att_2$ strategies feed the ID with nominal behavior.
Thus, logical ID systems cannot detect strategies $att_1$ and $att_2$.

\noindent \textbf{(2) Probabilistic $\epsilon$-safety ID system.} 
Our previous works \citep{meira-goes:2020towards, Fahim2024-wodes} provided a probabilistic approach to detect a sensor attack strategy using the notion of $\epsilon$-safety. 
This notion compares the probability of generating a nominal behavior versus an attacked one.
% Let us consider that it is unlikely for $adv$ to immediately move ahead when $ego$ is $1$ cell away, i.e., a probabilistic model of the system in Fig.~\ref{Example}.
According to \citep{Fahim2024-wodes}, this system is $\epsilon$-safe with respect to strategy $att_1$ for a confidence level of $0.9$.
The ID detects $att_1$ since the attacker eagerly inserts the fictitious event without considering its probability changes.
However, $\epsilon$-safety regarding strategy $att_1$ does not provide any information about detecting strategy $att_2$.
We need to verify $att_2$ to guarantee that $\epsilon$-safety holds, i.e., a new run of the verification procedure.

\noindent \textbf{(2) Probabilistic $\lambda$-sensor-attacks ID system.} 
We present an enhanced probabilistic ID approach that goes beyond detecting a specific sensor attack strategy. 
The notion of $\lambda$-sensor-attacks detectability can effectively detect all possible sensor attack strategies.
In Section~\ref{sect:solution}, we show that the collision avoidance system in Fig.~\ref{Example} is $\lambda$-sensor-attacks detectable with confidence level $0.9$.
It means that both $att_1$ and $att_2$ can be detected when using probabilistic information of the system.
% Building on the foundations of our previous works, this new approach significantly expands its applicability and robustness by detecting a broad range of attack strategies, including all complete, consistent, and successful sensor attack strategies. 
% For instance, the new approach 



\section{Modeling of Controlled Systems}\label{sect:sup} 
\subsection{Supervisory Control}\label{subsect:des}
We consider the supervisory layer of a feedback control system, where the uncontrolled system (plant) is modeled as a Deterministic Finite-State Automaton (DFA) in the discrete-event modeling formalism.
A DFA is defined by $G := (X_G,\Sigma,\delta_G,x_{0,G},X_{m,G})$, where $X_G$ is the finite set of states, $\Sigma$ is the finite set of events, $\delta_G:X_G\times\Sigma\rightarrow X_G$ is the partial transition function, $x_{0,G}$ is the initial state, and $X_{m,G}$ is the set of marked states.
The function $\delta_G$ is extended, in the usual manner, to the domain $X_G\times\Sigma^*$. 
The language and the marked language generated by $G$ are defined by $\lang(G) := \{s \in \Sigma^*\mid \delta_G(x_{0,G},s)!\}$ and $\lang_m(G) := \{s \in \lang(G)\mid \delta_G(x_{0,G},s)\in X_{m,G}\}$, where $!$ means that the function is defined.

For convenience, we define $\Gamma_G(x) := \{e\in\Sigma\mid\delta_G(x,e)!\}$ as the set of feasible events in state $x\in X_G$.
% We also use the following notation for any string $s\in \Sigma^*$.
For string $s$, the length of $s$ is denoted by $|s|$ whereas $s[i]$ denotes the $i^{th}$ event of $s$ such that $s = s[1]s[2]\ldots s[|s|]$.
The $i^{th}$ prefix of $s$ is defined by $s^i$, i.e., $s^i = s[1]s[2]\ldots s[i]$ and $s^0 = \epsilon$. 
% Moreover, we denote by $\bar{s}$ the set of all prefixes of $s$.
% Finally, we use $\mathbb{N}$ to be the set of natural numbers, $[n]$ and $[n]^+$ to be, respectively, the set of natural numbers and the set of positive natural numbers both bounded by $n\in \mathbb{N}$.

Considering the supervisory control theory of DES \citep{Ramadge:1987}, a \emph{supervisor} controls the plant $G$ by dynamically disabling events.
The limited actuation capability of the supervisor is modeled by partitioning the event set $\Sigma$ into the sets of controllable and uncontrollable events, $\Sigma_{c}$ and $\Sigma_{uc}$.
The supervisor cannot disable uncontrollable events. 
Therefore, the supervisor's control decisions are limited to the set $\Gamma:=\{\gamma\subseteq\Sigma\mid\Sigma_{uc} \subseteq \gamma\}$.
Formally, a supervisor is a mapping $S:\lang(G)\rightarrow\Gamma$ defined to satisfy specifications on $G$, e.g., avoid unsafe states in $G$, avoid deadlocks.

The closed-loop behavior of $G$ under the supervision of $S$ is denoted by $S/G$ and generates the closed-loop languages $\lang(S/G)$ and $\lang_m(S/G)$; see, e.g., \citep{Lafortune:2021,Wonham:2018}.
Herein, we assume that supervisor $S$ is realized by an automaton $R = (X_R,\Sigma,\delta_R,x_{0,R})$, i.e., $S(s) = \Gamma_R(\delta_R(x_{0,R},s))$.
With an abuse of notation, we use $S$ and $R$ interchangeably hereafter.
%For any string $s\in \Sigma^*$, we use the following notation. 
%We denote by $e^i_s$ the $i^{th}$ event of $s$ such that $s = e^1_se^2_s\ldots e^{|s|}_s$, where $|s|$ denotes the length of $s$.
%We denote by $s^i$ the $i^{th}$ prefix of $s$, namely $s^i = e^1_s\ldots e^i_s$ and $s^0 = \epsilon$. 
%Moreover, we denote by $\bar{s}$ the set of all prefixes of $s$.
%Finally, we use $\mathbb{N}$ to be the set of natural numbers, $[n]$ and $[n]^+$ to be, respectively, the set of natural numbers and the set of positive natural numbers both bounded by $n\in \mathbb{N}$. 



\subsection{Stochastic Supervisory Control}\label{subsect:sdes}
We consider a stochastic DES modeled as a probabilistic discrete event system (PDES) defined similarly to a DFA \citep{Lawford:1993,Garg:1999,Pantelic:2014}. 
A PDES is defined by the tuple $G := (X_G,\Sigma,\allowbreak \delta_G, P_G,x_{0,G},X_{m,G})$ where $X_G$, $\Sigma$, $\delta_G$, $x_{0,G}$ and $X_{m,G}$ are defined as in the DFA definition.
The probabilistic transition function $P_G:X_G\times\Sigma\times X_G\rightarrow [0,1]$ specifies the probability of moving from state $x$ to state $y$ with event $e$. 
Hereupon, we will use the notation $G$ to describe a PDES.

In this work, we assume that $G$ is deterministic where $\nexists y,y^*\in X_G$, $y^*\neq y$ such that $P_G(x,e,y)>0$ and $P_G(x,e,y^*)>0$.
In this manner, $\delta_G(x,e)= y$ if and only if $P_G(x,e,y) > 0$.
Moreover, we assume that each state in $G$ transitions with probability $1$ or deadlocks, i.e.,  $\sum_{e\in \Sigma} \sum_{y\in X_H} P_G(x,e,y) \in \{0,1\}$ for any $x \in X_G$.

\begin{example}  
Figure~\ref{fig:plant_G} provides an example of PDES $G$ modeling the collision avoidance motivating example described in Sect.~\ref{sect:motivation}.
The PDES $G$ has four states and three events.
States in $G$ model the relative distance among $ego$ and $adv$ whereas events $a,\ b$, and $c$ model the decrease, increase, and no change, respectively, on this relative distance.
The cars collide when the relative distance is equal to zero.
For simplicity, once the relative distance is greater than or equal to three, $adv$ escapes $ego$'s range.
States $0$ and $3$ are deadlock states, e.g., $\sum_{e\in \Sigma} P_G(0,e,\delta_G(x,e)) = 0$.
On the other hand, states $1$ and $2$ are live states with probability $1$.
The dynamics of $G$ is captured by the transitions in Fig.~\ref{fig:plant_G} where the label describes the event and probability of the transition, respectively.
For instance, transition $2\rightarrow 1$ has probability $0.1$ of occurring, e.g, $P_G(2,a,1) = 0.1$.
\end{example}

\begin{figure}[thpb]
\centering
\includegraphics[width=0.4\columnwidth]{Figs/plant_G.png}
\caption{PDES $G$ collision avoidance}
\label{fig:plant_G}
\vspace{-2em}
\end{figure}

Although the language and marked language of PDES $G$ are defined as in the DFA case, the notion of probabilistic languages (p-languages) of a PDES was introduced to characterize the probability of executing a string \citep{Garg:1999}.
The p-language of $G$, $L_p(G):\Sigma^*\rightarrow [0,1]$, is recursively defined for $s\in\Sigma^*$ and $e\in\Sigma$ as: 
$L_p(G)(\epsilon) := 1$, $L_p(G)(se) := L_p(G)(s)P_G(x,e,y)$ if $x=\delta_G(x_{0,G},s)$,  $e\in \Gamma_G(x)$ and $y = \delta_G(x,e)$, and $0$ otherwise.
% \vspace*{-0.3cm}
% \begin{align}
% \hspace{-.2cm}L_p(G)(\varepsilon) &:=1\\
% \hspace{-.2cm}L_p(G)(se) &:=\hspace{-.1cm}\left\lbrace\hspace{-.1cm}\begin{array}{ll}
% L_p(G)(s)P_G(x,e,y)&\text{ if }x = \delta_G(x_{0,G},s)\\
% &\ y=\delta_G(x,e)\\
% 0& \text{ otherwise}
% \end{array}\right.
% \end{align}
\begin{example}  
Continuing with our running example, we can obtain the probabilistic language of PDES $G$ in Fig.~\ref{fig:plant_G}.
For instance, the probability of executing string $abc \in \lang(G)$ is inductively computed by $P_G(2,a,1)P_G(1,b,2)P_G(2,c,2) = 0.1\times 0.1\times 0.8$.
\end{example}
% For more details on the probability space used in this paper, please see \citep{Garg:1999}. 
% One property from the measurable sets in this space is that two \emph{distinct} strings $s$ and $t$, such that no string is a prefix of the other one, generate independent measurable sets.
% This property implies that the probability of generating either one of these two strings is equal to $L_p(H)(s)+L_p(H)(t)$.
% This useful result will be exploited in both the problem formulation and the solution methodology.

For convenience, we write $P_G(x,e)$ to denote $P_G(x,e,\delta_G(x,e))$ whenever $\delta_G(x,e)!$, i.e., the probability of executing $e$ in state $x$.
% And we define $\Gamma(x):=\{e\in \Sigma\mid \delta_G(x,e)!\}$ as the set of active events in state $x$.
% We also define a \emph{marked path} in $G$.
% Intuitively, a marked path in $G$ is a finite sequence of (state, event) pairs starting in $x_{0,G}$, satisfying $\delta_G$, and ending in a marked state.
% \begin{definition}(Marked path)
% A \emph{marked path} in $G$ is a sequence $\rho = x_0e_0\dots x_{n-1}e_{n-1}x_n\in (X_G\times \Sigma)^*X_G$ such that $x_0 = x_{0,G}$, $x_{i+1} = \delta_G(x_i,e_i)$ for $i<n$, and $x_n\in X_{m,G}$.
% The length of $\rho$ is defined by $|\rho| = n+1$. 
% The set of all marked paths in $G$ is denoted as $Paths_m(G)$.
% \end{definition}

In the context of stochastic supervisory control theory, the plant $G$ is controlled by a supervisor $S$ as described in Section~\ref{subsect:des}.
In this work, we use the framework of supervisory control of PDES introduced by \citep{Kumar:2001}.
The supervisor is \emph{deterministic} and realized by DFA $R = (X_R,\Sigma,\allowbreak \delta_R,x_{0,R})$ as previously-described.
And although $R$ is deterministic, its events disablement proportionally increases the probability of the enabled ones.
Given a state $x\in X_G$, a state $y \in X_R$, and an event $e \in \Gamma_G(x)\cap\Gamma_R(y)$, the probability of $e$ being executed is given by the standard normalization:
\begin{equation}\label{eq:renormalization}
P_{R,G}((x_R,x_G),e,(y_R,y_G)) = \frac{P_G(x_G,e)}{\sum_{\sigma\in\Gamma_G(x_R)\cap\Gamma_R(y_R)}P_G(x_G,\sigma)}
\end{equation}
Due to this renormalization, the closed-loop behavior $R/G$ generates a p-language different, in general, than the p-language of $G$.
We can represent the closed-loop behavior $R/G$ as PDES $R||_p G$ where $||_p$ is defined based on the parallel composition $||$ and Eq.~\ref{eq:renormalization}.
The formal definition of $||_p$ is described in Appendix~\ref{app:parallel_prob}.



\begin{example}
Regarding our running example, the supervisor $R$ for PDES $G$ is shown in Fig.~\ref{fig:sup_R}.
This supervisor ensures that the plant never reaches state $0$.
The closed-loop representation of $R/G$ is shown in Fig.~\ref{fig:M_n}.
The supervisor disables event $a$ in state $1$ which results in disabling event $a$ in state $1$ in the plant.
For this reason, events from state $(1,1)$ in $R/G$ have been normalized based on Eq.~\ref{eq:renormalization}.
For example, $P_{R,G}((1,1),b,(2,2)) = \frac{0.1}{0.9}= 0.111\dots$.
\end{example}
\begin{figure}[thpb]
\begin{subfigure}[t]{0.45\columnwidth}
\centering
\includegraphics[width=0.75\columnwidth]{sup_R.png}
\caption{Supervisor $R$ collision avoidance}
\label{fig:sup_R}
\end{subfigure}
\ 
\begin{subfigure}[t]{0.45\columnwidth}
\centering
\includegraphics[width=0.75\columnwidth]{Figs/controlled-R-G.png}
\caption{Supervised system $R/G=R||_pG$}
\label{fig:M_n}
\end{subfigure}
\caption{Supervisor $R$ and controlled system $R/G$}
\label{fig:supervisor-controlled}
\vspace{-2em}
\end{figure}

For simplicity, we assume that the plant $G$ has one critical state, denoted $x_{crit} \in X_G$.
Moreover, we assume that every transition to $x_{crit}$ is controllable, i.e., $\delta_G(x,e) = x_{crit} \Rightarrow e\in \Sigma_c$ for any $x\in X_G$.
These assumptions are without loss of generality as we can generalize it to any regular language by state space refinement in the usual way \citep{Cho:1989,Lafortune:2021}. 
Supervisor $R$ ensures the critical state is not reachable in $R/G$ as in our running example.

% Intuitively, the probabilistic composition is defined similarly to the standard parallel composition.
% The probabilistic composition renormalizes the probability transition function $P_{R,G}$ based on the events enabled by supervisor $R$, i.e., $e\in \Gamma_G(x_G)\cap \Gamma_R(x_R)$.
% Therefore, $R/H$ generates a p-language different, in general, than the p-language of $G$. 

% Based on the probabilistic composition, we define $R/G:= R||_p G$ as the closed-loop representation of $R/G$.
% Moreover, since $R/G$ is a PDES, it has a p-language $L_p(R/G)$.

% In the context of stochastic supervisory control theory, the plant $G$ is controlled by a supervisor as in the previously-described supervisory control framework.
% Nonetheless, there are different ways of studying its closed-loop behavior \cite{Lawford:1993,Kumar:2001,Pantelic:2009}.
% In this paper, we use the results of supervisory control of stochastic DES introduced in \cite{Kumar:2001}, where only the plant behaves stochastically. 
% That is, both the specification and the supervisor are deterministic and defined as in the previously-described logical supervisory control framework.
% However, the supervisor \emph{alters} the probabilistic behavior of the plant via the control actions it takes (disabling events).
% Recall that in the supervisory control framework, the event set of $H$ is partitioned into controllable, $\Sigma_c$, and uncontrollable, $\Sigma_{uc}$, events, where the supervisor does not disable uncontrollable events.
% Conditions for the existence of a supervisor for the above control problem are provided in \cite{Kumar:2001}.

% Formalizing the previous discussion, the plant modeled by PFA $H$ is controlled by a supervisor modeled by DFA $R$.
% And although $R$ is deterministic, its events disablement proportionally increases the probability of the enabled ones.
% Given a state $x\in X_H$, a state $y \in X_R$, and an event $e \in \Gamma_H(x)\cap\Gamma_R(y)$, the probability of $e$ being executed is given by the standard normalization:
% \begin{equation}\label{eq:renormalization}
% Pr_{e}^{x,y} = \frac{Pr_H(x,e,\delta_H(x,e))}{\textstyle\sum\limits_{e'\in\Gamma_H(x)\cap\Gamma_R(y)}Pr_H(x,e',\delta_H(x,e'))} 
% \end{equation}
% The set $\Gamma_H(x)\cap\Gamma_R(y)$ describes the events that can be executed by $H$ in state $x$ restricted by the events enabled by $R$ in state $y$.
% If every event in $\Gamma_H(x)$ is enabled by $R$ in state $y$, then their probabilities of execution remains unaltered, i.e., the denominator in Eq.~(\ref{eq:reachprob}) is equal to $1$. 
% However, if at least one event in $\Gamma_H(x)$ is disabled by $R$ in state $y$,  its probability of execution is proportionally redistributed to the remaining enabled and executable events.
% Therefore, $R/H$ generates a p-language different, in general, than the p-language of $H$. 

% For simplicity and without loss of generality, we assume that the plant $H$ has one deadlock critical state and $R$ ensures that this state is not reachable in $R/H$.
% Specifically, we assume that the language $\lang(R/H)\subseteq\{s\in \lang(H)\mid \delta_H(x_{0,H},s)\neq x_{crit}\}$ where $x_{crit}\in X_H$ is the critical state and $\lang(R/H)$ is controllable, see, e.g., \cite{Lafortune:2008,Wonham:2018}.
% Normally, one would find a supervisor that generates the supremal controllable sublanguage of $\{s\in \lang(H)\mid \delta_H(x_{0,H},s)\neq x_{crit}\}$, but we do not assume such a supervisor is selected, see Example~II.1.
% For the definition of the supremal controllable sublanguage see, e.g., \cite{Lafortune:2008,Wonham:2018}.
% Lastly, we define the set of unsafe strings as $U_{uns} = \{s \in \Sigma^* \mid \delta_H(x_{0,H},s) = x_{crit}\}$ and the set of unsafe state pairs for the controlled system $R/H$ by $
% X_{uns} := \{x_{crit}\}\times X_R$

% This section reviews the concepts of probabilistic discrete event systems, stochastic supervisory control, and supervisory control under attacks.  



% \subsection{Probabilistic Discrete Event Systems}\label{subsect:pdes}

% In the discrete-event formalism, a probabilistic discrete event system (PDES) extends DES models as in \citep{Lafortune:2021,Wonham:2018} by introducing probabilistic information to them \citep{Lawford:1993,Garg:1999,Pantelic:2014}. 
% A PDES is defined by the tuple $G := (X_G,\Sigma,\allowbreak \delta_G, P_G,x_{0,G},X_{m,G})$ where $X_G$ is the finite set of states; $\Sigma$ is the finite set of events; $\delta_G:X_G\times\Sigma\rightarrow X_G$ is the partial transition function; $P_G:X_G\times\Sigma\times X_G\rightarrow [0,1]$ is the transition probability function; $x_{0,G}$ is the initial state; and $X_{m,G}$ is the set of marked states. 
% We define $\delta_G$ based on $P_G$ as $\delta_G(x,e)= y$ if and only if $P_G(x,e,y) > 0$, i.e., $\delta_G(x,e)!$ is defined.

% Herein, we follow the state probability termination criteria as in \cite{Lawford:1993, Pantelic:2014} which allows a termination probability of either $0$ or $1$. 
% Formally, the PDES $G$ satisfies $\sum_{y\in X_G \wedge e\in \Sigma} P_G(x,e,y) \in \{0,1\}$ for any $x\in X_G$.
% In other words, states are either deadlocks (no transition from them) or live with probability $1$ (no partial termination).

% \begin{example}  
% Fig.~\ref{Example} provides an example of PDES $G$ modeling the collision avoidance motivating example.
% The PDES $G$ has four states and three events.
% States $0$ and $3$ are deadlock states, e.g., $\sum_{e\in \Sigma} P_G(0,e,\delta_G(x,e)) = 0$.
% On the other hand, states $1$ and $2$ are live states with probability $1$.
% The dynamics of $G$ are provided by the transitions in Fig.~\ref{fig:plant_G} where the label describes the event and probability of the transition, respectively.
% For instance, transition $2\rightarrow 1$  has probability $0.1$ of occurring, i.e.g, $P_G(2,a,1) = 0.1$.
% \end{example}

% % Each state in $G$ represents the relative distance between $ego$ and $adv$.
% % The cars start with a relative distance equal to $2$.
% % Moreover, states $0$ and $3$ are deadlock states that model collision and out-of-range, respectively.
% % For simplicity, we assume that the tracking stops whenever $adv$ is out-of-range. 
% % The model has three events, $\{a, b, c\}$, that model $ego$ moving forward, $adv$ moving forward, $c$ cars remain in the same position.
% % The dynamics of $G$ are provided by the transitions in Fig.~\ref{fig:collision-plant} where the label describes the event and probability of the transition, respectively.

% The language and marked language of PDES $G$ are defined as usual, e.g., $\lang(G) = \{s \in \Sigma^*\mid \delta_G(x_{0,G},s)!\}$ \citep{Lafortune:2021}.
% In \citep{Garg:1999}, the notion of probabilistic languages (p-languages) of a PDES was introduced to characterize the probability of executing a string.

% \begin{definition}(p-language)
% The p-language of $G$, $L_p(G):\Sigma^*\rightarrow [0,1]$, is defined for $s\in\Sigma^*$ and $e\in\Sigma$ as:
% \vspace*{-0.3cm}
% \begin{align}
% \hspace{-.2cm}L_p(G)(\varepsilon) &:=1\\
% \hspace{-.2cm}L_p(G)(se) &:=\hspace{-.1cm}\left\lbrace\hspace{-.1cm}\begin{array}{ll}
% L_p(G)(s)P_G(x,e,y)&\text{ if }x = \delta_G(x_{0,G},s)\\
% &\ y=\delta_G(x,e)\\
% 0& \text{ otherwise}
% \end{array}\right.
% \end{align}
% \end{definition}
% % The notion of probabilistic languages (p-languages) of a PDES was introduced in \citep{Garg:1999}.

% \begin{example}  
% Continuing with our running example, we can obtain the probabilistic language of PDES $G$ as in Fig.~\ref{fig:plant_G}.
% For instance, the probability of executing string $abc \in \lang(G)$ is inductively computed by $P_G(2,a,1)P_G(1,b,2)P_G(2,c,2) = 0.1\times 0.1\times 0.8$.
% \end{example}

% % For more details on the probability space used in this paper, please see \citep{Garg:1999}. 
% % One property from the measurable sets in this space is that two \emph{distinct} strings $s$ and $t$, such that no string is a prefix of the other one, generate independent measurable sets.
% % This property implies that the probability of generating either one of these two strings is equal to $L_p(H)(s)+L_p(H)(t)$.
% % This useful result will be exploited in both the problem formulation and the solution methodology.

% For convenience, we write $P_G(x,e)$ to denote $P_G(x,e,\delta_G(x,e))$ whenever $\delta_G(x,e)!$ (is defined), i.e., the probability of executing $e$ in state $x$.
% And we define $\Gamma(x):=\{e\in \Sigma\mid \delta_G(x,e)!\}$ as the set of active events in state $x$.
% We also introduce the definition of a \emph{marked path} in $G$.
% Intuitively, a marked path in $G$ is a finite sequence of (state, event) pairs starting in $x_{0,G}$, satisfying $\delta_G$, and ending in a marked state.

% \begin{definition}(Marked path)
% A \emph{marked path} in $G$ is a sequence $\rho = x_0e_0\dots x_{n-1}e_{n-1}x_n\in (X_G\times \Sigma)^*X_G$ such that $x_0 = x_{0,G}$, $x_{i+1} = \delta_G(x_i,e_i)$ for $i<n$, and $x_n\in X_{m,G}$.
% The length of $\rho$ is defined by $|\rho| = n+1$. 
% The set of all marked paths in $G$ is denoted as $Paths_m(G)$.
% \end{definition}


% \subsection{Stochastic supervisory control}\label{subsect:sdes}
% We consider the supervisory level of a CPS where the uncontrolled system is modeled as a PDES $G$, \emph{the plant}.
% Limited control capabilities in the plant are modeled by partitioning the set of events $\Sigma$ into two disjoint sets, the set of controllable events $\Sigma_{c}$ and the set of uncontrollable events $\Sigma_{uc}$.
% In the discrete-event formalism, a supervisory controller, or simply \emph{supervisor}, for $G$ can dynamically disable controllable events to ensure desired behaviors such as ``safe" behaviors. 
% Formally, a supervisor is defined as a function $S:\Sigma^*\rightarrow\Gamma$, where $\Gamma = \{\gamma\subseteq\Sigma\mid\Sigma_{uc} \subseteq \gamma\}$ is the set of admissible control decisions.
% The closed-loop behavior of the controlled system is denoted by $S/G$ and defined by the language $\lang(S/G)$; see, e.g., \citep{Lafortune:2021}.

% In this work, we use the framework of supervisory control of PDES introduced by \citep{Kumar:2001}.
% The supervisor is \emph{deterministic} and realized by a deterministic finite-state automaton (DFA) $R = (X_R,\Sigma,\allowbreak \delta_R,x_{0,R})$ with $X_R,\Sigma,\delta_R$ and $x_{0,R}$ defined as in the PDES definition.
% From DFA $R$, we can extract a supervisor function $S$ as $S(s) = \Gamma_R(\delta(x_{0,R},s))$ for every $s\in \lang(R)$, see, e.g., \citep{Lafortune:2021,Wonham:2018}.
% The composition of supervisor $R$ and PDES $G$ defines a new PDES $R/G$ that represents the closed-loop behavior.

% \begin{definition}
% The probabilistic composition, $||_p$ between PDES $G$ and DFA $R$ is defined by $R||_p G = (X_R\times X_G, \Sigma, \delta_{R,G},P_{R,G}, (x_{0,R},x_{0,G}), X_{R}\times X_{m,G})$ where:
% \begin{equation}\label{eq:renorm}
% P_{R,G}((x_R,x_G),e,(y_R,y_G)) = \left\lbrace
% \begin{array}{ll}
% \frac{P_G(x_G,e)}{\sum_{\sigma\in\Gamma_G(x_R)\cap\Gamma_R(y_R)}P_G(x_G,\sigma)} &
% \ e\in \Gamma_R(x_R)\cap\Gamma_G(x_G) \\
% 0 & \text{otherwise}
% \end{array}
% \right.
% \end{equation}
% \end{definition}
% Intuitively, the probabilistic composition is defined similarly to the standard parallel composition.
% However, the probabilistic composition renormalizes the probability transition function $P_{R,G}$ based on the events enabled by supervisor $R$, i.e., $e\in \Gamma_G(x_G)\cap \Gamma_R(x_R)$.
% This probability renormalization is defined by Eq.~\ref{eq:renorm} where the probability of disabled events is redistributed to the enabled events.

% Based on the probabilistic composition, we define $R/G:= R||_p G$ as the closed-loop representation of $R/G$.
% Moreover, since $R/G$ is a PDES, it has a p-language $L_p(R/G)$.

% \begin{example}
% Regarding our running example, the supervisor $R$ for PDES $G$ is shown in Fig.~\ref{fig:supervisor_R}.
% This supervisor ensures that the plant never reaches state $0$.
% The closed-loop representation of $R/G$ is shown in Fig.~\ref{Mn}.
% The supervisor disables event $a$ in state $A$ which results in disabling event $a$ in state $1$ in the plant.
% For this reason, events from state $(A,1)$ in $R/G$ have been normalized based on Eq.~\ref{eq:renorm}.
% For example, $P_{R,G}((A,1),b,(2,B)) = \frac{0.1}{0.9}= 0.111\dots$.
% \end{example}

% For simplicity, we assume that the plant $G$ has one critical state, denoted $x_{crit} \in X_G$.
% This assumption is without loss of generality as we can generalize it to any regular language by state space refinement in the usual way \citep{Cho:1989,Lafortune:2021}. 
% Supervisor $R$ ensures the critical state is not reachable in $R/G$ as in our running example.
% % The set of unsafe strings is defined as $L_{crit} = \{s \in \Sigma^* \mid \delta_G(x_{0,G},s) = x_{crit}\}$. 




\section{Modeling of Control Systems under Sensor Attacks}\label{sect:preliminaries}
In a cyber-security context, we assume that the supervisory control system can be under sensor attack.
In sensor attacks, an attacker hijacks and controls a subset of the sensors to reach the critical state.
In this manner, this attacker modifies the closed-loop behavior of the system.
In this section, we review the supervisory control under sensor deception attacks as in \citep{Su:2018,meira-goes:2020synthesis,meira-goes:2023dealing}.
We focus on explaining the probabilistic attacker model and defining the control system under attack.
% In the appendix, we describe in detail the construction of PDES $R_a/G_a$ as defined in \cite{meira-goes:2021synthesis}. 

\subsection{Sensor Attacker} 
We follow the probabilistic sensor attacker model introduced in \citep{meira-goes:2021synthesis}.
A sensor deception attacker compromises a subset of sensor events, denoted $\Sigma_a\subseteq \Sigma$.
This attacker can modify the readings of these compromised events by inserting fictitious events into or deleting event readings from the supervisor.
To identify the attacker actions, insertion and deletion events are modeled using the sets $\Sigma_i = \{ins(e) \mid e \in \Sigma_a\}$ and $\Sigma_d=\{del(e)\mid e \in \Sigma_a\}$, respectively.
The union of the insertion, deletion, and plant event sets, $\Sigma_m = \Sigma\cup\Sigma_i\cup\Sigma_d$, encompasses the event set of the system under attack. 
Formally, the attacker is defined as:

\begin{definition}[Attack strategy]\label{def:attack_str}
An attack strategy with compromised event set $\Sigma_a$ is defined as a partial map $A: \Sigma_m^* \times (\Sigma\cup \{\epsilon\})\rightarrow \Sigma_m^*$ that satisfies for any $t \in \Sigma^*_m$ and $e \in \Sigma\cup \{\epsilon\}$:
\begin{enumerate}
    \item $A(\epsilon,\epsilon) \in \Sigma_{i}^*$ and $A(t,\epsilon) = \epsilon$ for $t\neq \epsilon$
    \item If $e\in \Sigma_a$, then $A(t,e)\in \{e,del(e)\}\Sigma_i^*$
    \item If $e\in \Sigma\setminus\Sigma_a$, then $A(t,e)\in \{e\}\Sigma_i^*$
\end{enumerate}
\end{definition}
The attack strategy $A$ defines a deterministic action based on the last event executed $e$ and modification history $t$.
We extend the function $A$ to concatenate these modifications for any string $s\in \Sigma^*$: $A(\epsilon) = A(\epsilon,\epsilon)$ and $A(s) = A(s^{|s|-1})A(A(s^{|s|-1}),s[|s|])$.
With an abuse of notation, we assume that attack strategy $A$ is encoded as a DFA $A = (X_A, \delta_A, \Sigma_m, x_{0,A})$ as in \citep{meira-goes:2020synthesis,meira-goes2021synthesistac}.
In Appendix~\ref{app:A_aut}, we show the conditions for this encoding.
Intuitively, each state of $A$ encodes one attacker's decision: insertion, deletion, or no attack. 

To identify how the attack actions affect the plant $G$ and supervisor $S$, we define projection operators to reason about events in $\Sigma_m$ in different contexts.
% For example, an insertion event $ins(e)$ is seen by the supervisor as a legitimate event $e$ whereas the $\varepsilon$ (empty) event is executed by the plant since $ins(e)$ is a fictitious event created by the attacker. 
We define projector operator $\Pi^G$ ($\Pi^S$) that projects events in $\Sigma_m$ to events in $\Sigma$ generated by the plant (observed by the supervisor).
Formally, $\Pi^G$ outputs the event that is executed in $G$, i.e., $\Pi^G(ins(e)) = \varepsilon$ and $\Pi^G(del(e)) = \Pi^G(e) = e$. 
Similarly, $\Pi^S$ outputs the event observed by the supervisor, i.e, $\Pi^S(del(e)) = \varepsilon$ and $\Pi^S(ins(e)) = \Pi^S(e) = e$. 
% Lastly, we define the mask operator $\mask:\Sigma_m\rightarrow\Sigma$ that simply removes insertion and deletion ``marks" from events, i.e., $\mask(del(e)) = \mask(ins(e)) = \mask(e) = e$.



\begin{example}
% \rmg{This example should just focus on the attack strategies. We will need figures different than Fig.~5.
% In the next subsection, you can talk about the closed-loop behavior with these strategies.
% The int values should be only mentioned later (next section). We haven't introduced it yet.}
% Figure~\ref{fig:attack-str} depicts two attack strategies for our running example as DFAs $A_1$ and $A_2$. In the attack strategy $A_1$, illustrated in Figure~\ref{fig:M_a}, the attacker does not use memory. Starting from state 2, the attacker inserts the event $b_{\text{ins}}$ at state 1, creating the path $a(b_{\text{ins}})$ that leads to the unsafe state. By introducing $b_{\text{ins}}$ when the relative distance between two cars is one, the attacker misleads the system into believing the distance is safe. This false perception allows the system to advance, ultimately causing the two cars to collide and transitioning the system to the critical unsafe state, $X_{\text{crit},a}$.

% The probability of this path occurring in the attacked system is $0.1(1)=0.1$, while in the nominal system, it is $0.1(0.111...)=0.0111...$.

% In contrast, the second strategy ($A_2$) assumes the attacker has memory, enabling them to track past states. This allows for a more coordinated sequence of actions, such as $aba(b_{\text{ins}})$, which misleads the system over multiple transitions and eventually drives it to the unsafe state.

%  The probability of this sequence originating from the attacker is $0.1(0.111...) \cdot 0.1(1)=0.00111...$ in the attacked system and $0.1(0.111...) \cdot 0.1(0.111...)=0.000123...$ in the nominal system.

%  For the first strategy $A_1$, the intrusion detection value given by Equation~\ref{eq:attack_ratio} is computed as:
% \[
% \text{int}_{A_1} = \frac{0.1}{0.1 + 0.01111111111} \approx 0.9.
% \]

% For the second strategy $A_2$, the intrusion detection value is computed as:
% \[
% \text{int}_{A_2} = \frac{0.00111111111}{0.00111111111 + 0.000123} \approx 0.9003.
% \]

% Both values are approximately equal, indicating similar outcomes for the two attack strategies. This means that the $\lambda$-sda detectability value for these attack strategies is approximately $0.9$, as shown in Equation~\ref{eq:attack_ratio_sdad}.
% Although the attacker employs memory in the second strategy, the $\lambda$-sda detectability value remains approximately equal to that of the memoryless strategy. This indicates that even with memory, the attack remains detectable.
% \rmg{Talk about the two strategies.}
Figure~\ref{fig:attack-strategies} illustrates two attack strategies encoded as automata $A_1$, Fig.~\ref{fig:A1}, and $A_2$, Fig.~\ref{fig:A2}.
Since attack strategies act on observations from the plant, the encoding defines attack strategies for string $s\in \lang(A_1)$ such that $s[|s|] \in \Sigma\cup\Sigma_d$, i.e., the last event is the event observation.
For example, strings $b, a\text{ins}(b)\text{del}(b) \in \lang(A_1)$ define attack strategy $A_1(\epsilon,b)$ and $A_1(a\text{ins}(b),b)$.
After observing $s$, the attack strategy is defined for the last observed event in $s$ followed by any possible insertion in the automaton.
For instance, string $ca\in \lang(A_1)$ defines the attack strategy $A_1(c,a) = a\text{ins}(b)$ since $ca\text{ins}(b) \in \lang(A_1)$ and $\nexists e \in \Sigma_i$ such that $ca\text{ins}(b)e \in \lang(A_1)$.
Similarly, for string $aa\in \lang(A_2)$, the strategy is $A_2(a,a) = a\text{ins}(b)$.
On the other hand, for string $aa\text{ins}(b)\text{del}(b)$, the strategy is $A_2(aa\text{ins}(b),b) = \text{del}(b)$ since state $2$ does not have any feasible insertion event.

Both attack strategies $A_1$ and $A_2$ insert event $b$ when the relative distance between the vehicles is $1$.
The key difference between attackers $A_1$ and $A_2$ is when each attacker inserts event $b$. 
Attacker $A_1$ always inserts $b$ whenever the relative distance is $1$, i.e., immediately after observing $a$ the attacker inserts.
On the other hand, attacker $A_2$ inserts $b$ after the second time the relative distance is $1$.
In Fig.~\ref{fig:A2}, the attacker inserts event $b$ only after two events $a$ happened, i.e., the second time the relative distance among the cars is equal to $1$ as in Fig.~\ref{fig:M_n}. 
% is satisfied.
% For instance, consider a sequence starting with $c a$. 
% As shown in Fig.~\ref{fig:plant_G}, this sequence causes the relative distance between the two cars to become $1$. 
% In this situation, $A_1$ immediately responds by inserting event $b$, resulting in the trace $c a \text{ins}(b)$.

% Attacker $A_2$, however, behaves differently. 
% It does not insert event $b$ immediately after observing $a$; instead, it alternates its behavior in an ``on-and-off" pattern. 
% Consider another path, $c a b a$, where, according to Fig.~\ref{fig:plant_G}, the relative distance between the two cars becomes $1$. 
% The first time this condition is met, $A_2$ skips inserting $b$. 
% However, when the relative distance becomes $1$ again, it responds by inserting $b$. 
% As a result, the trace produced by $A_2$ is $c a b a \text{ins}(b)$. 
% This demonstrates that $A_2$'s behavior depends on the history of events, as it decides whether to insert $b$ based on past occurrences.

% Both attack strategies $A_1$ and $A_2$ ultimately lead the system to the critical state, but they operate differently. Attacker $A_1$ acts without memory—it inserts event $b$ whenever the relative distance condition is met, regardless of past events. In contrast, attacker $A_2$ has memory, meaning its decision to insert event $b$ depends on whether the condition has been satisfied before.
\end{example}

\begin{figure}[thpb]
\begin{subfigure}[t]{0.45\columnwidth}
\centering
\includegraphics[width=0.7\columnwidth]{A1.png}
\caption{Attacker Strategy $A_1$}
\label{fig:A1}
\end{subfigure}
\ 
\begin{subfigure}[t]{0.45\columnwidth}
\centering
\includegraphics[width=0.87\columnwidth]{Figs/A2.png}
\caption{Attacker Strategy $A_2$}
\label{fig:A2}
\end{subfigure}
\caption{Two Attack strategies $A_1$ and $A_2$}
\label{fig:attack-strategies}
\vspace{-2em}
\end{figure}

\subsection{Controlled System under Attack}

The sensor attacker disrupts the \emph{nominal} controlled system $R/G$.
A new controlled behavior is generated when the attack function $A$ is placed in the communication channel between the plant and the supervisor.
We define a new supervisor, denoted by $S_A$, that composes $S$ with the attack strategy $A$.
\begin{definition}[Attacked supervisor]
Given supervisor $S$, a set of compromised events $\Sigma_a\subseteq \Sigma$, and an attack strategy $A$. The attacked supervisor is defined for $s\in \Sigma^*$: 
\begin{equation}
S_A(s) = (S\circ \Pi^S\circ A)(s)
\end{equation}
\end{definition}
% Formally,  is the resulting control action, under attack, after string $s$ has been executed by the system\footnote{$\circ$ is the function composition operator.}.
Based on $S_A$ and $G$, the closed-loop system language under attack is defined as $\mathcal{L}(S_A/G)\subseteq \lang(G)$ as in \citep{Lafortune:2021}. 
The system $S_A/G$ denotes the closed-loop system language under attack, or simply the \emph{attacked system}.
Moreover, $S_A/G$ also generates a p-language in the same manner as $R/G$.
% The language $\lang(S_A/G)$ is defined over $\Sigma$; and it generates a p-language in the same manner as $R/G$.

\begin{example}
We compute the controlled language of the attacked systems $S_{A_1}/G$ and $S_{A_2}/G$, for $A_1$ and $A_2$ as shown in Fig.~\ref{fig:attack-strategies}.
Figure~\ref{fig:SA/G} depicts these two controlled systems under attack.
Let us analyze the controlled system under attack strategy $A_1$. 
At the beginning, $A_1$ does not insert any event, i.e., $A_1(\epsilon,\epsilon) = \epsilon$.
In this manner, $S_{A_1}(\epsilon) = S(\epsilon) = \{a,b,c\}$. 
For this reason, events $a,b$, and $c$ are present in the initial of the automaton in Fig.~\ref{fig:SA1/G}.

Next, let us assume that the plant executed event $a$ for which $A_1(\epsilon, a) = a\text{ins}(b)$.
Recall that after event $a$ the distance between the cars is $1$, i.e., state $1$ in $G$ in Fig.~\ref{fig:plant_G}. 
Attack strategy $A_1$ makes the supervisor return to state $2$ with $S_{A_1}(a) = S(ab) = \{a,b,c\}$, i.e., all events are enabled when the plant is in state $1$.
Thus, the controlled system under attack $A_1$ behaves exactly as in $G$ (open-loop), Fig.~\ref{fig:plant_G}.
A similar analysis can be made for the controlled system under attack $A_2$, Fig.~\ref{fig:SA2/G}.
However, the first time the plant reaches state $1$ in $G$, the controlled system behaves normally as in Fig.\ref{fig:M_n}.
After that, the controlled system under attack $A_2$ behaves as in $G$.
% Using attack strategy $A_1$, the attacker inserts $b$ whenever the relative distance condition is met. For instance, while the plant executes the sequence $c a$, the attacker modifies it to $c a \text{ins}(b)$. 
% The supervisor, unaware of this manipulation, observes the sequence as $c a b$ and interprets $b$ as a legitimate event. Based on this observation, the supervisor assumes that the relative distance between the two cars has increased to $2$, even though the actual distance remains $1$. 
% This misinterpretation can lead the supervisor to allow unsafe actions, such as permitting the plant to execute event $a$ from state $1$, which could result in a collision. 
% This demonstrates how the closed-loop behavior $S_{A_1}/G$ is compromised due to the attacker's manipulation of observed events.

% Attack strategy $A_2$, however, uses a different tactic. Unlike $A_1$, it does nothing the first time the relative distance between the two cars becomes $1$. The attacker allows the plant to execute $c a b$ unmodified during this phase. However, the second time the relative distance becomes $1$, $A_2$ inserts $b$, causing the plant’s execution trace $c a b a$ to be modified to $c a b a \text{ins}(b)$. The supervisor observes this manipulated sequence as $c a b a b$ and again assumes that the relative distance between the cars has increased to $2$, despite the actual distance still being $1$. This deception can lead the supervisor to make similarly unsafe decisions, such as allowing event $a$ to execute from state $1$, resulting in a collision. Thus, while $A_2$ operates differently by incorporating memory, it ultimately disrupts the closed-loop behavior $S_{A_2}/G$ in a manner comparable to $A_1$.


\end{example}
\begin{figure}[thpb]
\begin{subfigure}[t]{0.45\columnwidth}
\centering
\includegraphics[width=0.85\columnwidth]{Figs/plant_G.png}
\caption{$S_{A_1}/G$}
\label{fig:SA1/G}
\end{subfigure}
\ 
\begin{subfigure}[t]{0.45\columnwidth}
\centering
\includegraphics[width=0.87\columnwidth]{Figs/SA2-G.png}
\caption{$S_{A_2}/G$}
\label{fig:SA2/G}
\end{subfigure}
\caption{Controlled system under attack}
\label{fig:SA/G}
\vspace{-2em}
\end{figure}

% The sensor attacker disrupts the \emph{nominal} controlled system $R/G$.
% This attacker can modify the event observations the supervisor receives by inserting events that \emph{have not happened} in the plant or by deleting events that \emph{have occurred} in the plant.
% Based on these attack actions, an attacked plant $G_a$ and attacked supervisor $R_a$ are defined to include \emph{every possible attack action} with respect to $\Sigma_a$, i.e., ``all-out" attacker as in \citep{Carvalho:2018}. 
% The controlled system under sensor attacks is then defined by a PDES $R_a/G_a$ similar as $R/G$.
% Considering these attacker actions, we model the controlled system under sensor attacks by modifying the plant $G$ and supervisor $R$ as in \citep{meira-goes:2021synthesis}.

% The attacked plant $G_a$ is a copy of $G$ with more transitions based on compromised sensors $\Sigma_a$.
% Insertion events are introduced to $G_a$ as self-loops with probability $1$ since fictitious insertions do not alter the state of the plant with probability $1$.
% On the other hand, deletion events are defined in $G_a$ with the same probability as their legitimate events because the attacker can only delete an event if this event has been executed in the plant $G$.
% The following insertion and deletion transitions are added to $G_a$ for any $e\in \Sigma_a$.
% \begin{align}
% \delta_{G_a}(x,ins(e)) = x, \quad & P_{G_a}(x,ins(e),x) = 1\label{eq:ins_plant}\\
% \delta_{G_a}(x,del(e)) = y, \quad & P_{G_a}(x,del(e),y) = P_{G}(x,e,y), \text{ if } \delta_G(x,e)!\label{eq:del_plant}
% \end{align}
% Figure~\ref{fig:plant_Ga} shows the $G_a$ for our running example where these new transitions are highlighted in red.
% % For each transition $y = \delta_G(x,e)$ in $G$, the following deletion transitions are added to $G_a$.
% % \begin{align}
% % \end{align}
% % Equation~\ref{eq:del_plant} models attacker deletions with the same probability as their legitimate events.

% % Figure~\ref{fig:attacked-plant} shows the attacked plant for our running example where new events are highlighted in red.
% % For illustration purposes, we omit the transitions in states $0$ and $3$ since they are deadlock states.
% Similarly to the construction of $G_a$, we define the attacked supervisor $R_a$ as a DFA.
% For the supervisor, insertions are observed as legitimate events while deletions do not change the state of the supervisor.
% For any event $e\in \Sigma_a$, the following transitions are added to $R_a$ on top of those already in $R$.
% \begin{align}
% \delta_{R_a}(x,ins(e)) &= \delta_R(x,e) \text{ if } \delta_R(x,e)! \label{eq:ins_sup}\\
% \delta_{R_a}(x,del(e)) &= x \text{ if } \delta_R(x,e)! \label{eq:del_sup}
% \end{align}
% Figure~\ref{fig:supervisor_Ra} shows the attacked supervisor for our running example.
% % Again for illustration purposes, we omit transitions from state~$3$.

% The controlled system under attack is defined as $R_a/G_a := R_a||_p G_a$.
% Fig.~\ref{fig:supervisor_Ra} shows the system $R_a/G_a$ for our running example.
% Note that, the system $R_a/G_a$ includes every possible sensor attack action since the model considers the worst-case attack scenario in which the attacker can attack whenever it is possible.
% Although this model is simple, it is well-suited to the problem we investigating since we want to detect every possible attack strategy. 
% Next, we show how to specialize this general closed-loop behavior under all possible attacks to specific attack strategies.

% % The PDES $M_a$ defines the language of the attacked system in $\Sigma^*_m$, i.e., with the marks of the attacker modifications.
% % We define the set of critical states in $M_a$ as:
% % $$X_{crit,a} = \{x \in X_{M_a}| \exists s\in \lang(M_a) \text{ s.t. } x_{crit} = \delta_{G_a}(x_{0,G_a},s)\}$$


% % \begin{example}\label{ex:optimalattackedsystem}
% % Back to our running example, let the attacker compromise all $adv$ events, i.e., $\Sigma_a = \{r_{adv},m_{adv}\}$.
% % For simplicity, we omit the full model of the attacker.
% % Intuitively, the attacker inserts a fictitious move by $adv$, $ins(m_{adv})$, when the relative distance between the cars is equal to one.
% % After the insertion, $ego$'s supervisor believes that the relative distance is equal to $2$ which allows $ego$ to advance.
% % If $ego$ moves one cell closer to $adv$ immediately after this insertion, the two cars collide, i.e., state $0$ is in $X_{crit,a}$.
% % Figure~\ref{fig:M_a} depicts the attacked system $M_a$.



% \begin{figure}[htbp]
%     \centering
%     % Subfigure for the attacked plant Ga
%     \begin{subfigure}[b]{0.45\textwidth}
%         \centering
%         \includegraphics[width=\textwidth]{Figs/11.png}
%         \caption{Attacked plant $G_a$}
%         \label{fig:plant_Ga}
%     \end{subfigure}
%     \hfill
%     % Subfigure for the attacked supervisor Ra
%     \begin{subfigure}[b]{0.41\textwidth}
%         \centering
%         \includegraphics[width=\textwidth]{Figs/10.png}
%         \caption{Attacked supervisor $R_a$}
%         \label{fig:supervisor_Ra}
%     \end{subfigure}
%     \caption{Illustration of the attacked systems $G_a$ and $R_a$}
%     \label{fig:attacked_systems}
% \end{figure}
% %\\
% %\begin{subfigure}{0.9\columnwidth}
% %\centering
% %\includegraphics[width=0.9\columnwidth]{13.png}
% %\caption{Attacked system $M_a$}
% %\label{fig:M_a}
% %\end{subfigure}
% %\vspace{-1em}
% %\caption{Attacked systems}
% %\label{fig:AS}
% %\end{figure}
% %\vspace{-1em}
% % \end{example}




\subsection{Class of Attack Strategies}

So far, we have used the general definition of attack strategies, Def.~\ref{def:attack_str}.
Herein, we have additional constraints that an attack strategy needs to satisfy.
We say the attack strategy is \emph{complete} if the attack strategy is defined for every new event observation the plant generates.
Moreover, we assume that the attack strategy is \emph{consistent} if it does not insert an event disabled by the supervisor.
Lastly, we consider \emph{successful} attack strategies as the ones that can reach the critical state, i.e., strategies that can cause damage.
\begin{definition}[Complete, Consistent, and Successful Strategies \citep{meira-goes:2021synthesis}]\label{def:complete_attacker}
An attack strategy $A$ is \emph{complete} w.r.t. $G$ and $S$ if for any $s$ in $\lang(S_A/G)$, we have that $A(s)$ is defined.
$A$ is \emph{consistent} if for any $e\in \Sigma$, $s\in \lang(S_A/G)$ such that $se \in\lang(S_A/G)$ with $A(s,e) = t$, then $S(\Pi^S(A(s)t^i))!$ and $t[i+1]\in S(\Pi^S(A(s)t^i))$ for all $i\in [|t|-1]$.
Lastly, $A$ is \emph{successful} if $\exists s\in \lang(S_A/G)$ such that $\delta_G(x_{0,G},s) = x_{crit}$.
We denote by $\Psi_A$ as the set of all complete, consistent, and successful attack strategies.  
\end{definition}
% \rmg{FYI: I removed the part of memoryless for now. I am thinking if it is needed.}

% Although the set $\Psi_A$ already restricts the attack strategies space, there are other potential ways of limiting the attacker strategies.
% Let $\mathcal{A}\subseteq\Psi_A$ denotes the \emph{attack set}.
% For example, we could consider $\attackset = \{A_1, A_2\}$ where $A_1,A_2$ are the two attack strategies as in Fig.~\ref{fig:attack-str}.
% We can generalize for any finite set of attack strategies $\attackset = \{A_1,A_2,\dots, A_n\}$ for given attack strategies $A_i\in \Psi_A$ for $1 \leq i\leq n$.

% For this work, we will consider two large classes of attack strategies: memoryless and finite-memory attack strategies.
% A memoryless attack strategy only depends on the states of $G$ and $R$ and is independent of previous attack modifications.
% Intuitively, the attack strategy is memoryless when the \emph{same} action is selected when the controlled system is in states $x_G\in X_G$ and $x_R\in X_R$.

% \begin{definition}[Memoryless attack strategies]\label{def:mem-att-str}
% Attack strategy $A$ is a \emph{memoryless} attack strategy if for all $s_1,s_2\in \lang(S_A/G)$, $e\in \Sigma$, and $s_1e,s_2e\in \lang(S_A/G)$ such that $\delta_G(x_{0,G},s_1) = \delta_G(x_{0,G},s_2)$ and $\delta_R(x_{0,R},\Pi^S(A(s_1))) = \delta_R(x_{0,R},\Pi^S(A(s_2)))$, then $A(s_1,e) = A(s_2,e)$.
% The set of all memoryless attack strategies is defined as $\attackset_{mem}$
% \end{definition}
% \begin{example}
% % \rmg{Back to examples of $A_1$ and $A_2$. $A_1$ is memoryless whereas $A_2$ is finite-memory.}
% In the context of our running example, we assume an attacker capable of manipulating the event $\Sigma_a = \{b\}$. 
% Figure ~\ref{fig:attack-str} illustrates an attack strategy $A_1$, which represents a deterministic and memoryless attack strategy. 
% In both attack strategies, the attacker targets the event $b$, which allows transitions between states 1 and 2. 
% By deleting $b$ when the system is in state 1, the attacker prevents the system from moving to state 2, a safer state. 
% This forces the system to remain in state 1, where the likelihood of executing the unsafe action $a$ increases. 
% The deletion attack reduces the system's flexibility in choosing alternative paths, thus increasing the probability of reaching the critical state (state 0). 
% In addition, the attacker can insert event $b$ when the system is in state 1, misleading the system into believing it has transitioned to state 2. 
% This deception may result in the system allowing the execution of action $a$, which is unsafe in state 1 but permissible in state 2. 
% Consequently, the system is more likely to transition directly to the critical state (state 0). 
% \end{example}




\section{Probabilistic Intrusion Detection of Sensor Deception Attacks}\label{sect:problem}
.pdfIn this section, we formulate two new problems regarding probabilistic detection of sensor attacks: the verification of $\lambda$-sensor-attack detectability and the optimal $\lambda^*$ for $\lambda$-sensor-attack detectability.
We start by formally describing the detection level of a given string and the detection language of an attack strategy. 
Following these descriptions, we define the notion of $\lambda$-sensor-attack detectability.
Next, we formulate two problems over this definition.
Lastly, we compare the definition of $\lambda$-sensor-attack detectability with the definition of $\varepsilon$-safety as in \citep{meira-goes:2020towards,Fahim2024-wodes}.


% \subsection{Overview}
% \rmg{TALK ABOUT MAP - We are doing a hypothesis test check - see reviewers' note. A figure here to explain it.}

% \textcolor{Brown}{In our system, the sensor attacker can disrupt the normal operations of the nominal system $M_n=R/G$, by inserting fictitious events, deleting actual events, or simply allowing events to proceed as normal. This model uses an attacked plant $G_a$ (see Fig.~\ref{fig:plant_Ga}) and an attacked supervisor $R_a$ (see fig.~\ref{fig:supervisor_Ra}) to represent all possible attack scenarios, creating the system $M_a=R_a/G_a$. The attacked plant $G_a$ is a copy of the original plant but includes extra transitions for these three actions: insertions, deletions, and normal transitions. Insertion actions act as self-loops that don’t alter the state, deletion actions occur with the same probability as legitimate events and normal transitions are carried out as they would in the original system. For a detailed explanation of the definitions and construction of the systems please refer to our previous papers \citep{meira-goes:2020towards, Fahim2024-wodes}.}

% \textcolor{Brown}{In our past studies \citep{meira-goes:2020towards, Fahim2024-wodes}, we utilized a single-attack strategy where the supervisor $R$ and the attacker $A$ were combined using parallel composition to form the system $(R||A)$. The attacker could interfere with the supervisor’s observations by either inserting or deleting events, but it had a fixed mode of operation, allowing it to perform only one of these actions throughout the system’s execution, without dynamically switching between them. In other words, the attacker's actions were predefined and consistent throughout the execution. The strategy did not adapt or change based on the state of the system.}

% \textcolor{Brown}{However, in our current study, the attacker can dynamically choose among three actions: insertion, deletion, and normal transitions. Our model introduces a set of strategies \(\{A_1, A_2, \ldots, A_n\}\), allowing the attacker to flexibly switch between different actions. This means the attacker is no longer limited to a single, static strategy but can adapt its actions based on the system's state. This adaptability allows the attacker to be more stealthy and dynamic, choosing in real-time which action—whether inserting, deleting, or allowing normal transitions—would be most effective for evading detection or causing disruption.
% }

% \textcolor{Brown}{In the study \citep{Fahim2024-wodes}, we introduced the concept of an \textit{intrusion detection value} (\textit{int}) to enhance the robustness of our system against sensor deception attacks. The primary goal of \textit{int} is to quantify the system’s ability to distinguish between legitimate and malicious behavior with a high degree of confidence.
% The \textit{intrusion detection value} is defined as:
% \begin{equation}
%     \text{int} := \inf_{s \in L_{det}} \frac{L_p(M_a)(s)}{L_p(M_n)(\Pi_R(s)) + L_p(M_a)(s)},
%     \label{eq:attack_ratio}
% \end{equation}
% Where:
% \begin{itemize}
%     \item $L_p(M_a)(s)$ is the probability that an observed string $s$ is generated by the attacked system ($M_a$).
%     \item $L_p(M_n)(\Pi_R(s))$ is the probability that the same string originates from the nominal system ($M_n$).
% \end{itemize}
% The core idea behind the \textit{int} metric is to provide a rigorous threshold for detecting whether a system has been compromised. Specifically, a system is considered $\epsilon$-safe if the computed intrusion detection value remains above a pre-defined threshold $\epsilon$. For a given string, a higher \textit{int} value indicates that the string is more likely to come from the attacker rather than the nominal system.
% }

% \textcolor{Brown}{Unlike traditional methods that assume a static attack strategy, our approach, termed \textit{``$\lambda$-sensor-deception-attack-detectable,''} extends the concept of \textit{int} to handle dynamic attackers. These attackers can adapt their strategies in real-time, making detection more challenging. To account for this increased complexity, we define a refined version of the intrusion detection value as:
% \begin{equation}
%     \text{dtc} := \inf_{\text{attackers}} \inf_{s \in L_{det}} \frac{L_p(M_a)(s)}{L_p(M_n)(\Pi_R(s)) + L_p(M_a)(s)}.
%     \label{eq:attack_ratio_sdad}
% \end{equation}}

% \textcolor{Brown}{This formulation introduces a double infimum, ensuring that the system remains resilient even against the most adaptive and stealthy attackers. By considering the worst-case scenario among all potential attack strategies, the system guarantees safety, with $\lambda$ representing the minimum threshold required to confidently detect diverse attack strategies.}

% \textcolor{Brown}{In essence, the new \textit{dtc} metric not only measures the likelihood of detecting an attack but also provides a robust framework for assessing system safety in environments where attackers can dynamically alter their behavior. This adaptability ensures that our system can maintain its integrity even in the presence of highly sophisticated adversaries.}


% Intuitively, the definition of $\lambda$-sensor-deception-attack-detectable, $\lambda$-sdad for short,  compares the probability of executing strings in a nominal system versus a possible set of attacked systems.
% Using the probabilistic information, $\lambda$-sdad informs when strings with the same projection, called \emph{ambiguous strings}, are more likely to have been executed by the attacked system than the nominal system.
% % We use an example to concretely demonstrate the definition of $\epsilon$-safety.

% \begin{example}
% \textcolor{Brown}{In our running example, we have constructed the nominal system $M_n$ in Fig~\ref{Mn} and provided a partial view of the attacked system $M_a$ in Fig.~\ref{fig:M_a}.
% % We investigate if it is possible to detect and prevent the attacker from reaching the critical state $0$ in $M_a$.
% The string $ab_{ins}a$ reaches the critical state in $M_a$.
% Since we want to prevent the system before reaching the critical state, we need to detect the attacker with a prefix of strings that reach the critical states, e.g., string $s = ab_{ins}a$.
% We denote strings such as $s$ as \emph{detection strings}.
% These strings represent the latest time for the ID module to identify the attacker and prevent it by disabling controllable events.}

% \textcolor{Brown}{If string $s$ is executed, the ID observes the projected string $\Pi_R(s) = ab_{ins}$.
% The detection mechanism must decide after observing $\Pi_R(s)$ if it originates from the nominal system $M_n$ or the attacked system $M_a$.
% Note that $\Pi_R(s)$ belongs to the $\lang(M_n)$; thus, traditional ID systems cannot detect the attacker \citep{Carvalho:2018,Lima:2019}.
% In $\epsilon$-safety, we compare the probability of executing $s$ in $M_a$ with the probability of executing $\Pi_R(s)$ in $M_n$.}
% % In other words, when the detection module observes $\Pi_R(s)$, $\epsilon$-safety checks if it is more likely to have been generated by $M_a$ than $M_n$. }
% \end{example}
% %\vspace{-1em}
% %\begin{figure}[thpb]
% %\centering
% %\includegraphics[width=0.55\columnwidth]{det-lang.pdf}
% %\caption{Detection language diagram}
% %\label{fig:detection_language}
% %\end{figure}


% % \begin{figure*}[thpb]
% % \centering
% % \begin{subfigure}[b]{0.75\textwidth}
% %     \includegraphics[width=\textwidth]{Figs/15.png}
% %     \caption{Attacked system $M_a$ with transitions leading to unsafe state.}
% %     \label{fig:M_a}
% % \end{subfigure}
% % \hfill
% % \begin{subfigure}[b]{0.75\textwidth}
% %     \includegraphics[width=\textwidth]{Figs/16.png} % Replace with the second image path
% %     \caption{Attacked system $M_a$ where the attacker uses memory.}
% %     \label{fig:M_a_alt}
% % \end{subfigure}
% % \caption{Comparison of the attacked system $M_a$ with and without attacker memory. (a) Shows the primary path to the unsafe state. (b) The attacker employs memory, resulting in an alternate path to the unsafe state.}
% % \label{fig:attack-str}
% % \end{figure*}


\subsection{Detection Value}

The attack detection problem is to determine if an observed behavior $s\in \lang(G)$ is generated by the nominal system $S/G$ or by an attacked system $S_A/G$.
Usually, a detection problem is described as a hypothesis-testing problem \citep{poor2013introduction}.
In our case, the null hypothesis $H_0$ is defined by the nominal system $S/G$ whereas the alternative hypothesis $H_1$ is described by an attack system $S_A/G$.

To identify from which system an observation is generated, we compare the two systems using their probabilistic language.  
Inspired by the \emph{maximum a posterior probability}, we calculate the likelihood of string $s\in \lang(G)$ by directly comparing the probability between $L_p(S/G)$ and $L_p(S_A/G)$. 
We define the \emph{detection level} of a string.

\begin{definition}[Detection level]
Let $s\in \lang(S_A/G)$, the detection level of $s$ with respect to $G$, $S$, and $A$ is:
\begin{equation}\label{eq:detection_level}
det(s) = 
\frac{L_p(S_A/G)\bigl(s \bigr)}{L_p(S_A/G)\bigl(s \bigr)+L_p(S/G)\bigl(\Pi^S(A(s))\bigr)}
\end{equation}
\end{definition}

Intuitively, $det(s)$ informs a ``detection value" of the attack strategy generating string $s\in S_A/G$.
It characterizes the likelihood of $s$ being generated by $S_A/G$ compared to $\Pi^S(A(s))$ being generated in $S/G$.
% In the numerator, we sum the probabilities of generating string $s$ in $S_A/G$, $L_p(S_A/G)(s)$, with the probability of generating string $\Pi^S(A(s))$ in $S/G$, $L_p(S/G)(\Pi^S(A(s)))$.
To generate $s \in \lang(S_A/G)$, the attack strategy feeds the supervisor with observation $\Pi^S(A(s))$.
For this reason, we compare the probabilities of generating $s$ in $S_A/G$ versus $\Pi^S(A(s))$ in $S/G$.
% In other words, the detector will observe string $\Pi^S(A(s))$ and it has to decide if it was a legitimate string from $S/G$ or if $s$ was generated by attack system $S_A/G$.


Note that if $\Pi^S(A(s))\notin \lang(S/G)$, then the value of $det(s) = 1$, i.e., the attack strategy reveals the attacker when generating $s\in\lang(S_A/G)$.
These are the \emph{only} types of attacks that logical detection systems can detect, e.g., \citep{Carvalho:2018, Lima:2019, lin2024diagnosability}.
On the other hand, the detection value is still useful when $\Pi^S(A(s))\in \lang(S/G)$.
For instance, if the attack strategy $A$ modifies the closed-loop system such that $L_p(S_A/G)(s)>L_p(S/G)(\Pi^S(A(s))$, then $det(s)>0.5$, i.e., it is more likely that the observation is coming from an attacked system instead the nominal.

\begin{example}
Let us characterize the detection value for a string in our running example with attack strategy $A_1$.
We select string $s = ca\in \lang(S_{A_1}/G)$ that has probability $L_p(S_{A_1}/G)(ca) = 0.8\times 0.1 = 0.08$ as shown in Fig.~\ref{fig:SA1/G}.
Attack strategy $A_1$ modifies $s$ to $A_1(s) = cains(b)$ as described by Fig.~\ref{fig:A1}.
The supervisor observes string $\Pi^S(cains(b)) = cab$ that has probability $L_p(S/G)(cab) = 0.8\times 0.1 \times 0.111\dots = 0.0088\cdots$.
The detection value computes the likelihood of generating string $ca$ in $S_{A_1}/G$ versus $cab$ in $S/G$.
In this case, the detection value is $det(ca) = 0.9$.
After observing $cab$, it is $90\%$ more likely that $ca$ was executed in $S_{A_1}/G$ compared to $cab$ being executed in $S/G$.
\end{example}



% Based on this discussion, we define the detection language $L^A_{det}\subseteq \lang(S_A/G)$ as the strings in which the detector must make a decision otherwise it is too late. 
% Recall that we assume that $x_{crit}$ is only reached via controllable events, i.e., $x \in X_Q$ such that $\delta_G(x,e) = x_{crit}$ implies $e\in \Sigma_c$.
% First, we define the \emph{critical language of $A$} by the strings in $\lang(S_A/G)$ that reach the critical state.
% \begin{equation}
% L_{crit}^A = \{s\in \lang(S_A/G)\mid \delta_G(x_{0,G},s) = x_{crit}\}
% \end{equation}
% As mentioned above, it is too late to detect the attack with strings in $L_{crit}^A$, i.e., the critical state has been reached.

% % First, we define $X_{det} = \{x\in X_G\setminus\{x_{crit}\}\mid \exists e\in \Sigma_c.\ \delta_G(x,e) = x_{crit}\}$. 
% % These states define 

\subsection{Detection Language}
Although the detection value is defined for every string in $s\in \lang(S_A/G)$, only some can drive the controlled system to a critical state. 
Recall that an attack strategy is only successful if it generates a string that reaches the critical state $x_{crit}$.
Therefore, these ``successful" strings must be detected by the attack detector.
On top of detecting these strings, the attack detection must identify the attack \emph{before} the system reaches a critical state. 
In other words, we can only mitigate attacks if the attack detection detects them before it is too late, i.e., before the critical state is reached.

Based on this discussion, we define the detection language $L^A_{det}\subseteq \lang(S_A/G)$ as the strings in which the detector must make a decision otherwise it is too late. 
Recall that we assume that $x_{crit}$ is only reached via controllable events, i.e., $x \in X_G$ such that $\delta_G(x,e) = x_{crit}$ implies $e\in \Sigma_c$.
First, we define the \emph{critical language of $A$} by the strings in $\lang(S_A/G)$ that reach the critical state.
\begin{equation}
L_{crit}^A = \{s\in \lang(S_A/G)\mid \delta_G(x_{0,G},s) = x_{crit}\}
\end{equation}
As mentioned above, it is too late to detect the attack with strings in $L_{crit}^A$, i.e., the critical state has been reached.
% Based on this discussion, we define the detection language $L_{det}^A\subseteq \lang(S_A/G)$ of attack strategy $A$. 
Given plant $G$, supervisor $S$, and attack strategy $A$, we define the detection language as 
\begin{equation}\label{eq:det_lang}
L_{det}^A = \{s\in \lang(S_A/G)\mid (\exists e\in \Sigma_c.\ se\in L_{crit}^A)\wedge (\forall \sigma\in \Sigma_c,\ i<|s|.\ s^i\sigma \notin L_{crit}^A)\}
\end{equation}

A string $s$ in $L^A_{det}$ can reach the critical state with a controllable event.
Moreover, no prefix of $s$ can reach the critical state with any controllable event.
In other words, string $s$ is the shortest string to be one controllable event away from a critical state. 

\begin{example}
Let us return to our running example with attack strategy $A_1$ to investigate its detection language.
The detection language is given by $L_{det}^{A_1} = \{a, ca, cca, \dots\}$.
Let $s = a$, then $A(s) = a ins(b)$, i.e., the attack inserts event $b$ immediately after $a$ occurs. 
This insertion makes the supervisor return to state $2$ while the plant remains in state $1$ as in Figs.~\ref{fig:plant_G} and \ref{fig:sup_R}.
Since the supervisor enables event $a$ in state $2$, if the plant executes $a$, then the critical state $0$ is reached.
Therefore, the detection system must decide after observing $\Pi^S(a ins(b)) = ab$ if it should disable event $a$.
Note that since observation $ab$ belongs to $\lang(S/G)$, it cannot be detected by logical detectors.
\end{example}


% To formally define $\epsilon$-safety, we define the detection language $L_{det}\subseteq \lang(M_a)$, i.e., strings in which the ID must decide.
% Figure~\ref{fig:detection_language} provides intuition behind the detection language.
% The unsafe region includes states in $M_a$ that can uncontrollably reach the critical state.
% In other words, these are states where it can be too late to prevent the attacker.
% \begin{align}
% Uns := &\{x \in X_{M_a}\mid \exists s\in \Sigma_m^*\text{ s.t. } \nonumber \\& (\delta_{M_a}(x,s)\in X_{crit,a})\wedge(\Pi_G(s)\in \Sigma_{uc}^*)\label{eq:unsafe-region}\}
% \end{align}
% In the attacked system $M_a$ in Fig.~\ref{fig:M_a}, the set of unsafe states only contains the critical state, $Uns = \{0\}$.
% Detection states can reach the unsafe region $Uns$ with one controllable transition.
% For example, state $1'$ in $M_a$, Fig~\ref{fig:M_a}, is a detection state since it can reach state $0$ via controllable event $m_{ego}$.
% Detection states are the last states where the ID can prevent the attacker from reaching the critical state.
% \begin{align}
% X_{det} := &\{x \in X_{M_a}\setminus Uns\mid \exists e\in (\Sigma_d\cap\Sigma)\text{ s.t. } \nonumber \\& (\delta_{M_a}(x,e)\in Uns) \wedge(\Pi_G(e)\in \Sigma_c)\label{eq:detection-state}\}
% \end{align}

% Finally, the detection language includes the shortest ambiguous strings that reach detection states:
% \begin{align}
% L_{det} := &\{s = e_0\dots e_n \in \lang(M_a)\mid (\Pi_R(s)\in \lang(M_n))\wedge\nonumber\\&(\delta_{M_a}(x_{0,M_a},s)\in X_{det})\wedge\nonumber\\&(\delta_{M_a}(x_{0,M_a},e_0\dots e_i)\notin X_{det},\ i<|s|)\}\label{eq:det-lang}
% \end{align}
% The three conditions in Eq.~\ref{eq:det-lang} specify in order: (1) ambiguous strings; (2) detection state reached; and (3) shortest strings.
% The detection language enforces when the detector must make a decision even though $L_{det}$ contains ambiguous strings.
% This language violates the GF-safe diagnosability as in \citep{Carvalho:2018} because $L_{det}$ only contains ambiguous strings whereas GF-safe diagnosability searches to disambiguate attack strings against nominal behavior.
% The detection language $L_{det}$ for $M_a$ in Fig.~\ref{fig:M_a} is defined by the marked language in the DFA shown in Fig.~\ref{fig:verifier} where a marked state is depicted by the black state edge in Fig.~\ref{fig:verifier}.
% We show how to construct this DFA in Sect.~\ref{sect:solution}.


\subsection{Probabilistic Sensor Detectability}
Based on the definitions of detection value, $det(s)$, and detection language, $L_{det}^A$, we define $\lambda$-sa detectability as follows:
% \begin{figure*}[thpb]
% \centering
% \includegraphics[width=0.6\textwidth]{overview-alg.pdf}
% \caption{Overview of our approach to verify $\epsilon$-safety. 
% }
% \label{fig:overview_solution}
% \end{figure*}

\begin{definition}[$\lambda$-sa detectable]\label{def:lambda-sa-det}
Given plant $G$, supervisor $S$, a set of compromised events $\Sigma_a\subseteq \Sigma$, and value $\lambda \in (0.5,1]$, the controlled system is \emph{$\lambda$-sensor-attack detectable}, or simply $\lambda$-sa, if
% Given a nominal controlled system $M_n$, a set of possibly compromised sensors $E_a\subseteq E_o$, a set of sensor attacker strategies $Att$, and $\lambda\in [0.5,1]$, the nominal system $M_n$ is \emph{$\lambda$-sensor-attack}, $\lambda$-sa detectable with respect to $E_a$, $Att$, and $\lambda$ if
\begin{equation}
dtc:=\inf_{A\in \Psi_A}\ \inf_{s\in L^A_{det}} det(s)\geq \lambda\label{eq:likelihood}
\end{equation}
\end{definition}

Intuitively, the controlled system is $\lambda$-sa if \emph{every} complete, consistent, and successful attack strategy significantly modifies the probability of the nominal controlled system.
The parameter $\lambda$ gives the confidence level of detection strings being more likely to be generated by the attacked systems.
For example, $0.9$-sa detectable means that detection strings for all attack strategies are at least $90\%$ more likely to have been generated in $S_A/G$ compared to their observation in $S/G$.

% Equation~\ref{eq:likelihood} computes the probability of executing a string in $M_a$ divided by the total probability of executing this string in $M_n$ and $M_a$.
% The $\epsilon$ parameter must be above $0.5$ since it provides the confidence of strings being more likely to be generated by the attacked systems.
% For string $s = m_{ego}ins(m_{adv})$ in Fig.~\ref{fig:M_a}, we have that $L_p(M_a)(s) = 0.1$.
% Similarly, we use $M_n$ to compute $L_p(M_n)(\Pi_R(s)) = L_p(M_n)(m_{ego}m_{adv}) = 0.011\dots$.
% Computing the ratio in Eq.~\ref{eq:likelihood}, we have $0.1/0.111\dots = 0.9$.
% When observing $\Pi_R(s)$, it is $90\%$ more likely for $s$ to be executed by $M_a$ than by $M_n$.

% \subsection{Verification and Synthesis problems}

Based on the definition of $\lambda$-sa detectability, we formulate two problems.
First, we define a verification problem to check if a controlled system is $\lambda$-sa detectable.

\begin{problem}[Verification of $\lambda$-sa]\label{prob:ver-lsa}
Given plant $G$, supervisor $R$, a set of compromised events $\Sigma_a\subseteq \Sigma$, and $\lambda \in (0.5,1]$, verify if the controlled system is $\lambda$-sa detectable
\[dtc \geq\lambda\]
\end{problem} 

Problem~\ref{prob:ver-lsa} verifies if the controlled system is $\lambda$-sa detectable for a given $\lambda$ value.
A natural question to ask is if there exists a maximum $\lambda$ value such that the controlled system is $\lambda$-sa detectable.
Formally, the problem is posed as follows.
\begin{problem}[Maximum $\lambda$-sa]\label{prob:optimal-lsa}
Given plant $G$, supervisor $R$, and set of compromised events $\Sigma_a\subseteq \Sigma$, find, if it exists, the maximum $\lambda^*$ such that the controlled system is $\lambda^*$-sa: 
$$\lambda^* := \sup\ \{\lambda\in (0.5,1] \mid S/G \text{ is } \lambda\text{-sa detectable}\}$$
\end{problem}

\begin{remark}
In  \citep{meira-goes:2020towards,Fahim2024-wodes}, the problem of $\varepsilon$-safety is defined.
The main difference between $\varepsilon$-safety and $\lambda$-sa detectability is the $\inf_{A\in \Psi_A}$ in Eq.~\ref{eq:detection_level}.
The $\varepsilon$-safety definition only considers \emph{one} attack strategy whereas $\lambda$-sa considers all possible complete, consistent, and successful attack strategies.
The $\lambda$-sa detectability reduces to $\varepsilon$-safety when $|\Psi_A| = 1$, i.e., a single attack strategy.
\end{remark}
% Problem~\ref{prob:verification_eps_safe} simply verifies if a given nominal system is $\epsilon$-safe with respect to a given attacked system.
% On the other hand, Problem~\ref{prob:optimal_eps_safe} searches for the largest $\epsilon^*$ such that $M_n$ is $\epsilon^*$-safe for a given attacked system.
% We addressed the error in \citep{meira-goes:2020towards} by using the $\max$ function.



\section{Solution Verification of $\lambda$-sa detectability}\label{sect:solution}
.pdf% An overview of this reduction is shown in Figure~\ref{fig:overview_solution}.
Figure~\ref{fig:overview_solution} provides an overview of our solution approach. 
First, we construct an attack system that encompasses all possible sensor attacks using the plant model $G$, supervisor $R$, and compromised event set $\Sigma_a$.
Intuitively, we construct PDES $M_n$ containing the nominal controlled behavior and $M_a$ containing all possible attacked behavior.
Next, we constructed a weighted verifier consisting of a DFA $V$ and weight function $w$.
The DFA $V$ marks the language in $L^A_{det}$ for all attacks in $\Psi_A$.
At the same time, $V$ combines the information of string executions in nominal and attack systems.
The function $w$ captures the probability ratio between executing transitions in the nominal controlled system and an attacked system.

\begin{figure}[h]
    \centering
    \includegraphics[width=1\textwidth]{Figs/overview-alg.pdf} 
    \caption{Overview on solution algorithm} 
    \label{fig:overview_solution}
\end{figure}
% The first step in this reduction is to construct an automaton that exactly marks the language in $L_{det}$.
% Inspired by the verifier automaton \citep{yoo2002polynomial}, the construction of this automaton has been first described in \citep{meira-goes:2020towards}.
% Intuitively, a verifier automaton $T$ is defined based on the nominal system $M_n$ and the attacked system $M_a$.
% The next step is to compute the ratio between an event being executed in $M_a$ versus the event in $M_n$ for every transition in $T$.
% Instead of directly computing the probability of executing a string in $L_{det}$, we compute the ratio of executing this string in $M_a$ over its projected string in $M_n$.
% Due to limited space, proofs are not omitted.
% refer to the corresponding references for more information.

Based on the weighted verifier, we pose a shortest path problem to identify the shortest path to reach a marked state in $V$, i.e., executing a string in a detection language $L^A_{det}$.
The shortest path problem outputs either: (1) a message that $V$ has ``negative cycles" or (2) a vector $\mathbf{sp}$ with the shortest path values from the initial state to each marked state in $V$.
Based on this output, we calculate the detection value $dtc$.
To make our approach concrete, we describe each step in detail using our running example.
Our goal is to verify if our running example is $0.9$-safe.



% \subsection{Overview}

\subsection{Construction of nominal system}
The nominal controlled system is built as described in Section~\ref{sect:preliminaries} using the probabilistic parallel composition $||_p$.
Formally, the nominal system is defined by $M_n = R||_pG$.
Figure~\ref{fig:M_n} depicts the nominal system for our running example.

\subsection{Construction of attacked system}\label{sub:controlled-system}
The sensor attacker disrupts the \emph{nominal} controlled system $R/G$.
Based on the attack actions, we construct the attacked plant $G_a$ and the attacked supervisor $R_a$ to include \emph{every possible attack action} with respect to $\Sigma_a$ as in \citep{meira-goes:2021synthesis}. 
In this manner, we can obtain a structure that contains all possible controlled systems under sensor attacks by composing $R_a$ and $G_a$.
% Considering these attacker actions, we model the controlled system under sensor attacks by modifying the plant $G$ and supervisor $R$ .

The attacked plant $G_a$ is a copy of $G$ with more transitions based on compromised sensors $\Sigma_a$.
Insertion events are introduced to $G_a$ as self-loops with probability $1$ since fictitious insertions do not alter the state of the plant with probability $1$.
On the other hand, deletion events are defined in $G_a$ with the same probability as their legitimate events because the attacker can only delete an event if this event has been executed in the plant $G$.
The following insertion and deletion transitions are added to $G_a$ for any $e\in \Sigma_a$.
\begin{align}
\delta_{G_a}(x,ins(e)) = x, \quad & P_{G_a}(x,ins(e),x) = 1\label{eq:ins_plant}\\
\delta_{G_a}(x,del(e)) = y, \quad & P_{G_a}(x,del(e),y) = P_{G}(x,e,y), \text{ if } \delta_G(x,e)!\label{eq:del_plant}
\end{align}
Figure~\ref{fig:Ga} shows the $G_a$ for our running example, the new transitions are highlighted in red.
Insertions in deadlock states are omitted for illustration purposes.

\begin{figure}[thpb]
\begin{subfigure}[t]{0.45\columnwidth}
\centering
\includegraphics[width=1\columnwidth]{Figs/Ga.png}
\caption{$G_a$}
\label{fig:Ga}
\end{subfigure}
\ 
\begin{subfigure}[t]{0.45\columnwidth}
\centering
\includegraphics[width=0.71\columnwidth]{Figs/Ra.png}
\caption{$R_a$}
\label{fig:Ra}
\end{subfigure}
\\
\begin{subfigure}[t]{1\columnwidth}
\centering
\includegraphics[width=0.55\columnwidth]{Figs/Ma.png}
\caption{$M_a$}
\label{fig:Ma}
\end{subfigure}
\caption{Attacked plant, supervisor, and system}
\label{fig:attacked-models}
\vspace{-2em}
\end{figure}

% For each transition $y = \delta_G(x,e)$ in $G$, the following deletion transitions are added to $G_a$.
% \begin{align}
% \end{align}
% Equation~\ref{eq:del_plant} models attacker deletions with the same probability as their legitimate events.

% Figure~\ref{fig:attacked-plant} shows the attacked plant for our running example where new events are highlighted in red.
% For illustration purposes, we omit the transitions in states $0$ and $3$ since they are deadlock states.
Similar to the construction of $G_a$, we define the attacked supervisor $R_a$ as a DFA.
For the supervisor, insertions are observed as legitimate events while deletions do not change the state of the supervisor.
For any event $e\in \Sigma_a$, the following transitions are added to $R_a$ on top of those already in $R$.
\begin{align}
\delta_{R_a}(x,ins(e)) &= \delta_R(x,e) \text{ if } \delta_R(x,e)! \label{eq:ins_sup}\\
\delta_{R_a}(x,del(e)) &= x \text{ if } \delta_R(x,e)! \label{eq:del_sup}
\end{align}
Figure~\ref{fig:Ra} shows the attacked supervisor for our running example.
% Again for illustration purposes, we omit transitions from state~$3$.

All possible controlled systems under attack are represented by $M_a = R_a||_p G_a$.
Figure~\ref{fig:Ma} shows the system $M_a$ for our running example.
Recall that states in $M_a$ are of the format $(x_{R},x_{G})$ for $x_R\in X_{R_a}$ and $x_G\in X_{G_a}$.
Note that the attacker can cause a mismatch between these states which was not possible in the nominal $M_n$ as in Fig.~\ref{fig:M_n}.
% Note that, the system $R_a/G_a$ includes every possible sensor attack action since the model considers the worst-case attack scenario in which the attacker can attack whenever it is possible.
Although this model is simple, it is well-suited to the problem we are investigating since we want to detect every possible attack strategy. 
Next, we have two propositions that link the language of $M_a$ with languages generated by attacked systems $\lang(S_A/G)$ for a given $A$.
The first proposition shows that a complete and consistent attack strategy generates a behavior in $M_a$.
\begin{proposition}\label{prop:Sa-Ma}
Let attack strategy $A$ be complete and consistent.
For every $s\in \lang(S_A/G)$, then $A(s) \in \lang(M_a)$.
\end{proposition}
\begin{proof}
It follows by the construction of $G_a$, $R_a$, and by $A$ being a complete and consistent attack strategy.
\end{proof}

Next, every string in $M_a$ can be generated by a complete and consistent attack strategy.
\begin{proposition}\label{prop:Ma-Sa}
For every $s\in \lang(M_a)$, there exists $A$ complete and consistent such that $\Pi^G(s)\in \lang(S_A/G)$.
\end{proposition}
\begin{proof}
We start by showing $\Rightarrow$ by constructing an attack strategy $A$ that generates $s\in \lang(M_a)$.
First, we can break $s$ into $0\leq k\leq|s|$ substrings such that $s = t_1\dots t_k$.
Moreover, each substring satisfies: $t_1\in \Sigma_i^*$ and $t_i\in (\Sigma\cup\Sigma_d)\Sigma_i^*$ for $1<i\leq k$.

Intuitively, we are breaking $s$ into $k$ substrings to be generated by the attack strategy $A$.
Recall that $A$ needs to satisfy the conditions in Def.~\ref{def:attack_str}.
Attack $A$ will output each of these $t_i$, e.g., $A(\epsilon,\epsilon) = t_1$, and $A(t_1,\mask(t_2[1])) = t_2$.
We construct $A$ for $s$ as follows:
\begin{align*}
A(\epsilon,\epsilon) &= t_1\\
A(t_1\dots t_j,\mask(t_{j+1}[1])) &= t_{j+1} \forall\ 1<j\leq k-1
\end{align*}
For other strings $t\in\Sigma_m^*\setminus\{\epsilon\}$ and event $e\in\Sigma$, the attack strategy is $A(t,e)=e$.
Attack strategy $A$ is complete and consistent by construction.

Now, we show that $\Pi^G(s)\in \lang(S_A/G)$ by showing that $\Pi^G(t_1\dots t_j)\in \lang(S_A/G)$.
We show this by recursively showing that $te\in \lang(G)$, $t\in \lang(S_A/G)$ and $e\in S_A(t)$ which implies that $te\in \lang(S_A/G)$.  
By definition of $\lang(S_A/G)$, $\epsilon = \Pi^G(t_1)\in \lang(S_A/G)$.

By construction of $t_1$ and $t_2$, we have $\Pi^G(t_1t_2) = \mask(t_2[1])$.
Now by construction of $R_a$, it follows that $x_R = \delta_{R_a}(x_{0,R_a},t_1) = \delta_R(x_{0,R},\Pi^S(t_1))$. 
Since $t_1t_2 \in \lang(M_a)$ and the definition of $R_a||_pG_a$, we have that $t_2[1]\in \Gamma_{R_a}(x_R)$ and $\mask(t_2[1])\in \Gamma_{R}(x_R)$.
Thus, the event $\mask(t_2[1])$ is allowed by $S(A(\epsilon,t_1)) = S_A(\epsilon)$.
As $t_1t_2[1]\in \lang(G_a)$, then $\Pi^G(t_1t_2) = \mask(t_2[1])\in \lang(G)$ by construction of $G_a$.
In summary, we have $\epsilon\in \lang(S_A/G)$, $\Pi^G(t_1t_2) = t_2[1]\in \lang(G)$, and $t_2[1]\in S_A(t_1)$, which implies that $\Pi^G(t_1t_2)\in \lang(S_A/G)$.
By similar recursive arguments, we can show that $\Pi^G(t_1\dots t_j) = t_2[1]\dots t_j[1]\in \lang(S_A/G)$ for any $1<j\leq k$.
\end{proof}


\subsection{Constructing the Weighted Verifier}
Once $M_a = R_a||_p G_a$ is constructed, we identify all possible languages $L_{det}^A$ for any $A\in \Pi_A$.
Recall that $L_{det}^A$ is defined by the shortest strings that are one controllable event away from a critical state.
Based on $M_a$, we define the detection states as follows:
\begin{equation}
X_{det} = \{(x_R,x_G)\in X_{M_a}\mid \exists e\in (\Sigma_d\cap\Sigma)\text{ s.t. } (e \in \Gamma_{R_a}(x_R))\wedge (\delta_{G_a}(x_{G},e)=x_{crit}\}
\end{equation}
In the $M_a$ in Fig.~\ref{fig:Ma}, the detection state is $(2,1)$ since the critical state $(1,0)$ is reached via controllable event $a$. 
The detection states $X_{det}$ are related to $L_{det}^A$ since they are states one controllable event away from a critical state.

% This relationship is captured by the following proposition.




Next, we need a structure where we can directly compare string executions in $S_A/G$ versus $S/G$.
Inspired by the verifier automaton in \citep{yoo2002polynomial}, we define the weighted verifier, DFA $V$, and weight function $w$.
The verifier automaton $V$ marks the strings in $L^A_{det}$ for any $A\in \Psi_A$.
Moreover, weights $w$ contain the probability information of executing transitions in attacked systems $S_A/G$ and nominal system $S/G$.
We start by constructing the verifier $V$ similarly to the steps described in \citep{meira-goes:2020towards}. 
% Let $M_n = R||_p G$ denote the nominal system and $M_a = R_a||_p G_a$ denote the attacked systems.
% in $\lang(M_n)\cap\Pi^S(\lang(M_a))$.
% This language captures strings in the nominal system that can be also generated by the attacked system, i.e., ambiguous strings.
% Moreover, in verifier $V$, we mark detection strings $L_{det}$.

\begin{definition}\label{def:verifier}
Given $M_n$, $M_a$, and the detection states $X_{det}$, we define verifier $V$ as: 
(1) $X_{V}\subseteq X_{R}\times X_G \times X_{R_a}\times X_{G_a}$; (2) $x_{0,V} = (x_{0,R},x_{0,G},x_{0,R_a},x_{0,G_a})$; (3) $\delta_{V}((x_1,x_2,x_3,x_4),e) = (y_1,y_2,y_3,y_4)$ if $\delta_{R,G}((x_1,x_2),\Pi^S(e)) = (y_1,y_2)$ and $\delta_{R_a,G_a}((x_3,x_4),e) = (y_3,y_4)$  for $e\in \Sigma_m$, $x_1,y_1\in X_R$, $x_2,y_2\in X_G$, $x_3,y_3\in X_{R_a}$, and $x_4,y_4\in X_{G_a}$ with $(x_3,x_4)\not\in X_{det}$, otherwise is undefined; and (4) $X_{m,V} = \{(x_1,x_2,x_3,x_4) \mid \ (x_3,x_4)\in X_{det}\}$.
% \begin{enumerate}
%     \item $X_{V}\subseteq X_{M_n}\times X_{M_a}$
%     \item $x_{0,V} = (x_{0,M_n},x_{0,M_a})$
%     \item $\delta_{V}((x_1,x_2),e) = (y_1,y_2)$ if $\delta_{M_n}(x_1,\Pi^S(e)) = y_1$ and $\delta_{M_a}(x_2,e) = y_2$ for $e\in \Sigma_m$ and $(x_1,x_2),(y_1,y_2)\in X_{M_n}\times X_{M_a}$ with $x_2\not\in X_{det}$;
%     \item $X_{m,V} = \{(x_1,x_2) \mid \ x_2\in X_{det}\}$.
% \end{enumerate}
\end{definition}

% In \citep{meira-goes:2020towards}, it was shown that $V$ marks  $L_{det}$.
With abuse of notation, we only describe $V$ by its accessible and co-accessible parts, i.e., the $Trim(V)$ operator as in \citep{Lafortune:2021} is applied after Def.~\ref{def:verifier}.
Figure~\ref{fig:verifier} depicts the verifier $V$ constructed based on Def.~\ref{fig:verifier}.
To construct this automaton, we start from the initial state $x_{0,V}$ and perform a reachability analysis to obtain the next states via $\delta_V$.
For example from state $(2,2,2,2)$ and event $a$, state $(1,1,1,1)$ is reached.
In this scenario, $ego$ moves one cell closer to $adv$ in both systems.
From state $(1,1,1,1)$ and event $ins(b)$, state $(2,2,2,1)$ is reached.
In this case, the nominal system moves to state $(2,2)$ since the insertion $ins(b)$ is observed as event of $b$.
However, the attacked system moves to state $(2,1)$ where the relative distance remains $1$.
State $(2,2,2,1)$ is a marked state since state $(2,1)$ is a detection state.
Next, we discuss the weights in the verifier.

\begin{figure}[thpb]
\centering
\includegraphics[width=0.65\columnwidth]{verifier.png}
\caption{Weighted verifier $V$}
\label{fig:verifier}
\end{figure}

% The automaton $T$ in Fig.~\ref{fig:DFA_t} has deadlock states, i.e., $Uns$ states.
% For this reason, we define the verifier $V$ using the coaccessible automaton of $T$, $CoAc(T)$ where the operator $CoAc$ is defined as in \citep{Lafortune:2021}.
We define weights for verifier $V$ based on the transition probabilities in $M_n$ and $M_a$.
For example, the transition in $V$ from state $(1,1,1,1)$ to state $(2,2,2,1)$ via event $ins(b)$ captures the information of the execution in $M_n$ and in $M_a$.
In the case of $M_n$, the nominal system observes event $b$, which has a probability of $0.111\dots$, Fig.~\ref{fig:M_n}.
In the attacked system, the probability of executing $ins(b)$ is equal to $1$, Fig.~\ref{fig:Ma}.
Thus, the ratio of executing this transition in the attacked system is $9$ times more likely than executing in the nominal system, i.e., $1/0.111\dots = 9$.
We use the logarithm of this ratio as a weight for this transition as shown in Fig.~\ref{fig:verifier}.
The use of the logarithm will become clear when we pose the shortest path problem.
% Figure~\ref{fig:verifier} shows the weighted verifier where weights are shown in blue.

Formally, we define the weight function $w:X_V\times \Sigma_m\times X_V\rightarrow \mathbb{R}$ for each transition $\delta_V((x_1,x_2,x_3,x_4),e) = (y_1,y_2,y_3,y_4)$ as:

\begin{equation}\label{eq:ratio}
w((x_1,x_2,x_3,x_4),e,(y_1,y_2,y_3,y_4)) = \log \frac{P_{M_a}((x_3,x_4),e,(y_3,y_4)}{P_{M_n}((x_1,x_2),\Pi^S(e),(y_1,y_2))}
\end{equation}
In Eq.~\ref{eq:ratio}, we have that $P_{M_n}((x_1,x_2),\epsilon,(y_1,y_2)) = 1$, i.e., the probability of executing the empty string is always one.


Based on the verifier $V$, we calculate the weights as shown in Fig.~\ref{fig:verifier} using Eq.~\ref{eq:ratio}.
For example, the weight for transition $\delta_V((2,2,2,2),b) = (1,1,1,1)$ is equal to $\log{(1)}=0$ since $P_{M_a}((2,2),b,(1,1)) = P_{M_n}((2,2),b,(1,1)) = 0.1$.

Propositions~\ref{prop:Sa-Ma} and~\ref{prop:Ma-Sa} have linked the strings in $M_a$ to strings in $S_A/G$ for an attack strategy.
Next, we show that the strings in $\lang_m(V)$ are related to strings in a detection language $L_A^{det}$.

\begin{proposition}\label{prop:det_lang}
A string $s\in \lang_m(V)$ if and only if $\exists A\in \Psi_A$ such that $\Pi^G(s)\in L_{det}^A$.
\end{proposition}
\begin{proof}
This proposition follows from Propositions~\ref{prop:Sa-Ma} and~\ref{prop:Ma-Sa} and the construction of $V$ and its detection states $X_{det}$    
\end{proof}

\subsection{Finding the Shortest Path}
The verifier automaton $V$ has the information to find the attack strategy $A$ and the string in $L^A_{det}$ with the smallest intrusion detection value, $dtc$, as in Def.~\ref{def:lambda-sa-det}.
This string is related to the shortest path from the initial state to any detection state, i.e., the string executed by the path.
To show this relationship, we present the relationship between the weight of a path in $V$ and the probabilities of executing this path in $M_n$ and $M_a$

Let us select the path $p = (2,2,2,2)a(1,1,1,1)ins(b)\allowbreak(2,2,2,1)$ from the verifier $V$ in Fig.~\ref{fig:verifier}.
This path is generated using attack strategy $A_1$ in Fig.~\ref{fig:A1}.
The weight of this path is the sum of each transition weight: $$\log(\frac{0.1}{0.1})+\log(\frac{1}{0.11\dots})=\log(1)+\log(9) = \log(1\times 9) = \log(9)$$
In Eq.~\ref{eq:ratio}, we define transition weights as the ratio of transition probabilities in $M_a$ over  $M_n$.
We rewrite the weight of path $p$:
\begin{align*}
\log(\frac{0.1}{0.1})+\log(\frac{1}{0.11\dots}) &= \log(\frac{0.1}{0.011\dots})\\&=\log(\frac{L_p(S_{A_1}/G)(a)}{L_p(S/G)(ab)})\\& =\log(9)
\end{align*}
Executing $p$ in $M_a$ is $9$ times more likely than executing in $M_n$.
% When the detection observes the string $ab$, it has high confidence that it was executed by an attacked system in $M_a$.

By finding the shortest path in $V$, we find the smallest ratio of executing a path in $M_a$ over executing in $M_n$.
The shortest path is defined as:
\begin{definition}[Shortest path]\label{def:shortest-path}
Given verifier $V$ with weight function $w$, the shortest marked path is defined as the path with the shortest sum of weights from the initial state to a marked state in $V$:
$$\inf_{\rho:=x_0e_0\dots x_{|\rho|}\in Paths_m(V)}\sum_{i=0}^{i<|\rho|}w(x_i,e_i,x_{i+1})$$
where $Path_m(G)$ is a path starting in $x_{0,G}$ and ending in $X_{m,G}$, $Path_m(G) = \{x_0e_0\dots x_n\in (X_G\times \Sigma)^*X_G \mid  x_0 = x_{0,G}\wedge x_{i+1} = \delta_G(x_i,e_i), i<n \wedge x_n\in X_{m,G}\}$.
\end{definition}

% We also define a \emph{marked path} in $G$.
% Intuitively, a marked path in $G$ is a finite sequence of (state, event) pairs starting in $x_{0,G}$, satisfying $\delta_G$, and ending in a marked state.
% \begin{definition}(Marked path)
% A \emph{marked path} in $G$ is a sequence $\rho = x_0e_0\dots x_{n-1}e_{n-1}x_n\in (X_G\times \Sigma)^*X_G$ such that $x_0 = x_{0,G}$, $x_{i+1} = \delta_G(x_i,e_i)$ for $i<n$, and $x_n\in X_{m,G}$.
% The length of $\rho$ is defined by $|\rho| = n+1$. 
% The set of all marked paths in $G$ is denoted as $Paths_m(G)$.
% \end{definition}
The following proposition formally ties the weight of a path in $V$ with the probability ration of executing a string in $M_a$ and $M_n$.
\begin{proposition}\label{prop:weight-Ma}
Given verifier $V$ and weight function $w$, for any $s\in \lang_m(V)$ with path $x_0s[1]x_1\dots s[|s|]x_{|s|}$ then
$$\sum_{i=0}^{|s|-1} w(x_i,s[i+1],x_{i+1}) = \log \left(\frac{L_p(S_A/G)(\Pi^G(s))}{L_p(S/G)(\Pi^S(s))} \right)$$
where attack strategy $A$ is constructed as in the proof of Prop.~\ref{prop:Ma-Sa}.
\end{proposition}
\begin{proof}
This result follows by the definition of the verifier, the weight function, and the logarithm property of multiplication, i.e., $log(ab) = log(a) +log(b)$.
Recall that each verifier state $x_i$ is defined by $(x_{i,1},x_{i,2},x_{i,3},x_{i,4})$ (Def.~\ref{def:verifier}).
It follows that:
\begin{align*}
\sum_{i=0}^{|s|-1}w(x_i,s[i+1],x_{i+1}) &= \sum_{i=0}^{|s|-1} \log\left(\frac{P_{M_a}((x_{i,3},x_{i,4}),s[i+1],(x_{i+1,3},x_{i+1,4})}{P_{M_n}((x_{i,1},x_{i,2}),\Pi^S(s[i+1]),(x_{i+1,1},x_{i+1,2}))}\right)\\
&= \log\left(\Pi_{i=0}^{|s|-1} \frac{P_{M_a}((x_{i,3},x_{i,4}),s[i+1],(x_{i+1,3},x_{i+1,4})}{P_{M_n}((x_{i,1},x_{i,2}),\Pi^S(s[i+1]),(x_{i+1,1},x_{i+1,2}))}\right)
\end{align*}
By Prop.~\ref{prop:Ma-Sa}, we can construct an attack strategy $A$ such that $\Pi^G(s)\in \lang(S_A/G)$.
Moreover, by construction of $M_a$, we have that $\Pi_{i=0}^{|s|-1} P_{M_a}((x_{i,3},x_{i,4}),s[i+1],(x_{i+1,3},x_{i+1,4}) = L_p(S_A/G)(\Pi^G(s))$.
And by construction of $M_n$, we have $\Pi_{i=0}^{|s|-1} P_{M_n}((x_{i,1},x_{i,2}),\Pi^S(s[i+1]),(x_{i+1,1},x_{i+1,2}) = L_p(S/G)(\Pi^S(s))$.
Therefore, we have:
\begin{align*}
\sum_{i=0}^{|s|-1}w(x_i,s[i+1],x_{i+1})  = \log\left(\frac{L_p(S_A/G)(\Pi^G(s))}{L_p(S/G)(\Pi^S(s))}\right)
\end{align*}
\end{proof}
% Although Dijkstra's algorithm has better time complexity than Bellman-Ford, it disallows negative cycles in the graph.
% In the weighted verifier $V$, it is possible to have negative transition weights when the probability of executing the transition in $M_n$ is larger than executing in $M_a$, i.e., $log(r)<1$ when $r<1$ in Eq.~\ref{eq:ratio}.
% Therefore, it is possible to have negative cycles in the verifier $V$.
% For this reason, we use the Bellman-Ford algorithm to compute the shortest path in $V$.

The shortest path problem is a well-known problem in graph theory with polynomial-time algorithm solutions, e.g., Bellman-Ford, and Dijkstra's algorithms \citep{cormen2022introduction}.
Since $V$ can have negative and positive weights, we use the Bellman-Ford algorithm to compute the shortest path in $V$.
The Bellman-Ford algorithm outputs either: (1) a vector, $\mathbf{sp} \in \mathbb{R}^{|X_{m,V}|}$, storing the smallest real values for paths from the initial state to marked states, or (2) an output saying that the graph has a ``negative cycle."

Back to our running example, we run the Bellman-Ford algorithm using the weighted verifier $V$ depicted in Fig.~\ref{fig:verifier}.
The verifier does not have negative cycles since all cycles have $0$ weight, i.e., $\log(1) = 0$.
The shortest path returns the vector $\mathbf{sp} = [\log(9)]$.

\subsection{Extracting the Intrusion Detection Value}
Solving the shortest path problem for the weighted verifier $V$ gives us the information to solve Problems~\ref{prob:ver-lsa} and~\ref{prob:optimal-lsa}.
Herein, we describe the solution for Problem~\ref{prob:ver-lsa}.
The solution for Problem~\ref{prob:optimal-lsa} follows the same steps.

The shortest path problem can return two possible outputs: (1) a vector, $\mathbf{sp}\in \mathbb{R}^{|X_{m,V}|}$, or (2) ``negative cycle."
In the case of output (2), Problem~\ref{prob:ver-lsa} returns that $M_n$ is not $\lambda$-sa detectable.
The shortest path can be arbitrarily small when $V$ has a negative cycle.
We can select a string in $M_a$ that reaches a marked state that its probability of execution is much smaller than in $M_n$, i.e., $L_p(M_a)(s)<< L_p(M_n)(\Pi^S(s))$.
In other words, we can construct an attacker $A$ as in Prop.~\ref{prop:Ma-Sa} that generates this string with an arbitrarily small detection value. 
% , i.e., it goes to $-\infty$ which means that the execution ratio goes to $0$.
% In consequence, the execution ratio of a string goes to $0$.
% The probability of executing this string in $M_a$ is much smaller than the probability of executing the projected string in $M_n$.
% Consequently, the $int$ value in Eq.~\ref{eq:likelihood} goes to $0$.
In this scenario, there exists an attacker that hides its probabilistic trace to be negligible.
% This result is described by the following theorem.

\begin{theorem}\label{theo:negative-cycle}
Let $M_n$, $M_a$, $\lambda \in (0.5,1]$ be given.
If the Bellman-Ford algorithm returns that the weighted verifier $V$ has a negative cycle, then $M_n$ is not $\lambda$-sa detectable.
\end{theorem}

\proof
When $V$ has a negative cycle, the shortest path has a limit as $-\infty$.
Thus, we can find a path $s\in \lang_m(V)$ with path $\rho = x_0s[1]x_1\dots s[|s|]x_{|s|+1}$ with weight smaller than $\log(\frac{\lambda}{1-\lambda})$.
$$\sum_{i=0}^{i<|s|}w(x_i,s[i],x_{i+1})< \log\left(\frac{\lambda}{1-\lambda}\right)$$
% It follows that:
% \begin{align*}
% \sum_{i=0}^{i<|s|}w(x_i,s[i],x_{i+1}) = \sum_{i=0}^{i<|s|} \log\left(\frac{P_{M_a}((x_{i,3},x_{i,4}),s[i],(x_{i+1,3},x_{i+1,4})}{P_{M_n}((x_{i,1},x_{i,2}),\Pi^S(s[i]),(x_{i+1,1},x_{i+1,2}))}\right)< \log\left(\frac{\lambda}{1-\lambda}\right)
% \end{align*}
% By Prop.~\ref{prop:Ma-Sa}, we can construct an attack strategy $A$ such that $s\in \lang(S_A/G)$.
% Moreover, by construction of $M_a$, we have that $\Pi_{i=0}^{|s|} P_{M_a}((x_{i,3},x_{i,4}),s[i],(x_{i+1,3},x_{i+1,4}) = L_p(S_A/G)(\Pi^G(s))$.
% And by construction of $M_n$, we have $\Pi_{i=0}^{|s|} P_{M_n}((x_{i,1},x_{i,2}),\Pi^S(s[i]),(x_{i+1,1},x_{i+1,2}) = L_p(S/G)(\Pi^S(s))$.
Using Prop.~\ref{prop:weight-Ma}, we have:
\begin{align*}
\log\left(\frac{L_p(S_A/G)(s)}{L_p(S/G)(\Pi^S(s))}\right)< \log\left(\frac{\lambda}{1-\lambda}\right)\\
\frac{L_p(S_A/G)(\Pi^G(s))}{L_p(S/G)(\Pi^S(s))}<\left(\frac{\lambda}{1-\lambda}\right)
\end{align*}
Manipulating the equation above, we get:
\begin{align*}
\frac{L_p(S_A/G)(\Pi^G(s))}{L_p(S_A/G)(\Pi^G(s))+L_p(S/G)(\Pi^S(s))}< \lambda 
\end{align*}
Since $s\in \lang_m(V)$ and by Prop.~\ref{prop:det_lang}, the system $M_n$ is not $\lambda$-sa detectable.
\endproof

In the case the Bellman-Ford algorithm returns the vector $\mathbf{sp}\in \mathbb{R}^{|X_{m,V}|}$, we can find the value $dtc$ in Eq.~\ref{eq:likelihood}.
Depending on this calculated value, we return (i) $M_n$ is $\lambda$-sa detectable, or (ii) $M_n$ is \textbf{not} $\lambda$-sa detectable.
We calculate the value $dtc$ with the lowest value $val$ in the vector $\mathbf{sp}$, i.e., $val = \min \textbf{sp}$.
\begin{align}
\frac{L_p(S_A/G)(\Pi^G(s))}{L_p(S/G)(\Pi^S(s))} = \exp(val)\nonumber\\
L_p(S_A/G)(\Pi^G(s)) = \exp(val)L_p(S/G)(\Pi^S(s))\label{eq:relation-ratio}
\end{align}
Where $s$ is the string in $V$ generated by the shortest path value obtained by the Bellman-Ford algorithm.
Using the relation in Eq.~\ref{eq:relation-ratio}, Eq.~\ref{eq:likelihood} gives us the value of $dtc$.
\begin{align}
dtc &= \frac{\exp(val)L_p(S/G)(\Pi^S(s))}{L_p(S/G)(\Pi^S(s))(1+\exp(val))}\nonumber \\
& =\frac{exp(val)}{1+exp(val)} \label{eq:final-int}
\end{align}

Therefore, the shortest path value $val$ to a marked state in $V$ provides us the $dtc$ as in Def.~\ref{def:lambda-sa-det}.
Formally, we have the following theorem.

\begin{theorem}\label{theo:vector-shortest}
Let $M_n$, $M_a$, $\lambda\in (0.5,1]$ be given.
Also, let the Bellman-Ford algorithm for the weighted verifier $V$ return vector $\mathbf{sp}\in \mathbb{R}^{|X_{m,V}|}$ with $val$ being the lowest value in $\mathbf{sp}$.
The system $M_n$ is $\lambda$-sa detectable if and only if 
\begin{equation}\label{eq:thm}
\frac{exp(val)}{1+exp(val)}\geq \lambda
\end{equation}
\end{theorem}

\proof \quad 
We start by proving $\Rightarrow$: If $M_n$ is $\lambda$-sa detectable, then Eq.~\ref{eq:thm} holds.
First, we assume that $M_n$ is  $\lambda$-sa detectable, which implies that $dtc = \inf_{A\in\Psi_A}\inf_{s\in L^A_{det}} det(s) \geq \lambda$.
Therefore, there exists an attack strategy $A\in \Psi_A$ and string $s\in L_{det}^A$ such that for all other $A'\in \Psi_A$ and $t\in L_{det}^{A'}$: $det(s)\leq det(t)$.
Let $A$ and $s$ be the attack strategy and string that satisfy the proposition above.
Using Prop.~\ref{prop:Sa-Ma}, we can find a string $u = A(s)\in \lang_m(V)$ such that $\Pi^G(u) = s$.
By Prop.~\ref{prop:weight-Ma}, we have: 
\begin{equation}\label{eq:w_u}
w_u = \exp\left(\sum_{i=0}^{|u|-1} w(x_i,u[i+1],x_{i+1})\right)  = \frac{L_p(S_A/G)(s)}{L_p(S/G)(\Pi^S(s))}
\end{equation}
where $u$ generates path $x_0u[1]\dots u[|u|]x_{|u|}$ in $V$.
Similar to Eq.~\ref{eq:final-int}, we can write the detection value of $s$ as $det(s) = \frac{w_u}{1+w_u}$.
Since we assumed that $M_n$ is $\lambda$-sa detectable $\frac{w_u}{1+w_u}\geq \lambda$.
It remains to be proved that $w_u = \exp(val)$.

We show that $w_u = \exp(val)$ by contradiction.
By the solution of the Bellman-Ford algorithm, we can construct a string $v\in \lang_m(V)$ such that $w_v = \exp(val)$, where $w_v$ is the weight of string $v$ calculated as in Eq.~\ref{eq:w_u}
Assume that $w_u\neq \exp(val)$.
If $w_u<\exp(val)$, we found a new path in $V$ with a weight smaller than the shortest path $val$, i.e., a contradiction.
If $w_u>\exp(val)$, we found a string $v\in \lang_m(V)$ with weight smaller than $w_u$.
Using Prop.~\ref{prop:weight-Ma}, we can calculate $det(\Pi^G(v)) = \frac{w_v}{w_v+1}$.
Since $w_u>w_v$, $w_u>0$, and $w_v>0$, it follows: 
\begin{equation}\label{eq:ineq-cont}
det(s) = det(\Pi^G(u)) = \frac{w_u}{w_u+1}> \frac{w_v}{w_v+1}>det(\Pi^G(v))
\end{equation}
However, Eq.~\ref{eq:ineq-cont} contradicts that $det(s)$ is the smallest detection value as we assumed above.
Therefore, it must be that $w_u = w_v = \exp(val)$ which concludes our $\Rightarrow$ proof.

Next, we need to prove $\Leftarrow$: If Eq.~\ref{eq:thm} holds, then $M_n$ is $\lambda$-sa detectable.
We can prove this statement by contradiction.
Let us assume that Eq.~\ref{eq:thm} holds and that $M_n$ \emph{is not} $M_n$ $\lambda$-sa detectable.
Since $M_n$ is not $\lambda$-sa detectable, then there exist an attack strategy $A$ and string $s\in L_{det}^A$ such that $det(s)<\lambda$.
Using Prop.~\ref{prop:Sa-Ma}, we can find an string $t\in \lang_m(V)$ such that $A(s) = t$ and $det(s) = \frac{w_t}{1+w_t}$ where $w_t$ is calculated as in Eq.~\ref{eq:w_u}.
We show that $w_t$ will be smaller than $exp(val)$, which contradicts the assumption that $val$ is the smallest weight in $V$.

Let $v\in \lang_m(V)$ such that $w_v = exp(val)$, where $w_v$ is the weight of string $v$ similar to Eq.\ref{eq:w_u}.
With $w_t$, it follows that $det(\Pi^G(v)) = \frac{w_v}{w_v+1}>\lambda$ by our assumption that Eq.~\ref{eq:thm} holds.
Using $det(s)<\lambda$ and $det(\Pi^G(v))>\lambda$, it follows that  $\frac{w_t}{1+w_t}<\frac{w_v}{w_v+1}$.
The last inequality implies that $w_t<w_v = \exp(val)$ since $w_v>0$ and $w_t>0$, i.e., string $t$ has weight less than $val$.
However, $val$ is the smallest weight in $V$, i.e., a contradiction.
Therefore, if Eq.~\ref{eq:thm} holds, then $M_n$ is $\lambda$-sa detectable.
This concludes our proof.
\endproof

Back to our running example, the shortest path from the Bellman-Ford algorithm for verifier in Fig.~\ref{fig:verifier} is $\mathbf{sp} = \log(9)$.
% In this scenario, the algorithm returned $\mathbf{sp} = [\log(9)]$.
Since there is a single marked state, $val = \log(9)$.
Following Thm.~\ref{theo:vector-shortest}, we have:
\begin{equation}
\frac{\exp(val)}{1+\exp(val)}=0.9\geq 0.9
\end{equation}
Thus, the system $M_n$ is $0.9$-safe.

\subsection{Complexity of $\lambda$-sa detectability}
Our last result is related to the complexity of solving Problems~\ref{prob:ver-lsa} and~\ref{prob:optimal-lsa} due to the Bellman-Ford algorithm.
\begin{theorem}\label{theo:complexity}
Solving Problem~\ref{prob:ver-lsa} has worst-case time-complexity of $O(|X_V|^2) = O(|X_G\times X_R|^4)$.
\end{theorem}
\begin{proof}
This result follows by the complexity of the Bellman-Ford algorithm and the construction of $V$.
\end{proof}


\section{Conclusion}\label{sect:conclusion}
This paper investigated a new sensor attack detection notion using probabilistic information. 
We proposed the notion of $\lambda$-sa detectability that ensures probabilistic detection for all complete, consistent, and successful sensor attack strategies.
This notion generalizes the previously defined $\epsilon$-safety, introduced in \citep{meira-goes:2020towards,Fahim2024-wodes}, by considering general classes of sensor attack strategies instead of a single strategy as in $\epsilon$-safety.
We show that $\lambda$-sa detectability can be verified in polynomial time by reducing our verification problem to a shortest path problem in a graph.
We leave for future work considering the detection of both sensor and actuator attacks. 
%\nolinenumbers
\bibliography{romulo}% common bib file

\appendix
\section{Appendix}
\subsection{Probabilistic Parallel Composition $||_p$} \label{app:parallel_prob}
%%MOVE TO APENDIX
\begin{definition}
The probabilistic composition, $||_p$ between PDES $G$ and DFA $R$ is defined by $R||_p G = (X_R\times X_G, \Sigma, \delta_{R,G},P_{R,G}, (x_{0,R},x_{0,G}), X_{R}\times X_{m,G})$ where:
\begin{equation}
P_{R,G}((x_R,x_G),e,(y_R,y_G)) = \left\lbrace
\begin{array}{ll}
\frac{P_G(x_G,e)}{\sum_{\sigma\in\Gamma_G(x_R)\cap\Gamma_R(y_R)}P_G(x_G,\sigma)} &
\ e\in \Gamma_R(x_R)\cap\Gamma_G(x_G) \\
0 & \text{otherwise}
\end{array}
\right.
\end{equation}
\end{definition}

\subsection{Construction of attack strategy using automaton $A$}\label{app:A_aut}

An automaton $A= (X_A, \Sigma_m, \delta_{A},x_{0,A})$ encodes an attack strategy if the following condition holds:
\begin{enumerate}
    \item[(1)] $\forall x\in X_A$, $e\in \Sigma_i$. $\delta_A(x,e)! \Rightarrow \forall e\in \Sigma_m\setminus\{e\}.\  \delta_A(x,e)\hspace{-0.1cm}\not{!}$ 
    \item[(2)] $\forall x\in X_A$, $e\in \Sigma_a$. $\delta_A(x,e)! \Leftrightarrow \delta_A(x,del(e))\hspace{-0.1cm}\not{!}$ 
\end{enumerate}
Condition (1) ensures that states with an insertion event do not have any other transitions. 
Since the attacker is deterministic, it can only insert one event when it has ``decided" to insert.
Condition (2) ensures that the attacker can only delete a compromised event or not attack this event, i.e., it cannot have both options since it is deterministic.

Next, we show how to extract an attack strategy from an automaton $A$.
First, for state $q\in X_A$, the function $Ins(q)$ returns a string of insertion events that can be executed from $q$ until it reaches a state in $A$ where it cannot insert events.
The function $Ins(q)$ is defined recursively as follows:
\begin{equation}
Ins(q) = \left\{ 
\begin{array}{cc}
  eIns(\delta_A(q,e))   & \text{if } \Gamma_A(q) = \{e\} \wedge e\in \Sigma_i\\
  \epsilon  &  \text{if } \Gamma_A(q)\cap \Sigma_i = \empty
\end{array}
\right.
\end{equation}

Now, we are ready to extract a strategy from $A$.
\begin{definition}
Given DFA $A = (X_A, \Sigma_m, \delta_{A},x_{0,A})$ satisfying the conditions above.
First, the strategy for $\epsilon$ is $A(\epsilon,\epsilon) = Ins(x_{0,A})$, i.e., a string of insertions from the initial state if they are present in A.
Next, the attack strategy generated by $A$ is constructed as follows for any state $q\in X_A$ and event $e\in \Sigma$:

\begin{equation}\label{eq:}
A(q,e) = \left\lbrace
\begin{array}{ll}
del(e) Ins(\delta_A(q,del(e)) & \text{if } del(e) \in \Gamma_A(q)\\
e Ins(\delta_A(q,e)) & \text{if } e\in \Gamma_A(q)\\
\text{undefined} & \text{otherwise}
\end{array}
\right.
\end{equation}
\end{definition}
The strategy is defined for state $q$ after observing event $e\in \Gamma_A(q)\cap \Sigma$ is: (1) maintain observation of event $e$ followed by a possible string of insertion event, or (2) replace $e$ with $del(e)$ followed by a possible string of insertion event.
Lastly, we can extend the function $A(q,e)$ for strings as in Def.~\ref{def:attack_str} for any  $s\in \lang(A)$ as $A(s,e) = A(\delta_A(x_{0,A},s),e)$.
The attack strategy is also undefined for any string $s\notin \lang(A)$.

\end{document}
