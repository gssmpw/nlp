\subsection{Supervisory Control}\label{subsect:des}
We consider the supervisory layer of a feedback control system, where the uncontrolled system (plant) is modeled as a Deterministic Finite-State Automaton (DFA) in the discrete-event modeling formalism.
A DFA is defined by $G := (X_G,\Sigma,\delta_G,x_{0,G},X_{m,G})$, where $X_G$ is the finite set of states, $\Sigma$ is the finite set of events, $\delta_G:X_G\times\Sigma\rightarrow X_G$ is the partial transition function, $x_{0,G}$ is the initial state, and $X_{m,G}$ is the set of marked states.
The function $\delta_G$ is extended, in the usual manner, to the domain $X_G\times\Sigma^*$. 
The language and the marked language generated by $G$ are defined by $\lang(G) := \{s \in \Sigma^*\mid \delta_G(x_{0,G},s)!\}$ and $\lang_m(G) := \{s \in \lang(G)\mid \delta_G(x_{0,G},s)\in X_{m,G}\}$, where $!$ means that the function is defined.

For convenience, we define $\Gamma_G(x) := \{e\in\Sigma\mid\delta_G(x,e)!\}$ as the set of feasible events in state $x\in X_G$.
% We also use the following notation for any string $s\in \Sigma^*$.
For string $s$, the length of $s$ is denoted by $|s|$ whereas $s[i]$ denotes the $i^{th}$ event of $s$ such that $s = s[1]s[2]\ldots s[|s|]$.
The $i^{th}$ prefix of $s$ is defined by $s^i$, i.e., $s^i = s[1]s[2]\ldots s[i]$ and $s^0 = \epsilon$. 
% Moreover, we denote by $\bar{s}$ the set of all prefixes of $s$.
% Finally, we use $\mathbb{N}$ to be the set of natural numbers, $[n]$ and $[n]^+$ to be, respectively, the set of natural numbers and the set of positive natural numbers both bounded by $n\in \mathbb{N}$.

Considering the supervisory control theory of DES \citep{Ramadge:1987}, a \emph{supervisor} controls the plant $G$ by dynamically disabling events.
The limited actuation capability of the supervisor is modeled by partitioning the event set $\Sigma$ into the sets of controllable and uncontrollable events, $\Sigma_{c}$ and $\Sigma_{uc}$.
The supervisor cannot disable uncontrollable events. 
Therefore, the supervisor's control decisions are limited to the set $\Gamma:=\{\gamma\subseteq\Sigma\mid\Sigma_{uc} \subseteq \gamma\}$.
Formally, a supervisor is a mapping $S:\lang(G)\rightarrow\Gamma$ defined to satisfy specifications on $G$, e.g., avoid unsafe states in $G$, avoid deadlocks.

The closed-loop behavior of $G$ under the supervision of $S$ is denoted by $S/G$ and generates the closed-loop languages $\lang(S/G)$ and $\lang_m(S/G)$; see, e.g., \citep{Lafortune:2021,Wonham:2018}.
Herein, we assume that supervisor $S$ is realized by an automaton $R = (X_R,\Sigma,\delta_R,x_{0,R})$, i.e., $S(s) = \Gamma_R(\delta_R(x_{0,R},s))$.
With an abuse of notation, we use $S$ and $R$ interchangeably hereafter.
%For any string $s\in \Sigma^*$, we use the following notation. 
%We denote by $e^i_s$ the $i^{th}$ event of $s$ such that $s = e^1_se^2_s\ldots e^{|s|}_s$, where $|s|$ denotes the length of $s$.
%We denote by $s^i$ the $i^{th}$ prefix of $s$, namely $s^i = e^1_s\ldots e^i_s$ and $s^0 = \epsilon$. 
%Moreover, we denote by $\bar{s}$ the set of all prefixes of $s$.
%Finally, we use $\mathbb{N}$ to be the set of natural numbers, $[n]$ and $[n]^+$ to be, respectively, the set of natural numbers and the set of positive natural numbers both bounded by $n\in \mathbb{N}$. 



\subsection{Stochastic Supervisory Control}\label{subsect:sdes}
We consider a stochastic DES modeled as a probabilistic discrete event system (PDES) defined similarly to a DFA \citep{Lawford:1993,Garg:1999,Pantelic:2014}. 
A PDES is defined by the tuple $G := (X_G,\Sigma,\allowbreak \delta_G, P_G,x_{0,G},X_{m,G})$ where $X_G$, $\Sigma$, $\delta_G$, $x_{0,G}$ and $X_{m,G}$ are defined as in the DFA definition.
The probabilistic transition function $P_G:X_G\times\Sigma\times X_G\rightarrow [0,1]$ specifies the probability of moving from state $x$ to state $y$ with event $e$. 
Hereupon, we will use the notation $G$ to describe a PDES.

In this work, we assume that $G$ is deterministic where $\nexists y,y^*\in X_G$, $y^*\neq y$ such that $P_G(x,e,y)>0$ and $P_G(x,e,y^*)>0$.
In this manner, $\delta_G(x,e)= y$ if and only if $P_G(x,e,y) > 0$.
Moreover, we assume that each state in $G$ transitions with probability $1$ or deadlocks, i.e.,  $\sum_{e\in \Sigma} \sum_{y\in X_H} P_G(x,e,y) \in \{0,1\}$ for any $x \in X_G$.

\begin{example}  
Figure~\ref{fig:plant_G} provides an example of PDES $G$ modeling the collision avoidance motivating example described in Sect.~\ref{sect:motivation}.
The PDES $G$ has four states and three events.
States in $G$ model the relative distance among $ego$ and $adv$ whereas events $a,\ b$, and $c$ model the decrease, increase, and no change, respectively, on this relative distance.
The cars collide when the relative distance is equal to zero.
For simplicity, once the relative distance is greater than or equal to three, $adv$ escapes $ego$'s range.
States $0$ and $3$ are deadlock states, e.g., $\sum_{e\in \Sigma} P_G(0,e,\delta_G(x,e)) = 0$.
On the other hand, states $1$ and $2$ are live states with probability $1$.
The dynamics of $G$ is captured by the transitions in Fig.~\ref{fig:plant_G} where the label describes the event and probability of the transition, respectively.
For instance, transition $2\rightarrow 1$ has probability $0.1$ of occurring, e.g, $P_G(2,a,1) = 0.1$.
\end{example}

\begin{figure}[thpb]
\centering
\includegraphics[width=0.4\columnwidth]{Figs/plant_G.png}
\caption{PDES $G$ collision avoidance}
\label{fig:plant_G}
\vspace{-2em}
\end{figure}

Although the language and marked language of PDES $G$ are defined as in the DFA case, the notion of probabilistic languages (p-languages) of a PDES was introduced to characterize the probability of executing a string \citep{Garg:1999}.
The p-language of $G$, $L_p(G):\Sigma^*\rightarrow [0,1]$, is recursively defined for $s\in\Sigma^*$ and $e\in\Sigma$ as: 
$L_p(G)(\epsilon) := 1$, $L_p(G)(se) := L_p(G)(s)P_G(x,e,y)$ if $x=\delta_G(x_{0,G},s)$,  $e\in \Gamma_G(x)$ and $y = \delta_G(x,e)$, and $0$ otherwise.
% \vspace*{-0.3cm}
% \begin{align}
% \hspace{-.2cm}L_p(G)(\varepsilon) &:=1\\
% \hspace{-.2cm}L_p(G)(se) &:=\hspace{-.1cm}\left\lbrace\hspace{-.1cm}\begin{array}{ll}
% L_p(G)(s)P_G(x,e,y)&\text{ if }x = \delta_G(x_{0,G},s)\\
% &\ y=\delta_G(x,e)\\
% 0& \text{ otherwise}
% \end{array}\right.
% \end{align}
\begin{example}  
Continuing with our running example, we can obtain the probabilistic language of PDES $G$ in Fig.~\ref{fig:plant_G}.
For instance, the probability of executing string $abc \in \lang(G)$ is inductively computed by $P_G(2,a,1)P_G(1,b,2)P_G(2,c,2) = 0.1\times 0.1\times 0.8$.
\end{example}
% For more details on the probability space used in this paper, please see \citep{Garg:1999}. 
% One property from the measurable sets in this space is that two \emph{distinct} strings $s$ and $t$, such that no string is a prefix of the other one, generate independent measurable sets.
% This property implies that the probability of generating either one of these two strings is equal to $L_p(H)(s)+L_p(H)(t)$.
% This useful result will be exploited in both the problem formulation and the solution methodology.

For convenience, we write $P_G(x,e)$ to denote $P_G(x,e,\delta_G(x,e))$ whenever $\delta_G(x,e)!$, i.e., the probability of executing $e$ in state $x$.
% And we define $\Gamma(x):=\{e\in \Sigma\mid \delta_G(x,e)!\}$ as the set of active events in state $x$.
% We also define a \emph{marked path} in $G$.
% Intuitively, a marked path in $G$ is a finite sequence of (state, event) pairs starting in $x_{0,G}$, satisfying $\delta_G$, and ending in a marked state.
% \begin{definition}(Marked path)
% A \emph{marked path} in $G$ is a sequence $\rho = x_0e_0\dots x_{n-1}e_{n-1}x_n\in (X_G\times \Sigma)^*X_G$ such that $x_0 = x_{0,G}$, $x_{i+1} = \delta_G(x_i,e_i)$ for $i<n$, and $x_n\in X_{m,G}$.
% The length of $\rho$ is defined by $|\rho| = n+1$. 
% The set of all marked paths in $G$ is denoted as $Paths_m(G)$.
% \end{definition}

In the context of stochastic supervisory control theory, the plant $G$ is controlled by a supervisor $S$ as described in Section~\ref{subsect:des}.
In this work, we use the framework of supervisory control of PDES introduced by \citep{Kumar:2001}.
The supervisor is \emph{deterministic} and realized by DFA $R = (X_R,\Sigma,\allowbreak \delta_R,x_{0,R})$ as previously-described.
And although $R$ is deterministic, its events disablement proportionally increases the probability of the enabled ones.
Given a state $x\in X_G$, a state $y \in X_R$, and an event $e \in \Gamma_G(x)\cap\Gamma_R(y)$, the probability of $e$ being executed is given by the standard normalization:
\begin{equation}\label{eq:renormalization}
P_{R,G}((x_R,x_G),e,(y_R,y_G)) = \frac{P_G(x_G,e)}{\sum_{\sigma\in\Gamma_G(x_R)\cap\Gamma_R(y_R)}P_G(x_G,\sigma)}
\end{equation}
Due to this renormalization, the closed-loop behavior $R/G$ generates a p-language different, in general, than the p-language of $G$.
We can represent the closed-loop behavior $R/G$ as PDES $R||_p G$ where $||_p$ is defined based on the parallel composition $||$ and Eq.~\ref{eq:renormalization}.
The formal definition of $||_p$ is described in Appendix~\ref{app:parallel_prob}.



\begin{example}
Regarding our running example, the supervisor $R$ for PDES $G$ is shown in Fig.~\ref{fig:sup_R}.
This supervisor ensures that the plant never reaches state $0$.
The closed-loop representation of $R/G$ is shown in Fig.~\ref{fig:M_n}.
The supervisor disables event $a$ in state $1$ which results in disabling event $a$ in state $1$ in the plant.
For this reason, events from state $(1,1)$ in $R/G$ have been normalized based on Eq.~\ref{eq:renormalization}.
For example, $P_{R,G}((1,1),b,(2,2)) = \frac{0.1}{0.9}= 0.111\dots$.
\end{example}
\begin{figure}[thpb]
\begin{subfigure}[t]{0.45\columnwidth}
\centering
\includegraphics[width=0.75\columnwidth]{sup_R.png}
\caption{Supervisor $R$ collision avoidance}
\label{fig:sup_R}
\end{subfigure}
\ 
\begin{subfigure}[t]{0.45\columnwidth}
\centering
\includegraphics[width=0.75\columnwidth]{Figs/controlled-R-G.png}
\caption{Supervised system $R/G=R||_pG$}
\label{fig:M_n}
\end{subfigure}
\caption{Supervisor $R$ and controlled system $R/G$}
\label{fig:supervisor-controlled}
\vspace{-2em}
\end{figure}

For simplicity, we assume that the plant $G$ has one critical state, denoted $x_{crit} \in X_G$.
Moreover, we assume that every transition to $x_{crit}$ is controllable, i.e., $\delta_G(x,e) = x_{crit} \Rightarrow e\in \Sigma_c$ for any $x\in X_G$.
These assumptions are without loss of generality as we can generalize it to any regular language by state space refinement in the usual way \citep{Cho:1989,Lafortune:2021}. 
Supervisor $R$ ensures the critical state is not reachable in $R/G$ as in our running example.

% Intuitively, the probabilistic composition is defined similarly to the standard parallel composition.
% The probabilistic composition renormalizes the probability transition function $P_{R,G}$ based on the events enabled by supervisor $R$, i.e., $e\in \Gamma_G(x_G)\cap \Gamma_R(x_R)$.
% Therefore, $R/H$ generates a p-language different, in general, than the p-language of $G$. 

% Based on the probabilistic composition, we define $R/G:= R||_p G$ as the closed-loop representation of $R/G$.
% Moreover, since $R/G$ is a PDES, it has a p-language $L_p(R/G)$.

% In the context of stochastic supervisory control theory, the plant $G$ is controlled by a supervisor as in the previously-described supervisory control framework.
% Nonetheless, there are different ways of studying its closed-loop behavior \cite{Lawford:1993,Kumar:2001,Pantelic:2009}.
% In this paper, we use the results of supervisory control of stochastic DES introduced in \cite{Kumar:2001}, where only the plant behaves stochastically. 
% That is, both the specification and the supervisor are deterministic and defined as in the previously-described logical supervisory control framework.
% However, the supervisor \emph{alters} the probabilistic behavior of the plant via the control actions it takes (disabling events).
% Recall that in the supervisory control framework, the event set of $H$ is partitioned into controllable, $\Sigma_c$, and uncontrollable, $\Sigma_{uc}$, events, where the supervisor does not disable uncontrollable events.
% Conditions for the existence of a supervisor for the above control problem are provided in \cite{Kumar:2001}.

% Formalizing the previous discussion, the plant modeled by PFA $H$ is controlled by a supervisor modeled by DFA $R$.
% And although $R$ is deterministic, its events disablement proportionally increases the probability of the enabled ones.
% Given a state $x\in X_H$, a state $y \in X_R$, and an event $e \in \Gamma_H(x)\cap\Gamma_R(y)$, the probability of $e$ being executed is given by the standard normalization:
% \begin{equation}\label{eq:renormalization}
% Pr_{e}^{x,y} = \frac{Pr_H(x,e,\delta_H(x,e))}{\textstyle\sum\limits_{e'\in\Gamma_H(x)\cap\Gamma_R(y)}Pr_H(x,e',\delta_H(x,e'))} 
% \end{equation}
% The set $\Gamma_H(x)\cap\Gamma_R(y)$ describes the events that can be executed by $H$ in state $x$ restricted by the events enabled by $R$ in state $y$.
% If every event in $\Gamma_H(x)$ is enabled by $R$ in state $y$, then their probabilities of execution remains unaltered, i.e., the denominator in Eq.~(\ref{eq:reachprob}) is equal to $1$. 
% However, if at least one event in $\Gamma_H(x)$ is disabled by $R$ in state $y$,  its probability of execution is proportionally redistributed to the remaining enabled and executable events.
% Therefore, $R/H$ generates a p-language different, in general, than the p-language of $H$. 

% For simplicity and without loss of generality, we assume that the plant $H$ has one deadlock critical state and $R$ ensures that this state is not reachable in $R/H$.
% Specifically, we assume that the language $\lang(R/H)\subseteq\{s\in \lang(H)\mid \delta_H(x_{0,H},s)\neq x_{crit}\}$ where $x_{crit}\in X_H$ is the critical state and $\lang(R/H)$ is controllable, see, e.g., \cite{Lafortune:2008,Wonham:2018}.
% Normally, one would find a supervisor that generates the supremal controllable sublanguage of $\{s\in \lang(H)\mid \delta_H(x_{0,H},s)\neq x_{crit}\}$, but we do not assume such a supervisor is selected, see Example~II.1.
% For the definition of the supremal controllable sublanguage see, e.g., \cite{Lafortune:2008,Wonham:2018}.
% Lastly, we define the set of unsafe strings as $U_{uns} = \{s \in \Sigma^* \mid \delta_H(x_{0,H},s) = x_{crit}\}$ and the set of unsafe state pairs for the controlled system $R/H$ by $
% X_{uns} := \{x_{crit}\}\times X_R$

% This section reviews the concepts of probabilistic discrete event systems, stochastic supervisory control, and supervisory control under attacks.  



% \subsection{Probabilistic Discrete Event Systems}\label{subsect:pdes}

% In the discrete-event formalism, a probabilistic discrete event system (PDES) extends DES models as in \citep{Lafortune:2021,Wonham:2018} by introducing probabilistic information to them \citep{Lawford:1993,Garg:1999,Pantelic:2014}. 
% A PDES is defined by the tuple $G := (X_G,\Sigma,\allowbreak \delta_G, P_G,x_{0,G},X_{m,G})$ where $X_G$ is the finite set of states; $\Sigma$ is the finite set of events; $\delta_G:X_G\times\Sigma\rightarrow X_G$ is the partial transition function; $P_G:X_G\times\Sigma\times X_G\rightarrow [0,1]$ is the transition probability function; $x_{0,G}$ is the initial state; and $X_{m,G}$ is the set of marked states. 
% We define $\delta_G$ based on $P_G$ as $\delta_G(x,e)= y$ if and only if $P_G(x,e,y) > 0$, i.e., $\delta_G(x,e)!$ is defined.

% Herein, we follow the state probability termination criteria as in \cite{Lawford:1993, Pantelic:2014} which allows a termination probability of either $0$ or $1$. 
% Formally, the PDES $G$ satisfies $\sum_{y\in X_G \wedge e\in \Sigma} P_G(x,e,y) \in \{0,1\}$ for any $x\in X_G$.
% In other words, states are either deadlocks (no transition from them) or live with probability $1$ (no partial termination).

% \begin{example}  
% Fig.~\ref{Example} provides an example of PDES $G$ modeling the collision avoidance motivating example.
% The PDES $G$ has four states and three events.
% States $0$ and $3$ are deadlock states, e.g., $\sum_{e\in \Sigma} P_G(0,e,\delta_G(x,e)) = 0$.
% On the other hand, states $1$ and $2$ are live states with probability $1$.
% The dynamics of $G$ are provided by the transitions in Fig.~\ref{fig:plant_G} where the label describes the event and probability of the transition, respectively.
% For instance, transition $2\rightarrow 1$  has probability $0.1$ of occurring, i.e.g, $P_G(2,a,1) = 0.1$.
% \end{example}

% % Each state in $G$ represents the relative distance between $ego$ and $adv$.
% % The cars start with a relative distance equal to $2$.
% % Moreover, states $0$ and $3$ are deadlock states that model collision and out-of-range, respectively.
% % For simplicity, we assume that the tracking stops whenever $adv$ is out-of-range. 
% % The model has three events, $\{a, b, c\}$, that model $ego$ moving forward, $adv$ moving forward, $c$ cars remain in the same position.
% % The dynamics of $G$ are provided by the transitions in Fig.~\ref{fig:collision-plant} where the label describes the event and probability of the transition, respectively.

% The language and marked language of PDES $G$ are defined as usual, e.g., $\lang(G) = \{s \in \Sigma^*\mid \delta_G(x_{0,G},s)!\}$ \citep{Lafortune:2021}.
% In \citep{Garg:1999}, the notion of probabilistic languages (p-languages) of a PDES was introduced to characterize the probability of executing a string.

% \begin{definition}(p-language)
% The p-language of $G$, $L_p(G):\Sigma^*\rightarrow [0,1]$, is defined for $s\in\Sigma^*$ and $e\in\Sigma$ as:
% \vspace*{-0.3cm}
% \begin{align}
% \hspace{-.2cm}L_p(G)(\varepsilon) &:=1\\
% \hspace{-.2cm}L_p(G)(se) &:=\hspace{-.1cm}\left\lbrace\hspace{-.1cm}\begin{array}{ll}
% L_p(G)(s)P_G(x,e,y)&\text{ if }x = \delta_G(x_{0,G},s)\\
% &\ y=\delta_G(x,e)\\
% 0& \text{ otherwise}
% \end{array}\right.
% \end{align}
% \end{definition}
% % The notion of probabilistic languages (p-languages) of a PDES was introduced in \citep{Garg:1999}.

% \begin{example}  
% Continuing with our running example, we can obtain the probabilistic language of PDES $G$ as in Fig.~\ref{fig:plant_G}.
% For instance, the probability of executing string $abc \in \lang(G)$ is inductively computed by $P_G(2,a,1)P_G(1,b,2)P_G(2,c,2) = 0.1\times 0.1\times 0.8$.
% \end{example}

% % For more details on the probability space used in this paper, please see \citep{Garg:1999}. 
% % One property from the measurable sets in this space is that two \emph{distinct} strings $s$ and $t$, such that no string is a prefix of the other one, generate independent measurable sets.
% % This property implies that the probability of generating either one of these two strings is equal to $L_p(H)(s)+L_p(H)(t)$.
% % This useful result will be exploited in both the problem formulation and the solution methodology.

% For convenience, we write $P_G(x,e)$ to denote $P_G(x,e,\delta_G(x,e))$ whenever $\delta_G(x,e)!$ (is defined), i.e., the probability of executing $e$ in state $x$.
% And we define $\Gamma(x):=\{e\in \Sigma\mid \delta_G(x,e)!\}$ as the set of active events in state $x$.
% We also introduce the definition of a \emph{marked path} in $G$.
% Intuitively, a marked path in $G$ is a finite sequence of (state, event) pairs starting in $x_{0,G}$, satisfying $\delta_G$, and ending in a marked state.

% \begin{definition}(Marked path)
% A \emph{marked path} in $G$ is a sequence $\rho = x_0e_0\dots x_{n-1}e_{n-1}x_n\in (X_G\times \Sigma)^*X_G$ such that $x_0 = x_{0,G}$, $x_{i+1} = \delta_G(x_i,e_i)$ for $i<n$, and $x_n\in X_{m,G}$.
% The length of $\rho$ is defined by $|\rho| = n+1$. 
% The set of all marked paths in $G$ is denoted as $Paths_m(G)$.
% \end{definition}


% \subsection{Stochastic supervisory control}\label{subsect:sdes}
% We consider the supervisory level of a CPS where the uncontrolled system is modeled as a PDES $G$, \emph{the plant}.
% Limited control capabilities in the plant are modeled by partitioning the set of events $\Sigma$ into two disjoint sets, the set of controllable events $\Sigma_{c}$ and the set of uncontrollable events $\Sigma_{uc}$.
% In the discrete-event formalism, a supervisory controller, or simply \emph{supervisor}, for $G$ can dynamically disable controllable events to ensure desired behaviors such as ``safe" behaviors. 
% Formally, a supervisor is defined as a function $S:\Sigma^*\rightarrow\Gamma$, where $\Gamma = \{\gamma\subseteq\Sigma\mid\Sigma_{uc} \subseteq \gamma\}$ is the set of admissible control decisions.
% The closed-loop behavior of the controlled system is denoted by $S/G$ and defined by the language $\lang(S/G)$; see, e.g., \citep{Lafortune:2021}.

% In this work, we use the framework of supervisory control of PDES introduced by \citep{Kumar:2001}.
% The supervisor is \emph{deterministic} and realized by a deterministic finite-state automaton (DFA) $R = (X_R,\Sigma,\allowbreak \delta_R,x_{0,R})$ with $X_R,\Sigma,\delta_R$ and $x_{0,R}$ defined as in the PDES definition.
% From DFA $R$, we can extract a supervisor function $S$ as $S(s) = \Gamma_R(\delta(x_{0,R},s))$ for every $s\in \lang(R)$, see, e.g., \citep{Lafortune:2021,Wonham:2018}.
% The composition of supervisor $R$ and PDES $G$ defines a new PDES $R/G$ that represents the closed-loop behavior.

% \begin{definition}
% The probabilistic composition, $||_p$ between PDES $G$ and DFA $R$ is defined by $R||_p G = (X_R\times X_G, \Sigma, \delta_{R,G},P_{R,G}, (x_{0,R},x_{0,G}), X_{R}\times X_{m,G})$ where:
% \begin{equation}\label{eq:renorm}
% P_{R,G}((x_R,x_G),e,(y_R,y_G)) = \left\lbrace
% \begin{array}{ll}
% \frac{P_G(x_G,e)}{\sum_{\sigma\in\Gamma_G(x_R)\cap\Gamma_R(y_R)}P_G(x_G,\sigma)} &
% \ e\in \Gamma_R(x_R)\cap\Gamma_G(x_G) \\
% 0 & \text{otherwise}
% \end{array}
% \right.
% \end{equation}
% \end{definition}
% Intuitively, the probabilistic composition is defined similarly to the standard parallel composition.
% However, the probabilistic composition renormalizes the probability transition function $P_{R,G}$ based on the events enabled by supervisor $R$, i.e., $e\in \Gamma_G(x_G)\cap \Gamma_R(x_R)$.
% This probability renormalization is defined by Eq.~\ref{eq:renorm} where the probability of disabled events is redistributed to the enabled events.

% Based on the probabilistic composition, we define $R/G:= R||_p G$ as the closed-loop representation of $R/G$.
% Moreover, since $R/G$ is a PDES, it has a p-language $L_p(R/G)$.

% \begin{example}
% Regarding our running example, the supervisor $R$ for PDES $G$ is shown in Fig.~\ref{fig:supervisor_R}.
% This supervisor ensures that the plant never reaches state $0$.
% The closed-loop representation of $R/G$ is shown in Fig.~\ref{Mn}.
% The supervisor disables event $a$ in state $A$ which results in disabling event $a$ in state $1$ in the plant.
% For this reason, events from state $(A,1)$ in $R/G$ have been normalized based on Eq.~\ref{eq:renorm}.
% For example, $P_{R,G}((A,1),b,(2,B)) = \frac{0.1}{0.9}= 0.111\dots$.
% \end{example}

% For simplicity, we assume that the plant $G$ has one critical state, denoted $x_{crit} \in X_G$.
% This assumption is without loss of generality as we can generalize it to any regular language by state space refinement in the usual way \citep{Cho:1989,Lafortune:2021}. 
% Supervisor $R$ ensures the critical state is not reachable in $R/G$ as in our running example.
% % The set of unsafe strings is defined as $L_{crit} = \{s \in \Sigma^* \mid \delta_G(x_{0,G},s) = x_{crit}\}$. 

