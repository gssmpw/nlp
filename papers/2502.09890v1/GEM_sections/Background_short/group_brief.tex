\subsection{Group and Group Symmetrization}

A \emph{group} is a mathematical structure comprising a set \( G \) and a binary operation \( \circ : G \times G \to G \) that combines two elements of \( G \), describing symmetries and transformations. A \emph{group action} defines how \( G \) operates on a set \( \Omega \) via a map \( \circ : G \times \Omega \to \Omega \) satisfying identity and compatibility properties. Here, \( \Omega \) is a set on which the group acts, such as a set of data points or geometric objects. In this paper, we only consider \textit{linear isometry groups}, i.e., any mapping \( g : \Omega \to \Omega \) can be written as \( g(x) \mapsto A_gx \) for some unitary matrix $A_g$. 

A function \( f : \Omega \to \mathbb{R} \) is \(\Gi\) if for all \( g \in G \), \( f(g \circ x) = f(x) \). Conversely, \( f : \Omega \to \Omega \) is \(\Ge\) if for all \( g \in G \), \( f(g \circ x) = g \circ f(x) \), ensuring the group action commutes with the transformation.

A probability distribution \( q(x) \) on \( \Omega \) is \(\Gi\) under \( G \) if for all \( g \in G \) and measurable subsets \( A \subset \Omega \), \( q(g \circ x \in A) = q(x \in A) \). This implies the distribution remains unchanged under transformations induced by the group. A conditional distribution \( q(y \mid x) \) is \(\Ge\) if for all \( g \in G \), \( q(g \circ y \mid g \circ x) = q(y \mid x) \). This ensures the conditional distribution transforms consistently with the group action, preserving symmetry between \( x \) and \( y \).


\paragraph{Group Symmetrization.}  
We introduce group symmetrization operators~\citepcolor{Theoequi, structureGAN}, which are essential for our approach. Let \( S_G \) be the symmetrization operator under group \( G \), transforming any distribution \( p(x) \) into a \(\Gs\) distribution \( p^G(x) \) defined as:
\begin{equation}
    p^G(x) := S_G(p(x)) := \int_G p(g \circ x) \mu_G(g) \mathrm{d}g, \nonumber
\end{equation}
where \( \mu_G \) is the unique Haar probability measure on \( G \). This \(\Gs\) distribution has several desirable properties: (1) \( p^G(x) \) is a \(\Gi\) distribution~\citepcolor{structureGAN}; (2) for any \(\Gi\) distribution \( q(x) \) and its approximating distribution \( p(x) \), the Wasserstein-1 distance between \(p^G(x)\) and \( q(x) \) is no greater than the Wasserstein-1 distance between \( p(x) \) and \( q(x) \)~\citepcolor{Theoequi}.

