\subsection{Molecular Conformer Prediction}

% \vinh{TODO: Include task description}
To evaluate our method in real-world scenarios, we apply it to molecular conformer generation, a crucial task in drug discovery. Given a molecule's 2D molecular graph \(G=(V,E)\), the goal is to predict its 3D conformations. A conformer \(C\) represents a mapping of each atom in \(V\) to a point in 3D space and can be viewed as a set of \(SE(3)\)-equivalent vectors in \(\mathbb{R}^{3n}\) where \(n=|V|\). We also consider the automorphism group of the graph, \(\text{Aut}(G)\), which consists of all graph automorphisms—bijective mappings of the vertex set that preserve the adjacency relationships in the graph.



We benchmark our method against several baselines, including Torsional Diffusion~\citepcolor{jing2022torsional}, MCF~\citepcolor{Swallow}, ETFlow and ETDiff (ETFlow with Gaussian prior and no special coupling)~\citepcolor{hassan2024flow} using GEOM-QM9, a subset of the GEOM dataset~\citepcolor{axelrod2022geom}, which contains molecules with an average of 11 atoms. Following previous works, we report precision and recall for Coverage and Average Minimum RMSD (AMR) between generated and gold-standard conformers provided in CREST~\citepcolor{CREST}. However, we argue that the previous threshold to classify good predictions, \(\delta = 0.5\text{\AA}\), is not tight enough. In fact, some important molecular properties such as energy can be very different within this range~\citepcolor{transitionstate}. Thus, to capture finer structural details, we use \(0.1\text{\AA}\) instead of \(0.5\text{\AA}\) for molecular conformer generation on GEOM-Q9. More details about these metrics are provided in~\cref{sub_app:QM9_metrics}.

\begin{table*}[ht!]
    \caption{Molecule conformer generation results on GEOM-QM9 ($\delta = 0.1\text{\AA}$).}
    \label{tab:qm9}
    \centering
    \renewcommand{\arraystretch}{1.1} % Default value: 1
    \begin{tabular}{lcccccccc}
    \toprule
     & \multicolumn{4}{c}{Recall} & \multicolumn{4}{c}{Precision} \\
     & \multicolumn{2}{c}{Coverage $\uparrow$} & \multicolumn{2}{c}{AMR $\downarrow$} & \multicolumn{2}{c}{Coverage $\uparrow$} & \multicolumn{2}{c}{AMR $\downarrow$} \\
        \hline
        & mean & median & mean & median & mean & median & mean & median\\
        \hline
        Torsional Diff. & 37.7 & 25.0 & 0.178 & 0.147 & 27.6 & 12.5 & 0.221 & 0.195 \\
        MCF             & 81.9 & \textbf{100.0} & 0.092 & 0.045 & 78.6 & \textbf{93.8} & \textbf{0.113} & {0.052} \\
        \hline
        ETDiff          & 80.7 & \textbf{100.0} & 0.092 & 0.038 & 71.4 & 75.0 & 0.162 & 0.093 \\
        \cellcolor[HTML]{D3D3D3}ETDiff + Ours   & \cellcolor[HTML]{D3D3D3}{81.0} & \cellcolor[HTML]{D3D3D3}\textbf{100.0} & \cellcolor[HTML]{D3D3D3}{0.089} & \cellcolor[HTML]{D3D3D3}\cellcolor[HTML]{D3D3D3}{0.037} & \cellcolor[HTML]{D3D3D3}{72.2} & \cellcolor[HTML]{D3D3D3}{80.0} & \cellcolor[HTML]{D3D3D3}{0.151} & \cellcolor[HTML]{D3D3D3}{0.077} \\
        \hline
        ETFlow          & 83.8 & \textbf{100.0} & 0.082 & 0.032 & 77.7 & 90.0 & 0.130 & 0.049 \\
        \cellcolor[HTML]{D3D3D3}ETFlow + Ours   & \cellcolor[HTML]{D3D3D3}\textbf{84.7} & \cellcolor[HTML]{D3D3D3}\textbf{100.0} & \cellcolor[HTML]{D3D3D3}\textbf{0.079} & \cellcolor[HTML]{D3D3D3}\textbf{0.028} & \cellcolor[HTML]{D3D3D3}\textbf{79.1} & \cellcolor[HTML]{D3D3D3}{92.5} & \cellcolor[HTML]{D3D3D3}{0.122} & \cellcolor[HTML]{D3D3D3}\textbf{0.046} \\
        \bottomrule
    \end{tabular}
\end{table*}
The results for molecular conformer generation on the GEOM-QM9 dataset ($\delta = 0.1\text{\AA}$) are summarized in Table~\ref{tab:qm9}. Our method consistently improves the performance of baseline models, as indicated by the shaded rows. Specifically, when applied to ETFlow, our approach achieves the highest mean Coverage (84.7\%) and the lowest mean AMR (0.079) for Recall, as well as the best median AMR (0.028). Similarly, notable improvements are observed when augmenting ETDiff, yielding superior AMR scores compared to the baseline. 
While these improvements require a slight increase in training time, the overhead remains minimal, at 2.9\% for ETDiff and 0.9\% for ETFlow. This ensures an efficient enhancement with significant performance gains for our method.

% \begin{table*}[ht!]
    \caption{Molecule conformer generation results on GEOM-DRUG ($\delta = 0.1\text{\AA}$). For both ETFlow-DIff and ETFlow-Ours, we sample conformations over $100$ time-steps.}
    \label{tab:qm9}
    \centering
    \renewcommand{\arraystretch}{1.1} % Default value: 1
    \begin{tabular}{lcccccccc}
    \toprule
     & \multicolumn{4}{c}{Recall} & \multicolumn{4}{c}{Precision} \\
     & \multicolumn{2}{c}{Coverage $\uparrow$} & \multicolumn{2}{c}{AMR $\downarrow$} & \multicolumn{2}{c}{Coverage $\uparrow$} & \multicolumn{2}{c}{AMR $\downarrow$} \\
        \hline
        & mean & median & mean & median & mean & median & mean & median\\
        \hline
        % CGCF            &  &  &  &  &  &  &  &  \\
        % GeoDiff         &  &  &  &  &  &  &  &  \\
        % GeoMol          &  &  &  &  &  &  &  &  \\
        Torsional Diff. &  &  &  &  &  &  &  &  \\
        MCF             &  &  &  &  &  &  &  &  \\
        \hline
        ETDiff          &  &  &  &  &  &  &  &  \\
        ETDiff + Ours   &  &  & 
        &  &  &  &  &  \\
        \hline
        ETFlow          &  &  &  &  &  &  &  &  \\
        ETFlow + Ours   &  &  &  &  &  &  &  &  \\
        \bottomrule
    \end{tabular}
\end{table*}

% \paragraph{Ablation study.} How important each type of equivariant group is.

% \paragraph{Qualitative.} Some visualization

