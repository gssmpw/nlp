\subsection{Group and Group Action}

A \emph{group} is a mathematical structure that consists of a set \( G \) along with a binary operation \( \cdot : G \times G \to G \), which combines two elements of the group. This operation must satisfy the following four properties:

\begin{enumerate}
    \item \textbf{Closure}: For any two elements \( g_1, g_2 \in G \), the result of the operation \( g_1 \cdot g_2 \) is also an element of \( G \).
    \item \textbf{Associativity}: For all \( g_1, g_2, g_3 \in G \), we have \( (g_1 \cdot g_2) \cdot g_3 = g_1 \cdot (g_2 \cdot g_3) \).
    \item \textbf{Identity}: There exists an identity element \( e \in G \) such that for any \( g \in G \), \( e \cdot g = g \cdot e = g \).
    \item \textbf{Invertibility}: For each \( g \in G \), there exists an inverse element \( g^{-1} \in G \) such that \( g \cdot g^{-1} = g^{-1} \cdot g = e \), where \( e \) is the identity element.
\end{enumerate}

In the context of symmetries, the elements of a group typically represent transformations, and the operation corresponds to combining these transformations. A \emph{group action} describes how a group \( G \) acts on a set \( \Omega \). Formally, a group action is a map \( \cdot : G \times \Omega \to \Omega \) that satisfies two key properties:

\begin{enumerate}
    \item \textbf{Identity element}: For every \( x \in \Omega \), the identity element \( e \in \Omega \) leaves \( x \) unchanged, i.e., \( e \cdot x = x \).
    \item \textbf{Compatibility}: For any two elements \( g, h \in G \) and any \( x \in \Omega \), the action is associative, i.e., \( g \cdot (h \cdot x) = (gh) \cdot x \).
\end{enumerate}

In this paper, we only consider \textit{linear isometry groups}, i.e., any mapping \( g : \Omega \to \Omega \) can be written as \( g(x) \mapsto A_gx \) for some unitary matrix $A_g \in \mathbb{R}^{d \times d}$. 
In applications such as molecular modeling, the set \( \Omega \) could represent molecular structures, and the group \( G \) might correspond to symmetry transformations, such as rotations, translations, and automorphism.

A function \( f : \Omega \to \mathbb{R} \) is said to be \emph{\( G \)-invariant} if for all \( g \in G \) and \( x \in \Omega \), the function satisfies:
\[
f(g \cdot x) = f(x).
\]
This means that the function value does not change under the action of any element of the group \( G \).

On the other hand, a function \( f : \Omega \to \Omega \) is said to be \emph{\( G \)-equivariant} if for all \( g \in G \), \( x \in \Omega \), the following condition holds:
\[
f(g \cdot x) = g \cdot f(x),
\]
where the action of \( G \) on \( \Omega \) is understood in an appropriate way, ensuring that the group action commutes with the transformation of \( f \). 

\vinh{Add definition of G-invariant distribution and G-equivariant conditional distribution. Remove the properties of groups. }
