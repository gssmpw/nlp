\begin{figure}[H]
    \centering
    % Top-left figure
    \begin{subfigure}[b]{0.32\textwidth} % Reduced width
        \centering
        \includegraphics[width=\textwidth]{GEM_figs/Toy_Exps/BS1000_2D_training_loss.pdf}
        \caption{Training Loss (first 1k iterations)}
        \label{fig:training_loss_2D}
    \end{subfigure}
    \hfill
    % Top-right figure
    \begin{subfigure}[b]{0.32\textwidth} % Reduced width
        \centering
    \includegraphics[width=\textwidth]{GEM_figs/Toy_Exps/baseline_vs_ours50.pdf}
        \caption{Qualitative Visualization}
        \label{fig:2D-qualitative}
    \end{subfigure}
    \hfill
    \begin{subfigure}[b]{0.32\textwidth} % Reduced width
        \centering
    \includegraphics[width=\textwidth]{GEM_figs/Toy_Exps/2D-rmsd_comparison.pdf}
        \caption{RMSD - lower is better}
        \label{fig:2D-rmsd}
    \end{subfigure}
    \caption{\textbf{2D toy experiment.} (a) Training loss curve illustrating the model's convergence during training. (b) Visualization of 100 samples generated by each model, where accurate samples have points lying within the brown circles. (c) Comparison of RMSD between model samples and the ground truth, showing the average RMSD over 1000 samples with sampling steps ranging from 10 to 50.}
    \label{fig:2D-toy}
\end{figure}