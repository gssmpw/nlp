\subsection{Group and Group Action}

A \emph{group} is a mathematical structure comprising a set \( G \) and a binary operation \( m : G \times G \to G \) that combines two elements of \( G \). Groups are fundamental in describing symmetries and transformations in mathematics and physics. A \emph{group action} defines how a group \( G \) operates on a set \( \Omega \). Formally, it is a bijective map \( \circ : G \times \Omega \to \Omega \) satisfying the identity and compatibility properties, ensuring the action is well-defined and consistent. Here, \( \Omega \) is a set on which the group acts, such as a set of data points or geometric objects.


A function \( f : \Omega \to \mathbb{R} \) is said to be \(\Gi\) if for all \( g \in G \) and \( x \in \Omega \), the function satisfies:
\[
f(g \circ x) = f(x).
\]
This means that the function value does not change under the action of any element of the group \( G \).

On the other hand, a function \( f : \Omega \to \Omega \) is said to be \(\Ge\) if for all \( g \in G \), \( x \in \Omega \), the following condition holds:
\[
f(g \circ x) = g \circ f(x),
\]
where the action of \( G \) on \( \Omega \) is understood in an appropriate way, ensuring that the group action commutes with the transformation of \( f \). In this paper, we only consider \textit{linear isometry groups}, i.e., any mapping \( g : \Omega \to \Omega \) can be written as \( g(x) \mapsto A_gx \) for some unitary matrix \( A_g \in \mathbb{R}^{d \times d} \). Examples of such isometry groups include the permutation group \( \mathsf{S}_n \), which consists of all possible permutations of \( n \) elements, the orthogonal group \( \mathsf{O}(d) \), which includes rotations and reflections in \( d \)-dimensional space, and the special orthogonal group \( \mathsf{SO}(d) \), which consists of rotations only. 

\paragraph{Invariant Distribution} 
A probability distribution \( p(x) \) defined on a set \( \Omega \) is said to be \(\Gi\) under the action of a group \( G \) if for all \( g \in G \) and measurable subsets \( A \subset \Omega \), the following holds:
\[
p(g \circ x \in A) = p(x \in A),
\]
where \( g \circ x \) denotes the action of \( g \in G \) on \( x \in \Omega \). Intuitively, this means that the distribution remains unchanged under the transformations induced by the group.

\paragraph{Equivariant Conditional Distribution} 
A conditional distribution \( p(y \mid x) \), where \( x \in \Omega \) and \( y \in \mathcal{Y} \), is said to be \(\Ge\) if for all \( g \in G \), the following condition holds:
\[
p(g \circ y \mid g \circ x) = p(y \mid x).
\]
This property ensures that the conditional distribution transforms consistently with the group action, preserving the symmetry in the relationship between \( x \) and \( y \). An example of an equivariant conditional distribution is the Gaussian kernel \( q(x \mid y) = \mathcal{N}(x; y, \sigma^2 I) \), which satisfies \( q(g \circ x \mid g \circ y) = q(x \mid y)\) for all isometry groups. 

\paragraph{Group Symmetrization.}  
We use group symmetrization operators~\citepcolor{Theoequi, structureGAN}, which play a crucial role in developing the theoretical framework for our approach. Let \( S_G \) be the symmetrization operator under group \( G \), transforming any distribution \( p(x) \) into a \(\Gs\) distribution:
\begin{equation}
    S_G[p](x) := \int_G p(g \circ x) \mathrm{d}\mu_G(g), \nonumber
\end{equation}
where \( \mu_G \) is the unique Haar probability measure on \( G \). Importantly, this \(\Gs\) distribution is a \(\Gi\) distribution~\citepcolor{structureGAN}. Group symmetrization can be useful to study the theoretical properties of equivariant distributions~\citepcolor{Theoequi, structureGAN} or to design equivariant neural networks~\citepcolor{frameaverage}.

% has several desirable properties: (1) \( p^G(x) \) is a \(\Gi\) distribution~\citepcolor{structureGAN}; (2) for any \(\Gi\) distribution \( q(x) \) and its approximating distribution \( p(x) \), the Wasserstein-1 distance between \(p^G(x)\) and \( q(x) \) is no greater than the Wasserstein-1 distance between \( p(x) \) and \( q(x) \)~\citepcolor{Theoequi}.

