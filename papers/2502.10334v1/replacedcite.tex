\section{Related Work}
In recent years, the integration of Deep Convolutional Generative Adversarial Networks (DCGANs) into the domain of medical image classification has gained significant attention from researchers. This approach addresses a common challenge in this field - the scarcity of diverse and extensive labeled datasets. Previous studies have primarily focused on the detection of medical conditions using deep learning models trained on limited data. Generative Adversarial Networks have been employed for data augmentation with the goal of enhancing the training of Convolutional Neural Networks ____. DCGANs offer a compelling solution by generating synthetic medical images that closely mimic real ones. These synthetic images can help bridge the gap in data scarcity and class imbalance, thereby enhancing the robustness and accuracy of medical image classifiers.\\
Zeid Baker et. al.____ studied and employed Generative Adversarial Networks to generate synthetic images that closely resemble the real dataset, with the aim of expanding the available dataset. The research comprised two distinct experiments. In the first experiment, they fine-tuned a Deep Convolutional Generative Adversarial Network  specifically for a given dataset. The second experiment focused on assessing how the introduction of synthetic data impacted the accuracy of a classification task. The researchers conducted these experiments using three different datasets: MNIST, Fashion-MNIST, and Flower photos. Their findings suggested that the effectiveness of DCGAN in increasing model accuracy depended on the nature of the dataset and highlighted the significant role that data preprocessing played in the performance of DCGANs, a consideration applicable to most machine learning algorithms.\\ 
Wu, Qiufeng et. al. ____employed deep convolutional generative adversarial networks to augment their dataset with generated images, alongside original images, for the purpose of identifying Tomato. Their findings demonstrated that the images generated by DCGAN not only expanded the dataset size but also introduced diverse characteristics, ultimately leading to improved model generalization. They utilized a GoogLeNet classifier to train and test the model on five different classes of tomato leaf images.