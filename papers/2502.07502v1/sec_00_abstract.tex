\begin{abstract}
[\textbf{Context and Motivation}]:
Cyber-Physical Systems (CPS) have become relevant in a wide variety of different domains, integrating hardware and software, often operating in an emerging and uncertain environment where human actors actively or passively engage with the CPS. To ensure correct and safe operation, and  self-adaptation, monitors are used for collecting and analyzing diverse runtime information.
[\textbf{Problem}]:
However, monitoring humans at runtime, collecting potentially sensitive information about their actions and behavior, comes with significant ramifications that can severely hamper the successful integration of human-machine collaboration. Requirements engineering (RE) activities must integrate diverse human values, including Privacy, Security, and Self-Direction during system design, to avoid involuntary data sharing or misuse.
%Furthermore, to ensure acceptance of a system that analyzes and processes sensitive human-related data, trustworthiness needs to be established already when defining core requirements, particularly for any involved human-monitoring components. 
[\textbf{Principal Ideas}]:
In this research preview, we focus on the importance of incorporating these aspects in the RE lifecycle of eliciting and  creating runtime monitors.
% Particularly, informed by research on human values and value tactics, we created a conceptual framework for creating runtime monitors that actively include human-related properties.
[\textbf{Contribution}]:
We derived an initial conceptual framework, building on the value taxonomy introduced by Schwartz and human value integrated Software Engineering by Whittle, further leveraging the concept of value tactics. The goal is to tie functional and non-functional monitoring requirements to human values and establish traceability between values, requirements, and actors. Based on this, we lay out a research roadmap guiding our ongoing work in this area.
\end{abstract}

%% original-full length abstract
% \begin{comment}

% \begin{abstract}
% [\textbf{Context and Motivation}]:
% Cyber-Physical Systems (CPS) have become relevant in a wide variety of different domains, including autonomous vehicles, drones performing search-and-rescue operations, and smart automation in factory shop floors.
% One inherent aspect is that CPS exhibit tight integration of hardware and software components, often operating in an emerging and uncertain environment where human actors actively or passively engage and interact with different parts of the CPS. To ensure correct and safe operation, and potentially enable self-adaptive behavior CPS, monitoring components are implemented for collecting and analyzing diverse runtime information.
% [\textbf{Problem}]:
% However, monitoring humans at runtime, and collecting potentially sensitive information about their actions and behavior, comes with significant ramifications that can severely hamper the successful integration of human-machine collaboration in a CPS. Diverse values such as Privacy, Security, Power, and Self-Direction need to be considered, already during system design, as part of requirements engineering (RE) activities to avoid involuntary data sharing or misuse.
% %Furthermore, to ensure acceptance of a system that analyzes and processes sensitive human-related data, trustworthiness needs to be established already when defining core requirements, particularly for any involved human-monitoring components. 
% [\textbf{Principal Ideas and Results}]:
% In this research preview, we focus on the importance of incorporating these aspects in the RE lifecycle of eliciting, specifying, and subsequently creating runtime monitors.
% % Particularly, informed by research on human values and value tactics, we created a conceptual framework for creating runtime monitors that actively include human-related properties.
% [\textbf{Contribution}]:
% We have derived an initial version of a conceptual framework, building on the value taxonomy introduced by Schwartz and initial works in human values integrated software engineering by Whittle, further leveraging the concept of value tactics to tie functional and non-functional monitoring requirements to human values and establish traceability between these three often conflicting dimensions.
% Based on this, we lay out a research roadmap guiding our ongoing work in this area. 
% \end{abstract}

% \end{comment}