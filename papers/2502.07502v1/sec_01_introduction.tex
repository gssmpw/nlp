\section{Introduction}
\label{sec:intro}
% motivation with shop floor / self driving cards

%\zp{UC example excel: \href{https://uibkacat-my.sharepoint.com/:x:/g/personal/zoe_pfister_uibk_ac_at/EZMFDkVHlqNLsB9-myleJgMBXRRsXLDdOEclEAfMKdDhlw?e=vCMj0Q}{Use-Case Vignettes}}

%\mv{context and motivation:}
Cyber-Physical Systems (CPS) have become increasingly prevalent in our society, including, for example, autonomous vehicles or drones.
%The use cases of these systems range from autonomous transportation and environmental monitoring to search-and-rescue missions. 
As part of shop-floor automation, Cyber-Physical Production Systems integrate robots to perform tasks like automated material transport~\cite{theunissen2018smart}.
Typically, CPS exhibit tight integration of hardware and software, combining virtual and physical spaces~\cite{broy2013engineering}. Moreover, CPS often operate in emerging, uncertain environments alongside human stakeholders, such as shop floor workers, drone operators, or pedestrians.

% Monitoring CPS is critical for ensuring safe operation and for enabling self-adaptive behavior, with monitors deployed that collect runtime data and perform constraint checks at runtime.
To ensure safe operation and enable self-adaptive behavior, it is critical to monitor CPS through deployed monitors that collect data and perform constraint checks at runtime.
%,zheng2016efficient}. 
In recent years, paradigms such as Human-Machine Teaming have emerged, where humans are more closely and actively collaborating with CPS.
Current monitoring solutions often consider safety concerns in conjunction with human-machine interaction~\cite{tan2011triple}.
To ensure safety in scenarios with tight human-machine interaction, mechanisms need to be in place not only for the monitoring of the system and collecting properties at runtime, but also of human actors with properties such as location, posture, or cognitive load~\cite{mukhopadhyay2014wearable}. Collecting and analyzing information about a human actor engaged with a CPS can provide additional benefits, for example, depending on the position, movement, and breath data of a shop-floor worker, the system may decide to send emergency personnel to investigate if an accident has occurred~\cite{nwakanma2021Detection}.

%In the context of human-machine collaboration and self-adaptation, monitoring data about the human operator can be leveraged by the system to notify emergency services in case it detects worker injury, prompt the user for inputs, or trigger certain actions when humans are detected during a search-and-rescue mission.

% For example, depending on the current environmental context and the state of a human operator, the system may decide to reduce the information provided to the operator and operate more autonomously, with the aim to reduce their cognitive load~\cite{cleland-huang2023HumanMachine}.
%\mv{Question and Problems:}
However, monitoring humans at runtime, and collecting information about their actions and behavior, comes with significant ramifications that can severely hinder the successful integration of human-machine collaboration in a CPS. Recently, Whittle~\etal[whittle2021Case] argued that \emph{\enquote{human values are heavily underrepresented in SE methods}}.
This is particularly true when sensitive data about humans is collected that can potentially be used in ways it was not intended. For example, Amazon has made headlines for monitoring the workers of their warehouses to track and assess worker performance\footnote{\url{https://tinyurl.com/amazon-monitoring}, accessed Nov 1, 2024.}.
This example reinforces that aspects such as \emph{Privacy} -- and human values in general -- are a major concern, as sensitive data might be stored, shared, or misused. In addition, \textit{ethical aspects} regarding the responsibility of actions must be investigated. 
Questions like, ``who is responsible for an action within a CPS with tight human-machine collaboration?'' and ``when should the system be allowed to act fully autonomously?'' must be considered during design and implementation of the system.
Besides privacy and ethics, it is crucial to consider the \textit{technical aspects} of human monitoring in CPS. Instead of ``simply'' collecting data from the system through sensors, human monitoring often requires additional hardware, such as cameras or wearables, further adding to the complexity.  %For example, monitoring the position of shop-floor workers can be achieved through LIDAR sensors~\cite{nwakanma2021Detection}. 

In this research preview, we focus on the importance of incorporating human values in the RE process, addressing challenges and shortcomings of existing approaches (\citesec{background}),  proposing an initial conceptual Value-Complemented Framework (\citesec{approach}), and discussing our research roadmap (\citesec{roadmap}).



%, particularly ones that integrate human-monitoring@runtime.
%We utilize the value taxonomy introduced by Schwartz~\cite{schwartz1992Universals}, and leverage the concept of value tactics~\cite{wohlrab2024Supporting} % (a concept based on architectural tactics~\cite{kim2009quality,marquez2023architectural})
% We first discuss challenges and shortcomings of existing approaches (\citesec{background}). %, and discussing the 4 crucial aspects of \emph{WHO} relevant stakeholders are, \emph{WHY} human-monitoring@runtime is important, \emph{WHAT} functionality is required, and \emph{HOW} human-monitoring@runtime could be successfully incorporated.
% Based on this, we propose an initial conceptual, Value-Complemented Framework for establishing  CPS monitoring support (\citesec{approach}) and discuss our ongoing research and research roadmap (\citesec{roadmap}).












% but also from a technical perspective, e.g. how and with which devices data can be collected, monitoring human actions poses additional requirements on a traditional runtime monitoring framework.

% From a software and process perspective, additional types of requirements need to be taken into consideration -- with new/different types of constraints to be evaluated at runtime. Instead of ``simply'' collecting data from the system (e.g., sensors, or status messages), human monitoring often requires additional hardware (such as cameras, wearable devices, etc.) that further adds to the complexity.  

%% aspects to consider (benefits vs. downsides:
% -> why do we want human monitoring in these aspects (process optimization, resource management, safety)
% privacy, misuse of data, ethical questions.. technical feasibility

%\mv{Ideas and Solution:}







% \zp{If we want to build upon whittle I would write something like \textit{We will discuss the extension of the classical requirements engineering view of capturing WHAT and HOW by the notion of WHY they have to follow specific requirements.}}
% % Whittle \etal~\cite{whittle2021Case} argue that Schwartz's taxonomy provides valid starting point when integrating human values in a software project.

% As part of this work, we focus on the following two research objectives:\\
% \emph{$\bullet$ RO1: Which aspects in human monitoring need to be considered in the requirements engineering process}.\\ %(what/why/how?)\\
% \textbf{Rationale:} With this first objective, we try to establish a basis for further investigation in this domain. It is important to clearly state \textit{why} and for \textit{what} purpose data is collected (explainability, openness). \\
% \emph{$\bullet$  RO2: How can these aspects be integrated into the requirements engineering process}.\\ %(advantages/drawbacks/trade-offs)\\
% \textbf{Rationale:} With the second resesarch objective, we argue about the need for clear guidelines concerning human values and ethics in CPS. Such guidelines may include discussion value priorization of different system stakeholders (i.e., users, managers, etc.) that lead to potential trade-off decisions.
% % With the second research objective we dive deeper into the human monitoring aspect arguing about different categories of application scenarios and their advantages, but also about hazards and drawbacks, and potential trade-offs that need to be made. 

% The remainder of this paper is structured as follows. In \citesec{background} we provide a motivating example and discuss related work in the area of .... In \citesec{approach} we then discuss the relevant aspects of human monitoring and establish an initial concept of... 
% We then, in \citesec{roadmap}, present a research roadmap, highlighting xx areas for our further endeavor, and conclude the paper in \citesec{conclusion}.