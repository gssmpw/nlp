\section{Background \& Motivating Example}
\label{sec:background}
% also relwork
% \mv{current state of the arg  - problems and challenges - why HMT is needed for CPS!}
%https://www.mdpi.com/1424-8220/23/13/6054
% \mv{
% challenges:
% -autonomy
% -heterogeneity
% complexity
% ...
% }
% ==> 
Runtime monitoring has been an active research area for many years, ranging from model-based approaches and self-adaptive systems, to formal verification and  model-checking of monitored properties~\cite{vierhauser2016requirements}. 
However, hardly any of these approaches has considered human actors, let alone human values or ethical considerations, when collecting potentially sensitive runtime data.
Human aspects are, to some extent, addressed in the Human-Computer Interaction (HCI) community, e.g., as part of supporting human operators in complex, safety-critical environments~\cite{schmid20}, but not directly in the context of runtime monitoring.
On the other hand, integrating human values into software engineering has received increased attention in recent years. To do so, we first need a taxonomy of potential human values.
One of the most prominent human value theories is Schwartz's theory~\cite{schwartz1992Universals}, which includes a list of 10 core universal values and various sub-values, further showcasing which values complement or oppose each other.
For example, the universal value \textit{Security} contains sub-values such as national security, which is opposed to the universal value \textit{Self-Direction} that includes privacy.
%In contrast, values like freedom and broad-mindedness complement each other.
Focusing on SE activities, Whittle \etal[whittle2021Case] have used the value taxonomy of Schwartz to enhance the requirements engineering process. 
%They argue that human values \enquote{should be considered as first-class citizens in software development}.
%In an experimental software project in collaboration with \textit{Clasp} -- a project with the aim to reduce social anxiety in autistic adults --, they defined relevant values for each stakeholder in the project. % iterative process
%Stakeholders included, for example, users of the final software, the product owners, and the project team itself.
% The values are context specific to the individual project. % value portraits
From a set of identified values, they then extracted new requirements.
% they don't filter it by all the values, but by a selection of four overarching things (openness to change, self-enhancement, self-transcendence, conservation
During an experimental software project, they found that the values defined as part of their process capture the \textit{why} of requirements engineering, complementing the traditional view of \textit{what} and \textit{how} of functional and non-functional requirements respectively.

Connecting these aspects, we have previously presented an approach to integrate human values into the RE process~\cite{wohlrab2024Supporting}.
In our \enquote{value-aware requirements engineering} framework, original system goals are enriched with relevant values and sub-values based on the taxonomy of Schwartz~\cite{schwartz1992Universals}.
%After the initial definition, values are prioritized, and consolidated into personas.
As a next step, so-called \enquote{value tactics} are selected for the values and sub-values. Value tactics are decisions that directly address a particular sub-value and can be used to derive functional and non-functional requirements.
An example value tactic for the value-sub-value pair \textit{benevolence-honesty} is to \textit{anonymize transactions}. These value tactics then enable system engineers to elicit detailed requirements, e.g., \textit{anonymize transactions} could lead to specific data protection requirements.

However, the individual aspects of system use cases, human values, and ultimately the resulting requirements are still somewhat disconnected, and streamlining these activities in a coherent RE process remains an open challenge. Particularly in the context of (human) runtime monitoring, this aspect is crucial as different values and their resulting requirements can have an immediate impact on, for example, what data can/must be collected and can be stored or must be anonymized/deleted.
To motivate the challenges and benefits of human-monitoring at runtime in CPS, we present an example Use Case (cf. Table~\ref{tab:motivating-example}) that promotes increasing worker safety in a shop-floor environment.
In the Use Case snippet, we define that the system must detect when a shop-floor worker enters a restricted or dangerous area (e.g., areas with autonomous robots). % which may lead to injuries
Through continuous monitoring of a worker's position, the system must notify the worker if they move beyond the specified boundary. Additionally, emergency personnel must be alerted if the worker does not respond within a certain amount of time.
A system implementing this use case can thus improve the safety of workers, but comes with privacy concerns.
Ultimately, the values \textit{security} (protecting individuals from threats) and \textit{independence} (personal privacy) need to be traded off against each other.
This paper proposes a framework to do that and arrive at a set of value-complemented monitoring requirements.

\begin{table}[hbt]
\footnotesize
\caption{Motivating Use Case Example.}
\renewcommand*{\arraystretch}{1.22}
\label{tab:motivating-example}
\begin{tabularx}{\columnwidth}{L{2.75cm}X}

\toprule
% Use Case related to & Detect Worker Emergency \\
Name & Detect worker entering hazardous area. \\
Description & The system tracks shop-floor workers and notifies them if they enter a restricted or dangerous area. \\
Pre-Conditions & (1) The shop floor worker is equipped with location tracking devices OR the shop floor is equipped with LiDAR to track workers. 

(2) The worker enters a restricted area. \\
Trigger & An administrator starts the tracking software OR The software continuously tracks workers in an area. \\
Success End Condition & The worker is notified when entering a restricted area. If they do not leave within a specified  time, additional emergency personnel are notified. \\

% Error Situations & The location tracking is defective. \\
Failure End Condition & The worker enters a restricted area but is not notified.\\
\midrule
Monitoring Targets & Shop-Floor Worker, Robot \\
Monitored Properties & movement data / worker position / tracking device OR LiDAR\\

\emph{Additional Actors} & Location Tracking Devices, Emergency Personnel, \enquote{Others}, Floor Manager / Shift Supervisor\\

\bottomrule
\end{tabularx}
\renewcommand*{\arraystretch}{1}
\end{table}

% \zp{describe motivating example and human values that arise from that in textual form and show the tables later. Bring amazon privacy examples (see verge article).}
% \zp{in a shop floor there are workers, robots, dangerous areas requiring safety aspects. }

% HMT examples: emergency response https://doi.org/10.3390/app11083662
% 
% Bashar: https://dl.acm.org/doi/abs/10.1145/3640310.3674095

% Requirements reflection: requirements as runtime entities
% N Bencomo, J Whittle, P Sawyer, A Finkelstein

% how does whittles process look like?

% wohlrab
% similar approach based on value derivation and persona creation. They then create 'value-tactics' which lead to the definition of requirements. ““value-aware requirements engineering”” 