\section{Research Roadmap \& Discussion}
\label{sec:roadmap}
%==> 1) Value Validation (Michael)
%% how to validate nfr/ethics
%%% coverage - >traceabilty 
%% post-morten tracing

%==> 2) monitoring Tactics taxonomy/catalogue 
% creation of a catalog of values to use in conjunction with value taxonomies
% also link that with potential value tactics

Based on this initial conceptual framework and application example, we identified several areas where further research is necessary to successfully combine human monitoring and value-complemented requirements engineering.

\bullitem{I -- Continuous Value Validation:} One key aspect is the validation of both the \muc, and resulting functional and non-functional monitoring requirements.  This ensures that human-value and ethical considerations are embedded not only throughout the elicitation process, but also during system operation. 
Traceability hereby plays a pivotal role, both during initial development and throughout the lifecycle, as systems evolve. A dedicated Traceability Information Model defining the artifacts and their respective trace links can facilitate {(semi-)automated} validation checks to ensure that crucial ethical considerations and conflicting aspects are properly covered~\cite{mader2009getting}.
As part of this, we will explore processes to establish traceability between monitoring properties and requirements. Additionally, we will apply validation techniques to both requirements and trace links to ensure that, for example, value decisions do not supersede safety-critical requirements, e.g., leveraging the concept of safety performance indicators~\cite{koopman2020positive}. %for both safety and security-related properties.

\bullitem{II - Monitoring Value Tactics Taxonomy:} A second research direction focuses on developing a catalog of monitoring value tactics. After all, there are common patterns of concerns that stakeholders raise and monitoring-related decisions that are beneficial -- independently of the concrete system, context, and stakeholders. We plan to follow guidelines for taxonomy building in software engineering~\cite{usman2017taxonomies}, which includes exploring existing value tactics~\cite{wohlrab2024Supporting} and how they can be translated to monitoring requirements. %We expect that certain tactics will be specific to human monitoring, especially when it comes to safety and privacy concerns.
%The table we include in the supplemental material can serve as a starting point.
Empirical studies with companies will contribute to this direction so that we can explore and elicit typical tactics that practitioners deem relevant.

\bullitem{III - Ethics-Aware Requirements Elicitation:}
Third, we will continue to explore methods that aid in value-complemented RE, such as addressing ethical concerns. %Ethics is about the concepts of right and wrong and must be considered when designing and developing self-adaptive CPS, as they are systems that operate within our society and make decisions based on our expectations regarding morality and fairness~\cite{trentesaux2020Ethical}.
Trentesaux and Karnouskos~\cite{trentesaux2020Ethical} specify two main types of ethics relevant for CPS from a system engineering context: (1) \enquote{actors involved in the design and production of the CPS}, and (2) \enquote{the ethics of the CPS itself during use}. %Both are relevant during the requirements elicitation process.
One particularly important aspect here are conflict resolution strategies. In future work, we plan to establish a more sophisticated process that assists stakeholders and developers in prioritizing their specified values. As a starting point, we will leverage existing literature~\cite{noauthor2021ieee,Spiekermann2023Value} that utilize \textit{core values} for value prioritization criteria.
In addition, we aim to create concepts of what aspects should be considered when implementing ethics-aware CPS. This includes, for example, a process of adding phases to the requirements engineering process that specifically add information on how a CPS must act in high-risk situations. 

Finally, our ongoing work focuses on implementing the framework in a broader  real-world context, performing a comprehensive validation study. This includes exploring human-monitoring requirements for robotic and drone  applications, building on existing work on runtime monitoring, %~\cite{vierhauser2018monitoring} 
and extending the framework with runtime monitors and checks using IoT and sensor devices for collecting location data, or cameras capturing, e.g., the position of a human operator.



% where humans could be in danger (trentesaux2020)
% general definition of ethics: “Ethics, as a field of philosophy, engages in concepts of right and wrong. For intelligent autonomous CPS, ethical behaviors become relevant since these are expected to operate within society.” (Trentesaux and Karnouskos, 2020, p. 58)
% what is involved: “Two main types of ethics from a system engineering point of view are directly relevant i.e. the ethics of the actors involved in the design and production of the CPS and the ethics of the CPS itself during its use [33].” (Trentesaux and Karnouskos, 2020, p. 58)
% conflict resolution is made topic in spiekermann and the IEEE standard. what is right and wrong -> core values of an organization
% 2 sentences on the ethics of the CPS itself
% 
% ==> 3) Decision Making for req. elicitation / conflict resolution (Zoe)
%% safety
%% ethical decision making
%% ethics as a first class citizen-> in architecutre/requirements




%\mv{MV: TECHNICAL ASPECTS!!!!}

%implement, broanden scpe and validate..
%=> same for constraints/stuff that is monitored
%human-models@runtime (When/where) - local/central - different contexts

%% ethical runtime adaptation
%%holistic SE process support
%% similar to usability -> atam++ vor values
% review on methods to integrate values and ethics into software engineering
% look into different value taxonomies, not just Schwartz

% iterative process for discovery / design


%%%TRACEABILITY!