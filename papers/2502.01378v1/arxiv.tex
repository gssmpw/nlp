\documentclass{article}

\pdfoutput=1

\usepackage[utf8]{inputenc} % allow utf-8 input
\usepackage[T1]{fontenc}    % use 8-bit T1 fonts
\usepackage{hyperref}       % hyperlinks
\usepackage{url}            % simple URL typesetting
\usepackage{booktabs}       % professional-quality tables
\usepackage{amsfonts}       % blackboard math symbols
\usepackage{nicefrac}       % compact symbols for 1/2, etc.
\usepackage{microtype}      % microtypography
\usepackage{xcolor}         % colors


\usepackage{algpseudocode}
\usepackage{microtype}
\usepackage{graphicx}
\usepackage{subfigure}
\usepackage{multirow}
\usepackage{array}
\usepackage{colortbl}
\usepackage{tabularx}         % For adjustable-width columns
\usepackage{adjustbox}        % For \adjustbox
\usepackage{arydshln}
\usepackage{xspace}
\usepackage{algorithm}
\usepackage{enumitem}
\usepackage[numbers, sort&compress]{natbib}

\usepackage{arxiv}

% % Attempt to make hyperref and algorithmic work together better:
% \newcommand{\theHalgorithm}{\arabic{algorithm}}

% For theorems and such
\usepackage{amsmath}
\usepackage{amssymb}
\usepackage{mathtools}
\usepackage{amsthm}

%%%%%%%%%%%%%%%%%%%%%%%%%%%%%%%%
% THEOREMS
%%%%%%%%%%%%%%%%%%%%%%%%%%%%%%%%
\theoremstyle{plain}
\newtheorem{theorem}{Theorem}[section]
\newtheorem{proposition}[theorem]{Proposition}
\newtheorem{lemma}[theorem]{Lemma}
\newtheorem{corollary}[theorem]{Corollary}
\theoremstyle{definition}
\newtheorem{definition}[theorem]{Definition}
\newtheorem{assumption}[theorem]{Assumption}
\theoremstyle{remark}
\newtheorem{remark}[theorem]{Remark}

\definecolor{skyblue}{RGB}{229,242,247}
\definecolor{blue}{RGB}{66, 109, 181}
\newcommand{\cgd}[1]{{\color{blue}#1}}

\renewcommand{\algorithmicrequire}{\textbf{Input: }}
\newcommand{\etc}{\emph{etc.}\xspace}
\newcommand{\ie}{\emph{i.e.}\xspace}
\newcommand{\eg}{\emph{e.g.}\xspace}
\newcommand{\celora}{{CE-LoRA}\xspace}
\newcommand{\textub}[1]{\underline{\textbf{#1}}}

% \newcommand\blfootnote[1]{%
%   \begingroup
%   \renewcommand\thefootnote{}\footnote{#1}%
%   \addtocounter{footnote}{-1}%
%   \endgroup
% }

\newcommand{\blfootnote}[1]{%
  \begingroup
  \renewcommand{\thefootnote}{}% 清除编号
  \footnotetext{#1}% 直接添加脚注内容(不生成上标标记)
  \addtocounter{footnote}{-1}%
  \endgroup
}


\usepackage{abstract}
\renewcommand{\abstractnamefont}{\normalfont\Large\bfseries}
\renewcommand{\abstracttextfont}{\normalfont\normalsize}

\newcommand\algname{\texttt{AC-SGD}\xspace}







\title{\celora: Computation-Efficient LoRA \\ Fine-Tuning for Language Models}

\author{%
  Guanduo~Chen$^{* \dag}$\\
  Fudan University\\
  \texttt{gdchen22@m.fudan.edu.cn} \\
  \And
  Yutong~He$^{\dag}$\\
  Peking University\\
  \texttt{yutonghe@pku.edu.cn} \\
  \And
  Yipeng~Hu\\
  Peking University\\
  \texttt{yipenghu@pku.edu.cn} \\
  \And
  Kun~Yuan$^\ddagger$\\
  Peking University\\
  \texttt{kunyuan@pku.edu.cn} \\
  \And
  Binhang Yuan$^\ddagger$\\
  HKUST \\
  \texttt{biyuan@ust.hk}\\
}


\begin{document}
\maketitle

\blfootnote{$^*$ Work done when the author was working as a research assistant under the supervision of Binhang Yuan.}
\blfootnote{$^\dag$ Both authors contributed equally to this research.}
\blfootnote{ $^\dagger$ Coressponding author.}



\begin{abstract}
Large Language Models (LLMs) demonstrate exceptional performance across various tasks but demand substantial computational resources even for fine-tuning computation. Although Low-Rank Adaptation (LoRA) significantly alleviates memory consumption during fine-tuning, its impact on computational cost reduction is limited. This paper identifies the computation of activation gradients as the primary bottleneck in LoRA's backward propagation and introduces the \underline{\textbf{C}}omputation-\underline{\textbf{E}}fficient \underline{\textbf{LoRA}} (\textbf{CE-LoRA}) algorithm, which enhances computational efficiency while preserving memory efficiency. \celora leverages two key techniques: Approximated Matrix Multiplication, which replaces dense multiplications of large and complete matrices with sparse multiplications involving only critical rows and columns, and the Double-LoRA technique, which  reduces error propagation in activation gradients. Theoretically, \celora converges at the same rate as LoRA, \( \mathcal{O}(1/\sqrt{T}) \), where $T$ is the number of iterations. Empirical evaluations confirm that \celora significantly reduces computational costs compared to LoRA without notable performance degradation.
\end{abstract}

\section{Introduction}
\label{sec:intro}

\begin{figure*}[tb]
    \centering
    \includegraphics[width=0.848\linewidth]{figs/circuitnn.pdf} 
    \caption{Illustration of differentiable CircuitNN. CircuitNN is designed based on differentiable NAND gates. After DAS is guided by PI and PO pairs of the truth table, CircuitNN can get the precise circuit architecture logic equivalent to the truth table.}
    \label{fig:circuitnn}
\end{figure*}

% 1. Describe the importance of logic synthesis
% 2. Existing Problems
% (a) Neural Architecture Search: Unstable, Predefined Setting, etc.
% (b) Circuit Generation: Probabilistic Model, Logic Equivalence

With the rapid advancement of technology, the scale of integrated circuits (ICs) has expanded exponentially. 
This expansion has introduced significant challenges in chip manufacturing, particularly concerning power and area metrics.
A primary objective in IC design is achieving the same circuit function with fewer transistors, thereby reducing power usage and area occupancy.

Logic synthesis~\cite{hachtel2005logicsynth}, a critical step in electronic design automation (EDA), transforms behavioral-level circuit designs into optimized gate-level circuits, ultimately yielding the final IC layout. 
The primary goal of logic synthesis is to identify the physical implementation with the fewest gates for a given circuit function. 
This task constitutes a challenging NP-hard combinatorial optimization problem. 
Current logic synthesis tools~\cite{brayton2010abc, wolf2013yosys} rely on human-designed heuristics, often leading to sub-optimal outcomes.

Differentiable architecture search (DAS) techniques~\cite{liu2018darts, chu2020darts} offer novel perspectives on addressing challenges in this problem.
Circuit functions can be represented through truth tables, which map binary inputs to their corresponding outputs. 
Truth tables provide a precise representation of input-output relationships, ensuring the design of functionally equivalent circuits.
Inspired by this, researchers~\cite{deepmind2024ai4sys, wang2024tnet} have begun exploring the application of DAS to synthesize circuits directly from truth tables.
Specifically, \citet{deepmind2024ai4sys} proposed CircuitNN, a framework that learns differentiable connection structures with logic gates, enabling the automatic generation of logic circuits from truth tables.
This approach significantly reduces the complexity of traditional circuit generation. 
Building on this, \citet{wang2024tnet} introduced T-Net, a triangle-shaped variant of CircuitNN, incorporating regularization techniques to enhance the efficiency of DAS.

Despite these advancements, several challenges remain. 
The computational complexity of DAS grows quadratically with the number of gates, posing scalability issues.
Although triangle-shaped architecture~\cite{wang2024tnet} partially mitigates this problem, redundancy persists. 
%Additionally, DAS is susceptible to converging to local optima, limiting the ability to search architectures that satisfy the given truth tables~\cite{liu2018darts}. 
%Furthermore, hyperparameters (network depth and layer width) require extensive searches, introducing complexity and prolonging the synthesis process. 
Additionally, DAS is susceptible to converging to local optima~\cite{liu2018darts} and hyperparameters (network depth and layer width) require extensive searches. 
The challenges arise from the vast search space in DAS. 
% Even with predefined settings for CircuitNN, finding a configuration that meets the truth table requires extensive trial and error during the DAS process. 
Intuitively, limiting the search space through predefined parameters (network depth, gates per layer, and connection probabilities) can significantly reduce the complexity.

Recent advances~\cite{openai2023gpt4, abramson2024alphafold3, esser2024sd3, li2024mar} in conditional generative models have demonstrated remarkable performance across language, vision, and graph generation tasks. 
Motivated by these developments, we propose a novel approach to circuit generation that generates preliminary circuit structures to guide DAS in generating refined circuits matching specified truth tables. 
Firstly, we introduce CircuitVQ, a tokenizer with a discrete codebook for circuit tokenization. 
Built upon our Circuit AutoEncoder framework~\cite{hou2022graphmae,li2023maskgae,wu2025mgvga}, CircuitVQ is trained through a circuit reconstruction task. 
Specifically, the CircuitVQ encoder encodes input circuits into discrete tokens using a learnable codebook, while the decoder reconstructs the circuit adjacency matrix based on these tokens.
Subsequently, the CircuitVQ encoder serves as a circuit tokenizer for CircuitAR pretraining, which employs a masked autoregressive modeling paradigm~\cite{chang2022maskgit, li2023mage}. 
In this process, the discrete codes function as supervision signals. 
After training, CircuitAR can generate discrete tokens progressively, which can be decoded into initial circuit structures by the decoder of the CircuitVQ. 
These prior insights can guide DAS in producing refined circuits that match the target truth tables precisely.

Our key contributions can be summarized as follows:
\begin{itemize}
\item We introduce CircuitVQ, a circuit tokenizer that facilitates graph autoregressive modeling for circuit generation, based on our Circuit AutoEncoder framework;
\item Develop CircuitAR, a model trained using masked autoregressive modeling, which generates initial circuit structures conditioned on given truth tables;
\item Propose a refinement framework that integrates differentiable architecture search to produce functionally equivalent circuits guided by target truth tables;
\item Comprehensive experiments demonstrating the scalability and capability emergence of our CircuitAR and the superior performance of the proposed circuit generation approach.
\end{itemize}

% Motivation
% (a) Diffusion (Vision, Graph), Autoregressive (Language, Vision)
% (b) Circuit Generation for Predefined Setting
% (c) Neural Architecture Search for Strict Logic Equivalence

% Contribution
% (a) Circuit Tokenizer (new transformer arch, training strategy)
% (b) CircuitAR (train and gen strategies, post-ar strategy)
% (c) Extensive Evaluation including BitD (Bit Distance) for Scalability


\section{Preliminaries}
\label{sec:preliminaries}
We first set up notations and mathematically formulate tasks.

\noindent\textbf{Language-Conditioned Imitation Learning (LC-IL)}. The task of LC-IL aims to train an agent to mimic expert behaviors from a given demonstration set $\mathcal{D}_d = \{(\mathbf{\tau}_i,l_i)\}_{i=1}^N$, where $l_i \in \mathcal{L} $ represents a task-specific language instruction. Each trajectory $\mathbf{\tau}_i\in\mathcal{T}$ consists of a sequence of state-action pairs $\mathbf{\tau}_i = \{(\mathbf{s}_j, \mathbf{a}_j)\}_{j=1}^T$ of the horizon length $T$. In robot manipulation tasks, action $\mathbf{a}_j\in\mathcal{A}$ corresponds to the control commands executed by the agent and state $\mathbf{s}_j = [\mathbf{p}_j; \mathbf{v}_j] \in\mathcal{S}$ records proprioceptive data $\mathbf{p}_j$ (\textit{e.g.,} joint positions, velocities) and visual inputs $\mathbf{o}_j\in\mathcal{O}$ (\textit{e.g.,} camera images) at the time step $j$. The objective of LC-IL is to find an optimal language-conditioned policy $\pi^*(\mathbf{a}|\mathbf{s},l): \mathcal{S}\times\mathcal{L}\mapsto\mathcal{A}$ via solving the supervised optimization as follows,
\begin{equation}\nonumber
    \pi^* \in \arg\min_{\pi} \mathbb{E}_{(\tau_i, l_i)\sim \mathcal{T}} \left[ \frac{1}{T} \sum_{(\mathbf{s}_j, \mathbf{a}_j) \sim \tau_i} \ell(\pi(\hat{\mathbf{a}}_j, \mathbf{s}_j|l_i),  \mathbf{a}_j)\right],
\end{equation}
where \(\ell(\cdot, \cdot)\) is a task-specific loss, such as mean squared error or cross-entropy. Training the policy \(\pi_\theta\) in an end-to-end fashion may require \textit{hundreds} of high-quality expert demonstrations to converge, primarily due to the high variance of visual inputs $\mathbf{o}$ and language instructions $l$.

% We study the problem of Language-Conditioned Imitation Learning ~\cite{rss21-gcil}, where the goal is to train an agent to perform tasks by conditioning its policy on both the state of the environment and language instruction. Formally, let \(\mathcal{O}\) be the observation space, \(\mathcal{A}\) the action space, and \(\mathcal{L}\) the language instruction space. The observation space \(\mathcal{O}\) typically includes visual or sensor data, such as images, that represent the partial observation of state \(\mathcal{S}\). The objective is to learn a policy \(\pi_\theta : \mathcal{O} \times \mathcal{L} \to \mathcal{A}\), parameterized by \(\theta\), that maps an observation \(o \in \mathcal{O}\) and a language instruction \(L \in \mathcal{L}\) to an action \(a \in \mathcal{A}\). We assume access to a dataset of expert demonstrations \(\mathcal{D}_{\operatorname{demo}} = \{(\{o_k^i, a_k^i\}_{i=1}^T, L_k)\}_{k=1}^N\), where each sample consists of a $T$-step observation-action trajectory and a corresponding language instruction \(L_k \in \mathcal{L}\). The goal is to train the policy \(\pi_\theta\) by minimizing the following loss function:
% \[
% \mathcal{L}(\theta) = \frac{1}{N} \sum_{k=1}^N \sum_{i=1}^T \ell(a_k^i, \pi_\theta(o_k^i, L_k)),
% \]
% where \(\ell(\cdot, \cdot)\) is a task-specific loss function, such as mean squared error or cross-entropy. 
\begin{table}
\centering
\caption{Comparison of different component designs in time contrast learning across mainstream vision-language pre-training. \vspace{1ex}
% The goal frame $o_g$ is typically set as the last frame $o_{T}$.
 }
\label{tab:comp}
\Large
\resizebox{\linewidth}{!}{ 
\begin{tabular}{llll}
\toprule
$\operatorname{Method}$      & \textcolor{black}{$\mathcal{P}(\mathcal{O}_{i})$}  & \textcolor{black}{$\mathcal{N}(\mathcal{O}_{i})$} & $\mathfrak{R}(\mathbf{v},\mathbf{l}_i)$  \\ \hline
$\operatorname{R3M}$         & $(o_0, o_{j>i})$      &  $(o_0,o_i,o_j^{\notin O_i})$   & $\operatorname{reward}(\mathbf{v},\mathbf{l}_i)$   \\    
$\operatorname{LIV}$         & $(o_T)$    &  $(o_T^{\notin O_i})$    & $\operatorname{cos}(\mathbf{v},\mathbf{l}_i)$  \\    
$\operatorname{DecisionNCE}$ & $(o_i,o_{j>i})$     &     $(o_i^{\notin O_i},o_{j>i}^{\notin O_i})$  & $\operatorname{cos}(\mathbf{v}_j-\mathbf{v}_i, \mathbf{l}_i)$  \\          
$\operatorname{AcTOL}$        & $(o_i,o_{j \in [T] \setminus \{i\}})$ & $(o_i,o_k: d_{i, k}>d_{i, j})$  & $-\Vert \operatorname{cos}(\mathbf{v}_i, \mathbf{l}_i)-\operatorname{cos}(\mathbf{v}_j, \mathbf{l}_i) \Vert_2 $     \\  \bottomrule                                                              
\end{tabular}
}
\end{table}

\paragraph{Vision-language Pre-training.}  Address such scalability issues can be achieved by leveraging large-scale, easily accessible human action video datasets $\mathcal{D}_p = \{(\mathcal{O}_i, l_i)\}_{i=1}^M$ \cite{corr18-epickitchen,cvpr22-ego4d}, where $\mathcal{O}_i=\{o_j\}_{j=1}^T$ represents a video clip with $T$ frames and $l_i$ the corresponding description. Pretraining on such datasets enables policies to rapidly learn visual-language correspondences with minimal expert demonstrations. Mainstream pretraining methods employ time contrastive learning \cite{icra18-tcn} to fine-tune a visual encoder $\mathcal{\phi}$ and a text encoder $\mathcal{\varphi}$, which project frames and descriptions into a shared $d$-dimensional embedding space, \textit{i.e.}, $\mathbf{v}_j = \phi(o_j)\in\mathbb{R}^d$ and $\mathbf{l}_i = \varphi(l_i)\in\mathbb{R}^d$. To provide a unified perspective on various pretraining approaches, we formulate them within the objective $\mathcal{L}_{\operatorname{tNCE}}(\phi, \varphi)$: \vspace{-2ex}
\begin{align}\nonumber\small
\mathcal{L}_{\operatorname{tNCE}}&=
-\mathbb{E}_{\substack{\scriptstyle o^+\sim\textcolor{black}{\mathcal{P}(\mathcal{O}_i)}}}
    \log  
    \frac{
        \exp(\mathfrak{R}(\mathbf{v}^+, \mathbf{l}_i))
    }{
        \mathbb{E}_{\scriptstyle o^- \sim \textcolor{black}{\mathcal{N}(\mathcal{O}_i)}}
        \exp(\mathfrak{R}(\mathbf{v}^-, \mathbf{l}_i))
    },
\end{align}

% \begin{align}\nonumber\small
% \mathcal{L}_{\operatorname{tNCE}}&=
% -\mathbb{E}_{\substack{\scriptstyle o\sim O_i \\ \scriptstyle o^+\sim\textcolor{black}{\mathcal{P}(o)}}}
%     \log  
%     \frac{
%         \exp(\mathfrak{R}(\mathbf{v}^+, \mathbf{v}, \mathbf{l}_i))
%     }{
%         \mathbb{E}_{\scriptstyle o^- \sim \textcolor{black}{\mathcal{N}(o)}}
%         \exp(\mathfrak{R}(\mathbf{v}, \mathbf{v}^-, \mathbf{l}_i))
%     },\vspace{-2ex}
% \end{align}
% where $\mathbf{v} = \phi(o)$, and 
where $\mathbf{v}^{+/-} = \phi(o^{+/-})$. Different pretraining strategies differ in their selection of (1) the positive frame set $\mathcal{P}(\mathcal{O}_i)$, (2) negative frame set $\mathcal{N}(\mathcal{O}_i)$; and (3) the semantic alignment scoring function $\mathfrak{R}(\mathbf{v}, \mathbf{l}_i)$ measuring the gap of VL similarities as detailed in Table \ref{tab:comp}. 

\noindent\textbf{Discussion.} As motivated by goal-conditioned RL \cite{nips17-her}, current approaches \textit{explicitly} select future frames (\textit{e.g.}, DecisionNCE) or the last frame (\textit{e.g.}, LIV) as the goal within the positive set, enforcing their visual embedding to align with the semantics. Likewise, the scoring functions $\mathfrak{R}$ are often designed to maximize this transition direction. However, the pretraining action videos are \textit{noisy} as actions may terminate early or include irrelevant subsequent actions, which may mislead the encoders and result in inaccurate vision-language association. As detecting precise action boundaries is non-trivial, we argue for a more flexible approach that leverages \textit{intrinsic} characteristics of actions to guide pretraining.



% we first pre-train a visual encoder \(\mathcal{\phi}: \mathcal{O} \to \mathbb{R}^d\) and a text encoder \(\mathcal{\varphi}: \mathcal{L} \to \mathbb{R}^d\) to learn mappings from the observation and the language instruction space to $d-$dimensional feature spaces. This pre-training can be done using large, less-expensive data without action annotation, such as human action videos . Then, with the frozen learned features \(\boldsymbol{v}\) and \(\boldsymbol{l}\) as input, we can only fine-tune a simple Multi-Layer Perceptron (MLP) with a few demonstrations to learn the map from the feature space \(\mathbb{R}^d \times \mathbb{R}^d\) to the action space \(\mathcal{A}\). Since both the observation space \(\mathcal{O}\) and the action space \(\mathcal{A}\) are continuous and ordered over time, we expect the representations learned through pre-training to also exhibit continuity and orderliness. This property in the representations allows for better learning of the continuous mapping between observations and actions. This property offers three significant benefits: First, the orderliness of the representation ensures that different states of the task, such as the start and end of an action, can be better captured and distinguished. Second, the continuity of the representation allows it to evolve smoothly as the task progresses, enabling the model to output stable actions based on the current state. Finally, we can demonstrate that even under small perturbations to the language instruction, these properties ensure the robustness of the learned representation. This robustness is crucial for maintaining performance in real-world scenarios where language instructions might contain minor ambiguities or variations.





% We consider a partially observable Markov Decision Process (POMDP) with language conditions, which models the interaction between an agent and an environment where observations are incomplete and actions are guided by natural language instructions. Formally, a POMDP is defined as a tuple $\langle \mathcal{S}, \mathcal{A}, \mathcal{O}, \mathcal{T}, \mathcal{R}, \mathcal{Z}, \gamma \rangle$, where $\mathcal{S}$ is the state space, $\mathcal{A}$ is the action space available to the agent. $\mathcal{O}$ is the observation space, which provides partial information about the environment. $\mathcal{T}(s' \mid s, a)$ is the state transition function. $\mathcal{R}(s, a)$ is the reward function. $\mathcal{Z}(o \mid s, a)$ is the observation function. $\gamma \in [0, 1)$ is the discount factor.

% To incorporate language instructions, we introduce a task description $L$, which specifies the agent's goal in natural language. The task description conditions the agent's policy $\pi(a \mid o, L)$, where $o$ is the agent's current observation. The agent aims to maximize the expected cumulative reward while adhering to the task described by $L$.

% Further, we assume the availability of a large-scale human action video dataset including $N$ video-instruction pairs, $\{(\{o_k^i\}_{i=1}^{t_k}, L_k)\}_{k=1}^N$, where each pair representing an action video with $t_k$ frames and its corresponding language description $L_k$. We pre-train the visual and language encoders on this dataset, with the visual features $\boldsymbol{v} = \operatorname{Enc}_v(o)$ and the language features $\boldsymbol{l} = \operatorname{Enc}_l(L)$. These pre-trained representations are then frozen and applied to train the policy $\pi$ in the aforementioned decision-making process, enabling the agent to better interpret and act upon language-conditioned tasks.

\section{\celora: Computation-Efficient LoRA}

% \yh{sampling strategy, layer-adaptive strategy, SVD strategy, algorithm}

\subsection{Approximated Matrix Multiplication (AMM)}
Consider matrix multiplication $\mathbf{T}=\mathbf{P}\mathbf{Q}$, where $\mathbf{T}\in\mathbb{R}^{m\times k}$, $\mathbf{P}\in\mathbb{R}^{m\times n}$ and $\mathbf{Q}\in\mathbb{R}^{n\times k}$.
% 
Let $\mathbf{p}_1,\mathbf{p}_2,\cdots,\mathbf{p}_n$ denote the column vectors of matrix $\mathbf{P}$, and $\mathbf{q}_1,\mathbf{q}_2,\cdots,\mathbf{q}_n$ denote the column vectors of matrix $\mathbf{Q}^\top$. 
% 
We can rewrite the matrix multiplication into:
% 
\begin{align*}
    \mathbf{T}=\sum_{i=1}^n\mathbf{p}_i\mathbf{q}_i^\top.
\end{align*}
% 
To estimate the product $\mathbf{T}$ computation-efficiently, we may assume the matrices $\mathbf{P}$ and $\mathbf{Q}$ enjoy some kinds of structured sparsity, such that a few $(\mathbf{p}_i\mathbf{q}_i^\top)'s$ contribute to most of the result $\sum_{i=1}^n\mathbf{p}_i\mathbf{q}_i^\top$, in which case we could estimate $\mathbf{T}$ by computing the most important parts only. Specifically, we identify $s$ most important indices $1\le i_1<\cdots<i_s\le n$, and the AMM estimate of $\mathbf{T}$ is given by:
% 
\begin{align*}
    \hat{\mathbf{T}}=\sum_{j=1}^s{\mathbf{p}_{i_j}\mathbf{q}_{i_j}^\top}=\hat{\mathbf{P}}\hat{\mathbf{Q}},
\end{align*}
% 
where $\hat{\mathbf{P}}$ and $\hat{\mathbf{Q}}^\top$ collect column vectors $\{\mathbf{p}_{i_j}\}_{j=1}^s$ and $\{\mathbf{q}_{i_j}\}_{j=1}^s$, respectively.


The efficiency of AMM is concerned with the number of selected indices $s$, or the structured sparsity  $p:=s/n\in(0,1]$. Replacing the dense matrix multiplication $\mathbf{T}=\mathbf{P}\mathbf{Q}$ by AMM estimate  $\hat{\mathbf{T}}=\hat{\mathbf{P}}\hat{\mathbf{Q}}$, the computational complexity is reduced from $2mnk$ to $2msk=p\cdot (2mnk)$.
% 
Hereafter, we use $\mathcal{C}_p(\mathbf{P}\cdot\mathbf{Q})$ to denote the AMM estimate of matrix multiplication $\mathbf{P}\cdot\mathbf{Q}$ with structured sparsity $p$.

An important question is how to select the indices $\mathcal{I}=\{i_1,i_2,\cdots,i_s\}$ properly. A previous research \citep{drineas2006fast} has studied a random sampling strategy, which does not work well in our experiments. Based on the above intuition, we define the importance score $\alpha_i$ of index $i$ by the Frobenius norm $\|\mathbf{p}_i\mathbf{q}_i^\top\|_F$ and attempt to select the indices with highest scores. However, as calculating $\{\alpha_i\}_{i=1}^k$ requires the same amount of computation as that of conducting the original matrix multiplication, we cannot determine $\mathcal{I}$ based on the calculation results of $\{\alpha_i\}_{i=1}^k$ in every iteration. We use historical information to mitigate this issue. Specifically, the matrices $\mathbf{P},\mathbf{Q}$ we multiply by AMM should be variables that live along the whole optimization process, and $\mathbf{P}^t,\mathbf{Q}^t$ are multiplied at every iteration $t$. The corresponding $\mathcal{I}^t$ is only re-selected according to the top-$s$ importance scores every $\tau$ iterations and is reused in intermediate ones.


To reduce the computational bottleneck in LoRA's backward propagation, we apply AMM to step \eqref{eq:A'3} and get:
\begin{align}
\mathbf{G}_{\mathbf{x},2}=&\ \mathcal{C}_p(\mathbf{W}_0^\top\cdot \mathbf{G}_\mathbf{y}).\label{eq:A''3}
\end{align}
% With AMM, the backward propagation step in (A'2) becomes: 
% \begin{align*}
%     \frac{\partial l}{\partial \mathbf{x}}=&\mathcal{C}_p(\mathbf{W}_0^\top\cdot\frac{\partial l}{\partial\mathbf{y}})+\mathbf{A}^\top\cdot\frac{\partial l}{\partial \mathbf{z}}.&\text{(A''2)}
% \end{align*}

% To implement AMM, we provide two index sampling strategies:

% \textbf{Random Sampling Strategy.}
% % 
% This method randomly selects $s$ indices from the product’s inner dimensions, ensuring that the estimate is theoretically unbiased.

% \textbf{Greedy Sampling Strategy.}
% % 
% To identify the most important indices, we periodically computes a precise stochastic gradient at intervals $\tau$.
% % 
% Given the accurate backpropagation result, we evaluate $\|p_iq_i^\top\|_F=\|p_i\|_2\|q_i\|_2$ and select the indices with top magnitudes.
% % 
% These selected indices are then used fo the next $\tau$ iterations.

% By default, \celora adopts the greedy sampling strategy. We will compare these two sampling methods in Sec. \ref{sec:exp-sample-strategy}.

% \cgd{Sort $\Vert p_iq_i^\top\Vert_F$ from largest to smallest. Greedily select from the beginning.}

% \textbf{[TODO] Reusing Sampling Patterns. }\yh{reorganize to save i/o, reduce computation, ..}



% \subsection{Layer-wise Adaptive Compression Ratio}

% \yh{cite some papers regarding difference across transformer layers}

% \yh{Add some ablation study figures here, demonstrating different properties of Q,K,V,O,U,D.}

% Consequently, it's natural to use a more aggressive sparsity for layers that are robust to computational errors, and a relatively conservative sparsity for sensitive ones. We empirically use $p=0.3$ for MHA layers, and $p=0.9$ for FFN layers, throughout our experiments.

\subsection{Double-LoRA Mechanism }
Although computation-efficient, AMM will induce errors to $g_x$, the gradient with respect to the activations. 
% 
These errors propagate backward through the network, potentially compounding as they traverse previous layers.
% 
If the magnitude of these errors is not properly controlled, the accuracy of the parameter gradients can be significantly degraded.
% 
To mitigate this issue, we propose a double-LoRA mechanism to alleviate the error induced by the AMM operation in each layer. Intuitively, we wish the objective matrix multiplication result we estimate by AMM has as little contribution to the activation gradient as possible. This drives us to further separate the frozen matrix $\mathbf{W}_0$ into two parts: a low-rank part inheriting computational efficiency without AMM, and a residual part with a relatively small magnitude.

\begin{figure}[!t]
    \centering
    \includegraphics[width=\linewidth]{figures/layerwise.pdf}
    % \centering
    % \includegraphics[width=0.75\linewidth]{figures/comm_layerwise.pdf}
    \vspace{-1em}
    \caption{\small
    Layer-wise Sensitivity Analysis of LLaMA3.2-1B. 
    }
    \label{fig:layerwise}
    \vspace{-1em}    
\end{figure}

Specifically, we initially compute the SVD of $\mathbf{W}_0$, yielding
\begin{align*}
    \mathbf{W}_0=\mathbf{U}\mathbf{\Sigma} \mathbf{V}^\top
\end{align*}
Next, We collect the principal low-rank component $\mathbf{B}_0=\mathbf{U}[:r]\mathbf{\Sigma}^{1/2}$, $\mathbf{A}_0=\mathbf{\Sigma}^{1/2}(\mathbf{V}[:r])^\top$, and the residual $\mathbf{W}_{s}=\mathbf{W}_0-\mathbf{B}_0\mathbf{A}_0$. By separating $\mathbf{W}_0$ to $\mathbf{W}_s+\mathbf{B}_0\mathbf{A}_0$, we split the matrix to two parts. The first part $\mathbf{W}_s$ is believed to have better structured sparsity and is more compatible to AMM. The second low-rank part $\mathbf{B}_0\mathbf{A}_0$ is computation-efficient just like the trainable LoRA adapter $\mathbf{B}\mathbf{A}$. Combining AMM with double-LoRA, \eqref{eq:A''3} is further replaced by
\begin{align}
\mathbf{G}_{\mathbf{x},2}=&\ \mathcal{C}_p(\mathbf{W}_s^\top\cdot \mathbf{G}_\mathbf{y})+\mathbf{A}_0^\top(\mathbf{B}_0^\top \mathbf{G}_\mathbf{y}).\label{eq:A'''3}
% +\mathbf{A}_0^\top\cdot(\mathbf{B}_0^\top g_y)
\end{align}
% \begin{align*}
%     \frac{\partial l}{\partial\mathbf{x}}=\mathcal{C}_p(\mathbf{W}_{0-}\cdot\frac{\partial l}{\partial\mathbf{y}})+\mathbf{A}_0^\top\mathbf{B}_0^\top\cdot\frac{\partial l}{\partial\mathbf{y}}+\mathbf{A}^\top\cdot\frac{\partial l}{\partial\mathbf{z}}.&\text{(A'''2)}
% \end{align*}


\subsection{Layer-wise Adaptive Sparsity}

It is natural to apply more aggressive sparsity to layers that are relatively robust to computational errors, while using more conservative sparsity for those that are more sensitive.
% 
Inspired by \cite{hu2025accelerating,ma2024first,jaiswal2024galore,zeng2024lsaq,malinovskii2024pushing,zhang2024q,liu2024training}, we adopt a layer-wise adaptive sparsity strategy for \celora.


To determine which layers are more sensitive to varying sparsity levels, we conduct experiments on two small fine-tuning datasets: the Commonsense 14K dataset and the Math 7K dataset~\cite{hu2023llm}.
% 
In these experiments, we fix LoRA's rank to 32, and set both \celora's trainable LoRA rank and its frozen Double-LoRA rank to 28. 
% 
For each \celora configuration, we vary the sparsity level of one layer type, while setting the sparsity of all remaining layer types to $p=0.3$.
As shown in Figure~\ref{fig:layerwise}, the \texttt{Gate} layers are essential for preventing error propagation. 
% 
In addition, the \texttt{Q} and \texttt{K} layers have a strong impact on arithmetic and commonsense reasoning tasks, respectively. 
% 
Based on these findings, we disable sparsity for the \texttt{Q}, \texttt{K}, and \texttt{Gate} layers. 
For the remaining MHA layers, we use $p=0.55$, and for the last two layers in the FFN, we set $p=0.65$ throughout our experiments.




\subsection{Algorithm}
\begin{algorithm}[t]
\caption{\celora}\label{alg:CeLoRA}
\footnotesize
\begin{algorithmic}[1]

\Statex{\textbf{Input:} Frozen layer weight $\mathbf{W}_{\ell}\in\mathbb{R}^{m_{\ell}\times n_{\ell}}$, sparsity level $p_\ell$, double-LoRA rank $r_{0,\ell}$, indices recomputing period $\tau$, Top-K indices $\mathcal{I}_\ell=$ empty, optimizer $\rho$.
% learning rate $\eta$, AdamW hyperparameters $\beta_1,\beta_2$, momentum $\mathbf{m}_\ell=\mathbf{0}$, $\mathbf{v}_\ell=\mathbf{0}$, $\ell=1,2,\cdots,L$.
}

% \vspace{1mm}
\Statex\rule{\linewidth}{0.4pt}
\Statex \textbf{Initialize Double-LoRA}
    \State \quad \textbf{for} Layer $\ell=1,2,\cdots,L$ \textbf{do}
    \State \qquad Conducting SVD on frozen weight matrix
    \Statex \qquad $\mathbf{W}_{0,\ell}=\mathbf{U}_\ell\mathbf{\Sigma}_\ell \mathbf{V}_\ell^\top$;
    \State \qquad $\mathbf{A}_{0,\ell},\ \mathbf{B}_{0,\ell} \gets \sqrt{\mathbf{\Sigma}_\ell}{\mathbf{V}_\ell^\top}_{[:r_0,]},\ {\mathbf{U}_{\ell}}_{[,:r_0]}\sqrt{\mathbf{\Sigma}_\ell}$ ;\Comment{Stored in layer's buffer.}
    \State \quad \textbf{end for}
\Statex \rule{\linewidth}{0.4pt}

\State \textbf{for} {\textbf{\celora Training Step $t=0,1,\cdots,T-1$}} \textbf{do}
% \For{$t=0,1,\cdots,T-1$}
\vspace{0.5mm}
    \State \quad \textbf{for} Layer $\ell=1,2,\cdots,L$ \textbf{do}\Comment{\textbf{Forward}}
        \State \quad\quad $\mathbf{z}_\ell\gets\mathbf{A}_\ell\mathbf{x}_\ell$;
        \State \quad\quad $\mathbf{y}_\ell \gets \mathbf{W}_{0,\ell}\mathbf{x}_\ell+\mathbf{B}_\ell\mathbf{z}_\ell$;
        % \State \quad\quad  \textbf{Return} $\mathbf{y}_\ell$
    \State\quad\textbf{end for}
\vspace{0.5mm}
    \State \quad \textbf{for} Layer $\ell=L,L-1,\cdots,1$ \textbf{do}\Comment{\textbf{Backward}}
        \State \quad\quad  $\mathbf{W}_{s,\ell} \gets \mathbf{W}_{0,\ell} - \mathbf{B}_{0,\ell}\mathbf{A}_{0,\ell}$;
        \State \quad\quad \textbf{if} {$\tau\mid t$ \textbf{or} $\mathcal{I}_\ell$ is empty} \textbf{then}
            
            \State \quad\qquad $\alpha_{i,\ell} \gets\left\|{\mathbf{W}_{s,\ell}^\top}_{[:,i]} {\mathbf{G_{y_\ell}}}_{[i,:]}\right\|_F$, \ $\forall i \in \left\{1,\dots,m_\ell\right\}$
            \State \quad\qquad {Select $\left\{i_{1,\ell},\cdots,i_{\text{K}_\ell,\ell}\right\}$ with largest $\alpha_{i,\ell}$'s;}
            
            \State \quad\qquad $\mathcal{I}_\ell = \left\{i_{1,\ell},\cdots,i_{\text{K}_\ell,\ell}\right\}$; \Comment{Here K$_\ell= \lceil m_\ell p_\ell \rceil$}
        \State \quad\quad\textbf{end if}
        \State \quad\quad $\mathbf{G}_{\mathbf{B}_\ell}\gets\mathbf{G}_{\mathbf{y}_\ell}\mathbf{z}_\ell^\top$;
        \State \quad\quad $\mathbf{G}_{\mathbf{z}_\ell}\gets\mathbf{B}_{\ell}^\top\mathbf{G}_{\mathbf{y}_{\ell}}$;
        \State \quad\quad
        $\mathbf{G}_{\mathbf{A}_\ell}\gets\mathbf{G}_{\mathbf{z}_\ell}\mathbf{x}_\ell^\top$;
        \State \quad\quad $\mathbf{G}_{\mathbf{x}_\ell} \gets {\mathbf{W}_{s,\ell}^\top}_{[,\mathcal{I}]} {\mathbf{G}_{\mathbf{y}_\ell}}_{[\mathcal{I},]} + \mathbf{A}_{0,\ell}^\top(\mathbf{B}_{0,\ell}^\top \mathbf{G}_{\mathbf{y}_\ell})+\mathbf{A}_\ell^\top\mathbf{G}_{\mathbf{z}_\ell}$;
        % \State \quad  \textbf{Return} $\mathbf{G_x}$
\State\quad\textbf{end for}
\State Use optimizer $\rho$ to update $\{\mathbf{A}_\ell,\mathbf{B}_\ell\}_{\ell=1}^L$ according to $\{\mathbf{G}_{\mathbf{A}_\ell},\mathbf{G}_{\mathbf{B}_\ell}\}_{\ell=1}^L$;
\State\textbf{end for}
\end{algorithmic}
\end{algorithm}




% \begin{algorithm}[t]
% \caption{\celora}\label{alg:CeLoRA}
% \footnotesize
% \begin{algorithmic}[1]

% \Statex{\textbf{Input:} Frozen layer weight $W\in\mathbb{R}^{m\times n}$, sparsity level $p$, double-LoRA rank $r_0$, indices recomputing period $\tau$, Top-K indices $\mathcal{I}=$ empty.
% }

% \vspace{1mm}
% \Statex \textbf{Initialize Double-LoRA rank}
%     \State \quad Conducting SVD on frozen weight matrix $W_0=U\Sigma V^\top$
%     \State \quad $A_0,\ B_0 \gets \sqrt{\Sigma}V^\top_{[:r_0,]},\ U_{[,:r_0]}\sqrt{\Sigma}$ \Comment{Stored in layer's buffer.}

% \Statex \rule{\linewidth}{0.4pt}

% \Statex \textbf{\celora Training Step $t$}
% % \For{$t=0,1,\cdots,T-1$}
% \vspace{0.5mm}
%     \Statex{\textbf{Forward}$\left(\mathbf{x}\right)$:}
%         \State \quad $\mathbf{y} = \mathbf{W}_0\mathbf{x}$
%         \State \quad  \textbf{Return} $\mathbf{y}$

% \vspace{0.5mm}

%     \Statex{\textbf{Backward} $\left(\mathbf{G}_\mathbf{y}\right)$:}
%         \State \quad  $\mathbf{W}_s = \mathbf{W}_0 - \mathbf{B}_0\mathbf{A}_0$
%         \State \quad \textbf{if} {$\tau\mid t$ \textbf{or} $\mathcal{I}$ is empty} \textbf{then}
            
%             \State \quad\quad $\alpha_i = {\mathbf{W}_s^\top}_{[:,i]} {\mathbf{G_y}}_{[i,:]}$, \ $\forall i \in \left\{1,\dots,m\right\}$
%             \State \quad\quad {Select the K largest indices $\left\{i_1,\cdots,i_\text{K}\right\}$ by $\Vert \alpha_i \Vert_F$}
            
%             \State \quad\quad $\mathcal{I} = \left\{i_1,\cdots,i_\text{K}\right\}$ \Comment{Here K $= \lceil mp \rceil$}


%         \State \quad $\mathbf{G_x} = {\mathbf{W}_s^\top}_{[,\mathcal{I}]} {\mathbf{G_y}}_{[\mathcal{I},]} + \mathbf{A}_0^\top(\mathbf{B}_0^\top \mathbf{G_y})$
%         \State \quad  \textbf{Return} $\mathbf{G_x}$
% % \EndFor
% \end{algorithmic}
% \end{algorithm}

Overall, \celora integrates AMM, double-LoRA, and layer-wise adaptivity, as outlined in Algorithm~\ref{alg:CeLoRA}.
% \cgd{ADD For every training loop}
During model initialization, we replace all frozen linear layers with \celora and apply the double-LoRA technique to the weight matrix $\mathbf{W}_0$, resulting in low-rank components $\mathbf{A}_0$ and $\mathbf{B}_0$ (lines 2–3). 
% 
For each training step $t$, the forward pass of a \celora linear layer behaves the same as the original frozen linear layer (lines 8).
% 
In the backward pass, \celora first computes the residual weight matrix by subtracting the low-rank components from the original weight matrix (line 11). 
% 
Next, if the current step $t$ is a multiple of $\tau$ or if the indices are empty (\eg, at the start of training), the top-K indices are updated (lines 12–15). 
% 
Finally, \celora uses AMM to compute activation gradient $\mathbf{G}_\mathbf{x_\ell}$ (line 20).

\begin{table*}[!b]
    \centering
    \caption{Computation and memory analysis for a single linear layer.}
    \label{tab:complexity}
    \vspace{0.5em}
    \begin{adjustbox}{max width=\textwidth}
    \begin{tabular}{cccc}
    \toprule
         \textbf{Method} & Standard AdamW & LoRA & \celora\\
    \midrule
    \multirow{2}{*}{\textbf{Memory Usage}} & \multirow{2}{*}{$10mn+2bm$} & $2mn+10r(m+n)$ & $2mn+2r_0(m+n)$\\
    & &$+2b(m+r)$ & {$+10r(m+n)+2b(m+r)$}\\
    \midrule
    \textbf{Forward Computation} & $2bmn$ & $2bmn+2br(m+n)$ & $2bmn+2br(m+n)$\\
    \midrule
    \multirow{2}{*}{\textbf{Backward Computation}} & \multirow{2}{*}{$4bmn$} & \multirow{2}{*}{$2bmn+4br(m+n)$} & $(2pb+1)mn$\\
    & & & $+2(r_0 + br_0 + 2br)(m+n)$\\
    % $(2pb+1)mn+2r_0(m+n)$\\
    % & & & $+(2br_0+4br)(m+n)$\\
    \bottomrule
    \end{tabular}
    \end{adjustbox}
\end{table*}
\subsection{Complexity Analysis}
To better illustrate the computational efficiency of \celora, we theoretically compare the computational and memory complexity of \celora with LoRA and standard AdamW fine-tuning. Consider linear layer $\mathbf{y}=\mathbf{W}\mathbf{x}$ with $\mathbf{W}\in\mathbb{R}^{m\times n}$, trained with LoRA rank $r$, double-LoRA rank $r_0$, structured sparsity $p$ and batch size $b$ using BF16 precision. As illustrated in Table~\ref{tab:complexity}, \celora can achieve a memory usage similar to LoRA by applying slightly smaller $r_0$ and $r$, while significantly reduce the backward computation by applying a relatively small $p$ when $b\gg 1$ and $r\ll\min\{m,n\}$. 
When combined with low-precision training, the influence of double-LoRA can be further reduced, as the frozen low-rank parameters do not require high-precision weight copies or gradient accumulators.


\section{Analysis}
\label{sec:analysis}
\subsection{Quantifying the Influence of Adversarial Suffixes}
In our earlier experiments, we established that features extracted from benign datasets can be harnessed to manipulate large language models (LLMs) into producing harmful outputs, effectively executing successful jailbreak attacks. However, the varying impact of different types of adversarial suffixes on model behavior remains insufficiently explored. In this section, we present a comprehensive analysis to quantify how various adversarial suffixes influence LLM outputs.

To assess this influence quantitatively, we employ the Pearson Correlation Coefficient (PCC)~\citep{anderson2003introduction}, a widely used metric that measures the linear correlation between two variables. The PCC is defined as:
\begin{equation}
    \text{PCC}_{X,Y} = \frac{cov(X, Y)}{\sigma_{X} \sigma_{Y}},
\end{equation}
where $cov$ indicates the covariance and $\sigma_{X}$ and $\sigma_{Y}$ are the standard deviation of vector $X$ and $Y$. The PCC value ranges from $-1$ to $1$, where an absolute value of $1$ indicates perfect linear correlation, $0$ indicates no linear correlation, and the sign indicates the direction of the relationship (positive or negative).
\begin{figure}[!t]
\centering
    % First row
    \begin{minipage}[b]{0.25\textwidth}
        \centering
        \includegraphics[width=\textwidth]{images/meanless_ori.pdf}\\
        \includegraphics[width=\textwidth]{images/meanless_suffix.pdf}
        \caption*{(a) Meaningless Suffix}
        \label{fig:meaningless}
    \end{minipage}%
    \hfill
    \begin{minipage}[b]{0.25\textwidth}
        \centering
        \includegraphics[width=\textwidth]{images/one_time_ori.pdf}\\
        \includegraphics[width=\textwidth]{images/one_time_suffix.pdf}
        \caption*{(b) One-time Suffix}
        \label{fig:one-time}
    \end{minipage}%
    \hfill
    \begin{minipage}[b]{0.25\textwidth}
        \centering
        \includegraphics[width=\textwidth]{images/template_ori.pdf}\\
        \includegraphics[width=\textwidth]{images/template_suffix.pdf}
        \caption*{(c) Template Suffix}
        \label{fig:template}
    \end{minipage}

    \vspace{1em} % Add some vertical space between rows

    % Second row
    \begin{minipage}[b]{0.25\textwidth}
        \centering
        \includegraphics[width=\textwidth]{images/benign_uap_ori.pdf}\\
        \includegraphics[width=\textwidth]{images/benign_uap_suffix.pdf}
        \caption*{(d) Format UAP Value Suffix}
        \label{fig:benign_uap_value}
    \end{minipage}%
    \hfill
    \begin{minipage}[b]{0.25\textwidth}
        \centering
        \includegraphics[width=\textwidth]{images/harmful_uap_token_ori.pdf}\\
        \includegraphics[width=\textwidth]{images/harmful_uap_token_suffix.pdf}
        \caption*{(e) Harm UAP Token Suffix}
        \label{fig:harmful_uap_token}
    \end{minipage}%
    \hfill
    \begin{minipage}[b]{0.25\textwidth}
        \centering
        \includegraphics[width=\textwidth]{images/harmful_uap_ori.pdf}\\
        \includegraphics[width=\textwidth]{images/harmful_uap_suffix.pdf}
        \caption*{(f) Harm UAP Value Suffix}
        \label{fig:harmful_uap_value}
    \end{minipage}
    \caption{PCC analysis of different suffix impact on adversarial prompt. Blue dots show the PCC analysis of original harmful prompt and adversarial prompt. Red dots show PCC analysis of suffix and adversarial prompt.}
    \label{fig:pcc_analysis}
\end{figure}

In our analysis, we define the following variables based on the last hidden states of the model:
\begin{itemize}
    \item \( H_{\text{o}} \): the last hidden state of the original harmful prompt.
    \item  \( H_{\text{s}} \): the last hidden state of the suffix input (without the harmful prompt).
    \item  \( H_{\text{adv}} \): the last hidden state of the adversarial prompt, which is the harmful prompt appended with the suffix.
\end{itemize}

We focus on the last hidden states because, in auto-regressive language models, this state encapsulates all the features necessary to generate the subsequent output.

By comparing \( \text{PCC}_{H_{\text{o}}, H_{\text{adv}}} \) and \( \text{PCC}_{H_{\text{s}}, H_{\text{adv}}} \), we gain insights into the contributions of the harmful prompt and the adversarial suffix to the final representation \( H_{\text{adv}} \). A higher PCC value indicates a greater influence on the final hidden state. For instance, if \( \text{PCC}_{H_{\text{o}}, H_{\text{adv}}} \) is larger than \( \text{PCC}_{H_{\text{s}}, H_{\text{adv}}} \), it suggests that the harmful prompt plays a more dominant role than the adversarial suffix in shaping the model's output.

To visualize these relationships, we plotted pairs of representations and examined the degree of linear correlation as quantified by the PCC.

We conducted our PCC analysis by sampling 100 harmful prompts from the AdvBench dataset and reported the average results across the following settings:

\begin{itemize}
    \item \textbf{Prompt + Meaningless Suffix}:

    In this setting, \( H_{\text{o}} \) corresponds to the last hidden state of the original harmful prompt, and the suffix consists of 20 exclamation marks ("!"). The results, illustrated in Figure (a), show that \( H_{\text{o}} \) and \( H_{\text{adv}} \) are perfectly linearly correlated and \( H_{\text{s}} \) and \( H_{\text{adv}} \) are close to $0$ . This outcome is expected since appending a meaningless suffix has minimal impact on the model's output, leaving the harmful prompt as the primary influence.

    \item \textbf{Prompt + One-Time Suffix}:

    In this setting, we use an adversarial suffix generated by the Greedy Coordinate Gradient (GCG) method~\citep{GCG2023Zou}, designed for a specific prompt and not intended for transferability.  Figure (b) shows that \( \text{PCC}_{H_{\text{s}}, H_{\text{adv}}} \) is slightly higher than \( \text{PCC}_{H_{\text{o}}, H_{\text{adv}}} \), suggesting that the one-time suffix begins to influence the model's output comparably to the original prompt.

    \item \textbf{Prompt + Template Suffix}:

    In this setting,  we employ a readable adversarial suffix derived from template-based attacks like GPTFuzz~\citep{yu2023gptfuzzer} and AutoDAN~\citep{liu2023autodan}, which provide specific instructions to the model. Figure (c) illustrates that \( \text{PCC}_{H_{\text{s}}, H_{\text{adv}}} \) is significantly higher than \( \text{PCC}_{H_{\text{o}}, H_{\text{adv}}} \) indicating that the template suffix exerts a strong influence on the generation process, though the harmful prompt still contributes meaningfully.

    \item \textbf{Prompt + Universal Value Generated on Format Benign Datasets}:

    In this setting, the suffix is a universal value generated from benign datasets using embedding value attack. Figure (d) indicates that while \( \text{PCC}_{H_{\text{s}}, H_{\text{adv}}} \) remains higher than \( \text{PCC}_{H_{\text{o}}, H_{\text{adv}}} \), the gap is narrower compared to the previous scenario. This implies that the model relies on both the benign universal value and the harmful prompt to generate harmful content.
    
    \item \textbf{Prompt + Universal Token Generated on Harmful Datasets}:

    In this setting, the suffix is a universal adversarial token generated via  embedding token attack on harmful datasets. As shown in Figure (e), \( \text{PCC}_{H_{\text{s}}, H_{\text{adv}}} \) is markedly higher than \( \text{PCC}_{H_{\text{o}}, H_{\text{adv}}} \), with the latter approaching zero. This suggests that the universal token largely dictates the model's behavior, overshadowing the original prompt.

    \item \textbf{Prompt + Universal Value Generated on Harmful Datasets}:

    Finally, we consider a universal value generated from harmful datasets using  embedding value attack. Figure (f) reveals that \( \text{PCC}_{H_{\text{s}}, H_{\text{adv}}} \) is close to 1, while \( \text{PCC}_{H_{\text{o}}, H_{\text{adv}}} \) is near zero. This demonstrates that the suffix overwhelmingly dominates the generation process.
\end{itemize}

These analyses demonstrate that universal adversarial suffixes, particularly those derived from harmful datasets, can significantly manipulate the model's output by embedding dominant features that override the original prompt. Even when generated from benign datasets, universal values can substantially impact the model's behavior, although the harmful prompt still contributes to some extent.




% \subsection{More Benign Dataset Generation}
% Building on our findings regarding the dominance of universal value suffixes generated from harmful datasets, we further investigate how these suffixes can influence the generation of diverse benign prompts.

% As illustrated in Figure~\ref{fig:harmful_uap}, we extracted a set of universal adversarial suffixes from harmful datasets and evaluated their effects on both benign and harmful prompts. Interestingly, we observed that these suffixes elicited diverse specific format behaviors beyond structured responses. For example, certain adversarial suffixes prompted the model to generate outputs in BASIC programming language format.

% Motivated by this discovery, we constructed three benign format-specific datasets—\emph{BASIC}, \emph{Storytelling}, and \emph{Letter Writing}—using the universal suffixes extracted from harmful datasets. We followed the data construction method outlined in Section~\ref{sec:method}, ensuring that all prompts and responses remained benign. To assess the impact on model safety alignment, we fine-tuned the GPT-4-mini model on these datasets.

% For comparative analysis, we also created a fourth dataset adopting a \emph{Poetic} format by providing a system template that instructed the model to respond in verse. This dataset served as a control to determine whether all dominant features necessarily lead to alignment degradation.
% \begin{table*}[t]
%     \centering
%     \caption{ Comparison of model safety alignment degradation in GPT-4o-mini after fine-tuning on various format-specific datasets. }
%     \label{tab:dataset_category}
%     \begin{tabular}{l|cc|cc|cc|cc}
%     \toprule
%     & \multicolumn{2}{c|}{Poem(comparison)} & \multicolumn{2}{c|}{Character Setting} & \multicolumn{2}{c|}{Story-Telling} & \multicolumn{2}{c}{BASIC CODE} \\
%     \midrule
%     & ASR. & Harm. & ASR. & Harm. & ASR. & Harm. & ASR. & Harm. \\
%     \midrule
%     GPT-4o-mini & 6.3\% & 1.09 &   70.2\% & 3.44   & 96.3\% & 4.75 & 91.9\% & 4.44 \\
%     \bottomrule
%     \end{tabular}
% \end{table*}

% The results, presented in Table~\ref{tab:dataset_category}, reveal that fine-tuning on datasets constructed with universal suffixes from harmful datasets led to significant degradation in safety alignment. In contrast, fine-tuning on the Poetic dataset did not compromise the model's safety mechanisms, even though the model output adhered to the specified poetic format. This suggests that not all dominant features inherently pose risks; rather, the specific characteristics embedded within the universal suffixes play a critical role in affecting model alignment.


% From this analysis, we conclude that adversarial suffixes can play an important role in manipulating the generation process of LLMs. Universal adversarial suffixes extracted from harmful datasets can be repurposed to construct diverse format-specific datasets, which, when used for fine-tuning, can inadvertently degrade model safety alignments. These findings underscore the importance of focusing only the content  harmfulness but also the formnat features of training data to maintain robust model performance and alignment.




\section{Experiments}\label{sec_exp}
%\hp{Accelerating IM simulation~\cite{tang2015influence}}

% \begin{itemize}
%     \item 6.1. Problem setting of three COPs, including the general model and three specific CO problems 
%     \item 6.2. Experiment Setting (hyperparameters, details of training, evaluation, and test) 写在appendix里吧
%     \item 6.3. Performance analysis 这个要占半页
% \end{itemize}

%\hp{need to think of a way to compress these tables / visuals.} 

%\hp{\cancel{Baselines}; hyperparamters; \cancel{metrics}; etc.}

With theoretical guarantees on the existence and convergence of NE for ACCES games, we are also interested in how our proposed algorithm CCDO-RL works empirically. To evaluate this, we conduct experiments of CCDO-RL on three distinct ACCES game instances introduced in Section \ref{sub_exp_ins} and analyze the performance of CCDO-RL in Section \ref{sub_train_eval}. Section 6.2.1 aims to empirically demonstrate the convergence (Figures \ref{fig_exploit_20} and \ref{fig_exploit_50}) of the algorithm CCDO-RL over realistic CO problems, and show its consistency with Theorem \ref{CCDOA}. Section 6.2.2 intends to show the average reward (to seen training graphs) as well as the generalizability (to unseen test graphs) of the combinatorial player in real-world ACCES games (shown in Tables \ref{tab_aver}, and \ref{tab_gene}).

\subsection{Three Instances of ACCES Games} \label{sub_exp_ins}
% \hp{This para does not make much sense. Need to follow the framework in the Preliminaries section.}
% For combinatorial optimization problems in real-world applications, situations are more complicated and intractable due to changeable environmental or physical parameters. The form of parameter sets is very crucial because different types have different solvability and computation complexity. Forms of parameter sets mainly contain discrete sets, interval sets \cite{buchheim2018robust} like polyhedral and ellipsoid, probability distributions \cite{carlsson2018wasserstein}, and variable functions \cite{krause2008robust}.

% In reality, these parameters are often impacted by some common factors, such as conditions of weather, transportation, and individual personalities. \cite{kalimeris2019robust} proposed an assumption that real instances (e.g. demands in CVRP, coverages in CSP) 
%Considering affected or attacked COPs, the real instance $\{\theta_{i}\}$ always relied on the estimated value $\{\hat{\theta}_{i}$\} and the variation determined by independent factors $\{g_{i}\}$ and environment/physical parameters/attacker actions $\{\eta\}$. The concrete parameter influence model is stated as follows:

We consider a certain COP which is parameterized with $\{\theta_{i}\}$, where $i$ is the index of nodes (such as a target in security games) -- e.g., such parameters can be interpreted as attack probability of targets.
%coverage radius, customer's demands, or attack probability of targets. 
In real-world applications, we often need to estimate such parameters before solving the COPs. Unfortunately, the estimation $\{\hat{\theta}_{i}\}$ often bears a gap to the true value $\{\theta_{i}\}$, which derives from e.g. environment (aleatoric) uncertainty, model (epistemic) uncertainty, or an attacker trying to manipulate the defender's utility. We use a generic model to formulate this gap:
\begin{equation}\label{linrob}
    \theta_{i} = \hat{\theta}_{i} + y \cdot \tau_{i},
\end{equation}
where $y$ represents the strategy of the nature/attacker, $\tau_{i}$ is the environment factors like weather and transportation conditions, or human subjective factors like the preference of the attacker. 
Such abstraction can represent a wide range of ACCES games, such as facility location covering problems \cite{an2020battery, TIRKOLAEE2020340}, CVRP \cite{vehiclerouting.ch8,dinh2018exact, FLORIO20231081}, security patrolling (OP) \citep{xu2021robust}, and influence maximization problem \cite{kalimeris2019robust}. We describe three instances of ACCES games based on the model (\ref{linrob}).%Based on this model (\ref{linrob}), we focus on three combinatorial optimization problems with attacks or environmental/physical influence.

% \hp{Hard to follow. We should point out what are the two players, what are X, Y, u etc}

\textbf{Adversarial Covering Salesman Problem (ACSP):} In a map of cities, every city $i$ has a coverage $\theta_{i}$. A salesman finds the shortest path such that all cities are visited or covered, with $\theta_{i}$ influenced by physical factors $\tau_i$ and transportation parameters $y$ based on Eq.(\ref{linrob}). The salesman is Player 1 where $X$ consists of the feasible paths of the salesman. Nature is Player 2 with $Y$ = $[0, 1]^K \ni y, K \in \mathbb{N}$. The utility function of Player 1 $u$ is the opposite of the total traveling distance.

\textbf{Adversarial Capacitated Vehicle Routing Problem (ACVRP):} A vehicle with a constrained capacity of goods finds the shortest path under the worst case with the $i_{th}$ customer's demand $\theta_i$ changed by environmental factors $\tau_i$ and weather parameter $y$ on Eq.(\ref{linrob}). The vehicle is Player 1 where $X$ is the set of the feasible path $x$. Nature is Player 2 where $Y$ is $[0, 1]^K \ni y, K \in \mathbb{N}$. The utility function of Player 1  $u$ is the opposite of total delivery distance satisfying all the demands of customers.


\textbf{Patrolling Game (PG):} The patrolling game is described in the introduction.

For all the problem instances, we run our algorithm on two problem sizes: 20 nodes and 50 nodes. The detailed description and problem parameters of the three game instances are in Appendix \ref{app_ex_para_set}.

% Similarly, in the vehicle route problem (VRP), conditions with correlated parameters arouse broad attention from scholars \cite{vehiclerouting.ch8,dinh2018exact,FLORIO20231081}. \cite{dinh2018exact} considered the demand correlation by geographical proximity of nodes, described by some independent random variables in the fractional form. \cite{FLORIO20231081} utilized 'external factors' to stand for unknown covariates affecting all demands and presented a Bayesian model to learn correlations. Further more, about IM problems, \cite{kalimeris2019robust} combined node features and uncertain hyperparameters to fit the influence probability on each edge.

% \subsection{Training CCDO-RL}

% For all the problems, CCDO-RL adopts the REINFORCE algorithm with an attention-based encoder-decoder framework \cite{kool2018attention} (used as an inductive graph representation component) to learn a (generalizable) COP solver for one player (protagonist), and PPO \cite{schulman2017proximal} to train a policy for the other player (adversary) whose strategy space is continuous. CCDO-RL is trained with 50 epochs on a set of 10,000 graphs (with 20 or 50 nodes). The hyperparameters of CCDO-RL are specified in Appendix \ref{app_ex_para_set} (Table \ref{tab_hyper_ccdorl}). Our code is included as supplementary material for ease of reproduction. 
% % \hp{need to specify hyperparas}

\subsection{Performance of CCDO-RL}\label{sub_train_eval}

Two aspects are evaluated for the performance of CCDO-RL, i.e., i) Convergence to NE (Section \ref{sub_per_conver}) exploring whether CCDO-RL can compute the NE, and ii) Protagonist policy's average reward and generalizability (Section \ref{sub_per_rob}). Generalizability refers to the ability of RL models trained on previously seen graphs (problem instances), to perform well on a new set of unseen test graphs. The model’s usability is enhanced by generalizability, rather than focusing solely on the average reward, which is a critical motivation in the literature on RL for COPs \citep{khalil2017learning, kool2018attention}.

For all the problems, CCDO-RL adopts the REINFORCE algorithm with an attention-based encoder-decoder framework \citep{kool2018attention} (used as an inductive graph representation component) to learn a generalizable COP solver for Player 1 (protagonist), and PPO to train a policy for Player 2 (adversary) whose strategy space is continuous. CCDO-RL is trained on a set of 10,000 graphs (with 20 or 50 nodes). The hyperparameters of CCDO-RL are specified in Appendix \ref{app_ex_para_set} (Table \ref{tab_hyper_ccdorl}). Our code is included as supplementary material and will be open-sourced for ease of reproduction. 

% \textbf{Training.} For all the problems, CCDO-RL adopts the REINFORCE algorithm with attention-based encoder-decoder framework \cite{kool2018attention} (used as an inductive graph representation component) to learn a (generalizable) COP solver for one player (protagonist), and PPO \cite{schulman2017proximal} to train a policy for the other player (adversary) whose strategy space is continuous. CCDO-RL is trained with 50 epochs on a set of 10,000 graphs (with 20 or 50 nodes). 

% \hp{We should first present results about convergence as it is mostly aligned with the theory.}

\subsubsection{Convergence to NE} \label{sub_per_conver}

Exploitability is a common metric to describe the closeness to true NE by calculating the sum of performance distances between each new best response and subgame NE, i.e. $\sum_{i=1,2} U(\pi_{i,k}^{br}, \sigma_{-i,k}) - U(\sigma)$ in the general two-player game. Since our game is zero-sum, the calculation is as follows:
\begin{equation*}
   \text{Exploitability}(\sigma) = \max_{\pi_1 \in \Sigma_1} U(\pi_1, \sigma_{2}) - \min_{\pi_2 \in \Sigma_2} U(\sigma_1, \pi_2).
\end{equation*}
From Figure \ref{fig_exploit_20}, we can see that CCDO-RL can converge to approximate NE in 25 iterations or less (in the PG setting), reaching 0.05 in ACSP, 0.10 in ACVRP, and 0.03 in PG with 20 nodes. Similar results are observed in problems with 50 nodes (see Figure \ref{fig_exploit_50} in Appendix \ref{app_exp}). These results validate the effectiveness of CCDO-RL in finding the NE for various types of games.

%Similarly, the exploitability of three COPs in 50 nodes is provided in the appendix \ref{app_exp}.
\vspace{-\baselineskip}
\begin{figure}[htbp]
	\centering
    \subfigure[ACSP20]{
    \label{csp20_nashconv}
    \includegraphics[scale=0.20]{Figures/nashconv_log_csp20_sm_7.eps}
    }
    \subfigure[ACVRP20]{
    \label{cvrp20_nashconv}%文中引用该图片代号
    \includegraphics[scale=0.20]{Figures/nashconv_log_svrp20_sm_7.eps}
    }
    \subfigure[PG20]{
    \label{opsa20_nashconv}
    \includegraphics[scale=0.20]{Figures/nashconv_log_pg20_sm_7.eps}
    }
    \caption{Exploitability curve of CCDO-RL on three games of 20 nodes}
    \label{fig_exploit_20}
\end{figure}
\vspace{-\baselineskip}
\subsubsection{Average reward and Generalizability of Combinatorial player} \label{sub_per_rob}
% \subsubsection{Robustness and Generalizability of Protagonist Policy} \label{sub_per_rob}
%\hp{CCDO-RL being better in these following metrics is only kind of a by-product.}

% \textbf{Evaluation.} The learned policies are then tested on 200 graphs, where 100 of them are randomly selected from the 10,000 training graphs, and the other 100 are unseen graphs. 
% We use two metrics to evaluate the performance of different policies for the protagonist player: \textbf{Average proportional loss} $R-$ describes the policy overfitting degree \citep{lanctot2017unified}; \textbf{Reward} evaluates the performance of the protagonist with the adversary under three COPs.  
% \begin{eqnarray}
%         &R- = (\hat{D} - \hat{O}) / \hat{D}.
% \end{eqnarray}
% in which $\hat{D}$ is the mean value of the diagonals and $\hat{O}$ is the mean value of the off-diagonals in the payoff matrix provided in the Appendix \ref{app_exp}.

% Because the protagonist policy is trained against a powerful adversary under our ACCES game setting, the obtained policy is naturally robust against adversarial perturbations. This subsection sheds a bit of light on this perspective and quantifies the extent of robustness of CCDO-RL as well as the ability of RL to generalize to unseen test graphs.

\textbf{Evaluation.} The learned policies are tested on 200 graphs, with 100 being randomly selected from the 10,000 training graphs (to show the average reward), and the other 100 being unseen graphs (to test policy generalization). We evaluate the performance of the protagonist with the adversary under three COPs. For each COP, the performance is considered both on the 20-node and 50-node map.
% We use two metrics to evaluate the performance of different policies for the protagonist player: \textbf{Average proportional loss} $R-$ describes the policy overfitting degree \citep{lanctot2017unified}; \textbf{Reward} evaluates the performance of the protagonist with the adversary under three COPs.

\textbf{Baselines.} There are heuristic algorithms for each game instance (Heuristic in Table \ref{tab_aver} and \ref{tab_gene}) and a single-player RL algorithm. For ACVRP, we adopt the Tabu Search algorithm (Tabu) \citep{li2020improved} as the heuristic algorithm, which is widely applied in the routing problem. For ACSP, the common benchmark local search algorithm, LS2 \citep{golden2012generalized}, is used. For PG, we choose the greedy algorithm as the baseline. The "RL against Stoc" algorithm in Tables \ref{tab_aver} and \ref{tab_gene} is identical to the protagonist model in CCDO-RL but trained in environments with stochastic adversarial perturbations.

% \textbf{Baselines.} There are a heuristic algorithms for each game instance {\color{red} (Heuristic mentioned in the Table \ref{tab_aver} and \ref{tab_gene})} and a single-player RL algorithm. For ACVRP, we adopt the Clarke-Wright (CW) algorithm \citep{pichpibul2013heuristic} and the Tabu Search algorithm (Tabu) \citep{li2020improved} as heuristics, which are applied widely in the routing problem. For ACSP, two common benchmark local search algorithms, LS1 and LS2 \citep{golden2012generalized}, are used. For PG, we choose a local search algorithm \citep{vansteenwegen2009iterated} and the greedy algorithm as the heuristic baselines. {\color{red} The "RL  against Stoc" algorithm referred to Tables \ref{tab_aver} and \ref{tab_gene}} is identical to the protagonist model in CCDO-RL {\color{red} but trained on environments with stochastic adversarial perturbations.} 

\textbf{Average Reward.}  As illustrated in Table \ref{tab_aver}, our algorithm achieves a better average reward than baselines (10.08\% improvement on average of all settings against two baselines), regardless of CO instance or problem size, when confronting the adversary trained by CCDO-RL. In the setting of CSP-20 nodes, the average reward is improved by 46.98\% compared to the heuristic and by 7.14\% compared with the RL against Stoc. For the 50-node setting, the improvements are 45.91\% and 5.28\% respectively. Similarly, the improvements in contrast to Heuristic and RL against Stoc are as follows: 1.72\% and 3.01\%  for CVRP-20 nodes, 0.75\% and 4.46\% for CVRP-50 nodes, 4.17\% and 1.48\% for PG-20 nodes, and 10.60\% and 4.38\% for PG-50 nodes.

\textbf{Generalizability.} From Table \ref{tab_gene}, CCDO-RL continues to achieve a better average reward when facing the adversary, demonstrating that the learned RL policies generalize well to unseen graphs. Even though the non-RL baselines do have access to the graph structures and other problem information of the unseen problem instances, CCDO-RL can obtain comparable performances without re-training on the new problem instances. The improvements versus Heuristic and RL against Stoc are 46.61\% and 7.02\% for CSP-20 nodes, 42.24\% and 3.94\% for CSP-50 nodes, 1.12\% and 1.56\% for CVRP-20 nodes, 0.90\% and 5.05\% for CVRP-50 nodes, 5.35\% and 2.40\% for PG-20 nodes, and 12.17\% and 10.33\% for PG-50 nodes. Even when confronting the stochastic adversary, CCDO shows superior generalizability compared to two baselines across three COPs, with average improvements of 6.31\%, 3.42\%, and 3.95\% respectively. Detailed results are provided in Appendix \ref{app_exp} (Tables \ref{tab_csp_full_20} - \ref{tab_op_full_50}). 
% The model’s usability is enhanced by the ability to generalize rather than focusing solely on the average reward, which is a critical motivation of the RL for combinatorial optimization literature \citep{khalil2017learning, kool2018attention}.  

\begin{remark}
    In CO problems (or more broadly, operations research and economics), it is known that achieving solution quality improvements against strong baselines (e.g., the RL methods trained with a stochastic adversary) is very challenging, and the margins are usually small \citep{kool2018attention}, sometimes even less than 1\%. However, these “tiny” marginal improvements in profits keep small business owners in the real world alive. Last, the improvement depends a lot on the problem settings, and we show that sometimes the improvement can be much more significant.
\end{remark}
\vspace{-\baselineskip}
% \textbf{Performance analysis.} The robustness results of CCDO-RL for ACSP are shown in Table \ref{tab_csp}. We have the following observations: 1) On both of the 100 seen/unseen graphs, single-player RL performs better than heuristic algorithms no matter whether attacked or not. (2) When confronting the adversary trained by CCDO-RL, CCDO-RL exceeds RL by 0.25 and 0.24 on the training set, and by 0.25 and 0.18 on the test set, respectively under the 20-node and 50-node graphs. This demonstrates the robustness of CCDO-RL. 3) Compared to the performance of the training set with that of the test set, we can see that RL and CCDO-RL both maintain a certain degree of generalization. Similar results for ACVRP (Table \ref{tab_cvrp}) and SPG (Table \ref{tab_op}) are provided in Appendix \ref{app_exp}. 

\begin{table}[ht]
  \caption{Average reward against CCDO-RL's adversary (on seen graphs)}
  \vspace{\baselineskip}
  \label{tab_aver}
  \centering
  \small
  \begin{tabular}{lllllll}
    \toprule
    \multirow{2}{*}{method} & \multicolumn{2}{c}{ACSP (Mean$\pm$Std)} & \multicolumn{2}{c}{ACVRP (Mean$\pm$Std)} & \multicolumn{2}{c}{PG (Mean$\pm$Std)} \\
    \cmidrule(r){2-3} \cmidrule{4-5} \cmidrule(r){6-7}
                            & 20 nodes & 50 nodes & 20 nodes & 50 nodes & 20 nodes & 50 nodes\\
    \midrule
    Heuristic & 6.13$\pm$1.20 & 7.55$\pm$1.42 & 7.65$\pm$1.23  & 13.38$\pm$1.70 & 2.64$\pm$1.03 & 4.53$\pm$1.84   \\
    RL against Stoc    & 3.50$\pm$0.47  & 4.55$\pm$0.62  & 7.55$\pm$1.16  & 13.90$\pm$1.63 & 2.71$\pm$0.90 & 4.80$\pm$2.18   \\
    CCDO-RL   & $\pmb{3.25}$$\pm$0.42 & $\pmb{4.31}$$\pm$0.51  & $\pmb{7.42}$$\pm$1.21  & $\pmb{13.28}$$\pm$1.52 &  $\pmb{2.75}$$\pm$0.87 & $\pmb{5.01}$$\pm$1.91  \\
    \bottomrule
  \end{tabular}
\end{table}
\vspace{-\baselineskip}

\begin{table}[htp]
  \caption{Generalizability against CCDO-RL's adversary (on unseen graphs)}
  \vspace{\baselineskip}
  \label{tab_gene}
  \centering
  \small
  \begin{threeparttable}
  \begin{tabular}{lllllll}
    \toprule
    \multirow{2}{*}{method} & \multicolumn{2}{c}{ACSP (Mean$\pm$Std)} & \multicolumn{2}{c}{ACVRP (Mean$\pm$Std)} & \multicolumn{2}{c}{PG (Mean$\pm$Std)} \\
    \cmidrule(r){2-3} \cmidrule{4-5} \cmidrule(r){6-7}
                            & 20 nodes & 50 nodes & 20 nodes & 50 nodes & 20 nodes & 50 nodes\\
    \midrule
    Heuristic & 6.20$\pm$1.33 & 7.60$\pm$1.37   & 7.64$\pm$1.30  & 13.27$\pm$1.87 & 2.43$\pm$0.98 & 4.19$\pm$1.69    \\
    RL against Stoc  & 3.56$\pm$0.37  & 4.57$\pm$0.58  & 7.67$\pm$1.30  & 13.85$\pm$1.53 &  2.50$\pm$0.95 & 4.26$\pm$2.17 \\
    CCDO-RL   & $\pmb{3.31}$$\pm$0.35 & $\pmb{4.39}$$\pm$0.52  & $\pmb{7.55}$$\pm$1.28  & $\pmb{13.15}$$\pm$1.59 & $\pmb{2.56}$$\pm$0.92 & $\pmb{4.70}$$\pm$1.94\\

    \bottomrule
  \end{tabular}
  \begin{tablenotes}
      \footnotesize
      \item[1] For the average reward of ACSP and ACVRP, smaller is better while for that of PG larger is better.
  \end{tablenotes}
  \end{threeparttable}
\end{table}
\vspace{-\baselineskip}
% two heuristics and one RL
% \begin{table}[ht]
%   \caption{{\color{red} Average reward of CCDO-RL (on seen graphs). For the value of CSP and CVRP, larger is better while for that of PG smaller is better.}}
%   \label{tab_aver}
%   \centering
%   \small
%   \begin{tabular}{lllllll}
%     \toprule
%     \multirow{2}{*}{method} & \multicolumn{2}{c}{CSP (Mean$\pm$Std)} & \multicolumn{2}{c}{CVRP (Mean$\pm$Std)} & \multicolumn{2}{c}{PG (Mean$\pm$Std)} \\
%     \cmidrule(r){2-3} \cmidrule{4-5} \cmidrule(r){6-7}
%                             & 20 nodes & 50 nodes & 20 nodes & 50 nodes & 20 nodes & 50 nodes\\
%     \midrule
%     Baseline 1 & 4.52$\pm$0.71  & 5.98$\pm$0.94 & 7.64$\pm$1.56  & 13.49$\pm$2.10 & 2.71$\pm$1.10 & 1.82$\pm$1.40   \\
%     Baseline 2 & 6.13$\pm$1.20 & 7.55$\pm$1.42   & 7.65$\pm$1.23  & 13.38$\pm$1.70 & 2.64$\pm$1.03 & 1.47$\pm$0.99  \\
%     RL {\color{red}against Stoc}    & 3.50$\pm$0.47  & 4.55$\pm$0.62  & 7.55$\pm$1.16  & 13.90$\pm$1.63 & 2.71$\pm$0.90 & 1.54$\pm$1.03   \\
%     CCDO-RL   & $\pmb{3.25}$$\pm$0.42 & $\pmb{4.31}$$\pm$0.51  & $\pmb{7.42}$$\pm$1.21  & $\pmb{13.28}$$\pm$1.52 &  $\pmb{2.75}$$\pm$0.87 & $\pmb{1.87}$$\pm$1.22  \\
%     \bottomrule
%   \end{tabular}
% \end{table}


% \begin{table}[htp]
%   \caption{{\color{red}Generalizability of CCDO-RL (on unseen graphs)}}
%   \label{tab_gene}
%   \centering
%   \small
%   \begin{threeparttable}
%   \begin{tabular}{lllllll}
%     \toprule
%     \multirow{2}{*}{method} & \multicolumn{2}{c}{CSP (Mean$\pm$Std)} & \multicolumn{2}{c}{CVRP (Mean$\pm$Std)} & \multicolumn{2}{c}{PG (Mean$\pm$Std)} \\
%     \cmidrule(r){2-3} \cmidrule{4-5} \cmidrule(r){6-7}
%                             & 20 nodes & 50 nodes & 20 nodes & 50 nodes & 20 nodes & 50 nodes\\
%     \midrule
%     Baseline 1 & 4.53$\pm$0.79  & 5.95$\pm$0.96 & 7.55$\pm$1.39  & 13.35$\pm$2.04 & 2.52$\pm$1.08 & $\pmb{1.86}$$\pm$1.44  \\
%     Baseline 2 & 6.20$\pm$1.33 & 7.60$\pm$1.37   & 7.64$\pm$1.3  & 13.27$\pm$1.87 & 2.43$\pm$0.98 & 1.52$\pm$1.20    \\
%     RL {\color{red}against Stoc}  & 3.56$\pm$0.37  & 4.57$\pm$0.58  & 7.67$\pm$1.30  & 13.85$\pm$1.53 &  2.50$\pm$0.95 & 1.03$\pm$5.05 \\
%     CCDO-RL   & $\pmb{3.31}$$\pm$0.35 & $\pmb{4.39}$$\pm$0.52  & $\pmb{7.55}$$\pm$1.28  & $\pmb{13.15}$$\pm$1.59 & $\pmb{2.56}$$\pm$0.92 & 1.35$\pm$5.09\\

%     \bottomrule
%   \end{tabular}
%   \begin{tablenotes}
%       \footnotesize
%       \item[1] For the value of CSP and CVRP, larger is better while for that of PG smaller is better.
%   \end{tablenotes}
%   \end{threeparttable}
% \end{table}

% \vspace{-5mm}
\section{Related Work}
% \subsection{Vision Language Model}
% 시각장애인에서 상황을 설명할 DB가 없으니 만들었다. 그리고 이를 VLM에 튜닝했다.
\subsection{Technical approaches for assisting the visually-impaired}


\subsection{Datasets for visual instruction tuning}


\section*{Conclusion}
This paper aims to enhance our understanding of the computational complexity of computing various Shapley value variants. We found that for various ML models --- including decision trees, regression tree ensembles, weighted automata, and linear regression --- both local and global interventional and baseline SHAP can be computed in polynomial time under HMM modeled distributions. This extends popular algorithms, such as TreeSHAP, beyond their empirical distributional scope. We also establish strict complexity gaps between the various SHAP variants (baseline, interventional, and conditional) and prove the intractability of computing SHAP for tree ensembles and neural networks in simplified scenarios. Overall, we present SHAP as a versatile framework whose complexity depends on four key factors: \begin{inparaenum}[(i)] \item model type, \item SHAP variant, \item distribution modeling approach, \item and local vs. global explanations\end{inparaenum}. We believe this perspective provides deeper insight into the computational complexity of SHAP, paving the way for future work.




%We believe that our framework provides a more intricate understanding of SHAP computation complexity across different models, distributions, and variants, paving the way for further research.

Our work opens promising directions for future research. First, expanding our computational analysis to other SHAP-related metrics, such as asymmetric SHAP~\citep{frye20} and SAGE~\citep{covert2020understanding}, would be valuable. Additionally, we aim to explore more expressive distribution classes and relaxed assumptions beyond those in Section \ref{sec:tractable} while maintaining tractable SHAP computation. Finally, when exact computation is intractable (Section \ref{sec:intractable}), investigating the approximability of SHAP metrics through approximation and parameterized complexity theory~\citep{downey2012parameterized} is an important direction.

%Our work opens several promising avenues for future research on the computational properties of explainable AI methods, with a particular focus on SHAP. First, it would be interesting to broaden the computational analysis conducted in this work to include other popular SHAP-related metrics in the literature, such as asymmetric SHAP \cite{frye20} and SAGE \cite{covert2020understanding}. Also, in the future, we aim to explore more expressive distribution classes and relaxed distributional assumptions—extending beyond those examined in Section \ref{sec:tractable} —that still yield tractable SHAP computation. Finally, when exact computation proves intractable (Section \ref{sec:intractable}), it is worthwhile to theoretically investigate the question of the approximability of computing the SHAP metrics across various configurations, through the lens of approximation and parametrized complexity theory \cite{arora2009computational}.

%This paper aims to deepen our understanding of the computational complexity involved in obtaining different Shapley value variants. We found that for a variety of ML models, including decision trees, tree ensembles for regression, weighted automata, and linear regression models — computing both local and global interventional and baseline SHAP can be done in polynomial time when distributions are modeled by HMMs. This extends the distributional scope of popular algorithms like TreeSHAP, which is limited to empirical distributions. Additionally, we demonstrate a strict complexity gap between SHAP variants, showing that interventional and baseline SHAP can be strictly easier to compute than conditional SHAP. Despite these positive results, we uncovered intractability for various SHAP variants in neural networks and tree ensembles. Finally, we provided generalized complexity relations across SHAP variants. We believe that our framework offers a deeper understanding of the complexity involved in computing SHAP across various variants, models, distributions, as well as in both local and global computations, laying the groundwork for future research.

% % \section*{Impact Statement}
% This paper presents work whose goal is to advance the field of Machine Learning.
% It investigates fundamental aspects of instruction-tuning of language models and should not have direct societal impacts or implications that should be discussed here specifically, to the best of the authors' knowledge. 

\bibliographystyle{unsrt}
% \bibliographystyle{plainnat}
\bibliography{ref}


\clearpage
\appendix
\section{Missing Proofs}\label{app:proof}

In this section, we provide detailed proofs for Theorem \ref{thm:celora}. We first prove the following lemma.

\begin{lemma}\label{lm:m}
    Under Assumptions \ref{asp:proper}-\ref{asp:contractive}, if $\beta_1\in(0,1)$, it holds that
    \begin{align}
        \sum_{t=0}^T\mathbb{E}[\|\mathbf{m}^t-\nabla f(\mathbf{x}^t)\|_2^2]\le&\frac{2\|\mathbf{m}^0-\nabla f(\mathbf{x}^0)\|_2^2}{\beta_1}+\frac{4L^2}{\delta\beta_1^2}\sum_{t=1}^T\|\mathbf{x}^t-\mathbf{x}^{t-1}\|_2^2\nonumber\\
        &+\left(1-\frac{\delta}{2}\right)(1+6\beta_1)\sum_{t=1}^T\mathbb{E}[\|\nabla f(\mathbf{x}^t)\|_2^2]+6T\beta_1\sigma^2.\label{eq:lm-m}
    \end{align}
\end{lemma}
\begin{proof}
    According to the update of momentum, we have
    \begin{align}
        \mathbf{m}^{t}-\nabla f(\mathbf{x}^{t})=&(1-\beta_1)(\mathbf{m}^{t-1}-\nabla f(\mathbf{x}^{t}))+\beta_1(\hat{\mathbf{g}}^t-\nabla f(\mathbf{x}^t)).\nonumber
    \end{align}
    Taking expectation we have
    \begin{align}
        \mathbb{E}[\|\mathbf{m}^t-\nabla f(\mathbf{x}^t)\|_2^2]=&\mathbb{E}[\|(1-\beta_1)(\mathbf{m}^{t-1}-\nabla f(\mathbf{x}^t))+\beta_1(\mathbb{E}[\hat{\mathbf{g}}^t]-\nabla f(\mathbf{x}^t))\|_2^2]\nonumber\\
        &+\beta_1^2\mathbb{E}[\|\hat{\mathbf{g}}^t-\mathbb{E}[\hat{\mathbf{g}}^t]\|_2^2].\label{eq:pflm-m-1}
    \end{align}
    For the first term, applying Jensen's inequality yields
    \begin{align}
        &\mathbb{E}[\|(1-\beta_1)(\mathbf{m}^{t-1}-\nabla f(\mathbf{x}^t)+\beta_1(\mathbb{E}[\hat{\mathbf{g}}^t]-\nabla f(\mathbf{x}^t))\|_2^2]\nonumber\\
        \le&(1-\beta_1)\mathbb{E}[\|\mathbf{m}^{t-1}-\nabla f(\mathbf{x}^{t-1})-\nabla f(\mathbf{x}^t)+\nabla f(\mathbf{x}^{t-1})\|_2^2]+\beta_1\mathbb{E}[\|\mathbb{E}[\hat{\mathbf{g}}^t]-\nabla f(\mathbf{x}^t)\|_2^2].\label{eq:pflm-m-2}
    \end{align}
    By Young's inequality, we have
    \begin{align}
        \mathbb{E}[\|\mathbf{m}^{t-1}-\nabla f(\mathbf{x}^{t-1})-\nabla f(\mathbf{x}^t)+\nabla f(\mathbf{x}^{t-1})\|_2^2]\le&\left(1+\frac{\delta\beta_1}{2}\right)\mathbb{E}[\|\mathbf{m}^{t-1}-\nabla f(\mathbf{x}^{t-1})\|_2^2]\nonumber\\
        &+\left(1+\frac{2}{\delta\beta_1}\right)\mathbb{E}[\|\nabla f(\mathbf{x}^t)-\nabla f(\mathbf{x}^{t-1})\|_2^2].\label{eq:pflm-m-3}
    \end{align}
    For the second term, applying Cauchy's inequality yields
    \begin{align}
        \mathbb{E}[\|\hat{\mathbf{g}}^t-\mathbb{E}[\hat{\mathbf{g}}^t]\|_2^2]\le&3\mathbb{E}\|\hat{\mathbf{g}}^t-\mathbf{g}^t\|_2^2+3\mathbb{E}[\|\mathbf{g}^t-\nabla f(\mathbf{x}^t)\|_2^2]+3\mathbb{E}[\|\nabla f(\mathbf{x}^t)-\mathbb{E}[\hat{\mathbf{g}}^t]\|_2^2]\nonumber\\
        \le&6(1-\delta)\mathbb{E}[\|\nabla f(\mathbf{x}^t)\|_2^2]+3(2-\delta)\sigma^2,\label{eq:pflm-m-4}
    \end{align}
    where the last inequality uses Assumption \ref{asp:stochastic} and \ref{asp:contractive}.
    % and
    % \begin{align}
    %     &\mathbb{E}[C(g^k)-\nabla f(x^k)\|_2^2]\nonumber\\
    %     \le&\left(1+\frac{\delta}{2}\right)\mathbb{E}[\|C(g^k)-g^k\|_2^2]\nonumber\\
    %     &+\left(1+\frac{2}{\delta}\right)\mathbb{E}[\|g^k-\nabla f(x^k)\|_2^2]\nonumber\\
    %     \le&\left(1-\frac{\delta}{2}\right)\mathbb{E}[\|\nabla f(x^k)\|_2^2]+\frac{4\sigma^2}{\delta}.\label{eq:pflm-m-4}
    % \end{align}
    Applying \eqref{eq:pflm-m-2}\eqref{eq:pflm-m-3}\eqref{eq:pflm-m-4} to \eqref{eq:pflm-m-1} and using Assumption \ref{asp:smoothness} and \ref{asp:contractive}, we obtain
    \begin{align}
    \mathbb{E}[\|\mathbf{m}^t-\nabla f(\mathbf{x}^t)\|_2^2]\le&\left(1-\beta_1\left(1-\frac{\delta}{2}\right)\right)\mathbb{E}[\|\mathbf{m}^{t-1}-\nabla f(\mathbf{x}^{t-1})\|_2^2]+\frac{2L^2}{\delta\beta_1}\mathbb{E}[\|\mathbf{x}^t-\mathbf{x}^{t-1}\|_2^2]\nonumber\\
    &+(\beta_1+6\beta_1^2)(1-\delta)\mathbb{E}[\|\nabla f(\mathbf{x}^t)\|_2^2]+3(2-\delta)\beta_1^2\sigma^2.\label{eq:pflm-m-5}
    \end{align}
    Summing \eqref{eq:pflm-m-5} for $t=1,2,\cdots,T$ yields \eqref{eq:lm-m}.
\end{proof}

Now we are ready to prove Theorem \ref{thm:celora}. We first restate the theorem below in Theorem \ref{thm:celora-restate}.

\begin{theorem}\label{thm:celora-restate}
    Under Assumptions \ref{asp:proper}-\ref{asp:contractive}, if $\beta_1\in(0,\delta/(24-12\delta))$ and $\eta\le\min\{1/2L,\sqrt{(\delta\beta_1^2)/(8L^2)}\}$, CeLoRA with momentum SGD converges as
    \begin{align}
        \frac{1}{T+1}\sum_{t=0}^T\mathbb{E}[\|\nabla f(\mathbf{x}^t)\|_2^2]\le&\frac{4[f(\mathbf{x}^0)-\inf_{\mathbf{x}}f(\mathbf{x})]}{\delta\eta(T+1)}+\frac{4\|\mathbf{m}^0-\nabla f(\mathbf{x}^0)\|_2^2]}{\delta\beta_1(T+1)}+\frac{12\beta_1\sigma^2}{\delta}.\label{eq:thm-restate}
    \end{align}
\end{theorem}
\begin{proof}
    By Assumption \ref{asp:smoothness}, we have
    \begin{align}
        f(\mathbf{x}^{t+1})-f(\mathbf{x}^t)\le&\langle\nabla f(\mathbf{x}^t),\mathbf{x}^{t+1}-\mathbf{x}^t\rangle+\frac{L}{2}\|\mathbf{x}^{t+1}-\mathbf{x}^t\|_2^2\nonumber\\
        =&\left\langle\frac{\mathbf{m}^t}{2},\mathbf{x}^{t+1}-\mathbf{x}^t\right\rangle+\left\langle\nabla f(\mathbf{x}^t)-\frac{\mathbf{m}^t}{2},\mathbf{x}^{t+1}-\mathbf{x}^t\right\rangle+\frac{L}{2}\|\mathbf{x}^{t+1}-\mathbf{x}^t\|_2^2\nonumber\\
        =&-\left(\frac{1}{2\eta}-\frac{L}{2}\right)\|\mathbf{x}^{t+1}-\mathbf{x}^t\|_2^2+\frac{\eta}{2}\|\nabla f(\mathbf{x}^t)-\mathbf{m}^t\|_2^2-\frac{\eta}{2}\|\nabla f(\mathbf{x}^t)\|_2^2.\label{eq:pfthm-1}
    \end{align}
    Taking expectation and summing \eqref{eq:pfthm-1} for $t=0,1,\cdots,T$ yields
    \begin{align}
        \inf_{\mathbf{x}}f(\mathbf{x})-f(\mathbf{x}^0)\le&\frac{\eta}{2}\sum_{t=0}^{T}\mathbb{E}[\|\nabla f(\mathbf{x}^t)-\mathbf{m}^t\|_2^2]-\left(\frac{1}{2\eta}-\frac{L}{2}\right)\sum_{t=0}^{T}\mathbb{E}[\|\mathbf{x}^{t+1}-\mathbf{x}^t\|_2^2]\nonumber\\
        &-\frac{\eta}{2}\sum_{t=0}^T\mathbb{E}[\|\nabla f(\mathbf{x}^t)\|_2^2].\label{eq:pfthm-2}
    \end{align}
    Applying Lemma \ref{lm:m} to \eqref{eq:pfthm-2} and noting that $\beta_1\in(0,\delta/(24-12\delta))$ implies $(1-\delta/2)(1+6\beta_1)\le1-\delta/4$, we obtain
    \begin{align}
        \frac{1}{T+1}\sum_{t=0}^T\mathbb{E}[\|\nabla f(\mathbf{x}^t)\|_2^2]\le&\frac{4[f(\mathbf{x}^0)-\inf_{\mathbf{x}}f(\mathbf{x})]}{\delta\eta(T+1)}+\frac{4\|\mathbf{m}^0-\nabla f(\mathbf{x}^0)\|_2^2}{\delta\beta_1(T+1)}+\frac{12\beta_1\sigma^2}{\delta}\nonumber\\
        &-\frac{4}{\delta\eta}\left(\frac{1}{2\eta}-\frac{L}{2}-\frac{2\eta L^2}{\delta\beta_1^2}\right)\sum_{t=0}^T\|\mathbf{x}^{t+1}-\mathbf{x}^t\|_2^2.\label{eq:pfthm-3}
    \end{align}
   Since $\eta\le\min\{1/2L,\sqrt{(\delta\beta_1^2)/(8L^2)}\}$ implies $1/(4\eta)\ge L/2$ and $1/(4\eta)\ge(2\eta L^2)/(\delta\beta_1^2)$, \eqref{eq:thm-restate} is a direct result of \eqref{eq:pfthm-3}.
\end{proof}
% % \section{Additional Experiments}\label{app:delta}

% \textbf{Empirical justification of Assumption \ref{asp:contractive}. } In order to justify \eqref{eq:asp-ecgk}, we conduct experiments on language model fine-tuning tasks on \cgd{[XXX]} model using COLA, RTE and MRPC datasets, three tasks in the GLUE benchmark. In these experiments, we alternatively calculate one iteration of full gradient AdamW and one epoch of random gradient AdamW, each with a learning rate of \texttt{1e-5} for a total of 80 cycles. We apply a rank of 64 for both LoRA and double-LoRA in CeLoRA, and apply a compression rate of $p_{\mathrm{FFN}}=0.9$ and $p_{\mathrm{MHA}}=0.4$. We calculate the relative error $\|\mathbb{E}_{\xi^k\sim\mathcal{D}}[C(x^k;\xi^k)]-\nabla f(x^k)\|^2/\|\nabla f(x^k)\|^2$ for every full-gradient step, as illustrated in Fig.~\ref{fig:ecgk}, where all relative errors are below 1.


% \begin{figure}
%     \centering
%     \begin{minipage}{0.33\textwidth}
%         \includegraphics[width=\textwidth]{figures/histogram_cola_delta_fullgrad.png}
%     \end{minipage}
%     \begin{minipage}{0.33\textwidth}
%         \includegraphics[width=\textwidth]{figures/histogram_mrpc_delta_fullgrad.png}
%     \end{minipage}\begin{minipage}{0.33\textwidth}
%         \includegraphics[width=\textwidth]{figures/histogram_rte_delta_fullgrad.png}
%     \end{minipage}
%     \caption{Relative errors on COLA (left), MRPC (middle) and RTE (right).}
%     \label{fig:ecgk}
% \end{figure}
% The experiment selects three tasks in GLUE(COLA,RTE,MRPC) for testing. For full batch gradient descent, the experiment alternately uses one full gradient descent and one epoch of random gradient descent for training, calculates the relative error of the LoRA model and CeLoRA model for each full gradient descent, and performs a total of 80 cycles. For the stochastic gradient descent , the experiment train for 1 epoch per task and calculate the relative error every 10 iterations. The LoRA rank is 64, the CeLoRA Double LoRA method rank is 64, the MLP sampling ratio is 0.9, and the MHA sampling ratio is 0.4, We use google/gemma-2b as experiment model and the optimizer is AdamW, learning rate is 1e-5.

% \textbf{Ablation study for Double LoRA}

%\section{Appendix 1}



%%%%%%%%%%%%%%%%%%%%%%%%%%%%%%%%%%%%%%%%%%%%%%%%%%%%%%%%%%%%

%\begin{figure}
    \centering
    \includegraphics[width=\linewidth]{figures/MCQA_checklist.pdf}
    \vspace{-4.75ex}
    \setlength{\fboxsep}{0pt}
    \caption{\small Example unanswerable MCQ from MMLU \cite{gema2024we}, along with rubric criteria from \citet{haladyna1989taxonomy} flagged by OpenAI's o1 \cite{jaech2024openai}.}
    \label{fig:checklist}
    \vspace{-1.7ex}
\end{figure}

\end{document}