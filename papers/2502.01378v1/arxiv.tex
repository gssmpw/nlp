\documentclass{article}

\pdfoutput=1

\usepackage[utf8]{inputenc} % allow utf-8 input
\usepackage[T1]{fontenc}    % use 8-bit T1 fonts
\usepackage{hyperref}       % hyperlinks
\usepackage{url}            % simple URL typesetting
\usepackage{booktabs}       % professional-quality tables
\usepackage{amsfonts}       % blackboard math symbols
\usepackage{nicefrac}       % compact symbols for 1/2, etc.
\usepackage{microtype}      % microtypography
\usepackage{xcolor}         % colors


\usepackage{algpseudocode}
\usepackage{microtype}
\usepackage{graphicx}
\usepackage{subfigure}
\usepackage{multirow}
\usepackage{array}
\usepackage{colortbl}
\usepackage{tabularx}         % For adjustable-width columns
\usepackage{adjustbox}        % For \adjustbox
\usepackage{arydshln}
\usepackage{xspace}
\usepackage{algorithm}
\usepackage{enumitem}
\usepackage[numbers, sort&compress]{natbib}

\usepackage{arxiv}

% % Attempt to make hyperref and algorithmic work together better:
% \newcommand{\theHalgorithm}{\arabic{algorithm}}

% For theorems and such
\usepackage{amsmath}
\usepackage{amssymb}
\usepackage{mathtools}
\usepackage{amsthm}

%%%%%%%%%%%%%%%%%%%%%%%%%%%%%%%%
% THEOREMS
%%%%%%%%%%%%%%%%%%%%%%%%%%%%%%%%
\theoremstyle{plain}
\newtheorem{theorem}{Theorem}[section]
\newtheorem{proposition}[theorem]{Proposition}
\newtheorem{lemma}[theorem]{Lemma}
\newtheorem{corollary}[theorem]{Corollary}
\theoremstyle{definition}
\newtheorem{definition}[theorem]{Definition}
\newtheorem{assumption}[theorem]{Assumption}
\theoremstyle{remark}
\newtheorem{remark}[theorem]{Remark}

\definecolor{skyblue}{RGB}{229,242,247}
\definecolor{blue}{RGB}{66, 109, 181}
\newcommand{\cgd}[1]{{\color{blue}#1}}

\renewcommand{\algorithmicrequire}{\textbf{Input: }}
\newcommand{\etc}{\emph{etc.}\xspace}
\newcommand{\ie}{\emph{i.e.}\xspace}
\newcommand{\eg}{\emph{e.g.}\xspace}
\newcommand{\celora}{{CE-LoRA}\xspace}
\newcommand{\textub}[1]{\underline{\textbf{#1}}}

% \newcommand\blfootnote[1]{%
%   \begingroup
%   \renewcommand\thefootnote{}\footnote{#1}%
%   \addtocounter{footnote}{-1}%
%   \endgroup
% }

\newcommand{\blfootnote}[1]{%
  \begingroup
  \renewcommand{\thefootnote}{}% 清除编号
  \footnotetext{#1}% 直接添加脚注内容(不生成上标标记)
  \addtocounter{footnote}{-1}%
  \endgroup
}


\usepackage{abstract}
\renewcommand{\abstractnamefont}{\normalfont\Large\bfseries}
\renewcommand{\abstracttextfont}{\normalfont\normalsize}

\newcommand\algname{\texttt{AC-SGD}\xspace}







\title{\celora: Computation-Efficient LoRA \\ Fine-Tuning for Language Models}

\author{%
  Guanduo~Chen$^{* \dag}$\\
  Fudan University\\
  \texttt{gdchen22@m.fudan.edu.cn} \\
  \And
  Yutong~He$^{\dag}$\\
  Peking University\\
  \texttt{yutonghe@pku.edu.cn} \\
  \And
  Yipeng~Hu\\
  Peking University\\
  \texttt{yipenghu@pku.edu.cn} \\
  \And
  Kun~Yuan$^\ddagger$\\
  Peking University\\
  \texttt{kunyuan@pku.edu.cn} \\
  \And
  Binhang Yuan$^\ddagger$\\
  HKUST \\
  \texttt{biyuan@ust.hk}\\
}


\begin{document}
\maketitle

\blfootnote{$^*$ Work done when the author was working as a research assistant under the supervision of Binhang Yuan.}
\blfootnote{$^\dag$ Both authors contributed equally to this research.}
\blfootnote{ $^\dagger$ Coressponding author.}



\begin{abstract}
Large Language Models (LLMs) demonstrate exceptional performance across various tasks but demand substantial computational resources even for fine-tuning computation. Although Low-Rank Adaptation (LoRA) significantly alleviates memory consumption during fine-tuning, its impact on computational cost reduction is limited. This paper identifies the computation of activation gradients as the primary bottleneck in LoRA's backward propagation and introduces the \underline{\textbf{C}}omputation-\underline{\textbf{E}}fficient \underline{\textbf{LoRA}} (\textbf{CE-LoRA}) algorithm, which enhances computational efficiency while preserving memory efficiency. \celora leverages two key techniques: Approximated Matrix Multiplication, which replaces dense multiplications of large and complete matrices with sparse multiplications involving only critical rows and columns, and the Double-LoRA technique, which  reduces error propagation in activation gradients. Theoretically, \celora converges at the same rate as LoRA, \( \mathcal{O}(1/\sqrt{T}) \), where $T$ is the number of iterations. Empirical evaluations confirm that \celora significantly reduces computational costs compared to LoRA without notable performance degradation.
\end{abstract}

\section{Introduction}


\begin{figure}[t]
\centering
\includegraphics[width=0.6\columnwidth]{figures/evaluation_desiderata_V5.pdf}
\vspace{-0.5cm}
\caption{\systemName is a platform for conducting realistic evaluations of code LLMs, collecting human preferences of coding models with real users, real tasks, and in realistic environments, aimed at addressing the limitations of existing evaluations.
}
\label{fig:motivation}
\end{figure}

\begin{figure*}[t]
\centering
\includegraphics[width=\textwidth]{figures/system_design_v2.png}
\caption{We introduce \systemName, a VSCode extension to collect human preferences of code directly in a developer's IDE. \systemName enables developers to use code completions from various models. The system comprises a) the interface in the user's IDE which presents paired completions to users (left), b) a sampling strategy that picks model pairs to reduce latency (right, top), and c) a prompting scheme that allows diverse LLMs to perform code completions with high fidelity.
Users can select between the top completion (green box) using \texttt{tab} or the bottom completion (blue box) using \texttt{shift+tab}.}
\label{fig:overview}
\end{figure*}

As model capabilities improve, large language models (LLMs) are increasingly integrated into user environments and workflows.
For example, software developers code with AI in integrated developer environments (IDEs)~\citep{peng2023impact}, doctors rely on notes generated through ambient listening~\citep{oberst2024science}, and lawyers consider case evidence identified by electronic discovery systems~\citep{yang2024beyond}.
Increasing deployment of models in productivity tools demands evaluation that more closely reflects real-world circumstances~\citep{hutchinson2022evaluation, saxon2024benchmarks, kapoor2024ai}.
While newer benchmarks and live platforms incorporate human feedback to capture real-world usage, they almost exclusively focus on evaluating LLMs in chat conversations~\citep{zheng2023judging,dubois2023alpacafarm,chiang2024chatbot, kirk2024the}.
Model evaluation must move beyond chat-based interactions and into specialized user environments.



 

In this work, we focus on evaluating LLM-based coding assistants. 
Despite the popularity of these tools---millions of developers use Github Copilot~\citep{Copilot}---existing
evaluations of the coding capabilities of new models exhibit multiple limitations (Figure~\ref{fig:motivation}, bottom).
Traditional ML benchmarks evaluate LLM capabilities by measuring how well a model can complete static, interview-style coding tasks~\citep{chen2021evaluating,austin2021program,jain2024livecodebench, white2024livebench} and lack \emph{real users}. 
User studies recruit real users to evaluate the effectiveness of LLMs as coding assistants, but are often limited to simple programming tasks as opposed to \emph{real tasks}~\citep{vaithilingam2022expectation,ross2023programmer, mozannar2024realhumaneval}.
Recent efforts to collect human feedback such as Chatbot Arena~\citep{chiang2024chatbot} are still removed from a \emph{realistic environment}, resulting in users and data that deviate from typical software development processes.
We introduce \systemName to address these limitations (Figure~\ref{fig:motivation}, top), and we describe our three main contributions below.


\textbf{We deploy \systemName in-the-wild to collect human preferences on code.} 
\systemName is a Visual Studio Code extension, collecting preferences directly in a developer's IDE within their actual workflow (Figure~\ref{fig:overview}).
\systemName provides developers with code completions, akin to the type of support provided by Github Copilot~\citep{Copilot}. 
Over the past 3 months, \systemName has served over~\completions suggestions from 10 state-of-the-art LLMs, 
gathering \sampleCount~votes from \userCount~users.
To collect user preferences,
\systemName presents a novel interface that shows users paired code completions from two different LLMs, which are determined based on a sampling strategy that aims to 
mitigate latency while preserving coverage across model comparisons.
Additionally, we devise a prompting scheme that allows a diverse set of models to perform code completions with high fidelity.
See Section~\ref{sec:system} and Section~\ref{sec:deployment} for details about system design and deployment respectively.



\textbf{We construct a leaderboard of user preferences and find notable differences from existing static benchmarks and human preference leaderboards.}
In general, we observe that smaller models seem to overperform in static benchmarks compared to our leaderboard, while performance among larger models is mixed (Section~\ref{sec:leaderboard_calculation}).
We attribute these differences to the fact that \systemName is exposed to users and tasks that differ drastically from code evaluations in the past. 
Our data spans 103 programming languages and 24 natural languages as well as a variety of real-world applications and code structures, while static benchmarks tend to focus on a specific programming and natural language and task (e.g. coding competition problems).
Additionally, while all of \systemName interactions contain code contexts and the majority involve infilling tasks, a much smaller fraction of Chatbot Arena's coding tasks contain code context, with infilling tasks appearing even more rarely. 
We analyze our data in depth in Section~\ref{subsec:comparison}.



\textbf{We derive new insights into user preferences of code by analyzing \systemName's diverse and distinct data distribution.}
We compare user preferences across different stratifications of input data (e.g., common versus rare languages) and observe which affect observed preferences most (Section~\ref{sec:analysis}).
For example, while user preferences stay relatively consistent across various programming languages, they differ drastically between different task categories (e.g. frontend/backend versus algorithm design).
We also observe variations in user preference due to different features related to code structure 
(e.g., context length and completion patterns).
We open-source \systemName and release a curated subset of code contexts.
Altogether, our results highlight the necessity of model evaluation in realistic and domain-specific settings.






% Consider a lasso optimization procedure with potentially distinct regularization penalties:
% \begin{align}
%     \hat{\beta} = \arg\min_{\beta}\{\|y-X\beta\|^2_2+\sum_{i=1}^{N}\lambda_i|\beta_i|\}.
% \end{align}
\subsection{Supervised Data-Driven Learning}\label{subsec:supervised}
We consider a generic data-driven supervised learning procedure. Given a dataset \( \mathcal{D} \) consisting of \( n \) data points \( (x_i, y_i) \in \mathcal{X} \times \mathcal{Y} \) drawn from an underlying distribution \( p(\cdot|\theta) \), our goal is to estimate parameters \( \theta \in \Theta \) through a learning procedure, defined as \( f: (\mathcal{X} \times \mathcal{Y})^n \rightarrow \Theta \) 
that minimizes the predictive error on observed data. 
Specifically, the learning objective is defined as follows:
\begin{align}
\hat{\theta}_f := f(\mathcal{D}) = \arg\min_{\theta} \mathcal{L}(\theta, \mathcal{D}),
\end{align}
where \( \mathcal{L}(\cdot,\mathcal{D}) := \sum_{i=1}^{n} \mathcal{L}(\cdot, (x_i, y_i))\), and $\mathcal{L}$ is a loss function quantifying the error between predictions and true outcomes. 
Here, $\hat{\theta}_f$ is the parameter that best explains the observed data pairs \( (x_i, y_i) \) according to the chosen loss function \( \mathcal{L} (\cdot) \).

\paragraph{Feature Selection.}
Feature selection aims to improve model \( f \)'s predictive performance while minimizing redundancy. 
%Formally, given data \( X \), response \( y \), feature set \( \mathcal{F} \), loss function \( \mathcal{L}(\cdot) \), and a feature limit \( k \), the objective is:
% \begin{align}
% \mathcal{S}^* = \arg \min_{\mathcal{S} \subseteq \mathcal{F}, |\mathcal{S}| \leq k} \mathcal{L}(y, f(X_\mathcal{S})) + \lambda R(\mathcal{S}),
% \end{align}
% where \( X_\mathcal{S} \) is the submatrix of \( X \) for selected features \( \mathcal{S} \), \( \lambda \) is a regularization parameter, and \( R(\mathcal{S}) \) penalizes feature redundancy.
 State-of-the-art techniques fall into four categories: (i) filter methods, which rank features based on statistical properties like Fisher score \citep{duda2001pattern,song2012feature}; (ii) wrapper methods, which evaluate model performance on different feature subsets \citep{kohavi1997wrappers}; (iii) embedded methods, which integrate feature selection into the learning process using techniques like regularization \citep{tibshirani1996LASSO,lemhadri2021lassonet}; and (iv) hybrid methods, which combine elements of (i)-(iii) \citep{SINGH2021104396,li2022micq}. This paper focuses on embedded methods via Lasso, benchmarking against approaches from (i)-(iii).

\subsection{Language Modeling}
% The objective of language modeling is to learn a probability distribution \( p_{LM}(x) \) over sequences of text \( x = (X_1, \ldots, X_{|x|}) \), such that \( p_{LM}(x) \approx p_{text}(x) \), where \( p_{text}(x) \) represents the true distribution of natural language. This process involves estimating the likelihood of token sequences across variable lengths and diverse linguistic structures.
% Modern large language models (LLMs) are trained on vast datasets spanning encyclopedias, news, social media, books, and scientific papers \cite{gao2020pile}. This broad training enables them to generalize across domains, learn contextual knowledge, and perform zero-shot learning—tackling new tasks using only task descriptions without fine-tuning \cite{brown2020gpt3}.
Language modeling aims to approximate the true distribution of natural language \( p_{\text{text}}(x) \) by learning \( p_{\text{LM}}(x) \), a probability distribution over text sequences \( x = (X_1, \ldots, X_{|x|}) \). Modern large language models, trained on diverse datasets \citep{gao2020pile}, exhibit strong generalization across domains, acquire contextual knowledge, and perform zero-shot learning—solving new tasks using only task descriptions—or few-shot learning by leveraging a small number of demonstrations \citep{brown2020gpt3}.
\paragraph{Retrieval-Augmented Generation (RAG).} Retrieval-Augmented Generation (RAG) enhances the performance of generative language models by  integrating a domain-specific information retrieval process  \citep{lewis2020retrieval}. The RAG framework comprises two main components: \textit{retrieval}, which extracts relevant information from external knowledge sources, and \textit{generation}, where an LLM generates context-aware responses using the prompt combined with the retrieved context. Documents are indexed through various databases, such as relational, graph, or vector databases \citep{khattab2020colbert, douze2024faiss, peng2024graphretrievalaugmentedgenerationsurvey}, enabling efficient organization and retrieval via algorithms like semantic similarity search to match the prompt with relevant documents in the knowledge base. RAG has gained much traction recently due to its demonstrated ability to reduce incidence of hallucinations and boost LLMs' reliability as well as performance \citep{huang2023hallucination, zhang2023merging}. 
 
% image source: https://medium.com/@bindurani_22/retrieval-augmented-generation-815c1ae438d8
\begin{figure}
    \centering
\includegraphics[width=1.03\linewidth]{fig/fig1.pdf}
\vspace{-0.6cm}
\scriptsize 
    \caption{Retrieval Augmented Generation (RAG) based $\ell_1$-norm weights (penalty factors) for Lasso. Only feature names---no training data--- are included in LLM prompt.} 
    \label{fig:rag}
\end{figure}
% However, for the RAG model to be effective given the input token constraints of the LLM model used, we need to effectively process the retrieval documents through a procedure known as \textit{chunking}.

\subsection{Task-Specific Data-Driven Learning}
LLM-Lasso aims to bridge the gap between data-driven supervised learning and the predictive capabilities of LLMs trained on rich metadata. This fusion not only enhances traditional data-driven methods by incorporating key task-relevant contextual information often overlooked by such models, but can also be especially valuable in low-data regimes, where the learning algorithm $f:\mathcal{D}\rightarrow\Theta$ (seen as a map from datasets $\mathcal{D}$ to the space of decisions $\Theta$) is susceptible to overfitting.

The task-specific data-driven learning model $\tilde{f}:\mathcal{D}\times\mathcal{D}_\text{meta}\rightarrow\Theta$ can be described as a metadata-augmented version of $f$, where a link function $h(\cdot)$ integrates metadata (i.e. $\mathcal{D}_\text{meta}$) to refine the original learning process. This can be expressed as:
\[
\tilde{f}(\mathcal{D}, \mathcal{D}_\text{meta}) := \mathcal{T}(f(\mathcal{D}),  h(\mathcal{D}_{\text{meta}})),
\]
where the functional $\mathcal{T}$ takes the original learning algorithm $f(\mathcal{D})$ and transforms it into a task-specific learning algorithm $\tilde{f}(\mathcal{D}, \mathcal{D}_\text{meta})$ by incorporating the metadata $\mathcal{D}_\text{meta}$. 
% In particular, the link function $h(\mathcal{D}_{\text{meta}})$ provides a structured mechanism summarizing the contextual knowledge.

There are multiple approaches to formulate $\mathcal{T}$ and $h$.
%to ``inform" the data-driven model $f$ of %meta knowledge. 
For instance, LMPriors \citep{choi2022lmpriorspretrainedlanguagemodels} designed $h$ and $\mathcal{T}$ such that $h(\mathcal{D}_{\text{meta}})$ first specifies which features to retain (based on a probabilistic prior framework), and then $\mathcal{T}$ keeps the selected features and removes all the others from the original learning objective of $f$. 
Note that this approach inherently is restricted as it selects important features solely based on $\mathcal{D}_\text{meta}$ without seeing $\mathcal{D}$.

In contrast, we directly embed task-specific knowledge into the optimization landscape through regularization by introducing a structured inductive bias. This bias guides the learning process toward solutions that are consistent with metadata-informed insights, without relying on explicit probabilistic modeling. Abstractly, this can be expressed as:
\begin{align}
    \!\!\!\!\!\hat{\theta}_{\tilde{f}} := \tilde{f}(\mathcal{D},\mathcal{D}
    _\text{meta})= \arg\min_{\theta} \mathcal{L}(\theta, \mathcal{D}) + \lambda R(\theta, \mathcal{D}_{\text{meta}}),
\end{align}
where \( \lambda \) is a regularization parameter, \( R(\cdot) \) is a regularizer, and $\theta$ is the prediction parameter.
%We explain our framework with more details in the following section.


% Our research diverges from both aforementioned approaches by positioning the LLM not as a standalone feature selector but as an enhancement to data-driven models through an embedded feature selection method, L-LASSO. L-LASSO incorporates domain expertise—auxiliary natural language metadata about the task—via the LLM-informed LASSO penalty, which is then used in statistical models to enhance predictive performance. This method integrates the rich, context-sensitive insights of LLMs with the rigor and transparency of statistical modeling, bridging the gap between data-driven and knowledge-driven feature selection approaches. To approach this task, we need to tackle two key components: (i). train an LLM that is expert in the task-specific knowledge; (ii). inform data-driven feature selector LASSO with LLM knowledge.

% In practice, this involves combining techniques like prompt engineering and data engineering to develop an effective framework for integrating metadata into existing data-driven models. We will go through this in detail in Section \ref{mthd} and \ref{experiment}.



\section{\celora: Computation-Efficient LoRA}

% \yh{sampling strategy, layer-adaptive strategy, SVD strategy, algorithm}

\subsection{Approximated Matrix Multiplication (AMM)}
Consider matrix multiplication $\mathbf{T}=\mathbf{P}\mathbf{Q}$, where $\mathbf{T}\in\mathbb{R}^{m\times k}$, $\mathbf{P}\in\mathbb{R}^{m\times n}$ and $\mathbf{Q}\in\mathbb{R}^{n\times k}$.
% 
Let $\mathbf{p}_1,\mathbf{p}_2,\cdots,\mathbf{p}_n$ denote the column vectors of matrix $\mathbf{P}$, and $\mathbf{q}_1,\mathbf{q}_2,\cdots,\mathbf{q}_n$ denote the column vectors of matrix $\mathbf{Q}^\top$. 
% 
We can rewrite the matrix multiplication into:
% 
\begin{align*}
    \mathbf{T}=\sum_{i=1}^n\mathbf{p}_i\mathbf{q}_i^\top.
\end{align*}
% 
To estimate the product $\mathbf{T}$ computation-efficiently, we may assume the matrices $\mathbf{P}$ and $\mathbf{Q}$ enjoy some kinds of structured sparsity, such that a few $(\mathbf{p}_i\mathbf{q}_i^\top)'s$ contribute to most of the result $\sum_{i=1}^n\mathbf{p}_i\mathbf{q}_i^\top$, in which case we could estimate $\mathbf{T}$ by computing the most important parts only. Specifically, we identify $s$ most important indices $1\le i_1<\cdots<i_s\le n$, and the AMM estimate of $\mathbf{T}$ is given by:
% 
\begin{align*}
    \hat{\mathbf{T}}=\sum_{j=1}^s{\mathbf{p}_{i_j}\mathbf{q}_{i_j}^\top}=\hat{\mathbf{P}}\hat{\mathbf{Q}},
\end{align*}
% 
where $\hat{\mathbf{P}}$ and $\hat{\mathbf{Q}}^\top$ collect column vectors $\{\mathbf{p}_{i_j}\}_{j=1}^s$ and $\{\mathbf{q}_{i_j}\}_{j=1}^s$, respectively.


The efficiency of AMM is concerned with the number of selected indices $s$, or the structured sparsity  $p:=s/n\in(0,1]$. Replacing the dense matrix multiplication $\mathbf{T}=\mathbf{P}\mathbf{Q}$ by AMM estimate  $\hat{\mathbf{T}}=\hat{\mathbf{P}}\hat{\mathbf{Q}}$, the computational complexity is reduced from $2mnk$ to $2msk=p\cdot (2mnk)$.
% 
Hereafter, we use $\mathcal{C}_p(\mathbf{P}\cdot\mathbf{Q})$ to denote the AMM estimate of matrix multiplication $\mathbf{P}\cdot\mathbf{Q}$ with structured sparsity $p$.

An important question is how to select the indices $\mathcal{I}=\{i_1,i_2,\cdots,i_s\}$ properly. A previous research \citep{drineas2006fast} has studied a random sampling strategy, which does not work well in our experiments. Based on the above intuition, we define the importance score $\alpha_i$ of index $i$ by the Frobenius norm $\|\mathbf{p}_i\mathbf{q}_i^\top\|_F$ and attempt to select the indices with highest scores. However, as calculating $\{\alpha_i\}_{i=1}^k$ requires the same amount of computation as that of conducting the original matrix multiplication, we cannot determine $\mathcal{I}$ based on the calculation results of $\{\alpha_i\}_{i=1}^k$ in every iteration. We use historical information to mitigate this issue. Specifically, the matrices $\mathbf{P},\mathbf{Q}$ we multiply by AMM should be variables that live along the whole optimization process, and $\mathbf{P}^t,\mathbf{Q}^t$ are multiplied at every iteration $t$. The corresponding $\mathcal{I}^t$ is only re-selected according to the top-$s$ importance scores every $\tau$ iterations and is reused in intermediate ones.


To reduce the computational bottleneck in LoRA's backward propagation, we apply AMM to step \eqref{eq:A'3} and get:
\begin{align}
\mathbf{G}_{\mathbf{x},2}=&\ \mathcal{C}_p(\mathbf{W}_0^\top\cdot \mathbf{G}_\mathbf{y}).\label{eq:A''3}
\end{align}
% With AMM, the backward propagation step in (A'2) becomes: 
% \begin{align*}
%     \frac{\partial l}{\partial \mathbf{x}}=&\mathcal{C}_p(\mathbf{W}_0^\top\cdot\frac{\partial l}{\partial\mathbf{y}})+\mathbf{A}^\top\cdot\frac{\partial l}{\partial \mathbf{z}}.&\text{(A''2)}
% \end{align*}

% To implement AMM, we provide two index sampling strategies:

% \textbf{Random Sampling Strategy.}
% % 
% This method randomly selects $s$ indices from the product’s inner dimensions, ensuring that the estimate is theoretically unbiased.

% \textbf{Greedy Sampling Strategy.}
% % 
% To identify the most important indices, we periodically computes a precise stochastic gradient at intervals $\tau$.
% % 
% Given the accurate backpropagation result, we evaluate $\|p_iq_i^\top\|_F=\|p_i\|_2\|q_i\|_2$ and select the indices with top magnitudes.
% % 
% These selected indices are then used fo the next $\tau$ iterations.

% By default, \celora adopts the greedy sampling strategy. We will compare these two sampling methods in Sec. \ref{sec:exp-sample-strategy}.

% \cgd{Sort $\Vert p_iq_i^\top\Vert_F$ from largest to smallest. Greedily select from the beginning.}

% \textbf{[TODO] Reusing Sampling Patterns. }\yh{reorganize to save i/o, reduce computation, ..}



% \subsection{Layer-wise Adaptive Compression Ratio}

% \yh{cite some papers regarding difference across transformer layers}

% \yh{Add some ablation study figures here, demonstrating different properties of Q,K,V,O,U,D.}

% Consequently, it's natural to use a more aggressive sparsity for layers that are robust to computational errors, and a relatively conservative sparsity for sensitive ones. We empirically use $p=0.3$ for MHA layers, and $p=0.9$ for FFN layers, throughout our experiments.

\subsection{Double-LoRA Mechanism }
Although computation-efficient, AMM will induce errors to $g_x$, the gradient with respect to the activations. 
% 
These errors propagate backward through the network, potentially compounding as they traverse previous layers.
% 
If the magnitude of these errors is not properly controlled, the accuracy of the parameter gradients can be significantly degraded.
% 
To mitigate this issue, we propose a double-LoRA mechanism to alleviate the error induced by the AMM operation in each layer. Intuitively, we wish the objective matrix multiplication result we estimate by AMM has as little contribution to the activation gradient as possible. This drives us to further separate the frozen matrix $\mathbf{W}_0$ into two parts: a low-rank part inheriting computational efficiency without AMM, and a residual part with a relatively small magnitude.

\begin{figure}[!t]
    \centering
    \includegraphics[width=\linewidth]{figures/layerwise.pdf}
    % \centering
    % \includegraphics[width=0.75\linewidth]{figures/comm_layerwise.pdf}
    \vspace{-1em}
    \caption{\small
    Layer-wise Sensitivity Analysis of LLaMA3.2-1B. 
    }
    \label{fig:layerwise}
    \vspace{-1em}    
\end{figure}

Specifically, we initially compute the SVD of $\mathbf{W}_0$, yielding
\begin{align*}
    \mathbf{W}_0=\mathbf{U}\mathbf{\Sigma} \mathbf{V}^\top
\end{align*}
Next, We collect the principal low-rank component $\mathbf{B}_0=\mathbf{U}[:r]\mathbf{\Sigma}^{1/2}$, $\mathbf{A}_0=\mathbf{\Sigma}^{1/2}(\mathbf{V}[:r])^\top$, and the residual $\mathbf{W}_{s}=\mathbf{W}_0-\mathbf{B}_0\mathbf{A}_0$. By separating $\mathbf{W}_0$ to $\mathbf{W}_s+\mathbf{B}_0\mathbf{A}_0$, we split the matrix to two parts. The first part $\mathbf{W}_s$ is believed to have better structured sparsity and is more compatible to AMM. The second low-rank part $\mathbf{B}_0\mathbf{A}_0$ is computation-efficient just like the trainable LoRA adapter $\mathbf{B}\mathbf{A}$. Combining AMM with double-LoRA, \eqref{eq:A''3} is further replaced by
\begin{align}
\mathbf{G}_{\mathbf{x},2}=&\ \mathcal{C}_p(\mathbf{W}_s^\top\cdot \mathbf{G}_\mathbf{y})+\mathbf{A}_0^\top(\mathbf{B}_0^\top \mathbf{G}_\mathbf{y}).\label{eq:A'''3}
% +\mathbf{A}_0^\top\cdot(\mathbf{B}_0^\top g_y)
\end{align}
% \begin{align*}
%     \frac{\partial l}{\partial\mathbf{x}}=\mathcal{C}_p(\mathbf{W}_{0-}\cdot\frac{\partial l}{\partial\mathbf{y}})+\mathbf{A}_0^\top\mathbf{B}_0^\top\cdot\frac{\partial l}{\partial\mathbf{y}}+\mathbf{A}^\top\cdot\frac{\partial l}{\partial\mathbf{z}}.&\text{(A'''2)}
% \end{align*}


\subsection{Layer-wise Adaptive Sparsity}

It is natural to apply more aggressive sparsity to layers that are relatively robust to computational errors, while using more conservative sparsity for those that are more sensitive.
% 
Inspired by \cite{hu2025accelerating,ma2024first,jaiswal2024galore,zeng2024lsaq,malinovskii2024pushing,zhang2024q,liu2024training}, we adopt a layer-wise adaptive sparsity strategy for \celora.


To determine which layers are more sensitive to varying sparsity levels, we conduct experiments on two small fine-tuning datasets: the Commonsense 14K dataset and the Math 7K dataset~\cite{hu2023llm}.
% 
In these experiments, we fix LoRA's rank to 32, and set both \celora's trainable LoRA rank and its frozen Double-LoRA rank to 28. 
% 
For each \celora configuration, we vary the sparsity level of one layer type, while setting the sparsity of all remaining layer types to $p=0.3$.
As shown in Figure~\ref{fig:layerwise}, the \texttt{Gate} layers are essential for preventing error propagation. 
% 
In addition, the \texttt{Q} and \texttt{K} layers have a strong impact on arithmetic and commonsense reasoning tasks, respectively. 
% 
Based on these findings, we disable sparsity for the \texttt{Q}, \texttt{K}, and \texttt{Gate} layers. 
For the remaining MHA layers, we use $p=0.55$, and for the last two layers in the FFN, we set $p=0.65$ throughout our experiments.




\subsection{Algorithm}
\begin{algorithm}[t]
\caption{\celora}\label{alg:CeLoRA}
\footnotesize
\begin{algorithmic}[1]

\Statex{\textbf{Input:} Frozen layer weight $\mathbf{W}_{\ell}\in\mathbb{R}^{m_{\ell}\times n_{\ell}}$, sparsity level $p_\ell$, double-LoRA rank $r_{0,\ell}$, indices recomputing period $\tau$, Top-K indices $\mathcal{I}_\ell=$ empty, optimizer $\rho$.
% learning rate $\eta$, AdamW hyperparameters $\beta_1,\beta_2$, momentum $\mathbf{m}_\ell=\mathbf{0}$, $\mathbf{v}_\ell=\mathbf{0}$, $\ell=1,2,\cdots,L$.
}

% \vspace{1mm}
\Statex\rule{\linewidth}{0.4pt}
\Statex \textbf{Initialize Double-LoRA}
    \State \quad \textbf{for} Layer $\ell=1,2,\cdots,L$ \textbf{do}
    \State \qquad Conducting SVD on frozen weight matrix
    \Statex \qquad $\mathbf{W}_{0,\ell}=\mathbf{U}_\ell\mathbf{\Sigma}_\ell \mathbf{V}_\ell^\top$;
    \State \qquad $\mathbf{A}_{0,\ell},\ \mathbf{B}_{0,\ell} \gets \sqrt{\mathbf{\Sigma}_\ell}{\mathbf{V}_\ell^\top}_{[:r_0,]},\ {\mathbf{U}_{\ell}}_{[,:r_0]}\sqrt{\mathbf{\Sigma}_\ell}$ ;\Comment{Stored in layer's buffer.}
    \State \quad \textbf{end for}
\Statex \rule{\linewidth}{0.4pt}

\State \textbf{for} {\textbf{\celora Training Step $t=0,1,\cdots,T-1$}} \textbf{do}
% \For{$t=0,1,\cdots,T-1$}
\vspace{0.5mm}
    \State \quad \textbf{for} Layer $\ell=1,2,\cdots,L$ \textbf{do}\Comment{\textbf{Forward}}
        \State \quad\quad $\mathbf{z}_\ell\gets\mathbf{A}_\ell\mathbf{x}_\ell$;
        \State \quad\quad $\mathbf{y}_\ell \gets \mathbf{W}_{0,\ell}\mathbf{x}_\ell+\mathbf{B}_\ell\mathbf{z}_\ell$;
        % \State \quad\quad  \textbf{Return} $\mathbf{y}_\ell$
    \State\quad\textbf{end for}
\vspace{0.5mm}
    \State \quad \textbf{for} Layer $\ell=L,L-1,\cdots,1$ \textbf{do}\Comment{\textbf{Backward}}
        \State \quad\quad  $\mathbf{W}_{s,\ell} \gets \mathbf{W}_{0,\ell} - \mathbf{B}_{0,\ell}\mathbf{A}_{0,\ell}$;
        \State \quad\quad \textbf{if} {$\tau\mid t$ \textbf{or} $\mathcal{I}_\ell$ is empty} \textbf{then}
            
            \State \quad\qquad $\alpha_{i,\ell} \gets\left\|{\mathbf{W}_{s,\ell}^\top}_{[:,i]} {\mathbf{G_{y_\ell}}}_{[i,:]}\right\|_F$, \ $\forall i \in \left\{1,\dots,m_\ell\right\}$
            \State \quad\qquad {Select $\left\{i_{1,\ell},\cdots,i_{\text{K}_\ell,\ell}\right\}$ with largest $\alpha_{i,\ell}$'s;}
            
            \State \quad\qquad $\mathcal{I}_\ell = \left\{i_{1,\ell},\cdots,i_{\text{K}_\ell,\ell}\right\}$; \Comment{Here K$_\ell= \lceil m_\ell p_\ell \rceil$}
        \State \quad\quad\textbf{end if}
        \State \quad\quad $\mathbf{G}_{\mathbf{B}_\ell}\gets\mathbf{G}_{\mathbf{y}_\ell}\mathbf{z}_\ell^\top$;
        \State \quad\quad $\mathbf{G}_{\mathbf{z}_\ell}\gets\mathbf{B}_{\ell}^\top\mathbf{G}_{\mathbf{y}_{\ell}}$;
        \State \quad\quad
        $\mathbf{G}_{\mathbf{A}_\ell}\gets\mathbf{G}_{\mathbf{z}_\ell}\mathbf{x}_\ell^\top$;
        \State \quad\quad $\mathbf{G}_{\mathbf{x}_\ell} \gets {\mathbf{W}_{s,\ell}^\top}_{[,\mathcal{I}]} {\mathbf{G}_{\mathbf{y}_\ell}}_{[\mathcal{I},]} + \mathbf{A}_{0,\ell}^\top(\mathbf{B}_{0,\ell}^\top \mathbf{G}_{\mathbf{y}_\ell})+\mathbf{A}_\ell^\top\mathbf{G}_{\mathbf{z}_\ell}$;
        % \State \quad  \textbf{Return} $\mathbf{G_x}$
\State\quad\textbf{end for}
\State Use optimizer $\rho$ to update $\{\mathbf{A}_\ell,\mathbf{B}_\ell\}_{\ell=1}^L$ according to $\{\mathbf{G}_{\mathbf{A}_\ell},\mathbf{G}_{\mathbf{B}_\ell}\}_{\ell=1}^L$;
\State\textbf{end for}
\end{algorithmic}
\end{algorithm}




% \begin{algorithm}[t]
% \caption{\celora}\label{alg:CeLoRA}
% \footnotesize
% \begin{algorithmic}[1]

% \Statex{\textbf{Input:} Frozen layer weight $W\in\mathbb{R}^{m\times n}$, sparsity level $p$, double-LoRA rank $r_0$, indices recomputing period $\tau$, Top-K indices $\mathcal{I}=$ empty.
% }

% \vspace{1mm}
% \Statex \textbf{Initialize Double-LoRA rank}
%     \State \quad Conducting SVD on frozen weight matrix $W_0=U\Sigma V^\top$
%     \State \quad $A_0,\ B_0 \gets \sqrt{\Sigma}V^\top_{[:r_0,]},\ U_{[,:r_0]}\sqrt{\Sigma}$ \Comment{Stored in layer's buffer.}

% \Statex \rule{\linewidth}{0.4pt}

% \Statex \textbf{\celora Training Step $t$}
% % \For{$t=0,1,\cdots,T-1$}
% \vspace{0.5mm}
%     \Statex{\textbf{Forward}$\left(\mathbf{x}\right)$:}
%         \State \quad $\mathbf{y} = \mathbf{W}_0\mathbf{x}$
%         \State \quad  \textbf{Return} $\mathbf{y}$

% \vspace{0.5mm}

%     \Statex{\textbf{Backward} $\left(\mathbf{G}_\mathbf{y}\right)$:}
%         \State \quad  $\mathbf{W}_s = \mathbf{W}_0 - \mathbf{B}_0\mathbf{A}_0$
%         \State \quad \textbf{if} {$\tau\mid t$ \textbf{or} $\mathcal{I}$ is empty} \textbf{then}
            
%             \State \quad\quad $\alpha_i = {\mathbf{W}_s^\top}_{[:,i]} {\mathbf{G_y}}_{[i,:]}$, \ $\forall i \in \left\{1,\dots,m\right\}$
%             \State \quad\quad {Select the K largest indices $\left\{i_1,\cdots,i_\text{K}\right\}$ by $\Vert \alpha_i \Vert_F$}
            
%             \State \quad\quad $\mathcal{I} = \left\{i_1,\cdots,i_\text{K}\right\}$ \Comment{Here K $= \lceil mp \rceil$}


%         \State \quad $\mathbf{G_x} = {\mathbf{W}_s^\top}_{[,\mathcal{I}]} {\mathbf{G_y}}_{[\mathcal{I},]} + \mathbf{A}_0^\top(\mathbf{B}_0^\top \mathbf{G_y})$
%         \State \quad  \textbf{Return} $\mathbf{G_x}$
% % \EndFor
% \end{algorithmic}
% \end{algorithm}

Overall, \celora integrates AMM, double-LoRA, and layer-wise adaptivity, as outlined in Algorithm~\ref{alg:CeLoRA}.
% \cgd{ADD For every training loop}
During model initialization, we replace all frozen linear layers with \celora and apply the double-LoRA technique to the weight matrix $\mathbf{W}_0$, resulting in low-rank components $\mathbf{A}_0$ and $\mathbf{B}_0$ (lines 2–3). 
% 
For each training step $t$, the forward pass of a \celora linear layer behaves the same as the original frozen linear layer (lines 8).
% 
In the backward pass, \celora first computes the residual weight matrix by subtracting the low-rank components from the original weight matrix (line 11). 
% 
Next, if the current step $t$ is a multiple of $\tau$ or if the indices are empty (\eg, at the start of training), the top-K indices are updated (lines 12–15). 
% 
Finally, \celora uses AMM to compute activation gradient $\mathbf{G}_\mathbf{x_\ell}$ (line 20).

\begin{table*}[!b]
    \centering
    \caption{Computation and memory analysis for a single linear layer.}
    \label{tab:complexity}
    \vspace{0.5em}
    \begin{adjustbox}{max width=\textwidth}
    \begin{tabular}{cccc}
    \toprule
         \textbf{Method} & Standard AdamW & LoRA & \celora\\
    \midrule
    \multirow{2}{*}{\textbf{Memory Usage}} & \multirow{2}{*}{$10mn+2bm$} & $2mn+10r(m+n)$ & $2mn+2r_0(m+n)$\\
    & &$+2b(m+r)$ & {$+10r(m+n)+2b(m+r)$}\\
    \midrule
    \textbf{Forward Computation} & $2bmn$ & $2bmn+2br(m+n)$ & $2bmn+2br(m+n)$\\
    \midrule
    \multirow{2}{*}{\textbf{Backward Computation}} & \multirow{2}{*}{$4bmn$} & \multirow{2}{*}{$2bmn+4br(m+n)$} & $(2pb+1)mn$\\
    & & & $+2(r_0 + br_0 + 2br)(m+n)$\\
    % $(2pb+1)mn+2r_0(m+n)$\\
    % & & & $+(2br_0+4br)(m+n)$\\
    \bottomrule
    \end{tabular}
    \end{adjustbox}
\end{table*}
\subsection{Complexity Analysis}
To better illustrate the computational efficiency of \celora, we theoretically compare the computational and memory complexity of \celora with LoRA and standard AdamW fine-tuning. Consider linear layer $\mathbf{y}=\mathbf{W}\mathbf{x}$ with $\mathbf{W}\in\mathbb{R}^{m\times n}$, trained with LoRA rank $r$, double-LoRA rank $r_0$, structured sparsity $p$ and batch size $b$ using BF16 precision. As illustrated in Table~\ref{tab:complexity}, \celora can achieve a memory usage similar to LoRA by applying slightly smaller $r_0$ and $r$, while significantly reduce the backward computation by applying a relatively small $p$ when $b\gg 1$ and $r\ll\min\{m,n\}$. 
When combined with low-precision training, the influence of double-LoRA can be further reduced, as the frozen low-rank parameters do not require high-precision weight copies or gradient accumulators.


\section{Analysis}
\label{sec:analysis}
In the following sections, we will analyze European type approval regulation\footnote{Strictly speaking, the German enabling act (AFGBV) does not regulate type-approval, but how test \& operating permits are issued for SAE-Level-4 systems. Type-approval regulation for SAE-Level-3 systems follows UN Regulation No. 157 (UN-ECE-ALKS) \parencite{un157}.} regarding the underlying notions of ``safety'' and ``risk''.
We will classify these notions according to their absolute or relative character, underlying risk sources, or underlying concepts of harm.

\subsection{Classification of Safety Notions}
\label{sec:safety-notions}
We will refer to \emph{absolute} notions of safety as conceptualizations that assume the complete absence of any kind of risk.
Opposed to this, \emph{relative} notions of safety are based on a conceptualization that specifically includes risk acceptance criteria, e.g., in terms of ``tolerable'' risk or ``sufficient'' safety.

For classifying notions of safety by their underlying risk (or rather ``hazard'') sources, and different concepts of harm, \Cref{fig:hazard-sources} provides an overview of our reasoning, which is closely in line with the argumentation provided by Waymo in \parencite{favaro2023}.
We prefer ``hazard sources'' over ``risk sources'', as a risk must always be related to a \emph{cause} or \emph{source of harm} (i.e., a hazard \parencite[p.~1, def. 3.2]{iso51}).
Without a concrete (scenario) context that the system is operating in, a hazard is \emph{latent}: E.g., when operating in public traffic, there is a fundamental possibility that a \emph{collision with a pedestrian} leads to (physical) harm for that pedestrian. 
However, only if an automated vehicle shows (potentially) hazardous behavior (e.g., not decelerating properly) \emph{and} is located near a pedestrian (context), the hazard is instantiated and leads to a hazardous event.
\begin{figure*}
    \includeimg[width=.9\textwidth]{hazard-sources0.pdf}
    \caption{Graphical summary of a taxonomy of risk related to automated vehicles, extended based on ISO 21448 (\parencite{iso21448}) and \parencite{favaro2023}. Top: Causal chain from hazard sources to actual harm; bottom: summary of the individual elements' contributions to a resulting risk. Graphic translated from \parencite{nolte2024} \label{fig:hazard-sources}}
\end{figure*}
If the hazardous event cannot be mitigated or controlled, we see a loss event in which the pedestrian's health is harmed.
Note that this hypothetical chain of events is summarized in the definition of risk:
The probability of occurrence of harm is determined by a) the frequency with which hazard sources manifest, b) the time for which the system operates in a context that exposes the possibility of harm, and c) by the probability with which a hazardous event can be controlled.
A risk can then be determined as a function of the probability of harm and the severity of the harm potentially inflicted on the pedestrian.

In the following, we will apply this general model to introduce different types of hazard sources and also different types of harm.
\cref{fig:hazard-sources} shows two distinct hazard sources, i.e., functional insufficiencies and E/E-failures that can lead to hazardous behavior.
ISO~21488 \parencite{iso21448} defines functional insufficiencies as insufficiencies that stem from an incomplete or faulty system specification (specification insufficiencies).
In addition, the standard considers insufficiencies that stem from insufficient technical capability to operate inside the targeted Operational Design Domain (performance insufficiencies).
Functional insufficiencies are related to the ``Safety of the Intended Functionality (SOTIF)'' (according to ISO~21448), ``Behavioral Safety'' (according to Waymo \parencite{waymo2018}), or ``Operational Safety'' (according to UN Regulation No. 157 \parencite{un157}).
E/E-Failures are related to classic functional safety and are covered exhaustively by ISO~26262 \parencite{iso2018}.
Additional hazard sources can, e.g., be related to malicious security attacks (ISO~21434), or even to mechanical failures that should be covered (in the US) in the Federal Motor Vehicle Safety Standards (FMVSS).

For the classification of notions of safety by the related harm, in \parencite{salem2024, nolte2024}, we take a different approach compared to \parencite{koopman2024}:
We extend the concept of harm to the violation of stakeholder \emph{values}, where values are considered to be a ``standard of varying importance among other such standards that, when combined, form a value pattern that reduces complexity for stakeholders [\ldots] [and] determines situational actions [\ldots].'' \parencite{albert2008}
In this sense, values are profound, personal determinants for individual or collective behavior.
The notion of values being organized in a weighted value pattern shows that values can be ranked according to importance.
For automated vehicles, \emph{physical wellbeing} and \emph{mobility} can, e.g., be considered values which need to be balanced to achieve societal acceptance, in line with the discussion of required tradeoffs in \cref{sec:terminology}.
For the analysis of the following regulatory frameworks, we will evaluate if the given safety or risk notions allow tradeoffs regarding underlying stakeholder values. 

\subsection{UN Regulation No. 157 \& European Implementing Regulation (EU) 2022/1426}
\label{sec:enabling-act}
UN Regulation No. 157 \parencite{un157} and the European Implementing Regulation 2022/1426 \parencite{eu1426} provide type approval regulation for automated vehicles equipped with SAE-Level-3 (UN Reg. 157) and Level 4 (EU 2022/1426) systems on an international (UN Reg. 157) and European (EU 2022/1426) level.

Generally, EU type approval considers UN ECE regulations mandatory for its member states ((EU) 2018/858, \parencite{eu858}), while the EU largely forgoes implementing EU-specific type approval rules, it maintains the right to alter or to amend UN ECE regulation \parencite{eu858}.

In this respect, the terminology and conceptualizations in the EU Implementing Act closely follow those in UN Reg. No. 157.
The EU Implementing Act gives a clear reference to UN Reg. No. 157 \parencite[][Preamble,  Paragraph 1]{eu1426}.
Hence, the documents can be assessed in parallel.
Differences will be pointed out as necessary.

Both acts are written in rather technical language, including the formulation of technical requirements (e.g., regarding deceleration values or speeds in certain scenarios).
While providing exhaustive definitions and terminology, neither of both documents provide an actual definition of risk or safety.
The definition of ``unreasonable'' risk in both documents does not define risk, but only what is considered \emph{unreasonable}. It states that the ``overall level of risk for [the driver, (only in UN Reg. 157)] vehicle occupants and other road users which is increased compared to a competently and carefully driven manual vehicle.''
The pertaining notions of safety and risk can hence only be derived from the context in which they are used.

\subsubsection{Absolute vs. Relative Notions of Safety}
In line with the technical detail provided in the acts, both clearly imply a \emph{relative} notion of safety and refer to the absence of \emph{unreasonable} risk throughout, which is typical for technical safety definitions.

Both acts require sufficient proof and documentation that the to-be-approved automated driving systems are ``free of unreasonable safety risks to vehicle occupants and other road users'' for type approval.\footnote{As it targets SAE-Level-3 systems, UN Reg. 157 also refers to the driver, where applicable.}
In this respect, both acts demand that the manufacturers perform verification and validation activities for performance requirements that include ``[\ldots] the conclusion that the system is designed in such a way that it is free from unreasonable risks [\ldots]''.
Additionally, \emph{risk minimization} is a recurring theme when it comes to the definition of Minimum Risk Maneuvers (MRM).

Finally, supporting the relative notions of safety and risk, UN Reg. 157 introduces the concept of ``reasonable foreseeable and preventable'' \parencite[Article 1, Clause 5.1.1.]{un157} collisions, which implies that a residual risk will remain with the introduction of automated vehicles.
\parencite[][Appendix 3, Clause 3.1.]{un157} explicitly states that only \emph{some} scenarios that are unpreventable for a competent human driver can actually be prevented by an automated driving system.
While this concept is not applied throughout the EU Implementing Act, both documents explicitly refer to \emph{residual} risks that are related to the operation of automated driving systems (\parencite[][Annex I, Clause 1]{un157}, \parencite[][Annex II, Clause 7.1.1.]{eu1426}).

\subsubsection{Hazard Sources}
Hazard sources that are explicitly differentiated in UN Reg. 157 and (EU) 2022/1426 are E/E-failures that are in scope of functional safety (ISO~26262) and functional insufficiencies that are in scope of behavioral (or ``operational'') safety (ISO~21448).
Both documents consistently differentiate both sources when formulating requirements.

While the acts share a common definition of ``operational'' safety (\parencite[][Article 2, def. 30.]{eu1426}, \parencite[][Annex 4, def. 2.15.]{un157}), the definitions for functional safety differ.
\parencite{un157} defines functional safety as the ``absence of unreasonable risk under the occurrence of hazards caused by a malfunctioning behaviour of electric/electronic systems [\ldots]'', \parencite{eu1426} drops the specification of ``electric/electronic systems'' from the definition.
When taken at face value, this definition would mean that functional safety included all possible hazard sources, regardless of their origin, which is a deviation from the otherwise precise usage of safety-related terminology.

\subsubsection{Harm Types}
As the acts lack explicit definitions of safety and risk, there is no consistent and explicit notion of different harm types that could be differentiated.

\parencite{un157} gives little hints regarding different considered harm types.
``The absence of unreasonable risk'' in terms of human driving performance could hence be related to any chosen performance metric that allows a comparison with a competent careful human driver including, e.g., accident statistics, statistics about rule violations, or changes in traffic flow.

In \parencite{eu1426}, ``safety'' is, implicitly, attributed to the absence of unreasonable risk to life and limb of humans.
This is supported by the performance requirements that are formulated:
\parencite[][Annex II, Clause 1.1.2. (d)]{eu1426} demands that an automated driving system can adapt the vehicle behavior in a way that it minimizes risk and prioritizes the protection of human life.

Both acts demand the adherence to traffic rules (\parencite[][Annex 2, Clause 1.3.]{eu1426}, \parencite[][Clause 5.1.2.]{un157}).
\parencite[][Annex II, Clause 1.1.2. (c)]{eu1426} also demands that an automated driving system shall adapt its behavior to surrounding traffic conditions, such as the current traffic flow.
With the relative notion of risk in both acts, the unspecific clear statement that there may be unpreventable accidents \parencite{un157}, and a demand of prioritization of human life in \parencite{eu1426}, both acts could be interpreted to allow developers to make tradeoffs as discussed in \cref{sec:terminology}.


\subsubsection{Conclusion}
To summarize, the UN Reg. 157 and the (EU) 2022/1426 both clearly support the technical notion of safety as the absence of unreasonable risk.
The notion is used consistently throughout both documents, providing a sufficiently clear terminology for the developers of automated vehicles.
Uncertainty remains when it comes to considered harm types: Both acts do not explicitly allow for broader notions of safety, in the sense of \parencite{koopman2024} or \parencite{salem2024}.
Finally, a minor weak spot can be seen in the definition of risk acceptance criteria: Both acts take the human driving performance as a baseline.
While (EU) 2022/1426 specifies that these criteria are specific to the systems' Operational Design Domain \parencite[][Annex II, Clause 7.1.1.]{eu1426}, the reference to the concrete Operational Design Domain is missing in UN Reg. 157.
Without a clearly defined notion of safety, however, it remains unclear, how aspects beyond net accident statistics (which are given as an example in \parencite[][Annex II, Clause 7.1.1.]{eu1426}), can be addressed practically, as demanded by \parencite{koopman2024}.

\subsection{German Regulation (StVG \& AFGBV)}
\label{sec:afgbv}
The German L3 (Automated Driving Act) and L4 (Act on Autonomous Driving) Acts from 2017 and 2021,\footnote{Formally, these are amendments to the German Road Traffic Act (StVG): 06/21/2017, BGBl. I p. 1648, 07/12/2021 BGBl. I p. 3108.} respectively, provide enabling regulation for the operation of SAE-Level-3 and 4 vehicles on German roads.
The German Implementing Regulation (\parencite{afgbv}, AFGBV) defines how this enabling regulation is to be implemented for granting testing permits for SAE-Level-3 and -4 and driving permits for SAE-Level-3 and -4 automated driving systems.\footnote{Note that these permits do not grant EU-wide type approval, but serve as a special solution for German roads only. At the same time, the AFGBV has the same scope as (EU) 2022/1426.}
With all three acts, Germany was the first country to regulate the approval of automated vehicles for a domestic market.
All acts are subject to (repeated) evaluation until the year 2030 regarding their impact on the development of automated driving technology.
An assessment of the German AFGBV and comparisons to (EU) 2022/1426 have been given in \cite{steininger2022} in German.

Just as for UN Reg. 157 and (EU) 2022/1426, neither the StVG nor the AFGBV provide a clear definition of ``safety'' or ``risk'' -- even though the "safety" of the road traffic is one major goal of the StVG and StVO.
Again, different implicit notions of both concepts can only be interpreted from the context of existing wording.
An additional complication that is related to the German language is that ``safety'' and ``security'' can both be addressed as ``Sicherheit'', adding another potential source of unclarity.
Literal Quotations in this section are our translations from the German act.

\subsubsection{Absolute vs. Relative Notions of Safety}
For assessing absolute vs. relative notions of safety in German regulation, it should be mentioned that the main goal of the German StVO is to ensure the ``safety and ease of traffic flow'' -- an already diametral goal that requires human drivers to make tradeoffs.\footnote{For human drivers, this also creates legal uncertainty which can sometimes only be settled in a-posteriori court cases.}
While UN and EU regulation clearly shows a relative notion of safety\footnote{And even the StVG contains sections that use wording such as ``best possible safety for vehicle occupants'' (§1d (4) StVG) and acknowledges that there are unavoidable hazards to human life (§1e (2) No. 2c)).}, the German AFGBV contains ambiguous statements in this respect:
Several paragraphs contain a demand for a hazard free operation of automated vehicles.
§4 (1) No. 4 AFGBV, e.g., states that ``the operation of vehicles with autonomous driving functions must neither negatively impact road traffic safety or traffic flow, nor endanger the life and limb of persons.''
Additionally, §6 (1) AFGBV states that the permits for testing and operation have to be revoked, if it becomes apparent that a ``negative impact on road traffic safety or traffic flow, or hazards to the life and limb of persons cannot be ruled out''.
The same wording is used for the approval of operational design domains regulated in §10 (1) No. 1.
A particularly misleading statement is made regarding the requirements for technical supervision instances which are regulated in §14 (3) AFGBV which states that an automated vehicle has to be  ``immediately removed from the public traffic space if a risk minimal state leads to hazards to road traffic safety or traffic flow''.
Considering the argumentation in \cref{sec:terminology}, that residual risks related to the operation of automated driving systems are inevitable, these are strong statements which, if taken at face value, technically prohibit the operation of automated vehicles.
It suggests an \emph{absolute} notion of safety that requires the complete absence of risk.  
The last statement above is particularly contradictory in itself, considering that a risk \emph{minimal} state always implies a residual risk.

In addition to these absolute safety notions, there are passages which suggest a relative notion of safety:
The approval for Operational Design Domains is coupled to the proof that the operation of an automated vehicle ``neither negatively impacts road traffic safety or traffic flow, nor significantly endangers the life and limb of persons beyond the general risk of an impact that is typical of local road traffic'' (§9 (2) No. 3 AFGBV).
The addition of a relative risk measure ``beyond the general risk of an impact'' provides a relaxation (cf. also \cite{steininger2022}, who criticizes the aforementioned absolute safety notion) that also yields an implicit acceptance criterion (\emph{statistically as good as} human drivers) similar to the requirements stated in UN Reg. 157 and (EU) 2022/1426.

Additional hints for a relative notion of safety can be found in Annex 1, Part 1, No. 1.1 and Annex 1, Part 2, No. 10.
Part 1, No 1.1 specifies collision-avoidance requirements and acknowledges that not all collisions can be avoided.\footnote{The same is true for Part 2, No. 10, Clause 10.2.5.}
Part 2, No. 10 specifies requirements for test cases.
It demands that test cases are suitable to provide evidence that the ``safety of a vehicle with an autonomous driving function is increased compared to the safety of human-driven vehicles''.
This does not only acknowledge residual risks, but also yields an acceptance criterion (\emph{better} than human drivers) that is different from the implied acceptance criterion given in §9 (2) No. 3 AFGBV.

\subsubsection{Hazard Sources}
Regarding hazard sources, Annex 1 and 3 AFGBV explicitly refer to ISO~26262 and ISO~21448 (or rather its predecessor ISO/PAS~21448:2019).
However, regarding the discussion of actual hazard sources, the context in which both standards are mentioned is partially unclear:
Annex 1, Clause 1.3 discusses requirements for path and speed planning.
Clause 1.3 d) demands that in intersections, a Time to Collision (TTC) greater than 3 seconds must be guaranteed.
If manufacturers deviate from this, it is demanded that ``state-of-the-art, systematic safety evaluations'' are performed.
Fulfillment of the state of the art is assumed if ``the guidelines of ISO~26262:2018-12 Road Vehicles -- Functional Safety are fulfilled''.
Technically, ISO~26262 is not suitable to define the state of the art in this context, as the requirements discussed fall in the scope of operational (or behavioral) safety (ISO~21448).
A hazard source ``violated minimal time to collision'' is clearly a functional insufficiency, not an E/E-failure.

Similar unclarity presents itself in Annex 3, Clause 1 AFGBV: 
Clause 1 specifies the contents of the ``functional specification''.
The ``specification of the functionality'' is an artifact which is demanded in ISO~21448:2022 (Clause 5.3) \parencite{iso21448}.
However, Annex 3, Clause 1 AFGBV states that the ``functional specification'' is considered to comply to the state of the art, if the ``functional specification'' adheres to ISO~26262-3:2018 (Concept Phase).
Again, this assumes SOTIF-related contents as part of ISO~26262, which introduces the ``Item Definition'' as an artifact, which is significantly different from the ``specification of the functionality'' which is demanded by ISO~21448.
Finally, Annex 3, Clause 3 AFGBV demands a ``documentation of the safety concept'' which ``allows a functional safety assessment''.
A safety concept that is related to operational / behavioral safety is not demanded.
Technically, the unclarity with respect to the addressed harm types lead to the fact that the requirements provided by the AFGBV do not comply with the state of the art in the field, providing questionable regulation.

\subsubsection{Harm Types}
Just like UN Reg. 157 and (EU) 2022/1426, the German StVG and AFGBV do not explicitly differentiate concrete harm types for their notions of safety.
However, the AFGBV mentions three main concerns for the operation of automated vehicles which are \emph{traffic flow} (e.g., §4 (1) No. 4 AFGBV), compliance to \emph{traffic law} (e.g., §1e (2) No. 2 StVG), and the \emph{life and limb of humans} (e.g., §4 (1) No. 4 AFGBV).

Again, there is some ambiguity in the chosen wording:
The conflict between traffic flow and safety has already been argued in \cref{sec:terminology}.
The wording given in §4 (1) No. 4 and §6 (1) AFGBV  demand to ensure (absolute) safety \emph{and} traffic flow at the same time, which is impossible (cf. \cref{sec:terminology}) from an engineering perspective.
§1e (2) No. 2 StVG defines that ``vehicles with an autonomous driving function must [\ldots] be capable to comply to [\ldots] traffic rules in a self-contained manner''.
Taken at face value, this wording implies that an automated driving system could lose its testing or operating permit as soon as it violates a traffic rule.
A way out could be provided by §1 of the German Traffic Act (StVO) which demands careful and considerate behavior of all traffic participants and by that allows judgement calls for human drivers.
However, if §1 is applicable in certain situations is often settled in court cases. 
For developers, the application of §1 StVO during system design hence remains a legal risk.

While there are rather absolute statements as mentioned above, sections of the AFGBV and StVG can be interpreted to allow tradeoffs:
§1e (2) No. 2 b) demands that a system,  ``in case of an inevitable, alternative harm to legal objectives, considers the significance of the legal objectives, where the protection of human life has highest priority''.
This exact wording \emph{could} provide some slack for the absolute demands in other parts of the acts, enabling tradeoffs between (tolerable) risk and mobility as discussed in \cref{sec:terminology}.
However, it remains unclear if this interpretation is legally possible.

\subsubsection{Conclusion}
Compared to UN Reg. 157 and (EU) 2022/1426, the German StVG and AFGBV introduce openly inconsistent notions of safety and risk which are partially directly contradictory:
The wording partially implies absolute and relative notions of safety and risk at the same time.
The implied validation targets (``better'' or ``as good as'' human drivers) are equally contradictory. 
The partially implied absolute notions of safety, when taken at face value, prohibit engineers from making the tradeoffs required to develop a system that is safe and provides customer benefit at the same time. 
In consequence, the wording in the acts is prone to introducing legal uncertainty.
This uncertainty creates additional clarification need and effort for manufacturers and engineers who design and develop SAE-Level-3 and -4 automated driving systems. The use of undefined legal terms not only makes it more difficult for engineers to comply with the law, but also complicates the interpretation of the law and leads to legal uncertainty.

\subsection{UK Automated Vehicles Act 2024 (2024 c. 10)}
The UK has issued a national enabling act for regulating the approval of automated vehicles on the roads in the UK.
To the best of our knowledge, concrete implementing regulation has not been issued yet.
Regarding terminology, the act begins with a dedicated terminology section to clarify the terms used in the act \parencite[Part 1, Chapter 1, Section 1]{ukav2024}.
In that regard, the act defines a vehicle to drive ```autonomously' if --- (a)
it is being controlled not by an individual but by equipment of the vehicle, and (b) neither the vehicle nor its surroundings are being monitored by an individual with a view to immediate intervention in the driving of the vehicle.''
The act hence covers SAE-Level-3 to SAE-Level-5 automated driving systems.

\subsubsection{Absolute vs. Relative Notions of Safety}
While not providing an explicit definition of safety and risk, the UK Automated Vehicles Act (``UK AV Act'') \parencite{ukav2024} explicitly refers to a relative notion of safety.
Part~1, Chapter~1, Section~1, Clause (7)~(a) defines that an automated vehicle travels ```safely' if it travels to an acceptably safe standard''.
This clarifies that absolute safety is not achievable and that acceptance criteria to prove the acceptability of residual risk are required, even though a concrete safety definition is not given.
The act explicitly tasks the UK Secretary of State\footnote{Which means, that concrete implementation regulation needs to be enacted.} to install safety principles to determine the ``acceptably safe standard'' in Part~1, Chapter~1, Section~1, Clause (7)~(a).
In this respect, the act also provides one general validation target as it demands that the safety principles must ensure that ``authorized automated vehicles will achieve a level of safety equivalent to, or higher than, that of careful and competent human drivers''.
Hence, the top-level validation risk acceptance criterion assumed for UK regulation is ``\emph{at least as good} as human drivers''.

\subsubsection{Hazard Sources}
The UK AV Act contains no statements that could be directly related to different hazard sources.
Note that, in contrast to the rest of the analyzed documents, the UK AV Act is enabling rather than implementing regulation.
It is hence comparable to the German StVG, which does not refer to concrete hazard sources as well.

\subsubsection{Types of Harm}
Even though providing a clear relative safety notion, the missing definition of risk also implies a lack of explicitly differentiable types of harm.
Implicitly, three different types of harm can be derived from the wording in the act.
This includes the harm to life and limb of humans\footnote{Part~1, Chapter~3, Section~25 defines ``aggravated offence where death or serious injury occurs'' \parencite{ukav2024}.}, the violation of traffic rules\footnote{Part~1, Chapter~1, Clause~(7)~(b) defines that an automated vehicle travels ```legally' if it travels with an acceptably low risk of committing a traffic infraction''}, and the cause of inconvenience to the public \parencite[Part~1, Chapter~1, Section~58, Clause (2)~(d)]{ukav2024}.

The act connects all the aforementioned types of harm to ``risk'' or ``acceptable safety''.
While the act generally defines criminal offenses for providing ``false or misleading information about safety'', it also acknowledges possible defenses if it can be proven that ``reasonable precautions'' were taken and that ``due diligence'' was exercised to ``avoid the commission of the offence''.
This statement could enable tradeoffs within the scope of ``reasonable risk'' to the life and limb of humans, the violation of traffic rules, or to the cause of inconvenience to the public, as we argued in \cref{sec:terminology}.

\subsubsection{Conclusion}
From the set of reviewed documents, the current UK AV Act is the one with the most obvious relative notions of safety and risk and the one that seems to provide a legal framework for permitting tradeoffs.
In our review, we did not spot major inconsistency beyond a missing definitions of safety and risk\footnote{Note that with the Office for Product Safety and Standards (OPSS), there is a British government agency that maintains an exhaustive and widely focussed ``Risk Lexicon'' that provides suitable risk definitions. For us, it remains unclear, to what extent this terminology is assumed general knowledge in British legislation.}.
The general, relative notion of safety and the related alleged ability for designers to argue well-founded development tradeoffs within the legal framework could prove beneficial for the actual implementation of automated driving systems.
While the act thus appears as a solid foundation for the market introduction of automated vehicles, without accompanying implementing regulation, it is too early to draw definite conclusions.

\section{Experiments}
\label{sec:exp}
Following the settings in Section \ref{sec:existing}, we evaluate \textit{NovelSum}'s correlation with the fine-tuned model performance across 53 IT datasets and compare it with previous diversity metrics. Additionally, we conduct a correlation analysis using Qwen-2.5-7B \cite{yang2024qwen2} as the backbone model, alongside previous LLaMA-3-8B experiments, to further demonstrate the metric's effectiveness across different scenarios. Qwen is used for both instruction tuning and deriving semantic embeddings. Due to resource constraints, we run each strategy on Qwen for two rounds, resulting in 25 datasets. 

\subsection{Main Results}

\begin{table*}[!t]
    \centering
    \resizebox{\linewidth}{!}{
    \begin{tabular}{lcccccccccc}
    \toprule
    \multirow{3}*{\textbf{Diversity Metrics}} & \multicolumn{10}{c}{\textbf{Data Selection Strategies}} \\
    \cmidrule(lr){2-11}
    & \multirow{2}*{\textbf{K-means}} & \multirow{2}*{\vtop{\hbox{\textbf{K-Center}}\vspace{1mm}\hbox{\textbf{-Greedy}}}}  & \multirow{2}*{\textbf{QDIT}} & \multirow{2}*{\vtop{\hbox{\textbf{Repr}}\vspace{1mm}\hbox{\textbf{Filter}}}} & \multicolumn{5}{c}{\textbf{Random}} & \multirow{2}{*}{\textbf{Duplicate}} \\ 
    \cmidrule(lr){6-10}
    & & & & & \textbf{$\mathcal{X}^{all}$} & ShareGPT & WizardLM & Alpaca & Dolly &  \\
    \midrule
    \rowcolor{gray!15} \multicolumn{11}{c}{\textit{LLaMA-3-8B}} \\
    Facility Loc. $_{\times10^5}$ & \cellcolor{BLUE!40} 2.99 & \cellcolor{ORANGE!10} 2.73 & \cellcolor{BLUE!40} 2.99 & \cellcolor{BLUE!20} 2.86 & \cellcolor{BLUE!40} 2.99 & \cellcolor{BLUE!0} 2.83 & \cellcolor{BLUE!30} 2.88 & \cellcolor{BLUE!0} 2.83 & \cellcolor{ORANGE!20} 2.59 & \cellcolor{ORANGE!30} 2.52 \\    
    DistSum$_{cosine}$  & \cellcolor{BLUE!30} 0.648 & \cellcolor{BLUE!60} 0.746 & \cellcolor{BLUE!0} 0.629 & \cellcolor{BLUE!50} 0.703 & \cellcolor{BLUE!10} 0.634 & \cellcolor{BLUE!40} 0.656 & \cellcolor{ORANGE!30} 0.578 & \cellcolor{ORANGE!10} 0.605 & \cellcolor{ORANGE!20} 0.603 & \cellcolor{BLUE!10} 0.634 \\
    Vendi Score $_{\times10^7}$ & \cellcolor{BLUE!30} 1.70 & \cellcolor{BLUE!60} 2.53 & \cellcolor{BLUE!10} 1.59 & \cellcolor{BLUE!50} 2.23 & \cellcolor{BLUE!20} 1.61 & \cellcolor{BLUE!30} 1.70 & \cellcolor{ORANGE!10} 1.44 & \cellcolor{ORANGE!20} 1.32 & \cellcolor{ORANGE!10} 1.44 & \cellcolor{ORANGE!30} 0.05 \\
    \textbf{NovelSum (Ours)} & \cellcolor{BLUE!60} 0.693 & \cellcolor{BLUE!50} 0.687 & \cellcolor{BLUE!30} 0.673 & \cellcolor{BLUE!20} 0.671 & \cellcolor{BLUE!40} 0.675 & \cellcolor{BLUE!10} 0.628 & \cellcolor{BLUE!0} 0.591 & \cellcolor{ORANGE!10} 0.572 & \cellcolor{ORANGE!20} 0.50 & \cellcolor{ORANGE!30} 0.461 \\
    \midrule    
    \textbf{Model Performance} & \cellcolor{BLUE!60}1.32 & \cellcolor{BLUE!50}1.31 & \cellcolor{BLUE!40}1.25 & \cellcolor{BLUE!30}1.05 & \cellcolor{BLUE!20}1.20 & \cellcolor{BLUE!10}0.83 & \cellcolor{BLUE!0}0.72 & \cellcolor{ORANGE!10}0.07 & \cellcolor{ORANGE!20}-0.14 & \cellcolor{ORANGE!30}-1.35 \\
    \midrule
    \midrule
    \rowcolor{gray!15} \multicolumn{11}{c}{\textit{Qwen-2.5-7B}} \\
    Facility Loc. $_{\times10^5}$ & \cellcolor{BLUE!40} 3.54 & \cellcolor{ORANGE!30} 3.42 & \cellcolor{BLUE!40} 3.54 & \cellcolor{ORANGE!20} 3.46 & \cellcolor{BLUE!40} 3.54 & \cellcolor{BLUE!30} 3.51 & \cellcolor{BLUE!10} 3.50 & \cellcolor{BLUE!10} 3.50 & \cellcolor{ORANGE!20} 3.46 & \cellcolor{BLUE!0} 3.48 \\ 
    DistSum$_{cosine}$ & \cellcolor{BLUE!30} 0.260 & \cellcolor{BLUE!60} 0.440 & \cellcolor{BLUE!0} 0.223 & \cellcolor{BLUE!50} 0.421 & \cellcolor{BLUE!10} 0.230 & \cellcolor{BLUE!40} 0.285 & \cellcolor{ORANGE!20} 0.211 & \cellcolor{ORANGE!30} 0.189 & \cellcolor{ORANGE!10} 0.221 & \cellcolor{BLUE!20} 0.243 \\
    Vendi Score $_{\times10^6}$ & \cellcolor{ORANGE!10} 1.60 & \cellcolor{BLUE!40} 3.09 & \cellcolor{BLUE!10} 2.60 & \cellcolor{BLUE!60} 7.15 & \cellcolor{ORANGE!20} 1.41 & \cellcolor{BLUE!50} 3.36 & \cellcolor{BLUE!20} 2.65 & \cellcolor{BLUE!0} 1.89 & \cellcolor{BLUE!30} 3.04 & \cellcolor{ORANGE!30} 0.20 \\
    \textbf{NovelSum (Ours)}  & \cellcolor{BLUE!40} 0.440 & \cellcolor{BLUE!60} 0.505 & \cellcolor{BLUE!20} 0.403 & \cellcolor{BLUE!50} 0.495 & \cellcolor{BLUE!30} 0.408 & \cellcolor{BLUE!10} 0.392 & \cellcolor{BLUE!0} 0.349 & \cellcolor{ORANGE!10} 0.336 & \cellcolor{ORANGE!20} 0.320 & \cellcolor{ORANGE!30} 0.309 \\
    \midrule
    \textbf{Model Performance} & \cellcolor{BLUE!30} 1.06 & \cellcolor{BLUE!60} 1.45 & \cellcolor{BLUE!40} 1.23 & \cellcolor{BLUE!50} 1.35 & \cellcolor{BLUE!20} 0.87 & \cellcolor{BLUE!10} 0.07 & \cellcolor{BLUE!0} -0.08 & \cellcolor{ORANGE!10} -0.38 & \cellcolor{ORANGE!30} -0.49 & \cellcolor{ORANGE!20} -0.43 \\
    \bottomrule
    \end{tabular}
    }
    \caption{Measuring the diversity of datasets selected by different strategies using \textit{NovelSum} and baseline metrics. Fine-tuned model performances (Eq. \ref{eq:perf}), based on MT-bench and AlpacaEval, are also included for cross reference. Darker \colorbox{BLUE!60}{blue} shades indicate higher values for each metric, while darker \colorbox{ORANGE!30}{orange} shades indicate lower values. While data selection strategies vary in performance on LLaMA-3-8B and Qwen-2.5-7B, \textit{NovelSum} consistently shows a stronger correlation with model performance than other metrics. More results are provided in Appendix \ref{app:results}.}
    \label{tbl:main}
    \vspace{-4mm}
\end{table*}


\begin{table}[t!]
\centering
\resizebox{\linewidth}{!}{
\begin{tabular}{lcccc}
\toprule
\multirow{2}*{\textbf{Diversity Metrics}} & \multicolumn{3}{c}{\textbf{LLaMA}} & \textbf{Qwen}\\
\cmidrule(lr){2-4} \cmidrule(lr){5-5} 
& \textbf{Pearson} & \textbf{Spearman} & \textbf{Avg.} & \textbf{Avg.} \\
\midrule
TTR & -0.38 & -0.16 & -0.27 & -0.30 \\
vocd-D & -0.43 & -0.17 & -0.30 & -0.31 \\
\midrule
Facility Loc. & 0.86 & 0.69 & 0.77 & 0.08 \\
Entropy & 0.93 & 0.80 & 0.86 & 0.63 \\
\midrule
LDD & 0.61 & 0.75 & 0.68 & 0.60 \\
KNN Distance & 0.59 & 0.80 & 0.70 & 0.67 \\
DistSum$_{cosine}$ & 0.85 & 0.67 & 0.76 & 0.51 \\
Vendi Score & 0.70 & 0.85 & 0.78 & 0.60 \\
DistSum$_{L2}$ & 0.86 & 0.76 & 0.81 & 0.51 \\
Cluster Inertia & 0.81 & 0.85 & 0.83 & 0.76 \\
Radius & 0.87 & 0.81 & 0.84 & 0.48 \\
\midrule
NovelSum & \textbf{0.98} & \textbf{0.95} & \textbf{0.97} & \textbf{0.90} \\
\bottomrule
\end{tabular}
}
\caption{Correlations between different metrics and model performance on LLaMA-3-8B and Qwen-2.5-7B.  “Avg.” denotes the average correlation (Eq. \ref{eq:cor}).}
\label{tbl:correlations}
\vspace{-2mm}
\end{table}

\paragraph{\textit{NovelSum} consistently achieves state-of-the-art correlation with model performance across various data selection strategies, backbone LLMs, and correlation measures.}
Table \ref{tbl:main} presents diversity measurement results on datasets constructed by mainstream data selection methods (based on $\mathcal{X}^{all}$), random selection from various sources, and duplicated samples (with only $m=100$ unique samples). 
Results from multiple runs are averaged for each strategy.
Although these strategies yield varying performance rankings across base models, \textit{NovelSum} consistently tracks changes in IT performance by accurately measuring dataset diversity. For instance, K-means achieves the best performance on LLaMA with the highest NovelSum score, while K-Center-Greedy excels on Qwen, also correlating with the highest NovelSum. Table \ref{tbl:correlations} shows the correlation coefficients between various metrics and model performance for both LLaMA and Qwen experiments, where \textit{NovelSum} achieves state-of-the-art correlation across different models and measures.

\paragraph{\textit{NovelSum} can provide valuable guidance for data engineering practices.}
As a reliable indicator of data diversity, \textit{NovelSum} can assess diversity at both the dataset and sample levels, directly guiding data selection and construction decisions. For example, Table \ref{tbl:main} shows that the combined data source $\mathcal{X}^{all}$ is a better choice for sampling diverse IT data than other sources. Moreover, \textit{NovelSum} can offer insights through comparative analyses, such as: (1) ShareGPT, which collects data from real internet users, exhibits greater diversity than Dolly, which relies on company employees, suggesting that IT samples from diverse sources enhance dataset diversity \cite{wang2024diversity-logD}; (2) In LLaMA experiments, random selection can outperform some mainstream strategies, aligning with prior work \cite{xia2024rethinking,diddee2024chasing}, highlighting gaps in current data selection methods for optimizing diversity.



\subsection{Ablation Study}


\textit{NovelSum} involves several flexible hyperparameters and variations. In our main experiments, \textit{NovelSum} uses cosine distance to compute $d(x_i, x_j)$ in Eq. \ref{eq:dad}. We set $\alpha = 1$, $\beta = 0.5$, and $K = 10$ nearest neighbors in Eq. \ref{eq:pws} and \ref{eq:dad}. Here, we conduct an ablation study to investigate the impact of these settings based on LLaMA-3-8B.

\begin{table}[ht!]
\centering
\resizebox{\linewidth}{!}{
\begin{tabular}{lccc}
\toprule
\textbf{Variants} & \textbf{Pearson} & \textbf{Spearman} & \textbf{Avg.} \\
\midrule
NovelSum & 0.98 & 0.96 & 0.97 \\
\midrule
\hspace{0.10cm} - Use $L2$ distance & 0.97 & 0.83 & 0.90\textsubscript{↓ 0.08} \\
\hspace{0.10cm} - $K=20$ & 0.98 & 0.96 & 0.97\textsubscript{↓ 0.00} \\
\hspace{0.10cm} - $\alpha=0$ (w/o proximity) & 0.79 & 0.31 & 0.55\textsubscript{↓ 0.42} \\
\hspace{0.10cm} - $\alpha=2$ & 0.73 & 0.88 & 0.81\textsubscript{↓ 0.16} \\
\hspace{0.10cm} - $\beta=0$ (w/o density) & 0.92 & 0.89 & 0.91\textsubscript{↓ 0.07} \\
\hspace{0.10cm} - $\beta=1$ & 0.90 & 0.62 & 0.76\textsubscript{↓ 0.21} \\
\bottomrule
\end{tabular}
}
\caption{Ablation Study for \textit{NovelSum}.}
\label{tbl:ablation}
\vspace{-2mm}
\end{table}

In Table \ref{tbl:ablation}, $\alpha=0$ removes the proximity weights, and $\beta=0$ eliminates the density multiplier. We observe that both $\alpha=0$ and $\beta=0$ significantly weaken the correlation, validating the benefits of the proximity-weighted sum and density-aware distance. Additionally, improper values for $\alpha$ and $\beta$ greatly reduce the metric's reliability, highlighting that \textit{NovelSum} strikes a delicate balance between distances and distribution. Replacing cosine distance with Euclidean distance and using more neighbors for density approximation have minimal impact, particularly on Pearson's correlation, demonstrating \textit{NovelSum}'s robustness to different distance measures.






% \vspace{-5mm}
\section{RELATED WORK}
\label{sec:relatedwork}
In this section, we describe the previous works related to our proposal, which are divided into two parts. In Section~\ref{sec:relatedwork_exoplanet}, we present a review of approaches based on machine learning techniques for the detection of planetary transit signals. Section~\ref{sec:relatedwork_attention} provides an account of the approaches based on attention mechanisms applied in Astronomy.\par

\subsection{Exoplanet detection}
\label{sec:relatedwork_exoplanet}
Machine learning methods have achieved great performance for the automatic selection of exoplanet transit signals. One of the earliest applications of machine learning is a model named Autovetter \citep{MCcauliff}, which is a random forest (RF) model based on characteristics derived from Kepler pipeline statistics to classify exoplanet and false positive signals. Then, other studies emerged that also used supervised learning. \cite{mislis2016sidra} also used a RF, but unlike the work by \citet{MCcauliff}, they used simulated light curves and a box least square \citep[BLS;][]{kovacs2002box}-based periodogram to search for transiting exoplanets. \citet{thompson2015machine} proposed a k-nearest neighbors model for Kepler data to determine if a given signal has similarity to known transits. Unsupervised learning techniques were also applied, such as self-organizing maps (SOM), proposed \citet{armstrong2016transit}; which implements an architecture to segment similar light curves. In the same way, \citet{armstrong2018automatic} developed a combination of supervised and unsupervised learning, including RF and SOM models. In general, these approaches require a previous phase of feature engineering for each light curve. \par

%DL is a modern data-driven technology that automatically extracts characteristics, and that has been successful in classification problems from a variety of application domains. The architecture relies on several layers of NNs of simple interconnected units and uses layers to build increasingly complex and useful features by means of linear and non-linear transformation. This family of models is capable of generating increasingly high-level representations \citep{lecun2015deep}.

The application of DL for exoplanetary signal detection has evolved rapidly in recent years and has become very popular in planetary science.  \citet{pearson2018} and \citet{zucker2018shallow} developed CNN-based algorithms that learn from synthetic data to search for exoplanets. Perhaps one of the most successful applications of the DL models in transit detection was that of \citet{Shallue_2018}; who, in collaboration with Google, proposed a CNN named AstroNet that recognizes exoplanet signals in real data from Kepler. AstroNet uses the training set of labelled TCEs from the Autovetter planet candidate catalog of Q1–Q17 data release 24 (DR24) of the Kepler mission \citep{catanzarite2015autovetter}. AstroNet analyses the data in two views: a ``global view'', and ``local view'' \citep{Shallue_2018}. \par


% The global view shows the characteristics of the light curve over an orbital period, and a local view shows the moment at occurring the transit in detail

%different = space-based

Based on AstroNet, researchers have modified the original AstroNet model to rank candidates from different surveys, specifically for Kepler and TESS missions. \citet{ansdell2018scientific} developed a CNN trained on Kepler data, and included for the first time the information on the centroids, showing that the model improves performance considerably. Then, \citet{osborn2020rapid} and \citet{yu2019identifying} also included the centroids information, but in addition, \citet{osborn2020rapid} included information of the stellar and transit parameters. Finally, \citet{rao2021nigraha} proposed a pipeline that includes a new ``half-phase'' view of the transit signal. This half-phase view represents a transit view with a different time and phase. The purpose of this view is to recover any possible secondary eclipse (the object hiding behind the disk of the primary star).


%last pipeline applies a procedure after the prediction of the model to obtain new candidates, this process is carried out through a series of steps that include the evaluation with Discovery and Validation of Exoplanets (DAVE) \citet{kostov2019discovery} that was adapted for the TESS telescope.\par
%



\subsection{Attention mechanisms in astronomy}
\label{sec:relatedwork_attention}
Despite the remarkable success of attention mechanisms in sequential data, few papers have exploited their advantages in astronomy. In particular, there are no models based on attention mechanisms for detecting planets. Below we present a summary of the main applications of this modeling approach to astronomy, based on two points of view; performance and interpretability of the model.\par
%Attention mechanisms have not yet been explored in all sub-areas of astronomy. However, recent works show a successful application of the mechanism.
%performance

The application of attention mechanisms has shown improvements in the performance of some regression and classification tasks compared to previous approaches. One of the first implementations of the attention mechanism was to find gravitational lenses proposed by \citet{thuruthipilly2021finding}. They designed 21 self-attention-based encoder models, where each model was trained separately with 18,000 simulated images, demonstrating that the model based on the Transformer has a better performance and uses fewer trainable parameters compared to CNN. A novel application was proposed by \citet{lin2021galaxy} for the morphological classification of galaxies, who used an architecture derived from the Transformer, named Vision Transformer (VIT) \citep{dosovitskiy2020image}. \citet{lin2021galaxy} demonstrated competitive results compared to CNNs. Another application with successful results was proposed by \citet{zerveas2021transformer}; which first proposed a transformer-based framework for learning unsupervised representations of multivariate time series. Their methodology takes advantage of unlabeled data to train an encoder and extract dense vector representations of time series. Subsequently, they evaluate the model for regression and classification tasks, demonstrating better performance than other state-of-the-art supervised methods, even with data sets with limited samples.

%interpretation
Regarding the interpretability of the model, a recent contribution that analyses the attention maps was presented by \citet{bowles20212}, which explored the use of group-equivariant self-attention for radio astronomy classification. Compared to other approaches, this model analysed the attention maps of the predictions and showed that the mechanism extracts the brightest spots and jets of the radio source more clearly. This indicates that attention maps for prediction interpretation could help experts see patterns that the human eye often misses. \par

In the field of variable stars, \citet{allam2021paying} employed the mechanism for classifying multivariate time series in variable stars. And additionally, \citet{allam2021paying} showed that the activation weights are accommodated according to the variation in brightness of the star, achieving a more interpretable model. And finally, related to the TESS telescope, \citet{morvan2022don} proposed a model that removes the noise from the light curves through the distribution of attention weights. \citet{morvan2022don} showed that the use of the attention mechanism is excellent for removing noise and outliers in time series datasets compared with other approaches. In addition, the use of attention maps allowed them to show the representations learned from the model. \par

Recent attention mechanism approaches in astronomy demonstrate comparable results with earlier approaches, such as CNNs. At the same time, they offer interpretability of their results, which allows a post-prediction analysis. \par



\section{Conclusion}
In this work, we propose a simple yet effective approach, called SMILE, for graph few-shot learning with fewer tasks. Specifically, we introduce a novel dual-level mixup strategy, including within-task and across-task mixup, for enriching the diversity of nodes within each task and the diversity of tasks. Also, we incorporate the degree-based prior information to learn expressive node embeddings. Theoretically, we prove that SMILE effectively enhances the model's generalization performance. Empirically, we conduct extensive experiments on multiple benchmarks and the results suggest that SMILE significantly outperforms other baselines, including both in-domain and cross-domain few-shot settings.

% \section*{Impact Statement}
\system advances cost-efficient AI by demonstrating how small on-device language models can collaborate with powerful cloud-hosted models to perform data-intensive reasoning. By reducing reliance on expensive remote inference, \system makes advanced AI more accessible and sustainable. This has broad societal implications, including lowering barriers to AI adoption and enhancing data privacy by keeping more computations local. However, careful consideration must be given to potential biases in small models and the security risks of local code execution. 

\bibliographystyle{unsrt}
% \bibliographystyle{plainnat}
\bibliography{ref}


\clearpage
\appendix
\section{Missing Proofs}\label{app:proof}

In this section, we provide detailed proofs for Theorem \ref{thm:celora}. We first prove the following lemma.

\begin{lemma}\label{lm:m}
    Under Assumptions \ref{asp:proper}-\ref{asp:contractive}, if $\beta_1\in(0,1)$, it holds that
    \begin{align}
        \sum_{t=0}^T\mathbb{E}[\|\mathbf{m}^t-\nabla f(\mathbf{x}^t)\|_2^2]\le&\frac{2\|\mathbf{m}^0-\nabla f(\mathbf{x}^0)\|_2^2}{\beta_1}+\frac{4L^2}{\delta\beta_1^2}\sum_{t=1}^T\|\mathbf{x}^t-\mathbf{x}^{t-1}\|_2^2\nonumber\\
        &+\left(1-\frac{\delta}{2}\right)(1+6\beta_1)\sum_{t=1}^T\mathbb{E}[\|\nabla f(\mathbf{x}^t)\|_2^2]+6T\beta_1\sigma^2.\label{eq:lm-m}
    \end{align}
\end{lemma}
\begin{proof}
    According to the update of momentum, we have
    \begin{align}
        \mathbf{m}^{t}-\nabla f(\mathbf{x}^{t})=&(1-\beta_1)(\mathbf{m}^{t-1}-\nabla f(\mathbf{x}^{t}))+\beta_1(\hat{\mathbf{g}}^t-\nabla f(\mathbf{x}^t)).\nonumber
    \end{align}
    Taking expectation we have
    \begin{align}
        \mathbb{E}[\|\mathbf{m}^t-\nabla f(\mathbf{x}^t)\|_2^2]=&\mathbb{E}[\|(1-\beta_1)(\mathbf{m}^{t-1}-\nabla f(\mathbf{x}^t))+\beta_1(\mathbb{E}[\hat{\mathbf{g}}^t]-\nabla f(\mathbf{x}^t))\|_2^2]\nonumber\\
        &+\beta_1^2\mathbb{E}[\|\hat{\mathbf{g}}^t-\mathbb{E}[\hat{\mathbf{g}}^t]\|_2^2].\label{eq:pflm-m-1}
    \end{align}
    For the first term, applying Jensen's inequality yields
    \begin{align}
        &\mathbb{E}[\|(1-\beta_1)(\mathbf{m}^{t-1}-\nabla f(\mathbf{x}^t)+\beta_1(\mathbb{E}[\hat{\mathbf{g}}^t]-\nabla f(\mathbf{x}^t))\|_2^2]\nonumber\\
        \le&(1-\beta_1)\mathbb{E}[\|\mathbf{m}^{t-1}-\nabla f(\mathbf{x}^{t-1})-\nabla f(\mathbf{x}^t)+\nabla f(\mathbf{x}^{t-1})\|_2^2]+\beta_1\mathbb{E}[\|\mathbb{E}[\hat{\mathbf{g}}^t]-\nabla f(\mathbf{x}^t)\|_2^2].\label{eq:pflm-m-2}
    \end{align}
    By Young's inequality, we have
    \begin{align}
        \mathbb{E}[\|\mathbf{m}^{t-1}-\nabla f(\mathbf{x}^{t-1})-\nabla f(\mathbf{x}^t)+\nabla f(\mathbf{x}^{t-1})\|_2^2]\le&\left(1+\frac{\delta\beta_1}{2}\right)\mathbb{E}[\|\mathbf{m}^{t-1}-\nabla f(\mathbf{x}^{t-1})\|_2^2]\nonumber\\
        &+\left(1+\frac{2}{\delta\beta_1}\right)\mathbb{E}[\|\nabla f(\mathbf{x}^t)-\nabla f(\mathbf{x}^{t-1})\|_2^2].\label{eq:pflm-m-3}
    \end{align}
    For the second term, applying Cauchy's inequality yields
    \begin{align}
        \mathbb{E}[\|\hat{\mathbf{g}}^t-\mathbb{E}[\hat{\mathbf{g}}^t]\|_2^2]\le&3\mathbb{E}\|\hat{\mathbf{g}}^t-\mathbf{g}^t\|_2^2+3\mathbb{E}[\|\mathbf{g}^t-\nabla f(\mathbf{x}^t)\|_2^2]+3\mathbb{E}[\|\nabla f(\mathbf{x}^t)-\mathbb{E}[\hat{\mathbf{g}}^t]\|_2^2]\nonumber\\
        \le&6(1-\delta)\mathbb{E}[\|\nabla f(\mathbf{x}^t)\|_2^2]+3(2-\delta)\sigma^2,\label{eq:pflm-m-4}
    \end{align}
    where the last inequality uses Assumption \ref{asp:stochastic} and \ref{asp:contractive}.
    % and
    % \begin{align}
    %     &\mathbb{E}[C(g^k)-\nabla f(x^k)\|_2^2]\nonumber\\
    %     \le&\left(1+\frac{\delta}{2}\right)\mathbb{E}[\|C(g^k)-g^k\|_2^2]\nonumber\\
    %     &+\left(1+\frac{2}{\delta}\right)\mathbb{E}[\|g^k-\nabla f(x^k)\|_2^2]\nonumber\\
    %     \le&\left(1-\frac{\delta}{2}\right)\mathbb{E}[\|\nabla f(x^k)\|_2^2]+\frac{4\sigma^2}{\delta}.\label{eq:pflm-m-4}
    % \end{align}
    Applying \eqref{eq:pflm-m-2}\eqref{eq:pflm-m-3}\eqref{eq:pflm-m-4} to \eqref{eq:pflm-m-1} and using Assumption \ref{asp:smoothness} and \ref{asp:contractive}, we obtain
    \begin{align}
    \mathbb{E}[\|\mathbf{m}^t-\nabla f(\mathbf{x}^t)\|_2^2]\le&\left(1-\beta_1\left(1-\frac{\delta}{2}\right)\right)\mathbb{E}[\|\mathbf{m}^{t-1}-\nabla f(\mathbf{x}^{t-1})\|_2^2]+\frac{2L^2}{\delta\beta_1}\mathbb{E}[\|\mathbf{x}^t-\mathbf{x}^{t-1}\|_2^2]\nonumber\\
    &+(\beta_1+6\beta_1^2)(1-\delta)\mathbb{E}[\|\nabla f(\mathbf{x}^t)\|_2^2]+3(2-\delta)\beta_1^2\sigma^2.\label{eq:pflm-m-5}
    \end{align}
    Summing \eqref{eq:pflm-m-5} for $t=1,2,\cdots,T$ yields \eqref{eq:lm-m}.
\end{proof}

Now we are ready to prove Theorem \ref{thm:celora}. We first restate the theorem below in Theorem \ref{thm:celora-restate}.

\begin{theorem}\label{thm:celora-restate}
    Under Assumptions \ref{asp:proper}-\ref{asp:contractive}, if $\beta_1\in(0,\delta/(24-12\delta))$ and $\eta\le\min\{1/2L,\sqrt{(\delta\beta_1^2)/(8L^2)}\}$, CeLoRA with momentum SGD converges as
    \begin{align}
        \frac{1}{T+1}\sum_{t=0}^T\mathbb{E}[\|\nabla f(\mathbf{x}^t)\|_2^2]\le&\frac{4[f(\mathbf{x}^0)-\inf_{\mathbf{x}}f(\mathbf{x})]}{\delta\eta(T+1)}+\frac{4\|\mathbf{m}^0-\nabla f(\mathbf{x}^0)\|_2^2]}{\delta\beta_1(T+1)}+\frac{12\beta_1\sigma^2}{\delta}.\label{eq:thm-restate}
    \end{align}
\end{theorem}
\begin{proof}
    By Assumption \ref{asp:smoothness}, we have
    \begin{align}
        f(\mathbf{x}^{t+1})-f(\mathbf{x}^t)\le&\langle\nabla f(\mathbf{x}^t),\mathbf{x}^{t+1}-\mathbf{x}^t\rangle+\frac{L}{2}\|\mathbf{x}^{t+1}-\mathbf{x}^t\|_2^2\nonumber\\
        =&\left\langle\frac{\mathbf{m}^t}{2},\mathbf{x}^{t+1}-\mathbf{x}^t\right\rangle+\left\langle\nabla f(\mathbf{x}^t)-\frac{\mathbf{m}^t}{2},\mathbf{x}^{t+1}-\mathbf{x}^t\right\rangle+\frac{L}{2}\|\mathbf{x}^{t+1}-\mathbf{x}^t\|_2^2\nonumber\\
        =&-\left(\frac{1}{2\eta}-\frac{L}{2}\right)\|\mathbf{x}^{t+1}-\mathbf{x}^t\|_2^2+\frac{\eta}{2}\|\nabla f(\mathbf{x}^t)-\mathbf{m}^t\|_2^2-\frac{\eta}{2}\|\nabla f(\mathbf{x}^t)\|_2^2.\label{eq:pfthm-1}
    \end{align}
    Taking expectation and summing \eqref{eq:pfthm-1} for $t=0,1,\cdots,T$ yields
    \begin{align}
        \inf_{\mathbf{x}}f(\mathbf{x})-f(\mathbf{x}^0)\le&\frac{\eta}{2}\sum_{t=0}^{T}\mathbb{E}[\|\nabla f(\mathbf{x}^t)-\mathbf{m}^t\|_2^2]-\left(\frac{1}{2\eta}-\frac{L}{2}\right)\sum_{t=0}^{T}\mathbb{E}[\|\mathbf{x}^{t+1}-\mathbf{x}^t\|_2^2]\nonumber\\
        &-\frac{\eta}{2}\sum_{t=0}^T\mathbb{E}[\|\nabla f(\mathbf{x}^t)\|_2^2].\label{eq:pfthm-2}
    \end{align}
    Applying Lemma \ref{lm:m} to \eqref{eq:pfthm-2} and noting that $\beta_1\in(0,\delta/(24-12\delta))$ implies $(1-\delta/2)(1+6\beta_1)\le1-\delta/4$, we obtain
    \begin{align}
        \frac{1}{T+1}\sum_{t=0}^T\mathbb{E}[\|\nabla f(\mathbf{x}^t)\|_2^2]\le&\frac{4[f(\mathbf{x}^0)-\inf_{\mathbf{x}}f(\mathbf{x})]}{\delta\eta(T+1)}+\frac{4\|\mathbf{m}^0-\nabla f(\mathbf{x}^0)\|_2^2}{\delta\beta_1(T+1)}+\frac{12\beta_1\sigma^2}{\delta}\nonumber\\
        &-\frac{4}{\delta\eta}\left(\frac{1}{2\eta}-\frac{L}{2}-\frac{2\eta L^2}{\delta\beta_1^2}\right)\sum_{t=0}^T\|\mathbf{x}^{t+1}-\mathbf{x}^t\|_2^2.\label{eq:pfthm-3}
    \end{align}
   Since $\eta\le\min\{1/2L,\sqrt{(\delta\beta_1^2)/(8L^2)}\}$ implies $1/(4\eta)\ge L/2$ and $1/(4\eta)\ge(2\eta L^2)/(\delta\beta_1^2)$, \eqref{eq:thm-restate} is a direct result of \eqref{eq:pfthm-3}.
\end{proof}
% % \section{Additional Experiments}\label{app:delta}

% \textbf{Empirical justification of Assumption \ref{asp:contractive}. } In order to justify \eqref{eq:asp-ecgk}, we conduct experiments on language model fine-tuning tasks on \cgd{[XXX]} model using COLA, RTE and MRPC datasets, three tasks in the GLUE benchmark. In these experiments, we alternatively calculate one iteration of full gradient AdamW and one epoch of random gradient AdamW, each with a learning rate of \texttt{1e-5} for a total of 80 cycles. We apply a rank of 64 for both LoRA and double-LoRA in CeLoRA, and apply a compression rate of $p_{\mathrm{FFN}}=0.9$ and $p_{\mathrm{MHA}}=0.4$. We calculate the relative error $\|\mathbb{E}_{\xi^k\sim\mathcal{D}}[C(x^k;\xi^k)]-\nabla f(x^k)\|^2/\|\nabla f(x^k)\|^2$ for every full-gradient step, as illustrated in Fig.~\ref{fig:ecgk}, where all relative errors are below 1.


% \begin{figure}
%     \centering
%     \begin{minipage}{0.33\textwidth}
%         \includegraphics[width=\textwidth]{figures/histogram_cola_delta_fullgrad.png}
%     \end{minipage}
%     \begin{minipage}{0.33\textwidth}
%         \includegraphics[width=\textwidth]{figures/histogram_mrpc_delta_fullgrad.png}
%     \end{minipage}\begin{minipage}{0.33\textwidth}
%         \includegraphics[width=\textwidth]{figures/histogram_rte_delta_fullgrad.png}
%     \end{minipage}
%     \caption{Relative errors on COLA (left), MRPC (middle) and RTE (right).}
%     \label{fig:ecgk}
% \end{figure}
% The experiment selects three tasks in GLUE(COLA,RTE,MRPC) for testing. For full batch gradient descent, the experiment alternately uses one full gradient descent and one epoch of random gradient descent for training, calculates the relative error of the LoRA model and CeLoRA model for each full gradient descent, and performs a total of 80 cycles. For the stochastic gradient descent , the experiment train for 1 epoch per task and calculate the relative error every 10 iterations. The LoRA rank is 64, the CeLoRA Double LoRA method rank is 64, the MLP sampling ratio is 0.9, and the MHA sampling ratio is 0.4, We use google/gemma-2b as experiment model and the optimizer is AdamW, learning rate is 1e-5.

% \textbf{Ablation study for Double LoRA}

%\section{Appendix 1}



%%%%%%%%%%%%%%%%%%%%%%%%%%%%%%%%%%%%%%%%%%%%%%%%%%%%%%%%%%%%

%\fbox{\begin{minipage}{38em}

\subsubsection*{Scaling Law Reproducilibility Checklist}\label{sec:checklist}


\small

\begin{minipage}[t]{0.48\textwidth}
\raggedright
\paragraph{Scaling Law Hypothesis (\S\ref{sec:power-law-form})}

\begin{itemize}[leftmargin=*]
    \item What is the form of the power law?
    \item What are the variables related by (included in) the power law?
    \item What are the parameters to fit?
    \item On what principles is this form derived?
    \item Does this form make assumptions about how the variables are related?
    % \item How are each of these variables counted? (For example, how is compute cost/FLOPs counted, if applicable? How are parameters of the model counted?)
    % \item Are code/code snippets provided for calculating these variables if applicable? 
\end{itemize}


\paragraph{Training Setup (\S\ref{sec:model_training})}
\begin{itemize}[leftmargin=*]
    \item How many models are trained?
    \item At which sizes?
    \item On how much data each? On what data? Is any data repeated within the training for a model?
    \item How are model size, dataset size, and compute budget size counted? For example, how are parameters of the model counted? Are any parameters excluded (e.g., embedding layers)?
    \item Are code/code snippets provided for calculating these variables if applicable?
    % embedding  For example, how is compute cost counted, if applicable? 
    \item How are hyperparameters chosen (e.g., optimizer, learning rate schedule, batch size)? Do they change with scale?
    \item What other settings must be decided (e.g., model width vs. depth)? Do they change with scale?
    \item Is the training code open source?
    % \item How is the correctness of the scaling law considered SHOULD WE?
\end{itemize}

\end{minipage}
\begin{minipage}[t]{0.48\textwidth}
\raggedright


\paragraph{Data Collection(\S\ref{sec:data})}
\begin{itemize}[leftmargin=*]
    \item Are the model checkpoints provided openly?
    % \item Are these checkpoints modified in any way before evaluation? (say, checkpoint averaging)
    % \item If the above is done, is code for modifying the checkpoints provided?
    \item How many checkpoints per model are evaluated to fit each scaling law?
    \item What evaluation metric is used? On what dataset?
    \item Are the raw evaluation metrics modified, e.g., through loss interpolation, centering around a mean, scaling logarithmically, etc?
    \item If the above is done, is code for modifying the metric provided? 
\end{itemize}

\paragraph{Fitting Algorithm (\S\ref{sec:opt})}
\begin{itemize}[leftmargin=*]
    \item What objective (loss) is used?
    \item What algorithm is used to fit the equation?
    \item What hyperparameters are used for this algorithm?
    \item How is this algorithm initialized?
    \item Are all datapoints collected used to fit the equations? For example, are any outliers dropped? Are portions of the datapoints used to fit different equations?
    \item How is the correctness of the scaling law considered? Extrapolation, Confidence Intervals, Goodness of Fit?
\end{itemize}

\end{minipage}

% \paragraph{Other}
% \begin{itemize}
%     \item Is code for 
% \end{itemize}

\end{minipage}}

\end{document}