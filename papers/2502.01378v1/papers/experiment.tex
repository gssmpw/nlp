\section{Experiments}

In this section, we present a comprehensive set of experiments to evaluate the convergence performance and computational efficiency of \celora, and compare it against the baseline method.


\subsection{Experimental Setup}

\textbf{Datasets.}
We follow the benchmark design outlined in \cite{hu2023llm} and evaluate \celora on two popular reasoning benchmarks:

\begin{itemize}[topsep=5pt, leftmargin=1em]
\vspace{-0.5em}
\item \textbf{Commonsense Reasoning}: This dataset includes eight tasks: BoolQ \cite{clark-etal-2019-boolq}, PIQA \cite{bisk2020piqa}, SocialQA \cite{sap2019socialiqa}, HellaSwag \cite{zellers2019hellaswag}, WinoGrande \cite{sakaguchi2021winogrande}, ARC-challenge \cite{clark2018think}, ARC-easy \cite{clark2018think}, and OpenbookQA \cite{mihaylov2018can}. In our experiments, we fine-tune all models using the Commonsense 170K dataset \cite{hu2023llm}, which is constructed by combining the training sets from these eight tasks.

\vspace{-0.5em}
\item \textbf{Arithmetic Reasoning}:
This benchmark consists of seven subsets: MultiArith \cite{roy2016solving}, GSM8K \cite{cobbe2021training}, AddSub \cite{hosseini2014learning}, AQuA \cite{ling-etal-2017-program}, SingleEq \cite{koncel-kedziorski-etal-2015-parsing}, SVAMP \cite{patel-etal-2021-nlp} and MAWPS \cite{koncel-kedziorski-etal-2016-mawps}.
We fine-tune the models on the Math 10k dataset \cite{hu2023llm}, which includes training data from GSM8K, MAWPS, and AQuA, augmented by language models with chain-of-thought reasoning steps.
\vspace{-0.5em}
\end{itemize}

\textbf{Fine-tuned models and hyper-parameters.} We fine-tune LLaMA-2-7B, LLaMA-2-13B \cite{touvron2023llama2openfoundation}, and LLaMA-3.1-8B \cite{llama3modelcard} using both \celora and LoRA.
% 
The adapter is applied to all linear layers in each transformer block, including \texttt{Q}, \texttt{K}, \texttt{V}, \texttt{O}, \texttt{Up}, \texttt{Gate}, and \texttt{Down}.
% 
Unless specified otherwise, all \celora experiments replace the frozen \texttt{V}, \texttt{O}, \texttt{Up}, and \texttt{Down} layers with \celora layers.
% 
The sparsity levels are set as follows: $p_\text{V} = p_\text{O} = 0.55$ and $p_\text{Up} = p_\text{Down} = 0.65$. 
% 
For consistency, the same set of hyperparameters is applied across both methods for each model size.
% 
All experiments are conducted using the BF16 format to optimize memory usage.
% 

\begin{table*}[!b]
    \caption{Comparison among eight commonsense reasoning tasks for the LLaMA2-7B/13B, LLaMA3.1-8B models}
    \vspace{0.5em}
    \label{table:common_accuracy}
    \centering
    \begin{adjustbox}{max width=\textwidth}

    \begin{tabular}{l l c c c c c c c c c c}
        \toprule
        \textbf{Model} & \textbf{Method} & \textbf{Rank} & \textbf{BoolQ} & \textbf{PIQA} & \textbf{SIQA} & \textbf{HellaSwag} & \textbf{Wino} & \textbf{ARC-e} & \textbf{ARC-c} & \textbf{OBQA} & \textbf{Avg. $\uparrow$} \\
        \midrule
        % \textcolor{gray}{GPT-3.5} & - & - & \textcolor{gray}{83.8} & \textcolor{gray}{56.4} & \textcolor{gray}{85.3} & \textcolor{gray}{38.9} & \textcolor{gray}{88.1} & \textcolor{gray}{69.9} & \textcolor{gray}{87.4} & \textcolor{gray}{72.8} \\
        % \midrule
        \multirow{4}{*}{LLaMA2-7B} 
            & LoRA & 16 & 71.99 & 84.49 & 81.73 & 94.45 & 85.95 & 87.63 & 73.21 & 83.80 & 82.91 \\
            & \cellcolor{skyblue}\celora & \cellcolor{skyblue}14 & \cellcolor{skyblue}70.24 & \cellcolor{skyblue}82.59 & \cellcolor{skyblue}79.27 & \cellcolor{skyblue}93.17 & \cellcolor{skyblue}82.72 & \cellcolor{skyblue}85.56 & \cellcolor{skyblue}70.65 & \cellcolor{skyblue}79.80 & \cellcolor{skyblue}80.50 \\
            & LoRA & 64 & 72.26 & 84.88 & 82.70 & 94.97 & 86.42 & 88.55 & 74.74 & 86.40 & 83.87 \\
            & \cellcolor{skyblue}\celora & \cellcolor{skyblue}56 & \cellcolor{skyblue}71.68 & \cellcolor{skyblue}85.20 & \cellcolor{skyblue}82.09 & \cellcolor{skyblue}94.61 & \cellcolor{skyblue}83.98 & \cellcolor{skyblue}87.29 & \cellcolor{skyblue}73.29 & \cellcolor{skyblue}84.20 & \cellcolor{skyblue}82.79 \\
        \midrule
        \multirow{4}{*}{LLaMA2-13B}
            & LoRA & 16 & 75.32 & 88.03 & 83.21 & 81.14 & 92.34 & 88.20 & 96.08 & 88.71 & 86.63 \\
            & \cellcolor{skyblue}\celora & \cellcolor{skyblue}14 & \cellcolor{skyblue}73.21 & \cellcolor{skyblue}86.62 & \cellcolor{skyblue}82.14 & \cellcolor{skyblue}94.65 & \cellcolor{skyblue}86.27 & \cellcolor{skyblue}90.15 & \cellcolor{skyblue}77.22 & \cellcolor{skyblue}84.80 & \cellcolor{skyblue}84.38 \\
            & LoRA & 64 & 75.72 & 88.85 & 84.39 & 96.34 & 88.71 & 92.42 & 81.83 & 89.60 & 87.23 \\
            & \cellcolor{skyblue}\celora & \cellcolor{skyblue}56 & \cellcolor{skyblue}74.01 & \cellcolor{skyblue}86.51 & \cellcolor{skyblue}83.11 & \cellcolor{skyblue}92.74 & \cellcolor{skyblue}87.92 & \cellcolor{skyblue}91.37 & \cellcolor{skyblue}79.61 & \cellcolor{skyblue}85.40 & \cellcolor{skyblue}85.08 \\
        \midrule
        \multirow{4}{*}{LLaMA3-8B} 
            & LoRA & 16 & 75.84 & 90.86 & 83.52 & 96.93 & 89.90 & 94.07 & 84.47 & 88.8 & 88.05 \\
            & \cellcolor{skyblue}\celora & \cellcolor{skyblue}14 & \cellcolor{skyblue}72.08 & \cellcolor{skyblue}89.72 & \cellcolor{skyblue}82.65 & \cellcolor{skyblue}96.24 & \cellcolor{skyblue}88.32 & \cellcolor{skyblue}93.35 & \cellcolor{skyblue}83.36 & \cellcolor{skyblue}87.60 & \cellcolor{skyblue}86.66 \\
            & LoRA & 64 & 75.63 & 90.21 & 83.32 & 96.38 & 88.95 & 93.39 & 84.04 & 89.20 & 87.64 \\
            & \cellcolor{skyblue}\celora & \cellcolor{skyblue}56 & \cellcolor{skyblue}73.36 & \cellcolor{skyblue}89.66 & \cellcolor{skyblue}82.40 & \cellcolor{skyblue}95.76 & \cellcolor{skyblue}86.42 & \cellcolor{skyblue}93.14 & \cellcolor{skyblue}82.68 & \cellcolor{skyblue}87.60 & \cellcolor{skyblue}86.38 \\
        \bottomrule

        % \multirow{4}{*}{LLaMA3-8B} 
        %     & Full FT & 100 & 99.2 & 62.0 & 93.9 & 26.8 & 96.7 & 74.0 & 91.2 & 77.7 \\
        %     & LoRA & 0.70 & 99.5 & 61.6 & 92.7 & 25.6 & 96.3 & 73.8 & 90.8 & 77.2 \\
        %     & LoRA & 0.71 & 98.8 & 62.7 & 92.2 & 26.8 & 96.9 & 74.0 & 91.2 & 77.5 \\
        %     & \cellcolor{skyblue}SFT (\celora) & \cellcolor{skyblue}0.70 & \cellcolor{skyblue}\textbf{99.7} & \cellcolor{skyblue}\textbf{65.8} & \cellcolor{skyblue}\textbf{93.7} & \cellcolor{skyblue}\textbf{31.5} & \cellcolor{skyblue}\textbf{97.8} & \cellcolor{skyblue}76.0 & \cellcolor{skyblue}\textbf{92.4} & \cellcolor{skyblue}\textbf{79.6} \\
        % \bottomrule
    \end{tabular}
    \end{adjustbox}
    % \vspace{-1em}
\end{table*}

\subsection{Statistical Efficiency of \celora}

\begin{figure*}[!t]
  \centering
    \includegraphics[width=\linewidth]{figures/loss.pdf}
  \vspace{-5mm}
  \caption{Loss curve of commonsense reasoning fine-tune task. Each row in the figure corresponds to a different trainable parameter setting, while each column represents base models: LLaMA2-7B/13B and LLaMA3.1-8B.}
  \label{fig:loss_curve}
\end{figure*}

In this set of experiments, we evaluate the convergence performance of \celora using two critical metrics: the accuracy achieved on each benchmark and the trajectory of the fine-tuning loss across training iterations. 
% 
By monitoring these metrics, we aim to gain insights into how quickly and effectively \celora converges compared to LoRA.
%


\textbf{Accuracy}. %\label{sec:benchmark-accuracy}
We trained both \celora and LoRA under low-rank (LoRA rank of $16$, \celora rank of $14$) and high-rank (LoRA rank of $64$, \celora rank of $56$) configurations across two reasoning datasets for one epoch.
Table \ref{table:common_accuracy} and Table \ref{table:math_accuracy} summarize the results for the commonsense and arithmetic reasoning benchmarks.
% 
The experimental outcomes demonstrate that, across all LoRA rank settings in both benchmarks, \celora achieves fine-tuning accuracy that is nearly identical to that of LoRA, with an average difference in results of 1.58\%.
%
These findings suggest that our approach has a negligible impact on the original LoRA fine-tuning accuracy.
% 
The slight differences in accuracy between \celora and LoRA on the test sets can primarily be attributed to the scaling of \celora's rank, which was adjusted to ensure a fair experimental comparison.


\begin{table*}[!b]
    \caption{Performance comparison of LoRA and \celora on seven arithmetic reasoning tasks.}
    \label{table:math_accuracy}
    \vspace{0.5em}
    \centering
    \begin{adjustbox}{max width=\textwidth}
    \begin{tabular}{l l c c c c c c c c c}
        \toprule
        \textbf{Model} & \textbf{Method} & \textbf{Rank} & \textbf{MultiArith} & \textbf{GSM8K} & \textbf{AddSub} & \textbf{AQuA} & \textbf{SingleEq} & \textbf{SVAMP} & \textbf{MAWPS} & \textbf{Avg. $\uparrow$} \\
        \midrule
        % \textcolor{gray}{GPT-3.5} & - & - & \textcolor{gray}{83.8} & \textcolor{gray}{56.4} & \textcolor{gray}{85.3} & \textcolor{gray}{38.9} & \textcolor{gray}{88.1} & \textcolor{gray}{69.9} & \textcolor{gray}{87.4} & \textcolor{gray}{72.8} \\
        % \midrule

        % \multirow{4}{*}{LLaMA2-7B} 
        %     & LoRA & 16 & 80.00 & 33.28 & 85.32 & 25.98 & 84.45 & 54.00 & 77.73 & 62.97 \\
        %     & \cellcolor{skyblue}\celora & \cellcolor{skyblue}14 & \cellcolor{skyblue}74.50 & \cellcolor{skyblue}24.41 & \cellcolor{skyblue}76.71 & \cellcolor{skyblue}22.05 & \cellcolor{skyblue}77.56 & \cellcolor{skyblue}48.30 & \cellcolor{skyblue}71.85 & \cellcolor{skyblue}56.48 \\
        %     & LoRA & 64 & 90.00 & 33.43 & 81.77 & 26.38 & 83.46 & 53.60 & 79.83 & 64.07 \\
        %     & \cellcolor{skyblue}\celora & \cellcolor{skyblue}56 & \cellcolor{skyblue}77.67 & \cellcolor{skyblue}28.89 & \cellcolor{skyblue}81.52 & \cellcolor{skyblue}21.26 & \cellcolor{skyblue}81.69 & \cellcolor{skyblue}48.40 & \cellcolor{skyblue}77.31 & \cellcolor{skyblue}59.53 \\
        % \midrule
        
        % \multirow{4}{*}{LLaMA2-13B}
        %     & LoRA & 16 & 91.00 & 48.98 & 84.81 & 27.16 & 88.98 & 68.20 & 81.09 & 70.03 \\
        %     & \cellcolor{skyblue}\celora & \cellcolor{skyblue}14 & \cellcolor{skyblue}85.17 & \cellcolor{skyblue}39.65 & \cellcolor{skyblue}79.49 & \cellcolor{skyblue}30.32 & \cellcolor{skyblue}85.63 & \cellcolor{skyblue}59.70 & \cellcolor{skyblue}79.41 & \cellcolor{skyblue}65.62 \\
        %     & LoRA & 64 & 93.17 & 49.66 & 88.10 & 32.68 & 88.58 & 67.90 & 81.93 & 71.59 \\
        %     & \cellcolor{skyblue}\celora & \cellcolor{skyblue}56 & \cellcolor{skyblue}88.33 & \cellcolor{skyblue}44.35 & \cellcolor{skyblue}80.51 & \cellcolor{skyblue}30.71 & \cellcolor{skyblue}88.78 & \cellcolor{skyblue}61.40 & \cellcolor{skyblue}80.67 & \cellcolor{skyblue}67.82 \\
        % \midrule
        
        \multirow{4}{*}{LLaMA3-8B} 
            & LoRA & 16 & 94.50 & 64.59 & 90.89 & 47.24 & 92.13 & 76.30 & 88.66 & 79.19 \\
            & \cellcolor{skyblue}\celora & \cellcolor{skyblue}14 & \cellcolor{skyblue}94.00 & \cellcolor{skyblue}62.09 & \cellcolor{skyblue}91.14 & \cellcolor{skyblue}44.88 & \cellcolor{skyblue}93.50 & \cellcolor{skyblue}75.00 & \cellcolor{skyblue}90.76 & \cellcolor{skyblue}78.77 \\
            & LoRA & 64 & 96.33 & 65.50 & 90.63 & 49.61 & 92.91 & 81.2 & 89.50 & 80.81 \\
            & \cellcolor{skyblue}\celora & \cellcolor{skyblue}56 & \cellcolor{skyblue}96.17 & \cellcolor{skyblue}62.02 & \cellcolor{skyblue}88.86 & \cellcolor{skyblue}47.64 & \cellcolor{skyblue}93.31 & \cellcolor{skyblue}77.10 & \cellcolor{skyblue}89.08 & \cellcolor{skyblue}79.17 \\
        \bottomrule
        
        % \multirow{4}{*}{LLaMA3-8B} 
        %     & Full FT & 100 & 99.2 & 62.0 & 93.9 & 26.8 & 96.7 & 74.0 & 91.2 & 77.7 \\
        %     & LoRA & 0.70 & 99.5 & 61.6 & 92.7 & 25.6 & 96.3 & 73.8 & 90.8 & 77.2 \\
        %     & DoRA & 0.71 & 98.8 & 62.7 & 92.2 & 26.8 & 96.9 & 74.0 & 91.2 & 77.5 \\
        %     & \cellcolor{skyblue}SFT (\celora) & \cellcolor{skyblue}0.70 & \cellcolor{skyblue}\textbf{99.7} & \cellcolor{skyblue}\textbf{65.8} & \cellcolor{skyblue}\textbf{93.7} & \cellcolor{skyblue}\textbf{31.5} & \cellcolor{skyblue}\textbf{97.8} & \cellcolor{skyblue}76.0 & \cellcolor{skyblue}\textbf{92.4} & \cellcolor{skyblue}\textbf{79.6} \\
        % \bottomrule
    \end{tabular}
    \end{adjustbox}
    
\end{table*}
\textbf{Loss curve}.
Figure \ref{fig:loss_curve} illustrates the loss curves of both \celora and LoRA under different rank settings across the three models on the commonsense reasoning fine-tuning task.
% 
In each setting, \celora's loss curves nearly overlap with those of its LoRA counterparts, indicating similar convergence behaviors.
% 
These results highlight the effectiveness of our method, empirically demonstrating that \celora can achieve nearly the same convergence capability as the original LoRA while potentially offering computational savings. 
% 
The overlapping loss curves suggest that \celora does not introduce additional convergence challenges and maintains training stability comparable to LoRA.



\subsection{Computation Efficiency}

In these experiments, we measure \celora's training efficiency by comparing the average training step latency of a single-layer \celora with a single-layer LoRA. 
% 
All experiments are conducted on a single \texttt{NVIDIA-HGX-H20-(96GB)} GPU to maintain consistent hardware conditions. 
% 
For a fair comparison, both \celora and LoRA employ the same trainable rank of $64$. 
% 
We run experiments on three different model weight sizes---$(8192, 8192)$, $(4096, 4096)$, and $(2048, 2048)$---using a fixed batch size of $16$ and a sequence length of $8192$. 
% 
To measure average training step latency, each configuration is tested over 100 runs.%, each consisting of 500 training iterations. 
% 
The first 10 iterations of each run are considered warmup and are excluded from latency measurements to mitigate initialization overhead.



Figure~\ref{fig:latency} compares the results of LoRA and \celora with various sparsity levels and shows that \celora achieves a consistent reduction in overall training time, with a maximum of $36.3\%$ speedup. 
% 
As illustrated, \celora’s forward pass latency closely matches that of LoRA's due to the unchanged forward logic of the frozen layer. 
% 
However, in the backward pass, \celora outperforms LoRA by up to $3.39\times$ with some aggressive sampling rate.
% 
The observed improvements in wall-clock speed are primarily attributed to two key factors:
% 
(\underline{i}) \celora effectively reduces the theoretical floating-point operations required during backpropagation for frozen layers. 
(\underline{ii}) We developed specialized CUDA kernels tailored for low-rank computations inherent in \celora's backpropagation process, which optimize memory access patterns, resulting in enhanced computational efficiency and reduced latency.

\begin{figure*}[!t]
  \centering
    \includegraphics[width=\linewidth]{figures/latency.pdf}
  \vspace{-5mm}
  \caption{Comparison of training latency for \celora and LoRA at various sparsity levels (i.e., $\frac{1}{2}$, $\frac{1}{4}$, $\frac{1}{8}$, $\frac{1}{16}$) across three model shapes: $(8192, 8192)$, $(4096, 4096)$, and $(2048, 2048)$. \celora provides significant speedups in the backward pass, leading to a maximum of $36.3\%$ overall reduction in end-to-end training time compared to LoRA.}
  \label{fig:latency}
\end{figure*}