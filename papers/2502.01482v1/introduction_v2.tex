\section{Introduction} \label{sec:intro}

Remote observation of distributed sources is key to many Internet of things (IoT) applications, ranging from agricultural and environmental monitoring, to industrial control and asset tracking \cite{Bedewy21_TIT,Yates20_Survey,Uysal21_Semantic}. In these contexts, a possibly large number of devices sample physical processes of interest, and deliver updates to a common receiver for further processing or decision making \cite{Pappas24_TCOM,Soleymani20_valueInfo}. To accommodate the sporadic and potentially unpredictable traffic generated by low-complexity terminals over a shared wireless channel, random access schemes based on variations of the ALOHA policy \cite{Abramson77:PacketBroadcasting} are typically employed, as epitomized by several commercial solutions \cite{LoRa,SigFox}.

The relevance of these applications has steered recent research towards understanding how the channel access policies shall be tuned, aiming to provide the monitor with an accurate estimate of the state of tracked processes. In this perspective, first steps were taken tackling freshness, captured by means of age of information (AoI) \cite{Yates17:AoI_SA,Modiano18_AoI,Munari21_TCOM_AoI,Uysal21_AlohaThresh,Bidokhti22_TIT,Badia22_NetwLetters,Badia24_TMC,Liew20_INFOCOM,Munari23_TCOM}. These studies were later refined to account for the actual status of the sources and the estimate available at the receiver, e.g., via age of incorrect information \cite{Ephremides19_AoII,Chiariotti23:ICC,Munari24_ICC,Chiariotti25_INFOCOM}, false alarm and missed detection probabilities \cite{Pappas24_MOBIHOC,Munari23_Asilomar}, as well as information-theoretic inspired \cite{Cocco23_JSAIT,Liew22_TIT,Pappas23_WiOpt} and other metrics \cite{Soleymani20_valueInfo,Pappas24_WiOpt,Yates20_Survey}.

In this work, we take the lead from two practical observations. Fist, many IoT networks operate without feedback, so that nodes are unaware of the outcome of their transmissions, e.g., \cite{LoRa}. Second, we note that control and monitoring tasks are typically performed at the application layer, which, in turn, may receive from lower layers a limited amount of information. In most systems, indeed, only messages generated by the tracked source would be forwarded to the application, possibly together with general information on the network configuration, e.g., number of registered nodes. This configuration hinders the realization of advanced estimators, e.g., \cite{Cocco23_JSAIT}, which exploit cross-layer information such as level of interference, lack of transmissions or reception of messages from other sources, to refine knowledge on the process. 

Based on this, we consider a basic solution, which simply retains the last received status update as current estimate \-- sometimes also referred to as martingale estimator \cite{Ulukus24_Martingale,Munari23_Asilomar}. Leaning on this, we study the problem of monitoring two-state Markov sources over a slotted random access channel without feedback, and characterize performance in terms of the uncertainty at the receiver given the available information, i.e., the last obtained message on the state of a process of interest and when such value was measured, corresponding to the AoI. Specifically, we focus on variations of ALOHA in which each terminal probabilistically transmits an update based on the current and previous value of the tracked process. We provide an analytical derivation of the conditional entropy at the receiver, and compare the behavior of three main strategies: a benchmark scheme where access probabilities are set to maximize throughput, regardless of the source evolution; a reactive approach in which nodes access the medium only when a change in source state takes place; and a balanced strategy, whose access probabilities are obtained to minimize the uncertainty metric. Our study reveals non-trivial insights, showing how optimizing channel throughput does not necessarily lead to a lower uncertainty at the receiver. Moreover, we show that, while for symmetric sources (i.e., with balanced transitions between the two states), a reactive policy is optimal, more complex solutions that also foresee transmission in the absence of source transitions are to be preferred when asymmetric processes are monitored.

\emph{Notation}: Matrices are indicated in capital boldface, e.g. $\mathbf A$, whereas column (row) vectors in lowercase boldface, e.g. $\mathbf v$ ($\mathbf v^{\mathsf T}$). We denote a discrete random variable (r.v.) and its realization by lower- and uppercase letters, e.g., $X$ and $x$. Its probability mass function is indicated as \mbox{$\mathsf P(X=x) = p(x)$}, and the conditional distribution of $X$ given $Y$ is expressed as $p(x\given y)$. For a homogeneous discrete time Markov chain $X_n$, $n\in\mathbb N$, with finite space state $\mathcal X$, $q_{ij}$ is the one-step transition probability from state $i$ to $j$, whereas $q_{ij}(\ell)$ is the $\ell$-step transition probability. Finally, $\mathbf 1_\ell$ is the $\ell$ dimensional column vector with all ones, and $\mathbf I_\ell$ the $\ell\times\ell$ identity matrix.%——, i.e., the probability of being in $j$ at step $\ell$ starting in $i$ at time $0$.%Let $\mathbf P$ be the $k\times k$ $\ell$-step transition matrix of the Markov chain $X_n$, and let $x$ be the $i$-th element of the ordered state space $\mathcal X$. Then, we denote as $\mathbf e_i$ the $k\times 1$ column vector composed of all zeros, except for a $1$ the in the $i$-th position.  


