\section{Results and Discussion}
\label{sec:results}

%To gauge the impact of different channel access policies on the uncertainty at the receiver, we will
We focus on three main strategies, labeled \emph{random}, \emph{reactive}, and \emph{balanced}. The first is obtained by setting $\pTx_{\mc_{n-1}\mcn} = 1/\nodes$, and corresponds to an approach that maximizes the throughput, transmitting regardless of the evolution of the tracked sources. In turn, the reactive scheme fully ties access to source transitions, sending an update if a state change occurs, and remaining silent otherwise, i.e., $\pTxVec=[0,1,1,0]$. Finally, we denote as balanced an approach in which \pTxVec\ has been optimized to minimize $H(\Mcn\given\Agen,\Estn)$.

As a preliminary step, the tightness of the approximation in \eqref{eq:ps} is verified in \figr\ref{fig:prob_verification}. Here, markers denote the calculation of \mbox{$\pr(\Mcn{=}0\given\agen,\Estn{=}0)$} and $\pr(\Agen{=}\agen,\Estn{=}0)$ derived via \eqref{eq:pmfXnGivenDeltaXnHat}, \eqref{eq:condPMFAge_complete}, \eqref{eq:pXn}, whereas solid lines are the outcome of simulations which jointly track the behavior of each node, characterizing the success probability exactly.

\begin{figure}
    \subfloat[]{
        \includegraphics[width=.49\columnwidth]{./Figures/condPMF_Xn.pdf}}
    \hspace*{.3em}
    \subfloat[]{
        \includegraphics[width=.49\columnwidth]{./Figures/jointPMF.pdf}}
    \caption{Comparison of analysis (under the approximation \eqref{eq:ps}) and simulation (considering the exact interference behavior of the network). Results obtained for $\nodes=50$ nodes, $\qZO=\qOZ=0.02$, $\pTxVec = [0,1,1,0]$.}
    \label{fig:prob_verification}
\end{figure}



We then start by studying symmetric sources ($\eta=1$). Considering $\alpha{=}\beta{=}0.02$, Fig.~\ref*{fig:pmfH} reports the cumulative distribution function of the uncertainty at the receiver at a generic point in time, given the current value of AoI and estimate, obtained analytically as $\pr(\hspace{.1em}\condent(\agen,\estn){\leq} \zeta )= \!\!\sum_{\mathcal A} p(\agen,\est)$, where \mbox{$\mathcal A = \{(\agen,\estn) \!\!: \condent(\agen,\estn)\leq \zeta)\}$}.
%\begin{align}
%    \pr\big(\hspace{.1em}\condent(\agen,\estn)\leq \zeta \,\big)\,\,= \!\!\sum_{\substack{(\agen,\estn): \\ \condent(\agen,\estn)\leq \zeta}} \!\!\!p(\agen,\est).
%\end{align}
Clearly, the larger the number of nodes, the higher the uncertainty, as updates from a specific source are delivered more sporadically. More interestingly, the use of a reactive strategy reduces the receiver uncertainty. From this standpoint, although transmitting only in case of state change may leave the receiver with an incorrect estimate for longer times compared to randomly sending updates (see, e.g., \cite{Munari24_ICC}), the effect is more than counterbalanced by the beneficial impact of tying channel access to the source evolution. A solid intuition for this can be obtained considering the simple case $\nodes=1$. If a reactive strategy is employed, the receiver retains certain knowledge on source state as soon as the first message is received ($\condent(\agen,\estn) = 0$). Conversely, if the source transmits randomly, a change of state might have occurred without the receiver being notified, leading to an uncertainty that grows over time until a new update is received. The effect holds also as more nodes contend, with the random policy exhibiting a steeper rise over time of $\condent(\agen,\estn)$, as illustrated in the example of Fig.~\ref{fig:timeline_comparison}.

\begin{figure}
    \centering
    \includegraphics[width=.85\columnwidth]{./Figures/pmfH.pdf}
    \vspace{-1em}
    \caption{CDF of $\condent(\agen,\est)$, considering symmetric sources with $\alpha=0.02$.}
    \label{fig:pmfH}
    \vspace{-1em}
\end{figure}

\begin{figure}
    \centering
    \includegraphics[width=.9\columnwidth]{./Figures/entropyTimeline_RanVsRea.pdf}
    \caption{Example of time evolution of the entropy $\condent(\agen,\estn)$ for symmetric sources, with $\qZO=0.02$, $\nodes=50$. For the random strategy (dash-dotted lines), the transmission probability has been set to $\pTx=1/\nodes$. The steeper rise of the receiver uncertainty of this approach compared to the reactive one (solid line) is evident at the very beginning, as well as after the second reset of $\condent(\agen,\estn)$ for the random scheme.}
    \vspace{-1em}
    \label{fig:timeline_comparison}
\end{figure}



To further elaborate, \figr\ref*{fig:symmetric}a reports $H(\Mcn\given\Agen,\Estn)$ as a function of the number of nodes. As discussed, for large populations, the average uncertainty converges for all policies to the stationary entropy of the source $\mathsf H(\Mc)$. Focus first on the random and balanced schemes. For the latter, the optimized transmission probabilities are shown by the circle-marked line in \figr\ref*{fig:symmetric}b, referring to the left $y$-axis, and compared for reference to the $\lambda{=}1/\nodes$ access probability of the random scheme. The numerical optimization leads to $\pTx_{00}{=}\pTx_{11}{=}0$ regardless of \nodes, i.e., a node shall send no update if the source does not transition. On the other hand, a pure reactive solution is optimal ($\pTx_{01}{=}\pTx_{10}{=}1$) as long as the channel is not congested. This is confirmed by the green dash-dotted curve, referred to the right $y$-axis, showing the  average number of incoming packets per slot at the receiver (i.e., channel load). As \nodes\ increases, having nodes transmit only in case of state change, yet with probability smaller than $1$, is convenient to curb collisions. These  trends pinpoint a trade-off between throughput (maximized for load $1$ pkt/slot) and uncertainty at the receiver, with the latter improved by transmitting fewer yet more informative packets. This is further buttressed by the yellow, dash-dotted curve in \figr\ref*{fig:symmetric}, outlining a modification of the plain reactive scheme in which nodes transmit also in case of no source change, with probability chosen such that the overall average channel load is always $1$ pkt/slot, to maximize throughput. An increase in the receiver uncertainty is apparent.

\begin{figure}
    \subfloat[]{
        \includegraphics[width=.48\columnwidth]{./Figures/HvsNodes_symmetric_slim.pdf}}
    \hspace*{.1em}
    \subfloat[]{
        \includegraphics[width=.5\columnwidth]{./Figures/opt_pTx_vsNodes_symmetric_slim.pdf}}
    \caption{(a) $\ent(\Mcn\given\Agen,\Estn)$ vs number of nodes, symmetric sources ($\qZO=0.02$); (b) transmission probabilities (random and balanced strategies).}
    \label{fig:symmetric}
    \vspace{-1em}
\end{figure}


\begin{figure}
    \subfloat[]{
        \includegraphics[width=.45\columnwidth]{./Figures/HVsEta_slim.pdf}}
    \hspace*{.1em}
    \subfloat[]{
        \includegraphics[width=.47\columnwidth]{./Figures/opt_pTx_vsEta_asymmetric_g08.pdf}}
    \caption{(a) $\ent(\Mcn\given\Agen,\Estn)$ vs asymmetry factor $\eta$; (b) optimized transmission probabilities for the balanced strategy. $\nodes=50$; source parameters set to have on average $\nodes(\pi_0\alpha {+} \pi_1\beta){=}0.8$ state transitions per slot in the network.}
    \label{fig:asymm}
    \vspace{-1em}
\end{figure}

The situation changes significantly when asymmetric sources are to be monitored. To study this, we report in \figr\ref*{fig:asymm}a $H(\Estn\given\Agen,\Estn)$ against the asymmetry factor $\eta$ for $\nodes{=}50$ nodes. The source parameters $(\alpha$, $\beta)$ have been set such that, for any $\eta$ the average number of state transitions in the network is constant, i.e., $\nodes(\pi_0\alpha {+} \pi_1\beta){=}0.8$. As expected, as $\eta$ increases, sources spend more time in one state (i.e, $1$), leading to lower uncertainty. Two facts shall be stressed, though. First, resorting to a pure reactive policy is once more be preferred to the use of a random solution, all the more so with high asymmetry. However, in contrast to what discussed for the symmetric case, a solely reactive approach no longer minimizes the uncertainty, as done by the balanced scheme. This is clarified in \figr\ref{fig:asymm}b, reporting the optimized transmission probabilities ($\lambda_{11}{=}0$ in all cases, is not shown for clarity). To reduce uncertainty, a transition to the less visited state $0$ shall be immediately notified ($\pTx_{01}{=}1$). As asymmetry grows, updates shall also be sent when the source remains in such state ($\pTx_{00}$ rising with $\eta$), compensating for a possible loss of the initial notification due to collision, and informing the receiver of the less likely conditions being experienced. Such messages become more important than notifying a return to the most common state, and $\lambda_{01}$ goes to $0$ to reduce congestion.





\section{Conclusions}
For remote monitoring of two-states Markov sources over random access channels, we characterized analytically the uncertainty of a simple estimator which retains the last received value as current estimate. Our study highlighted important trade-offs, showing that reactive schemes are to be preferred in the case of symmetric sources, even at the expense of operating the system at lower throughput. Instead, when asymmetric sources are to be tracked, a hybrid solution that also foresees transmissions in absence of a state transition can lead to lower entropy. In all cases, the presented framework allows for a simple optimization, providing useful protocol design hints.